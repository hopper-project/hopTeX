\documentclass[12pt]{amsart}

\usepackage{amsmath,amssymb,amsthm}
\usepackage[all]{xy}
\usepackage[dvipdfm,colorlinks=true]{hyperref}

\topmargin=0cm
\textheight=22cm
\textwidth=17cm
\oddsidemargin=-0.5cm
\evensidemargin=-0.5cm

\numberwithin{equation}{section}

\SelectTips{eu}{12}

\newtheorem{theorem}{Theorem}[section]
\newtheorem{proposition}[theorem]{Proposition}
\newtheorem{lemma}[theorem]{Lemma}
\newtheorem{corollary}[theorem]{Corollary}
\newtheorem{claim}[theorem]{Claim}

\theoremstyle{definition}
\newtheorem{definition}[theorem]{Definition}
\newtheorem{example}[theorem]{Example}
\newtheorem{conjecture}[theorem]{Conjecture}
\newtheorem{problem}[theorem]{Problem}

\theoremstyle{remark}
\newtheorem{remark}[theorem]{Remark}

\title{Samelson products in $p$-regular ${\mathrm{SO}}(2n)$}

\author{Daisuke Kishimoto}
\address{Department of Mathematics, Kyoto University, Kyoto, 606-8502, Japan}
\email{kishi@math.kyoto-u.ac.jp}
\author{Mitsunobu Tsutaya}
\address{Department of Mathematics, Kyoto University, Kyoto, 606-8502, Japan}
\email{tsutaya@math.kyoto-u.ac.jp}
\subjclass[2010]{55Q15}
\keywords{Samelson product, $p$-regular Lie group, special orthogonal group}

\begin{document}

\maketitle

\baselineskip 16pt

\begin{abstract}
A Lie group is called $p$-regular if it has the $p$-local homotopy type of a product of spheres. (Non)triviality of the Samelson products of the inclusions of the factor spheres into ${\mathrm{SO}}(2n)_{(p)}$ for $p$-regular ${\mathrm{SO}}(2n)$ is determined, which completes the list of (non)triviality of such Samelson products in $p$-regular simple Lie groups.
\end{abstract}

\section{Introduction and statement of the result}

Let $G$ be a compact connected Lie group. By the classical result of Hopf, it is well known that there is a rational homotopy equivalence
$$G\simeq_{(0)}S^{2n_1-1}\times\cdots\times S^{2n_\ell-1}$$
where $n_1\le\cdots\le n_\ell$. The sequence $\{n_1,\ldots,n_\ell\}$ is called the type of $G$. Here is the list of the types of simple Lie groups.

\begin{table}[htbp]
\centering
\begin{tabular}{l|l||l|l}
\hline
$\mathrm{SU}(n)$&$2,3,\ldots,n$&$\mathrm{G}_2$&$2,6$\\
${\mathrm{SO}}(2n+1)$&$2,4,\ldots,2n$&$\mathrm{F}_4$&$2,6,8,12$\\
$\mathrm{Sp}(n)$&$2,4,\ldots,2n$&$\mathrm{E}_6$&$2,5,6,8,9,12$\\
${\mathrm{SO}}(2n)$&$2,4,\ldots,2n-2,n$&$\mathrm{E}_7$&$2,6,8,10,12,14,18$\\
&&$\mathrm{E}_8$&$2,8,12,14,18,20,24,30$\\\hline
\end{tabular}
\end{table}

\noindent Serre generalizes the above rational homotopy equivalence to a $p$-local homotopy equivalence such that when $G$ is semisimple, there is a $p$-local homotopy equivalence 
\begin{equation}
\label{p-regular}
G\simeq_{(p)}S^{2n_1-1}\times\cdots\times S^{2n_\ell-1}
\end{equation}
if and only if $p\ge n_\ell$, in which case $G$ is called $p$-regular. In this paper we are interested in the standard multiplicative structure of the $p$-localization $G_{(p)}$ when $G$ is $p$-regular, and then we assume that $G$ is a simple Lie group in the above table and is $p$-regular throughout this section. Recall that for a homotopy associative H-space $X$ with inverse and maps $\alpha:A\to X,\beta:B\to X$, the correspondence
$$A\wedge B\to X,\quad(x,y)\mapsto\alpha(x)\beta(y)\alpha(x)^{-1}\beta(y)^{-1}$$
is called the Samelson product of $\alpha,\beta$ in $X$ and is denoted by $\langle\alpha,\beta\rangle$. One easily sees that for investigating the multiplicative structure of $G_{(p)}$, the Samelson products $\langle\epsilon_i,\epsilon_j\rangle$ play the fundamental role as in \cite{KK}, where $\epsilon_i$ is the inclusion $S^{2n_i-1}\to S^{2n_1-1}_{(p)}\times\cdots\times S^{2n_\ell-1}_{(p)}\simeq G_{(p)}$ into the $i$-th factor. So it is our task to determine (non)triviality of these Samelson products. 

We here make a remark on the choice of $\epsilon_i$ which depends on the $p$-local homotopy equivalence \eqref{p-regular}. Recall from \cite[Theorem 13.4]{T} that 
\begin{equation}
\label{pi(S)}
\pi_*(S^{2m-1}_{(p)})=0\quad\text{for}\quad 2m-1<*<2m+2p-4.
\end{equation}
Then we see that $\pi_{2n_i-1}(G_{(p)})$ is a free ${\mathbb{Z}}_{(p)}$-module for all $i$, and so $\pi_{2n_i-1}(G_{(p)})\cong{\mathbb{Z}}_{(p)}$ for all $i$ and $G\ne{\mathrm{SO}}(2n)$ since the entries of the type are distinct for $G\ne{\mathrm{SO}}(2n)$ as in the above table. Hence for $G\ne{\mathrm{SO}}(2n)$ we may choose any generator of $\pi_{2n_i-1}(G_{(p)})\cong{\mathbb{Z}}_{(p)}$ as $\epsilon_i$. For $G={\mathrm{SO}}(2n)$ we will make an explicit choice of $\epsilon_i$ below.

We first consider the Samelson products $\langle\epsilon_i,\epsilon_j\rangle$ in $G_{(p)}$ when $G$ is the classical group execpt for ${\mathrm{SO}}(2n)$.

\begin{theorem}
\label{classical}
Let $G$ be the $p$-regular classical group except for ${\mathrm{SO}}(2n)$, and let $\epsilon_i$ be a generator of $\pi_{2n_i-1}(G_{(p)})\cong{\mathbb{Z}}_{(p)}$ for the type $\{n_1,\ldots,n_\ell\}$ of $G$. Then 
$$\langle\epsilon_i,\epsilon_j\rangle\ne 0\quad\text{if and only if}\quad n_i+n_j>p.$$
\end{theorem} 

\begin{proof}
If $G=\mathrm{SU}(n),\mathrm{Sp}(n)$, nontriviality of the Samelson products follows from the result of Bott \cite{B} and triviality follows from the fact that $\pi_{2*}(G_{(p)})=0$ for $*<p$ which is deduced from \eqref{pi(S)}. Since there is a homotopy equivalence as loop spaces $\mathrm{Sp}(n)_{(p)}\simeq{\mathrm{SO}}(2n+1)_{(p)}$ due to Friedlander \cite{F}, the case of ${\mathrm{SO}}(2n+1)_{(p)}$ is the same as $\mathrm{Sp}(n)_{(p)}$.
\end{proof}

We next consider the Samelson products $\langle\epsilon_i,\epsilon_j\rangle$ in $G_{(p)}$ when $G$ is the exceptional Lie group. Some of these Samelson products are calculated in \cite{HK2,KK}, and (non)triviality of all these Samelson products is determined in \cite{HKO} as follows. 

\begin{theorem}
[\cite{HKO}]
\label{HKO}
Let $G$ be a $p$-regular compact connected exceptional simple Lie group, and let $\epsilon_i$ be a generator of $\pi_{2n_i-1}(G_{(p)})\cong{\mathbb{Z}}_{(p)}$ for the type $\{n_1,\ldots,n_\ell\}$ of $G$. Then
$$\langle\epsilon_i,\epsilon_j\rangle\ne 0\quad\text{if and only if}\quad n_i+n_j=n_k+p-1\text{ for some }k.$$
\end{theorem}

Thus the only remaining case is ${\mathrm{SO}}(2n)$. The difficulty of this case is caused by the middle dimensional sphere $S^{2n-1}_{(p)}$ in ${\mathrm{SO}}(2n)_{(p)}$ which vanishes by the inclusion ${\mathrm{SO}}(2n)\to{\mathrm{SO}}(2n+1)$, where there are some partial results on the Samelson products including this sphere as in \cite{Ma,HK1}. The purpose of this paper is to show that a sufficient condition for nontriviality of the Samelson products $\langle\epsilon_i,\epsilon_j\rangle$ in $G_{(p)}$ (Lemma \ref{criterion}) used in \cite{KO,HK1,HK2,HKO} is actually a necessary and sufficient condition, and is to apply it to determination of (non)triviality of all the Samelson products $\langle\epsilon_i,\epsilon_j\rangle$ in ${\mathrm{SO}}(2n)_{(p)}$. To state the result on ${\mathrm{SO}}(2n)_{(p)}$, we choose the maps $\epsilon_i$. Let $\epsilon_i$ be the composite 
$$S^{4i-1}\to{\mathrm{SO}}(2n-1)_{(p)}\xrightarrow{\rm incl}{\mathrm{SO}}(2n)_{(p)}$$ 
for $i=1,\ldots,n-1$, where the first arrow is a generator of $\pi_{4i-1}({\mathrm{SO}}(2n-1)_{(p)})\cong{\mathbb{Z}}_{(p)}$. Let $\theta:S^{2n-1}\to{\mathrm{SO}}(2n)_{(p)}$ be the map corresponding to the adjoint of the fiber inclusion of the canonical homotopy fiber sequence
$$S^{2n}\to B{\mathrm{SO}}(2n)\to B{\mathrm{SO}}(2n+1).$$

\begin{theorem}
\label{main}
Let $\epsilon_i,\theta$ be the above maps into ${\mathrm{SO}}(2n)_{(p)}$ for $p$-regular ${\mathrm{SO}}(2n)$. All nontrivial Samelson products of $\epsilon_i,\theta$ in ${\mathrm{SO}}(2n)_{(p)}$ are
$$\langle\epsilon_i,\epsilon_j\rangle\quad\text{for}\quad 2i+2j>p\quad\text{and}\quad\langle\epsilon_{\frac{p-1}{2}},\theta\rangle=\langle\theta,\epsilon_{\frac{p-1}{2}}\rangle,\;\langle\theta,\theta\rangle\quad\text{for}\quad p=2n-1.$$
\end{theorem}

\section{Proof of Theorem \ref{main}}

Let $G$ be a $p$-regular compact connected Lie group of type $\{n_1,\ldots,n_\ell\}$ throughout this section. We set notation for $G$. Since $G$ is $p$-regular, we have
$$H^*(BG_{(p)};{\mathbb{Z}}/p)={\mathbb{Z}}/p[x_,\ldots,x_\ell],\qquad|x_i|=2n_i.$$
We fix this presentation of the mod $p$ cohomology of $BG_{(p)}$. Note that 
$$H^*(G_{(p)};{\mathbb{Z}}/p)=\Lambda(e_1,\ldots,e_\ell)$$ 
for the suspension $e_i$ of $x_i$. Clearly, there are maps $\epsilon_i:S^{2n_i-1}\to G_{(p)}$ satisfying 
\begin{equation}
\label{epsilon}
(\Sigma\epsilon_i)^*\circ\iota_1^*(x_j)=\begin{cases}\Sigma u_{2n_i-1}&i=j\\0&i\ne j\end{cases}
\end{equation}
for $i=1,\ldots,\ell$, where $\iota_1:\Sigma G_{(p)}\to BG_{(p)}$ is the canonical map and $u_k$ is a generator of $H^k(S^k;{\mathbb{Z}}/p)\cong{\mathbb{Z}}/p$. Then the composite
$$S^{2n_1-1}\times\cdots\times S^{2n_\ell-1}\xrightarrow{\epsilon_1\times\cdots\times\epsilon_\ell}G_{(p)}\times\cdots\times G_{(p)}\to G_{(p)}$$
induces a $p$-local homotopy equivalence where the second map is the multiplication, and we identify $G_{(p)}$ with $S^{2n_1-1}_{(p)}\times\cdots\times S^{2n_\ell-1}_{(p)}$ by this $p$-local homotopy equivalence. The following lemma is first used in \cite{KO} and is the main tool in the proof of Theorem \ref{HKO} given in \cite{HKO}, here we reproduce the proof for completeness of the present paper.

\begin{lemma}
[{\cite[Proof of Theorem 1.1]{KO}}]
\label{criterion}
If $\mathcal{P}^1x_k$ includes the term $cx_ix_j$ ($c\ne 0$), the Samelson product $\langle\epsilon_i,\epsilon_j\rangle$ is nontrivial.
\end{lemma}

\begin{proof}
Suppose $\langle\epsilon_i,\epsilon_j\rangle=0$ under the assumption that $\mathcal{P}^1x_k$ includes the term $cx_ix_j$ ($c\ne 0$). Let $\bar{\epsilon}_m:S^{2n_m}\to BG_{(p)}$ be the adjoint of $\epsilon_m$. Then by \eqref{epsilon}, we have $\bar{\epsilon}_m^*(x_m)=u_{2m}$. By adjointness of Samelson products and Whitehead products, the Whitehead product $[\bar{\epsilon}_i,\bar{\epsilon}_j]$ in $BG_{(p)}$ is trivial, and then there is a map $\mu:S^{2n_i}\times S^{2n_j}\to BG_{(p)}$ satisfying $\mu\vert_{S^{2n_i}\vee S^{2n_j}}=\bar{\epsilon}_i\vee\bar{\epsilon}_j$. So we get $\mu^*(x_i)=u_{2n_i}\otimes 1$ and $\mu^*(x_j)=1\otimes u_{2n_i}$, and hence 
$$0\ne cu_{2n_i}\otimes u_{2n_j}=\mu^*(cx_ix_j)=\mu^*(\mathcal{P}^1x_k)=\mathcal{P}^1\mu^*(x_k)=0$$
where the last equality follows from triviality of $\mathcal{P}^1$ on $H^*(S^{2n_i}\times S^{2n_j};{\mathbb{Z}}/p)$. This is a contradiction.
\end{proof}

We elaborate Lemma \ref{criterion} to prove that its converse is true. Let $P^2G_{(p)}$ be the projective plane of $G_{(p)}$, i.e. there is a cofiber sequence
\begin{equation}
\label{P^2G}
\Sigma G_{(p)}\wedge G_{(p)}\xrightarrow{H}\Sigma G_{(p)}\xrightarrow{\rho_1}P^2G_{(p)}
\end{equation}
where $H$ is the Hopf construction. Put $\bar{x}_i=\iota_2^*(x_i)$ for the natural map $\iota_2:P^2G_{(p)}\to BG_{(p)}$. Since $\iota_1=\iota_2\circ\rho_1$ for the inclusion $\rho_1:\Sigma G_{(p)}\to P^2G_{(p)}$ and $\iota_1^*(x_i)=\Sigma e_i$, we have $\rho_1^*(\bar{x}_i)=\Sigma e_i$. By \cite[Section 3]{L}, we also have $\delta_1^*(\Sigma^2e_i\otimes e_j)=\bar{x}_i\bar{x}_j$ for the connecting map $\delta_1:P^2G_{(p)}\to\Sigma^2G_{(p)}\wedge G_{(p)}$ of the cofiber sequence \eqref{P^2G}. Fix $1\le i,j\le\ell$ and consider the map
$$\Phi=\Sigma\langle\epsilon_i,\epsilon_j\rangle-[\Sigma\epsilon_i,\Sigma\epsilon_j]:\Sigma S^{2n_i-1}\wedge S^{2n_j-1}\to\Sigma G_{(p)}$$
where $[-,-]$ denotes the Whitehead product. The map $\Phi$ is connected with the Hopf construction $H$ through the map constructed by Morisugi \cite[Theorem 5.1]{Mo} such that there is a map $\xi:S^{2n_i-1}\wedge S^{2n_j-1}\to G_{(p)}\wedge G_{(p)}$ satisfying
$$\Phi=H\circ\Sigma\xi\quad\text{and}\quad\xi^*(e_i\otimes e_j)=u_{2n_i-1}\otimes u_{2n_j-1}.$$
Then we get a homotopy commutative diagram of homotopy cofiber sequences
$$\xymatrix{\Sigma G_{(p)}\ar[r]^{\rho_2}\ar@{=}[d]&C_\Phi\ar[r]^(.3){\delta_2}\ar[d]^{\lambda_1}&\Sigma^2 S^{2n_i-1}\wedge S^{2n_j-1}\ar[d]^{\Sigma^2\xi}\\
\Sigma G_{(p)}\ar[r]^{\rho_1}&P^2G_{(p)}\ar[r]^(.45){\delta_1}&\Sigma^2 G_{(p)}\wedge G_{(p)}}$$
implying that
\begin{equation}
\label{lambda1}
\rho_2^*\circ\lambda_1^*(\bar{x}_k)=\Sigma e_k\quad\text{and}\quad\lambda_1^*(\bar{x}_i\bar{x}_j)=\delta_2^*(\Sigma^2u_{2n_i-1}\otimes u_{2n_j-1}).
\end{equation}
We now suppose $\langle\epsilon_i,\epsilon_j\rangle\ne 0$. As in \cite{KK}, for some $1\le k\le\ell$ we have $n_k+p-1=n_i+n_j$ and 
$$\pi_k\circ\langle\epsilon_i,\epsilon_j\rangle=c\alpha_1\quad(0\ne c\in{\mathbb{Z}}/p)$$ 
where $\alpha_1$ is a generator of $\pi_{2n_k+2p-4}(S^{2n_k-1})\cong{\mathbb{Z}}/p$ \cite[Proposition 13.6]{T} and $\pi_k:G_{(p)}=S^{2n_1-1}_{(p)}\times\cdots\times S^{2n_\ell-1}_{(p)}\to S^{2n_k-1}_{(p)}$ is the projection. Then for the map 
$$\widehat{\Phi}=c\Sigma\alpha_1-[\Sigma\pi_k\circ\epsilon_i,\Sigma\pi_k\circ\epsilon_j]:\Sigma S^{2n_i-1}\wedge S^{2n_j-1}\to\Sigma S^{2n_k-1}_{(p)}$$
there is a homotopy commutative diagram of homotopy cofiber sequences
$$\xymatrix{\Sigma G_{(p)}\ar[r]^{\rho_2}\ar[d]^{\Sigma\pi_k}&C_\Phi\ar[r]^(.3){\delta_2}\ar[d]^{\lambda_2}&\Sigma^2 S^{2n_i-1}\wedge S^{2n_j-1}\ar@{=}[d]\\
\Sigma S^{2n_k-1}\ar[r]^(.62){\rho_3}&C_{\widehat{\Phi}}\ar[r]^(.3){\delta_3}&\Sigma^2 S^{2n_i-1}\wedge S^{2n_j-1}.}$$
Since $\alpha_1$ is detected by the Steenrod operation $\mathcal{P}^1$, the mod $p$ cohomology of $C_{\widehat{\Phi}}$ is given by
$$\widetilde{H}^*(C_{\widehat{\Phi}};{\mathbb{Z}}/p)=\langle a_{2n_k},a_{2n_i+2n_j}\rangle,\quad\mathcal{P}^1a_{2n_k}=ca_{2n_i+2n_j}$$
such that $\delta_3^*(\Sigma^2u_{2n_i-1}\otimes u_{2n_j-1})=a_{2n_i+2n_j}$ and $\rho_3^*(a_{2n_k})=\Sigma u_{2n_k-1}$. Then by \eqref{lambda1}, we get $\rho_2^*\circ\lambda_2^*(a_{2n_k})=\rho_2^*\circ\lambda_1^*(\bar{x}_k)$. By the homotopy cofiber sequence $\Sigma G_{(p)}\xrightarrow{\rho_2}C_\Phi\xrightarrow{\delta_2}\Sigma^2S^{2n_i-1}\wedge S^{2n_j-1}$ one can see that the inclusion $\rho_2:\Sigma G_{(p)}\to C_\Phi$ is injective in the mod $p$ cohomology of dimension $2n_k$, and then we obtain $\lambda_2^*(a_{2n_k})=\lambda_1^*(\bar{x}_k)$. By \eqref{lambda1}, we also have $\lambda_2^*(a_{2n_i+2n_j})=\lambda_1^*(\bar{x}_i\bar{x}_j)$. Hence since $\mathcal{P}^1a_{2n_k}=ca_{2n_i+2n_j}$, we get that $\mathcal{P}^1\bar{x}_k$ includes the term $c\bar{x}_i\bar{x}_j$. Thus since $\iota_2^*(x_m)=\bar{x}_m$ for $m=1,\ldots,\ell$, $\mathcal{P}^1x_k$ must include the term $cx_ix_j$. Therefore we have established the following.

\begin{theorem}
\label{P}
The Samelson product $\langle\epsilon_i,\epsilon_j\rangle$ in $G_{(p)}$ is nontrivial if and only if for some $k$, $\mathcal{P}^1x_k$ includes the term $cx_ix_j$ with $c\ne 0$.
\end{theorem}

\begin{proof}
[Proof of Theorem \ref{main}]
Let $p_i,e_n\in H^*(B{\mathrm{SO}}(2n)_{(p)};{\mathbb{Z}}/p)$ be the mod $p$ reduction of the $i$-th universal Pontrjagin class for $i=1,\ldots,n-1$ and the Euler class respectively. Then 
$$H^*(B{\mathrm{SO}}(2n)_{(p)};{\mathbb{Z}}/p)={\mathbb{Z}}/p[p_1,\ldots,p_{n-1},e_n]$$
and the maps $\epsilon_i$ and $\theta$ correspond to $p_i$ and $e_n$ respectively in the sense of \eqref{epsilon}. Since the inclusion ${\mathrm{SO}}(2n-1)_{(p)}\to{\mathrm{SO}}(2n)_{(p)}$ has a left homotopy inverse, it follows from Theorem \ref{classical} that the Samelson product $\langle\epsilon_i,\epsilon_j\rangle$ is nontrivial if and only if $2i+2j>p$. To detect the Samelson products $\langle\epsilon_i,\theta\rangle=\langle\theta,\epsilon_i\rangle$ and $\langle\theta,\theta\rangle$ by Theorem \ref{P}, we compute the quadratic terms of $\mathcal{P}^1p_i$ and $\mathcal{P}^1e_n$ including $e_n$. Recall that for a maximal torus $T$ of ${\mathrm{SO}}(2n)$ and the natural map $\iota:BT_{(p)}\to B{\mathrm{SO}}(2n)_{(p)}$, we have
$$H^*(BT_{(p)};{\mathbb{Z}}/p)={\mathbb{Z}}/p[t_1,\ldots,t_n],\quad|t_i|=2$$
such that $\iota^*(p_i)$ is the $i$-th elementary symmetric polynomial in $t_1^2,\ldots,t_n^2$ and $\iota^*(e_n)=t_1\cdots t_n$. In particular, $\iota$ is injective in the mod $p$ cohomology. Since $\iota^*(\mathcal{P}^1p_i)=\mathcal{P}^1\iota^*(p_i)$ is a symmetric polynomial in $t_1^2,\ldots,t_n^2$ and $p>2n-2$ for the $p$-regularity of ${\mathrm{SO}}(2n)$, if a quadratic term of $\mathcal{P}^1p_i$ includes $e_n$, it must be a multiple of $e_n^2$ and $(i,p)=(1,2n-1)$. Then we get
$$\mathcal{P}^1p_i\equiv\begin{cases}(-1)^{\frac{p-1}{2}}e_n^2&(i,p)=(1,2n-1)\\0&\text{otherwise}\end{cases}\mod(p_1,\ldots,p_{n-1})^2$$
where the first congruence is obtained by applying the Newton formula to $\iota^*(\mathcal{P}^1p_1)=\mathcal{P}^1\iota^*(p_1)=\mathcal{P}^1(t_1^2+\cdots+t_n^2)=2((t_1^2)^{\frac{p+1}{2}}+\cdots+(t_n^2)^{\frac{p+1}{2}})$. (cf. \cite{Ma}) We also have $\iota^*(\mathcal{P}^1e_n)=\mathcal{P}^1\iota^*(e_n)=\mathcal{P}^1(t_1\cdots t_n)=t_1\cdots t_n((t_1^2)^{\frac{p-1}{2}}+\cdots+(t_n^2)^{\frac{p-1}{2}})$. Then by the Newton formula and $p>2n-2$, 
$$\mathcal{P}^1e_n\equiv\begin{cases}(-1)^{\frac{p-1}{2}}\tfrac{p-1}{2}e_np_{\frac{p-1}{2}}&p=2n-1\\0&\text{otherwise}\end{cases}\mod(p_1,\ldots,p_{n-1})^2.$$
(cf. \cite{HK1}) Therefore by Theorem \ref{P} the proof of Theorem \ref{main} is completed.
\end{proof}

\begin{thebibliography}{HKO}
\bibitem[B]{B}R. Bott, {\it A note on the Samelson products in the classical groups}, Comment. Math. Helv. {\bf 34} (1960), 249-256.
\bibitem[F]{F}E.M. Friedlander, {\it Exceptional isogenies and the classifying spaces of simple Lie groups}, Ann. of Math. {\bf	101} (1975), 510-520.
\bibitem[HK1]{HK1}H. Hamanaka and A. Kono, {\it A note on the Samelson products in $\pi_*({\mathrm{SO}}(2n))$ and the group $[{\mathrm{SO}}(2n),{\mathrm{SO}}(2n)]$}, Topology Appl. {\bf 154} (2007), no. 3, 567-572.
\bibitem[HK2]{HK2}H. Hamanaka and A. Kono, {\it A note on Samelson products and mod $p$ cohomology of  classifying spaces of the exceptional Lie groups}, Topology Appl. {\bf 157} (2010), no. 2, 393-400.
\bibitem[HKO]{HKO}S. Hasui, D. Kishimoto, and A. Ohsita, {\it Samelson products in $p$-regular exceptional Lie groups}, available at \url{arXiv:1403.4998}.
\bibitem[KK]{KK}S. Kaji and D. Kishimoto, {\it Homotopy nilpotency in $p$-regular loop spaces}, Math. Z., {\bf 264} (2010), no.1, 209-224.
\bibitem[KO]{KO}A. Kono, H. \={O}shima, {\it Commutativity of the group of self homotopy classes of Lie groups}, Bull. London Math. Soc. {\bf 36} (2004) 37-52.
\bibitem[L]{L}J. Lin,  {\it H-spaces with Finiteness Conditions}, Handbook of Algebraic Topology, North-Hollan Elsevier (1995), pp. 1095-1141 Chapter 22.
\bibitem[Ma]{Ma}M. Mahowald, {\it A Samelson product in ${\mathrm{SO}}(2n)$}, Bul. Soc. Math. Mexicana {\bf 10} (1965) 80-83.
\bibitem[Mo]{Mo}K. Morisugi, {\it Hopf construction, Samelson products and suspension maps}, Contemporary Math. {\bf 239} (1999), 225-238.
\bibitem[T]{T}H. Toda, {\it Composition Methods in Homotopy Groups of Spheres}, Ann. of Math. Studies {\bf 49}, Princeton Univ. Press, Princeton N.J., 1962.
\end{thebibliography}

\end{document}

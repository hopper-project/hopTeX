\documentclass{compositio}
\usepackage{amssymb}
\usepackage{faktor}
\usepackage[all,cmtip]{xy}
\usepackage{amsmath,amscd}
\usepackage{mathrsfs}

\newtheorem*{lemma2}{Proposition}
\newtheorem*{lemma3}{Proposition}
\newtheorem*{lemma4}{Proposition}
\theoremstyle{definition}\newtheorem{definition}{Definition}[section]
\theoremstyle{theorem}\newtheorem{lemma}[definition]{Lemma}
\theoremstyle{remark}\newtheorem*{conventions}{Conventions}
\theoremstyle{remark}\newtheorem*{acknowledgments}{Acknowledgments}
\theoremstyle{remark}\newtheorem*{outline}{Outline}
\theoremstyle{remark}\newtheorem*{questions}{Questions}
\theoremstyle{remark}\newtheorem{example}[definition]{Example}
\theoremstyle{definition}\newtheorem{construction}[definition]{Construction}
\theoremstyle{definition}\newtheorem*{convention}{Convention}
\theoremstyle{definition}\newtheorem*{conjecture}{Conjecture}
\theoremstyle{theorem}\newtheorem{theorem}[definition]{Theorem}
\theoremstyle{theorem}\newtheorem{paradigm}[definition]{Paradigm}
\theoremstyle{remark}\newtheorem{remark}[definition]{Remark}
\theoremstyle{corollary}\newtheorem{corollary}[definition]{Corollary}
\theoremstyle{theorem}\newtheorem{proposition}[definition]{Proposition}
\theoremstyle{definition}\newtheorem{question}[definition]{Question}

\title[Betti numbers and stability via factorization homology]{Betti numbers and stability for configuration spaces via factorization homology}
\author{Ben Knudsen}
\email{knudsen@math.northwestern.edu}
\address{Department of Mathematics\\Northwestern University\\Evanston, IL 60208}
\date{} 

\begin{document}

\classification{57R19 (primary), 18G55 (secondary).}
\keywords{Configuration spaces, stability, factorization homology, Lie algebras, $E_n$-algebras.}

\begin{abstract}
Using factorization homology, we realize the rational homology of the unordered configuration spaces of an arbitrary manifold $M$, possibly with boundary, as the homology of a Lie algebra constructed from the compactly supported cohomology of $M$. By locating the homology of each configuration space within the Chevalley-Eilenberg complex of this Lie algebra, we extend theorems of B\"{o}digheimer-Cohen-Taylor and F\'{e}lix-Thomas and give elementary proofs of homological stability results of Church and Randal-Williams. Our method lends itself to explicit calculations, examples of which we include.
\end{abstract}

\maketitle

\section{Introduction}

We study the configuration space $B_k(M)$ of $k$ unordered points in a manifold $M$, defined as $$B_k(M)={\mathrm{Conf}}_k(M)_{\Sigma_k}:=\faktor{\big\{(x_1,\ldots,x_k)\in M^k:x_i\neq x_j \text{ for } i\neq j\big\}}{\Sigma_k},$$ where $\Sigma_k$ acts by permuting the $x_i$. If $M$ has boundary, we consider the quotient of $B_k(M)$ by the subspace of configurations with at least one point lying in $\partial M$, denoted $B_k(M,\partial M)$. This definition reproduces the first when $\partial M=\varnothing$, since in this case $B_k(M,\varnothing)=B_k(M)_+$. 

Our main theorem concerns the homology of these spaces.

\begin{theorem}\label{with grading} Let $M$ be an $n$-manifold and $\mathbb{F}$ a field of characteristic zero. \begin{enumerate}
\item There is an isomorphism of bigraded vector spaces $$\bigoplus_{k\geq0}\widetilde{H}_*(B_k(M,\partial M);\mathbb{F})\cong H_*^{\mathrm{Lie}}(H_{c}^{-*}(M;{\mathrm{Lie}}(\mathbb{F}^w[n-1]))).$$
\item If $n$ is even, there is an isomorphism of bigraded vector spaces $$\bigoplus_{k\geq0}\widetilde{H}_*(B_k(M,\partial M);\mathbb{F}^w)\cong H_*^{\mathrm{Lie}}(H_{c}^{-*}(M;{\mathrm{Lie}}(\mathbb{F}[n-1]))).$$
\item If $n$ is odd, there is an isomorphism of bigraded vector spaces $$\bigoplus_{k\geq0}\widetilde{H}_*(B_k(M,\partial M);\mathbb{F}^w)[k]\cong H_*^{\mathrm{Lie}}(H_{c}^{-*}(M;{\mathrm{Lie}}(\mathbb{F}[n]))).$$
\end{enumerate}
Here, $\mathbb{F}^w$ is the orientation sheaf of $M$, $H_*^{\mathrm{Lie}}$ denotes Lie algebra homology, ${\mathrm{Lie}}$ is the free graded Lie algebra functor, and in each case the left is graded by cardinality and the right by powers of the Lie generator.
\end{theorem}

For parallelizable manifolds, the first assertion is a result of \cite{Francis}. To the author's knowledge, the theorem is new in all other cases.

The study of configuration spaces is classical. To name some highlights, the space $B_k(\mathbb{R}^2)$ is a classifying space for the braid group on $k$ strands (see \cite{Artin}); the space ${\mathrm{Conf}}_k(\mathbb{R}^n)$ has the homotopy type of the space of $k$-ary operations of the little $n$-disks operad and so plays a central role in the theory of $n$-fold loop spaces (see e.g. \cite{CLM}, \cite{May}, \cite{Segal}); certain spaces of ``labeled'' configurations provide models for more general types of mapping spaces (see \cite{Bodigheimer}, \cite{McDuff}, \cite{Salvatore}, \cite{Segal}); and, according to a striking theorem of \cite{LS}, the homotopy type of $B_k(M)$ is not an invariant of the homotopy type of $M$. 

Running counter to this last fact is a string of results asserting that the homology of configuration spaces is surprisingly simple, provided one is willing to work over a field of characteristic zero. In \cite{BCT}, B\"{o}digheimer-Cohen-Taylor showed that the Betti numbers of $B_k(M)$ are determined by those of $M$ when $M$ is orientable and of odd dimension, and, in \cite{FT}, F\'{e}lix-Thomas showed that, in the even-dimensional case, the Betti numbers of $B_k(M)$ are determined by the rational cohomology ring of $M$, as long as $M$ is closed, orientable, and nilpotent. We recover extensions of these results as immediate consequences of Theorem \ref{with grading}.

\begin{corollary}\label{main corollary} Let $n$ be fixed. \begin{enumerate}\item The groups $\widetilde{H}_*(B_k(M,\partial M);\mathbb{F})$ depend only on
\begin{itemize}
\item the graded abelian group $H_c^{-*}(M;\mathbb{F}^w)$ if $n$ is odd; or
\item the cup product $H_{c}^{-*}(M;\mathbb{F}^w)^{\otimes 2}\to H_c^{-*}(M;\mathbb{F})$ if $n$ is even.
\end{itemize}
\item The groups $\widetilde{H}_*(B_k(M,\partial M);\mathbb{F}^w)$ depend only on the algebra $H_c^{-*}(M;\mathbb{F})$.
\end{enumerate}
\end{corollary}

The computational power of Theorem \ref{with grading} lies in the bigrading, which permits one to isolate the homology of a single configuration space within the Chevalley-Eilenberg complex computing the appropriate Lie homology. Employing this strategy, we show that the chain complexes computing $H_*(B_k(M);\mathbb{F})$ exhibited in \cite{BCT} and \cite{FT} are isomorphic to subcomplexes of the Chevalley-Eilenberg complex; precise statements appear in \S\ref{formulas}. Better yet, in dealing with the entire Chevalley-Eilenberg complex at once, one is able to perform computations for all $k$ simultaneously (see \S\ref{examples}).

Parallel to the program of determining the homology of $B_k(M)$ is that of understanding its limiting behavior as $k$ tends to infinity, the program of \emph{homological stability} (see \cite{Church}, \cite{CEF}, \cite{ORWstab}, and the references therein). The general theme is that the Betti numbers of $B_k(M)$ stabilize with $k$, despite the absence of a map of spaces $B_k(M)\to B_{k+1}(M)$ in general, and despite the failure of stabilization over the integers. Again, characteristic zero is special. 

Regarding stability, we prove the following:

\begin{theorem}\label{stability theorem}
Choose $n>1$ and let $M$ be a connected $n$-manifold without boundary. There is a map $\varphi_k:H_*(B_{k+1}(M);\mathbb{F})\to H_*(B_k(M);\mathbb{F})$ that is an isomorphism \begin{itemize}
\item for $*<k$ when $M$ is orientable and $n=2$; and
\item for $*\leq k$ in all other cases.
\end{itemize}
\end{theorem}

This result improves on the stable range asserted in \cite{Church} and very slightly on that of \cite{ORWstab}. As in the former work, our stable range can be further improved if the low degree Betti numbers of $M$ vanish. As the example of the Klein bottle shows, the bound $*\leq k$ is sharp in the sense that no better stable range holds for all manifolds that are not orientable surfaces.

Conceptually, we think of Theorem \ref{with grading} as providing, in the form of the Chevalley-Eilenberg complex, an explanation and organizing principle for the behavior of configuration spaces in characteristic zero. The germ of our approach, and the source of the connection to Lie algebras, is the calculation, due to Arnold and Cohen, of the homology of the configuration spaces of $\mathbb{R}^n$, which is the fundamental result of the subject (see \cite{Arnold}, \cite{CLM}). Factorization homology, our primary tool in this work, is used here as a means of globalizing that local calculation. 

At a more formal level, we rely on the fact that the factorization homology of $M$, with coefficients taken in a certain free algebra, can be computed in two different ways. On the one hand, according to Proposition \ref{free}, it has an expression in terms of the configuration spaces of $M$. On the other hand, the free algebra may be thought of as a kind of enveloping algebra, and a calculation of \cite{Francis} then identifies the same invariant as Lie algebra homology. On the face of it, these calculations only coincide for framed manifolds; we show that they agree in general in characteristic zero.

The paper following the introduction is split into three sections. In \S2, we review the basic ideas of factorization homology and make the connection with Lie algebras. The results advertised above are proven in \S3, and \S4 is concerned with explicit examples. In keeping with our metamathematical goal of making the case for factorization homology as a computational tool, we do not focus on the technical underpinnings of the theory. The reader may find these in \cite{AF}, \cite{AFT}, \cite{FrancisTangent}, \cite{Francis}, and \cite{Lurie}, for example.

\begin{questions}
Our work invites a number of questions.
\begin{enumerate}
\item According to \S\ref{formulas}, the dual of the Chevalley-Eilenberg complex for our Lie algebra coincides with the direct sum over $k$ of the $\Sigma_k$-invariants in the Cohen-Taylor spectral sequence  computing the cohomology of ${\mathrm{Conf}}_k(M)$, where $M$ is orientable (see \cite{CohenTaylor}, \cite{Totaro}). Is there a corresponding interpretation of the full spectral sequence that might shed light on the rational homology of ${\mathrm{Conf}}_k(M)$? 
\item Our results form a comprehensive additive picture of unordered configuration spaces rationally. Can any of the multiplicative structure be seen in the Chevalley-Eilenberg complex?
\item We show that homological stability is induced by taking a cap product with a ``fundamental class'' in Lie algebra cohomology (see \S\ref{stability}). This fact, considered in the framework of \cite{AF}, suggests that, in this case at least, homological stability may be thought of as a kind of Poincar\'{e} duality phenomenon. Is there more to this connection?
\item It seems likely to the author that our stability map coincides with the transfer map employed in \cite{Church}. Is this the case, and, if so, is there an explanation for this coincidence?
\end{enumerate}
\end{questions}

\begin{conventions}\begin{enumerate}
\item In accordance with the literature on factorization homology, we work in an $\infty$-categorical context, where for us an $\infty$-category will always mean a quasicategory. The standard reference here is \cite{HTT}. We will need to ask very little of the theory of quasicategories, and the reader is safe ignoring this point and substituting ``homotopy colimit'' for ``colimit'' everywhere, for example.
\item Every manifold is smooth and may be embedded as the interior of a compact manifold with boundary (such an embedding is not part of the data). We view manifolds as objects of the $\infty$-category ${\mathrm{Mfld}}_n$, the coherent nerve of the topological category of $n$-manifolds and smooth embeddings, which is symmetric monoidal under disjoint union.
\item Our homology theories are valued in the $\infty$-category ${\mathrm{Ch}}_\mathbb{F}$, the coherent nerve of the simplicial category of chain complexes over $\mathbb{F}$, a field of characteristic zero. With the single exception of Theorem \ref{Eilenberg-Steenrod}, ${\mathrm{Ch}}_\mathbb{F}$ is understood to be symmetric monoidal under tensor product. 
\item All chain complexes are homologically graded. If $V$ is a chain complex, we define $V[k]$ by $(V[k])_n=V_{n-k},$ and, for $x\in V$, the corresponding element in $V[k]$ is denoted $\sigma^kx$. Cohomology is concentrated in negative degrees; to reinforce this point, we write $H^{-*}(X)$ for the graded vector space whose degree $-k$ part is the $k$th cohomology group of $X$.
\item If $X$ is a space and $V$ is a chain complex, the \emph{tensor of $X$ and $V$} is the chain complex $$X\otimes V:=C_*(X)\otimes V.$$
\end{enumerate}
\end{conventions}

\begin{acknowledgments}
The author is grateful to John Francis for suggesting this project, and for his help and patience. He would also like to thank Lee Cohn, Boris Hanin, Sander Kupers, Jeremy Miller, Martin Palmer, and Dylan Wilson for their comments on earlier versions of this paper; Eric Wofsey and Joel Specter for suggested examples; and Aaron Mazel-Gee and Jordan Ellenberg for enlightening conversations that led to significant improvements of the paper.
\end{acknowledgments}

\section{Factorization Homology}

\subsection{Homology theories}\label{factorization homology} In this section, we review the basic notions of factorization homology. The primary reference is \cite{Francis}. As there, our point of view is that factorization homology is a natural theory of homology for manifolds. To illustrate in what sense this is so, we first recall the classical characterization of ordinary homology, phrased in a way that invites generalization.

\begin{theorem}[(Eilenberg-Steenrod)]\label{Eilenberg-Steenrod}
Let $V$ be a chain complex. There is a symmetric monoidal functor $C_*(-; V)$ from spaces with disjoint union to chain complexes with direct sum, called \emph{singular homology with coefficients in $V$}, which is characterized up to natural quasi-isomorphism by the following properties:
\begin{enumerate}
\item $C_*(\mathrm{pt}; V)\simeq V$;
\item the natural map $$C_*(X_1;V)\bigoplus_{C_*(X_0;V)}C_*(X_2;V)\to C_*\bigg(X_1\coprod_{X_0} X_2;V\bigg)$$ is a quasi-isomorphism for every diagram of cofibrations $X_1\leftarrow X_0\rightarrow X_2$.
\end{enumerate}
\end{theorem}

Property (ii), a local-to-global principle equivalent to the usual excision axiom, is the reason that homology is computable and hence useful. 

Of course, ordinary homology is homotopy invariant. In the study of manifolds, the equivalence relation of interest is often finer than homotopy equivalence, and one could hope for a theory better suited to such geometric investigations. To discover what form this theory might take, let us contemplate a generic symmetric monoidal functor $F:({\mathrm{Mfld}}_n,\amalg)\to ({\mathrm{Ch}}_\mathbb{F},\otimes)$. By analogy with Theorem \ref{Eilenberg-Steenrod}, we ask that $F$ be determined by its value on $\mathbb{R}^n$, the basic building block in the construction of $n$-manifolds. Unlike a point, Euclidean space has interesting internal structure.

\begin{definition}
An \emph{$n$-disk algebra} is a symmetric monoidal functor $A:{\mathrm{Disk}}_n\to {\mathrm{Ch}}_\mathbb{F}$, where ${\mathrm{Disk}}_n\subset {\mathrm{Mfld}}_n$ is the full subcategory spanned by manifolds diffeomorphic to $\amalg_k\mathbb{R}^n$ for some $k\in\mathbb{Z}_{\geq0}$.\footnote{Following \cite{Francis}, we avoid the ``framed $E_n$-algebra'' moniker, for fear of the inevitable confusion arising from the fact that ``framed'' algebras do not correspond to ``framed'' embeddings.}
\end{definition}

In other words, ${\mathrm{Disk}}_n$ is the category of operations associated to the endomorphism operad of $\mathbb{R}^n$ in ${\mathrm{Mfld}}_n$, and an $n$-disk algebra is an algebra over this operad. In contrast, the endomorphism operad of a point in topological spaces is the commutative operad, and every chain complex is canonically and essentially uniquely a commutative algebra in $({\mathrm{Ch}}_\mathbb{F},\oplus)$.

Taking the extra structure of $\mathbb{R}^n$ into account, there is an analogous classification theorem:

\begin{theorem}[(Francis)]
Let $A$ be an $n$-disk algebra. There is a symmetric monoidal functor $\int_{(-)}A:{\mathrm{Mfld}}_n\to {\mathrm{Ch}}_\mathbb{F}$, called \emph{factorization homology with coefficients in $A$}, which is characterized up to natural quasi-isomorphism by the following properties:
\begin{enumerate}
\item $\int_{\mathbb{R}^n}A\simeq A$ as $n$-disk algebras;
\item the natural map $$\int_{M_1}A\bigotimes^\mathbb{L}_{\int_{M_0\times\mathbb{R}}A}\int_{M_2}A\to \int_MA$$ is a quasi-isomorphism for every decomposition $M\cong M_1\amalg_{M_0\times\mathbb{R}}M_2$ with $M_1$ and $M_2$ embedded $n$-manifolds and $M_0$ an embedded $n-1$-manifold.
\end{enumerate}
\end{theorem}

\subsection{Construction}
Just as the functor of singular chains is but one model for ordinary homology, factorization homology may be constructed in several equivalent ways. The construction that we will favor is as follows.

Let $A:{\mathrm{Disk}}_n\to {\mathrm{Ch}}_\mathbb{F}$ be an $n$-disk algebra. Then factorization homology with coefficients in $A$ is the (homotopy) left Kan extension in the following diagram:
$$\xymatrix{
{\mathrm{Disk}}_n\ar[r]^A\ar@{^{(}->}[d] &{\mathrm{Ch}}_\mathbb{F}\\
{\mathrm{Mfld}}_n \ar@{-->}[ur]_{\int_{(-)}A}
}$$ Explicitly, it may be calculated as the (homotopy) colimit $$\int_MA\simeq \operatorname*{\mathrm{colim}}_{{\mathrm{Disk}}_{n_{/M}}}A.$$

\begin{remark}\label{spectral sequence remark}
From this description, it follows that there is a spectral sequence $$H\bigg(\int_MH(A)\bigg)\implies H\bigg(\int_MA\bigg)$$ (see Proposition \ref{spectral sequence}). The results of this paper can also be obtained by understanding the differentials in this spectral sequence. We intend to revisit this subject in future work.
\end{remark}

\subsection{Adjunctions} We introduce several functors that will be important for us in what follows. The reference here is \cite{AF}.

Within the $\infty$-category ${\mathrm{Disk}}_n$ there is a Kan complex with a single vertex, the object $\mathbb{R}^n$, whose endomorphisms are ${\mathrm{Emb}}(\mathbb{R}^n,\mathbb{R}^n)\simeq O(n)$, so that we may identify this Kan complex with $BO(n)$. Restricting to this subcategory defines a forgetful functor $${\mathrm{Disk}}_n\text{-}\mathrm{alg}\to \mathrm{Fun}(BO(n),{\mathrm{Ch}}_\mathbb{F}).$$ The latter is equivalent to the $\infty$-category of chain complexes equipped with an action of $C_*(O(n))$, which we refer to simply as $O(n)$-\emph{modules}. This functor admits a left adjoint ${\mathrm{Disk}}_n(-)$, the free $n$-disk algebra generated by an $O(n)$-module. 

\begin{remark}
Any $n$-disk algebra is an $E_n$-algebra, and there is an equivalence of $E_n$-algebras $${\mathrm{Disk}}_n(\mathrm{triv}(V))\simeq E_n(V),$$ where $\mathrm{triv}(V)$ is the trivial $O(n)$-module on the chain complex $V$, and $E_n(V)$ is the free $E_n$-algebra generated by $V$.
\end{remark}

Evaluation on $\mathbb{R}^n$ defines a still more forgetful functor, which we think of as associating to an algebra its underlying chain complex. The situation is summarized in the following commuting diagram of adjunctions, in which the straight arrows are forgeful functors:
$$\xymatrix{
{\mathrm{Disk}}_n\text{-}\mathrm{alg}\ar[rr]\ar[ddrr]&&O(n)\text{-}\mathrm{mod}\ar@/_1pc/[ll]_{{\mathrm{Disk}}_n(-)}\ar[dd]\\\\
&&{\mathrm{Ch}}_\mathbb{F}\ar@/_1pc/[uu]_{O(n)\otimes(-)}\ar@/^1pc/[uull]
}$$

In particular, for a chain complex $V$, the free $n$-disk algebra on $V$ is naturally equivalent to ${\mathrm{Disk}}_n(O(n)\otimes V).$ More generally, there is the following description.

\begin{proposition}
There is a natural equivalence $${\mathrm{Disk}}_n(K)\simeq\bigoplus_{k\geq0}{\mathrm{Emb}}(\amalg_k\mathbb{R}^n,-)\otimes_{\Sigma_k\ltimes O(n)^k}K^{\otimes k}$$ of functors from ${\mathrm{Disk}}_n$ to ${\mathrm{Ch}}_\mathbb{F}$.
\end{proposition}
\begin{proof}
In the case $K=O(n)\otimes V$, ${\mathrm{Disk}}_n(K)$ is equivalent to the free $n$-disk algebra on the chain complex $V$, so that \begin{align*}{\mathrm{Disk}}_n(K)&\simeq\bigoplus_{k\geq0}{\mathrm{Emb}}(\amalg_k\mathbb{R}^n,-)\otimes_{\Sigma_k}V^{\otimes k}\\
&\cong\bigoplus_{k\geq0}{\mathrm{Emb}}(\amalg_k\mathbb{R}^n,-)\otimes_{\Sigma_k\ltimes O(n)^k}(O(n)^{\otimes k}\otimes V^{\otimes k})\\
&\cong\bigoplus_{k\geq0}{\mathrm{Emb}}(\amalg_k\mathbb{R}^n,-)\otimes_{\Sigma_k\ltimes O(n)^k}K^{\otimes k}.
\end{align*} Since a general $K$ may be expressed as a split geometric realization of free $O(n)$-modules, and since ${\mathrm{Disk}}_n(-)$, as a left adjoint, preserves geometric realizations, it suffices to show that the right hand side shares this property. But both $O(n)\text{-}\mathrm{mod}$ and ${\mathrm{Disk}}_n\text{-}\mathrm{alg}$ are monadic over ${\mathrm{Ch}}_\mathbb{F}$, so on both sides the geometric realization is computed in ${\mathrm{Ch}}_\mathbb{F}$, and the right hand side clearly preserves colimits in chain complexes.
\end{proof}

\subsection{Frame bundles}\label{free algebras}

The object of this section is to compute the factorization homology of the free $n$-disk algebra generated by an $O(n)$-module. This calculation can be deduced from a more general result of \cite{AF}, but a fairly simple and direct proof suffices for our purposes.

For a manifold $M$, let ${\mathrm{Fr}}(M)\to M$ denote the corresponding principal $O(n)$-bundle. Since ${\mathrm{Conf}}_k(M)$ is an open submanifold of $M^k$, its structure group is canonically reducible to $O(n)^k$, and we denote the corresponding principal $O(n)^k$-bundle by ${\mathrm{Fr}}_k({\mathrm{Conf}}_k(M))$.

\begin{proposition}\label{free}
There is a natural equivalence $$\int_M{\mathrm{Disk}}_n(K)\simeq \bigoplus_{k\geq0}{\mathrm{Fr}}_{k}({\mathrm{Conf}}_k(M))\otimes_{\Sigma_k\ltimes O(n)^k}K^{\otimes k},$$ where $K$ is an $O(n)$-module.
\end{proposition}

\begin{proof}
The natural map $$\operatorname*{\mathrm{colim}}_{{\mathrm{Disk}}_{n_{/M}}}{\mathrm{Emb}}(\amalg_k\mathbb{R}^n,-)\to{\mathrm{Emb}}(\amalg_k\mathbb{R}^n,M)$$ is an equivalence (see \cite{Lurie}, for example), so that we have \begin{align*}\int_M{\mathrm{Disk}}_n(M)&\simeq\operatorname*{\mathrm{colim}}_{{\mathrm{Disk}}_{n_{/M}}}\bigg(\bigoplus_{k\geq0}{\mathrm{Emb}}(\amalg_k\mathbb{R}^n,-)\otimes_{\Sigma_k\ltimes O(n)^k}K^{\otimes k}\bigg)\\
&\simeq \bigoplus_{k\geq0}\bigg(\operatorname*{\mathrm{colim}}_{{\mathrm{Disk}}_{n_{/M}}}{\mathrm{Emb}}(\amalg_k\mathbb{R}^n,-)\bigg)\otimes_{\Sigma_k\ltimes O(n)^k}K^{\otimes k}\\
&\simeq\bigoplus_{k\geq0}{\mathrm{Emb}}(\amalg_k\mathbb{R}^n,M)\otimes_{\Sigma_k\ltimes O(n)^k}K^{\otimes k}.
\end{align*} 

\noindent To conclude, we note that evaluation at the origin defines a projection ${\mathrm{Emb}}(\amalg_k\mathbb{R}^n,M)\to {\mathrm{Conf}}_k(M),$ and the natural map ${\mathrm{Emb}}(\amalg_k\mathbb{R}^n,M)\to{\mathrm{Fr}}_k{\mathrm{Conf}}_k(M)$ covering the identity is an equivalence of $O(n)^k$-spaces over ${\mathrm{Conf}}_k(M)$, with inverse equivalence supplied by a choice of metric, for example.
\end{proof}

\begin{corollary}\label{homology}
There is a natural equivalence $$\int_M{\mathrm{Disk}}_n(\mathbb{F})\simeq \bigoplus_{k\geq0}C_*(B_k(M);\mathbb{F}).$$
\end{corollary}

\subsection{Orientations}\label{covers}

Proposition \ref{free} may also be used to study the twisted homology of $B_k(M)$. To pursue this direction, we must first identify the orientation cover $\widetilde{B_k(M)}$ of $B_k(M)$. To this end, notice that the orientation cover $$\widetilde{{\mathrm{Conf}}_k(M)}\to {\mathrm{Conf}}_k(M)\to B_k(M)$$ has structure group $\Sigma_k\times C_2$ when considered as a bundle over $B_k(M)$, the automorphism corresponding to $-1\in C_2$ reverses orientation, and the automorphism corresponding to $\tau\in \Sigma_k$ reverses orientation if ${\mathrm{sgn}}(\tau)=-1$ and $n$ is odd and preserves orientation otherwise. Therefore, the action of the subgroup $$H:=\{(\tau,{\mathrm{sgn}}^n(\tau))\mid\tau\in \Sigma_k\}\subset \Sigma_k\times C_2$$ is orientation preserving, and we deduce the following:

\begin{proposition} $\widetilde{B_k(M)}\cong \widetilde{{\mathrm{Conf}}_k(M)}_H$ as covers of $B_k(M)$.
\end{proposition}

For a chain complex $V$, let $V^\sigma$ denote the sign representation of $C_2$ on $V$ and $V^{\sigma_n}$ the $O(n)$-module obtained from the latter by restriction along the determinant. Recall that, for an $n$-manifold $N$, the homology of $N$ twisted by the orientation character may be computed as the homology of the complex $$C_*(N;\mathbb{F}^w):=\widetilde N\otimes_{C_2}\mathbb{F}^\sigma\cong {\mathrm{Fr}}(N)\otimes_{O(n)}\mathbb{F}^{\sigma_n}.$$

\begin{proposition}\label{twisted homology} Let $M$ be an $n$-manifold.
\begin{enumerate}
\item If $n$ is even, there is a natural equivalence $$\int_M{\mathrm{Disk}}_n(\mathbb{F}^{\sigma_n})\simeq \bigoplus_{k\geq0}C_*(B_k(M);\mathbb{F}^w).$$
\item If $n$ is odd, there is a natural equivalence $$\int_M{\mathrm{Disk}}_n(\mathbb{F}^{\sigma_n}[1])\simeq\bigoplus_{k\geq0} C_*(B_k(M); \mathbb{F}^w)[k].$$
\end{enumerate}
\end{proposition}
\begin{proof} \begin{enumerate}
\item
We have that \begin{align*}
{\mathrm{Fr}}_{k}({\mathrm{Conf}}_k(M))\otimes_{\Sigma_k\ltimes O(n)^k}(\mathbb{F}^{\sigma_{n}})^{\otimes k}&\cong {\mathrm{Fr}}({\mathrm{Conf}}_k(M))\otimes_{\Sigma_k\ltimes O(nk)}\mathbb{F}^{\sigma_{nk}}
\\
&\cong\widetilde{{\mathrm{Conf}}_k(M)}\otimes_{\Sigma_k\times C_2}\mathbb{F}^\sigma\\
&\cong\widetilde{{\mathrm{Conf}}_k(M)}_{\Sigma_k}\otimes_{C_2}\mathbb{F}^\sigma\\ 
&\cong\widetilde{B_k(M)}\otimes_{C_2}\mathbb{F}^\sigma, 
\end{align*} where we have used the commuting of the diagram $$\xymatrix{
O(n)^k\ar[r]\ar[d]_{\det^k} &O(nk)\ar[d]^{\det}\\
C_2^k\ar[r]^{\text{multiply}} &C_2.
}$$ and the fact that $H=\Sigma_k\times \{1\}$ when $n$ is even. The claim follows after summing over $k$ and applying Proposition \ref{free}.

\item Similarly, we have that \begin{align*}
{\mathrm{Fr}}_{k}({\mathrm{Conf}}_k(M))\otimes_{\Sigma_k\ltimes O(n)^k}(\mathbb{F}^{\sigma_{n}}[1])^{\otimes k}&\cong {\mathrm{Fr}}({\mathrm{Conf}}_k(M))\otimes_{\Sigma_k\ltimes O(nk)}(\mathbb{F}^{\sigma_{nk}}\otimes\mathbb{F}[1]^{\otimes k})
\\
&\cong\widetilde{{\mathrm{Conf}}_k(M)}\otimes_{\Sigma_k\times C_2}(\mathbb{F}^\sigma\otimes\mathbb{F}[1]^{\otimes k})\\
&\cong\widetilde{{\mathrm{Conf}}_k(M)}_{H}\otimes_{C_2}\mathbb{F}^\sigma[k]\\ 
&\cong\widetilde{B_k(M)}\otimes_{C_2}\mathbb{F}^\sigma[k], 
\end{align*} where we have used that $\mathbb{F}^\sigma\otimes\mathbb{F}[1]^{\otimes k}$ is a trivial $H$-module and that $[\Sigma_k\times C_2:H]=2$.
\end{enumerate}
\end{proof}

\subsection{Boundaries and augmentations}\label{boundaries}

We pause to extend the vocabulary developed in \S\ref{factorization homology} to include manifolds with boundary. In order not to become encumbered by details not relevant to our present purpose, we will content ourselves with a brief outline. The reader interested in the details should consult \cite{AF}, or \cite{AFT}, in which manifolds with much more general types of singularities are considered.

The set-up is identical. Our domain is the symmetric monoidal $\infty$-category ${\mathrm{Mfld}}_n^\partial$ obtained from the topological category with objects $n$-manifolds with boundary and morphisms embeddings respecting boundaries. There is a full subcategory ${\mathrm{Disk}}_n^\partial$, whose objects are disjoint unions of copies of $\mathbb{R}^n$ and $\mathbb{H}^n:=\mathbb{R}_{\geq0}\times\mathbb{R}^{n-1}.$ The category of symmetric monoidal functors $$A:({\mathrm{Disk}}_n^\partial,\amalg) \to({\mathrm{Ch}}_\mathbb{F},\otimes)$$ is denoted ${\mathrm{Disk}}_n^\partial\text{-}\mathrm{alg}$, and factorization homology is defined in the same way and satisfies an analogous form of excision. An object of ${\mathrm{Disk}}_n^\partial\text{-}\mathrm{alg}$ is roughly the data of an $n$-disk algebra $A$, corresponding to $\mathbb{R}^n$, and an $(n-1)$-disk algebra $B$ \emph{over} $A$, corresponding to $\mathbb{R}^n\to\mathbb{H}^n$.\footnote{The category ${\mathrm{Disk}}_n^\partial$ is the category of operations of a close relative of the ``Swiss-cheese'' operad considered in \cite{Voronov}. The Swiss-cheese operad is obtained by considering manifolds with boundary under \emph{framed} embeddings.} We will not try to make this idea precise, since we only need the following degenerate special case.

\begin{definition}
An \emph{augmented $n$-disk algebra} is a pair $(A,\varepsilon)$ with $A$ an $n$-disk algebra and $\varepsilon:A\to \mathbb{F}$ a map of $n$-disk algebras. 
\end{definition}

An augmented $n$-disk algebra defines an object of ${\mathrm{Disk}}_n^\partial\text{-}\mathrm{alg}$. When we mention such algebras in the context of a manifold with boundary, we leave the augmentation implicit. 

\begin{remark}
For any $O(n)$-module $K$, ${\mathrm{Disk}}_n(K)$ is canonically augmented via the unique map $K\to 0$. 
\end{remark}

Our calculations of factorization homology carry over into this new context. More precisely, Corollary \ref{homology} holds for manifolds with boundary with $C_*(B_k(M);\mathbb{F})$ replaced by $\widetilde{C}_*(B_k(M,\partial M);\mathbb{F})$, and the same is true of Proposition \ref{twisted homology} if we define $\widetilde{C}_*(B_k(M,\partial M);\mathbb{F}^w)$ as \emph{relative} homology with local coefficients (recall that $B_k(M,\partial M)$ is a quotient of $B_k(M)$). These expanded statements follow in the same way from the correct version of Proposition \ref{free} for manifolds with boundary, for which the reader is referred to \cite{AF}.

\subsection{Enveloping algebras}\label{Lie algebras}

It has long been known that configuration spaces are intimately related to Lie algebras (see \cite{Cohen} and \cite{CohenTaylor}, for example). For us, the connection will be made through factorization homology. This section presents a somewhat informal outline of the place of Lie algebras in the theory; further details are available in \cite{Francis}.

Let $A$ be an $E_n$-algebra. Then, since $E_n(2)\simeq {\mathrm{Conf}}_2(\mathbb{R}^n)\simeq S^{n-1}$, there is a map $$m: S^{n-1}\otimes A^{\otimes 2}\to A.$$ The homology of $S^{n-1}$ is concentrated in degrees 0 and $n-1$, so that, up to homotopy, there are two maps $$m_0:A^{\otimes2}\to A\qquad\text{and}\qquad m_{n-1}: A^{\otimes 2}\to A[1-n]$$ defining a commutative multiplication on $A$ and a Lie bracket on $A[n-1]$, again up to homotopy. The Jacobi identity for $m_{n-1}$ follows from the so-called Yang-Baxter relations in $H_*({\mathrm{Conf}}_2(\mathbb{R}^n))$ (see \cite{FH}), and $O(n)$, acting on $S^{n-1}$ by degree plus and minus one maps, interchanges it with the opposite bracket.

The fact that this discussion illustrates is the existence of a forgetful functor producing a Lie algebra from an  $E_n$-algebra.\footnote{We are in characteristic zero and so need not be troubled by the distinction between algebras for an operad and algebras for the cofibrant replacement of that operad; see \cite{MayOperads}, for example.} This forgetful functor admits a left adjoint $U_n$, the $n$-enveloping algebra functor, which, when $n=1$, reproduces the ordinary universal enveloping algebra. Note that $U_n(\mathfrak{g})$ is canonically augmented via the unique Lie algebra map $\mathfrak{g}\to0$, so that it is sensible to consider the factorization homology of a manifold with boundary with coefficients in $U_n(\mathfrak{g})$.

The following calculation appears in \cite{Francis}; see also \cite{FG} and \cite{Gwilliam}.

\begin{theorem}[(Francis)]\label{enveloping calculation}
There is a natural equivalence $$\int_M U_n(\mathfrak{g})\simeq C_*^{\mathrm{Lie}}(C_c^{-*}(M;\mathfrak{g})),$$ where $M$ is an $n$-manifold, possibly with boundary, and $C_*^{\mathrm{Lie}}$ is the functor of Lie chains. In particular, $$U_n(\mathfrak{g})\simeq C_*^{\mathrm{Lie}}( C_c^{-*}(\mathbb{R}^n;\mathfrak{g})).$$
\end{theorem}

\begin{remark}
Since $C_c^{-*}(\mathbb{R}^n)$ is an $n$-disk algebra under direct sum, and since the functor of Lie chains is symmetric monoidal, the second assertion has the important consequence that $U_n(\mathfrak{g})$, \emph{a priori} merely an $E_n$-algebra, naturally obtains the structure of an $n$-disk algebra. Consequently, the expression $\int_M U_n(\mathfrak{g})$ is sensible not only when $M$ is framed, but for arbitrary $M$, and the first assertion holds in this generality.
\end{remark}

\begin{remark}
One way to think about this theorem is the following. By excision, the factorization homology of the closed $n$-disk is equivalent to the $n$-fold bar construction, which implements (derived) Koszul duality for $E_n$-algebras (see \cite{Francis} for a discussion of this phenomenon). Since $C_*^{\mathrm{Lie}}$ implements duality for Lie algebras, and since duality interchanges induction and restriction, it follows formally that $\int_{D^n}U_n(\mathfrak{g})\simeq C_*^{\mathrm{Lie}}(\mathfrak{g})$ as $E_n$-coalgebras. There is the dual notion of factorization \emph{co}homology, and, since the Chevalley-Eilenberg chain complex is cocommutative, we have the particularly simple expression $\int^MC_*^{\mathrm{Lie}}(\mathfrak{g})\simeq C_*^{\mathrm{Lie}}(C^{-*}(M;\mathfrak{g}))$. The appearance of compactly supported cohomology can then be understood as an avatar of Poincar\'{e} duality (see \cite{AF}).
\end{remark}

Returning to our discussion above, if $A$ is now an $n$-disk algebra rather than merely an $E_n$-algebra, then $A$ determines a shifted Lie algebra in $O(n)$-modules, but now with $O(n)$ acting on the suspension coordinates. A full discussion of this phenomenon and the corresponding enveloping algebra is beyond the scope of this paper. Since the analogue of Theorem \ref{enveloping calculation} is true in that context, we will content ourselves with making it our definition. 

As a matter of notation, if $K$ is an $O(n)$-module, we denote by $K_M$ the local system on $M$ induced by $K$ via the map $M\to BO(n)$ classifying $TM$.

\begin{definition}
Let $\mathfrak{g}$ be a Lie algebra in $O(n)$-modules. The $n$-\emph{enveloping algebra} of $\mathfrak{g}$ is the $n$-disk algebra $$U_n(\mathfrak{g}):=C_*^{\mathrm{Lie}}(C_c^{-*}(\mathbb{R}^n;\mathfrak{g}_{\mathbb{R}^n})),$$ where $O(n)$ acts diagonally on $C_c^{-*}(\mathbb{R}^n;\mathfrak{g}_{\mathbb{R}^n})$.
\end{definition}

This definition recovers the earlier one when the action of $O(n)$ is trivial, and it is arranged so as to provide the equivalence $$\int_M U_n(\mathfrak{g})\simeq C_*^{\mathrm{Lie}}(C_c^{-*}(M;\mathfrak{g}_M))$$ for every manifold with boundary and every Lie algebra in $O(n)$-modules. As a final note, via the functor $C_*^{\mathrm{Lie}}$, the factorization homology $\int_MU_n(\mathfrak{g})$ is endowed with the structure of a cocommutative coalgebra in $n$-disk algebras, a fact that will be important in what follows. When $M=\mathbb{R}^n$, this is an analogue of the familiar fact that the ordinary enveloping algebra of a Lie algebra is a Hopf algebra. 

\section{Configuration Spaces}

\subsection{Comparing algebras} \label{proof one} In this section, we prove the first half of Theorem \ref{with grading}. The connection between the two calculations of factorization homology presented thus far is made through a chain of equivalences of algebras.

\begin{lemma}
Let $K$ be an $O(n)$-module and $\underline{K}$ its underlying chain complex. There is a natural equivalence $$E_n(\underline{K})\simeq {\mathrm{Disk}}_n(K)$$ of augmented $E_n$-algebras.
\end{lemma}
\begin{proof} 
As $E_n$-algebras, ${\mathrm{Disk}}_n(O(n)\otimes V)\cong E_n(O(n)\otimes V)$ (see \cite{Wahl}). To complete the proof, we write $K$ as a split geometric realization of free $O(n)$-modules and note that $n$-disk algebras are monadic over $O(n)$-modules, and that both $O(n)$-modules and $E_n$-algebras are monadic over chain complexes. Thus all such colimits are computed in chain complexes and commute with every functor in sight.
\end{proof}

\begin{remark}
Thinking topologically, the generic example of an $n$-disk algebra in spaces is an $n$-fold loop space on an $O(n)$-space $X$ (see \cite{SalvatoreWahl} and \cite{Wahl}). In this context, the statement is that, \emph{as an $n$-fold loop space}, the homotopy type of $\Omega^n X$ does not depend on the action of $O(n)$ on $X$. 
\end{remark}

The corresponding result for enveloping algebras is immediate from Theorem \ref{enveloping calculation} and the definition we have chosen, but we record it nonetheless.

\begin{lemma}
Let $\mathfrak{g}$ be a Lie algebra in $O(n)$-modules and $\underline{\mathfrak{g}}$ its underlying Lie algebra. There is a natural equivalence $$U_n(\underline{\mathfrak{g}})\simeq U_n(\mathfrak{g})$$ of augmented $E_n$-algebras.
\end{lemma}

Connecting the two is the following formal observation.

\begin{lemma}
Let $V$ be a chain complex. There is a natural equivalence $$E_n(V)\simeq U_n({\mathrm{Lie}}(V[n-1]))$$ of augmented $E_n$-algebras, where ${\mathrm{Lie}}$ is the free Lie algebra functor.
\end{lemma}
\begin{proof}
The functors in question are left adjoints to the same forgetful functor.
\end{proof}

As noted in \cite{Francis}, this equivalence of $E_n$-algebras, together with Theorem \ref{enveloping calculation} and Proposition \ref{free}, is enough to identify the homology of configuration spaces with Lie homology for framed manifolds. In characteristic zero, more is true.

Recall that the Lie homology of a Lie algebra $\mathfrak{g}$ may be computed by means of the Chevalley-Eilenberg complex: $$H_*^{\mathrm{Lie}}(\mathfrak{g})\cong H({\mathrm{Sym}}(\mathfrak{g}[1]), d_\mathfrak{g}+D_{CE}),$$ with $D_{CE}$ determined as a graded coderivation by the formula $$D_{CE}(\sigma v\wedge \sigma w)=\sigma [v,w].$$ See \cite{FHT}, for example.

\begin{proposition}\label{equivalence}
Let $V$ be a chain complex. There are natural equivalences of augmented $n$-disk algebras 

$${\mathrm{Disk}}_n(V^{\sigma_n})\simeq U_n({\mathrm{Lie}}(V[n-1]))$$ and $${\mathrm{Disk}}_n(V)\simeq U_n({\mathrm{Lie}}(V^{\sigma_n}[n-1])).$$  In particular, these algebras determine equivalent homology theories.
\end{proposition}
\begin{proof}
The action of $O(n)$ on $C_c^{-*}(\mathbb{R}^n)$ is via the stable J-homomorphism, which is trivial over $\mathbb{F}$ above degree zero (see \cite{Adams}). Thus, up to homotopy, the action factors through the determinant, and we may choose quasi-isomorphisms of $O(n)$-modules $$C_c^{-*}(\mathbb{R}^n;V[n])\simeq V^{\sigma_n}$$ and $$C_c^{-*}(\mathbb{R}^n; V^{\sigma_n}[n])\simeq V$$ naturally in $V$. 

Now, for the first equivalence, we note that the inclusion $$ C_c^{-*}(\mathbb{R}^n;V[n])\to ({\mathrm{Sym}}( C_c^{-*}(\mathbb{R}^n;{\mathrm{Lie}}(V[n-1]))[1]),d+D_{CE})
$$ of $O(n)$-modules induces a map of $n$-disk algebras $${\mathrm{Disk}}_n(V^{\sigma_n})\simeq {\mathrm{Disk}}_n( C_c^{-*}(\mathbb{R}^n; V[n]))\to C_*^{\mathrm{Lie}}(C_c^{-*}(\mathbb{R}^n;{\mathrm{Lie}}(V[n-1]))) \simeq U_n({\mathrm{Lie}}(V[n-1]))$$ by the universal property of the free algebra. To show that this map is an equivalence, it suffices to observe that the underlying map of $E_n$-algebras is an equivalence by the preceding lemmas.

The argument establishing the second equivalence is identical, except that we instead consider the inclusion of $O(n)$-modules $$C_c^{-*}(\mathbb{R}^n;V^{\sigma_n}[n]_{\mathbb{R}^n})\to ({\mathrm{Sym}}( C_c^{-*}(\mathbb{R}^n;{\mathrm{Lie}}(V^{\sigma_n}[n-1])_{\mathbb{R}^n})[1]),d+D_{CE})\simeq U_n({\mathrm{Lie}}(V^{\sigma_n}[n-1]))
$$ and note that ${\mathrm{Lie}}(V^{\sigma_n}[n-1])\cong {\mathrm{Lie}}(V[n-1])$ as ordinary Lie algebras.
\end{proof}

Now, for any integer $s$, the Lie algebras $C_c^{-*}(M;{\mathrm{Lie}}(\mathbb{F}[s]))$ and $C_c^{-*}(M;{\mathrm{Lie}}(\mathbb{F}^{\sigma_n}[s])_M)$ are formal, a fact whose proof we defer momentarily. Combining this observation with Corollary \ref{homology}, Proposition \ref{twisted homology}, Theorem \ref{enveloping calculation}, Proposition \ref{equivalence}, the remarks at the close of \S\ref{boundaries}, and the observation that $(\mathbb{F}^{\sigma_n})_M\cong\mathbb{F}^w$, we obtain natural isomorphisms 

\begin{align*}&\bigoplus_{k\geq0}\widetilde H_*(B_k(M,\partial M);\mathbb{F})\cong H_*^{\mathrm{Lie}}(H_c^{-*}(M;{\mathrm{Lie}}(\mathbb{F}^w[n-1])))\\&\bigoplus_{k\geq0}\widetilde{H}_*(B_k(M,\partial M);\mathbb{F}^w)\cong H_*^{\mathrm{Lie}}(H_{c}^{-*}(M;{\mathrm{Lie}}(\mathbb{F}[n-1])))\\
&\bigoplus_{k\geq0}\widetilde{H}_*(B_k(M,\partial M);\mathbb{F}^w)[k]\cong H_*^{\mathrm{Lie}}(H_{c}^{-*}(M;{\mathrm{Lie}}(\mathbb{F}[n]))),\end{align*} the second valid for $n$ even and the third for $n$ odd. Thus, all that is left to be done in proving Theorem \ref{with grading} is to identify the extra gradings on each side. We will pursue this goal shortly.

\subsection{Formality results}
It remains to prove the claim of formality.

\begin{lemma}
Let $$0\to\mathfrak{h}\to\mathfrak{e}\to\mathfrak{g}\to0$$ be an exact sequence of Lie algebras in ${\mathrm{Ch}}_\mathbb{F}$ with $\mathfrak{g}$ and $\mathfrak{h}$ abelian. Assume that $\mathfrak{g}$ acts trivially on $\mathfrak{h}$ and that the underlying sequence of chain complexes splits. Then $\mathfrak{e}$ is formal.
\end{lemma}
\begin{proof}
The hypotheses imply that the bracket on $\mathfrak{e}\cong\mathfrak{g}\oplus\mathfrak{h}$ is given by $$[(g_1,h_1),(g_2,h_2)]=f(g_1,g_2)$$ for some (not uniquely defined) map $f:{\mathrm{Sym}}^2(\mathfrak{g}[1])[-2]\to\mathfrak{h}$, and the bracket on $H(\mathfrak{e})\cong H(\mathfrak{g})\oplus H(\mathfrak{h})$ is determined in the same way by $f_*$.

Choose quasi-isomorphisms $\varphi:\mathfrak{g}\to H(\mathfrak{g})$ and $\psi:\mathfrak{h}\to H(\mathfrak{h})$. Without loss of generality, we may assume that both maps induce the identity on homology. Let $\bar\psi$ be a quasi-inverse to $\psi$. Then $(\bar\psi\circ f_*\circ\varphi^{\wedge2})_*=f_*$, so $$\bar\psi\circ f_*\circ\varphi^{\wedge 2}-f=d_{\mathfrak{h}}G+Gd_{\mathrm{Sym}}$$ for some homotopy operator $G:{\mathrm{Sym}}^2(\mathfrak{g}[1])[-2]\to \mathfrak{h}[-1]$. 

Now, since $\mathfrak{g}$ is abelian and acts trivially, this equation may be written as $$D^{CE}G=\bar\psi\circ f_*\circ\varphi^{\wedge 2}-f,$$ where $D^{CE}$ denotes the Chevalley-Eilenberg differential in $C^*_\mathrm{Lie}(\mathfrak{g},\mathfrak{h})$. Since extensions of $\mathfrak{g}$ by the module $\mathfrak{h}$ are classified by $H^2_\mathrm{Lie}(\mathfrak{g},\mathfrak{h})$, it follows that $f$ and $\bar\psi\circ f_*\circ\varphi^{\wedge2}$ determine isomorphic extensions, so that we may take $f=\bar\psi\circ f_*\circ\varphi^{\wedge 2}$ after choosing a different splitting. But then $\psi\circ f=f_*\circ \varphi^{\wedge 2}$, so that the composite $$\begin{CD}
\mathfrak{e}@>\cong>>\mathfrak{g}\oplus\mathfrak{h}@>(\varphi,\psi)>>H(\mathfrak{g})\oplus H(\mathfrak{h})@>\cong>>H(\mathfrak{e})
\end{CD}$$ is a map of Lie algebras. Since it is also a quasi-isomorphism of chain complexes, the proof is complete.
\end{proof}

Let $\mathfrak{g}$ be a Lie algebra and $B$ a commutative algebra in ${\mathrm{Ch}}_\mathbb{F}$. Then $B\otimes\mathfrak{g}$ is a Lie algebra with bracket defined by $$[a \otimes v, b\otimes w]=(-1)^{|v||b|}ab\otimes [v,w].$$

\begin{corollary}\label{lemma}
For any $s\in\mathbb{Z}$ and any commutative algebra $B$, $B\otimes\mathrm{Lie}(\mathbb{F}[s])$ is formal.
\end{corollary}
\begin{proof}
If $s$ is even, then $\mathrm{Lie}(\mathbb{F}[s])$ is abelian, and so is $B\otimes\mathrm{Lie}(\mathbb{F}[s])$, so there is nothing to prove. If $s$ is odd, then $\mathrm{Lie}(\mathbb{F}[s])$ has as basis $v$ in degree $s$ and $[v,v]$ in degree $2s$. The exact sequence $$\begin{CD}0@>>>\mathbb{F}[2s]@>[v,v]>>\mathrm{Lie}(\mathbb{F}[s])@>>>\mathbb{F}[s]@>>>0\end{CD}$$ satisfies the hypotheses of the lemma, and, since tensoring with $B$ preserves these hypotheses, the assertion follows.
\end{proof}

We can now prove the claim.

\begin{proposition}
For any $s\in\mathbb{Z}$ and any manifold $M$, the Lie algebras $C_c^{-*}(M;{\mathrm{Lie}}(\mathbb{F}[s]))$ and $C_c^{-*}(M;{\mathrm{Lie}}(\mathbb{F}^{\sigma_n}[s])_M)$ are formal.
\end{proposition}
\begin{proof} To reduce clutter, we denote the Lie algebra in question by $\mathfrak{h}$ in both cases.

In the first case, $\mathfrak{h}\cong C_c^{-*}(M)\otimes{\mathrm{Lie}}(\mathbb{F}[s])$, and the previous corollary applies.

In the second case, if $s$ is even, then $\mathfrak{h}$ is abelian, and there is nothing to prove. If $s$ is odd, then, as a chain complex, $$\mathfrak{h}\cong C_c^{-*}(M;\mathbb{F}^w)[s]\oplus C_c^{-*}(M;\mathbb{F})[2s],$$ and the bracket is induced by the cup product $$C_c^{-*}(M;\mathbb{F}^w)^{\otimes 2}\to C_c^{-*}(M;\mathbb{F}^w\otimes\mathbb{F}^w)\cong C_c^{-*}(M;\mathbb{F}).$$ Thus there is an exact sequence of Lie algebras $$\begin{CD}0@>>> C_c^{-*}(M;\mathbb{F})[2s]@>>>\mathfrak{h} @>>> C_c^{-*}(M;\mathbb{F}^w)[s]@>>>0,\end{CD}$$ which evidently satisfies the hypotheses of the lemma.
\end{proof}

\subsection{Filtrations}\label{proof two}

In this section, we complete the proof of Theorem \ref{with grading}. The first step is to notice that the functor $F:{\mathrm{Ch}}_\mathbb{F}\to{\mathrm{Ch}}_\mathbb{F}$ defined by $$F(V)=H(U_n({\mathrm{Lie}}(V[n-1]))\cong{\mathrm{Sym}}({\mathrm{Lie}}(V[n-1])[1-n]).$$ has been identified with the homology of two different functors, each of which carries a natural filtration.

For the first, notice that, by formal properties of adjunctions, ${\mathrm{Lie}}(V[n-1])[1-n]$ is the free $n-1$-shifted Lie algebra on the chain complex $V$. The free algebra on an operad, considered as an endofunctor on chain complexes, is a split analytic functor, and every such functor is canonically filtered; in this case, the filtration is the usual Lie filtration by bracket length. We extend this filtration to $U_n({\mathrm{Lie}}(V[n-1]))$ by requiring that it be compatible with the inclusion and coproduct, and we refer to the induced grading on $F$ as the \emph{weight grading}.

The free $E_n$-algebra is also a split analytic functor, with $r$th filtered piece given explicitly by $$\bigoplus_{0\leq k\leq r}{\mathrm{Conf}}_k(\mathbb{R}^n)\otimes_{\Sigma_k}(-)^{\otimes k}.$$ We refer to the induced grading on $F$ as the \emph{cardinality grading}.

\begin{proposition}\label{filtration}
The isomorphisms on homology induced by the equivalences of Proposition \ref{equivalence} interchange the weight and cardinality gradings.
\end{proposition}
\begin{proof}
We proceed by induction on the symmetric filtration of $$F(V)\cong {\mathrm{Sym}}({\mathrm{Lie}}(H(V)[n-1])[1-n]).$$

For the base case, recall that the inclusion of ${\mathrm{Lie}}(H(V)[n-1])[1-n]$ into $F(V)$ is induced by the inclusion of the shifted Lie operad into the $n$-Poisson operad, which is the homology of $E_n$ (see \cite{GJ}). In particular, this inclusion is a map of split analytic functors, and, since both gradings were defined to be induced by the canonical filtrations of the respective split analytic functors, the base case is proven.

For the induction step, suppose the proposition is known for polynomials of degree less than $r$, and choose a monomial $x=x_1\cdots x_r\in F(V)$, where $x_i\in {\mathrm{Lie}}(H(V)[n-1])[1-n]$ has weight $w_i$. We wish to show that $x$ has cardinality degree $w:=w_1+\cdots+w_i$. The coproduct on $U_n({\mathrm{Lie}}(V[n-1]))$ is a map of $E_n$-algebras and hence a map of split analytic functors; in particular, the reduced coproduct $\overline\Delta:F(V)\to F(V)\otimes F(V)$ preserves the cardinality grading, so it will suffice to show that $\overline\Delta(\alpha)$ has cardinality degree $w$ in $F(V)\otimes F(V)$. But $\overline\Delta(\alpha)$ is a sum of terms of the form $\beta_1\otimes \beta_2$, where the $\beta_i$ are monomials in the $x_i$ of degree strictly less than $r$ whose weights add to $w$. Since the claim holds for polynomials of degree less than $r$ by induction, we conclude that $\beta_1\otimes \beta_2$ has cardinality degree $w$ in $F(V)\otimes F(V)$, as desired.
\end{proof}

Proposition \ref{filtration} completes the proof of the theorem in the special case $M=\mathbb{R}^n$. The final step is to globalize this result using the spectral sequence alluded to in Remark \ref{spectral sequence remark}.

\begin{proposition}\label{spectral sequence} Let $M$ be an $n$-manifold and $A$ an $n$-disk algebra.
There is a natural first-quadrant spectral sequence $$E^2_{p,q}\cong H_{p+q}(M\otimes H(A))\implies H_{p+q}\bigg(\int_MA\bigg),$$ with differential $d^r$ of bidegree $(-r,r-1)$.
\end{proposition}
\begin{proof}
From the construction of $\int_M A$ as a homotopy colimit, its homology is computed by the bicomplex $$\cdots\to\bigoplus_{U_1\to U_0\to M}A(U_1)\to \bigoplus_{U_0\to M} A(U_0),$$ where the sums are indexed by composable strings of embeddings of disjoint unions of disks into one another. Filtering by skeleta in the usual way, we obtain a spectral sequence $$E^1_{p,q}=\bigoplus_{U_p\to\cdots\to U_0\to M}H_q(A(U_p))\implies H_{p+q}\bigg(\int_M A\bigg).$$ The differential $d^1$ is the horizontal differential in the bicomplex $$W=\bigg(\cdots\to\bigoplus_{U_1\to U_0\to M}H(A(U_1))\to \bigoplus_{U_0\to M} H(A(U_0))\bigg).$$ Therefore, $$E^2_{p,q}\cong H_p(H_q(W))\cong H_{p+q}(\mathrm{Total}(W))\cong H_{p+q}\bigg(\int_M H(A)\bigg),$$ since the vertical differential in $W$ vanishes, and since $W$ is precisely the bicomplex that computes the homology of $\int_MH(A)$. Since $H(A)$ is commutative, we may apply the following proposition to identify the $E^2$ page with the tensor.
\end{proof}

\begin{proposition}
Let $B$ be a commutative algebra in ${\mathrm{Ch}}_\mathbb{F}$, considered as an $n$-disk algebra. Then $$\int_MB\simeq M\otimes B,$$ where the tensor is computed in commutative algebras.
\end{proposition}

For a proof, see \cite{Francis}.

\begin{remark}
Horel considers the same spectral sequence in a somewhat different context in \cite{Horel}.
\end{remark}

\begin{proof}[Proof of Theorem \ref{with grading}]
From the equivalence of Proposition \ref{equivalence}, we obtain an isomorphism of spectral sequences,

$$\xymatrix{
H(M\otimes F(V))\ar@{=>}[r]\ar[d]^\wr & \displaystyle H\bigg(\int_M U_n({\mathrm{Lie}}(V^{\sigma_n}[n-1]))\bigg)\ar[d]^\wr\\
H(M\otimes F(V))\ar@{=>}[r] & \displaystyle H\bigg(\int_M {\mathrm{Disk}}_n(V)\bigg).
}$$

The filtration of $U_n({\mathrm{Lie}}(V^{\sigma_n}[n-1]))$ by weight and the filtration of ${\mathrm{Disk}}_n(V)$ by cardinality induce filtrations on the respective factorization homologies, so that these spectral sequences are each trigraded. By Proposition \ref{filtration}, the isomorphism preserves the extra grading on $E^2$ and hence on $E^\infty$. 

This establishes the claim for the first isomorphism in the statement of the theorem. The second and third follow by the same argument after a change of notation.
\end{proof}

\subsection{Formulas}\label{formulas} In this section, we use Theorem \ref{with grading} to produce chain complexes computing the homology of individual configuration spaces, reproducing the results alluded to in the introduction. To the author's knowledge, these results are new in this generality.

\begin{corollary}
If $n$ is odd, there is an isomorphism $$\widetilde{H}_*(B_k(M,\partial M);\mathbb{F})\cong {\mathrm{Sym}}^k(\widetilde{H}_*(M,\partial M;\mathbb{F})).$$
\end{corollary}
\begin{proof}
Since $n$ is odd, the Lie algebra in question is abelian, so that the Chevalley-Eilenberg complex has no differential, and the weight grading coincides with the usual grading of the symmetric algebra. The claim follows after replacing shifted, twisted, compactly supported cohomology with relative homology using Poincar\'{e} duality.
\end{proof}

When $M$ is orientable, we recover the result of \cite{BCT}, as formulated in dual form in \cite{FelixTanre}, in which the isomorphism on cohomology is shown to be an isomorphism of algebras.

\begin{corollary}
If $n$ is even, $\widetilde{H}_*(B_k(M,\partial M);\mathbb{F})$ is isomorphic to the homology of the complex $$\bigg(\bigoplus_{i=0}^{\lfloor\frac{k}{2}\rfloor}{\mathrm{Sym}}^{k-2i}(H^{-*}_c(M;\mathbb{F}^w))[n]\otimes{\mathrm{Sym}}^{i}(H^{-*}_c(M;\mathbb{F}))[2n-1],\,D\bigg),$$ where the differential $D$ is defined as a coderivation by the equation $$D(\sigma^n\alpha\wedge\sigma^n\beta)=(-1)^{(n-1)|\beta|}\sigma^{2n-1}(\alpha\smile\beta).$$
\end{corollary}
\begin{proof}
It suffices by Theorem \ref{with grading} to identify the complex in question with the weight $k$ part of the Chevalley-Eilenberg complex for $\mathfrak{g}=H_c^{-*}(M;{\mathrm{Lie}}(\mathbb{F}^w[n-1]))$, which as a graded vector space is given by $${\mathrm{Sym}}(\mathfrak{g}[1])\cong{\mathrm{Sym}}(H_c^{-*}(M;\mathbb{F}^w)[n])\otimes {\mathrm{Sym}}(H_c^{-*}(M;\mathbb{F})[2n-1]),$$ with differential determined as a coderivation by the bracket of $\mathfrak{g}$, which is none other than the shifted cup product shown above (the sign is determined by the usual Koszul rule of signs). Since the generators of the first tensor factor have weight 1 and those of the second tensor factor weight 2, the subcomplex of total weight $k$ is exactly the sum shown above. 
\end{proof}

When $M$ is closed, orientable, and nilpotent, we recover the linear and Poincar\'{e} dual of the result of \cite{FT}, as formulated in \cite{FelixTanre}. When $M$ is a punctured surface, we recover the result of \cite{BC}.

\begin{remark}
In \cite{CohenTaylor}, a spectral sequence $(E_r,d_r)$ is constructed for oriented manifolds $M$ converging to $H^{-*}({\mathrm{Conf}}_k(M);\mathbb{F})$ (see also \cite{Totaro}). It follows from these results and the work of \cite{FelixTanre} that there is a quasi-isomorphism $$\bigg(\int_M {\mathrm{Disk}}_n(\mathbb{F})\bigg)^\vee\simeq \bigoplus_{k\geq0}(E_1,d_1)^{\Sigma_k}.$$
\end{remark}

Regarding twisted homology, we have the following:

\begin{corollary}
If $n$ is even, $\widetilde{H}_*(B_k(M,\partial M);\mathbb{F}^w)$ is isomorphic to homology of the complex $$\bigg(\bigoplus_{i=0}^{\lfloor\frac{k}{2}\rfloor}{\mathrm{Sym}}^{k-2i}(H^{-*}_c(M;\mathbb{F}))[n]\otimes{\mathrm{Sym}}^{i}(H^{-*}_c(M;\mathbb{F}))[2n-1],\,D\bigg).$$ If $n$ is odd, $\widetilde{H}_*(B_k(M,\partial M);\mathbb{F}^w)$ is isomorphic to the $k$th desuspension of the homology of the complex $$\bigg(\bigoplus_{i=0}^{\lfloor\frac{k}{2}\rfloor}{\mathrm{Sym}}^{k-2i}(H^{-*}_c(M;\mathbb{F}))[n+1]\otimes{\mathrm{Sym}}^{i}(H^{-*}_c(M;\mathbb{F}))[2n+1],\,D\bigg).$$ In both cases, $D$ is determined as before by the cup product.
\end{corollary}

To the author's knowledge, this result is new in all cases except when $M$ is orientable and $n$ is even, so that $B_k(M)$ is orientable.

\subsection{Stability}\label{stability} This section assembles the proof of Theorem \ref{stability theorem}. Throughout, unless otherwise noted, $M$ will be connected, without boundary, and of dimension $n>1$.

For a Lie algebra $\mathfrak{g}$, recall that the linear dual of the Lie chains of $\mathfrak{g}$ are the Lie cochains, so that there is a pairing $$\langle-,-\rangle:C^{-*}_{\mathrm{Lie}}(\mathfrak{g})\otimes C_*^{\mathrm{Lie}}(\mathfrak{g})\to \mathbb{F}.$$ Given a functional $\lambda$, the composite $$\begin{CD}
C_*^{\mathrm{Lie}}(\mathfrak{g})@>\lambda\otimes1>>C^{-*}_{\mathrm{Lie}}(\mathfrak{g})\otimes C_*^{\mathrm{Lie}}(\mathfrak{g})@>1\otimes\Delta>> C^{-*}_{\mathrm{Lie}}(\mathfrak{g})\otimes C_*^{\mathrm{Lie}}(\mathfrak{g})\otimes C_*^{\mathrm{Lie}}(\mathfrak{g})@>\langle-,-\rangle\otimes 1>>C_*^{\mathrm{Lie}}(\mathfrak{g})
\end{CD}$$ is called the \emph{cap product with $\lambda$}, denoted $\lambda\frown(-)$. This map respects the weight grading everywhere and induces a map of the same name on homology when $\lambda$ is closed. 

Now, take $\mathfrak{g}=H_c^{-*}(M;{\mathrm{Lie}}(\mathbb{F}^w[n-1]))$, and choose a generator of $H_0(M;\mathbb{F})\cong\mathbb{F}$ (here we use that $M$ is connected). Denote the Poincar\'{e} dual element by $[M]\in H_c^{n}(M;\mathbb{F}^w)$, and let $p$ denote the image of $\sigma^n[M]$ under the inclusion $$H_c^{-*}(M;\mathbb{F}^w)[n]\to {\mathrm{Sym}}(\mathfrak{g}[1]).$$ Note that $p$ has weight 1 and homological degree 0, so that capping with the dual functional $p^\vee$ preserves homological degree and lowers weight by 1. 

We will need only one calculation of this cap product. It is important to note that, when we speak of divisibility and multiplication in the Chevalley-Eilenberg complex, we refer only to the formal manipulation of monomials in ${\mathrm{Sym}}(\mathfrak{g}[1])$ and not to any kind of differential graded algebra structure. 

\begin{lemma}\label{cap formula}
In ${\mathrm{Sym}}(\mathfrak{g}[1])$, the equation $$p^\vee\frown p^rx=rp^{r-1}x$$ holds for any monomial $x$ not divisible by $p$ and any $r>0$.
\end{lemma}
\begin{proof}Write $\Delta(x)=x\otimes 1+1\otimes x+\sum_j y_j\otimes y_j'$ with $y_j$ and $y_j'$ lying in the augmentation ideal and not divisible by $p$. Then we have \begin{align*}\Delta(p^rx)&=\Delta(p)^r\Delta(x)=(p\otimes 1+1\otimes p)^r(x\otimes 1+1\otimes x+\sum_j y_j\otimes y_j')\\
&=\sum_{i=0}^r\binom{r}{i}\bigg(p^ix\otimes p^{r-i}+p^i\otimes p^{r-i}x+\sum_jp^iy_j\otimes p^{r-i}y_j'\bigg),
\end{align*} whence $$p^\vee\frown p^rx=\sum_{i=0}^r\binom{r}{i}\bigg(\langle p^\vee,p^ix\rangle p^{r-i}+\langle p^\vee, p^{i}\rangle p^{r-i}x\rangle+\sum_j\langle p^\vee, p^iy_j\rangle p^{r-i}y_j'\bigg)=rp^{r-1}x.$$ Here we have used the assumption that $p$ does not divide $x$ to conclude that $p^ix$ is not a scalar multiple of $p$ for any $i$, and we have used the fact that $y_j$ lies in the augmentation ideal and is not divisible by $p$ to conclude that $p^i y_j$ is not a scalar multiple of $p$ for any $(i,j)$.
\end{proof}

The central observation behind our approach to stability is the following.

\begin{lemma}\label{divisibility}
Unless $n=2$ and $M$ is orientable, any monomial $x\in{\mathrm{Sym}}(\mathfrak{g}[1])$ with weight greater than $|x|$ is divisible by $p$. Moreover, if $n=2$ and $M$ is orientable, any nonzero monomial $x$ with weight greater than $|x|+1$ is divisible by $p$.
\end{lemma}
\begin{proof}
Write $x=x_1\cdots x_r$ with $x_i\in \mathfrak{g}[1]$. Since $x$ has weight greater than $|x|$, $x_j$ has weight greater than $|x_j|$ for some $j$. Since $x_j\in \mathfrak{g}[1]$, either $x_j=\sigma^n\alpha$ or $x_j=\sigma^{2n-1}\beta$, where $\alpha\in H_c^{-*}(M;\mathbb{F}^w)$ and $\beta\in H^{-*}_c(M;\mathbb{F})$. In the first case, $x_j$ has weight 1, so $$-|\alpha|>n-1\implies -|\alpha|=n,$$ from which it follows that $\alpha$ is a scalar multiple of $[M]$, so that $x_j$ is a scalar multiple of $p$. In the second case, $x_j$ has weight 2, so $$-|\beta|>2n-3\implies n<3,$$ since $-|\beta|\leq n$. Thus, provided $n\neq 2$, this is a contradiction, and we are finished. 

If $n=2$, the inequality becomes $-|\beta|>1$, so that $-|\beta|=2$. Suppose that $M$ is non-orientable; then $H_c^{2}(M;\mathbb{F})\cong H_0(M;\mathbb{F}^w)=0,$ and no such $\beta$ exists. Thus, in this case as well, we have a contradiction.

Assume now that $n=2$, that $M$ is orientable, that the weight of $x$ is greater than $|x|+1$, and that $p$ does not divide $x$. As above, we must have $-|\beta|=2$, so that $x_j=\sigma^{3}[M]$ after scaling. Then we may write $x=\sigma^3[M]\cdot y$, where $y$ has homological degree $|x|-1$ and weight greater than $|x|-1$, since $\sigma^3[M]$ has homological degree 1 and weight 2. Since $p$ does not divide $x$, $p$ does not divide $y$, and the same argument shows that $\sigma^{3}[M]$ divides $y$. But $\sigma^{3}[M]$ is an exterior generator, so $x=0$, a contradiction.
\end{proof}

\begin{lemma}\label{differential}
Let $w$ be a homogeneous element of ${\mathrm{Sym}}(\mathfrak{g}[1])$ of weight $k$ such that $|w|< k+n-1$. Then $$D_{CE}(pw)=pD_{CE}(w).$$
\end{lemma}
\begin{proof}
By linearity, we may assume that $w$ is a monomial. We may also assume that $M$ is compact and orientable, for otherwise no element of $H_c^{-*}(M;\mathbb{F}^w)$ cups nontrivially with $[M]$, and the claim is immediate from the definition of $D_{CE}$ and the fact that $|p|=0$ is even. When $M$ is compact and orientable, $[M]$ cups nontrivially only with scalar multiples of $1\in H^{-*}(M;\mathbb{F})$. The corresponding element in the Chevalley-Eilenberg complex has weight 1 and degree $n$ and so cannot appear as a factor in $w$, since $|w|-k<n-1$.
\end{proof}

We are now equipped to state the result that will imply Theorem \ref{stability}. Denote by $C(k)$ the subcomplex of the Chevalley-Eilenberg complex spanned by the weight $k$ monomials. Capping with $p^\vee$ defines a map $\Phi_k:C(k+1)\to C(k)$ whose induced map on homology we denote by $$\varphi_k:H_*(B_{k+1}(M);\mathbb{F})\to H_*(B_{k}(M);\mathbb{F}).$$ Recall that the $r$th \emph{brutal truncation} of a chain complex $V$ is the chain complex $\tau_{\leq r}V$ defined by $$(\tau_{\leq r}V)_i=\begin{cases}
V_i,\qquad &i\leq r,\\
0,\qquad &\text{else.}
\end{cases}$$ Truncation is a functor on chain complexes in the obvious way.

\begin{proposition}
For each $k$, $\varphi_k$ is an isomorphism \begin{itemize}
\item for $*<k$ when $M$ is orientable and $n=2$; and
\item for $*\leq k$ in all other cases.
\end{itemize}
\end{proposition}
\begin{proof}
We give the proof in the case when $M$ is not an orientable 2-manifold. The proof in the exceptional case differs only in notation.

Our first claim is that $\tau_{\leq k}\Phi_k$ is an isomorphism of chain complexes. To see this, let $y\in \tau_{\leq k}C(k+1)$ be a monomial. By Lemma \ref{divisibility}, we may write $y=p^rx$, with $x$ a monomial not divisible by $p$ and $r>0$, and Lemma \ref{cap formula} then implies that $\Phi_k(y)=rp^{r-1}x,$ from which it follows that $\Phi_k$ is injective on monomials and hence injective. On the other hand, let $z\in \tau_{\leq k}C(k)$ be a monomial. Then $pz\in \tau_{\leq k}C(k+1)$ is also a monomial, and $\Phi_k(pz)$ is a scalar multiple of $z$, again by Lemma \ref{cap formula}. It follows that $\Phi_k$ is surjective on monomials and hence surjective.

It follows from this claim and the definition of the brutal truncation that $\varphi_k$ is an isomorphism for $*<k$, and that $\Phi_k$ induces a bijection on $k$-cycles and an injection on $k$-boundaries. Thus, to prove the assertion, it suffices to show that $\Phi_k$ induces a surjection on $k$-boundaries. 

Let $v\in C(k)$ be a $k$-boundary, so that we may write $v=D_{CE}(w)$. The claim will follow once we are assured of the relation $D_{CE}(pw)=pD_{CE}(w)$, for then we will have that $pv$ is a boundary and $\Phi_k(pv)=v$, as desired, where the last equality follows by linearity from Lemma \ref{cap formula} and the fact that $v$ has equal weight and degree and so cannot be divisible by $p$. Since $w$ has weight $k$ and degree $k+1$, the necessary relation follows from Lemma \ref{differential}.

\end{proof}

\begin{remark}
Let $\mathbb{K}$ denote the Klein bottle. As shown in \S\ref{examples}, $$\dim H_*(B_k(\mathbb{K});\mathbb{F})=\begin{cases}
1,\qquad &i\in\{0,1,2, k+1\},\\
2,\qquad &3\leq i\leq k,\\
0,\qquad \text{else.}
\end{cases}$$ In particular, $H_{k+1}(B_{k+1}(\mathbb{K});\mathbb{F})\ncong H_{k+1}(B_{k}(\mathbb{K});\mathbb{F})$, so that our bound is sharp in the sense that no better stable range holds for all manifolds that are not orientable surfaces.
\end{remark}

\begin{remark}
If $M$ is orientable and $H_*(M;\mathbb{F})=0$ for $1\leq *\leq r-1$, then $H_c^{-*}(M;\mathbb{F})=0$ for $n-r+1\leq -*\leq n-1$, and the argument of Lemma \ref{divisibility} shows that a monomial $x$ is divisible by $p$ provided its weight is greater than $\frac{|x|}{r}+1$. This improved estimate leads to an improved stable range, as in \cite{Church}.
\end{remark}

\begin{remark}
In \cite{MillerKupers}, factorization homology is used to obtain homological stability results for various constructions on open manifolds. The approach there is through certain ``partial algebras'' and appears unrelated to ours.
\end{remark}

\section{Examples}\label{examples}

We close with a selection of computations illustrating the following general procedure for determining the rational homology of the configuration space of $k$ points in an $n$-manifold $M$:

\begin{enumerate}
\item determine the appropriate compactly supported cohomology of $M$;
\item perform a Lie algebra homology calculation;
\item count basis elements of weight $k$.
\end{enumerate}

\noindent It should be emphasized that the second and third tasks are of the sort that a computer could perform; only the first has any real content. It is also worth noting that the Chevalley-Eilenberg complex allows one to obtain answers simultaneously for all $k$, effectively reducing infinitely many computations to one.

\begin{convention} In the following examples, a variable decorated with a tilde has weight 2, while an unadorned variable has weight 1.
\end{convention}

\subsection{Punctured Euclidean space}

As a warm-up and base case, we recover the classical computation of $H_*(B_k(\mathbb{R}^n);\mathbb{F})$. Since there are no cup products in the compactly supported cohomology of $\mathbb{R}^n$, there are no differentials in the corresponding Chevalley-Eilenberg complex. Thus $H_*(B_k(\mathbb{R}^n);\mathbb{F})$ is identified with the subspace of $\mathbb{F}[x]$ spanned by $x^k$ when $n$ is odd, while for $n$ even the identification is with the subspace of $$\mathbb{F}[x]\otimes\Lambda[\tilde x],\qquad |x|=0,\,|\tilde x|=n-1$$ spanned by elements of weight $k$, a basis for which is given by $\{x^k, x^{k-2}\tilde x\}$. We conclude, for all $k>1$, that $$H_*(B_k(\mathbb{R}^n);\mathbb{F})\cong\begin{cases}
\mathbb{F},\qquad\qquad& n\text{ odd}\\
\mathbb{F}\oplus\mathbb{F}[n-1], &n\text{ even}.
\end{cases}$$

Now, choose $\bar p=\{p_1,\ldots, p_m\}\in\mathbb{R}^n$. There is a homotopy equivalence $(\mathbb{R}^n\setminus\bar p)^+\simeq S^{n}\vee (S^1)^{\vee m},$ so that $H_c^{-*}(\mathbb{R}^n\setminus\bar p;\mathbb{F})\cong\mathbb{F}^m[-1]\oplus\mathbb{F}[-n].$ There are no cup products, so there can be no differentials.

If $n$ is odd, Theorem \ref{with grading} identifies $H_*(\mathbb{R}^n\setminus\bar p;\mathbb{F})$ with the weight $k$ part of $$\mathbb{F}[x,y_1,\ldots, y_m],\qquad |x|=0,\, |y_i|=n-1,$$ and an easy induction now shows that $$ \dim H_*(B_k(\mathbb{R}^n\setminus\bar p);\mathbb{F})=\begin{cases}
\binom{m+i-1}{i},\quad &*=i(n-1), \,\,0\leq i\leq k,\\
0,\quad &\text{else.}
\end{cases}$$ (it is helpful to recall that $\binom{m+i-1}{i}$ is the number of ways to choose $i$ not necessarily distinct elements from a set of $m$ elements).

If $n$ is even, then the corresponding vector space is the weight $k$ part of $$\mathbb{F}[x, \tilde y_1,\ldots, \tilde y_m]\otimes\Lambda[\tilde x, y_1,\ldots, y_m],\qquad |x|=0,\,|y_i|=|\tilde x|=n-1,\, |\tilde y_i|=2n-2.$$ Counting inductively in terms of less punctured Euclidean spaces, one finds that $$H_*(B_k(\mathbb{R}^n\setminus \bar p;\mathbb{F})\cong \bigoplus_{l=0}^k\bigoplus_{j_1+\cdots+j_m=l} H_{*-l(n-1)}(B_{k-l}(\mathbb{R}^n);\mathbb{F}),$$ from which it follows easily that $$\dim H_*(B_k(\mathbb{R}^n\setminus \bar p;\mathbb{F})=\begin{cases}
\binom{m+i-1}{m-1}+\binom{m+i-2}{m-1}, \quad &*=i(n-1), \, 0\leq i<k,\\	
\binom{m+k-1}{m-1},\quad &*=k(n-1),\\
0,\quad &\text{ else}
\end{cases}$$ (it is helpful to recall that $\binom{m+i-1}{m-1}$ is the number of ways to write $i$ as the sum of $m$ non-negative integers).

It should be clear from this example that Theorem \ref{with grading} reduces calculations to counting problem whenever $n$ is odd or the relevant compactly supported cohomology has no cup products.

\subsection{Punctured torus}
Since $H_c^{-*}(T^2\setminus \mathrm{pt};\mathbb{F})\cong \widetilde H^{-*}(T^2;\mathbb{F}),$ the relevant Lie algebra is isomorphic to $$\mathfrak{h}\oplus \mathbb{F}\langle \tilde a, \tilde b, c\rangle$$ where $\mathfrak{h}=\mathbb{F}\langle a, b, \tilde c\rangle$ as a vector space, $|a|=|b|=|\tilde c|=0$, $|\tilde a|=|\tilde b|=1$, $|c|=-1$, and the bracket is defined by the equation $$[a,b]=\tilde c.$$

The Lie homology of $\mathfrak{h}$ is calculated by the complex $$(\Lambda[ x, y, \tilde z], d(xy)=\tilde z),$$ (where for ease of notation we have set $x=\sigma a$ and so on), a basis for the homology of which is easily seen to be given by the image in homology of the set $\{1,x, y, x\tilde z,  y\tilde z, xy\tilde z\}.$ Thus we have an identification of $H_*(B_k(T^2\setminus \mathrm{pt});\mathbb{F})$ with the weight $k$ part of $$\mathbb{F}\langle 1,x,y,x\tilde z,y\tilde z,xy\tilde z\rangle\otimes \mathbb{F}[\tilde x,\tilde y,z],\qquad |z|=0,\, |x|=|y|=|\tilde z|=1,\,|\tilde x|=|\tilde y|=2.$$ Counting, we find that $$\dim H_i(B_k(T^2\setminus\mathrm{pt});\mathbb{F})=\begin{cases}
\frac{3i-1}{2}+1,\quad &*=2i+1<k,\\
\frac{3i}{2}+1,\quad &*=2i<k,\\
k+1,\quad &*=k \text{ odd,}\\
\frac{k}{2}+1,\quad &*=k \text{ even,}\\
0,\quad &\text{else.}
\end{cases}$$ The same methods applied to the case of a punctured surface of genus $g$ produce a cumbersome formula that nevertheless depends only on $k$ and $g$. We do not record this formula here.

An amusing comparison can be seen by taking $k=2$ in the above formula, which yields $$H_*(B_2(T^2\setminus \mathrm{pt});\mathbb{F})\cong \mathbb{F}\oplus\mathbb{F}^2[1]\oplus\mathbb{F}^2[2].$$ On the other hand, from the preceding example, one calculates that $$H_*(B_2(\mathbb{R}^2\setminus \{p_1,p_2\};\mathbb{F})\cong \mathbb{F}\oplus\mathbb{F}^3[1]\oplus\mathbb{F}^3[2].$$ Thus, despite the fact that the punctured torus and the twice-punctured plane are homotopy equivalent, having $S^1\vee S^1$ as a common deformation retract, their configuration spaces are not homotopy equivalent. The explanation is that the two manifolds are not \emph{proper} homotopy equivalent; as is shown in \cite{Francis}, the stable homotopy type of $B_k(M)$ is a proper homotopy invariant of $M$ when $M$ is oriented.

\subsection{Real projective space}
Let $n$ be even, so that $\mathbb{RP}^n$ is non-orientable. Then, as a ring, $H_c^{-*}(\mathbb{RP}^n;\mathbb{F})\cong\mathbb{F}$, and the Lie homology of interest is $H_*^{\mathrm{Lie}}({\mathrm{Lie}}(\mathbb{F}[n-1]))\cong\mathbb{F}\oplus \mathbb{F}[n],$ whence, for $k>1$, $$H_*(B_k(\mathbb{RP}^n);\mathbb{F}^w)=0.$$

As for the untwisted homology, we note that $H_c^{-*}(\mathbb{RP}^n;\mathbb{F}^w)\cong \mathbb{F}[-n]$ by Poincar\'{e} duality, so that the cup product map $H_c^{-*}(\mathbb{RP}^n;\mathbb{F}^w)^{\otimes 2}\to H_c^{-*}(\mathbb{RP}^n;\mathbb{F})$ is trivial for degree reasons. Thus $$H_c^{-*}(\mathbb{RP}^n; {\mathrm{Lie}}(\mathbb{F}^w[n-1]))\cong \mathbb{F}[-1] \oplus \mathbb{F}[2n-2]$$ is abelian, so that $H_*(B_k(\mathbb{RP}^n);\mathbb{F})$ is isomorphic to the weight $k$ part of $$\mathbb{F}[x]\otimes \Lambda[\tilde y],\quad |x|=0,\, |\tilde y|=2n-1.$$ Hence for all $k>1$, $$H_*(B_k(\mathbb{RP}^n);\mathbb{F})\cong\mathbb{F}\oplus\mathbb{F}[2n-1].$$

\subsection{Klein bottle, twisted}
Let $\mathbb{K}$ denote the Klein bottle. Then $H_c^{-*}(\mathbb{K};\mathbb{F})\cong \mathbb{F}\oplus\mathbb{F}[-1]$, with the generator in degree zero acting as a unit for the multiplication. As a vector space, the Lie algebra in question is $\mathfrak{g}:=\mathbb{F}\langle a, \tilde a, b, \tilde b\rangle$, where $|b|=0$, $|a|=|\tilde b|=1$, and $|\tilde a|=2$, and the bracket is defined by the equations $$[a,a]=\tilde a\qquad\qquad [a,b]=-\tilde b.$$ The subspace spanned by $\{b,\tilde b\}$ is an ideal realizing $\mathfrak{g}$ as an extension $$0\to \mathbb{F}\langle b,\tilde b\rangle\to\mathfrak{g}\to {\mathrm{Lie}}(\mathbb{F}\langle a\rangle)\to0,$$ so that we may avail ourselves of the Lyndon-Hochschild-Serre spectral sequence $$E^2_{p,q}\cong H^{\mathrm{Lie}}_p({\mathrm{Lie}}(\mathbb{F}\langle a\rangle);H^{\mathrm{Lie}}_q(\mathbb{F}\langle b,\tilde b\rangle))\implies H^{\mathrm{Lie}}_{p+q}(\mathfrak{g}).$$ There are no differentials for degree reasons, and the $E^2$ page is computed as the homology of the complex $$0\to \mathbb{F}\langle a\rangle[1]\otimes {\mathrm{Sym}}(\mathbb{F}\langle b,\tilde b\rangle[1])\to {\mathrm{Sym}}(\mathbb{F}\langle b,\tilde b\rangle[1])\to 0,$$ where the differential is the action of $a$. It follows that a basis for $H_*^{\mathrm{Lie}}(\mathfrak{g})$ is given by $\{\sigma a\otimes (\sigma\tilde b)^i,\sigma b\otimes(\sigma\tilde b)^j\mid i,j\geq0\}.$ Counting monomials of weight $k$, we find that $$H_*(B_k(\mathbb{K});\mathbb{F}^w)\cong\begin{cases}
\mathbb{F}[k]\oplus \mathbb{F}[k+1],\qquad &k\text{ odd,}\\
0,\qquad\qquad &k\text{ even.}
\end{cases}$$

\subsection{Non-orientable surfaces}
Let $N_h=(\mathbb{RP}^2)^{\#h}$. Using the method of the previous example, one could proceed to obtain a general formula for the twisted homology of $B_k(N_h)$. Here we will determine the corresponding untwisted homology. We have $$H_c^{-*}(N_h;\mathbb{F})\cong\mathbb{F}\oplus\mathbb{F}[-1]^{h-1},\qquad H_c^{-*}(N_h; \mathbb{F}^w)\cong\mathbb{F}[-1]^{h-1}\oplus\mathbb{F}[-2],$$ so that there can be no cup products. Thus $H_*(B_k(N_h);\mathbb{F})$ is the weight $k$ part of $$\mathbb{F}[x, \tilde y_1,\ldots, \tilde y_{h-1}]\otimes \Lambda[\tilde z, w_1,\ldots, w_{h-1}],\quad |x|=0,\,|w_i|=1,\,|\tilde y_i|=2,\,|\tilde z|=3.$$ Counting inductively as in the example of punctured Euclidean space, we find that $$H_*(B_k(N_h);\mathbb{F})\cong \bigoplus_{l=0}^k\bigoplus_{j_1+\cdots j_{h-1}=l}H_{*-l}(B_{k-l}(\mathbb{RP}^2);\mathbb{F}),$$ from which it follows that $$\dim H_*(B_k(N_h);\mathbb{F})=\begin{cases}
\binom{h+*-2}{h-2},\qquad &*\in\{0,1,2, k+1\},\\
\binom{h+*-2}{h-2}+\binom{h+*-5}{h-2},\qquad &3\leq *\leq k,\\
0,\qquad &\text{else.}
\end{cases}$$

\subsection{Open and closed M\"{o}bius band} Let $\mathbb{M}$ denote the closed M\"{o}bius band. Then since $\mathbb{M}$ has the same compactly supported cohomology ring as the Klein bottle, our earlier calculation shows that 

$$\widetilde{H}_*(B_k(\mathbb{M},\partial \mathbb{M});\mathbb{F}^w)\cong\begin{cases}
\mathbb{F}[k]\oplus \mathbb{F}[k+1],\qquad &k\text{ odd,}\\
0,\qquad\qquad &k\text{ even.}
\end{cases}$$ On the other hand, $H_c^{-*}(\mathbb{M};\mathbb{F}^w)=0$ by Poincar\'{e} duality, so that $H_c^{-*}(\mathbb{M}; {\mathrm{Lie}}(\mathbb{F}^w[1]))\cong H^{-*}(\mathbb{M};\mathbb{F})[2]$ is abelian, and $\widetilde{H}_*(B_k(\mathbb{M},\partial \mathbb{M});\mathbb{F})$ is the weight $k$ part of $$\mathbb{F}[\tilde x]\otimes\Lambda[\tilde y],\quad |\tilde x|=2,\,|\tilde y|=3,$$ whence $$\widetilde{H}_*(B_k(\mathbb{M},\partial \mathbb{M});\mathbb{F})\cong\begin{cases}
0,\qquad & k\text{ odd,}\\
\mathbb{F}[k]\oplus\mathbb{F}[k+1],\qquad & k\text{ even.}\\
\end{cases}$$

The situation with the corresponding open manifold is quite different. Since $(\mathring{\mathbb{M}})^+\cong\mathbb{RP}^2$, $H_c^{-*}(\mathring{\mathbb{M}};\mathbb{F})=0$, so that $$H_*(B_k(\mathring{\mathbb{M}});\mathbb{F}^w)=0$$ for all $k>1$. On the other hand, $H_c^{-*}(\mathring{\mathbb{M}};\mathbb{F}^w)\cong\mathbb{F}[-1]\oplus\mathbb{F}[-2]$ by Poincar\'{e} duality, so that $H_*(B_k(\mathring{\mathbb{M}});\mathbb{F})$ is the weight $k$ part of $$\mathbb{F}[x]\otimes\Lambda[y],\qquad |x|=0,\,| y|=1,$$ whence $$H_*(B_k(\mathring{\mathbb{M}});\mathbb{F})\cong \mathbb{F}[0]\oplus\mathbb{F}[1]$$ for all $k\geq1$.

\begin{thebibliography}{100}
\bibitem[Ada66]{Adams} J. Adams. On the groups $J(X)$-IV. Topology 5 (1966) 21-71.
\bibitem[Arn69]{Arnold} V. Arnold. The cohomology ring of the group of dyed braids. Mat. Zametiki 5 (1969), 227-231.
\bibitem[Art47]{Artin} E. Artin. Theory of braids. Ann. Math. 48 (1947) 101-126.
\bibitem[AF14]{AF} D. Ayala and J. Francis. Poincar\'{e}/Koszul duality. In preparation.
\bibitem[AFT12]{AFT} D. Ayala, J. Francis, and H. Tanaka. Structured singular manifolds and factorization homology. arXiv:1206.5164.
\bibitem[Bod87]{Bodigheimer} C.-F. B\"{o}digheimer. Stable splittings of mapping spaces. Lecture Notes in Math. Vol. 1286, Springer-Verlag, Berlin-New York, 1987, 174-187. 
\bibitem[BC88]{BC} C.-F. B\"{o}digheimer and F. Cohen. Rational cohomology of configuration spaces of surfaces. Lecture Notes in Math., Vol. 1361, Springer-Verlag, Berlin-New York, 1988, 7-13.
\bibitem[BCT89]{BCT} C.-F. B\"{o}digheimer, F. Cohen, and L. Taylor. On the homology of configuration spaces. Topology 28 (1989) no. 1, 111-123.
\bibitem[Chu12]{Church} T. Church. Homological stability for configuration spaces of manifolds. Invent. Math. 188 (2012) 2, 465-504.
\bibitem[CEF12]{CEF} T. Church, J. Ellenberg, and B. Farb. FI-modules: a new approach to stability for $S_n$-representations. arXiv:1204.4533.
\bibitem[Coh95]{Cohen} F. Cohen. On configuration spaces, their homology, and Lie algebras. J. Pure Appl. Algebra 100 (1995), no. 1-3, 19-42.
\bibitem[CLM76]{CLM} F. Cohen, T. Lada, and J. P. May. The Homology of Iterated Loop Spaces. Lecture Notes in Math. Vol. 533, Springer-Verlag, Berlin-New York, 1976.
\bibitem[CT78]{CohenTaylor} F. Cohen and L. Taylor. Computation of Gelfand-Fuks cohomology, the cohomology of function spaces, and the cohomology of configuration spaces. Geometric applications of homotopy theory I. Lecture Notes in Math., Vol. 657, Springer-Verlag, Berlin-New York, 1978.
\bibitem[ES45]{ES} S. Eilenberg and N. Steenrod. Axiomatic approach to homology theory. Proc. Nat. Acad. Sci. U.S.A. 31 (1945) 117Ð120.
\bibitem[FH00]{FH} E. Fadell and S. Husseini. Geometry and Topology of Configuration Spaces. Springer-Verlag, Berlin-New York, 2000.
\bibitem[FT03a]{FelixTanre} Y. F\'{e}lix and D. Tanr\'{e}. The cohomology algebra of unordered configuration spaces. arXiv:0311323v1.
\bibitem[FT00b]{FT} Y. F\'{e}lix and J.-C. Thomas. Rational Betti numbers of configuration spaces. Topol. Appl. 102 (2000) no. 2, 139-149.
\bibitem[FHT00]{FHT} Y. F\'{e}lix, S. Halperin, and J.-C. Thomas. Rational Homotopy Theory. Graduate Texts in Math. Vol. 205, Springer-Verlag, Berlin-New York, 2000. 
\bibitem[Fra13a]{Francis} J. Francis. Factorization homology of topological manifolds. arXiv:1206.5522.
\bibitem[Fra13b]{FrancisTangent} J. Francis. The tangent complex and Hochschild cohomology of $E_n$-rings. Compos. Math. 149 (2013) no. 3, 430-480.
\bibitem[FG12]{FG} J. Francis and D. Gaitsgory. Chiral Koszul duality. Sel. Math. (N.S.) 18 (2012) no. 1, 27-87.
\bibitem[GJ94]{GJ} E. Getzler and J. Jones. Operads, homotopy algebra and iterated integrals for double loop spaces. arxiv:hep-th/9403055v1.
\bibitem[Gwi12]{Gwilliam} O. Gwilliam. Factorization algebras and free field theories. Thesis (Ph.D.) Northwestern, 2012.
\bibitem[Hor14]{Horel} G. Horel. Higher Hochschild cohomology of the Lubin-Tate ring spectrum. arXiv:1311.2805v2.
\bibitem[KM13]{MillerKupers} A. Kupers and J. Miller. Homological stability for topological chiral homology of completions. arXiv:1311.5203v1.
\bibitem[LS05]{LS} R. Longoni and P. Salvatore. Configuration spaces are not homotopy invariant. Topology 44 (2005) no. 2, 375-380.
\bibitem[Lur09a]{HTT} J. Lurie. Higher topos theory. Annals of Mathematics Studies, 170. Princeton University Press, Princeton, NJ, 2009.
\bibitem[Lur09b]{Lurie} J. Lurie. Derived algebraic geometry VI: $E_k$-algebras. arXiv:0911.0018.
\bibitem[Lur11a]{LurieX} J. Lurie. Derived algebraic geometry X: Formal moduli problems. Preprint available at math.harvard.edu/$\sim$lurie/papers/DAG-X.pdf.
\bibitem[Lur11b]{HA} J. Lurie. Higher algebra. Manuscript available at math.harvard.edu/$\sim$lurie.
\bibitem[May72]{May} J. P. May. The Geometry of Iterated Loop Spaces. Lecture Notes in Math., Vol. 271, Springer-Verlag, Berlin-New York, 1972.
\bibitem[May97]{MayOperads} J. P. May. Operads, algebras and modules. Operads: Proceedings of Renaissance Conferences. American Mathematical Society, 1997. 
\bibitem[McD75]{McDuff} D. McDuff. Configuration spaces of positive and negative particles. Topology 14 (1975) 91-107.
\bibitem[Ran13a]{ORWstab} O. Randal-Williams. Homological stability for unordered configuration spaces. Quarterly Journal of Mathematics 64 (1) (2013) 303-326.
\bibitem[Ran13b]{ORWspheres} O. Randal-Williams. ``Topological chiral homology'' and configuration spaces of spheres. Morfismos, to appear.
\bibitem[Sal01]{Salvatore} P. Salvatore. Configuration spaces with summable labels. Progr. Math., 196 (2001), 375-396.
\bibitem[SW03]{SalvatoreWahl} P. Salvatore and N. Wahl. Framed discs operads and Batalin-Vilkovisky algebras. Quart. J. Math. 54 (2003) 213-231. 
\bibitem[Seg73]{Segal} G. Segal. Configuration-spaces and iterated loop-spaces. Invent. Math. 21 (1973) 213-221.
\bibitem[Tot96]{Totaro} B. Totaro. Configuration spaces of algebraic varieties. Topology 35 (1996) no. 4, 1057-1067.
\bibitem[Vor98]{Voronov} A. Voronov. The Swiss-cheese operad. 	arXiv:9807037.
\bibitem[Wah01]{Wahl} N. Wahl. Ribbon braids and related operads. Thesis (Ph.D.) Oxford, 2001.
\end{thebibliography}

\end{document}

\documentclass{amsart}
\usepackage{geometry}
 \geometry{
 a4paper,
 total={210mm,297mm},
 left=20mm,
 right=20mm,
 top=20mm,
 bottom=20mm,
 }
\usepackage{graphicx}
\usepackage{amssymb}
\usepackage{natbib}
 
 

\newtheorem{theorem}{Theorem}
\newtheorem{corollary}{Corollary}
\newtheorem{lemma}{Lemma}
\newtheorem{proposition}{Proposition}
\newtheorem{remark}{Remark}
\newtheorem{definition}{Definition}
\newtheorem{assumption}{Assumption}
\newtheorem{note}{Note}
\newtheorem{example}{Example}
\include{preamble-14122004-arxiv}
\begin{document}

\title[Proportional Switching]{Proportional switching in FIFO networks}

\author{Maury Bramson}
\address{School of Mathematics, University of Minnesota, 206 Church St. SE, MN 55455, USA.}
\email{bramson@math.umn.edu}
\thanks{The research of the first author was partially supported by NSF grants DMS-1105668 and DMS-1203201.}
\author{Bernardo D'Auria}
\address{Departmento de Estad\'istica, Universidad Carlos III de Madrid, Calle Madrid 126, 28903 Getafe, Madrid, Spain.}\email{bernardo.dauria@uc3m.es}
\thanks{The research of the second author was partially supported by the Spanish Ministry of Economy and Competitiveness Grants MTM2013-42104-P via FEDER funds;he  thanks the ICMAT (Madrid, Spain) Research Institute that kindly hosted him while developing this project.}
\author{Neil Walton}
\address{Korteweg-de Vries Institute for Mathematics, University of Amsterdam, Science Park 904, 1098 XH Amsterdam, NL.}\email{n.s.walton@uva.nl}
\thanks{The research of the third author was funded by the VENI research programme, which is financed by the
Netherlands Organisation for Scientific Research (NWO).}

\keywords{Proportional Scheduler, BackPressure,  Kelly networks, bandwidth sharing networks, Massouli\'e networks, switch networks, proportional fairness.}

\date{}

\begin{abstract}

We consider a family of discrete time multihop switched queueing networks where each packet moves along a fixed route. In this setting, BackPressure is the canonical choice of scheduling policy; this policy has the virtues of possessing a maximal stability region and not requiring explicit knowledge of traffic arrival rates.  BackPressure has certain structural weaknesses because implementation requires information about each route, and queueing delays can grow super-linearly with route length. For large networks, where 
packets over
many routes are processed by a queue, or where packets over a route are processed by many queues, these limitations can be prohibitive. 

In this article, we introduce a scheduling policy for FIFO networks, the Proportional Scheduler, which is based on the proportional fairness criterion. We show that, like BackPressure, the Proportional Scheduler has a maximal stability region and does not require explicit knowledge of traffic arrival rates.
The Proportional Scheduler has the advantage that information about the network\rq{}s route structure  is not required for scheduling, 
which substantially improves the policy's performance for large networks.
For instance, packets can be routed with only next-hop information and new nodes can be added to the network with only knowledge of the scheduling constraints. 

\end{abstract}

\maketitle

\section{Introduction}
We consider, in this paper, a family of discrete time multihop switched queueing networks where packets move along fixed routes. Switched networks were first introduced in \cite{TaEp92}; this
terminology was first employed in \cite{ShWi12} for a discrete time queueing network whose queues are served simultaneously subject to certain global scheduling constraints and with packets moving along fixed routes. 
Here, we consider a variant of this model that allows more than one route through
each queue; the scheduling constraints and queueing discipline employed here do not depend on the routes of the packets.

 Applications of switched networks include 
 wireless ad-hoc networks,
 Internet routers, call centers with cross trained staff,
 data centers, and
 urban road traffic scheduling. 
For such applications, and in general,
maximum stability is a highly desirable feature for the scheduling policy.
Roughly stated, for switched networks, a scheduling policy is maximally stable if, for every arrival rate for which there exists a stable policy, the policy stabilizes the network.
When the vector of arrival rates is known, one can specify a stable policy
by choosing a random schedule whose average service rate at each queue dominates the corresponding arrival rate.
However, in practice, explicit knowledge of arrival rates is often not available, particularly when rates may vary. Since the seminal work of \cite{TaEp92},  BackPressure has been the canonical maximally stable scheduling policy 
for multihop switched networks. An important feature of this policy is that, in addition to it
being maximally stable, only information on the current state is required for a scheduling decision.

When packets are routed through different network components, policies such as BackPressure require detailed knowledge of this routing.  Such policies are similar in nature to the classical single class Jackson queueing networks if one identifies each route with a queue, so that each queue now contains only a single customer type. However, in many applications, this single class interpretation may not be practical  because the information required can increase rapidly in the number of queues of the original network.  It is also not common practice --
for instance, an Internet router will maintain a first-in, first-out (FIFO) queue for each outgoing link rather than a queue for each route, because the number of links to and from a router is orders of magnitude smaller than the number of routes it  processes. 
As a result, in many practical situations, the model simplifies if one reinterprets 
queues as being multiclass, i.e., permitting different packet classes and hence different routing of packets passing through the queue. 

In our setting, queues will be multiclass but will serve packets according to a scheduling policy allocating service among queues that depends only on the number of packets at each queue; 
the discipline at each queue will be FIFO and packets will all be of unit size. 
In this setting, we will show that this switching policy, the Proportional Scheduler, is maximally stable.

There are well-known examples of disciplines that are not maximally stable 
for multiclass queues. See, for instance, \cite{LuKu91}, \cite{RySt92} and \cite{Br94} for instability examples that can easily be adapted to switched networks.  The examples in \cite{Br94} are for the FIFO discipline, but our setting does not fall within this framework since all packets require an equal amount of service.

Our policy, the \emph{Proportional Scheduler}, can be described roughly as follows: for a set of FIFO queues $\mJ$, a vector of queue lengths $(Q_j: j\in\mJ)\in \bZ_+^{|\mJ|}$, and a convex set of schedules $ <\!\mS \!> \subset \bR^{|\mJ|}$, the Proportional Scheduler serves packets according to a vector of expected rates,  $\vecsigma \in \bR^{|\mJ|}$, that solves the proportional fair optimization problem
\begin{align}\label{IntroPF}
\text{maximize}\qquad \sum_{j\in\mJ} Q_j \log \sigma_j \qquad
\text{over}\qquad  \vecsigma \in <\! \mS \! >.
\end{align}
Packets within a queue are to be served according to a FIFO queueing discipline; once served, a packet goes to the next queue on its route.

We  prove that maximal stability holds for this policy by employing a fluid model analysis of our system. This combines the approaches of \cite{Ma07}, for bandwidth networks, and \cite{Br96a}, for FIFO Kelly networks; the latter networks can be viewed as the special case of switched networks where the allocation of
service to different queues is fixed irrespective of the state of the system.

We also compare our policy to the BackPressure policy. Unlike BackPressure, the Proportional Scheduler does not require knowledge of the route used by each packet. Thus, routing structure and scheduling decisions are distinct
for the Proportional Scheduler policy. 
As we later discuss in more detail, there are various benefits of the Proportional Scheduler's functional structure: 
a) rather than distinguishing packets according to their routes (either in the memory of the scheduling algorithm or by physically maintaining different queues), packets can be served at their queue in a first-in, first-out order, 
b) a packet can be routed knowing only its next hop, rather than knowing its entire route,
c) when adding new network components, one only needs to know the component's scheduling constraints and not the entire route or class structure of the network to implement the policy, and
d) for BackPressure, messages must be sent between queues, which the Proportional Scheduler does not require. 

\iffalse
The above observations focus on the functional properties of the Proportional Scheduler. There is a rich vein of work connecting the proportional fair optimization with reversible queueing systems; we compare the stochastic structure of the stationary distribution induced by the Proportional Scheduler with certain quasi-reversible queueing networks.
  Quasi-reversible processes over product state spaces often have a product form stationary distribution. When scheduling constraints are independent -- that is when the constraints of a number of queues do not restrict the schedule for the remainder of the network -- we argue that the queue sizes for distinct components should be, at least approximately, statistically independent. Thus, we can postulate that these observations hold for the Proportional Scheduler.

The Proportional Scheduler is based on the proportional fair optimization (\ref{IntroPF}) introduced in \cite{Ke97}; this optimization is known to have a close relationship with reversible queueing networks, see, e.g.,  \cite{Sc79}. 
Employing this connection, we wish to show in a future paper that, if the schedule of each network component does not restrict the schedule used by the remainder of the network then, in the heavy traffic limit,  the components of the network will be stochastically independent.
We briefly discuss this relationship and some of its implications in the current paper.  
\fi

\iffalse
We discuss the Proportional Scheduler in comparison to the BackPressure policy of \cite{TaEp92}. Like the Proportional Scheduler the BackPressure policies are maximally stable and only current queue state information is required to make a scheduling decision. However, the queue state information required by BackPressue is significantly larger than that of the Proportional Scheduler

[DESCIRBE PF POLICY AND FIFO BRAMSON AND THEN GO BACK TO BACK PRESSURE ]

 For switched networks where packets are served along fixed routes, we prove the maximum stability property for a switched network of FIFO queues which are scheduled under a certain proportional switch policy.
In comparison with existing policies, multiclass queueing substantially reduces the queueing overhead of our switched policy. 

For switched queue networks, the BackPressure policies are the canonical scheduling algorithm.
These policies, first described by \cite{TaEp92}, are maximally stable and only current queue state information is required to make a scheduling decision. 
For this reason the BackPressure policies have proven to be extremely popular.
 
By design, the BackPressure policy minimizes the drift of a Lyapunov function.
As a consequence, the BackPressure policies are class-based priority policies; they must maintain a queue at each node for each route-class. Thus, as mentioned above, the queueing structure maintained at a node implementing the BackPressure policy grows substantially with a network's size. 
[<-- MOVE THESE SENTENCES INTO MAIN TEXT SECTION]

Further the BackPressure policies must exchange information on queue lengths between queues to implement the policy.  In addition to the required information exchange between queues, an artefact of this is that the queueing delays typically grow quadratically with the length of a route, see \cite{BSS11}. These points are discussed later in Section ??.

In this paper, we consider a multiclass FIFO queued switched network.  We consider scheduling policy based on the proportionally fair scheduling. We prove that this scheduler is maximally stable for FIFO switched networks.  The proportional scheduler can be described roughly as follows. For a set of FIFO queues $\mJ$, a vector of queue lengths $(Q_j: j\in\mJ)\in \bZ_+^{|\mJ|}$ and, allowing for randomisation, a set of schedules $ <\!\mS \!> \subset \bR^{|\mJ|}$, the proportionally fair optimization is the following
\begin{align}\label{IntroPF}
&\text{maximize}&&\sum_{j\in\mJ} Q_j \log \sigma_j
&\text{over} &&  \vecsigma \in <\! \mS \! >.
\end{align}
The mean number of packets served by the proportional scheduler solves this optimisation.  Packets within a queue are served according to a FIFO queueing discipline. Once served a packet goes to the next queue on its route.
As we will discuss, there are close connections between this scheduler and reversible queueing networks.
Notice to implement the policy just described, the scheduler does not need to know the route used by each packet. Routing decisions and scheduling decisions are separated.

The main result of this paper, Theorem \ref{mainthrm}, proves the maximum stability for proportional scheduler in FIFO switched networks with fixed routing. As we have discussed and will discuss in more detail in the paper, there are several important consequences to this result: [DONT ENNUMERATE KEEP IN LINE]
\begin{enumerate}
\item
By allowing for FIFO multiclass queueing, our policy better separates routing and scheduling structure of the network. For instance, a queue can be maintained for each link of a network component rather than each route processed by a network component. 

\item 
We do not need to know the entire route of a packet. Packet's can be routed by only knowing the next hop of a packet. 

\item 
We can more easily add new network components. We only needs to be aware of the components scheduling constraints but its does not need to entire the route-class structure of the network to implement the policy.

\item 
Independent networked components can be decomposed. If the schedule of one network component does not effect the schedule used by another part of the network then the optimization \eqref{IntroPF} is separable, and thus can be solved in parallel. This is compared to BackPressure where messages must be sent between these components and hence introduces functional dependence.

\item 
The network is approximately quasi-reversible and hence approximately product form. This we describe in more detail in Section ??. This implies that network components that can function independently are approximately statistically independent. This implies that delays grow linearly along long routes of independent functioning components, compared with BackPressure where delays grow quadratically.
\end{enumerate}

\fi

\subsection{Relevant Literature}
The main results of this paper combine a number of results, methods, and models from the theory of queueing networks over the last three decades. We review this relevant literature  in a roughly chronological order.

\emph{A. Classical Queueing Networks:} The development of Jackson networks is one of the earliest substantial developments in the theory of stochastic networks.
In Jackson networks, a single class of customers is routed probabilistically between queues in the network.
Influenced by the Input Theorem of \cite{Bu56}, \cite{Ja63} found that the stationary distribution of these networks can be written in product form, meaning that its stationary distribution is the product of simple terms. 

\cite{BCMP75} and \cite{Ke75,Ke79} 
broadened this family of queueing networks by permitting queues to be multiclass, and hence allowing more than one route through each queue. 
Using quasi-reversibility, they showed that, for certain service disciplines including FIFO, the stationary distribution must be of product form.  (Quasi-reversibility will 
also be
employed in the context of proportional fairiness in Part D below.)

\emph{B. Switched Networks, BackPressure and their Applications:} Switched queueing networks and, more specifically, the BackPressure scheduling policies, were first introduced by \cite{TaEp92} as a model of wireless communication. As mentioned above, a switched queueing network is a discrete time queueing network with constraints on which queues can be served simultaneously.
The BackPressure policies have proven popular because they maximize a network's stability region while not requiring explicit estimation of traffic arrival rates;
for a comprehensive review of the BackPressure policies, see \cite{GNT06}.

The BackPressure policies have been generalized and specialized in numerous directions. 
In contrast to our work, the defining feature of a BackPressure policy is that the policy minimizes the drift of a Lyapunov function subject to the scheduling constraints of the network. For single hop networks -- where packets are served only once before departing the network -- the BackPressure policy is often referred to as the MaxWeight policy.
As a model of Internet protocol routers, \cite{MMAW99} applied this paradigm to the example of input-queued switches.
 \cite{AKRSVW04} consider power functions when defining their MaxWeight Lyapunov function and thus generalized the set of MaxWeight/BackPressure policies.
Additional extensions are considered by \cite{Me09} and \cite{ESP05} and further generalizations to cone polices are considered by \cite{ArBa03}.
In different stochastic senses, BackPressure and MaxWeight can be shown to optimize certain workload functions:
 in heavy traffic, see \cite{St04}; in large deviations, see \cite{Ve07}; in overload, see \cite{ShWi11}. Further, heavy tailed arrivals are analyzed by \cite{JMMT11}.
   
 
There are numerous application areas associated with switched networks. For these areas, BackPressure is often the canonical choice. Applications  include
 wireless ad-hoc networks, in \cite{TaEp92},
 Internet routers, in \cite{MMAW99},
 call centers with cross trained staff, in \cite{MaSt04},
 data centers, in \cite{ShWi11},
 urban road traffic scheduling, in \cite{Va13}, and
 stochastic processing networks, which would include numerous manufacturing and general processing settings, in \cite{DaLi05}.

\emph{C. Instability, Fluid Stability and Stability of Queueing Networks:}
Most classical queueing networks are positive recurrent when the network is subcritical.  By subcritical, we mean that each network resource experiences a load that is strictly less than the resource's capacity. It had been thought that subcritical networks were always stable under a work conserving policy. However, a series of examples constructed in the mid-nineties showed that this is not the case.   For instance, see \cite{LuKu91}, \cite{RySt92} and \cite{Br94}.

This led to new approaches for determining the stability region for queueing networks. In particular, \cite{RySt92} and \cite{Da95} developed a fluid model approach where the stability of a queueing network can be determined from that of an associated fluid model. This theory is surveyed in by \cite{Br08}.
A fluid analysis of multiclass FIFO queueing networks was first given by \cite{Br96a}. Our fluid analysis uses a similar approach.

Recent work of \cite{DiSh13} further considers the issue of finding natural scheduling policies that lead to stability whenever each server is nominally underloaded. Similar to adversarial queueing frameworks, e.g., \cite{BKRSW01}, stability is achieved by prioritizing queue service according to a least-routed-first-priority discipline.  
This differs from the approach taken in this paper where the service discipline does not use the routing structure of the network  to achieve maximum stability.
Further recent work of \cite{JJS13} considers stability results for switched networks where packets are queued per-link rather than per-route, as is typically applied for BackPressure. Here stability is achieved by running an appropriate {queueing system} in the memory of the algorithm. By estimating queue sizes and loads in this way, stability is achievable for switched networks as long as routes do not form a loop. Once again this differs from the approach of this paper, where routes are general and only current queue size information is required to execute the scheduling policy.

\emph{D. Massouli\'e Networks, Proportional Fairness and Quasi-Reversibility:}
A further class of  Internet models was introduced in \cite{MaRo99} and
\cite{massoulie2000bandwidth}
in a processor-sharing framework where resources are shared subject to constraints on these resources. 
Similar to MaxWeight and BackPressure, these policies are often defined by an optimization that is maximally stable.  (Unlike BackPressure, the construction does not minimize the instantaneous drift of a Lyapunov function.) Stability proofs for these systems can be found in \cite{BoMa01}, \cite{Ye05}, \cite{GW09} and \cite{PAFA12}. Due to their proliferation to different areas, these models have taken various names, such as bandwidth sharing networks, stochastic flow level models, and resource sharing networks; here, we refer to these networks as \emph{Massouli\'e networks}. See \cite{HMSY14} for a recent discussion of the benefits and varied applications of this resource sharing paradigm.

In this paper, we allocate resources according to the proportional fair optimization, 
which was first introduced by \cite{Ke97}. Proportional fairness has been 
used in the allocation of bandwidth in modern 3G telephone networks, see, e.g., \cite{VTL02} and \cite{KuWh04}.  The stability of proportional fairness in Massouli\'e networks was first shown by \cite{DLK99}; an important generalization of this stability analysis is given in \cite{Ma07}. Further progress on the stability and large deviations behavior of proportional fairness can be found in \cite{JoLo14}. Heavy traffic analysis of proportional fair policies can be found in \cite{KKLW07i}, \cite{Ye12}, and \cite{VZZ14}. \cite{St04} has investigated resource pooling for MaxWeight policies. \cite{KKLW07ii},  \cite{KKLW07i}, \cite{KMW09} and \cite{Ye12} discussed product form resource pooling properties associated with the proportional fairness in heavy traffic and large deviations regimes.

Influenced by \cite{Wh85}, the quasi-reversibility and insensitivity property in Massouli\'e networks was studied by \cite{BoPr02,BoPr03,BoPr04} and \cite{Za07}. 
For connections between proportional fairness and the queueing networks of \cite{Ke75} and \cite{BCMP75}, see, e.g., \cite{Sc79,Ke89,MaRo99,Wa09}, and \cite{MOR13}.
A short, general description of the relationship between proportional fairness, maximum stability and quasi-reversibility can also be found in \cite{Wa11B}.
 As we discuss later, the Lyapunov function in \cite{Ma07} is relevant to our analysis. 

\emph{E. Resource Sharing in Switched Networks:}
The sharing of network resources is a key property of the proportional fair optimization. Other resource sharing policies exist, e.g., the weighted $\alpha$-fair policies of \cite{MoWa00}.  Only recently, authors have begun to consider these policies in the context of switched networks; 
to the best of our knowledge, application of $\alpha$-fairness to switched networks was first made by \cite{ShWi11} and \cite{Zh12}. 

\iffalse
In recent years, there has been much progress on the analysis of decentralized throughput optimal algorithms for switched networks. 
In particular, see \cite{JiWa10} and \cite{ShSh12}, where
the proofs are given for single hop networks.  When the components of networks communicate, as in the present paper, further analysis is needed. We hope that the work here will be helpful in extensions from the single hop setting to multihop networks in analyzing networks of communicating components.
\fi

\subsection{Organization} 
The remainder of the paper is organized as follows. In  Section \ref{Model}, we define a family of FIFO switched networks and the Proportional Scheduler. The
main result of the paper, Theorem \ref{mainthrm}, states that the
corresponding network is positive recurrent for all subcritical arrival rates. In Section \ref{Comparison}, we discuss the properties of the Proportional Scheduler in comparison to BackPressure; the
section is not needed to understand Theorem \ref{mainthrm}, but it is important in order to understand its consequences.  
In Section \ref{PROOF}, we prove Theorem \ref{mainthrm}.  We begin by characterizing
the fluid model that is associated with the Proportional Scheduler and then,  in (\ref{def:L.fun}-\ref{def:entropy}) and Proposition \ref{HDiff}, define a Lyapunov function for the fluid model and calculate its derivative.    This is applied, in Theorem \ref{FluidStable}, to prove fluid stability,  from which we conclude, in Proposition \ref{propFMQN}, that the corresponding
stochastic network is positive recurrent.  Proposition \ref{propFMQN} and certain other steps in the proof of Theorem \ref{mainthrm} will be proved in the appendix.

\section{FIFO Network Model, Proportional Scheduler, and Main Result}\label{Model}

\subsection{Network and Scheduling Set Notation}
\label{subsection2.1}
Let $\mJ$ be a finite set of queues, with cardinality $|\mJ|$ and  indexed by $j$.
A \emph{schedule} is a vector $\sigma=(\sigma_j : j\in\mJ) \in \bZ_+^{|\mJ|}$, where $\bZ_+^{|\mJ|}$ denotes the non-negative integers.  We will denote by
$\mS$ a finite set of schedules satisfying: (1) If $\vecsigma\in\mS$, then
$\tilde{\vecsigma}\in\mS$ for each $\tilde{\vecsigma}\leq\vecsigma$   
(with vector inequalities  $\tilde{\vecsigma}\leq\vecsigma$ being interpreted componentwise, i.e., $\tilde{\sigma}_j \leq \sigma_j$ for all $j\in\mJ$).  Note that
the vector of all zeros belongs to $\mS$.
(2) For each $j\in\mJ$, there exists some schedule $\vecsigma\in\mS$ such that $\sigma_j>0$.    For a vector $\vecQ=(Q_j: j\in\mJ) \in \bZ_+^\mJ$, we denote by $\mS_Q$ the schedules in $\mS$ with $\sigma_j \leq Q_j$ for $j\in\mJ$, and 
denote by $\sigma_{\max}:=\max\{ \sigma_j : j\in\mJ, \vecsigma\in\mS \}$
the \emph{maximum component} in the set of schedules.
We define $\coS$ to be the convex combination of points in $\mS$ and assume that $\coS$ has non-empty interior.
The \emph{subcritical region} $\mC$ of this network is the interior of $\coS$.

Each packet in the network is assumed to belong to a class at a given time $t$. The notation of a packet\rq{}s class will be used to uniquely identify the route of the packet, the queue it is at, and the stage along its route.
We denote by $\mK$ the set of classes of packets in the network.
A route through the network is a vector of classes
$r=(k^r_i : i=1,\ldots,|r|)\in\mK^{|r|}$, with size $|r|\in\bN$. Each class is assumed to occur along a unique route, with the class occurring  exactly once along its route; $\mR$ denotes the set of routes through this network.

With each class $k\in\mK$, we associate a unique route denoted $r(k)$ and a unique queue $j(k)$. For notational convenience, we add an additional ``outside\rq{}\rq{} class denoted by $\out$:  For each class $k\in\mK$, we let the function $b(k) \in\mK \cup \{\out\}$ denote the class \emph{before} class $k$ on route $r$; if $k$ is the first class on a route, we then set $b(k)=\out$. Similarly, the function $n(k) \in\mK \cup \{\out\}$ will denote the \emph{next} class on route $r$. If $k\in\mK$ is the last class on a route, then $n(k)=\out$. 

For a given route $r\in\mR$, we define the input class $i(r)\in\mK$ to be the first class on route $r$, i.e., $i(r)=k^r_1$, and the output class $o(r)\in\mK$ to be the last class on route $r$, i.e., $o(r)=k^r_{|r|}$. The subsets of input and output classes are denoted by $\mK^i$ and  $\mK^o$, respectively.
For notational convenience, we write $k \in j$ to indicate that class $k$ is at queue $j$, i.e., $j(k)=j$, and $k\in r$ to indicate that class $k$ occurs on route $r$, i.e., $r(k)=r$. Unless stated otherwise,  $| \cdot |$ denotes the $L^1$ norm.

\iffalse
\begin{remark}\label{ScheduleRemark}
\NW{In the model defined above we assume, for concreteness, that a schedule is deterministic. However, we note that a straight-forward extension would allow random service. Such a setting would be of interest for wireless communication networks, where packets transmitted by the sender may fail to transmit due to interference. Here each schedule would be a bounded random variable on $\bZ_+^{|\mJ|}$ with mean $\sigma$. We would then let the set $\mS$ index the set of mean values for these random variables. Our positive recurrence result, Theorem 1, holds as since the functional law of large numbers and thus our fluid model and fluid stability results which apply in this case.}
\end{remark}
\fi

\iffalse
If $x\in\mJ$, it denotes the amount of packets arrived at queue $x$,  in case $x\in\mR$, it gives  the amount of packets arrived at route $x$ and, finally, if $x\in\mK$, it is the amount of arrived packets of class $x$.
In general to easy the notation for the previous three cases we are going to use the indexes $j\in\mJ$, $r\in\mR$ and $k\in\mK$, having $A_j(t)$,  $A_r(t)$ and $A_k(t)$, respectively.
\fi

\subsection{Network Quantities and Equations}\label{FIFOeq}

Our principal objects of interest are discrete time FIFO networks and the
Proportional Scheduler, which are described here and in the next subsection.  We begin by introducing the state primitives for first-in, first-out (FIFO) networks having time index $t\in\bZ_+$.  Analogous primitives and equations will be employed in Section
\ref{PROOF} for the corresponding fluid model, where time will instead be continuous.  

Throughout the paper, the indices $j, k,r$ will be used to refer to, respectively, queues $\mJ$, classes $\mK$ and routes $\mR$. Regardless of the index $x=j,k,r$, we will denote by $A_x(t)$ the cumulative number of arrivals by time $t$, by $D_x(t)$ the cumulative departures by time $t$, and by $Q_x(t)$ the queue size at time $t$.
For example, $A_j(t)$ is the total number of arrivals at queue $j$ by time $t$, $D_k(t)$ is the total number of packets that have departed from class $k$, and $Q_r(t)$ is the number of packets that have arrived at but not departed from route $r$.
For each  $k\in\mK$ and $s\in\bN$, the function $\Gamma_k(s)$ denotes the number of packets of class $k$ that will be served after $s$ jobs are served from queue $j(k)$.

The processes $A_x(t), D_x(t)$\text{ and }$\Gamma_k(t)$ are non-negative and non-decreasing, with $A_x(0) = D_x(0)=\Gamma_k(0) =0$ for $x\in \mJ\cup \mK\cup \mR$ and $k\in\mK$.  The queue size process $Q_x(t)$ is non-negative for $x\in \mJ\cup \mK\cup \mR$. Given these natural conditions, the following fundamental equations hold for a FIFO switched network.
\begin{align}
& Q_x(t) = Q_x(0)+ A_x(t) - D_x(t), \label{eq:Incr}\\
 
&\sum_{k \in j} \Gamma_k(t) = t, \label{eq:gammak} \\
\Big( &\frac{D_j(t) -D_j(s)}{t-s} : {j\in\mJ}\Big) \in \coS   \label{eq:Dcov} \quad \text{for } t>s, \\
&D_k(t) = \Gamma_k(D_j(t)), \label{eq:Dk}\\%\label{eq:Ak}
&A_k(t) + Q_k(0) = \Gamma_k(A_j(t) + Q_j(0)), \label{eq:Ak}\\ &A_k(t) = D_{b(k)}(t), \label{eq:ADk}
\end{align}
where $x\in \mJ\cup \mK\cup \mR$, $j\in\mJ$, $k\in\mK$ and $r\in\mR$ for (\ref{eq:Incr}-\ref{eq:Ak}), and $k\in\mK \backslash \mK^i$ in \eqref{eq:ADk}.
Finally, we define the arrivals/departures for routes by 
\begin{align}\label{eq:ADr}
A_r(t) &= A_{i(r)}(t), \qquad\qquad D_r(t) = D_{o(r)}(t).
\end{align}

The above equations (\ref{eq:Incr}-\ref{eq:ADk}) can be interpreted as follows: (\ref{eq:Incr}) is standard, \eqref{eq:gammak} states that the $t$th job served from queue $j$ is from class $k\in j$, and \eqref{eq:Dcov} states that the departure process must be achievable within the constraints of the network scheduling. 
In \eqref{eq:Dk}, $\Gamma_k(D_j(t))$ is the number of class $k$ jobs served after $D_j(t)$ units of service;
\eqref{eq:Ak} states that all of the packets that were originally at class $k$ or arrived there by time $t$ will have been served after $A_j(t) + Q_j(0)$ packets have been served at  $j$ and so, if the $A_j(t)$th packet arrival is of class $k$, then the $(A_j(t) + Q_j(0))$th packet departure is also of class $k$, i.e., the queueing discipline is FIFO; \eqref{eq:ADk} gives the routing between classes.
Further relationships can also be deduced from the above equations. For instance, \eqref{eq:gammak} and \eqref{eq:Ak} imply
$A_j(t) = \sum_{k \in j} A_k(t)$ and 
 $ D_j(t) = \sum_{k \in j} D_k(t)$, 
and \eqref{eq:Incr}, \eqref{eq:ADr} and \eqref{eq:ADk} imply
$Q_r(t) = \sum_{k\in r} Q_k(t)$.

We remark that, when the service rates $\sigma_j$ are constant, the FIFO property given by \eqref{eq:Ak} is equivalent to the FIFO property (2.5) in \cite{Br96a}. The FIFO condition there is given in terms of the workload at the queue; such an interpretation of workload does not immediately transfer to our setting, since service rates can vary.   Also note that (\ref{eq:Ak})
does not restrict the order in which packets that have arrived simultaneously at a queue are served; we shall allow any such order.

In addition to equations (\ref{eq:Incr}-\ref{eq:ADk}), we also assume that the number of arrivals at each route $r$ over different times is i.i.d. with mean $a_r\in (0,\infty)$, and that the arrivals at different routes occur independently.  
For each class $k\in r$, we set $a_k = a_{r}$, and  for each queue $j\in\mJ$, we set $a_j = \sum_{k \in j} a_k$, with $\veca=(a_j : j\in\mJ)$.
We assume that the time required for the service of each packet is
deterministic and equal to $1$.  With this in mind, we say that the mean arrival vector $\veca$ is 
\emph{subcritical} for a given network when $\veca \in \mC$;
note that, since $\mC$ is open, this implies that $(1+\epsilon)\veca \in \mC$
for some $\epsilon > 0$. This provides a natural extension to the definition of subcriticality used for multiclass queueing networks (with a fixed allocation of service for each queue).

We also note that, for a network with the FIFO property, one needs to
specify the initial state of the system by including $\Gamma_k(s)$, for $0\leq s \leq Q_j(0)$, in order to uniquely specify the corresponding Markov process.  
As mentioned above, one also needs to define a tie-breaking rule when two or more packets from different classes arrive at a queue at the same time, 
for which we specify in the functions $\Gamma_k$  which packet arrived  ``first".

\iffalse
From our notational convention, it is clear that $A_j(t), A_k(t)$ and $A_r(t)$ each refer to different arrival processes: the cummulative arrivals at queue $j$, the cummulative arrivals at class $k$ and the cummulative arrivals on route $r$.
We let $Q_j(t)\in\bZ_+$ be the queue size of queue $j$ at time $t$. Let $A_j(t)\in\bZ_+$ be the total number of arrivals at queue $j\in\mJ$ by time $t$ -- including external, internal arrivals and packets initial in the queue at time zero. Let $D_j(t)\in\bZ_+$ be the total number of departures from queue $j$ by time $t$. Thus we have that
\begin{equation}
Q_j(t) = A_j(t) - D_j(t).
\end{equation}
Note in the above definition, we define $Q_j(0)=A_j(0)$.
Given the class $k\in\mK$, we let $A_k(t)\in\bZ_+$ and $D_k(t)\in\bZ_+$ be the total number of class $k$ packets to, respectively, arrive and depart from queue $j(k)$.
For a route $r\in\mR$ the number of arrived and departed packets, $A_r(t)$ and $D_r(t)$, is given by the number of arrived packets of class $i(r)$, respectively departed packets of class $o(r)$,  from and to the outside, $\out$. That is, $A_r(t)=A_{i(r)}(t)$ and $D_r(t)=D_{o(r)}(t)$.
\fi

\subsection{Proportional Scheduler}
A scheduling policy is a sequence of schedules $\vecpi(t)\in\mS$ with $\vecpi(t) \leq \vecQ(t)$ component-wise
 that determines the service of packets at each queue, i.e., $\vecD(t)-\vecD(t-1)=\vecpi(t)$. The proportional fair scheduler, which is the main focus of this paper, is defined as follows.
 
For $\vecQ= (Q_j: j\in\mJ)\in\bZ_+^{|\mJ|}$, let $\vecsigma(\vecQ)= (\sigma_j(\vecQ) : j\in\mJ)\in \coS$ be a solution to the following optimization problem:
\begin{align}\label{PFOpt}
&\text{maximize}\qquad\sum_{j\in\mJ} Q_j \log \sigma_j\qquad
\text{over} \qquad  \vecsigma \in <\! \mS_{\vecQ} \! >;
\end{align}
when $Q_j=0$, set $\sigma_j(\vecQ)=0$, with the convention that
$0\log 0 = 0$. (Note that $\vecQ \mapsto \vecsigma(\vecQ)$ is invariant under scalar multiplication, i.e., $\vecsigma_j(c \, \vecQ) = \vecsigma_j(\vecQ)$ for $c>0$.) 
The solution to this optimization need not belong to the set of schedules $\mS$. However, since $\vecsigma \in <\! \mS_{\vecQ} \! >$, $\vecsigma$ can be expressed as a convex combination of points in $\mS_{\vecQ}$, i.e., there exists random $ \boldsymbol{\pi}(\vecQ)= ( \pi_j(\vecQ) : j\in\mJ)$, with support in $\mS$ and such that, for $j\in\mJ$,
\begin{subequations}\label{DServe}
\begin{equation}
\bE \pi_j(\vecQ) = \sigma_j(\vecQ). \label{eq:Dep}
\end{equation}
The Proportional Scheduler is defined to be any policy having the sequence of
schedules $\vecpi (t)$, where
\begin{equation}\label{Dchange}
 \vecsigma(t) = \operatorname*{arg\,max} \quad \sum_{j\in\mJ} Q_j(t-1) \log \sigma_j \quad\text{over} \quad \vecsigma \in <\! \mS_{\vecQ(t-1)} \! >
\end{equation}
\end{subequations}
and $\vecpi (t)$ satisfies the analog of (\ref{eq:Dep}).
(Note that $\vecsigma (t)$ is uniquely defined because of the strict concavity of the objective function in (\ref{PFOpt}) and because $\sigma_j(t)=0$ when $Q_j(t)=0$, although $\vecpi (t)$ need not be uniquely defined (depending on $\mS$).)

We will refer to any discrete time Markov chain satisfying the FIFO switched network equations  (\ref{eq:Incr}-\ref{eq:ADk}), along with the policy in 
\eqref{DServe} and the arrival and service assumptions in the next to last paragraph of  Subsection \ref{FIFOeq},
as a \emph{proportional switched network}.  The state space of the Markov chain is assumed to be that induced by the number of packets at each queue together with their respective ordering within the queue.  (With a slight abuse of terminology, we blur here the distinction between the Markov chain and the underlying switched network.)

\subsection{Main Result}
A standard fact is that, under any policy, a queueing network cannot be positive recurrent when the vector of arrival rates $\veca$ lies outside the network's subcritical region $\mC$.  Hence,
$\mC$ is the network's greatest possible stability region.
The main result in this paper is the following converse, which shows that the Proportional Scheduler achieves its greatest possible stability region.

\begin{theorem}\label{mainthrm}
Suppose that the vector of arrival rates $\veca = (a_j : j\in\mJ) \in \mC$.  Then the corresponding proportional switched network is positive recurrent.
\end{theorem}

The well-known BackPressure models are also positive recurrent for all subcritical arrival rates.
In the next section, we will compare the implementability of the two models.

\iffalse
The processes $\Gamma$, $A$ and $D$, indexed by $k\in\mK$, are sufficient to maintain a FIFO order of service at the queue. From the definitions of $\Gamma$, $A$ and $D$, the following conditions are ensured by FIFO order of service: for $k \in j$ and $t_2 \geq t_1$,
\begin{subequations}\label{eq:A.and.D.deriv}
\begin{align}
A_k(t_2) - A_k(t_1)= \Gamma_k(A_j(t_2)) - \Gamma_k(A_j(t_1)),\\
D_k(t_2) - D_k(t_1)= \Gamma_k(D_j(t_2)) - \Gamma_k(D_j(t_1)),
\end{align}
\end{subequations}
We can then define the state of class $k \in j$ packets at time $t$ with the vector
\begin{equation}
X_k(t)= ( \Gamma_k(s) : s=D_j(t), D_j(t)+1, ..., A_j(t)).
\end{equation}
From this we can define the state of the FIFO queueing networks by
\begin{equation} \label{eq:state.vec}
  X(t)=(X_k(t) :  k \in \mK) \ .
\end{equation}
\fi

\iffalse
\subsection{Markov Description}
The FIFO network equations (\ref{eq:Incr}-\ref{eq:ADk}) are not sufficient to give a Markov description of our queueing network. Note that if two jobs arrive simultaneously at a queue, the equations (\ref{eq:Incr}-\ref{eq:ADk}) do not specify the order that packets arrive in.
It is necessary to specify conditions the yield a Markov queueing process in the prelimit. This is the main point of this section. This is issue does not effect our fluid limit or our fluid model analysis. So, if one assumes a Markov description exists, this section can be skipped on first reading.

We need to specify the order by which packets arrive at each queue. Given a vector of arrivals $\vecalpha = (\alpha_k: k\in \mK)\in\bZ_+^{|\mK|}$ with $\alpha_j:=\sum_{k\in j} \alpha_k$ and $u\in [0,1]$, we define the \emph{multiplexing functions} at node $j\in\mJ$ by functions, $(g_k : k \in j)$,  satisfying
\begin{equation*}
g_k(s ; \vecalpha, \, u) \in \{ 0, 1 \}
\end{equation*}
for $k \in j$ and $s = 1, ..., \alpha_{j}$, and also satisfying
\begin{equation*}
\sum_{k\in j} g_k(s ; \vecalpha, \, u) =1.
\end{equation*}
Basically, the multiplex function, $g_k(s ; \vecalpha, \, u)=1$, indicates that the $s$th arrival in queue $j$ is of class $k$ when a vector of arrivals $\vecalpha$ occurs. The variable $u\in [0,1]$ is a uniform random variable used to perturb the order of packets, if required.

We let $U(t)$ be independent uniform random variables on $[0,1]$ for each time $t\in\bZ_+$. We let $\vecA(t)=(A_k(t) : k\in \mK)$ give the class arrival process. For each $k\in j$, we define the processes $(\Gamma_k(s): k\in\mK)$ by
\begin{equation}\label{multiplex}
\Delta\Gamma_k(A_j(t) + s)= g_k(s; \Delta \vecA_k(t), U(t))
\end{equation}
for $s=1,\ldots,\Delta A_j(t)$.

\iffalse
that returns a vector of size $k_j$, indexed by $k \in j$.
The $k$-th component of such vector is a non-decreasing function,
\begin{equation}
g_k((a_k, k\in j), \, \xi; \,s) \in \bN  \quad  s = 0,1, \ldots, |a|
\end{equation}
such that $\sum_{k \in j} f_k(\ldots;\,s) =s$ and $f_k(\ldots; \, |a|)=a_k$.

The Markovian evolution of the network is defined as follows.

The total number of served packets at time $t$ and at node $j\in\mJ$ are given by
\begin{equation}
D_j(t) =  D_j(t-1) + \vecsigma_{j(k)}(\vecQ(t-1))
\end{equation}
and the ones of class $k\in j$ are
\begin{equation}
D_k(t) = D_k(t-1) + \Gamma(D_{j(k)}(t)) -  \Gamma(D_{j(k)}(t-1)) \ .
\end{equation}
For $k\in\mK^i$, set
\begin{equation}
A_k(t) = A_k(t-1) + \Theta_k(t)
\end{equation}
where $(\Theta_k(t), t>0)$ is a sequence of i.i.d. random variables with mean $a_{r(k)}$
and for  $k\in\mK\setminus\mK^i$  set
\begin{equation}
A_k(t) = A_k(t-1) + D_{b(k)}(t) -  D_{b(k)}(t-1) \ .
\end{equation}
Let, for $j\in\mJ$
\begin{equation}
A_j(t) = \sum_{k \in j} A_k(t)
\end{equation}
\fi

We define the state of our discrete time FIFO queueing network by
\begin{equation}\label{eq:Markov.Chain.X}
\vecX (t)=(\Gamma_k(s) : j\in\mJ, k\in j, s=D_j(t)+1,...,A_j(t)),
\end{equation}
Notice $\vecX$ belongs to countable state space $\mathcal X = \cup_{n\in\bN}\{ 0,1\}^n$. Notice that the next state $\vecX(t+1)$ is a function of $\vecX(t)$ (the previous state), iidrvs $(\Delta A_r(t): r\in\mR)$ (the arrival increments), iidrv $U(t)$ (the random ordering of packet arrivals) and $\vecpi(\vecQ(t))$ (the random scheduling decision) which by assumption is conditionally independent of $\vecQ(t)$. Thus $\vecX(t+1)$ is a function of parameters that are conditionally independent of $\vecX(t)$. This implies $\vecX$ is Markov. In otherwords, we have just proved that the following proposition holds.

\begin{proposition}
Given FIFO network equations  (\ref{eq:Incr}-\ref{eq:ADk}) hold, given multiplexing of packets \eqref{multiplex} holds for iid uniform $[0,1]$ random variables $U(t)$, $t\in\bZ_+$ and given the arrival increments $(\Delta A_r(t) : r\in\mR)$, $t\in\bZ_+$ are iidrvs, the process $\vecX(t)$, $t\in\bZ_+$ is a discrete time countable state space Markov chain.
\end{proposition}
\fi

\section{Comparison with the BackPressure Policy}
 \label{Comparison}\label{BP Compare}
The BackPressure policy is currently  the canonical policy for scheduling in multi-hop switched networks.
Here, we present examples to illustrate practical differences between this policy and proportional fair switching. 
This section is not required for the remainder of the paper.

\subsection{Definition of BackPressure}
We define and briefly describe the BackPressure policy using the notation introduced in Section \ref{Model}, with the reader being referred to \cite{TaEp92} for a more detailed description of the policy and its properties. The network is defined there in terms of a directed graph, but the reader can easily check that the definition we give below is equivalent; this alternative format is employed to avoid unnecessary complications. (Note that  queues $j\in\mJ$ are referred as links in \cite{TaEp92}, and  classes are referred to as queues there.) 

For a given buffer size distribution $\vecQ=(Q_k: \, k\in\mK)$, the BackPressure policy can be defined as follows:
\begin{enumerate}
\item For each queue $j \in \mJ$, introduce the weights 
\begin{equation}\label{BPwOpt}
w_j(\vecQ)= \max_{k \in j} \left\{   Q_k - Q_{n(k)}\right\},
\end{equation}
with $Q_{n(k)}= 0$ for $n(k)=\out$.  
Denote by $k^*_j(\vecQ)$ one of the classes where the above maximization is achieved.
\item Over the set of schedules $\mS$, solve the optimization
\begin{equation}\label{BPOpt}
\sigma^*(\vecQ) \in \operatorname*{arg\,max}_{\sigma\in\mS} \;\; \sum_{j \in \mJ} \sigma_j w_j(\vecQ) \ .
\end{equation}
\item When $w_j(\vecQ)>0$,  schedule $\sigma_j^*(\vecQ)$ packets from class $k^*_j(\vecQ)$ at queue $j \in \mJ$ and no packets from any of the other classes at $j$ during the next time increment; 
when $w_j(\vecQ)\le 0$, do not schedule any packets at the queue.
\end{enumerate}
 
Assuming that the number of  arrivals of packets at each route  over different times is i.i.d. and the arrivals at different routes occur independently, then    
the BackPressure policy is positive recurrent whenever 
the arrival rates are subcritical.
This is shown by employing a quadratic Lyapunov function;
the BackPressure policy, in fact, maximizes the negative drift of
the Lyapunov function. 

\subsection{Queueing Structure}
\label{QueueingS}
The robust stability of the BackPressure policy is a compelling feature. However, a crucial disadvantage of the BackPressure approach is that it leads to a priority policy that requires explicit information about the classes and routes throughout the network; in many practical circumstances, the compilation of such information is not feasible.
\begin{figure}[h!]
  \centering
\includegraphics[width=0.40\textwidth]{Tree-14122004-arxiv.pdf}
\caption{A tree network of degree $d=3$ and diameter $D=6$. There are $12$ leaf nodes and thus $132= 12\cdot 11$ directed routes between leaf nodes. Of these routes, $96=12\cdot 8$ routes pass through the node $j$ (colored grey).
\label{Tree} }
\end{figure}

Consider, for example, a tree network of diameter $D$, with each node having degree $d$, and whose routes are the shortest paths connecting the leaf nodes.  Figure \ref{Tree} depicts a tree of degree $d=3$ and diameter $D=6$; at each node,
the BackPressure policy employs the knowledge of all routes passing through the node.
Focussing on the central (or root) node $j$ for concreteness, and 
denoting by $b(d,D) = |k\in j|$ the number of routes passing through the node,
  it is easy to verify that
\begin{equation}\label{ExBPbound1}
b(d,D)  = d(d-1)^{(D -1)}.
\end{equation}
A more efficient implementation of the BackPressure policy in this setting is possible 
by distinguishing only between the future routing of packets at a node.
Still, under this variant, 
\begin{equation}\label{ExBPbound2}
 b(d,D) = d(d-1)^{({D}/{2}-1)}.
\end{equation}

Each of the above implementations of the BackPressure policy requires
knowledge of a number of routes emanating from each queue that is growing rapidly  with respect to both the
diameter $D$ and the network size.  If implemented,
either policy would present serious memory problems for such trees.
In the general setting, one would expect analogous implementation problems because of the rapid growth of the number of classes needed at each site. 

In contrast to this, for the Proportional Scheduler, each node only requires knowledge of the number of packets destined for its $d$
adjacent nodes (and not the number of packets in each class).
This amount of information is far less than that required by the BackPressure policies  in \eqref{ExBPbound1} and \eqref{ExBPbound2} and 
does not increase as the network size increases, since it
depends only on the local structure of the network and not on, e.g., the
final destination of each packet.
So, although the above example is used for reasons of exposition, 
the Proportional Scheduler is far closer in nature to the next-hop routing used in modern IP routers on the Internet or that might be used in a wireless ad-hoc network. We note, for instance, that modern Internet routers maintain tens of FIFO queues that aggregate tens of thousands of flows (route classes) \cite{mckeown1999islip,appenzeller2004sizing}. Furthermore, when these components are connected together to form a network, queue state information is not explicitly exchanged.

\subsection{Decomposition}
Decomposition is essential to the decentralized implementation of a policy. 
A property of BackPressure optimization is that, once queue length comparisons have been made between links (see \eqref{BPwOpt}), the optimization can be decomposed. In particular, if the scheduling of one subset of queues $\mJ_1$ does not effect the scheduling choice of a complementary subset $\mJ_2$ (i.e., $\mS=\mS_1\times\mS_2$), then the BackPressure optimization can be decomposed as 
\begin{equation*}
\max_{\sigma\in\mS} \Big\{ \sum_{j \in \mJ} \sigma_j w_j(\vecQ) \Big\}= \max_{\sigma\in\mS_1} \Big\{  \sum_{j \in \mJ_1} \sigma_j w_j(\vecQ) \Big\} + \max_{\sigma\in\mS_2} \Big\{  \sum_{j \in \mJ_2} \sigma_j w_j(\vecQ) \Big\}.
\end{equation*}
The sub-problems involving the two optimizations on the right can then be solved independently, leading to a decomposed implementation of the policy. We remark, however, that the above optimization may not completely decompose since the weight calculations $w_j(Q)$ typically require comparisons with the sizes of upstream queues, and hence some information exchange will occur between network components. 

No such queue size comparisons are required for the proportional fair optimization that is employed to define the Proportional Scheduler, and so the optimization can be completely decomposed as
\begin{equation*}
\max_{\sigma\in\coS} \Big\{  \sum_{j \in \mJ} Q_j \log \sigma_j \Big\}=  \max_{\sigma\in <\! \mS_1\!>}\Big\{ \sum_{j \in \mJ_1} Q_j \log \sigma_j \Big\} + \max_{\sigma\in <\! \mS_2\!>} \Big\{  \sum_{j \in \mJ_2} Q_j \log \sigma_j \Big\}
\end{equation*}
for complementary components having 
independent scheduling.
This leads to more potential applications when the network decomposes, in comparison with the BackPressure policy.

\subsection{Delay on Long Routes}\label{delay}
As illustrated by the example in Subsection \ref{QueueingS},
the complexity of the BackPressure policy is an issue when there are many routes.  Other complications might also arise, even when all packets have the same route.
Consider, for example, the network in Figure \ref{Line}, which
is a special case  of the linear networks considered by \cite{BSS11} (in Theorem 2) and \cite{St11}, for the BackPressure policy. 
\begin{figure}[h!]
  \centering
\includegraphics[width=0.95\textwidth]{Spin_Line-14122004-arxiv.pdf}
\caption{A linear network with $J$ links. Each packet must pass through each link.
\label{Line} }
\end{figure}

In this example, there are $J\ge 2$ links in series, with packets entering at link $1$ and being sequentially processed through links $1,2,\ldots,J$;
there are no constraints preventing simultaneous service at different links and,
once served, a packet moves to the next link along its route.
\iffalse
We assume there is a maximum number of allowed failures at a given link, say $\bar q$.\footnote{We remark that, in this setting, we can define BackPressure policy with the probability $1-q_j$ in expression \eqref{BPOpt}. 
Further, with the assumed technical condition, the probabilistic routing in this example is a special case of the routing structure considered in this paper for the proportionally scheduler.}
\fi
Arrivals at queue $1$ occur according to a Bernoulli process, with mean parameter $a>0$, and one packet is served per unit time at each nonempty queue. (Recall that a Bernoulli process is a counting process whose increments are i.i.d. Bernoulli random variables.)
\cite{BSS11} and  \cite{St11} showed that,
 if the network is subcritically loaded with $1/2 <a<1$,  then, under the BackPressure policy,
 the sum of the expected queue sizes in equilibrium grows quadratically in $J$, that is,
\begin{equation}
\label{eqofprop1}
\sum_{j=1}^J \bE \big[ Q_{j} \big] \geq  c \, J^2 
\end{equation}
for some constant $c>0$ not depending on $J$.
\iffalse
\begin{proposition}\label{pr:linear.network.BP}
In a linear network with $J\ge 2$ links, if the network is subcritically loaded and $\alpha_j > 1/2$ for all $j$, then, under the BackPressure policy,
 the sum of the expected queue sizes in equilibrium grows quadratically in $J$; in particular,
\begin{equation}
\label{eqofprop1}
\sum_{j=1}^J \bE \big[ Q_{j} \big] \geq  c \, J^2 
\end{equation}
for some constant $c>0$ not depending on $J$.
\end{proposition}
\fi
Since one unit of time is required to serve each packet at a queue, it is immediate from (\ref{eqofprop1}) that the flow delay for packets satisfies the same lower bound.

The basic idea behind the demonstration of (\ref{eqofprop1}) is that 
BackPressure only serves a packet on the $j$th link when $Q_j  - Q_{j+1} > 0$, which will occur in equilibrium with probability $a$ (since the average arrival and departure rates are equal), whereas, in equilibrium, $Q_j  - Q_{j+1} \ge -1$ must always hold (since service from queue $j$ occurs only when $Q_j - Q_{j+1} > 0$).  Since $a > 1/2$, there will be linear growth in the expected size of successive queues from the last queue $Q_J$ to the first queue $Q_1$, which
implies the quadratic growth in $J$ of the sum of the expected queue sizes.
In general, when routing is more involved,
one should expect the sum of the queue sizes in equilibrium to grow quadratically with route length for routes with close to critical arrival rates since, as before,  
BackPressure will not serve links with a negative queue size differential.
(For details, see \cite{BSS11} or \cite{St11}.)

The Proportional Scheduler  (as well as any other work conserving scheduler)  will exhibit completely different behavior for this linear network since there are no constraints preventing simultaneous service at different links. In equilibrium, queue $1$ will have either $1$ or $0$ packets corresponding to whether or not an arrival occurred in the last time slot. Therefore, the output of a queue is a Bernoulli process that is independent of the current state of the queue. Arguing inductively, this implies that the queue sizes at different queues are independent, and hence that the sum of the queue sizes for the network is binomially distributed with parameters $J$ and $a$. Consequently, under the Proportional Scheduler policy, the sum of the expected value of the queue sizes in equilibrium grows only linearly in $J$, with
\begin{equation}\label{eqofprop2}
\sum_{j=1}^J \bE \big[Q_j \big]   = a J .
\end{equation}
\iffalse
\begin{proposition}\label{pr:linear.network.PF}
In a linear network with $J$ links, if each link is subcritically loaded, then, under the Proportional Scheduler policy, the total queue size of the network in equilibrium grows only linearly with $J$, i.e.,
\begin{equation*}
\sum_{j=1}^J \bE \big[Q_j \big]   \leq c \, J 
\end{equation*}
for some constant $c>0$ not depending on $J$.
\end{proposition}
\fi

The above reasoning that was employed for \eqref{eqofprop2} is an elementary variant of Burke's Output Theorem (as in, e.g., \cite{Bu56} or \cite{HsBu76}). 

\iffalse
The argument can be briefly sketched as follows: The queues in the linear network behave like birth-death processes and hence are reversible. 
Since arrivals at the first queue occur according to a Bernoulli process that is independent of the current state, reversibility implies that the output process of this queue is again an independent Bernoulli process.
By induction, this will imply that, in equilibrium, each queue behaves independently of the others, which will prevent individual queues from growing. 
\fi

\iffalse
\subsection{Product Form Resource Pooling and Heavy Traffic Limits} \label{ProdFormResPooling}

Suppose that a network can be decomposed into two subsets that model different physical  components of a queueing system (e.g., two  input-queued switches). Mathematically, this would imply that the scheduling of the two components are not subject to any common constraints, and thus the set of schedules $\mS$ can be written as the set product $\mS= \mS_1\! \times\! \mS_2$ corresponding to these two components. Intuitively, 
under appropriate limits involving an increasing number of packets in the system, the two subnetworks should behave as if they were stochastically independent.

We refer to this general phenomenon of the independence of components of a system as \emph{product form resource pooling} (PFRP).  This phenomena was demonstrated by \cite{KKLW07i}, in Theorem 5.3, in the heavy traffic setting for proportional fair systems; they observed there that the covariance and reflection angles under heavy traffic limits are equal to those of the product form queueing networks from \cite{BoPr03}, and \cite{MaRo98}. More recently, in the context of single hop switched networks, product form resource pooling has been used to prove optimal queue size scaling of switched networks (see \cite{ShWaZh12}).

We motivate here why the PFRP property should hold for multihop proportional switched networks under heavy traffic limits; a rigorous justification involves work in progress.  We begin by considering \emph{Store-Forward} (SF) networks, 
which were introduced by  \cite{BoPr03}, and
which employ a different scheduler that is easier to analyze but that asymptotically behaves identically to the proportional scheduler.  
In essence, a SF network is a continuous time queueing network whose service rates are determined by the throughput of a closed multiclass queueing network, with fixed capacity, whose parameters depend on the current state of the network.
A detailed description is left to the appendix.  The service rates  at individual queues will be denoted by $\sigma_j^{SF}(\vecQ)$,  $j\in\mJ$.

The following proposition states that SF networks whose components have independent scheduling have a product form stationary distribution.  The result will be
proved in the Appendix.
It is a natural extension of classical product form results to the setting of switched networks. 

\begin{proposition}\label{PFRPProp}
Consider a SF network whose scheduling set is the product of independent scheduling components, i.e.,  $\mS= \mS^1\! \times\! ...\! \times\! \mS^N$, where $\mS$ is the set of schedules for the network and $\mS^n$ are the schedules for its components.  For such a network, the stationary queue size vectors associated with each of these components  are independent.
\end{proposition}
\iffalse
The proof of the above proposition interprets the scheduling set $\coS$  as  a polytope with a finite number of facets:
\[
\coS=\left \{ \sigma\in \bR_+^\mJ : \sum_{j\in\mJ} A_{lj} \sigma_j \leq 1,\, l\in\mL \right\}.
\]
Here, the set $\mL$ indexes the facets or \emph{resource pools} of the scheduling polytope $\coS$ and $A=(A_{lj}: l\in\mL, j\in\mJ)$ is a non-negative matrix. As was first observed by \cite{KKLW07i}, for proportional fair systems, there is an independent random variable associated with each facet of the above polytope and the queue sizes can be determined as a linear function of these. When the scheduling polytope is of product form, i.e., $\mS= \mS_1\! \times\! ...\! \times\! \mS_N$, then the queue sizes in different components are linear combinations of different independent random variables and so are themselves independent.
\fi

It is a simple consequence of Proposition \ref{PFRPProp} that the
heavy traffic limits of SF networks satisfying the assumptions of the proposition have stationary distributions with independent components.
It was first observed by \cite{KKLW07i} that product form results of this type hold for proportional fair systems;
related observations have also been made by \cite{Sc79,Ke89}, and \cite{MaRo99}.
In  \cite{Wa09}, it is shown that the SF policy is asymptotically proportionally fair, with
\begin{equation}\label{LimSF}
\vecsigma^{SF}(c \, \vecQ) \xrightarrow[c\rightarrow\infty]{} \vecsigma(\vecQ) \in \operatorname*{arg\,max}_{\sigma\in\mC} \sum_j Q_j \log \sigma_j \n .
\end{equation}
As a consequence of this and the preceding observation, stationary measures of networks with the Proportional Scheduler optimization whose components have independent scheduling should have, under heavy traffic limits, the same product decomposition as SF networks.  However, this assertion remains to be proved.

As an illustration of this heavy traffic limit decomposition, consider  a network of input-queued switches, which are the queueing architecture used by modern internet routers. Each switch has scheduling constraints given by the set of bipartite matchings, and different switches can be linked together to form a network. Since the scheduling constraints between different routers do not interfere, for the Proportional Scheduler, the queue lengths at the different input-queued switches will be asymptotically independent under a heavy traffic load.

\iffalse
[CORRECT NEXT TWO PARAS]
Notice that, under BackPressure if we increased the processing rate of, say, the first queue in the line network, then, because BackPressure requires a positive queue size differential, the queue size at that node would still be proportional to the length of the queue on step ahead. Thus, increasing capacity in one part of the network will not substantially reduce the delay of packets traversing the line.
By contrast, for the Proportional Scheduler, the independence between components implies increasing capacity in one part of the network will decrease queue sizes without affecting other components of the network. In the next subsection, we discuss this independence of queue lengths in the context of heavy traffic limits.
\fi

\iffalse
\begin{remark}[Independence in Interference Graphs]
The above result would suggest that independence only holds when $\mS$ is a product set. This, however, is not the case. 
We discuss how the argument for independence given for Proposition \ref{PFRP} extends to scheduling on interference graphs. The proofs for the results described here follow more-or-less identically to the proof of Proposition \ref{PFRP}. Similar discussions to this remark can be found in \cite{ShWaZh12}.

The scheduling set of an interference graph corresponds to a graph $G=(\mJ,E)$. Here no node two neighbouring vertices can be scheduled simultaneously and thus the set of feasible schedules are the independent sets of this graph:
\begin{equation}
\mS = \{ \sigma\in\{ 0 ,1 \}^\mJ : \sigma_{j}+\sigma_{j'} \leq 1, (j,j')\in E \}.
\end{equation}
In general, the set $<\mS>$ does not take the same form as above (by which we mean the set $\{0,1\}$ replaced by the interval $[0,1]$). However, if we consider a perfect graph -- meaning that every induced subgraph has a chromatic number equal to induced graph's largest complete subgraph, the set $< \mS >$, or more concretely, by the Strong Perfect Subgraph Theorem, a graph not containing and odd holes or anti-holes of length 5 or more -- such graphs have their stable set, $< \mS >$ is expressable in the form
\begin{equation}
<\mS> =  \bigg\{ \sigma\in\{ 0 ,1 \}^\mJ : \sum_{j\in \mC} \sigma_j \leq 1,\,\, \mC\subset \mJ\text{ is a clique}\bigg\}.
\end{equation}
In otherwords, the resource pools of the network are the set of maximal cliques. We index these be $l\in\mL$.
Examples of perfect graphs include trees, even cycles, bipartite graphs, and square grids. Since under the Store-Forward allocation, stationary queue sizes, $Q_j$, are expressible as the sums of random variables
\begin{equation}
Q_j=\sum_{l\in\mL} M_{lj},
\end{equation}
where the random the vectors $(M_{lj}:j\in\mJ)$ are independent over $l\in\mL$ and geonominal (a multinomial distribution with a geometrically distributed number of trials).  Thus we see that if two nodes are not neighbours, $(j,j')\in\bE$, then they do not share a common clique thus there queue sizes are independent, $Q_j\indep Q_{j'}$.  More generally, for perfect graphs, the queues for nodes in an independent set are mutually independent random variables.

We note that these results do not extend to imperfect graphs when, for instance, the graph contains an odd cycle. For instance a cycle of length $5$ has a stable set given by
\begin{equation}
<\mS> = \{ \sigma\in [0,1]^5 : \sigma_j + \sigma_{j'}\leq 1, (j,j')\in E, \text{ and }  \sigma_1+\sigma_2+\sigma_3+\sigma_4+\sigma_5 \leq 2  \}.
\end{equation}
Due to the final \emph{full rank} constraint for this cycle graph,  all queues are dependent under the Store-Forward policy. The dependence is given by the components of an independent geonomial distribution. See Figure \ref{Graphs} below for  examples. 
\begin{figure}[h!]
  \centering
\includegraphics[width=0.95\textwidth]{Graphs.pdf}
\caption{Four examples of interference graphs: a complete bipartite graph (input-queue switch), a square grid, a triangular grid and a odd length cycle. The first three networks are perfect graphs. In each example, the nodes coloured grey are mutually independent.
\label{Graphs} }
\end{figure}
\end{remark}
\fi
 

\iffalse
\begin{lemma}
If $x$ and $y$ belong to the interval $[0,K]$ for $K>0$ then
\begin{equation}
y\log \left(\frac{y}{x}\right) - (y-x) \geq \frac{1}{2 K} \left(y-x \right)^2.
\end{equation}
\end{lemma}
\proof{Proof}
Consider the function
\begin{equation}
F(z) = z\log z - (z-1).
\end{equation}
Differentiating we see that $F\rq{}(z)= \log z$ and $F\rq{}\rq{}(z) = z^{-1}$. Taking a Taylor expansion around $z=1$ we have that, for some $\theta$ in the interval containing $1$ and $z$,
\begin{align*}
F(z) &= F(z) - F(1) \\
 &= F\rq{}(1) (z-1) + \frac{1}{2} F\rq{}\rq{}(\theta) (z-1)^2\\
& = \frac{1}{2\theta} (z-1)^2 \geq \frac{1}{2 (1\vee z )} (z-1)^2
\end{align*}
The last equality, holds since $\theta \leq (1 \vee z)$. Applying this bound, we observe that
\begin{align*}
y\log \left(\frac{y}{x}\right) - (y-x) &= x F(\tfrac{y}{x}) \\
 &=\frac{x}{2 (1\vee \frac{y}{x} )} \left(\frac{y}{x}-1\right)^2\\
& \geq \frac{1}{2 (x\vee y) } (y-x)^2 \\
& \geq \frac{1}{2 K} (y-x)^2.
\end{align*}
This yields the required bound.
{}\endproof
\fi 

\fi

\section{Maximum Stability Proof} \label{PROOF}
In this section, we prove Theorem \ref{mainthrm}, for which we employ fluid model
techniques introduced in  \cite{Da95}.
The basic idea is to show that the Markov chain converges under appropriate scaling to a deterministic fluid model, 
and then to show that the queue lengths of all normalized fluid model solutions converge to $0$ within
a fixed time.
This latter step requires most of the work in the proof 
and is shown by defining an appropriate Lyapunov function that
decreases to $0$ by this time.  

From a mathematical point of view, the main contribution of this argument is the extension of fluid model techniques to a case with varying service speeds.
Here, this requires identifying the Lyapunov function in \eqref{def:entropy}, showing it to be non-increasing in Corollary \ref{cor:H.non-increasing},
and then showing that this Lyapunov function in fact decreases at a fixed rate by applying Lemmas \ref{bound.change.Q} and \ref{lm:lowerbound.q.sigma}. 
The proofs of certain technical lemmas and propositions are relegated to the appendix.

\subsection{Fluid Model}\label{fluidmodel}

We will employ, in Section \ref{PROOF}, a fluid model, with the functions  
$A_x(t)$, $D_x(t)$, $Q_x(t)$, and $\Gamma_k(t)$, 
that satisfies the same conditions (\ref{eq:Incr}-\ref{eq:ADk}) as the discrete time switched network of Section \ref{Model}, and such that
$A_x(t), D_x(t)$\text{ and }$\Gamma_k(t)$ are non-negative and non-decreasing,
and $Q_x(t)$ is non-negative, with $A_x(0) = D_x(0)=\Gamma_k(0) =0$ for $x\in \mJ\cup \mK\cup \mR$ and $k\in\mK$.
Note that each of the conditions (\ref{eq:Incr}-\ref{eq:ADk}) is invariant under the ``law of large numbers" scaling, where both time and space are scaled equally. 

In addition to these conditions, the route arrival and queue departure processes of the fluid model will be assumed to satisfy
\begin{align}
A_r(t) =& a_r t \label{Fluid:1},\\
Q_j(t) >0 \quad \text{implies} \quad   D_j\rq{}&(t) = \sigma_j ( \vecQ(t))\,, \label{Fluid:3}
\end{align}
for given $a_r > 0$ and for $t>s>0$, $j\in\mJ$ and $r\in\mR$, with $\vecsigma(\vecQ)$ denoting a solution to the proportional fair optimization
\begin{equation}\label{PF}
\vecsigma(\vecQ) \in \operatorname*{arg\,max}_{\vecsigma\in \coS} \,\,\sum_{j\in\mJ} Q_j \log \sigma_j.
\end{equation}
As in Section \ref{Model}, we set $\sigma_j(\vecQ)=0$ whenever $Q_j=0$ and, for each class $k\in r$ and queue $j\in\mJ$, set $a_k = a_{r}$, $a_j = \sum_{k \in j} a_k$, and $\veca=(a_j : j\in\mJ)$.
We will refer to the conditions (\ref{eq:Incr}-\ref{eq:ADk}) and (\ref{Fluid:1}-\ref{PF}) on $\vecA(t)$, $\vecD(t)$, $\vecQ(t)$, and $\vecGamma (t)$ as \emph{proportional switch fluid model equations}, or collectively, as the \emph{proportional switch fluid model} (or \emph{fluid model}, for short).

Together, the above conditions imply with a little work that, for $x\in \mJ\cup \mK\cup \mR $, the components $A_x (t)$, $D_x (t)$, $Q_x (t)$, and $\Gamma_x (t)$ are all Lipschitz continuous and thus almost everywhere differentiable.
Note that $D'_j(t)=0$ need not hold when $Q_j(t)=0$, since work may
enter and leave an empty queue of a fluid model.  Also note that a solution of the fluid model equations corresponding
to a given initial condition need not be the unique such solution.  

\iffalse
\begin{remark}
As in Section \ref{Model}, we employ here the convention that $\sigma_j(\vecQ)=0$ whenever $Q_j=0$.  However, there may be other optimal solutions to this optimization. In particular, it is important to note that, $D'_j(t)$ solves the proportional fair optimization \eqref{PFOpt} but, in general, $D'_j(t)=0$ need not hold when $Q_j(t)=0$. In other words, an empty queue may still be processing work in our fluid model.
\end{remark}
\fi
\subsection{Lyapunov Function and its Derivative}\label{LyaDeriv}

In this subsection, we define the entropy function $H(t)$ and prove that it has negative derivative for non-empty subcritical systems.
We first introduce the functions
\begin{align}
L(t) = 
& \sum_{\jInJ} Q_j(t) \, \log D'_j(t) \label{def:L.fun},  \\
M(t) = 
& \sum_\jInJ \sum_\kInj 
  \int_{D_j(t)}^{Q_j(0)+A_j(t)}
    \Gamma'_k(s) \log \frac{\Gamma'_k(s)}{a_{k}} \, ds \label{def:M.fun};
\end{align}
$H(t)$ is then defined as
\begin{equation}\label{def:entropy}
H(t) 
 = L(t) + M(t) \ .
\end{equation}
The entropy function $H(t)$ builds on entropy functions in \cite{Ma07} and \cite{Br96a}.
The term L(t) is derived in \cite{Ma07} as the large deviations rate function of a reversible network that approximates proportional fairness; the term 
$M(t)$ is similar to the entropy function employed in \cite{Br96a} for FIFO 
multiclass queueing networks.

The function  $L(t)$ is well defined everywhere (with, as earlier, the convention
 that $0 \log 0 = 0$).  Note that  
 
\begin{equation}
\label{eq22'}
\begin{split}
Q_j(t)  =  Q_j(0) + A_j(t) - D_j(t) 
=  &\sum_\kInj (\Gamma_k(Q_j(0) +A_j(t)) - \Gamma_k(D_j(t))) \\ 
=  &\sum_\kInj \int_{D_j(t)}^{Q_j(0)+A_j(t)} \Gamma'_k(s) \, ds
\ ;
\end{split}
\end{equation}
the lower and upper bounds $D_j(t)$ and $Q_j(0)+A_j(t)$ for the integral in (\ref{eq22'}) can be thought of as the amounts of packet mass having departed from the queue $j$ by time $t$ and having departed by the later time when the last of the mass already at $j$ at time $t$ departs from the queue.  With (\ref{eq22'}) in mind, note that $M(t)$ is the sum over the different queues of a weighted version of the packet mass at each queue $j$ at time $t$, based on the departure rates of the mass at the queues.

The following lemma shows that the function $H(t)$ is always non-negative for any solution of the fluid model.

\begin{lemma}\label{prop:H.not.zero}
For any $t\geq0$,
$H(t)\ge 0.$
Moreover, whenever the arrival rate vector $\veca$ belongs to the interior of the scheduling set $\mC$, then $H(t)=0$ iff $\vecQ(t)=0$.
\end{lemma}
\begin{proof}
By (\ref{eq22'}),
$$
Q_j(t) \, \log a_j =
  \sum_\kInj \int_{D_j(t)}^{Q_j(0)+A_j(t)}
    \Gamma'_k(s) \log a_j \, ds 
$$
for each $\jInJ$; adding and subtracting such terms
allows one to rewrite $H(t)$ as
\begin{equation}\label{HforPositivity}
H(t) 
 = \sum_\jInJ 
  \int_{D_j(t)}^{Q_j(0)+A_j(t)} \sum_\kInj 
    \Gamma'_k(s) \log \left(\Gamma'_k(s) \frac{a_j}{a_k}\right) \, ds
  + \sum_{\jInJ} Q_j(t) \, \log \frac{D'_j(t)}{a_j} \ .
\end{equation}
\iffalse
If we define the discrete probability distributions $\vecp$ and  $\vecq$, over the states $\kInj$, by $\vecp=(\Gamma'_k(s) : \kInj )$ and $\vecq= ( a_k/a_j : k \in j)$, we have that
$$
\sum_\kInj 
    \Gamma'_k(s) \log \left(\Gamma'_k(s) \frac{a_j}{a_k}\right)
= D(\vecp||\vecq).
$$
where $D(\vecp||\vecq)$ denotes the \emph{relative entropy} between the distributions $\vecp$ and $\vecq$.
\fi 
By  Jensen's inequality,  
\begin{equation}
\label{unlabeledeq}
\sum_\kInj 
    \Gamma'_k(s) \log \left(\Gamma'_k(s) \frac{a_j}{a_k}\right) = -\sum_\kInj 
    \Gamma'_k(s) \log \left( \frac{1}{ \Gamma'_k(s)} \frac{a_k}{a_j}\right)  \geq -\log \left(  \sum_\kInj 
    \Gamma'_k(s)    \frac{1}{ \Gamma'_k(s)} \frac{a_k}{a_j}\right) =0.
\end{equation}
From this, it follows that the first summation in \eqref{HforPositivity} is non-negative.
The non-negativity of the second term in (\ref{HforPositivity}) follows since  the service rates $D'_j(t)$ solve the optimization problem \eqref{PF} and  $\veca\in \mC$, and so, in particular,
$$
\sum_{\jInJ} Q_j(t) \, \log D'_j(t)
\geq 
\sum_{\jInJ} Q_j(t) \, \log a_j \ .
$$
Together with (\ref{unlabeledeq}), this implies $H(t)\geq 0$.

When the fluid model is subcritical, one can improve on the inequality in the last display to show that $H(t)$ is positive when $\vecQ(t) \neq 0$.  For this, note that
$\veca \in \mC$ implies the existence of an  $\epsilon > 0$ such that
$$ (1+\epsilon) \, \veca \in \mC \ .$$
 Since $\vecD'(t)$ is an optimal scheduling, 
it follows that, for $\vecQ(t) \not = 0$,
\begin{equation}\label{HQ}
\sum_{\jInJ} Q_j(t) \, \log \frac{D'_j(t)}{a_j}
\geq 
\sum_{\jInJ} Q_j(t) \, \log (1+\epsilon) > 0 \ .
\end{equation}
Consequently, $H(t)=0$ implies $Q(t)=0$.  Trivially, $\vecQ(t)=0$ implies $H(t)=0$.
\end{proof}
 
We next show that  $H(t)$ is decreasing when $\vecQ(t)\not=0$, by proving $H'(t)$ is negative.
Since we do not know in advance that $L(t)$ is sufficiently regular to
apply the Fundamental Theorem of Calculus to
$\int_s^t L'(u)\,du$, we will employ the following
technical lemma, which is based on Lemma 5 of \cite{Ma07}.

\begin{lemma}\label{L.deriv.a.e.} 
There exists a time $T>0$ not depending on the initial queue state such that:

i) For $t \geq T|\vecQ(0)|$, 
\begin{equation}\label{eq:EnvelopeTheorem}
L'(t) = 
\sum_\jInJ Q'_j(t) \log D'_j(t) \quad \text{a.e.}
\end{equation}

ii) There exists a constant $\kappa_L>0$ such that, for all $t\geq s \geq 0$,  
\begin{equation}\label{eq:L.Lipschitz.up}
L(t)-L(s) \leq \kappa_L (t-s).
\end{equation}

iii) For all $t\geq s\geq T|\vecQ(0)|$,
\begin{equation*}
L(t) - L(s) \leq \int_s^t L'(u) \, du.
\end{equation*}
\end{lemma}
A proof of this lemma is given in the Appendix. 

Since the processes $\vecA(t)$ and $\vecD(t)$ are Lipschitz, it is easy to see that the term $M(t)$, which integrates an a.e. bounded function, is Lipschitz and hence a.e. differentiable.  
Thus, the following corollary is immediate from Lemma \ref{L.deriv.a.e.}.

\begin{corollary}\label{H.deriv.a.e.} 
i) There exists a constant $\kappa_H>0$ such that, for $t\geq s \geq 0$,  
\begin{equation}\label{eq:H.Lipschitz.up}
H(t)-H(s) \leq \kappa_H (t-s).
\end{equation}
ii) There exists a time $T>0$ not depending on the initial state such that $H(t)$ is a.e. differentiable on $t \in [T|\vecQ(0)|, \infty)$ and, for given $s,t$ with $t\ge s\ge T|\vecQ(0)|$,
\begin{equation}\label{eq:upp.Lipschitz}
H(t) - H(s) \leq \int_s^t H'(u) \, du.
\end{equation}
\end{corollary}

 We now focus on computing $H'(t)$. The following proposition shows that  $H'(t)$ is the sum of two terms: the (unnormalized) relative entropy between route departure and arrival rates, and the (unnormalized) relative entropy between queue arrival and departure rates. This decomposition will play an important role in the proof of Theorem \ref{FluidStable}.  

\begin{proposition}\label{HDiff}
For $T$ as in Corollary \ref{H.deriv.a.e.}, and $t\ge T|\vecQ(0)|$, 
\begin{equation}\label{H(t).deriv}
H'(t) = 
- \sum_\rInR  D'_r(t) \log \frac{D'_r(t)}{A_r'(t)} 
- \sum_\jInJ A'_j(t) \, \log \frac{A'_j(t)}{D'_j(t)} \quad a.e. 
\end{equation}
\end{proposition}
As earlier, we set $0\log 0 = 0$; we also set $0\log (0/0) = 0$ here and note
that the set where $\{A_j'(t) > 0, D_j'(t) = 0\}$ has measure $0$ for each $j$.
For brevity, when no confusion is likely, we will often omit the quantifier ``a.e." in our computations.
\begin{proof}[Proof of Proposition \ref{HDiff}.]
Differentiating the expression for $H(t)$ in \eqref{def:entropy}
and applying \eqref{eq:EnvelopeTheorem} gives
\begin{align*}
H'(t) = 
& \sum_{j\in\mJ}\sum_{k\in j} A'_j(t) \, \Gamma'_k(A_j(t)) \log \frac{\Gamma'_k(A_j(t))}{a_k} \nonumber \\
& -\sum_{j\in\mJ}\sum_{k\in j} D'_j(t) \, \Gamma'_k(D_j(t)) \log \frac{\Gamma'_k(D_j(t))}{a_k}
+ \sum_\jInJ Q'_j(t) \, \log D'_j(t) .
\end{align*}
On the other hand, differentiation of the expressions \eqref{eq:Dk} and \eqref{eq:Ak} implies that
\begin{equation*}
D'_k(t) = D'_j(t)\Gamma'_k(D_j(t)), \qquad A'_k(t) = A'_j(t)\Gamma'_k(A_j(t)),
\end{equation*}
and substitution of these terms into $H'(t)$ gives
\begin{align*}
H'(t) = 
& \sum_{j\in\mJ}\sum_{k\in j} A'_k(t) \log \frac{A'_k(t)}{A'_j(t) \, a_k}
-\sum_{j\in\mJ}\sum_{k\in j} D'_k(t) \log \frac{D'_k(t)}{D'_j(t) \, a_k} \nonumber \\
& + \sum_\jInJ (A'_j(t)- D'_j(t)) \, \log D'_j(t) \\
= & \sum_{j\in\mJ}\sum_{k\in j}  \left(A'_k(t) \log \frac{A'_k(t)}{a_k} 
- D'_k(t) \log \frac{D'_k(t)}{a_k} \right)
+ \sum_\jInJ A'_j(t) \, \log \frac{D'_j(t)}{A'_j(t)} \nonumber .
\end{align*}
Application of \eqref{eq:ADk} shows that the double sum above telescopes over the classes in each route;
since  $A'_r(t) \equiv a_k$ for each $k\in r$, only the terms due to 
 external departures $D'_r(t)$ do not cancel out,  which implies
 \eqref{H(t).deriv}.
\end{proof}

It is now easy to see that $H'(t)\le 0$ for suitably large $t$.
\begin{corollary}\label{cor:H.non-increasing}
For $T$ as in Corollary \ref{H.deriv.a.e.} and $t>T|\vecQ(0)|$, $H'(t) \le 0$ a.e.
\end{corollary}
\begin{proof}
For each choice of $a,\! d\geq0$, one has  $a \log (a/d) \geq a-d$ (as before,
 $0\log 0 = 0$,  $0\log (0/0)= 0$). Applying this to \eqref{H(t).deriv} implies that
\begin{equation}\label{eq:cons.flow}
H'(t) \leq  \sum_{r\in\mR} \big( A'_r(t) - D'_r(t) \big) -  \sum_{j\in\mJ} \big( A'_j(t) - D'_j(t) \big)
=\sum_\rInR Q'_r(t)
- \sum_\jInJ Q'_j(t) = 0 \ ,
\end{equation}
with the final equality following since 
the rates of change in total queue size summed over all queues and summed over all routes are equal.
\end{proof}

\subsection{Fluid Stability and Positive Recurrence}\label{PosRec}

We wish to show that, in an appropriate averaged sense, $H(t)$ is always decreasing at a rate bounded away from zero when $\vecQ(t)\not=0$.  It will follow quickly from this that $\vecQ(t) = 0$ for $t\ge \gamma |\vecQ(0)|$, with $\gamma$
not depending on the particular fluid model solution; when this behavior occurs, the fluid
model is said to be \emph{stable}.

The following lemma shows that, for subcritical networks, queue sizes must vary over time and hence there is a ``mismatch" between arrival rates and departure rates at some of the queues. As we will see, this together with the relative entropy terms in Proposition \ref{HDiff} will show the Lyapunov function $H(t)$ decreases to $0$ at a uniform rate.
The lemma is proven in Subsection \ref{AppendixD} of the Appendix.  

\begin{lemma} \label{bound.change.Q}
Assume that the arrival rate vector $\veca \in \mC$.  Then
there exist constants $c,  \kappa>0$ such that, whenever $|\vecQ(0)|>0$, 
\begin{equation}
\label{eqnewkappa}
   |Q_j(t) - Q_j(0)| \geq \kappa \, |\vecQ(0)|\ 
\end{equation}
for some $j\in\mJ$ 
and $t \le c \, |\vecQ(0)|$.
\end{lemma}
\iffalse
\begin{lemma}\label{lm:log.bound}
For $x,\, y \in (0,K]$ 
\begin{equation}\label{eq:log.bound}
y\log \left(\frac{y}{x}\right) - (y-x) \geq \frac{1}{2 K} \left(y-x \right)^2.
\end{equation}
\end{lemma}
\begin{proof}
Let 
$F(z) = z\log z - (z-1)$, with $F\rq{}(z)= \log z$ and $F\rq{}\rq{}(z) = z^{-1}$. Taking a Taylor expansion around $1$, with $F(1)=0$, for some $\theta \in(1,z)$, 
$$
F(z) = F\rq{}(1) (z-1) + \frac{1}{2} F\rq{}\rq{}(\theta) (z-1)^2 
= \frac{1}{2\theta} (z-1)^2 \geq \frac{1}{2 (1\vee z )} (z-1)^2 \ .
$$
The last equality, holds since $\theta \leq (1 \vee z)$. 
Noticing that the left hand side of \eqref{eq:log.bound} is equal to $x\, F(y/x)$ the result follos by the following relation
$$
 x F(y/x) 
 \geq \frac{x}{2 (1\vee \frac{y}{x} )} \left(\frac{y}{x}-1\right)^2
= \frac{1}{2 (x\vee y) } (y-x)^2 
 \geq \frac{1}{2 K} (y-x)^2 \ .
$$
\end{proof}
\fi

\iffalse
\begin{lemma}\label{lm:H.bound}
For appropriate constants $h_1, \, h_2>0$, 
\begin{equation}\label{eq:H.bounds}
h_1 |\vecQ(t)|
\leq H(t) \leq 
h_2 |\vecQ(t)|
\end{equation}
\end{lemma}
\begin{proof}
\end{proof}
\fi

The following elementary lemma is also proved in the Appendix.

\begin{lemma}\label{lm:log.bound}
For $x,\, y \in (0,K]$ and any $K>0$,
\begin{equation*}
y\log \left(\frac{y}{x}\right) - (y-x) \geq \frac{1}{2 K} \left(y-x \right)^2.
\end{equation*}
\end{lemma}

We will use the two preceding lemmas to give the following lower bound on the rate of decrease of $H(t)$.  The basic argument for the proposition is similar to that leading up to (4.24) in \cite{Br96a},
although the reasoning differs in a few ways.  

\begin{proposition}\label{pr:der.H.boud.away.0}
Assume that the arrival rate vector $\veca$ belongs to the subcritical region $\mC$. 
For $T$ as in Corollary \ref{H.deriv.a.e.},  
there exist constants $c_1, c_2 >0$ such that, for $t\geq T|\vecQ(0)|$,
\begin{equation}\label{eq:H.dec.bound}
  H( t + c_1\, |\vecQ(t)|) - H(t) \le - c_2  \, |\vecQ(t)|  \,  .
\end{equation}
\end{proposition}
\begin{proof}
 
Let $K$ be the Lipschitz constant bounding the processes $D'_r(t),$ $A'_r(t),$ $D'_j(t),$ and $A'_j(t),$ over all $r\in\mR$ and $j\in\mJ$. From Proposition \ref{HDiff} and Lemma \ref{lm:log.bound}, 
\begin{align*}
H'(t) 
\leq & - \sum_\rInR \frac{1}{2 K} \left(D'_r(t)-A'_r(t) \right)^2 
- \sum_\jInJ \frac{1}{2 K} \left(A'_j(t)-D'_j(t) \right)^2\\
& - \sum_\rInR (D'_r(t)-A'_r(t))  - \sum_\jInJ Q'_j(t)\\
= & - \sum_\rInR \frac{1}{2 K} \left(D'_r(t)-A'_r(t) \right)^2 
- \sum_\jInJ \frac{1}{2 K} \left(Q'_j(t)\right)^2 
\leq - \frac{1}{2 K} \sum_{j\in\mJ} \left(Q'_j(t)\right)^2  .
\end{align*}
On account of \eqref{eq:upp.Lipschitz} in Corollary \ref{H.deriv.a.e.}, this implies
\begin{equation}
\label{eqquadraticbd}
H(t) - H(s) \leq -\frac{1}{2 K}  \sum_{j\in\mJ} \int_s^t \left(Q'_j(u)\right)^2 du. 
\end{equation}

By Lemma \ref{bound.change.Q}, there exists a $j\in\mJ$ and a value of $t$ with $t \leq s + c_1 \, |\vecQ(s)|$ such that
\begin{equation*}
 |Q_j(t) - Q_j(s)| \geq \kappa \, |\vecQ(s)|.
\end{equation*}
It therefore follows from the Cauchy-Schwarz Inequality that 

\begin{align}
\kappa \, |\vecQ(s)| &
\leq \int_s^{s+c_1 \, |\vecQ(s)|} \left| Q'_j(u) \right| \, du \leq \left( \int_s^{s+c_1 \, |\vecQ(s)|} \left( Q'_j(u) \right)^2  du \right)^{\frac{1}{2}} \, \sqrt{c_1 \, |\vecQ(s)|}  \nonumber
\end{align} 
for this choice of $j$, and hence
$$ 
\int_s^{s+c_1 \, |\vecQ(s)|} \left( Q'_j(u) \right)^2 du \geq \frac{\kappa^2}{c_1} |\vecQ(s)|\ .
$$
Applying the last inequality to  \eqref{eqquadraticbd}, one obtains
\begin{equation*}
H(s+c_1 \, |\vecQ(s)|) - H(s) \leq - \frac{\kappa^2}{2Kc_1} |\vecQ(s)| \,,
\end{equation*}
and so   (\ref{eq:H.dec.bound}) follows by setting $c_2=\kappa^2/2Kc_1$.
\iffalse
By \eqref{eq:H.bounds} we have  
\begin{equation*}
H(t+c_1 \, |\vecQ(t)|) - H(t) \leq - \frac{K_5}{c_1}  \frac{1}{h_1} \, H(t)\ ,
\end{equation*}
and by letting $ \epsilon = K_5 / (h_1 \, c_1 )$ we have that
$H(t+c_1 \, |\vecQ(t)|) \leq ( 1 - \epsilon ) \, H(t)$.
Finally since the function $H(t)$ is non-increasing and non-negative it must be that 
\begin{equation*}
H(u) \leq ( 1 - \epsilon ) \, H(t).
\end{equation*}
for all $u \geq t + c_1 |\vecQ(t)|$.
\fi
\end{proof}

Employing Lemma \ref{prop:H.not.zero}, Corollary \ref{H.deriv.a.e.}, and Proposition \ref{pr:der.H.boud.away.0}, we now show our main result for proportional switch
fluid models.  

\begin{theorem} \label{FluidStable}
If $\veca \in \mC$, then the proportional switch fluid model is stable.
\end{theorem}
\begin{proof}
The proof follows the reasoning in \cite{Br96a}, which we repeat for completeness.
Let $c_1$ and $c_2$ be as in in Proposition \ref{pr:der.H.boud.away.0}, set $t_0=T|\vecQ(0)|$ for $T$ as in Corollary \ref{H.deriv.a.e.}, and define
\begin{equation}\label{eq:t_i.rec}
t_{i+1} = t_i + c_1 \, |\vecQ(t_i)|. \ 
\end{equation}
By Proposition \ref{pr:der.H.boud.away.0}, since $H(t)\ge 0$, 
$$
H(t_0) \geq
H(t_0) - H(t_{i+1}) \geq
\sum_{l=0}^i (H(t_l) - H(t_{l+1})) \geq \sum_{l=0}^i c_2 |\vecQ(t_l)| 
= \sum_{l=0}^i \frac{c_2}{c_1} (t_{l+1}-t_l) 
= \frac{c_2}{c_1} (t_{i+1} -t_{0})\,,
$$
and so, as $i \to \infty$, one has
$t_\infty =\lim_{i\rightarrow\infty} t_i \leq c_1 \, H(t_0)/c_2 + t_0 < \infty$.
By the continuity of $\vecQ(t)$  and the definition of $t_{\infty}$, 
$\vecQ(t_\infty)=0$, and consequently, since $H(t)$ is non-increasing,
\begin{equation*}
H(t)=0 \quad \mbox{ for } t\geq \frac{c_1}{c_2}  H(t_0) + t_0.
\end{equation*}
On the other hand, by Corollary \ref{H.deriv.a.e.}, $H(t_0)\leq \kappa_H t_0= \kappa_H T |\vecQ(0)|$.
Setting $c_3=c_1 \kappa_H T/c_2+T$, this together with Lemma \ref{prop:H.not.zero} implies 
\begin{equation}\label{eq:Q.hitting.time.of.zero}
\vecQ(t)=0 \quad \mbox{ for } t\geq c_3 \, |\vecQ(0)|\,,
\end{equation}
as required.
\end{proof}

The main result in the paper, Theorem \ref{mainthrm}, follows immediately from Theorem  \ref{FluidStable} and the following proposition.
\begin{proposition}
\label{propFMQN}
Suppose that the proportional switch fluid model is stable.  Then the corresponding proportional switched network is positive recurrent.
\end{proposition} 

We postpone the proof of Proposition \ref{propFMQN} to Subsection \ref{Appendix F} of the Appendix.  We  will follow there the approach of \cite{Da95} as presented in \cite{Br08}, although the argument is considerably shorter in the present setting and requires only several pages. 

\bibliographystyle{apalike}
\bibliography{references-14122004-arxiv}

\appendix 

\iffalse
\section{Results from Section \ref{BP Compare}}

\begin{proof}[Proof of Proposition \ref{pr:linear.network.BP}.]
Set $\epsilon=\min_{j} \{ 2\alpha_j-1\}>0$, and let $\vecQ(t)$ denote the process starting from its equilibrium (with $\vecQ$ denoting the equilibrium itself). 
The BackPressure policy will only serve a link $j$ at time $t$ if the queue length at the downstream queue is strictly smaller, i.e., $Q_{j}(t)>Q_{j+1}(t)$; hence, if a packet has been served at a link $j$ by time $t$, then
\begin{equation}
\label{eqnewA3.1}
Q_{j}(t) - Q_{j+1}(t) \geq -1.
\end{equation}
Letting $t\rightarrow\infty$ shows that (\ref{eqnewA3.1}) always holds for the equilibrium $\vecQ$.

In equilibrium, packets cross a link $j$ at long term rate $\alpha_j$, which implies from the policy that
\begin{equation*}
\bP \left( Q_{j} - Q_{j+1}> 0\right) = \alpha_{j}.
\end{equation*}
Consequently,
\begin{equation*}
\bE \left[ Q_{j}\!\! - Q_{j+1} \right] \geq \bP (Q_{j} - Q_{j+1}> 0) - \bP (-1 \leq Q_{j} - Q_{j+1} \leq 0)
=  2\alpha_{j} -1 \ge \epsilon.
\end{equation*}
It follows from this that, for $j\leq J-1$,
\begin{equation*}
\bE \big[ Q_{j} \big] \geq \bE \big[ Q_{j} - Q_{J} \big] = \sum_{j\rq{}=j}^{J-1} \bE \big[Q_{j\rq{}} - Q_{j\rq{}+1} \big] 
 \geq \epsilon(J-j)
\end{equation*}
and so
\begin{equation*}
 \sum_{j=1}^J \bE \big[  Q_{j} \big] \geq \sum_{j=1}^{J-1} \epsilon (J-j) =  \frac{\epsilon}{2}J(J-1) \ge cJ^2 
\end{equation*}
for $c = \epsilon/4 $, as required.
\end{proof}

\begin{proof}[Proof of Proposition \ref{pr:linear.network.PF}.]
Consider a queue $Q$ that serves a job with probability $\sigma$ and has arrivals from an independent Bernoulli processes with of parameter $a>0$.
The process $Q$ is discrete time Markov chain that increases by $1$ with probability $a(1-\sigma)$ and, provided $Q>0$, decreases by $1$ with probability $(1- a)\sigma$. Once can check from the detailed balance equations that, provided $a < \sigma$, this Markov chain is reversible with stationary distribution
\begin{subequations}\label{LineED}
\begin{equation}
\pi(q) =(1- \rho) \rho^q,
\end{equation}
where
\begin{equation}
\rho=\frac{a(1-\sigma)}{(1-a)\sigma},
\end{equation}
\end{subequations}
and thus the expected number queue size is
\begin{equation}
\bE Q = \frac{\rho}{1-\rho} = \frac{a (1-\sigma)}{\sigma -a }.
\end{equation}
Since future arrivals are Bernoulli and independent of the current queue state, reversibility implies that the output process of the queue is a Bernoulli process independent of the current queue state.

The first queue of the line network, see Figure \ref{Line}, behaves exactly as the reversible queue described above. Thus, in equilibrium, its output processes is an independent Bernoulli process. As a consequence, the queue is independent of all downstream queues. This is the input process to the second queue. Hence, the second queue behaves exactly the same as the reversible queue described above and its output is an independent Bernoulli processes. Continuing to argue so forth we see that, in equilibrium, each queue $j$ is independent of all other queues and has distribution given by \eqref{LineED} with $\sigma=\sigma_j$.

Thus the total expected queue length of this network is
\begin{equation}
\bE \bigg[ \sum_{j=1}^N Q_j \bigg]  = \sum_{j\in\mJ} \frac{a (1-\sigma_j)}{\sigma_j -a }.
\end{equation}
Through standard manipulations it can be shown that if $a/\sigma_j < 1- \epsilon$ then
\begin{equation}
\frac{a (1-\sigma_j)}{\sigma_j -a } \leq \frac{1}{\epsilon}.
\end{equation}
Thus,
\begin{equation}
\bE \bigg[ \sum_{j=1}^N Q_j \bigg]  \leq \frac{N}{\epsilon},
\end{equation}
as required.
\end{proof}

[COMMENT OUT THE PROPOSITION BELOW]
\begin{proof}[Proof of Proposition \ref{pr:linear.network.PF}.]
The proposition can be proved by using the standard approach for quasi-reversible
networks as in, e.g., Kelly (1979), which we therefore just summarize. 

Restricting attention to the first queue in the network, one can
show that the probability in equilibrium of any state $s$ in the queue is
$B^{-1}( a/(1-a))^q$,
where $q$ is the number of packets in the state and $B > 1$ is an appropriate normalizing constant.
Hence, the queue size $Q_1$ at the first queue is bounded by 
\begin{equation}
\label{theexpectationformula}
\bE Q_1 \leq \sum_{q=0}^\infty q \left( \frac{an'_1}{1-a} \right)^{q} = \frac{n'_1 a (1-a)}{(1 -a(n'_1+1))^2}\,,
\end{equation}
where $n'_1 = n_1 +1$.  Using time reversal, one can also show that the output process from the queue is a 
Bernoulli process that, up to any time $t$, is independent of the state of the original queue process at time $t$.  This output process is also the input process for the second queue in the network.

Arguing inductively, it follows that, in equilibrium, the expectation of the queue size
$Q_j$ at the $jth$ queue satisfies the analog of (\ref{theexpectationformula}), and
hence that
the total expected queue length of the entire network bounded by
\begin{equation*}
\sum_{j\in\mJ}  \frac{n'_j a (1-a)}{(1 -a(n'_j+1))^2},
\end{equation*}
where $n'_j = n_j +1$.
Through standard manipulations it can be shown that, if $\alpha_j= an'_j \le 1- \epsilon$, then\footnote{Being a little more careful the dependence on $\epsilon$ can be improved in this bound.}
\begin{equation*}
\frac{n_ja (1-a_j)}{(1 -a(n_j+1))^2} \leq \frac{1}{\epsilon^2}.
\end{equation*}
Thus,
\begin{equation*}
\bE \bigg[ \sum_{j=1}^J Q_j \bigg]  \leq \frac{N}{\epsilon^2},
\end{equation*}
as required.
\end{proof}
\fi

\iffalse
\section{Proof of Proposition \ref{PFRPProp} from Section \ref{ProdFormResPooling}}
A Store-Forward (SF) network is defined by employing the stationary distribution of a corresponding
processor sharing network.
We first define this processor sharing network. For finite index sets $\mL$ and $\mJ$, and given vector 
$(a_j: j\in \mJ)$ and matrix $(A_{\ell j}: \ell \in \mL, j \in \mJ)$, with $a_j >0$, $A_{\ell j} \ge 0$, and $\sum_{j} A_{lj} \, a_j < 1$
for $\ell\in \mL$ and $j\in \mJ$, we consider the processor sharing network with $\mL$ queues (which we henceforth refer to as \emph{resource pools} rather than \emph{queues}) and $\mJ$ \emph{arrival streams} such that \emph{jobs} enter the network according to $|\mJ|$ independent Poisson processes at rates $a_j$, with jobs from each stream visiting
each resource pool exactly once, in some prescribed order, and requiring an
exponentially distributed amount of service with mean $A_{\ell j}$ at resource pool $\ell$.  

Denoting by $\vm = (m_{\ell j}: \ell \in \mL, j \in \mJ)$ the vector for the number of 
jobs at each resource pool from each stream, then, by employing quasi-reversibility (see, e.g., \cite{Ke79}), it is standard that the above processor sharing network has stationary distribution
\begin{equation}
\label{aphi}
\pi(\vm)=
\prod_{l \in \mL}   \left( \left({1-a_l}\right)^{-1} {{m}_l \choose {m}_{lj}: j\in\mJ}\prod_{\substack{j \in\mJ }} \left({A_{\ell j}a_j}\right)^{{m}_{\ell j}}\right),
\end{equation}
where $a_{\ell}=\sum_{j} A_{\ell j} \, a_j < 1$ is the load and $m_{\ell}=\sum_{j} m_{\ell j}$ is the number of jobs at resource pool $\ell$.  Note that the product form
representation of $\pi(\vm)$ in (\ref{aphi}) implies the behavior at different
resource pools is stochastically independent.  

We denote by
\begin{equation}
\label{eqnewforq}
\pi(\vecq) := \pi\Big(\Big\{ \vm : \sum_{l\in\mL} m_{lj}=q_j, j\in\mJ \Big\}\Big) 
\end{equation}
the stationary distribution for the number of jobs in each stream. 
The conditional distribution obtained by dividing \eqref{aphi} by \eqref{eqnewforq} is the stationary distribution of the closed processor sharing queueing network corresponding to the (open) processor sharing network just described, where, instead of arrival streams, there are always
  $\vecq$ jobs in the network that repeatedly traverse their routes (with each job immediately returning to the beginning of its route after completion of its previous excursion along the route). See \citet[][Chapter 3.4]{Ke79} for further details.
Another straight-forward calculation shows that, given the number of jobs on each route is $\vecq$, the stationary throughput of jobs on route $j$ is
\begin{equation}\label{Store-Forward}
\sigma^{SF}_j(\vecq):= \frac{\pi(\vecq-\ve_j)}{\pi(\vecq)}\,;
\end{equation}
the function $\vecsigma^{SF}(\vecq)=(\sigma_j^{SF}(\vecq): j\in\mJ)$ is referred to in the literature as the \emph{Store-Forward} (SF) allocation associated with the above
processor sharing network (see \cite{BoPr03}). 
Note that $\vecsigma^{SF}(\cdot)$ depends solely on the pair $(\veca, \vecA)$.

We now construct the Store-Forward queueing network.  
We choose an arrival rate $\veca$ and scheduling set $\mS$, where $\veca >0$ and $\mS$ is as in Subsection
\ref{subsection2.1}.  The convex hull $<\mS>$ of $\mS$ can be expressed in terms of
a finite number of linear constraints, 
\begin{equation}\label{Polytope}
\coS =\Big \{ \sigma\in\bR_+^\mJ : \sum_{j\in\mJ} A_{lj} \sigma \leq 1,\, l\in\mL \Big\},
\end{equation}
where $\vecA = (A_{l,j}: l\in \mL, j\in \mJ)$ is non-negative and is of full row rank.  Using $\veca$
and $\vecA$, we define the \emph{Store-Forward} (SF) queueing network corresponding
to $(\veca, \vecA)$ as follows:
Employing the notation in Section \ref{subsection2.1}, routing is indexed by $r\in\mR$, with
packets visiting classes $k\in\mK$ at queues $j\in\mJ$ along their routes.  Packets 
are assumed to arrive at the routes $r$ according to Poisson processes having rates $a_r$, 
and packets are queued in a FIFO manner at each $j\in\mJ$, where they have independent service
 requirements that are exponentially distributed with mean $1$.  Let $\vecq=(q_j : j\in\mJ)$ denote the vector of queue lengths at each $j$ and let $\vecGamma=(\Gamma_k^s : k\in\mK\,,s=1,...,q_j)$, where $\Gamma_k^s$, $s=1,...,q_j$, is the number of class $k$ packets at queue $j$ between the $1$st and $s$th positions (inclusive). 

Packets at the head of each FIFO queue $j$ are to be served at rate $\sigma_j^{SF}(\vecq)$ given by the SF allocation \eqref{Store-Forward} with pair $(\veca,\vecA)$. A class $k$ packet on route $r$ completing service at a queue is then assumed to move to the next class $k'=n(k)$
along its route, where it joins the end of the corresponding queue $j(k')$. 
 For numerous routing topologies (one of which we soon discuss), the corresponding SF network 
will have a product form stationary distribution.  (We note that the SF network is a continuous time
queueing network, whereas the proportional switched network is in discrete time.)

Standard quasi-reversibility arguments (see \cite{Ke79}) imply that the stationary distribution of this SF network is
\begin{equation}
\label{FIFOSF}
\pi^{SF} (\vecq,\vecGamma) = {\pi(\vecq)} \prod_{j\in\mJ} \prod_{k\in j} \left(\frac{a_k}{a_j}\right)^{\Gamma_k^{q_j}},
\end{equation}
where $\pi (\vecq)$ is as in (\ref{eqnewforq}).  (Note that $\vecGamma$ specifies the ordering
of the packets at each queue.)  It follows directly from (\ref{FIFOSF}) that the stationary
probability at $\vecq$ equals $\pi (\vecq)$.  
(Recall that $\pi (\vecq)$ is the
sum of independent random variables corresponding to $m_{lj}$, $l\in\mL$,
in (\ref{aphi}).) 

We now demonstrate Proposition \ref{PFRPProp}.
\begin{proof}[Proof of Proposition \ref{PFRPProp}.]
\iffalse
This policy is first analyzed for its insensitivity property by \cite{BoPr03}.
The connection between product form queueing networks, Store-Forward and proportional fairness is detailed by \cite{Wa09}.
For $\vm=(m_{jl}:j\in\mJ, l\in\mL)\in\bZ_+^{\mJ\times\mL}$, we consider the product form distribution
\begin{equation}
\phi(\vm)\propto 
\prod_{l \in \mL} \left({{m}_l \choose {m}_{lj}: j\in\mJ}\prod_{\substack{j \in\mJ }} \left(\frac{A_{lj}a_j}{C_l}\right)^{{m}_{lj}}\right).
\end{equation}
Here we apply the convention $m_l:=\sum_{j\in\mJ} m_{jl}$, $l\in\mL$. In the above distribution, if $A_{lj}=0$ then $m_{jl}=0$. The key observation is that under this distribution, the values of $m_{jl}$ and $m_{j\rq{}l\rq{}}$ are given by independent random variables when $l\neq l\rq{}$. The Store-Forward policy the policy defined to achieve the stationary distribution
\begin{equation}\label{phiP}
P^{SF}(\vq)= \sum_{{{\vm}} \in U(\vecq)}  \phi(\vm).
\end{equation}
where $U(\vecq)=\{ \vm\in\bZ_+^{\mL\times\mJ} :  \sum_{l\in\mL} m_{lj} = q_j,\, j\in\mJ\}$. So $P^{SF}(\vecq)$ is the probability the sum $m_{lj}$ over $l$ is $q_j$. Thus can be achieved by an Whittle policy $\sigma^{SF}_j(\vecq)$ defined by a potential $\Phi(\vecq)=P^{SF}(\vecq)$. 
\fi

Let $\vM$ be a random matrix with distribution given by 
$\pi (\cdot)$ in (\ref{aphi}), let
$M_{lj}$ and $M_{l}$ be the corresponding individual components and row sums, and let
$Q_j$ be the corresponding column sums;
we will write $j\in l$ when $A_{lj} > 0$.
The random variables
$M_l$ are geometrically distributed and, given that $M_{l} =m_{l}$, the 
random variables $(M_{lj}: j\in l)$ are multinomial, with $m_l$ trials and parameters $(A_{lj}a_j/a_l : j\in l)$; moreover, the row random vectors $(M_{lj} : j\in l)$ are independent for distinct $l$. 
For $A_{lj}=0$, it follows that $M_{lj}=0$ with probability 1.

By assumption, $\mS= \mS^1\! \times\! ...\! \times\! \mS^N$ and so $<\! \mS \!>= <\! \mS^1\!>\!\times ... \times\! <\! \mS^N \!>$; 
consequently, the matrix $\vA$  (resp., $\vM$) is block diagonal, with the sub-matrices $\vA^1,...,\vA^N$ (resp., $ \vM^1,\ldots, \vM^N$) corresponding to each respective scheduling component. 
(We will also use the notation $j\in n$ if $A_{lj}^n > 0$ for some $l \in \mL$.)
The submatrices $\vM^1,\ldots, \vM^N$ are independent by the independence of the row vectors of $\vM$.
Hence, for $n\neq n'$ and $j\in n$, $j'\in n'$,  the column sums 
\begin{equation*}
Q_j = \sum_{l : j\in l} M_{lj}^n\qquad\text{and}\qquad Q_{j'} =\sum_{l : j'\in l} M_{lj'}^{n'}
\end{equation*}
are independent.

By the same argument,  the queue size vectors  $\vecQ^n=(Q_j: j\in n)$ are independent
over different components $n$.  (Note that $\vQ^n = \mathbf{1}^T \vM^n$.)
It follows from the independence of $\vQ^n$ that $\pi(\vecq)$ can be expressed in the form $\pi(\vecq)=\prod_{n=1}^N \pi^n(\vecq^n)$, and thus the stationary distribution \eqref{FIFOSF} for such SF networks can be expressed in the form
\[
\pi^{SF} (\vecq,\vecGamma) = \prod_{n=1}^N \left[  {\pi^n(\vecq^n)} \prod_{j\in n} \prod_{k\in j} \left(\frac{a_k}{a_j}\right)^{\Gamma_k^{q_j}}\right].
\]
That is, the stationary distribution of such a SF network factors according to its scheduling components, irrespective of the routing between queues.
\end{proof}

\fi

\section{Lemmas from Section \ref{LyaDeriv}}

The main aim of this section is to prove Lemma \ref{L.deriv.a.e.}. We first present the supporting Lemmas \ref{SigCont} - \ref{pr:D'.positive}.  Lemma \ref{SigCont} 
is proved in  Lemma A.3 of \cite{KeWi04}.

\begin{lemma}\label{SigCont} 
The function $\vecQ \mapsto \sigma_j(\vecQ)$ is continuous on the set $\{ \vecQ : Q_j>0\}$.
\end{lemma}

\iffalse
The $D_j(t)$ are Lipschitz, then by  \eqref{eq:ADk}
also the $A_k(t)$ are Lipschitz, whenever the classes are not source classes, 
finally for the source classes the Lipschitz property it is due to the 
i.i.d. assumption and the finite means.
\fi

\iffalse
\begin{proposition}\label{deriv.D}
\begin{subequations}
\begin{align}\label{PF.forD}
\vecD'(t) &\in \operatorname*{arg\,max}_{\sigma \in \mC} \sum_{j\in\mJ} Q_j(t) \log \sigma_j & t - \mbox{a.e.} \\
D'_j(t) &= A'_j(t) & \mbox{if } Q'_j(t)=0
\end{align}
\end{subequations}
\end{proposition}
\fi

When $Q_j/|\vecQ|$ is bounded away from $0$, Lemma \ref{lm:lower.bound.sigma} states that the corresponding
coordinate $\sigma_j(\vecQ)$ of the proportional fair optimization in (\ref{PF}) is also bounded away from $0$.
\begin{lemma}\label{lm:lower.bound.sigma}
Let $\vecsigma(\vecQ)$ be a solution to the proportional fair optimization (\ref{PF}), where $|\vecQ| >0$.
Then, for any $\epsilon>0$, there exists a $\delta>0$ such that, for any $j$, 
 $\sigma_j(\vecQ)\geq \delta$ whenever $Q_j \geq \epsilon |\vecQ|$.
\end{lemma}
\begin{proof}
Scaling $\vecQ$ by $|\vecQ|$, it suffices to prove the result for $\sum_{j'\in \mJ}Q_{j'} = 1$, when $Q_j \ge \epsilon$.  If the result were not true, then, for some $\epsilon>0$ and $j$, there would exist a sequence $\vecQ^{(k)}$ with $Q_j^{(k)}>\epsilon$ and  $\sigma_j(\vecQ^{(k)}) \rightarrow 0$ as $k\rightarrow\infty$.
It would follow that
\begin{equation*}
\sum_{j' \in\mJ } Q^{(k)}_{j'} \log \sigma_{j'}(\vecQ^{(k)}) \leq Q_j^{(k)} \log \sigma_j(\vecQ^{(k)}) + (1- Q_j^{(k)}) \log \sigma_{\text{max}}
 \xrightarrow[k\rightarrow\infty]{} -\infty,
\end{equation*}
where $\sigma_{\text{max}}$ is defined at the beginning of Subsection \ref{subsection2.1}.
Since the maximum in (\ref{PF}) is bounded from below (by any fixed choice of $\vecsigma$), this gives a contradiction.
\iffalse
Consider the value
$$ s(\epsilon) = \inf_{\vecalpha:q_{j'} \geq \epsilon} \hat\sigma_{j'} $$
and assume that it is equal to $0$. The we can find a sequence $\vecalpha^{[k]}$
such that the corresponding optimal solutions $\hat\vecsigma^{[k]}$ have the $j'$ component vanishing.

However we would have that 
\begin{align}
q^{[k]}_{j'} \log \sigma^{[k]}_{j'} + \sum_{\tiny\begin{array}{c}j \in\mJ \\ j\not=j'\end{array}} q^{[k]}_j \log \sigma^{[k]}_j 
\leq
\epsilon \log \sigma^{[k]}_{j'} + |\mJ| \sigmamax \to -\infty
\end{align}
that is a contradiction, since $\hat\vecsigma^{[1]}$ would be a feasible solution with higher value of the objective function
$$\inf_{\vecalpha:q_{j'} \geq \epsilon} \left(
 q_{j'} \log \hat\sigma^{[1]}_{j'} 
+ \sum_{\tiny\begin{array}{c}j \in\mJ \\ j\not=j'\end{array}} q_j \log \hat\sigma^{[1]}_j 
\right) 
\geq 
|\mJ| \log \left( \min_{j \in\mJ}  \hat\sigma^{[1]}_j \right).$$
\fi
\end{proof}

Lemma \ref{pr:D'.positive} bounds $D_k(t)/t$ from below for large $t$.  A related result is given in Lemma 5 of \cite{Ma07}.

\begin{lemma}\label{pr:D'.positive} 
There exists $T >0$, not depending on $\vecQ(0)$,  such that, for any class $k$ and $t\ge T|\vecQ(0)|$,  
\noindent i) for appropriate $\beta >0$,
 $D_k(t) >   \beta t$  and \noindent ii)  $D'_k(t)>0$ a.e.

\iffalse
For each $k\in\mK$ 
  \begin{equation}\label{eq:D.unbounded}
  \limsup_{t\rightarrow\infty} \frac{D_k(t)}{t} > 0\, .
  \end{equation}
\fi
\end{lemma}
\begin{proof}

\noindent i) We argue by induction along each route, assuming for a class $k$, times $s$ with $s\ge T_1|\vecQ(0)|$ for a given $T_1\ge 1$, and a given $\alpha>0$, that $A_k(s)\geq \alpha s$.
We first show the analog of the desired result, but for $D_j(t)$ instead of $D_k(t)$, where $k\in j$.
If $D_j(s) < \alpha s/2$, then $Q_j(s)\geq Q_k(s)\geq \alpha s/2 $, and so, when $s\ge T_1|\vecQ(0)|$,
\begin{equation*}
\frac{Q_j(s)}{\sum_{j\rq{}\in\mJ} Q_{j\rq{}}} \geq \frac{\alpha s/2 }{\sum_{j'\in\mJ} (a_{j'} s + Q_{j'}(0)) } \geq \epsilon
\end{equation*}
for appropriate $\epsilon>0$. 
Therefore, by Lemma \ref{lm:lower.bound.sigma}, $D'_j(s) \geq \delta$, where $\delta$ does not depend on the initial queue size distribution; from this, it follows
that, if $D_j(s) < \alpha s/2$ for all $s\in [t/2,t]$ with $t\ge 2T_1|\vecQ(0)|$, then $D_j(t)\ge \delta t/2$.
Combining the last inequality with the case when $D_j(s) \ge \alpha s/2$ for some $s\in [t/2,t]$,  and setting $ \delta'= \min\{ \alpha/4 , \delta/2 \}$, one obtains $D_j(t) \geq  \delta't$ for $t\ge 2T_1|\vecQ(0)|$ in both cases.

We now show $D_k(t)$ also increases linearly in time.  Note that $A_j(t) \le \bar{a} t $, where $\bar{a}= \max_j{a_j}+ |\mK| |\sigma_{\max}|$. Applying this and setting $\gamma=  \delta'/2\bar{a}$, one can check that 
$D_j(t) \geq A_j(\gamma t) + |\vecQ(0)|$ for $t\ge T_2|\vecQ(0)|$, with $T_2 = 2 \max(T_1, 1/\delta')$ .
Using \eqref{eq:Dk} and the FIFO assumption \eqref{eq:Ak}, one therefore obtains
\begin{equation*}
D_k(t) = \Gamma_k(D_j(t)) \geq \Gamma_k(A_j(\gamma t) + Q_j(0)) = A_k(\gamma t) + Q_k(0) \geq \gamma \alpha t
\end{equation*}
for $t\ge T_2|\vecQ(0)|$, where the last inequality follows from the induction assumption.  Since $A_k(t)=a_r t$ for the first class on a route $r$, applying (\ref{eq:ADk}) and the argument just given, we can inductively show that for all classes on a given route $r$, $D_k(t) > \beta t$ for $t \ge T|\vecQ(0)|$, for appropriate choices of $\beta,T>0$. This is the desired result.

\iffalse
Seeking a contradiction, suppose $\lim_{t\rightarrow\infty} D_k(t)/t = 0$ and $\limsup_{t\rightarrow \infty} A_k(t)/t >0$ for some class $k$ at some queue $j$. 
Then $\limsup_{k\rightarrow\infty} Q_k(t)/t >0$, thus a positive proportion of the work in the network must be class $k$ work distributed along queue $j$. $Q_j(t)$ is greater than $Q_k(t)$. By the previous lemma and since $Q_j(t)$ is Lipschitz, $D\rq{}_j(t)$ must be bounded below for a positive proportion of time. Thus $\limsup_{t\rightarrow\infty} D_j(t)/t>0$. Further since service is FIFO and arrival rates to queue $j$ are bounded, over time positive proportions of class $k$ arrivals must depart. So one sees that $\limsup_{t\rightarrow\infty} D_k(t)/t>0$. This yields a contradiction. 

Clearly $\limsup_{t\rightarrow \infty} A_k(t)/t>0$ holds for each input class.  
The remainder of the proof applies the argument above by induction along the classes of each route.
\fi
\noindent ii) We again argue by induction along each route, assuming for a class $k$ that
$A'_k(t) > 0$ a.e. on $t\ge T_3|\vecQ(0)|$ for some $T_3\ge T$, where $T$ is given in part i); we henceforth restrict our attention to times $t$ where the arrival and departure processes are differentiable.  It follows under this assumption that 
\begin{equation}\label{Ddash}
D'_j(t)>0 \qquad \text{for } t\ge T_3|\vecQ(0)|
\end{equation}
because, if $Q_j(t) > 0$, then $D_j'(t)>0$ by Lemma \ref{lm:lower.bound.sigma}, whereas, if $Q_j(t)=0$, then $Q_j'(t)=0$ and so $D_j'(t)=A_j'(t)\geq A_k'(t)>0$. 

By \eqref{eq:Ak} and the definition of $\Gamma'_k$,  for $t\ge T_3|\vecQ(0)|$,
\begin{equation}
\label{eqnew42}
\Gamma'_k (A_j(t)+ Q_j(0)) = \frac{A'_k(t)}{A'_j(t)} > 0\,.
\end{equation} 
Also, by part i), for  $t \ge T_4 |\vecQ(0)|$, with $T_4 = \max(T , \beta^{-1} (A_j(T_3 |\vecQ(0)|)/|\vecQ(0)| +1))$, 
one has $D_j(t) \geq A_j(T_3 |\vecQ(0)|) + Q_j(0)$; together with (\ref{eqnew42}), this implies that 
\begin{equation}\label{Gdash}
\Gamma'_k(D_j(t))>0 
\end{equation}
for $t\geq T_4 |\vecQ(0)|$.
Using the same upper bound for $A_j(t)$ as in part i), one can show that $T_4 \le T_5$ for $T_5 = \max(T_3 , \beta^{-1}(\bar{a}T_3 + 1))$.
So, \eqref{Ddash} and \eqref{Gdash} together imply that 
\[
D_k'(t)= D_j'(t) \Gamma'_k(D_j(t))>0
\] 
for $t \ge T_5 |\vecQ(0)|$. Applying (\ref{eq:ADk}), one can then inductively repeat this argument along the classes of each route to give the desired result after a new choice of $T$.
\end{proof}

We now employ Lemmas \ref{SigCont} and  \ref{pr:D'.positive} to demonstrate Lemma \ref{L.deriv.a.e.}.

\iffalse
We note that by an elementary induction argument, it must be the case that for each class $k\in\mK$, $D_k(t) \to \infty$  as $t \to \infty $. It this were not true, take the first class $k$ on a route $r$ for which $D_k(t)$ is bounded. The queue size of a class prior to $k$ must then grow linearly with time 

there would be a positive proportion of the work arriving in the system$j$ or in queues leading to $j$
\fi

\iffalse
\begin{proof}
We prove \eqref{eq:D.unbounded} by showing that for each route $r \in\mR$ there is no class
$k\in r$ such that $D_{j(k)}(\infty) < \infty$. 
We do this by showing that if the above condition holds than after some finite time, all the queues
serving packets from the same route $r$ and belonging to classes $k'$ coming in order before class $k$, say $k' \leq k$,
will emit packets on that route at least at some constant rate.

This would lead to a contradiction, since to have $D_{j(k)}(\infty) < \infty$, 
the output rate at queue $j(k)$ must drop to zero  as $t\to\infty$.

Assume now that for $k \in \mR$ and $D_{j(k)}(\infty) = K > \infty$.
Then by conservation of flows we have 
\begin{equation}
\sum_{\substack{k'\in r \\k'\leq k}} Q_{k'}(t) - Q_{k'}(t) = a_r \, t - D_{j(k)}(t)
\end{equation}
so that choosing $T$ enough large and $\epsilon$ small we have that there exists 
a class $k' \in r$ with $k' \leq k$ such that
\begin{equation}
  Q_{k'}(t) \ge \epsilon \, t , \quad t \geq T\ .
\end{equation}
Having
\begin{equation}
  \sum_j Q_j(t) \leq \sum_j Q_j(0) + \sum_r a_r  \, t , \quad t \geq T
\end{equation}
it follows that for some $\epsilon'$ it also holds that
\begin{equation}
  \frac{Q_{k'}(t)}{\sum_j Q_j(t)} \leq \epsilon' \, t , \quad t \geq T
\end{equation}
that together with Lemma \ref{lm:lower.bound.sigma} implies the existence of $\delta'$ such that
\begin{equation}
D'_{k'}(t) \geq \delta' , \quad t \geq T\ .
\end{equation}
If $k'<k$, we can repeat the same argument considering the route $r$ as starting in $n(k')$ and with arrival rate $\delta'$ till reaching a contradiction.
\end{proof}
\fi

\iffalse
\begin{proposition}\label{pr:D'.positive}
For each $j\in\mJ$ there exists $T_j>0$ such that 
  \begin{equation}\label{eq:D'.positive}
  D'_j(t)>0 \quad t>T_j \quad a.e. 
  \end{equation}
\end{proposition}
\begin{proof}
We show that if there is a class $k \in j$ such that $A'_k(t) >0$ for some $t>T$, 
than \eqref{eq:D'.positive} holds. 
Since this assumption holds for $j(k)$ with $k\in\mK^i$ and $T=0$, by induction it implies that 
\eqref{eq:D'.positive} holds for any $j\in\mJ$.

By Proposition \ref{deriv.D}, a.e. for $t>T$
\begin{align}
& Q_j(t)>0 \implies D'_j(t)>0 \\
& Q_j(t)=0 \implies Q'_j(t)=0 \implies D'_j(t) \geq A'_j(t) \geq A'_k(t)
\end{align}

In addition, by Proposition \ref{pr:D.unbounded} there exists a time $T'>T$ after that all initial content $Q_j(0)$ has been served,
that is $D_j(T') \geq Q_j(0)$.
For all $t > T'$,
\begin{equation}
\Gamma'_k (A_j(t)) = \frac{A'_k(t)}{A'_j(t)} > 0
\end{equation} 
that implies 
\begin{equation}
  D'_k(t) = \Gamma'_k(D_j(t)) \, D_j(t) >0
\end{equation}
\end{proof}
\fi

\begin{proof}[Proof of Lemma \ref{L.deriv.a.e.}.]
i)  We first obtain lower and upper bounds for  $L(t+h) - L(t)$.
The processes $\vecD(t)$ and $\vecQ(t)$ are almost everywhere differentiable and,  by part ii) of Lemma \ref{pr:D'.positive}, there exists a $T>0$ such that $D_j'(t) >0$ a.e. on $t \geq T|\vecQ(0)|$ for each queue $j$; choose a $t$ satisfying these conditions.
\iffalse
From \eqref{def:entropy}, $H(t)$ admits derivative as soon as the term 
\begin{equation}
L(t) = \sum_{j\in\mJ} Q_j(t) \log D'_j(t)
\end{equation}
does.
\fi
Since $\vecD'(t)$ is suboptimal for the proportional fair optimization (\ref{PF}) with queue size vector $\vecQ(t+h)$ for given $h$,
\begin{equation}
\label{eqnew44}
\begin{split}
L(t+h) - L(t) &= \sum_\jInJ \left(Q_j(t+h)\log D'_j(t+h)-Q_j(t)  \log D'_j(t)\right) \\
&\geq \sum_\jInJ \left(Q_j(t+h)-Q_j(t)\right)  \log D'_j(t).  
\end{split}
\end{equation}
Also, since $\vecD'(t+h)$ is suboptimal for queue size vector $\vecQ(t)$, 
\begin{equation}
\label{eqnew45}
\begin{split}
L(t+h) - L(t) 
\leq  & \sum_{j\in\mJ} \left(Q_j(t+h)-Q_j(t)\right)  \log D'_j(t+h) \\
\leq  & \sum_{\substack{j\in\mJ:\\ Q_j(t) >0}} \left(Q_j(t+h)-Q_j(t)\right)  
\log D'_j(t+h) +\sum_{\substack{j\in\mJ:\\ Q_j(t) =0}} Q_j(t+h)
\log \sigma_{\max} .
\end{split}
\end{equation}
(After splitting the sum over $j$ into two parts depending on whether or not $Q_j(t)=0$, we employed $\log D_j'(t) \leq \log \sigma_{\max}$ in the summation over $j$ with $Q_j(t)=0$.)

After dividing by $h$, we consider the limiting behavior of the left hand sides of (\ref{eqnew44}) and (\ref{eqnew45}), as $h\rightarrow 0$, by employing the bounds on the right hand sides of the equations.
Since $\vecQ'(t)$ exists, after dividing by $h$, the right hand side of  (\ref{eqnew44}) converges to
\begin{equation}\label{Diffy}
\sum_\jInJ Q'_j(t) \log D'_j(t) 
\end{equation}
as  $h\rightarrow 0$. On the other hand,  by Lemma \ref{SigCont},  $D'_j(t)=\sigma_j(\vecQ(t))$ is continuous where $Q_j(t)>0$, and so $D'_j(t+h) \rightarrow D'_j(t)$ as $h\rightarrow 0$. Recalling that $Q'_j(t)=0$ when $Q_j(t)=0$, the first term on the right hand side of (\ref{eqnew45}) also converges to (\ref{Diffy}) after dividing by $h$. 
Moreover, the last term in (\ref{eqnew45}) converges to $0$ after dividing by $h$, since
$Q'_j(t)=0$ for such $j$.  Putting these limits together, (\ref{eq:EnvelopeTheorem}) follows.

ii) Most of the work consists of computing an upper bound for the upper right Dini derivative of the function $L(\cdot)$  at time $t$, which is given by
\begin{equation}\label{eq:L.upper.Dini.deriv}
D^+L(t) = \limsup_{h\searrow 0} \frac{ L(t+h) - L(t)}{h} \, .
\end{equation}
By the same reasoning as in (\ref{eqnew45}), for
$h>0$,
\begin{equation}
\label{eqnew48'}
\begin{split}
\frac{L(t+h) - L(t)}{h} 
&\leq  \sum_{j\in\mJ} \frac{Q_j(t+h)-Q_j(t)}{h} \log D'_j(t+h) \\
& \leq  
\sum_{\substack{j\in\mJ:\\ Q_j(t) =0}} \frac{Q_j(t+h)}{h} \log D'_j(t+h)  
 + \sum_{\substack{j\in\mJ:\\ Q_j(t) >0}}  \frac{Q_j(t+h)-Q_j(t)}{h} \log D'_j(t+h).
\end{split}
\end{equation}
Denoting by $K_A$ and $K_Q$ the Lipschitz constants for the processes $\vecA(t)$ and $\vecQ(t)$, and employing $\vecQ(t) = \vecA(t)-\vecD(t)$, it follows that this is at most
\begin{equation*} 
(K_Q +K_A) |\mJ| \log \sigmamax
- \sum_{\substack{j\in\mJ:\\ Q_j(t) >0}}  \frac{D_j(t+h)-D_j(t)}{h} \log D'_j(t+h). 
\end{equation*}
By Lemma \ref{SigCont} , $D'_j(t+h)$ is continuous at $h=0$ when $Q_j(t)>0$.  Taking the $\limsup$ of the left hand side of (\ref{eqnew48'}), it follows from the above inequalities that
\begin{align}\label{eq:upper.right.derivative}
D^+L(t) \leq  
(K_Q +K_A) |\mJ| \log \sigmamax  - \sum_{\substack{j\in\mJ:\\ Q_j(t) >0}}  D'_j(t) \log D'_j(t) \leq \,  \kappa_L \ 
\end{align}
for some constant $\kappa_L$, since  $x\,\log x$ is bounded from below. It then follows from standard results on Dini derivatives that
\begin{equation}\label{LDev}
L(t)-L(s) \leq \kappa_L (t-s)
\end{equation}
(see, for instance, Theorem 3.4.5 on page 65 of \cite{kannan1996advanced}), which is the desired inequality.

iii) For $n\in\bN$ and fixed $s$, $t$, and $u$, with $u\in [s,t]$, define the dyadic floor and ceiling  $\lfloor u \rfloor_n$ and $\lceil u \rceil_n$ of $u$ by 
\begin{align*}
\lfloor u \rfloor_n &:= \max_{k\in\bZ_+}\{ k (t-s) 2^{-n} + s :  k (t-s) 2^{-n} + s \leq u\}, \\
 \lceil u \rceil_n &:=\min_{k\in\bZ_+}\{ k (t-s) 2^{-n} + s :  k (t-s) 2^{-n} + s > u\};
\end{align*}
note that $\lceil u \rceil_n -\lfloor u \rfloor_n= (t-s)2^{-n}$. By interpolating terms and rewriting them in integral form, one has
\begin{align*}
 L(t) - L(s)
 = & \sum_{k=1}^{2^{n}} \Big( L (k(t-s)2^{-n}+s)- L ((k-1)(t-s)2^{-n}+s) \Big) \\
 =  & \int_s^{t} \frac{ L (\lceil u \rceil_n)- L (\lfloor u \rfloor_n)}{\lceil u \rceil_n -\lfloor u \rfloor_n}  \, du 
  =   \lim_{n \rightarrow \infty} \int_s^t\frac{ L (\lceil u \rceil_n)- L (\lfloor u \rfloor_n)}{\lceil u \rceil_n -\lfloor u \rfloor_n}  \,  du \\
\leq & \int_s^t \limsup_{n \rightarrow \infty}   \frac{ L (\lceil u \rceil_n)- L (\lfloor u \rfloor_n)}{\lceil u \rceil_n -\lfloor u \rfloor_n}  \,  du 
=   \int_s^t L'(u) du.
\end{align*}
The limit in the third equality holds trivially since the expression is not a function of $n$; the inequality follows from the Fatou Inequality since the integrand is bounded above by (\ref{eq:L.Lipschitz.up}); and the last equality follows by using (\ref{eq:EnvelopeTheorem}).
\end{proof}

\iffalse
Defining 
 $\bar f_n(x) = f_n(\left\lceil 2^n \, (x-s)/(t-s) \right\rceil)$ 
and
 $\underline f_n(x) = f_n(\left\lfloor  2^n \, (x-s)/(t-s)  \right\rfloor)$ 
it follows that 
\begin{align}
 L(t) - L(s)
 =   \int_s^t \frac{L (\bar f_n(x))- L (\underline f_n(x))}{(t-s)2^{-n}}   \,  dx 
 = &  \lim_{n \rightarrow \infty} \int_s^t \frac{L (\bar f_n(x))- L (\underline f_n(x))}{(t-s)2^{-n}}   \,  dx \nonumber 
\end{align}
\fi
\iffalse
Since $| \bar f_n(x) - \underline f_n(x)| \leq 2^{-n}$, using the upperbound \eqref{eq:L.Lipschitz.up} and the bounded convergence theorem we get
\begin{align}
 L(t) - L(s)
\leq  &  \int_s^t 
 \lim\sup_{n \rightarrow \infty} 
\frac{L (\bar f_n(x))- L (\underline f_n(x))}{\bar f_n(x) - \underline f_n(x)} \, 
\frac{\bar f_n(x) - \underline f_n(x)}{(t-s)2^{-n}}   \, dx  \nonumber \\
= & \int_s^t \frac{dL(x)}{dt} dx.
\end{align}
where we used that $\bar f_n(x), \, \underline f_n(x) \to x$ and the second fraction converges to $1$.
\fi

\section{Stability Results from Section \ref{PosRec}} \label{AppendixD}

For the proof of Lemma \ref{bound.change.Q}, we require the following technical lemma. 

\begin{lemma}\label{lm:lowerbound.q.sigma}
For any $\veca\in\mC$, there exist $\epsilon >0$ and $c>0$, such that, for any $\vecQ\in\bR_+^{|\mJ|}$ with $|\vecQ|>0$, 
 one has $Q_j \geq c \, |\vecQ|$ and $\sigma_j(\vecQ) \ge (1+\epsilon) a_j $ for some queue $j$.
\end{lemma}
\begin{proof}
Since the assertion is trivial if $a_j =0$ for some $j$, we can assume that $a_j >0$ for all $j$.

Since $\veca \in \mC$, there exists an $\epsilon>0$ such that 
$(1+\epsilon)^2 \, \veca \in  \mC$. Moreover, since $(1+\epsilon)^2 \veca$ is  suboptimal, 
\begin{equation*}
\sum_\jInJ Q_j \log\sigma_j(\vecQ) 
\geq \sum_\jInJ Q_j \log \left( (1+\epsilon)^2 \, a_j \right);
\end{equation*}
hence
$$
\sum_\jInJ Q_j \log \frac{\sigma_j(\vecQ)}{(1+\epsilon) a_j} 
\geq |\vecQ|\log (1+\epsilon),
$$
which implies the existence of at least one queue, say $j$, with
$$
Q_j \log \frac{\sigma_j(\vecQ)}{(1+\epsilon) a_j} 
\geq \frac{|\vecQ|}{|\mJ|} \log (1+\epsilon) 
> 0.
$$
This implies that $\sigma_j(\vecQ) > (1+\epsilon) a_j$, as the logarithm on the right hand side is positive, and also that
$$ \frac{Q_j}{|\vecQ|} \geq \frac{1}{|\mJ|}  \log (1+\epsilon) \left(\log \frac{\sigma_j(\vecQ)}{(1+\epsilon) a_j}\right)^{-1} \geq \frac{1}{|\mJ|} \log (1+\epsilon) \left(\log \frac{\sigma_{\max}}{(1+\epsilon) a_{\min}}\right)^{-1} \ , $$
where $a_{\min} := \min \{a_j: \jInJ\}$.
Denoting by $c>0$ the right hand side of this expression, the desired lower bound on $Q_j$ follows.
\end{proof}

\iffalse
\begin{lemma} \label{bound.change.Q}
There exist constants $c, \, K>0$, such that, for $|\vecQ(0)|>0$, some $j\in\mJ$ 
and some time $t < c \, |\vecQ(0)|$
\begin{equation}
   |Q_j(t) - Q_j(0)| \geq K \, |\vecQ(0)|\ .
\end{equation}
\end{lemma}
\fi

We now prove Lemma \ref{bound.change.Q}, which gives a lower bound on the amount $Q_j(t)$ will change over time for some $j$. The main idea is that, if
$Q_j(t)$ remains nearly constant, then, using
Lemma \ref{lm:lowerbound.q.sigma}, it will follow that one of the queues $j$ will empty at a faster rate than mass enters its route, which will provide a contradiction over a long enough time interval. 

\begin{proof}[Proof of Lemma \ref{bound.change.Q}.]
For $|\vecQ(0)|>0$, it follows from  Lemma \ref{lm:lowerbound.q.sigma} that, for some queue $j$, 
$\sigma_{j}(\vecQ(0)) \ge  ( 1+ \epsilon) a_{j}$, where $\veca$ is the arrival rate vector.
Denote by $\tilde Q_{j}(0)$ the quantity of packets at queue $j$ at time $0$, together with the packets already in the network then
that will eventually be routed through $j$.
Setting $\tilde T = 2 \tilde Q_{j}(0) /(\epsilon \, a_{j})$ and $T = 2 |\vecQ(0)|/(\epsilon \, \amin)$ (with $\amin := \min_{j' \in \mJ} a_{j'}$), one has $T \geq \Tilde T$; note that $\amin > 0$. 

It is not possible that
\begin{equation}
\label{eqnewlemma3}
\sigma_{j}(\vecQ(t)) > a_{j} \left( 1+ \frac{\epsilon}{2} \right) \qquad \text{for all } t \in [0, T]. 
\end{equation}
For, if (\ref{eqnewlemma3}) were to hold, then
$\int_0^T  \sigma_{j}(\vecQ(t)) \, dt > a_{j} \left( 1+ \frac{\epsilon}{2} \right) \, T $,
from which the contradiction
\begin{align}
0 & \,\leq \, \tilde Q_{j}(T) 
  \, = \, \tilde Q_{j}(0) + a_{j} \, T - \int_0^T \sigma_{j}(\vecQ(t)) \, dt \nonumber  \,<\, \tilde Q_{j}(0) - a_{j}  \frac{\epsilon}{2}  \, T 
  \, \leq \, \tilde Q_{j}(0) - a_{j}  \frac{\epsilon}{2}  \, \tilde T \,=\, 0 \nonumber
\end{align}
would follow. Hence, for some $t \in [0, T]$,
$\sigma_{j}(\vecQ(t)) \leq a_{j} \left( 1+ \frac{\epsilon}{2} \right) $;
since the index $j$ was obtained from Lemma \ref{lm:lowerbound.q.sigma}, 
this implies that 
\begin{equation}\label{sigma.distance}
\sigma_{j}(\vecQ(0)) - \sigma_{j}(\vecQ(t)) \ge  a_{j} \frac{\epsilon}{2} \ . 
\end{equation}
By Lemma \ref{SigCont}, there exists $\delta > 0$ such that,
for any $\vecQ$ with $|\vecQ - \vecQ(0)| \leq \delta \, |\vecQ(0)|$, 
\begin{equation}
\label{eqafter53}
|\sigma_{j}(\vecQ) - \sigma_{j}(\vecQ(0))| \leq a_{j} \frac{\epsilon}{4}\,.
  \end{equation}
It therefore follows from \eqref{sigma.distance} that,
with possibly a new choice of $j$,
\begin{equation} \label{eq:delta}
| Q_j(t) - Q_j(0) | > \frac{\delta}{|\mJ|} \, |\vecQ(0)|\,.
\end{equation}
 
We still need to show that the bound in (\ref{eqnewkappa}) is uniform in
$|\vecQ(0)| >0$.  The choice of $\epsilon$, which is determined by Lemma
\ref{lm:lowerbound.q.sigma}, does not depend on $\vecQ(0)$.
Employing Lemma \ref{SigCont}
and  the compactness of $\{\vecQ\in\bR_+^{|\mJ|}:|\vecQ|=1\}$, $\delta$ in \eqref{eq:delta} can also be chosen so as not to depend on
$\vecQ(0)$ since $Q_j (0)/\vecQ(0) \ge c >0$ by
Lemma \ref{lm:lowerbound.q.sigma}, for $j$ in (\ref{eqafter53}).
Setting $\kappa = \delta / |\mJ|$, we obtain (\ref{eqnewkappa}).
\end{proof}

The elementary Lemma \ref{lm:log.bound} follows from a straightforward Taylor expansion.

\begin{proof}[Proof of Lemma \ref{lm:log.bound}.]
Setting
$F(z) = z\log z - (z-1)$, one has $F(1) = F\rq{}(1)= 0$ and $F\rq{}\rq{}(z) = z^{-1}$.  Expanding around $1$ gives 
$$
F(z) = \frac{1}{2} F\rq{}\rq{}(\theta) (z-1)^2 
= \frac{1}{2\theta} (z-1)^2 \geq \frac{1}{2 \max(1,z)} (z-1)^2 
$$
 for some $\theta$ between $1$ and $z$.
Since
\[
x\, F(y/x)=y\log \left(\frac{y}{x}\right) - (y-x)
\]
for $x,y >0$, the desired inequality follows from
$$
 x F(y/x) 
 \geq \frac{x}{2 \max(1,y/x )} \left(\frac{y}{x}-1\right)^2
= \frac{1}{2 \max(x,y) } (y-x)^2 
 \geq \frac{1}{2 K} (y-x)^2 
$$
for $K\ge \max(x,y)$.
\end{proof}

\section{Fluid Limit}
\label{appendFMFL}

\iffalse
We define the processes $A_k(t)$ also for $t\leq0$ such that $Q_k(0)=A_k(0)$.
For the discrete and the continuous model the following equations hold
\begin{align}
D_j(t) &= \sum_{s=0}^t \sigma_j(\vecQ(s)) \label{eq:sigmaj}\\
D_j(t) &= \int_0^t \sigma_j(\vecQ(s)) \, ds \label{eq:cont.sigmaj} \ .
\end{align}
For the discrete time process, for $t > 0$
\begin{equation}\label{eq.arr.disc.time}
A_r(t+1)-A_r(t) \quad  r\in\mR
\end{equation}
are i.i.d. random variables with mean $a_r$, while for  the continuous fluid model and $t>0$
\begin{equation}\label{eq.fluid.model}
A_r(t) - A_r(0) = a_r \, t \quad r\in\mR \ .
\end{equation}
Let $\invA_j(\cdot)$ be the inverse function of $A_j(\cdot)$ that is
\begin{equation}
\invA_j(A_j(t))=\inf\{s:A(s)\geq t\}
\end{equation}
such that $A_j(\invA_j(s))=s$.
Notice, in the second equality that we do not allow for impulsive arrivals.
We have that
\begin{equation}
\Gamma_k(s) = A_k(\invA_j(s))
\end{equation}
For the discrete time model
\begin{equation}
\gamma_k(s) = \Gamma_k(s+1) - \Gamma_k(s) 
\end{equation}
and in the continuos time model
\begin{equation}
\gamma_k(s) = \Gamma'_k(s)  
\end{equation}
\fi

In Section \ref{PROOF}, we employed Proposition \ref{propFMQN} to conclude that
the positive recurrence of a proportional switched network follows from the stability
of the corresponding proportional switch fluid model.  
One of the main steps in showing Proposition \ref{propFMQN} 
is Proposition \ref{FluidLimit}, which states that
the proportional switch fluid model in Section \ref{fluidmodel} is the limit of scaled discrete time switched networks.
We demonstrate  Proposition \ref{FluidLimit} in this subsection; our approach is similar to the standard fluid limit approaches in \cite{Da95} and \cite{Br08}. 

 We consider a family of proportional switched networks indexed by $c\in\bZ_+$, employing
analogous notation 
to that introduced in Section \ref{Model}, e.g., $A^c_x$, $D^c_x$, $Q^c_x$ for $x\in \mJ \cup \mK \cup \mR$.  We will assume that the sum of the initial queue sizes of the queueing network is equal
to $c$, that is,
$c=|\vecQ^c(0)| = \sum_{j\in\mJ} Q^c_j(0)$.
To simplify the proof, we couple the processes on the same probability space, assuming that the external arrival processes are identical for different $c$, that is, for $c\in\bZ_+$ and $r\in\mR$,
\begin{equation*}
 A^c_r(t) =  A_r(t).
\end{equation*}
 
For $x\in \mJ \cup \mK \cup \mR$ and $k\in\mK$, we introduce the 
scaled processes $\bar{A}^c_x(t)$, $\bar{D}^c_x(t)$, $\bar{Q}^c_x(t)$, and $\bar{\Gamma}^c_x(t)$
on $t\ge 0$, by setting
\begin{equation}\label{eq:fluid.scaled.processes}
\bar{A}^c_x(t) = \frac{A_x^c( ct )}{c},\;\;\bar{D}^c_x(t) = \frac{D_x^c(ct)}{c},\;\;\bar{Q}^c_x(t) = \frac{Q_x^c(ct)}{c},\;\; \bar{\Gamma}^c_k(t) = \frac{\Gamma^c_k(ct)}{c}  
\end{equation}
for $t\in\{0,c^{-1},2c^{-1}, 3c^{-1},...\}$, and interpolating linearly.
The following limit result holds for these processes. 
Here, $G = G_1 \cap G_2$ is a set with $\mathbb{P}(G) = 1$, where $G_1$ and $G_2$ will be specified shortly.
\begin{proposition}\label{FluidLimit}
For each $\omega \in G$, any scaled subsequence  $ (\bar{A}^{c_i}_x, \bar{D}^{c_i}_x,\bar{Q}^{c_i}_x,\bar{\Gamma}_k^{c_i} : x\in \mJ \cup\mK \cup \mR, \,k\in\mK,\,c_1 < c_2 < \ldots)$ of a sequence of proportional switched networks contains a further subsequence that converges uniformly on compact time intervals. Moreover, any such limit satisfies the proportional switch fluid model equations (\ref{eq:Incr}-\ref{eq:ADk}) and (\ref{Fluid:1}-\ref{PF}). 
\end{proposition}

In order to demonstrate Proposition \ref{FluidLimit}, we recall that, for each $r\in \mR$,  $(A_r(t)-A_r(t-1): t\in \mathbb{Z}_+)$ is a collection of i.i.d. random variables with  
mean $a_r < \infty$. It thus follows from
the Strong Law of Large Numbers that, on a set $G_1$ with $\mathbb{P}(G_1) = 1$, 
one has that, for each $r\in\mR$,  
$\bar{A}^c_r(t)\to a_r t$ as $c\to \infty$
uniformly on compact time intervals. 

We also introduce, for each $j$, the martingales
\[
M_j^c(t) = \sum_{\tau=1}^t \left[ \pi_j({\vecQ}^{c}(\tau)) - \sigma_j({\vecQ}^{c}(\tau)) \right] ,
\]
 where $\pi_j(\cdot)$ and $\sigma_j(\cdot)$  are as in (\ref{DServe}), and set
$\bar{M}^c_j(t) = M^c_j(ct)/c$.
Since the increments of $\bar{M}^c_j(t)$ are uniformly bounded, it will follow from standard 
martingale bounds that, on a set with probability $1$, $\bar{M}_j^c(t) \to 0$ as $c\to \infty$, uniformly on compact time intervals for each $j\in\mJ$:

\begin{lemma} 
\label{lastlemma}
On a set $G_2$ with $\mathbb{P}(G_2) = 1$,
\[
\sup_{t\leq T} \Big|\bar{M}_j^c(t)  \Big|\xrightarrow[c\rightarrow \infty]{} 0
\]
 for all $j\in\mJ$ and $T>0$.
\end{lemma}

We will demonstrate Lemma \ref{lastlemma} after the proof of Proposition \ref{FluidLimit}. 

The proof of Proposition \ref{FluidLimit} is relatively straightforward. Most of the work consists
of showing the above limits satisfy the equations (\ref{eq:Incr}-\ref{eq:ADk}) and (\ref{Fluid:1}-\ref{PF}); the invariance of
 the equations (\ref{eq:Incr}-\ref{eq:ADk}) under fluid scaling is an important ingredient. 

\begin{proof}[Proof of Proposition \ref{FluidLimit}.]

By the Arzel\`a-Ascoli Theorem (cf. \cite{Bi99}), any sequence of equicontinuous functions $X^{c_i} (t)$ on $[0,T]$, $T\in (0, \infty)$, such that $\sup_{c_i} |X^{c_i} (0)| < \infty$, has a converging subsequence with respect to the uniform norm.  In order to apply the theorem, we will show that the sequences
$ (\bar{A}^{c_i}_x, \bar{D}^{c_i}_x,\bar{Q}^{c_i}_x,\bar{\Gamma}_k^{c_i})$ of $4$-tuples
satisfy both properties for $\omega \in G_1$.
\iffalse
\begin{equation}\label{ModConv}
\lim_{\delta \rightarrow \infty} \sup_{X\in S} w_X(\delta) = 0,
\end{equation}
where $w_X(\delta)$ is the modulus of continuity 
\begin{equation*}
w_X(\delta) = \sup_{\substack{0\leq s,t \leq T:\\ |t-s|< \delta }} \left| X(t) - X(s) \right|.
\end{equation*}
\fi

Since $|\bar{\vecQ}^{c_i}(0)|=1$ and the other variables are initially $0$, the second property
clearly holds.  In order to show equicontinuity, we first note that each packet requires one unit of service, and so $\bar{D}^c_x(t)$ is Lipschitz for $x\in\mJ\cup \mR \cup \mK$, $k\in\mK$; by
equation \eqref{eq:gammak}, so is $\bar{\Gamma}^c_k(t)$.
Hence, both variables are equicontinuous. 

On the other hand, equicontinuity of $\bar{A}^{c_i}_r (t)$ follows by applying 
the Functional Strong Law of Large Numbers to $\bar{A}^{c_i}_r(t)$ for increasingly fine meshes in $t$. Equicontinuity for the remaining variables $\bar{A}^{c_i}_x (t)$ and $\bar{Q}^{c_i}_x (t)$ follows from straightforward calculations using the equations (\ref{eq:Incr}-\ref{eq:ADk}). Consequently, by the Azel\`a-Ascoli Theorem, every subsequence $\{ (\bar{A}^{c_i}_x, \bar{D}^{c_i}_x,\bar{Q}^{c_i}_x,\bar{\Gamma}_k^{c_i} : x\in \mJ \cup\mK \cup \mR, k\in\mK) \}$ has a further subsequence that converges uniformly on $[0,T]$.

We now show that equations (\ref{eq:Incr}-\ref{eq:ADk}) and (\ref{Fluid:1}-\ref{PF}) hold for 
the limit of any 
converging subsequence of $\{ (\bar{A}^{c_i}_x, \bar{D}^{c_i}_x,\bar{Q}^{c_i}_x,\bar{\Gamma}_k^{c_i} : x\in \mJ \cup\mK \cup \mR, k\in\mK) \}$; we will demonstrate this on $G_1$ for
all equations except (\ref{Fluid:3}-\ref{PF}).
With the exception of (\ref{eq:Dk}-\ref{eq:Ak}), equations (\ref{eq:Incr}-\ref{eq:ADk}) 
clearly hold for any limit since each holds under the prelimit. We must take a little more care for expressions 
(\ref{eq:Dk}-\ref{eq:Ak}),  because some minor error is introduced by linear interpolating terms; the bound
\begin{equation*}
\left| D^{c}_k(s) - \Gamma^{c}_k(D^c_j(s)) \right| \leq \left| D^c_k(\lceil s \rceil) - D^c_k(\lfloor s\rfloor)\right| \leq \sigma_{\max}
\end{equation*}
shows that this error vanishes in the limit. 
On $G_1$, \eqref{Fluid:1} follows directly from the Strong Law of Large Numbers. 

We still need to demonstrate \eqref{Fluid:3} for $\sigma (\vecQ)$ satisfying (\ref{PF}), which we show for $\omega \in G_2$,
where $G_2$ will be chosen as in Lemma \ref{lastlemma}.
Let $(Q_j,D_j : j\in\mJ)$ be the limit of a subsequence of $\{ (\bar{Q}_j^{c_i}, \bar{D}_j^{c_i}: j\in\mJ ) \}$. Suppose that $Q_j(t)>\epsilon$ for some $\epsilon>0$ and a given $j$.  Then, by the continuity of $\vecQ(t)$, there exists $\delta>0$ such that $Q_j(t+h)>\epsilon$ for all $h$ satisfying $|h|\le \delta$. Since convergence is uniform, there also exists $c\rq{}$ such that, for all $c_i\ge c\rq{}$, $\bar{Q}^{c_i}_j(t+h)>\epsilon$ for all such $h$. 

For $u\in\bR_+$, denote by $\lfloor u \rfloor_{c_i}$ be the largest value of $k c_{i}^{-1}$, $k\in\bZ$, that is at most $u$.
The scaled departures over $(\lfloor t \rfloor_{c_i}, \lfloor t+h \rfloor_{c_i}]$ can be rewritten as 
\begin{align}
&\bar{D}^{c_i}_j(\lfloor t+h \rfloor_{c_i}) - \bar{D}^{c_i}_j(\lfloor t \rfloor_{c_i})\notag 
= \frac{1}{c_i}\sum_{\tau=\lfloor t \rfloor_{c_i}+1/c_i}^{\lfloor t+h \rfloor_{c_i}} \pi_j(\bar{\vecQ}^{c_i}(\tau))\notag \\
=  &\frac{1}{c_i}\sum_{\tau=\lfloor t \rfloor_{c_i}+1/c_i}^{\lfloor t+h \rfloor_{c_i}} \sigma_j(\bar{\vecQ}^{c_i}(\tau))  + \bar{M}^{c_i}_j(\lfloor t+h \rfloor_{c_i}) - \bar{M}^{c_i}_j(\lfloor t \rfloor_{c_i})\notag \\
= & \int_{\lfloor t \rfloor_{c_i} +1/c_i}^{\lfloor t+h \rfloor_{c_i}} \sigma_j(\bar{\vecQ}^{c_i}(\lfloor \tau \rfloor_{c_i}))   d \tau + \bar{M}^{c_i}_j(\lfloor t+h \rfloor_{c_i}) - \bar{M}^{c_i}_j(\lfloor t \rfloor_{c_i})
\label{FinalBound}\,,
\end{align}
where the summations are understood to be over values in $\{0,c_{i}^{-1}, 2c_{i}^{-1},...\}$.
The square bracketed terms on the last line of \eqref{FinalBound} are bounded.
For $\omega\in G_2$, with $G_2$ as in Lemma \ref{lastlemma}, the last two terms in \eqref{FinalBound} converge uniformly to zero. 
On the other hand, by uniform convergence, the terms $\bar{\vecQ}^{c_i}(\lfloor \tau \rfloor_{c_i})$ on the subsequence converge to $\vecQ_j(\tau)$. Also, $\sigma_j(\vecQ)$ is bounded and continuous on $Q_j>0$. Thus, applying the Bounded Convergence Theorem, 
it follows from (\ref{FinalBound}) that
\begin{equation*}
{D}_j(t+h) - {D}_j(t) = \int_{t}^{t+h} \sigma_j({\vecQ}(\tau))   d \tau\,,
\end{equation*}
where $\sigma_j$ is as in (\ref{PF}).
Dividing by $h$ and taking the limit $h\searrow 0$ implies $D\rq{}_j(t) = \sigma_j(\vecQ(t))$, and hence that \eqref{Fluid:3} also holds.
\end{proof}

\smallskip

\begin{proof}[Proof of Lemma \ref{lastlemma}.]
For $\epsilon > 0$, let
$\tau_\epsilon$ denote the stopping time for the event $\{ |M_j^c(t)| \ge c\epsilon  \}$.  
Since, for each $c$, $M_j^c$ is a martingale with increments uniformly bounded by 
$\sigma_{\text{max}}$, it follows by applying 
the Azuma-Hoeffding Inequality \cite[E14.2]{Wi91} to $\{ M_j^{c}(\min(\tau_\epsilon, s)) \}_{s\leq cT}$ that
\begin{align*}
  \bP \left(  \max_{t: t \leq c T} \left| M_j^{c}(t)\right| \ge c\epsilon  \right) 
= & \bP\Big( \big|M_j^c\big(\min(\tau_\epsilon, c \, T)\big)\big| \ge c\epsilon \Big) \leq  2 \exp\left\{ - \frac{c^2\epsilon^2}{2 cT \sigma_{\max}^2 } \right\}.
\end{align*}
After summing over $c\in\bN$, the right hand side of the above inequality is finite and hence,
by the Borel-Cantelli Lemma, 
\[
\limsup_{c\rightarrow \infty}  \max_{t: t\leq c T} \frac{1}{c} \left| M_j^{c}(t) \right|  \leq \epsilon
\]
holds almost surely.  Since $\epsilon >0$ is arbitrary, this implies the claim in the lemma.
\end{proof}

\iffalse
\NW{We briefly comment that if we allowed for schedules to be random, as discussed in Remark \ref{ScheduleRemark}, then the convergence of   \eqref{FinalBound} would still hold under the Martingale Convergence Theorem. And thus fluid model under random schedules (with expected values $\mS$) is given by the same fluid model equations. Since our fluid analysis does not require the set of schedules to be integral, positive recurrence holds in this case also.}
\fi

\section{Positive Recurrence}
\label{Appendix F}

In this subsection, we demonstrate Proposition \ref{propFMQN}, which states that 
positive recurrence for a FIFO proportional switched network follows
from the fluid model stability of the corresponding  proportional switch fluid model.  The
main tools are Proposition \ref{FluidLimit} of the previous subsection and the Multiplicative Foster\rq{}s Criterion.

\begin{proof}[Proof of Proposition \ref{propFMQN}.]

One can check that, for each $t\geq 0$, the sequence of queue sizes 
$\{ |\bar{\vecQ}^{c}(t)| \}_c$ of the
scaled proportional switched networks in Proposition \ref{FluidLimit}
is uniformly integrable. This follows quickly from the inequality
\begin{equation}
\label{equnifint}
|\bar{\vecQ}^{c}(t)| =\sum_{j\in\mJ} \bar{Q}_j^{c}(t)\leq \sum_{j\in\mJ} \frac{Q^{c}_j(0)}{c} + \sum_{r\in\mR} \frac{A^{c}_r(t)}{c}
\end{equation}
since $A^{c}_r(t)$ is a sum of i.i.d. random variables with finite mean (see, e.g., \citet[ Lemma 4.13, (4.81)]{Br08}). 

On the other hand, by Proposition \ref{FluidLimit}, on a set of probability one, every subsequence of
$\bar{\vecQ}^{c}(t)$ has a further subsequence that converges uniformly on compact time
 intervals to a fluid model solution $\vecQ(t)$ of (\ref{eq:Incr}--\ref{eq:ADk}) and (\ref{Fluid:1}--\ref{PF}), with $|\vecQ(0)|=1$. By Theorem \ref{FluidStable}, this fluid model is stable; hence all fluid model solutions with $|\vecQ(0)|=1$  satisfy $|\vecQ(t)| = 0$ for $t\ge \gamma$, with $\gamma$ not depending on the particular
fluid model solution. 

It follows from this that, on a set of probability one, every subsequence of 
$|\bar{\vecQ}^{c}(\gamma)|$ has a further subsequence converging to zero; consequently,  
$|\bar{\vecQ}^{c}(\gamma)|$ also converges to zero on a set of probability one. By the above uniform integrability and almost sure convergence of $|\bar{\vecQ}^{c}(\gamma)|$, it follows that
\begin{equation*}
\lim_{c\rightarrow\infty} \bE |\bar{\vecQ}^{c}(\gamma)|  =0\,,
\end{equation*}
and so, if $c\ge \tilde{c}$ for appropriate $\tilde{c}$, then
$\bE |\bar{\vecQ}^{c}(\gamma)|   \le 1/2$.  

The last inequality is equivalent to 
$\bE |\vecQ^{c}(\gamma c)|   \le c/2$; together with (\ref{equnifint}), it implies that, for large enough $\kappa$ and all $c$, 
\begin{equation}
\label{eqMFC}
\bE |\vecQ^{c}(\gamma (\max(c,\kappa)))|   \le \max(c,\kappa)/2 \,.
\end{equation}
The inequality (\ref{eqMFC}) satisfies the main premise of the Multiplicative Foster\rq{}s Criterion \cite[Proposition 4.6, (4.28)]{Br08}, and therefore implies the positive recurrence of the corresponding
proportional switched network, which completes the proof of the proposition.  
(The criterion is stated in \cite{Br08} for continuous time
Markov processes; however, both the criterion and its proof carry over to discrete time.  Alternatively, the
discrete time Markov process can be embedded in continous time.  The petite set assumption in the criterion is automatically satisfied in our framework since the empty set will be hit with uniformly high
probability from sets with a bounded number of packets.)
\iffalse
Since the state space is discrete, we have that for any $K>0$ the set
$$F_K = \{ x \in \mathcal X: ||x|| \leq K\}$$
is finite.
From \eqref{eq:Q.hitting.time.of.zero} and \eqref{eq:fluid.scaled.processes} we have that, with $T=c_5$
$$
\limsup_{\vecQ(0) \to\infty} \bE_{\vecQ(0)}[||\bar X(T)||] 
= \limsup_{\vecQ(0)\to\infty} \frac{\bE_{\vecQ(0)}[||\bar X(|\vecQ(0)| \, T)||] }{|\vecQ(0)|} = 0 \ .
$$
The property follows from a derivate result of the Foster's criterion, see Corollary 9.8, page 259 in \cite{Ro03}.
\fi
\end{proof}

\end{document}


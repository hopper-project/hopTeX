\documentclass[a4paper,11pt]{amsart}

\usepackage{amssymb}
\usepackage{mathrsfs}
\usepackage{amsmath,amssymb,amsthm,latexsym,amscd,mathrsfs}
\usepackage{indentfirst}
\usepackage{stmaryrd}
\usepackage{graphicx}
\usepackage{extarrows}
\usepackage{amssymb}
\usepackage{amsmath,amssymb,amsthm,latexsym,amscd,mathrsfs}
\usepackage{indentfirst}
\usepackage{multirow}
\usepackage{latexsym}
\usepackage{amsfonts}
\usepackage{color}
\usepackage{pictexwd,dcpic}
\usepackage{graphicx}
\usepackage{psfrag}
\usepackage{hyperref}
\usepackage{comment}

 \setlength{\parindent}{2em}
 \setlength{\parskip}{3pt plus1pt minus2pt}
 \setlength{\baselineskip}{20pt plus2pt minus1pt}
 \setlength{\textheight}{21 true cm}
 \setlength{\textwidth}{14.5true cm}
  \setlength{\headsep}{10truemm}
  \addtolength{\hoffset}{-12mm}
 
 

\numberwithin{equation}{section} \theoremstyle{plain}
\newtheorem{thm}{Theorem}[section]
\newtheorem*{thm0}{Theorem}
\newtheorem{prop}[thm]{Proposition}
\newtheorem{lem}[thm]{Lemma}
\newtheorem{cor}[thm]{Corollary}
\newtheorem{defn}[thm]{Definition}
\newtheorem{conj}[thm]{Conjecture}
\newtheorem{exmp}[thm]{Example}
\newtheorem{prob}[thm]{Problem}
\newtheorem*{prob0}{Problem}
\newtheorem{rem}[thm]{Remark}
\newtheorem*{rem0}{Remark}
\newtheorem{ques}[thm]{Question}
\newtheorem*{ques0}{Question}
\newtheorem*{acknow}{Acknowledgments}
\newtheorem{ack}{Acknowledgements}   
 \makeatletter

\makeatother
\title[DDVV inequality for Hermitian matrices]{DDVV-type inequality for Hermitian matrices}
\author[J.Q. Ge]{Jianquan Ge}
\address{School of Mathematical Sciences, Laboratory of Mathematics and Complex Systems, Beijing Normal
University, Beijing 100875, P.R. CHINA.}
\email{jqge@bnu.edu.cn}

\author[S. Xu]{Song Xu}
\email{mike9710@163.com}

\author[H. Y. You]{Hangyu You}
\email{779640002@qq.com}

\author[Y. Zhou]{Yi Zhou}
\email{zyzy8369@163.com}

\subjclass[2010]{15A45, 15B57, 53C42.}
\date{}
\keywords{DDVV inequality; Hermitian matrices; Commutator.}

\thanks{The first author is partially supported by the NSFC (No. 11331002, 11522103). }
\begin{document}
\maketitle

\begin{abstract}
About 2007, the DDVV conjecture (also called normal scalar curvature conjecture) in submanifold geometry was proven independently and differently by Lu \cite{Lu11} and Ge-Tang \cite{GT08}.
Its algebraic version is an optimal inequality about the sum of all squared norms of the commutators of real symmetric matrices.
This kind of matrices inequality turns out to be useful not only in geometry but also in other subjects, e.g., physics, random matrix, etc.
Ge \cite{Ge14} established such matrices inequality for real skew-symmetric matrices and applied to give a Simons-type inequality for (generalized) Yang-Mills fields in Riemannian submersions geometry.

In this paper we establish this DDVV-type inequality for Hermitian and skew-Hermitian matrices, following the method of Ge-Tang \cite{GT08}.
\end{abstract}

\section{Introduction}\label{introduction}
In submanifold geometry, it is very important to know more close relations between intrinsic invariants and extrinsic invariants.
In 1999, De Smet, Dillen, Verstraelen and Vrancken \cite{DDVV99} proposed the normal scalar curvature conjecture (DDVV conjecture):
\begin{conj}\label{DDVVgeom}
    Let $M^n\rightarrow N^{n+m}(c)$ be an isometric immersed $n$-dimensional submanifold in the real space form with constant sectional curvature $c$.
    Then between the scalar curvature $\rho$ (intrinsic invariant), the mean curvature $H$ and the normal scalar curvature $\rho^\bot$ (extrinsic invariants),
    there is the following pointwise inequality:
    $$\rho+\rho^\bot\leq\|H\|^2+c.$$
\end{conj}
There is an equivalent algebraic version:
\begin{conj}\label{DDVValg}
     Let $B_1,\cdots,B_m$ be $n\times n$ real symmetric matrices. Then we have the matrices inequality:
$$\sum^m_{r,s=1}\|[B_r,B_s]\|^2\leq{\left(}\sum^m_{r=1}\|B_r\|^2{\right)}^2,$$
where $[A,B]=AB-BA$ is the commutator and
$\|B\|^2=\operatorname{tr}(BB^t)$ is the squared norm.
\end{conj}
This conjecture had generated considerable interest in the past decades.
Many partial results towards the proof, generalizations (e.g., to Lorentz version and other ambient spaces)
and applications (e.g., in scalar curvature pinching problem, physics and random matrix theory) were obtained.
About 2007, the DDVV conjecture was completely proven by Lu \cite{Lu11} and Ge-Tang \cite{GT08} independently and differently.
The equality condition of the DDVV inequality in Conjecture \ref{DDVValg} was also given, so was that in Conjecture \ref{DDVVgeom} at a point of the submanifold $M$.
However, submanifolds achieving the equality everywhere of the DDVV inequality (so-called Wintgen ideal submanifolds) are still unclassified, although many partial results and studies are available in literature (cf. \cite{CL08}, \cite{DT09}, \cite{XLMW14}, etc.).
Very recently the geometric DDVV inequality was strengthened and the equality case was classified on the focal submanifolds of isoparametric hypersurfaces in unit spheres by Ge-Tang-Yan \cite{GTY16}.
For more detailed history of the DDVV conjecture we refer the readers to the papers \cite{Lu07}, \cite{LW11}, \cite{GT11} and references therein.

For the convenience of readers, we restate the algebraic DDVV inequality and its equality condition in the following.
Throughout this paper a $K:=O(n)\times O(m)$ (resp. $\widetilde{K}:=U(n)\times O(m)$) action on $(B_1,\cdots,B_m)$ means that\\
$$(P,R)\cdot(B_1,\cdots,B_m):=(P^*B_1P,\cdots,P^*B_mP)\cdot R,$$ for $(P,R)\in K$ (resp. $(P,R)\in \widetilde{K}$).
\begin{thm}\label{DDVVsymthm}(Lu \cite{Lu11}, Ge-Tang\cite{GT08})
Let $B_1,\cdots,B_m$ be $n\times n$ real symmetric matrices. Then we have
$$\sum^m_{r,s=1}\|[B_r,B_s]\|^2\leq{\left(}\sum^m_{r=1}\|B_r\|^2{\right)}^2,$$
where the equality holds if and only if under some $K$ action all $B_r$'s are zero except for $2$
matrices which can be written as $diag(A_1,0)$ and $diag(A_2,0)$, where $0\in M(n-2)$ is the zero matrix of order $n-2$ and
$$A_1:=
\begin{pmatrix}
    \lambda & 0 \\
    0 & -\lambda \\
\end{pmatrix},
\quad
A_2:=
\begin{pmatrix}
    0 & \lambda \\
    \lambda & 0 \\
\end{pmatrix},
$$
where $\lambda\geq0$, $\lambda\in\mathbb{R}$.
\end{thm}
As a ``double dual", Ge \cite{Ge14} established the DDVV-type inequality for real skew-symmetric matrices (dual to real symmetric matrices),
and applied to give a Simons-type inequality for (generalized) Yang-Mills fields in Riemannian submersions geometry (dual to Simons inequality for minimal submanifolds of spheres in submanifold geometry \cite{Lu11}). The dual phenomenon between Yang-Mills fields and minimal submanifolds was initially investigated by Tian \cite{Ti00}.
\begin{thm}\label{DDVVskewthm}(Ge \cite{Ge14})
Let $B_1,\cdots,B_m$ be $n\times n$ real skew-symmetric matrices.
\begin{enumerate}
  \item If\ $n=3$, then we have
  $$\sum^m_{r,s=1}\|[B_r,B_s]\|^2\leq\frac13{\left(}\sum^m_{r=1}\|B_r\|^2{\right)}^2,$$
where the equality holds if and only if under some $K$ action all $B_r$'s are zero except for $3$
matrices which can be written as
$$C_1:=
\begin{pmatrix}
    0 & \lambda & 0 \\
    -\lambda & 0 & 0 \\
    0 & 0 & 0
\end{pmatrix},
C_2:=
\begin{pmatrix}
    0 & 0 & \lambda \\
    0 & 0 & 0 \\
    -\lambda & 0 & 0
\end{pmatrix},
C_3:=
\begin{pmatrix}
    0 & 0 & 0 \\
    0 & 0 & \lambda \\
    0 & -\lambda & 0
\end{pmatrix}
$$
  \item If $n\geq4$, then we have
  $$\sum^m_{r,s=1}\|[B_r,B_s]\|^2\leq\frac23{\left(}\sum^m_{r=1}\|B_r\|^2{\right)}^2,$$
where the equality holds if and only if under some $K$ action all $B_r$'s are zero except for $3$
matrices which can be written as $diag(D_1,0)$, $diag(D_2,0)$ and $diag(D_3,0)$, where $0\in M(n-4)$ is the zero matrix of order $n-4$ and
$$D_1:=
\begin{pmatrix}
    0 & \lambda & 0 & 0 \\
    -\lambda & 0 & 0 & 0 \\
    0 & 0 & 0 & \lambda \\
    0 & 0 & -\lambda & 0
\end{pmatrix},
D_2:=
\begin{pmatrix}
    0 & 0 & \lambda & 0 \\
    0 & 0 & 0 & -\lambda \\
    -\lambda & 0 & 0 & 0 \\
    0 & \lambda & 0 & 0
\end{pmatrix},
D_3:=
\begin{pmatrix}
    0 & 0 & 0 & \lambda \\
    0 & 0 & \lambda & 0 \\
    0 & -\lambda & 0 & 0 \\
    -\lambda & 0 & 0 & 0
\end{pmatrix},
$$
where $\lambda\geq0,\lambda\in\mathbb{R}$.
\end{enumerate}
\end{thm}

These DDVV-type matrices inequalities are naturally related to the B\"{o}ttcher-Wenzel conjecture \cite{BW05}:
\begin{conj}\label{BWconj}
Let $X, Y$ be two $n\times n$ matrices. Then
$$\|[X, Y]\|^2\leq 2\|X\|^2\|Y\|^2.$$
\end{conj}
This conjecture has been proven independently by Lu \cite{Lu11}, \cite{Lu12}, B\"{o}ttcher-Wenzel \cite{BW08} and Vong-Jin \cite{VJ08}.
In particular, this inequality holds also for arbitrary complex matrices, which is not a mere triviality as shown by B\"{o}ttcher and Wenzel \cite{BW08}.
A unified generalization of the DDVV inequality and the B\"{o}ttcher-Wenzel inequality has been conjectured and ongoing studied by Lu and Wenzel \cite{LW11}.

In this paper, we extend the DDVV-type inequalities from real matrices to complex matrices.
As in the real case, we consider the complex matrices with symmetries, namely, the Hermitian matrices and the skew-Hermitian matrices.
In fact, since $\sqrt{-1}B$ is skew-Hermitian for any Hermitian matrix $B$, the inequality for skew-Hermitian case is the same as the Hermitian case, which is different with the real case in Theorem \ref{DDVVsymthm} and Theorem \ref{DDVVskewthm}.
\begin{thm}\label{DDVVHermthm}
Let $B_1,\cdots,B_m$ be $n\times n$ (skew-)Hermitian matrices.
\begin{enumerate}
\item If\ $m\geq3$, then we have
$$
\sum^m_{r,s=1}\|\[B_r,B_s{\right]}\|^2\leq \frac43{\left(}\sum^m_{r=1}\|B_r\|^2{\right)}^2,
$$
where the equality holds if and only if under some $\widetilde{K}$ action all $B_r$'s are zero except for $3$
matrices which can be written as $diag(H_1,0)$, $diag(H_2,0)$ and $diag(H_3,0)$, where $0\in M(n-2)$ is the zero matrix of order $n-2$ and
$$H_1:=
\begin{pmatrix}
\lambda & 0 \\
0 & -\lambda \\
\end{pmatrix},
H_2:=
\begin{pmatrix}
0 & \lambda \\
\lambda & 0 \\
\end{pmatrix},
H_3:=
\begin{pmatrix}
0 & -\lambda\sqrt{-1} \\
\lambda\sqrt{-1} & 0 \\
\end{pmatrix},
$$
where $\lambda\geq0,\lambda\in\mathbb{R}$.
\item If $m=2$, then we have
$$
\sum^2_{r,s=1}\|\[B_r,B_s{\right]}\|^2\leq {\left(}\sum^2_{r=1}\|B_r\|^2{\right)}^2
$$
where the equality holds if and only if under some $\widetilde{K}$ action $B_1=diag(H_1,0)$ and $B_2=diag(\cos \theta H_2+\sin \theta H_3,0)$.
\end{enumerate}
\end{thm}
The second inequality of Theorem \ref{DDVVHermthm} is implied by the B\"{o}ttcher-Wenzel inequality
and its proof can be even traced back to Chern-do Carmo-Kabayashi \cite{CDK} for real symmetric matrices; hence we omit its proof.
The equality condition follows simply by restricting the matrices to be Hermitian (cf. \cite{BW08}).
It seems that the first inequality of Theorem \ref{DDVVHermthm} would also hold for arbitrary complex matrices, at least for real matrices. 
The reason is that any matrix can be written as the sum of Hermitian (symmetric) and skew-Hermitian (skew-symmetric) matrices,
which enables one to combine their orthonormal basis and use this combined basis to compute the commutators as in (\ref{comm-basis}, \ref{comm-basis1}, \ref{comm-basis2}).
As for possible geometric applications of this DDVV-type inequality for Hermitian matrices, we believe that it would also derive a Simons-type inequality in K\"{a}hler (complex) geometry with certain ``Hermitian" tensors instead of second fundamental tensor in submanifold geometry as in the real case.

The proof follows the method of Ge and Tang \cite{GT08}, \cite{Ge14}, though the computation is more complicated.
\section{Notations and preparatory lemmas}
Throughout this paper, we denote the space of $m\times n$ real matrices by $M(m,n)$, the space of $n\times n$ real matrices by $M(n)$, and the space of $n\times n$ Hermitian matrices $HM(n)$, which has dimension $N:=n^2$.

For every $(i,j)$ with $1\leq i,j\leq n$, let
$$
\check{E}_{ij}:=\
\begin{cases}
E_{ii}                                 &if\ i=j,\\
(E_{ij}+E_{ji})/\sqrt{2}               &if\ i<j,\\
\sqrt{-1}(E_{ij}-E_{ji})/\sqrt{2}      &if\ i>j,
\end{cases}
$$
where $E_{ij}\in M(n)$ is the matrix with 1 in position $(i,j)$ and 0 elsewhere. Clearly $\{\check{E}_{ij}\}$ is an orthonormal basis of $HM(n)$. Let us put an ordering on the index set $S:=\{(i,j)\ |\ 1\leq i,j\leq n\}$ by saying
\begin{equation}\label{eqno1}
(i,j)<(k,l)\Longleftrightarrow i<k, or \ i=k\ and \ j<l.
\end{equation}
We use this ordering to index elements of $S$ with a single (Greek) index in the range $\{1,...,N\}$.

For $\alpha =(i,j)\leq \beta =(k,l)$ in $S$, direct calculations imply
\begin{equation}\label{eqno2}
\left\|{\left[}\check{E}_\alpha,\check{E}_\beta{\right]}\right\|^2=
\begin{cases}
2         &if\ i=l<j=k,\\
\\
1         &if\ i=j=k<l\ |\ i<j=k=l\ |\ i=j=l<k\ |\ j<i=k=l,\\
\\
          &\ \ \ \ \ i<j=k<l\ |\ i=k<j<l\ |\ i<k<j=l\ |\ j=l<i<k\ |\\
\frac12   &if\ j<l<i=k\ |\ j<i=l<k\ |\ i<j=l<k\ |\ l<i<j=k\ |\\
          &\ \ \ \ \ i<l<j=k\ |\ i=l<j<k\ |\ i=l<k<j\ |\ j<k=i<l,\\
\\
0         &otherwise,
\end{cases}
\end{equation}
and for any $\alpha, \beta\in S$,
\begin{equation}\label{eqno3}
\sum_{\gamma}{\langle}[\check{E}_\alpha,\check{E}_\gamma],[\check{E}_\beta,\check{E}_\gamma]{\rangle}=2n\delta_{\alpha\beta}-2\delta_{\alpha}\delta_{\beta},
\end{equation}
where $\delta_{\alpha\beta}=\delta_{ik}\delta_{jl},\delta_{\alpha}=\delta_{ij},\delta_{\beta}=\delta_{kl}$, and $\<A,B{\rangle}=\operatorname{Re}{\left[}\operatorname{tr}\(AB^*{\right)}{\right]}$.

Let $\{\check{Q}_\alpha\}_{\alpha\in S}$ be any orthonormal basis of $HM(n)$. Then there exists a unique
orthogonal matrix $Q\in O(N)$ such that
\begin{equation}\label{Qeq}
(\check{Q}_1,\cdots,\check{Q}_N)=(\check{E}_1,\cdots,\check{E}_N)Q,
 \end{equation}
 that is, $\check{Q}_\alpha=\sum_\beta q_{\beta\alpha}\check{E}_\beta$ for $Q=(q_{\alpha\beta})_{N\times N}$, and if $\check{Q}_\alpha=(\check{q}_{ij}^\alpha)_{n\times n}$, $\beta=(i,j), \gamma=(j,i)$,
\begin{equation}\label{qalpha}
\check{q}_{ij}^\alpha=\overline{\check{q}_{ji}^\alpha}=
\begin{cases}
q_{\beta\alpha}                                               &if\ i=j,\\
(q_{\beta\alpha}-q_{\gamma\alpha}\sqrt{-1})/\sqrt{2}          &if\ i<j.
\end{cases}
\end{equation}

Let $\lambda_1, \cdots, \lambda_n$ be $n$ real numbers satisfying $\sum_{i=1}^n\lambda_i^2=1$ and $\lambda_1\geq\cdots\geq\lambda_n$. Define
$I_1:=\{j\ |\ \lambda_i-\lambda_j>\frac2{\sqrt{3}}\}, I_2:=\{i\ |\ \lambda_i-\lambda_j>\frac2{\sqrt{3}}\}$, and $$I:=\{(i,j)\in S\ |\ \lambda_i-\lambda_j>\frac2{\sqrt{3}}\}.$$
Let $n_0$ be the number of elements of $I$. Then $({1}\times I_1)\bigcup(I_2\times {n})\subset I\subset S$.

\begin{lem}\label{lem1}
Either $I=\{1\}\times I_1$ or $I=I_2\times\{n\}$.
\end{lem}
\begin{proof}
If $n_0=0$, the three sets are all empty. If $n_0=1$, the single element must be $(1,n)$, and the three sets are equal. If $n_0\geq2$, let $(1,n)$ and $(i_1,j_1)$ be two different elements of $I$, that is
$\lambda_1-\lambda_n\geq\lambda_{i_1}-\lambda_{j_1}>\frac2{\sqrt{3}}$ and $(1,n)\neq(i_1,j_1)$. We assert that either  $i_1=1$ and $j_1\neq n$ or $i_1\neq1$ and $j_1=n$, which shows exactly that $I=\{1\}\times I_1\cup I_2\times\{n\}$. Otherwise, $1$,$i_1$,$j_1$ and $n$ would be four different elements in $\{1,\cdots,n\}$, and thus
$$
1\geq\lambda^2_1+\lambda^2_{i_1}+\lambda^2_{j_1}+\lambda^2_n\geq\frac{1}2(\lambda_1-\lambda_n)^2+\frac{1}2(\lambda_{i_1}-\lambda_{j_1})^2>\frac{4}{3}
$$
is a contradiction. Next, without loss of generality, we assume  $(i_1,j_1)\in\{1\}\times I_1$. Then it will be seen that $I_2\times\{n\}={(1,n)}$, and thus $I=\{1\}\times I_1$, which completes the proof. Otherwise, if there is another element, say $(i_2,n)$, in $I_2\times\{n\}$, then $i_1=1$, $j_1$, $i_2$ and $n$ are four different elements in $\{1,\cdots,n\}$, and we come to the same contradiction as above.
\end{proof}

\begin{lem}\label{lem2}
We have $\sum_{(i,j)\in I}[(\lambda_i-\lambda_j)^2-\frac4{3}]\leq\frac{2}3,$ where the equality holds 
if and only if $n_0=1$ and $\lambda_1=-\lambda_n=\frac{\sqrt{2}}2,\lambda_2=\cdots\lambda_{n-1}=0$.
\end{lem}
\begin{proof}
Without loss of generality,we can assume that $I=\{1\}\times I_1$ by Lemma \ref{lem1}. Then
$$
\begin{aligned}
\sum_{(i,j)\in I}[(\lambda_i-\lambda_j)^2-\frac43]&=\sum_{j\in I_1}(\lambda^2_1+\lambda^2_j-2\lambda_1\lambda_j)-\frac43n_0\\
                                                  &=n_0\lambda^2_1+\sum_{j\in I_1}\lambda^2_j-2\lambda_1\sum_{j\in I_1}\lambda_j-\frac43n_0\\
                                                  &\leq(n_0+1)\lambda^2_1+\sum_{j\in I_1}\lambda^2_j+\Big(\sum_{j\in I_1}\lambda_j\Big)^2-\frac43n_0\\
                                                  &\leq(n_0+1)\Big(\lambda^2_1+\sum_{j\in I_1}\lambda^2_j\Big)-\frac43n_0\\
                                                  &\leq(n_0+1)\sum^n_{i=1}\lambda^2_i-\frac43n_0\\
                                                  &=1-\frac13n_0\leq \frac23,
\end{aligned}
$$
where the equality condition is easily seen from the proof.
\end{proof}

\begin{lem}\label{lem3}
For any $Q\in O(N)$, given any $\alpha\in S$ and any subset $J_\alpha\subset S$, we have $$\sum_{\beta\in{J}_\alpha}{\left(}\left\|{\left[}\check{Q}_\alpha,\check{Q}_\beta{\right]}\right\|^2-\frac{4}3{\right)}\leq\frac{4}3.$$
\end{lem}
\begin{proof}
For $\alpha\in S$, we can assume without loss of generality $\check{Q}_\alpha=diag{\left(}\lambda_1,\cdots,\lambda_n{\right)}$, with $\sum_i\lambda^2_i=1$, $\lambda_1\geq\cdots\geq\lambda_n$ and $I=\{1\}\times I_1$ by Lemma \ref{lem1}. Then by Lemma \ref{lem2} and Equation (\ref{qalpha}),
$$
\begin{aligned}
\sum_{\beta\in J_\alpha}\left(\left\|{\left[}\check{Q}_\alpha,\check{Q}_\beta{\right]}\right\|^2-\frac43\right)
&=\sum_{\beta\in J_\alpha}{\left(}\sum_{i,j=1}^n{\left(}\lambda_i-\lambda_j{\right)}^2|\check{q}^\beta_{ij}|^2-\frac43\cdot1{\right)}\\
&=\sum_{\beta\in J_\alpha}\sum_{(i,j)=\gamma\in S}{\left(}{\left(}\lambda_i-\lambda_j{\right)}^2-\frac43{\right)}\cdot|\check{q}^\beta_{ij}|^2\\
&=\sum_{\beta\in J_\alpha}\sum_{(i,j)=\gamma\in S}{\left(}{\left(}\lambda_i-\lambda_j{\right)}^2-\frac43{\right)}\cdot\frac12\(q^2_{\gamma\beta}+q^2_{\tau\beta}{\right)}\\
&\leq2\sum_{(i,j)=\gamma, i<j}{\left(}{\left(}\lambda_i-\lambda_j{\right)}^2-\frac43{\right)}\cdot\sum_{\beta\in J_\alpha}\frac12\(q^2_{\gamma\beta}+q^2_{\tau\beta}{\right)}\\
&\leq2\sum_{(i,j)=\gamma\in I}{\left(}{\left(}\lambda_i-\lambda_j{\right)}^2-\frac43{\right)}\cdot\sum_{\beta\in S}\frac12\(q^2_{\gamma\beta}+q^2_{\tau\beta}{\right)}\\
&=2\sum_{(i,j)=\gamma\in S}{\left(}{\left(}\lambda_i-\lambda_j{\right)}^2-\frac43{\right)}\cdot 1\leq\frac{4}3,
\end{aligned}
$$
where $\tau=(j,i)$, the equality in the last line is because of $Q\in O(N)$.
\end{proof}

\begin{lem}\label{lem4}
We have $\sum_{\beta\in S}\left\|{\left[}\check{Q}_\alpha,\check{Q}_\beta{\right]}\right\|^2\leq2n$ for any $ Q\in O(N)$ and any $\alpha\in S$.
\end{lem}
\begin{proof}
It follows from Equations (\ref{eqno3}, \ref{Qeq}, \ref{qalpha}) that
$$
\begin{aligned}
\sum_{\beta\in S}\left\|{\left[}\check{Q}_\alpha,\check{Q}_\beta{\right]}\right\|^2
&=\sum_{\beta\gamma\tau\xi\eta}q_{\gamma\alpha}q_{\xi\alpha}q_{\tau\beta}q_{\eta\beta}{\langle}[\check{E}_\gamma,\check{E}_\tau],[\check{E}_\xi,\check{E}_\eta]{\rangle}\\
&=\sum_{\gamma\xi}q_{\gamma\alpha}q_{\xi\alpha}\sum_\tau{\langle}[\check{E}_\gamma,\check{E}_\tau],[\check{E}_\xi,\check{E}_\tau]{\rangle}\\
&=\sum_{\gamma\xi}q_{\gamma\alpha}q_{\xi\alpha}\cdot(2n\delta_{\gamma\xi}-2\delta_\gamma\delta_\xi)\\
&=2n\sum_\gamma q^2_{\gamma\alpha}-2\Big(\sum_i\check{q}^\alpha_{ii}\Big)^2\\
&\leq 2n.
\end{aligned}
$$
\end{proof}

Now let $\varphi:M(m,n)\longrightarrow M(\binom{m}{2},\binom{n}{2})$ be the map defined by $\varphi\(A{\right)}_{\(i,j{\right)}\(k,l{\right)}}:=A\binom{kl}{ij}$, where $1\leq i<j\leq m$, $1\leq k<l\leq n$, and $A\binom{kl}{ij}=a_{ik}a_{jl}-a_{il}a_{jk}$ is the discriminant of the $2\times2$ submatrix of $A$ that is the intersection of rows $i$ and $j$ with columns $k$ and $l$, arranged with the same order as in (\ref{eqno1}). We have easily $\varphi\(I_n{\right)}=I_{\binom{n}{2}}$, $\varphi\(A{\right)}^t=\varphi\(A^t{\right)}$, and the following lemma.

\begin{lem}\label{lem5}
The map $\varphi$ preserves the matrix product, that is, $\varphi(AB)=\varphi(A)\varphi(B)$ holds for $A\in M(m,k)$ and $B\in M(k,n)$.
\end{lem}

We will also need a result of linear algebra for proving the equality case.
\begin{lem}\label{lem6}
Let $A$, $B$ be two matrices in $M(N,m)$. Then $AA^t=BB^t$ if and only if $A=BR$ for some $R\in O(m)$.
\end{lem}

 \section{Proof of the main results}
 Let $B_1,\cdots,B_m$ be any $n\times n$ Hermitian matrices ($n\geq2$). Their coefficients in the standard basis $\{\check{E}_\alpha|\alpha\in S\}$ are determined by a matrix $B\in M\(N,m{\right)}$ as $$\(B_1,\cdots,B_m{\right)}={\left(}\check{E}_1,\cdots,\check{E}_N\)B.$$ Range $\{\(r,s{\right)}|1\leq r<s\leq m\}$ and $\{{\left(}\alpha,\beta{\right)}|1\leq\alpha<\beta\leq N\}$ by lexicographic order as in (\ref{eqno1}). We arrange $\{[B_r,B_s]\}_{r<s}$ and $\{[\check{E}_\alpha,\check{E}_\beta]\}_{\alpha<\beta}$ into $\binom{m}{2}$- and $\binom{N}{2}$-vectors, respectively.

We first observe that
\begin{equation}\label{comm-basis}
\Big([B_1, B_2],\cdots,[B_{m-1}, B_m]\Big)=\Big([\check{E}_1, \check{E}_2],\cdots,[\check{E}_{N-1}, \check{E}_N]\Big)\cdot \varphi(B).
\end{equation}

 Let $C(\check{E})$ denote the matrix in $M(\binom{N}{2})$
defined by
\begin{equation}\label{comm-basis1}
C(\check{E})_{(\alpha,\beta)(\gamma,\tau)}:=\langle~
[\check{E}_{\alpha}, \check{E}_{\beta}],~ [\check{E}_{\gamma},
\check{E}_{\tau}]~ \rangle,
\end{equation}
for $1\leq\alpha<\beta\leq N$,
$1\leq\gamma<\tau\leq N$. Moreover we will use the same notation for
$\{B_r\}$ and $\{\check{Q}_{\alpha}\}$, \emph{i.e.}, the $\binom{m}{2}\times\binom{m}{2}$ matrix $C(B):=\Big(\langle~
[B_{r_1}, B_{s_1}],~ [B_{r_2}, B_{s_2}]~ \rangle\Big)$ and the $\binom{N}{2}\times\binom{N}{2}$ matrix $C(Q):=\Big(\langle~
[\check{Q}_{\alpha}, \check{Q}_{\beta}],~ [\check{Q}_{\gamma},
\check{Q}_{\tau}]~ \rangle\Big)$
respectively. Then it is obvious from Equation (\ref{comm-basis}) that
\begin{equation}\label{comm-basis2}
C(B)=\varphi(B^t)C(\check{E})\varphi(B), \hskip 0.3cm C(Q)=\varphi(Q^t)C(\check{E})\varphi(Q).
\end{equation}

Since $BB^t$ is a $N\times N$ semi-positive definite matrix, there
exists an orthogonal matrix $Q\in SO(N)$ such that
\begin{equation}\label{BBdiag}
BB^t=Q~diag(x_1,\cdots,x_N)~ Q^t, \quad \textit{where}~~x_{\alpha}\geq 0,~ 1\leq\alpha\leq N.
\end{equation}
Thus
\begin{equation}\label{BBnorm}
\sum_{r=1}^m\|B_r\|^2=\|B\|^2 =\sum_{\alpha=1}^Nx_{\alpha}
\end{equation}
 and
hence by Lemma \ref{lem5} and Equation (\ref{comm-basis2}),
\begin{eqnarray}\label{commtransf}
\sum_{r,s=1}^m\|[B_r, B_s]\|^2&=&2\operatorname{tr}~C(B)=2\operatorname{tr}\hskip 0.1cm
\varphi(B^t)C(\check{E})\varphi(B)\\
&=&2\operatorname{tr}~\varphi(BB^t)C(\check{E})=2\operatorname{tr}~
\varphi(diag(x_1,\cdots,x_N))C(Q)\nonumber\\
&=&\sum_{\alpha,\beta=1}^Nx_{\alpha}x_{\beta}\|[\check{Q}_{\alpha},
\check{Q}_{\beta}]\|^2.\nonumber
\end{eqnarray}

 

Combining the Equations (\ref{BBdiag}, \ref{BBnorm}, \ref{commtransf}), the inequality (1) of Theorem \ref{DDVVHermthm} is now transformed into the following:
\begin{equation}\label{DDVVtransf}
\sum_{\alpha,\beta=1}^Nx_\alpha x_\beta\left\|{\left[}\check{Q}_\alpha,\check{Q}_\beta{\right]}\right\|^2\leq\frac43{\left(}\sum_{\alpha=1}^Nx_\alpha{\right)}^2,\quad \forall x\in R^N_+,~ \forall Q\in SO\(N{\right)},
\end{equation}
where $R^N_+:=\{x=\(x_1,\cdots,x_N{\right)}\in R^N \mid x_\alpha\geq0,1\leq\alpha\leq N\}$.

\textbf{Proof of (\ref{DDVVtransf}), i.e., Theorem \ref{DDVVHermthm}}: Firstly, we define the function:
$$f_Q(x)=F(x,Q):=\sum_{\alpha,\beta=1}^Nx_{\alpha}x_{\beta}\|[\check{Q}_{\alpha},
\check{Q}_{\beta}]\|^2-\frac{4}{3}\Big(\sum_{\alpha=1}^Nx_{\alpha}\Big)^2.$$
Then $F$ is a continuous function defined on $\mathbb{R}^N\times
SO(N)$ and thus uniformly continuous on any compact subset of
$\mathbb{R}^N\times SO(N)$. Let
$\Delta:=\{x\in\mathbb{R}^N_{+}~|~\sum_{\alpha}x_{\alpha}=1\}$
and for any sufficiently small $\varepsilon>0$,
$\Delta_{\varepsilon}:=\{x\in
\Delta~|~x_{\alpha}\geq \varepsilon, 1\leq\alpha\leq N\}$.
Also let
$$G:=\{Q\in SO(N)~|~f_Q(x)\leq 0, ~for~all~ x\in
\Delta\},$$
$$G_{\varepsilon}:=\{Q\in SO(N)~|~f_Q(x)< 0,
~for~all~ x\in \Delta_{\varepsilon}\}.$$ We claim that
$G=\lim_{\varepsilon\rightarrow 0}G_{\varepsilon}=SO(N).$ Note that
this implies (\ref{DDVVtransf}) by the homogeneity of $f_Q$ and thus proves Theorem \ref{DDVVHermthm}. In
fact we can show
\begin{equation}\label{G epsilon}
G_{\varepsilon}=SO(N) ~\quad \textit{for  any sufficiently small } \varepsilon>0.
\end{equation}
To prove (\ref{G epsilon}), we use the continuity method, in which
we must prove the following three properties:
\begin{itemize}
\item[\textbf{(a)}]\label{step1} $I_N\in G_{\varepsilon}$ (and thus
$G_{\varepsilon}\neq\emptyset$);
\item[\textbf{(b)}]\label{step2} $G_{\varepsilon}$ is open in $SO(N)$;
\item[\textbf{(c)}]\label{step3} $G_{\varepsilon}$ is closed in $SO(N)$.
\end{itemize}

 Since $F$ is uniformly continuous on
$\triangle_{\varepsilon}\times SO(N)$, \textbf{(b)} is obvious.\\
\textbf{Proof of (a)}.  For any $ x\in\Delta_\epsilon$, applying for Equation (\ref{eqno2}) we get
$$\begin{aligned}
  f_{I_N}\(x{\right)} = & \sum_{\alpha,\beta=1}^Nx_\alpha{x}_\beta\|{\left[}\check{E}_\alpha,\check{E}_\beta{\right]}\|^2-\frac43\\
   =& ~4\sum_{i<j}x_{ij}x_{ji}+2\sum_{i<j}(x_{ii}x_{ij}+x_{ij}x_{jj}+x_{ii}x_{ji}+x_{ji}x_{jj})+\\
    & ~\sum_{i<j<k}(x_{ij}x_{jk}+x_{ij}x_{ik}+x_{ik}x_{jk}+x_{ji}x_{ki}
       +x_{ki}x_{kj}+x_{ji}x_{kj}+x_{ij}x_{kj}+\\
    & ~x_{jk}x_{ki}+x_{ik}x_{kj}+x_{ij}x_{ki}+x_{ik}x_{ji}+x_{ji}x_{jk})-\frac43\Big(\sum^n_{i,j=1}x_{ij}\Big)^2\\
\leq&  \sum_{i<j}\Big((x_{ij}+x_{ji})^2+2(x_{ii}+x_{jj})(x_{ij}+x_{ji})\Big)+\\
    &  \sum_{i<j<k}{\left(} (x_{ik}+x_{jk}+x_{ki}+x_{kj})(x_{ij}+x_{ji})+(x_{ik}+x_{ki})(x_{jk}+x_{kj}) {\right)}-\frac43\Big(\sum^n_{i,j=1}x_{ij}\Big)^2\\
   <& ~0,
\end{aligned}$$
  which means that $ I_N\in G_\varepsilon$. \hfill  $\qed$

\textbf{Proof of (c)}.
  We only need to prove the following \textbf{a priori estimate}: \textit{Suppose $f_Q\(x{\right)}\leq0$ for every $x\in\Delta_\varepsilon$. Then $f_Q\(x{\right)}<0$ for every $x\in\Delta_\varepsilon$.}

  The proof of this estimate is as follows: If there is a point $y\in\Delta_\varepsilon$ such that $f_Q\(y{\right)}=0$, we can assume without loss of generality that for some $1\leq\gamma\leq N$,
  $$y\in\Delta_\varepsilon^\gamma:=\{x\in\Delta_\varepsilon \mid x_\alpha>\varepsilon \textit{ for } \alpha\leq\gamma,\textit{ and } x_\beta=
  \varepsilon \textit{ for } \beta>\gamma\}.$$
   Then $y$ is a maximum point of $f_Q\(x{\right)}$ in the cone spanned by $\Delta_\varepsilon$ and an interior maximum point of $f_Q\(x{\right)}$ in $\Delta_\varepsilon^\gamma$.
      Hence, applying the Lagrange Multiplier Method, there exist numbers
$b_{\gamma+1},\cdots,b_N$ and a number $a$ such that
\begin{equation}\label{partial f}
\begin{array}{ll}
\Big(\frac{\partial f_Q}{\partial x_1}(y),\cdots,\frac{\partial
f_Q}{\partial x_{\gamma}}(y)\Big)=2a(1,\cdots,1),&\\
\Big(\frac{\partial f_Q}{\partial
x_{\gamma+1}}(y),\cdots,\frac{\partial f_Q}{\partial
x_{N}}(y)\Big)=2(b_{\gamma+1},\cdots,b_N)&
\end{array}
\end{equation}
or equivalently
\begin{equation}\label{partial f 2}
\sum_{\beta=1}^Ny_{\beta}(\|[\check{Q}_{\alpha},
\check{Q}_{\beta}]\|^2)-\frac{4}{3}=\Big\{
\begin{array}{ll}
 a & \alpha\leq\gamma,\\
b_{\alpha}& \alpha>\gamma.
\end{array}
\end{equation}
Hence
$$
f_Q(y)=\Big(\sum_{\alpha=1}^{\gamma}y_{\alpha}\Big)a+\Big(\sum_{\alpha=\gamma+1}^Nb_{\alpha}\Big)\varepsilon
=0\quad and \quad
\sum_{\alpha=1}^{\gamma}y_{\alpha}+(N-\gamma)\varepsilon=1.
$$
Meanwhile, by the homogeneity of $f_Q$ we see $\frac{\partial f_Q}{\partial \nu}(y)=2(a\gamma
+\sum_{\alpha=\gamma+1}^Nb_{\alpha})\leq 0$, where
$\nu=(1,\cdots,1)$ is the vector normal to $\Delta$ in
$\mathbb{R}^N$. For any sufficiently small $\varepsilon$ (such as
$\varepsilon<1/N$), it follows from the above three formulas that
$a\geq 0$. Without loss of generality, we assume
$y_1=max\{y_1,\cdots,y_{\gamma}\}>\varepsilon$. Let $$J:=\{\beta\in
S~|~ \|[\check{Q}_{1}, \check{Q}_{\beta}]\|^2\geq \frac{4}{3}\},$$
and let $n_1$ be the number of elements of $J$.
 Now combining Lemma \ref{lem3}, Lemma \ref{lem4} and Equation (\ref{partial f 2}) will give a
contradiction as follows:

  

\begin{equation}\label{contrad}
\begin{aligned}
\frac43\leq\frac43+a&=\sum^N_{\beta=2}y_\beta\left\|{\left[}\check{Q}_1,\check{Q}_\beta{\right]}\right\|^2\\
                    &=\sum_{\beta\in J}y_\beta{\left(}\left\|{\left[}\check{Q}_1,\check{Q}_\beta{\right]}\right\|^2-\frac43{\right)}+\frac43\sum_{\beta\in J}y_\beta+\sum_{\beta\in S\backslash J}y_\beta\left\|{\left[}\check{Q}_1,\check{Q}_\beta{\right]}\right\|^2\\
                    &\leq y_1\sum_{\beta\in J}{\left(}\left\|{\left[}\check{Q}_1,\check{Q}_\beta{\right]}\right\|^2-\frac43{\right)}+\frac43\sum_{\beta\in J}y_\beta+\sum_{\beta\in S\backslash J}y_\beta\left\|{\left[}\check{Q}_1,\check{Q}_\beta{\right]}\right\|^2\\
                    &\leq \frac43y_1+\frac43\sum_{\beta\in J}y_\beta+\sum_{\beta\in S\backslash J}y_\beta\left\|{\left[}\check{Q}_1,\check{Q}_\beta{\right]}\right\|^2\\
                    &\leq\frac43\sum^N_{\beta=1}y_\beta=\frac43.
\end{aligned}
\end{equation}
Thus
\begin{equation}\label{n1N}
a=0\quad and \quad \sum_{\beta\in J}\|[\check{Q}_1,
\check{Q}_{\beta}]\|^2=\frac{4}{3}(n_1+1)\leq 2n<\frac{4}{3}N \quad (n\geq2).
\end{equation}
Hence $S\backslash(J\cup\{1\})\neq\emptyset$, and the last ``$\leq$" in
(\ref{contrad}) should be ``$<$" by the definition of $J$ and the
positivity of $y_{\beta}$ for $\beta\in S\backslash(J\cup\{1\})$.\hfill
$\Box$

Now we consider the equality condition of (1) of Theorem \ref{DDVVHermthm} in view of the proof of the a priori estimate. When the equality holds, all inequalities in (\ref{contrad}) and thus in Lemmas \ref{lem2} and \ref{lem3} achieve the equality. For the eigenvalues $\{\lambda_1,\ldots, \lambda_n\}$ of $\check{Q}_1$ in (\ref{contrad}), because of the equality condition of Lemma \ref{lem2}, $n_0=1$ and $I=\{(1,n)\}$. Without loss of generality, let $\check{Q}_1=\frac1{\sqrt{2}}(E_{11}-E_{22})$ (replace the index $n$ with $2$ for simplicity). Because of the equality condition of Lemma \ref{lem3}, $q_{\gamma\beta}=q_{\tau\beta}=0$ for $\gamma=(1,2)\in I$, $\tau=(2,1)$ and $\beta\in S\backslash J$; $q_{\gamma\beta}=q_{\tau\beta}=0$ for any $\gamma\in S\backslash \{(1,2),(2,1)\}$ and $\beta\in J$. Then we have $n_1=2$ since $Q$ is orthogonal.
Moreover, for any $\beta\in J$, $\operatorname{rank}(\check{Q}_\beta)=2$, $\check{q}_{ij}^\beta=0$ for $(i,j)\in S\backslash \{(1,2),(2,1)\}$ and  $\check{q}_{12}^\beta=\overline{\check{q}_{21}^\beta}\neq0$. These yield that the two matrices $\check{Q}_\beta$ ($\beta\in J$) should be in the form of $diag(H_2,0)$ and $diag(H_3,0)$ up to a rotation.
Notice also that $y_\beta=y_1$ by (\ref{contrad}) for $\beta\in J$. Finally by (\ref{BBdiag}) and Lemma \ref{lem6} we obtain the equality condition of Theorem \ref{DDVVHermthm}.

\begin{ack}
The first author is very grateful to Professor Zhiqin Lu for sending the preprint paper \cite{LW11} and valuable discussions. 
\end{ack}

\begin{thebibliography}{123}
\bibitem{BW05}
A. B\"{o}ttcher and D. Wenzel, \emph{How big can the commutator of two matrices be and how big is it typically?} Linear Algebra Appl. \textbf{403} (2005), 216--228.
\bibitem{BW08}
A. B\"{o}ttcher and D. Wenzel, \emph{The Frobenius norm and the commutator}, Linear Algebra Appl. \textbf{429} (8-9) (2008), 1864--1885.
\bibitem{CDK}
S. S. Chern, M. do Carmo and S. Kobayashi, \emph{Minimal
submanifolds of a sphere with second fundamental form of constant
length}, in: Functional Analysis and Related Fields, Proc. Conf. for
M. Stone, Univ. Chicago, Chicago, Ill., 1968, Springer, New York,
1970, pp. 59-75.
\bibitem{CL08}
T, Choi and Z.Lu, \emph{On the DDVV Conjecture and the Comass in Calibrated Geometry (I)}, Math. Z. \textbf{260} (2008), 409--429.
\bibitem{DT09}
M. Dajczer and R. Tojeiro, \emph{Submanifolds of codimension two attaining equality in an extrinsic inequality}, Math. Proc. Cambridge Philos. Soc. \textbf{146} (2009), no. 2, 461--474.
\bibitem{DDVV99}
P. J. De Smet, F. Dillen, L. Verstraelen and L. Vrancken, \emph{A pointwise inequality in submanifold
theory}, Arch. Math. (Brno), \textbf{35} (1999), 115--128.
\bibitem{Ge14}
J. Q. Ge, \emph{DDVV-type inequality for skew-symmetric matrices and Simons-type inequality for Riemannian submersions}, Advances in Math. \textbf{251} (2014), 62--86.
\bibitem{GT08}
J. Q. Ge and Z. Z. Tang, \emph{A proof of the DDVV conjecture and its equality case}, Pacific J. Math., \textbf{237}(1)
2008, 87-95.
\bibitem{GT11}
J. Q. Ge and Z. Z. Tang, \emph{A survey on the DDVV conjecture}, \textit{Harmonic maps and differential geometry}, 247--254, Contemp. Math., \textbf{542}, Amer. Math. Soc., Providence, RI, 2011.
\bibitem{GTY16}
J. Q. Ge, Z. Z. Tang and W. J. Yan, \emph{Normal scalar curvature inequality on the focal submanifolds of isoparametric hypersurfaces},  2016,  arXiv:1610.03912.
\bibitem{Lu07}
Z. Lu, \emph{Recent developments of the DDVV conjecture}, Bull. Transilv. Univ. Bra\c{s}ov Ser. B (N.S.) \textbf{14}(49) (2007), 133-143.
\bibitem{Lu11}
Z. Lu, \emph{Normal scalar curvature conjecture and its applications}, J. Funct. Anal. \textbf{261} (2011), 1284--1308.
\bibitem{Lu12}
Z. Lu, \emph{Remarks on the B\"{o}ttcher-Wenzel inequality}, Linear Algebra Appl. \textbf{436} (2012), no. 7, 2531--2535.
\bibitem{LW11}
Z. Lu and D. Wenzel, \emph{Commutator estimates comparising the Frobenius norm - Looking back and forth}, preprint, 2011.
\bibitem{Ti00}
G. Tian, \emph{Gauge theory and calibrated geometry,I}, Ann. Math. \textbf{151} (2000) 193--268.
\bibitem{VJ08}
S. W. Vong and X. Q. Jin, \emph{Proof of B\"{o}ttcher and Wenzel's
conjecture}, Oper. Matrices \textbf{2} (2008), no. 3, 435--442.
\bibitem{XLMW14}
Z.X. Xie, T.Z. Li, X. Ma and C.P. Wang, \emph{M\"{o}bius geometry of three-dimensional Wintgen ideal submanifolds in $\mathbb{S}^5$}, Sci. China Math. \textbf{57} (2014), no. 6, 1203--1220.
\end{thebibliography}
\end{document}


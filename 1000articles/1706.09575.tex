
\documentclass[a4paper,oneside,english]{amsart}
\usepackage[T1]{fontenc}
\usepackage[latin9]{inputenc}
\usepackage{mathrsfs}
\usepackage{amsbsy}
\usepackage{amstext}
\usepackage{amsthm}
\usepackage{amssymb}
\usepackage{stmaryrd}
\usepackage[all]{xy}
\makeatletter
\pdfpageheight\paperheight
\pdfpagewidth\paperwidth
\numberwithin{equation}{section}
\numberwithin{figure}{section}
\theoremstyle{plain}
\newtheorem{thm}{\protect\theoremname}
\theoremstyle{definition}
\newtheorem{defn}[thm]{\protect\definitionname}
\theoremstyle{remark}
\newtheorem{rem}[thm]{\protect\remarkname}
\theoremstyle{definition}
\newtheorem{example}[thm]{\protect\examplename}
\theoremstyle{plain}
\newtheorem{prop}[thm]{\protect\propositionname}
\theoremstyle{plain}
\newtheorem{cor}[thm]{\protect\corollaryname}
\theoremstyle{plain}
\newtheorem{lem}[thm]{\protect\lemmaname}
\usepackage[dvipsnames,svgnames,x11names,hyperref]{xcolor}
\usepackage[colorlinks=true,linkcolor=Dark Red, citecolor=Dark Green,linktoc=all]{hyperref}
\usepackage{lmodern}

\usepackage{amssymb}
\usepackage{amsfonts}
\usepackage{egothic}
\usepackage[T1]{fontenc}
\usepackage{url}
\usepackage{amsthm}
\usepackage{aliascnt}
\usepackage{lipsum}
\usepackage{mathrsfs}
\usepackage{esint}
\usepackage{url}
\usepackage{amsmath}
\usepackage{dsfont}
\usepackage{bbm}
\usepackage{pdflscape}
\usepackage{bbm}
\usepackage{bussproofs}

\newlength{\lhs} 
\newlength{\rhs} 
\makeatother
\usepackage{babel}
\providecommand{\corollaryname}{Corollary}
\providecommand{\definitionname}{Definition}
\providecommand{\examplename}{Example}
\providecommand{\lemmaname}{Lemma}
\providecommand{\propositionname}{Proposition}
\providecommand{\remarkname}{Remark}
\providecommand{\theoremname}{Theorem}

\begin{document}

\title{Distributive Laws via Admissibility}

\author{Charles R. Walker}

\keywords{KZ-doctrines, lax-idempotent pseudomonads, pseudo-distributive laws}

\subjclass[2000]{18A35, 18C15, 18D05}

\address{Department of Mathematics, Macquarie University, NSW 2109, Australia}

\email{charles.walker1@mq.edu.au}

\thanks{The author acknowledges the support of an Australian Government Research
Training Program Scholarship.}

\date{\today}
\begin{abstract}
This paper concerns the problem of lifting a KZ doctrine $P$ to the
2-category of pseudo $T$-algebras for some pseudomonad $T$. Here
we show that this problem is equivalent to giving a pseudo-distributive
law (meaning that the lifted pseudomonad is automatically KZ), and
that such distributive laws may be simply described algebraically
and are essentially unique (as known to be the case in the (co)KZ
over KZ setting). 

Moreover, we give a simple description of these distributive laws
using Bunge and Funk's notion of admissible morphisms for a KZ doctrine
(the principal goal of this paper). We then go on to show that the
2-category of KZ doctrines on a 2-category is biequivalent to a poset.

We will also discuss here the case of lifting a locally fully faithful
KZ doctrine, which we noted earlier enjoys most of the axioms of a
Yoneda structure, and show that an oplax-lax bijection is exhibited
on the lifted ``Yoneda structure'' similar to Kelly's doctrinal
adjunction. We also briefly discuss how this bijection may be viewed
as a coherence result for oplax functors out of the bicategories of
spans and polynomials, but leave the details for a future paper.

\tableofcontents{}
\end{abstract}

\maketitle

\section{Introduction}

It is well known that to give a lifting of a monad to the algebras
of another monad is to give a distributive law \cite{beckdist}. More
generally, to give a lifting of a pseudomonad to the pseudoalgebras
of another pseudomonad is to give a pseudo-distributive law \cite{marm1999,cheng2003}.
However, in this paper we are interested in the problem of lifting
a Kock-Z\"{o}berlein pseudomonad $P$ (also known as a lax idempotent
pseudomonad), as introduced by Kock \cite{kock1972} and Z\"{o}berlein
\cite{zober1976}, to the pseudoalgebras of some pseudomonad $T$.
These KZ pseudomonads are a particular type of pseudomonad which captures
the idea of the free cocompletion under a class of colimits $\Phi$.

But what does it mean to give a lifting of a KZ doctrine to the setting
of pseudoalgebras such that the lifted pseudomonad is also KZ? One
objective of this paper is to show that this problem is equivalent
to giving a pseudo-distributive law (meaning a lifting of this pseudomonad
automatically inherits the KZ structure), and consequently that such
pseudo-distributive laws have a couple of simple descriptions. One
simple description being purely algebraic (a generalization and simplification
of a description given in \cite{marm1999}), and another description
(which is new) purely in terms of left Kan extensions and Bunge and
Funk's admissible maps of a KZ doctrine \cite{bungefunk}. In fact,
Bunge and Funk's admissible maps are a central tool in the proof of
these results. We also see that these distributive laws are essentially
unique, a generalization capturing \cite[Theorem 7.4]{marm2012} and
strengthening parts of \cite[Prop. 4.1]{marm2002}.

These two descriptions of a pseudo-distributive law correspond to
two different descriptions of a KZ pseudomonad. The first, which from
now on we call a KZ pseudomonad, is to be the usual algebraic definition
as given by Kock \cite{kock1972}; the second, which we call a KZ
doctrine, is to be the description in terms of left Kan extensions
due to Marmolejo and Wood \cite[Definition 3.1]{marm2012}.

Bunge and Funk showed that admissibility in the setting of a KZ pseudomonad
also has both an algebraic definition and a definition in terms of
left Kan extensions. Indeed, Bunge and Funk defined a morphism $f$
to be admissible in the context of a KZ doctrine $P$ when $Pf$ has
a right adjoint \cite[Definition 1.1]{bungefunk}, and showed that
this notion of admissibility also has a description in terms of left
Kan extensions \cite[Prop. 1.5]{bungefunk}. We refer to this as $P$-admissibility.

The central idea here is that instead of thinking about the problem
of lifting a KZ doctrine algebraically, we think about the problem
in terms of algebraic left Kan extensions. Moreover, this notion of
admissibility is crucial here as it allows us to show that certain
left extensions exist and are preserved.

A well known and motivating example the reader may keep in mind is
the KZ doctrine for the free small cocompletion on locally small categories,
with its lifting to the setting of monoidal categories described by
Im and Kelly \cite{uniconvolution} via the Day convolution \cite{dayconvolution}.

In Section \ref{background} we give the necessary background for
this paper, and recall the basic definitions of pseudomonads, pseudo
algebras and morphisms between pseudo algebras. In particular, we
recall the notion of a KZ pseudomonad and KZ doctrine and some results
concerning them. These notions will be used regularly throughout the
paper.

In Section \ref{doctrinalpartialadjunctions} we generalize one of
the results of Kelly's paper ``Doctrinal adjunction'' \cite{doctrinal}
to the setting of relative adjunctions (also known as partial adjunctions
or absolute left liftings), showing that given an oplax $T\text{-morphism}$
structure $\xi$ on $I$ and a lax $T\text{-morphism}$ structure
$\beta$ on $R$ as below, we get an oplax $T\text{-morphism}$ structure
$\alpha$ on $L$ (unique such that $\eta$ is a generalized $T$-transformation)
\[
\xymatrix@=1em{\left(\mathcal{B},T\mathcal{B}\overset{y}{\rightarrow}\mathcal{B}\right){\ar^-{{\left(R,\beta\right)}}[{rr}]} &  & \left(\mathcal{C},T\mathcal{C}\overset{z}{\rightarrow}\mathcal{C}\right)\ar@{}[ld]|-{\stackrel{\eta}{\Longleftarrow}}\\
 & \;\\
 &  & \left(\mathcal{A},T\mathcal{A}\overset{x}{\rightarrow}\mathcal{A}\right)\ar[uu]_{\left(I,\xi\right)}\ar[uull]^{\left(L,\alpha\right)}
}
\]
when we have an underlying 2-cell $\eta$ exhibiting $L$ as a partial
left adjoint.

In Section \ref{doctrinalleftextensions} we recall some results of
Koudenburg \cite{roald2015} concerning algebraic left extensions.
In particular, given a lax $T\text{-morphism}$ structure $\xi$ on
$I$ and an oplax $T\text{-morphism}$ structure $\alpha$ on $L$
as in the above diagram, we get a lax $T\text{-morphism}$ structure
$\beta$ on $R$ (again unique such that $\eta$ is a generalized
$T$-transformation) when we have an underlying 2-cell $\eta$ exhibiting
$R$ as a left extension and moreover $z$ is separately cocontinuous
with respect to $T$. 

In Section \ref{doctrinalyonedastructures} we show that Yoneda structures
(and in particular locally fully faithful KZ doctrines lifted to pseudo
algebras, which enjoy all the Yoneda structure axioms apart from the
right ideal property \cite{yonedakz}) enjoy an oplax-lax bijective
correspondence as in the case of Kelly's Doctrinal adjunction \cite{doctrinal}.
In particular, for a locally fully faithful KZ doctrine $P$ which
has been lifted to pseudo $T$-algebras where $L\colon\mathcal{A}\to\mathcal{B}$
is $P$-admissible and $\left(R_{L},\varphi_{L}\right)$ is the corresponding
left extension underlying the diagram
\[
\xymatrix@=1em{\left(\mathcal{B},T\mathcal{B}\overset{y}{\rightarrow}\mathcal{B}\right){\ar^-{{\left(R_{L},\beta\right)}}[{rr}]} &  & \left(P\mathcal{A},TP\mathcal{A}\overset{z_{x}}{\rightarrow}P\mathcal{A}\right)\ar@{}[ld]|-{\stackrel{\varphi_{L}}{\Longleftarrow}}\\
 & \;\\
 &  & \left(\mathcal{A},T\mathcal{A}\overset{x}{\rightarrow}\mathcal{A}\right)\ar[uu]_{\left(y_{\mathcal{A}},\xi_{x}\right)}\ar[uull]^{\left(L,\alpha\right)}
}
\]
we have a bijection between oplax structures $\alpha$ on $L$ and
lax structures $\beta$ on $R_{L}$. An interesting application of
this oplax-lax bijection is as a coherence result for the bicategories
of spans and polynomials (and in particular the oplax functors out
of these bicategories). We briefly discuss the applications here,
but leave this to be explored in more detail in a forthcoming paper.

In Section \ref{liftingkzdoctrines}, which is the bulk of this paper,
we use Bunge and Funk's notion of admissibility to generalize results
of Marmolejo and Wood concerning pseudo-distributive laws of (co)KZ
doctrines over KZ doctrines, such as the simple form of such distributive
laws or essential uniqueness of them. Our first improvement here is
to show that an axiom concerning the (co)KZ doctrine may be dropped,
allowing us to generalize these results to pseudo-distributive laws
of \emph{any} pseudomonad over a KZ doctrine. For example, this level
of generality allows us to capture the case studied by Im and Kelly
\cite{uniconvolution}; showing that the lifting of the small cocompletion
from categories to monoidal categories is essentially unique.

In addition, we use this simplification to give a simple algebraic
description of a pseudo-distributive law of a pseudomonad over a KZ
pseudomonad, consisting only of a pseudonatural transformation and
three invertible modifications subject to three coherence axioms (or
one invertible modification subject to one coherence axiom if we are
only lifting to lax algebras), and prove this definition is equivalent
to the usual notion of pseudo-distributive law. However, the main
new result of this section is a simple description of pseudo-distributive
laws over a KZ doctrine purely in terms of left Kan extensions and
admissibility.

Furthermore, through these calculations we find that in the presence
of a such a distributive law, the lifting of a KZ doctrine $P$ to
pseudo-$T$-algebras (for a pseudomonad $T$) is automatically a KZ
doctrine. The proof of these results is highly technical, relying
on $T$ preserving $P$-admissible maps; however, the results of this
section are summarized in Theorem \ref{liftkzequiv}. 

In Section \ref{consequencesandexamples} we study some properties
of the lifted KZ doctrine $\widetilde{P}$, such as classifying the
$\widetilde{P}$-cocomplete $T$-algebras as those for which the underlying
object is $P$-cocomplete and the algebra map separately cocontinuous,
thus justifying the usual definition of algebraic cocompleteness.
We also compare our results to that of Im-Kelly \cite{uniconvolution},
but seen from the KZ doctrine viewpoint.

After checking that the 2-category of KZ doctrines on a 2-category
is biequivalent to a poset, we go on to give some examples in which
we apply our results. Our first example concerns the case of the small
cocompletion and monoidal categories, and our second example concerns
multi-adjoints as studied by Diers \cite{diers}.

\section{Pseudomonads and KZ Pseudomonads\label{background}}

In this section we give the background knowledge necessary for this
paper, and recall the basic definitions of pseudomonads, pseudo algebras,
morphisms between pseudo algebras, as these notions will be used regularly
throughout the paper. We then go on the recall the notion of a KZ
pseudomonad, which is a type pseudomonad capturing the idea of the
cocompletion under some class of colimits $\Phi$, and give their
basic properties and some examples.

The notion of pseudonatural transformation is the (weak) 2-categorical
version of natural transformation. There are weaker notions also of
lax and oplax natural transformations however those will not be used
here. Modifications, defined below, take the place of morphisms between
pseudonatural transformations.
\begin{defn}
A \emph{pseudonatural transformation} between pseudofunctors $t\colon F\to G\colon\mathscr{A}\to\mathscr{B}$
where $\mathscr{A}$ and $\mathscr{B}$ are bicategories provides
for each 1-cell $f\colon\mathcal{A}\to\mathcal{B}$ in $\mathscr{A}$,
1-cells $t_{\mathcal{A}}$ and $t_{\mathcal{B}}$ and an invertible
2-cell in $\mathscr{B}$ 
\[
\xymatrix@=1em{F\mathcal{A}\ar[rr]^{Ff}\ar[dd]_{t_{\mathcal{A}}} &  & F\mathcal{B}\ar[dd]^{t_{\mathcal{B}}}\\
 & \ar@{}[]|-{\overset{t_{f}}{\implies}}\\
G\mathcal{A}\ar[rr]_{Gf} &  & G\mathcal{B}
}
\]
as above, satisfying coherence conditions outlined in \cite[Definition 2.2]{kelly1974}.
Given two pseudonatural transformations $t,s\colon F\to G\colon\mathscr{A}\to\mathscr{B}$
as above, a \emph{modification} $\alpha\colon s\to T$ consists of,
for every object $\mathcal{A}\in\mathscr{A}$, a 2-cell $\alpha_{\mathcal{A}}\colon t_{\mathcal{A}}\to s_{\mathcal{A}}$
such that for each 1-cell $f\colon\mathcal{A}\to\mathcal{B}$ in $\mathscr{A}$
we have the equality $\alpha_{\mathcal{B}}\cdot Ff\circ t_{f}=s_{f}\circ Gf\cdot\alpha_{\mathcal{A}}$.
\end{defn}

\subsection{Pseudomonads and their Algebras}

The following is the (weak) 2-categorical version of monad.
\begin{defn}
A \emph{pseudomonad} on a 2-category $\mathscr{C}$ consists of a
pseudofunctor and pseudonatural transformations
\[
T\colon\mathscr{C}\to\mathscr{C},\qquad u\colon1_{\mathscr{C}}\to T,\qquad m\colon T^{2}\to T
\]
as well as invertible modifications
\[
\xymatrix@=1em{T\ar[rr]^{uT}\ar[rdrd]_{\textnormal{id}} &  & T^{2}\ar[dd]_{m} &  & T\ar[ll]_{Tu}\ar[ldld]^{\textnormal{id}} &  & T^{3}\ar[rr]^{Tm}\ar[dd]_{mT} &  & T^{2}\ar[dd]^{m}\\
 & \;\ar@{}[ru]|-{\overset{\alpha}{\Longleftarrow}} &  & \ar@{}[lu]|-{\overset{\beta}{\Longleftarrow}}\\
 &  & T &  &  &  & T^{2}\ar[rr]_{m} &  & T\ar@{}[lulu]|-{\overset{\gamma}{\Longleftarrow}}
}
\]
which satisfy the two coherence axioms
\[
\xymatrix@=1em{T^{4}\ar[rr]^{T^{2}m}\ar[dd]_{mT^{2}}\ar[rdrd]^{TmT} &  & T^{3}\ar[rrdd]^{Tm} &  &  &  & T^{4}\ar[rr]^{T^{2}m}\ar[dd]_{mT^{2}} &  & T^{3}\ar[rrdd]^{Tm}\ar[dd]^{mT}\\
 &  & \ar@{}[]|-{\overset{T\gamma}{\Longleftarrow}} &  &  &  &  & \ar@{}[]|-{\overset{m_{m}^{-1}}{\Longleftarrow}}\\
T^{3}\ar[rrdd]_{mT} & \ar@{}[]|-{\overset{\gamma T}{\Longleftarrow}} & T^{3}\ar[rr]^{Tm}\ar[dd]_{mT} &  & T^{2}\ar[dd]^{m} & = & T^{3}\ar[rrdd]_{mT}\ar[rr]_{Tm} &  & T^{2}\ar[rrdd]^{m} & \ar@{}[]|-{\overset{\gamma}{\Longleftarrow}} & T^{2}\ar[dd]^{m}\\
 &  &  & \ar@{}[]|-{\overset{\gamma}{\Longleftarrow}} &  &  &  &  & \ar@{}[]|-{\overset{\gamma}{\Longleftarrow}}\\
 &  & T^{2}\ar[rr]_{m} &  & T &  &  &  & T^{2}\ar[rr]_{m} &  & T
}
\]
and
\[
\xymatrix@=1em{ &  &  &  & T^{2}\ar[rrd]^{m} &  &  &  &  &  & T^{3}\ar[rrd]^{Tm}\\
T^{2}\ar[rr]^{TuT} &  & T^{3}\ar[rru]^{Tm}\ar[rrd]_{mT} &  & \ar@{}[]|-{\Downarrow\gamma} &  & T & = & T^{2}\ar[rur]^{TuT}\ar[rrd]_{TuT}\ar[rrrr] &  & \ar@{}[u]|-{\Downarrow T\alpha}\ar@{}[d]|-{\Downarrow\beta T} &  & T^{2}\ar[rr]^{m} &  & T\\
 &  &  &  & T^{2}\ar[rru]_{m} &  &  &  &  &  & T^{3}\ar[urr]_{mT}
}
\]
\end{defn}
\begin{rem}
One should note here that there are three useful consequences of these
pseudomonad axioms \cite[Proposition 8.1]{marm1997} originally due
to Kelly \cite{kellymaclanecoherence}. Of these, we will only need
the consequence that 
\[
\xymatrix@=1em{ &  &  &  & T^{2}\ar[rrd]^{m} &  &  &  &  &  & T\ar[rrd]^{uT}\\
1_{\mathscr{C}}\ar[rr]^{u} &  & T\ar[rru]^{uT}\ar[rrd]_{Tu}\ar[rrrr] &  & \ar@{}[u]|-{\Downarrow\alpha}\ar@{}[d]|-{\Downarrow\beta} &  & T & = & 1_{\mathscr{C}}\ar[rru]^{u}\ar[rrd]_{u} &  & \ar@{}[]|-{\Downarrow u_{u}^{-1}} &  & T^{2}\ar[rr]^{m} &  & T\\
 &  &  &  & T^{2}\ar[rru]_{m} &  &  &  &  &  & T\ar[rru]_{Tu}
}
\]

\end{rem}
Given a pseudomonad $\left(T,u,m\right)$ on a 2-category $\mathscr{C}$
we may consider $T$-algebras, $T$-morphisms and $T$-transformations,
as well as their weaker notions where conditions only hold up coherent
isomorphism of 2-cells.
\begin{defn}
Given a pseudomonad $\left(T,u,m\right)$ on a 2-category $\mathscr{C}$,
a \emph{lax $T$-algebra} consists of an object $\mathcal{A}\in\mathscr{C}$,
a 1-cell $x:T\mathcal{A}\to\mathcal{A}$ and 2-cells
\[
\xymatrix{T^{2}\mathcal{A}\ar[d]_{m_{\mathcal{A}}}\ar[r]^{Tx}\ar@{}[rd]|-{\Downarrow\mu} & T\mathcal{A}\ar[d]^{x} & \mathcal{A}\ar[rr]^{\textnormal{id}}\ar[rd]_{u_{\mathcal{A}}} & \;\ar@{}[d]|-{\Downarrow\nu} & \mathcal{A}\\
T\mathcal{A}\ar[r]_{x} & \mathcal{A} &  & T\mathcal{A}\ar[ur]_{x}
}
\]
such that both 
\[
\xymatrix@=1em{ &  & \ar@{}[d]|-{\Downarrow\nu} &  &  &  &  &  & \ar@{}[dd]|-{\Downarrow T\nu} &  & T\mathcal{A}\ar[rrdd]^{x}\ar@{}[dddd]|-{\Downarrow\mu}\\
\mathcal{A}\ar[rr]^{u_{\mathcal{A}}}\ar@/^{2pc}/[rrrr]^{\textnormal{id}}\ar@{}[drdr]|-{\Downarrow u_{x}} &  & T\mathcal{A}\ar[rr]^{x}\ar@{}[ddrr]|-{\Downarrow\mu} &  & \mathcal{A}\\
 &  &  &  &  &  & T\mathcal{A}\ar@/^{1pc}/[rrrruu]^{\textnormal{id}}\ar[rr]_{Tu_{\mathcal{A}}}\ar@/_{1pc}/[rrrrdd]_{\textnormal{id}} &  & T^{2}\mathcal{A}\ar[uurr]_{Tx}\ar[ddrr]_{m_{\mathcal{A}}}\ar@{}[dd]|-{\cong} &  &  &  & \mathcal{A}\\
T\mathcal{A}\ar[rr]_{u_{T\mathcal{A}}}\ar@/_{2pc}/[rrrr]_{\textnormal{id}}\ar[uu]^{x} &  & T^{2}\mathcal{A}\ar[uu]_{Tx}\ar[rr]_{m_{\mathcal{A}}} &  & T\mathcal{A}\ar[uu]_{x}\\
 &  & \ar@{}[u]|-{\cong} &  &  &  &  &  & \; &  & T\mathcal{A}\ar[urur]_{x}
}
\]
paste to the identity 2-cell at $x$, known as the left and right
unit axioms respectively. Moreover, the associativity axiom asks that
we have the equality 
\[
\xymatrix@=1em{ &  & T^{2}\mathcal{A}\ar[rr]^{Tx}\ar[rdrd]^{m_{\mathcal{A}}}\ar@{}[dddd]|-{\Downarrow m_{x}^{-1}} &  & T\mathcal{A}\ar[rrdd]^{x}\ar@{}[dd]|-{\Downarrow\mu} &  &  &  &  &  & T^{2}\mathcal{A}\ar[rr]^{Tx}\ar@{}[dd]|-{\Downarrow T\mu} &  & T\mathcal{A}\ar[rrdd]^{x}\ar@{}[dddd]|-{\Downarrow\mu}\\
\\
T^{3}\mathcal{A}\ar[urur]^{T^{2}x}\ar[rrdd]_{m_{T\mathcal{A}}} &  &  &  & T\mathcal{A}\ar[rr]^{x}\ar@{}[dd]|-{\Downarrow\mu} &  & \mathcal{A} & = & T^{3}\mathcal{A}\ar[rr]^{Tm_{\mathcal{A}}}\ar[uurr]^{T^{2}x}\ar[rdrd]_{m_{T\mathcal{A}}} &  & T^{2}\mathcal{A}\ar[uurr]^{Tx}\ar[ddrr]_{m_{\mathcal{A}}}\ar@{}[dd]|-{\cong} &  &  &  & \mathcal{A}\\
\\
 &  & T^{2}\mathcal{A}\ar[rr]_{m_{\mathcal{A}}}\ar[uurr]_{Tx} &  & T\mathcal{A}\ar[rruu]_{x} &  &  &  &  &  & T^{2}\mathcal{A}\ar[rr]_{m_{\mathcal{A}}} &  & T\mathcal{A}\ar[rruu]_{x}
}
\]
If the above 2-cells $\nu$ and $\mu$ are isomorphisms, we call this
a \emph{pseudo $T$-algebra}. If $\nu$ and $\mu$ are identity 2-cells,
we call this a \emph{strict $T$-algebra}.
\end{defn}
These $T$-algebras may be regarded as the objects of a category,
with morphisms of (pseudo) $T$-algebras defined as follows.
\begin{defn}
An \emph{oplax $T$-morphism} of pseudo $T$-algebras
\[
\left(L,\alpha\right)\colon\left(\mathcal{A},T\mathcal{A}\stackrel{x}{\rightarrow}\mathcal{A}\right)\to\left(\mathcal{B},T\mathcal{B}\stackrel{y}{\rightarrow}\mathcal{B}\right)
\]
consists of a 1-cell $L\colon\mathcal{A}\to\mathcal{B}$ and a 2-cell
\[
\xymatrix{T\mathcal{B}\ar[r]^{y}\ar@{}[rd]|-{\Uparrow\alpha} & \mathcal{B}\\
T\mathcal{A}\ar[r]_{x}\ar[u]^{TL} & \mathcal{A}\ar[u]_{L}
}
\]
such that 
\[
\xymatrix{ & \;\\
\mathcal{B}\ar[r]^{u_{\mathcal{B}}}\ar@{}[rd]|-{\Uparrow u_{L}}\ar@/^{2.5pc}/[rr]^{\textnormal{id}_{\mathcal{B}}} & T\mathcal{B}\ar[r]^{y}\;\ar@{}[rd]|-{\Uparrow\alpha}\ar@{}[u]|-{\cong} & \mathcal{B}\\
\mathcal{A}\ar[u]^{L}\ar[r]_{u_{\mathcal{A}}}\ar@/_{2.5pc}/[rr]_{\textnormal{id}_{\mathcal{B}}} & T\mathcal{A}\ar[r]_{x}\ar[u]^{TL}\ar@{}[d]|-{\cong} & \mathcal{A}\ar[u]_{L}\\
 & \;
}
\]
is the identity 2-cell on $L$, and for which
\[
\xymatrix{ & T\mathcal{B}\ar[rd]^{y}\\
T^{2}\mathcal{B}\ar[r]^{Ty}\ar@{}[rd]|-{\Uparrow T\alpha}\ar[ru]^{m_{\mathcal{B}}} & T\mathcal{B}\ar[r]^{y}\ar@{}[rd]|-{\Uparrow\alpha}\ar@{}[u]|-{\cong} & \mathcal{B} & \ar@{}[d]|-{=} & T^{2}\mathcal{B}\ar[r]^{m_{\mathcal{B}}}\ar@{}[dr]|-{\Uparrow m_{L}} & T\mathcal{B}\ar[r]^{y}\;\ar@{}[rd]|-{\Uparrow\alpha} & \mathcal{B}\\
T^{2}\mathcal{A}\ar[u]^{T^{2}L}\ar[r]_{Tx}\ar[rd]_{m_{\mathcal{A}}} & T\mathcal{A}\ar[r]_{x}\ar[u]^{TL} & \mathcal{A}\ar[u]_{L} & \; & T^{2}\mathcal{A}\ar[u]^{T^{2}L}\ar[r]_{m_{\mathcal{A}}} & T\mathcal{A}\ar[r]_{x}\ar[u]^{TL} & \mathcal{A}\ar[u]_{L}\\
 & T\mathcal{A}\ar[ur]_{x}\ar@{}[u]|-{\cong}
}
\]

If the 2-cell $\alpha$ goes in the opposite direction, this is the
definition of a \emph{lax $T$-morphism, }and if $\alpha$ is invertible
this is then the definition of a \emph{pseudo $T$-morphism.}
\end{defn}
The usual definition of $T$-transformation between oplax or lax $T$-morphisms
is not general enough for our purposes as we will be considering situations
in which we have both oplax and lax $T$-morphisms, and so we define
$T$-transformations as based on the double category viewpoint \cite{adjointdouble}.
Such transformations are sometimes referred to as generalized $T$-transformations.
\begin{defn}
Suppose we are given a square of morphisms of pseudo $T$-algebras
\[
\xymatrix@=1em{\left(\mathcal{B},y\right)\ar[rr]^{\left(R,\beta\right)} &  & \left(\mathcal{C},z\right)\ar@{}[ldld]|-{\stackrel{\zeta}{\Longleftarrow}}\\
\\
\left(\mathcal{D},w\right)\ar[rr]_{\left(M,\varepsilon\right)}\ar[uu]^{\left(N,\varphi\right)} &  & \left(\mathcal{A},x\right)\ar[uu]_{\left(I,\xi\right)}
}
\]
where the vertical maps are oplax $T$-morphisms and the horizontal
maps are lax $T$-morphisms. A \emph{$T$-transformation }$\zeta$
as in the above square is a 2-cell $\zeta:IM\to RN$ for which we
have the equality of the two sides of the cube
\[
\xymatrix{ & T\mathcal{B}\ar[r]^{y}\ar@{}[d]|-{\Uparrow\varphi} & \mathcal{B}\ar[rd]^{R}\ar@{}[dd]|-{\Uparrow\zeta} &  &  &  & T\mathcal{B}\ar[r]^{y}\ar[dr]^{TR}\ar@{}[dd]|-{\Uparrow T\zeta} & \mathcal{B}\ar[rd]^{R}\ar@{}[d]|-{\Uparrow\beta}\\
T\mathcal{D}\ar[r]^{w}\ar[rd]_{TM}\ar[ru]^{TN} & \mathcal{D}\ar[rd]_{M}\ar[ru]^{N}\ar@{}[d]|-{\Uparrow\varepsilon} &  & \mathcal{C} & = & T\mathcal{D}\ar[rd]_{TM}\ar[ru]^{TN} &  & T\mathcal{C}\ar[r]^{z}\ar@{}[d]|-{\Uparrow\xi} & \mathcal{C}\\
 & T\mathcal{A}\ar[r]_{x} & \mathcal{A}\ar[ur]_{I} &  &  &  & T\mathcal{A}\ar[r]_{x}\ar[ur]_{TI} & \mathcal{A}\ar[ur]_{I}
}
\]

We will call the 2-category of pseudo $T$-algebras, pseudo $T$-morphisms,
and $T$-transformations $\text{ps}$-$T$-alg (we may consider squares
where both horizontal maps are identities or both vertical maps are
identities to recover the usual notions of transformation between
lax/oplax/pseudo $T$-morphisms). \end{defn}
\begin{rem}
\label{Ttransremark} Note that in this language it makes sense to
talk about the unit and counit of the adjunction where the left adjoint
is oplax and the right adjoint lax. Indeed the oplax-lax bijective
correspondence in Kelly's doctrinal adjunction \cite{doctrinal} is
unique such the counit ${\varepsilon}$ (and unit $\eta$) of the adjunction
extends to a $T$-transformation\footnote{This is shown in more generality in the next section.}.
Note that in this setting of a doctrinal adjunction $L\dashv R$ it
makes sense to view the unit and counit as $T$-transformations as
we have squares 
\[
\xymatrix@=1em{\left(\mathcal{B},y\right)\ar[rr]^{\left(\textnormal{id},\textnormal{id}\right)} &  & \left(\mathcal{B},y\right)\ar@{}[ldld]|-{\stackrel{\varepsilon}{\Longleftarrow}} &  & \left(\mathcal{B},y\right)\ar[rr]^{\left(R,\beta\right)} &  & \left(\mathcal{A},x\right)\ar@{}[ldld]|-{\stackrel{\eta}{\Longleftarrow}}\\
\\
\left(\mathcal{B},y\right)\ar[rr]_{\left(R,\beta\right)}\ar[uu]^{\left(\textnormal{id},\textnormal{id}\right)} &  & \left(\mathcal{A},x\right)\ar[uu]_{\left(L,\alpha\right)} &  & \left(\mathcal{A},x\right)\ar[rr]_{\left(\textnormal{id},\textnormal{id}\right)}\ar[uu]^{\left(L,\alpha\right)} &  & \left(\mathcal{A},x\right)\ar[uu]_{\left(\textnormal{id},\textnormal{id}\right)}
}
\]

As a convention, will will usually omit these identity $T$-morphisms.
The reader should just remember that it makes sense to consider $T$-transformations
from a lax followed by an oplax $T$-morphism, into an oplax followed
by a lax $T$-morphism, and that any such transformation may be uniquely
expressed as a square in the form of the above definition by inserting
the appropriate identity $T$-morphisms; which is what we have done
in the case of the unit and counit above.\end{rem}
\begin{example}
One may define the category of $\mathbf{Cat}$ (the category of locally
small categories) enriched graphs, denoted $\mathbf{CatGrph}$, with
objects given as families of hom-categories 
\[
\left(\mathscr{C}\left(X,Y\right)\colon X,Y\in\textnormal{ob}\mathscr{C}\right)
\]
 and morphisms consisting of locally defined functors 
\[
\left(F_{X,Y}:\mathscr{C}\left(X,Y\right)\to\mathscr{D}\left(FX,FY\right)\colon X,Y\in\mathscr{C}\right)
\]
which have not been endowed with the structure of a bicategory or
a lax/oplax functor respectively \cite{companion}. This gives rise
to, via a suitable 2-monad $T$ on $\mathbf{CatGrph}$, the 2-category
of bicategories, oplax functors and icons \cite{icons}. We may replace
oplax here with ``lax'' or ``pseudo'' and still have a 2-category.
Note that inside this 2-category lives the one object bicategories
(isomorphic to monoidal categories), giving the 2-category of monoidal
categories, lax/oplax/strong monoidal functors and monoidal transformations
(which may also be constructed directly via a suitable 2-monad \cite{icons}).
\end{example}

\subsection{KZ Pseudomonads}

A KZ pseudomonad is a type of pseudomonad which captures the idea
of the cocompletion under some class of colimits $\Phi$. We first
recall the well known algebraic definition of a KZ pseudomonad \cite{kock1972}. 
\begin{defn}
\label{defkzpseudomonad} A \emph{KZ pseudomonad $\left(P,y,\mu\right)$
}on a 2-category $\mathscr{C}$ consists of a pseudomonad $\left(P,y,\mu\right)$
on $\mathscr{C}$ along with a modification $\theta\colon Py\to yP$
for which
\[
\xymatrix@=1em{ &  &  &  &  &  &  &  & P\ar[rrd]^{yP}\\
1_{\mathscr{C}}\ar[rr]^{y} &  & P\ar@/^{1pc}/[rr]^{yP}\ar@/_{1pc}/[rr]_{Py}\ar@{}[rr]|-{\Uparrow\theta} &  & P^{2} & = & 1_{\mathscr{C}}\ar[rru]^{y}\ar[rrd]_{y} &  & \ar@{}[]|-{\Uparrow y_{y}} &  & P^{2}\\
 &  &  &  &  &  &  &  & P\ar[rru]_{Py}
}
\]
and
\[
\xymatrix@=1em{\\
 &  & \ar@{}[d]|-{\Uparrow\alpha}\\
P\ar@/^{0.7pc}/[rr]^{yP}\ar@/_{0.7pc}/[rr]_{Py}\ar@{}[rr]|-{\Uparrow\theta}\ar@/^{2.5pc}/[rrrr]^{\textnormal{id}_{P}}\ar@/_{2.5pc}/[rrrr]_{\textnormal{id}_{P}} &  & P^{2}\ar[rr]^{\mu} &  & P & = & \textnormal{id}_{\textnormal{id}_{P}}\\
 &  & \ar@{}[u]|-{\Uparrow\beta}\\
\\
}
\]

\end{defn}
The above is the algebraic description of a KZ pseudomonad; however
there is another description in terms of left Kan extensions given
by Marmolejo and Wood \cite{marm2012} which we refer to as a KZ doctrine.
\begin{defn}
\cite[Definition 3.1]{marm2012} A \emph{KZ doctrine $\left(P,y\right)$
}on a 2-category $\mathscr{C}$ consists of 

(i) An assignation on objects $P\colon\textnormal{ob}\mathscr{C}\to\textnormal{ob}\mathscr{C}$;

(ii) For every object $\mathcal{A}\in\mathscr{C}$, a 1-cell $y_{\mathcal{A}}\colon\mathcal{A}\to P\mathcal{A}$;

(iii) For every pair of objects $\mathcal{A}\text{ and }\mathcal{B}$
and 1-cell $F\colon\mathcal{A}\to P\mathcal{B}$, a left extension
\[
\xymatrix@=1em{P\mathcal{A}\ar@{->}[rr]^{\overline{F}}\ar@{}[rd]|-{\stackrel{c_{F}}{\Longleftarrow}} &  & P\mathcal{B}\\
 & \; &  &  & (2.1)\\
\mathcal{A}\ar[rruu]_{F}\ar[uu]^{y_{\mathcal{A}}}
}
\]
of $F$ along $y_{\mathcal{A}}$ exhibited by an isomorphism $c_{F}$
as above. 

Moreover, we require that:

(a) For every object $\mathcal{A}\in\mathscr{C}$, the left extension
of $y_{\mathcal{A}}$ as in (2.1) is given by
\[
\xymatrix@=1em{P\mathcal{A}\ar@{->}[rr]^{\textnormal{id}_{P\mathcal{A}}} &  & P\mathcal{A}\ar@{}[ld]|-{\stackrel{\textnormal{id}}{\Longleftarrow}}\\
 & \;\\
 &  & \mathcal{A}\ar[uu]_{y_{\mathcal{A}}}\ar[ulul]^{y_{\mathcal{A}}}
}
\]

Note that this means $c_{y_{\mathcal{A}}}$ is equal to the identity
2-cell on $y_{\mathcal{A}}$. 

(b) For any 1-cell $G\colon\mathcal{B}\to P\mathcal{C}$, the corresponding
left extension $\overline{G}\colon P\mathcal{B}\to P\mathcal{C}$
preserves the left extension $\overline{F}$ in (2.1) .\end{defn}
\begin{rem}
These two descriptions are equivalent in the sense that each gives
rise to the other \cite{marm2012,marm1997}. In Section \ref{consequencesandexamples}
we will express this relationship as a biequivalence between the 2-category
of KZ pseudomonads and the preorder of KZ doctrines.
\end{rem}
The following definitions in terms of left extensions are equivalent
to the preceding notions of pseudo $P$-algebra and $P$-homomorphism,
in the sense that we have an equivalence between the two resulting
2-categories of pseudo $P$-algebras arising from the two different
definitions \cite[Theorems 5.1,5.2]{marm2012}.
\begin{defn}
[\cite{marm2012}] Given a KZ doctrine $\left(P,y\right)$ on a 2-category
$\mathscr{C}$, we say an object $\mathcal{X}\in\mathscr{C}$ is \emph{$P$-cocomplete}
if for every $G\colon\mathcal{B}\to\mathcal{X}$ 
\[
\xymatrix@=1em{P\mathcal{B}\ar@{->}[rr]^{\overline{G}}\ar@{}[rd]|-{\stackrel{c_{G}}{\Longleftarrow}} &  & \mathcal{X} &  &  & P\mathcal{A}\ar@{->}[rr]^{\overline{F}}\ar@{}[rd]|-{\stackrel{c_{F}}{\Longleftarrow}} &  & P\mathcal{B}\ar[rr]^{\overline{G}} &  & \mathcal{X}\\
 & \; &  &  &  &  & \;\\
\mathcal{B}\ar[rruu]_{G}\ar[uu]^{y_{\mathcal{B}}} &  &  &  &  & \mathcal{A}\ar[rruu]_{F}\ar[uu]^{y_{\mathcal{A}}}
}
\]
there exists a left extension $\overline{G}$ as on the left exhibited
by an isomorphism $c_{G}$, and moreover this left extension respects
the left extensions $\overline{F}$ as in the diagram on the right.
We say a 1-cell $E\colon\mathcal{X}\to\mathcal{Y}$ between $P$-cocomplete
objects $\mathcal{X}$ and $\mathcal{Y}$ is a \emph{$P$-homomorphism
}when it respects all left extensions along $y_{\mathcal{B}}$ into
$\mathcal{X}$ for every object $\mathcal{B}$. \end{defn}
\begin{rem}
It is clear that $P\mathcal{A}$ is $P$-cocomplete for every $\mathcal{A}\in\mathscr{C}$. 
\end{rem}
We now recall the notion of $P$-admissibility in the setting of a
KZ doctrine $P$. This notion of admissibility is useful for showing
that certain left extensions exist, and moreover are preserved, and
will be used throughout this paper.
\begin{defn}
\label{admequiv} Given a KZ doctrine $\left(P,y\right)$ on a 2-category
$\mathscr{C}$, we say a 1-cell $L\colon\mathcal{A}\to\mathcal{B}$
is \emph{$P$-admissible} if any of the following equivalent conditions
are met:

(1) in the left diagram below: 
\[
\xymatrix@=1em{\mathcal{B}\ar@{->}[rr]^{R_{L}} &  & P\mathcal{A}\ar@{}[ld]|-{\stackrel{\varphi_{L}}{\Longleftarrow}} &  &  & \mathcal{B}\ar@{->}[rr]^{R_{L}} &  & P\mathcal{A}\ar@{}[ld]|-{\stackrel{\varphi_{L}}{\Longleftarrow}}\ar[rr]^{\overline{H}}\ar@{}[rd]|-{\stackrel{c_{H}}{\Longleftarrow}} &  & \mathcal{X}\\
 & \; &  &  &  &  & \; &  & \;\\
 &  & \mathcal{A}\ar[uu]_{y_{\mathcal{A}}}\ar[ulul]^{L} &  &  &  &  & \mathcal{A}\ar[uu]_{y_{\mathcal{A}}}\ar[ulul]^{L}\ar[rruu]_{H}
}
\]
 there exists a left extension $\left(R_{L},\varphi_{L}\right)$ of
$y_{\mathcal{A}}$ along $L$, and moreover the left extension is
preserved by any $\overline{H}$ as in the right diagram where $\mathcal{X}$
is $P$-cocomplete.

(2) every $P$-cocomplete object $\mathcal{X}\in\mathscr{C}$ admits,
and $P$-homomorphism preserves, left extensions along $L$. This
says that for any given 1-cell $K:\mathcal{A}\to\mathcal{X}$, where
$\mathcal{X}$ is $P$-cocomplete, there exists a 1-cell $J$ and
2-cell $\delta$ as in the left diagram below 
\[
\xymatrix@=1em{\mathcal{B}\ar@{->}[rr]^{J} &  & \mathcal{X}\ar@{}[ld]|-{\stackrel{\delta}{\Longleftarrow}} &  &  & \mathcal{B}\ar@{->}[rr]^{J} &  & \mathcal{X}\ar@{}[ld]|-{\stackrel{\delta}{\Longleftarrow}}\ar[rr]^{E} &  & \mathcal{Y}\\
 & \; &  &  &  &  & \; &  & \;\\
 &  & \mathcal{A}\ar[uu]_{K}\ar[ulul]^{L} &  &  &  &  & \mathcal{A}\ar[uu]_{K}\ar[ulul]^{L}
}
\]
exhibiting $J$ as a left extension, and moreover this left extension
is preserved by any $P$-homomorphism $E\colon\mathcal{X}\to\mathcal{Y}$
for $P$-cocomplete $\mathcal{Y}$ as in the right diagram.

(3) $PL:=\textnormal{lan}_{L}$ given as the left extension
\[
\xymatrix@=1em{P\mathcal{A}\ar@{->}[rr]^{PL}\ar@{}[rdrd]|-{\stackrel{c_{y_{\mathcal{B}\cdot L}}}{\Longleftarrow}} &  & P\mathcal{B}\\
\\
\mathcal{A}\ar[rr]_{L}\ar[uu]^{y_{\mathcal{A}}} &  & \mathcal{B}\ar[uu]_{y_{\mathcal{B}}}
}
\]
has a right adjoint. \end{defn}
\begin{rem}
For a proof that the descriptions (1), (2) and (3) above are equivalent,
we refer the reader to \cite{bungefunk} or \cite{yonedakz}.\end{rem}
\begin{defn}
Given a KZ doctrine $\left(P,y\right)$ on a 2-category $\mathscr{C}$,
we say a 1-cell $L\colon\mathcal{A}\to\mathcal{B}$ is \emph{$P$-fully
faithful} if $PL$ is fully faithful.\end{defn}
\begin{rem}
The importance of the \emph{$P$-fully faithful }maps stems from the
fact that for a $P$-admissible map $L\colon\mathcal{A}\to\mathcal{B},$
this $L$ is $P$-fully faithful if and only if every left extension
along $L$ into a $P$-cocomplete object is exhibited by an isomorphism
\cite[Remark 24]{yonedakz}. Clearly each $y_{\mathcal{A}}$ is both
$P$-admissible and $P$-fully faithful.
\end{rem}
For any given KZ doctrine $P$ on a 2-category $\mathscr{C}$ a natural
question to ask is: what are the $P$-cocomplete objects; $P$-homomorphisms;
$P$-admissible maps and $P$-fully faithful maps? Let us consider
a couple of examples.
\begin{example}
A well known example of a KZ doctrine is the free small cocompletion
operation on locally small categories, which sends a locally small
category $\mathcal{A}$ to its category of small presheaves. In particular,
when $\mathcal{A}$ is small the free small cocompletion is $P\mathcal{A}=\left[\mathcal{A}^{\textnormal{op}},\mathbf{Set}\right]$.
In this example, the $P$-cocomplete objects are those locally small
categories which are small cocomplete and the $P$-homomorphisms are
those functors between such categories preserving small colimits.
The $P$-admissible maps are those functors $L\colon\mathcal{A}\to\mathcal{B}$
for which $\mathcal{B}\left(L-,-\right)\colon\mathcal{B}\to\left[\mathcal{A}^{\textnormal{op}},\mathbf{Set}\right]$
factors through $P\mathcal{A}$. The $P$-fully faithful maps are
the fully faithful functors.
\end{example}
Another example is the free \emph{large} cocompletion KZ doctrine
on locally small categories. The reader should keep in mind a theorem
of Freyd showing that any locally small category which admits all
large colimits is a preorder. Consequently, a locally small category
is large cocomplete precisely when it is a preorder with all large
joins. This KZ doctrine has some unusual properties. For example it
is a cocompletion KZ doctrine with unit components not always fully
faithful. Moreover, every functor is admissible against the large
cocompletion. We define this KZ doctrine $P\colon\mathbf{Cat}\to\mathbf{Cat}$
by the assignment 
\[
\begin{aligned}P\colon\textnormal{ob}\;\mathbf{Cat}\to\textnormal{ob}\;\mathbf{Cat}\colon & \mathcal{A}\mapsto\left[\mathcal{A}^{\textnormal{op}},\mathbbm2\right]\end{aligned}
\]
with unit maps for each $\mathcal{A}\in\mathbf{Cat}$ given by
\[
y_{\mathcal{A}}\colon\mathcal{A}\to\left[\mathcal{A}^{\textnormal{op}},\mathbbm2\right]\colon X\mapsto\mathcal{A}\left\langle -,X\right\rangle 
\]
where each $\mathcal{A}\left\langle -,X\right\rangle $ is defined
as 
\[
\mathcal{A}\left\langle -,X\right\rangle \colon\mathcal{A}^{\textnormal{op}}\to\mathbbm2\colon S\mapsto\begin{cases}
1, & \exists\;S\overset{f}{\longrightarrow}X\textnormal{ in }\mathcal{A}\\
0, & \textnormal{otherwise}
\end{cases}
\]

For any functor $F\colon\mathcal{A}\to\mathcal{D}$ where $\mathcal{D}$
is a preordered category with all large joins (such as $P\mathcal{B}$
for any $\mathcal{B}$) we may define a left extension $\overline{F}\colon\left[\mathcal{A}^{\textnormal{op}},\mathbbm2\right]\to\mathcal{D}$
as in the left diagram
\[
\xymatrix@=1em{\left[\mathcal{A}^{\textnormal{op}},\mathbbm2\right]\ar[rr]^{\overline{F}} &  & \mathcal{D}\\
 & \;\ar@{}[ul]|-{\overset{\textnormal{id}}{\Longleftarrow\;}} &  &  & {\displaystyle \overline{F}\left(H\right)=\sup_{X\in\mathcal{A}\colon HX=1}FX}\\
\mathcal{A}\ar[uu]^{y_{\mathcal{A}}}\ar[rruu]_{F}
}
\]
by the assignment on the right.

For this KZ doctrine, the $P$-cocomplete objects are the large cocomplete
categories and the $P$-homomorphisms are the order and join preserving
maps between such categories. Every map is $P$-admissible, and a
map $L\colon\mathcal{A}\to\mathcal{B}$ is $P$-fully faithful precisely
when there exists a map $X\to Y$ in $\mathcal{A}$ if and only if
there exists a map $LX\to LY$ in $\mathcal{B}$.

\section{Doctrinal Partial Adjunctions and Left Extensions}

\subsection{Doctrinal Partial Adjunctions\label{doctrinalpartialadjunctions}}

In this section we study how pseudomonads interact with absolute left
liftings (also called partial adjunctions or relative adjunctions),
which we now define. The goal of this section is to generalize the
oplax-lax bijection given in Kelly's doctrinal adjunction \cite{doctrinal}.
In particular, we show that we get an induced oplax structure on a
partial left adjoint under suitable conditions, which gives a lifting
of the partial adjunction to the setting of pseudo algebras in a suitable
sense. 
\begin{defn}
Suppose we are given a diagram of the form
\[
\xymatrix@=1em{\mathcal{B}\ar[rr]^{R} &  & \mathcal{C}\ar@{}[ld]|-{\stackrel{\eta}{\Longleftarrow}}\\
 & \; &  &  & (3.1)\\
 &  & \mathcal{A}\ar[uu]_{I}\ar[uull]^{L}
}
\]
in a 2-category $\mathscr{C}$ equipped with a 2-cell $\eta:I\to R\cdot L$.
We call such a diagram a \emph{partial adjunction} and say that $L$
is a \emph{partial left adjoint} to $R$ if given any 1-cells $M$
and $N$ as below, for any 2-cell $\zeta:I\cdot M\to R\cdot N$ there
exists a unique $\overline{\zeta}:L\cdot M\to N$ such that $\zeta$
is equal to the pasting
\[
\xymatrix@=1em{\mathcal{B}\ar[rr]^{R} &  & \mathcal{C}\ar@{}[ld]|-{\stackrel{\eta}{\Longleftarrow}}\\
 & \;\ar@{}[ld]|-{\stackrel{\overline{\zeta}}{\Longleftarrow}\;}\\
\mathcal{D}\ar[rr]_{M}\ar[uu]^{N} &  & \mathcal{A}\ar[uu]_{I}\ar[uull]^{L}
}
\]
That is, pasting 2-cells of the form $\overline{\zeta}$ above with
$\eta$ defines a bijection of 2-cells.\end{defn}
\begin{rem}
It is an easy and well known exercise to check that we have an adjunction
$L\dashv R:\mathcal{B}\to\mathcal{A}$ with unit $\eta$ in a 2-category
$\mathscr{C}$ if and only if 
\[
\xymatrix@=1em{\mathcal{B}\ar[rr]^{R} &  & \mathcal{A}\ar@{}[ld]|-{\stackrel{\eta}{\Longleftarrow}}\\
 & \;\\
 &  & \mathcal{A}\ar[uu]_{\textnormal{id}_{\mathcal{A}}}\ar[uull]^{L}
}
\]
exhibits $L$ as a partial left adjoint. 
\end{rem}
We now define a notion of partial adjunction in the context of pseudo
$T$-algebras and $T$-morphisms. 
\begin{defn}
Suppose we are given oplax $T$-morphisms $\left(I,\xi\right)$ and
$\left(L,\alpha\right)$ and a lax $T$-morphism $\left(R,\beta\right)$
equipped with a $T$-transformation $\eta$ (as in Remark \ref{Ttransremark}
with appropriate identities) as in the diagram

\[
\xymatrix@=1em{\left(\mathcal{B},T\mathcal{B}\overset{y}{\rightarrow}\mathcal{B}\right)\ar[rr]^{\left(R,\beta\right)} &  & \left(\mathcal{C},T\mathcal{C}\overset{z}{\rightarrow}\mathcal{C}\right)\ar@{}[ld]|-{\stackrel{\eta}{\Longleftarrow}}\\
 & \;\\
 &  & \left(\mathcal{A},T\mathcal{A}\overset{x}{\rightarrow}\mathcal{A}\right)\ar[uu]_{\left(I,\xi\right)}\ar[uull]^{\left(L,\alpha\right)}
}
\]
We call such a diagram a \emph{$T$-partial adjunction }if for any
given pseudo $T$-algebras $\left(\mathcal{D},w\right),\left(\mathcal{U},u\right)$
and $\left(\mathcal{S},s\right)$, lax $T$-morphism $\left(M,{\varepsilon}\right)$,
and oplax $T$-morphism $\left(N,\varphi\right)$ as below
\[
\xymatrix@=1em{\left(\mathcal{B},T\mathcal{B}\overset{y}{\rightarrow}\mathcal{B}\right)\ar[rr]^{\left(R,\beta\right)} &  & \left(\mathcal{C},T\mathcal{C}\overset{z}{\rightarrow}\mathcal{C}\right)\ar@{}[ld]|-{\stackrel{\eta}{\Longleftarrow}}\\
 & \;\ar@{}[ld]|-{\stackrel{\overline{\zeta}}{\Longleftarrow}\quad\quad}\\
\left(\mathcal{D},T\mathcal{D}\overset{w}{\rightarrow}\mathcal{D}\right)\ar[rr]_{\left(M,{\varepsilon}\right)}\ar[uu]^{\left(N,\varphi\right)} &  & \left(\mathcal{A},T\mathcal{A}\overset{x}{\rightarrow}\mathcal{A}\right)\ar[uu]_{\left(I,\xi\right)}\ar[ulul]^{\left(L,\alpha\right)}
}
\]
pasting $T$-transformations of the form $\overline{\zeta}$ above
with the $T$-transformation $\eta$ defines the bijection of $T$-transformations:
\[
\xymatrix@=1em{\left(\mathcal{B},y\right)\ar@{=}[rr] &  & \left(\mathcal{B},y\right)\ar@{}[ldld]|-{\stackrel{\overline{\zeta}}{\Longleftarrow}} &  &  & \left(\mathcal{B},y\right)\ar[rr]^{\left(R,\beta\right)} &  & \left(\mathcal{C},z\right)\ar@{}[ldld]|-{\stackrel{\zeta}{\Longleftarrow}}\\
 &  &  & \ar@{}[r]|-{\sim} & \;\\
\left(\mathcal{D},w\right)\ar[rr]_{\left(M,{\varepsilon}\right)}\ar[uu]^{\left(N,\varphi\right)} &  & \left(\mathcal{A},x\right)\ar[uu]_{\left(L,\alpha\right)} &  &  & \left(\mathcal{D},w\right)\ar[rr]_{\left(M,{\varepsilon}\right)}\ar[uu]^{\left(N,\varphi\right)} &  & \left(\mathcal{A},x\right)\ar[uu]_{\left(I,\xi\right)}
}
\]

This operation of pasting the $T$-transformation $\overline{\zeta}$
with $\eta$ is given by pasting the underlying 2-cells; for a verification
that this does yield a $T$-transformation, we refer the reader to
the final two diagrams of the proof of Proposition \ref{docparadj}.\end{defn}
\begin{rem}
We may be more general here by replacing $\left(M,{\varepsilon}\right)$
and $\left(N,\varphi\right)$ by a lax followed by an oplax, and an
oplax followed by a lax $T$-morphism respectively. However, this
level of generality will not be necessary for this paper.
\end{rem}
We now give the doctrinal properties enjoyed by partial adjunctions.
\begin{prop}
\label{docparadj} Suppose we are given a partial adjunction
\[
\xymatrix@=1em{\mathcal{B}\ar[rr]^{R} &  & \mathcal{C}\ar@{}[ld]|-{\stackrel{\eta}{\Longleftarrow}}\\
 & \;\\
 &  & \mathcal{A}\ar[uu]_{I}\ar[uull]^{L}
}
\]
in a 2-category $\mathscr{C}$ equipped with a pseudomonad $\left(T,u,m\right)$.
Suppose further that 
\[
\left(\mathcal{A},T\mathcal{A}\stackrel{x}{\longrightarrow}\mathcal{A}\right),\;\left(\mathcal{B},T\mathcal{B}\stackrel{y}{\longrightarrow}\mathcal{B}\right),\;\left(\mathcal{C},T\mathcal{C}\stackrel{z}{\longrightarrow}\mathcal{C}\right)
\]
are pseudo $T\text{-algebras}$. Then given an oplax $T\text{-morphism}$
structure $\xi$ on $I$ and a lax $T\text{-morphism}$ structure
$\beta$ on $R$, there exists a unique oplax $T\text{-morphism}$
structure $\alpha$ on $L$ such that $\eta$ extends to a $T$-transformation.
Moreover, this partial adjunction is then lifted to the $T$-partial
adjunction
\[
\xymatrix@=1em{\left(\mathcal{B},T\mathcal{B}\overset{y}{\rightarrow}\mathcal{B}\right)\ar[rr]^{\left(R,\beta\right)} &  & \left(\mathcal{C},T\mathcal{C}\overset{z}{\rightarrow}\mathcal{C}\right)\ar@{}[ld]|-{\stackrel{\eta}{\Longleftarrow}}\\
 & \;\\
 &  & \left(\mathcal{A},T\mathcal{A}\overset{x}{\rightarrow}\mathcal{A}\right)\ar[uu]_{\left(I,\xi\right)}\ar[uull]^{\left(L,\alpha\right)}
}
\]
\end{prop}
\begin{proof}
Given our 2-cells 
\[
\xymatrix@=1em{T\mathcal{A}\ar[dd]_{TI}\ar[rr]^{x}\ar@{}[rrdd]|-{\Downarrow\xi} &  & \mathcal{A}\ar[dd]^{I} &  & T\mathcal{B}\ar[dd]_{TR}\ar[rr]^{x}\ar@{}[rrdd]|-{\Uparrow\beta} &  & \mathcal{B}\ar[dd]^{R}\\
\\
T\mathcal{C}\ar[rr]_{z} &  & \mathcal{C} &  & T\mathcal{C}\ar[rr]_{z} &  & \mathcal{C}
}
\]
exhibiting $I$ as an oplax $T\text{-morphism}$ and $R$ as a lax
$T\text{-morphism}$, we can take our oplax constraint cell of $L$
(which we call $\alpha$) as the unique solution to
\[
\xymatrix{T\mathcal{B}\ar[r]^{y}\;\ar@{}[rdd]|-{\Uparrow\alpha} & \mathcal{B}\ar[rd]^{R} &  &  & T\mathcal{B}\ar[r]^{y}\ar[dr]^{TR} & \mathcal{B}\ar[rd]^{R}\ar@{}[d]|-{\Uparrow\beta}\\
 & \;\ar@{}[r]|-{\Uparrow\eta} & \mathcal{C} & = & \;\ar@{}[r]|-{\Uparrow T\eta} & T\mathcal{C}\ar[r]^{z}\ar@{}[d]|-{\Uparrow\xi} & \mathcal{C}\\
T\mathcal{A}\ar[r]_{x}\ar[uu]^{TL} & \mathcal{A}\ar[uu]_{L}\ar[ur]_{I} &  &  & T\mathcal{A}\ar[r]_{x}\ar[uu]^{TL}\ar[ur]^{TI} & \mathcal{A}\ar[ur]_{I}
}
\]
which exists since $L$ is a partial left adjoint. That is, $\alpha$
is the unique oplax structure on $L$ for which $\eta:I\to R\cdot L$
extends to a $T$-transformation. The unit coherence property follows
since pasting with the unit and the unit isomorphisms gives 
\[
\xymatrix{ & \; &  &  &  &  & \;\\
\mathcal{B}\ar[r]^{u_{\mathcal{B}}}\ar@{}[rdd]|-{\Uparrow u_{L}}\ar@/^{2.5pc}/[rr]^{\textnormal{id}_{\mathcal{B}}} & T\mathcal{B}\ar[r]^{y}\;\ar@{}[rdd]|-{\Uparrow\alpha}\ar@{}[u]|-{\cong} & \mathcal{B}\ar[rd]^{R} &  &  & \mathcal{B}\ar[r]^{u_{\mathcal{B}}}\ar@{}[rdd]|-{\Uparrow u_{L}}\ar@/^{2.5pc}/[rr]^{\textnormal{id}_{\mathcal{B}}} & T\mathcal{B}\ar[r]^{y}\ar[dr]^{TR}\ar@{}[u]|-{\cong} & \mathcal{B}\ar[rd]^{R}\ar@{}[d]|-{\Uparrow\beta}\\
 &  & \;\ar@{}[r]|-{\Uparrow\eta} & \mathcal{C} & = &  & \;\ar@{}[r]|-{\Uparrow T\eta} & T\mathcal{C}\ar[r]^{z}\ar@{}[d]|-{\Uparrow\xi} & \mathcal{C}\\
\mathcal{A}\ar[uu]^{L}\ar[r]_{u_{\mathcal{A}}}\ar@/_{2.5pc}/[rr]_{\textnormal{id}_{\mathcal{A}}} & T\mathcal{A}\ar[r]_{x}\ar[uu]^{TL}\ar@{}[d]|-{\cong} & \mathcal{A}\ar[uu]_{L}\ar[ur]_{I} &  &  & \mathcal{A}\ar[uu]^{L}\ar[r]_{u_{\mathcal{A}}}\ar@/_{2.5pc}/[rr]_{\textnormal{id}_{\mathcal{A}}} & T\mathcal{A}\ar[r]_{x}\ar[uu]^{TL}\ar[ur]^{TI}\ar@{}[d]|-{\cong} & \mathcal{A}\ar[ur]_{I}\\
 & \; &  &  &  &  & \;
}
\]
and the right hand side reduces, by pseudonaturality of the unit and
then the fact that $\xi$ and $\beta$ satisfy the unit coherence
property, to:
\[
\xymatrix{ & \;\\
\mathcal{B}\ar[r]^{u_{\mathcal{B}}}\ar[dr]_{R}\ar@{}[ddr]|-{\Uparrow\eta}\ar@/^{2.5pc}/[rr]^{\textnormal{id}_{\mathcal{B}}} & T\mathcal{B}\ar[r]^{y}\ar[dr]_{TR}\ar@{}[d]|-{\Uparrow u_{R}^{-1}}\ar@{}[u]|-{\cong} & \mathcal{B}\ar[rd]^{R}\ar@{}[d]|-{\Uparrow\beta} &  &  & \mathcal{B}\ar[dr]^{R}\\
 & \mathcal{C}\ar[r]^{u_{\mathcal{C}}}\ar@{}[d]|-{\Uparrow u_{I}} & T\mathcal{C}\ar[r]^{z}\ar@{}[d]|-{\Uparrow\xi} & \mathcal{C} & = & \;\ar@{}[r]|-{\Uparrow\eta} & \mathcal{C}\\
\mathcal{A}\ar[r]_{u_{\mathcal{A}}}\ar[ur]^{I}\ar[uu]^{L}\ar@/_{2.5pc}/[rr]_{\textnormal{id}_{\mathcal{A}}} & T\mathcal{A}\ar[r]_{x}\ar[ur]^{TI}\ar@{}[d]|-{\cong} & \mathcal{A}\ar[ur]_{I} &  &  & \mathcal{A}\ar[ur]_{I}\ar[uu]^{L}\\
 & \;
}
\]
As $\eta$ exhibits $L$ as a partial left adjoint, we can cancel
it to obtain the desired unit coherence property. For the multiplicative
coherence property we note that using the definition of $\alpha$
twice gives
\[
\xymatrix{ & T\mathcal{B}\ar[rd]^{y} &  &  &  &  & T\mathcal{B}\ar[rd]^{y}\\
T^{2}\mathcal{B}\ar[r]^{Ty}\ar@{}[rdd]|-{\Uparrow T\alpha}\ar[ru]^{m_{\mathcal{B}}} & T\mathcal{B}\ar[r]^{y}\ar@{}[rdd]|-{\Uparrow\alpha}\ar@{}[u]|-{\cong} & \mathcal{B}\ar[rd]^{R} &  &  & T^{2}\mathcal{B}\ar[r]^{Ty}\ar[dr]_{T^{2}R}\ar@{}[ddr]|-{\Uparrow T^{2}\eta}\ar[ru]^{m_{\mathcal{B}}} & T\mathcal{B}\ar[r]^{y}\ar[dr]^{TR}\ar@{}[d]|-{\Uparrow T\beta}\ar@{}[u]|-{\cong} & \mathcal{B}\ar[rd]^{R}\ar@{}[d]|-{\Uparrow\beta}\\
 &  & \;\ar@{}[r]|-{\Uparrow\eta} & \mathcal{C} & = &  & T^{2}\mathcal{C}\ar[r]^{Tz}\ar@{}[d]|-{\Uparrow T\xi} & T\mathcal{C}\ar[r]^{z}\ar@{}[d]|-{\Uparrow\xi} & \mathcal{C}\\
T^{2}\mathcal{A}\ar[uu]^{T^{2}L}\ar[r]_{Tx}\ar[rd]_{m_{\mathcal{A}}} & T\mathcal{A}\ar[r]_{x}\ar[uu]^{TL} & \mathcal{A}\ar[uu]_{L}\ar[ur]_{I} &  &  & T^{2}\mathcal{A}\ar[uu]^{T^{2}L}\ar[r]_{Tx}\ar[ru]^{T^{2}I}\ar[rd]_{m_{\mathcal{A}}} & T\mathcal{A}\ar[r]_{x}\ar[ur]^{TI} & \mathcal{A}\ar[ur]_{I}\\
 & T\mathcal{A}\ar[ur]_{x}\ar@{}[u]|-{\cong} &  &  &  &  & T\mathcal{A}\ar[ur]_{x}\ar@{}[u]|-{\cong}
}
\]
and simplifying the right hand side by using that both $\xi$ and
$\beta$ satisfy the multiplication coherence property and then pseudonaturality
of $m$ gives 
\[
\xymatrix{T^{2}\mathcal{B}\ar[r]^{m_{\mathcal{B}}}\ar[dr]_{T^{2}R}\ar@{}[ddr]|-{\Uparrow T^{2}\eta} & T\mathcal{B}\ar[r]^{y}\ar[dr]^{TR}\ar@{}[d]|-{\Uparrow m_{R}^{-1}} & \mathcal{B}\ar[rd]^{R}\ar@{}[d]|-{\Uparrow\beta} &  &  & T^{2}\mathcal{B}\ar[r]^{m_{\mathcal{B}}}\ar@{}[ddr]|-{\Uparrow m_{L}} & T\mathcal{B}\ar[r]^{y}\ar[dr]^{TR} & \mathcal{B}\ar[rd]^{R}\ar@{}[d]|-{\Uparrow\beta}\\
 & T^{2}\mathcal{C}\ar[r]^{m_{\mathcal{C}}}\ar@{}[d]|-{\Uparrow m_{I}} & T\mathcal{C}\ar[r]^{z}\ar@{}[d]|-{\Uparrow\xi} & \mathcal{C} & = &  & \;\ar@{}[r]|-{\Uparrow T\eta} & T\mathcal{C}\ar[r]^{z}\ar@{}[d]|-{\Uparrow\xi} & \mathcal{C}\\
T^{2}\mathcal{A}\ar[uu]^{T^{2}L}\ar[r]_{m_{\mathcal{A}}}\ar[ru]^{T^{2}I} & T\mathcal{A}\ar[r]_{x}\ar[ur]^{TI} & \mathcal{A}\ar[ur]_{I} &  &  & T^{2}\mathcal{A}\ar[uu]^{T^{2}L}\ar[r]_{m_{\mathcal{A}}} & T\mathcal{A}\ar[r]_{x}\ar[ur]^{TI}\ar[uu]^{TL} & \mathcal{A}\ar[ur]_{I}
}
\]
which by definition of $\alpha$ is precisely
\[
\xymatrix{T^{2}\mathcal{B}\ar[r]^{m_{\mathcal{B}}}\ar@{}[ddr]|-{\Uparrow m_{L}} & T\mathcal{B}\ar[r]^{y}\;\ar@{}[rdd]|-{\Uparrow\alpha} & \mathcal{B}\ar[rd]^{R}\\
 &  & \;\ar@{}[r]|-{\Uparrow\eta} & \mathcal{C}\\
T^{2}\mathcal{A}\ar[uu]^{T^{2}L}\ar[r]_{m_{\mathcal{A}}} & T\mathcal{A}\ar[r]_{x}\ar[uu]^{TL} & \mathcal{A}\ar[uu]_{L}\ar[ur]_{I}
}
\]
giving the required multiplicative coherence condition on $\alpha$,
again since $\eta$ exhibits $L$ as a partial left adjoint.

We now check this gives a $T$-partial adjunction. As the underlying
map $\overline{\zeta}\mapsto\zeta$ given by pasting with $\eta$
is bijective, it is trivial that the assignment of $T$-transformations
$\overline{\zeta}\mapsto\zeta$ is injective as the underlying data
is the same. For surjectivity, suppose we are given a $T$-transformation
$\zeta$, that is a $\zeta$ such that
\[
\xymatrix@=1em{ &  & T\mathcal{U}\ar[rr]^{u} &  & \mathcal{B}\ar[rdrd]^{R} &  &  &  &  &  & T\mathcal{B}\ar[rrdd]^{TR}\ar[rr]^{y} &  & \mathcal{B}\ar[rrdd]^{F}\\
\\
T\mathcal{D}\ar[rr]^{w}\ar[ruur]^{TN}\ar[rdrd]_{TM} &  & \mathcal{D}\ar[rdrd]_{M}\ar[ruur]^{N}\ar@{}[rrrr]|-{\Uparrow\zeta}\ar@{}[dd]|-{\Uparrow{\varepsilon}}\ar@{}[uu]|-{\Uparrow\varphi} &  &  &  & \mathcal{C} & = & T\mathcal{D}\ar[ruur]^{TN}\ar[rdrd]_{TM}\ar@{}[rrrr]|-{\Uparrow T\zeta} &  &  &  & T\mathcal{C}\ar[rr]^{z}\ar@{}[dd]|-{\Uparrow\xi}\ar@{}[uu]|-{\Uparrow\beta} &  & \mathcal{C}\\
\\
 &  & T\mathcal{S}\ar[rr]_{s} &  & \mathcal{A}\ar[urur]_{I} &  &  &  &  &  & T\mathcal{A}\ar[urur]_{TI}\ar[rr]_{x} &  & \mathcal{A}\ar[urur]_{G}
}
\]
We can factor this as 
\[
\xymatrix@=1em{ &  & T\mathcal{B}\ar[rr]^{y} &  & \mathcal{B}\ar[rrdd]^{R} &  &  &  &  &  & T\mathcal{B}\ar[rr]^{y}\ar[rdrd]_{TR} &  & \mathcal{B}\ar[rdrd]^{R}\\
\\
T\mathcal{D}\ar[rr]^{w}\ar[ruur]^{TN}\ar[rdrd]_{TM} &  & \mathcal{D}\ar[rdrd]_{M}\ar[ruur]^{N}\ar@{}[rr]|-{\Uparrow\overline{\zeta}}\ar@{}[dd]|-{\Uparrow{\varepsilon}}\ar@{}[uu]|-{\Uparrow\varphi} &  & \; & \;\ar@{}[]|-{\Uparrow\eta} & \mathcal{C} & = & T\mathcal{D}\ar[ruur]^{TN}\ar[rdrd]_{TM}\ar@{}[rr]|-{\Uparrow T\overline{\zeta}\quad} &  & \;\ar@{}[rr]|-{\Uparrow T\eta} &  & T\mathcal{C}\ar[rr]^{z}\ar@{}[uu]|-{\Uparrow\beta} &  & \mathcal{C}\\
\\
 &  & T\mathcal{A}\ar[rr]_{x} &  & \mathcal{A}\ar[uurr]_{I}\ar[uuuu]^{L} &  &  &  &  &  & T\mathcal{A}\ar[uuuu]^{TL}\ar[rr]_{x}\ar[urur]_{TI} &  & \mathcal{A}\ar[urur]_{I}\ar@{}[uu]|-{\Uparrow\xi}
}
\]
and by definition of $\alpha$ this may be written as 
\[
\xymatrix@=1em{ &  & T\mathcal{B}\ar[rr]^{y} &  & \mathcal{B}\ar[rrdd]^{R} &  &  &  &  &  & T\mathcal{B}\ar[rr]^{y} &  & \mathcal{B}\ar[rdrd]^{R}\\
\\
T\mathcal{D}\ar[rr]^{w}\ar[ruur]^{TN}\ar[rdrd]_{TM} &  & \mathcal{D}\ar[rdrd]_{M}\ar[ruur]^{N}\ar@{}[rr]|-{\Uparrow\overline{\zeta}}\ar@{}[dd]|-{\Uparrow{\varepsilon}}\ar@{}[uu]|-{\Uparrow\varphi} &  & \; & \;\ar@{}[]|-{\Uparrow\eta} & \mathcal{C} & = & T\mathcal{D}\ar[ruur]^{TN}\ar[rdrd]_{TM}\ar@{}[rr]|-{\Uparrow T\overline{\zeta}\quad} &  & \;\ar@{}[rr]|-{\Uparrow\alpha} &  & \;\ar@{}[rr]|-{\Uparrow\eta} &  & \mathcal{C}\\
\\
 &  & T\mathcal{A}\ar[rr]_{x} &  & \mathcal{A}\ar[uurr]_{I}\ar[uuuu]^{L} &  &  &  &  &  & T\mathcal{A}\ar[uuuu]^{TL}\ar[rr]_{x} &  & \mathcal{A}\ar[urur]_{I}\ar[uuuu]_{L}
}
\]
Since $\eta$ exhibits $L$ as a partial left adjoint, we may cancel
the $\eta$ to find that $\overline{\zeta}$ is indeed a $T$-transformation.\end{proof}
\begin{rem}
Notice that the last two diagrams in the proof above show that if
both $\overline{\zeta}$ and $\eta$ are $T$-transformations, then
the composite is as well.
\end{rem}
The following example is an easy application of this result.
\begin{example}
\label{compositeoplax} For bicategories $\mathscr{A},\mathscr{B}$
and $\mathscr{C}$ and a diagram 
\[
\xymatrix{\mathscr{A}\ar@{-->}[r]^{F}\ar@/_{1pc}/[rr]_{H} & \mathscr{B}\ar[r]^{G} & \mathscr{C}}
\]
where $G$ is a lax and locally fully faithful functor, $H$ is an
oplax functor, $F$ is a locally defined functor
\[
\left(F_{X,Y}:\mathscr{A}\left(X,Y\right)\to\mathscr{B}\left(FX,FY\right):X,Y\in\mathscr{A}\right)
\]
and $G\cdot F=H$ locally, it follows that $F$ extends to an oplax
functor.

To see this, recall that the fully faithfulness of each $G_{M,N}$
(for objects $M,N\in\mathscr{B}$) may be characterized by saying
that each 
\[
\xymatrix@=1em{\mathscr{B}\left(M,N\right)\ar@{->}[rr]^{G_{M,N}} &  & \mathscr{C}\left(HM,HN\right)\ar@{}[dl]|-{\stackrel{\text{id}}{\Longleftarrow}}\\
 & \mbox{\;}\\
 &  & \mathscr{B}\left(M,N\right)\ar@{->}[ulul]^{\textnormal{id}_{\mathscr{B}\left(M,N\right)}}\ar[uu]_{G_{M,N}}
}
\]
is an absolute lifting \cite[Example 2.18]{weberyoneda}. As this
absolute left lifting is preserved upon whiskering by 
\[
F_{X,Y}:\mathscr{A}\left(X,Y\right)\to\mathscr{B}\left(FX,FY\right)
\]
we have the family of partial adjunctions
\[
\xymatrix@=1em{\mathscr{B}\left(FX,FY\right)\ar@{->}[rr]^{G_{FX,FY}} &  & \mathscr{C}\left(HX,HY\right)\ar@{}[dl]|-{\stackrel{\text{id}}{\Longleftarrow}}\\
 & \mbox{\;}\\
 &  & \mathscr{A}\left(X,Y\right)\ar@{->}[ulul]^{F_{X,Y}}\ar[uu]_{H_{X,Y}}
}
\]
Endowing with the bicategory structure of $\mathscr{A}$, and full
sub-bicategory structures of $\mathscr{B}$ and $\mathscr{C}$ restricted
to objects in the images of $F$ and $H$ respectively, we see by
Proposition \ref{docparadj} that $F$ extends to an oplax functor
$F:\mathscr{A}\to\mathscr{B}$.

Of course this may be stated more generally in the setting of a pseudo
$T$-algebras (and also by only asking $GF\cong H$ on the underlying
2-category).
\end{example}

\begin{rem}
In Kelly's setting of a doctrinal adjunction, if both the left and
right adjoint are lax, exhibited by a counit and unit which are $T$-transformations
of lax $T$-morphisms, then the induced oplax structure on the left
adjoint is inverse to the given lax structure. In this partial adjunction
case the best we can say is that if $\left(I,\xi\right)$ is pseudo,
$\left(L,\alpha^{\ast}\right)$ lax, and $\eta\colon\left(I,\xi\right)\to\left(R,\beta\right)\cdot\left(L,\alpha^{\ast}\right)$
a $T$-transformation of lax $T$-morphisms, then the induced oplax
structure on $L$ given as $\alpha$ satisfies $\alpha^{\ast}\cdot\alpha=\textnormal{id}_{L\cdot x}$.
This means the identity 2-cell is a generalized $T$-transformation
from $\left(L,\alpha\right)$ to $\left(L,\alpha^{*}\right)$, but
not necessarily the other way.
\end{rem}

\subsection{Doctrinal Left Extensions\label{doctrinalleftextensions}}

We now look at how pseudomonads interact with left extensions. This
is in the same spirit as the above section on partial adjunctions,
but not completely analogous. In particular, here we require that
the pseudo $T$-algebra we are extending into satisfies an algebraic
cocompleteness property. In this section we give conditions under
which a left extension lifts to a suitable notion of left extension
in the setting of pseudo $T$-algebras and $T$-morphisms. The results
of this section are mostly due to Koudenburg in a double category
setting \cite{roald2015}. 
\begin{defn}
Suppose we are given a diagram of the form
\[
\xymatrix@=1em{\mathcal{B}\ar[rr]^{R} &  & \mathcal{C}\ar@{}[ld]|-{\stackrel{\eta}{\Longleftarrow}}\\
 & \;\\
 &  & \mathcal{A}\ar[uu]_{I}\ar[uull]^{L}
}
\]
We say that $R$ is exhibited as a left extension by $\eta$ when
pasting 2-cells $\varphi:R\to M$ with the 2-cell $\eta:I\to R\cdot L$
as in the diagram 
\[
\xymatrix@=1em{\mathcal{B}\ar[rr]^{R}\ar@/^{1.5pc}/[rr]_{\Uparrow\varphi}^{M} &  & \mathcal{C}\ar@{}[ld]|-{\stackrel{\eta}{\Longleftarrow}}\\
 & \;\\
 &  & \mathcal{A}\ar[uu]_{I}\ar[uull]^{L}
}
\]
defines a bijection between 2-cells $R\to M$ and 2-cells $I\to M\cdot L$.
\end{defn}
We now give a suitable notion of when a lax $T$-morphism may be regarded
as a left extension in the setting of pseudo $T$-algebras.
\begin{defn}
Suppose we are given an oplax $T$-morphism $\left(L,\alpha\right)$
and lax $T$-morphisms $\left(R,\beta\right)$ and $\left(I,\sigma\right)$
between pseudo $T$-algebras equipped with a $T$-transformation $\eta$
as in the diagram

\[
\xymatrix@=1em{\left(\mathcal{B},T\mathcal{B}\overset{y}{\rightarrow}\mathcal{B}\right)\ar[rr]^{\left(R,\beta\right)} &  & \left(\mathcal{C},T\mathcal{C}\overset{z}{\rightarrow}\mathcal{C}\right)\ar@{}[ld]|-{\stackrel{\eta}{\Longleftarrow}}\\
 & \;\\
 &  & \left(\mathcal{A},T\mathcal{A}\overset{x}{\rightarrow}\mathcal{A}\right)\ar[uu]_{\left(I,\sigma\right)}\ar[uull]^{\left(L,\alpha\right)}
}
\]
We call such a diagram a \emph{$T$-left extension }if for any given
pseudo $T$-algebra $\left(\mathcal{D},w\right)$, lax $T$-morphism
$\left(M,{\varepsilon}\right)$ and oplax $T$-morphism $\left(N,\varphi\right)$
as below

\[
\xymatrix@=1em{ & \left(\mathcal{D},T\mathcal{D}\overset{w}{\rightarrow}\mathcal{D}\right)\ar[rd]^{\left(M,{\varepsilon}\right)}\\
\left(\mathcal{B},T\mathcal{B}\overset{y}{\rightarrow}\mathcal{B}\right)\ar[rr]_{\left(R,\beta\right)}\ar[ur]^{\left(N,\varphi\right)} & \;\ar@{}[u]|-{\Uparrow\overline{\zeta}} & \left(\mathcal{C},T\mathcal{C}\overset{z}{\rightarrow}\mathcal{C}\right)\ar@{}[ld]|-{\stackrel{\eta}{\Longleftarrow}}\\
 & \;\\
 &  & \left(\mathcal{A},T\mathcal{A}\overset{x}{\rightarrow}\mathcal{A}\right)\ar[uu]_{\left(I,\sigma\right)}\ar[uull]^{\left(L,\alpha\right)}
}
\]
pasting $T$-transformations of the form $\overline{\zeta}$ above
with the $T$-transformation $\eta$ defines the bijection of $T$-transformations:
\[
\xymatrix@=1em{\left(\mathcal{D},w\right)\ar[rr]^{\left(M,{\varepsilon}\right)} &  & \left(\mathcal{C},z\right)\ar@{}[ldld]|-{\stackrel{\overline{\zeta}}{\Longleftarrow}} &  &  & \left(\mathcal{D},w\right)\ar[rr]^{\left(M,{\varepsilon}\right)} &  & \left(\mathcal{C},z\right)\ar@{}[ldld]|-{\stackrel{\zeta}{\Longleftarrow}}\\
 &  &  & \ar@{}[r]|-{\sim} & \; & \left(\mathcal{B},y\right)\ar[u]^{\left(N,\varphi\right)}\\
\left(\mathcal{B},y\right)\ar[rr]_{\left(R,\beta\right)}\ar[uu]^{\left(N,\varphi\right)} &  & \left(\mathcal{C},z\right)\ar[uu]_{\left(\textnormal{id},\textnormal{id}\right)} &  &  & \left(\mathcal{A},x\right)\ar[rr]_{\left(I,\sigma\right)}\ar[u]^{\left(L,\alpha\right)} &  & \left(\mathcal{C},z\right)\ar[uu]_{\left(\textnormal{id},\textnormal{id}\right)}
}
\]

To see that if $\overline{\zeta}$ and $\eta$ are both $T$-transformations
then so is the composite, we refer the reader the last two diagrams
of the proof of Proposition \ref{docleftext}.
\end{defn}
The left extension property will not be enough to get these $T$-left
extensions, and so we will also require an algebraic cocompleteness
property.
\begin{defn}
\label{Tpreserved} We say a left extension $\left(H,\varphi\right)$
as on the left below is \emph{$T$-preserved} by a 1-cell $z:T\mathcal{C}\to\mathcal{D}$
when 
\[
\xymatrix@=1em{\mathcal{B}\ar[rr]^{H} &  & \mathcal{C}\ar@{}[ld]|-{\stackrel{\varphi}{\Longleftarrow}} &  &  & T\mathcal{B}\ar[rr]^{TH} &  & T\mathcal{C}\ar@{}[ld]|-{\stackrel{T\varphi}{\Longleftarrow}}\ar[r]^{z} & \mathcal{D}\\
 & \; &  &  &  &  &  &  & \;\ar@{}[ul]|-{\stackrel{\textnormal{id}}{\Longleftarrow}\quad\;\;}\\
 &  & \mathcal{X}\ar[uu]_{F}\ar[uull]^{G} &  &  &  &  & T\mathcal{X}\ar[uu]^{TF}\ar[uull]^{TG}\ar[uur]_{z\cdot TF}
}
\]
the pasting diagram on the right exhibits $\left(z\cdot TH,z\cdot T\varphi\right)$
as a left extension. If for every $F$ and $G$ the left extension
$\left(H,\varphi\right)$ as above is $T$-preserved by $z$ we then
say $z$ is \emph{$T$-cocontinuous}.\end{defn}
\begin{rem}
Given a pseudo $T$-algebra $\left(\mathcal{C},T\mathcal{C}\overset{z}{\rightarrow}\mathcal{C}\right)$
if we ask that the underlying object $\mathcal{C}$ is cocomplete
in the sense that all left extensions into $\mathcal{C}$ exist, and
moreover that the algebra structure map $z$ is $T$-cocontinuous,
then this is the notion of algebraic cocompleteness as given by Weber
\cite[Definition 2.3.1]{markextension} (except that we are not using
pointwise left extensions here). In the setting monoidal categories,
this condition of $z$ being $T$-cocontinuous is the analogue of
asking the tensor product be separately cocontinuous; see \cite[Prop. 2.3.2]{markextension}.
\end{rem}
We now recall a result for algebraic left extensions mostly due to
Koudenburg \cite{roald2015} (though we avoid working in a double
categorical setting). We will include some details of the proof as
we will need them later.
\begin{prop}
\label{docleftext} Suppose we are given a diagram
\[
\xymatrix@=1em{\mathcal{B}\ar[rr]^{R} &  & \mathcal{C}\ar@{}[ld]|-{\stackrel{\eta}{\Longleftarrow}}\\
 & \;\\
 &  & \mathcal{A}\ar[uu]_{I}\ar[uull]^{L}
}
\]
which exhibits $R$ as a left extension in a 2-category $\mathscr{C}$
equipped with a pseudomonad $\left(T,u,m\right)$. Suppose further
that 
\[
\left(\mathcal{A},T\mathcal{A}\stackrel{x}{\longrightarrow}\mathcal{A}\right),\;\left(\mathcal{B},T\mathcal{B}\stackrel{y}{\longrightarrow}\mathcal{B}\right),\;\left(\mathcal{C},T\mathcal{C}\stackrel{z}{\longrightarrow}\mathcal{C}\right)
\]
are pseudo $T\text{-algebras}$. Suppose even further that the left
extension $\left(R,\eta\right)$ is $T$-preserved by $z$. Then given
a lax $T\text{-morphism}$ structure $\sigma$ on $I$ and an oplax
$T\text{-morphism}$ structure $\alpha$ on $L$, there exists a unique
lax $T\text{-morphism}$ structure $\beta$ on $R$ such that $\eta$
extends to a $T$-transformation. Moreover, this left extension is
then lifted to the $T$-left extension
\[
\xymatrix@=1em{\left(\mathcal{B},T\mathcal{B}\overset{y}{\rightarrow}\mathcal{B}\right)\ar[rr]^{\left(R,\beta\right)} &  & \left(\mathcal{C},T\mathcal{C}\overset{z}{\rightarrow}\mathcal{C}\right)\ar@{}[ld]|-{\stackrel{\eta}{\Longleftarrow}}\\
 & \;\\
 &  & \left(\mathcal{A},T\mathcal{A}\overset{x}{\rightarrow}\mathcal{A}\right)\ar[uu]_{\left(I,\sigma\right)}\ar[uull]^{\left(L,\alpha\right)}
}
\]
\end{prop}
\begin{proof}
Given our structure cells $\sigma$ and $\alpha$ as below
\[
\xymatrix{T\mathcal{A}\ar[d]_{TI}\ar[r]^{x}\ar@{}[rd]|-{\Uparrow\sigma} & \mathcal{A}\ar[d]^{I} &  & T\mathcal{A}\ar[d]_{TL}\ar[r]^{x}\ar@{}[rd]|-{\Downarrow\alpha} & \mathcal{A}\ar[d]^{L}\\
T\mathcal{C}\ar[r]_{z} & \mathcal{C} &  & T\mathcal{B}\ar[r]_{y} & \mathcal{B}
}
\]
 our lax constraint cell for $R$ is given as the unique $\beta$
for which $\eta$ extends to a $T$-transformation, that is

\[
\xymatrix{ & T\mathcal{B}\ar[r]^{y} & \mathcal{B}\ar[dd]^{R} &  &  & T\mathcal{B}\ar[r]^{y}\ar[dd]_{TR} & \mathcal{B}\ar[dd]^{R}\\
T\mathcal{A}\ar[ru]^{TL}\ar[r]^{x}\ar[rd]_{TI} & \mathcal{A}\ar[dr]^{I}\ar[ur]^{L}\ar@{}[r]|-{\Uparrow\eta}\ar@{}[u]|-{\Uparrow\alpha}\ar@{}[d]|-{\Uparrow\sigma} & \; & = & T\mathcal{A}\ar[ru]^{TL}\ar[rd]_{TI}\ar@{}[r]|-{\Uparrow T\eta\quad} & \;\ar@{}[r]|-{\Uparrow\beta} & \;\\
 & T\mathcal{C}\ar[r]_{z} & \mathcal{C} &  &  & T\mathcal{C}\ar[r]_{z} & \mathcal{C}
}
\]
as $z\cdot T\eta$ exhibits $z\cdot TR$ as a left extension. For
a verification that $\left(R,\beta\right)$ is a lax $T$-morphism
one may see \cite[Theorem 2.44]{markextension}.

We now check this extends to a $T$-left extension. Suppose we are
given a pseudo $T$-algebra $\left(\mathcal{D},w\right)$, lax $T$-morphism
$\left(M,{\varepsilon}\right)$ and oplax $T$-morphism $\left(N,\varphi\right)$
as below
\[
\xymatrix@=1em{ & \left(\mathcal{D},T\mathcal{D}\overset{w}{\rightarrow}\mathcal{D}\right)\ar[rd]^{\left(M,{\varepsilon}\right)}\\
\left(\mathcal{B},T\mathcal{B}\overset{y}{\rightarrow}\mathcal{B}\right)\ar[rr]_{\left(R,\beta\right)}\ar[ur]^{\left(N,\varphi\right)} & \;\ar@{}[u]|-{\Uparrow\overline{\zeta}} & \left(\mathcal{C},T\mathcal{C}\overset{z}{\rightarrow}\mathcal{C}\right)\ar@{}[ld]|-{\stackrel{\eta}{\Longleftarrow}}\\
 & \;\\
 &  & \left(\mathcal{A},T\mathcal{A}\overset{x}{\rightarrow}\mathcal{A}\right)\ar[uu]_{\left(I,\xi\right)}\ar[uull]^{\left(L,\alpha\right)}
}
\]
We need to check that the assignment of $T$-transformations $\overline{\zeta}\mapsto\zeta$
defined by composing with $\eta$ is bijective. Again, injectivity
is trivial as the the underlying data is the same. We need only check
surjectivity, meaning that for a given $T$-transformation $\zeta$
the corresponding $\overline{\zeta}$ is also a $T$-transformation.
Indeed given a $\zeta$ such that

\[
\xymatrix@=1em{ &  & T\mathcal{D}\ar[rr]^{w}\ar@{}[rd]|-{\Uparrow\varphi} &  & \mathcal{D}\ar[dddd]^{M} &  &  &  & T\mathcal{D}\ar[dddd]^{TM}\ar[rr]^{w} &  & \mathcal{D}\ar[dddd]^{M}\\
 & T\mathcal{B}\ar[rr]^{y}\ar[ru]^{TN}\ar@{}[rd]|-{\Uparrow\alpha} &  & \mathcal{B}\ar[ur]^{N} &  &  &  & T\mathcal{B}\ar[ur]^{N}\\
T\mathcal{A}\ar[rr]^{x}\ar[rdrd]_{TI}\ar[ru]^{TL} &  & \mathcal{A}\ar[rdrd]_{I}\ar[ur]^{L}\ar@{}[dd]|-{\Uparrow\sigma}\ar@{}[rr]|-{\Uparrow\zeta} &  & \; & = & T\mathcal{A}\ar[rdrd]_{I}\ar[ur]^{L}\ar@{}[rr]|-{\Uparrow T\zeta} &  & \;\ar@{}[rr]|-{\Uparrow{\varepsilon}} &  & \;\\
\\
 &  & T\mathcal{C}\ar[rr]_{z} &  & \mathcal{C} &  &  &  & T\mathcal{C}\ar[rr]_{z} &  & \mathcal{C}
}
\]
We can factor as 

\[
\xymatrix@=1em{ &  & T\mathcal{D}\ar[rr]^{w}\ar@{}[rd]|-{\Uparrow\varphi} &  & \mathcal{D}\ar[dddd]^{M} &  &  &  & T\mathcal{D}\ar[dddd]^{TM}\ar[rr]^{w} &  & \mathcal{D}\ar[dddd]^{M}\\
 & T\mathcal{B}\ar[rr]^{y}\ar[ru]^{TN}\ar@{}[rd]|-{\Uparrow\alpha} &  & \mathcal{B}\ar[ur]^{N}\ar[rddd]^{R}\ar@{}[rd]|-{\Uparrow\overline{\zeta}} &  &  &  & T\mathcal{B}\ar[ur]^{N}\ar[rddd]^{TR}\ar@{}[rd]|-{\Uparrow T\overline{\zeta}}\\
T\mathcal{A}\ar[rr]^{x}\ar[rdrd]_{TI}\ar[ru]^{TL} &  & \mathcal{A}\ar[rdrd]_{I}\ar[ur]^{L}\ar@{}[dd]|-{\Uparrow\sigma}\ar@{}[rr]|-{\Uparrow\eta} &  & \; & = & T\mathcal{A}\ar[rdrd]_{I}\ar[ur]^{L}\ar@{}[rr]|-{\Uparrow T\eta\quad} &  & \;\ar@{}[rr]|-{\Uparrow{\varepsilon}} &  & \;\\
\\
 &  & T\mathcal{C}\ar[rr]_{z} &  & \mathcal{C} &  &  &  & T\mathcal{C}\ar[rr]_{z} &  & \mathcal{C}
}
\]
which by definition of $\beta$ is 

\[
\xymatrix@=1em{ &  & T\mathcal{D}\ar[rr]^{w}\ar@{}[rd]|-{\Uparrow\varphi} &  & \mathcal{D}\ar[dddd]^{M} &  &  &  & T\mathcal{D}\ar[dddd]^{TM}\ar[rr]^{w} &  & \mathcal{D}\ar[dddd]^{M}\\
 & T\mathcal{B}\ar[rr]^{y}\ar[ru]^{TN}\ar[rddd]^{TR}\ar@{}[rdddrr]|-{\Uparrow\beta} &  & \mathcal{B}\ar[ur]^{N}\ar[rddd]^{R}\ar@{}[rd]|-{\Uparrow\overline{\zeta}} &  &  &  & T\mathcal{B}\ar[ur]^{N}\ar[rddd]^{TR}\ar@{}[rd]|-{\Uparrow T\overline{\zeta}}\\
T\mathcal{A}\ar[rdrd]_{TI}\ar[ru]^{TL}\ar@{}[rdr]|-{\Uparrow T\eta} &  &  &  & \; & = & T\mathcal{A}\ar[rdrd]_{I}\ar[ur]^{L}\ar@{}[rr]|-{\Uparrow T\eta\quad} &  & \;\ar@{}[rr]|-{\Uparrow{\varepsilon}} &  & \;\\
 &  & \;\\
 &  & T\mathcal{C}\ar[rr]_{z} &  & \mathcal{C} &  &  &  & T\mathcal{C}\ar[rr]_{z} &  & \mathcal{C}
}
\]
which since $z\cdot T\eta$ exhibits $z\cdot TR$ as a left extension
gives 

\[
\xymatrix@=1em{ &  & T\mathcal{D}\ar[rr]^{w}\ar@{}[dd]|-{\Uparrow\varphi} &  & \mathcal{D}\ar[dddd]^{M} &  &  &  & T\mathcal{D}\ar[dddd]^{TM}\ar[rr]^{w} &  & \mathcal{D}\ar[dddd]^{M}\\
\\
T\mathcal{B}\ar[rr]^{y}\ar[rdrd]_{TR}\ar[urur]^{TN} &  & \mathcal{B}\ar[rdrd]^{R}\ar[ruru]^{N}\ar@{}[rr]|-{\Uparrow\overline{\zeta}}\ar@{}[dd]|-{\Uparrow\beta} &  & \; & = & T\mathcal{B}\ar[rdrd]_{TR}\ar[ruru]^{TN}\ar@{}[rr]|-{\Uparrow T\overline{\zeta}} &  & \;\ar@{}[rr]|-{\Uparrow{\varepsilon}} &  & \;\\
\\
 &  & T\mathcal{C}\ar[rr]_{z} &  & \mathcal{C} &  &  &  & T\mathcal{C}\ar[rr]_{z} &  & \mathcal{C}
}
\]
so that $\overline{\zeta}$ is indeed a $T$-transformation.
\end{proof}

\subsection{Doctrinal ``Yoneda Structures''\label{doctrinalyonedastructures}}

Kelly \cite{doctrinal} showed that given an adjunction $L\dashv R$
which lifts to pseudo algebras, oplax structures on the left adjoint
are in bijection with lax structures on the right adjoint. The goal
of this section is to give a similar result for diagrams such as (3.1)
(as appear in Yoneda structures \cite{yonedastructures}, or in the
setting of a locally fully faithful KZ doctrine \cite{yonedakz}). 

We state the following as one of the main results of this paper, due
to its applications as a coherence result for the bicategories of
spans and polynomials which will be explained later in this section.
\begin{thm}
Suppose we are given a pseudomonad $\left(T,u,m\right)$ on a 2-category
$\mathscr{C}$ along with a diagram in $\mathscr{C}$ of the form
\[
\xymatrix@=1em{\mathcal{B}\ar[rr]^{R} &  & \mathcal{C}\ar@{}[ld]|-{\stackrel{\eta}{\Longleftarrow}}\\
 & \;\\
 &  & \mathcal{A}\ar[uu]_{I}\ar[uull]^{L}
}
\]
for which the 2-cell $\eta$ exhibits both $R$ as a left extension
and $L$ as an absolute left lifting. Suppose we are given pseudo
$T$-algebra structures
\[
\left(\mathcal{A},T\mathcal{A}\stackrel{x}{\longrightarrow}\mathcal{A}\right),\qquad\left(\mathcal{B},T\mathcal{B}\stackrel{y}{\longrightarrow}\mathcal{B}\right),\qquad\left(\mathcal{C},T\mathcal{C}\stackrel{z}{\longrightarrow}\mathcal{C}\right)
\]
and that the left extension $\left(R,\eta\right)$ is $T$-preserved
by $z$ . Suppose further that there exists a 2-cell 
\[
\xymatrix@=1em{T\mathcal{C}\ar[rr]^{z} &  & \mathcal{C}\\
 & \ar@{}[]|-{\Uparrow\xi}\\
T\mathcal{A}\ar[rr]_{x}\ar[uu]^{TI} &  & \mathcal{A}\ar[uu]_{I}
}
\]
exhibiting $I$ as a pseudo $T$-morphism. Then 2-cells $\alpha$
as on the left exhibiting $L$ as an oplax $T$-morphism
\[
\xymatrix@=1em{T\mathcal{B}\ar[rr]^{y} &  & \mathcal{B} &  &  &  &  & T\mathcal{B}\ar[rr]^{y}\ar[dd]_{TR} &  & \mathcal{B}\ar[dd]^{R}\\
 & \ar@{}[]|-{\Uparrow\alpha} &  &  &  &  &  &  & \ar@{}[]|-{\Uparrow\beta}\\
T\mathcal{A}\ar[rr]_{x}\ar[uu]^{TL} &  & \mathcal{A}\ar[uu]_{L} &  &  &  &  & T\mathcal{C}\ar[rr]_{z} &  & \mathcal{C}
}
\]
are in bijection with 2-cells $\beta$ as on the right exhibiting
$R$ as a lax $T$-morphism.\end{thm}
\begin{proof}
We need only check that the propositions concerning partial adjunctions
and left extensions are inverse to each other. But this is just a
consequence of the fact that we can go between the defining equalities
for these propositions
\[
\xymatrix{T\mathcal{B}\ar[r]^{y}\;\ar@{}[rdd]|-{\Uparrow\alpha} & \mathcal{B}\ar[rd]^{R} &  &  & T\mathcal{B}\ar[r]^{y}\ar[dr]^{TR} & \mathcal{B}\ar[rd]^{R}\ar@{}[d]|-{\Uparrow\beta}\\
 & \;\ar@{}[r]|-{\Uparrow\eta} & \mathcal{C} & = & \;\ar@{}[r]|-{\Uparrow T\eta} & T\mathcal{C}\ar[r]^{z}\ar@{}[d]|-{\Uparrow\xi} & \mathcal{C}\\
T\mathcal{A}\ar[r]_{x}\ar[uu]^{TL} & \mathcal{A}\ar[uu]_{L}\ar[ur]_{I} &  &  & T\mathcal{A}\ar[r]_{x}\ar[uu]^{TL}\ar[ur]^{TI} & \mathcal{A}\ar[ur]_{I}
}
\]
and
\[
\xymatrix{ & T\mathcal{B}\ar[r]^{y} & \mathcal{B}\ar[dd]^{R} &  &  & T\mathcal{B}\ar[r]^{y}\ar[dd]_{TR} & \mathcal{B}\ar[dd]^{R}\\
T\mathcal{A}\ar[ru]^{TL}\ar[r]^{x}\ar[rd]_{TY} & \mathcal{A}\ar[dr]^{Y}\ar[ur]^{L}\ar@{}[r]|-{\Uparrow\eta}\ar@{}[u]|-{\Uparrow\alpha}\ar@{}[d]|-{\Uparrow\xi^{-1}} & \; & = & T\mathcal{A}\ar[ru]^{TL}\ar[rd]_{TY}\ar@{}[r]|-{\Uparrow T\eta\quad} & \;\ar@{}[r]|-{\Uparrow\beta} & \;\\
 & T\hat{\mathcal{A}}\ar[r]_{z} & \hat{\mathcal{A}} &  &  & T\hat{\mathcal{A}}\ar[r]_{z} & \hat{\mathcal{A}}
}
\]
by pasting with $\xi$ and $\xi^{-1}$.\end{proof}
\begin{example}
Suppose that we have a lifting of a locally fully faithful KZ doctrine
$P$ to the setting of pseudo-$T$-algebras for a given pseudomonad
$T$ (we will discuss in detail in the next section what it means
for a KZ doctrine to lift to pseudoalgebras). Then for every $P$-admissible
morphism $L$ with corresponding left extension $R_{L}$, we have
a bijection between extensions of $L$ to an oplax $T$-morphism and
extensions of $R_{L}$ to a lax $T$-morphism. Moreover, for each
oplax structure $\alpha$ on $L$ the 2-cell $\varphi_{L}$ as in
\[
\xymatrix@=1em{\left(\mathcal{B},T\mathcal{B}\overset{y}{\rightarrow}\mathcal{B}\right){\ar^-{{\left(R_{L},\beta\right)}}[{rr}]} &  & \left(P\mathcal{A},TP\mathcal{A}\overset{z_{x}}{\rightarrow}P\mathcal{A}\right)\ar@{}[ld]|-{\stackrel{\varphi_{L}}{\Longleftarrow}}\\
 & \;\\
 &  & \left(\mathcal{A},T\mathcal{A}\overset{x}{\rightarrow}\mathcal{A}\right)\ar[uu]_{\left(y_{\mathcal{A}},\xi_{x}\right)}\ar[uull]^{\left(L,\alpha\right)}
}
\]
is a $T$-transformation exhibiting $\left(L,\alpha\right)$ as a
$T$-partial left adjoint and $\left(R_{L},\beta\right)$ as a $T$-left
extension, provided $\alpha$ and $\beta$ are related via this bijection.
\end{example}
The motivating application of this result is not to give an analogous
result to doctrinal adjunction, but instead the observation that it
may be seen as a coherence result. In particular, consider the following
special case of this theorem concerning the bicategory of spans $\mathbf{Span}\left(\mathcal{E}\right)$.
The analogous result for the bicategory of polynomials with cartesian
2-cells $\mathbf{Poly}_{c}\left(\mathcal{E}\right)$ will be considered
in a future paper. 

For the following corollary, we recall that locally defined functors
are the morphisms of $\mathbf{CatGrph}$, and $\mathbf{CatGrph}$
gives rise to bicategories and oplax/lax functors via a suitable 2-monad
\cite{icons}.
\begin{cor}
Suppose we are given a small\footnote{Note that one may work in a larger universe to work around this condition.}
category with pullbacks $\mathcal{E}$ and a bicategory $\mathscr{C}$
with the same objects, as well as locally defined functors
\[
L_{X,Y}:\mathbf{Span}\left(\mathcal{E}\right)\left(X,Y\right)\to\mathscr{C}\left(X,Y\right)
\]
with corresponding left extensions $\left(R_{L}\right)_{X,Y}$ as
components in the diagram 
\[
\xymatrix@=1em{\mathscr{C}\ar@{->}[rr]^{R_{L}} &  & \hat{\mathbf{Span}}\left(\mathcal{E}\right)\ar@{}[ld]|-{\quad\stackrel{\varphi_{L}}{\Longleftarrow}}\\
 & \;\\
 &  & \mathbf{Span}\left(\mathcal{E}\right)\ar[uu]_{Y}\ar@{->}[uull]^{L}
}
\]
Then enrichments of the locally defined functor $L$ to an oplax functor
are in bijection with enrichments of the locally defined functor $R_{L}$
to a lax functor.
\end{cor}
To see why this is useful, recall that composition of spans is given
by taking the terminal diagram of the form
\[
\xymatrix{ &  & \bullet\ar@{-->}[dd]\ar@{-->}[ld]\ar@{-->}[rd]\ar@/_{2pc}/@{-->}[ldld]\ar@/^{2pc}/@{-->}[rdrd]\\
 & \bullet\ar[ld]_{f}\ar[rd]^{g} &  & \bullet\ar[ld]_{h}\ar[rd]^{k}\\
\bullet &  & \bullet &  & \bullet
}
\]
and so when evaluating the composite of two spans we may recover the
two morphisms of spans in the above diagram; that is, there is a relationship
between the way 2-cells are defined and how composition of 1-cells
is defined.

This relationship between composition and 2-cells is captured in Day's
convolution formula \cite{dayconvolution}, and causes the coend defining
the Day convolution to collapse to a more workable sum. In particular,
composition in $\hat{\mathbf{Span}}\left(\mathcal{E}\right)$ is given
by the convolution formula
\[
{\displaystyle GF\left(s^{*};t\right)=\sum_{T\stackrel{h}{\longrightarrow}Y}F\left(s^{*};h\right)G\left(h^{*};t\right)}
\]
where $s^{*};t$ is an arbitrary span from $X$ to $Z$ through $Y$,
and $F$ and $G$ are presheaves on $\mathbf{Span}\left(\mathcal{E}\right)\left(X,Y\right)$
and $\mathbf{Span}\left(\mathcal{E}\right)\left(Y,Z\right)$ respectively.
As a result, it is easier to show that a locally defined functor $L\colon\mathbf{Span}\left(\mathcal{E}\right)\to\mathscr{C}$
is oplax by instead showing that the corresponding $R:\mathscr{C}\to\hat{\mathbf{Span}}\left(\mathcal{E}\right)$
is lax. Indeed, the reader should notice here that the problem of
showing $L$ is oplax involves pullbacks, whereas the equivalent problem
of showing $R$ is lax does not (once this convolution formula has
been established).

To demonstrate that this in fact a coherence result, we observe the
following consequence of this convolution formula collapsing.
\begin{cor}
To show that a locally defined functor 
\[
L:\mathbf{Span}\left(\mathcal{E}\right)\to\mathscr{C}
\]
 is oplax, it suffices to define oplax constraints $\varphi_{\mathfrak{r},\mathfrak{s}}$
only for pairs $\mathfrak{t}$ and $\mathfrak{s}$ of the form
\[
\xymatrix{ & \bullet\ar[dl]_{r}\ar[dr]^{t} &  &  & \bullet\ar[dl]_{t}\ar[dr]^{s}\\
\bullet &  & \bullet & \bullet &  & \bullet
}
\]
and check naturality for these, and check associativity only on triples
\[
\xymatrix{ & \bullet\ar[dl]_{r}\ar[dr]^{t} &  &  & \bullet\ar[dl]_{t}\ar[dr]^{h} &  &  & \bullet\ar[dl]_{h}\ar[dr]^{s}\\
\bullet &  & \bullet & \bullet &  & \bullet & \bullet &  & \bullet
}
\]
as well as the usual identity conditions\footnote{The identity conditions could be reduced slightly but it is not worth it as these conditions are usually simple.}.
Then $\varphi$ uniquely extends to an oplax structure on $L$.\end{cor}
\begin{proof}
Note that this is enough to show that the corresponding locally defined
functor $\mathscr{C}\to\hat{\mathbf{Span}}\left(\mathcal{E}\right)$
is lax, which is equivalent to showing $\mathbf{Span}\left(\mathcal{E}\right)\to\mathscr{C}$
is oplax.
\end{proof}

\subsection{Applications}

A more involved application along the same lines deals not with the
bicategory of spans, but instead $\mathbf{Poly}_{c}\left(\mathcal{E}\right)$,
the bicategory of polynomials with cartesian 2-cells as studied by
Gambino, Kock and Weber \cite{weber,gambinokock}. We see that due
to the complicated nature of composition in $\mathbf{Poly}_{c}\left(\mathcal{E}\right)$,
showing that a locally defined functor $L:\mathbf{Poly}\left(\mathcal{E}\right)\to\mathscr{C}$
is oplax becomes a large calculation (especially for the associativity
coherence conditions); however if we instead show that $R_{L}:\mathscr{C}\to\hat{\mathbf{Poly}_{c}}\left(\mathcal{E}\right)$
is lax our work will be reduced significantly; in fact by this method
we can completely avoid coherences involving composition of distributivity
pullbacks.

In a soon forthcoming paper we will exploit this fact in more detail
to give a complete proof of the universal properties of polynomials
which avoids the majority of the coherence conditions.

\section{Pseudo-Distributive Laws over KZ Doctrines\label{liftingkzdoctrines}}

It was shown by Marmolejo that pseudo-distributive laws of a (co)KZ
doctrine over a KZ doctrine have a particularly simple form \cite[Definition 11.4]{marm1999}.
Here we show that the axiom concerning the (co)KZ doctrine may be
dropped, thus showing that pseudo-distributive laws of any pseudomonad
over a KZ doctrine have this simple form. Hence the problem of lifting
a cocompletion operation to the 2-category of pseudo algebras may
be easily understood. The other main observation we make here is that
if a KZ doctrine lifts to a pseudomonad on the 2-category of pseudo
algebras, then this pseudomonad is a KZ doctrine automatically. In
the proof we make use of this admissibility perspective; in fact,
the preservation of $P$-admissible maps by a pseudomonad $T$ where
$P$ is a KZ doctrine is crucial here. The proof of these results
is quite technical, though the results are summarized in Theorem \ref{liftkzequiv}.

\subsection{Notions of Pseudo-Distributive Laws}

Beck \cite{beckdist} defined a distributive law of a monad $\left(T,u,m\right)$
over another monad $\left(P,y,\mu\right)$ on a category $\mathcal{C}$
to be a natural transformation $\lambda\colon TP\to PT$ rendering
commutative the four diagrams
\[
\xymatrix@=1em{ & TP\ar[rr]^{\lambda} &  & PT &  &  &  & TP\ar[rr]^{\lambda} &  & PT\\
 &  & \;\ar@{}[ru]|-{=} &  &  &  &  &  & \;\ar@{}[ru]|-{=}\\
 &  &  & P\ar[uu]_{Pu}\ar[lluu]^{uP} &  &  &  &  &  & T\ar[uu]_{yT}\ar[lluu]^{Ty}\\
TTP\ar[dd]_{mP}\ar[rr]^{T\lambda}\;\ar@{}[rddrrr]|-{=} &  & TPT\ar[rr]^{\lambda T} &  & PTT\ar[dd]^{Pm} &  & TPP\ar@{}[rddrrr]|-{=}\ar[dd]_{T\mu}\ar[rr]^{\lambda P} &  & PTP\ar[rr]^{P\lambda} &  & PPT\ar[dd]^{\mu T}\\
\\
TP\ar[rrrr]_{\lambda} &  &  &  & PT &  & TP\ar[rrrr]_{\lambda} &  &  &  & PT
}
\]

A well known example on $\mathbf{Set}$ is the canonical distributive
law of the monad for monoids over the monad for abelian groups (whose
composite is the monad for rings). 

More generally, one may talk about a pseudo-distributive law of a
pseudomonad over another pseudomonad on a 2-category \cite{marm1999,kellycoherence,thesisTanaka,cheng2003}.
In this generalization the four conditions above are replaced by four
pieces of data (four invertible modifications) which are then required
to satisfy multiple coherence axioms, which we will omit here.
\begin{defn}
\label{dist} A \emph{pseudo-distributive law }of a pseudomonad $\left(T,u,m\right)$
over a pseudomonad $\left(P,y,\mu\right)$ on a 2-category $\mathscr{C}$
consists of a pseudonatural transformation $\lambda\colon TP\to PT$,
along with four invertible modifications $\omega_{1},\omega_{2},\omega_{3}$
and $\omega_{4}$ in place of the four equalities above. These four
modifications are subject to the nine\footnote{This may be reduced to eight axioms \cite{marm2008}.}
coherence axioms given in \cite[Section 4]{marm1999}. As a convention,
we choose the direction of these four modifications to be from right
to left in the above four diagrams. 
\end{defn}
In this section, as in the background, we differentiate between ``KZ
doctrine'' defined in terms of left extensions, and ``KZ pseudomonad''
defined algebraically. 

We now define a pseudo-distributive law over such a KZ pseudomonad,
though showing this data and these coherence conditions suffice will
take some work. 
\begin{defn}
\label{distkzpseudomonad} Suppose we are given a 2-category $\mathscr{C}$
equipped with a pseudomonad $\left(T,u,m\right)$ and a KZ pseudomonad
$\left(P,y,\mu\right)$. Then a \emph{pseudo-distributive law over
a KZ pseudomonad} $\lambda\colon TP\to PT$ consists of a pseudonatural
transformation $\lambda\colon TP\to PT$ along with three invertible
modifications\footnote{Note the direction of the modifications are different in \cite{marm1999}. We use here the direction in which they will naturally arise from left extension and admissiblilty properties. Our direction agrees with that of \cite[Section 4]{tholen}.}
\[
\xymatrix@=1em{TP\ar[rr]^{\lambda} &  & PT &  & TP\ar[rr]^{\lambda} &  & PT &  & TTP\ar[dd]_{mP}\ar[rr]^{T\lambda}\;\ar@{}[rddrrr]|-{\overset{\omega_{3}}{\Longleftarrow}} &  & TPT\ar[rr]^{\lambda T} &  & PTT\ar[dd]^{Pm}\\
 & \;\ar@{}[ru]|-{\overset{\omega_{1}}{\Longleftarrow}} &  &  &  & \;\ar@{}[ru]|-{\overset{\omega_{2}}{\Longleftarrow}}\\
 &  & P\ar[uu]_{Pu}\ar[lluu]^{uP} &  &  &  & T\ar[uu]_{yT}\ar[lluu]^{Ty} &  & TP\ar[rrrr]_{\lambda} &  &  &  & PT
}
\]
subject to the three coherence axioms: 
\[
\xymatrix@=1em{ &  &  &  &  &  &  &  & TP\ar[ddll]_{TyP}\ar[rr]^{\lambda}\ar[dd]_{yTP}\ar@{}[ddrr]|-{\stackrel{y_{\lambda}^{-1}}{\Longleftarrow}\qquad\;\;} &  & PT\ar@/^{1pc}/[dd]^{PyT}\ar@/_{1pc}/[dd]_{yPT}\ar@{}[dd]|-{\stackrel{\theta T}{\Longleftarrow}}\\
TP\ar@/^{1pc}/[dd]^{TPy}\ar@/_{1pc}/[dd]_{TyP}\ar@{}[dd]|-{\stackrel{T\theta}{\Longleftarrow}}\ar[rr]^{\lambda}\ar@{}[drdr]|-{\qquad\quad\stackrel{\lambda_{y}}{\Longleftarrow}} &  & PT\ar[dd]^{PTy}\ar[rrdd]^{PyT} &  & \overset{\textnormal{coh }1}{=} &  &  & \;\ar@{}[dr]|-{\overset{\omega_{2}P}{\Longleftarrow}}\\
 &  &  & \;\ar@{}[dl]|-{\overset{P\omega_{2}}{\Longleftarrow}} &  & \; & TPP\ar[rr]_{\lambda P} &  & PTP\ar[rr]_{P\lambda} &  & PPT\ar[rr]_{\mu T} &  & PT\\
TPP\ar[rr]_{\lambda P} &  & PTP\ar[rr]_{P\lambda} &  & PPT\ar[rr]_{\mu T} &  & PT
}
\]
\[
\xymatrix@=1em{ &  &  &  &  &  &  & TP\ar[rd]^{\lambda}\\
P\ar[rr]^{uP}\ar@{}[drdr]|-{\stackrel{u_{y}}{\Longleftarrow}} &  & TP\ar[rr]^{\lambda}\ar@{}[dr]|-{\overset{\omega_{2}}{\Longleftarrow}} &  & PT &  & P\ar[rr]_{Pu}\ar[ru]^{uP}\ar@{}[drdr]|-{\stackrel{y_{u}^{-1}}{\Longleftarrow}} & \;\ar@{}[u]|-{\;\Uparrow\omega_{1}} & PT\\
 &  &  & \; &  & \overset{\textnormal{coh }2}{=}\\
1\ar[rr]_{u}\ar[uu]^{y} &  & T\ar[uu]_{Ty}\ar[uurr]_{yT} &  &  &  & 1\ar[rr]_{u}\ar[uu]^{y} &  & T\ar[uu]_{yT}
}
\]
\[
\xymatrix@=1em{ &  &  &  &  &  &  &  &  & TP\ar[rddrr]^{\lambda}\\
\\
TTP\ar[rr]^{mP} &  & TP\ar@{}[dldl]|-{\stackrel{m_{y}}{\Longleftarrow}}\ar[rr]^{\lambda}\ar@{}[dr]|-{\overset{\omega_{2}}{\Longleftarrow}} &  & PT &  & TTP\ar[rruru]^{mP}\ar[rr]^{T\lambda} &  & TPT\ar[rr]^{\lambda T} & \;\ar@{}[uu]|-{\Uparrow\omega_{3}} & PTT\ar[rr]^{Pm}\ar@{}[dd]|-{\stackrel{y_{m}^{-1}}{\Longleftarrow}} &  & PT\\
 &  &  & \; &  & \overset{\textnormal{coh }3}{=} &  & \;\ar@{}[ur]|-{\overset{T\omega_{2}}{\Longleftarrow}} &  & \;\ar@{}[ul]|-{\overset{\omega_{2}T}{\Longleftarrow}}\\
TT\ar[rr]_{m}\ar[uu]^{T^{2}y} &  & T\ar[uu]_{Ty}\ar[uurr]_{yT} &  &  &  &  &  & TT\ar[rr]_{m}\ar[uull]^{T^{2}y}\ar[rruu]_{yT^{2}}\ar[uu]^{TyT} &  & T\ar[uurr]_{yT}
}
\]
\end{defn}
\begin{rem}
(1) We will see later that $\omega_{1}$ and $\omega_{3}$ are uniquely
determined by $\omega_{2}$, due to the last two axioms and left extension
properties. (2) Actually, even the naturality cells of $\lambda$
may be determined given $\omega_{2}$ and the first coherence axiom.
(3) The last two coherence axioms ensure that our KZ pseudomonad $P$
lifts to pseudo $T$-algebras, as opposed to only lifting to lax $T$-algebras. 
\end{rem}
We will need a notion of separately cocontinuous in the context of
KZ doctrines, and so we define the following. 
\begin{defn}
\label{defTadmccts} Suppose we are given a 2-category $\mathscr{C}$
equipped with a pseudomonad $\left(T,u,m\right)$ and a KZ doctrine
$\left(P,y\right)$. With $T$-preservation of left extensions as
in Definition \ref{Tpreserved}, we define a 1-cell $z\colon T\mathcal{X}\to\mathcal{C}$
where $\mathcal{X}$ and $\mathcal{C}$ are $P$-cocomplete objects
to be:
\begin{enumerate}
\item \emph{$T_{P}$-cocontinuous} when every left extension along a unit
component $y_{\mathcal{A}}\colon\mathcal{A}\to P\mathcal{A}$ into
$\mathcal{X}$ is $T$-preserved by $z$;
\item \emph{$T_{P}$-adm-cocontinuous} when every left extension along a
$P$-admissible map $L\colon\mathcal{A}\to\mathcal{B}$ into $\mathcal{X}$
is $T$-preserved by $z$;
\end{enumerate}
\end{defn}
\begin{rem}
We will see in the next section that these two notions are equivalent.
\end{rem}
We are now ready to give the definition of a pseudo-distributive law
over a KZ doctrine in terms of left extensions.
\begin{defn}
\label{distkzdoctrine} Suppose we are given a 2-category $\mathscr{C}$
equipped with a pseudomonad $\left(T,u,m\right)$ and a KZ doctrine
$\left(P,y\right)$. Then a \emph{pseudo-distributive law over a KZ
doctrine} $\lambda\colon TP\to PT$ consists of the following assertions:

(0) $T$ preserves $P$-admissible maps;

\noindent and for every $\mathcal{A}\in\mathscr{C}$,

(1) the exhibiting 2-cell $\omega_{2}^{\mathcal{A}}$ of the left
extension $\lambda_{\mathcal{A}}$\footnote{The left extension is unique up to coherent isomorphism, and exists since $Ty_{\mathcal{A}}$ is $P$-admissible.}
in
\[
\xymatrix@=1em{TP\mathcal{A}\ar[rr]^{\lambda_{\mathcal{A}}} &  & PT\mathcal{A}\ar@{}[dl]|-{\stackrel{\omega_{2}^{\mathcal{A}}}{\Longleftarrow}}\\
 & \;\\
 &  & T\mathcal{A}\ar[uu]_{y_{T\mathcal{A}}}\ar[uull]^{Ty_{\mathcal{A}}}
}
\]
is invertible\footnote{Equivalently one could ask $PTy_{\mathcal{A}}$  is fully faithful \cite[Prop. 23]{yonedakz}.};

(2) each $\lambda_{\mathcal{A}}$ is $T_{P}$-cocontinuous\footnote{Equivalently one could ask that $\lambda_{\mathcal{A}}$ is $T_{P}$-adm-cocontinuous. This is shown later.}; 

(3) the respective diagrams
\[
\xymatrix@=1em{P\mathcal{A}\ar[rr]^{u_{P\mathcal{A}}} &  & TP\mathcal{A}\ar@{}[dldl]|-{\stackrel{u_{y}^{\mathcal{A}}}{\Longleftarrow}}\ar[rr]^{\lambda_{\mathcal{A}}}\ar@{}[dr]|-{\stackrel{\omega_{2}^{\mathcal{A}}}{\Longleftarrow}} &  & PT\mathcal{A} &  & T^{2}P\mathcal{A}\ar[rr]^{m_{P\mathcal{A}}} &  & TP\mathcal{A}\ar@{}[dldl]|-{\stackrel{m_{y}^{\mathcal{A}}}{\Longleftarrow}}\ar[rr]^{\lambda_{\mathcal{A}}}\ar@{}[dr]|-{\stackrel{\omega_{2}^{\mathcal{A}}}{\Longleftarrow}} &  & PT\mathcal{A}\\
 &  &  & \; &  &  &  &  &  & \;\\
\mathcal{A}\ar[rr]_{u_{\mathcal{A}}}\ar[uu]^{y_{\mathcal{A}}} &  & T\mathcal{A}\ar[uu]_{Ty_{\mathcal{A}}}\ar[uurr]_{y_{T\mathcal{A}}} &  &  &  & T^{2}\mathcal{A}\ar[rr]_{m_{\mathcal{A}}}\ar[uu]^{T^{2}y_{\mathcal{A}}} &  & T\mathcal{A}\ar[uu]_{Ty_{\mathcal{A}}}\ar[uurr]_{y_{T\mathcal{A}}}
}
\]

exhibit both $\lambda_{\mathcal{A}}\cdot u_{P\mathcal{A}}$ and $\lambda_{\mathcal{A}}\cdot m_{P\mathcal{A}}$
as left extensions.\end{defn}
\begin{rem}
Note that a pseudo-distributive law as defined above is unique, as
it contains only assertions, and these assertions are invariant under
the choice of left left extension (unique up to coherent isomorphism).
\end{rem}

\subsection{The Main Theorem}

We are now ready to state the main result of this section (and this
paper), justifying our definitions above.
\begin{thm}
\label{liftkzequiv} Suppose we are given a 2-category $\mathscr{C}$
equipped with a pseudomonad $\left(T,u,m\right)$ and a KZ pseudomonad
$\left(P,y,\mu\right)$. Then the following are equivalent:

(a) $P$ lifts to a KZ doctrine $\widetilde{P}$ on $\text{ps-}T\text{-alg}$;

(b) $P$ lifts to a KZ pseudomonad $\widetilde{P}$ on $\text{ps-}T\text{-alg}$;

(c) $P$ lifts to a pseudomonad $\widetilde{P}$ on $\text{ps-}T\text{-alg}$;

(d) There exists a pseudo-distributive law over a KZ doctrine $\lambda\colon TP\to PT$;

(e) There exists a pseudo-distributive law over a KZ pseudomonad $\lambda\colon TP\to PT$;

(f) There exists a pseudo-distributive law $\lambda\colon TP\to PT$.

\noindent Moreover, if any of the above are true, then 

(g) $P$ lifts to a KZ doctrine $\widetilde{P}_{\textnormal{lax}}$
on $\text{ps-}T\text{-alg}_{\textnormal{lax}}$;

(h) $P$ lifts to a KZ doctrine $\widetilde{P}_{\textnormal{oplax}}$
on $\text{ps-}T\text{-alg}_{\textnormal{oplax}}$;

(i) if $y_{\mathcal{A}}$ is fully faithful for every $\mathcal{A}\in\mathscr{C}$,
then every $Ty_{\mathcal{A}}$ is fully faithful;

(j) the KZ structure cell $\theta\colon Py\to yP$ for $P$ is also
the KZ structure cell for $\widetilde{P}$.
\end{thm}
The proof of this theorem is lengthy, and so we will prove $\left(e\right)\implies\left(d\right),\left(d\right)\implies\left(a\right)$
and $\left(g\right)$ in the subsequent subsections. Before moving
on to these subsections, we give the rest of the proof.
\begin{proof}
[Proof of Theorem \ref{liftkzequiv}] The following implications prove
the logical equivalence asserted in the statement of Theorem \ref{liftkzequiv}.

$\left(a\right)\implies\left(b\right)\colon$ That a KZ doctrine gives
rise to a pseudomonad is shown in \cite[Theorem 4.1]{marm2012}.

$\left(b\right)\implies\left(c\right)\colon$ This is trivial.

$\left(c\right)\implies\left(f\right)\colon$ For the correspondence
between pseudo-distributive laws and liftings to pseudo $T$-algebras
see \cite[Theorem 5.4]{cheng2003}.

$\left(f\right)\implies\left(e\right)\colon$ Given a pseudo-distributive
law $\lambda\colon TP\to PT$ where $P$ is a KZ pseudomonad, to check
this is a distributive law over a KZ pseudomonad in the sense of Definition
\ref{distkzpseudomonad} we need only check the first axiom. But this
axiom follows from coherences $7$ and $8$ as given in \cite[Section 4]{marm1999}
along with the second coherence axiom of a KZ pseudomonad as in Definition
\ref{defkzpseudomonad}.

$\left(e\right)\implies\left(d\right)\colon$ This is shown later
in Theorem \ref{eimpliesd}.

$\left(d\right)\implies\left(a\right),\left(g\right)\colon$ This
is shown later in Theorem \ref{dimpliesa}. 

$\left(h\right)\colon$ $P$ lifts to a KZ doctrine $\widetilde{P}_{\textnormal{oplax}}$
on $\text{ps-}T\text{-alg}_{\textnormal{oplax}}$ since given an oplax
structure cell $\varphi$ we get an oplax structure cell $\overline{\varphi}$
\[
\xymatrix@=1em{\left(P\mathcal{A},z_{x}\right)\ar[rr]^{\left(\overline{F},\overline{\varphi}\right)}\ar@{}[dr]|-{\stackrel{c_{F}}{\Longleftarrow}} &  & \left(P\mathcal{B},z_{r}\right)\\
 & \;\\
\left(\mathcal{A},x\right)\ar[uurr]_{\left(F,\varphi\right)}\ar[uu]^{\left(y_{\mathcal{A}},\xi_{x}\right)}
}
\]
given as unique the solution to 
\[
\xymatrix@=1em{TP\mathcal{B}\ar[rr]^{z_{r}}\ar@{}[ddrr]|-{\Uparrow\overline{\varphi}} &  & P\mathcal{B} &  &  &  & TP\mathcal{B}\ar[rr]^{z_{r}}\ar@{}[rrdddd]|-{\Uparrow\varphi} &  & P\mathcal{B}\\
\\
TP\mathcal{A}\ar[rr]^{z_{x}}\ar@{}[ddrr]|-{\Uparrow\xi_{x}}\ar[uu]^{T\overline{F}} &  & P\mathcal{A}\ar[uu]_{\overline{F}} & = & TP\mathcal{A}\ar[uurr]^{T\overline{F}} & \ar@{}[]|-{\quad\cong Tc_{F}} &  &  &  & \ar@{}[]|-{\cong c_{F}} & P\mathcal{A}\ar[lluu]_{\overline{F}}\\
 & \;\\
T\mathcal{A}\ar[uu]^{Ty_{\mathcal{A}}}\ar[rr]_{x} &  & \mathcal{A}\ar[uu]_{y_{\mathcal{A}}} &  &  &  & T\mathcal{A}\ar[uull]^{Ty_{\mathcal{A}}}\ar[rr]_{x}\ar[uuuu]_{TF} &  & \mathcal{A}\ar[urru]_{y_{\mathcal{A}}}\ar[uuuu]_{F}
}
\]
with the conditions for an being an oplax $T$-morphism following
from Proposition \ref{claim} (Part 3) proven later. Note that the
induced oplax structure cell for 
\[
\xymatrix@=1em{\left(P\mathcal{A},z_{x}\right)\ar[rr]^{\left(\overline{F},\overline{\varphi}\right)}\ar@{}[dr]|-{\stackrel{c_{F}}{\Longleftarrow}} &  & \left(P\mathcal{B},z_{r}\right)\ar[rr]^{\left(\overline{G},\overline{\tau}\right)} &  & \left(P\mathcal{C},z_{k}\right)\\
 & \;\\
\left(\mathcal{A},x\right)\ar[uurr]_{\left(F,\varphi\right)}\ar[uu]^{\left(y_{\mathcal{A}},\xi_{x}\right)}
}
\]
is still $\left(\overline{G},\overline{\tau}\right)\cdot\left(\overline{F},\overline{\varphi}\right)$.
To see that $\left(\overline{F},\overline{\varphi}\right)$ is a left
extension in the sense of transformations, suppose we are given a
transformation $\sigma\colon\left(F,\varphi\right)\to\left(H,\psi\right)\cdot\left(y_{\mathcal{A}},\xi_{x}\right)$,
then the induced cell $\overline{\sigma}\colon\overline{F}\to H$
is a transformation since

\[
\xymatrix@=1em{ & TP\mathcal{B}\ar[rr]^{z_{r}}\ar@{}[ddrr]|-{\Uparrow\overline{\varphi}} &  & P\mathcal{B} &  &  & TP\mathcal{B}\ar[rr]^{z_{r}}\ar@{}[dd]|-{\Uparrow\psi} &  & P\mathcal{B}\\
\;\ar@{}[r]|-{\overset{T\overline{\sigma}}{\Longleftarrow}} & \; &  &  &  &  &  & \;\ar@{}[r]|-{\overset{\overline{\sigma}}{\Longleftarrow}} & \;\\
 & TP\mathcal{A}\ar[rr]^{z_{x}}\ar@{}[ddrr]|-{\Uparrow\xi_{x}}\ar[uu]_{T\overline{F}}\ar@/^{2pc}/[uu]^{TH} &  & P\mathcal{A}\ar[uu]_{\overline{F}} & = &  & TP\mathcal{A}\ar[rr]^{z_{x}}\ar@{}[ddrr]|-{\Uparrow\xi_{x}}\ar@/^{2pc}/[uu]^{TH} &  & P\mathcal{A}\ar[uu]_{\overline{F}}\ar@/^{2pc}/[uu]^{H}\\
 &  & \; &  &  &  &  & \;\\
 & T\mathcal{A}\ar[uu]^{Ty_{\mathcal{A}}}\ar[rr]_{x} &  & \mathcal{A}\ar[uu]_{y_{\mathcal{A}}} &  &  & T\mathcal{A}\ar[uu]^{Ty_{\mathcal{A}}}\ar[rr]_{x} &  & \mathcal{A}\ar[uu]_{y_{\mathcal{A}}}
}
\]
as a consequence of $\sigma$ being a transformation. By Proposition
\ref{docleftext} density still works in the setting of oplax $T$-morphisms;
this being why we proved the general case of Proposition \ref{docleftext}
in terms of composites of lax and oplax morphisms.

$\left(i\right)\colon$Just note the naturality square
\[
\xymatrix@=1em{PT\mathcal{A}\ar[rr]^{PTy_{\mathcal{A}}}\ar@{}[drdr]|-{\cong y_{Ty}^{\mathcal{A}}} &  & PTP\mathcal{A}\\
\\
T\mathcal{A}\ar[rr]_{Ty_{\mathcal{A}}}\ar[uu]^{y_{T\mathcal{A}}} &  & TP\mathcal{A}\ar[uu]_{y_{TP\mathcal{A}}}
}
\]
and that $PTy_{\mathcal{A}}$ is the left adjoint of a reflection
(and thus fully faithful), or simply that we have the isomorphism
$\omega_{2}$.

$\left(j\right)\colon$ Just apply Proposition \ref{docleftext} to
the naturality square 
\[
\xymatrix@=1em{P\mathcal{A}\ar[rr]^{Py_{\mathcal{A}}}\ar@{}[rrdd]|-{\cong y_{y}^{\mathcal{A}}} &  & P^{2}\mathcal{A}\\
\\
\mathcal{A}\ar[rr]_{y_{\mathcal{A}}}\ar[uu]^{y_{\mathcal{A}}} &  & P\mathcal{A}\ar[uu]_{y_{P\mathcal{A}}}
}
\]
noting that each $y_{\mathcal{A}}$ extends to a pseudo $T$-morphism,
to find the KZ structure cell for $\widetilde{P}$.\end{proof}
\begin{rem}
In view of this theorem, we note the following:
\begin{enumerate}
\item Recall that a pseudo-distributive law over a KZ doctrine is unique;
in particular the components of $\lambda$ are unique up to coherent
isomorphism as they are left extensions. This tells us such a lifting
is essentially unique. This may be seen as a generalization of \cite[Theorem 7.4]{marm2012}.
\item The reader may note the correspondence between the last two axioms
of a pseudo-distributive law over a KZ pseudomonad, and the last axiom
of a pseudo-distributive law over a KZ doctrine. These make sure we
have a lifting to pseudoalgebras, not just lax algebras.
\item From the proof of this theorem (and similarly for the lax algebra
setting), one may see that the first coherence axiom of a pseudo-distributive
law over a KZ pseudomonad is equivalent to preservation of admissible
maps, in the presence of such a pseudonatural transformation $\lambda$
and invertible modification $\omega_{2}$.
\end{enumerate}
\end{rem}

\begin{cor}
When the conditions of Theorem \ref{liftkzequiv} are met, the lifted
pseudomonad arising from the pseudo-distributive law is automatically
KZ.\end{cor}
\begin{proof}
There exists an equivalence of pseudomonads between this lifted pseudomonad
and the KZ pseudomonad resulting from the lifted KZ doctrine in the
theorem since the pseudo-distributive law is essentially unique.
\end{proof}

\subsection{Distributive Laws over KZ Monads to those over KZ Doctrines}

We will devote this entire subsection to showing that a pseudo-distributive
law over a KZ pseudomonad, as in Definition \ref{distkzpseudomonad},
gives rise to a pseudo-distributive law over a KZ doctrine, as in
Definition \ref{distkzdoctrine}. This is $\left(e\right)\implies\left(d\right)$
of Theorem \ref{liftkzequiv}. As this is the most difficult implication
to show, we will break the proof up into a number of propositions
and lemmata, starting with the following. 
\begin{prop}
\label{lambdabeck} Suppose we are given a 2-category $\mathscr{C}$
equipped with a pseudomonad $\left(T,u,m\right)$ and a KZ doctrine
$\left(P,y\right)$. Further suppose that for each object $\mathcal{A}\in\mathscr{C}$,
$Ty_{\mathcal{A}}$ is \emph{$P$-}admissible, and the left extension\footnote{This left extension exists since $Ty_{\mathcal{A}}$ is $P$-admissible.}
which we denote $\lambda_{\mathcal{A}}$ in
\[
\xymatrix@=1em{TP\mathcal{A}\ar[rr]^{\lambda_{\mathcal{A}}} &  & PT\mathcal{A}\ar@{}[dl]|-{\stackrel{\omega_{2}^{\mathcal{A}}}{\Longleftarrow}}\\
 & \;\\
 &  & T\mathcal{A}\ar[uu]_{y_{T\mathcal{A}}}\ar[uull]^{Ty_{\mathcal{A}}}
}
\]
is exhibited by an isomorphism denoted $\omega_{2}^{\mathcal{A}}$.
Then for every \emph{$P$-}admissible 1-cell $L\colon\mathcal{A}\to\mathcal{B}$
such that $TL\colon T\mathcal{A}\to T\mathcal{B}$ is also \emph{$P$-}admissible,
the following square on the left commutes up to coherent isomorphism;
with $\textnormal{res}_{L}$ for each $P$-admissible $L$ defined
as the left extension on the right
\[
\xymatrix@=1em{TP\mathcal{B}\ar[rr]^{\lambda_{\mathcal{B}}}\ar[dd]_{T\textnormal{res}_{L}}\ar@{}[rdrd]|-{\cong} &  & PT\mathcal{B}\ar[dd]^{\textnormal{res}_{TL}} &  &  &  &  & P\mathcal{A} &  & P\mathcal{B}\ar[ll]_{\textnormal{res}_{L}}\\
 &  &  &  &  &  &  &  & \;\ar@{}[ur]|-{\stackrel{c_{R_{L}}}{\implies}}\\
TP\mathcal{A}\ar[rr]_{\lambda_{\mathcal{A}}} &  & PT\mathcal{A} &  &  &  &  & \mathcal{A}\ar[uu]^{y_{\mathcal{A}}}\ar[rr]_{L}\ar@{}[ur]|-{\stackrel{\varphi_{L}}{\implies}\quad} &  & \mathcal{B}\ar[uu]_{y_{\mathcal{B}}}\ar[ulul]|-{R_{L}}
}
\]
Here coherent means that that pasting with this isomorphism takes
us between the left and right diagrams: 
\[
\xymatrix@=1em{PT\mathcal{A} &  & TP\mathcal{A}\ar[ll]_{\lambda_{\mathcal{A}}}\ar@{}[dl]|-{\stackrel{}{\cong}\omega_{2}^{\mathcal{A}}} &  & TP\mathcal{B}\ar[ll]_{T\textnormal{res}_{L}} &  & PT\mathcal{A} &  & PT\mathcal{B}\ar[ll]_{\textnormal{res}_{TL}} &  & TP\mathcal{B}\ar[ll]_{\lambda_{\mathcal{B}}}\\
 & \; &  & \;\ar@{}[ur]|-{\quad\cong}\ar@{}[dl]|-{\stackrel{T\eta_{L}}{\implies}} &  &  &  & \;\ar@{}[r]|-{\stackrel{\eta_{TL}}{\implies}} & \; & \;\ar@{}[ur]|-{\quad\cong\omega_{2}^{\mathcal{B}}}\ar@{}[l]|-{\cong}\\
 &  & T\mathcal{A}\ar[uu]|-{Ty_{\mathcal{A}}}\ar[rr]_{TL}\ar[uull]^{y_{T\mathcal{A}}} &  & T\mathcal{B}\ar[uu]_{Ty_{\mathcal{B}}}\ar[uull]_{TR_{L}} &  &  &  & T\mathcal{A}\ar[rr]_{TL}\ar[uull]^{y_{T\mathcal{A}}} &  & T\mathcal{B}\ar[uu]_{Ty_{\mathcal{B}}}\ar[uull]|-{y_{T\mathcal{B}}}\ar[uullll]_{\widetilde{TR}}
}
\]
 Moreover, if the left diagram exhibits $R_{L}$ as a left extension
\[
\xymatrix@=1em{\mathcal{B}\ar[rr]^{R_{L}} &  & P\mathcal{A}\ar@{}[dl]|-{\stackrel{\varphi_{L}}{\Longleftarrow}} &  &  &  & T\mathcal{B}\ar[rr]^{TR_{L}} &  & TP\mathcal{A}\ar@{}[dl]|-{\stackrel{T\varphi_{L}}{\Longleftarrow}}\ar[rr]^{\lambda_{\mathcal{A}}}\ar@{}[rd]|-{\cong\omega_{2}^{\mathcal{A}}} &  & PT\mathcal{A}\\
 & \; &  &  &  &  &  & \; &  & \;\\
 &  & \mathcal{A}\ar[uull]^{L}\ar[uu]_{y_{\mathcal{A}}} &  &  &  &  &  & T\mathcal{A}\ar[uull]^{TL}\ar[uu]_{Ty_{\mathcal{A}}}\ar[uurr]_{y_{T\mathcal{A}}}
}
\]
then the right diagram exhibits $\lambda_{\mathcal{A}}\cdot TR_{L}$
as a left extension. \end{prop}
\begin{proof}
Firstly, we consider the diagram
\[
\xymatrix@=1em{PT\mathcal{A} &  & TP\mathcal{A}\ar[ll]_{\lambda_{\mathcal{A}}}\ar@{}[dl]|-{\stackrel{}{\cong}\omega_{2}^{\mathcal{A}}} &  & TP\mathcal{B}\ar[ll]_{T\textnormal{res}_{L}}\\
 & \; &  & \;\ar@{}[ur]|-{\quad\cong}\ar@{}[dl]|-{\stackrel{T\eta_{L}}{\implies}}\\
 &  & T\mathcal{A}\ar[uu]^{Ty_{\mathcal{A}}}\ar[rr]_{TL}\ar[uull]^{y_{T\mathcal{A}}} &  & T\mathcal{B}\ar[uu]_{Ty_{\mathcal{B}}}\ar[uull]_{TR}
}
\]
and note that $\lambda_{\mathcal{A}}\cdot T\textnormal{res}_{L}$
is a left extension since we have the bijection 

\settowidth{\rhs}{$H\cdot T\textnormal{lan}_{L}\cdot Ty_{\mathcal{A}}$} 
\settowidth{\lhs}{$\lambda_{\mathcal{A}}\cdot T\textnormal{res}_{L}$}
\begin{prooftree}
\Axiom$\makebox[\lhs][r]{$\lambda_{\mathcal{A}}\cdot T\textnormal{res}_{L}$} {\ \rightarrow\ } \makebox[\rhs][l]{$H$}$
\RightLabel{$\qquad \textnormal{mates}$} 
\UnaryInf$\makebox[\lhs][r]{$\lambda_{\mathcal{A}}$} {\ \rightarrow\ } \makebox[\rhs][l]{$H\cdot T\textnormal{lan}_{L}$}$ 
\RightLabel{$\qquad \text{since \ensuremath{\lambda_{\mathcal{A}}} is a left extension}$}
\UnaryInf$\makebox[\lhs][r]{$y_{T\mathcal{A}}$} {\ \rightarrow\ } \makebox[\rhs][l]{$H\cdot T\textnormal{lan}_{L}\cdot Ty_{\mathcal{A}}$}$ 
\RightLabel{$\qquad PL\cdot y_{\mathcal{A}}\cong y_{\mathcal{B}}\cdot L$}
\UnaryInf$\makebox[\lhs][r]{$y_{T\mathcal{A}}$} {\ \rightarrow\ } \makebox[\rhs][l]{$H\cdot Ty_{\mathcal{B}}\cdot TL$}$ 
\end{prooftree}and one may check this is the correct exhibiting morphism using \cite[Remark 16]{yonedakz}.
We may also consider the diagram
\[
\xymatrix@=1em{PT\mathcal{A} &  & PT\mathcal{B}\ar[ll]_{\textnormal{res}_{TL}} &  & TP\mathcal{B}\ar[ll]_{\lambda_{\mathcal{B}}}\\
 & \;\ar@{}[r]|-{\stackrel{\eta_{TL}}{\implies}} & \; & \;\ar@{}[ur]|-{\quad\cong\omega_{2}^{\mathcal{B}}}\ar@{}[l]|-{\cong}\\
 &  & T\mathcal{A}\ar[rr]_{TL}\ar[uull]^{y_{T\mathcal{A}}} &  & T\mathcal{B}\ar[uu]_{Ty_{\mathcal{B}}}\ar[uull]|-{y_{T\mathcal{B}}}\ar[uullll]_{\widetilde{TR}}
}
\]
where $\widetilde{TR}$ is defined as the left extension arising from
$TL$ being \emph{$P$-}admissible (that is $R_{TL})$. Now since
$Ty_{\mathcal{B}}$ is \emph{$P$-}admissible the extension $\lambda_{\mathcal{B}}$
is preserved by $\textnormal{res}_{TL}$. We then apply the pasting
lemma for left extensions (the dual of \cite[Prop. 1]{yonedastructures})
to see the outside diagram is a left extension. By uniqueness of left
extensions, our square then commutes up to the said coherent isomorphism.

Now, to show that
\[
\xymatrix@=1em{T\mathcal{B}\ar[rr]^{TR_{L}} &  & TP\mathcal{A}\ar@{}[dl]|-{\stackrel{T\varphi_{L}}{\Longleftarrow}}\ar[rr]^{\lambda_{\mathcal{A}}}\ar@{}[rd]|-{\stackrel{\omega_{2}^{\mathcal{A}}}{\Longleftarrow}} &  & PT\mathcal{A}\\
 & \; &  & \;\\
 &  & T\mathcal{A}\ar[uull]^{TL}\ar[uu]|-{Ty_{\mathcal{A}}}\ar[uurr]_{y_{T\mathcal{A}}}
}
\]
is a left extension, it suffices to show that $\lambda_{\mathcal{A}}\cdot TR_{L}\cong\widetilde{TR}$
with $\widetilde{TR}$ just defined above. This is the case since
all regions in the following diagram commute up to isomorphism
\[
\xymatrix@=1em{ &  & \ar@{}[d]|-{\cong} &  & TP\mathcal{A}\ar[drr]^{\lambda_{\mathcal{A}}}\ar@{}[d]|-{\cong}\\
T\mathcal{B}\ar@/_{1.5pc}/[rrrr]_{y_{T\mathcal{B}}}\ar[rr]^{Ty_{\mathcal{B}}}\ar@/_{3pc}/[rrrrrr]_{\widetilde{TR}}\ar@/^{1pc}/[rrrru]^{TR_{L}} &  & TP\mathcal{B}\ar[rr]^{\lambda_{\mathcal{B}}}\ar[urr]^{T\textnormal{res}_{L}} &  & PT\mathcal{B}\ar[rr]^{\textnormal{res}_{TL}} &  & PT\mathcal{A}\\
 &  & \ar@{}[u]|-{\cong\omega_{2}^{\mathcal{B}}} &  & \ar@{}[]|-{\cong}
}
\]
and it is easy to check $\varphi_{TL}$ pasted with this gives the
required pasting of $\omega_{2}^{\mathcal{A}}$ with $T\varphi_{L}$
using the way the isomorphism for the square is defined.\end{proof}
\begin{rem}
Note that the above proposition tells us something about the components
of $\lambda$ being separately cocontinuous, without any assumptions
on pseudonaturality of $\lambda$. This may seem unusual in view of
the following proposition, in which we show pseudonaturality of $\lambda$
is precisely equivalent to the $T_{P}$-cocontinuity of its components.\end{rem}
\begin{lem}
\label{lambdaccts} Suppose we are given a 2-category $\mathscr{C}$
equipped with a pseudomonad $\left(T,u,m\right)$ and a KZ doctrine
$\left(P,y\right)$. Further suppose that for each object $\mathcal{A}\in\mathscr{C}$,
$Ty_{\mathcal{A}}$ is \emph{$P$-}admissible and the left extension
which we call $\lambda_{\mathcal{A}}$ in
\[
\xymatrix@=1em{TP\mathcal{A}\ar[rr]^{\lambda_{\mathcal{A}}} &  & PT\mathcal{A}\ar@{}[dl]|-{\stackrel{\omega_{2}^{\mathcal{\mathcal{A}}}}{\Longleftarrow}}\\
 & \;\\
 &  & T\mathcal{A}\ar[uu]_{y_{T\mathcal{A}}}\ar[uull]^{Ty_{\mathcal{A}}}
}
\]
is exhibited by an isomorphism $\omega_{2}^{\mathcal{\mathcal{A}}}$.
Then the naturality squares for $\lambda$
\[
\xymatrix@=1em{TP\mathcal{B}\ar[rr]^{\lambda_{\mathcal{B}}} &  & PT\mathcal{B}\\
\\
TP\mathcal{A}\ar[rr]_{\lambda_{\mathcal{A}}}\ar[uu]^{TPL}\ar@{}[urur]|-{\Uparrow\lambda_{L}} &  & PT\mathcal{A}\ar[uu]_{PTL}
}
\]
commute up to coherent isomorphism, with coherent meaning
\[
\xymatrix@=1em{ &  & TP\mathcal{B}\ar[rrd]^{\lambda_{\mathcal{B}}}\\
TP\mathcal{A}\ar[rr]^{\lambda_{\mathcal{A}}}\ar[rru]^{TPL} &  & PT\mathcal{A}\ar@{}[dl]|-{\stackrel{\omega_{2}^{\mathcal{\mathcal{A}}}}{\Longleftarrow}}\ar[rr]^{PTL}\ar@{}[rdrd]|-{\quad\cong y_{TL}}\ar@{}[u]|-{\Uparrow\lambda_{L}} &  & PT\mathcal{B} &  & TP\mathcal{A}\ar[rr]^{TPL}\ar@{}[ddrr]|-{\cong Ty_{L}} &  & TP\mathcal{B}\ar[rr]^{\lambda_{\mathcal{B}}} &  & PT\mathcal{B}\\
 & \; &  &  &  & = &  & \; &  & \;\ar@{}[ul]|-{\cong\omega_{2}^{\mathcal{B}}}\\
 &  & T\mathcal{A}\ar[uu]_{y_{T\mathcal{A}}}\ar[uull]^{Ty_{\mathcal{A}}}\ar[rr]_{TL} &  & T\mathcal{B}\ar[uu]_{y_{T\mathcal{B}}} &  & T\mathcal{A}\ar[uu]^{Ty_{\mathcal{A}}}\ar[rr]_{TL} &  & T\mathcal{B}\ar[uurr]_{y_{T\mathcal{B}}}\ar[uu]^{Ty_{\mathcal{B}}}
}
\]
that is, the condition for $\omega_{2}$ to be a modification, if
and only if each $\lambda_{\mathcal{A}}$ is $T_{P}$-cocontinuous.\end{lem}
\begin{proof}
The following implications prove the logical equivalence.

$\left(\implies\right)\colon$ Suppose that the naturality squares
commute up to coherent isomorphism. Then given any extension as on
the left
\[
\xymatrix@=1em{P\mathcal{A}\ar[rr]^{PF}\ar@{}[rrdd]|-{\cong y_{F}} &  & P^{2}\mathcal{B}\ar[rr]^{\textnormal{res}_{y_{\mathcal{B}}}} &  & P\mathcal{B} &  & TP\mathcal{A}\ar[rr]^{TPF}\ar@{}[rrdd]|-{\cong Ty_{F}} &  & TP^{2}\mathcal{B}\ar[rr]^{T\textnormal{res}_{y_{\mathcal{B}}}} &  & TP\mathcal{B}\ar[rr]^{\lambda_{\mathcal{B}}} &  & PT\mathcal{B}\\
 &  &  & \;\ar@{}[ul]|-{\cong} &  &  &  &  &  & \;\ar@{}[ul]|-{\cong}\\
\mathcal{A}\ar[uu]^{y_{\mathcal{A}}}\ar[rr]_{F} &  & P\mathcal{B}\ar[uu]_{y_{P\mathcal{B}}}\ar[uurr]_{\textnormal{id}_{P\mathcal{B}}} &  &  &  & T\mathcal{A}\ar[uu]^{Ty_{\mathcal{A}}}\ar[rr]_{TF} &  & TP\mathcal{B}\ar[uu]_{Ty_{P\mathcal{B}}}\ar[uurr]_{\textnormal{id}_{TP\mathcal{B}}}
}
\]
we need to check that the right diagram is a left extension. To see
this we note that the pasting
\[
\xymatrix@=1em{ &  & PT\mathcal{A}\ar[rr]^{PTF}\ar@{}[dd]|-{\Uparrow\lambda_{F}^{-1}} &  & PTP\mathcal{B}\ar[ddrr]^{\textnormal{res}_{Ty_{\mathcal{B}}}}\ar@{}[dd]|-{\cong}\\
\\
TP\mathcal{A}\ar[rr]^{TPF}\ar@{}[rrdd]|-{\cong Ty_{F}}\ar[uurr]^{\lambda_{\mathcal{A}}} &  & TP^{2}\mathcal{B}\ar[rr]^{T\textnormal{res}_{y_{\mathcal{B}}}}\ar[uurr]^{\lambda_{P\mathcal{B}}} &  & TP\mathcal{B}\ar[rr]^{\lambda_{\mathcal{B}}} &  & PT\mathcal{B}\\
 &  &  & \;\ar@{}[ul]|-{\cong}\\
T\mathcal{A}\ar[uu]^{Ty_{\mathcal{A}}}\ar[rr]_{TF} &  & TP\mathcal{B}\ar[uu]_{Ty_{P\mathcal{B}}}\ar[uurr]_{\textnormal{id}_{TP\mathcal{B}}}
}
\]
is equal to the pasting
\[
\xymatrix@=1em{TP\mathcal{A}\ar[rr]^{\lambda_{\mathcal{A}}} &  & PT\mathcal{A}\ar[rr]^{PTF}\ar@{}[rdrd]|-{\cong y_{TF}} &  & PTP\mathcal{B}\ar[rr]^{\textnormal{res}_{Ty_{\mathcal{B}}}} &  & PT\mathcal{B}\\
 & \;\ar@{}[ur]|-{\cong\omega_{2}^{\mathcal{A}}} &  &  &  & \;\ar@{}[ul]|-{\cong}\\
 &  & T\mathcal{A}\ar[rr]_{TF}\ar[uu]^{y_{T\mathcal{A}}}\ar[uull]^{Ty_{\mathcal{A}}} &  & TP\mathcal{B}\ar[uu]_{y_{TP\mathcal{B}}}\ar[rruu]_{\lambda_{\mathcal{B}}}
}
\]
This is shown by first using the coherence condition on $\lambda_{F}^{-1}$,
and then using the coherence condition from Proposition \ref{lambdabeck}.
Note also this last diagram is a left extension since $Ty_{\mathcal{A}}$
is \emph{$P$-}admissible (using preservation of left extensions by
$P$-homomorphisms). 

$\left(\Longleftarrow\right)\colon$ Now $PTL\cdot\lambda_{\mathcal{A}}$
is an extension of $y_{T\mathcal{B}}\cdot TL$ along $Ty_{\mathcal{A}}$
since $Ty_{\mathcal{A}}$ is \emph{$P$-}admissible. Also $\lambda_{\mathcal{B}}\cdot TPL$
is such a left extension as $\lambda_{\mathcal{B}}$ is $T_{P}$-cocontinuous,
giving us an isomorphism of left extensions $\lambda_{F}$ coherent
as in the statement of the lemma.\end{proof}
\begin{rem}
A Beck condition is satisfied here; the square in Proposition \ref{lambdabeck}
is the mate of $\lambda_{L}$. Indeed, this may be seen by pasting
with $\lambda_{L}$ and checking it satisfies the defining coherence
condition of the isomorphism (making use of \cite[Remark 16]{yonedakz}
and the coherence condition on $\lambda_{L}$).\end{rem}
\begin{lem}
\label{w2isext} Suppose we are given a 2-category $\mathscr{C}$
equipped with a pseudomonad $\left(T,u,m\right)$ and a KZ pseudomonad
$\left(P,y,\mu\right)$. Suppose further that we are given a pseudo-distributive
law over a KZ pseudomonad $\lambda\colon TP\to PT$. Then for each
$\mathcal{A}\in\mathscr{C}$, $Ty_{\mathcal{A}}$ is \emph{$P$-}admissible,
exhibited by an adjunction
\[
PTy_{\mathcal{A}}\dashv\mu_{T\mathcal{A}}\cdot P\lambda_{\mathcal{A}}
\]
Moreover, the diagrams as on the left exhibit each $\lambda_{\mathcal{A}}$
as a left extension, 
\[
\xymatrix@=1em{TP\mathcal{A}\ar[rr]^{\lambda_{\mathcal{A}}} &  & PT\mathcal{A}\ar@{}[dl]|-{\stackrel{\omega_{2}^{\mathcal{\mathcal{A}}}}{\Longleftarrow}} &  &  &  & TPP\mathcal{A}\ar[rr]^{\lambda_{P\mathcal{A}}}\ar@{}[rddrrr]|-{\stackrel{\omega_{4}^{\mathcal{\mathcal{A}}}}{\Longleftarrow}}\ar[dd]_{T\mu_{\mathcal{A}}} &  & PTP\mathcal{A}\ar[rr]^{P\lambda_{\mathcal{A}}} &  & PPT\mathcal{A}\ar[dd]^{\mu_{T\mathcal{A}}}\\
 & \;\\
 &  & T\mathcal{A}\ar[uu]_{y_{T\mathcal{A}}}\ar[uull]^{Ty_{\mathcal{A}}} &  &  &  & TP\mathcal{A}\ar[rrrr]_{\lambda_{\mathcal{A}}} &  &  &  & PT\mathcal{A}
}
\]
and there exists canonical isomorphisms as on the right for each $\mathcal{A}$.\end{lem}
\begin{proof}
(1) We define $\textnormal{res}_{Ty_{\mathcal{A}}}$, also called
$\overline{\lambda_{\mathcal{A}}}$, to be $\mu_{T\mathcal{A}}\cdot P\lambda_{\mathcal{A}}$.
This is exhibited as the left extension
\[
\xymatrix@=1em{PTP\mathcal{A}\ar[rr]^{P\lambda_{\mathcal{A}}}\ar@{}[rdrd]|-{\quad\cong y_{\lambda}^{\mathcal{A}}} &  & P^{2}T\mathcal{A}\ar[rr]^{\mu_{T\mathcal{A}}}\ar@{}[rd]|-{\cong} &  & PT\mathcal{A}\\
 &  &  & \;\\
TP\mathcal{A}\ar[rr]_{\lambda_{\mathcal{A}}}\ar[uu]^{y_{TP\mathcal{A}}} &  & PT\mathcal{A}\ar[uu]_{y_{PT\mathcal{A}}}\ar[rruu]_{\textnormal{id}_{PT\mathcal{A}}}
}
\]
using that a KZ pseudomonad gives rise to a KZ doctrine \cite{marm2012}.
We define $\eta$ as the unique solution to
\[
\xymatrix@=1em{ &  &  &  &  &  & PT\mathcal{A}\ar[rr]^{PTy_{\mathcal{A}}}\ar@{}[rdrd]|-{\cong y_{Ty}^{\mathcal{A}}} &  & PTP\mathcal{A}\ar[rr]^{P\lambda_{\mathcal{A}}}\ar@{}[rdrd]|-{\quad\cong y_{\lambda}^{\mathcal{A}}} &  & P^{2}T\mathcal{A}\ar[rr]^{\mu_{T\mathcal{A}}}\ar@{}[rd]|-{\cong} &  & PT\mathcal{A}\\
 &  &  & PTP\mathcal{A}\ar[rd]^{\overline{\lambda_{\mathcal{A}}}} &  & = &  &  &  &  &  & \;\\
T\mathcal{A}\ar[rr]^{y_{T\mathcal{A}}} &  & PT\mathcal{A}\ar[rr]_{\textnormal{id}}\ar[ru]^{PTy_{\mathcal{A}}} & \;\ar@{}[u]|-{\Uparrow\eta} & PT\mathcal{A} &  & T\mathcal{A}\ar[rr]_{Ty_{\mathcal{A}}}\ar[uu]^{y_{T\mathcal{A}}}\ar@/_{2pc}/[rrrr]_{y_{T\mathcal{A}}} &  & TP\mathcal{A}\ar[rr]_{\lambda_{\mathcal{A}}}\ar[uu]_{y_{TP\mathcal{A}}}\ar@{}[d]|-{\cong\omega_{2}^{\mathcal{A}}} &  & PT\mathcal{A}\ar[uu]_{y_{PT\mathcal{A}}}\ar[rruu]_{\textnormal{id}}\\
 &  &  & \; &  &  &  &  & \;
}
\]
Note that the unit is then given by 

\[
\xymatrix@=1em{ &  &  &  & \ar@{}[lld]|-{\cong}\\
 &  & \;\ar@{}[d]|-{\cong P\omega_{2}^{\mathcal{A}}}\\
PT\mathcal{A}\ar[rr]^{PTy_{\mathcal{A}}}\ar@/^{2pc}/[rrrr]^{Py_{T\mathcal{A}}}\ar@/^{4pc}/[rrrrrr]^{\textnormal{id}_{PT\mathcal{A}}} &  & PTP\mathcal{A}\ar[rr]^{P\lambda_{\mathcal{A}}} &  & P^{2}T\mathcal{A}\ar[rr]^{\mu_{T\mathcal{A}}} &  & PT\mathcal{A}
}
\]
as $y\colon1\to P$ is a pseudonatural transformation. We define ${\varepsilon}$
as the unique 2-cell rendering the following pasting diagrams equal

\[
\xymatrix@=1em{ &  &  &  & \;\ar@{}[d]|-{\Uparrow{\varepsilon}} &  &  &  &  & \ar@{}[]|-{\cong\omega_{2}P^{\mathcal{A}}}\\
TPA\ar[rr]^{y_{TP\mathcal{A}}}\ar@/_{2pc}/[rrrr]_{\lambda_{\mathcal{A}}} &  & PTP\mathcal{A}\ar[rr]^{\overline{\lambda_{\mathcal{A}}}}\ar@/^{2pc}/[rrrr]^{\textnormal{id}_{PTP\mathcal{A}}} &  & PT\mathcal{A}\ar[rr]^{PTy_{\mathcal{A}}} &  & PTP\mathcal{A} & TP\mathcal{A}\ar@/^{1pc}/[rr]^{Ty_{P\mathcal{A}}}\ar@/_{1pc}/[rr]_{TPy_{\mathcal{A}}}\ar@{}[rr]|-{\Uparrow T\theta}\ar@/^{4pc}/[rrrr]^{y_{TP\mathcal{A}}}\ar@/_{1pc}/[rrdd]_{\lambda_{\mathcal{A}}} &  & TP^{2}\mathcal{A}\ar[rr]^{\lambda_{P\mathcal{A}}} &  & PTP\mathcal{A}\\
 &  & \;\ar@{}[u]|-{\cong}\\
 &  &  &  &  &  &  &  &  & PTA\ar@/_{1pc}/[rruu]_{PTy_{\mathcal{A}}}\ar@{}[uu]|-{\cong\lambda_{y}^{\mathcal{A}}}\\
\\
}
\]
One could also define ${\varepsilon}$ directly in terms of $\theta$,
but that makes the proof more complicated. We now check the triangle
identities:
\[
\xymatrix@=1em{PTy_{\mathcal{A}}\ar[ddrrr]_{\textnormal{id}_{PTy_{\mathcal{A}}}}{\ar^-{{PTy_{\mathcal{A}}\cdot\eta}}[{rrr}]} &  &  & PTy_{\mathcal{A}}\cdot\overline{\lambda_{\mathcal{A}}}\cdot PTy_{\mathcal{A}}\ar[dd]^{{\varepsilon}\cdot PTy_{\mathcal{A}}} &  & \overline{\lambda_{\mathcal{A}}}\ar[ddrrr]_{\textnormal{id}_{\overline{\lambda_{\mathcal{A}}}}}{\ar^-{{\eta\cdot\overline{\lambda_{\mathcal{A}}}}}[{rrr}]} &  &  & \overline{\lambda_{\mathcal{A}}}\cdot PTy_{\mathcal{A}}\cdot\overline{\lambda_{\mathcal{A}}}\ar[dd]^{\overline{\lambda_{\mathcal{A}}}\cdot{\varepsilon}}\\
\\
 &  &  & PTy_{\mathcal{A}} &  &  &  &  & \overline{\lambda_{\mathcal{A}}}
}
\]

The left triangle identity, which is equivalent to asking for equality
when whiskered by $y_{T\mathcal{A}}$, amounts to asking that the
pasting

\[
\xymatrix@=1em{ &  & PT\mathcal{A}\ar@/^{2pc}/[rrddrrdd]^{PTy_{\mathcal{A}}}\\
\\
 &  & TP\mathcal{A}\ar[rrdd]^{Ty_{P\mathcal{A}}}\ar@/^{1.5pc}/[rrrrdd]^{y_{TP\mathcal{A}}}\ar@{}[dddd]|-{\left(2\right)\cong Ty_{y}^{\mathcal{A}}}\ar@{}[uu]|-{\cong y_{Ty}^{\mathcal{A}}} &  & \;\ar@{}[dd]|-{\quad\quad\left(1\right)\cong\omega_{2}P^{\mathcal{A}}}\\
\\
T\mathcal{A}\ar[rruu]^{Ty_{\mathcal{A}}}\ar[rrdd]_{Ty_{\mathcal{A}}}\ar@/_{4pc}/[rrddrr]_{y_{T\mathcal{A}}}\ar@/^{2pc}/[uurruu]^{y_{T\mathcal{A}}} &  &  &  & TP^{2}\mathcal{A}\ar[rr]^{\lambda_{P\mathcal{A}}}\ar@{}[dd]|-{\left(4\right)\cong\lambda_{y}^{\mathcal{A}}} &  & PTP\mathcal{A}\\
\\
 &  & TP\mathcal{A}\ar[rruu]_{TPy_{\mathcal{A}}}\ar[rr]_{\lambda_{\mathcal{A}}}\ar@{}[d]|-{\left(3\right)\cong\omega_{2}^{\mathcal{A}}} &  & PT\mathcal{A}\ar[uurr]_{PTy_{\mathcal{A}}}\\
 &  & \;
}
\]
is the identity. This is because $\left(1\right)+\left(2\right)+\left(3\right)+\left(4\right)$
is naturality for $y$ since $\omega_{2}$ is a modification. The
right triangle identity, which is equivalent to asking for equality
when whiskered by $y_{TP\mathcal{A}}$, amounts to asking that the
pasting

\[
\xymatrix@=1em{ &  &  &  & PT\mathcal{A}\ar@/^{1pc}/[ddddrr]^{y_{PT\mathcal{A}}}\ar@/^{1.5pc}/[ddddrrrr]^{\textnormal{id}_{PT\mathcal{A}}}\\
\\
\\
 &  & \ar@{}[]|-{\cong\omega_{2}P^{\mathcal{A}}} &  &  &  & \ar@{}[r]|-{\cong} & \;\\
TP\mathcal{A}\ar@/^{1pc}/[rr]^{Ty_{P\mathcal{A}}}\ar@/_{1pc}/[rr]_{TPy_{\mathcal{A}}}\ar@{}[rr]|-{\Uparrow T\theta^{\mathcal{A}}}\ar@/^{4pc}/[rrrr]^{y_{TP\mathcal{A}}}\ar@/_{1pc}/[rrdd]_{\lambda_{\mathcal{A}}}\ar@/^{3pc}/[rrrruuuu]^{\lambda_{\mathcal{A}}} &  & TP^{2}\mathcal{A}\ar[rr]^{\lambda_{P\mathcal{A}}} &  & PTP\mathcal{A}\ar[rr]^{P\lambda_{\mathcal{A}}}\ar@{}[uuuu]|-{\cong y_{\lambda}^{\mathcal{A}}} &  & P^{2}T\mathcal{A}\ar[rr]^{\mu_{T\mathcal{A}}} &  & PT\mathcal{A}\\
 &  & \; &  & \;\ar@{}[rru]|-{\cong P\omega_{2}^{\mathcal{A}}}\\
 &  & PT\mathcal{A}\ar@/_{1pc}/[rruu]^{PTy_{\mathcal{A}}}\ar@{}[uu]|-{\cong\lambda_{y}^{\mathcal{A}}}\ar@/_{1pc}/[rruurr]_{Py_{T\mathcal{A}}}\ar@/_{2pc}/[rruurrrr]_{\textnormal{id}_{PT\mathcal{A}}} &  &  & \ar@{}[rruur]|-{\cong}
}
\]
is the identity. This is the first axiom for a pseudo-distributive
law over a KZ pseudomonad, though noting the second coherence axiom
of a KZ pseudomonad as in Definition \ref{defkzpseudomonad}.

(2) We must first check that each $\lambda_{\mathcal{A}}$ is a left
extension as in the distributive law axiom 
\[
\xymatrix@=1em{TP\mathcal{A}\ar[rr]^{\lambda_{\mathcal{A}}} &  & PT\mathcal{A}\ar@{}[dl]|-{\cong\omega_{2}^{\mathcal{A}}}\\
 & \;\\
 &  & T\mathcal{A}\ar[uu]_{y_{T\mathcal{A}}}\ar[uull]^{Ty_{\mathcal{A}}}
}
\]
As $Ty_{\mathcal{A}}$ is \emph{$P$-}admissible, we know by \cite[Remark 16]{yonedakz}
we may take the extension 
\[
\xymatrix@=1em{TP\mathcal{A}\ar[rr]^{y_{TP\mathcal{A}}} &  & PTP\mathcal{A}\ar[rr]^{\textnormal{res}_{Ty_{\mathcal{A}}}} &  & PT\mathcal{A}\\
\;\ar@{}[rr]|-{\quad\quad\stackrel{y_{Ty}^{\mathcal{A}}}{\Longleftarrow}} &  & PT\mathcal{A}\ar[u]^{PTy_{\mathcal{A}}}\ar@{}[rru]|-{\stackrel{\eta\cdot y_{T\mathcal{A}}}{\Longleftarrow}}\\
 &  &  &  & T\mathcal{A}\ar[uu]_{y_{T\mathcal{A}}}\ar@/^{2pc}/[ululll]^{Ty_{\mathcal{A}}}\ar[ull]^{y_{T\mathcal{A}}}
}
\]
where $\eta$ is the unit of $PL\dashv\textnormal{res}_{L}$ as just
defined. Substituting in the definition of $\eta$ and pasting the
isomorphisms as below tells us the pasting 
\[
\xymatrix@=1em{ &  & \;\ar@{}[rd]|-{\cong}\\
 & PT\mathcal{A}\ar[rr]^{y_{PT\mathcal{A}}}\ar@/^{3pc}/[rrrd]^{\textnormal{id}_{PT\mathcal{A}}}\ar@{}[rd]|-{\cong y_{\lambda}^{\mathcal{A}}} &  & P^{2}T\mathcal{A}\ar[rd]|-{\mu_{T\mathcal{A}}}\\
TP\mathcal{A}\ar[rr]^{y_{TP\mathcal{A}}}\ar[ru]^{\lambda_{\mathcal{A}}} &  & PTP\mathcal{A}\ar[ur]^{P\lambda_{\mathcal{A}}} & \;\ar@{}[l]|-{\cong P\omega_{2}^{\mathcal{A}}}\ar@{}[r]|-{\cong} & PT\mathcal{A}\\
 &  & \; & \;\\
 &  &  & PT\mathcal{A}\ar[uul]^{PTy_{\mathcal{A}}}\ar[rruul]_{\textnormal{id}}\ar[ruulu]_{Py_{T\mathcal{A}}} & \;\ar@{}[l]|-{=}\\
 &  & \;\ar@{}[uu]|-{\stackrel{y_{Ty}^{\mathcal{A}}}{\Longleftarrow}}\\
 &  &  &  & T\mathcal{A}\ar[uuuu]_{y_{T\mathcal{A}}}\ar@/^{2pc}/[uluuulll]^{Ty_{\mathcal{A}}}\ar[uul]^{y_{T\mathcal{A}}}
}
\]
gives $\lambda_{\mathcal{A}}$ as a left extension. Note that this
pasting is equal to $\omega_{2}$ as a consequence of $\omega_{2}$
being a modification as well as the pseudomonad axiom
\[
\xymatrix@=1em{ &  & P\mathcal{A}\ar[rrdd]^{y_{P\mathcal{A}}}\ar@/^{1pc}/[ddrrrr]^{\textnormal{id}_{P\mathcal{A}}}\\
 &  &  &  & \ar@{}[]|-{\cong}\\
\mathcal{A}\ar[rruu]^{y_{\mathcal{A}}}\ar[rrdd]_{y_{\mathcal{A}}} &  & \ar@{}[]|-{\cong y_{y}^{\mathcal{A}}} &  & P^{2}\mathcal{A}\ar[rr]^{\mu_{\mathcal{A}}} &  & P\mathcal{A} & = & \textnormal{id}_{y_{\mathcal{A}}}\\
 &  &  &  & \ar@{}[]|-{\cong}\\
 &  & P\mathcal{A}\ar[rruu]_{Py_{\mathcal{A}}}\ar@/_{1pc}/[uurrrr]_{\textnormal{id}_{P\mathcal{A}}}
}
\]

(3) We have the left extensions
\[
\xymatrix@=1em{TP^{2}\mathcal{A}\ar[rr]^{\lambda_{P\mathcal{A}}} &  & PTP\mathcal{A}\ar[rr]^{P\lambda_{\mathcal{A}}}\ar@{}[rddr]|-{\quad\cong y_{\lambda}^{\mathcal{A}}} &  & PPT\mathcal{A}\ar[rr]^{\mu_{T\mathcal{A}}} &  & PT\mathcal{A}\\
 & \;\ar@{}[ur]|-{\cong\omega_{2}P^{\mathcal{A}}} &  &  &  & \ar@{}[ul]|-{\cong}\\
 &  & TP\mathcal{A}\ar[uu]_{y_{TP\mathcal{A}}}\ar[uull]^{Ty_{P\mathcal{A}}}\ar[rr]_{\lambda_{\mathcal{A}}} &  & PT\mathcal{A}\ar[uu]_{y_{PT\mathcal{A}}}\ar[rruu]_{\textnormal{id}_{PT\mathcal{A}}}
}
\]
and
\[
\xymatrix@=1em{TP^{2}\mathcal{A}\ar[rr]^{T\mu_{\mathcal{A}}} &  & TP\mathcal{A}\ar[rr]^{\lambda_{\mathcal{A}}} &  & PT\mathcal{A}\\
 & \;\ar@{}[ul]|-{\Uparrow TP_{\textnormal{id}}}\\
TP\mathcal{A}\ar[rruu]_{T\textnormal{id}_{P\mathcal{A}}}\ar[uu]^{Ty_{P\mathcal{A}}}
}
\]
since $Ty_{P\mathcal{A}}$ is \emph{$P$-}admissible and $\lambda_{\mathcal{A}}$
is $T_{P}$-cocontinuous (by Lemma \ref{lambdaccts}) respectively,
giving us our isomorphism of left extensions $\omega_{4}^{\mathcal{A}}$.
Note that this means $\omega_{4}$ satisfies coherence axiom 7 of
\cite{marm1999}.\end{proof}
\begin{prop}
\label{preserveadm} Suppose we are given a 2-category $\mathscr{C}$
equipped with a pseudomonad $\left(T,u,m\right)$ and a KZ pseudomonad
$\left(P,y,\mu\right)$. Suppose further that we are given a pseudo-distributive
law over a KZ pseudomonad $\lambda\colon TP\to PT$. Then $T$ preserves
$P$-admissible maps.\end{prop}
\begin{proof}
Suppose we are given a $P$-admissible map $L\colon\mathcal{A}\to\mathcal{B}$.
We show that $TL\colon T\mathcal{A}\to T\mathcal{B}$ is $P$-admissible,
exhibited by an adjunction
\[
PTL\dashv\mu_{T\mathcal{A}}\cdot P\lambda_{\mathcal{A}}\cdot PT\textnormal{res}_{L}\cdot PTy_{\mathcal{B}}:=\mathbf{R}_{L}
\]
with this right adjoint exhibited as the left extension
\[
\xymatrix@=1em{PT\mathcal{B}\ar[rr]^{PTy_{\mathcal{B}}}\ar@{}[rdrd]|-{\cong y_{Ty}^{\mathcal{B}}\quad} &  & PTP\mathcal{B}\ar[rr]^{PT\textnormal{res}_{L}}\ar@{}[rdrd]|-{\cong y_{T\textnormal{res}_{L}}} &  & PTP\mathcal{A}\ar[rr]^{P\lambda_{\mathcal{A}}}\ar@{}[rdrd]|-{\quad\cong y_{\lambda}^{\mathcal{A}}} &  & P^{2}T\mathcal{A}\ar[rr]^{\mu_{T\mathcal{A}}}\ar@{}[rd]|-{\cong} &  & PT\mathcal{A}\\
 &  &  &  &  &  &  & \;\\
T\mathcal{B}\ar[uu]^{y_{T\mathcal{B}}}\ar[rr]_{Ty_{\mathcal{B}}} &  & TP\mathcal{B}\ar[uu]^{y_{TP\mathcal{B}}}\ar[rr]_{T\textnormal{res}_{L}} &  & TP\mathcal{A}\ar[rr]_{\lambda_{\mathcal{A}}}\ar[uu]_{y_{TP\mathcal{A}}} &  & PT\mathcal{A}\ar[uu]_{y_{PT\mathcal{A}}}\ar[rruu]_{\textnormal{id}_{PT\mathcal{A}}}
}
\]
Denote the unit and counit of $PL\dashv\textnormal{res}_{L}$ as $\eta$
and ${\varepsilon}$. We define $n$ as the unique solution to
\[
\xymatrix@=1em{ &  &  & PT\mathcal{B}\ar[rd]^{\mathbf{R}_{L}}\\
T\mathcal{A}\ar[rr]^{y_{T\mathcal{A}}} &  & PT\mathcal{A}\ar[rr]_{\textnormal{id}_{PT\mathcal{A}}}\ar[ru]^{PTL} & \;\ar@{}[u]|-{\Uparrow n} & PT\mathcal{A}
}
\]
being
\[
\xymatrix@=1em{PT\mathcal{A}\ar[rr]^{PTL}\ar@{}[rdrd]|-{\cong y_{TL}} &  & PT\mathcal{B}\ar[rr]^{PTy_{\mathcal{B}}}\ar@{}[rdrd]|-{\cong y_{Ty}^{\mathcal{B}}\quad} &  & PTP\mathcal{B}\ar[rr]^{PT\textnormal{res}_{L}}\ar@{}[rdrd]|-{\cong y_{T\textnormal{res}_{L}}} &  & PTP\mathcal{A}\ar[rr]^{P\lambda_{\mathcal{A}}}\ar@{}[rdrd]|-{\quad\cong y_{\lambda}^{\mathcal{A}}} &  & P^{2}T\mathcal{A}\ar[rr]^{\mu_{T\mathcal{A}}}\ar@{}[rd]|-{\cong} &  & PT\mathcal{A}\\
 &  &  &  &  &  &  &  &  & \;\\
T\mathcal{A}\ar[rr]_{TL}\ar[uu]^{y_{T\mathcal{A}}}\ar@/_{1pc}/[rrrrdd]_{Ty_{\mathcal{A}}}\ar@/_{6pc}/[rrrrrrrr]_{y_{T\mathcal{A}}} &  & T\mathcal{B}\ar[uu]^{y_{T\mathcal{B}}}\ar[rr]_{Ty_{\mathcal{B}}} &  & TP\mathcal{B}\ar[uu]^{y_{TP\mathcal{B}}}\ar[rr]_{T\textnormal{res}_{L}}\ar@{}[rdd]|-{\Uparrow T\eta} &  & TP\mathcal{A}\ar[rr]_{\lambda_{\mathcal{A}}}\ar[uu]_{y_{TP\mathcal{A}}} &  & PT\mathcal{A}\ar[uu]_{y_{PT\mathcal{A}}}\ar[rruu]_{\textnormal{id}_{PT\mathcal{A}}}\\
 &  & \ar@{}[r]|-{\cong Ty_{L}} & \;\\
 &  &  &  & TP\mathcal{A}\ar[uu]^{TPL}\ar@/_{1pc}/[rruu]_{\textnormal{id}_{TP\mathcal{A}}} & \;\\
 &  &  &  & \ar@{}[]|-{\cong\omega_{2}^{\mathcal{A}}}
}
\]

Note that the unit $n$ is then given by 
\[
\xymatrix@=1em{PT\mathcal{A}\ar[rr]^{PTL}\ar@/_{1pc}/[rrrrdd]^{PTy_{\mathcal{A}}}\ar@/_{5pc}/[rrrrrrrr]_{Py_{T\mathcal{A}}}\ar@/_{7pc}/[rrrrrrrrrr]_{\textnormal{id}_{PT\mathcal{A}}} &  & PT\mathcal{B}\ar[rr]^{PTy_{\mathcal{B}}} &  & PTP\mathcal{B}\ar[rr]^{PT\textnormal{res}_{L}}\ar@{}[rdd]|-{\Uparrow PT\eta} &  & PTP\mathcal{A}\ar[rr]^{P\lambda_{\mathcal{A}}} &  & P^{2}T\mathcal{A}\ar[rr]^{\mu_{T\mathcal{A}}} &  & PT\mathcal{A}\\
 &  &  & \ar@{}[ul]|-{\Uparrow PTy_{L}}\\
 &  &  &  & PTP\mathcal{A}\ar[uu]^{PTPL}\ar@/_{1pc}/[rruu]_{\textnormal{id}_{PTP\mathcal{A}}} & \;\\
 &  &  &  &  & \;\ar@{}[ul]|-{\cong P\omega_{2}^{\mathcal{A}}}\\
 &  &  &  &  & \; & \ar@{}[l]|-{\cong}
}
\]
as $y\colon1\to P$ is a pseudonatural transformation. We define $e$
as the unique solution to
\[
\xymatrix@=1em{ &  &  &  & \;\ar@{}[d]|-{\Uparrow e} &  &  &  &  &  & \ar@{}[d]|-{\cong\omega_{2}^{\mathcal{B}}}\\
T\mathcal{B}\ar[rr]^{y_{T\mathcal{B}}}\ar@/_{2pc}/[rrrr]_{\lambda_{\mathcal{A}}\cdot T\textnormal{res}_{L}\cdot Ty_{\mathcal{B}}} &  & PT\mathcal{B}\ar[rr]^{\mathbf{R}_{L}}\ar@/^{2pc}/[rrrr]^{\textnormal{id}_{PT\mathcal{B}}} &  & PT\mathcal{A}\ar[rr]^{PTL} &  & PT\mathcal{B} & = & T\mathcal{B}\ar@/_{0pc}/[dd]_{Ty_{\mathcal{B}}}\ar@/^{0pc}/[rr]^{Ty_{\mathcal{B}}}\ar@/^{2pc}/[rrrr]^{y_{T\mathcal{B}}} &  & TP\mathcal{B}\ar[rr]^{\lambda_{\mathcal{B}}} &  & PT\mathcal{B}\\
 &  & \;\ar@{}[u]|-{\cong} &  &  &  &  &  & \ar@{}[r]|-{\cong} & \ar@{}[rd]|-{\Uparrow T{\varepsilon}} &  & \ar@{}[]|-{\cong\lambda_{L}}\\
 &  &  &  &  &  &  &  & TP\mathcal{B}\ar[rruu]^{T\textnormal{id}_{P\mathcal{B}}}\ar@/_{0pc}/[rr]_{T\textnormal{res}_{L}} &  & TP\mathcal{A}\ar[uu]^{TPL}\ar[rr]_{\lambda_{\mathcal{A}}} &  & PT\mathcal{A}\ar[uu]_{PTL}\\
\\
}
\]

We now check the following triangle identities.
\[
\xymatrix@=1em{PTL\ar[ddrrr]_{\textnormal{id}_{PTL}}{\ar^-{{PTL\cdot n}}[{rrr}]} &  &  & PTL\cdot\mathbf{R}_{L}\cdot PTL\ar[dd]^{e\cdot PTL} &  & \mathbf{R}_{L}\ar[ddrrr]_{\textnormal{id}_{\mathbf{R}_{L}}}{\ar^-{{n\cdot\mathbf{R}_{L}}}[{rrr}]} &  &  & \mathbf{R}_{L}\cdot PTL\cdot\mathbf{R}_{L}\ar[dd]^{\mathbf{R}_{L}\cdot e}\\
\\
 &  &  & PTL &  &  &  &  & \mathbf{R}_{L}
}
\]

The left triangle identity (or equivalently the left triangle identity
whiskered by $y_{T\mathcal{A}}$) easily follows from the whiskered
definitions of $n$ and $e$ as well as the corresponding triangle
identity for $PL\dashv\textnormal{res}_{L}$, and $\omega_{2}$ being
a modification. The right triangle identity (or that whiskered by
$y_{T\mathcal{B}}$) is more complicated. This amounts to checking
that
\[
\xymatrix@=1em{ &  & \ar@{}[d]|-{\cong\omega_{2}^{\mathcal{B}}}\\
T\mathcal{B}\ar@/_{0pc}/[dd]_{Ty_{\mathcal{B}}}\ar@/^{0pc}/[rr]^{Ty_{\mathcal{B}}}\ar@/^{2pc}/[rrrr]^{y_{T\mathcal{B}}} &  & TP\mathcal{B}\ar[rr]^{\lambda_{\mathcal{B}}} &  & PT\mathcal{B}\ar[rr]^{PTy_{\mathcal{B}}}\ar@{}[rrdd]|-{\quad\quad\cong PTy_{L}} &  & PTP\mathcal{B}\ar[rr]^{PT\textnormal{res}_{L}} &  & PTP\mathcal{A}\ar[rr]^{P\lambda_{\mathcal{A}}} &  & P^{2}T\mathcal{A}\ar[rr]^{\mu_{T\mathcal{A}}} &  & PT\mathcal{A}\\
\ar@{}[r]|-{\cong} & \ar@{}[rd]|-{\Uparrow T{\varepsilon}} &  & \ar@{}[]|-{\cong\lambda_{L}} &  &  &  & \;\ar@{}[ul]|-{\Uparrow PT\eta}\\
TP\mathcal{B}\ar[rruu]^{T\textnormal{id}_{P\mathcal{B}}}\ar@/_{0pc}/[rr]_{T\textnormal{res}_{L}} &  & TP\mathcal{A}\ar[uu]^{TPL}\ar[rr]_{\lambda_{\mathcal{A}}} &  & PT\mathcal{A}\ar[uu]_{PTL}\ar[rr]_{PTy_{\mathcal{A}}}\ar@/_{4pc}/[rrrrrruu]_{Py_{T\mathcal{A}}}\ar@/_{6pc}/[rrrrrrrruu]_{\textnormal{id}_{PT\mathcal{A}}} &  & PTP\mathcal{A}\ar[uu]_{PTPL}\ar[uurr]_{\textnormal{id}_{PTP\mathcal{A}}} & \; & \ar@{}[l]|-{\cong P\omega_{2}^{\mathcal{A}}}\\
 &  &  &  &  &  &  &  &  & \ar@{}[]|-{\cong}
}
\]
is just the isomorphism $\mathbf{R}_{L}\cdot y_{T\mathcal{B}}\cong\lambda_{\mathcal{A}}\cdot T\textnormal{res}_{L}\cdot Ty_{\mathcal{B}}$.
The first step here is to reduce the problem to something that looks
more like the first axiom of a pseudo-distributive law over a KZ doctrine.
Upon using that $\omega_{2}$ is a modification and the second coherence
axiom of Definition \ref{defkzpseudomonad} the problem reduces to
showing that
\[
\xymatrix@=1em{T\mathcal{B}\ar@/_{0pc}/[dd]_{Ty_{\mathcal{B}}}\ar@/^{0pc}/[rr]^{Ty_{\mathcal{B}}} &  & TP\mathcal{B}\ar[rr]^{\lambda_{\mathcal{B}}} &  & PT\mathcal{B}\ar[rr]^{PTy_{\mathcal{B}}}\ar@{}[rrdd]|-{\quad\quad\cong PTy_{L}} &  & PTP\mathcal{B}\ar[rr]^{PT\textnormal{res}_{L}} &  & PTP\mathcal{A}\ar[rr]^{P\lambda_{\mathcal{A}}} &  & P^{2}T\mathcal{A}\ar[rr]^{\mu_{T\mathcal{A}}} &  & PT\mathcal{A}\\
\ar@{}[r]|-{\cong} & \ar@{}[rd]|-{\Uparrow T{\varepsilon}} &  & \ar@{}[]|-{\cong\lambda_{L}} &  &  &  & \;\ar@{}[ul]|-{\Uparrow PT\eta}\\
TP\mathcal{B}\ar[rruu]^{T\textnormal{id}_{P\mathcal{B}}}\ar@/_{0pc}/[rr]_{T\textnormal{res}_{L}} &  & TP\mathcal{A}\ar[uu]^{TPL}\ar[rr]_{\lambda_{\mathcal{A}}} &  & PT\mathcal{A}\ar[uu]_{PTL}\ar[rr]_{PTy_{\mathcal{A}}}\ar@/_{4pc}/[rrrrrruu]_{Py_{T\mathcal{A}}} &  & PTP\mathcal{A}\ar[uu]_{PTPL}\ar[uurr]_{\textnormal{id}_{PTP\mathcal{A}}} & \; & \ar@{}[l]|-{\cong P\omega_{2}^{\mathcal{A}}}\\
\\
}
\]
is
\[
\xymatrix@=1em{ & PT\mathcal{B}\ar[rr]^{PTy_{\mathcal{B}}}\ar@{}[rd]|-{\cong\lambda_{y}^{\mathcal{B}}} &  & PTP\mathcal{B}\ar[rr]^{PT\textnormal{res}_{L}}\ar@{}[rd]|-{\cong\lambda_{\textnormal{res}_{L}}} &  & PTP\mathcal{A}\ar[rr]^{P\lambda_{\mathcal{A}}}\ar@{}[dr]|-{\quad\cong y_{\lambda}^{\mathcal{A}}} &  & P^{2}T\mathcal{A}\ar[rr]^{\mu_{T\mathcal{A}}} &  & PT\mathcal{A}\\
TP\mathcal{B}\ar[rr]^{TPy_{\mathcal{B}}}\ar[ur]^{\lambda_{\mathcal{B}}} &  & TP^{2}\mathcal{B}\ar[rr]^{TP\textnormal{res}_{L}}\ar[ur]^{\lambda_{P\mathcal{B}}} &  & TP^{2}\mathcal{A}\ar@{}[r]|-{\cong\omega_{2}P^{\mathcal{A}}}\ar[ur]^{\lambda_{P\mathcal{A}}} & \; & \; & \;\ar@{}[]|-{\overset{\theta T}{\Longleftarrow}}\\
 & T\mathcal{B}\ar[rr]_{Ty_{\mathcal{B}}}\ar[ul]^{Ty_{\mathcal{B}}}\ar@{}[ru]|-{\cong Ty_{y}^{\mathcal{B}}} &  & TP\mathcal{B}\ar[rr]_{T\textnormal{res}_{L}}\ar[ul]^{Ty_{P\mathcal{B}}}\ar@{}[ru]|-{\cong Ty_{\textnormal{res}_{L}}} &  & TP\mathcal{A}\ar[rr]_{\lambda_{\mathcal{A}}}\ar[uu]_{y_{TP\mathcal{A}}}\ar[ul]^{Ty_{P\mathcal{A}}} &  & PT\mathcal{A}\ar@/^{1pc}/[uu]^{y_{PT\mathcal{A}}}\ar@/_{1pc}/[uu]_{Py_{T\mathcal{A}}}
}
\]
Now, using the first axiom of a pseudo-distributive law over a KZ
doctrine, and pasting $\lambda_{y}^{\mathcal{B}}$, $\lambda_{y}^{\mathcal{A}}$
and $\lambda_{\textnormal{res}_{L}}$ to the other side the problem,
and canceling $P\omega_{2}^{\mathcal{A}}$, this reduces to showing
that
\[
\xymatrix@=1em{ &  &  &  & TP^{2}\mathcal{B}\ar[rr]^{TP\textnormal{res}_{L}}\ar[rrdd]^{\lambda_{P\mathcal{B}}}\ar@{}[dd]|-{\cong\lambda_{y}^{\mathcal{B}}} &  & TP^{2}\mathcal{A}\ar[rrdd]^{\lambda_{P\mathcal{A}}}\ar@{}[dd]|-{\cong\lambda_{\textnormal{res}_{L}}}\\
\\
T\mathcal{B}\ar@/_{0pc}/[dd]_{Ty_{\mathcal{B}}}\ar@/^{0pc}/[rr]^{Ty_{\mathcal{B}}} &  & TP\mathcal{B}\ar[rr]^{\lambda_{\mathcal{B}}}\ar[uurr]^{TPy_{\mathcal{B}}} &  & PT\mathcal{B}\ar[rr]^{PTy_{\mathcal{B}}}\ar@{}[rrdd]|-{\quad\quad\cong PTy_{L}} &  & PTP\mathcal{B}\ar[rr]^{PT\textnormal{res}_{L}} &  & PTP\mathcal{A}\ar[rr]^{P\lambda_{\mathcal{A}}} &  & P^{2}T\mathcal{A}\ar[rr]^{\mu_{T\mathcal{A}}} &  & PT\mathcal{A}\\
\ar@{}[r]|-{\cong} & \ar@{}[rd]|-{\Uparrow T{\varepsilon}} &  & \ar@{}[]|-{\cong\lambda_{L}} &  &  &  & \;\ar@{}[ul]|-{\Uparrow PT\eta}\\
TP\mathcal{B}\ar[rruu]^{T\textnormal{id}_{P\mathcal{B}}}\ar@/_{0pc}/[rr]_{T\textnormal{res}_{L}} &  & TP\mathcal{A}\ar[uu]^{TPL}\ar[rr]_{\lambda_{\mathcal{A}}}\ar[rrdd]_{TPy_{\mathcal{A}}} &  & PT\mathcal{A}\ar[uu]_{PTL}\ar[rr]_{PTy_{\mathcal{A}}}\ar@{}[dd]|-{\cong\lambda_{y}^{\mathcal{A}}} &  & PTP\mathcal{A}\ar[uu]_{PTPL}\ar[uurr]_{\textnormal{id}_{PTP\mathcal{A}}} & \;\\
\\
 &  &  &  & TP^{2}\mathcal{A}\ar[rruu]_{\lambda_{P\mathcal{A}}}
}
\]
is
\[
\xymatrix@=1em{TP\mathcal{B}\ar[rr]^{TPy_{\mathcal{B}}} &  & TP^{2}\mathcal{B}\ar[rr]^{TP\textnormal{res}_{L}} &  & TP^{2}\mathcal{A}\ar[rr]^{\lambda_{P\mathcal{A}}} &  & PTP\mathcal{A}\ar[rr]^{P\lambda_{\mathcal{A}}} &  & P^{2}T\mathcal{A}\ar[rr]^{\mu_{T\mathcal{A}}} &  & PT\mathcal{A}\\
\; & \ar@{}[l]|-{\cong Ty_{y}^{\mathcal{A}}} & \; & \ar@{}[ld]|-{\cong Ty_{\textnormal{res}_{L}}} & \;\ar@{}[]|-{\overset{T\theta^{\mathcal{A}}}{\Longleftarrow}}\\
T\mathcal{B}\ar[rr]_{Ty_{\mathcal{B}}}\ar[uu]^{Ty_{\mathcal{B}}} &  & TP\mathcal{B}\ar[rr]_{T\textnormal{res}_{L}}\ar[uu]^{Ty_{P\mathcal{B}}} &  & TP\mathcal{A}\ar@/^{1pc}/[uu]^{Ty_{P\mathcal{A}}}\ar@/_{1pc}/[uu]_{TPy_{\mathcal{A}}}
}
\]
But as $\lambda$ is pseudonatural (and as we can then cancel $\lambda_{y}^{\mathcal{A}}$),
we can simplify the first to
\[
\xymatrix@=1em{T\mathcal{B}\ar@/_{0pc}/[dd]_{Ty_{\mathcal{B}}}\ar@/^{0pc}/[rr]^{Ty_{\mathcal{B}}} &  & TP\mathcal{B}\ar[rr]^{TPy_{\mathcal{B}}} &  & TP^{2}\mathcal{B}\ar[rr]^{TP\textnormal{res}_{L}} &  & TP^{2}\mathcal{A}\ar[rr]^{\lambda_{P\mathcal{A}}} &  & PTP\mathcal{A}\ar[rr]^{P\lambda_{\mathcal{A}}} &  & P^{2}T\mathcal{A}\ar[rr]^{\mu_{T\mathcal{A}}} &  & PT\mathcal{A}\\
\ar@{}[r]|-{\cong} & \ar@{}[rd]|-{\Uparrow T{\varepsilon}} &  & \ar@{}[]|-{\quad\quad\cong TPy_{L}} &  & \ar@{}[ul]|-{\Uparrow TP\eta}\\
TP\mathcal{B}\ar[rruu]^{T\textnormal{id}_{P\mathcal{B}}}\ar@/_{0pc}/[rr]_{T\textnormal{res}_{L}} &  & TP\mathcal{A}\ar[uu]_{TPL}\ar[rr]_{TPy_{\mathcal{A}}} &  & TP^{2}\mathcal{A}\ar[rruu]_{TP\textnormal{id}_{P\mathcal{A}}}\ar[uu]_{TP^{2}L}
}
\]
Since we have the isomorphism $\omega_{4}^{\mathcal{A}}$ as in Lemma
\ref{w2isext}, and as $T\theta^{\mathcal{B}}\cdot Ty_{\mathcal{B}}$
is invertible justifying pasting with $T\theta^{\mathcal{B}}$, we
may reduce the problem to showing that
\[
\xymatrix@=1em{T\mathcal{B}\ar@/_{0pc}/[dd]_{Ty_{\mathcal{B}}}\ar@/^{0pc}/[rr]^{Ty_{\mathcal{B}}} &  & TP\mathcal{B}\ar@/_{0.5pc}/[rr]_{TPy_{\mathcal{B}}}\ar@/^{0.5pc}/[rr]^{Ty_{P\mathcal{B}}} & \ar@{}[]|-{\Uparrow T\theta^{\mathcal{B}}} & TP^{2}\mathcal{B}\ar[rr]^{TP\textnormal{res}_{L}} &  & TP^{2}\mathcal{A}\ar[rr]^{T\mu_{\mathcal{A}}} &  & TP\mathcal{A}\ar[rr]^{\lambda_{\mathcal{A}}} &  & PT\mathcal{A}\\
\ar@{}[r]|-{\cong} & \ar@{}[rd]|-{\Uparrow T{\varepsilon}} &  & \ar@{}[dr]|-{\cong TPy_{L}} &  & \ar@{}[ul]|-{\Uparrow TP\eta}\\
TP\mathcal{B}\ar[rruu]^{T\textnormal{id}_{P\mathcal{B}}}\ar@/_{0pc}/[rr]_{T\textnormal{res}_{L}} &  & TP\mathcal{A}\ar[uu]_{TPL}\ar[rr]_{TPy_{\mathcal{A}}} &  & TP^{2}\mathcal{A}\ar[rruu]_{TP\textnormal{id}_{P\mathcal{A}}}\ar[uu]_{TP^{2}L}
}
\]
is
\[
\xymatrix@=1em{TP\mathcal{B}\ar@/_{0.5pc}/[rr]_{TPy_{\mathcal{B}}}\ar@/^{0.5pc}/[rr]^{Ty_{P\mathcal{B}}} & \ar@{}[]|-{\Uparrow T\theta^{\mathcal{B}}} & TP^{2}\mathcal{B}\ar[rr]^{TP\textnormal{res}_{L}} &  & TP^{2}\mathcal{A}\ar[rr]^{T\mu_{\mathcal{A}}} &  & TP\mathcal{A}\ar[rr]^{\lambda_{\mathcal{A}}} &  & PT\mathcal{A}\\
\; & \ar@{}[d]|-{\cong Ty_{y}^{\mathcal{B}}} & \; & \ar@{}[d]|-{\cong Ty_{\textnormal{res}_{L}}} & \;\ar@{}[]|-{\overset{T\theta^{\mathcal{A}}}{\Longleftarrow}}\\
T\mathcal{B}\ar[rr]_{Ty_{\mathcal{B}}}\ar[uu]^{Ty_{\mathcal{B}}} & \; & TP\mathcal{B}\ar[rr]_{T\textnormal{res}_{L}}\ar[uu]^{Ty_{P\mathcal{B}}} & \; & TP\mathcal{A}\ar@/^{1pc}/[uu]^{Ty_{P\mathcal{A}}}\ar@/_{1pc}/[uu]_{TPy_{\mathcal{A}}}
}
\]
Now using that $\theta\colon Py\to yP$ is a modification on the top,
and that $\theta^{\mathcal{B}}\cdot y_{y}^{\mathcal{B}}$ collapses
to the identity on the bottom and moving $Ty_{\textnormal{res}_{L}}$
across, the problem reduces to showing that 
\[
\xymatrix@=1em{ &  &  &  & TP\mathcal{A}\ar[ddrr]^{Ty_{P\mathcal{A}}}\\
 &  &  &  & \ar@{}[]|-{\cong Ty_{\textnormal{res}_{L}}}\\
T\mathcal{B}\ar@/_{0pc}/[dd]_{Ty_{\mathcal{B}}}\ar@/^{0pc}/[rr]^{Ty_{\mathcal{B}}} &  & TP\mathcal{B}\ar@/^{0pc}/[rr]^{Ty_{P\mathcal{B}}}\ar[uurr]^{T\textnormal{res}_{L}} & \; & TP^{2}\mathcal{B}\ar[rr]^{TP\textnormal{res}_{L}} &  & TP^{2}\mathcal{A}\ar[rr]^{T\mu_{\mathcal{A}}} &  & TP\mathcal{A}\ar[rr]^{\lambda_{\mathcal{A}}} &  & PT\mathcal{A}\\
\ar@{}[r]|-{\cong} & \ar@{}[rd]|-{\Uparrow T{\varepsilon}} &  & \ar@{}[u]|-{\cong Ty_{PL}} &  & \ar@{}[ul]|-{\Uparrow TP\eta}\\
TP\mathcal{B}\ar[rruu]^{T\textnormal{id}_{P\mathcal{B}}}\ar@/_{0pc}/[rr]_{T\textnormal{res}_{L}} &  & TP\mathcal{A}\ar[uu]_{TPL}\ar@/_{0.5pc}/[rr]_{TPy_{\mathcal{A}}}\ar@/^{0.5pc}/[rr]^{Ty_{P\mathcal{A}}} & \ar@{}[]|-{\Uparrow T\theta^{\mathcal{A}}} & TP^{2}\mathcal{A}\ar[rruu]_{TP\textnormal{id}_{P\mathcal{A}}}\ar[uu]_{TP^{2}L}
}
\]
is
\[
\xymatrix@=1em{T\mathcal{B}\ar[rr]^{Ty_{\mathcal{B}}} &  & TP\mathcal{B}\ar[rr]^{T\textnormal{res}_{L}} &  & TP\mathcal{A}\ar@/_{0.5pc}/[rr]_{TPy_{\mathcal{A}}}\ar@/^{0.5pc}/[rr]^{Ty_{P\mathcal{A}}} & \ar@{}[]|-{\Uparrow T\theta^{\mathcal{A}}} & TP^{2}\mathcal{A}\ar[rr]^{T\mu_{\mathcal{A}}} &  & TP\mathcal{A}\ar[rr]^{\lambda_{\mathcal{A}}} &  & PT\mathcal{A}}
\]
The next step is to cancel $T\theta^{\mathcal{A}}$ above, justified
as $T\mu_{\mathcal{A}}\cdot T\theta^{\mathcal{A}}$ is invertible,
and then to use pseudo naturality of $y\colon1\to P$. In particular
we paste with the equality $P\left(\textnormal{id}_{P\mathcal{A}}\right)\cdot y_{P\mathcal{A}}=y_{P\mathcal{A}}\cdot\textnormal{id}_{P\mathcal{A}}$
(we may assume without loss of generality that $P\left(\textnormal{id}_{\mathcal{X}}\right)=\textnormal{id}_{P\mathcal{X}}$)
to change the $TP\eta$ to a $T\eta$ and apply the corresponding
triangle identity for $PL\dashv\textnormal{res}_{L}$. This gives
the result.\end{proof}
\begin{rem}
Note that here, as well as in the preceding lemma, we only used $\omega_{2}$
being an invertible modification and the first axiom for a pseudo-distributive
law over a KZ doctrine along with pseudo naturality of $\lambda$.

We are now ready to prove the main result of this subsection.\end{rem}
\begin{thm}
\label{eimpliesd} In the statement of Theorem \ref{liftkzequiv}
(e) implies (d).\end{thm}
\begin{proof}
Firstly note by Proposition \ref{preserveadm} that $T$ preserves
\emph{$P$-}admissible maps. We then have the following.

(1) We know by Lemma \ref{w2isext} that each $\lambda_{\mathcal{A}}$
is a left extension exhibited by the distributive law data as in 
\[
\xymatrix@=1em{TP\mathcal{A}\ar[rr]^{\lambda_{\mathcal{A}}} &  & PT\mathcal{A}\ar@{}[dl]|-{\cong\omega_{2}^{\mathcal{A}}}\\
 & \;\\
 &  & T\mathcal{A}\ar[uu]_{y_{T\mathcal{A}}}\ar[uull]^{Ty_{\mathcal{A}}}
}
\]

(2) We note that each $\lambda_{\mathcal{A}}$ is $T_{P}$-cocontinuous
as a consequence of Lemma \ref{lambdaccts} and $\omega_{2}$ being
a modification.

(3) That the diagrams
\[
\xymatrix@=1em{P\mathcal{A}\ar[rr]^{u_{P\mathcal{A}}} &  & TP\mathcal{A}\ar@{}[dldl]|-{\cong u_{y}^{\mathcal{A}}}\ar[rr]^{\lambda_{\mathcal{A}}}\ar@{}[dr]|-{\cong\omega_{2}^{\mathcal{A}}} &  & PT\mathcal{A} &  & T^{2}P\mathcal{A}\ar[rr]^{m_{P\mathcal{A}}} &  & TP\mathcal{A}\ar@{}[dldl]|-{\cong m_{y}^{\mathcal{A}}}\ar[rr]^{\lambda_{\mathcal{A}}}\ar@{}[dr]|-{\cong\omega_{2}^{\mathcal{A}}} &  & PT\mathcal{A}\\
 &  &  & \; &  &  &  &  &  & \;\\
\mathcal{A}\ar[rr]_{u_{\mathcal{A}}}\ar[uu]^{y_{\mathcal{A}}} &  & T\mathcal{A}\ar[uu]_{Ty_{\mathcal{A}}}\ar[uurr]_{y_{T\mathcal{A}}} &  &  &  & T^{2}\mathcal{A}\ar[rr]_{m_{\mathcal{A}}}\ar[uu]^{T^{2}y_{\mathcal{A}}} &  & T\mathcal{A}\ar[uu]_{Ty_{\mathcal{A}}}\ar[uurr]_{y_{T\mathcal{A}}}
}
\]
exhibit both $\lambda_{\mathcal{A}}\cdot u_{P\mathcal{A}}$ and $\lambda_{\mathcal{A}}\cdot m_{P\mathcal{A}}$
as left extensions is due to the last two axioms for a pseudo-distributive
law over a KZ pseudomonad, and noting that in 
\[
\xymatrix@=1em{T^{2}P\mathcal{A}\ar[rr]^{T\lambda_{\mathcal{A}}} &  & TPT\mathcal{A}\ar@{}[dl]|-{\stackrel{T\omega_{2}^{\mathcal{A}}}{\Longleftarrow}}\ar[rr]^{\lambda_{T\mathcal{A}}}\ar@{}[dr]|-{\stackrel{\omega_{2}T^{\mathcal{A}}}{\Longleftarrow}} &  & PT^{2}\mathcal{A}\ar[rr]^{Pm_{\mathcal{A}}}\ar@{}[dd]|-{\cong y_{m}^{\mathcal{A}}} &  & PT\mathcal{A}\\
 & \; &  & \;\\
 &  & T^{2}\mathcal{A}\ar[uu]_{Ty_{T\mathcal{A}}}\ar[uull]^{T^{2}y_{\mathcal{A}}}\ar[uurr]_{y_{T^{2}\mathcal{A}}}\ar[rr]_{m_{\mathcal{A}}} &  & T\mathcal{A}\ar[uurr]_{y_{T\mathcal{A}}}
}
\]
the composite $Pm_{\mathcal{A}}\cdot\lambda_{T\mathcal{A}}\cdot T\lambda_{\mathcal{A}}$
is a left extension since each $\lambda_{\mathcal{A}}$ $T$-preserves
left extensions of unit components along admissible maps as in Proposition
\ref{lambdabeck} meaning we know that $\lambda_{T\mathcal{A}}\cdot T\lambda_{\mathcal{A}}$
is a left extension above, and so since $T^{2}y_{\mathcal{A}}$ is
\emph{$P$-}admissible by Proposition \ref{preserveadm} it then remains
a left extension when whiskered with $Pm_{\mathcal{A}}$ as required.
\end{proof}

\subsection{Lifting a KZ Doctrine to Algebras via a Distributive Law}

In this subsection we show that given a pseudo-distributive law over
a KZ doctrine $P$, we may lift $P$ to a KZ doctrine $\widetilde{P}$
on the 2-category of pseudo $P$-algebras. This is $\left(d\right)\implies\left(a\right)$
of Theorem \ref{liftkzequiv}. However, before we show this implication
we will first need to verify the following proposition.
\begin{prop}
\label{claim} Suppose we are given statement $\left(d\right)$ of
Theorem \ref{liftkzequiv}. It then follows that:

$\quad$(0) $T$ preserves $P$-admissible maps, 

\noindent and for every given pseudo $T$-algebra $\left(\mathcal{A},T\mathcal{A}\overset{x}{\rightarrow}\mathcal{A}\right)$, 

$\quad$(1) there exists a 1-cell $z_{x}$ given as the left extension
via an isomorphism $\xi_{x}$ 
\[
\xymatrix{TP\mathcal{A}\ar[r]^{z_{x}} & P\mathcal{A}\\
T\mathcal{A}\ar[r]_{x}\ar[u]^{Ty_{\mathcal{A}}}\ar@{}[ur]|-{\Uparrow\xi_{x}} & \mathcal{A}\ar[u]_{y_{\mathcal{A}}}
}
\]

which we call the Day convolution at $x$;

$\quad$(2) each $z_{x}$ is $T_{P}$-cocontinuous;

$\quad$(3) the respective diagrams
\[
\xymatrix@=1em{P\mathcal{A}\ar[rr]^{u_{P\mathcal{A}}} &  & TP\mathcal{A}\ar@{}[dldl]|-{\Uparrow u_{y}^{\mathcal{A}}}\ar[rr]^{z_{x}}\ar@{}[ddrr]|-{\Uparrow\xi_{x}} &  & P\mathcal{A} &  & T^{2}P\mathcal{A}\ar[rr]^{m_{P\mathcal{A}}} &  & TP\mathcal{A}\ar@{}[dldl]|-{\Uparrow m_{y}^{\mathcal{A}}}\ar[rr]^{z_{x}}\ar@{}[ddrr]|-{\Uparrow\xi_{x}} &  & P\mathcal{A}\\
 &  &  & \; &  &  &  &  &  & \;\\
\mathcal{A}\ar[rr]_{u_{\mathcal{A}}}\ar[uu]^{y_{\mathcal{A}}} &  & T\mathcal{A}\ar[uu]_{Ty_{\mathcal{A}}}\ar[rr]_{x} &  & \mathcal{A}\ar[uu]_{y_{\mathcal{A}}} &  & T^{2}\mathcal{A}\ar[rr]_{m_{\mathcal{A}}}\ar[uu]^{T^{2}y_{\mathcal{A}}} &  & T\mathcal{A}\ar[uu]_{Ty_{\mathcal{A}}}\ar[rr]_{x} &  & \mathcal{A}\ar[uu]_{y_{\mathcal{A}}}
}
\]

exhibit both $z_{x}\cdot u_{P\mathcal{A}}$ and $z_{x}\cdot m_{P\mathcal{A}}$
as left extensions. \end{prop}
\begin{proof}
(0) This property is straight from the definition. We include this
property here so that this proposition may be taken as one the equivalent
conditions of Theorem \ref{liftkzequiv}. We will remark about this
later in this subsection. (1) The left extension $\left(z_{x},\xi_{x}\right)$
is given by
\[
\xymatrix@=1em{TP\mathcal{A}\ar[rr]^{\lambda_{\mathcal{A}}} &  & PT\mathcal{A}\ar@{}[dl]|-{\stackrel{\omega_{2}^{\mathcal{A}}}{\Longleftarrow}}\ar[rr]^{Px}\ar@{}[drdr]|-{\stackrel{y_{x}^{-1}}{\Longleftarrow}} &  & P\mathcal{A}\\
 & \;\\
 &  & T\mathcal{A}\ar[uu]_{y_{T\mathcal{A}}}\ar[uull]^{Ty_{\mathcal{A}}}\ar[rr]_{x} &  & \mathcal{A}\ar[uu]_{y_{\mathcal{A}}}
}
\]
Here the left extension $\lambda_{\mathcal{A}}$ is preserved by $Px$
as $Ty_{\mathcal{A}}$ is \emph{$P$-}admissible. (2) Now suppose
we are given a pseudo $T$-algebra $\left(\mathcal{B},T\mathcal{B}\overset{r}{\rightarrow}\mathcal{B}\right)$
and a left extension as on the left
\[
\xymatrix@=1em{P\mathcal{A}\ar[rr]^{\overline{F}} &  & P\mathcal{B} &  &  &  & TP\mathcal{A}\ar[rr]^{T\overline{F}} &  & TP\mathcal{B}\ar[rr]^{z_{r}} &  & P\mathcal{B}\\
 & \ar@{}[ul]|-{\overset{c_{F}}{\Longleftarrow}} &  &  &  &  &  & \ar@{}[ul]|-{\overset{Tc_{F}}{\Longleftarrow}}\\
\mathcal{A}\ar[uu]^{y_{\mathcal{A}}}\ar[uurr]_{F} &  &  &  &  &  & T\mathcal{A}\ar[uu]^{Ty_{\mathcal{A}}}\ar[uurr]_{TF}
}
\]
To see that right diagram is also a left extension, note that $z_{r}:=Pr\cdot\lambda_{\mathcal{B}}$
and so the $T_{P}$-cocontinuity of $\lambda_{\mathcal{B}}$ may be
extended to that of $z_{r}$ as each $Ty_{\mathcal{A}}$ is \emph{$P$-}admissible.
(3) Again, since each $Ty_{\mathcal{A}}$ is \emph{$P$-}admissible,
the left extensions
\[
\xymatrix@=1em{P\mathcal{A}\ar[rr]^{u_{P\mathcal{A}}} &  & TP\mathcal{A}\ar@{}[dldl]|-{\overset{u_{y}^{\mathcal{A}}}{\Longleftarrow}}\ar[rr]^{\lambda_{\mathcal{A}}}\ar@{}[dr]|-{\overset{\omega_{2}^{\mathcal{A}}}{\Longleftarrow}} &  & PT\mathcal{A} &  & T^{2}P\mathcal{A}\ar[rr]^{m_{P\mathcal{A}}} &  & TP\mathcal{A}\ar@{}[dldl]|-{\overset{m_{y}^{\mathcal{A}}}{\Longleftarrow}}\ar[rr]^{\lambda_{\mathcal{A}}}\ar@{}[dr]|-{\overset{\omega_{2}^{\mathcal{A}}}{\Longleftarrow}} &  & PT\mathcal{A}\\
 &  &  & \; &  &  &  &  &  & \;\\
\mathcal{A}\ar[rr]_{u_{\mathcal{A}}}\ar[uu]^{y_{\mathcal{A}}} &  & T\mathcal{A}\ar[uu]_{Ty_{\mathcal{A}}}\ar[uurr]_{y_{T\mathcal{A}}} &  &  &  & T^{2}\mathcal{A}\ar[rr]_{m_{\mathcal{A}}}\ar[uu]^{T^{2}y_{\mathcal{A}}} &  & T\mathcal{A}\ar[uu]_{Ty_{\mathcal{A}}}\ar[uurr]_{y_{T\mathcal{A}}}
}
\]
are preserved upon composing with $Px$, and upon pasting with the
isomorphism $y_{x}$.\end{proof}
\begin{rem}
Note that from this proposition one may recover statement (e) of Theorem
\ref{liftkzequiv}. This is since given the data of this proposition,
one may recover a choice of each $\lambda_{\mathcal{A}}$ and its
exhibiting 2-cell $\omega_{2}^{\mathcal{A}}$ as a left extension,
by taking the pasting
\[
\xymatrix@=1em{TP\mathcal{A}\ar[rr]^{TPu_{\mathcal{A}}}\ar@{}[ddrr]|-{\Uparrow T\left(y_{u}^{\mathcal{A}}\right)^{-1}\quad} &  & TPT\mathcal{A}\ar[rr]^{z_{m_{\mathcal{A}}}}\ar@{}[ddrr]|-{\Uparrow\xi_{m_{\mathcal{A}}}} &  & PT\mathcal{A}\\
\\
T\mathcal{A}\ar[uu]^{Ty_{\mathcal{A}}}\ar[rr]_{Tu_{\mathcal{A}}} &  & T^{2}\mathcal{A}\ar[uu]^{Ty_{T\mathcal{A}}}\ar[rr]_{m_{\mathcal{A}}} &  & T\mathcal{A}\ar[uu]_{y_{T\mathcal{A}}}
}
\]

Indeed, pseudo-naturality of $\lambda\colon TP\to PT$ will then be
equivalent to the components of $\lambda$ being $T_{P}$-cocontinuous,
which is inherited from this condition on each $z_{m_{\mathcal{A}}}$.
From condition (3) of the proposition we recover the additional two
coherence conditions of such a pseudo-distributive law over a KZ pseudomonad.
The first coherence axiom for this distributive law may be recovered
from $T$ preserving $P$-admissible maps. Indeed, we recall that
since each $Ty_{\mathcal{A}}$ is $P$-admissible, $PTy_{\mathcal{A}}$
has a right adjoint (and in particular a right adjoint as constructed
as in Lemma \ref{w2isext}, since we may apply the proof of \cite[Prop. 15]{yonedakz}
to construct this adjoint), and one of the triangle identities for
this adjunction is equivalent to this coherence axiom.

It will then follow, after we have proven Theorem \ref{liftkzequiv},
that this proposition may be taken as one of the equivalent conditions
of this theorem.
\end{rem}
The following proposition will be useful in the proof that $\left(d\right)$
implies $\left(a\right)$.
\begin{prop}
\label{admcocontinuousequiv}Suppose we are given a 2-category $\mathscr{C}$
equipped with a pseudomonad $\left(T,u,m\right)$ and a KZ doctrine
$\left(P,y\right)$. Further suppose that we are given a pseudo-distributive
law over a KZ doctrine $\lambda\colon TP\to PT$. Then for any two
$P$-cocomplete objects $\mathcal{C}$ and $\mathcal{D}$, a 1-cell
$u\colon T\mathcal{C}\to\mathcal{D}$ is $T_{P}$-cocontinuous if
and only if it is $T_{P}$-adm-cocontinuous. \end{prop}
\begin{proof}
Supposing that $u$ is $T_{P}$-cocontinuous we check that $u$ is
necessarily $T_{P}$-adm-cocontinuous. To see this, we first note
that we have an induced isomorphism of left extensions as a consequence
of having the two left extensions
\[
\xymatrix@=1em{TP\mathcal{C}\ar[rr]^{\lambda_{\mathcal{C}}} &  & PT\mathcal{C}\ar[rr]^{Pu} &  & P\mathcal{D}\ar[rr]^{\left(y_{\mathcal{D}}\right)_{\ast}} &  & \mathcal{D} &  & TP\mathcal{C}\ar[rr]^{T\left(y_{\mathcal{C}}\right)_{\ast}} &  & T\mathcal{C}\ar[rr]^{u} &  & \mathcal{D}\\
 & \;\ar@{}[ul]|-{\stackrel{\omega_{2}^{\mathcal{C}}}{\Longleftarrow}} &  &  &  &  &  &  &  & \;\ar@{}[ul]|-{\stackrel{Tc_{\textnormal{id}_{\mathcal{C}}}}{\Longleftarrow}} & \ar@{}[]|-{\cong}\\
T\mathcal{C}\ar[uu]^{Ty_{\mathcal{C}}}\ar[rruu]_{y_{T\mathcal{C}}}\ar[rr]_{u} &  & \mathcal{D}\ar[rruu]_{y_{\mathcal{D}}}\ar@{}[uu]|-{\stackrel{y_{u}^{-1}}{\Longleftarrow}}\ar@/_{1pc}/[rurrur]_{\textnormal{id}_{\mathcal{D}}} &  & \ar@{}[uu]|-{\stackrel{c_{\textnormal{id}_{\mathcal{D}}}}{\Longleftarrow}} &  &  &  & T\mathcal{C}\ar[uu]^{Ty_{\mathcal{C}}}\ar[rruu]_{T\textnormal{id}_{\mathcal{C}}}\ar@/_{1pc}/[rurrur]_{u}
}
\]
We must check the left extension
\[
\xymatrix@=1em{\mathcal{B}\ar[rr]^{R_{L}} &  & P\mathcal{A}\ar@{}[dl]|-{\stackrel{\varphi_{L}}{\Longleftarrow}}\ar[rr]^{PH} &  & P\mathcal{C}\ar[rr]^{\left(y_{\mathcal{C}}\right)_{\ast}} &  & \mathcal{C}\\
 & \; &  & \; &  & \;\ar@{}[ul]|-{\stackrel{c_{\textnormal{id}_{\mathcal{C}}}}{\Longleftarrow}}\\
 &  & \mathcal{A}\ar[uull]^{L}\ar[uu]_{y_{\mathcal{A}}}\ar[rr]_{H}\ar@{}[uurr]|-{\stackrel{y_{H}^{-1}}{\Longleftarrow}} &  & \mathcal{C}\ar[uu]^{y_{\mathcal{C}}}\ar[rruu]_{\textnormal{id}_{\mathcal{C}}}
}
\]
is $T$-preserved by $u$. Indeed on applying $T$ and whiskering
by $u$, and then pasting with this isomorphism of left extensions
and a naturality isomorphism of $\lambda$ (which we have by Lemma
\ref{lambdaccts}) we obtain 
\[
\xymatrix@=1em{ &  & PT\mathcal{A}\ar[rr]^{PTH} &  & PT\mathcal{C}\ar[rr]^{Pu} &  & P\mathcal{D}\ar[rrdd]^{\left(y_{\mathcal{D}}\right)_{\ast}} & \;\\
\\
T\mathcal{B}\ar[rr]^{TR_{L}} &  & TP\mathcal{A}\ar@{}[dl]|-{\stackrel{T\varphi_{L}}{\Longleftarrow}}\ar[rr]^{TPH}\ar[uu]^{\lambda_{\mathcal{A}}}\ar@{}[uurr]|-{\stackrel{\lambda_{H}^{-1}}{\Longleftarrow}} &  & TP\mathcal{C}\ar[rr]^{T\left(y_{\mathcal{C}}\right)_{\ast}}\ar[uu]_{\lambda_{\mathcal{C}}}\ar@{}[uurrr]|-{\cong} &  & T\mathcal{C}\ar[rr]^{u} &  & \mathcal{D}\\
 & \; &  & \; &  & \;\ar@{}[ul]|-{\stackrel{Tc_{\textnormal{id}_{\mathcal{C}}}}{\Longleftarrow}}\\
 &  & T\mathcal{A}\ar[uull]^{TL}\ar[uu]_{Ty_{\mathcal{A}}}\ar[rr]_{TH}\ar@{}[uurr]|-{\stackrel{Ty_{H}^{-1}}{\Longleftarrow}} &  & T\mathcal{C}\ar[uu]^{Ty_{\mathcal{C}}}\ar[rruu]_{T\textnormal{id}_{\mathcal{C}}}
}
\]
Then noting that pasting with invertible 2-cells preserves left extensions
and that 
\[
\xymatrix@=1em{T\mathcal{B}\ar[rr]^{TR_{L}} &  & TP\mathcal{A}\ar@{}[dl]|-{\stackrel{T\varphi_{L}}{\Longleftarrow}}\ar[rr]^{\lambda_{\mathcal{A}}}\ar@{}[dr]|-{\stackrel{\omega_{2}}{\Longleftarrow}} &  & PT\mathcal{A}\ar[rr]^{PTH} &  & PT\mathcal{C}\ar[rr]^{Pu} &  & P\mathcal{D}\ar[rr]^{\left(y_{\mathcal{D}}\right)_{\ast}} &  & \mathcal{D}\\
 & \; &  & \;\\
 &  & T\mathcal{A}\ar[uull]^{TL}\ar[uu]_{Ty_{\mathcal{A}}}\ar[rruu]_{y_{T\mathcal{A}}}
}
\]
is a left extension as a consequence of $TL$ being $P$-admissible
(thus the left extension $\lambda_{\mathcal{A}}\cdot TR_{L}$ in Proposition
\ref{lambdabeck} being preserved), we have the result.\end{proof}
\begin{thm}
\label{dimpliesa} In the statement of Theorem \ref{liftkzequiv}
(d) implies both (a) and (g).\end{thm}
\begin{proof}
Firstly, we observe that each $z_{x}$ is $T_{P}$-adm-cocontinuous
as a consequence of Proposition \ref{admcocontinuousequiv}. It follows
that we have the left extensions 
\[
\xymatrix@=1em{T^{2}P\mathcal{A}\ar[rr]^{Tz_{x}} &  & TP\mathcal{A}\ar@{}[dldl]|-{\Uparrow T\xi_{x}}\ar[rr]^{z_{x}}\ar@{}[ddrr]|-{\Uparrow\xi_{x}} &  & P\mathcal{A} &  & T^{3}P\mathcal{A}\ar[rr]^{T^{2}z_{x}} &  & T^{2}P\mathcal{A}\ar[rr]^{Tz_{x}}\ar@{}[dldl]|-{\Uparrow T^{2}\xi_{x}} &  & TP\mathcal{A}\ar@{}[dldl]|-{\Uparrow T\xi_{x}}\ar[rr]^{z_{x}}\ar@{}[ddrr]|-{\Uparrow\xi_{x}} &  & P\mathcal{A}\\
 &  &  & \; &  &  &  &  &  &  &  & \;\\
T^{2}\mathcal{A}\ar[rr]_{Tx}\ar[uu]^{T^{2}y_{\mathcal{A}}} &  & T\mathcal{A}\ar[uu]_{Ty_{\mathcal{A}}}\ar[rr]_{x} &  & \mathcal{A}\ar[uu]_{y_{\mathcal{A}}} &  & T^{3}\mathcal{A}\ar[rr]_{T^{2}x}\ar[uu]^{T^{3}y_{\mathcal{A}}} &  & T^{2}\mathcal{A}\ar[rr]_{Tx}\ar[uu]_{T^{2}y_{\mathcal{A}}} &  & T\mathcal{A}\ar[uu]_{Ty_{\mathcal{A}}}\ar[rr]_{x} &  & \mathcal{A}\ar[uu]_{y_{\mathcal{A}}}
}
\]
upon noting that each $T^{2}y_{\mathcal{A}}$ and $T^{3}y_{\mathcal{A}}$
is $P$-admissible. 

Secondly, we check that each $\left(P\mathcal{A},z_{x}\right)$ is
a pseudo $T$-algebra. We define our algebra structure maps as the
unique solutions to the following equalities (and note they are invertible
as they are isomorphisms of left extensions by Proposition \ref{claim})
\[
\xymatrix{ & TP\mathcal{A}\ar[dr]^{z_{x}} &  &  &  & TP\mathcal{A}\ar[r]^{z_{x}}\ar@{}[rd]|-{\cong\xi_{x}} & P\mathcal{A}\\
P\mathcal{A}\ar[rr]^{\textnormal{id}_{P\mathcal{A}}}\ar[ur]^{u_{P\mathcal{A}}} & \;\ar@{}[u]|-{\Uparrow\sigma_{x}} & P\mathcal{A} & = & P\mathcal{A}\ar[ur]^{u_{P\mathcal{A}}}\ar@{}[r]|-{\cong u_{y}^{\mathcal{A}}} & T\mathcal{A}\ar[r]_{x}\ar[u]^{Ty_{\mathcal{A}}}\ar@{}[d]|-{\cong} & \mathcal{A}\ar[u]_{y_{\mathcal{A}}}\\
 &  & \mathcal{A}\ar[u]_{y_{\mathcal{A}}}\ar[ull]_{\quad\stackrel{\textnormal{id}}{\Longleftarrow}}^{y_{\mathcal{A}}} &  & \mathcal{A}\ar[ur]_{u_{\mathcal{A}}}\ar@/_{1pc}/[rru]_{\textnormal{id}_{\mathcal{A}}}\ar[u]^{y_{\mathcal{A}}} & \;\\
 & TP\mathcal{A}\ar[dr]^{z_{x}} &  &  &  & TP\mathcal{A}\ar[r]^{z_{x}}\ar@{}[rd]|-{\cong\xi_{x}} & P\mathcal{A}\\
T^{2}P\mathcal{A}\ar[r]^{Tz_{x}}\ar[ur]^{m_{P\mathcal{A}}}\ar@{}[rd]|-{\cong T\xi_{x}} & TP\mathcal{A}\ar[r]^{z_{x}}\ar@{}[rd]|-{\cong\xi_{x}}\ar@{}[u]|-{\Uparrow\delta_{x}} & P\mathcal{A} & = & T^{2}P\mathcal{A}\ar[ur]^{m_{P\mathcal{A}}}\ar@{}[r]|-{\cong m_{y}^{\mathcal{A}}} & T\mathcal{A}\ar[r]_{x}\ar[u]^{Ty_{\mathcal{A}}}\ar@{}[d]|-{\cong} & \mathcal{A}\ar[u]_{y_{\mathcal{A}}}\\
T^{2}\mathcal{A}\ar[r]_{Tx}\ar[u]^{T^{2}y_{\mathcal{A}}} & T\mathcal{A}\ar[r]_{x}\ar[u]^{Ty_{\mathcal{A}}} & \mathcal{A}\ar[u]_{y_{\mathcal{A}}} &  & T^{2}\mathcal{A}\ar[ur]_{m_{\mathcal{A}}}\ar[u]^{T^{2}y_{\mathcal{A}}}\ar[r]_{Tx} & T\mathcal{A}\ar[ur]_{x}
}
\]
Note that these are the axioms for $\xi_{x}$ to exhibit $y_{\mathcal{A}}$
as a pseudo $T$-morphism. This may be seen to give an algebra structure
as the equalities
\[
\xymatrix@=1em{ & \;\\
 & T^{2}P\mathcal{A}\ar[rr]^{m_{P\mathcal{A}}}\ar[rd]^{Tz_{x}}\ar@{}[u]|-{\cong} &  & TP\mathcal{A}\ar[rd]^{z_{x}}\\
TP\mathcal{A}\ar[rr]_{\textnormal{id}}\ar[ur]|-{Tu_{P\mathcal{A}}}\ar@/^{3pc}/[rrur]^{\textnormal{id}_{TP\mathcal{A}}} & \;\ar@{}[u]|-{\Uparrow T\sigma_{x}} & TP\mathcal{A}\ar[rr]^{z_{x}}\ar@{}[rdrd]|-{\cong\xi_{x}}\ar@{}[ur]|-{\Uparrow\delta_{x}} &  & P\mathcal{A} &  & TP\mathcal{A}\ar[rr]^{z_{x}}\ar@{}[rdrd]|-{\cong\xi_{x}} &  & P\mathcal{A}\\
 & \;\ar@{}[ul]|-{=} &  &  &  & =\\
T\mathcal{A}\ar[rrrr]_{x}\ar[uu]^{Ty_{\mathcal{A}}}\ar[uurr]_{Ty_{\mathcal{A}}} &  &  &  & \mathcal{A}\ar[uu]_{y_{\mathcal{A}}} &  & T\mathcal{A}\ar[uu]^{Ty_{\mathcal{A}}}\ar[rr]_{x} &  & \mathcal{A}\ar[uu]_{y_{\mathcal{A}}}\\
 & \; & \; & TP\mathcal{A}\ar@/^{1pc}/[rdd]^{z_{x}}\\
 & T^{2}P\mathcal{A}\ar[rr]^{Tz_{x}}\ar[rru]^{m_{P\mathcal{A}}}\ar@{}[u]|-{\quad\;\cong} &  & TP\mathcal{A}\ar[rd]^{z_{x}}\ar@{}[u]|-{\Uparrow\delta_{x}}\\
TP\mathcal{A}\ar[rr]^{z_{x}}\ar[ur]|-{u_{TP\mathcal{A}}}\ar@/^{2pc}/[rrruu]^{\textnormal{id}_{TP\mathcal{A}}} &  & P\mathcal{A}\ar[rr]_{\textnormal{id}_{P\mathcal{A}}}\ar[ur]^{u_{P\mathcal{A}}}\ar@{}[ul]|-{\cong u_{z_{x}}} & \;\ar@{}[u]|-{\Uparrow\sigma_{x}} & P\mathcal{A} &  & TP\mathcal{A}\ar[rr]^{z_{x}}\ar@{}[rdrd]|-{\cong\xi_{x}} &  & P\mathcal{A}\\
 &  &  & \;\ar@{}[ru]|-{=} &  & =\\
T\mathcal{A}\ar[rrrr]_{x}\ar[uu]^{Ty_{\mathcal{A}}}\ar@{}[rruu]|-{\cong\xi_{x}} &  &  &  & \mathcal{A}\ar[uu]_{y_{\mathcal{A}}}\ar[uull]^{y_{\mathcal{A}}} &  & T\mathcal{A}\ar[uu]^{Ty_{\mathcal{A}}}\ar[rr]_{x} &  & \mathcal{A}\ar[uu]_{y_{\mathcal{A}}}
}
\]
and the equality between

\[
\xymatrix@=1em{T^{2}P\mathcal{A}\ar[rr]^{m_{P\mathcal{A}}} &  & T^{2}P\mathcal{A}\ar[rr]^{m_{P\mathcal{A}}}\ar[ddrr]^{Tz_{x}} &  & TP\mathcal{A}\ar[rddr]^{z_{x}}\\
\; & \;\ar@{}[l]|-{\cong}\\
T^{3}P\mathcal{A}\ar[rr]^{T^{2}z_{x}}\ar[ruru]^{Tm_{P\mathcal{A}}}\ar[uu]^{m_{TP\mathcal{A}}} &  & T^{2}P\mathcal{A}\ar[rr]^{Tz_{x}}\ar@{}[uu]|-{\Uparrow T\delta_{x}} &  & TP\mathcal{A}\ar[rr]^{z_{x}}\ar@{}[uu]|-{\Uparrow\delta_{x}} &  & P\mathcal{A}\\
\\
T^{3}\mathcal{A}\ar[rr]_{T^{2}x}\ar[uu]^{T^{3}y_{\mathcal{A}}}\ar@{}[rruu]|-{\cong T^{2}\xi_{x}} &  & T^{2}\mathcal{A}\ar[rr]_{Tx}\ar[uu]_{T^{2}y_{\mathcal{A}}}\ar@{}[rruu]|-{\cong T\xi_{x}} &  & T\mathcal{A}\ar[rr]_{x}\ar[uu]_{Ty_{\mathcal{A}}}\ar@{}[rruu]|-{\cong\xi_{x}} &  & \mathcal{A}\ar[uu]_{y_{\mathcal{A}}}
}
\]
and

\[
\xymatrix@=1em{ &  &  &  & TP\mathcal{A}\ar@/^{1pc}/[rdrddd]^{z_{x}}\\
\\
 &  & T^{2}P\mathcal{A}\ar[rr]^{Tz_{x}}\ar[rruu]^{m_{P\mathcal{A}}} &  & TP\mathcal{A}\ar[rddr]^{z_{x}}\ar@{}[uu]|-{\Uparrow\delta_{x}}\\
\\
T^{3}P\mathcal{A}\ar[rr]^{T^{2}z_{x}}\ar[rruu]^{m_{TP\mathcal{A}}} &  & T^{2}P\mathcal{A}\ar[rr]^{Tz_{x}}\ar[uurr]^{m_{P\mathcal{A}}}\ar@{}[uu]|-{\Uparrow m_{z_{x}}} &  & TP\mathcal{A}\ar[rr]^{z_{x}}\ar@{}[uu]|-{\Uparrow\delta_{x}} &  & P\mathcal{A}\\
\\
T^{3}\mathcal{A}\ar[rr]_{T^{2}x}\ar[uu]^{T^{3}y_{\mathcal{A}}}\ar@{}[rruu]|-{\cong T^{2}\xi_{x}} &  & T^{2}\mathcal{A}\ar[rr]_{Tx}\ar[uu]_{T^{2}y_{\mathcal{A}}}\ar@{}[rruu]|-{\cong T\xi_{x}} &  & T\mathcal{A}\ar[rr]_{x}\ar[uu]_{Ty_{\mathcal{A}}}\ar@{}[rruu]|-{\cong\xi_{x}} &  & \mathcal{A}\ar[uu]_{y_{\mathcal{A}}}
}
\]
easily follow from the respective conditions on $\left(\mathcal{A},x\right)$
being a pseudo $T$-algebra and the definitions of $\delta_{x}$ and
$\sigma_{x}$.

We now use the above to define our KZ doctrine 
\[
\widetilde{P}\colon\text{ps-\ensuremath{T}-alg}\to\text{ps-\ensuremath{T}-alg}
\]
We use the assignment on objects $\left(\mathcal{A},x\right)\mapsto\left(P\mathcal{A},z_{x}\right)$.
We take our units as the pseudo $T$-morphisms $\left(y_{\mathcal{A}},\xi_{x}\right)\colon\left(\mathcal{A},x\right)\to\left(P\mathcal{A},z_{x}\right)$.
Now suppose that we are given a pseudo $T$-morphism $\left(F,\phi\right)\colon\left(\mathcal{A},x\right)\to\left(P\mathcal{B},z_{r}\right)$
as in the diagram
\[
\xymatrix@=1em{\left(P\mathcal{A},z_{x}\right)\ar[rr]^{\left(\overline{F},\overline{\phi}\right)}\ar@{}[dr]|-{\stackrel{c_{F}}{\Longleftarrow}} &  & \left(P\mathcal{B},z_{r}\right)\\
 & \;\\
\left(\mathcal{A},x\right)\ar[uurr]_{\left(F,\phi\right)}\ar[uu]^{\left(y_{\mathcal{A}},\xi_{x}\right)}
}
\]
Since $z_{r}$ is $T_{P}$-cocontinuous, we may apply Proposition
\ref{docleftext} to find a lax $T$-morphism $\left(\overline{F},\overline{\phi}\right)$
as above. Indeed the lax structure map $\overline{\phi}$ is given
as the unique solution to 
\[
\xymatrix{ & TP\mathcal{A}\ar[r]^{z_{x}} & P\mathcal{A}\ar[dd]^{\overline{F}} &  &  & TP\mathcal{A}\ar[r]^{z_{x}}\ar[dd]_{T\overline{F}} & P\mathcal{A}\ar[dd]^{\overline{F}}\\
T\mathcal{A}\ar[r]^{x}\ar[ur]^{Ty_{\mathcal{A}}}\ar[rd]_{TF} & \mathcal{A}\ar[dr]^{F}\ar[ur]^{y_{\mathcal{A}}}\ar@{}[r]|-{\Uparrow c_{F}}\ar@{}[u]|-{\Uparrow\xi_{x}} & \; & = & T\mathcal{A}\ar[ru]^{Ty_{\mathcal{A}}}\ar[rd]_{TF}\ar@{}[r]|-{\Uparrow Tc_{F}\quad} & \;\ar@{}[r]|-{\Uparrow\overline{\phi}} & \;\\
 & TP\mathcal{B}\ar[r]_{z_{r}}\ar@{}[u]|-{\Uparrow\phi} & P\mathcal{B} &  &  & TP\mathcal{B}\ar[r]_{z_{r}} & P\mathcal{B}
}
\]
But we notice that
\[
\xymatrix@=1em{TP\mathcal{A}\ar[rr]^{T\overline{F}}\ar@{}[dr]|-{\stackrel{Tc_{F}}{\Longleftarrow}} &  & TP\mathcal{B}\ar[rr]^{z_{r}} &  & P\mathcal{B} &  & TP\mathcal{A}\ar[rr]^{z_{x}} &  & P\mathcal{A}\ar[rr]^{\overline{F}} &  & P\mathcal{B}\\
 & \; &  &  &  &  &  &  &  & \;\ar@{}[ul]|-{\stackrel{c_{F}}{\Longleftarrow}}\\
T\mathcal{A}\ar[uu]^{Ty_{\mathcal{A}}}\ar[rruu]_{TF} &  &  &  &  &  & T\mathcal{A}\ar[rr]_{x}\ar[uu]^{Ty_{\mathcal{A}}}\ar@{}[urur]|-{\Uparrow\xi_{x}}\ar@/_{0.5pc}/[rrrd]_{TF} &  & \mathcal{A}\ar[uu]_{y_{\mathcal{A}}}\ar[uurr]_{F}\ar@{}[dr]|-{\cong\phi}\\
 &  &  &  &  &  &  &  &  & TP\mathcal{B}\ar[uuru]_{z_{r}}
}
\]
are both left extensions since $z_{r}$ is $T_{P}$-cocontinuous and
$Ty_{\mathcal{A}}$ is \emph{$P$-}admissible respectively. It follows
that the lax $T$-morphism structure map $\overline{\phi}$ is an
isomorphism of left extensions, making $\left(\overline{F},\overline{\phi}\right)$
a pseudo $T$-morphism. The case where we only assume $\left(F,\phi\right)$
to be a lax $T$-morphism (which gives only a lax structure on $\overline{F}$)
corresponds to (g) of Theorem \ref{liftkzequiv}.

We now check that such left extensions are preserved by other left
extensions of this form. Suppose we are given two left extensions
of pseudo $T$-algebras and pseudo $T$-morphisms
\[
\xymatrix@=1em{\left(P\mathcal{A},z_{x}\right)\ar[rr]^{\left(\overline{F},\overline{\phi}\right)}\ar@{}[dr]|-{\stackrel{c_{F}}{\Longleftarrow}} &  & \left(P\mathcal{B},z_{r}\right) &  &  & \left(P\mathcal{B},z_{r}\right)\ar[rr]^{\left(\overline{G},\overline{\sigma}\right)}\ar@{}[dr]|-{\stackrel{c_{G}}{\Longleftarrow}} &  & \left(P\mathcal{C},z_{h}\right)\\
 & \; &  &  &  &  & \;\\
\left(\mathcal{A},x\right)\ar[uurr]_{\left(F,\phi\right)}\ar[uu]^{\left(y_{\mathcal{A}},\xi_{x}\right)} &  &  &  &  & \left(\mathcal{B},r\right)\ar[uurr]_{\left(G,\sigma\right)}\ar[uu]^{\left(y_{\mathcal{B}},\xi_{r}\right)}
}
\]
To see that
\[
\xymatrix@=1em{\left(P\mathcal{A},z_{x}\right)\ar[rr]^{\left(\overline{F},\overline{\phi}\right)} &  & \left(P\mathcal{B},z_{r}\right)\ar[rr]^{\left(\overline{G},\overline{\sigma}\right)} &  & \left(P\mathcal{C},z_{h}\right)\\
\ar@{}[rr]|-{\stackrel{\left(\overline{G},\overline{\sigma}\right)c_{F}}{\Longleftarrow}} & \; & \left(P\mathcal{B},z_{r}\right)\ar[rur]_{\left(\overline{G},\overline{\sigma}\right)}\\
\left(\mathcal{A},x\right)\ar[urr]_{\left(F,\phi\right)}\ar[uu]^{\left(y_{\mathcal{A}},\xi_{x}\right)}
}
\]
is a left extension we need only observe that the $T$-morphism structure
on $\overline{G}\overline{F}$ resulting from an application of Proposition
\ref{docleftext} (on the outside diagram) is given by composing $\overline{\phi}$
and $\overline{\sigma}$ as above. This is shown by pasting the defining
diagram for $\overline{\phi}$ with $\overline{\sigma}$ which gives

\[
\xymatrix{ & TP\mathcal{A}\ar[r]^{z_{x}} & P\mathcal{A}\ar[dd]^{\overline{F}} &  &  & TP\mathcal{A}\ar[r]^{z_{x}}\ar[dd]_{T\overline{F}} & P\mathcal{A}\ar[dd]^{\overline{F}}\\
T\mathcal{A}\ar[r]^{x}\ar[ur]^{Ty_{\mathcal{A}}}\ar[rd]_{TF} & \mathcal{A}\ar[dr]^{F}\ar[ur]^{y_{\mathcal{A}}}\ar@{}[r]|-{\Uparrow c_{F}}\ar@{}[u]|-{\Uparrow\xi_{x}} & \; &  & T\mathcal{A}\ar[ru]^{Ty_{\mathcal{A}}}\ar[rd]_{TF}\ar@{}[r]|-{\Uparrow Tc_{F}\quad} & \;\ar@{}[r]|-{\Uparrow\overline{\phi}} & \;\\
 & TP\mathcal{B}\ar[r]_{z_{r}}\ar@{}[u]|-{\Uparrow\phi}\ar[dd]_{T\overline{G}} & P\mathcal{B}\ar[dd]^{\overline{G}} & = &  & TP\mathcal{B}\ar[r]_{z_{r}}\ar[dd]_{T\overline{G}} & P\mathcal{B}\ar[dd]^{\overline{G}}\\
 & \;\ar@{}[r]|-{\Uparrow\overline{\sigma}} & \; &  &  & \;\ar@{}[r]|-{\Uparrow\overline{\sigma}} & \;\\
 & TP\mathcal{C}\ar[r]_{z_{h}} & P\mathcal{C} &  &  & TP\mathcal{C}\ar[r]_{z_{h}} & P\mathcal{C}
}
\]
which is  the defining diagram for the induced lax structure on $\overline{G}\cdot\overline{F}$
from an application of Proposition \ref{docleftext}.

It is an easy consequence of Proposition \ref{docleftext} that each
$\left(y_{\mathcal{A}},\xi_{x}\right)$ is dense. Indeed since $z_{x}$
$T$-preserves the left extension
\[
\xymatrix@=1em{P\mathcal{A}\ar[rr]^{\textnormal{id}_{P\mathcal{A}}} &  & P\mathcal{A}\\
 & \ar@{}[ul]|-{=}\\
\mathcal{A}\ar[uu]^{y_{\mathcal{A}}}\ar[uurr]_{y_{\mathcal{A}}}
}
\]
the density property may be lifted to pseudo-$T$-algebras applying
Proposition \ref{docleftext}.
\end{proof}

\section{Consequences and Examples\label{consequencesandexamples}}

In this section we point out some consequences of Theorem \ref{liftkzequiv}
proven in the previous section, and in particular some properties
of the lifted KZ doctrine $\widetilde{P}$ on $\textnormal{ps-}T\textnormal{-alg}$. 

\subsection{The lifted KZ doctrine}

Algebraic cocompleteness is usually defined by asking that the underlying
object be cocomplete, and that the algebra structure map be separately
cocontinuous. The following proposition justifies this definition.
\begin{prop}
Suppose any of the equivalent conditions of Theorem \ref{liftkzequiv}
are satisfied. Then a pseudoalgebra $\left(\mathcal{A},x\right)$
is $\widetilde{P}$-cocomplete if and only if $\mathcal{A}$ is $P$-cocomplete
and $x\colon T\mathcal{A}\to\mathcal{A}$ is $T_{P}$-cocontinuous. \end{prop}
\begin{proof}
$\left(\implies\right)\colon$ Suppose that $\left(\mathcal{A},x\right)$
is a $\widetilde{P}$-cocomplete pseudo $T$-algebra. Then, by doctrinal
adjunction \cite{doctrinal}, the pseudo $T$-morphism $\left(y_{\mathcal{A}},\xi_{x}\right)$
has a reflection left adjoint $\left(\left(y_{\mathcal{A}}\right)_{\ast},\left(\xi_{x}^{-1}\right)_{\ast}\right)$
for which $\left(\xi_{x}^{-1}\right)_{\ast}$ is defined by the mates
correspondence and is invertible. That is, we have isomorphisms 
\[
\xymatrix@=1em{TP\mathcal{A}\ar[rr]^{z_{x}}\ar@{}[ddrr]|-{\Downarrow\xi_{x}^{-1}} &  & P\mathcal{A} &  &  &  & TP\mathcal{A}\ar[rr]^{z_{x}}\ar@{}[ddrr]|-{\Downarrow\left(\xi_{x}^{-1}\right)_{\ast}}\ar[dd]_{T\left(y_{\mathcal{A}}\right)_{\ast}} &  & P\mathcal{A}\ar[dd]^{\left(y_{\mathcal{A}}\right)_{\ast}}\\
 & \; &  &  &  &  &  & \;\\
T\mathcal{A}\ar[uu]^{Ty_{\mathcal{A}}}\ar[rr]_{x} &  & \mathcal{A}\ar[uu]_{y_{\mathcal{A}}} &  &  &  & T\mathcal{A}\ar[rr]_{x} &  & \mathcal{A}
}
\]
Now $\left(y_{\mathcal{A}}\right)_{\ast}\dashv y_{\mathcal{A}}$ via
a reflection adjoint so $\mathcal{A}$ is $P$-cocomplete. We thus
check that $x\colon T\mathcal{A}\to\mathcal{A}$ is $T_{P}$-cocontinuous.
Suppose we are given a left extension as on the left
\[
\xymatrix@=1em{P\mathcal{D}\ar[rr]^{\overline{F}}\ar@{}[dr]|-{\Uparrow c_{F}} &  & \mathcal{A} &  &  & TP\mathcal{D}\ar[rr]^{\overline{F}}\ar@{}[dr]|-{\Uparrow Tc_{F}} &  & T\mathcal{A}\ar[rr]^{x} &  & \mathcal{A}\\
 & \; &  &  &  &  & \;\\
\mathcal{D}\ar[rruu]_{F}\ar[uu]^{y_{\mathcal{D}}} &  &  &  &  & T\mathcal{D}\ar[rruu]_{TF}\ar[uu]^{Ty_{\mathcal{D}}}
}
\]
We check that the right diagram is a left extension. We first note
this is equivalent to showing that $x$ $T$-preserves left extensions
as on the left below
\[
\xymatrix@=1em{P\mathcal{D}\ar[rr]^{PF}\ar@{}[drdr]|-{\Uparrow c_{y_{\mathcal{A}}\cdot F}} &  & P\mathcal{A}\ar[rr]^{\left(y_{\mathcal{A}}\right)_{\ast}}\ar@{}[dr]|-{\Uparrow c_{\textnormal{id}_{\mathcal{A}}}} &  & \mathcal{A} &  & TP\mathcal{D}\ar[rr]^{TPF}\ar@{}[drdr]|-{\Uparrow Tc_{y_{\mathcal{A}}\cdot F}} &  & TP\mathcal{A}\ar[rr]^{T\left(y_{\mathcal{A}}\right)_{\ast}}\ar@{}[dr]|-{\Uparrow Tc_{\textnormal{id}_{\mathcal{A}}}} &  & T\mathcal{A}\ar[rr]^{x} &  & \mathcal{A}\\
 &  &  & \; &  &  &  &  &  & \;\\
\mathcal{D}\ar[rr]_{F}\ar[uu]^{y_{\mathcal{D}}} &  & \mathcal{A}\ar[uu]_{y_{\mathcal{A}}}\ar[uurr]_{\textnormal{id}_{\mathcal{A}}} &  &  &  & T\mathcal{D}\ar[rr]_{TF}\ar[uu]^{Ty_{\mathcal{D}}} &  & T\mathcal{A}\ar[uu]_{Ty_{\mathcal{A}}}\ar[uurr]_{T\textnormal{id}_{\mathcal{A}}}
}
\]
and so it suffices to check the right diagram is a left extension.
This is seen upon pasting with the isomorphism $\left(\xi_{x}^{-1}\right)_{\ast}$
as $z_{x}$ is $T_{P}$-cocontinuous and $\left(y_{\mathcal{A}}\right)_{\ast}$
is a left adjoint (and hence preserves all left extensions).

$\left(\implies\right)\colon$ Suppose that $\mathcal{A}$ is $P$-cocomplete
and $x$ is $T_{P}$-cocontinuous. Then $\left(\mathcal{A},x\right)$
is $\widetilde{P}$-cocomplete as $\left(\mathcal{A},x\right)$ admits
left extensions along $\left(y_{\mathcal{A}},\xi_{x}\right)$ by Proposition
\ref{docleftext}, and showing that such left extensions admit a pseudo
$T$-morphism structure and are preserved is a similar calculation
to that in the proof of Theorem \ref{dimpliesa}.\end{proof}
\begin{rem}
By the same proof, we have the same classification of $\widetilde{P}_{\textnormal{lax}}$-cocomplete
pseudo $P$-algebras. The $\widetilde{P}_{\textnormal{oplax}}$-cocomplete
pseudo $P$-algebras are those with an underlying $P$-cocomplete
object, as a consequence of doctrinal adjunction \cite{doctrinal}.
The maps $\left(F,\phi\right)$ which are $\widetilde{P}/\widetilde{P}_{\textnormal{lax}}/\widetilde{P}_{\textnormal{oplax}}$-cocontinuous
are all classified by those maps for which the underlying $F$ is
$P$-cocontinuous.
\end{rem}
Given a KZ doctrine $P$ on a 2-category $\mathscr{C}$ we have an
equivalence given by composition with the unit $y_{\mathcal{A}}$,
namely $\mathscr{C}_{\textnormal{ccts}}\left(P\mathcal{A},\mathcal{B}\right)\simeq\mathscr{C}\left(\mathcal{A},\mathcal{B}\right)$,
with $\mathscr{C}_{\textnormal{ccts}}\left(P\mathcal{A},\mathcal{B}\right)$
containing left extensions of maps $\mathcal{A}\to\mathcal{B}$ along
the unit $y_{\mathcal{A}}$. This is clearly essentially surjective
as for an $F\colon\mathcal{A}\to\mathcal{B}$ we may take $\overline{F}\colon P\mathcal{A}\to\mathcal{B}$,
and fully faithful as $y_{\mathcal{A}}$ is dense. We can thus recover
Im and Kelly's following result.
\begin{cor}
[Im-Kelly \cite{uniconvolution}] Suppose we are given a 2-category
$\mathscr{C}$ equipped with a pseudomonad $\left(T,u,m\right)$ and
a KZ doctrine $\left(P,y\right)$. Suppose any of the equivalent conditions
of Theorem \ref{liftkzequiv} are met. Then for every pair of pseudo
$T$-algebras $\left(\mathcal{A},x\right)$ and $\left(\mathcal{B},r\right)$
where $\mathcal{B}$ is $P$-cocomplete, composition with the unit
$\left(y_{\mathcal{A}},\xi_{x}\right)$ defines the equivalence 
\[
\mathbf{Oplax}\left[\left(\mathcal{A},x\right),\left(\mathcal{B},r\right)\right]\simeq\mathbf{Oplax}_{\textnormal{ccts}}\left[\left(P\mathcal{A},z_{x}\right),\left(\mathcal{B},r\right)\right]
\]
where a morphism of pseudo $T$-algebras is cocontinuous when the
underlying morphism is. Suppose further that $r$ is $T_{P}$-cocontinuous,
then composition with the unit $\left(y_{\mathcal{A}},\xi_{x}\right)$
also defines the equivalences
\[
\begin{aligned}\mathbf{Lax}\left[\left(\mathcal{A},x\right),\left(\mathcal{B},r\right)\right] & \simeq\mathbf{Lax}_{\textnormal{ccts}}\left[\left(P\mathcal{A},z_{x}\right),\left(\mathcal{B},r\right)\right]\\
\mathbf{Pseudo}\left[\left(\mathcal{A},x\right),\left(\mathcal{B},r\right)\right] & \simeq\mathbf{Pseudo}_{\textnormal{ccts}}\left[\left(P\mathcal{A},z_{x}\right),\left(\mathcal{B},r\right)\right]
\end{aligned}
\]
Moreover, the above three equivalences restrict to \emph{$P$-}admissible
underlying morphisms.\end{cor}
\begin{proof}
Observe that the pseudo $T$-algebras $\left(\mathcal{B},r\right)$
with $\mathcal{B}$ being $P$-cocomplete and $r$ being $T_{P}$-cocontinuous
are $\widetilde{P}$-cocomplete as a consequence of Proposition \ref{docleftext}.
Similarly in the setting of lax $T$-morphisms. For the oplax case,
the $P$-cocompleteness of the underlying object is sufficient (as
in the proof of the above theorem) as Proposition \ref{docleftext}
does not apply here (aside from density).

Moreover, we note that if $\overline{L}\colon P\mathcal{A}\to\mathcal{B}$
is \emph{$P$-}admissible then so is the composite $\overline{L}\cdot y_{\mathcal{A}}\cong L$
due to closure under composition. If $L$ is \emph{$P$-}admissible,
then $\overline{L}$ has a right adjoint by \cite[Lemma 12]{yonedakz},
and so $P\overline{L}$ does too.\end{proof}
\begin{prop}
Suppose any of the equivalent conditions of Theorem \ref{liftkzequiv}
are satisfied. Assume $\left(L,\alpha\right)\colon\left(\mathcal{A},x\right)\to\left(\mathcal{B},y\right)$
is a pseudo $T$-morphism and $L\colon\mathcal{A}\to\mathcal{B}$
is $P$-admissible. Then $\left(L,\alpha\right)$ is $\widetilde{P}$-admissible
if and only if for every $\widetilde{P}$-cocomplete pseudoalgebra
$\left(\mathcal{C},z\right)$ and pseudo $T$-morphism $\left(I,\xi\right)$
as in the diagram
\[
\xymatrix@=1em{\left(\mathcal{B},y\right)\ar[rr]^{\left(R,\beta\right)} &  & \left(\mathcal{C},z\right)\ar@{}[dl]|-{\stackrel{\eta}{\Longleftarrow}}\\
 & \;\\
 &  & \left(\mathcal{A},x\right)\ar[uu]_{\left(I,\xi\right)}\ar[uull]^{\left(L,\alpha\right)}
}
\]
the induced lax structure cell $\beta$ on the underlying left extension
$R$ as in Proposition \ref{docleftext} is invertible. Moreover,
for pseudo, lax and oplax $\left(L,\alpha\right)$ respectively,

1. $\left(L,\alpha\right)$ is $\widetilde{P}$-admissible iff $\widetilde{P}\left(L,\alpha\right)$
has a pseudo right adjoint;

2. $\left(L,\alpha\right)$ is $\widetilde{P}_{\textnormal{lax}}$-admissible
iff $\widetilde{P}\left(L,\alpha\right)$ is pseudo;

3. $\left(L,\alpha\right)$ is $\widetilde{P}_{\textnormal{oplax}}$-admissible
iff $\widetilde{P}\left(L,\alpha\right)$ has a pseudo right adjoint.\end{prop}
\begin{proof}
The first part of this proposition follows an equivalent characterization
of $P$-admissibility as given by Bunge and Funk and discussed in
\cite{bungefunk,yonedakz}, along with Proposition \ref{docleftext}.
The last three are a direct consequence of doctrinal adjunction \cite{doctrinal}.\end{proof}
\begin{rem}
Note that if $P$ (and thus $\widetilde{P}$) is locally fully faithful,
then $\widetilde{P}\left(L,\alpha\right)$ being pseudo implies $\left(L,\alpha\right)$
is. Indeed the lax structure cell $\alpha$ when whiskered by $y_{\mathcal{A}}$
is invertible (a direct consequence of how the structure cell of $\widetilde{P}\left(L,\alpha\right)$
is defined in Proposition \ref{docleftext}). As $y_{\mathcal{A}}$
is fully faithful, this means $\alpha$ is invertible. Hence, in this
case, Statement 2 of the above proposition is equivalent to saying
$\left(L,\alpha\right)$ is pseudo.
\end{rem}

\subsection{The Preorder of KZ Doctrines on a 2-Category}

In the following discussion of morphisms between KZ pseudomonads and
doctrines we will omit most of the details, as this would take us
beyond the scope of this paper. Moreover, the calculations are quite
similar to earlier. 
\begin{defn}
Given KZ pseudomonads $\left(P,y,\mu\right)$ and $\left(P',y',\mu'\right)$
on a 2-category $\mbox{\ensuremath{\mathscr{C}}}$, a \emph{morphism
of KZ pseudomonads} $P\implies P'$ (corresponding to a lifting of
the identity on $\mathscr{C}$) consists of a pseudonatural transformation
$\alpha\colon P\to P'$ and an invertible modification 
\[
\xymatrix@=1em{P\ar[rr]^{\alpha} &  & P'\\
 & \ar@{}[ur]|-{\overset{\psi_{y}}{\Longleftarrow}}\\
 &  & 1\ar[uu]_{y'}\ar[lluu]^{y}
}
\]
such that 
\[
\xymatrix@=1em{P\ar[rr]^{\alpha}\ar@/^{0.7pc}/[dd]^{Py}\ar@/_{0.7pc}/[dd]_{yP} &  & P'\ar[dd]^{P'y}\ar[rrdd]^{P'y'} &  &  &  &  &  &  &  & P\ar[rr]^{\alpha}\ar[dd]_{y'P}\ar[dlld]_{yP} &  & P'\ar@/^{0.7pc}/[dd]^{P'y'}\ar@/_{0.7pc}/[dd]_{y'P'}\\
\ar@{}[]|-{\stackrel{\theta}{\Longleftarrow}} & \ar@{}[r]|-{\stackrel{\alpha_{y}}{\Longleftarrow}\quad} & \; & \ar@{}[ld]|-{\;\quad\stackrel{P'\psi_{y}}{\Longleftarrow}} &  &  &  & = &  & \ar@{}[rd]|-{\stackrel{\psi_{y}P}{\Longleftarrow}\quad\;} & \; & \ar@{}[l]|-{\stackrel{\left(y'\right)_{\alpha}^{-1}}{\Longleftarrow}} & \ar@{}[]|-{\stackrel{\theta'}{\Longleftarrow}}\\
PP\ar[rr]_{\alpha P} &  & P'P\ar[rr]_{P'\alpha} &  & P'P'\ar[rr]_{\mu'} &  & P' &  & PP\ar[rr]_{\alpha P} &  & P'P\ar[rr]_{P'\alpha} &  & P'P'\ar[rr]_{\mu'} &  & P'
}
\]
\end{defn}
\begin{lem}
Given a morphism of KZ pseudomonads as above, the 2-cell $\psi_{y}$
exhibits $\alpha$ as a left extension of $y'$ along $y$.\end{lem}
\begin{proof}
Note that $P'y\dashv\mu'\cdot P'\alpha$ (note this right adjoint
is $\overline{\alpha}$) with unit $\eta$ given by

\[
\xymatrix@=1em{ & P'P\ar[r]^{P'\alpha} & P'P'\ar[rd]^{\mu'}\\
P'\ar[ru]^{P'y}\ar@/_{1pc}/[rur]_{P'y'}\ar@/_{2pc}/[rrr]_{\textnormal{id}_{P'}} & \ar@{}[u]|-{\quad\cong P'\psi_{y}} & \ar@{}[d]|-{\cong} & P'\\
 &  & \;
}
\]
We define the counit ${\varepsilon}$ as the unique 2-cell for which
\[
\xymatrix@=1em{ &  &  &  & \;\ar@{}[drr]|-{\underset{}{\Uparrow{\varepsilon}}} &  &  &  &  &  &  &  & \ar@{}[]|-{\cong\psi_{y}P}\\
P\ar[rr]^{y'P}\ar@/_{0.5pc}/[rrd]_{\alpha} &  & P'P\ar[rr]^{P'\alpha}\ar@/^{2pc}/[rrrrrr]^{\textnormal{id}} &  & P'P'\ar[rr]^{\mu'} &  & P'\ar[rr]^{P'y} &  & P'P &  & P\ar@/^{0.7pc}/[rr]^{y_{P}}\ar@/_{0.7pc}/[rr]_{Py}\ar@{}[rr]|-{\Uparrow\theta}\ar@/^{3pc}/[rrrr]^{y'P}\ar@/_{0.7pc}/[rrd]_{\alpha} &  & PP\ar[rr]^{\alpha P} & \; & P'P\\
 &  & P'\ar@/_{0.5pc}/[rru]^{y'P}\ar@{}[u]|-{\cong\left(y'\right)_{\alpha}^{-1}}\ar@/_{0.7pc}/[rrrru]_{\textnormal{id}_{P'}} &  & \ar@{}[u]|-{\cong} &  &  &  &  &  &  &  & P'\ar@/_{0.8pc}/[rru]_{P'y}\ar@{}[ru]|-{\cong\alpha_{y}}
}
\]
We will omit the triangle identities (as this is almost the same calculation
as earlier). The result then follows from \cite[Remark 16]{yonedakz}
and naturality and pseudomonad coherence axioms.\end{proof}
\begin{rem}
Note that we automatically have an isomorphism
\[
\xymatrix@=1em{PP\ar[rr]^{\alpha\ast\alpha}\ar[dd]_{\mu} &  & P'P'\ar[dd]^{\mu'}\\
 & \ar@{}[]|-{\cong}\\
P\ar[rr]_{\alpha} &  & P'
}
\]
Indeed $\alpha\cdot\mu$ may be seen as a left extension of $Py\cdot y$
along $y'$ exhibited by the bijections 

\settowidth{\rhs}{$H\cdot Py\cdot y$} 
\settowidth{\lhs}{$\alpha\cdot\mu$}
\begin{prooftree}
\Axiom$\makebox[\lhs][r]{$\alpha\cdot\mu$} {\ \rightarrow\ } \makebox[\rhs][l]{$H$}$
\UnaryInf$\makebox[\lhs][r]{$\alpha$} {\ \rightarrow\ } \makebox[\rhs][l]{$H\cdot Py$}$
\UnaryInf$\makebox[\lhs][r]{$y'$} {\ \rightarrow\ } \makebox[\rhs][l]{$H\cdot Py\cdot y$}$ 
\end{prooftree}Moreover, $\mu'\cdot\alpha\ast\alpha$ may be seen a left extension
of $yP\cdot y$ along $y'$ by using the formula for this left extension
as in \cite[Remark 16]{yonedakz} and composing with appropriate isomorphisms.
Indeed $P'y\dashv\mu'\cdot P'\alpha$ and hence $P'yP\dashv\mu'\cdot P'\alpha P$.
We then recall that $yP\cdot y\cong Py\cdot y$.\end{rem}
\begin{defn}
Given KZ doctrines $\left(P,y\right)$ and $\left(P',y'\right)$ on
a 2-category $\mathscr{C}$ a \emph{morphism of KZ doctrines} $P\implies P'$
consists of the assertions that: 
\begin{enumerate}
\item every $P$-admissible map is also $P'$-admissible;
\item for each $\mathcal{A}\in\mathscr{C}$, the resulting 2-cell exhibiting
the left extension $\alpha_{\mathcal{A}}$ 
\[
\xymatrix@=1em{P\mathcal{A}\ar[rr]^{\alpha_{\mathcal{A}}} &  & P'\mathcal{A}\\
 & \ar@{}[ur]|-{\overset{\psi_{y}^{\mathcal{A}}}{\Longleftarrow}}\\
 &  & \mathcal{A}\ar[uu]_{y'_{\mathcal{A}}}\ar[lluu]^{y_{\mathcal{A}}}
}
\]
is invertible;
\item for each $\mathcal{A},\mathcal{B}\in\mathscr{C}$, left extensions
along $y_{\mathcal{A}}$ into $P\mathcal{B}$ are preserved by $\alpha_{\mathcal{B}}$.\footnote{Consequently, components of $\alpha$ are $P$-homomorphisms.}
\end{enumerate}
\end{defn}
\begin{lem}
Suppose we are given two KZ doctrines $\left(P,y\right)$ and $\left(P',y'\right)$
on a 2-category $\mathscr{C}$, with corresponding KZ pseudomonads
$\left(P,y,\mu\right)$ and $\left(P',y',\mu'\right)$. Then morphisms
$P\implies P'$ of KZ doctrines are in bijection with morphisms $P\implies P'$
of KZ pseudomonads (identified via uniqueness of left extensions up
to coherent isomorphism).\end{lem}
\begin{proof}
Given that every $P$-admissible map is also $P'$-admissible, we
know that $P'y$ has a right adjoint (and that we have a left extension
$\alpha$ as above, assumed invertible). In particular, this right
adjoint may be constructed as in \cite[Prop. 15]{yonedakz}, and thus
we have an adjunction $P'y\dashv\mu'\cdot P'\alpha$ with unit and
counit as above. The triangle identities then force the coherence
condition. Pseudonaturality of $\alpha$ is equivalent to the preservation
condition.

Conversely, given a morphism of KZ pseudomonads (which gives rise
to a usual morphism of pseudomonads) we know that every $P'$-cocomplete
object is also $P$-cocomplete (as the cocomplete objects may be characterized
as algebras), and similarly for homomorphisms. Hence given a $P$-admissible
map $L\colon\mathcal{A}\to\mathcal{B}$ and map $K\colon\mathcal{A}\to\mathcal{X}$
for a $P'$-cocomplete object (and thus also $P$-cocomplete) $\mathcal{X}$,
there exists a left extension $J\colon\mathcal{B}\to\mathcal{X}$
which is preserved by any $P'$-homomorphism (as such is necessarily
a $P$-homomorphism also). Thus $L$ is $P'$-admissible.
\end{proof}
Combining this with the results of \cite{marm2012}, we have the following
proposition.
\begin{prop}
Given a 2-category $\mathscr{C}$, the assignation of \cite[Theorems 4.1,4.2]{marm2012}
underlies a biequivalence
\[
\mathbf{KZdoc}\left(\mathscr{C}\right)\simeq\mathbf{KZps}\left(\mathscr{C}\right)
\]
where $\mathbf{KZps}\left(\mathscr{C}\right)$ is the 2-category of
KZ pseudomonads, morphisms of KZ pseudomonads and isomorphisms of
left extensions, and $\mathbf{KZdoc}\left(\mathscr{C}\right)$ is
the preorder of KZ doctrines and morphisms of KZ doctrines.
\end{prop}

\subsection{Examples}

Consider the 2-monad $T$ on locally small categories for strict monoidal
categories, and take $P$ to be the free small cocompletion KZ doctrine
on locally small categories. Note that the pseudo-$T$-algebras are
unbiased monoidal categories (equivalent to (strict) monoidal categories
\cite{leinster}) and so we may write $\textnormal{ps-}T\textnormal{-alg}\simeq\textnormal{MonCat}_{\textnormal{ps}}$
with the latter being the 2-category of monoidal categories, strong
monoidal functors and monoidal transformations. 

Given a monoidal category $\left(\mathcal{A},\varotimes\right)$ we
may define a monoidal structure on $P\mathcal{A}$ by Day's convolution
formula
\[
F\varotimes_{\textnormal{Day}}G:=\int^{a,b\in\mathcal{A}}\mathcal{A}\left(-,a\varotimes b\right)\times Fa\times Gb
\]
for small presheaves $F$ and $G$ on $\mathcal{A}$. Note that $F\varotimes_{\textnormal{Day}}G$
is then small, see \cite[Section 7]{daylack2007}. This can be shown
to give a monoidal structure by the arguments of Day \cite{dayconvolution},
equivalent to the structure of a pseudo-$T$-algebra. As the convolution
algebra structure map is separately cocontinuous (and hence $T_{P}$-cocontinuous
\cite[Prop. 2.3.2]{markextension}) we have enough of Proposition
\ref{claim} to show condition (a) of Theorem \ref{liftkzequiv} is
met.

We thus know that $T$ preserves $P$-admissible maps. This says that
if we suppose that $L\colon\mathcal{A}\to\mathcal{B}$ is $P$-admissible,
meaning that each $\mathcal{B}\left(L-,b\right)$ is a small colimit
of representables, then each 
\[
T\mathcal{B}\left(TL\boldsymbol{-},\mathbf{b}\right)=T\mathcal{B}\left[\left(L-,\cdots L-\right),\left(b_{1},\cdots,b_{n}\right)\right]=\prod_{j=1}^{n}\mathcal{B}\left(L-,b_{j}\right)
\]
 is also a small colimit of representables. Let us consider the preservation
of the admissibility of $L=y_{\mathcal{A}}$ (which is equivalent
to preservation for all $L$). This gives the following example.
\begin{prop}
Let $X,Y\colon\mathcal{A}^{\textnormal{op}}\to\mathbf{Set}$ be two
small presheaves on $\mathcal{A}$. Then 
\[
X\times Y\colon\left(\mathcal{A}\times\mathcal{A}\right)^{\textnormal{op}}\to\mathbf{Set},\qquad\left(a_{1},a_{2}\right)\mapsto X\left(a_{1}\right)\times Y\left(a_{2}\right)
\]
is a small presheaf on $\mathcal{A}\times\mathcal{A}$.\end{prop}
\begin{proof}
Note that $Ty_{\mathcal{A}}$ is $P$-admissible, and hence 
\[
TP\mathcal{A}\left(Ty_{\mathcal{A}}\boldsymbol{-},\mathbf{X}\right)\colon\left(T\mathcal{A}\right)^{\textnormal{op}}\to\mathbf{Set}
\]
 is a small presheaf on $T\mathcal{A}$ for each $\mbox{\ensuremath{\mathbf{X}}}=\left(X_{1},\cdots,X_{n}\right)$
in $TP\mathcal{A}$. In particular, if we take $\mbox{\ensuremath{\mathbf{X}}}=\left(X,Y\right)$
then
\[
\begin{aligned}TP\mathcal{A}\left(y_{\mathcal{A}}\boldsymbol{-},\mathbf{X}\right) & =\begin{cases}
TP\mathcal{A}\left[\left(y_{\mathcal{A}}-,y_{\mathcal{A}}-\right),\left(X,Y\right)\right], & \mathbf{a}\in\left(\mathcal{A}\times\mathcal{A}\right)^{\textnormal{op}}\\
\emptyset, & \textnormal{otherwise}
\end{cases}\\
 & =\begin{cases}
X\left(-\right)\times Y\left(-\right), & \mathbf{a}\in\left(\mathcal{A}\times\mathcal{A}\right)^{\textnormal{op}}\\
\emptyset, & \textnormal{otherwise}
\end{cases}
\end{aligned}
\]
is a small presheaf on $\sum_{n\in\mathbb{N}}A^{n}$ and so $X\left(-\right)\times Y\left(-\right)$
is a small presheaf on $\mathcal{A}\times\mathcal{A}$.
\end{proof}
Our results also apply to the less general setting of distributing
(co)KZ doctrines over KZ doctrines. The following is such an example.
\begin{example}
Consider the KZ doctrine for the free coproduct completion 
\[
\mathbf{Fam}_{\Sigma}\colon\mathbf{Cat}\to\mathbf{Cat}.
\]
 Here a map $L\colon\mathcal{A}\to\mathcal{B}$ is $\mathbf{Fam}_{\Sigma}$-admissible
when $\mathbf{Fam}_{\Sigma}L$ is a left adjoint; that is, when $L$
is a left multiadjoint. This is to say that for any $Z\in\mathcal{B}$
there exists a family of morphisms $\left(h_{i}\colon LX_{i}\to Z\right)_{i\in\mathcal{I}}$
which is universal in the sense that given any $k\colon LX\to Z$
there exists a unique pair $\left(i,f\right)$ with $i\in\mathcal{I}$
and $f\colon X\to X_{i}$ such that $h_{i}\cdot Lf=k$ \cite{diers}.

It is well known the free product completion $\mathbf{Fam}_{\Pi}$
distributes over this doctrine \cite[Section 8]{marm2012}. Thus,
as a consequence of Theorem \ref{liftkzequiv}, we see that if a functor
$L$ is a left multiadjoint, then the functor $\mathbf{Fam}_{\Pi}L$
is a left multiadjoint also.
\end{example}
\bibliographystyle{siam}
\bibliography{references}

\end{document}


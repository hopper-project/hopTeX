\documentclass[a4paper]{amsart}
\usepackage{graphicx,color}
\usepackage{url}
\usepackage{enumerate}

\newtheorem{theorem}{Theorem}[section]
\newtheorem{corollary}[theorem]{Corollary}
\newtheorem{lemma}[theorem]{Lemma}
\newtheorem{proposition}[theorem]{Proposition}
\newtheorem{claim}{Claim}[theorem]
\theoremstyle{definition}
\newtheorem{definition}[theorem]{Definition}
\newtheorem{problem}[theorem]{Problem}
\newtheorem{remark}[theorem]{Remark}
\newtheorem{example}[theorem]{Example}

\numberwithin{equation}{section}

\begin{document}
\title[The covariogram  and Fourier-Laplace transform]{The covariogram
and Fourier-Laplace\\ transform in ${\mathbb{C}}^n$}
\author{Gabriele Bianchi}
\address{Dipartimento di Matematica e Informatica, Universit\`a di Firenze, 
Viale Morgagni 67/A, Firenze, Italy I-50134}
\email{gabriele.bianchi@unifi.it}
\subjclass[2010]{Primary 42B10, 52A20; Secondary 32A50, 32A60, 60D05}
\keywords{autocorrelation, convex body, covariogram, cross covariogram, Fourier transform, geometric tomography, phase retrieval, Pompeiu problem, set covariance}
\date{\today}

\begin{abstract}
The covariogram $g_{K}$ of a convex body $K$ in ${\mathbb{R}}^n$ is the function which associates to each
$x\in{\mathbb{R}}^n$ the volume of the intersection of $K$ with $K+x$. Determining $K$ from the knowledge of $g_K$ is known as the covariogram problem. It is equivalent to determining the characteristic function $1_K$ of $K$ from the modulus of its Fourier-Laplace transform ${\widehat{{{1_K}}}}$, a particular instance of the phase retrieval problem.

We prove that when $K$ is sufficiently smooth and in any dimension $n$, $K$ is determined by $g_K$ in the class of sufficiently smooth bodies. We also prove the corresponding result for the problem of determining two planar convex bodies $H$ and $K$ from the knowledge of the area of the intersection of $H$ with the translates of $K$ (i.e. from the cross covariogram).
The proofs use in an essential way a study  of the asymptotic behavior at infinity of the zero set  of the Fourier-Laplace  transform ${\widehat{{{1_K}}}}$  in ${\mathbb{C}}^n$ done by T.~Kobayashi.

We also discuss the relevance for the covariogram problem of known determination results for the phase retrieval problem and the difficulty of finding explicit geometric conditions on $K$ which grant that the entire function ${\widehat{{{1_K}}}}$ cannot be factored as the product of non-trivial entire functions. This shows a connection between the covariogram problem and the Pompeiu problem.
\end{abstract}

\maketitle

\section{Introduction}
Let $H$ and $K$ be convex bodies in  ${\mathbb{R}}^n$, $n \geq 2$, and let ${{\mathop{\lambda_n}}}$ stand for the $n$-dimensional Lebesgue measure.
The \emph{cross covariogram} ${g_{{H,K}}}$ of $H$ and $K$ is the
function defined for $x\in{\mathbb{R}}^n$ by
\[
{g_{{H,K}}}(x)={{\mathop{\lambda_n}}}(H\cap (K+x)).
\]
This function coincides with the convolution of  the characteristic function ${1}_H$ of $H$ with the characteristic function ${1}_{-K}$ of the reflection of $K$ in the origin, that is,
\begin{equation}\label{convoluzione}
g_{H,K} ={1}_H\ast {1}_{-K}.
\end{equation}

The function $g_{K,K}$  was introduced by G.~Matheron in his book~\cite[Section~4.3]{Matheron-1975} on random sets, is denoted by $g_K$ and is called \emph{covariogram} or \emph{set covariance} of $K$. Observe that $g_K$ is clearly unchanged by  translations or reflections of $K$ (in this paper the term \emph{reflection}  always means reflection in a point).
The data provided by \( g_K \) can be interpreted in several ways within different contexts, using purely geometric, functional-analytic and probabilistic terminology.
As a result, covariograms of convex bodies and other sets appear naturally in various research areas including convex geometry, image analysis, geometric shape and pattern matching, phase retrieval in Fourier analysis, crystallography and geometric probability. See Baake and Grimm~\cite{BaakeGrimm}, Bianchi, Gardner and Kiderlen~\cite{BiGaKi11} and references therein, Matheron~\cite{Matheron-1975} and Schymura \cite{Schymura-2011}.

The following  problem  was posed by G.~Matheron in 1986 and has received much attention in recent years.

\smallskip

\textbf{Covariogram Problem.} \emph{Does the covariogram determine a
convex body, among all convex bodies, up to translations and
reflections?}

\smallskip

The answer to the covariogram problem is positive for every planar convex body (see Averkov and Bianchi~\cite{averkov-bianchi-2009}), it is positive for convex polytopes in ${\mathbb{R}}^3$ (see Bianchi~\cite{Bianchi-2009-polytopes}) but the case of a general convex body in ${\mathbb{R}}^3$ is still open, and in every dimension $n\geq4$ there are examples of nondetermination, as well as positive results in some subclasses of the class of convex bodies (see Goodey, Schneider and Weil~\cite{Goodey-Schneider-Weil-1997}, Bianchi~\cite{Bianchi-2005}).

The \emph{Phase Retrieval Problem} asks for the determination of  a function $f\in L^2({\mathbb{R}}^n)$ with compact support from the knowledge of the modulus of its of Fourier transform ${\widehat{{{f}}}}(x)$ for $x\in{\mathbb{R}}^n$, up to the inherent ambiguities. 
In view of \eqref{convoluzione} the Fourier transform of $g_K$ coincides with $|{\widehat{{{1_K}}}}|^2$. Therefore the covariogram problem coincides with a particular instance of the phase retrieval problem where the unknown function $f$ is the characteristic function of a convex body and this function has to be determined by the modulus of its Fourier transform within the class of characteristic functions of convex bodies.

The Fourier-Laplace transform plays an important role in the uniqueness aspects of the phase retrieval problem. When $f\in L^2({\mathbb{R}}^n)$ has compact support then ${\widehat{{{f}}}}$ is an entire function (i.e. is holomorphic on the entire ${\mathbb{C}}^n$).
Sanz and Huang \cite{Sanz-Huang-1984}, Barakat and Newsam~\cite{Barakat-Newsam-1984} and Stefanescu~\cite{Stefanescu-1985} prove that the non-uniqueness in the determination  is related to the possibility of factoring ${\widehat{{{f}}}}$ as the product of two non-trivial entire functions.
What is the significance of these results for the covariogram problem?
\emph{Is it possible to find explicit geometric conditions on a convex body $K$ that grant that ${\widehat{{{1_K}}}}$ cannot be factored?}
This seems to be a difficult problem.
For instance the subproblem consisting in understanding for which $K$  the function ${\widehat{{{1_K}}}}$ can be factored as the product of a polynomial $P({{\zeta}})$ and of an entire function is equivalent to understanding for which $K$ a certain differential problem has a solution with compact support $K$. 
When $P({{\zeta}})={{\zeta}}_1^2+\dots+{{\zeta}}_n^2-c$ this problem has been studied in many papers, because proving that a solution to this problem exists for some $c>0$ only if $K$ is a ball is equivalent to the famous Pompeiu problem, a long-standing open problem in integral geometry. In Section~\ref{sec_phase_retr} we explain all this in detail.

In this paper we use different aspects of the Fourier-Laplace transform. The zero set $\{{{\zeta}}\in{\mathbb{C}}^n : {\widehat{{{1_K}}}}({{\zeta}})=0\}$ has been studied extensively, both for the role that it plays in attempts to solve the  Pompeiu problem (see, for instance, Berenstein~\cite{Berenstein-1980} and Garofalo and Segala~\cite{Garofalo-Segala-1991}) and to understand which geometric information it carries (see Benguria et al.~\cite{Benguria-Levitin-Parnovski-2009}, Kobayashi~\cite{Kob1, Kob2}).
In particular, Kobayashi~\cite{Kob1, Kob2} studies the asymptotic behavior at infinity of this zero set in the case of a $C^\infty_+$ convex body (the subscript $+$ means that ${{\partial}} K$ is assumed to have Gauss curvature positive everywhere), and relates this behavior to the width function of $K$ and to the Gauss curvature of ${{\partial}} K$. See~Theorem~\ref{teo_kobayashi} of this paper for the precise statement. 

In this paper we are interested in the consequences of this study for the covariogram problem. To this end we first prove Kobayashi's result under lower regularity assumptions (see Theorem~\ref{teo_kobayashi_cm} for the precise statement). Then we use this extension  as a key to prove a positive answer to the covariogram problem for smooth convex bodies in every dimension.
\begin{theorem}\label{teo_cov_smooth}Let $n\geq 2$ and define ${{r(n)}}=8$ when $n=2,4,6$, ${{r(n)}}=9$ when $n=3,5,7$
and ${{r(n)}}=[(n-1)/2]+5$ when $n\geq8$. Let $H$ and $K$  be  convex bodies in ${\mathbb{R}}^n$ of class $C^{{r(n)}}_+$. Then $g_H=g_K$ implies that $H$ and $K$ coincide, up to translations and reflections.
\end{theorem}
The class of $C^{{r(n)}}_+$ convex bodies is the one where we are able to extend Kobayashi's result.

The covariogram problem has been solved more that ten years ago in the class of planar $C^2_+$ convex bodies~\cite{Bianchi-Segala-Volcic-2002}, but it is still open for $n$-dimensional $C^2_+$ convex bodies, even when $n=3$.
It can be proved that if $K\subset{\mathbb{R}}^n$ is a $C^2_+$ convex body then, for each $u\in {S^{n-1}}$, $g_K$ provides the non-ordered pair $\{{\tau}_K(u),{\tau}_K(-u)\}$ consisting in the Gauss curvature of ${{\partial}} K$ at the points of $\partial K$ with outer normal $u$ and $-u$. Thus if $H$ is in the class $C^2_+$ and $g_H=g_K$, the continuity of the curvature implies that given any component $V$ of $\{u\in {S^{n-1}} : {\tau}_K(u)\neq {\tau}_K(-u)\}$, after possibly a reflection of $H$, we have
\begin{equation}\label{curvatures_locally_equal}
{\tau}_H(v)={\tau}_K(v)\quad\text{for each $v\in V$.}
\end{equation}
If~\eqref{curvatures_locally_equal} were true for each $v\in{S^{n-1}}$ then $H$ and $K$ would coincide, up to a translation, by the uniqueness part in Minkowski's Theorem \cite[Th. 7.2.1]{Sc}.
However, a priori the reflection that makes~\eqref{curvatures_locally_equal} valid may vary from component to component. 
An important difference between $n=2$ and $n>2$, something which has been used in overcoming this difficulty in ${\mathbb{R}}^2$, is the fact that in the plane \eqref{curvatures_locally_equal} implies that a portion of ${{\partial}} H$ is a translation of a portion of ${{\partial}} K$, while this is not true anymore in ${\mathbb{R}}^n$ when $n>2$.
Kobayashi's result is the key to prove that the reflection that makes~\eqref{curvatures_locally_equal} valid does not vary from component to component. In order to explain this we observe that~\eqref{convoluzione}, with $H=K$, implies
\[
{\widehat{{{g_K}}}}({{\zeta}})={\widehat{{{1_K}}}}({{\zeta}}){\overline{{{\widehat{{{1_K}}}}\left({\overline{{{\zeta}}}}\right)}}}
\]
(the bar denotes conjugation) because ${\widehat{{{1_{-K}}}}}(\zeta)={\overline{{{\widehat{{{1_{K}}}}}\left({\overline{\zeta}}\right)}}}$. Thus
\[
 \{{{\zeta}}\in{\mathbb{C}}^n : {\widehat{{{g_K}}}}({{\zeta}})=0\}=\{{{\zeta}}\in{\mathbb{C}}^n : {\widehat{{{1_K}}}}({{\zeta}})=0\}\cup{\overline{{\{{{\zeta}}\in{\mathbb{C}}^n : {\widehat{{{1_K}}}}({{\zeta}})=0\}}}}.
\]
Determining $K$ from $g_K$ can be proved equivalent to resolve the ambiguity in determining  $\{{\widehat{{{1_K}}}}=0\}$ from $\{{\widehat{{{1_K}}}}=0\}\cup{\overline{{\{{\widehat{{{1_K}}}}=0\}}}}$. 
In this context the previous ambiguity in determining the reflection can be seen in the following terms: given a  component of $\{{\widehat{{{1_K}}}}=0\}\cap\{{{\zeta}}\in{\mathbb{C}}^n : {\operatorname{Im}} {{\zeta}}\neq0\}$ then, after possibly a reflection of $H$, this component is contained in $\{{\widehat{{{1_H}}}}=0\}$. We have to prove that this reflection of $H$ does not vary from component to component.
The key ingredient in proving this fact when the body is regular enough is the description of the asymptotic behavior of $\{{\widehat{{{1_K}}}}=0\}$ given in  Kobayashi's Theorem~\ref{teo_kobayashi} and  the analyticity  of the maps involved in this description.

Kobayashi's Theorem~\ref{teo_kobayashi} and our extension Theorem~\ref{teo_kobayashi_cm} enable us to prove also a result regarding the equivalent of the covariogram problem for the cross-covariogram. Here one asks for the determination of a pair $(H,K)$ of convex bodies from $g_{H,K}$, up to the inherent ambiguities. (We express these ambiguities with the term \emph{trivial associates}.)
Bianchi~\cite{B4} solves this problem in the class of pairs of convex polygons. It proves that there are two families of different pairs of parallelograms with the same cross-covariogram and (surprisingly, in our opinion) that  $g_{H,K}$ determines $(H,K)$ except for the members of these families (see Theorem~\ref{teo_cov_congiunto_poligoni} in  Section~\ref{sec_cross_cov} for the precise statement).
In this paper we are able to prove that no counterexample exists among pairs of sufficiently regular planar convex bodies.
\begin{theorem}\label{teo_cross_cov_smooth}Let $H, K, H'$ and $K'$ be  planar convex bodies of class $C^8_+$. Then
$g_{H,K}=g_{H',K'}$ implies that $(H,K)$ and $(H',K')$ are trivial associates.
\end{theorem}

We conclude the introduction by mentioning another contribution to the covariogram problem proved using results of the theory of functions of several complex variables.  A natural question is the following one.
\smallskip

\emph{Is it possible to read in $g_E$ symmetry properties of the set $E$?}
\smallskip

Note that $g_E$ is always an even function, independently of any symmetry property of $E$. Lawton~\cite[Corollary~1]{Lawton-1981} implies the following result.
\begin{theorem}[Lawton~\cite{Lawton-1981}]\label{teo_radial_symmetry}
Let $n\geq 2$ and $E\subset{\mathbb{R}}^n$ be a regular compact set and assume that  $g_E$ is radially symmetric. Then a translation of $E$ is radially symmetric and $E$ is determined by $g_E$, up to translation  and reflection, in the class of regular compact sets.
\end{theorem}
W.~Lawton proves the corresponding result for real-valued $L^2({\mathbb{R}}^n)$ functions with compact support as a consequence of a representation formula for entire functions of exponential type such that the modulus of their restriction to ${\mathbb{R}}^n$ is radially symmetric and in $L^2({\mathbb{R}}^n)$.

\section{Definitions, notations and preliminaries}\label{sec_definitions}
\subsection{Basic definitions and notation}
As usual, ${S^{n-1}}$ denotes the unit sphere and  $o$ the
origin in the Euclidean $n$-space ${\mathbb{R}}^n$. If $x,y\in{\mathbb{R}}^n$, then $\left<x,y\right>$ is the scalar product of $x$ and $y$, while $|x|$ is the norm of $x$. 
If ${{\zeta}}\in{\mathbb{C}}^n$ and ${{\zeta}}=x+{\operatorname{i}} y$, with $x,y\in{\mathbb{R}}^n$, then ${\operatorname{Re}}{{\zeta}}$ and ${\operatorname{Im}} {{\zeta}}$ denote respectively $x$ and $y$. Moreover $|{{\zeta}}|=(|{\operatorname{Re}}{{\zeta}}|^2+|{\operatorname{Im}} {{\zeta}}|^2)^{1/2}$ denotes the norm of $z$. If $u\in{S^{n-1}}$, then $u^{\perp}$ is the $(n-1)$-dimensional subspace orthogonal to $u$. 
For ${{\delta}}>0$ and $x\in{\mathbb{R}}^n$, $B(x,{{\delta}})$ denotes $\{y\in{\mathbb{R}}^n : |y-x|<{{\delta}}\}$. When ${{\zeta}}\in{\mathbb{C}}^n$, $B({{\zeta}},{{\delta}})$ is defined similarly. We write ${{\lambda}}_n$ for $n$-dimensional Lebesgue measure in ${\mathbb{R}}^n$. We define $\omega_n$ the surface area of the unit ball in ${\mathbb{R}}^n$.

For $t\in{\mathbb{R}}$ let $t_+=\max\{t,0\}$, $t_-=\max\{-t,0\}$ and let $[t]$ denote the integer part of $t$.

We denote by $\partial E$,  ${{\operatorname{int}}} E$,  ${{\operatorname{cl}}} E$, and $1_E$ the {\it boundary}, {\it
interior}, \emph{closure}, and {\em characteristic function} of a set $E$ in ${\mathbb{R}}^n$, respectively. A compact set $E\subset{\mathbb{R}}^n$ is \emph{regular} if $E={{\operatorname{cl}}}\  {{\operatorname{int}}} E$. A set is {\it $o$-symmetric} if it is centrally symmetric, with center at the origin.
If $E$ and $F$ are sets in ${\mathbb{R}}^n$, then $E+F=\{x+y: x\in E, y\in F\}$ denotes their {\em Minkowski sum}.

Given a function $f$ defined on a subset of ${\mathbb{R}}^n$, ${{\operatorname{supp}}} f$, $\nabla f$ and $D^2 f$ denote its support, its gradient and its Hessian, respectively. We say that $f\in{\mathbb{C}}^\infty_0({\mathbb{R}}^n)$ if $f$ is $m$-times differentiable for each $m\in{\mathbb{N}}$ and ${{\operatorname{supp}}} f$ is compact.

\subsection{Convex geometry and covariogram}
A {\em convex body} in ${\mathbb{R}}^n$ is a compact convex set with nonempty interior. The treatise of
Schneider \cite{Sc} is an excellent general reference for convex geometry.
The function
$$h_K(u)=\max\{\left<u,y\right>: y\in K\},$$
for $u\in{\mathbb{R}}^n$, is the {\it support function} of $K$ and
$${{{w}}}_K(u)=h_K(u)+h_K(-u),$$
its {\it width function}.  Any $K \in {\mathcal K}^n$
is uniquely determined by its support function.

We say that a convex body $K$ is in the class $C^m$, for $m\in{\mathbb{N}}$, if it is a $m$-differentiable manifold.  We say that $K\in{\mathbb{C}}^m_+$, for $m\geq2$, if $K\in C^m$ and the \emph{Gauss curvature} of ${{\partial}} K$ is positive everywhere.    We say that  $K\in{\mathbb{C}}^\infty_+$ if $K\in C^m_+$ for each $m\in{\mathbb{N}}$. 

When $K\in{\mathbb{C}}^2_+$, $\nu_K:{{\partial}} K\to{S^{n-1}}$ denotes the \emph{Gauss map} and ${\tau}_K(u)$ denotes the Gauss curvature of ${{\partial}} K$ at the point  $\nu^{-1}(u)$ on ${{\partial}} K$ with outer normal $u\in{S^{n-1}}$.
Let ${{\mathop{W}}}_K(u)$ denote the \emph{Weingarten map}, i.e. the differential of the {Gauss map} $\nu_K$ of ${{\partial}} K$ computed at $\nu^{-1}_K(u)$.
The eigenvalues of ${{\mathop{W}}}_K(u)$ are the principal curvatures of ${{\partial}} K$ at $\nu^{-1}_K(u)$ and their product equals the Gauss curvature  ${\tau}_K(u)$.

The covariogram and the cross covariogram have been defined in the introduction.

Let $H$, $H'$, $K$ and $K'$ be convex bodies in  ${\mathbb{R}}^n$.
The translation of $H$ and $K$ by the same vector, and the substitution of $H$ with $-K$ and of $K$ with $-H$, leave $g_{H,K}$ unchanged.  We call $(H,K)$ and $(H',K')$ \emph{trivial associates} when one pair is obtained by the other one via a combination of the two operations above, that is, when either $(H,K)=(H'+x,K'+x)$ or $(H,K)=(-K'+x,-H'+x)$, for some $x\in {\mathbb{R}}^n$.

We have $g_{H,K}(x)=0$  if and only if $x\notin H+(-K)$, so the support of $g_{H,K}$ is $H+(-K)$. Since the support function is linear with respect to Minkowski addition we have
\begin{equation}\label{width_of_support}
{{{w}}}_{{{\operatorname{supp}}} g_{H,K}}={{{w}}}_H+{{{w}}}_K.
\end{equation}

\subsection{Fourier-Laplace  and Radon transform}
An \emph{entire function} is a complex-valued function that is holomorphic over the whole ${\mathbb{C}}^n$. An entire function $f$ is of \emph{exponential type} if there exist $a, b\in{\mathbb{R}}$ and $m\in\mathbb{Z}$ such that $|f({{\zeta}})|\leq a(1+|{{\zeta}}|)^m e^{b |{\operatorname{Im}} {{\zeta}}|}$, for each ${{\zeta}}\in{\mathbb{C}}^n$.

The \emph{Fourier-Laplace transform} of a function $f\in L^2({\mathbb{R}}^n)$ with compact support is defined for $\zeta\in{\mathbb{C}}^n$ as
\[
{\widehat{{{f}}}}(\zeta)=\int_{{\mathbb{R}}^n}e^{{\operatorname{i}} x\cdot \zeta}f(x)\ dx.
\]
By the Paley-Wiener Theorem ${\widehat{{{f}}}}$ is an entire function of exponential type whose restriction to ${\mathbb{R}}^n$ belongs to $L^2$ if and only if $f\in L^2({\mathbb{R}}^n)$ and has compact support. The version of this theorem for distributions asserts that ${\widehat{{{f}}}}$ is an entire function of exponential type  if and only if $f$ is a distribution with compact support. See \cite[Theorem~7.23]{Rudin-91}.   Distributions will enter this paper only very marginally and we refer to Rudin~\cite{Rudin-91} for their definition.

Taking Fourier transforms in \eqref{convoluzione} and using the identity
\begin{equation}\label{riflessione_in_Cn}
{\widehat{{{1_{-K}}}}}(\zeta)={\overline{{{\widehat{{{1_{K}}}}}\left({\overline{\zeta}}\right)}}},
\end{equation}
valid for every ${{\zeta}}\in{\mathbb{C}}^n$, we obtain the relation
\begin{equation}\label{convoluzione_in_cn}
{\widehat{{{g_K}}}}(\zeta)
={\widehat{{{1_K}}}}(\zeta)\,{\overline{{{\widehat{{{1_{K}}}}}\left({\overline{\zeta}}\right)}}}.
\end{equation}

Given a convex body $K$ in ${\mathbb{R}}^n$, $t\in{\mathbb{R}}$ and $u\in {S^{n-1}}$,  we denote by $S_K(u,t)$ the \emph{Radon transform} of $1_K$
\begin{equation*}
S_K(u,t)={{\lambda}}_{n-1}\left(K\cap(u^\perp+t)\right).
\end{equation*}

\section{The information that is easy to read in the covariogram of a $C^2_+$ convex body} \label{sec_c2plus}
This section is devoted to the following result.
\begin{proposition}\label{teo_information_c2+_bodies}
Let $K$ be a convex body of class $C^2_+$ in ${\mathbb{R}}^n$.
\begin{enumerate}[(I)]
\item\label{prop_sum_reverse_weingarten} 
The width function ${{{w}}}_K$ determines 
\begin{equation*}
{{\operatorname{trace}}}\left({{\mathop{W}}}_K(u)^{-1}\right)+{{\operatorname{trace}}}\left({{\mathop{W}}}_K(-u)^{-1}\right)
\end{equation*}
for each $u\in {S^{n-1}}$. In particular when $n=2$ it determines 
\begin{equation}\label{equal_sum_radii_curvature}
\frac1{{\tau}_K(u)}+\frac1{{\tau}_K(-u)}.
\end{equation}
\item\label{prop_comport_asint_covario} 
Let $u\in{S^{n-1}}$ and $p=\nu^{-1}_K(u)-\nu^{-1}_K(-u)$. The knowledge of $g_K$ in a neighborhood of $p$ determines
\[
{{\mathop{W}}}_K(u)^{-1}+{{\mathop{W}}}_K(-u)^{-1}\quad\text{and}\quad \det\left({{\mathop{W}}}_K(u)+{{\mathop{W}}}_K(-u)\right).
\]
In particular, it determines
\begin{equation}\label{product_gauss_curv}
 {\tau}_K(u){\tau}_K(-u).
\end{equation}
\item\label{prop_sum_radii_curvature}
The knowledge of $g_K$ in a neighborhood of $o$ determines the expression in~\eqref{equal_sum_radii_curvature} for each $u\in{S^{n-1}}$.
\item\label{nonordered_curvatures}
The covariogram $g_K$ determines $\{{\tau}_K(u),{\tau}_K(-u)\}$ for each $u\in{S^{n-1}}$.
\end{enumerate}
\end{proposition}

The point $p$ in Assertion~\eqref{prop_comport_asint_covario} is the point of the boundary of ${{\operatorname{supp}}} g_K$ with outer normal $u$, by the identity ${{\operatorname{supp}}} g_K=K+(-K)$ and  \cite[Th. 1.7.5(c)]{Sc}. Studying the  behavior of $g_K$ near $p$ is equivalent to studying the behavior of the volume of $K\cap(K+x)$ for $x$ such that $K\cap(K+x)$ is contained in a small neighborhood of $\nu^{-1}_K(u)$ and its boundary consists of a portion of  ${{\partial}} K$ near $\nu^{-1}_K(u)$ and of (a translation of) a portion of ${{\partial}} K$ near $\nu^{-1}_K(-u)$.

The next two lemmas are needed to prove Assertion~~\eqref{prop_comport_asint_covario}.
\begin{lemma}\label{lem_identita_matrici}
 If $A$ and $B$ are non-singular matrices so that $A+B$ is also non-singular then $A^{-1}+B^{-1}$ is non-singular,
\begin{gather}
 A-A(A+B)^{-1}A=B(A+B)^{-1}A=A(A+B)^{-1}B=\left(A^{-1}+B^{-1}\right)^{-1},\label{identita_matrici}
 \intertext{and}
\det\left(\left(A^{-1}+B^{-1}\right)^{-1}\right)=\frac{\det A\det B}{\det\left(A+B\right)}.\label{identita_determinanti}
\end{gather}
\end{lemma}
\begin{proof}
We have
\begin{align*}
A-A(A+B)^{-1}A=&A(A+B)^{-1}(A+B)-A(A+B)^{-1}A\\
=&A(A+B)^{-1}\left(A+B-A\right) \\
=&A(A+B)^{-1}B.
\end{align*}
In a similar way, substituting the first summand $A$ in $A-A(A+B)^{-1}A$ by $(A+B)(A+B)^{-1}A$, one proves  $A-A(A+B)^{-1}A=B(A+B)^{-1}A$. These two identities imply
\begin{align*}
\left( A-A(A+B)^{-1}A\right)\left(A^{-1}+B^{-1}\right)
=&\left( A-A(A+B)^{-1}A\right)A^{-1}+\\
&\quad\quad\quad+\left( A-A(A+B)^{-1}A\right)B^{-1}\\
=&B(A+B)^{-1}AA^{-1}+A(A+B)^{-1}BB^{-1}\\
=&(A+B)(A+B)^{-1}=I
\end{align*}
and also $\left(A^{-1}+B^{-1}\right)\left( A-A(A+B)^{-1}A\right)=I$. This proves~\eqref{identita_matrici}. The last equality in~\eqref{identita_matrici} and standard properties of the determinant imply~\eqref{identita_determinanti}.
\end{proof}

\begin{lemma}\label{lem_comport_asintot_covario}
Let $A$, $B$ be symmetric $(n-1) \times (n-1)$ positive-definite matrices. Let $t\in{\mathbb{R}}$, $t>0$, and  $q\in{\mathbb{R}}^{n-1}$ be such that $2t-\left<\left(A^{-1}+B^{-1}\right)^{-1}q, q\right>\geq0$.
Let $f_1$, $f_2:{\mathbb{R}}^{n-1}\to{\mathbb{R}}$  be the quadratic functions
\[
f_1(x)=t-\frac{1}{2}\left<A(x-q),x-q\right>,\quad f_2(x)=\frac{1}{2}\left<Bx,x\right>.
\]
 Then the volume  of the region in ${\mathbb{R}}^{n}$ bounded by the graphs of $f_1$ and $f_2$  is
\begin{multline*}
 {{\lambda}}_n\left\{(x,x')\in{\mathbb{R}}^{n-1}\times{\mathbb{R}} : f_2(x)\leq x'\leq f_1(x)\right\}\\
 =\frac{\omega_{n-1} 2^{(n+1)/2}}{n^2-1}
 \frac{ \left( 2t-\left<\left(A^{-1}+B^{-1}\right)^{-1}q, q\right>\right)^{(n+1)/2}}
{\sqrt{\det(A+B)}}.
\end{multline*}
\end{lemma}
\begin{proof}
We have
\begin{equation}\label{espressione1}
f_1(x)-f_2(x)=t-\frac{1}{2}\Big( \left<(A+B)x, x\right>-\left<Ax, q\right>-\left<Aq, x\right>+\left<Aq, q\right> \Big).
\end{equation}
Let us consider the expression  in parentheses in the right hand side of \eqref{espressione1}. By adding and subtracting $\left<Aq, (A+B)^{-1}Aq\right>$, by rewriting $\left<Ax, q\right>$ as $\left<(A+B)x,(A+B)^{-1}Aq\right>$ (a consequence of the symmetry of $A$ and $B$) and by regrouping some terms, we  obtain
\begin{multline*}
 \left<(A+B)x, x\right>-\left<Ax, q\right>-\left<Aq, x\right>+\left<Aq, q\right>=\\
 =\left<(A+B)y, y\right>+\left<\left(A-A(A+B)^{-1}A\right)q, q\right>,
\end{multline*}
where $y=x-(A+B)^{-1}Aq$.  This formula and Lemma \ref{lem_identita_matrici} imply
\[
  f_1(x(y))-f_2(x(y))=s-\frac1{2}\left<(A+B)y, y\right>,
\]
where  $s=t-(1/2)\left<\left(A^{-1}+B^{-1}\right)^{-1}q, q\right>$.

Let $V(q,t)$ denote the volume that we wish to compute. It is
\begin{align*}
 V(q,t)
=&\int_{\{x\in{\mathbb{R}}^{n-1} : f_1(x)-f_2(x)\geq0\}}f_1(x)-f_2(x)\ dx\\
=&\int_{\{y\in{\mathbb{R}}^{n-1} : 2s-\left<(A+B)y, y\right>\geq0\}}s-\frac1{2}\left<(A+B)y, y\right>\ dy.
\end{align*}
Since  an orthogonal transformation does not change $V(q,t)$, we may assume that the symmetric matrix $A+B$ is diagonal. Let ${{\lambda}}_i$ be the $i$-th element of the diagonal of $(A+B)$ and let $w=(\sqrt{{{\lambda}}_1}y_1,\dots,\sqrt{{{\lambda}}_{n-1}}y_{n-1})$. We have
\begin{align*}
 V(q,t)
=&\frac1{\sqrt{\det(A+B)}}\int_{\{w\in{\mathbb{R}}^{n-1}:|w|^2\leq 2s\}}s-\frac{|w|^2}{2}\ dw\\
=&\frac{\omega_{n-1}}{\sqrt{\det(A+B)}}\int_0^{\sqrt{2s}}r^{n-2}\left(s-\frac{r^2}2\right)\,dr\\
=&\frac{\omega_{n-1} 2^{(n+1)/2}}{n^2-1} \frac{s^{(n+1)/2}}{\sqrt{\det(A+B)}}.
\end{align*}
Writing $s$ in terms of $q$ and $t$ concludes the proof.
\end{proof}

\begin{proof}[Proof of Proposition~\ref{teo_information_c2+_bodies}]

\emph{Assertion~\eqref{prop_sum_reverse_weingarten}.} This is a consequence of Theorems 3.3.2 and 3.3.5 in~\cite{Gar95ed2}.

\emph{Assertion~\eqref{prop_comport_asint_covario}.} Let us compute the  asymptotic expansion of $g_K$ near $p$. Changing, if necessary, the coordinate system we may assume  $u=(0,\dots,0,-1)$ and $\nu_K^{-1}(u)=o$. 
Let $\nu_K^{-1}(-u)=(a,s)\in{\mathbb{R}}^{n-1}\times{\mathbb{R}}$. We have $p=-(a,s)$. Let $A$ and $B$ be the matrices representing respectively ${{\mathop{W}}}_K(-u)$ and ${{\mathop{W}}}_K(u)$ in an orthonormal basis in $u^\perp={\mathbb{R}}^{n-1}$.

We prove that when  $q\in{\mathbb{R}}^{n-1}$ and $t>0$ are such that
\begin{equation}\label{condition_t_and_q}
2t-\left<\left(A^{-1}+B^{-1}\right)^{-1}q\cdot q\right>>0,
\end{equation}
we have
\begin{equation}\label{svil_asintot}
 g_K\big(q-a,t-s\big)=c\frac
{ \left( 2t-\left<\left(A^{-1}+B^{-1}\right)^{-1}q, q\right>\right)^{(n+1)/2}}
{\sqrt{\det(A+B)}}
\left(1+\epsilon(q,t)\right),
\end{equation}
where $c=\omega_{n-1}2^{(n+1)/2}/(n^2-1)$ and
\[
 \lim_{\substack{(q,t)\to0\\ \text{\eqref{condition_t_and_q} holds true}}} \epsilon(q,t)=0.
\]
The boundary of $K\cap (K+(q-a,t-s))$, the set whose volume is measured by $g_K\big(q-a,t-s\big)$, consists of a portion of ${{\partial}} K$ near $(a,s)$ translated by the vector $(q-a,t-s)$ and of a portion of ${{\partial}} K$ near $o$.
Since $K$ is sufficiently smooth, $\partial K$ can be approximated in a neighborhood of $(a,s)$ (up to terms of higher order) by the graph of the paraboloid $\{(x,x')\in{\mathbb{R}}^{n-1}\times{\mathbb{R}} : x'=s-\left<A(x-a),(x-a)\right>\}$.
Similarly, $\partial K$ can be approximated in a neighborhood of $o$ (up to terms of higher order) by the graph of the paraboloid $\{(x,x')\in{\mathbb{R}}^{n-1}\times{\mathbb{R}} : x'=\left<Bx, x\right>\}$.

For $q\in{\mathbb{R}}^{n-1}$ and $t>0$ sufficiently small and satisfying~\eqref{condition_t_and_q}, $K\cap (K+(q-a,t-s))$ is contained in
\[
 \{(x,x'): \left<B_1(q,t)x, x\right>\le x'\le t-\left<A_1(q,t)(x-q), x-q\right> \}
\]
and contains
\[
 \{(x,x'): \left<B_2(q,t)x, x\right>\le x'\le t-\left<A_2(q,t)(x-q), x-q\right> \},
\]
where $A_1(q,t)$, $A_2(q,t)$, $B_1(q,t)$ and $B_2(q,t)$ are symmetric positive-definite matrices such that
$A_1(q,t)<A <A_2(q,t)$, $B_1(q,t)<B <B_2(q,t)$,
\[
 \lim_{q\rightarrow o\,,t\rightarrow 0^+} A_1(q,t)=\lim_{q\rightarrow o\,,t\rightarrow 0^+} A_2(q,t)=A,
\]
and
\[
 \lim_{q\rightarrow o\,,t\rightarrow 0^+} B_1(q,t)=\lim_{q\rightarrow o\,,t\rightarrow 0^+} B_2(q,t)=B.
\]
By Lemma~\ref{lem_comport_asintot_covario} the volume  of these two sets is
\[
c\frac
{ \left( 2t-\left(A_i^{-1}+B_i^{-1}\right)^{-1}q\cdot q\right)^{(n+1)/2}}
{\sqrt{\det(A_i+B_i)}},
\]
$i=1,2$. Since the difference between the previous expression for $i=1$ and that for $i=2$ is ${{\operatorname{o}}}(2t+|q|^2)$, as $q$ tends to $o$ and $t$ tends to $0^+$, we have \eqref{svil_asintot}. This asymptotic expansion proves the first claim of the proposition.

To prove the last claim it suffices to observe that ${\tau}_K(u)=\det B$, ${\tau}_K(-u)=\det A$ and to apply~\eqref{identita_determinanti}.

\emph{Assertion~\eqref{prop_sum_radii_curvature}.} Matheron~\cite[p.~86]{Matheron-1975}  proves that for each $v\in{S^{n-1}}$ we have
\[
\frac{\partial^+ g_K}{\partial v}(o)=-{{\lambda}}_{n-1}\left(K|v^\perp\right),
\]
where $\partial^+ /\partial v$ denotes left directional derivative  in direction $v$, and $K|v^\perp$ denotes  the orthogonal projection of $K$ on $v^\perp$.
\cite[Theorem 3.3.2]{Gar95ed2} proves that the knowledge of ${{\lambda}}_{n-1}\left(K|v^\perp\right)$ for each $v\in{S^{n-1}}$ determines  the expression in~\eqref{equal_sum_radii_curvature}.

\emph{Assertion~\eqref{nonordered_curvatures}.} The  expressions in~\eqref{equal_sum_radii_curvature} and~\eqref{product_gauss_curv} determine $\{{\tau}_K(u), {\tau}_K(-u)\}$.
\end{proof}

\section{Proof of Kobayashi result under lower regularity assumption} \label{sec_kobayashi_cm}
Let 
 \[
S=\{z\in{\mathbb{C}}^n : z={{\zeta}} u, \text{ with }{{\zeta}}\in{\mathbb{C}}, u\in {S^{n-1}}\}.
 \]
In $S$ we identify ${{\zeta}} u$ and $(-{{\zeta}})(-u)$, for each ${{\zeta}}\in{\mathbb{C}}$ and $u\in{S^{n-1}}$. 
Let
\[
{{\mathcal Z}}(K)=\{\zeta\in{\mathbb{C}}^n\ :\ {\widehat{{{1_K}}}}(\zeta)=0\}.
\]

\begin{theorem}[T.~Kobayashi \cite{Kob1}]\label{teo_kobayashi} Let $S$ be defined as above. Let $K$ be a convex body in ${\mathbb{R}}^n$ of class  $C^\infty_+$. Then there exists  a positive integer $m(K)$ such that 
\[
 {{\mathcal Z}}(K)\cap S=\left( \bigcup_{m=m(K)}^\infty{{\mathcal Z}}_m(K) \right)\bigcup C(K),
\]
where the union is disjoint, $C(K)$ is a bounded set and, for each integer $m\geq m(K)$, ${{\mathcal Z}}_m(K)$ is analytically diffeomorphic to ${S^{n-1}}$.
More precisely, for each integer $m\geq m(K)$, there exists an analytic map $F_{m,K}:{S^{n-1}}\to{\mathbb{C}}$ such that
\begin{equation}\label{rappresentazione_Zm_con_Fm}
 {{\mathcal Z}}_m(K)=\{ F_{m,K}(u)\,u\ : u\in{S^{n-1}}\},
\end{equation}
we have
\begin{equation}\label{rappresentazione_mappa_analitica}
F_{m,K}(u)=\frac{\pi(4m+n-1)}{2w_K(u)}+{\operatorname{i}}\ \frac{\ln{\tau}_K(-u)-\ln{\tau}_K(u)}{2w_K(u)}+{{\operatorname{O}}}\left(\frac1{m}\right),
\end{equation}
and the error term ${{\operatorname{O}}}(1/m)$ in \eqref{rappresentazione_mappa_analitica} tends to $0$, as $m$ tends to infinity, uniformly with respect to $u\in {S^{n-1}}$.
\end{theorem}

We are interested in lowering the  regularity assumption on $K$ needed for the conclusions of Theorem~\ref{teo_kobayashi} to hold. We are able to prove the following result.

\begin{theorem}\label{teo_kobayashi_cm}Let ${{r(n)}}$ be as in Theorem~\ref{teo_cov_smooth}. If the convex body $K$ in ${\mathbb{R}}^n$ is of class $C^{{r(n)}}_+$ then the conclusions of Theorem~\ref{teo_kobayashi} hold.
\end{theorem}

We remark that the regularity assumption in Theorem~\ref{teo_kobayashi_cm} is analogous to that required in some studies of the asymptotic behavior of ${\widehat{{{1_K}}}}(x)$ as $x\in{\mathbb{R}}^n$  and $|x|$ tends to infinity (see, for instance, Herz~\cite{Herz-1962}).

The proof of Theorem \ref{teo_kobayashi} is presented  both in \cite[Theorem 2.3.6]{Kob1} and in \cite{Kob2}. 
The Fourier-Laplace transform ${\widehat{{{1_K}}}}({{\zeta}} u)$, for ${{\zeta}}\in{\mathbb{C}}$ and $u\in{S^{n-1}}$, is written as the Fourier-Laplace transform of the Radon transform $S_K(u,t)$ with respect to the single variable $t$, i.e.
\begin{equation}\label{radon_plus_fourier_oned}
{\widehat{{{1_K}}}}({{\zeta}} u)=\int_{-\infty}^{\infty}S_K(u,t)e^{{\operatorname{i}} t{{\zeta}}}\,dt.
\end{equation}
Some results proved in \cite{Kob2} and regarding the zero set of the Fourier-Laplace transform of functions of a single variable are then applied to this expression.
We refer in particular to \cite[Corollary 2.20]{Kob2}, which gives the asymptotic behavior at infinity of the zeros of the Fourier-Laplace transform of a function and an estimate on the dimensions of a compact set containing the remaining zeros.
It is the application of this corollary which yields the conclusions of Theorem~\ref{teo_kobayashi}, and both \cite[Lemma 3.14]{Kob2} and \cite[Lemma 2.2.8]{Kob1} prove that  $S_K(u,\cdot)$ satisfies the assumptions of this corollary when $K$ belongs to $C^\infty_+$.
The next lemma proves that $S_K(u,\cdot)$ satisfies the assumptions of \cite[Corollary 2.20]{Kob2} also when $K\in C^{{r(n)}}_+$. 

Let $\psi(x):{\mathbb{R}}\to[0,1]$ be a $C^\infty$ function such that ${{\operatorname{supp}}} \psi\subset[-2,2]$ and  $\psi(x)\equiv1$ when $x\in[-1,1]$.
\begin{lemma}\label{lem_kobayashi_cm}
Let ${{r(n)}}$ be as in Theorem~\ref{teo_cov_smooth} and let $K\subset{\mathbb{R}}^n$ be a convex body of class $C^{{r(n)}}_+$. Let $V=\{(u,t)\in{S^{n-1}}\times{\mathbb{R}}: -h_K(-u)<t<h_K(u)\}$ and, for $u\in{S^{n-1}}$, let
\begin{gather*}
a_0(u)=\frac{(2\pi)^{\frac{n-1}2}}{\Gamma(\frac{n+1}2)\sqrt{{\tau}_K(-u)}},\quad\quad
b_0(u)=\frac{(2\pi)^{\frac{n-1}2}}{\Gamma(\frac{n+1}2)\sqrt{{\tau}_K(u)}},\\
\phi(u,t)=\psi\left(\frac{5t}{{{{w}}}_K(u)}\right).\\
\end{gather*}
Then the following assertions hold:
\begin{enumerate}[(I)]
\item\label{ass:lem_kobayashi_cm_I} The Radon transform $S_K$ is continuous in ${S^{n-1}}\times{\mathbb{R}}$ and its support is ${{\operatorname{cl}}}(V)$. Moreover $S_K$ is differentiable ${{r(n)}}$ times with respect to $t$ at every $(u,t)\in V$ and each of these derivatives is continuous in  $t$ and in $u$;
\item\label{ass:lem_kobayashi_cm_II} For every $u\in{S^{n-1}}$ there exist $a_1(u), a_2(u), b_1(u), b_2(u)\in{\mathbb{R}}$ such that
\begin{multline}\label{taylor_radon}
 S_K(u,t)-\sum_{j=0}^2a_j(u)\Big(t+h_K(-u)\Big)^{\frac{n-1}2+j}_+
\phi\big(u,t+h_K(-u)\big)+\\-\sum_{j=0}^2b_j(u)\Big(t-h_K(u)\Big)^{\frac{n-1}2+j}_-
\phi\big(u,t-h_K(u)\big),
\end{multline}
as a function of $t$, belongs to $C^{\left[(n-1)/2\right]+2}({\mathbb{R}})$, and its derivative of order $\left[(n-1)/2\right]+3$ exists in $(-h_K(-u), h_K(u))$.
\item\label{ass:lem_kobayashi_cm_III} The expressions  $|a_1(u)|$, $|a_2(u)|$, $|b_1(u)|$, $|b_2(u)|$ and 
\[
 \sup_{-h_K(-u)< t< h_K(u)}\left|\frac{{{\partial}}^{\frac{n-1}{2}+3}\big(\text{expression in \eqref{taylor_radon}}\big)}{{{\partial}} t^{\frac{n-1}{2}+3}}\right|
\]
are  bounded from above uniformly with respect to $u$ in ${S^{n-1}}$.
\end{enumerate}
\end{lemma}
\begin{proof}
\emph{Assertion~\eqref{ass:lem_kobayashi_cm_I}.} The claims regarding the continuity and the support of $S_K$ are obvious.  
The claim regarding the derivatives is essentially proved in \cite[Lemma~2.4]{Kol05}. This lemma proves that when $K$ is $o$-symmetric then $S_K$ is differentiable ${{r(n)}}$ times with respect to $t$ at every $(u,t)$ such that $|t|$ is sufficiently small, and at  $(u,t)$ each of these derivatives is continuous in  $t$ and in $u$. However the $o$-symmetry of $K$  is not needed in the proof. 
Thus, let $(u_0,t_0)\in V$ and let $x_0\in{{\operatorname{int}}} K\cap (u_0+t_0)$. Note that ${{\operatorname{int}}} K\cap (u_0+t_0)\neq\emptyset$, by definition of $V$. Let us apply \cite[Lemma~2.4]{Kol05} with $x_0$ playing the role of the origin. It proves that   the function
\[
(u,s)\to{{\lambda}}_{n-1}\left(K\cap\left\{x : \left<x-x_0,u\right>=s\right\}\right)
\]
is differentiable ${{r(n)}}$ times with respect to $s$, and each of these derivatives is continuous in  $s$ and in $u$ whenever $|s|$ is sufficiently small.
The previous function coincides with $S_K(u,s+\left<x_0,u\right>)$. This implies the  requested property of $S_K(u,t)$  at each $(u,t)$ such that $|\left<x_0,u\right>-t|$ is sufficiently small, that is in a neighborhood of  $(u_0,t_0)$.

\emph{Assertion~\eqref{ass:lem_kobayashi_cm_II}.}
Since the expression in \eqref{taylor_radon}, as a function of $t$, vanishes outside $(-h_K(-u), h_K(u))$, it  belongs to $C^{{r(n)}}$ in that interval,  and ${{r(n)}}\geq[(n-1)/2]+3$, it suffices to prove the assertion in a neighborhood of each endpoint of that interval. We will do it in a neighborhood of $h_K(u)$, since the proof for the other endpoint is similar.
 
Let $u_0\in {S^{n-1}}$ and let $U\subset{S^{n-1}}$ be a neighborhood of  $u_0$. Let $u\in U$. Let $e_1(u),\dots,e_{n-1}(u)\in {\mathbb{R}}^n$ denote an orthonormal base of $u^\perp$ which is a $C^\infty$ function of $u$, and for $y=(y_1,\dots,y_{n-1})\in{\mathbb{R}}^{n-1}$ let $L(u,y)=\sum_{i=1}^{n-1}y_i e_i(u)$. 
For each $u\in U$ we parametrize ${{\partial}} K$ in a neighborhood $W(u)$ of $\nu^{-1}_K(u)$ as 
\[
 {{\partial}} K\cap W(u)=\{\nu^{-1}_K(u)+L(u,y)-f(u,y) u : y\in V\},
\]
where $V\subset{\mathbb{R}}^{n-1}$ is a suitable neighborhood of $o$. For each $(u,y)\in U\times V$, $f(u,y)$ is non-negative and convex with respect to $y$. We have
\[
 f(u,o)=0,\quad\nabla_y f(u,o)=0
\]
and the eigenvalues of $D^2_y f(u,o)$ are the principal curvatures of ${{\partial}} K$ at $\nu_K^{-1}(u)$. Moreover $f$ and $\nabla_y f$ belong to $C^{{{r(n)}}-1}(U\times V)$.

In order to express $S_K(u,t)$ in terms of $f$, for $u\in U$ and $t$ in a left neighborhood of $h_k(u)$, let us parametrize in terms of $f$ the sections of $K$ with  hyperplanes orthogonal to $u$ and close  to the hyperplane  supporting $K$ at $\nu_K^{-1}(u)$.
Let us start by expressing $f(u,\cdot)$ in polar coordinates $(r,{{\theta}})$, letting the parameter $r$ free to take also negative values. More precisely, for $(u,r,{{\theta}})\in U\times(-{{\varepsilon}},{{\varepsilon}})\times {S^{n-2}}$, for a sufficiently small ${{\varepsilon}}>0$, we define $f_0(u,r,{{\theta}})=f(u,r{{\theta}})$.
The properties of $f$ imply that there is a continuous function $f_1$ such that
\[
f_0(u,r,{{\theta}})=r^2f_1(u,r,{{\theta}}).
\]
Note that $f_1$ and $\nabla_{(r,{{\theta}})}f_1$ belong to $C^{{{r(n)}}-3}\left(U\times(-{{\varepsilon}},{{\varepsilon}})\times{S^{n-2}}\right)$.
Since $f_1(u,0,{{\theta}})>0$, for each $(u,{{\theta}})\in U\times {S^{n-2}}$, after possibly changing $U$ and ${{\varepsilon}}$, we may assume $f_1(u,r,{{\theta}})>0$ in $U\times(-{{\varepsilon}},{{\varepsilon}})\times{S^{n-2}}$.
By the Implicit Function Theorem there exist $U'\subset U$ neighborhood of $u_0$, ${{\delta}}>0$ and a function $R:U'\times (-{{\delta}},{{\delta}})\times{S^{n-2}}\to(-{{\varepsilon}},{{\varepsilon}})$ such that
\[
 R(u,t,{{\theta}})\sqrt{f_1\left(u,R(u,t,{{\theta}}),{{\theta}}\right)}=t.
\]
The regularity of $f_1$ and the fact that $R=0$ if and only $t=0$ imply that $R$ and $\nabla _{(t,{{\theta}})}R$ belong to $C^{{{r(n)}}-3}\left(U'\times(-{{\delta}},{{\delta}})\times{S^{n-2}}\right)$.  Since $f_1(u,r,{{\theta}})=f_1(u,-r,-{{\theta}})$, we have
$-R(u,t,{{\theta}})\sqrt{f_1\left(u,-R(u,t,{{\theta}}),-{{\theta}}\right)}=-t$, and this implies
\begin{equation}\label{lem_kobayashi_cm_a}
-R(u,t,{{\theta}})=R(u,-t,-{{\theta}}).
\end{equation}
The geometrical meaning of $R$ is the following.
Let $u\in U'$ and $t\in(h_K(u)-{{\delta}}^2,h_K(u))$. The section $K\cap\{x\in{\mathbb{R}}^n : x\cdot u=t\}$ is bounded by the surface
\[
 \Big\{ \nu^{-1}_K(u)+L(u,r{{\theta}})-\left(h_K(u)-t\right) u :
 r>0, {{\theta}}\in{S^{n-2}},  f(u,r\theta)=h_K(u)-t\Big\}.
\]
Expressing this in terms of $R$ and remembering that the sign of $R$ coincides with the sign of $t$, we have
\begin{multline*}
 K\cap\{x\in{\mathbb{R}}^n : x\cdot u=t\}=
 \Big\{ \nu^{-1}_K(u)+L(u,r{{\theta}})-\left(h_K(u)-t\right) u : \\
 :0\leq r\leq R\left(u,\sqrt{h_K(u)-t},{{\theta}}\right), {{\theta}}\in{S^{n-2}}\Big\}.
\end{multline*}
The previous formula allow us to  express $S_K$ in terms of $R$ as follows:
\begin{equation*}
 S_K(u,t)=
\begin{cases}
           0&\text{when $t\geq h_K(u)$;}\\
	  \displaystyle \int_{S^{n-2}} \frac{R\left(u,\sqrt{h_K(u)-t},{{\theta}}\right)^{n-1}}{n-1}\,d{{\theta}}& \text{when $t\in(h_K(u)-{{\delta}}^2,h_K(u))$,}
\end{cases}
\end{equation*}
where $d{{\theta}}$ denotes $(n-2)$-dimensional Hausdorff measure. 

Let us write the Taylor expansion of $R(u,t,\theta)$ in $t$ at $t=0$. In order to simplify the notations we set $m={{r(n)}}-2$ and we omit to explicitly write the dependence of $R$, and of some other functions, on $u$ and on ${{\theta}}$.
We have
\begin{equation}\label{taylor_exp_R}
 R(t)=\sum_{i=1}^{m}c_i t^i+r(t),
\end{equation}
for suitable coefficients $c_i=c_i(u,{{\theta}})$ (which depend continuously on $u$) and with the remainder  $r(t)=r(u,t,{{\theta}})$ written as
\[
r(t)=\int_0^t\left(\frac{{{\partial}}^mR(s)}{{{\partial}} s^m}-\frac{{{\partial}}^mR(0)}{{{\partial}} s^m}\right)\frac{(t-s)^{m-1}}{(m-1)!}\, ds.
\]
For $k=0,\dots,m$ and $t\in(0,{{\delta}})$, it is easy to derive from the previous expression of $r$ the following bounds
\begin{equation}\label{stima_resto_a}
 \frac{{{\partial}}^k r(t)}{{{\partial}} t^k}\leq \sup_{s\in[0,t]} \left|\frac{{{\partial}}^mR(s)}{{{\partial}} s^m}-\frac{{{\partial}}^mR(0)}{{{\partial}} s^m}\right|\ t^{m-k}.
\end{equation}
Let us prove that, for $j$ positive integer, $k=0,\dots,m$  and $t\in(h_K-{{\delta}}^2,h_K)$, we have
\begin{equation}\label{stima_resto}
 \frac{{{\partial}}^k r^j\left(\sqrt{h_K-t}\right)}{{{\partial}} t^k}\leq
 d_{j,k} \left(\sup_{s\in[0,t]} \left|\frac{{{\partial}}^mR(s)}{{{\partial}} s^m}-\frac{{{\partial}}^mR(0)}{{{\partial}} s^m}\right|\right)^j(h_K-t)^{\frac{m j}2-k}.
\end{equation}
for a suitable positive constant $d_{j,k}$ which depends only on $j$ and $k$.
Indeed, using  \cite[Formula $3_n$]{McK56} to express the $k$-th derivative of a composite function, we have 
\begin{align*}
\frac{{{\partial}}^k r\left(\sqrt{h_K-t}\right)}{{{\partial}} t^k}=&
\sum_{i=1}^k \frac{{{\partial}}^i r(s)}{{{\partial}} s^i}|_{s=\sqrt{h_K-t}}\sum_{j=0}^i \frac{(-1)^{i-j}}{j! (i-j)!}\left(h_K-t\right)^\frac{i-j}2\frac{{{\partial}}^k \left(h_K-t\right)^\frac{j}2}{{{\partial}} t^k}\\
=&\sum_{i=1}^k  \frac{{{\partial}}^i r(s)}{{{\partial}} s^i}|_{s=\sqrt{h_K-t}}(h_K-t)^{\frac{i}2-k}\sum_{j=0}^i \frac{(-1)^{i-j+k}}{j! (i-j)!}\times\\
&\quad\quad\quad\quad\times\frac{j}2\left(\frac{j}2-1\right)\dots\left(\frac{j}2-k+1\right).
\end{align*}
This formula and \eqref{stima_resto_a} prove~\eqref{stima_resto} when $j=1$. In order to prove~\eqref{stima_resto} when $j>1$ it suffices to use \cite[Formula $9_n$]{McK56} to express the derivative in the left-hand side of~\eqref{stima_resto} in terms of derivatives of $r(\sqrt{h_K-t})$ and to use~\eqref{stima_resto} with $j=1$. We omit the details.

Note that the continuity of $({{\partial}}^m/{{\partial}} t^m) R(t)$ implies that the left-hand side in~\eqref{stima_resto} is ${{\operatorname{o}}}\left((h_K-t)^{\frac{m j}2-k}\right)$ as $t<h_K$ tends to $h_K$.

Let us now apply all these estimates to our case. For $t\in(h_K-{{\delta}}^2,h_K)$ we write
\begin{equation*}\label{sviluppo_asintotico_SK}
\begin{split}
S_K(u,t)=& 
\frac1{n-1}\int_{S^{n-2}} \left(\sum_{i=1}^{m}c_i (h_K-t)^{i/2}+r\left(\sqrt{h_K-t}\right)\right)^{n-1}\,d{{\theta}}\\
=&\frac1{n-1}\int_{S^{n-2}} \left(\sum_{i=1}^{m}c_i (h_K-t)^{i/2}\right)^{n-1}\,d{{\theta}}+\\
&\quad+\frac1{n-1}\int_{S^{n-2}}\sum_{j=1}^{n-1}\binom{n-1}{j} r^j\left(\sqrt{h_K-t}\right)\left(\sum_{i=1}^{m}c_i (h_K-t)^{i/2}\right)^{n-1-j}\,d{{\theta}}.
\end{split}\end{equation*}
Let $I_1(t)=I_1(u,t)$ and $I_2(t)=I_2(u,t)$ denote respectively the first and the second integral after the last equality sign in the previous formula.
The integral $I_1(t)$ can be written as 
\begin{equation}\label{first_integral}
I_1(t)= \frac1{n-1}\sum_{l=(n-1)}^{m(n-1)}(h_K-t)^{l/2}\int\limits_{S^{n-2}}\sum_{\substack{i_1,\dots,i_{n-1}=1,\dots,m\\i_1+\dots+i_{n-1}=l}}c_{i_1}\dots c_{i_{n-1}}\,d{{\theta}}.
\end{equation}
 Formula~\eqref{lem_kobayashi_cm_a} implies, for $i=1,\dots,m$,
\begin{equation}\label{proprieta_c_i}
 c_i(u,-{{\theta}})=(-1)^{i-1} c_i(u,{{\theta}}).
\end{equation}
 When $l-(n-1)$ is odd   the coefficient of $(h_K-t)^{l/2}$ in \eqref{first_integral} vanishes, since  the associated integrand is and odd function of ${{\theta}}$ by \eqref{proprieta_c_i}.
 It is known (see \cite{Kob1}) that the coefficient of $(h_K-t)^{(n-1)/2}$ in the previous expression coincides with $b_0(u)$. Let $b_1(u)$ and $b_2(u)$ denote respectively the coefficients of $(h_K-t)^{\frac{n-1}{2}+1}$ and of $(h_K-t)^{\frac{n-1}{2}+2}$ in \eqref{first_integral}.
 When $t\in(h_K-{{\delta}}^2,h_K)$ then
\begin{equation}\label{Iunomenoprimitermini}
 I_1(t)-b_0(u)(t-h_K)_-^\frac{n-1}{2}-b_1(u)(t-h_K)_-^{\frac{n-1}{2}+1}-b_2(u)(t-h_K)_-^{\frac{n-1}{2}+2}
\end{equation}
is a linear combination of powers of $h_K-t$ with exponents higher than or equal to $(n-1)/2+3$. Thus if we extend the definition of $I_1(t)$ to $(h_K-{{\delta}}^2,h_K+{{\delta}}^2)$ by putting $I_1(t)=0$ when $t\in[h_K,h_K+{{\delta}}^2)$, the expression in \eqref{Iunomenoprimitermini}   belongs to $C^{\left[(n-1)/2\right]+2}(h_K-{{\delta}}^2,h_K+{{\delta}}^2)$. Moreover, when $t\in(h_K-{{\delta}}^2,h_K)$,  the  $[(n-1)/2]+3$ derivative with respect to $t$ of the expression in \eqref{Iunomenoprimitermini}  is equal to
\[
 \sum_{l=(n-1)+6}^{m(n-1)}e_l(h_K-t)^{l/2-\left[\frac{n-1}{2}\right]-3} \int\limits_{S^{n-2}}\sum_{\substack{i_1,\dots,i_{n-1}=1,\dots,m\\i_1+\dots+i_{n-1}=l}}c_{i_1}\dots c_{i_{n-1}}\,d{{\theta}},
\]
for suitable constants $e_l$ depending only on $n$ and $l$. Since all powers of $h_K-t$ in this derivative have non-negative exponents, its absolute value is uniformly bounded in $(h_K-{{\delta}}^2,h_K)$. Since the coefficients $c_1,\dots, c_m$ depend continuously on $u$, this bound is locally uniform with respect to $u$.

The function $I_2(t)$ is a linear combination of terms of the form
\begin{equation*}
(h_K-t)^{l/2}\int_\Snduec_{i_1}\dots c_{i_{n-1-j}}r^j\left(\sqrt{h_K-t}\right)\,d{{\theta}},
\end{equation*}
with $j=1,\dots,n-1$, $l=n-1-j,\dots,m(n-1-j)$, $i_1,\dots,i_{n-1-j}=1,\dots,n-1-j$, $i_1+\dots+i_{n-1-j}=l$.
Let $k\in\{0,\dots,[(n-1)/2]+3\}$ and let $t\in(h_K-{{\delta}}^2,h_K)$. Since $m\geq[(n-1)/2]+3$, the derivative of order $k$ of this term computed at $t$   exists and is a linear combination  of terms of the form
\begin{equation*}
\frac{{{\partial}}^{k-p}(h_K-t)^{l/2}}{{{\partial}} t^{k-p}}\int_{S^{n-2}} c_{i_1}\dots c_{i_{n-1-j}} \frac{{{\partial}}^p r^j\left(\sqrt{h_K-t}\right)}{{{\partial}} t^p} \,d{{\theta}},
\end{equation*}
with $0\leq p\leq k$. 
In view of~\eqref{stima_resto} the derivative of order $k$ of $I_2(t)$ is a linear combination  of terms which are continuous and whose asymptotic behavior as $t<h_K$ tends to $h_K $ is $o\left((h_K-t)^{\frac{mj+l}2-k}\right)$. Note that this asymptotic behavior is locally uniform with respect to $u$.
 Since
\[
m\geq \begin{cases}6 &\text{when $n$ is even,}\\7 &\text{when $n$ is odd}\end{cases}
\]
the exponent $(mj+l)/2-k$ is non-negative for $k$, $j$ and $l$ in the ranges described above (because  $mj+l\geq m+n-2$ and $k\leq [(n-1)/2]+3$).
This concludes the proof of Assertion~\eqref{ass:lem_kobayashi_cm_II}.

\emph{Assertion~\eqref{ass:lem_kobayashi_cm_III}.}
The coefficients $b_1(u)$ and $b_2(u)$ in \eqref{Iunomenoprimitermini} are integrals over ${S^{n-2}}$ of polynomials in the coefficients $c_1,\dots, c_5$ of \eqref{taylor_exp_R}.
These coefficients are, up to constants, the derivatives with respect to $t$, up to order five, of $f_1$ at $t=0$, or equivalently,  the derivatives with respect to $t$, of order up to seven, of $f$ at $t=0$.
The regularity of $f$ implies that  $b_1(u)$ and $b_2(u)$ depends continuously on $u$. The same is true for $b_0(u)$, due to its explicit representation and the regularity of ${\tau}_K$.
The assertion regarding the boundedness of the $[(n-1)/2]+3$  derivative with respect to $t$ of the expression in~\eqref{taylor_radon} in a left neighborhood of $h_K(u)$ is a consequence of what we have proved above regarding the
$[(n-1)/2]+3$ derivative of the expression in~\eqref{Iunomenoprimitermini} and of $I_2(t)$.
\end{proof}

\begin{proof}[Proof of Theorem \ref{teo_kobayashi_cm}]Let us write ${\widehat{{{1_K}}}}({{\zeta}} u)$ as in \eqref{radon_plus_fourier_oned} and let us apply \cite[Corollary 2.20]{Kob2} to the Fourier-Laplace transform of $S_K(u,t)$ with respect to $t$
\begin{equation}\label{FT_radon}
 {\widehat{{{S_K(u,t)}}}}({{\zeta}}):=\int_{-\infty}^{\infty}S_K(u,t)e^{{\operatorname{i}} t{{\zeta}}}\,dt.
\end{equation}

We apply this corollary with $f=S_K(u,t)$, $\lambda=(n-1)/2$, $A(f)={{{w}}}_K(u)$, $a_0(f)=a_0(u)$ and $b_0(f)=b_0(u)$ (where $a_j(u)$ and $b_j(u)$, for $j=0,1,2$, are the functions defined in the statement of Lemma~\ref{lem_kobayashi_cm}).  In the corollary the symbol $\|f\|_{\mathcal{C}^2(\lambda)}$ is defined as
\begin{multline*}
 \sum_{j=0}^2{{{w}}}_K(u)^{\frac{n-1}{2}+j}\left(|a_j(u)|+|b_j(u)|\right)+\\
 +{{{w}}}_K(u)^{\frac{n-1}{2}+3}\sup_{-h_K(-u)< t< h_K(u)}\left|\frac{{{\partial}}^{\frac{n-1}{2}+3}\big(\text{expression in \eqref{taylor_radon}}\big)}{{{\partial}} t^{\frac{n-1}{2}+3}}\right|.
\end{multline*}
By Lemma~\ref{lem_kobayashi_cm}, $\sup_{u\in{S^{n-1}}}\|f\|_{\mathcal{C}^2(\lambda)}$ is bounded from above.

\cite[Corollary 2.20]{Kob2} proves that for each $u\in{S^{n-1}}$ there exist a positive integer $m(K,u)$, a positive number $d(K,u)$ and a finite set $C(K,u)\subset{\mathbb{C}}$ such that the zero set of ${\widehat{{{S_K(u,t)}}}}$ consists of  $C(K,u)$ and, for each $m\geq m(K,u)$, of one simple zero in each  of the two balls (in ${\mathbb{C}}$)
\begin{equation}\label{balls_containing_zero}
 B\left(
 {{\gamma}}\frac{\pi(4m+n-1)}{2w_K(u)}+{\operatorname{i}}\ \frac{\ln {\tau}_K(-u)-\ln {\tau}_K(u)}{2w_K(u)},\frac{d(K,u)}{m}
 \right),
 \quad\text{${{\gamma}}=1,-1$.}
\end{equation}
Moreover $m(K,u)$, $d(K,u)$ and the radius of a ball centered at $o$ and containing $C(K,u)$ are bounded from above uniformly with respect to $u\in{S^{n-1}}$ in terms of $\sup_{u\in{S^{n-1}}}\|f\|_{\mathcal{C}^2(\lambda)}$, $\inf_{u\in{S^{n-1}}}{{{w}}}_K(u)$, $\sup_{u\in{S^{n-1}}}{{{w}}}_K(u)$, $\inf_{u\in{S^{n-1}}}{\tau}_K(u)$ and $\sup_{u\in{S^{n-1}}}{\tau}_K(u)$.

Let $m(K)=\sup_{u\in{S^{n-1}}}m(K,u)$ and, for each $m\geq m(K)$, let $F_{m,K}(u)$ be the zero of ${\widehat{{{S_K(u,t)}}}}$ contained in the ball in~\eqref{balls_containing_zero} corresponding to ${{\gamma}}=1$. (The one corresponding to ${{\gamma}}=-1$ coincides with $-F_{m,K}(-u)$.)
Due to \eqref{radon_plus_fourier_oned} the intersection of the zero set of ${\widehat{{{1_K}}}}$ with the ray $\{z={{\zeta}} u:{{\zeta}}\in{\mathbb{C}}\}$ consists of a bounded set and of $\cup_{m\geq m(K)}\{F_{m.K}(u)u, -F_{m,K}(-u)u\}$.

To complete the proof it remains to prove that the map $F_{m,K}:{S^{n-1}}\to{\mathbb{C}}$ is analytic.
This can be proved exactly as in \cite[Lemma 2.4.25]{Kob1}. The proof is based on the fact that ${\widehat{{{1_K}}}}$ is holomorphic, on the analytic Implicit Function Theorem and on the fact that if ${{\zeta}} u$ is a zero of ${\widehat{{{1_K}}}}$ and if ${\operatorname{Re}}{{\zeta}}$ is sufficiently large, then
\[
 \frac{{\partial}}{{{\partial}}{{\zeta}}}1_K({{\zeta}} u)\neq 0.
\]
The proof of this last formula is based on  the asymptotic expansion of $({{\partial}}/{{{\partial}}{{\zeta}}})1_K({{\zeta}} u)$ given by Formula (2.4.27) in \cite{Kob1}.
To prove this formula for $K\in{\mathbb{C}}^{{r(n)}}_+$ one argues as follows. The function $({{\partial}}/{{{\partial}}{{\zeta}}})1_K({{\zeta}} u)$ coincides with the Fourier-Laplace transform with respect to $t$ of ${\operatorname{i}} t S_K(u,t)$.
Since a result analogous to Lemma \ref{lem_kobayashi_cm} holds for ${\operatorname{i}} t S_K(u,t)$, \cite[Lemma 2.13]{Kob2} (with $f={\operatorname{i}} t S_K(u,t)$, ${{\lambda}}=(n-1)/2$, $a_0(f)=-{\operatorname{i}} a_0(u)h_K(-u)$, $b_0(f)={\operatorname{i}} b_0(u)h_K(u)$, ${{\alpha}}(f)=-h_K(-u)$, ${{\beta}}(f)=h_K(u)$ and $p({{\lambda}})=\Gamma\left((n+1)/2\right)e^{{\operatorname{i}}\pi(n+1)/4}$) applies  and yields \cite[Formula (2.4.27)]{Kob1}.
\end{proof}

\section{Covariogram problem for regular bodies}\label{sec_cov}
Kobayashi result enters the proof of Theorem~\ref{teo_cov_smooth} only through the next proposition. The key  point in the proof of this proposition is the fact that the maps $F_{m,K}$, introduced in the statement of Theorem~\ref{teo_kobayashi}, are analytic.
\begin{proposition}\label{prop_ratio_radii_curvature}
Let $H$, $K$ be convex bodies of class $C^{{r(n)}}_+$ with $g_H=g_K$. Then
\begin{align*}\label{equal_ratio_radii_curvature}
\text{either}\quad\frac{{\tau}_H(-u)}{{\tau}_H(u)}=& \frac{{\tau}_K(-u)}{{\tau}_K(u)}\quad\text{for each } u\in{S^{n-1}}\\
\text{or}\quad\frac{{\tau}_H(u)}{{\tau}_H(-u)}=& \frac{{\tau}_K(-u)}{{\tau}_K(u)}\quad\text{for each } u\in{S^{n-1}}.
\end{align*}
\end{proposition}

\begin{proof}
The identity $g_H=g_K$, \eqref{width_of_support} and \eqref{convoluzione_in_cn} imply
\begin{equation}\label{equal_width}
{{{w}}}_H={{{w}}}_K
\end{equation}
and, for $\zeta\in{\mathbb{C}}^n$,
\begin{equation*}
{\widehat{{{1_H}}}}(\zeta)\,
{\overline{{{\widehat{{{1_{H}}}}}\left({\overline{\zeta}}\right)}}}=
{\widehat{{{1_K}}}}(\zeta)\,
{\overline{{{\widehat{{{1_{K}}}}}\left({\overline{\zeta}}\right)}}}.
\end{equation*}
Thus we have
\begin{equation}\label{identita_zeri_covario}
{{\mathcal Z}}(H)\bigcup{\overline{{{{\mathcal Z}}(H)}}}={{\mathcal Z}}(K)\bigcup{\overline{{{{\mathcal Z}}(K)}}}.
\end{equation}
Let us use the notations introduced in the statement of Theorem~\ref{teo_kobayashi}. Let us choose $m_0>m(K),m(H)$ such that ${{\mathcal Z}}_{m,K}\cap C(H)=\emptyset$ for each $m\geq m_0$. 
Theorem \ref{teo_kobayashi_cm} and \eqref{identita_zeri_covario} imply that for each $m\geq m_0$ and for each $u\in{S^{n-1}}$ we have
\[
 F_{m,K}(u)u\in \bigcup_{l=m(H)}^\infty \left({{\mathcal Z}}_l(H)\bigcup{\overline{{{{\mathcal Z}}_l(H)}}}\right).
\]
The representation of ${{\mathcal Z}}_l(H)$ provided by \eqref{rappresentazione_Zm_con_Fm}  implies that there exists $l=l(m,u)$ such that either $F_{m,K}(u)=F_{l,H}(u)$ or $F_{m,K}(u)={\overline{{F_{l,H}(u)}}}$.
In both cases the representation of the real parts  of $F_{m,K}$ and $F_{l,H}$ given in \eqref{rappresentazione_mappa_analitica}, together with \eqref{equal_width}, implies that there exists $m_1\geq m_0$ such that   $l=m$ for each $m\geq m_1$ and $u\in{S^{n-1}}$. Summarizing, for each $u\in{S^{n-1}}$ and $m\geq m_1$  either we have
\begin{gather}
 F_{m,K}(u)=F_{m,H}(u)\label{alternativa1}
\intertext{or we have}
F_{m,K}(u)={\overline{{F_{m,H}(u)}}}.\label{alternativa2}
\end{gather}
A priori the choice may vary from $u$ to $u$. We may assume that $K$ is not centrally symmetric because otherwise $K$ is a reflection or translation of $H$ (it is an easy consequence of the Brunn-Minkowski inequality as explained, for instance,  in~\cite[p. 204]{Bianchi-2005}) and the claim follows. We may thus assume that ${\tau}_K$ is not an even function. Formula~\eqref{rappresentazione_mappa_analitica} implies that there exists $m_2\geq m_1$ and a relatively open connected subset $U$ of ${S^{n-1}}$ such that
\begin{equation}\label{zero_set_K_with_im_positive}
{\operatorname{Im}} F_{m,K}(u)>0,
\end{equation}
for each $m\ge m_2$ and $u\in U$. The alternatives  \eqref{alternativa1} and \eqref{alternativa2} imply that ${\operatorname{Im}} F_{m,H}(u)\neq0$ when $u\in U$.
Formula \eqref{riflessione_in_Cn} implies that passing from $H$ to $-H$ corresponds to conjugating $F_{m,H}$.  Thus, possibly after a reflection of $H$, there is $m_3\geq m_2$ and  a relatively open set $V\subset U$ such that
\begin{equation}\label{zero_set_H_with_im_positive}
{\operatorname{Im}} F_{m,H}(u)>0,
\end{equation}
for each $m\geq m_3$ and $u\in V$. Formulas~\eqref{alternativa1}, \eqref{alternativa2}, \eqref{zero_set_K_with_im_positive} and~\eqref{zero_set_H_with_im_positive}  imply
\begin{equation*}
 F_{m,H}(u)=F_{m,K}(u)
\end{equation*}
for each $m\geq m_3$ and $u\in V$. Since $F_{m,K}$ and $F_{m,H}$ are analytic maps from ${S^{n-1}}$ to ${\mathbb{C}}$, their coincidence in $V$ implies their coincidence on the whole ${S^{n-1}}$, i.e.
\[
 F_{m,H}=F_{m,K}.
\]
This and  \eqref{rappresentazione_mappa_analitica} conclude the proof.
\end{proof}

\begin{proof}[Proof of Theorem~\ref{teo_cov_smooth}]
 Propositions~\ref{teo_information_c2+_bodies}-\eqref{nonordered_curvatures} and~\eqref{prop_ratio_radii_curvature} imply that, possibly after a reflection of $H$, we have
\[
{\tau}_H(u)={\tau}_K(u)
\]
for each $u\in{S^{n-1}}$. The uniqueness part in Minkowski's Theorem \cite[Th. 7.2.1]{Sc} implies that $H$ and $K$ coincide, up to translations.
\end{proof}
\begin{remark}
Theorem~\ref{teo_cov_smooth} only proves that the covariogram determines a ${\mathbb{C}}^{{r(n)}}_+$ body among ${\mathbb{C}}^{{r(n)}}_+$ bodies. We are not able to prove that the determination holds among  all convex bodies.
\end{remark}

\section{Cross covariogram problem for regular bodies}\label{sec_cross_cov}
\smallskip

\textbf{Cross covariogram Problem.}
\emph{Does $g_{H,K}$ determine the pair $(H,K)$ of convex bodies among all pairs of convex bodies, up to trivial associates?}
\smallskip

Bianchi~\cite{B4} gives a complete answer to this problem when $H$ and $K$ are convex polygons. In order to explain this result let us introduce some families of sets.
\begin{example}\label{parallelograms}
Let ${{\alpha}}$, ${{\beta}}$, ${{\gamma}}$, ${{\delta}}$, ${{\alpha}}'$, ${{\beta}}'$, ${{\gamma}}'$ and ${{\delta}}'$ be positive real numbers, $m\in{\mathbb{R}}$, $y,y'\in{\mathbb{R}}^2$, $I_1=[(-1,0),(1,0)]$, $I_2=1/\sqrt{2}\ [(-1,-1),(1,1)]$, $I_3=[(0,-1),(0,1)]$, $I_4=1/\sqrt{2}\ [(1,-1),(-1,1)]$ and $I_5=(1/\sqrt{1+m^2})\,[(-m,-1),(m,1)]$. Assume either  $m=0$, ${{\alpha}}'\neq {{\gamma}}'$ and ${{\beta}}'\neq{{\delta}}'$ or else $m\neq 0$ and ${{\alpha}}'\neq {{\gamma}}'$. We define four pairs of parallelograms as follows (see Figure~\ref{fig_four_parall}):
\begin{align*}
{{\mathcal H}}_1&={{\alpha}} I_1+{{\beta}} I_2,\quad& {{\mathcal K}}_1&={{\gamma}} I_3+{{\delta}} I_4+y;\\
{{\mathcal H}}_2&={{\alpha}} I_1+{{\delta}} I_4,\quad&  {{\mathcal K}}_2&={{\beta}} I_2+{{\gamma}} I_3+y;\\
{{\mathcal H}}_3&={{\alpha}}' I_1+{{\beta}}' I_3,\quad& {{\mathcal K}}_3&={{\gamma}}' I_1+{{\delta}}' I_5+y';\\
{{\mathcal H}}_4&={{\gamma}}' I_1+{{\beta}}' I_3,\quad& {{\mathcal K}}_4&={{\alpha}}' I_1+{{\delta}}' I_5+y'.
\end{align*}

\begin{figure}
\begin{center}

\end{center}
\caption{We have $g_{{{\mathcal H}}_1,{{\mathcal K}}_1}=g_{{{\mathcal H}}_{2},{{\mathcal K}}_{2}}$ and $g_{{{\mathcal H}}_3,{{\mathcal K}}_3}=g_{{{\mathcal H}}_{4},{{\mathcal K}}_{4}}$. Moreover, up to affine transformations, these are the only pairs of planar convex polygons with equal cross covariogram.}
\label{fig_four_parall}
\end{figure}

\cite{B4} proves that for $i=1,3$, we have $g_{{{\mathcal H}}_i,{{\mathcal K}}_i}=g_{{{\mathcal H}}_{i+1},{{\mathcal K}}_{i+1}}$ but $({{\mathcal H}}_i,{{\mathcal K}}_i)$ is not a trivial associate of $({{\mathcal H}}_{i+1},{{\mathcal K}}_{i+1})$. It also proves that, in the class of convex polygons and up to an affine transformation, the previous counterexamples are the only ones.
\end{example}

\begin{theorem}[Bianchi~\cite{B4}]\label{teo_cov_congiunto_poligoni}
Let $H$ and $K$ be convex polygons and $H'$ and $K'$ be planar  convex bodies
with $g_{H,K}=g_{H',K'}$. Assume that there exist no affine transformation ${{\mathcal T}}$ and no different  indices $i,j$, with either $i,j\in\{1,2\}$ or  $i,j\in\{3,4\}$, such that $({{\mathcal T}} H,{{\mathcal T}} K)$ and $({{\mathcal T}} H',{{\mathcal T}} K')$ are trivial associates of $({{\mathcal H}}_i,{{\mathcal K}}_i)$ and of $({{\mathcal H}}_j,{{\mathcal K}}_j)$, respectively.
Then  $(H,K)$ is a trivial associate of $(H',K')$.
\end{theorem}

\begin{proof}[Proof of Theorem~\ref{teo_cross_cov_smooth}]
Formula~\eqref{convoluzione} implies
\[
 {\widehat{{{1_H}}}}(\zeta)\,{{\widehat{{{1_{-K}}}}}(\zeta)}={\widehat{{{1_{H'}}}}}(\zeta)\,{{\widehat{{{1_{-K'}}}}}(\zeta)}
\]
and, as a consequence,
\[
{{\mathcal Z}}(H)\bigcup{{{\mathcal Z}}(-K)}={{\mathcal Z}}(H')\bigcup{{{\mathcal Z}}(-K')}, 
\]
where, for a convex body $L\subset{\mathbb{R}}^n$, ${{\mathcal Z}}(L)=\{{{\zeta}}\in{\mathbb{C}}^n : {\widehat{{{1_L}}}}({{\zeta}})=0\}$.
This identity implies, by Theorem~\ref{teo_kobayashi_cm}, the existence of positive integers $m_i$, $i=1,2,3,4$,  such that
for each $u\in{S^{1}}$
\begin{multline}\label{union_of_zero_curves}
\left\{F_{m,H}(u):m\geq m_1\right\}\bigcup\left\{F_{m,-K}(u):m\geq m_2\right\}=\\
=\left\{F_{m,H'}(u):m\geq m_3\right\}\bigcup\left\{F_{m,-K'}(u):m\geq m_4\right\}.
\end{multline}

We first show that for each $u\in{S^{1}}$ we have
\begin{equation}\label{equal_set_widths}
 \{{{{w}}}_H(u),{{{w}}}_K(u)\}=\{{{{w}}}_{H'}(u),{{{w}}}_{K'}(u)\}.
\end{equation}
Formula~\eqref{width_of_support} implies
\begin{equation}\label{equal_sums_widths}
 {{{w}}}_H+{{{w}}}_{K}={{{w}}}_{H'}+{{{w}}}_{K'}.
\end{equation}
Let $u\in{S^{n-1}}$. If one of the elements of $\{{{{w}}}_H(u),{{{w}}}_K(u)\}$ belongs to $\{{{{w}}}_{H'}(u),{{{w}}}_{K'}(u)\}$, then~\eqref{equal_sums_widths} implies~\eqref{equal_set_widths}. If ${{{w}}}_H(u)={{{w}}}_K(u)$ and ${{{w}}}_{H'}(u)={{{w}}}_{K'}(u)$, then again~\eqref{equal_sums_widths} implies~\eqref{equal_set_widths}. Thus we may assume that one of the four numbers ${{{w}}}_H(u)$, ${{{w}}}_K(u)$, ${{{w}}}_{H'}(u)$ and ${{{w}}}_{K'}(u)$ is strictly larger than the other ones. Let us assume
\begin{equation}\label{strictly_larger_widths}
 {{{w}}}_H(u)>\max\left\{{{{w}}}_K(u),{{{w}}}_{H'}(u), {{{w}}}_{K'}(u)\right\}.
\end{equation}
(The other cases can be treated similarly.) By~\eqref{union_of_zero_curves} for each $m\geq m_1$ and $i=0,1,2$ there exists $l_i=l_i(m,u)$ such that
\begin{align}
 F_{m+i,H}(u)&=F_{l_i,H'}(u)\label{cc_alternative1}\\
 \text{or }\quad F_{m+i,H}(u)&=F_{l_i,-K'}(u). \label{cc_alternative2}
\end{align}
If \eqref{cc_alternative1} (if \eqref{cc_alternative2}) holds for a particular value of $i$ we say that \eqref{cc_alternative1}$_i$ (\eqref{cc_alternative2}$_i$, respectively) holds. 
At least one between \eqref{cc_alternative1}$_0$ and \eqref{cc_alternative2}$_0$ holds for infinitely many values of $m$, and let us assume that this happen for~\eqref{cc_alternative1}$_0$ (the other case can be treated similarly).
Note that~\eqref{cc_alternative1}$_0$ and~\eqref{cc_alternative1}$_1$ do not hold together when $m$ is sufficiently large. Indeed if they do we have $F_{m+1,H}(u)-F_{m,H}(u)=F_{l_1,H'}(u)-F_{l_0,H'}(u)$. On the other hand we have 
\begin{gather*}
  {\operatorname{Re}}\left(F_{m+1,H}(u)-F_{m,H}(u)\right)=\frac{2\pi}{{{{w}}}_H(u)}+{{\operatorname{O}}}\left(\frac1{m}\right),\\
 {\operatorname{Re}}\left(F_{l_1,H'}(u)-F_{l_0,H'}(u)\right)=\frac{2\pi(l_1-l_0)}{{{{w}}}_{H'}(u)}+{{\operatorname{O}}}\left(\frac1{m}\right)
\end{gather*}
(the term ${{\operatorname{O}}}(1/m)$ in the first line may differ from that in the second line) and the right-hand side of the first equation is strictly less than the right-hand side of the second equation when $m$ is sufficiently large, due to~\eqref{strictly_larger_widths} and $l_0<l_1$.  A similar argument proves that~\eqref{cc_alternative2}$_1$ and~\eqref{cc_alternative2}$_2$ do not hold together when $m$ is sufficiently large.
Thus~\eqref{cc_alternative1}$_0$ and~\eqref{cc_alternative1}$_2$ hold for all $m$ in an infinite set $I$. When $m\in I$ we have
\begin{gather*}
  {\operatorname{Re}}\left(F_{m+2,H}(u)-F_{m,H}(u)\right)=\frac{4\pi}{{{{w}}}_H(u)}+{{\operatorname{O}}}\left(\frac1{m}\right),\\
 {\operatorname{Re}}\left(F_{l_2,H'}(u)-F_{l_0,H'}(u)\right)=\frac{2\pi(l_2-l_0)}{{{{w}}}_{H'}(u)}+{{\operatorname{O}}}\left(\frac1{m}\right).
\end{gather*}
Arguing as above proves that $l_2-l_0=1$ when $m\in I$ and $m$ is large enough. This implies ${{{w}}}_H(u)=2{{{w}}}_{H'}(u)$. Thus Theorem~\ref{teo_kobayashi_cm} implies that when \eqref{cc_alternative1}$_0$ holds we have 
 \begin{equation*}
  \frac{\pi (4m+1)}{2{{{w}}}_H(u)}=\frac{\pi (4l_0+1)}{{{{w}}}_H(u)}+{{\operatorname{O}}}\left(\frac1{m}\right).
 \end{equation*}
This implies
\[
m-2l_0=1/4+{{\operatorname{O}}}(1/m), 
\]
This equality does not hold when $m$ is large, because $m-2l_0\in\mathbb{Z}$ while $1/4+{{\operatorname{O}}}(1/m)\notin\mathbb{Z}$. This contradiction concludes the proof of~\eqref{equal_set_widths}.

Assume that there exists a relatively open subset $U$ in ${S^{1}}$ such that
\begin{equation}\label{locally_diff_widths}
 {{{w}}}_H(u)\neq{{{w}}}_K(u)\quad\text{for each $u\in U$.}
\end{equation}
Up to restricting $U$ we may assume that
\begin{align}
\text{either ${{{w}}}_H(u)={{{w}}}_{H'}(u)$ and ${{{w}}}_K(u)={{{w}}}_{K'}(u)$ for each $u\in U$}\label{cc_alternative3}\\
 \text{or ${{{w}}}_H(u)={{{w}}}_{K'}(u)$ and ${{{w}}}_K(u)={{{w}}}_{H'}(u)$ for each $u\in U$}\label{cc_alternative4}.
\end{align}
Let us assume that \eqref{cc_alternative3} holds. (The other case can be treated similarly.)
Formulas~\eqref{rappresentazione_mappa_analitica}, \eqref{union_of_zero_curves}, \eqref{locally_diff_widths} and \eqref{cc_alternative3} imply that for each integer $m\geq m_1$ and for each $u\in U$ we have
\begin{equation}\label{id1}
 F_{m,H}(u)=F_{m,H'}(u)\quad\text{ and }\quad F_{m,-K}(u)=F_{m,-K'}(u)
\end{equation}
Since the four maps appearing in~\eqref{id1} are analytic maps from ${S^{1}}$ to ${\mathbb{C}}^2$, we have $F_{m,H}(u)=F_{m,H'}(u)$ and $F_{m,-K}(u)=F_{m,-K'}(u)$ for each $u\in{S^{1}}$. The equalities of the real parts imply
\begin{equation}\label{equal_widths_global}
 {{{w}}}_H={{{w}}}_{H'}\quad\text{ and }\quad{{{w}}}_K={{{w}}}_{K'}.
\end{equation}
The equalities of the imaginary parts imply
\begin{equation}\label{equal_ratio_curv_2}
\frac{{\tau}_{H}(-u)}{{\tau}_{H}(u)}=\frac{{\tau}_{H'}(-u)}{{\tau}_{H'}(u)}\text{ and }
\frac{{\tau}_{-K}(-u)}{{\tau}_{-K}(u)}=\frac{{\tau}_{-K'}(-u)}{{\tau}_{-K'}(u)}
\end{equation}
for each $u\in{S^{1}}$. 
By Proposition~\ref{teo_information_c2+_bodies}-\eqref{prop_sum_reverse_weingarten} the identities \eqref{equal_widths_global} imply
\begin{equation*}
\begin{aligned}
\frac1{{\tau}_{H}(u)}+\frac1{{\tau}_{H}(-u)}&=\frac1{{\tau}_{H'}(u)}+\frac1{{\tau}_{H'}(-u)}, \\
\frac1{{\tau}_{-K}(u)}+\frac1{{\tau}_{-K}(-u)}&=\frac1{{\tau}_{-K'}(u)}+\frac1{{\tau}_{-K'}(-u)}
\end{aligned}
\end{equation*}
for each $u\in{S^{1}}$. All these conditions imply ${\tau}_H={\tau}_{H'}$ and ${\tau}_{-K}={\tau}_{-K'}$.
The uniqueness part in Minkowski's Theorem \cite[Th. 7.2.1]{Sc} implies   $H=H'+x_1$ and $K=K'+x_2$, for suitable $x_1,x_2\in{\mathbb{R}}^2$.
The identity $H+(-K)={{\operatorname{supp}}}\, g_{H,K}={{\operatorname{supp}}}\, g_{H',K'}=H'+(-K')$ implies $x_1=x_2$. This concludes the proof under Assumption~\eqref{locally_diff_widths}.

If~\eqref{locally_diff_widths} does not hold, then~\eqref{equal_sums_widths} implies
\begin{equation}\label{all_equal_widths_global}
{{{w}}}_H={{{w}}}_{H'}={{{w}}}_K={{{w}}}_{K'}.
\end{equation}
We again distinguish two cases according to whether 
\begin{equation}\label{equal_ratio_curv}
\frac{{\tau}_{H}(-u)}{{\tau}_{H}(u)}=\frac{{\tau}_{-K}(-u)}{{\tau}_{-K}(u)}
\end{equation}
holds for each $u\in {S^{1}}$ or not.
If~\eqref{equal_ratio_curv} holds for each $u\in {S^{1}}$ then, arguing as we have done above we conclude that $H=-K$. This implies 
\[
 F_{m,H}=F_{m,-K}
\]
for each $m$. This, \eqref{rappresentazione_mappa_analitica}, \eqref{union_of_zero_curves} and  \eqref{all_equal_widths_global} imply $F_{m,H}=F_{m,H'}=F_{m,-K'}$ for each $m$ sufficiently large. This implies ${\tau}_{H}(-u)/{\tau}_{H}(u)={\tau}_{H'}(-u)/{\tau}_{H'}(u)={\tau}_{-K'}(-u)/{\tau}_{-K'}(u)$ for each $u\in {S^{1}}$ and $H=H'=-K'$. The proof is concluded in this case too.

It remains to consider the possibility that there exists a relatively open subset $U$ of ${S^{1}}$ such that~\eqref{equal_ratio_curv} is false  for each $u\in U$. This and~\eqref{all_equal_widths_global} imply that when $u\in U$ the real parts of $F_{m,H}(u)$ and $F_{m,-K}(u)$ coincide  but their imaginary parts differ.
Formula~\eqref{union_of_zero_curves} and the analyticity of $F_{m,H}$, $F_{m,-K}$, $F_{m,H'}$ and $F_{m,-K'}$ imply that we have
\begin{gather}
\text{$F_{m,H}=F_{m,H'}$ and $F_{m,-K}=F_{m,-K'}$ for infinitely many $m$}\label{cc_alternative5}\\
\text{or $F_{m,H}=F_{m,-K'}$ and $F_{m,-K}=F_{m,H'}$ for infinitely many $m$.}\label{cc_alternative6}
\end{gather}
If~\eqref{cc_alternative5} holds then we have~\eqref{equal_ratio_curv_2} and we conclude as before. When~\eqref{cc_alternative6} holds the proof is concluded by similar arguments.
\end{proof}

\section{The Covariogram problem and irreducibility of ${\widehat{{{1_K}}}}$}
\label{sec_phase_retr}

We say that an entire function $g$ is \emph{irreducible} if $g$ cannot be written as the product of two entire functions $g_1$, $g_2$ with $g_1\neq{{\alpha}} g$, for each ${{\alpha}}\in{\mathbb{C}}$,  and both $\{{{\zeta}}\in{\mathbb{C}}^n : g_1({{\zeta}})=0\}$ and $\{{{\zeta}}\in{\mathbb{C}}^n : g_2({{\zeta}})=0\}$ non-empty. Let $f\in L^2({\mathbb{R}}^n)$ have compact support.
Sanz and Huang~\cite{Sanz-Huang-1984} proves that if ${\widehat{{{f}}}}$ is irreducible then $f$ is determined, up to trivial associates, by the knowledge of $|{\widehat{{{f}}}}(x)|$ for all $x\in{\mathbb{R}}^n$.
Barakat and Newsam~\cite{Barakat-Newsam-1984} and Stefanescu~\cite{Stefanescu-1985} prove that if $f_1$ and $f_2$ belong to $L^2({\mathbb{R}}^2)$, have compact support, are not trivial associates and $|{\widehat{{{f_1}}}}(x)|=|{\widehat{{{f_2}}}}(x)|$ for all $x\in{\mathbb{R}}^2$, then there exist two entire functions $g_1$ and $g_2$ such that $\{{{\zeta}}\in{\mathbb{C}}^2 : g_1({{\zeta}})=0\}$ and $\{{{\zeta}}\in{\mathbb{C}}^2 : g_2({{\zeta}})=0\}$ are both non-empty and
\begin{equation}\label{factor_barakat}
{\widehat{{{f_1}}}}({{\zeta}})=g_1({{\zeta}})g_2({{\zeta}})\quad\text{ and }\quad {\widehat{{{f_2}}}}({{\zeta}})=e^{{\operatorname{i}}(c+ \left<d,{{\zeta}}\right>)}g_1({{\zeta}}){\overline{{g_2\left({\overline{{{\zeta}}}}\right)}}},
\end{equation}
for a suitable $c\in{\mathbb{R}}$ and $d\in{\mathbb{R}}^2$. Stefanescu~\cite{Stefanescu-personal} believes that a similar result holds true in any dimension $n\geq2$. It is not known whether the property that ${\widehat{{{f}}}}$ is not irreducible implies that $f$ is not determined by $|{\widehat{{{f}}}}|$.

What is the significance of these results for the covariogram problem? Assume that $n=n_1+n_2$, with $n_1$, $n_2$ positive integers, and that the convex body $K\subset{\mathbb{R}}^n={\mathbb{R}}^{n_1}\times{\mathbb{R}}^{n_2}$ can be written as
\begin{equation}\label{decomp_direct_sum}
 K=K_1+K_2
\end{equation}
with $K_1\subset {\mathbb{R}}^{n_1}$ and $K_2\subset{\mathbb{R}}^{n_2}$ convex bodies which are not centrally symmetric.
Then  $K'=K_1+(-K_2)$ is not  a translation or reflection of $K$ and $g_K=g_{K'}$ (see Bianchi~\cite{Bianchi-2009-polytopes}). All known examples of non-determination for the covariogram problem arise, up to a linear transformation, by a decomposition as in~\eqref{decomp_direct_sum}. This decomposition generates a factorization of ${\widehat{{{1_K}}}}$ as in \eqref{factor_barakat}. Indeed
\[
 1_{K}=\delta_{K_1}\ast \delta_{K_2}\quad\text{and}\quad 1_{K'}=\delta_{K_1}\ast \delta_{-{K_2}},
\]
where $\delta_{K_1}$ and ${{\delta}}_{K_2}$ are the distributions defined for $\phi\in C^\infty_0({\mathbb{R}}^{n})$ by
\[
 \delta_{K_1}(\phi)=\int_{K_1} \phi(x,0)\, dx,\quad \delta_{K_2}(\phi)=\int_{K_2} \phi(0,y)\, dy
\]
(here $x\in {\mathbb{R}}^{n_1}$, $y\in{\mathbb{R}}^{n_2}$ and $dx$ and $dy$ denote, respectively, Lebesgue measure in ${\mathbb{R}}^{n_1}$ and in ${\mathbb{R}}^{n_2}$) and ${{\delta}}_{-{K_2}}$ is defined similarly. By the Paley-Wiener Theorem ${\widehat{{{{{\delta}}_{K_1}}}}}$, ${\widehat{{{{{\delta}}_{K_2}}}}}$ and ${\widehat{{{{{\delta}}_{-{K_2}}}}}}$ are entire functions in ${\mathbb{C}}^{n}$ of exponential type. Clearly ${\widehat{{{\delta_{-{K_2}}}}}}({{\zeta}})={\overline{{{\widehat{{{\delta_{K_2}}}}}\left({\overline{{{\zeta}}}}\right)}}}$ and we have
\[
{\widehat{{{1_{K}}}}}={\widehat{{{\delta_{K_1}}}}}{\widehat{{{\delta_{K_2}}}}}\quad\text{and}\quad {\widehat{{{1_{K'}}}}}({{\zeta}})={\widehat{{{\delta_{K_1}}}}}({{\zeta}}) {\overline{{{\widehat{{{\delta_{K_2}}}}}\left({\overline{{{\zeta}}}}\right)}}},
\]
as in~\eqref{factor_barakat}.

In view of these results it would be interesting to \emph{find explicit geometric conditions on a convex body $K$ which grants that ${\widehat{{{1_K}}}}$ is irreducible}. Regarding the difficulty in answering to this question, consider the following subproblem.
\smallskip

\emph{Understand for which convex bodies $K$ the function  ${\widehat{{{1_K}}}}$ is the product of a non-trivial polynomial and an entire function.}
\smallskip

Let us introduce some notation. Given a polynomial $P({{\zeta}})=\sum_{|l|\leq m}c_l {{\zeta}}^l$, where $m$ is a positive integer,  $l=(l_1,\dots,l_n)$ denotes a multi-index, $c_l\in{\mathbb{C}}$,  $|l|=l_i+\dots+l_n$ and ${{\zeta}}^l={{\zeta}}_1^{l_1}\dots{{\zeta}}_n^{l_n}$, let  $P(D)$ denote the differential operator
\[
P(D)=\sum_{|l|\leq m}({\operatorname{i}})^{-|l|}c_l\left({{\partial}}^{l_1}/{{\partial}} x_1^{l_1}\right)\dots\left({{\partial}}^{l_n}/{{\partial}} x_n^{l_n}\right),
\]
where ${{\partial}}^{0}/{{\partial}} x_i^{0}$ denotes the identity operator.
\cite[Theorem~8.4]{Rudin-91} states that
\[
{\widehat{{{1_K}}}}=Pf,
\]
with $f$ entire  and $P$ a polynomial, if and only if the problem
\begin{equation}\label{differential_problem}
 P(D)u=1_K,
\end{equation}
has a solution $u$ in the class of distributions with support contained in $K$. Here ${\widehat{{{u}}}}=f$ and~\eqref{differential_problem} has to be understood in the sense of distributions. The Theorem of supports for convolutions~\cite[Theorem~4.3.3]{Hormander-1983} and elementary considerations imply that if a solution $u$ to~\eqref{differential_problem} exists then its support is $K$. 

A particular instance of this problem has received much attention. When $P({{\zeta}})={{\zeta}}_1^2+\dots+{{\zeta}}_n^2-c$, for some $c>0$, \eqref{differential_problem} becomes
\begin{equation}\label{schiffer_conj}
\begin{cases}
 \Delta u+c u=-1 &\text{in $K$}\\
 u=\frac{{{\partial}} u}{{{\partial}} \nu}=0 &\text{on ${{\partial}} K$}
\end{cases}
\end{equation}
($\nu$ denotes the exterior normal to ${{\partial}} K$). Let $E\subset{\mathbb{R}}^n$ be a bounded simply connected Lipschitz domain. The Pompeiu problem is a conjecture asserting that there exists  a non-zero continuous function $f:{\mathbb{R}}^n\to{\mathbb{R}}$ such that
\[
\int_{{{\mathcal T}}(E)} f\, dx=0\quad\text{for all rigid motions ${{\mathcal T}}$ in ${\mathbb{R}}^n$}
\]
only when $E$ is a ball. It is known that the Pompeiu problem is equivalent to proving that a solution to~\eqref{schiffer_conj} exists for some $c>0$ only if $K$ is a ball (see Berenstein~\cite{Berenstein-1980}). Up to our knowledge these problems are still open.

The example of a ball implies that the irreducibility condition is not necessary for determination by covariogram. Indeed, when $K$ is a ball a solution to~\eqref{schiffer_conj} exists and ${\widehat{{{1_K}}}}$ factors. On the other hand, in any dimension a ball $K$ is uniquely determined by $g_K$, as  Theorem~\ref{teo_radial_symmetry} implies.

\begin{center}
\textsc{Acknowledgements}
\end{center}

We thanks Ralph Howard, Alexander Koldobsky and Ion Sabba Stefanescu  for discussions on some aspects of this paper.

\bibliographystyle{amsplain}
\begin{thebibliography}{Bia09a}
\footnotesize

\bibitem[AB09]{averkov-bianchi-2009} G.~Averkov and G.~Bianchi, \emph{Confirmation of {M}atheron's conjecture on the covariogram of a planar convex body}, J. Eur. Math. Soc. {11} (2009), 1187--1202.

\bibitem[BLP09]{Benguria-Levitin-Parnovski-2009} R.~Benguria, M.~Levitin, and L.~Parnovski, \emph{Fourier transform, null variety, and Laplacian's eigenvalues}, J. Funct. Anal. {257} (2009), 2088--2123.

\bibitem[Ber80]{Berenstein-1980} C.~A.~Berenstein, \emph{An inverse spectral theorem and its relation to the Pompeiu problem}, J. Analyse Math. {37} (1980), 128--144.

\bibitem[BG07]{BaakeGrimm} M.~Baake and U.~Grimm, \emph{Homometric model sets and window covariograms}, Z. Krist. {222} (2007), 54--58.

\bibitem[BN84]{Barakat-Newsam-1984}  R.~Barakat and G.~Newsam, \emph{Necessary conditions for a unique solution to two-dimensional phase recovery}, J. Math. Phys. 25 (1984), 3190--3193.

\bibitem[Bia05]{Bianchi-2005} G.~Bianchi, \emph{Matheron's conjecture for the covariogram problem}, J. London Math. Soc. (2) {71} (2005), 203--220.

\bibitem[Bia09a]{Bianchi-2009-polytopes} \bysame, \emph{The covariogram determines three-dimensional convex polytopes}, Adv. Math. {220} (2009), 1771--1808.
 
\bibitem[Bia09b]{B4} \bysame, \emph{The cross covariogram of a pair of polygons determines both polygons, with a few exceptions}, Adv. in Appl. Math. {42} (2009), 519--544.
 
\bibitem[BSV02]{Bianchi-Segala-Volcic-2002} G.~Bianchi, F.~Segala and A.~Vol\v{c}i\v{c}, \emph{The solution of the covariogram problem for plane $C^2_+$ bodies}, J. Differential Geom. {60} (2002), 177--198.

\bibitem[BGK11]{BiGaKi11} G.~Bianchi, R.~J.~Gardner, and M.~Kiderlen, \emph{Phase retrieval for characteristic functions of convex bodies and reconstruction from covariograms}, J. Amer. Math. Soc. {24} (2011), 293--343.

\bibitem[Gar06]{Gar95ed2} R.~J.~Gardner, \emph{Geometric tomography}, second ed., Encyclopedia of Mathematics and its Applications, vol.~58. Cambridge University Press, Cambridge, 2006.

\bibitem[GS91]{Garofalo-Segala-1991}  N.~Garofalo and F.~Segala, \emph{Asymptotic expansions for a class of Fourier integrals and applications to the Pompeiu problem}, J. Analyse Math. {56} (1991), {1--28},

\bibitem[GSW97]{Goodey-Schneider-Weil-1997} P.~Goodey, R.~Schneider, and W.~Weil,  \emph{On the determination of convex bodies by projection functions}, Bull. London Math. Soc. {29} (1997), 82--88.

\bibitem[Her62]{Herz-1962} C.~S.~Herz, \emph{Fourier transforms related to convex sets}, Ann. of Math. {75} (1962), 81--92.

\bibitem[Hor83]{Hormander-1983} L.~H\"{o}rmander, \emph{The analysis of linear partial differential operators I}, Springer-Verlag, Berlin, 1983.
 
\bibitem[Kob89]{Kob1} T.~Kobayashi, \emph{Asymptotic behaviour of the null variety for a convex domain in a non-positively curved space form}, J. Fac. Sci. Univ. Tokyo Sect. IA Math. 36 (1989), 389--478. 

\bibitem[Kob91]{Kob2} \bysame, \emph{Convex domains and the Fourier transform on spaces of constant curvature}, Lecture notes of the Unesco-Cimpa School on ''Invariant differential operators on Lie groups and homogeneous spaces''  at WuHan University in P. R. China, 1991 (P.~Torasso, ed.), \url{http://www.ms.u-tokyo.ac.jp/~toshi/pub/21.html}.

\bibitem[Kol05]{Kol05} A.~Koldobsky, \emph{Fourier analysis in convex geometry}, Mathematical Surveys and Monographs, 116. American Mathematical Society, Providence, RI, 2005.

\bibitem[Law81]{Lawton-1981}  W.~Lawton, \emph{Uniqueness results for the phase-retrieval problem for radial functions}, J. Opt. Soc. Amer. 71 (1981), 1519--1522.

\bibitem[Mat75]{Matheron-1975} G.~Matheron, \emph{Random sets and integral geometry}, Wiley, New York, 1975.

\bibitem[McK56]{McK56} M.~McKiernan, \emph{On the $n$-th derivative of composite functions}, Amer. Math. Monthly 63 (1956), 331--333.

\bibitem[Rud91]{Rudin-91} W.~Rudin, \emph{Functional analysis},  second ed., International Series in Pure and Applied Mathematics. McGraw-Hill, Inc., New York, 1991.

\bibitem[Ste85]{Stefanescu-1985}  I.~S.~Stefanescu, \emph{On the phase retrieval problem in two dimensions}, J. Math. Phys. 26 (1985), 2141--2160.

\bibitem[Ste13]{Stefanescu-personal} \bysame, \emph{personal communication,} 2013.

\bibitem[SH84]{Sanz-Huang-1984} J.~L.~C.~Sanz and T.~S.~Huang, \emph{Phase reconstruction from magnitude of band-limited multidimensional signals}, J. Math. Anal. Appl. 104 (1984), 302--308.

\bibitem[Sch11]{Schymura-2011}
D.~Schymura, \emph{Probabilistic Matching of Solid Shapes in Arbitrary Dimension}, Ph.D. thesis, Freie Universit{\"a}t Berlin, 2011.

\bibitem[Sch93]{Sc}
R.~Schneider, \emph{Convex bodies: the Brunn-Minkowski theory}, Cambridge University Press, Cambridge, 1993.

\end{thebibliography}

\end{document}


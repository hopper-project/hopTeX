\documentclass[12pt]{amsart}
\usepackage{amsfonts,amssymb,amsmath,amsthm}
\usepackage{url}
\usepackage{enumerate}
\usepackage{hyperref}
\usepackage{fancyhdr}
\urlstyle{sf}
\newtheorem{theorem}{Theorem}[section]
\newtheorem{lemma}[theorem]{Lemma}
\newtheorem{conjecture}[theorem]{Conjecture}
\newtheorem{proposition}[theorem]{Proposition}
\newtheorem{corollary}[theorem]{Corollary}
\newtheorem{example}[theorem]{Example}
\newtheorem{examples}[theorem]{Examples}
\theoremstyle{definition}
\newtheorem{definition}[theorem]{Definition}
\newtheorem{definitions}[theorem]{Definitions}
\newtheorem{heuristic}[theorem]{Heuristic}
\newtheorem{remark}[theorem]{Remark}
\newtheorem{remarks}[theorem]{Remarks}
\numberwithin{equation}{section}
\textheight=630pt
\textwidth=32.5pc
\voffset=-15mm
\parindent=0cm

\let\sst=\scriptscriptstyle

\let\ds=\displaystyle
\let\wt=\widetilde

\author[Georges Gras]{Georges Gras}
\address{Villa la Gardette \\ chemin Ch\^ateau Gagni\`ere \\ F--38520 Le Bourg d'Oisans.
{\rm\url{http://www.researchgate.net/profile/Georges_Gras}}}
\email{g.mn.gras@wanadoo.fr}

\keywords{Global units; Leopoldt conjecture; Kummer--Leopoldt constant; 
$p$-adic regulator; Abelian $p$-ramification}

\subjclass{Primary 11R37; Secondary 11R27}

\begin{document}
 
\title[The $p$-adic Kummer--Leopoldt Constant] 
{The $p$-adic Kummer--Leopoldt Constant  \\  Normalized $p$-adic Regulator}

\date{January 24, 2017}

\begin{abstract} The $p$-adic Kummer--Leopoldt constant $\kappa_K$ 
of a number field $K$ is (under the Leopoldt conjecture for $p$ in $K$) the 
least integer $c$ such that for all $n$ large enough, any global unit 
$\varepsilon$ of $K$, which is locally (at the $p$-places) a $p^{n+c}$th power, is 
necessarily a $p^n$th power of a global unit of~$K$. This constant has been 
computed by J. Assim and T. Nguyen Quang Do using Iwasawa's technics. 
In this short Note, we give an elementary $p$-adic proof of their result and
the class field theory interpretation of $\kappa_K$ by means of 
Abelian $p$-ramification theory over $K$. 
Then, from this, we give a natural definition of the normalized $p$-adic 
regulator of $K$ for any $K$ and any $p\geq 2$.
\end{abstract}

\maketitle

\section{Notations}\label{section1} Let $K$ be a number field and let
$p\geq 2$ be a prime number.
We assume in the sequel that $K$ satisfies the Leopoldt conjecture at $p$. Denote by 
$E_K$ the group of principal global units of $K$ (i.e., units 
$\varepsilon \equiv 1 \! \pmod{ \prod_{{\mathfrak p} \mid p} {\mathfrak p}}$) and by 
$$U_K := \Big \{u \in {\displaystyle\mathop{\raise 2.0pt \hbox{$\prod$}}\limits}_{{\mathfrak p} \mid p}K_{\mathfrak p}^\times, \ \,
u = 1+x, \  x \in {\displaystyle\mathop{\raise 2.0pt \hbox{$\prod$}}\limits}_{{\mathfrak p} \mid p} {\mathfrak p} \Big\}\, \ \ \& \ \ \,
W_K = {\rm tor}_{{\mathbb{Z}}_p}^{}(U_K),$$ 

the ${\mathbb{Z}}_p$-module of local units at $p$ and its  torsion subgroup, where $K_{\mathfrak p}$ 
is the ${\mathfrak p}$-completion of $K$. 

\smallskip
The $p$-adic logarithm ${\rm log}$ is defined on $1+x$, 
$x \in \prod_{{\mathfrak p} \mid p} {\mathfrak p}$, 
by means of the series ${\rm log} (1+ x) = 
{\displaystyle\mathop{\raise 2.0pt \hbox{$\sum$}}\limits}_{i \ge 1}\, (-1)^{i+1} \, {\hbox{\footnotesize $\displaystyle \frac{{x^i}}{{i}}$}} \in {\displaystyle\mathop{\raise 2.0pt \hbox{$\prod$}}\limits}_{{\mathfrak p} \mid p}K_{\mathfrak p}$.
The kernel of ${\rm log}$ in $U_K$ is $W_K$.

\smallskip
We consider the diagonal embedding $E_K \to U_K$ and its natural extension 
$E_K \otimes {\mathbb{Z}}_p \to U_K$ whose image is $\overline E_K$, the topological 
closure of $E_K$ in $U_K$. In the sequel, these embeddings shall be understood.

\section{The Kummer--Leopoldt constant}\label{section2}

This notion comes from the Kummer lemma (see \cite[Theorem 5.36]{W3}), that is to say,
if the prime number $p$ is ``regular'', the cyclotomic field $K={\mathbb{Q}}(\mu_p)$ 
satisfies the following property for the whole group $E'_K$ of global units of $K$: 

\smallskip
\centerline{\it any $\varepsilon \in E'_K$, congruent to a rational
integer modulo $p$, is a $p$th power in $E'_K$.}

\smallskip
In fact, $\varepsilon \equiv a \pmod p$, with $a \in {\mathbb{Z}}$, implies 
$\varepsilon^{p-1} \equiv a^{p-1} \equiv 1 \pmod p$. So we shall write, equivalently, the 
Kummer property with principal units:

\smallskip
\centerline{\it any $\varepsilon \in E_K$, congruent to $1$ modulo $p$, is a $p$th power in $E_K$.}

\medskip
From \cite{A}, \cite{L}, \cite{O}, \cite{S}, \cite{W1}, \cite{W2} 
(see the rather intricate history in \cite{AN}) one can study this property 
and its generalizations with various technics. 
Give the following definition (from \cite{AN}):

\begin{definition} \label{defkappa}
Let $K$ be a number field satisfying the Leopoldt conjecture at the prime
$p\geq 2$. Let $E_K$ be the group of principal global units of $K$.
We call Kummer--Leopoldt constant (denoted $\kappa = \kappa_K$), 
the smallest integer $c$ such that the following condition is fulfilled:

\smallskip
\centerline{\it for all $n \gg 0$, any unit $\varepsilon \in E_K$, such that 
$\varepsilon \in U_K^{p^{n+c}}$, is necessarily in $E_K^{p^n}$.}
\end{definition}

\begin{remark}\label{rema}
The existence of $\kappa$ comes from the classical characterization 
 of Leopoldt's conjecture given, for instance, in \cite[Theorem III.3.6.2 (iv)]{Gr1}.
If the Leopoldt conjecture is not satisfied, we can find a sequence
$\varepsilon_n \in E_K \setminus E_K^p$ such that ${\rm log} (\varepsilon_n)
\to 0$ (i.e., $\varepsilon_n \in U_K^{p^m} \!\cdot W_K$, with $m \to \infty$ 
when $n\to\infty$); since $W_K$ is finite, taking a suitable power of 
$\varepsilon_n$, we see that $\kappa$ does not exist in that case.
\end{remark}

We have the following $p$-adic result giving $p^\kappa$ under 
the Leopoldt conjecture:

\begin{theorem} \label{thm} 
Denote by $E_K$ the group of principal global units of $K$, then by $U_K$ 
the ${\mathbb{Z}}_p$-module of principal local units at $p$ and by $W_K$ its torsion subgroup.
Let $\kappa_K$ (Definition \ref{defkappa}) be the Kummer--Leopoldt constant; 
then $p^{\kappa_K}$ is the exponent of the finite group 
${\rm tor}^{}_{{\mathbb{Z}}_p} \big ({\rm log} (U_K) / {\rm log} (\overline E_K)\big )$, 
where $\overline E_K$ is the closure of $E_K$ in $U_K$ (hence 
${\rm log} (\overline E_K) = {\mathbb{Z}}_p\, {\rm log} (E_K)$).
\end{theorem}

\begin{proof} (i) ($\kappa$ is suitable). Let $n \gg 0$ and let $\varepsilon \in E_K$ 
be such that $\varepsilon = u^{p^{n+\kappa}}$, $u \in U_K$ 
(thus $\varepsilon$ is arbitrary near from 1, depending on the choice of $n\gg 0$).
Then ${\rm log}(\varepsilon) =
p^n \cdot (p^\kappa \cdot {\rm log}(u)) = p^n \cdot  {\rm log}(\overline \eta)$
with $\overline \eta \in \overline E_K$. So, writting 
$$\hbox{$\overline \eta = \eta(N)\cdot u_N$, $\ \eta(N) \in E_K$,
$\  u_N \equiv 1\!\! \pmod {p^N}$, $\ N\to\infty$,}$$

we get
${\rm log}(\varepsilon) = p^n \cdot {\rm log}(\eta(N)) +p^n \cdot  {\rm log}(u_N)$ giving 
$$\hbox{$\varepsilon = \eta(N)^{p^n}\! \cdot u_N^{p^n}\! \cdot \xi$, 
$\ \xi \in W_K$.} $$

But $\xi$ is near from $1$, whence $\xi=1$. Then
$$\hbox{$\varepsilon = \eta(N)^{p^n}\!\! \cdot u'_N$, $\ u'_N \to 1$ as $N\to\infty$;}$$

so $u'_N = \varepsilon  \cdot \eta(N)^{-p^n}$ is a global unit, arbitrary near from 1, 
hence, because of the Leopoldt conjecture (loc. cit. in Remark \ref{rema}), 
of the form $\varphi_N^{p^n}$ with $\varphi_N \in E_K$, $\varphi_N\to1$ as $N \to\infty$ 
(recall that $n$ is arbitrary but fixed), giving $\varepsilon = 
\eta(N)^{p^n} \cdot \varphi_N^{p^n}$, whence:
$$\varepsilon \in E_K^{p^n}. $$

(ii) ($\kappa$ is the least solution). Suppose there exists $c < \kappa$ 
having the property given in Definition \ref{defkappa}.

\smallskip
Let $u_0 \in U_K$ be such that ${\rm log}(u_0)$ is of {\it order} $p^\kappa$ in
${\rm  tor}^{}_{{\mathbb{Z}}_p} \big ({\rm log} (U_K) / {\rm log} (\overline E_K)\big )$;
then ${\rm log}(u_0^{p^\kappa}) = {\rm log}(\overline \varepsilon_0)$, 
$\overline \varepsilon_0 \in \overline E_K$. This is equivalent to
$$\hbox{ $u_0^{p^\kappa} = \overline \varepsilon_0 \cdot \xi_0 = 
\varepsilon(N) \cdot u_N \cdot \xi_0$, $\ \varepsilon(N)\in E_K$,
$\ u_N \in U_K$, $\ \xi_0 \in W_K$} $$

(with $u_N \to 1$ when $N\to\infty$), hence, for any $n\gg0$, 
$$u_0^{p^{n+\kappa}} = \varepsilon(N)^{p^n} \cdot u_N^{p^n}. $$

Taking $N$ large enough, we can suppose that 
$u_N = v_N^{p^{2 \kappa}}$, $v_N \in U_K$ arbitrary near from 1;
because of the above relation, ${\rm log}(v_N)$ is of finite order modulo 
${\rm log} (\overline E_K)$, thus  ${\rm log} (v_N^{p^\kappa}) \in {\rm log} (\overline E_K)$.
This is sufficient, for $u'_0 := u_0 \cdot v_N^{-p^{\kappa}}$, to get ${\rm log}(u'_0)$ of 
{\it order} $p^\kappa$ modulo ${\rm log} (\overline E_K)$. So we can write:
$$\varepsilon(N)^{p^n} = u_0^{p^{n+\kappa}}\cdot u_N^{-p^n} 
= u_0^{p^{n+\kappa}}\cdot (v_N^{-p^\kappa})^{p^{n+\kappa}}
= u'^{p^{n+\kappa}}_0 \in U_K^{p^{n+(\kappa-c)+c }}, $$

but, by assumption on $c$, we obtain $\varepsilon(N)^{p^n} = \eta_0^{p^{n+(\kappa-c)}}$, 
$\eta_0 \in E_K$; thus, the above relation $u'^{p^{n+\kappa}}_0 = 
u'^{p^{n+(\kappa-c)+c}}_0 = \eta_0^{p^{n+(\kappa-c)}}$, yields to:

\medskip
\centerline{$p^c \cdot {\rm log}(u'_0) = {\rm log}(\eta_0) \in {\rm log}(E_K)$ (absurd).}

\vspace{-0.45cm}
\end{proof}

\section{Interpretation of $\kappa_K$ -- Fundamental exact sequence}\label{section3}

Consider the following diagram, under the Leopoldt conjecture for $p$ in $K$
(diagram given in \cite{Gr1}, \S\,III.2, (c), \!Fig.~2.2):
\unitlength=0.9cm 

$$\vbox{\hbox{\hspace{-2.8cm}  \begin{picture}(11.5,5.8)
\put(6.4,4.50){\line(1,0){1.3}}
\put(8.7,4.50){\line(1,0){2.0}}
\put(3.85,4.50){\line(1,0){1.4}}
\put(9.2,4.15){\footnotesize${\mathcal W}_K$}
\put(4.2,2.50){\line(1,0){1.25}}

\bezier{350}(3.8,4.8)(7.6,6.6)(11.0,4.8)
\put(7.2,5.8){\footnotesize${\mathcal T}_K$}

\bezier{350}(6.3,4.7)(8.6,5.5)(10.8,4.7)
\put(8.2,5.25){\footnotesize${\mathcal T}_K^{\sst E}$}
\put(3.50,2.9){\line(0,1){1.25}}
\put(3.50,0.9){\line(0,1){1.25}}
\put(5.7,2.9){\line(0,1){1.25}}

\bezier{300}(3.9,0.5)(4.7,0.5)(5.6,2.3)
\put(5.2,1.3){\footnotesize${{\mathcal C}\hskip-2pt{\ell}}_K$}

\bezier{300}(6.2,2.5)(8.5,2.6)(10.8,4.3)
\put(8.4,2.7){\footnotesize$U_K/\overline E_K$}

\put(10.85,4.4){$H_K^{\rm pr}$}
\put(5.3,4.4){$\widetilde K\! H_K$}
\put(7.9,4.4){$H_K^{\sst E}$}
\put(6.8,4.14){\footnotesize${\mathcal R}_K$}
\put(3.3,4.4){$\widetilde K$}
\put(5.5,2.4){$H_K$}
\put(2.7,2.4){$\widetilde K \!\cap \! H_K$}
\end{picture}   }} $$
\unitlength=1.0cm

where $\widetilde K$ is the compositum of the ${\mathbb{Z}}_p$-extensions of $K$, 
${{\mathcal C}\hskip-2pt{\ell}}_K$ the $p$-class group in the ordinary sense,
$H_K$ the $p$-Hilbert class field, $H_K^{\rm pr}$ the maximal Abelian $p$-ramified
(i.e., unramified outside $p$) pro-$p$-extension of $K$; then we put
${\mathcal T}_K^{\sst E} := {\rm  tor}^{}_{{\mathbb{Z}}_p} ({\rm Gal}(H_K^{\rm pr} / H_K)) 
\subseteq {\mathcal T}_K :=  {\rm  tor}^{}_{{\mathbb{Z}}_p}({\rm Gal}(H_K^{\rm pr}/K))$, and
${\mathcal R}_K \simeq {\rm Gal}(H_K^E / \widetilde KH_K)$. 

\smallskip
By class field theory, ${\rm Gal}(H_K^{\rm pr} / H_K)$ is canonically isomorphic to 
$U_K/\overline E_K$ in which the image of ${\mathcal W}_K := W_K / \mu_K$ 
(where $\mu_K$ is the group of roots of unity of $K$ of
$p$-power order) fixes $H_K^{\sst E}$, which is (except possibly if $p=2$ in the 
``special case'') equal to the Bertrandias--Payan field $H_K^{\rm bp}$, 
${\rm Gal}(H_K^{\rm bp} / \widetilde K)$ being then the 
Bertrandias--Payan module (also evoked in \cite{AN} about the calculation of 
$\kappa$; see also \cite[R\'ef\'erences]{Gr2} for some history about this module). 

\begin{lemma} We have (without any assumption on $K$ and $p$) 
the following exact sequence:
\begin{align*}
 1{\relbar{\mathrel{\mkern-4mu}}\rightarrow} W_K \big / {\rm tor}_{{\mathbb{Z}}_p}^{}(\overline E_K) {\relbar{\mathrel{\mkern-4mu}}\relbar{\mathrel{\mkern-4mu}}{\relbar{\mathrel{\mkern-4mu}}\rightarrow}} &
\  {\rm tor}_{{\mathbb{Z}}_p}^{} \big(U_K \big / \overline E_K \big) \\ 
&  \mathop {\relbar{\mathrel{\mkern-4mu}}\relbar{\mathrel{\mkern-4mu}}{\relbar{\mathrel{\mkern-4mu}}\rightarrow}}^{{\rm log}} \  {\rm tor}_{{\mathbb{Z}}_p}^{}\big({\rm log}\big 
(U_K \big) \big / {\rm log} (\overline E_K) \big) {\relbar{\mathrel{\mkern-4mu}}\rightarrow} 0.
\end{align*}
\end{lemma}

\begin{proof} From \cite[Lemma 4.2.4]{Gr1}: the surjectivity comes from the fact that if
$u \in U_K$ is such that $p^n {\rm log}(u) = {\rm log}(\overline\varepsilon)$, 
$\overline\varepsilon \in \overline E_K$, then $u^{p^n} = \overline\varepsilon \cdot \xi$ 
for $\xi \in W_K$, hence there exists $m\geq n$ such that $u^{p^m} \in \overline E_K$, 
whence $u \in  {\rm tor}_{{\mathbb{Z}}_p}^{} \big(U_K \big / \overline E_K \big)$; $u$ is a preimage.
If $u$ is such that ${\rm log}(u) \in {\rm log}(\overline E_K)$, 
$u = \overline \varepsilon \cdot \xi$ as above, giving the kernel equal to
$\overline E_K \cdot W_K /\overline E_K = W_K/ {\rm tor}_{{\mathbb{Z}}_p}^{}(\overline E_K)$.
\end{proof}

Assuming the Leopoldt conjecture, we have
${\rm tor}_{{\mathbb{Z}}_p}^{}(\overline E_K) = \mu_K$ (\cite[Corollary III.3.6.3]{Gr1}).
Thus, $W_K \big / {\rm tor}_{{\mathbb{Z}}_p}^{}(\overline E_K) = W_K /\mu_K = {\mathcal W}_K$
and the exact sequence in the Lemma becomes, with the notations of the diagram:
$$1{\relbar{\mathrel{\mkern-4mu}}\rightarrow} {\mathcal W}_K  {\relbar{\mathrel{\mkern-4mu}}\relbar{\mathrel{\mkern-4mu}}{\relbar{\mathrel{\mkern-4mu}}\rightarrow}} 
{\mathcal T}_K^{\sst E}  \mathop {\relbar{\mathrel{\mkern-4mu}}\relbar{\mathrel{\mkern-4mu}}{\relbar{\mathrel{\mkern-4mu}}\rightarrow}}^{{\rm log}}  {\mathcal R}_K {\relbar{\mathrel{\mkern-4mu}}\rightarrow} 0,$$

\begin{corollary} Under the Leopoldt conjecture for $p$ in $K$,
the Kummer--Leopoldt constant $\kappa$ of $K$ is $0$ if and only if
${\mathcal R}_K=1$, which is equivalent to ${\raise1.5pt \hbox{${\scriptscriptstyle \#}$}} {\mathcal T}_K = 
[H_K : \wt K \cap H_K] \cdot  {\raise1.5pt \hbox{${\scriptscriptstyle \#}$}} {\mathcal W}_K$.
\end{corollary}

\begin{corollary} If the prime number $p$ is regular, then $\kappa=0$
for the field ${\mathbb{Q}}(\mu_p)$ (Kummer's lemma).
\end{corollary}

\begin{proof} In this case we know that the cyclotomic field ${\mathbb{Q}}(\mu_p)$ 
is regular in the meaning of \cite[Th\'eor\`eme \& D\'efinition 2.1]{GJ}, 
so ${\mathcal T}_K = 1$ giving the result. 
\end{proof}

For instance, if ${\mathcal T}_K = 1$ (in which case $\kappa_K=0$)
then in any $p$-primitively ramified $p$-extension $L$ of $K$
(definition and examples in \cite[IV.3, (b); IV.3.5.1]{Gr1},
after \cite[Theorem 1, II.2 ]{Gr4} in the direction of the notion of 
``$p$-rational field''), we get ${\mathcal T}_L = 1$ whence $\kappa_L=0$.

\smallskip
Many generalizations are available which are left to the reader. 

\section{Normalized $p$-adic regulator of a number field}

\begin{definition} 
Under the Leopoldt conjecture for $p$ in $K$, we call 
$${\mathcal R}_K := {\rm tor}^{}_{{\mathbb{Z}}_p} \big ({\rm log} (U_K) / 
{\rm log} (\overline E_K)\big) \simeq  {\rm Gal}(H_K^E / \widetilde KH_K), $$ 

(or its order) the {\it normalized $p$-adic regulator} of $K$, whatever the 
number field $K$ and the prime number $p\geq 2$. 
\end{definition}

For instance, in the totally real case, from Coates's formula \cite[Appendix]{C}, 
we get easily:

\centerline{$\ds {\raise1.5pt \hbox{${\scriptscriptstyle \#}$}} {\mathcal R}_K \sim \frac{1}{2} \cdot
\frac{\big({\mathbb{Z}}_p : {\rm log}({\rm N}_{K/{\mathbb{Q}}}(U_K)) \big)}
{ {\raise1.5pt \hbox{${\scriptscriptstyle \#}$}} {\mathcal W}_K \cdot {\displaystyle\mathop{\raise 2.0pt \hbox{$\prod$}}\limits}_{{\mathfrak p} \mid p}{\rm N} {\mathfrak p}}
\cdot \frac {R_K}{\sqrt {D_K}}$,}

where $\sim$ means equality up to a $p$-adic unit factor, where $R_K$ 
is the usual $p$-adic regulator and $D_K$ the discriminant of $K$. 

\smallskip
In the real Galois case, with $p\ne 2$ unramified in $K/{\mathbb{Q}}$, 
we get, as defined in \cite[Definition 2.3]{Gr3},
$\ds {\raise1.5pt \hbox{${\scriptscriptstyle \#}$}} {\mathcal R}_K \sim \frac{R_K}{p^{[K : {\mathbb{Q}}]-1}}\cdot$

\smallskip
One computes that 
${\raise1.5pt \hbox{${\scriptscriptstyle \#}$}} {\mathcal R}_K=1$ for all imaginary quadratic fields and all $p$.

\begin{thebibliography}{HD}

\bibitem[A]{A} B. Angl\`es,  {\it Units and norm residue symbol}, Acta Arithmetica XCVIII, 1 (2001), 33--51.
\url{https://eudml.org/doc/278965}

\bibitem[AN]{AN} J. Assim, T. Nguyen Quang Do, {\it Sur la constante de 
Kummer--Leopoldt d'un corps de nombres}, Manuscripta Math.115, 1 (2004), 55--72.

\url{http://link.springer.com/article/10.1007/s00229-004-0482-9}

\bibitem[C]{C} J. Coates, {\it $p$-adic $L$-functions and Iwasawa's theory}, 
 Algebraic number fields: L-functions and Galois properties
 (Proc. Sympos., Univ. Durham, Durham, 1975),  Academic Press, London (1977), 269--353. 

\bibitem[GJ]{GJ} G. Gras, J-F. Jaulent,  {\it Sur les corps de nombres r\'eguliers}, Math. Z.
202 (1989), 343--365. \url{https://eudml.org/doc/174095}

\bibitem[Gr1]{Gr1} G. Gras, {\it Class Field Theory: from theory to practice},
SMM, Springer-Verlag 2003;  second corrected printing 2005.
\url{https://www.researchgate.net/publication/268005797}

\bibitem[Gr2]{Gr2}  G. Gras, {\it Sur le module de Bertrandias--Payan dans une $p$-extension -- 
Noyau de capitulation}, Publ. Math\'ematiques de Besan\c con,
Alg\`ebre et Th\'eorie des Nombres (2016), 25--44. 
\url{https://www.researchgate.net/publication/294194005}

\bibitem[Gr3]{Gr3} G. Gras, {\it Les $\theta$-r\'egulateurs locaux d'un nombre alg\'ebrique :
Conjectures $p$-adiques}, Canadian Journal of Mathematics 68, 3 (2016), 571--624. 

\url{http://dx.doi.org/10.4153/CJM-2015-026-3}

\bibitem[Gr4]{Gr4} G. Gras, {\it  Remarks on $K_2$ of number fields}, Jour. Number Theory 23
(1986), 322--335.
\url{http://www.sciencedirect.com/science/article/pii/0022314X86900776}

\bibitem[L]{L} F. Lorenz,  {\it Some remarks on Leopoldt's conjecture}, Algebra i Analiz 10, 6 
(1998), 144--155; translation in St. Petersburg Math. J. 10, 6 (1999), 1005--1013.

\url{http://www.mathnet.ru/links/9a4fa81fa0cb834ed18c3c4811f4ed43/}

\bibitem[O]{O} M. Ozaki {\it Kummer's lemma for ${\mathbb{Z}}_p$-extensions over totally real number fields},
Acta Arithmetica LXXXI, 1 (1997), 37--43. \url{https://eudml.org/doc/207053}

\bibitem[S]{S} J. Sands,  {\it Kummer's and Iwasawa's version of Leopoldt's conjecture}, Canad.
Math. Bull. 31, 1 (1988), 338--346.\url{http://cms.math.ca/openaccess/cmb/v31/cmb1988v31.0338-0346.pdf}

\bibitem[W1]{W1}  L.C. Washington,  {\it Units of irregular cyclotomic fields},  
Ill. J. Math. 23 (1979), 635--647.
\url{https://projecteuclid.org/download/pdf_1/euclid.ijm/1256047937}

\bibitem[W2]{W2}  L.C. Washington,  {\it Kummer's lemma for prime power cyclotomic fields}, 
Jour. Number Theory 40 (1992), 165--173.

\url{http://www.sciencedirect.com/science/article/pii/0022314X9290037P}

\bibitem[W3]{W3} L.C. Washington, {\it Introduction to cyclotomic fields}, 
Graduate Texts in Math. 83, Springer enlarged second edition 1997.

\end{thebibliography}

\end{document} 


\documentclass[11pt]{amsart}
\usepackage[numbers, square]{natbib}
\usepackage{amssymb}
\usepackage{amsmath}
\usepackage{amsfonts}
\usepackage{geometry}
\usepackage{graphicx}
\usepackage{amsthm}
\usepackage{hyperref}
\setcounter{MaxMatrixCols}{30}

\usepackage[dvipsnames]{xcolor}

\providecommand{\U}[1]{\protect\rule{.1in}{.1in}}
\marginparwidth -1cm \oddsidemargin 0cm \evensidemargin 0cm
\topmargin -1.5cm \textheight 232mm \textwidth 160mm
\theoremstyle{plain}
\newtheorem{acknowledgement}{Acknowledgement}
\newtheorem{algorithm}{Algorithm}
\newtheorem{axiom}{Axiom}
\newtheorem{case}{Case}
\newtheorem{claim}{Claim}
\newtheorem{conclusion}{Conclusion}
\newtheorem{condition}{Condition}
\newtheorem{conjecture}{Conjecture}
\newtheorem{corollary}{Corollary}
\newtheorem{criterion}{Criterion}
\newtheorem{definition}{Definition}
\newtheorem{example}{Example}
\newtheorem{exercise}{Exercise}
\newtheorem{lemma}{Lemma}
\newtheorem{notation}{Notation}
\newtheorem{problem}{Problem}
\newtheorem{proposition}{Proposition}
\newtheorem{remark}{Remark}
\newtheorem{solution}{Solution}
\newtheorem{summary}{Summary}
\newtheorem{theorem}{Theorem}
\numberwithin{equation}{section}

\begin{document}
\title[Quantitative uniqueness ]
{Quantitative uniqueness of solutions to second order elliptic equations with
singular lower order terms}
\author{ Blair Davey and Jiuyi Zhu}
\address{
Department of Mathematics\\
The City College of New York\\
New York, NY 10031, USA\\
Email: bdavey@ccny.cuny.edu }
\address{
Department of Mathematics\\
Louisiana State University\\
Baton Rouge, LA 70803, USA\\
Email:  zhu@math.lsu.edu }
\thanks{\noindent{Davey is supported in part by the Simons Foundation Grant number 430198.
Zhu is supported in part by \indent NSF grant DMS-1656845}}
\date{}
\subjclass[2010]{35J15, 35J10, 35A02.}
\keywords {Carleman estimates, unique continuation,
 singular lower order terms, vanishing order}

\begin{abstract}
In this article, we study the quantitative uniqueness of solutions to second order elliptic equations with singular lower order terms.
We quantify the strong unique continuation property by estimating the maximal vanishing order of solutions.
That is, when $u$ is a non-trivial solution to ${\Delta} u + W \cdot {\nabla} u + V u = 0$ in some open, connected subset of ${\ensuremath{\mathbb{R}}}^n$, where $n \ge 3$, we characterize the vanishing order of solutions in terms of the norms of $V$ and $W$ in their respective Lebesgue spaces.
Using these maximal order of vanishing estimates, we also establish quantitative unique continuation at infinity results for solutions to ${\Delta} u + W \cdot {\nabla} u + V u = 0$ in ${\ensuremath{\mathbb{R}}}^n$.
The main tools in our work are new versions of $L^p\to L^q$ Carleman estimates for a
range of $p$- and $q$-values.
\end{abstract}

\maketitle
\section{Introduction}

In this paper, we investigate the quantitative uniqueness of solutions to second order elliptic equations with singular lower order terms.
A partial differential operator $P$, along with a function space $S$, is said to have the {\em strong unique continuation property} if whenever $u \in S$ is a solution to $P u = 0$ in ${\Omega} {\subset} {\ensuremath{\mathbb{R}}}^n$, and $u$ vanishes to infinite order at some point $x_0 \in {\Omega}$, then, necessarily, $u \equiv 0$ throughout ${\Omega}$.
If $P$ has the strong unique continuation property, then it is interesting to determine the fastest rate at which a solution can vanish without being trivial.
We call this rate the {\em maximal order of vanishing}.
 Within this article, we study the maximal order of vanishing associated to elliptic operators of the form $P = {\Delta} + W \cdot {\nabla} + V$, where $W$ and $V$ are singular potentials that belong to some Lebesgue spaces.
We characterize the vanishing order in terms of the Lebesgue norms of the potential functions $V$ and $W$.
Then, using the maximal order of vanishing estimates, we employ a scaling technique to prove unique continuation at infinity theorems.

There is extensive literature regarding the strong unique continuation properties of solutions to second order elliptic equations with singular lower order terms.
Jerison and Kenig \cite{JK85} proved that the strong unique continuation property holds for operators of the form ${\Delta}+ V$ 
provided that $V \in L^{n/2}_{loc}{\left( {{\ensuremath{\mathbb{R}}}^n} \right) }$ for $n \ge 3$.
For operators of the form ${\Delta} + W \cdot {\nabla}$, the strong unique continuation property was proved by Kim \cite{Kim89} whenever $W\in L^{s}$ with $s=\frac{3n-2}{2}$, $n \ge 3$.
In \cite{Wol90}, Wolff improved the exponent to $s=\max\{n, \ \frac{3n-4}{2}\}$, $n \ge 3$.
Then Regbaoui \cite{Reg99} reduced the exponent to $s >\frac{7n-2}{6}$, which improved all previous results when $n > 5$.
More recently, Koch and Tataru \cite{KT01} proved the strong unique continuation property when $W\in l^1_{\omega}(L^n)$ with small norm.
Roughly speaking, the space $l^1_\omega (L^n)$ consists of functions whose $L^n$-norms are $l^1_\omega$ ``summable"  on dyadic annuli.
For a general elliptic operator of the form ${\Delta} + W \cdot {\nabla} + V$, if the lower order terms satisfy the following integrability conditions
$$ V\in L^{n/2} \quad \mbox{and} \quad W\in L^{s} \ \mbox{with} \ s>n, $$
then the strong unique continuation property holds, see. e.g. \cite{KT01}.
Therefore, there is a large class of elliptic operators for which we can study quantitative uniqueness.

Quantitative uniqueness is an important concept in geometry.
Let $\mathcal{M}$ denote a smooth, compact Riemannian manifold.
It is well-known that all zeros of non-trivial solutions to second order linear equations on $\mathcal{M}$ are of finite order.
For the classical eigenfunctions of $\mathcal{M}$, those $\phi_{\lambda}$ for which
 $$-{\mbox{$\triangle$}}_{g} \phi_\lambda=\lambda \phi_\lambda \quad \quad \mbox{in} \ \mathcal{M},$$
Donnelly and Fefferman in \cite{DF88}, \cite{DF90} showed that the maximal vanishing order of $\phi_\lambda$ on $\mathcal{M}$ is everywhere less than $C\sqrt{\lambda}$, where $C$ depends only on the manifold $\mathcal{M}$.
In \cite{Kuk98}, Kukavica considered the vanishing order of solutions to the Schr\"{o}dinger equation
\begin{equation}
-{\mbox{$\triangle$}} u=V(x)u.
\label{schro}
\end{equation}
Kukavica showed that if $V \in W^{1, \infty}$, then the upper bound for the vanishing order is less than $C(1+\|V\|_{W^{1, \infty}})$, where $\|V\|_{W^{1, \infty}}=\|V\|_{L^\infty}+\|\nabla V\|_{L^\infty}$.
Based on Donnelly and Fefferman's work in \cite{DF88}, where they considered $V(x)=\lambda$, Kukavica's upper bound is not sharp.
Recently, the vanishing order for solutions to (\ref{schro}) was shown to be less than $C(1+\sqrt{\|V\|_{W^{1, \infty}}})$ independently by Bakri in \cite{Bak12} and Zhu in \cite{Zhu16} using different
methods.
Since the exponent on the norm of $V$ agrees with that of the eigenfunctions in Donnelly and Fefferman's work, the results of Bakri and Zhu are sharp.

In their work on Anderson localization, Bourgain and Kenig \cite{BK05} studied quantitative uniqueness of solutions to Schr\"odinger equations.
They considered equations of the form \eqref{schro} in $B_{10} {\subset} {\ensuremath{\mathbb{R}}}^n$, where $V \in L^\infty$ with $\|V\|_{L^\infty}\leq M$ for some $M > 1$, and showed that if $u$ is a solution to \eqref{schro} with ${\left\vert\left\vert {u}\right\vert\right\vert}_{L^{\infty}{\left( {B_{10}} \right) }} \le \hat C$ and ${\left\vert\left\vert {u}\right\vert\right\vert}_{L^{\infty}{\left( {B_1} \right) }}\geq 1$, then
\begin{equation}
\|u\|_{L^\infty (B_r)}\geq a_1r^{a_2 M^{2/3}} \quad
\quad \mbox{as}\ r\to 0, \label{like}
\end{equation}
 where $a_1$ and $a_2$ depend only on $n$ and $\hat{C}$.
The estimate \eqref{like} indicates that the order of vanishing for solutions is less than $C(1+\|V\|_{L^\infty}^{2/3})$.
They used their vanishing order result to establish a unique continuation at infinity result: If $u$ is a bounded, normalized, non-trivial solution to \eqref{schro} in ${\ensuremath{\mathbb{R}}}^n$, where ${\left\vert\left\vert {V}\right\vert\right\vert}_{L^{\infty}{\left( {{\ensuremath{\mathbb{R}}}^n} \right) }} \le 1$, then
\begin{align}
\inf{\left\{{{\left\vert\left\vert {u}\right\vert\right\vert}_{L^{\infty}{\left( {B_1{\left( {x_0} \right) }} \right) }} : {\left\vert{x_0}\right\vert} = R}\right\}} \ge c \exp{\left( {- C R^{4/3} \log R} \right) }.
\label{quantEst}
\end{align}
The estimate \eqref{quantEst} is a quantitative form of Meshkov's qualitative unique continuation theorem from \cite{Mes92}.
In fact, Meshkov's examples in \cite{Mes92} indicate that the exponent of $4/3$ in \eqref{quantEst} is sharp for potential functions $V : {\ensuremath{\mathbb{R}}}^2 \to {\ensuremath{\mathbb{C}}}$.

In \cite{Dav14}, Davey generalized the result described in \eqref{quantEst} to solutions to more general elliptic equations of the form ${\Delta} u + W \cdot {\nabla} u + V u = {\lambda} u$, where ${\lambda} \in {\ensuremath{\mathbb{C}}}$, and $V$, $W$ are complex-valued potential functions with pointwise decay at infinity.
She proved estimates of the form \eqref{quantEst} in which the power depends explicitly on the decay rates of $V$ and $W$.
Due to newly constructed Meshkov-type examples (see also \cite{Dav15}), these estimates are known to be sharp in certain settings.
As a step towards proving these theorems, she derived order of vanishing estimates in the setting where $V$ and $W$ belong to $L^{\infty}$.
Her results indicate that the order of vanishing for such solutions is less than $C(1+\|W\|_{L^\infty}^{2}+\|V\|_{L^\infty}^{2/3})$.
The results in \cite{Dav14} were generalized to variable-coefficient operators by Lin and Wang in \cite{LW14}.

Motivated in part by Donnelly and Fefferman's work, Kenig \cite{Ken07} asked if the order of vanishing can be reduced to $C(1+\|V\|_{L^\infty}^{1/2})$ for real-valued $u$ and $V$.
Through the scaling argument demonstrated in \cite{BK05}, an affirmative answer to Kenig's question would imply a quantitative form of Landis' conjecture in the real-valued setting.
In the late 1960s, E.M.~Landis conjectured that if $u$ is a bounded solution to ${\Delta} u - V u = 0$ in ${\ensuremath{\mathbb{R}}}^n$, where $V$ is a bounded function and ${\left\vert{u{\left( {x} \right) }}\right\vert} \lesssim \exp{\left( {- c {\left\vert{x}\right\vert}^{1+}} \right) }$, then $u \equiv 0$.
Under the assumption that $V$ is bounded, real-valued, and almost everywhere non-negative, Kenig, Silvestre and Wang proved a quantitative form of Landis' conjecture in the plane \cite{KSW15}.
They also proved analogous theorems for equations with bounded, real-valued magnetic potentials.
Towards proving these quantitative Landis-type theorems, they first established order of vanishing estimates for second-order elliptic equations with bounded lower order terms.
Davey, Kenig and Wang \cite{DKW16} generalized the results from \cite{KSW15} to variable-coefficient operators.

Recently, attention has shifted to elliptic equations with singular lower order terms.
In \cite{KW15}, Kenig and Wang studied the quantitative uniqueness of solutions to second order elliptic equations with a drift term in the plane.
They proved vanishing order estimates for solutions to ${\Delta} u + W \cdot {\nabla} u = 0$ under the assumption that $W \in L^s{\left( {{\ensuremath{\mathbb{R}}}^2} \right) }$ for some $s \in {\left[ {2, {\infty}} \right) }$, then proved unique continuation at infinity theorems with the usual scaling technique.
Their proofs build on the complex analytic tools that were developed in \cite{KSW15}.
Since the techniques developed in \cite{KSW15}, \cite{KW15} and \cite{DKW16} rely on the relationship between real-valued solutions to elliptic equations in the plane and the Beltrami equation, it seems that these method cannot be adapted to $n\geq 3$.

 In \cite{KT16}, Klein and Tsang studied quantitative uniqueness for solutions to ${\Delta} u + V u = 0$, where $V \in L^t$ for some $t \ge n \ge 3$.
 The main tool in \cite{KT16} is an $L^2$ Carleman estimate similar to those that appeared in \cite{BK05}, \cite{Ken07}, and \cite{Dav14}.
Although it is not explicitly stated, the  results in \cite{KT16}
imply that the vanishing order of solutions is less than
$C(1+\|V\|_{L^t}^{\frac{ 2t}{3t-2n} })$. It appears that the methods
in \cite{KT16} do not apply to the study of quantitative uniqueness
for elliptic equations with unbounded gradient potentials. In this
paper, through the use more sophisticated Carleman estimates, we can
reduce the aforementioned bounds. Furthermore, we can treat $V \in
L^t$ with $t < n$ as well as singular gradient potentials. Finally,
we point out that in \cite{MV12}, Malinnikova and Vessella studied a
different quantitative uniqueness problem for elliptic operators
with singular lower order terms.
They derived estimates for the norms of solutions on arbitrary compact subsets of the domain from information about the smallness of solutions on subsets of positive measure.

We assume throughout that $n \ge 3$. In a forthcoming paper, we hope
to consider the quantitative uniqueness for general elliptic
equations with singular weights in $n=2$ dimensions. We use the
notation $B_{r}{\left( {x_0} \right) } {\subset} {\ensuremath{\mathbb{R}}}^n$ to denote the ball of radius $r$
centered at $x_0$. When the center is understood from the context,
we simply write $B_{r}$.

Our first goal is to study the vanishing order of solutions to general second order elliptic equations with singular lower order terms.
Then we use these estimates to prove quantitative unique continuation at infinity theorems.
Let ${\Omega}$ be an open, connected subset of ${\ensuremath{\mathbb{R}}}^n$.
For some $s > n$ and $t > \frac n 2$, assume that $W \in L^s_{loc}{\left( {\Omega} \right) }$ and $V \in L^t_{loc}{\left( {\Omega} \right) }$.
Suppose $u$ is a non-trivial solution to
\begin{equation}
\triangle u+ W(x)\cdot \nabla u+V(x) u=0.
\label{goal}
\end{equation}
A priori, we assume that $u \in W^{1,2}_{loc}{\left( {\Omega} \right) }$ is a weak solution to \eqref{goal} in ${\Omega}$.
However, the computations within Section \ref{CarlEst} imply that there exists a $p \in {\left( {\frac{2n}{n+2}, 2} \right] }$, depending on $s$ and $t$, such that $W(x)\cdot \nabla u+V(x) u \in L^p_{loc}{\left( {\Omega} \right) }$.
By regularity theory, it follows that $u \in W_{loc}^{2,p}{\left( {\Omega} \right) }$ and therefore $u$ is a solution to \eqref{goal} almost everywhere in ${\Omega}$.
Moreover, by de Giorgi-Nash-Moser theory, we have that $u \in L^{\infty}_{loc}{\left( {\Omega} \right) }$.
Therefore, when we say that $u$ is a solution to \eqref{goal} in ${\Omega}$, it is understood that $u$ belongs to $L^{\infty}_{loc}{\left( {\Omega} \right) } \cap W^{1,2}_{loc}{\left( {\Omega} \right) } \cap W^{2,p}_{loc}{\left( {\Omega} \right) }$ and $u$ satisfies equation \eqref{goal} almost everywhere in ${\Omega}$.

We state the first order of vanishing result.

\begin{theorem}
Assume that for some $s \in {\left( {\frac{3n-2}{2}, {\infty}} \right] }$ and $t \in
{\left( { n{\left( {\frac{3n-2}{5n-2}} \right) }, {\infty}} \right] }$, ${\left\vert\left\vert {W}\right\vert\right\vert}_{L^s{\left( {B_{10}} \right) }} \le
K$ and ${\left\vert\left\vert {V}\right\vert\right\vert}_{L^t{\left( {B_{10}} \right) }} \le M$. Let $u$ be a solution to
\eqref{goal} in $B_{10}$. Assume that $u$ is bounded and normalized as follows:
\begin{align}
& \|u\|_{L^\infty(B_{6})}\leq \hat{C}
\label{bound} \\
& \|u\|_{L^\infty(B_{1})}\geq 1.
\label{normal}
\end{align}
Then the vanishing order of $u$ in  $B_{1}$ is less than $C{\left( {1 +
C_1 K^\kappa + C_2 M^\mu} \right) }$. That is, for any $x_0\in B_1$ and every
$r$ sufficiently small,
\begin{align*}
\|u\|_{L^{\infty}(B_{r}(x_0))}
 &\ge c r^{C{\left( {1 + C_1 K^\kappa + C_2 M^\mu} \right) }},
\end{align*}
where
${\displaystyle} \kappa = \left\{\begin{array}{ll}
\frac{4s}{2s - {\left( {3n-2} \right) }} & t > \frac{sn}{s+n} \medskip \\
\frac{4t}{{\left( {5 - \frac 2 n} \right) }t - {\left( {3n-2} \right) }} & n{\left( {\frac{3n-2}{5n-2}} \right) }
< t \le \frac{sn}{s+n}
\end{array}\right.$,
${\displaystyle} \mu = \left\{\begin{array}{ll}
\frac{4s}{6s - {\left( {3n-2} \right) }} & t \ge s \medskip \\
\frac{4 s t}{6 s t + {\left( {n+2} \right) }t -4ns} & \frac{sn}{s+n} < t < s \medskip \\
\frac{4t}{{\left( {5 - \frac 2 n} \right) }t - {\left( {3n-2} \right) }} & n{\left( {\frac{3n-2}{5n-2}} \right) }
< t \le \frac{sn}{s+n}
\end{array}\right.$,
$c = c{\left( {n, \hat C, K, M} \right) }$, $C = C{\left( {n, \hat C} \right) }$, $C_1 = C_1{\left( {n,
s, t} \right) }$, and $C_2 = C_2{\left( {n, s, t} \right) }$. \label{thh}
\end{theorem}

As in \cite{BK05}, a scaling argument shows that the following unique continuation at infinity estimate follows from Theorem \ref{thh}.
Each unique continuation at infinity theorem is presented in terms of a lower bound for $\mathcal{M}{\left( {R} \right) }$, where
\begin{equation}
\mathcal{M}{\left( {R} \right) } := \inf{\left\{{{\left\vert\left\vert {u}\right\vert\right\vert}_{L^{\infty}{\left( {B_1{\left( {x_0} \right) }} \right) }} : {\left\vert{x_0}\right\vert} = R}\right\}}.
\label{MRDef}
\end{equation}
(Compare with Bourgain and Kenig's estimate given in \eqref{quantEst}.)

\begin{theorem}
Assume that for some $s \in {\left( {\frac{3n-2}{2}, {\infty}} \right] }$ and $t \in {\left( { n{\left( {\frac{3n-2}{5n-2}} \right) }, {\infty}} \right] }$, ${\left\vert\left\vert {W}\right\vert\right\vert}_{L^s{\left( {{\ensuremath{\mathbb{R}}}^n} \right) }} \le A_1$ and ${\left\vert\left\vert {V}\right\vert\right\vert}_{L^t{\left( {{\ensuremath{\mathbb{R}}}^n} \right) }} \le A_0$.
Let $u$ be a solution to \eqref{goal} in ${\ensuremath{\mathbb{R}}}^n$.
Assume that ${\left\vert\left\vert {u}\right\vert\right\vert}_{L^{\infty}{\left( {{\ensuremath{\mathbb{R}}}^n} \right) }} \le C_0$ and ${\left\vert{u{\left( {0} \right) }}\right\vert} \ge 1$.
Then for $R >> 1$,
\begin{equation*}
\mathcal{M}{\left( {R} \right) } \ge c \exp{\left[{-C R^\Pi \log R}\right]},
\end{equation*}
where
${\displaystyle}
\Pi = \left\{\begin{array}{ll}
\frac{4{\left( {s-n} \right) }}{2s - {\left( {3n-2} \right) }} & t > \frac{sn}{s+n} \medskip \\
\frac{4 {\left( {t-n\frac t s} \right) }}{{\left( {5 - \frac 2 n} \right) }s - {\left( {3n-2} \right) } \frac s t} & n{\left( {\frac{3n-2}{5n-2}} \right) } < t \le \frac{sn}{s+n}
\end{array}\right.
$, $c = c{\left( {n, C_0, A_0, A_1} \right) }$, \\ and $C = C{\left( {n, s, t, A_1,
A_0} \right) }$. \label{UCVW}
\end{theorem}

Now we consider solutions to equation \eqref{goal} with $V(x)\equiv0$, i.e. solutions to
\begin{equation}
\triangle u+W(x) \cdot \nabla u=0.
\label{drift}
\end{equation}
An immediate consequence of Theorem \ref{thh} is the vanishing order of solutions for second order elliptic equations with drift.

\begin{corollary}
Assume that ${\left\vert\left\vert {W}\right\vert\right\vert}_{L^s{\left( {B_{10}} \right) }} \le K$ for some $s \in
{\left( {\frac{3n-2}{2}, {\infty}} \right] }$. Let $u$ be a solution to \eqref{drift}
in $B_{10}$. Assume that $u$ is bounded and normalized in the sense
of \eqref{bound} and \eqref{normal}. Then the vanishing order of $u$
in $B_{1}$ is less than $C{\left( {1 + C_1 K^\kappa } \right) }$. That is, for any
$x_0\in B_1$ and every $r$ sufficiently small,
\begin{align*}
 \|u\|_{L^{\infty}(B_{r}(x_0))}
 &\ge c r^{C{\left( {1 + C_1 K^\kappa } \right) }},
\end{align*}
where ${\displaystyle} \kappa = \frac{4s}{2s - {\left( {3n-2} \right) }}$, $c = c{\left( {n, \hat
C, K} \right) }$, $C = C{\left( {n, \hat C} \right) }$, and $C_1 = C_1{\left( {n, s} \right) }$.
\label{thhCor}
\end{corollary}

The following unique continuation estimate follows from Corollary \ref{thhCor} in the same way that Theorem \ref{UCVW} follows from Theorem \ref{thh}.

\begin{corollary}
Assume that ${\left\vert\left\vert {W}\right\vert\right\vert}_{L^s{\left( {{\ensuremath{\mathbb{R}}}^n} \right) }} \le A_1$ for some $s \in {\left( {\frac{3n-2}{2}, {\infty}} \right] }$.
Let $u$ be a solution to \eqref{drift} in ${\ensuremath{\mathbb{R}}}^n$.
Assume that ${\left\vert\left\vert {u}\right\vert\right\vert}_{L^{\infty}{\left( {{\ensuremath{\mathbb{R}}}^n} \right) }} \le C_0$ and ${\left\vert{u{\left( {0} \right) }}\right\vert} \ge 1$.
Then for $R >> 1$,
\begin{equation*}
\mathcal{M}{\left( {R} \right) } \ge c \exp{\left( {-C R^\Pi \log R} \right) }
\end{equation*}
where ${\displaystyle} \Pi = \frac{4{\left( {s-n} \right) }}{2s - {\left( {3n-2} \right) }}$, $c = c{\left( {n,
C_0, A_1} \right) }$, and $C = C{\left( {n, s, A_1} \right) }$. \label{UCW}
\end{corollary}

Finally, we consider solutions to equation without a gradient potential,
\begin{equation}
\triangle u+V(x) u=0 .
\label{goal1}
\end{equation}

\begin{theorem}
Assume that ${\left\vert\left\vert {V}\right\vert\right\vert}_{L^t{\left( {B_{R_0}} \right) }} \le M$ for some $t \in {\left( {
\frac{4 n^2}{7n+2}, {\infty}} \right] }$. Let $u$ be a solution to \eqref{goal1}
in $B_{10}$. Assume that $u$ is bounded and normalized in the sense
of \eqref{bound} and \eqref{normal}. Then the vanishing order of $u$
in $B_{1}$ is  less than $C{\left( {1 + C_2M^\mu} \right) }$. That is, for any
$x_0\in B_1$ and every $r$ sufficiently small,
\begin{align*}
 \|u\|_{L^{\infty}(B_{r}(x_0))}
 &\ge c r^{C{\left( {1 + C_2M^\mu} \right) }},
\end{align*}
where for any positive sufficiently small constant ${\varepsilon} <
\min{\left\{{\frac{7t+\frac{2t}n-4n}{2}, \frac{(2t-n)(n+2)}{2n} }\right\}}$, \\
${\displaystyle} \mu = \left\{\begin{array}{ll}
\frac{4t}{6t - {\left( {3n-2} \right) }} & t > n \medskip \\
\frac{4t} {7t+\frac{2t}n-4n-{\varepsilon}} & \frac{4n^2}{7n+2} < t \le n
\end{array}\right.$,
$c=c{\left( {n, \hat C, M} \right) }$, $C = C{\left( {n, \hat C} \right) }$, and $C_2 =
C_2{\left( {n, s, t, {\varepsilon}} \right) }$. \label{thhh}
\end{theorem}

Using the maximal order of vanishing estimate from the previous theorem, we may prove quantitative unique continuation at infinity estimates.
Notice that the value of $\Pi$ here is much smaller than the one in Theorem \ref{UCVW}.
The following theorem may be interpreted as a quantitative form of Landis' conjecture for singular potentials.

\begin{theorem}
Assume that ${\left\vert\left\vert {V}\right\vert\right\vert}_{L^t{\left( {{\ensuremath{\mathbb{R}}}^n} \right) }} \le A_0$ for some $t \in {\left( { \frac{4 n^2}{7n+2}, {\infty}} \right] }$.
Let $u$ be a solution to \eqref{goal1} in ${\ensuremath{\mathbb{R}}}^n$.
Assume that ${\left\vert\left\vert {u}\right\vert\right\vert}_{L^{\infty}{\left( {{\ensuremath{\mathbb{R}}}^n} \right) }} \le C_0$ and ${\left\vert{u{\left( {0} \right) }}\right\vert} \ge 1$.
Then for $R >> 1$,
\begin{equation*}
\mathcal{M}{\left( {R} \right) } \ge c \exp{\left[{-C R^\Pi \log R}\right]},
\end{equation*}
where for any positive sufficiently small constant ${\varepsilon} <
\min{\left\{{\frac{7t+\frac{2t}n-4n}{2}, \frac{(2t-n)(n+2)}{2n} }\right\}}$, \\
${\displaystyle} \Pi =  \left\{\begin{array}{ll}
\frac{4{\left( {2t-n} \right) }}{6t - {\left( {3n-2} \right) }} & t > n \medskip \\
\frac{4{\left( {2t-n} \right) }} {7t+\frac{2t}n-4n-{\varepsilon}} & \frac{4n^2}{7n+2} < t \le n
\end{array}\right.$,
$c = c{\left( {n, C_0, A_0} \right) }$, and $C = C{\left( {n, t, A_0, {\varepsilon}} \right) }$.
\label{UCV}
\end{theorem}

As indicated by the surveyed literature, there are two fruitful
approaches to studying the quantitative uniqueness of solutions to
elliptic equations with lower order terms. The Carleman approach,
which uses weighted-norm estimates to derive three-ball
inequalities, is well-suited to complex-valued solutions in domains
of arbitrary dimensions. This is the approach that was used in
\cite{DF88}, \cite{DF90}, \cite{BK05}, \cite{Ken07}, \cite{Dav14},
\cite{LW14}. See also \cite{Zhu15} for an application of Carleman
estimates to the derivation of sharp vanishing order estimates for Steklov
eigenfunctions. On the other hand, the Beltrami approach, which is
based on the relationship between real-valued solutions in the plane
and solutions to first-order complex-valued equations, cannot be
used in higher dimensions, but does produce sharper results, see
\cite{KSW15}, \cite{KW15}, \cite{DKW16}. Since we focus on $n \ge
3$, we follow the Carleman approach.

In order to obtain quantitative uniqueness results, Carleman estimates are employed to derive three-ball inequalities.
Then the ``propagation of smallness" argument is used to obtain maximal order of vanishing estimates.
In much of the literature discussed above, $L^2 \to L^2$ Carleman estimates were used to prove maximal order of vanishing estimates.
In this paper, we establish new quantitative $L^p \to L^q$ Carleman inequalities with $\frac{2n}{n+2}<p \leq 2 \le q \le \frac{2n}{n-2}$.
This Carleman estimate is quantitative in the sense that we show the dependence on $\tau$, the constant that may be made arbitrarily large.
The ranges of $p$ and $q$, in combination with the H\"older inequality, allow us to consider equations of the form \eqref{goal}, where $W  \in L^{s}$ and $V \in L^{t}$ for some $s, t < {\infty}$.

To prove our Carleman estimates, we decompose the Laplacian into first order operators and prove a collection of Carleman estimates for these operators.
The $L^2 \to L^2$ Carleman estimates are proved using the standard integration by parts approach.
For the $L^p \to L^2$ estimates, we use the eigenfunction estimates of Sogge \cite{Sog86} along with the techniques developed by Jerison in \cite{Jer86} and Regbaoui in \cite{Reg99}.
By combining Carleman estimates for the first order constituents of ${\Delta}$, applying a Sobolev inequality, and interpolating, we arrive at the general Carleman estimate given in Theorem \ref{Carlpq}.

Once we have the general Carleman estimates, the order of vanishing results are proved in much the same way as in \cite{BK05} and \cite{Ken07}, for example.
The unique continuation at infinity theorems follow from the maximal order of vanishing estimates through the scaling argument presented in \cite{BK05}.

The outline of the paper is as follows.
Section \ref{CarlEst} is devoted to obtaining Carleman estimates for the second order elliptic operators with singular lower order terms.
In Section \ref{CarlProofs}, the major $L^p\to L^q$ Carleman estimates for ${\Delta}$ are established.
We also derive a quantitative Caccioppoli inequality.
In section \ref{vanOrd}, we deduce three-ball inequalities from the Carleman estimates.
Then, the vanishing order is obtained via the propagation of smallness argument.
The scaling argument is presented in Section \ref{QuantUC} where we prove the quantitative unique continuation at infinity theorems.
The letters $c$ and $C$ denote generic positive constants that do not depends on $u$, and may vary from line to line.

\section{Carleman estimates }
\label{CarlEst}

In this section, we state the crucial tools, the quantitative $L^p-L^{ q}$ type Carleman estimates.
Let $r=|x-x_0|$ and set
$$\phi(r)=\log r+\log(\log r)^2.$$
We use the notation $\|u\|_{L^p(r^{-n} dx)}$ to denote the $L^p$ norm with weight $r^{-n}$, i.e. $\|u\|_{L^p(r^{-n} dx)} = {\displaystyle} {\left( {\int |u{\left( {x} \right) }|^p r^{-n}\, dx} \right) }^\frac{1}{p}$.
Our quantitative $L^p - L^q$ Carleman estimate for the Laplacian is as follows.

\begin{theorem}
Let $\frac{2n}{n+2} < p \le 2 \leq q \leq \frac{2n}{n-2}$. There
exists a constant $C$ and a sufficiently small $R_0$ such that for any
$u\in C^{\infty}_{0}{\left( {B_{R_0}(x_0)\backslash{\left\{{x_0}\right\}} } \right) }$ and
$\tau>1$, one has
\begin{align}
&\tau^{{\beta}_0} \|(\log r)^{-1} e^{-\tau \phi(r)}u\|_{L^q(r^{-n}dx)} +
\tau^{{\beta}_1} \|(\log r )^{-1} e^{-\tau \phi(r)}r \nabla
u\|_{L^2(r^{-n}dx)}
\nonumber \medskip\\
&\leq  C \|(\log r ) e^{-\tau \phi(r)} r^2 \triangle u\|_{L^p(r^{-n} dx)} ,
\label{mainCar}
\end{align}
where ${\beta}_0 = \frac 3 2 -
\frac{{\left( {3n-2} \right) }{\left( {2-p} \right) }}{8p}-\frac{n(q-2)}{2q}$ and ${\beta}_1 =
\frac{1}{2}-\frac{{\left( {3n-2} \right) }{\left( {2-p} \right) }}{8p}$.
 \label{Carlpq}
\end{theorem}

\begin{remark}
We may at times use the notation ${\beta}_0{\left( {q} \right) }$ to remind the reader that ${\beta}_0$ depends on $q$.
This notation will be useful when we work with multiple $q$-values.
\end{remark}

The proof of Theorem \ref{Carlpq} is given in the next section.

Now we use Theorem \ref{Carlpq} to establish the following $L^p\to L^2$ Carleman estimates for second order elliptic equations of the form \eqref{goal}.
We show that for an appropriate choice of $p$, and for $\tau$ sufficiently large, we may replace the Laplacian with a more general elliptic operator.
In each of the following three theorems, we use H\"older's inequality and the triangle inequality to go from Theorem \ref{Carlpq} to a Carleman estimate for an elliptic operator with lower order terms.
Since the argument is the simplest, we start with drift operators ($V \equiv 0$) corresponding to equations of the form \eqref{drift}.

\begin{theorem}
Assume that for some $s \in {\left( {\frac{3n-2}{2}, {\infty}} \right] }$, ${\left\vert\left\vert {W}\right\vert\right\vert}_{L^s{\left( {B_{R_0}} \right) }} \le K$.
Then there exist constants $C_0$, $C_1$, and sufficiently small $R_0 < 1$  such that for any $u\in C^{\infty}_{0}(B_{R_0}(x_0)\setminus {\left\{{x_0}\right\}})$ and large positive constant
$$\tau \ge 1+ C_1 K^{\kappa},$$
one has
\begin{align}
\tau^{{\beta}_0} \|(\log r)^{-1} e^{-\tau \phi(r)}u\|_{L^2(r^{-n}dx)}
&\leq  C_0 \|(\log r ) e^{-\tau \phi(r)} r^2{\left( { \triangle u + W \cdot {\nabla} u} \right) }\|_{L^p(r^{-n} dx)} ,
\label{main1}
\end{align}
where $\kappa = \frac{4s}{2s - {\left( {3n-2} \right) }}$, $p = \frac{2s}{s+2}$, and ${\beta}_0 = {\beta}_0{\left( {2} \right) }$ as defined in Theorem \ref{Carlpq}.
Moreover, $C_0 = C$ from Theorem \ref{Carlpq}, and $C_1 = C_1{\left( {n, s} \right) }$.
\label{CarlpqW}
\end{theorem}

\begin{proof}
By \eqref{mainCar} in Theorem \ref{Carlpq} with $q = 2$ and the triangle inequality,
\begin{align}
&\tau^{{\beta}_0} \|(\log r)^{-1} e^{-\tau \phi(r)}u\|_{L^2(r^{-n}dx)}
+\tau^{{\beta}_1} \|(\log r )^{-1} e^{-\tau \phi(r)}r \nabla u\|_{L^2(r^{-n}dx)} \nonumber \\
&\leq  C \|(\log r ) e^{-\tau \phi(r)} r^2 \triangle u\|_{L^p(r^{-n} dx)}  \nonumber \\
&\le C\|(\log r) e^{-\tau \phi(r)} r^2 (\triangle u+ W\cdot \nabla u)\|_{L^p(r^{-n} dx)}
+ C\|(\log r) e^{-\tau \phi(r)} r^2 W\cdot \nabla u \|_{L^p(r^{-n} dx)}.
\label{triIneq0}
\end{align}
Therefore, to reach the conclusion of the theorem, we need to absorb the second term on the right of \eqref{triIneq0} into the lefthand side.

Set $p = \frac{2s}{s+2}$.
By assumption, $s > \frac{3n-2}{2} > n$, so it follows that $p \in {\left( {\frac{2n}{n+2}, 2} \right] }$.
Since $\frac{1}{p}=\frac{1}{s}+\frac{1}{2}$, then by an application of H\"older's inequality we have
\begin{align}
&\|(\log r) e^{-\tau\phi(r)} r^2 W\cdot\nabla u\|_{L^p(r^{-n} dx)} \nonumber \\
&\le  \|W\|_{L^{s}{\left( {B_{R_0}} \right) }} \|(\log r) e^{-\tau \phi(r)} r^{-\frac{n}{p}+\frac{n}{2}+2}|\nabla u|\|_{L^2(r^{-n} dx)} \nonumber \\
&\le  \|W\|_{L^{s}{\left( {B_{R_0}} \right) }} \|{\left( {\log r} \right) }^2 r^{1 + \frac n 2 - \frac n p}\|_{L^{\infty}{\left( {B_{R_0}} \right) }} \|(\log r)^{-1} e^{-\tau \phi(r)} r |\nabla u|\|_{L^2(r^{-n} dx)} \nonumber \\
&\le c K \|(\log r)^{-1} e^{-\tau \phi(r)} r |\nabla u|\|_{L^2(r^{-n} dx)},
\label{hod1}
\end{align}
where $p \in {\left( {\frac{2n}{n+2}, 2} \right] }$ implies that $1 + \frac n 2 - \frac n p > 0$, so that ${\left( {\log r} \right) }^2 r^{1 + \frac n 2 - \frac n p}$ is bounded on $B_{R_0}$.

By combining \eqref{triIneq0} and \eqref{hod1}, we see that
\begin{align*}
&\tau^{{\beta}_0} \|(\log r)^{-1} e^{-\tau \phi(r)}u\|_{L^2(r^{-n}dx)}
+\tau^{{\beta}_1} \|(\log r )^{-1} e^{-\tau \phi(r)}r \nabla u\|_{L^2(r^{-n}dx)} \nonumber \\
&\le C\|(\log r) e^{-\tau \phi(r)} r^2 (\triangle u+ W\cdot \nabla u)\|_{L^p(r^{-n} dx)}
+ c C K \|(\log r)^{-1} e^{-\tau \phi(r)} r |\nabla u|\|_{L^2(r^{-n} dx)}.
\end{align*}
Since ${\beta}_1 = \frac 1 2 - \frac{3n-2}{4s} = \frac{2s - {\left( {3n-2} \right) }}{4s} > 0$, if we choose $\tau \ge 1 + {\left( {c C K} \right) }^{\frac 1 {{\beta}_1}}$, then we may absorb the second term on the right into the lefthandside.
That is,
\begin{align*}
\tau^{{\beta}_0} \|(\log r)^{-1} e^{-\tau \phi(r)}u\|_{L^2(r^{-n}dx)} 
&\le C\|(\log r) e^{-\tau \phi(r)} r^2 (\triangle u+ W\cdot \nabla u)\|_{L^p(r^{-n} dx)},
\end{align*}
giving the conclusion of the theorem.
\end{proof}

Now we consider more general elliptic operators.

\begin{theorem}
Assume that for some $s \in {\left( {\frac{3n-2}{2}, {\infty}} \right] }$ and $t \in {\left( { n{\left( {\frac{3n-2}{5n-2}} \right) }, {\infty}} \right] }$, ${\left\vert\left\vert {W}\right\vert\right\vert}_{L^s{\left( {B_{R_0}} \right) }} \le K$ and ${\left\vert\left\vert {V}\right\vert\right\vert}_{L^t{\left( {B_{R_0}} \right) }} \le M$.
Then there exist constants $C_0$, $C_1$, $C_2$, and sufficiently small $R_0 < 1$  such that for any $u\in C^{\infty}_{0}(B_{R_0}(x_0)\setminus {\left\{{x_0}\right\}})$ and large positive constant
$$\tau \ge 1+ C_1 K^{\kappa} + C_2 M^{\mu},$$
one has
\begin{align}
\tau^{{\beta}_0} \|(\log r)^{-1} e^{-\tau \phi(r)}u\|_{L^2(r^{-n}dx)}
&\leq  C_0 \|(\log r ) e^{-\tau \phi(r)} r^2{\left( { \triangle u + W \cdot {\nabla} u + V u} \right) }\|_{L^p(r^{-n} dx)} ,
\label{main1}
\end{align}
where
${\displaystyle} \kappa = \left\{\begin{array}{ll}
\frac{4s}{2s - {\left( {3n-2} \right) }} & t > \frac{sn}{s+n} \medskip \\
\frac{4t}{{\left( {5 - \frac 2 n} \right) }t - {\left( {3n-2} \right) }} & n{\left( {\frac{3n-2}{5n-2}} \right) }
< t \le \frac{sn}{s+n}
\end{array}\right.$,
${\displaystyle} \mu = \left\{\begin{array}{ll}
\frac{4s}{6s - {\left( {3n-2} \right) }} & t \ge s \medskip \\
\frac{4 s t}{6 s t + {\left( {n+2} \right) }t -4ns} & \frac{sn}{s+n} < t < s \medskip \\
\frac{4t}{{\left( {5 - \frac 2 n} \right) }t - {\left( {3n-2} \right) }} & n{\left( {\frac{3n-2}{5n-2}} \right) }
< t \le \frac{sn}{s+n}
\end{array}\right.$,
${\displaystyle} p = \left\{\begin{array}{ll}
\frac{2s}{s+2} & t > \frac{sn}{s+n} \medskip \\
\frac{2 n t }{2n - 2t + nt } & n{\left( {\frac{3n-2}{5n-2}} \right) } < t \le
\frac{sn}{s+n}
\end{array}\right.$,
and ${\beta}_0 = {\beta}_0{\left( {2} \right) }$ as defined in Theorem \ref{Carlpq}.
Moreover, $C_0 = 2C$, where $C$ is from Theorem \ref{Carlpq}, $C_1 = C_1{\left( {n, s, t} \right) }$, and $C_2 = C_2{\left( {n,s,t} \right) }$.
\label{CarlpqVW}
\end{theorem}

\begin{proof}
If we add inequality \eqref{mainCar} from Theorem \ref{Carlpq} with $q = 2$ to the same inequality with $q$ arbitrary, we see that
\begin{align}
&\tau^{{\beta}_0{\left( {2} \right) }} \|(\log r)^{-1} e^{-\tau \phi(r)}u\|_{L^2(r^{-n}dx)}
+\tau^{{\beta}_0{\left( {q} \right) }} \|(\log r)^{-1} e^{-\tau \phi(r)}u\|_{L^q(r^{-n}dx)} \nonumber \\
&+\tau^{{\beta}_1} \|(\log r )^{-1} e^{-\tau \phi(r)}r \nabla u\|_{L^2(r^{-n}dx)} \nonumber \\
&\le 2 C\|(\log r) e^{-\tau \phi(r)} r^2 (\triangle u)\|_{L^p(r^{-n} dx)} \nonumber \\
&\le 2 C\|(\log r) e^{-\tau \phi(r)} r^2 (\triangle u+ W\cdot \nabla u + V u)\|_{L^p(r^{-n} dx)} \nonumber \\
&+ 2 C\|(\log r) e^{-\tau \phi(r)} r^2 W\cdot \nabla u \|_{L^p(r^{-n} dx)}
+ 2 C\|(\log r) e^{-\tau \phi(r)} r^2 V u \|_{L^p(r^{-n} dx)},
\label{triIneq}
\end{align}
where the last line follows from the triangle inequality.
Therefore, to reach the conclusion of the lemma, we need to choose $p$ and $q$, and make $\tau >> 1$, so that we may absorb the last two terms into the lefthand side.

An application of H\"older' inequality shows that if $p \in {\left( {\frac{2n}{n+2}, 2} \right] }$, then
\begin{align}
& \|(\log r) e^{-\tau\phi(r)} r^2 W\cdot\nabla u\|_{L^p(r^{-n} dx)} \nonumber \medskip \\
&\le  \|W\|_{L^{\frac{2p}{2-p}}{\left( {B_{R_0}} \right) }} \|{\left( {\log r} \right) }^2 r^{1 + \frac n 2 - \frac n p}\|_{L^{\infty}{\left( {B_{R_0}} \right) }} \|(\log r)^{-1} e^{-\tau \phi(r)} r |\nabla u|\|_{L^2(r^{-n} dx)} \nonumber \\
&\le c \|W\|_{L^{\frac{2p}{2-p}}{\left( {B_{R_0}} \right) }} \|(\log r)^{-1} e^{-\tau \phi(r)} r |\nabla u|\|_{L^2(r^{-n} dx)},
\label{hod2}
\end{align}
since $1 + \frac n 2 - \frac n p > 0$.
Similarly, if we further assume that $q \ge p$, then
\begin{align}
\|(\log r) e^{-\tau \phi(r)} r^2 V u \|_{L^p(r^{-n} dx)}
&\le  \|V\|_{L^{\frac{pq}{q-p}}{\left( {B_{R_0}} \right) }} \|{\left( {\log r} \right) }^2 r^{2 + \frac n q - \frac n p}\|_{L^{\infty}{\left( {B_{R_0}} \right) }}
\nonumber \\ &\times\|(\log r)^{-1} e^{-\tau \phi(r)} u\|_{L^q(r^{-n} dx)} \nonumber \\
&\le c \|V\|_{L^{\frac{pq}{q-p}}{\left( {B_{R_0}} \right) }} \|(\log r)^{-1} e^{-\tau \phi(r)} u\|_{L^q(r^{-n} dx)}.
\label{hod3}
\end{align}

{\bf Case 1: $t \in {\left[{s, {\infty}}\right]}$} \\
If $t \ge s$, then we choose $p = \frac{2s}{s+2}$ and $q = 2$.
Since $s > \frac{3n-2}{2}$, then $p$ is in the appropriate range.
As $\frac{2p}{2-p} = \frac{pq}{q-p} = s \le t$, then substituting \eqref{hod2} and \eqref{hod3} into \eqref{triIneq}, and using that $\|V\|_{L^{s}} \le c \|V\|_{L^{t}}$ by H\"older's inequality, we have that
\begin{align*}
&2\tau^{{\beta}_0{\left( {2} \right) }} \|(\log r)^{-1} e^{-\tau \phi(r)}u\|_{L^2(r^{-n}dx)}
+\tau^{{\beta}_1} \|(\log r )^{-1} e^{-\tau \phi(r)}r \nabla u\|_{L^2(r^{-n}dx)} \nonumber \\
&\le C\|(\log r) e^{-\tau \phi(r)} r^2 (\triangle u+ W\cdot \nabla u + V u)\|_{L^p(r^{-n} dx)} \nonumber \\
&+ 2 c C K \|(\log r)^{-1} e^{-\tau \phi(r)} r |\nabla u|\|_{L^2(r^{-n} dx)}
+ 2 c C M \|(\log r)^{-1} e^{-\tau \phi(r)} u\|_{L^q(r^{-n} dx)}.
\end{align*}
In this case, ${\beta}_0{\left( {2} \right) } = \frac{3}{2} - \frac{3n-2}{4s}$, ${\beta}_1 = \frac{1}{2} - \frac{3n-2}{4s}$, and the lower bound on $s$ ensures that ${\beta}_1 > 0$.
If we choose $\tau \ge 1 + {\left( {2 c C K} \right) }^{\frac 1 {{\beta}_1}}+ {\left( {2 c C M} \right) }^{\frac 1 {{\beta}_0{\left( {2} \right) }}} $, then we may absorb the last two terms on the right into the lefthandside to reach the conclusion of the theorem.

{\bf Case 2: $t \in {\left( {\frac{sn}{s+n}, s} \right) }$} \\
In this case, we choose $p = \frac{2s}{s+2}$ and $q = \frac{2st}{st+2t-2s}$.
As before, $p$ falls in the appropriate range and the bounds on $t$ ensure that $q \in {\left( {2, \frac{2n}{n-2}} \right) }$.
Since $\frac{2p}{2-p} = s$ and $\frac{pq}{q-p} = t$, then upon substituting \eqref{hod2} and \eqref{hod3} into \eqref{triIneq}, we see that
\begin{align}
&\tau^{{\beta}_0{\left( {2} \right) }} \|(\log r)^{-1} e^{-\tau \phi(r)}u\|_{L^2(r^{-n}dx)}
+\tau^{{\beta}_0{\left( {q} \right) }} \|(\log r)^{-1} e^{-\tau \phi(r)}u\|_{L^q(r^{-n}dx)} \nonumber \\
&+\tau^{{\beta}_1} \|(\log r )^{-1} e^{-\tau \phi(r)}r \nabla u\|_{L^2(r^{-n}dx)} \nonumber \\
&\le 2 C\|(\log r) e^{-\tau \phi(r)} r^2 (\triangle u+ W\cdot \nabla u + V u)\|_{L^p(r^{-n} dx)} \nonumber \\
&+ 2 c C K \|(\log r)^{-1} e^{-\tau \phi(r)} r |\nabla u|\|_{L^2(r^{-n} dx)}
+ 2 c C M \|(\log r)^{-1} e^{-\tau \phi(r)} u\|_{L^q(r^{-n} dx)}.
\label{boundSub}
\end{align}
Now ${\beta}_0{\left( {q} \right) } = \frac 3 2 + \frac{n+2}{4s} - \frac n t$ and ${\beta}_1 =\frac{1}{2}-\frac{3n-2}{4s}$ are both positive.
We again take $\tau \ge 1 + {\left( {2 c C K} \right) }^{\frac 1 {{\beta}_1}}+ {\left( {2 c C M} \right) }^{\frac 1 {{\beta}_0{\left( {q} \right) }}} $ in order to absorb the last two terms into the lefthand side.

{\bf Case 3: $t \in {\left( {n{\left( {\frac{3n-2}{5n-2}} \right) }, \frac{sn}{s+n}} \right] }$} \\
This time we choose $p = \frac{2nt}{2n-2t+nt}$ and $q = \frac{2n}{n-2}$.
Since $\frac n 2 < n{\left( {\frac{3n-2}{5n-2}} \right) }$ and $\frac{sn}{s+n} < n$, then $p \in {\left( {\frac{2n}{n+2}, 2} \right) }$.
Since $\frac{2p}{2-p} = \frac{nt}{n-t} \le s$, then an application of H\"older's inequality shows that $\|W\|_{L^{\frac{2p}{2-p}}} \le c \|W\|_{L^{s}}$.
Noting that $\frac{pq}{q-p} = t$, we again show that \eqref{boundSub} holds.
Now ${\beta}_1 = \frac 1 2 - \frac{3n-2}{4}{\left( {\frac 1 t - \frac 1 n} \right) }  = \frac {5n-2}{4n} - \frac{3n-2}{4t}$ and the lower bound on $t$ implies that ${\beta}_1 > 0$.
Since ${\beta}_0{\left( {q} \right) } = {\beta}_1$, then ${\beta}_0 > 0$ as well.
If we choose $\tau \ge 1 + {\left( {4 c C K} \right) }^{\frac 1 {{\beta}_1}} + {\left( {4 c C M} \right) }^{\frac 1 {{\beta}_0{\left( {q} \right) }}}$, we can absorb the last two terms into the lefthand side to reach the conclusion.
\end{proof}

Now we consider the second order elliptic equation $\triangle u+V(x)u=0$.
The proof is similar to the previous one.

\begin{theorem}
Assume that ${\left\vert\left\vert {V}\right\vert\right\vert}_{L^t{\left( {B_{R_0}} \right) }} \le M$ for some $t > \frac{4 n^2}{7n+2}$.
Then there exist constants $C_0$, $C_2$, and sufficiently small $R_0 < 1$  such that for any $u\in C^{\infty}_{0}(B_{R_0}(x_0)\setminus {\left\{{x_0}\right\}})$ and large positive constant
$$\tau \ge 1 + C_2 M^{\mu},$$
one has
\begin{align}
\tau^{{\beta}_0} \|(\log r)^{-1} e^{-\tau \phi(r)}u\|_{L^2(r^{-n}dx)}
&\leq  C_0 \|(\log r ) e^{-\tau \phi(r)} r^2{\left( { \triangle u + V u} \right) }\|_{L^p(r^{-n} dx)} ,
\label{main3}
\end{align}
where for any positive sufficiently small constant ${\varepsilon} <
\min{\left\{{\frac{7t+\frac{2t}n-4n}{2}, \frac{(2t-n)(n+2)}{2n} }\right\}}$, we
have ${\displaystyle} \mu = \left\{\begin{array}{ll}
\frac{4t}{6t - {\left( {3n-2} \right) }} & t > n \medskip \\
\frac{4t} {7t+\frac{2t}n-4n-{\varepsilon}} & \frac{4n^2}{7n+2} < t \le n
\end{array}\right.$,
${\displaystyle} p = \left\{\begin{array}{ll}
\frac{2t}{t+2} & t > n \medskip \\
\frac{2n}{n + 2 -  \frac{2n{\varepsilon}}{{\left( {n+2} \right) }t} }& \frac{4n^2}{7n+2} < t
\le n
\end{array}\right.$,
and ${\beta}_0 = {\beta}_0{\left( {2} \right) }$ as defined in Theorem \ref{Carlpq}.
Moreover, $C_0 = 2 C$, where $C$ is from Theorem \ref{Carlpq}, and $C_2 = C_2{\left( {n,t, {\varepsilon}} \right) }$.
\label{CarlpqV}
\end{theorem}

\begin{proof}
As in the previous proof, if we add inequality \eqref{mainCar} from Theorem \ref{Carlpq} with $q = 2$ to the same inequality with $q$ arbitrary, we see that
\begin{align}
&\tau^{{\beta}_0{\left( {2} \right) }} \|(\log r)^{-1} e^{-\tau \phi(r)}u\|_{L^2(r^{-n}dx)}
+\tau^{{\beta}_0{\left( {q} \right) }} \|(\log r)^{-1} e^{-\tau \phi(r)}u\|_{L^q(r^{-n}dx)} \nonumber \\
&\leq 2 C \|(\log r ) e^{-\tau \phi(r)} r^2 \triangle u\|_{L^p(r^{-n} dx)}  \nonumber \\
&\le 2 C\|(\log r) e^{-\tau \phi(r)} r^2 (\triangle u + V u)\|_{L^p(r^{-n} dx)}
+ 2 C\|(\log r) e^{-\tau \phi(r)} r^2 V u \|_{L^p(r^{-n} dx)},
\label{triIneqV}
\end{align}
where we have used the triangle inequality to reach the last line.
We need to choose $p, q$ so that the last term on the right can be absorbed into the second term on the left while making $\tau$ minimally large.
As shown in Theorem \ref{CarlpqVW}, if $p \in {\left( {\frac{2n}{n+2}, 2} \right] }$ and $q \ge p$, then
\begin{align}
\|(\log r) e^{-\tau \phi(r)} r^2 V u \|_{L^p(r^{-n} dx)}
&\le c \|V\|_{L^{\frac{pq}{q-p}}{\left( {B_{R_0}} \right) }} \|(\log r)^{-1} e^{-\tau \phi(r)} u\|_{L^q(r^{-n} dx)}.
\label{hod4}
\end{align}
Again, we will work in different cases corresponding to different ranges of $t$.

{\bf Case 1: $t > n$} \\
Set $p = \frac{2t}{t+2}$ and $q = 2$.
Since $t > n$, then $p \in \left(\frac{2n}{n+2}, 2\right]$, as required.
As $\frac{pq}{q-p} = t$, then by combining \eqref{triIneqV} and \eqref{hod4}, we see that
\begin{align*}
&2\tau^{{\beta}_0{\left( {2} \right) }} \|(\log r)^{-1} e^{-\tau \phi(r)}u\|_{L^2(r^{-n}dx)}
 \nonumber \\
&\le 2 C\|(\log r) e^{-\tau \phi(r)} r^2 (\triangle u + V u)\|_{L^p(r^{-n} dx)}
+ 2 c C M \|(\log r)^{-1}e^{-\tau \phi(r)} u\|_{L^q(r^{-n}dx)}.
\end{align*}
Since ${\beta}_0{\left( {2} \right) } = \frac 3 2 - \frac{3n-2}{4t} > 0$, if we ensure that $\tau \ge 1 + {\left( {2 c C M} \right) }^{\frac 1 {{\beta}_0}}$, then the second term on the right may be absorbed into the left and the conclusion of the theorem follows.

{\bf Case 2:} $t \in{\left( {\frac{4n^2}{7n+2}, n} \right] }$. \\
Choose ${\varepsilon} \in {\left( {0, \min{\left\{{\frac{7t+\frac{2t}n-4n}{2},
\frac{(2t-n)(n+2)}{2n}}\right\}}} \right) }$ to be arbitrarily small. Set $p =
\frac{2n}{n + 2 - \frac{2n{\varepsilon}}{{\left( {n+2} \right) }t} }$ and take $q =
\frac{pt}{t-p}$. Since ${\varepsilon}
<\frac{(2t-n)(n+2)}{n}<\frac{(n+2)t}{n}$, then $p \in
{\left( {\frac{2n}{n+2}, 2} \right) }$ and $q \in {\left( {2, \frac{2n}{n-2}} \right) }$. As
$\frac{pq}{q-p} = t$, then upon substituting \eqref{hod4} into
\eqref{triIneqV}, we see that
\begin{align*}
&\tau^{{\beta}_0{\left( {2} \right) }} \|(\log r)^{-1} e^{-\tau \phi(r)}u\|_{L^2(r^{-n}dx)}
+\tau^{{\beta}_0{\left( {q} \right) }} \|(\log r)^{-1} e^{-\tau \phi(r)}u\|_{L^q(r^{-n}dx)} \nonumber \\
&\le 2 C\|(\log r) e^{-\tau \phi(r)} r^2 (\triangle u + V u)\|_{L^p(r^{-n} dx)}
+ 2 c C M \|(\log r)^{-1} e^{-\tau \phi(r)} u\|_{L^q(r^{-n} dx)}
\label{triIneqV}
\end{align*}
Now ${\beta}_0{\left( {q} \right) } = \frac {7t+\frac{2t}n-4n-{\varepsilon}}{4t} >0$ (since $2{\varepsilon} <7t+\frac{2t}n-4n$) and if we choose $\tau \ge 1 + {\left( {2 c C M} \right) }^{\frac 1 {{\beta}_0{\left( {q} \right) }}}$, the conclusion of the theorem follows.
\end{proof}

\section{Proof of $L^p\to L^{q}$ Carleman estimates }
\label{CarlProofs}

In this section, we prove the crucial tool in the whole paper, i.e. the $L^p - L^q$ Carleman estimate stated in Theorem \ref{Carlpq}.
To prove our Carleman estimate, we first establish some intermediate Carleman estimates for first-order operators.

We introduce polar coordinates in $\mathbb R^n\backslash \{0\}$ by setting $x=r\omega$, with $r=|x|$ and $\omega=(\omega_1,\cdots,\omega_n)\in S^{n-1}$.
Further, we use a new coordinate $t=\log r$.
Then
$$ \frac{\partial }{\partial x_j}=e^{-t}(\omega_j\partial_t+  \Omega_j), \quad 1\leq j\leq n, $$
where $\Omega_j$ is a vector field in $S^{n-1}$.
It is well known that vector fields $\Omega_j$ satisfy
$$ \sum^{n}_{j=1}\omega_j\Omega_j=0 \quad \mbox{and} \quad
\sum^{n}_{j=1}\Omega_j\omega_j=n-1.$$
In the new coordinate system, the Laplace operator takes the form
\begin{equation}
e^{2t} \triangle=\partial^2_t u+(n-2)\partial_t+\triangle_{\omega},
\label{laplace}
\end{equation}
where ${\displaystyle} \triangle_\omega=\sum_{j=1}^n \Omega^2_j$ is the Laplace-Beltrami operator on $S^{n-1}$.
The eigenvalues for $-\triangle_\omega$ are $k(k+n-2)$, $k\in \mathbb{N}$, where $\mathbb{N}$ denotes the set of nonnegative integers.
The corresponding eigenspace is $E_k$, the space of spherical harmonics of degree $k$.
It follows that
$$\| \triangle_\omega v\|^2_{L^2(dtd\omega)}=\sum_{k\geq 0} k^2(k+n-2)^2\| v_k\|^2_{L^2(dtd\omega)}$$
and
\begin{equation}
\sum_{j=1}^n\| \Omega_j v\|^2_{L^2(dtd\omega)}
=\sum_{k\geq 0} k(k+n-2)\|v_k\|^2_{L^2(dtd\omega)},
\label{lll}
\end{equation}
where $v_k$ denotes the projection of $v$ onto $E_k$.
Here $\|\cdot\|_{L^2(dtd\omega)}$ denotes the $L^2$ norm on $(-\infty, \infty)\times S^{n-1}$.

 Let
$$\Lambda=\sqrt{\frac{(n-2)^2}{4}-\triangle_\omega}.$$ The operator $\Lambda$ is a
first-order elliptic pseudodifferential operator on $L^2(S^{n-1})$.
The eigenvalues for the operator $\Lambda$ are $k+\frac{n-2}{2}$, with corresponding eigenspace $E_k$.
That is, for any $v\in C^\infty_0(S^{n-1})$,
\begin{equation}
{\left( {\Lambda - \frac{n-2}{2}} \right) } v= \sum_{k\geq 0}k P_k v,
\label{ord}
\end{equation}
where $P_k$ is the projection operator from $L^2(S^{n-1})$ onto $E_k$.
We remark that the projection operator, $P_k$, acts only on the angular variables.
In particular, $P_k v{\left( {t, {\omega}} \right) } = P_k v{\left( {t, \cdot} \right) } {\left( {\omega} \right) }$.

Now define
\begin{equation} L^\pm=\partial_t+\frac{n-2}{2}\pm \Lambda.
\label{use}
\end{equation} From the equation (\ref{laplace}), it follows that
\begin{equation*}
e^{2t} \triangle=L^+L^-=L^-L^+.
\end{equation*}

Recall that we introduced the weight function
$$\phi(r)=\log r+\log(\log r)^2.$$
With $r=e^t$, define the weight function in terms of $t$ to be
$$\varphi(t)=\phi(e^t)=t+\log t^2.$$
We are only interested in those values of $r$ that are sufficiently small.
Since $r\to 0$ if and only if $t\to-\infty$ then, in terms of the new coordinate $t$, we study the case when $t$ is sufficiently close to $-\infty$.

We first establish an $L^2- L^2$ Carleman inequality for the operator $L^+$.

\begin{lemma}
If ${\left\vert{t_0}\right\vert}$ is sufficiently large, then for any $v \in C^{\infty}_0{\left( {{\left( {-{\infty}, -{\left\vert{t_0}\right\vert}} \right) } \times S^{n-1}} \right) }$, we have that
\begin{eqnarray}
\tau {\left\vert\left\vert {t^{-1} e^{-\tau \varphi(t)}v}\right\vert\right\vert}_{L^2(dtd\omega )}
&+&{\left\vert\left\vert {t^{-1}e^{-\tau \varphi(t)} \partial_t v}\right\vert\right\vert}_{L^2(dtd\omega )}
+\sum_{j=1}^n {\left\vert\left\vert {t^{-1}e^{-\tau \varphi(t)} \Omega_j v }\right\vert\right\vert}_{L^2(dtd\omega )}   \nonumber \medskip\\
&\leq& C{\left\vert\left\vert {t^{-1} e^{-\tau \varphi(t)} L^+v}\right\vert\right\vert}_{L^2(dtd\omega )}.
\label{cond}
 \end{eqnarray}
\label{Car22}
\end{lemma}

\begin{proof}
Recall that $L^+=\partial_t+\Lambda+\frac{n-2}{2}$.
Let $v=e^{\tau\varphi(t)}u$.
A computation shows that
\begin{align}
L^+_{\tau} u
&:= e^{- \tau {\varphi}{\left( {t} \right) }} L^+ v
= e^{- \tau {\varphi}{\left( {t} \right) }} {\left( {\partial_t+\Lambda+\frac{n-2}{2}} \right) } e^{\tau\varphi(t)}u \nonumber \\
&= \partial_t u
+ \Lambda u
+  \frac{n-2}{2} u
+ \tau {\varphi}^\prime{\left( {t} \right) } u \nonumber \\
&=  \partial_t u + \sum_{k \ge 0} {\left( {k+ n-2} \right) } u_k + \tau {\left( {1 + 2 t^{-1}} \right) } u,
\label{Lpt}
\end{align}
where the last line follows from an application of \eqref{ord} with $u_k = P_k u$.
It is clear that
\begin{eqnarray}
{\left\vert\left\vert {t^{-1}e^{-\tau\varphi(t)}L^+v}\right\vert\right\vert}^2_{L^2(dtd\omega)} &=&
\iint t^{-2}{\left( {\partial_t u+\Lambda u+\tau u+2\tau t^{-1} u +\frac{n-2}{2}u } \right) }^2 dtd\omega \nonumber \\
&=& \iint t^{-2} {\left\vert{\partial_t u}\right\vert}^2 dtd{\omega} +  2 \iint t^{-2}
{ \partial}_t u \sum_{k \ge 0} {\left( {k+ n-2} \right) } u_k \, dt d{\omega}
\nonumber \\
&+&  2 \tau \iint t^{-2}  {\left( {1 + 2 t^{-1}} \right) }
u { \partial}_t u \, dt d{\omega} \nonumber\\
&+& \iint t^{-2}{\left( {\Lambda u+\tau u+2\tau t^{-1} u +\frac{n-2}{2}
u} \right) }^2 dtd{\omega} .
\label{comb1}
\end{eqnarray}
 Keeping in mind that $t < 0$, integration by parts then gives
\begin{eqnarray}
2 \iint t^{-2} { \partial}_t u \sum_{k \ge 0} {\left( {k+ n-2} \right) } u_k \, dt d{\omega}
&=&  \iint t^{-2} \sum_{k \ge 0} {\left( {k+ n-2} \right) } { \partial}_t {\left\vert{u_k}\right\vert}^2 \, dt d{\omega} \nonumber \\
&=& - 2 \sum_{k \ge 0} {\left( {k+ n-2} \right) } {\left\vert\left\vert {t^{-3/2} u_k}\right\vert\right\vert}^2_{L^2{\left( { dt
d{\omega}} \right) }}, \label{ana1}
\end{eqnarray}
and
\begin{eqnarray}
 2 \tau \int t^{-2} { \partial}_t u {\left( {1 + 2 t^{-1}} \right) } u
&=& \tau \int t^{-2} {\left( {1 + 2 t^{-1}} \right) } { \partial}_t{\left\vert{u}\right\vert}^2 dt d{\omega} \nonumber\\
&=& - 2 \tau {\left\vert\left\vert { t^{-3/2}u}\right\vert\right\vert}^2_{L^2{\left( {dt d{\omega}} \right) }} + 6 \tau {\left\vert\left\vert {
t^{-2} u}\right\vert\right\vert}^2_{L^2{\left( {d t d{\omega}} \right) }} . \label{ana2}
\end{eqnarray}
By the definition of $\Lambda$, we have
\begin{eqnarray}
&&{\left\vert\left\vert { t^{-1}{\left( {\Lambda +\tau +2\tau t^{-1} +\frac{n-2}{2}} \right) } u
}\right\vert\right\vert}^2_{L^2(dtd\omega)} \nonumber \\ &&=  \sum_{k \ge 0} \iint
t^{-2}{\left[{k+ n-2 + \tau {\left( {1 + 2 t^{-1}} \right) }}\right]}^2 {\left\vert{u_k}\right\vert}^2 dt d{\omega}.
\end{eqnarray}
Since
\begin{align*}
{\left[{k+n -2 + \tau+2\tau t^{-1}}\right]}^2
\geq \frac{3\tau^2}{4} +4(\tau t^{-1})^2 +k(k+n-2) - 2 t^{-1}{\left( {k + n - 2} \right) } - 2 \tau t^{-1}
\end{align*}
for $|t_0|$ large enough, then
\begin{align*}
&{\left\vert\left\vert { t^{-1}{\left( {\Lambda +\tau +\tau t^{-1} +\frac{n-2}{2}} \right) }
u}\right\vert\right\vert}^2_{L^2(dtd\omega)} \ge
\frac{3\tau^2}{4}{\left\vert\left\vert {t^{-1}u}\right\vert\right\vert}^2_{L^2(dtd\omega)}
+ 4\tau^2 {\left\vert\left\vert {t^{-2}u}\right\vert\right\vert}^2_{L^2(dtd\omega)} \\
&+\sum_{k\geq 0}  k(k+n-2) {\left\vert\left\vert {t^{-1}u_k}\right\vert\right\vert}^2_{L^2(dtd\omega)}
+ 2 \sum_{k\geq0} {\left( {k + n - 2} \right) } {\left\vert\left\vert {t^{-3/2}u_k}\right\vert\right\vert}^2_{L^2(dtd\omega)}
+ 2 \tau {\left\vert\left\vert {t^{-3/2}u_k}\right\vert\right\vert}^2_{L^2(dtd\omega)}.
\end{align*}
Combining the previous estimate with \eqref{comb1}, \eqref{ana1} and \eqref{ana2}, it follows
that
\begin{align*}
{\left\vert\left\vert {t^{-1} e^{-\tau\varphi(t)}L^+v}\right\vert\right\vert}^2_{L^2(dtd\omega)}
&\ge {\left\vert\left\vert {t^{-1} \partial_t u }\right\vert\right\vert}^2_{L^2(dtd\omega)}
+ \frac{\tau^2}{2} {\left\vert\left\vert {t^{-1}u}\right\vert\right\vert}^2_{L^2(dtd\omega)}
+ 2\tau^2 {\left\vert\left\vert {t^{-2}u}\right\vert\right\vert}^2_{L^2(dtd\omega)} \\
&+ \sum_{k\geq 0}  k(k+n-2){\left\vert\left\vert {t^{-1}u_k}\right\vert\right\vert}^2_{L^2(dtd\omega)}
+ \frac{\tau^2}{4} {\left\vert\left\vert {t^{-1}u}\right\vert\right\vert}^2_{L^2(dtd\omega)} .
\end{align*}
Recalling that $u = e^{- \tau {\varphi}{\left( {t} \right) }} v$, 
we see by the triangle inequality that
\begin{align*}
{\left\vert\left\vert {t^{-1}{ \partial}_t u }\right\vert\right\vert}^2_{L^2(dtd\omega)}
+  \frac{\tau^2}{2} {\left\vert\left\vert {t^{-1} u}\right\vert\right\vert}^2_{L^2(dtd\omega)}
+  2\tau^2 {\left\vert\left\vert {t^{-2}  u }\right\vert\right\vert}^2_{L^2(dtd\omega)}
&\ge \frac 1 5 {\left\vert\left\vert {t^{-1}e^{- \tau {\varphi}{\left( {t} \right) }}\partial_t v }\right\vert\right\vert}^2_{L^2(dtd\omega)} .
\end{align*}
Substituting this expression into the previous inequality, and using an application of \eqref{lll} gives
\begin{align*}
C {\left\vert\left\vert {t^{-1} e^{-\tau\varphi(t)}L^+v}\right\vert\right\vert}^2_{L^2(dtd\omega)}
&\ge \tau^2 {\left\vert\left\vert {t^{-1}e^{- \tau {\varphi}{\left( {t} \right) }} v}\right\vert\right\vert}^2_{L^2(dtd\omega)}
+ {\left\vert\left\vert {t^{-1}e^{- \tau {\varphi}{\left( {t} \right) }}\partial_t v }\right\vert\right\vert}^2_{L^2(dtd\omega)} \\
&+ \sum_{j=1}^n\| t^{-1} e^{- \tau {\varphi}{\left( {t} \right) }} \Omega_j  v\|^2_{L^2(dtd\omega)},
\end{align*}
since ${\Omega}_j$ acts only on the angular variables.
This implies \eqref{cond}.
\end{proof}

Using a similar process, we also establish an $L^2 - L^2$ Carleman estimate for $L^-$.
Notice that the power on $\tau$ is different here from above.

\begin{lemma}
For every $v \in C^{\infty}_c{\left( {(-\infty, \ t_0)\times S^{n-1}} \right) }$, it
holds that
\begin{equation}
\|t^{-{1}} e^{-\tau \varphi(t)} v\|_{L^2(dtd\omega)} \leq C
\tau^{-\frac 1 2} \| e^{-\tau \varphi(t)} L^- v\|_{L^2(dtd\omega)}.
\label{key-q22}
\end{equation}
\label{CarL-qq}
\end{lemma}

\begin{proof}
Recall that $L^-=\partial_t+\frac{n-2}{2}-\Lambda$.
Let $v=e^{\tau\varphi(t)}u$.
Direct computations show that
\begin{align*}
L^-_{\tau} u &:= e^{- \tau {\varphi}{\left( {t} \right) }} L^- v
= e^{- \tau {\varphi}{\left( {t} \right) }} {\left( {\partial_t+\frac{n-2}{2}-\Lambda} \right) } e^{\tau\varphi(t)}u \nonumber \\
&= \partial_t u -\Lambda u +  \frac{n-2}{2} u
+ \tau {\varphi}^\prime{\left( {t} \right) } u \nonumber \\
&=  \partial_t u -\sum_{k \ge 0} k u_k + \tau {\left( {1 + 2 t^{-1}} \right) } u,
\end{align*}
where the last line follows from the application of \eqref{ord} with
$u_k = P_k u$.
It is true that
\begin{eqnarray*}
\|e^{-\tau\varphi(t)}L^-v\|^2_{L^2(dtd\omega)} &=&
\iint {\left[{\partial_t u+{\left( {\frac{n-2}{2}-\Lambda} \right) } u+\tau u+2\tau t^{-1} u  }\right]}^2 dtd\omega \nonumber \\
&=& \iint  {\left\vert{\partial_t u}\right\vert}^2 dtd{\omega}
+  \iint {\left[{{\left( {\frac{n-2}{2}-\Lambda} \right) } u+\tau u+2\tau t^{-1} u }\right]}^2
dtd{\omega}  \\
&+&  2 \iint  { \partial}_t u \sum_{k
\ge 0} (-k) u_k \, dt d{\omega}
+  2 \tau \iint   {\left( {1 + 2 t^{-1}} \right) }
u { \partial}_t u \, dt d{\omega} .
\end{eqnarray*}
Integration by parts then gives
\begin{eqnarray*}
2 \iint  { \partial}_t u \sum_{k \ge 0} {\left( {-k} \right) } u_k \, dt d{\omega} =0
\end{eqnarray*}
and
\begin{eqnarray*}
 2 \tau \int  { \partial}_t u {\left( {1 + 2 t^{-1}} \right) } u
= \tau \int  {\left( {1 + 2 t^{-1}} \right) } { \partial}_t{\left\vert{u}\right\vert}^2 dt d{\omega} 
= 2 \tau {\left\vert\left\vert { t^{-1}u}\right\vert\right\vert}^2_{L^2{\left( {dt d{\omega}} \right) }}.
\end{eqnarray*}
Since
\begin{eqnarray*}
&&\|{\left[{{\left( {\frac{n-2}{2}-\Lambda} \right) } u+\tau u+2\tau t^{-1} u
}\right]}\|^2_{L^2(dtd\omega)}
\geq 0.
\end{eqnarray*}
then it follows that
\begin{align*}
{\left\vert\left\vert { e^{-\tau\varphi(t)}L^-v}\right\vert\right\vert}^2_{L^2(dtd\omega)} \ge 2 \tau {\left\vert\left\vert {t^{-1}u}\right\vert\right\vert}^2_{L^2(dtd\omega)} .
\end{align*}
Recalling that $u = e^{- \tau {\varphi}{\left( {t} \right) }} v$,
 this implies the estimate \eqref{key-q22}.
\end{proof}

Our next task is to establish $L^p - L^2$ Carleman estimates for the operator $L^-$.
For these results, we require the following lemma which relies on the eigenfunction estimates of Sogge \cite{Sog86}.

\begin{lemma}
Let $N, M \in {\ensuremath{\mathbb{N}}}$ and let ${\left\{{c_k}\right\}}$ be a sequence of numbers such that ${\left\vert{c_k}\right\vert} \le 1$ for all $k$.
For any $v \in L^2{\left( {S^{n-1}} \right) }$, we have that
\begin{align}
\|\sum^M_{k=N} c_k P_k v\|_{L^2(S^{n-1})} &\leq C
{\left[{M^{\frac{n-2}{2}} {\left( { \sum^M_{k=N}
|c_k|^2} \right) }^{\frac{n}{2}}}\right]}^{\frac{1}{p}-\frac{1}{2}}\|
v\|_{L^p(S^{n-1})} \label{haha}
\end{align}
for all all $\frac{2n}{n+2}\leq p\leq 2$. \label{upDown}
\end{lemma}

\begin{proof}
Sogge's \cite{Sog86} eigenfunction estimates state that there is a constant $C$, depending only on $n \ge 3$, such that for any $v\in L^2(S^{n-1})$,
\begin{equation}
\| P_k v\|_{L^{\frac{2n}{n-2}}(S^{n-1})}
\leq Ck^{1-\frac{2}{n}} \|v\|_{L^{\frac{2n}{n+2}}(S^{n-1})}.
\label{sogg}
\end{equation}
Recall that $P_k v = v_k$ is the projection of $v$ onto the space of spherical harmonics of degree $k$.
By orthogonality, H\"older's inequality, and \eqref{sogg},
\begin{eqnarray}
\| P_k v\|^2_{L^{2}(S^{n-1})}
&\leq & \| P_k v\|_{L^{\frac{2n}{n-2}}(S^{n-1})} \|  v\|_{L^{\frac{2n}{n+2}}(S^{n-1})} 
\leq  Ck^{1-\frac{2}{n}}\|  v\|^2_{L^{\frac{2n}{n+2}}(S^{n-1})}.
\label{up}
\end{eqnarray}
It is obvious that
\begin{equation}
 \| P_k v\|_{L^{2}(S^{n-1})} \leq \|v\|_{L^{2}(S^{n-1})}.
 \label{stay}
\end{equation}

Interpolating \eqref{up} and \eqref{stay} gives that
\begin{align}
\| P_k v\|_{L^{2}(S^{n-1})} &\leq C k^{\frac{(n-2)(2-p)}{4p}}
\|v\|_{L^{p}(S^{n-1})} \label{indu}
\end{align}
for all $\frac{2n}{n+2}\leq p\leq 2$.

Now we consider a more general case of the previous inequality.
Let $\{c_k\}$ be a sequence of numbers with $|c_k|\leq 1$.
For all $N\leq M$, by H\"older's inequality, it follows that
\begin{equation*}
\|\sum^{M}_{k=N} c_k P_k v\|^2_{L^2(S^{n-1})}
\leq C{\left( {\sum^M_{k=N} |c_k|^2 \|P_k v\|_{L^{\frac{2n}{n-2}}(S^{n-1})}} \right) }\|v\|_{L^{\frac{2n}{n+2}}(S^{n-1})}.
\end{equation*}
An application of Sogge's estimate \eqref{sogg} shows that
\begin{equation*}
\|\sum^M_{k=N} c_k P_k v\|_{L^2(S^{n-1})}
\leq C M^\frac{n-2}{2n} {\left( {\sum^M_{k=N} |c_k|^2} \right) }^{\frac{1}{2}} \|v\|_{L^{\frac{2n}{n+2}}(S^{n-1})}.
\end{equation*}
For any sequence ${\left\{{d_k}\right\}}$ such that each $|d_k|\leq 1$, it is true that
\begin{equation}
\|\sum^M_{k=N} d_k P_k v\|_{L^2(S^{n-1})}
\leq \| v\|_{L^2(S^{n-1})}.
\label{seqSame}
\end{equation}
As before, we interpolate the last two inequalities (with $d_k = c_k$) and conclude that \eqref{haha} holds.

\end{proof}

Now we prove an $L^p-L^2$ type Carleman estimate for the operator $L^-$.

\begin{lemma}
For every $v \in C^{\infty}_c{\left( {(-\infty, \ t_0)\times S^{n-1}} \right) }$ and $\frac{2n}{n+2}< p < 2$,
\begin{equation}
\|t^{-{1}} e^{-\tau \varphi(t)} v\|_{L^2(dtd\omega)} \leq C
\tau^\beta \|t e^{-\tau \varphi(t)} L^- v\|_{L^p(dtd\omega)},
\label{key-}
\end{equation}
where ${\beta} = - \frac 1 2 + \frac{{\left( {3n-2} \right) }{\left( {2-p} \right) }}{8p}$.
\label{CarL-p}
\end{lemma}

\begin{proof}
To prove this lemma, we introduce the conjugated operator $L^-_\tau$ of $L^-$, defined by
$$L^-_\tau u=e^{-\tau\varphi(t)}L^-(e^{\tau \varphi(t)}u).$$
With $v=e^{\tau \varphi(t)}u$, inequality \eqref{key-} is equivalent to
\begin{equation}
\|t^{-{1}}u \|_{L^2(dtd\omega)}\leq C \tau^\beta \|t L^-_\tau
u\|_{L^p(dtd\omega)}. \label{keyu}
\end{equation}

From \eqref{ord} and \eqref{use}, the operator $L^-_\tau$ takes the form
\begin{equation}
L^-_\tau=\sum_{k\geq 0} (\partial_t+\tau \varphi'(t)-k)P_k.
\label{ord1}
\end{equation}
Since ${\displaystyle} \sum_{k\geq 0} P_k v= v,$ we split ${\displaystyle} \sum_{k \ge 0} P_k v$ into two sums.
Let $M=\lceil 2\tau\rceil$ and define
$$ P^+_\tau=\sum_{k> M}P_k, \quad \quad  P^-_\tau=\sum_{k=0}^{M}P_k.      $$
In order to prove the \eqref{keyu}, it suffices to show that
\begin{equation}
\| t^{-1} P^+_\tau {u} \|_{L^2(dtd\omega)} \leq \tau^{\beta}\|
t{L_\tau^- u} \|_{L^{{p}}(dtd\omega)}
\label{key1}
\end{equation}
and
\begin{equation}
\| t^{-1} P^-_\tau {u}\|_{L^2(dtd\omega)}\leq \tau^\beta\| t
{L_\tau^- u} \|_{L^{{p}}(dtd\omega)},
\label{key2}
\end{equation}
for all $u \in C^{\infty}_c{\left( {(-\infty, \ t_0)\times S^{n-1}} \right) }$ and $\frac{2n}{n+2}< p < 2$.
The sum of \eqref{key1} and \eqref{key2} will yield \eqref{keyu}, which implies \eqref{key-}.
We first establish \eqref{key1}.
From \eqref{ord1}, we have the first order differential equation
\begin{equation}
P_k L^-_\tau u= (\partial_t+\tau \varphi'(t)-k)P_k u.
\label{sord}
\end{equation}
For $u\in C^\infty_{0}{\left( { (-\infty, \ t_0)\times S^{n-1}} \right) }$
, solving the first order differential equation gives that
\begin{equation}
P_k u(t, \omega)
=-\int_{-\infty}^{\infty} H(s-t)e^{k(t-s)+\tau{\left[{\varphi(s)-\varphi(t)}\right]}} P_k L^-_\tau u (s, \omega)\, ds,
\label{star}
\end{equation}
where $H(z)=1$ if $z\geq 0$ and $H(z)=0$ if $z<0$.

For $k\geq M \ge 2 \tau$, we obtain that
\begin{equation*}
H(s-t)e^{k(t-s)+\tau{\left[{\varphi(s)-\varphi(t)}\right]}}
\leq e^{-\frac{1}{2}k|t-s|}
\end{equation*}
for all $s, t\in (-\infty, \ t_0)$.
Taking the $L^2{\left( {S^{n-1}} \right) }$-norm in \eqref{star} gives that
\begin{equation*}
\|P_k u(t, \cdot)\|_{L^2(S^{n-1})}
\leq  \int_{-\infty}^{\infty} e^{-\frac{1}{2}k|t-s|} \|P_k L^-_\tau u(s, \cdot)\|_{L^2(S^{n-1})} \,ds.
\end{equation*}
With the aid of \eqref{indu}, we get
\begin{equation*}
\|P_k u(t, \cdot)\|_{L^2(S^{n-1})}\leq C k^{\frac{(n-2)(2-p)}{4p}}
\int_{-\infty}^{\infty} e^{-\frac{1}{2}k|t-s|} \|L^-_\tau u(s,
\cdot)\|_{L^p(S^{n-1})} \,ds
\end{equation*}
for all $\frac{2n}{n+2}\leq p\leq 2$.
Applying Young's inequality for convolution then yields
\begin{equation*}
\|P_k u\|_{L^2(dt d\omega)}
\leq C k^{\frac{(n-2)(2-p)}{4p}} {\left( {\int_{-\infty}^{\infty} e^{-\frac{\sigma}{2}k|z|} dz} \right) }^{\frac{1}{\sigma}}\|L^-_\tau u\|_{L^p(dtd\omega)}
\end{equation*}
with $\frac{1}{\sigma}=\frac{3}{2}-\frac{1}{p}$.
A calculation shows that
$$ {\left( {\int_{-\infty}^{\infty} e^{-\frac{\sigma}{2}k|z|}} \right) }^{\frac{1}{\sigma}}\leq C k^{\frac{1}{p} - \frac{3}{2}}.  $$
Therefore,
\begin{equation*}
\|P_k u\|_{L^2(dt d\omega)}\leq C
k^{-1+\frac{n{\left( {2-p} \right) }}{4p}} \|L^-_\tau
u\|_{L^p(dtd\omega)}.
\end{equation*}
Squaring and summing up $k> M$ gives that
$$\sum_{k> M} \|P_k u\|^2_{L^2(dt d\omega)}
\leq C \sum_{k> M} k^{-2+\frac{n{\left( {2-p} \right) }}{2p}} \|L^-_\tau u\|^2_{L^p(dtd\omega)}.$$
Since $\frac{2n}{n+2}<p$, then $-2+\frac{n{\left( {2-p} \right) }}{2p} <-1$.
Thus, ${\displaystyle} \sum_{k> M} k^{-2+\frac{n{\left( {2-p} \right) }}{2p}}$ converges.
Note that at the borderline, where $p=\frac{2n}{n+2}$,  we have that $-2+\frac{n{\left( {2-p} \right) }}{2p}=-1$ and then the series ${\displaystyle} \sum_{k\geq M} k^{-2+\frac{n{\left( {2-p} \right) }}{2p}}$ diverges.
Further calculations show that
$$\sum_{k> M} k^{-2+\frac{n{\left( {2-p} \right) }}{2p}}\leq C M^{-1+\frac{n{\left( {2-p} \right) }}{2p}} \leq
C \tau^{-1+\frac{n{\left( {2-p} \right) }}{2p}}.$$
Recalling that $2{\beta} =- 1 + \frac{{\left( {3n-2} \right) }{\left( {2-p} \right) }}{4p}$, since $p < 2$ and $n\geq 3$, then
$-1+\frac{n{\left( {2-p} \right) }}{2p} \leq 2 \beta$.
Therefore,
\begin{equation*}
\|  P^+_\tau u\|_{L^2(dtd\omega)}\leq C\tau^{\beta}\| L^-_\tau
u\|_{L^p(dtd\omega)},
\end{equation*}
which implies estimate \eqref{key1} since $u$ and $L_\tau^-u$ are supported on ${\left( {-{\infty}, -{\left\vert{t_0}\right\vert}} \right) } \times S^{n-1}$, where ${\left\vert{t_0}\right\vert} \ge 1$.

Fix $t\in (-\infty, \ t_0)$ and set $N=\lceil\tau \varphi^\prime(t)\rceil$.
Recall that $\varphi(t)=t+\log t^2$.
An application Taylor's theorem (on a dyadic decomposition of $\frac s t$) shows that for all $s, t \in (-\infty, \ t_0)$
\begin{align}
\varphi(s)-\varphi(t)
&= \varphi'(t)(s-t)+\frac{1}{2}\varphi''(s_0)(s-t)^2
= \varphi'(t)(s-t)-\frac{1}{(s_0)^2}(s-t)^2,
\label{Taylor}
\end{align}
where $s_0$ is some number between $s$ and $t$.
If $s>t$, then
$$S_k(s, t) = e^{k(t-s) + \tau{\left[{{\varphi}{\left( {s - {\varphi}{\left( {t} \right) }} \right) }}\right]}} \leq e^{-(k-\tau\varphi'(t))(s-t)-\frac{\tau}{t^2}(s-t)^2},$$
so that
\begin{equation}
H(s-t) S_k(s, t)\leq e^{-|k- N 
||s-t|-\frac{\tau}{t^2}(s-t)^2}.
\label{case1}
\end{equation}
First we consider the case $N\leq k\leq M$.
From \eqref{star}, we sum over $k$ to get
\begin{equation}
\| \sum^M_{k=N} P_k u(t, \cdot)\|_{L^2(S^{n-1})}
\leq \int_{-\infty}^{\infty} \| \sum^M_{k=N} H(s-t)S_k(s,t) P_k L^-_\tau u(s, \cdot) \|_{L^2(S^{n-1})}\, ds.
\label{back}
\end{equation}
With $c_k= H(s-t)S_k(s,t)$, it is clear that $|c_k|\leq 1$.
Therefore, Lemma \ref{upDown} is applicable, so we may apply estimate \eqref{haha} to obtain
\begin{eqnarray}
&&\|\sum^M_{k=N}H(s-t)S_k(s,t) P_k L^-_\tau u(s, \cdot)\|_{L^2(S^{n-1})}\leq \nonumber \medskip\\
&& C {\left[{ \tau^{\frac{n-2}{{2}}}{\left( {\sum^M_{k=N}
H(s-t)|S_k(s,t)|^2} \right) }^{\frac{n}{2}}}\right]}^{\frac{1}{p}-\frac{1}{2}}
 \|L^-_\tau u(s, \cdot) \|_{L^p(S^{n-1})}
\label{mixSum}
\end{eqnarray}
for all $\frac{2n}{n+2}< p < 2$. Now we use the inequality
\eqref{case1} to bound ${\displaystyle} \sum^M_{k=N} H(s-t)|S_k(s,t)|^2$.
\begin{align}
\sum^M_{k=N} H(s-t)|S_k(s,t)|^2
&\leq {\left( {\sum^M_{k=N+1} e^{-2|k- N ||s-t|}+1} \right) }e^{ -\frac{2\tau}{t^2}(s-t)^2}
\nonumber \\
&\leq  C {\left( { \frac{1}{|s-t|}+1} \right) } e^{ -\frac{\tau}{t^2}(s-t)^2} .
\label{sumBnd}
\end{align}
Therefore, from \eqref{mixSum},
\begin{eqnarray*}
&&\|\sum^M_{k=N}H(s-t)S_k(s,t) P_k L^-_\tau u(s,
\cdot)\|_{L^2(S^{n-1})} \nonumber \medskip\\ &&
\leq  C \tau^{\alpha_1} (|s-t|^{-\alpha_2}+1)e^{-\frac{\alpha_2\tau}{t^2}(s-t)^2}\| L^-_\tau u(s, \cdot) \|_{L^p(S^{n-1})},
\end{eqnarray*}
where $\alpha_1=\frac{(n-2)(2-p)}{4p}$ and $\alpha_2=\frac{n(2-p)}{4p}$.
We see that
$$ e^{-\frac{\alpha_2\tau}{t^2}(s-t)^2}\leq C_j {\left( {1+\frac{\alpha_2\tau}{t^2}(s-t)^2} \right) }^{-j}   $$
for all $j\geq 0$ so that with $j=\frac{1}{2}$, we have
\begin{equation}
e^{-\frac{\alpha_2\tau}{t^2}(s-t)^2}\leq C|t|{\left( {1+\sqrt{\tau}|s-t|} \right) }^{-1}.
\label{expBnd}
\end{equation}
Thus,
$$ \|\sum^M_{k=N}H(s-t)S_k(s,t) P_k L^-_\tau u(s,\cdot)\|_{L^2(S^{n-1})}
\leq \frac{C\tau^{\alpha_1}|t|(|s-t|^{-\alpha_2}+1)\| L^-_\tau u(s,
\cdot) \|_{L^p(S^{n-1})}}{1+\sqrt{\tau}|s-t|}.
$$
It follows 
from \eqref{back} that
\begin{equation}
|t|^{-1}\| \sum^M_{k=N} P_k u(t, \cdot)\|_{L^2(S^{n-1})} \leq C
\tau^{\alpha_1} \int_{-\infty}^{\infty}\frac{(|s-t|^{-\alpha_2}+1)\|
L^-_\tau u(s, \cdot) \|_{L^p(S^{n-1})}}{1+\sqrt{\tau}|s-t|}.
\label{mile}
\end{equation}

For $k\leq N-1$, we solve the first order differential equation
\eqref{sord} as
\begin{equation}
P_k u(t, \omega)=\int_{-\infty}^{\infty} H(t-s)S_k(s, t) P_k L^-_\tau u{\left( {s, {\omega}} \right) }\, ds
\label{star1}.
\end{equation}
It follows from \eqref{Taylor} that for any $s, t$
\begin{equation}
H(t-s) S_k(s, t)\leq e^{-|N- 1 - k  
||s-t|-\frac{\tau}{s^2}(t-s)^2}.
\label{case2}
\end{equation}
Arguing as before, we use the upper bound for $H(t-s)S_k(s,t)$ in \eqref{case2} 
to similarly conclude that
\begin{equation}
\| \sum_{k=0}^{N-1} P_k u(t, \cdot)\|_{L^2(S^{n-1})} \leq
C\tau^{\alpha_1} \int_{-\infty}^{\infty}
\frac{|s|(|s-t|^{-\alpha_2}+1)\| L^-_\tau u(s, \cdot)
\|_{L^p(S^{n-1})}}{1+\sqrt{\tau}|s-t|}.
\label{mile1}
\end{equation}
Since $s,t $ are in $(-\infty, \ t_0)$ with $|t_0|$ large enough, we combine estimates \eqref{mile} and \eqref{mile1} to arrive at
\begin{equation*}
|t|^{-1}\| P_\tau^- u(t, \cdot)\|_{L^2(S^{n-1})} \leq C
\tau^{\alpha_1} \int_{-\infty}^{\infty}
\frac{|s|(|s-t|^{-\alpha_2}+1)\| L^-_\tau u(s, \cdot)
\|_{L^p(S^{n-1})}}{1+\sqrt{\tau}|s-t|}.
\end{equation*}
Applying Young's inequality for convolution, we get
\begin{equation*}
\| t^{-1} P_\tau^- u (t, \cdot) \|_{L^2(dtd\omega)}
\leq C\tau^{\alpha_1} {\left[{\int_{-\infty}^{\infty} {\left( {\frac{ |z|^{-\alpha_2}+1} {1+\sqrt{\tau}|z|}} \right) }^{\sigma} \, dz}\right]}^{\frac{1}{\sigma}} \| tL^-_\tau u \|_{L^p(dtd\omega)},
\end{equation*}
where $\frac{1}{\sigma}=\frac{3}{2}-\frac{1}{p}$.
Since ${\alpha}_2 \in {\left( {0, \frac 1 2} \right) }$ and ${\sigma} \in {\left( {1, 1 + \frac 1 {n-1}} \right) }$ for our
range of $p$, then a direct calculation shows that
$${\left[{\int_{-\infty}^{\infty} {\left( {\frac{ |z|^{-\alpha_2}+1} {1+\sqrt{\tau}|z|}} \right) }^{\sigma} \, dz}\right]}^{\frac{1}{\sigma}} \leq C\tau^{-\frac{1}{2{\sigma}}+\frac{\alpha_2}{2}}.
$$
 Therefore,
\begin{equation*}
\| t^{-1} P_\tau^- u (t, \cdot) \|_{L^2(dtd\omega)}\leq
C\tau^{-\frac{1}{2{\sigma}}+\frac{\alpha_2}{2}+\alpha_1} \| t L^-_\tau
u \|_{L^p(dtd\omega)}.
\end{equation*}
Since
$-\frac{1}{2{\sigma}}+\frac{\alpha_2}{2}+\alpha_1=\beta$,
this completes \eqref{key2}, and the lemma is proved.
\end{proof}

We now have all of the ingredients needed to prove the general $L^p
- L^q$ Carleman estimate for the Laplace operator given in Theorem
\ref{Carlpq}. The lemmas that we have established within this
section were adapted from the ideas in \cite{Reg99} and
\cite{Jer86}. Note that in \cite{Reg99}, a version of the following
theorem is proved for $p=\frac{14n-4}{7n+10}$ and $q = 2$. Similar
$L^p - L^2$ Carleman estimates have been shown in  \cite{BKRS88}.

\begin{proof}[Proof of Theorem \ref{Carlpq}]
Let $u\in C^{\infty}_{0}{\left( {B_{R_0}(x_0)\backslash{\left\{{x_0}\right\}} } \right) }$.
We first consider the case of $p \in {\left( {\frac{2n}{n-2}, 2} \right) }$.
An application of \eqref{cond} from Lemma \ref{Car22} applied to $v$, followed by an
application of \eqref{key-} from Lemma \ref{CarL-p} applied to $L^+v$ shows that
\begin{align*}
\tau {\left\vert\left\vert {t^{-1} e^{-\tau \varphi(t)}v}\right\vert\right\vert}_{L^2(dtd\omega )}
&+{\left\vert\left\vert {t^{-1}e^{-\tau \varphi(t)} \partial_t v}\right\vert\right\vert}_{L^2(dtd\omega )}
+\sum_{j=1}^n {\left\vert\left\vert {t^{-1}e^{-\tau \varphi(t)} \Omega_j v }\right\vert\right\vert}_{L^2(dtd\omega )}   \\
&\leq C{\left\vert\left\vert {t^{-1} e^{-\tau \varphi(t)} L^+v}\right\vert\right\vert}_{L^2(dtd\omega )} \\
&\le C\tau^{-\frac 1 2 + \frac{{\left( {3n-2} \right) }{\left( {2-p} \right) }}{8p}}{\left\vert\left\vert {t
e^{-\tau \varphi(t)} L^- L^+v}\right\vert\right\vert}_{L^p(dtd\omega )}.
\end{align*}
Now consider when $p = 2$.
Lemma \ref{Car22} combined with \eqref{key-q22} from Lemma \ref{CarL-qq} implies that
\begin{align*}
\tau {\left\vert\left\vert {t^{-1} e^{-\tau \varphi(t)}v}\right\vert\right\vert}_{L^2(dtd\omega )}
&+{\left\vert\left\vert {t^{-1}e^{-\tau \varphi(t)} \partial_t v}\right\vert\right\vert}_{L^2(dtd\omega )}
+\sum_{j=1}^n {\left\vert\left\vert {t^{-1}e^{-\tau \varphi(t)} \Omega_j v }\right\vert\right\vert}_{L^2(dtd\omega )}   \\
&\leq C{\left\vert\left\vert {t^{-1} e^{-\tau \varphi(t)} L^+v}\right\vert\right\vert}_{L^2(dtd\omega )} \\
&\le C\tau^{-\frac 1 2}{\left\vert\left\vert { e^{-\tau \varphi(t)} L^-
L^+v}\right\vert\right\vert}_{L^2(dtd\omega )} \\ &\le C\tau^{-\frac 1 2}{\left\vert\left\vert {t e^{-\tau
\varphi(t)} L^- L^+v}\right\vert\right\vert}_{L^2(dtd\omega )}.
\end{align*}
From the last two inequalities, recalling the definitions of $t$, ${\varphi}$, and $L^\pm$, we get that for any $p \in {\left( {\frac{2n}{n+2}, 2} \right] }$,
\begin{align}
&\tau \|(\log r)^{-1} e^{-\tau \phi(r)}u\|_{L^2(r^{-n}dx)}
+ \|(\log r )^{-1} e^{-\tau \phi(r)}r \nabla u\|_{L^2(r^{-n}dx)}
\nonumber \medskip\\
&\leq  C \tau^{\beta}  \|(\log r ) e^{-\tau \phi(r)} r^2 \triangle u\|_{L^p(r^{-n}dx)},
\label{dodo}
\end{align}
where we have set ${\beta} = - \frac 1 2 + \frac{{\left( {3n-2} \right) }{\left( {2-p} \right) }}{8p}$.

Notice that by the triangle inequality
\begin{align}
\|\nabla [(\log r)^{-1} e^{-\tau \phi(r)}  u r^\frac{-n+2}{2}]\|_{L^2}
&\leq \| (\log r)^{-1} e^{-\tau \phi(r)} u
\|_{L^2(r^{-n}dx)}+ \| (\log r)^{-2} e^{-\tau \phi(r)} u
\|_{L^2(r^{-n}dx)}  \nonumber \medskip\\
&+\tau  \| (\log r)^{-1} e^{-\tau \phi(r)}\phi'(r) r u
\|_{L^2(r^{-n}dx)} \nonumber \medskip\\ &+\frac{n}{2}\|(\log r)^{-1}
e^{-\tau \phi(r)}  u \|_{L^2(r^{-n}dx)}
+\|(\log r)^{-1} e^{-\tau \phi(r)}  r \nabla u \|_{L^2(r^{-n}dx)} \nonumber \medskip \\
&\leq C\tau \| (\log r)^{-1} e^{-\tau \phi(r)} u \|_{L^2(r^{-n}dx)}
\nonumber \medskip\\  &+C \| (\log r)^{-1} e^{-\tau \phi(r)} r
\nabla u \|_{L^2(r^{-n}dx)}, \label{dodo1}
\end{align}
where we have used that $\phi'(r)=\frac{1}{r}+\frac{2}{r\log r} \le \frac 1 r$ since $r \le R_0 \le 1$.
By the Sobolev embedding theorem,
\begin{align}
 \|(\log r)^{-1} e^{-\tau \phi(r)} u\|_{L^\frac{2n}{n-2}(r^{-n}dx)}
&= \ \|(\log r)^{-1} e^{-\tau \phi(r)} u r^\frac{-n+2}{2}\|_{L^\frac{2n}{n-2}{\left( {B_{R_0}} \right) }} \nonumber \\
&\le c \|\nabla [(\log r)^{-1} e^{-\tau \phi(r)}  u r^\frac{-n+2}{2}]\|_{L^2{\left( {B_{R_0}} \right) }} \nonumber \\
&\le C\tau \| (\log r)^{-1} e^{-\tau \phi(r)} u \|_{L^2(r^{-n}dx)}
\nonumber \\ &+C \| (\log r)^{-1} e^{-\tau \phi(r)} r \nabla u \|_{L^2(r^{-n}dx)}  \nonumber \\
&\leq  C \tau^{\beta} \|(\log r ) e^{-\tau \phi(r)} r^2 \triangle
u\|_{L^p(r^{-n} dx)}, \label{L2*Est}
\end{align}
where the last two inequalities are due to \eqref{dodo1} and \eqref{dodo}, respectively.
From (\ref{dodo}), it is clear that
\begin{equation}
\|(\log r)^{-1} e^{-\tau \phi(r)}u\|_{L^2(r^{-n}dx)}
\leq  C \tau^{{\beta} -1} \|(\log r ) e^{-\tau \phi(r)} r^2 \triangle u\|_{L^p(r^{-n} dx)}.
\label{L2Est}
\end{equation}
We are going to do a interpolation with the last two inequalities.
Choose ${\lambda} \in {\left( {0,1} \right) }$ so that $q = 2 {\lambda} + {\left( {1-{\lambda}} \right) } \frac{2n}{n-2}$.
By H\"older's inequality,
\begin{align*}
\|(\log r)^{-1} e^{-\tau \phi(r)}u\|_{L^q(r^{-n}dx)}
&\le \|(\log r)^{-1} e^{-\tau \phi(r)}u\|_{L^2(r^{-n}dx)}^{\frac{2{\lambda}}{q}} \|(\log r)^{-1} e^{-\tau \phi(r)}u\|_{L^{\frac{2n}{n-2}}(r^{-n}dx)}^{\frac{2n{\left( {1-{\lambda}} \right) }}{{\left( {n-2} \right) }q}}.
\end{align*}
Since ${\lambda}  = \frac{2q - n{\left( {q - 2} \right) } }{4}$, if we set ${\theta} = \frac{2{\lambda}}{q} = \frac{2q  - n{\left( {q -2} \right) }}{2q}$, then $1 - {\theta} = \frac{n{\left( {q-2} \right) }}{2q} = \frac{2n{\left( {1-{\lambda}} \right) }}{{\left( {n-2} \right) }q}$ and we have that $0\leq \theta\leq 1$.
Therefore,
\begin{align*}
& \|(\log r)^{-1} e^{-\tau \phi(r)}u\|_{L^q(r^{-n}dx)} \\
&\leq  \|(\log r)^{-1} e^{-\tau \phi(r)}u\|_{L^{2}(r^{-n}dx)}^{\theta} \|(\log
r)^{-1} e^{-\tau\phi(r)}u\|_{L^\frac{2n}{n-2}(r^{-n}dx)}^{1-\theta} \\
&\le {\left[{C\tau^{\beta-1} \|(\log r ) e^{-\tau \phi(r)} r^2 \triangle u\|_{L^p(r^{-n} dx)}}\right]}^{\theta}
{\left[{C \tau^\beta  \|(\log r ) e^{-\tau \phi(r)} r^2 \triangle u\|_{L^p(r^{-n} dx)}}\right]}^{1 - {\theta}} \\
&= C \tau^{\beta-{\theta}} \|(\log r ) e^{-\tau \phi(r)} r^2 \triangle u\|_{L^p(r^{-n} dx)},
\end{align*}
where the last inequality follows from \eqref{L2Est} and \eqref{L2*Est}.
That is, for any $2\leq q\leq \frac{2n}{n-2}$,
\begin{equation*}
\tau^{\frac 3 2 - \frac{{\left( {3n-2} \right) }{\left( {2-p} \right) }}{8p} - \frac{n{\left( {q -2} \right) }}{2q} }\|(\log r)^{-1} e^{-\tau
\phi(r)}u\|_{L^q(r^{-n}dx)} \leq C \|(\log r ) e^{-\tau \phi(r)} r^2
\triangle u\|_{L^p(r^{-n} dx)}.
\end{equation*}
Since \eqref{dodo} implies that
$$\tau^{\frac 1 2 - \frac{{\left( {3n-2} \right) }{\left( {2-p} \right) }}{8p}}   \|(\log r )^{-1} e^{-\tau \phi(r)}r \nabla u\|_{L^2(r^{-n}dx)} \leq  C \|(\log r ) e^{-\tau \phi(r)} r^2 \triangle u\|_{L^p(r^{-n}dx)},$$
adding the previous two inequalities gives the proof of Theorem \ref{Carlpq}.
\end{proof}

We prove a quantitative  Caccioppoli inequality for the second order elliptic equation (\ref{goal}) with singular lower order terms.
This Caccioppoli inequality is known, but since we want to show how the estimate depends on the norms of $W$ and $V$, we present the details of the proof.

\begin{lemma}
Assume that for some $s \in {\left( {n, {\infty}} \right] }$ and $t \in {\left( { \frac n 2, {\infty}} \right] }$, ${\left\vert\left\vert {W}\right\vert\right\vert}_{L^s{\left( {B_{R}} \right) }} \le K$ and ${\left\vert\left\vert {V}\right\vert\right\vert}_{L^t{\left( {B_{R}} \right) }} \le M$.
Let $u$ be a solution to \eqref{goal} in $B_R$.
Then there exists a constant $C$, independent of $W$ and $V$, such that
\begin{equation}
\|\nabla u\|^2_{L^2(B_r)}\leq
C{\left[{\frac{1}{(R-r)^2}+K^{\frac{2s}{s-n}}+M^{\frac{2t}{2t-n}}}\right]}\|
u\|^2_{L^2(B_R)}
\end{equation}
for any $r<R$.
\label{CaccLem}
\end{lemma}

\begin{proof}
We need to decompose $W$ and $V$.
Let
$$ W(x)=\overline{W}_{K_0}+W_{K_0},   \quad  V(x)=\overline{V}_{M_0}+V_{M_0},        $$
where $${\overline{W}}_{K_0}=W(x)\chi_{{\left\{{|W(x)|\leq \sqrt{K_0}}\right\}}}, \quad
{W}_{K_0}=W(x)\chi_{{\left\{{|W(x)|>\sqrt{K_0}}\right\}}},  $$ and
$${\overline{V}}_{M_0}=V(x)\chi_{{\left\{{|V(x)|\leq \sqrt{M_0}}\right\}}}, \quad
{V}_{M_0}=V(x)\chi_{{\left\{{|V(x)|>\sqrt{M_0}}\right\}}}, \quad  $$
for some $K_0, M_0$ to be determined.

For any $q$ with $1\leq q\leq s$, we have that
\begin{equation}
\|W_{K_0}\|_{L^q}\leq
K_0^{-\frac{s-q}{2q}}\|W_{K_0}\|_{L^{s}}^{\frac{s}{q}}\leq
K_0^{-\frac{s-q}{2q}}\|W\|_{L^{s}}^{\frac{s}{q}} \leq
K_0^{-\frac{s-q}{2q}}K^{\frac{s}{q}}.
\label{mmo}
\end{equation}
Similarly, for any $q$ with $1\leq q\leq t$, it holds
that
\begin{equation}
\|V_{M_0}\|_{L^q}\leq
M_0^{-\frac{t-q}{2q}}\|V_{M_0}\|_{L^{t}}^{\frac{t}{q}}\leq
M_0^{-\frac{t-q}{2q}}\|V\|_{L^{t}}^{\frac{t}{q}} \leq
M_0^{-\frac{t-q}{2q}}M^{\frac{t}{q}}.
\label{who}
\end{equation}

Let $\eta$ be a smooth cut-off function such that $\eta(x) \equiv 1$ in $B_r$,
$\eta(x) \equiv 0$ outside $B_R$ with $B_R\subset  B_1$. Then $|\nabla \eta|\leq \frac{C}{|R-r|}$.
Multiplying both sides of equation (\ref{goal}) by $\eta^2 u$ and integrating by parts, we obtain
\begin{equation}
\int V\eta^2 u^2 \, dx+\int W\cdot \nabla u \, \eta^2 u \, dx-2\int \nabla u\cdot\nabla \eta \, \eta \, u \, dx
=\int |\nabla u|^2 \eta^2 \,
dx.
\label{cacc}
\end{equation}
We estimate the terms on the left side of \eqref{cacc}. To
control $\int V\eta^2 u^2 \, dx$, we have
\begin{equation*}
\int V \eta^2 u^2\, dx \leq \int |{\overline{V}_{M_0}}|\eta^2 u^2 \,
dx+ \int |{{V}_{M_0}}|\eta^2 u^2 \, dx.
\end{equation*}
It is clear that
\begin{equation}
\int |{\overline{V}_{M_0}}|\eta^2 u^2 \, dx\leq
M_0^{\frac{1}{2}}\int \eta^2 u^2 \, dx.
\label{sss}
\end{equation}
By H\"older's inequality, \eqref{who} with $q=\frac{n}{2}$, and Sobolev imbedding, we get
\begin{eqnarray}
{\left\vert{\int {V}_{M_0}\eta^2 u^2 \, dx }\right\vert}
&\leq&{\left( {\int|{{V}_{M_0}}|^{\frac{n}{2}}\, dx} \right) }^{\frac{2}{n}}
{\left( {\int {\left\vert{\eta^2u^2}\right\vert}^{\frac{n}{n-2}}\, dx} \right) }^{\frac{n-2}{n}} \nonumber \medskip\\
&\leq & C {M_0}^{-\frac{2t-n}{2n}}M^{\frac{2t}{n}}\int |\nabla
(\eta u)|^2 \, dx. \label{mmm}
\end{eqnarray}
Taking ${M_0}^{-\frac{2t-n}{2n}}M^{\frac{2t}{n}}=c$ for some
small constant, i.e
$M_0=(c^{-1}K^{\frac{2t}{n}})^{\frac{2n}{2t-n}}$, from
(\ref{sss}) and (\ref{mmm}), we get
\begin{eqnarray}
\int V \eta^2 u^2\, dx & \leq & C M^{\frac{2t}{2t-n}}\int
|\eta u|^2 \, dx+\frac{1}{16}\int |\nabla (\eta u)|^2 \, dx
\nonumber
\medskip \\
& \leq &C M^{\frac{2t}{2t-n}}\int |\eta u|^2 \,
dx+\frac{1}{8}\int |\nabla \eta|^2 u^2 \, dx+\frac{1}{8}\int |\nabla
u|^2 \eta^2 \, dx.
\label{more1}
\end{eqnarray}

We estimate the second term in the lefthand side of (\ref{cacc}).
It is true that
\begin{equation}
\int W \cdot \nabla u \eta^2 u\, dx = \int |{\overline{W}_{K_0}}|
|\nabla u| \eta^2 u\, dx+ \int |{W}_{K_0}| | \nabla u| \eta^2 u \, dx.
\label{w1}
\end{equation}
By Young's inequality, we have
\begin{equation}
\int |{\overline{W}_{K_0}}|  |\nabla u| \eta^2 u\, dx\leq \frac{1}{16}
\int |\nabla u|^2\eta^2\,dx +CK_0 \int \eta^2 u^2 \, dx.
\label{w2}
\end{equation}
By H\"older's inequality, \eqref{mmo} with $q=n$, and Sobolev imbedding, we get
\begin{eqnarray}
\int {W}_{K_0} \cdot \nabla u \eta^2 u \, dx.
&\leq & {\left( {\int |{{W}_{K_0}}|^{{n}}\, dx} \right) }^{\frac{1}{n}} {\left( {\int |\nabla u\cdot \eta
|^{2}\, dx} \right) }^{\frac{1}{2}} {\left( {\int |u \eta |^{\frac{2n}{n-2}}\,
dx} \right) }^{\frac{n-2}{2n}}
\nonumber \medskip \\
& \leq & K_0^{-\frac{s-n}{2n}}K^{\frac{s}{n}}\||\nabla
u|\eta\|_{L^2}
\|\nabla(\eta u)\|_{L^2} \nonumber \medskip \\
&\leq & K_0^{-\frac{s-n}{2n}}K^{\frac{s}{n}}\||\nabla
u|\eta|\|^2_{L^2}+K_0^{-\frac{s-n}{2n}}K^{\frac{s}{n}}
\|\nabla(\eta u)\|^2_{L^2}.
\label{w3}
\end{eqnarray}
We choose $K_0^{-\frac{s-n}{2n}}K^{\frac{s}{n}}=c$ for some
small number $c$, that is,
$K_0=c^{\frac{-2n}{s-n}}K^{\frac{2s}{s-n}}$. Combining
(\ref{w1}), (\ref{w2}) and (\ref{w3}) gives that
\begin{eqnarray}
\int W \cdot \nabla u \, \eta^2 u\, dx &\leq& \frac{1}{8} \||\nabla u|
\eta\|^2_{L^2}+CK^{\frac{2s}{s-n}}
\|u\eta\|^2_{L^2}+\frac{1}{16}\|\nabla (\eta
u)\|^2_{L^2} \nonumber \\
&\leq & \frac{3}{8}\||\nabla u|
\eta\|^2_{L^2}+CK^{\frac{2s}{s-n}}
\|u\eta\|^2_{L^2}+\frac{1}{2}\||\nabla \eta| u \|^2_{L^2}.
\label{more2}
\end{eqnarray}
Together with (\ref{cacc}), (\ref{more1}) and (\ref{more2}), we
obtain
\begin{equation*}
\int |\nabla u|^2 \eta^2
\leq C {\left( {M^{\frac{2t}{2t-n}} +CK^{\frac{2s}{s-n}}} \right) } \int |u\eta|^2\,dx+ C \int
|\nabla \eta|^2u^2\,dx.
\end{equation*}
By the assumptions on $\eta$, we arrive at the conclusion in the
lemma.
\end{proof}

\section{Vanishing order}
\label{vanOrd}

Using the Carleman estimate in Theorem \ref{CarlpqVW}, we establish
a three-ball inequality that serves as the main tool in the proof of
Theorem \ref{thh}. We consider solutions to \eqref{goal} with first
order term $W$ and zeroth order term $V$. The arguments we present
are similar to those that appear in \cite{Ken07}.
By the translation invariance of the equations, the following three-ball inequality holds for balls centered at any $x_0 \in {\ensuremath{\mathbb{R}}}^n$ .
Without loss of generality, we assume that $x_0$ is the origin.

\begin{lemma}
Let $0 < r_0< r_1< R_1 < R_0$, where $R_0 < 1$ is sufficiently small.
Assume that for some $s \in {\left( { \frac{3n-2}{2}, {\infty}} \right] }$, $t \in {\left( { n{\left( {\frac{3n-2}{5n-2}} \right) }, {\infty}} \right] }$, $\|W\|_{L^s{\left( {B_{R_0}} \right) }} \le K$ and $\|V\|_{L^t{\left( {B_{R_0}} \right) }} \le M$.
Let $u$ be a solution to \eqref{goal} in $B_{R_0}$.
Then,
\begin{align}
\|u\|_{L^\infty {\left( {B_{3r_1/4}} \right) }} &\le C F{\left( {r_1} \right) }^{\frac n 2}  |\log
r_1| {\left[{ (K+|\log r_0|)F{\left( {r_0} \right) } \|u\|_{L^{\infty}(B_{2r_0})}}\right]}^{k_0} \nonumber \\
&\times {\left[{(K+|\log
R_1|)  F{\left( {R_1} \right) }\|u\|_{L^{\infty}(B_{R_1})}}\right]}^{1 - k_0} \nonumber \\
&+C F{\left( {r_1} \right) }^{\frac n 2} {\left( {\frac{R_1 }{r_1}} \right) }^{\frac n 2} {\left( {1 +\frac{|\log r_0|}{K}} \right) } \nonumber \\
&\times \exp{\left[{{\left( {1 + C_1 K^\kappa + C_2 M^\mu} \right) } {\left( {\phi{\left( {\frac{R_1}{2}} \right) }-\phi(r_0)} \right) }}\right]} \|u\|_{L^{\infty}(B_{2r_0})},
\label{three}
\end{align}
where ${\displaystyle} k_0 =
\frac{\phi(\frac{R_1}{2})-\phi(r_1)}{\phi(\frac{R_1}{2})-\phi(r_0)}$,
$F{\left( {r} \right) } = 1 + r K^{\frac{s}{s-n}} + r M^{\frac{t}{2t-n}}$, and
$\kappa$ and $\mu$ are as given in Theorem \ref{CarlpqVW}.
\end{lemma}

\begin{proof}
Let $r_0< r_1< R_1$.
Choose a smooth function $\eta\in C^\infty_{0}(B_{R_0})$ with $B_{2R_1}\subset B_{R_0}$.
We use the notation ${\left[{a,b}\right]}$ to denote a closed annulus with inner radius $a$ and outer radius $b$.
Let
$$D_1={\left[{\frac{3}{2}r_0, \frac{1}{2}R_1 }\right]}, \quad  \quad
D_2= {\left[{r_0, \frac{3}{2}r_0}\right]}, \quad \quad
D_3={\left[{\frac{1}{2}R_1, \frac{3 }{4}R_1}\right]}.$$ Let $\eta=1$ on $D_1$
and $\eta=0$ on $[0, \ r_0]\cup {\left[{\frac{3}{4}R_1, \ R_1}\right]}$. We
have $|\nabla \eta|\leq \frac{C}{r_0}$ and $|\nabla^2\eta|\leq
\frac{C}{r_0^2}$ on $D_2$. Similarly, $|\nabla \eta|\leq
\frac{C}{R_1}$ and $|\nabla^2 \eta|\leq\frac{C}{R_1^2}$ on $D_3$.

Since $u$ is a solution to \eqref{goal} in $B_{R_0}$, then, as per the discussion before the statement of Theorem \ref{thh}, $u \in L^{\infty}{\left( {B_{R_1}} \right) } \cap W^{1,2}{\left( {B_{R_1}} \right) } \cap W^{2,p}{\left( {B_{R_1}} \right) }$.
Therefore, by regularization, the estimate in Theorem \ref{CarlpqVW} holds for $\eta u$.
Substituting $\eta u$ into the Carleman estimates in Theorem \ref{CarlpqVW} and using that $u$ is a solution to equation \eqref{goal}, we get
\begin{align*}
\tau^{{\beta}_0} \|(\log r)^{-1} e^{-\tau \phi(r)} \eta u\|_{L^2(r^{-n}dx)}
&\leq  C_0 \|(\log r ) e^{-\tau \phi(r)} r^2{\left( { \triangle {\left( {\eta u} \right) } + W \cdot {\nabla}{\left( {\eta u} \right) } + V \eta u} \right) }\|_{L^p(r^{-n} dx)} \\
&=  C_0 \|(\log r ) e^{-\tau \phi(r)} r^2{\left( { \triangle \eta \, u + 2 {\nabla} \eta \cdot {\nabla} u + W \cdot {\nabla} \eta \, u } \right) }\|_{L^p(r^{-n} dx)},
\end{align*}
whenever
$$\tau \ge 1+ C_1 K^{\kappa} + C_2 M^{\mu}.$$

Then
\begin{equation}
\tau^{\beta_0} \|(\log r)^{-1} e^{-\tau \phi(r)} u\|_{L^2(D_1, r^{-n}dx )}\leq J,
\label{jjj}
\end{equation}
where
$$ J= C_0 \|(\log r) e^{-\tau \phi(r)} r^2 (\triangle \eta \, u+W\cdot \nabla \eta \, u + 2\nabla \eta \cdot \nabla u)\|_{L^p(D_2\cup D_3, r^{-n} dx)}.$$
An application of H\"older's inequality shows that
\begin{align*}
\|(\log r) e^{-\tau \phi(r)} r^2 \triangle \eta \, u\|_{L^p(D_2\cup D_3, r^{-n} dx)}
&\le \| {\left( {\log r} \right) } r^{2} {\Delta} \eta\|_{L^{\infty}{\left( {D_2} \right) }} \| r^{- \frac n p}\|_{L^{\frac{2p}{2-p}}{\left( {D_2} \right) }}\| e^{-\tau \phi(r)} u\|_{L^2(D_2)} \\
&+\| {\left( {\log r} \right) } r^{2} {\Delta} \eta\|_{L^{\infty}{\left( {D_3} \right) }} \| r^{- \frac n p}\|_{L^{\frac{2p}{2-p}}{\left( {D_3} \right) }}\| e^{-\tau \phi(r)} u\|_{L^2(D_3)}
\end{align*}
and
\begin{align*}
\|(\log r) e^{-\tau \phi(r)} r^2 {\nabla} \eta \cdot {\nabla} u\|_{L^p(D_2\cup D_3, r^{-n} dx)}
&\le \| {\left( {\log r} \right) } r^{2} {\nabla} \eta\|_{L^{\infty}{\left( {D_2} \right) }} \| r^{- \frac n p}\|_{L^{\frac{2p}{2-p}}{\left( {D_2} \right) }}\| e^{-\tau \phi(r)} {\nabla} u\|_{L^2(D_2)} \\
&+\| {\left( {\log r} \right) } r^{2} {\nabla} \eta\|_{L^{\infty}{\left( {D_3} \right) }} \| r^{- \frac n p}\|_{L^{\frac{2p}{2-p}}{\left( {D_3} \right) }}\| e^{-\tau \phi(r)} {\nabla} u\|_{L^2(D_3)}.
\end{align*}
As in the proof of Theorem \ref{CarlpqVW} (see the computations in \eqref{hod2}), since $\frac{2p}{2-p} \le s$, then
\begin{align*}
&\|(\log r) e^{-\tau\phi(r)} r^2 W\cdot\nabla \eta \, u\|_{L^p(r^{-n} dx, D_2 \cup D_3)} \\
&\le c \|W\|_{L^{s}{\left( {D_2} \right) }} {\left\vert\left\vert {{\nabla} \eta}\right\vert\right\vert}_{L^{\infty}{\left( {D_2} \right) }}\| e^{-\tau \phi(r)} r u \|_{L^2(D_2, r^{-n} dx)} \\
&+ c \|W\|_{L^{s}{\left( {D_3} \right) }} {\left\vert\left\vert {{\nabla} \eta}\right\vert\right\vert}_{L^{\infty}{\left( {D_3} \right) }}\| e^{-\tau \phi(r)} r u \|_{L^2(D_3, r^{-n} dx)} \\
&\le cK r_0^{-\frac{n}{2}}\|e^{-\tau \phi(r)} u\|_{{L^2(D_2)}}
+ cK R_1^{-\frac{n}{2}}\|e^{-\tau \phi(r)} u \|_{L^2(D_3)},
\end{align*}
where we have used the bounds on ${\left\vert{{\nabla} \eta}\right\vert}$.
By the bounds of $\eta$ in $D_2$ and $D_3$, we obtain
\begin{eqnarray*}
J &\leq & C |\log r_0| r_0^{-\frac{n}{2}}{\left( {\|e^{-\tau \phi(r)}
u\|_{{L^2(D_2)}}+ r_0 \|e^{-\tau \phi(r)} \nabla
u\|_{{L^2(D_2)}}} \right) }
+CK r_0^{-\frac{n}{2}}\|e^{-\tau \phi(r)}u\|_{{L^2(D_2)}} \nonumber \medskip \\
&+& C|\log R_1|R_1^{-\frac{n}{2}}{\left( {\|e^{-\tau \phi(r)} u\|_{{L^2(D_3)}}+ R_1
\|e^{-\tau \phi(r)} \nabla u\|_{{L^2(D_3)}}} \right) }
+CK R_1^{-\frac{n}{2}}\|e^{-\tau \phi(r)} u\|_{{L^2(D_3)}}.
\end{eqnarray*}
It follows that
\begin{eqnarray*}
J &\leq & C{\left( {K + |\log r_0|} \right) } r_0^{-\frac{n}{2}} e^{-\tau \phi(r_0)}  {\left( {\| u\|_{{L^2(D_2)}}+ r_0
\|\nabla u\|_{{L^2(D_2)}}} \right) } \nonumber \medskip \\
&+& C{\left( {K+ |\log R_1|} \right) } R_1^{-\frac{n}{2}}e^{-\tau \phi{\left( {\frac{R_1}{2}} \right) }} {\left( {\|
u\|_{{L^2(D_3)}}+ R_1 \| \nabla u\|_{{L^2(D_3)}}} \right) },
\end{eqnarray*}
where we have used that $e^{-\tau\phi(r)}$ is a decreasing function with respect to $r$.
By the Caccioppoli inequality in Lemma \ref{CaccLem},
\begin{equation*}
\|\nabla u\|_{L^2(D_2)}\leq
C{\left( {\frac{1}{r_0}+K^{\frac{s}{s-n}}+M^{\frac{t}{2t-n}}} \right) }\|
u\|_{L^2{\left( {B_{2r_0}\backslash B_{{r_0}/{2}}} \right) }}
\end{equation*}
and
\begin{equation*}
\|\nabla u\|_{L^2(D_3)}\leq
C{\left( {\frac{1}{R_1}+K^{\frac{s}{s-n}}+M^{\frac{t}{2t-n}}} \right) }\|
u\|_{L^2{\left( {B_{R_1}\backslash B_{{R_1}/{4}}} \right) }}.
\end{equation*}
Therefore,
\begin{eqnarray*}
J
&\leq& C{\left( {K+|\log r_0|} \right) } r_0^{-\frac{n}{2}}e^{-\tau
\phi(r_0)}{\left( {1+ r_0K^{\frac{s}{s-n}}+r_0M^{\frac{t}{2t-n}}} \right) }\|u\|_{L^2(B_{2r_0})}  \nonumber \medskip \\
&+& C{\left( {K+|\log R_1|} \right) } {R_1}^{-\frac{n}{2}}e^{-\tau
\phi{\left( {\frac{R_1}{2}} \right) }}{\left( {1+R_1K^{\frac{s}{s-n}}+R_1M^{\frac{t}{2t-n}}} \right) }
\|u\|_{L^2(B_{R_1})}.
\end{eqnarray*}
Set $D_4=\{r\in D_1, \ r\leq r_1\}$.
From \eqref{jjj} and that $\tau \ge 1$ and ${\beta}_0 > 0$, we have,
\begin{align*}
\| u\|_{L^2 (D_4)}
&\le \tau^{\beta_0}\| u\|_{L^2 (D_4)}
\le \tau^{\beta_0} \|e^{\tau\phi(r)}(\log r) r^{\frac{n}{2}}\|_{L^{\infty}{\left( {D_2} \right) }} \| (\log r)^{-1} e^{-\tau\phi(r)} u\|_{L^2 (D_4, r^{-n}dx)}  \\
&\le e^{\tau \phi(r_1)} |\log r_1| r_1^{\frac{n}{2}} J,
\end{align*}
where we have used that $e^{\tau\phi(r)}(\log r) r^{\frac{n}{2}}$ is increasing on $D_1$ for $R_0$ sufficiently small.
Adding $\|u\|_{L^2 {\left( {B_{3r_0/2}} \right) }}$ to both sides of the last inequality and using the bound on $J$ from above, we get
\begin{align*}
\| u\|_{L^2 (B_{r_1})}
&\le C |\log r_1|  {\left( {K+|\log r_0|} \right) } {\left( {\frac{r_1}{r_0}} \right) }^{\frac{n}{2}}e^{\tau {\left[{\phi(r_1)- \phi(r_0)}\right]}}{\left( {1+ r_0K^{\frac{s}{s-n}}+r_0M^{\frac{t}{2t-n}}} \right) }\|u\|_{L^2(B_{2r_0})} \\
&+ C |\log r_1| {\left( {K+|\log R_1|} \right) }
{\left( {\frac{r_1}{R_1}} \right) }^{\frac{n}{2}}e^{\tau{\left[{\phi(r_1) -
\phi{\left( {\frac{R_1}{2}} \right) }}\right]}}\nonumber \\
&\times{\left( {1+R_1K^{\frac{s}{s-n}}+R_1M^{\frac{t}{2t-n}}} \right) }
\|u\|_{L^2(B_{R_1})}.
\end{align*}
Let $U_1 =\|u\|_{L^2(B_{2r_0})}$, $U_2=\|u\|_{L^2(B_{R_1})}$ and define
\begin{align*}
A_1 &= C |\log r_1|  {\left( {K+|\log r_0|} \right) } {\left( {\frac{r_1}{r_0}} \right) }^{\frac{n}{2}} {\left( {1+ r_0K^{\frac{s}{s-n}}+r_0M^{\frac{t}{2t-n}}} \right) } \\
A_2 &= C |\log r_1| {\left( {K+|\log R_1|} \right) } {\left( {\frac{r_1}{R_1}} \right) }^{\frac{n}{2}} {\left( {1+R_1K^{\frac{s}{s-n}}+R_1M^{\frac{t}{2t-n}}} \right) }.
\end{align*}
Then the previous inequality simplifies to
\begin{eqnarray}
\| u\|_{L^2 (B_{r_1})}  &\leq& A_1 {\left[{\frac{\exp {\left( {\phi(r_1)} \right) }}{\exp{\left( { \phi(r_0)} \right) }}}\right]}^\tau U_1 + A_2 {\left[{\frac{\exp{\left( { \phi(r_1)} \right) }}{\exp{\left( { \phi{\left( {\frac{R_1}{2}} \right) }} \right) }}}\right]}^\tau U_2.
\label{D4est}
\end{eqnarray}
Introduce
$$\frac{1}{k_0}=\frac{\phi(\frac{R_1}{2})-\phi(r_0)}{\phi(\frac{R_1}{2})-\phi(r_1)}.$$
Recall that $\phi(r)=\log r+\log (\log r)^2$.
If $r_1$ and $R_1$ are fixed, and $r_0\ll r_1$, i.e. $r_0$ is sufficiently small, then $\frac{1}{k_0}\simeq \log \frac{1}{r_0}$.
Let
$$\tau_1 =\frac{k_0}{\phi{\left( {\frac{R_1}{2}} \right) }-\phi(r_1)}\log{\left( {\frac{A_2{U}_2}{A_1 {U}_1}} \right) }.$$
If $\tau_1 \ge 1 + C_1 K^\kappa + C_2 M^\mu$, then the above calculations are valid with $\tau = \tau_1$ and by substituting $\tau_1$ into \eqref{D4est}, we get
\begin{eqnarray}
\| u\|_{L^2 (B_{r_1})}  &\leq& 2{\left( {A_1 U_1} \right) }^{k_0}{\left( {A_2 U_2} \right) }^{1 - k_0}.
\label{mix1}
\end{eqnarray}
On the other hand, if $\tau_1 < 1 + C_1 K^\kappa + C_2 M^\mu$, then
\begin{align*}
U_2
< \frac{A_1}{A_2} \exp{\left[{{\left( {1 + C_1 K^\kappa + C_2 M^\mu} \right) } {\left( {\phi{\left( {\frac{R_1}{2}} \right) }-\phi(r_0)} \right) }}\right]} U_1.
\end{align*}
The last inequality implies that
\begin{equation}
\|u\|_{L^2 (B_{r_1})}
\le C {\left( {\frac{R_1}{r_0}} \right) }^{\frac n 2} {\left( {1 +\frac{|\log r_0|}{K}} \right) } e^{{\left( {1 + C_1 K^\kappa + C_2 M^\mu} \right) } {\left( {\phi{\left( {\frac{R_1}{2}} \right) }-\phi(r_0)} \right) }} \|u\|_{L^2(B_{2r_0})}.
\label{mix2}
\end{equation}
By combining \eqref{mix1} and \eqref{mix2}, we arrive at
\begin{align}
\| u\|_{L^2 (B_{r_1})}
&\le C  |\log r_1| r_1^{\frac n 2}{\left[{ r_0^{-\frac{n}{2}}{\left( {K+|\log r_0|} \right) } {\left( {1+ r_0K^{\frac{s}{s-n}}+r_0M^{\frac{t}{2t-n}}} \right) } \|u\|_{L^2(B_{2r_0})}}\right]}^{k_0} \nonumber \\
&\times {\left[{R_1^{-\frac{n}{2}}{\left( {K+|\log R_1|} \right) } {\left( {1+R_1K^{\frac{s}{s-n}}+R_1M^{\frac{t}{2t-n}}} \right) } \|u\|_{L^2(B_{R_1})}}\right]}^{1 - k_0} \nonumber \\
&+C {\left( {\frac{R_1}{r_0}} \right) }^{\frac n 2} {\left( {1 +\frac{|\log r_0|}{K}} \right) }
e^{{\left( {1 + C_1 K^\kappa + C_2 M^\mu} \right) }
{\left( {\phi{\left( {\frac{R_1}{2}} \right) }-\phi(r_0)} \right) }} \|u\|_{L^2(B_{2r_0})}.
 \label{end2}
\end{align}
By elliptic regularity (see for example \cite{HL11}, \cite{GT01}), we see that
\begin{align}
\|u\|_{L^\infty(B_r)}
\le C{\left( {1 + r K^{\frac{s}{s-n}} + r M^{\frac{t}{2t-n}}} \right) }^{\frac n 2}r^{- \frac n 2}\|u\|_{L^2(B_{2r})}.
\label{ell}
\end{align}
From \eqref{end2} and \eqref{ell}, we get the three-ball inequality in the $L^\infty$-norm that is given in \eqref{three}.
\end{proof}

The inequality \eqref{three} is the three-ball inequality we use in
the proof of Theorem \ref{thh}. We first use \eqref{three} in the
propagation of smallness argument to establish a lower bound for the
solution in $B_r$. Similar arguments have been performed in
\cite{Zhu16}. Then we use \eqref{three} again to establish the order
of vanishing estimate.

\begin{proof} [Proof of Theorem  \ref{thh}] Without loss of
generality, we may assume that $x_0$ is the origin.  With
$r_0=\frac{r}{2}$, $r_1=4r$ and $R_1=10r$, it follows from
\eqref{three} that
\begin{align}
\|u\|_{L^\infty {\left( {B_{3r}} \right) }}
&\le C {\left( {1 + K^{\frac{s}{s-n}} + M^{\frac{t}{2t-n}}} \right) }^{1+\frac n 2} {\left( {K+|\log r|} \right) } |\log r|  \|u\|_{L^{\infty}(B_{r})}^{k_0} \|u\|_{L^{\infty}(B_{10r})}^{1 - k_0} \nonumber \\
&+C{\left( {1 + \frac{{\left\vert{\log r}\right\vert}}{K}} \right) } \exp{\left[{ {\left( {1 + C_1 K^\kappa + C_2 M^\mu} \right) } {\left( {\phi{\left( {5r} \right) }-\phi{\left( {\frac r 2} \right) }} \right) } }\right]} \|u\|_{L^{\infty}(B_{r})},
\label{refi}
\end{align}
where ${\displaystyle} k_0 = \frac{\phi(5r)-\phi(4r)}{\phi(5r)-\phi{\left( {\frac r 2} \right) }}$.
We can check that
$$c\leq \phi(5r)-\phi{\left( {\frac{r}{2}} \right) }\leq C \quad \mbox{and} \quad c\leq \phi(5r)-\phi(4r)\leq C,$$
where $C$ and $c$ are positive constants that do not depend on $r$.
Therefore, $k_0$ is independent of $r$ in this case.

We choose a small $r < \frac 1 2$ such that
$$\sup_{B_r(0)}|u|={\varepsilon},$$
where ${\varepsilon}>0$.
Since \eqref{normal} implies ${\displaystyle} \sup_{|x|\leq 1}|u(x)|\geq 1$, there exists some $\bar x\in B_1$ such that ${\displaystyle} {\left\vert{u(\bar x)}\right\vert}=\sup_{|x|\leq 1}|u(x)|\geq 1$.
We select a sequence of balls, each with radius $r$, centered at $x_0=0, \ x_1, \ldots, x_d$ so that
$x_{i+1}\in B_{r}(x_i)$ for every $i$, and $\bar x\in B_{r}(x_d)$.
Note that the number of balls, $d$, depends on the radius $r$ which is to be fixed.
Employing the $L^\infty$-version of  three-ball inequality (\ref{refi}) at the origin and the boundedness assumption of $u$ given in \eqref{bound}, we get
\begin{align*}
\|u\|_{L^\infty {\left( {B_{3r}(0)} \right) }}
&\le C {\varepsilon}^{k_0} {\left( {1+\frac{|\log r|}{K}} \right) } {\left\vert{\log r}\right\vert}{\left( {1 + K^{\frac{s}{s-n}} + M^{\frac{t}{2t-n}}} \right) }^{1+\frac n 2}  K  \nonumber \\
&+ {\varepsilon} {\left( {1 + \frac{|\log r|}{K}} \right) } \exp{\left[{C{\left( {1 + C_1 K^\kappa + C_2 M^\mu} \right) } }\right]}.
\end{align*}
Since $B_r(x_{i+1})\subset B_{3r}(x_{i})$, then for every $i = 1, 2, \ldots, d$,
\begin{equation}
\|u\|_{L^\infty (B_r(x_{i+1}))}\leq  \|u\|_{L^\infty
(B_{3r}(x_{i}))}. \label{bbb}
\end{equation}
Repeating the above argument with balls centered at $x_i$ and using \eqref{bbb}, we obtain
\begin{equation*}
\|u\|_{L^\infty (B_{3r}(x_{i}))}
\leq C_i {\varepsilon}^{D_i} {\left( {1+\frac{|\log r|}{K}} \right) }^{E_i} |\log r|^{F_i}  \exp{\left[{H_i{\left( {1 + C_1 K^\kappa + C_2 M^\mu} \right) } }\right]}
\end{equation*}
for $i=0, 1, \cdots, d$, where $C_i$ is a constant depending on $d$ and $\hat C$, and $D_i$, $E_i$, $F_i$ $H_i$ are constants depending on $n$ and $d$.
By the fact that ${\left\vert{u(\bar x)}\right\vert} \geq 1$ and $\bar x \in B_{3r}(x_d)$, we obtain
\begin{equation*}
{\varepsilon}
\ge C \exp{\left[{-C {\left( {1 + C_1 K^\kappa + C_2 M^\mu} \right) } }\right]}{\left( {1+\frac{|\log r|}{K}} \right) }^{-C} |\log r|^{-C},
\end{equation*}
where $C$ depends on $d$, $n$, and $\hat C$.

Now we fix the radius $r$ as a small number so that $d$ is a fixed constant.
We are going to use the three-ball inequality again with a different set of radii.
Let $\frac{3}{4}r_1=r$, $R_1=10r$ and let $r_0 << r$, i.e. $r_0$ is sufficiently small with respect to $r$.
Then, by the three-ball inequality \eqref{three},
$$ {\varepsilon} \leq {\rm I} +\Pi,$$
where
\begin{align*}
{\rm I} &= C F{\left( {r} \right) }^{\frac n 2}  |\log r| {\left[{ (K+|\log r_0|)
F{\left( {r_0} \right) } \|u\|_{L^{\infty}(B_{2r_0})}}\right]}^{k_0} {\left[{ (K+|\log
10r|) F{\left( {10 r} \right) } \|u\|_{L^{\infty}(B_{10 r})}}\right]}^{1 - k_0} \\
\Pi &= C F{\left( {r} \right) }^{\frac n 2} {\left( {\frac{r }{r_0 }} \right) }^{\frac n 2} {\left( {1
+\frac{|\log r_0|}{K}} \right) } e^{{\left( {1 + C_1 K^\kappa + C_2 M^\mu} \right) }
{\left( {\phi{\left( {5r} \right) }-\phi(r_0)} \right) }} \|u\|_{L^{\infty}(B_{2r_0})},
\end{align*}
with ${\displaystyle} k_0 = \frac{\phi(5r)-\phi(\frac 4 3 r)}{\phi(5r)-\phi(r_0)}$ and $F{\left( {r} \right) } = 1 + r K^{\frac{s}{s-n}} + r M^{\frac{t}{2t-n}}$.

On one hand, if ${\rm I} \leq \Pi$, then
\begin{align*}
&C \exp{\left[{-C {\left( {1 + C_1 K^\kappa + C_2 M^\mu} \right) } }\right]}{\left( {1+\frac{|\log r|}{K}} \right) }^{-C} |\log r|^{-C}
\le {\varepsilon} \le 2 \Pi \\
&\le 2 C F{\left( {r} \right) }^{\frac n 2} {\left( {\frac{r }{r_0 }} \right) }^{\frac n 2} {\left( {1 +\frac{|\log r_0|}{K}} \right) } e^{{\left( {1 + C_1 K^\kappa + C_2 M^\mu} \right) } {\left( {\phi{\left( {5r} \right) }-\phi(r_0)} \right) }} \|u\|_{L^{\infty}(B_{2r_0})}.
\end{align*}
If $r_0 << r$ so that $\phi{\left( {r_0} \right) }-{\left( {C + \phi{\left( {5r} \right) }} \right) } \ge c \phi{\left( {r_0} \right) }$, then since $r$ is a fixed small positive constant, we obtain
\begin{align*}
\|u\|_{L^{\infty}(B_{2r_0})}
&\ge C r_0^{C {\left( {1 + C_1 K^\kappa + C_2 M^\mu} \right) } }.
\end{align*}
On the other hand, if $\Pi \leq {\rm I}$, then
\begin{align*}
&C \exp{\left[{-C {\left( {1 + C_1 K^\kappa + C_2 M^\mu} \right) } }\right]}{\left( {1+\frac{|\log r|}{K}} \right) }^{-C} |\log r|^{-C}
\le {\varepsilon} \le 2 {\rm I} \\
&\le 2 C F{\left( {r} \right) }^{\frac n 2} |\log r| {\left[{(K+|\log r_0|) F{\left( {r_0} \right) }
\|u\|_{L^{\infty}(B_{2r_0})}}\right]}^{k_0} {\left[{(K+|\log 10r|)  F{\left( {10 r} \right) }
\|u\|_{L^{\infty}(B_{10 r})}}\right]}^{1 - k_0},
\end{align*}
or, raising both sides to $\frac{1}{k_0}$ and using that
$\|u\|_{L^\infty{\left( {B_{10r}} \right) }}\leq \hat{C}$ from \eqref{bound}, we obtain
\begin{align*}
 \|u\|_{L^{\infty}(B_{2r_0})}
 &\ge C\frac{1}{|\log r_0|} {\left( {\frac{C}{\hat C {\left\vert{\log r}\right\vert}^C {\left( {K+|\log r|} \right) }^C F{\left( {1} \right) }^{1+ \frac n {2}} }} \right) }^{\frac 1 {k_0}} \exp{\left[{-\frac{C}{k_0} {\left( {1 + C_1 K^\kappa + C_2 M^\mu} \right) } }\right]}.
\end{align*}
Recalling that $\frac{1}{k_0}\simeq\log \frac{1}{r_0}$ if $ r_0\ll r$, since $r$ is a fixed small positive constant, we obtain
\begin{align*}
 \|u\|_{L^{\infty}(B_{2r_0})}
 &\ge C r_0^{C{\left( {1 + C_1 K^\kappa + C_2 M^\mu} \right) }},
\end{align*}
as before.
This completes the proof of Theorem \ref{thh}.
\end{proof}

To prove Theorem \ref{thhh}, we require another three-ball inequality.

\begin{lemma}
Let $0 < r_0< r_1< R_1 < R_0$, where $R_0 < 1$ is sufficiently small.
Assume that for some $t \in {\left( { \frac{4n^2}{7n+2}, {\infty}} \right] }$, $\|V\|_{L^t{\left( {B_{R_0}} \right) }} \le M$.
Let $u$ be a solution to \eqref{goal1} in $B_{R_0}$.
Then,
\begin{align}
\|u\|_{L^\infty {\left( {B_{3r_1/4}} \right) }}
&\le C |\log r_1| F{\left( {r_1} \right) }^{\frac n 2} {\left( {|\log r_0| F{\left( {r_0} \right) } \|u\|_{L^{\infty}(B_{2r_0})}} \right) }^{k_0} {\left( { |\log R_1|F{\left( {R_1} \right) } \|u\|_{L^{\infty}(B_{R_1})}} \right) }^{1 - k_0} \nonumber  \\
&+C F{\left( {r_1} \right) }^{\frac n 2}  {\left( {\frac{R_1}{r_1}} \right) }^{\frac n
2}\frac{{\left\vert{\log r_0}\right\vert}}{{\left\vert{\log R_1}\right\vert}} e^{{\left( {1 + C_2 M^\mu} \right) }
{\left( {\phi{\left( {\frac{R_1}{2}} \right) }-\phi(r_0)} \right) }} \|u\|_{L^{\infty}(B_{2r_0})},
\label{three2}
\end{align}
where ${\displaystyle} k_0 =
\frac{\phi(\frac{R_1}{2})-\phi(r_1)}{\phi(\frac{R_1}{2})-\phi(r_0)}$,
$F{\left( {r} \right) } = 1 + r M^{\frac{t}{2t-n}}$, and $\mu$ is as given in
Theorem \ref{CarlpqV}.
\end{lemma}

\begin{proof}
Let $r_0< r_1< R_1$.
Choose a smooth function $\eta\in C^\infty_{0}(B_{R_0})$ with $B_{2R_1}\subset B_{R_0}$.
As before, let
$$D_1={\left[{\frac{3}{2}r_0, \frac{1}{2}R_1 }\right]}, \quad  \quad
D_2= {\left[{r_0, \frac{3}{2}r_0}\right]}, \quad \quad
D_3={\left[{\frac{1}{2}R_1, \frac{3 }{4}R_1}\right]}.$$
Let $\eta=1$ on $D_1$ and $\eta=0$ on $[0, \ r_0]\cup {\left[{\frac{3}{4}R_1, \ R_1}\right]}$.
Then $|\nabla \eta|\leq \frac{C}{r_0}$ and $|\nabla^2\eta|\leq \frac{C}{r_0^2}$ on $D_2$, and $|\nabla \eta|\leq \frac{C}{R_1}$ and $|\nabla^2 \eta|\leq\frac{C}{R_1^2}$ on $D_3$.

Since $u$ is a solution to \eqref{goal1} in $B_{R_0}$, then $u \in L^{\infty}{\left( {B_{R_1}} \right) } \cap W^{1,2}{\left( {B_{R_1}} \right) } \cap W^{2,p}{\left( {B_{R_1}} \right) }$, so by regularization, the estimate in Theorem \ref{CarlpqV} holds for $\eta u$.
Using that $u$ is a solution to equation \eqref{goal1}, we get
\begin{align*}
\tau^{{\beta}_0} \|(\log r)^{-1} e^{-\tau \phi(r)} \eta u\|_{L^2(r^{-n}dx)}
&\leq  C_0 \|(\log r ) e^{-\tau \phi(r)} r^2{\left( { \triangle \eta \, u + 2 {\nabla} \eta \cdot {\nabla} u  } \right) }\|_{L^p(r^{-n} dx)},
\end{align*}
whenever
$$\tau \ge 1+ C_2 M^{\mu}.$$

Then
\begin{equation*}
\tau^{\beta_0} \|(\log r)^{-1} e^{-\tau \phi(r)} u\|_{L^2(D_1, r^{-n}dx )}\leq J,
\end{equation*}
where
$$ J= C_0 \|(\log r) e^{-\tau \phi(r)} r^2 (\triangle \eta \, u + 2\nabla \eta \cdot \nabla u)\|_{L^p(D_2\cup D_3, r^{-n} dx)}.$$
Set $D_4=\{r\in D_1, \ r\leq r_1\}$.
From the previous inequality and that $\tau \ge 1$ and ${\beta}_0 > 0$, we have, as before, that
\begin{align}
\| u\|_{L^2 (D_4)}
&\le e^{\tau \phi(r_1)} |\log r_1| r_1^{\frac{n}{2}} J.
\label{JBound}
\end{align}

As in the previous proof,
\begin{eqnarray*}
J &\leq & C |\log r_0| r_0^{-\frac{n}{2}} e^{-\tau \phi(r_0)}  {\left( {\| u\|_{{L^2(D_2)}}+ r_0
\|\nabla u\|_{{L^2(D_2)}}} \right) } \nonumber \medskip \\
&+& C|\log R_1| R_1^{-\frac{n}{2}}e^{-\tau \phi{\left( {\frac{R_1}{2}} \right) }} {\left( {\|
u\|_{{L^2(D_3)}}+ R_1 \| \nabla u\|_{{L^2(D_3)}}} \right) }.
\end{eqnarray*}
By the Caccioppoli inequality in Lemma \ref{CaccLem},
\begin{equation*}
\|\nabla u\|_{L^2(D_2)}\leq
C{\left( {\frac{1}{r_0}+M^{\frac{t}{2t-n}}} \right) }\|
u\|_{L^2{\left( {B_{2r_0}\backslash B_{{r_0}/{2}}} \right) }}
\end{equation*}
and
\begin{equation*}
\|\nabla u\|_{L^2(D_3)}\leq
C{\left( {\frac{1}{R_1}+M^{\frac{t}{2t-n}}} \right) }\|
u\|_{L^2{\left( {B_{R_1}\backslash B_{{R_1}/{4}}} \right) }}.
\end{equation*}
Therefore,
\begin{eqnarray*}
J
&\leq& C |\log r_0| r_0^{-\frac{n}{2}}e^{-\tau \phi(r_0)}{\left( {1+r_0M^{\frac{t}{2t-n}}} \right) }\|u\|_{L^2(B_{2r_0})}  \nonumber \medskip \\
&+& C |\log R_1| {R_1}^{-\frac{n}{2}}e^{-\tau \phi{\left( {\frac{R_1}{2}} \right) }}{\left( {1+R_1M^{\frac{t}{2t-n}}} \right) }
\|u\|_{L^2(B_{R_1})}.
\end{eqnarray*}
Adding $\|u\|_{L^2 {\left( {B_{3r_0/2}} \right) }}$ to both sides of \eqref{JBound}, we get
\begin{align*}
\| u\|_{L^2 (B_{r_1})}
&\le C |\log r_1|  |\log r_0| {\left( {\frac{r_1}{r_0}} \right) }^{\frac{n}{2}}e^{\tau {\left[{\phi(r_1)- \phi(r_0)}\right]}}{\left( {1+ r_0M^{\frac{t}{2t-n}}} \right) }\|u\|_{L^2(B_{2r_0})} \\
&+ C |\log r_1| |\log R_1| {\left( {\frac{r_1}{R_1}} \right) }^{\frac{n}{2}}e^{\tau{\left[{\phi(r_1) -
\phi{\left( {\frac{R_1}{2}} \right) }}\right]}}{\left( {1+R_1M^{\frac{t}{2t-n}}} \right) } \|u\|_{L^2(B_{R_1})}.
\end{align*}
Let $U_1 =\|u\|_{L^2(B_{2r_0})}$, $U_2=\|u\|_{L^2(B_{R_1})}$, and this time define
\begin{align*}
A_1 &= C |\log r_1|  |\log r_0| {\left( {\frac{r_1}{r_0}} \right) }^{\frac{n}{2}} {\left( {1+r_0M^{\frac{t}{2t-n}}} \right) } \\
A_2 &= C |\log r_1| |\log R_1| {\left( {\frac{r_1}{R_1}} \right) }^{\frac{n}{2}} {\left( {1+R_1M^{\frac{t}{2t-n}}} \right) }.
\end{align*}
Then the previous inequality simplifies to \eqref{D4est}.
With $k_0$ as before, i.e. $k_0=\frac{\phi(\frac{R_1}{2})-\phi(r_1)}{\phi(\frac{R_1}{2})-\phi(r_0)}$,
let
$$\tau_1 =\frac{k_0}{\phi{\left( {\frac{R_1}{2}} \right) }-\phi(r_1)}\log{\left( {\frac{A_2{U}_2}{A_1 {U}_1}} \right) }.$$
If $\tau_1 \ge 1 + C_2 M^\mu$, then the above calculations are valid with $\tau = \tau_1$ and by substituting $\tau_1$ into \eqref{D4est}, we get \eqref{mix1}.
On the other hand, if $\tau_1 < 1 + C_2 M^\mu$, then
\begin{align*}
U_2
< \frac{A_1}{A_2} \exp{\left[{{\left( {1 + C_2 M^\mu} \right) } {\left( {\phi{\left( {\frac{R_1}{2}} \right) }-\phi(r_0)} \right) }}\right]} U_1.
\end{align*}
The last inequality implies that
\begin{equation}
\|u\|_{L^2 (B_{r_1})}
\le C {\left( {\frac{R_1}{r_0}} \right) }^{\frac n 2}\frac{{\left\vert{\log r_0}\right\vert}}{{\left\vert{\log R_1}\right\vert}} e^{{\left( {1 + C_2 M^\mu} \right) } {\left( {\phi{\left( {\frac{R_1}{2}} \right) }-\phi(r_0)} \right) }} \|u\|_{L^2(B_{2r_0})}.
\label{mix22}
\end{equation}
By combining \eqref{mix1} and \eqref{mix22}, we arrive at
\begin{align}
\| u\|_{L^2 (B_{r_1})}
&\le C r_1^{\frac n 2} |\log r_1|  {\left( { r_0^{- \frac n 2} |\log r_0| {\left( {1+r_0M^{\frac{t}{2t-n}}} \right) } \|u\|_{L^2(B_{2r_0})}} \right) }^{k_0} \nonumber \\
&\times{\left( { R_1^{- \frac n 2}|\log R_1| {\left( {1+R_1M^{\frac{t}{2t-n}}} \right) } \|u\|_{L^2(B_{R_1})}} \right) }^{1 - k_0} \nonumber  \\
&+C {\left( {\frac{R_1}{r_0}} \right) }^{\frac n 2}\frac{{\left\vert{\log r_0}\right\vert}}{{\left\vert{\log R_1}\right\vert}} e^{{\left( {1 + C_2 M^\mu} \right) } {\left( {\phi{\left( {\frac{R_1}{2}} \right) }-\phi(r_0)} \right) }} \|u\|_{L^2(B_{2r_0})}.
 \label{end22}
\end{align}
By elliptic regularity
\begin{align}
\|u\|_{L^\infty(B_r)}
\le C{\left( {1 + r M^{\frac{t}{2t-n}}} \right) }^{\frac n 2}r^{- \frac n 2}\|u\|_{L^2(B_{2r})}.
\label{ell2}
\end{align}
From \eqref{end22} and \eqref{ell2}, we get the three-ball inequality in the $L^\infty$-norm that is given in \eqref{three2}.
\end{proof}

The proof of Theorem \ref{thhh} follows the arguments in the proof of Theorem \ref{thh}, except that we use \eqref{three2} in place of \eqref{three}.

\section{Unique continuation at infinity}
\label{QuantUC}

Using the scaling arguments established in \cite{BK05}, we show how the unique continuation estimates at infinity follow from the maximal order of vanishing estimates.

\begin{proof}[Proof of Theorem \ref{UCVW}]
Let $u$ be a solution to \eqref{goal} in ${\ensuremath{\mathbb{R}}}^n$.
Fix $x_0 \in {\ensuremath{\mathbb{R}}}^n$ and set ${\left\vert{x_0}\right\vert} = R$.
Let $u_R(x) = u(x_0 + Rx)$. 
Define $W_R{\left( {x} \right) } = R \, W{\left( {x_0 + R x} \right) }$ and $V_R{\left( {x} \right) } = R^2 V{\left( {x_0 + R x} \right) }$.
For any $r > 0$,
\begin{align*}
{\left\vert\left\vert {W_R}\right\vert\right\vert}_{L^s{\left( {B_r{\left( {0} \right) }} \right) }}
&= {\left( {\int_{B_r{\left( {0} \right) }} {\left\vert{W_R{\left( {x} \right) }}\right\vert}^s dx} \right) }^{\frac 1 s}
= {\left( {\int_{B_r{\left( {0} \right) }} {\left\vert{R \, W{\left( {x_0 + R x} \right) }}\right\vert}^s dx} \right) }^{\frac 1 s} \\
&= R^{1 - \frac n s } {\left( {  \int_{B_r{\left( {0} \right) }} {\left\vert{W{\left( {x_0 + R x} \right) }}\right\vert}^s d{\left( {Rx} \right) } } \right) }^{\frac 1 s}
= R^{1 - \frac n s } {\left\vert\left\vert {W}\right\vert\right\vert}_{L^s{\left( {B_{r R}{\left( {x_0} \right) }} \right) }}.
\end{align*}
and
\begin{align*}
{\left\vert\left\vert {V_R}\right\vert\right\vert}_{L^t{\left( {B_r{\left( {0} \right) }} \right) }}
&= R^{2 - \frac n t } {\left( {  \int_{B_r{\left( {0} \right) }} {\left\vert{V{\left( {x_0 + R x} \right) }}\right\vert}^t d{\left( {Rx} \right) } } \right) }^{\frac 1 t}
= R^{2 - \frac n t } {\left\vert\left\vert {V}\right\vert\right\vert}_{L^t{\left( {B_{r R}{\left( {x_0} \right) }} \right) }}.
\end{align*}
Therefore, ${\displaystyle} {\left\vert\left\vert {W_R}\right\vert\right\vert}_{L^s{\left( {B_{10}{\left( {0} \right) }} \right) }} = R^{1 - \frac n s} {\left\vert\left\vert {W}\right\vert\right\vert}_{L^s{\left( {B_{10R}{\left( {x_0} \right) }} \right) }} \le A_1 R^{1 - \frac n s}$ and ${\displaystyle} {\left\vert\left\vert {V_R}\right\vert\right\vert}_{L^t{\left( {B_{10}{\left( {0} \right) }} \right) }} \le A_0 R^{2 - \frac n t}$.
Moreover,
\begin{align*}
& {\Delta} u_R{\left( {x} \right) } + W_R{\left( {x} \right) } \cdot {\nabla} u_R{\left( {x} \right) }  + V_R{\left( {x} \right) } u_R{\left( {x} \right) } \\
&= R^2 {\Delta} u{\left( {x_0 + R x} \right) } + R^2 W{\left( {x_0 + Rx} \right) } \cdot {\nabla} u{\left( {x_0 + R x} \right) }  + R^2 V{\left( {x_0 + R x} \right) }u{\left( {x_0 + R x} \right) }
= 0.
\end{align*}
Therefore, $u_R$ is a solution to a scaled version of \eqref{goal} in $B_{10}$.
Clearly,
\begin{align*}
{\left\vert\left\vert {u_R}\right\vert\right\vert}_{L^{\infty}{\left( {B_{10}} \right) }}
&= {\left\vert\left\vert {u}\right\vert\right\vert}_{L^{\infty}{\left( {B_{10R}{\left( {x_0} \right) }} \right) }} \le C_0.
\end{align*}
Note that for ${\displaystyle}\widetilde{x_0} := -x_0/R$, ${\displaystyle}| \widetilde{x_0}| = 1$ and ${\left\vert{u_R(\widetilde{x_0})}\right\vert} = {\left\vert{u(0)}\right\vert} \ge 1$ so that ${\displaystyle}{\left\vert\left\vert {u_R}\right\vert\right\vert}_{L^{\infty}(B_1)} \ge 1$.
Thus, if $R$ is sufficiently large, then we may apply Theorem \ref{thh} to $u_R$ with $K = A_1 R^{1 - \frac n s}$, $M = A_2 R^{2 - \frac n t}$, and $\hat C = C_0$ to get
\begin{align*}
{\left\vert\left\vert {u}\right\vert\right\vert}_{L^{\infty}{\left( {{B_{1}(x_0)}} \right) }} = & {\left\vert\left\vert {u_R}\right\vert\right\vert}_{L^{\infty}{\left( {B_{1/R}(0)} \right) }}  \\
\ge & c(1/R)^{^{C{\left[{1 + C_1 {\left( {A_1 R^{1 - \frac n s}} \right) }^\kappa + C_2 {\left( {A_2 R^{2 - \frac n t} } \right) }^\mu}\right]}}} \\
= &c \exp{\left\{{-C{\left[{1 + C_1 {\left( {A_1 R^{1 - \frac n s}} \right) }^\kappa + C_2 {\left( {A_2 R^{2 - \frac n t} } \right) }^\mu}\right]} \log R}\right\}}.
\end{align*}
Since
\begin{align*}
\max{\left\{{\kappa{\left( {1 - \frac n s} \right) }, \mu{\left( {2 - \frac n t} \right) } }\right\}}= \Pi := \left\{\begin{array}{ll}
\frac{4{\left( {s-n} \right) }}{2s - {\left( {3n-2} \right) }} & t > \frac{sn}{s+n}, \medskip \\
\frac{4 {\left( {t-n\frac t s} \right) }}{{\left( {5 - \frac 2 n} \right) }s - {\left( {3n-2} \right) } \frac s
t} & n{\left( {\frac{3n-2}{5n-2}} \right) } < t \le \frac{sn}{s+n},
\end{array}\right.
\end{align*}
then
\begin{align*}
{\left\vert\left\vert {u}\right\vert\right\vert}_{L^{\infty}{\left( {{B_{1}(x_0)}} \right) }}
\ge &c \exp{\left[{-C{\left( {1 + C_1 A_1^\kappa + C_2 A_2^\mu } \right) } R^\Pi \log R}\right]}
\end{align*}
and the conclusion of the theorem follows.
\end{proof}

Corollary \ref{UCW} follows from Theorem \ref{UCVW} with $V \equiv 0$.
Finally, we present the proof of Theorem \ref{UCV}.

\begin{proof}[Proof of Theorem \ref{UCV}]
Let $u$ be a solution to \eqref{goal1} in ${\ensuremath{\mathbb{R}}}^n$.
Fix $x_0 \in {\ensuremath{\mathbb{R}}}^n$ and set ${\left\vert{x_0}\right\vert} = R$.
As above, let $u_R(x) = u(x_0 + Rx)$ and $V_R{\left( {x} \right) } = R^2 V{\left( {x_0 + R x} \right) }$.
Then, ${\displaystyle} {\left\vert\left\vert {V_R}\right\vert\right\vert}_{L^t{\left( {B_{10}{\left( {0} \right) }} \right) }} \le A_0 R^{2 - \frac n t}$ and
\begin{align*}
 {\Delta} u_R{\left( {x} \right) } + V_R{\left( {x} \right) } u_R{\left( {x} \right) }
&= R^2 {\Delta} u{\left( {x_0 + R x} \right) } + R^2 V{\left( {x_0 + R x} \right) }u{\left( {x_0 + R x} \right) }
= 0.
\end{align*}
Therefore, $u_R$ is a solution to a scaled version of \eqref{goal1} in $B_{10}$.
Clearly,
\begin{align*}
{\left\vert\left\vert {u_R}\right\vert\right\vert}_{L^{\infty}{\left( {B_{10}} \right) }}
&= {\left\vert\left\vert {u}\right\vert\right\vert}_{L^{\infty}{\left( {B_{10R}{\left( {x_0} \right) }} \right) }} \le C_0.
\end{align*}
Note that for ${\displaystyle}\widetilde{x_0} := -x_0/R$, ${\displaystyle}| \widetilde{x_0}| = 1$ and ${\left\vert{u_R(\widetilde{x_0})}\right\vert} = {\left\vert{u(0)}\right\vert} \ge 1$ so that ${\displaystyle}{\left\vert\left\vert {u_R}\right\vert\right\vert}_{L^{\infty}(B_1)} \ge 1$.
Thus, if $R$ is sufficiently large, then we may apply Theorem \ref{thhh} to $u_R$ with $M = A_2 R^{2 - \frac n t}$, and $\hat C = C_0$ to get
\begin{align*}
{\left\vert\left\vert {u}\right\vert\right\vert}_{L^{\infty}{\left( {{B_{1}(x_0)}} \right) }}
&= {\left\vert\left\vert {u_R}\right\vert\right\vert}_{L^{\infty}{\left( {B_{1/R}(0)} \right) }}
\ge c(1/R)^{^{C{\left[{1 + C_2 {\left( {A_2 R^{2 - \frac n t} } \right) }^\mu}\right]}}} \\
&= c \exp{\left\{{-C{\left[{1 + C_2 {\left( {A_2 R^{2 - \frac n t} } \right) }^\mu}\right]} \log R}\right\}}.
\end{align*}
Since
\begin{align*}
\mu{\left( {2 - \frac n t} \right) } = \Pi :=  \left\{\begin{array}{ll}
\frac{4{\left( {2t-n} \right) }}{6t - {\left( {3n-2} \right) }} & t > n, \medskip \\
\frac{4{\left( {2t-n} \right) }} {7t+\frac{2t}n-4n-{\varepsilon}} & \frac{4n^2}{7n+2} < t
\le n,
\end{array}\right.
\end{align*}
then
\begin{align*}
{\left\vert\left\vert {u}\right\vert\right\vert}_{L^{\infty}{\left( {{B_{1}(x_0)}} \right) }}
\ge &c \exp{\left[{-C{\left( {1 + C_2 A_2^\mu } \right) } R^\Pi \log R}\right]},
\end{align*}
as required.
\end{proof}

\begin{thebibliography}{BKRS88}

\bibitem[Bak12]{Bak12}
Laurent Bakri.
\newblock Quantitative uniqueness for {S}chr{\"o}dinger operator.
\newblock {\em Indiana Univ. Math. J.}, 61(4):1565--1580, 2012.

\bibitem[BK05]{BK05}
Jean Bourgain and Carlos~E. Kenig.
\newblock On localization in the continuous {A}nderson-{B}ernoulli model in
  higher dimension.
\newblock {\em Invent. Math.}, 161(2):389--426, 2005.

\bibitem[BKRS88]{BKRS88}
B.~Barcel\'o, C.~E. Kenig, A.~Ruiz, and C.~D. Sogge.
\newblock Weighted {S}obolev inequalities and unique continuation for the
  {L}aplacian plus lower order terms.
\newblock {\em Illinois J. Math.}, 32(2):230--245, 1988.

\bibitem[Dav14]{Dav14}
Blair Davey.
\newblock Some quantitative unique continuation results for eigenfunctions of
  the magnetic {S}chr\"odinger operator.
\newblock {\em Comm. Partial Differential Equations}, 39(5):876--945, 2014.

\bibitem[Dav15]{Dav15}
Blair Davey.
\newblock A {M}eshkov-type construction for the borderline case.
\newblock {\em Differential Integral Equations}, 28(3-4):271--290, 2015.

\bibitem[DF88]{DF88}
Harold Donnelly and Charles Fefferman.
\newblock Nodal sets of eigenfunctions on {R}iemannian manifolds.
\newblock {\em Invent. Math.}, 93(1):161--183, 1988.

\bibitem[DF90]{DF90}
H.~Donnelly and C.~Fefferman.
\newblock Growth and geometry of eigenfunctions of the {L}aplacian.
\newblock In {\em Analysis and partial differential equations}, volume 122 of
  {\em Lecture Notes in Pure and Appl. Math.}, pages 635--655. Dekker, New
  York, 1990.

\bibitem[DKW16]{DKW16}
Blair Davey, Carlos Kenig, and Jenn-Nan Wang.
\newblock The {L}andis conjecture for variable coefficient second-order
  elliptic {PDE}s.
\newblock arXiv:1510.04762, to appear in Trans. Amer. Math. Soc., 2016.

\bibitem[GT01]{GT01}
David Gilbarg and Neil~S. Trudinger.
\newblock {\em Elliptic partial differential equations of second order}.
\newblock Classics in Mathematics. Springer-Verlag, Berlin, 2001.
\newblock Reprint of the 1998 edition.

\bibitem[HL11]{HL11}
Qing Han and Fanghua Lin.
\newblock {\em Elliptic partial differential equations}, volume~1 of {\em
  Courant Lecture Notes in Mathematics}.
\newblock Courant Institute of Mathematical Sciences, New York, second edition,
  2011.

\bibitem[Jer86]{Jer86}
David Jerison.
\newblock Carleman inequalities for the {D}irac and {L}aplace operators and
  unique continuation.
\newblock {\em Adv. in Math.}, 62(2):118--134, 1986.

\bibitem[JK85]{JK85}
David Jerison and Carlos~E. Kenig.
\newblock Unique continuation and absence of positive eigenvalues for
  {S}chr\"odinger operators.
\newblock {\em Ann. of Math. (2)}, 121(3):463--494, 1985.
\newblock With an appendix by E. M. Stein.

\bibitem[Ken07]{Ken07}
Carlos~E. Kenig.
\newblock Some recent applications of unique continuation.
\newblock In {\em Recent developments in nonlinear partial differential
  equations}, volume 439 of {\em Contemp. Math.}, pages 25--56. Amer. Math.
  Soc., Providence, RI, 2007.

\bibitem[Kim89]{Kim89}
Yeon~Mi Kim.
\newblock {\em Carleman inequalities and strong unique continuation}.
\newblock ProQuest LLC, Ann Arbor, MI, 1989.
\newblock Thesis (Ph.D.)--Massachusetts Institute of Technology.

\bibitem[KSW15]{KSW15}
Carlos Kenig, Luis Silvestre, and Jenn-Nan Wang.
\newblock On {L}andis' {C}onjecture in the {P}lane.
\newblock {\em Comm. Partial Differential Equations}, 40(4):766--789, 2015.

\bibitem[KT01]{KT01}
Herbert Koch and Daniel Tataru.
\newblock Carleman estimates and unique continuation for second-order elliptic
  equations with nonsmooth coefficients.
\newblock {\em Comm. Pure Appl. Math.}, 54(3):339--360, 2001.

\bibitem[KT16]{KT16}
Abel Klein and C.~S.~Sidney Tsang.
\newblock Quantitative unique continuation principle for {S}chr{\"o}dinger
  operators with singular potentials.
\newblock {\em Proc. Amer. Math. Soc.}, 144(2):665--679, 2016.

\bibitem[Kuk98]{Kuk98}
Igor Kukavica.
\newblock Quantitative uniqueness for second-order elliptic operators.
\newblock {\em Duke Math. J.}, 91(2):225--240, 1998.

\bibitem[KW15]{KW15}
Carlos Kenig and Jenn-Nan Wang.
\newblock Quantitative uniqueness estimates for second order elliptic equations
  with unbounded drift.
\newblock {\em Math. Res. Lett.}, 22(4):1159--1175, 2015.

\bibitem[LW14]{LW14}
Ching-Lung Lin and Jenn-Nan Wang.
\newblock Quantitative uniqueness estimates for the general second order
  elliptic equations.
\newblock {\em J. Funct. Anal.}, 266(8):5108--5125, 2014.

\bibitem[Mes92]{Mes92}
V.~Z. Meshkov.
\newblock On the possible rate of decay at infinity of solutions of second
  order partial differential equations.
\newblock {\em Math USSR SB.}, 72:343--361, 1992.

\bibitem[MV12]{MV12}
E.~Malinnikova and S.~Vessella.
\newblock Quantitative uniqueness for elliptic equations with singular lower
  order terms.
\newblock {\em Math. Ann.}, 353(4):1157--1181, 2012.

\bibitem[Reg99]{Reg99}
R.~Regbaoui.
\newblock Unique continuation for differential equations of {S}chr{\"o}dinger's
  type.
\newblock {\em Comm. Anal. Geom.}, 7(2):303--323, 1999.

\bibitem[Sog86]{Sog86}
Christopher~D. Sogge.
\newblock Oscillatory integrals and spherical harmonics.
\newblock {\em Duke Math. J.}, 53(1):43--65, 1986.

\bibitem[Wol90]{Wol90}
Thomas~H. Wolff.
\newblock Unique continuation for {$\vert \Delta u\vert \le V\vert \nabla
  u\vert $} and related problems.
\newblock {\em Rev. Mat. Iberoamericana}, 6(3-4):155--200, 1990.

\bibitem[Zhu15]{Zhu15}
Jiuyi Zhu.
\newblock Doubling property and vanishing order of {S}teklov eigenfunctions.
\newblock {\em Comm. Partial Differential Equations}, 40(8):1498--1520, 2015.

\bibitem[Zhu16]{Zhu16}
Jiuyi Zhu.
\newblock Quantitative uniqueness of elliptic equations.
\newblock {\em Amer. J. Math.}, 138(3):733--762, 2016.

\end{thebibliography}

\end{document}


\documentclass{amsart}

\oddsidemargin 6pt \evensidemargin 6pt \marginparwidth 48pt
\marginparsep 10pt
\topmargin -18pt \headheight 12pt \headsep 25pt  \footskip 30pt
\textheight 625pt \textwidth 431pt \columnsep 10pt \columnseprule 0pt

\usepackage[all]{xy}
\usepackage{mathabx}
\usepackage{appendix}

\theoremstyle{plain}
\newtheorem{hypo}{Hypothesis}
\newtheorem{thm}{Theorem}[subsection]
\newtheorem{prop}[thm]{Proposition}
\newtheorem{lemma}[thm]{Lemma}
\newtheorem{cor}[thm]{Corollary}
\newtheorem{question}[thm]{Question}
\newtheorem{problem}[thm]{Problem}
\newtheorem{conj}[thm]{Conjecture}
\newtheorem{secthm}{Theorem}[section]
\newtheorem{secprop}[secthm]{Proposition}
\newtheorem{seclem}[secthm]{Lemma}
\newtheorem{seccor}[secthm]{Corollary}

\theoremstyle{definition}
\newtheorem{dfn}[thm]{Definition}
\newtheorem{secdfn}[secthm]{Definition}
\newtheorem{example}[thm]{Example}
\newtheorem{exs}[thm]{Examples}

\theoremstyle{remark}
\newtheorem{rem}[thm]{Remark}
\newtheorem{obs}[thm]{Observation}

 
 

\begin{document}

\title{Deformations of complexes for finite dimensional algebras}

\author{Frauke M. Bleher}
\address{F.B.: Department of Mathematics\\University of Iowa\\
14 MacLean Hall\\Iowa City, IA 52242-1419, U.S.A.}
\email{frauke-bleher@uiowa.edu}
\thanks{The first author was supported in part by NSF Grant DMS-1360621.}
\author{Jos\'{e} A. V\'{e}lez-Marulanda}
\address{J.V.: Department of Mathematics \& Computer Science\\Valdosta State University\\
2072 Nevins Hall\\Valdosta, GA 31698-0040, U.S.A.}
\email{javelezmarulanda@valdosta.edu}

\subjclass[2000]{Primary 16G10; Secondary 16G20, 20C20}
\keywords{Deformation rings, complexes, finite dimensional algebras}

\begin{abstract}
Let $k$ be a field and let $\Lambda$ be a finite dimensional $k$-algebra. We prove that every bounded complex $V^\bullet$
of finitely generated $\Lambda$-modules has a well-defined versal deformation ring $R(\Lambda,V^\bullet)$ which is
a complete local commutative Noetherian $k$-algebra with residue field $k$. We also prove that certain nice two-sided tilting
complexes between $\Lambda$ and another finite dimensional $k$-algebra $\Gamma$ preserve these versal deformation
rings. We apply these results to the derived equivalence class of a particular family of algebras of dihedral type
which were introduced by Erdmann and shown by Holm to be not derived equivalent to any block of a group algebra.
\end{abstract}

\maketitle

\section{Introduction}
\label{s:intro}
\setcounter{equation}{0}

Let $k$ be a field of arbitrary characteristic, and let $W$ be a complete local commutative
Noetherian ring with residue field $k$. Suppose $G$ is a profinite group and $V^\bullet$ is an
object in the derived category $D^-(k[[G]])$ of bounded above complexes of pseudocompact modules 
over the completed group algebra $k[[G]]$ of $G$ over $k$. It was proved in \cite{comptes,bcderived}
that Mazur's deformation theory of  $k[[G]]$-modules of finite $k$-dimension can be generalized to the study of 
deformations of complexes $V^\bullet$ with finitely many non-zero cohomology groups all of which are finite dimensional over $k$.
In particular, $V^\bullet$ always has a versal deformation ring $R(G,V^\bullet)$ which is a complete local commutative
Noetherian $W$-algebra with residue field $k$. Moreover, the case of modules corresponds to the case
when $V^\bullet$ has precisely one non-zero cohomology group. Additionally, it was shown in \cite{derivedeq}
that $R(G,V^\bullet)$ is preserved by certain nice derived equivalences when $G$ is a finite group and
$k[[G]]$ is replaced by a block $B$ of $kG$.

In this paper, we consider the case when $W$ is taken to be equal to $k$ and $k[[G]]$, respectively $B$, is replaced by an
arbitrary finite dimensional $k$-algebra $\Lambda$. Our main motivation to examine this case is as follows.
Suppose $G$ is a finite group and $V^\bullet$ is a bounded complex of finitely generated $kG$-modules which all
belong to a block $B$ of $kG$. Assume $\Lambda$ is a finite dimensional $k$-algebra such that there is a 
sufficiently nice two-sided tilting complex between $\Lambda$ and $B$ (see Definition \ref{def:nice2sidedtilting}) 
and $V^\bullet_\Lambda$ is the complex of $\Lambda$-modules corresponding to $V^\bullet$ under this
two-sided tilting complex. For example, $\Lambda$ can be taken to be the basic algebra of $B$.
Then the nice two-sided titlting complex provides a correspondence between the deformations of $V^\bullet$ and
the deformations of $V^\bullet_\Lambda$ over all complete local commutative Noetherian 
$k$-algebras with residue field $k$, and thus
between the versal deformation rings $R(G,V^\bullet)$ and $R(\Lambda,V^\bullet_\Lambda)$ (for a 
precise statement, see Theorem \ref{thm:deformations}). This allows us to use
methods from the representation theory of finite dimensional 
algebras to study deformation rings of bounded complexes of group representations. Moreover,
Rickard showed that a derived equivalence between $B$ and $\Lambda$ induces a 
stable equivalence of Morita type between $B$ and $\Lambda$ (see \cite[Cors. 5.3 and 5.5]{rickard1}). 
Since the components of the stable
Auslander-Reiten quivers of $B$ and $\Lambda$ match up under this stable equivalence, the
knowledge of the universal deformation rings of all $\Lambda$-modules whose stable endomorphism rings
are isomorphic to $k$ can be used to determine the universal deformation rings of all $B$-modules
whose stable endomorphism rings are isomorphic to $k$. In view of the famous conjecture by Brou\'{e} 
(see \cite{broue,rickardICM} and their references) that
blocks $B$ of $kG$ with abelian defect group $D$ are derived equivalent to blocks $b$ of $N_G(D)$,
this opens up the possibility of drastically reducing and simplifying computations of deformation rings.

Let now  $\Lambda$ be an arbitrary finite dimensional $k$-algebra and let $D^-(\Lambda)$ be the derived category
of bounded above complex of pseudocompact $\Lambda$-modules. In Section \ref{s:udr} we show how the arguments in 
\cite{comptes,bcderived} can be modified to prove the following result. For a more precise statement, see Theorem \ref{thm:bigthm}.

\begin{secthm}
\label{thm:main1}
Let $V^\bullet$ be an object of $D^-(\Lambda)$ such that $V^\bullet$ only has finitely many non-zero cohomology groups,
all of which have finite $k$-dimension. Then $V^\bullet$ always has a versal deformation ring $R(\Lambda,V^\bullet)$ 
which is a complete local commutative Noetherian $k$-algebra with residue field $k$. Moreover, $R(\Lambda,V^\bullet)$ is universal 
if the endomorphism ring of $V^\bullet$ in $D^-(\Lambda)$ is isomorphic to $k$.
\end{secthm}

The main challenge of the proof of Theorem \ref{thm:main1} is to ensure that the arguments in \cite{comptes,bcderived}
can be modified to work with arbitrary finite dimensional algebras $\Lambda$ that may be neither Frobenius nor self-injective.
In Section \ref{s:derivedequivalences}, we generalize the arguments in \cite{derivedeq} to prove the following result. 
For a more precise statement, see Theorem \ref{thm:deformations}.

\begin{secthm}
\label{thm:main2}
Suppose $\Gamma$ is another finite dimensional $k$-algebra such that $\Lambda$ and $\Gamma$ are derived equivalent.
Then there exists a nice two-sided tilting complex $P^\bullet$ of finitely generated $\Gamma$-$\Lambda$-bimodules such that
if $V^\bullet$ is a bounded complex of finitely generated $\Lambda$-modules and ${V'}^\bullet=P^\bullet\otimes_\Lambda V^\bullet$,
then the versal deformation rings $R(\Lambda,V^\bullet)$ and $R(\Gamma,{V'}^\bullet)$ are isomorphic.
\end{secthm}

Again, the challenge is to modify the arguments in \cite{derivedeq} so that they work for arbitrary finite dimensional $k$-algebras
$\Lambda$ that may be  neither Frobenius nor self-injective.
Moreover, we show in Proposition \ref{prop:stabmordef} that arbitrary stable equivalences of Morita type between
{self-injective} algebras preserve versal deformation rings of modules. In Section \ref{s:examples}, we show how the main results
from Sections \ref{s:udr} and \ref{s:derivedequivalences} can be applied to the derived equivalence
class of a particular family of algebras of dihedral type which was introduced by Erdmann in \cite{erd} and denoted by $D(3\mathcal{R})$. Note
that Holm showed in \cite[Sect. 3.2]{holm} that none of the algebras in the family $D(3\mathcal{R})$ is derived equivalent to a block of
a group algebra. Theorems \ref{thm:derivedex2},
\ref{thm:derivedex3} and Proposition \ref{prop:derivedexvelez} demonstrate how the knowledge of the universal
deformation rings of certain $D(3\mathcal{R})$-modules can be used to determine the universal deformation
rings of modules for another algebra $\Lambda$ which is just known to be derived equivalent to $D(3\mathcal{R})$.
Finally, in Appendix \ref{s:continuity}, we revisit the proof of the continuity of the deformation functor defined in 
\cite{bcderived}.
 

\section{Versal deformation rings for complexes over finite dimensional algebras}
\label{s:udr}
\setcounter{equation}{0}

In \cite{comptes,bcderived}, it was proved that if $k$ is a field of positive characteristic,
$G$ is a profinite group satisfying a certain finiteness
condition and $V^\bullet$ is quasi-isomorphic to a bounded complex of pseudocompact 
$k[[G]]$-modules, then $V^\bullet$ always has a versal deformation ring.
Moreover, it was proved that if the endomorphism ring of $V^\bullet$ 
in the derived category of bounded above complexes of pseudocompact $k[[G]]$-modules 
is isomorphic to $k$, then this versal deformation ring is universal. 

It is the goal of this section to prove an analogous result when $k$ is  an arbitrary field
and $k[[G]]$ is replaced by an arbitrary finite dimensional $k$-algebra $\Lambda$.
In Section \ref{s:setup}, we will recall some preliminary results on pseudocompact rings and modules,
define quasi-lifts and deformations of complexes, and state our main result of this section, Theorem \ref{thm:bigthm}.
In Section \ref{s:quasilifts}, we analyze the structure of quasi-lifts over Artinian rings. In Section \ref{s:prelims},
we provide some more results on complexes and quasi-lifts. In Section
\ref{s:proofbigthm}, we prove Theorem \ref{thm:bigthm}. In Section \ref{s:12terms}, 
we consider quasi-lifts of one-term and two-term complexes, and completely split complexes.

\subsection{Pseudocompact rings and modules, quasi-lifts, and deformations of complexes}
\label{s:setup}

Recall from \cite{brumer} 
that a pseudocompact ring $\Omega$ is a complete Hausdorff topological ring that admits a system 
of open neighborhoods of $0$ consisting of two-sided ideals $I$ for which $\Omega/I$ is an Artinian ring.
In particular, every finite dimensional algebra $\Lambda$ over any field $k$
is a pseudocompact ring, by choosing all two-sided ideals of $\Lambda$ as an 
open neighborhood basis of $0$. In other words, $\Lambda$ is a pseudocompact ring with the discrete
topology. Moreover, since $k$ is a commutative pseudocompact ring (again with the discrete topology),
$\Lambda$ is a pseudocompact $k$-algebra.
A $\Lambda$-module $M$ is called pseudocompact if $M$ is a complete Hausdorff topological 
$\Lambda$-module that has a basis of open neighborhoods  of $0$ consisting of submodules $N$ for which
$M/N$ has finite length as $\Lambda$-module. In particular, each finitely generated $\Lambda$-module
$M$ is pseudocompact, by choosing all submodules of $M$ as an open neighborhood basis of $0$
(i.e. by giving $M$ the discrete topology).
Let $D^-(\Lambda)$ be the derived category of bounded above complexes of pseudocompact
$\Lambda$-modules. 

\begin{hypo}
\label{hypo:fincoh}
Throughout this paper, we assume that $k$ is an arbitrary field, $\Lambda$ is a finite dimensional
$k$-algebra, and $V^\bullet $ is a
complex in $D^-(\Lambda)$ that has  only finitely many non-zero cohomology
groups, all of which have finite $k$-dimension.
\end{hypo}

Define $\hat{\mathcal{C}}$ to be the category of all complete local commutative Noetherian 
$k$-algebras with residue field $k$. The morphisms in $\hat{\mathcal{C}}$ are 
continuous $k$-algebra homomorphisms that induce the identity on $k$.
Let $\mathcal{C}$ be the subcategory of  Artinian objects in $\hat{\mathcal{C}}$.

For $R \in \mathrm{Ob}(\hat{\mathcal{C}})$, define $R\Lambda=R\otimes_k\Lambda$. 
Then $R$ is a commutative pseudocompact ring, and
$R\Lambda$ is a pseudocompact $R$-algebra. Define $\mathrm{PCMod}(R\Lambda)$ 
to be the category of pseudocompact $R\Lambda$-modules.

Pseudocompact rings, algebras and modules have been studied, for example, in 
\cite{ga1,ga2} and \cite{brumer}. For the convenience of the reader, we state some useful facts
from these references.

\begin{rem}
\label{rem:pseudocompact}
Let $R \in \mathrm{Ob}(\hat{\mathcal{C}})$.
\begin{enumerate}
\item[(i)] The ring $R\Lambda$ is the inverse limit 
of Artinian quotient rings. 
An $R\Lambda$-module is pseudocompact if and only if it is the inverse limit of 
$R\Lambda$-modules of finite length. 
Moreover, an $R\Lambda$-module has finite length 
if and only if it has finite length as an $R$-module.
The category $\mathrm{PCMod}(R\Lambda)$ is an abelian
category with exact inverse limits. 

\item[(ii)] A pseudocompact $R\Lambda$-module $M$ is said to be topologically free on a set
$X=\{x_i\}_{i\in I}$ if $M$ is isomorphic to the product of a family $(R\Lambda_i)_{i\in I}$ where
$R\Lambda_i=R\Lambda$ for all $i$.
Every topologically free pseudocompact $R\Lambda$-module is a projective object in 
$\mathrm{PCMod}(R\Lambda)$, and every pseudocompact
$R\Lambda$-module is the quotient of a topologically free $R\Lambda$-module. Hence
$\mathrm{PCMod}(R\Lambda)$ has enough projective objects. 

\item[(iii)] Suppose $M$ and $N$ are pseudocompact $R\Lambda$-modules. Then we define the right 
derived functors $\mathrm{Ext}^n_{R\Lambda}(M,N)$ by using a projective resolution of $M$. 
\end{enumerate}
\end{rem}

\begin{rem}
\label{rem:extrafree}
Let $R$ be an object in $\hat{\mathcal{C}}$ with maximal ideal $m_R$. Suppose that 
$[(R/m_R^i)X_i]$ is an abstractly free $(R/m_R^i)$-module on the finite topological space $X_i$
for all $i$,
and that $\{X_i\}_i$ forms an inverse system. Define $X=\displaystyle \lim_{\stackrel{
\longleftarrow}{i}} X_i$ and $[[R X]] = \displaystyle \lim_{\stackrel{\longleftarrow}{i}} 
[(R/m_R^i) X_i]$. Then
$[[R X]]$ is a topologically free pseudocompact $R$-module on $X$. 
\end{rem}

\begin{rem}
\label{rem:topflat}
Suppose $R \in \mathrm{Ob}(\hat{\mathcal{C}})$, and $\Omega=R$ or $\Omega=R\Lambda$. 
Let $M$ be a right (resp. left) pseudocompact $\Omega$-module
\begin{enumerate}
\item[(i)]
Let $\hat{\otimes}_{\Omega}$ denote the completed tensor product in the category 
$\mathrm{PCMod}(\Omega)$ (see \cite[\S 2]{brumer}). Then $M\hat{\otimes}_\Omega- $ 
(resp. $-\hat{\otimes}_\Omega M$) is a right exact functor. Moreover,  $M$ is said to be 
topologically flat, if the functor $M\hat{\otimes}_{\Omega}-$ (resp. $-\hat{\otimes}_\Omega M$)
is exact.

\item[(ii)]
By \cite[Lemma 2.1(iii)]{brumer} and \cite[Prop. 3.1]{brumer}, $M$ is topologically flat if and only 
if $M$ is projective.

\item[(iii)]
If $M$ is finitely generated as a pseudocompact $\Omega$-module, it follows from
\cite[Lemma 2.1(ii)]{brumer} that the functors
$M \otimes_\Omega -$ and $M\hat{\otimes}_\Omega -$ (resp. $-\otimes_\Omega M$ and 
$-\hat{\otimes}_\Omega M$) are naturally isomorphic.

\item[(iv)]
If $\Omega=R$ and $M$ is a pseudocompact $R$-module,
it follows from \cite[Proof of Prop. 0.3.7]{ga2} and \cite[Cor. 0.3.8]{ga2} that $M$ is 
topologically flat if and only if $M$ is topologically free if and only if $M$ is abstractly flat.
In particular, if $R$ is Artinian, a pseudocompact $R$-module is topologically flat  if and only
if it is abstractly free.
\end{enumerate}
\end{rem}

For $R \in \mathrm{Ob}(\hat{\mathcal{C}})$, let $C^-(R\Lambda)$
be the abelian category of complexes of pseudocompact $R\Lambda$-modules that are bounded above, let $K^-(R\Lambda)$ be the homotopy category of $C^-(R\Lambda)$, and
let $D^-(R\Lambda)$ be the derived category of $K^-(R\Lambda)$. 
Let $[1]$ denote the translation functor on $C^-(R\Lambda)$ (resp. $K^-(R\Lambda)$, 
resp. $D^-(R\Lambda)$), 
i.e. $[1]$ shifts complexes
one place to the left and changes the signs of the differentials.
Recall that a homomorphism in $C^-(R\Lambda)$
is a quasi-isomorphism if and only if the induced homomorphisms on 
all the cohomology groups are bijective.

\begin{rem}
\label{rem:leftderivedtensor}
Let $R \in \mathrm{Ob}(\hat{\mathcal{C}})$, and let $\mathcal{P}_R$ be the additive subcategory of 
$\mathrm{PCMod}(R\Lambda)$ of projective objects.
By Remark \ref{rem:pseudocompact}(ii) and \cite[Thm. 10.4.8]{Weibel}, the natural functor 
$K^-(\mathcal{P}_R)\to D^-(R\Lambda)$ is an equivalence of triangulated categories. 
Let $\sigma_R: D^-(R\Lambda)\to K^-(\mathcal{P}_R)$ denote its quasi-inverse.

Suppose $S$ is a pseudocompact $R$-module. If $S\in \mathrm{Ob}(\hat{\mathcal{C}})$ and there
exists a morphism $\alpha:R\to S$ defining the $R$-module structure on $S$, let $R_S=S$.
Otherwise let $R_S=R$. Consider the completed tensor product functor
$$S\hat{\otimes}_R-:K^-(R\Lambda)\to K^-(R_S\Lambda)\,.$$
By \cite[Thm. 10.5.6]{Weibel}, its left derived functor $S\hat{\otimes}^{\mathbf{L}}_R-:
D^-(R\Lambda)\to D^-(R_S\Lambda)$ is the following
composition of functors of triangulated categories:
\begin{equation}
\label{eq:leftderiveddef}
D^-(R\Lambda)\xrightarrow{\sigma_R}K^-(\mathcal{P}_R)\xrightarrow{S\hat{\otimes}_R-}
K^-(R_S\Lambda)\xrightarrow{q_S}D^-(R_S\Lambda)
\end{equation}
where $q_S:K^-(R_S\Lambda)\to D^-(R_S\Lambda)$ is the localization functor.
In other words, if $X^\bullet$ is in $\mathrm{Ob}(K^-(R\Lambda))$, then 
there exists an isomorphism $\rho_X:X^\bullet\to \sigma_R(X^\bullet)$ in $D^-(R\Lambda)$ and
\begin{equation}
\label{eq:leftderived}
S\hat{\otimes}^{\mathbf{L}}_R X^\bullet=S\hat{\otimes}_R\;\sigma_R(X^\bullet)\;.
\end{equation}
\end{rem}

The following definitions and remarks are adapted from \cite[Sect. 2]{bcderived} to our situation.

\begin{dfn}
\label{def:fintor}
We will say that a complex $M^\bullet$ in $K^-(R\Lambda)$
has \emph{finite pseudocompact $R$-tor dimension}, 
if there exists an integer $N$ such that for all pseudocompact
$R$-modules $S$, and for all integers $i<N$, ${\mathrm{H}}^i(S\hat{\otimes}^{\mathbf{L}}_R M^\bullet)=0$.
If we want to emphasize the integer $N$ in this definition, we say $M^\bullet$ has 
\emph{finite pseudocompact $R$-tor dimension at $N$}.
\end{dfn}

\begin{rem}
\label{rem:dumbdumb}
Suppose $M^\bullet$ is a complex in $K^-(R\Lambda)$ of topologically flat, hence topologically free, 
pseudocompact $R$-modules that has finite pseudocompact $R$-tor dimension
at $N$. Then the bounded complex ${M'}^\bullet$, which is
obtained from $M^\bullet$ by replacing $M^N$ by
${M'}^N=M^N/\delta^{N-1}(M^{N-1})$ and by setting ${M'}^i = 0$ if $i < N$,
is quasi-isomorphic to $M^\bullet$ and
has topologically free pseudocompact terms over $R$.
\end{rem}

\begin{dfn}
\label{def:lifts} 
A \emph{quasi-lift}
of $V^\bullet$ over an object $R$ of
$\hat{\mathcal{C}}$ is a pair $(M^\bullet,\phi)$ consisting of a complex
$M^\bullet$ in
$D^-(R\Lambda)$ that has finite pseudocompact $R$-tor  dimension
together with an isomorphism
\hbox{$\phi: k \hat{\otimes}^{\mathbf{L}}_R M^\bullet \to V^\bullet$} in $D^-(\Lambda)$.
Two
quasi-lifts $(M^\bullet, \phi)$ and $({M'}^\bullet,\phi')$ are
\emph{isomorphic} if there is an isomorphism
$f:M^\bullet \to {M'}^\bullet$ in $D^-(R\Lambda)$ with
$\phi'\circ(k\,\hat{\otimes}_R^{\mathbf{L}} f)=\phi$.
A \emph{deformation}
of $V^\bullet$ over $R$ is an isomorphism class of quasi-lifts of $V^\bullet$.

A \emph{proflat quasi-lift} of $V^\bullet$ over an object $R$ of $\hat{\mathcal{C}}$
is a quasi-lift $(M^\bullet,\phi)$ of $V^\bullet$ over $R$ whose cohomology
groups are topologically flat, and hence topologically free, pseudocompact $R$-modules.
A \emph{proflat deformation} of $V^\bullet$ over $R$ is an isomorphism class
of proflat quasi-lifts of $V^\bullet$.
\end{dfn}

\begin{rem}
\label{rem:difference}
There exist quasi-lifts that are not isomorphic to proflat quasi-lifts in $D^-(R\Lambda)$.
For example, suppose $\Lambda=k$
and $V^\bullet = k \xrightarrow{0} k$ is the two-term complex concentrated in dimensions
$-1$ and $0$ with trivial boundary map.  Then the two-term complex
$M^\bullet = k[[t]] \xrightarrow{t} k[[t]]$ concentrated in dimensions $-1$ and $0$
defines a quasi-lift of $V^\bullet$ over $k[[t]]$. However,  since $M^\bullet$ is isomorphic to
the one-term complex $k[[t]]/tk[[t]]\cong k$ concentrated in dimension $0$,
this quasi-lift is not isomorphic to a  proflat quasi-lift of $V^\bullet$ over $k[[t]]$.
\end{rem}

\begin{rem}
\label{rem:profree}
The following two statements are proved in the same way as \cite[Lemmas 2.9 and 2.11]{bcderived}
by using Remark \ref{rem:pseudocompact}(ii) and the fact that
a bounded above complex of topologically free pseudocompact modules splits completely.
\begin{enumerate} 
\item[(i)] Suppose $R \in \mathrm{Ob}(\hat{\mathcal{C}})$ and $(M^\bullet,\phi)$ is a quasi-lift of $V^\bullet$ 
over $R$. Then there exists a quasi-lift $(P^\bullet,\psi)$ of $V^\bullet$ over $R$ which is isomorphic to 
$(M^\bullet,\phi)$ such that the terms of $P^\bullet$ are topologically free pseudocompact 
$R\Lambda$-modules.
\item[(ii)] Suppose $R \in \mathrm{Ob}(\hat{\mathcal{C}})$ and $(M^\bullet,\phi)$ is a proflat quasi-lift of 
$V^\bullet$ over $R$. Then ${\mathrm{H}}^n(M^\bullet)$ is an abstractly free $R$-module of rank $d_n=
\mathrm{dim}_k\,{\mathrm{H}}^n(V^\bullet)$ for all $n$.
Moreover, for any $R'\in\mathrm{Ob}(\hat{\mathcal{C}})$ and for any morphism 
$\alpha:R\to R'$ in $\hat{\mathcal{C}}$, there is a natural
$R'$-linear isomorphism $R'\hat{\otimes}_R {\mathrm{H}}^n(M^\bullet) \cong {\mathrm{H}}^n(R'\hat{\otimes}_R^{\mathbf{L}} 
M^\bullet)$.
\end{enumerate}
\end{rem}

\begin{dfn}
\label{def:functordef}
Let $\hat{F} = \hat{F}_{V^\bullet}:\hat{\mathcal{C}} \to \mathrm{Sets}$
(resp. $\hat{F}^{\mathrm{f\/l}} = \hat{F}^{\mathrm{f\/l}}_{V^\bullet}:\hat{\mathcal{C}} \to \mathrm{Sets}$)
be the map which sends an object $R$ of $\hat{\mathcal{C}}$ to the set
$\hat{F}(R)$ (resp. $\hat{F}^{\mathrm{f\/l}}(R)$) of all deformations (resp. all proflat deformations)
of $V^\bullet$ over $R$, and which sends
a morphism $\alpha:R\to R'$ in $\hat{\mathcal{C}}$ to the set map
$\hat{F}(R)\to \hat{F}(R')$ (resp. $\hat{F}^{\mathrm{f\/l}}(R)\to \hat{F}^{\mathrm{f\/l}}(R')$)
induced by $(M^\bullet,\phi) \mapsto (R'\hat{\otimes}_{R,\alpha}^{\mathbf{L}}
M^\bullet,\phi_\alpha)$. Here $\phi_\alpha$ denotes the composition
$k\hat{\otimes}^{\mathbf{L}}_{R'} (R'\hat{\otimes}_{R,\alpha}^{\mathbf{L}} M^\bullet)
\cong k \hat{\otimes}^{\mathbf{L}}_R M^\bullet \xrightarrow{\phi} V^\bullet$.
Let $F = F_{V^\bullet}$ (resp. $F^{\mathrm{f\/l}} = F^{\mathrm{f\/l}}_{V^\bullet}$)
be the restriction of $\hat{F}$ (resp. $\hat{F}^{\mathrm{f\/l}}$) to the subcategory $\mathcal{C}$
of Artinian objects in $\hat{\mathcal{C}}$.
In the following, we will use the subscript $\mathcal{D}$ to denote the empty
condition if we consider the map $\hat{F}$, and the condition of having
topologically free cohomology groups
if we consider the map $\hat{F}^{\mathrm{f\/l}}$. In particular, the notation $\hat{F}_{\mathcal{D}}$
will be used to refer to both $\hat{F}$ and $\hat{F}^{\mathrm{f\/l}}$.

Let $k[\varepsilon]$, where $\varepsilon^2=0$, denote the ring of dual numbers over
$k$. The set $F_{\mathcal{D}}(k[\varepsilon])$ is called the \emph{tangent space} to 
$F_{\mathcal{D}}$, denoted by $t_{F_{\mathcal{D}}}$. 
\end{dfn}

\begin{rem}
\label{rem:functor}
Let ${F}_{\mathcal{D}}$ and
$\hat{F}_{\mathcal{D}}$ be as in Definition $\ref{def:functordef}$.
Using similar arguments as in the proof of \cite[Prop. 2.12]{bcderived}, it follows that
the map $\hat{F}_{\mathcal{D}}$ is a functor $\hat{\mathcal{C}}\to\mathrm{Sets}$.
Moreover, $\hat{F}^{\mathrm{f\/l}}$ is a subfunctor of $\hat{F}$ in the sense that there is
a natural transformation $\hat{F}^{\mathrm{f\/l}}\to\hat{F}$ which is injective.
If ${V'}^\bullet$ is a complex in $D^-(\Lambda)$ satisfying Hypothesis 
$\ref{hypo:fincoh}$ such
that there is an isomorphism $\nu:V^\bullet\to {V'}^\bullet$ in $D^-(\Lambda)$, 
then the natural transformation
$\hat{F}_{\mathcal{D},V^\bullet}\to \hat{F}_{\mathcal{D},{V'}^\bullet}$ $($resp.
${F}_{\mathcal{D},V^\bullet}\to {F}_{\mathcal{D},{V'}^\bullet}${}$)$ induced by
$(M^\bullet,\phi)\mapsto (M^\bullet,\nu\circ\phi)$ is an isomorphism of functors.
\end{rem}

The following theorem is the main result of Section \ref{s:udr}.

\begin{thm}
\label{thm:bigthm} Assume Hypothesis $\ref{hypo:fincoh}$, and let ${F}_{\mathcal{D}}$ and
$\hat{F}_{\mathcal{D}}$ be as in Definition $\ref{def:functordef}$.
\begin{enumerate}
\item[(i)] 
The functor
$F_{\mathcal{D}}$ has a pro-representable hull 
$R_{\mathcal{D}}(\Lambda,V^\bullet)\in \mathrm{Ob}(\hat{\mathcal{C}})$ 
$($c.f. \cite[Def. 2.7]{Sch} and \cite[\S 1.2]{Maz}$)$, and 
the functor $\hat{F}_{\mathcal{D}}$ is continuous
$($c.f. \cite{Maz}$)$. 

\item[(ii)]
If $F_{\mathcal{D}}=F$, then 
there is a $k$-vector space isomorphism $h: t_F \to
\mathrm{Ext}^1_{D^-(\Lambda)}(V^\bullet,V^\bullet)$.
If $F_{\mathcal{D}}=F^{\mathrm{f\/l}}$, then 
the composition of the natural map $t_{F^{\mathrm{f\/l}}}\to t_F$ and $h$ induces an isomorphism between
$t_{F^{\mathrm{f\/l}}}$ and
the kernel of the natural map $\mathrm{Ext}^1_{D^-(\Lambda)}(V^\bullet,V^\bullet)
\to \mathrm{Ext}^1_{D^-(k)}(V^\bullet,V^\bullet)$ given by forgetting the
$\Lambda$-action.

\item[(iii)]
If $\mathrm{Hom}_{D^-(\Lambda)}(V^\bullet,V^\bullet)= k$, then $\hat{F}_{\mathcal{D}}$ is represented
by $R_{\mathcal{D}}(\Lambda,V^\bullet)$. 
\end{enumerate}
\end{thm}

\begin{rem}
\label{rem:newrem}
By Theorem \ref{thm:bigthm}(i), there exists 
a quasi-lift $(U_{\mathcal{D}}(\Lambda,V^\bullet),\phi_U)$ of $V^\bullet$ over $R_{\mathcal{D}}(\Lambda,V^\bullet)$ 
with the following property. For each $R\in \mathrm{Ob}(\hat{\mathcal{C}})$, the map
$\mathrm{Hom}_{\hat{\mathcal{C}}}(R_{\mathcal{D}}(\Lambda,V^\bullet),R) \to \hat{F}_{\mathcal{D}}(R)$ 
induced by $\alpha \mapsto (R\hat{\otimes}^{\mathbf{L}}_{R_{\mathcal{D}}(\Lambda,V^\bullet),\alpha} U_{\mathcal{D}}(\Lambda,V^\bullet),\phi_{U,\alpha})$ is surjective,
and this map is bijective if $R$ is the ring of dual numbers $k[\varepsilon]$ over $k$
where $\varepsilon^2=0$.

In general,
the isomorphism type of the pair consisting of the
pro-representable hull $R_{\mathcal{D}}(\Lambda,V^\bullet)$ 
and the quasi-lift $(U_{\mathcal{D}}(\Lambda,V^\bullet),\phi_U)$ of $V^\bullet$ over $R_{\mathcal{D}}(\Lambda,V^\bullet)$ is
unique up to non-canonical isomorphism.
If $R_{\mathcal{D}}(\Lambda,V^\bullet)$ represents $\hat{F}_{\mathcal{D}}$,
the pair $(R_{\mathcal{D}}(\Lambda,V^\bullet),(U_{\mathcal{D}}(\Lambda,V^\bullet),\phi_U))$ is uniquely determined up 
to canonical isomorphism.
\end{rem}

\begin{dfn}
\label{def:newdef}
Using the notation of Theorem \ref{thm:bigthm} and Remark \ref{rem:newrem}, 
in case $\hat{F}_{\mathcal{D}}=\hat{F}$, we call 
$R_{\mathcal{D}}(\Lambda,V^\bullet)=R(\Lambda,V^\bullet)$
the \emph{versal deformation ring} of $V^\bullet$ and the isomorphism class of the quasi-lift 
$(U(\Lambda,V^\bullet),\phi_U)$ is called the \emph{versal deformation} of $V^\bullet$.
In case $\hat{F}_{\mathcal{D}}=\hat{F}^{\mathrm{f\/l}}$,
we call $R_{\mathcal{D}}(\Lambda,V^\bullet)=R^{\mathrm{f\/l}}(\Lambda,V^\bullet)$ the 
\emph{versal proflat deformation ring} of $V^\bullet$ and 
the isomorphism class of the quasi-lift $(U^{\mathrm{f\/l}}(\Lambda,V^\bullet),\phi_U)$ is called
the \emph{versal proflat deformation} of $V^\bullet$.

If $R_{\mathcal{D}}(\Lambda,V^\bullet)$ represents $\hat{F}_{\mathcal{D}}$, then
$R(\Lambda,V^\bullet)$ (resp. $R^{\mathrm{f\/l}}(\Lambda,V^\bullet)$) will be
called the \emph{universal deformation ring} (resp. \emph{the universal proflat deformation ring})
of $V^\bullet$, and the isomorphism class of
$(U(\Lambda,V^\bullet),\phi_U)$ (resp. $(U^{\mathrm{f\/l}}(\Lambda,V^\bullet),\phi_U)$) will be called the
\emph{universal deformation} (resp. the \emph{universal proflat deformation}) of $V^\bullet$.
\end{dfn}

\begin{rem}
\label{rem:bigthm}
\hspace*{1em}
\begin{itemize}
\item[(i)]
By part (ii) of Theorem \ref{thm:bigthm},
the tangent space $t_{F^{\mathrm{f\/l}}}$ consists of those elements
$$\gamma \in\mathrm{Ext}^1_{D^-(\Lambda)}(V^\bullet,V^\bullet)=\mathrm{Hom}_{D^-(\Lambda)}(
V^\bullet,V^\bullet[1])$$ which induce the trivial map on cohomology. In other words,
the $k$-vector space maps $\gamma^i:{\mathrm{H}}^i(V^\bullet)\to {\mathrm{H}}^{i+1}(V^\bullet)$ which are
induced by $\gamma$ have to be zero for all $i$.
\item[(ii)]
It follows from part (ii) of Theorem \ref{thm:bigthm} that there exists a non-canonical
surjective continuous $k$-algebra homomorphism
$f_{\mathrm{f\/l}}: R(\Lambda,V^\bullet) \to R^{\mathrm{f\/l}}(\Lambda,V^\bullet)$.
\item[(iii)]
If $V^\bullet$ consists of a single module $V_0$ in dimension $0$,
the versal deformation ring $R(\Lambda,V^\bullet)$ coincides with the versal deformation
ring $R(\Lambda,V_0)$ studied in \cite{blehervelez} (see Proposition \ref{prop:modulecase}).
\end{itemize}
\end{rem}

To prove Theorem \ref{thm:bigthm}, we adapt the argumentation in \cite{bcderived} to our situation.
Our first task is to adapt results from \cite[Sects.  3, 4 and 14]{bcderived} which
prove key properties of quasi-lifts of $V^\bullet$.

\subsection{Properties of quasi-lifts of $V^\bullet$}
\label{s:quasilifts}

In this subsection, 
we analyze the structure of quasi-lifts of $V^\bullet$ over Artinian objects $R$ in $\mathcal{C}$.
The following full subcategories of $C^{-}(R\Lambda)$, $K^{-}(R\Lambda)$ and 
$D^{-}(R\Lambda)$ play an important role in this situation.

\begin{dfn}
\label{dfn:fincategory}
Let $R \in \mathrm{Ob}(\mathcal {C})$ be Artinian.  Define $C^{-}_{\mathrm{f\/in}}(R\Lambda)$ 
(resp. $K^{-}_{\mathrm{f\/in}}(R\Lambda)$, resp. $D^{-}_{\mathrm{f\/in}}(R\Lambda)$) to
be the full subcategory of $C^{-}(R\Lambda)$ (resp. $K^{-}(R\Lambda)$, resp. 
$D^{-}(R\Lambda)$) whose objects are those complexes
$M^\bullet$ of finite pseudocompact $R$-tor dimension having finitely many 
non-zero cohomology groups, all of which have  finite $R$-length.  
\end{dfn}

\begin{rem}
\label{rem:whatever}
Suppose $R$ is an Artinian object in $\mathrm{Ob}(\mathcal {C})$.
\begin{enumerate}
\item[(i)]
By Remark \ref{rem:pseudocompact}(i), an $R\Lambda$-module has finite
length if and only if it has finite length as an $R$-module.
Since $R$ is local Artinian, an $R$-module has finite $R$-length if 
and only if it has finite $k$-length. 
\item[(ii)] Suppose $N^\bullet$ is a complex in $D^-_{\mathrm{f\/in}}(R\Lambda)$ and
$X^\bullet$ is a complex in $D^-(R\Lambda)$ such that there is an isomorphism
$\xi:X^\bullet \to N^\bullet$ in $D^-(R\Lambda)$. Then $X^\bullet$ is an object of
$D^-_{\mathrm{f\/in}}(R\Lambda)$ and $\xi$ is an isomorphism in $D^-_{\mathrm{f\/in}}(R\Lambda)$. This follows
since having finite pseudocompact $R$-tor dimension is an invariant of isomorphisms in
$D^-(R\Lambda)$, and since such isomorphisms induce isomorphisms between
the cohomology groups. In particular, if $A^\bullet\to B^\bullet$ is a quasi-isomorphism
in $K^-(R\Lambda)$ and one of $A^\bullet$ or $B^\bullet$ is an object in $K^{-}_{\mathrm{f\/in}}(R\Lambda)$,
then so is the other.
\item[(iii)]
Let $N^\bullet,N_1^\bullet,N_2^\bullet$ be complexes in $D^-_{\mathrm{f\/in}}(R\Lambda)$ 
such that \textit{all their terms have finite $k$-length}, and let $g:N_1^\bullet\to N_2^\bullet$ be a
morphism in $D^{-}_{\mathrm{f\/in}}(R\Lambda)$. By Remark \ref{rem:pseudocompact}(ii), and 
since  $R\Lambda$ is Artinian,
there exist  bounded above complexes $M^\bullet$, $M_1^\bullet$ and 
$M_2^\bullet$ of abstractly free finitely generated $R\Lambda$-modules such that there
are isomorphisms $\beta:N^\bullet\to M^\bullet$ and
$\beta_i: N_i^\bullet\to M_i^\bullet$ in $D^{-}_{\mathrm{f\/in}}(R\Lambda)$ ($i=1,2$). 
Then $f=\beta_2\,g\,\beta_1^{-1}$ is a 
morphism $f:M_1^\bullet \to M_2^\bullet$ in $D^{-}_{\mathrm{f\/in}}(R\Lambda)$.
Let $\mathcal{P}_R$ be the additive subcategory of $\mathrm{PCMod}(R\Lambda)$ of 
projective objects.
By  \cite[Thm. 10.4.8]{Weibel}, the natural functor $K^-(\mathcal{P}_R)
\to D^-(R\Lambda)$ is an equivalence of categories. Hence $f$
can be taken to be a morphism in $K^{-}_{\mathrm{f\/in}}(R\Lambda)$. 
\end{enumerate}
\end{rem}

The following two results, Lemmas \ref{lem:corollary3.6} and \ref{lem:lemma3.8},
establish key properties of objects and morphisms in $D^{-}_{\mathrm{f\/in}}(R\Lambda)$.
They replace \cite[Cor. 3.6 and Lemma 3.8]{bcderived} in our situation. Since
$\Lambda$ is Artinian, some of the statements can be simplified.

\begin{lemma}
\label{lem:corollary3.6}
Suppose $R \in \mathrm{Ob}(\mathcal {C})$ is Artinian, and $N^\bullet$, 
$N_1^\bullet$ and $N_2^\bullet$ are objects in $D^{-}_{\mathrm{f\/in}}(R\Lambda)$. 
Let $g:N_1^\bullet \to 
N_2^\bullet$ be a morphism in $D^{-}_{\mathrm{f\/in}}(R\Lambda)$. 
\begin{enumerate}
\item[(i)]  There exists a bounded above complex $M^\bullet$ of abstractly free finitely generated 
$R\Lambda$-modules, and an isomorphism $\beta:N^\bullet \to M^\bullet$ in 
$D^{-}_{\mathrm{f\/in}}(R\Lambda)$.

\item[(ii)]  There exist bounded above complexes $M_1^\bullet$ and $M_2^\bullet$ of 
abstractly free finitely generated $R\Lambda$-modules, a morphism
$f:M_1^\bullet \to M_2^\bullet$ in $K^-_{\mathrm{f\/in}}(R\Lambda)$, and
isomorphisms $\beta_i:N_i^\bullet \to M_i^\bullet$ in
$D^{-}_{\mathrm{f\/in}}(R\Lambda)$ $(i=1,2)$ such that $f = \beta_{2}\, g \,
\beta_1^{-1}$ as morphisms in $D^{-}_{\mathrm{f\/in}}(R\Lambda)$. 
\end{enumerate}
\end{lemma}

\begin{proof}
In view of Remark \ref{rem:whatever}(iii), the main ingredient in the proof is the following claim,
which is proved using similar arguments  as
in the proof of \cite[Lemma 3.4(i)]{bcderived}.

\medskip

\noindent
\textsc{Claim 1.} 
Suppose  $N^\bullet$ is an object in  $D^{-}_{\mathrm{f\/in}}(R\Lambda)$ satisfying
${\mathrm{H}}^j(N^\bullet)=0$ for $j<n$. Then there exists an exact sequence of complexes
\begin{equation}
\label{eq:cutting}
0 \to U^\bullet \xrightarrow{\iota} N^\bullet \to  {N'}^\bullet \to 0
\end{equation}
in $C^{-}_{\mathrm{f\/in}}(R\Lambda)$ such that $U^\bullet$ is acyclic, and such that the terms of ${N'}^\bullet$
have finite $k$-length and satisfy ${N'}^j=0$ for $j<n$. 

\medskip

Part (i) of Lemma \ref{lem:corollary3.6} now follows from Claim 1 and Remark \ref{rem:whatever}(iii).
To prove part(ii) of Lemma \ref{lem:corollary3.6}, we use Claim 1 to see that
there exist bounded complexes
${N_1'}^\bullet$ and ${N_2'}^\bullet$ such that all their terms have finite $k$-length, together
with quasi-isomorphisms $\gamma_i:N_i^\bullet \to {N'_i}^\bullet$ in $C^{-}_{\mathrm{f\/in}}(R\Lambda)$
($i=1,2$) that are surjective on terms. Let $g'=\gamma_2 \,g \, \gamma_1^{-1}:{N'_1}^\bullet
\to {N'_2}^\bullet$, so $g'$ is a morphism in $D^{-}_{\mathrm{f\/in}}(R\Lambda)$.
Using Remark \ref{rem:whatever}(iii),
there exist bounded above complexes $M_1^\bullet$ and $M_2^\bullet$ of 
abstractly free finitely generated $R\Lambda$-modules, a morphism
$f:M_1^\bullet \to M_2^\bullet$ in $K^-_{\mathrm{f\/in}}(R\Lambda)$, and
isomorphisms $\beta'_i:{N_i'}^\bullet \to M_i^\bullet$ in
$D^{-}_{\mathrm{f\/in}}(R\Lambda)$ $(i=1,2)$ such that $f = \beta'_{2}\, g'\,
{\beta'_1}^{-1}$ as morphisms in $D^{-}_{\mathrm{f\/in}}(R\Lambda)$. Letting $\beta_i=\beta_i'\circ\gamma_i$
($i=1,2$), part (ii) follows.
\end{proof}

\begin{dfn}
\label{def:Definition3.7}
In the situation of Lemma \ref{lem:corollary3.6}(i), we say \emph{we can replace $N^\bullet$
by  $M^\bullet$}. In the situation of Lemma \ref{lem:corollary3.6}(ii), we say
\emph{we can replace $N_i^\bullet$ by $M_i^\bullet$ $(i=1,2)$, and 
$g$ by $f$}.
\end{dfn}

\begin{lemma}
\label{lem:lemma3.8}
Suppose $M^\bullet$ is an object in $D^{-}_{\mathrm{f\/in}}(R\Lambda)$ such that ${\mathrm{H}}^j(M^\bullet)=0$
for $j<n$. Then $M^\bullet$ has finite pseudocompact $R$-tor dimension at $n$.
\end{lemma}

\begin{proof}
By Lemma \ref{lem:corollary3.6}(i), we may assume that $M^\bullet$ is a bounded above
complex of abstractly free finitely generated $R\Lambda$-modules. Hence all terms
of $M^\bullet$ are abstractly free finitely generated $R$-modules.
By Remark \ref{rem:dumbdumb}, there exists an integer $n_1\leq n$ such that $M^{n_1}/
\delta^{n_1-1}(M^{n_1-1})$ is a topologically free pseudocompact  $R$-module. Since $M^{n_1}$
is a finitely generated $R$-module, it follows that this is an abstractly free finitely generated $R$-module. 
To prove Lemma \ref{lem:lemma3.8}, it is enough to show that  $M^{n}/\delta^{n-1}(M^{n-1})$ is an 
abstractly free finitely generated $R$-module. 
This is proved exactly in the same way as in the proof of 
\cite[Lemma 3.8]{bcderived}.
\end{proof}

The following remark replaces  \cite[Cors. 3.10 and 3.11]{bcderived} in our situation. Note that since
$\Lambda$ is already Artinian, we do not need to deal with quotient algebras of $\Lambda$.

\begin{rem}
\label{rem:corollary3.1011}
Suppose $R,S,T\in\mathrm{Ob}(\mathcal{C})$ are Artinian objects with morphisms 
$R\xrightarrow{\alpha} T \xleftarrow{\beta} S$ in $\mathcal{C}$. 
Let $X^\bullet$ be an object in $D^{-}_{\mathrm{f\/in}}(R\Lambda)$, and let $Z^\bullet$ be an object in
$D^{-}_{\mathrm{f\/in}}(S\Lambda)$. 
Suppose $\tau:T\hat{\otimes}^{\mathbf{L}}_{S} Z^\bullet \to T\hat{\otimes}^{\mathbf{L}}_{R} 
X^\bullet$ is a morphism in $D^-_{\mathrm{f\/in}}(T\Lambda)$. By Remark \ref{rem:leftderivedtensor},
$$ T\hat{\otimes}^{\mathbf{L}}_{R} X^\bullet= T\hat{\otimes}_R\,\sigma_R(X^\bullet)
\quad\mbox{and}\quad
T\hat{\otimes}^{\mathbf{L}}_{S} Z^\bullet =T\hat{\otimes}_S\,\sigma_S(Z^\bullet)$$
where $\sigma_R(X^\bullet)$ (resp. $\sigma_S(Z^\bullet)$) is an object of 
$\mathcal{P}_R$ (resp. $\mathcal{P}_S$) and there exists an isomorphism
$\rho_X:X^\bullet\to \sigma_R(X^\bullet)$ in $D^-(R\Lambda)$
(resp. $\rho_Z:Z^\bullet\to\sigma_S(Z^\bullet)$ in $D^-(S\Lambda)$).
By Remark \ref{rem:whatever}(ii), $\sigma_R(X^\bullet)$ and $\rho_X$ (resp. $\sigma_S(Z^\bullet)$
and $\rho_Z$) are in $D^-_{\mathrm{f\/in}}(R\Lambda)$ (resp. $D^-_{\mathrm{f\/in}}(S\Lambda)$).

By Lemma \ref{lem:corollary3.6}(i), there exists a bounded above complex $\tilde{X}^\bullet$
(resp. $\tilde{Z}^\bullet$) of abstractly free finitely generated $R\Lambda$-modules
(resp. $S\Lambda$-modules) and an isomorphism 
$\tilde{\beta}: \tilde{X}^\bullet\to X^\bullet$ in $D^{-}_{\mathrm{f\/in}}(R\Lambda)$
(resp. $\tilde{\gamma}: \tilde{Z}^\bullet\to Z^\bullet$ in $D^{-}_{\mathrm{f\/in}}(S\Lambda)$). 
Let $\beta=\rho_X\circ\tilde{\beta}$ and $\gamma=\rho_Z\circ\tilde{\gamma}$.
Then $\beta:\tilde{X}^\bullet\to \sigma_R(X^\bullet)$ (resp. 
$\gamma:\tilde{Z}^\bullet\to\sigma_S(Z^\bullet)$) is an isomorphism in $D^{-}_{\mathrm{f\/in}}(R\Lambda)$
(resp. $D^{-}_{\mathrm{f\/in}}(S\Lambda)$). Since $\tilde{X}^\bullet,\sigma_R(X^\bullet)$ (resp.
$\tilde{Z}^\bullet,\sigma_S(Z^\bullet)$) are objects in $K^-(\mathcal{P}_R)$ (resp. $K^-(\mathcal{P}_S)$),
$\beta$ (resp. $\gamma$) can be taken to be a morphism in $K^{-}_{\mathrm{f\/in}}(R\Lambda)$
(resp. $K^{-}_{\mathrm{f\/in}}(S\Lambda)$). Moreover, $T\hat{\otimes}_R\beta=T\hat{\otimes}^{\mathbf{L}}_{R} \beta$ 
(resp. $T\hat{\otimes}_S\gamma=T\hat{\otimes}^{\mathbf{L}}_{S} \gamma$)
is an isomorphism in $K^{-}_{\mathrm{f\/in}}(T\Lambda)$, and $\tau:T\hat{\otimes}_S\,\sigma_S(Z^\bullet)
\to T\hat{\otimes}_R\,\sigma_R(X^\bullet)$ can be taken to be a morphism in $K^{-}_{\mathrm{f\/in}}(T\Lambda)$.
Define
\begin{equation}
\label{eq:replace}
\tilde{\tau}=(T\hat{\otimes}_R\beta)^{-1}\circ\tau\circ(T\hat{\otimes}_S\gamma):
T\hat{\otimes}_S\tilde{Z}^\bullet \to T\hat{\otimes}_R\tilde{X}^\bullet \,.
\end{equation}
Then we can replace $X^\bullet$ (resp. $Z^\bullet$) by $\tilde{X}^\bullet$ (resp. $\tilde{Z}^\bullet$),
in the sense of Definition \ref{def:Definition3.7}, and we can replace $\tau$ by $\tilde{\tau}$.
Note that $\beta$ (resp. $\gamma$) only depends on $\tilde{X}^\bullet$ and 
$\tilde{\beta}$ (resp. $\tilde{Z}^\bullet$ and $\tilde{\gamma}$).
\end{rem}

Using Remark \ref{rem:profree}(i), the following result is proved in a similar way to \cite[Lemma 3.1]{bcderived}.

\begin{lemma}
\label{lem:lemma3.1}
Suppose $(M^\bullet,\phi)$ is a quasi-lift of $V^\bullet$ over some Artinian object
$R \in \mathrm{Ob}(\mathcal{C})$. Then $M^\bullet$ is an object of $D^{-}_{\mathrm{f\/in}}(R\Lambda)$. 
More precisely, ${\mathrm{H}}^i(M^\bullet)$ is a subquotient of  an abstractly
free $R$-module of rank $d_i=\mathrm{dim}_k\,{\mathrm{H}}^i(V^\bullet)$ for all $i$.
\end{lemma}

The following result  summarizes the main properties of quasi-lifts and proflat quasi-lifts
of $V^\bullet$ over arbitrary objects $R$ in $\hat{\mathcal{C}}$. 
The proof is very similar to the proof of \cite[Thm. 2.10]{obstructions},
once we replace the results from \cite{bcderived} by the corresponding results stated above.
For the convenience of the reader, we provide some of the details.

\begin{thm} 
\label{thm:derivedresult}  
Suppose that ${\mathrm{H}}^i(V^\bullet) = 0$ unless $n_1 \le i \le n_2$.  Every quasi-lift of $V^\bullet$ over 
an object $R$ of $\hat{\mathcal{C}}$ is isomorphic to a quasi-lift $(P^\bullet, \psi)$ for
a complex $P^\bullet$ with the following properties:
\begin{enumerate}
\item[(i)] The terms of $P^\bullet$ are topologically free $R\Lambda$-modules.
\item[(ii)] The cohomology group ${\mathrm{H}}^i(P^\bullet)$ is finitely generated 
as an abstract $R$-module for all $i$, and ${\mathrm{H}}^i(P^\bullet) = 0$ unless $n_1 \le i \le n_2$. 
\item[(iii)]   One has ${\mathrm{H}}^i(S\hat{\otimes}^{\mathbf{L}}_RP^\bullet)=0$  for all pseudocompact $R$-modules 
$S$ unless $n_1 \le i \le n_2$.
\end{enumerate}
\end{thm}

\begin{proof}
Let $R$ be an object of $\hat{\mathcal{C}}$, and let $(M^\bullet,\phi)$ be a quasi-lift of $V^\bullet$ over $R$.
By Remark \ref{rem:profree}(i), it follows that there exists a quasi-lift $(P^\bullet,\psi)$ of $V^\bullet$
over $R$ which is isomorphic to the quasi-lift $(M^\bullet,\phi)$ and which satisfies property (i).
It remains to verify properties (ii) and (iii). By (i), we can assume that the terms of $P^\bullet$ 
are topologically free pseudocompact $R\Lambda$-modules. In particular, the functors 
$-\hat{\otimes}^{\mathbf{L}}_RP^\bullet$ and $-\hat{\otimes}_RP^\bullet$ are
naturally isomorphic. Let $m_R$ denote the maximal ideal of $R$, and let $n$ be an
arbitrary positive integer.
By Lemmas \ref{lem:lemma3.8} and \ref{lem:lemma3.1}, ${\mathrm{H}}^i((R/m_R^n)\hat{\otimes}_RP^\bullet)=0$ for $i>n_2$ 
and $i<n_1$. Moreover, for $n_1\le i\le n_2$,
${\mathrm{H}}^i((R/m_R^n)\hat{\otimes}_RP^\bullet)$ is a subquotient of  an abstractly free $(R/m_R^n)$-module
of rank $d_i=\mathrm{dim}_k\,{\mathrm{H}}^i(V^\bullet)$,
and $(R/m_R^n)\hat{\otimes}_RP^\bullet$ has finite pseudocompact $(R/m_R^n)$-tor dimension at
$N=n_1$. Since 
$P^\bullet\cong \displaystyle \lim_{\stackrel{\longleftarrow}{n}}\, (R/m_R^n)\hat{\otimes}_RP^\bullet$ 
and since by Remark \ref{rem:pseudocompact}(i), the category $\mathrm{PCMod}(R)$ has 
exact inverse limits, it  follows that for all pseudocompact $R$-modules $S$
$${\mathrm{H}}^i(S\hat{\otimes}_RP^\bullet)= \lim_{\stackrel{\longleftarrow}{n}}\,{\mathrm{H}}^i\left(
(S/m_R^nS)\hat{\otimes}_{R/m_R^n}\left((R/m_R^n)\hat{\otimes}_R P^\bullet\right)\right)$$ 
for all $i$. Hence Theorem \ref{thm:derivedresult} follows.
\end{proof}

\subsection{More results on complexes and quasi-lifts}
\label{s:prelims}

In this subsection, we first provide some results from Milne \cite{milne}, which
are adapted from \cite[Sect. 14]{bcderived} to our situation.
Note that in  \cite[Lemma VI.8.17]{milne} (resp.
\cite[Lemma VI.8.18]{milne}), the condition ``$\pi$ is surjective on
terms'' (resp. ``$\psi$ is surjective on terms'') is necessary in the statement.

\begin{lemma}
\label{lem:Lemma14.1}
{\rm (\cite[Lemma VI.8.17]{milne})}
Let $R\in\mathrm{Ob}(\hat{\mathcal{C}})$, and
let $M^\bullet \xrightarrow{\phi} L^\bullet \xleftarrow{\pi} N^\bullet$
be morphisms in $C^-(R\Lambda)$ such
that $\pi$ is a quasi-isomorphism which is surjective on terms.
If $M^\bullet$ is a complex of topologically free pseudocompact $R\Lambda$-modules,
there exists a morphism $\psi:M^\bullet\to N^\bullet$ in $C^-(R\Lambda)$ such that $\pi\psi=\phi$.
\end{lemma}

\begin{rem}
\label{rem:Remark14.2}
Suppose $R,R_0\in\mathrm{Ob}(\hat{\mathcal{C}})$
so that $R_0$ is a quotient ring
of $R$.  We write $X\to X_0$, $\phi\to \phi_0$ for the functor
$R_0\hat{\otimes}_R -$.
Suppose $M$, $N$ are
topologically free pseudocompact $R\Lambda$-modules.
Then every continuous $R_0\Lambda$-module homomorphism $\pi:M_0\to N_0$
can be lifted to a continuous $R\Lambda$-module homomorphism $\phi:M\to N$
so that $\pi=\phi_0$.
\end{rem}

\begin{lemma}
\label{lem:Lemma14.3}
{\rm (\cite[Sublemma VI.8.20]{milne})}
Let $R,R_0\in\mathrm{Ob}(\hat{\mathcal{C}})$ so that
$R_0$ is a quotient ring of $R$. As in Remark $\ref{rem:Remark14.2}$,
we write $X\to X_0$, $\phi\to
\phi_0$ for the functor $R_0\hat{\otimes}_R -$.
Let $\phi:L^\bullet\to M^\bullet$ be a
morphism in $C^-(R\Lambda)$ of complexes of topologically free pseudocompact
$R\Lambda$-modules. Then any morphism $L_0^\bullet\to M_0^\bullet$ 
in $C^-(R_0\Lambda)$ that is homotopic to
$\phi_0$ is of the form $\psi_0$, where $\psi:L^\bullet\to M^\bullet$ is a morphism
in $C^-(R\Lambda)$ which is homotopic to $\phi$.
\end{lemma}

\begin{lemma}
\label{lem:Lemma14.4}
{\rm (\cite[Lemma VI.8.18]{milne})}
Let $R,R_0\in\mathrm{Ob}(\mathcal{C})$ be Artinian so that $R_0$ is a quotient ring of $R$.
As in Remark $\ref{rem:Remark14.2}$, we write $X\to X_0$, $\phi\to
\phi_0$ for the functor $R_0\hat{\otimes}_R -$. Let $M^\bullet$ $($resp. $N^\bullet${}$)$ be
a {bounded above} complex of abstractly free finitely generated  $R\Lambda$-modules $($resp.
$R_0\Lambda$-modules$)$, and let $\psi$ be
a quasi-isomorphism $\psi:M_0^\bullet {\rightarrow}N^\bullet$ in $C^-(R_0\Lambda)$
which is surjective on terms. Then there
exist a {bounded above} complex $L^\bullet$ of {abstractly free finitely generated}
$R\Lambda$-modules,
a quasi-isomorphism
$\phi:M^\bullet\to L^\bullet$ in $C^-(R\Lambda)$, and an isomorphism 
$\rho:L^\bullet_0 {\rightarrow}
N^\bullet$ in $C^-(R_0\Lambda)$, such that $\rho\, \phi_0=\psi$.
\end{lemma}

The following remark (which replaces \cite[Remark 5.2]{bcderived} in our situation)
shows how one can relate a morphism $f$ in $C^-(R\Lambda)$ to 
a morphism $g$ in $C^-(R\Lambda)$ that is surjective on terms.

\begin{rem}
\label{rem:Remark5.2}
Suppose $R\in\mathrm{Ob}(\hat{\mathcal{C}})$. Let $M^\bullet$ and $N^\bullet$ be
two bounded above complexes of pseudocompact $R\Lambda$-modules, and
let $f:M^\bullet\to N^\bullet$ be a morphism in $C^-(R\Lambda)$. 
Let $P^\bullet$ be a bounded above complex  of topologically free pseudocompact
$R\Lambda$-modules such that there is a quasi-isomorphism
$P^\bullet \to N^\bullet$ in $C^-(R\Lambda)$
which is surjective on terms. Then
the mapping cone $C^\bullet$ of  $P^\bullet[-1] \xrightarrow{\mathrm{id}} P^\bullet[-1]$
is an acyclic complex, and there is a morphism $\pi: C^\bullet\to N^\bullet$ 
in $C^-(R\Lambda)$ which is
surjective on terms. Define $g:M^\bullet\oplus C^\bullet\to N^\bullet$ by $g=(f,\pi)$, and 
define $s:M^\bullet\to M^\bullet\oplus C^\bullet$ by $s=\left(\begin{array}{c}\mathrm{id}_{
M^\bullet}\\0
\end{array}\right)$. Then $g$ is surjective on terms, $s$ is a quasi-isomorphism and $g\,s=f$.

Suppose there is a surjective morphism $R_1\to R$ in $\hat{\mathcal{C}}$, 
and there is a bounded above complex
$X^\bullet$ of topologically free pseudocompact $R_1\Lambda$-modules such that $M^\bullet=
R\hat{\otimes}_{R_1}X^\bullet$. Since $R\Lambda=R\hat{\otimes}_{R_1}R_1\Lambda$, there exists
a bounded above complex $Q^\bullet$ of topologically free pseudocompact
$R_1\Lambda$-modules with
$P^\bullet=R\hat{\otimes}_{R_1}Q^\bullet$. Hence $C^\bullet=R\hat{\otimes}_{R_1}D^\bullet$,
where $D^\bullet$ is the mapping cone of $Q^\bullet[-1] \xrightarrow{\mathrm{id}} Q^\bullet[-1]$, and $M^\bullet\oplus C^\bullet=R\hat{\otimes}_{R_1}(X^\bullet\oplus Q^\bullet)$.
\end{rem}

Next, we look at quasi-lifts of $V^\bullet$ in the case when  the endomorphism ring of $V^\bullet$ 
in $D^-(\Lambda)$ is isomorphic to $k$. We  need the following remark.

\begin{rem}
\label{rem:Lemma4.2}
Define $\hat{F}_1$ (resp. $F_1$) to be the functor from
$\hat{\mathcal{C}}$ (resp. $\mathcal{C}$) to the category $\mathrm{Sets}$
which sends $R$ to the set $\hat{F}_1(R)$ (resp.
$F_1(R)$) of isomorphism classes in $D^-(R\Lambda)$ of quasi-lifts of $V^\bullet$
which are bounded  above complexes of topologically free pseudocompact $R\Lambda$-modules.

Using Remark \ref{rem:profree}(i), it follows as in the proof of \cite[Lemma 4.2]{bcderived} that the 
natural transformation $\hat{F}_1\to \hat{F}$ $($resp.
$F_1 \to F${}$)$ is an isomorphism of functors.
\end{rem}

Using Remark \ref{rem:Lemma4.2} instead of \cite[Lemma 4.2]{bcderived} and
Lemma \ref{lem:Lemma14.3} instead of \cite[Lemma 14.3]{bcderived}, the following result 
is proved in a similar way to
\cite[Prop. 4.3]{bcderived}.

\begin{prop}
\label{prop:liftendos}
Suppose $\mathrm{Hom}_{D^-(\Lambda)}(V^\bullet,V^\bullet)= k$.
Then $\mathrm{Hom}_{D^-(R\Lambda)}(M^\bullet,M^\bullet)= R$
for every quasi-lift $(M^\bullet,\phi)$ of $V^\bullet$ over an Artinian object 
$R\in\mathrm{Ob}(\mathcal{C})$.
\end{prop}

\subsection{Proof of Theorem \ref{thm:bigthm}}
\label{s:proofbigthm}

In this subsection, we prove Theorem \ref{thm:bigthm}. We follow the argumentation in
Sections 5 through 7 in \cite{bcderived} and explain the key steps.

\medskip

\noindent{\sc Step 1}. \textit{Schlessinger's criteria $(H1)$ and $(H2)$
$($see \cite[Thm. 2.11]{Sch}$)$
are always satisfied for $F_{\mathcal{D}}$. In case
$\mathrm{Hom}_{D^-([[kG]])}(V^\bullet,V^\bullet)= k$,
$(H4)$ is also satisfied.}

\medskip

\noindent\textit{Proof of Step $1$.} This is proved using the same arguments as in the proof of \cite[Prop. 5.1]{bcderived},
where we replace the results from \cite[Sects. 3, 4 and 14]{bcderived} by the results proved in Sections
\ref{s:quasilifts} and \ref{s:prelims}. More precisely, 
Remark \ref{rem:corollary3.1011} replaces \cite[Cor. 3.10]{bcderived},
Remark \ref{rem:Remark5.2} replaces \cite[Remark 5.2]{bcderived},
Lemma \ref{lem:Lemma14.4} replaces \cite[Lemma 14.4]{bcderived},
Remark \ref{rem:profree}(ii) replaces \cite[Lemma 2.11]{bcderived},
Remark \ref{rem:functor} replaces \cite[Prop. 2.12]{bcderived},
Proposition \ref{prop:liftendos} replaces \cite[Prop. 4.3]{bcderived}, and
Lemma \ref{lem:Lemma14.3} replaces \cite[Lemma 14.3]{bcderived}.

\medskip

\noindent{\sc Step 2}. \textit{There is a $k$-vector space isomorphism 
$$h: \;t_F = F(k[\varepsilon]) \longrightarrow \mathrm{Hom}_{D^-(\Lambda)}(V^\bullet,V^\bullet[1])=
\mathrm{Ext}^1_{D^-(\Lambda)}(V^\bullet, V^\bullet)\,.$$
Moreover, composing the natural injection $t_{F^{\mathrm{f\/l}}}\to t_F$ with $h$, we obtain an isomorphism between 
$t_{F^{\mathrm{f\/l}}}$ and the kernel of the natural forgetful map
\begin{equation}
\label{eq:forget}
\mathrm{Ext}^1_{D^-(\Lambda)}(V^\bullet,V^\bullet)
\to \mathrm{Ext}^1_{D^-(k)}(V^\bullet,V^\bullet)\,.
\end{equation}}

\medskip

\noindent\textit{Proof of Step $2$.} This is proved using the same arguments as in the proof of 
\cite[Lemmas 6.1 and 6.3]{bcderived}, where 
Remark \ref{rem:functor} replaces \cite[Prop. 2.12]{bcderived}, 
Remark \ref{rem:Lemma4.2} replaces \cite[Lemma 4.2]{bcderived}, and
Lemma \ref{lem:Lemma14.1} replaces \cite[Lemma 14.1]{bcderived}.
Note that in \cite{bcderived}  the word ``triangle'' stands for ``distinguished triangle".

\medskip

\noindent{\sc Step 3}. \textit{Schlessinger's criterion $(H3)$ is satisfied, i.e.
the $k$-dimension of the tangent space $t_{F_{\mathcal{D}}}$ is finite.}

\medskip

\noindent\textit{Proof of Step $3$.} This is proved using the same arguments as in the proof of 
\cite[Prop. 6.4]{bcderived}, except that we replace the assumption in \cite{bcderived} 
that $G$ has finite pseudocompact cohomology by the following fact. If $M_1,M_2$ are 
pseudocompact $\Lambda$-modules that are finite dimensional over $k$, then
$\mathrm{Ext}^j_\Lambda(M_1,M_2)$ is finite dimensional over $k$. Note that
by Remark \ref{rem:pseudocompact}(iii), $\mathrm{Ext}^j_\Lambda(M_1,M_2)$ is computed by using
a projective resolution of $M_1$ in $\mathrm{PCMod}(\Lambda)$. 
Since $\mathrm{dim}_k\,M_1$ is finite, there exists a resolution of
$M_1$ in $\mathrm{PCMod}(\Lambda)$ consisting of abstractly free finitely generated
$\Lambda$-modules. In other words, $\mathrm{Ext}^j_\Lambda(M_1,M_2)$ can be identified
with the corresponding Ext group in the category of finitely generated $\Lambda$-modules.

\medskip

\noindent{\sc Step 4}. \textit{The functor $\hat{F}_{\mathcal{D}}:\hat{\mathcal{C}}\to \mathrm{Sets}$
is continuous. In other words, for all objects $R$ in $\hat{\mathcal{C}}$
with maximal ideal $m_R$ we have
$$\hat{F}_{\mathcal{D}}(R)=\lim_{\stackrel{\longleftarrow}{i}} \hat{F}_{\mathcal{D}}(R/m_R^i)\, .$$
}

\medskip

\noindent\textit{Proof of Step $4$.} 
By Remark \ref{rem:functor} and Lemma \ref{lem:corollary3.6}(i),
we may assume, without loss of generality,
that $V^\bullet$ is a bounded above complex of abstractly free finitely generated 
$\Lambda$-modules.
Let $R$ be an object in $\hat{\mathcal{C}}$ with maximal ideal $m_R$.
We consider the natural map
\begin{equation}
\label{eq:themap0}
\Xi_{\mathcal{D}}: \hat{F}_{\mathcal{D}}(R) \to \lim_{\stackrel{\longleftarrow}{i}}
\hat{F}_{\mathcal{D}}(R/m_R^i)
\end{equation}
defined by $\Xi_{\mathcal{D}}((M^\bullet,\phi))= \{(\,(R/m_R^i) \hat{\otimes}^{\mathbf{L}}_R
M^\bullet,(R/m_R^i) \hat{\otimes}^{\mathbf{L}}_R\phi \;)\}_{i=1}^\infty$.

We prove that $\Xi_{\mathcal{D}}$ is surjective using the same arguments as in the proof of 
\cite[Prop. 7.2]{bcderived}, where
Remark \ref{rem:corollary3.1011} replaces \cite[Cor. 3.11]{bcderived},
Remark \ref{rem:Remark5.2} replaces \cite[Remark 5.2]{bcderived},
Remark \ref{rem:extrafree} replaces \cite[Remark 2.3]{bcderived}, 
Lemma \ref{lem:lemma3.1} replaces \cite[Lemma 3.1]{bcderived}, 
Lemma \ref{lem:lemma3.8} replaces \cite[Lemma 3.8]{bcderived},
Remark \ref{rem:pseudocompact}(i) replaces \cite[Remark 2.2(i)]{bcderived}, and
Remark \ref{rem:profree}(ii) replaces \cite[Lemma 2.11]{bcderived}.

To prove that $\Xi_{\mathcal{D}}$ is injective, we notice that in the proof of 
\cite[Prop. 7.2]{bcderived} an assumption was made to arrive at a morphism 
$f_i$ as in \cite[Eq. (7.5)]{bcderived}, which needs more explanation. We will provide the
necessary arguments in Appendix \ref{s:continuity}. The proof of the injectivity
of $\Xi_{\mathcal{D}}$ in our situation follows then using the same arguments as in the proof of 
Proposition \ref{prop:continuityagain}, where
Remark \ref{rem:functor} replaces \cite[Prop. 2.12]{bcderived},
Lemma \ref{lem:corollary3.6}(i) replaces \cite[Corollary 3.6(i)]{bcderived},
Remark \ref{rem:corollary3.1011} replaces \cite[Cor. 3.11]{bcderived},
Remark \ref{rem:Remark5.2} replaces \cite[Remark 5.2]{bcderived},
Remark \ref{rem:extrafree} replaces \cite[Remark 2.3]{bcderived},
Lemma \ref{lem:lemma3.1} replaces \cite[Lemma 3.1]{bcderived},
Lemma \ref{lem:lemma3.8} replaces \cite[Lemma 3.8]{bcderived},
Remark \ref{rem:pseudocompact}(i) replaces \cite[Remark 2.2(i)]{bcderived},
Remark \ref{rem:profree}(ii) replaces \cite[Lemma 2.11]{bcderived}, 
Remark \ref{rem:Lemma4.2} replaces \cite[Lemma 4.2]{bcderived},
Lemma \ref{lem:Lemma14.1} replaces \cite[Lemma 14.1]{bcderived}, and
Lemma \ref{lem:Lemma14.3} replaces \cite[Lemma 14.3]{bcderived}.

\medskip

This completes the proof of Theorem \ref{thm:bigthm}.

\subsection{Special complexes $V^\bullet$}
\label{s:12terms}

In this section we consider several cases of more special complexes $V^\bullet$. Namely,
we consider one-term and two-term complexes, and completely split complexes.
The results are adapted from \cite[Sects. 9 and 11]{bcderived}. 

\begin{prop}
\label{prop:modulecase}
Assume Hypothesis $\ref{hypo:fincoh}$, and let ${F}_{\mathcal{D}}$ and
$\hat{F}_{\mathcal{D}}$ be as in Definition $\ref{def:functordef}$.
Suppose $V^\bullet$ has exactly one non-zero cohomology group $C$, which has finite 
$k$-dimension. Then
$R(\Lambda,V^\bullet)$ coincides with the versal deformation ring $R(\Lambda,C)$ considered
in \cite{blehervelez}; in particular, $R(\Lambda,V^\bullet)=R^{\mathrm{f\/l}}(\Lambda,V^\bullet)$.
The groups $\mathrm{Hom}_{D^-(\Lambda)}(V^\bullet,V^\bullet)$
and  $\mathrm{Hom}_{\Lambda}(C,C)$ are isomorphic.
\end{prop}

\begin{proof}
This is proved using the same arguments as in the proof of 
\cite[Prop. 9.1]{bcderived}, where
Lemma \ref{lem:lemma3.1} replaces \cite[Lemma 3.1]{bcderived},
Lemma \ref{lem:lemma3.8} replaces \cite[Lemma 3.8]{bcderived}, 
Remark \ref{rem:dumbdumb} replaces \cite[Remark 2.6]{bcderived}, and
the deformation functor $\hat{F}_C$ considered in \cite{blehervelez} replaces Mazur's deformation functor
considered in \cite{Maz}.
\end{proof}

\begin{rem}
\label{rem:2term}
Suppose $V^\bullet$ has precisely two non-zero cohomology groups. Without
loss of generality, we can assume these groups are $U_0={\mathrm{H}}^0(V^\bullet)$ and
$U_{-n}={\mathrm{H}}^{-n}(V^\bullet)$ for some $n > 0$,  both of finite $k$-dimension by Hypothesis
\ref{hypo:fincoh}. 
We will also regard $U_0,U_{-n}$ as complexes concentrated in dimension $0$. 
In particular, we obtain a distinguished triangle  in $D^-(\Lambda)$ of the form
\begin{equation}
\label{eq:anotherone}
U_{-n}[n] \xrightarrow{\iota} V^\bullet \xrightarrow{\pi} U_0
\xrightarrow{\beta} U_{-n}[n+1]\;.
\end{equation}
By the triangle axioms, there exists a complex $C(\beta)^\bullet$
in $D^-(\Lambda)$, which is unique up to isomorphism, such that 
$$U_0 \xrightarrow{\beta}  U_{-n}[n+1] \to C(\beta)^\bullet \to U_0[1]$$
is a distinguished triangle in $D^-(\Lambda)$. In other words, 
$V^\bullet \cong C(\beta)^\bullet[-1]$ in $D^-(\Lambda)$.

The statements and proofs of 
\cite[Prop. 9.3, Cor. 9.4, Props. 9.6 and 9.7]{bcderived} can be adapted to our situation. Here 
is a summary of some of the main results:

\begin{enumerate}
\item[(i)] If
the endomorphism rings of $U_0$ and $U_{-n}$ are both given
by scalars, then $\mathrm{Hom}_{D^-(\Lambda)}(V^\bullet,V^\bullet)= k$
if and only if
\begin{enumerate}
\item[(a)] $\mathrm{Ext}^n_{\Lambda}(U_0,U_{-n})=0$, and
\item[(b)] there exists a nontrivial element
$\beta\in\mathrm{Ext}^{n+1}_{\Lambda}(U_0,U_{-n})\cong
\mathrm{Hom}_{D^-(\Lambda)}(U_0,U_{-n}[n+1])$
such that $V^\bullet\cong C(\beta)^\bullet[-1]$
in $D^-(\Lambda)$.
\end{enumerate}

\item[(ii)]
We have the following description of the tangent space of the proflat deformation
functor $\hat{F}^{\mathrm{f\/l}}$ using Remark \ref{rem:bigthm}(i):

If $n\geq 2$, then $t_{F^{\mathrm{f\/l}}}=t_F$.
If $n=1$, then $t_{F^{\mathrm{f\/l}}}$ is the subspace of
$t_F=\mathrm{Ext}^1_{D^-(\Lambda)}(V^\bullet,V^\bullet)=
\mathrm{Hom}_{D^-(\Lambda)}(V^\bullet,V^\bullet[1])$
consisting of those elements $f\in t_F$ satisfying
$\pi[1]\circ f\circ \iota=0$ in $\mathrm{Hom}_{D^-(\Lambda)}(U_{-1}[1],U_0[1])$.

\item[(iii)]
Suppose that $U_{-n}$ and $U_0$ have universal deformation rings $R_{-n}$ and $R_0$
and universal lifts $(X_{-n},\psi_{-n})$ and $(X_0,\psi_0)$, respectively,
in the sense of \cite{blehervelez}, and that
$$\mathrm{dim}_k\, \mathrm{Ext}^1_{\Lambda}(U_{-n},U_{-n}) +
\mathrm{dim}_k \,\mathrm{Ext}^1_{\Lambda}(U_{0},U_{0}) =
\mathrm{dim}_k \;t_{F^{\mathrm{f\/l}}}\;.$$
Suppose furthermore that there exists a proflat quasi-lift
$(M^\bullet,\phi)$ of $V^\bullet$ over $R_{-n}\hat{\otimes}_k R_0$
such that
$${\mathrm{H}}^{-n}(M^\bullet)\cong (R_{-n}\hat{\otimes}_k R_0)\hat{\otimes}_{R_{-n}}X_{-n}
\quad \mbox{ and } \quad
{\mathrm{H}}^0(M^\bullet)\cong (R_{-n}\hat{\otimes}_k R_0)\hat{\otimes}_{R_0} X_0\;.$$
Then the versal proflat deformation ring $R^{\mathrm{f\/l}}(\Lambda,V^\bullet)$ is universal and
isomorphic to $R_{-n}\hat{\otimes}_k R_0$.
\end{enumerate}
\end{rem}

\begin{rem}
\label{rem:split}
Suppose $V^\bullet$ is isomorphic in
$D^-(\Lambda)$ to a complex whose boundary maps are trivial.  Thus
in $D^-(\Lambda)$, $V^\bullet$ is isomorphic to the direct sum
$\bigoplus_i {\mathrm{H}}^{i}(V^\bullet)[-i]$, where there are only finitely
many non-zero terms in this sum and all terms have finite $k$-dimension by Hypothesis
\ref{hypo:fincoh}.  Without loss of generality,
we can assume that all the boundary maps of $V^\bullet$ are trivial,
so ${\mathrm{H}}^{i}(V^\bullet) = V^{i}$ for all $i$.

In this situation, a \emph{split quasi-lift}
of $V^\bullet$ over an object $R$ of
$\hat{\mathcal{C}}$ is a proflat quasi-lift $(M^\bullet,\phi)$ which is
isomorphic in $D^-(R\Lambda)$ to a complex whose boundary maps are trivial.
A split deformation is the isomorphism class of a split quasi-lift.
Let $\hat{F}^{\mathrm{sp}} = \hat{F}^{\mathrm{sp}}_{V^\bullet}:\hat{\mathcal{C}} \to \mathrm{Sets}$
be the functor which sends
each object $R$ of $\hat{\mathcal{C}}$ to the set
$\hat{F}^{\mathrm{sp}}(R)$ of all split deformations of $V^\bullet $ over $R$.

The statements and proofs of 
\cite[Lemma 11.2 and Prop. 11.3]{bcderived} can be adapted to our situation. Here 
is a summary of some of the main results:

\begin{enumerate}
\item[(i)]
The functor $\hat{F}^{\mathrm{sp}}$ is isomorphic to the product of the functors on $\hat{\mathcal {C}}$ 
associated to deformations, in the sense of \cite{blehervelez},
of the non-zero cohomology groups of $V^\bullet$ considered as $\Lambda$-modules.
Moreover, the functor $\hat{F}^{\mathrm{sp}}$  is naturally isomorphic to the functor
$\hat{F}^{\mathrm{f\/l}}$.

\item[(ii)]
The versal split deformation ring $R^{\mathrm{sp}}(\Lambda,V^\bullet)$ associated to
the split deformation functor $\hat{F}^{\mathrm{sp}}$
is the tensor product $\hat{\bigotimes}_i R(\Lambda,V^i)$ over $k$ of the versal deformation
rings, in the sense of \cite{blehervelez}, of the non-zero cohomology groups of $V^\bullet$.
A versal split deformation is given by the direct sum
$$U^{\mathrm{sp}}(\Lambda,V^\bullet) = \bigoplus_i R^{\mathrm{sp}}(\Lambda,V^\bullet)
\hat{\otimes}_{R(\Lambda,V^i)} U(\Lambda,V^i)[-i]$$
where $i$ runs over those integers for which ${\mathrm{H}}^i(V^\bullet) \ne \{0\}$.

\item[(iii)] 
The natural map on tangent spaces $\tau: F^{\mathrm{sp}}(k[\varepsilon]) \to F(k[\varepsilon])$ may be
identified with the natural inclusion
$$\iota: \bigoplus_i
 \mathrm{Ext}^1_{\Lambda}(V^i, V^i)
\to \mathrm{Ext}^1_{D^{-}(\Lambda)}(V^\bullet,V^\bullet) = \bigoplus_{i,j}\mathrm{Ext}^{1+i-j}_{
\Lambda}(V^i,V^j)\,.
$$

\item[(iv)]
We have a non-canonical surjective continuous $k$-algebra homomorphism 
$f_{\mathrm{sp}}:R(\Lambda,V^\bullet) \to R^{\mathrm{sp}}(\Lambda,V^\bullet)$.
If the versal deformation ring $R(\Lambda,V^i)$ is universal for all $i$,
then $R^{\mathrm{sp}}(\Lambda,V^\bullet)$ is a universal split deformation ring.  If in addition
$f_{\mathrm{sp}}$ is an isomorphism, then $R(\Lambda,V^\bullet)$ is a universal deformation ring for $V^\bullet$.
\end{enumerate}
\end{rem}

\section{Derived equivalences and stable equivalences of Morita type}
\label{s:derivedequivalences}
\setcounter{equation}{0}

In \cite{derivedeq}, it was proved that if $k$ is a field of positive characteristic, and
$A$ and $B$ are block algebras of finite groups over a complete local commutative 
Noetherian ring with residue field $k$, then a split-endomorphism
two-sided tilting complex (as introduced by Rickard \cite{rickard2}) for the derived categories
of bounded complexes of finitely generated modules over $A$, resp. $B$,
preserves the versal deformation rings of bounded complexes of finitely generated
modules over $kA$, resp. $kB$.

It is the goal of Section \ref{s:derivedeq} to prove an analogous result, Theorem \ref{thm:deformations}, 
when $k$ is  an arbitrary field and $A$ and $B$ are replaced by arbitrary finite dimensional $k$-algebras.
In Section \ref{s:stableeq}, we will then study the connection to stable equivalences of Morita type for
self-injective algebras; see Propositions \ref{prop:stabmordef} and \ref{prop:stable1}. We assume the notation of Section \ref{s:udr}.

If $S$ is a ring,  $S\mbox{-mod}$ denotes
the category of finitely generated left  $S$-modules. 
Let $C^b(S\mbox{-mod})$  be the category of bounded complexes in $S\mbox{-mod}$,
let $K^b(S\mbox{-mod})$ be the homotopy category of $C^b(S\mbox{-mod})$,
and let $D^b(S\mbox{-mod})$  be the derived category of $K^b(S\mbox{-mod})$.

\subsection{Deformations of complexes and derived equivalences}
\label{s:derivedeq}

Recall from Section \ref{s:udr} that we view the finite dimensional $k$-algebra $\Lambda$
as a pseudocompact $k$-algebra with the discrete topology, and every finitely generated
$\Lambda$-module as a pseudocompact $\Lambda$-module with the discrete topology.
In particular, $D^b(\Lambda\mbox{-mod})$ can be identified with a full subcategory of
the derived category $D^-(\Lambda)$ of bounded above complexes of pseudocompact 
$\Lambda$-modules.

Suppose $\Gamma$ is another finite dimensional $k$-algebra, and $R\in\mathrm{Ob}(\hat{\mathcal{C}})$
is arbitrary. Then $R\Lambda$ and $R\Gamma$ are free $R$-modules of finite rank.
Rickard proved in \cite{rickard1} that the derived categories $D^b(R\Lambda\mbox{-mod})$ and
$D^b(R\Gamma\mbox{-mod})$ are equivalent as triangulated categories if
and only if there is a bounded complex $P_R^\bullet$ of finitely
generated $R\Gamma$-$R\Lambda$-bimodules and a bounded complex $Q_R^\bullet$ of finitely
generated $R\Lambda$-$R\Gamma$-bimodules such that
\begin{eqnarray}
\label{eq:2tilting}
Q_R^\bullet \otimes_{R\Gamma}^\LLP_R^\bullet &\cong& R\Lambda \quad
\,\mbox{in $D^b((R\Lambda\otimes_R R\Lambda^{op})\mbox{-mod})$, and}\\
P_R^\bullet \otimes^{\mathbf{L}}_{R\Lambda}Q_R^\bullet
&\cong& R\Gamma \quad
\mbox{in $D^b((R\Gamma\otimes_R R\Gamma^{op})\mbox{-mod})$.} \nonumber
\end{eqnarray}
If $P_R^\bullet$ and $Q_R^\bullet$ exist, then the functors
\begin{eqnarray}
\label{eq:2tiltfunct} 
P_R^\bullet \otimes^{\mathbf{L}}_{R\Lambda} -: &D^b(R\Lambda\mbox{-mod}) \to
D^b(R\Gamma\mbox{-mod}) &\quad \mbox{ and}\\
Q_R^\bullet \otimes^{\mathbf{L}}_{R\Gamma} -: &D^b(R\Gamma\mbox{-mod})
\to D^b(R\Lambda\mbox{-mod}) &\nonumber
\end{eqnarray}
are equivalences of derived categories, and $Q_R^\bullet$ is
isomorphic to $\mathbf{R}\mathrm{Hom}_{R\Gamma}(P_R^\bullet,R\Gamma)$ in the
derived category of $R\Lambda$-$R\Gamma$-bimodules. The complexes $P_R^\bullet$
and $Q_R^\bullet$ are called \emph{two-sided tilting complexes} (see \cite[Def. 4.2]{rickard1}).

By \cite[Prop. 3.1]{rickard1}, $P_R^\bullet$ is isomorphic in $D^b(R\Gamma\mbox{-mod})$ 
(resp. in $D^b(R\Lambda^{op}\mbox{-mod})$) to a bounded complex $X^\bullet$ of finitely generated
projective left $R\Gamma$-modules (resp. a bounded complex $Y^\bullet$ of
finitely generated projective right $R\Lambda$-modules). 
Since $R\Gamma$ and $R\Lambda$ are free $R$-modules,
projective bimodules for these algebras 
are projective as left and right modules. Let 
$$T^\bullet: \qquad \cdots \to T^{n-1}\xrightarrow{\delta^{n-1}_T} T^{n}\xrightarrow{\delta^{n}_T}
T^{n+1}\to \cdots$$
be a projective
$R\Gamma$-$R\Lambda$-bimodule resolution of $P_R^\bullet$ such that all terms of $T^\bullet$
are finitely generated projective $R\Gamma$-$R\Lambda$-bimodules. By adding an acyclic complex
of finitely generated projective $R\Gamma$-$R\Lambda$-bimodules to $T^\bullet$ if necessary,
we can find quasi-isomorphisms
\begin{eqnarray}
\label{eq:qissurjective}
f:\quad &T^\bullet \to X^\bullet&\mbox{in $C^-(R\Gamma\mbox{-mod})$ and}\\
g:\quad &T^\bullet \to Y^\bullet&\mbox{in }C^-(R\Lambda^{op}\mbox{-mod})
\nonumber
\end{eqnarray}
which are surjective on terms.
More precisely, we construct the acyclic complex of finitely generated projective $R\Gamma$-$R\Lambda$-bimodules 
to be added to $T^\bullet$ inductively, starting from the right. Without loss of generality, assume $T^i=0=X^i=Y^i$ for $i>0$. 
Let $F^{-1}$ be a finitely generated free $R\Gamma$-$R\Lambda$-bimodule such that there exists a surjective $R\Lambda$-module homomorphism
$\phi_X^{-1}:F^{-1}\to X^{-1}$ and a surjective $R\Gamma$-module homomorphism $\phi_Y^{-1}:F^{-1}\to Y^{-1}$. 
Let $\psi_X^0:F^{-1}\to X^0$ be the composition $d_X^{-1}\circ\phi_X^{-1}$, and let 
$\psi_Y^0:F^{-1}\to Y^0$ be the composition $d_Y^{-1}\circ\phi_Y^{-1}$. Since $f^0:T^0\to X^0$ and $g^0:T^0\to Y^0$ induce
isomorphisms on the zero-th cohomology groups, it follows that $\tilde{f}^0=(\psi_X^0,f^0):F^{-1}\oplus T^0 \to X^0$ and $\tilde{g}^0=(\psi_Y^0,g^0): F^{-1}\oplus T^0\to Y^0$
are surjective. In other words, we have added the acyclic complex $F^{-1}\xrightarrow{\mathrm{id}}F^{-1}$, concentrated in degrees $-1$ and $0$, to $T^\bullet$
to ensure that the resulting homomorphisms $\tilde{f}^0$ and $\tilde{g}^0$ are surjective. Inductively, we now work our way to the left and add acyclic complexes of
finitely generated free $R\Gamma$-$R\Lambda$-bimodules of the form $F^{i-1}\xrightarrow{\mathrm{id}}F^{i-1}$, concentrated in degrees $i-1$ and $i$, to $T^\bullet$,
for $i\le -1$ such that there exists a surjective $R\Lambda$-module homomorphism $\phi_X^{i-1}:F^{i-1}\to X^{i-1}$ and a surjective $R\Gamma$-module homomorphism 
$\phi_Y^{i-1}:F^{i-1}\to Y^{i-1}$. We also let $\psi_X^i:F^{i-1}\to X^i$ be the composition $d_X^{i-1}\circ\phi_X^{i-1}$, and we let 
$\psi_Y^i:F^{i-1}\to Y^i$ be the composition $d_Y^{i-1}\circ\phi_Y^{i-1}$, to receive surjective $R\Gamma$-module morphisms 
$\tilde{f}^i=(\psi_X^i,\phi_X^i,f^i):F^{i-1}\oplus F^i\oplus T^i\to X^i$ and surjective $R\Lambda$-module homomorphisms 
$\tilde{g}^i=(\psi_Y^i,\phi_Y^i,g^i):F^{i-1}\oplus F^i\oplus T^i\to Y^i$ for all $i\le -1$.
Since $\mathrm{Ker}(f)$ (resp. $\mathrm{Ker}(g)$) is an acyclic complex of projective
left $R\Gamma$-modules (resp. projective right $R\Lambda$-modules), it splits completely.
If $X^i=0$ and $Y^i=0$ for all $i\le n$, it follows that 
$\mathrm{Ker}(\delta^{n}_T)$ is isomorphic to the kernel of the $n$-th boundary map of
$\mathrm{Ker}(f)$ (resp. $\mathrm{Ker}(g)$) as a left $R\Gamma$-module (resp. right $R\Lambda$-module).
But the latter kernel is projective as a left $R\Gamma$-module (resp. right $R\Lambda$-module).
Hence if we truncate $T^\bullet$ at $n$
$$\cdots \to 0\to \mathrm{Ker}(\delta^{n}_T)\to T^n\xrightarrow{\delta^{n}_T}
T^{n+1}\to \cdots$$
to produce a bounded complex isomorphic to $P_R^\bullet$ 
in $D^b((R\Gamma\otimes_R R\Lambda^{op})\mbox{-mod})$ with the additional property that all the
terms of this complex are projective as left and as right modules and that all terms, except possibly
the leftmost non-zero term, are actually projective as bimodules.
Similarly, we can assume that all terms of
$Q_R^\bullet$ are projective as left $R\Lambda$-modules and as right
$R\Gamma$-modules and that all  terms, except possibly
the leftmost non-zero term, are actually
projective as $R\Lambda$-$R\Gamma$-bimodules. In particular, we can take
$Q_R^\bullet$ to be the $R\Gamma$-dual
\begin{equation}
\label{eq:lambdadual}
\widetilde{P}_R^\bullet = \mathrm{Hom}_{R\Gamma}(P_R^\bullet,R\Gamma)\,.
\end{equation}
In this situation, (\ref{eq:2tilting}) is equivalent to
\begin{eqnarray}
\label{eq:2tiltbetter} 
\widetilde{P}_R^\bullet \otimes_{R\Gamma}P_R^\bullet 
&\cong& R\Lambda \qquad\,
\mbox{in $D^b((R\Lambda\otimes_R R\Lambda^{op})\mbox{-mod})$, and}\\
P_R^\bullet \otimes_{R\Lambda}\widetilde{P}_R^\bullet
&\cong&  R\Gamma \qquad
\mbox{in $D^b((R\Gamma\otimes_R R\Gamma^{op})\mbox{-mod})$.} \nonumber
\end{eqnarray}

\begin{dfn}
\label{def:nice2sidedtilting}
Let $R\in\mathrm{Ob}(\hat{\mathcal{C}})$. Suppose $P_R^\bullet$ is a bounded complex of finitely 
generated $R\Gamma$-$R\Lambda$-bimodules.
\begin{enumerate}
\item[(a)]
We call $P_R^\bullet$
a \emph{nice two-sided tilting
complex}, if all terms of $P_R^\bullet$ are projective as left $R\Gamma$-modules and
as right $R\Lambda$-modules and (\ref{eq:2tiltbetter}) is satisfied.

\item[(b)]
If $R=k$, then we write $P^\bullet=P_R^\bullet$ and $\widetilde{P}^\bullet = \mathrm{Hom}_{\Gamma}(P^\bullet,\Gamma)$. In other words, $P^\bullet$ is a \emph{nice two-sided tilting complex} if $P^\bullet$ is a 
bounded complex of finitely generated $\Gamma$-$\Lambda$-bimodules such that all terms of
$P^\bullet$ are projective as left $\Gamma$-modules and as right $\Lambda$-modules and such that
$\widetilde{P}^\bullet \otimes_{\Gamma}P^\bullet \cong \Lambda$ in 
$D^b((\Lambda\otimes_k \Lambda^{op})\mbox{-mod})$, and
$P^\bullet \otimes_{\Lambda}\widetilde{P}^\bullet \cong \Gamma$ in $D^b((\Gamma\otimes_k \Gamma^{op})\mbox{-mod})$.
\end{enumerate}
\end{dfn}

\begin{rem}
\label{rem:splitendo}
If $\Lambda$ and $\Gamma$ are symmetric $k$-algebras, then 
$R\Lambda$ and $R\Gamma$ are symmetric $R$-algebras for all
$R\in\mathrm{Ob}(\hat{\mathcal{C}})$, in the sense that $R\Lambda$ (resp. $R\Gamma$),
considered as a bimodule over itself, is isomorphic to its $R$-linear dual
$\mathrm{Hom}_R(R\Lambda,R)$ (resp. $\mathrm{Hom}_R(R\Gamma,R)$). In particular,
the functors $\mathrm{Hom}_{R\Gamma}(-,R\Gamma)$ and $\mathrm{Hom}_R(-,R)$ are
then naturally isomorphic. Therefore we may take $Q_R^\bullet$
to be the $R$-dual
\begin{equation}
\label{eq:kdual}
\widecheck{P}_R^\bullet = \mathrm{Hom}_R(P_R^\bullet,R)\,.
\end{equation}
Rickard calls a bounded complex $P_R^\bullet$ of finitely generated
$R\Gamma$-$R\Lambda$-bimodules a \emph{split-endomorphism two-sided tilting
complex}, if all terms of $P_R^\bullet$ are projective as left $R\Gamma$-modules and
as right $R\Lambda$-modules and (\ref{eq:2tiltbetter}) is satisfied with 
$\widetilde{P}_R^\bullet$ replaced by $\widecheck{P}_R^\bullet$ (see \cite[p. 336]{rickard2}).
\end{rem}

\begin{lemma}
\label{lem:lifting} 
Suppose $\Lambda$ and $\Gamma$ are finite dimensional $k$-algebras as above,
$P^\bullet$ is a nice two-sided tilting complex in 
$D^b((\Gamma\otimes_k\Lambda^{op})\mbox{-$\mathrm{mod}$})$ 
as in Definition $\ref{def:nice2sidedtilting}$, and
$R\in\mathrm{Ob}(\hat{\mathcal{C}})$. Then $P_R^\bullet=R\otimes_k
P^\bullet$ is a nice two-sided tilting complex in 
$D^b((R\Gamma\otimes_R R\Lambda^{op})\mbox{-$\mathrm{mod}$})$.
Moreover, 
\begin{eqnarray}
\label{eq:useful} 
\widetilde{P}_R^\bullet \hat{\otimes}_{R\Gamma}P_R^\bullet
&\cong&  R\Lambda \qquad
\mbox{in $D^-(R\Lambda\otimes_R R\Lambda^{op})$, and}\\
P_R^\bullet \hat{\otimes}_{R\Lambda}\widetilde{P}_R^\bullet 
&\cong& R\Gamma \qquad\,
\mbox{in $D^-(R\Gamma\otimes_R R\Gamma^{op})$.} \nonumber
\end{eqnarray}
In particular, the functors $P_R^\bullet\hat{\otimes}_{R\Gamma}-$ and
$\widetilde{P}_R^\bullet\hat{\otimes}_{R\Lambda}-$ provide quasi-inverse equivalences: 
\begin{eqnarray}
\label{eq:quasiinverses}
P_R^\bullet \hat{\otimes}_{R\Lambda} -: &D^-(R\Lambda) \to
D^-(R\Gamma)\,,& \mbox{and}\\
\widetilde{P}_R^\bullet \hat{\otimes}_{R\Gamma} -: &D^-(R\Gamma)
\to D^-(R\Lambda)\,.& \nonumber
\end{eqnarray}
\end{lemma}

\begin{proof}
We have
$$R\otimes_k(\Lambda\otimes_k \Lambda^{op})\cong R\Lambda\otimes_R R\Lambda^{op}$$
and 
$$R\otimes_k(\Gamma\otimes_k\Gamma^{op})\cong R\Gamma\otimes_R R\Gamma^{op}$$
as pseudocompact $R$-algebras. Therefore,
it follows from \cite[Lemma 4.3]{rickard1} that $P_R^\bullet=R\otimes_kP^\bullet$ is a two-sided tilting
complex in $D^b((R\Gamma\otimes_R R\Lambda^{op})\mbox{-$\mathrm{mod}$})$.
Note that $R\otimes_k\widetilde{P}^\bullet=R\otimes_k\mathrm{Hom}_{\Gamma}(P^\bullet,\Gamma)
\cong \mathrm{Hom}_{R\Gamma}(P_R^\bullet,R\Gamma)=\widetilde{P}_R^\bullet$ 
as complexes of $R\Lambda$-$R\Gamma$-bimodules.
Since all terms of $P^\bullet$ are projective as left $\Gamma$-modules and as right $\Lambda$-modules,
it follows that all terms of $P_R^\bullet$ are projective as left $R\Gamma$-modules and
as right $R\Lambda$-modules.
Since $R\otimes_k(\widetilde{P}^\bullet \otimes_{\Gamma}P^\bullet)
\cong \widetilde{P}_R^\bullet \otimes_{R\Gamma}P_R^\bullet$
in $C^b((R\Lambda\otimes_R R\Lambda^{op})\mbox{-mod})$
and since $R\otimes_k(P^\bullet \otimes_{\Lambda}\widetilde{P}^\bullet )
\cong P_R^\bullet \otimes_{R\Lambda}\widetilde{P}_R^\bullet$ 
in $C^b((R\Gamma\otimes_R R\Gamma^{op})\mbox{-mod})$, we obtain
(\ref{eq:2tiltbetter}). In other words, $P_R^\bullet$ is a nice two-sided tilting complex in 
$D^b((R\Gamma\otimes_R R\Lambda^{op})\mbox{-$\mathrm{mod}$})$.

To prove the remaining statement of Lemma \ref{lem:lifting}, we consider the isomorphisms
in (\ref{eq:2tiltbetter}) more closely. First, we note that since the terms of 
$P_R^\bullet$ are finitely generated projective left $R\Gamma$-modules and finitely generated
projective right $R\Lambda$-modules and since the terms of $\widetilde{P}_R^\bullet$ are finitely 
generated projective left $R\Lambda$-modules and finitely generated
projective right $R\Gamma$-modules, we can replace the tensor products by completed tensor
products. Moreover, since $\widetilde{P}^\bullet \otimes_{\Gamma}P^\bullet\cong  \Lambda$
in $D^b((\Lambda\otimes_k \Lambda^{op})\mbox{-mod})$, it follows that this is also true
in the derived category $D^-(\Lambda\otimes_k \Lambda^{op})$ of bounded above
pseudocompact $\Lambda$-$\Lambda$-bimodules. Similarly, 
$P^\bullet \otimes_{\Lambda}\widetilde{P}^\bullet\cong  \Gamma$ in the derived category 
$D^-(\Gamma\otimes_k \Gamma^{op})$ of bounded above
pseudocompact $\Gamma$-$\Gamma$-bimodules.  Therefore, we obtain
\begin{eqnarray}
\label{eq:oyoyoy} 
R\otimes_k(\widetilde{P}^\bullet \otimes_{\Gamma}P^\bullet )
&\cong&  R\otimes_k\Lambda=R\Lambda \qquad
\mbox{in $D^-(R\Lambda\otimes_R R\Lambda^{op})$, and}\\
R\otimes_k(P^\bullet \otimes_{\Lambda}\widetilde{P}^\bullet)
&\cong& R\otimes_k\Gamma\,=R\Gamma \qquad
\mbox{in $D^-(R\Gamma\otimes_R R\Gamma^{op})\,.$} \nonumber
\end{eqnarray}
Since 
$$R\otimes_k(\widetilde{P}^\bullet \otimes_{\Gamma}P^\bullet )
\cong \widetilde{P}_R^\bullet \otimes_{R\Gamma}P_R^\bullet
\cong \widetilde{P}_R^\bullet \hat{\otimes}_{R\Gamma}P_R^\bullet$$ 
in $C^-(R\Lambda\otimes_R R\Lambda^{op})$
and since 
$$R\otimes_k(P^\bullet \otimes_{\Lambda}\widetilde{P}^\bullet)
\cong P_R^\bullet \otimes_{R\Lambda}\widetilde{P}_R^\bullet
\cong P_R^\bullet \hat{\otimes}_{R\Lambda}\widetilde{P}_R^\bullet$$
in $C^-(R\Gamma\otimes_R R\Gamma^{op})$, 
the isomorphisms in (\ref{eq:useful}) follow.
This proves that the functors in (\ref{eq:quasiinverses}) are quasi-inverses
between the derived categories of bounded above complexes of pseudocompact 
$R\Lambda$- and $R\Gamma$-modules, completing the proof of Lemma \ref{lem:lifting}.
\end{proof}

\begin{thm}
\label{thm:deformations}
Suppose $\Lambda$ and $\Gamma$ are finite dimensional $k$-algebras as above,
and $P^\bullet$ is a nice two-sided tilting complex in 
$D^b((\Gamma\otimes_k\Lambda^{op})\mbox{-$\mathrm{mod}$})$
as in Definition $\ref{def:nice2sidedtilting}$. 
Let $V^\bullet$ be a bounded complex
of finitely generated $\Lambda$-modules, and let ${V'}^\bullet = P^\bullet\otimes_\Lambda
V^\bullet$. Then $R(\Lambda,V^\bullet)$ and $R(\Gamma,{V'}^\bullet)$ are isomorphic
in $\hat{\mathcal{C}}$.
\end{thm}

\begin{proof}
We use the notation from Lemma \ref{lem:lifting}.
Let $R\in\mathrm{Ob}(\hat{\mathcal{C}})$ and let $(M^\bullet, \phi)$ be a quasi-lift of
$V^\bullet$ over $R$. By Remark \ref{rem:profree}(i), we can assume that the terms of
$M^\bullet$ are topologically free $R\Lambda$-modules. Since $M^\bullet$ has finite
pseudocompact $R$-tor dimension, we can truncate $M^\bullet$ to obtain a quasi-lift 
$(N^\bullet, \psi)$ of $V^\bullet$ which is isomorphic to $(M^\bullet, \phi)$ such that $N^\bullet$
is a bounded complex of $R\Lambda$-modules, all of which are topologically free as $R$-modules. 

Define ${N'}^\bullet=P_R^\bullet \hat{\otimes}_{R\Lambda}N^\bullet$,
so ${N'}^\bullet$ is an object of $D^-(R\Gamma)$.
Since $P_R^\bullet$ is a bounded complex of finitely generated projective right $R\Lambda$-modules and 
since $N^\bullet$ is a bounded complex of topologically free
$R$-modules, it follows that ${N'}^\bullet$ is also a bounded complex of topologically free $R$-modules.
But this means that there exists an integer $n$ such that for all pseudocompact $R$-modules $S$
and all integers $i< n$ we have ${\mathrm{H}}^i(S\hat{\otimes}_R{N'}^\bullet)=
{\mathrm{H}}^i(S\hat{\otimes}^{\mathbf{L}}_R{N'}^\bullet)=0$. In other words, ${N'}^\bullet$ has finite pseudocompact
$R$-tor dimension.

Next we note that we can view $D^-(\Lambda)$ as the full subcategory of $D^-(R\Lambda)$
consisting of bounded above complexes of pseudocompact $R\Lambda$-modules
on which the maximal ideal $m_R$ of $R$ acts trivially. Moreover,
on $D^-(\Lambda)$ the functor $P_R^\bullet \hat{\otimes}_{R\Lambda} -$
coincides with the functor $P^\bullet \otimes_{\Lambda} -:D^-(\Lambda)\to D^-(\Gamma)$.
Define $\psi'=P_R^\bullet \hat{\otimes}_{R\Lambda}\psi$, so $\psi'$ is an isomorphism in 
$D^-(\Lambda)$. Then 
$${N'}^\bullet\hat{\otimes}_R^\LLk ={N'}^\bullet\hat{\otimes}_Rk =
(P_R^\bullet \hat{\otimes}_{R\Lambda}N^\bullet)\hat{\otimes}_Rk=
P_R^\bullet \hat{\otimes}_{R\Lambda}(N^\bullet\hat{\otimes}_Rk)
\xrightarrow{\psi'} P_R^\bullet \hat{\otimes}_{R\Lambda}V^\bullet
= {V'}^\bullet$$
which means $({N'}^\bullet,\psi')$ is a quasi-lift of ${V'}^\bullet$. 
It follows that for each $R\in\mathrm{Ob}(\hat{\mathcal{C}})$,
the functor $P_R^\bullet \hat{\otimes}_{R\Lambda}-$ induces a bijection
$\tau_R$ from the set of deformations of $V^\bullet$ over $R$ onto the set of deformations of 
${V'}^\bullet$ over $R$. 

It remains to show that the maps $\tau_R$ are natural with respect to morphisms $\alpha:R\to R'$ in 
$\hat{\mathcal{C}}$. Considering $(N^\bullet, \psi)$ and $({N'}^\bullet, \psi')$ as above, it suffices to
show that $(R'\hat{\otimes}_{R,\alpha}{N'}^\bullet,(\psi')_\alpha)$ and 
$(P_{R'}^\bullet \hat{\otimes}_{R'\Lambda} (R'\hat{\otimes}_{R,\alpha}N^\bullet),P_{R'}^\bullet \hat{\otimes}_{R'\Lambda}(\psi_\alpha))$ are isomorphic as quasi-lifts of ${V'}^\bullet$.
Since all the terms of $P_R^\bullet$ are finitely generated projective right $R\Lambda$-modules and
since all the terms of $N^\bullet$ are topologically free as $R$-modules, it follows
that there is a natural isomorphism
$$f\;:\quad {N'}^\bullet \hat{\otimes}_{R,\alpha}R'= 
(P_R^\bullet \hat{\otimes}_{R\Lambda}N^\bullet)\hat{\otimes}_{R,\alpha}R'
\to P_{R'}^\bullet \hat{\otimes}_{R'\Lambda}(N^\bullet\hat{\otimes}_{R,\alpha} R')$$
in $D^-(R'\Gamma)$ (in fact in $C^-(R'\Gamma)$). Moreover, the diagram
$$
\xymatrix{
\left({N'}^\bullet\hat{\otimes}_{R,\alpha}R'\right)\hat{\otimes}_{R'}k
\ar[rr]^(.45){f\hat{\otimes}_{R'}k} \ar[d]_{\cong}& &
\left(P_{R'}^\bullet \hat{\otimes}_{R'\Lambda}(N^\bullet\hat{\otimes}_{R,\alpha} R')\right)\hat{\otimes}_{R'}k
\ar[d]^{\cong}\\
{N'}^\bullet\hat{\otimes}_Rk\ar[rd]_{\psi'} && P_{R'}^\bullet \hat{\otimes}_{R'\Lambda}(N^\bullet\hat{\otimes}_R k)\ar[ld]^{P_{R'}^\bullet \hat{\otimes}_{R'\Lambda}\psi}
\\&{V'}^\bullet&
}$$
commutes in $D^-(\Gamma)$, where the vertical isomorphisms are resulting from the associativity of completed tensor products. Hence $(R'\hat{\otimes}_{R,\alpha}{N'}^\bullet,(\psi')_\alpha) \cong (P_{R'}^\bullet \hat{\otimes}_{R'\Lambda} (R'\hat{\otimes}_{R,\alpha}N^\bullet),P_{R'}^\bullet \hat{\otimes}_{R'\Lambda}(\psi_\alpha))$.

This means that the functors $\hat{F}_{V^\bullet}$ and $\hat{F}_{{V'}^\bullet}$ are naturally 
isomorphic, which implies Theorem \ref{thm:deformations}.
\end{proof}

\subsection{Stable equivalences of Morita type for self-injective algebras}
\label{s:stableeq}

We use the notation introduced in Section \ref{s:derivedeq}. Moreover, we assume throught this section
that both $\Lambda$ and $\Gamma$ are \textbf{self-injective} finite dimensional $k$-algebras.

We first collect some useful facts, which were proved as Claims 1, 2 and 6
in the proof of \cite[Thm. 2.6]{blehervelez}, only using the assumption that $\Lambda$ is a self-injective
finite dimensional $k$-algebra. As before, $\mathcal{C}$ denotes the full subcategory of $\hat{\mathcal{C}}$ 
consisting of Artinian objects.

\begin{rem}
\label{rem:josepaper}
Suppose $\Lambda$ is a self-injective finite dimensional $k$-algebra.
Let $R,R_0$ be in $\mathcal{C}$, and let $\pi:R\to R_0$ be a surjection in $\mathcal{C}$.
Let $M$, $Q$ (resp. $M_0$, $Q_0$) be finitely generated $R\Lambda$-modules
(resp. $R_0\Lambda$-modules) and assume that $Q$ (resp. $Q_0$) is projective. 
Suppose there are
$R_0\Lambda$-module isomorphisms $g:R_0\otimes_{R,\pi} M \to M_0$,
$h:R_0\otimes_{R,\pi} Q\to Q_0$. 
\begin{enumerate}
\item[(i)] 
If $\nu_0\in \mathrm{Hom}_{R_0\Lambda}(M_0, Q_0)$, then there
exists $\nu\in \mathrm{Hom}_{R\Lambda}(M, Q)$ with $\nu_0=
h\circ(R_0\otimes_{R,\pi}\nu)\circ g^{-1}$.
\item[(ii)]
If $\sigma_0\in \mathrm{End}_{\Lambda}(M_0)$  factors through a projective 
$R_0 \Lambda$-module, then there exists $\sigma
\in  \mathrm{End}_{R\Lambda}(M)$ such that $\sigma$ factors through a projective
$R\Lambda$-module and $\sigma_0=g\circ (R_0\otimes_{R,\pi}\sigma)\circ g^{-1}$.
\end{enumerate}
Let $P$ be a finitely generated projective $\Lambda$-module. 
Let $\iota_R:k\to R$ be the unique morphism in $\mathcal{C}$ endowing $R$ with a $k$-algebra structure,
and let $\pi_R:R\to k$ be the morphism of $R$ to its residue field $k$ in $\mathcal{C}$. Then 
$\pi_R\circ \iota_R$ is the identity on $k$, and $P_R=R\otimes_{k,\iota_R} P$ is a
projective $R\Lambda$-module cover of $P$, which is unique up to isomorphism. 
In particular, $(P_R,\pi_{R,P})$ is a lift of $P$ over $R$, where $\pi_{R,P}$ is the natural
isomorphism $k\otimes_{R,\pi_R}(R\otimes_{k,\iota_R}P)\to P$ of $\Lambda$-modules.
\begin{enumerate}
\item[(iii)]
Suppose there is a commutative
diagram of finitely generated $R\Lambda$-modules
\begin{equation}
\label{eq:whatever}
\xymatrix{
0\ar[r]&P_R\ar[d]\ar[r]^{g}&T\ar[d]\ar[r]^{h}& C\ar[d]\ar[r]&0\\
0\ar[r]&P\ar[r]^{\overline{g}}&k\otimes_R T\ar[r]^{\overline{h}}&k\otimes_R C\ar[r]&0}
\end{equation}
in which $T$ and $C$ are free over $R$ and the bottom row arises by tensoring the top row with 
$k$ over $R$ and using the $\Lambda$-module isomorphism $\pi_{R,P}:k\otimes_{R,\pi_R} P_R\to P$. 
Then the top row of $(\ref{eq:whatever})$ splits as a sequence of $R\Lambda$-modules.
\end{enumerate}
\end{rem}

We need the following result, which is a generalization of \cite[Thm. 2.6(b)]{blehervelez}.

\begin{lemma}
\label{lem:addproj}
Suppose $\Lambda$ is a self-injective finite dimensional $k$-algebra and that 
$V$ and $P$ are finitely generated non-zero
$\Lambda$-modules and $P$ is projective. Then $P$ has a universal deformation ring $R(\Lambda,P)$
which is isomorphic to $k$, and the versal deformation ring $R(\Lambda,V\oplus P)$ is isomorphic
to the versal deformation ring $R(\Lambda,V)$.
\end{lemma}

\begin{proof}
Since $P$ is a projective $\Lambda$-module, it follows that $\mathrm{Ext}^1_\Lambda(P,P)=0$, which
implies by \cite[Prop. 2.1]{blehervelez} that the versal deformation ring of $P$ is isomorphic to $k$. 
For each $R\in\mathrm{Ob}(\hat{\mathcal{C}})$, let $\iota_R:k\to R$ be the unique morphism in 
$\hat{\mathcal{C}}$ endowing $R$ with a $k$-algebra structure, and let $\pi_R:R\to k$ be the morphism 
of $R$ to its residue field $k$ in $\hat{\mathcal{C}}$. Then $\pi_R\circ \iota_R$ is the identity morphism of $k$. 
This means that $k$ is the universal deformation ring of $P$.

Let $R\in\mathrm{Ob}(\mathcal{C})$ be Artinian. Define a map
\begin{eqnarray}
\label{eq:bijproj}
\mathrm{Def}_\Lambda(V,R)&\to&\mathrm{Def}_\Lambda(V\oplus P, R)\\
{[M,\phi]} &\mapsto&[M\oplus P_R,\phi\oplus \pi_{R,P}]\nonumber
\end{eqnarray}
where $P_R=R\otimes_{k,\iota_R}P$ and $\pi_{R,P}:k\otimes_{R,\pi_R} P_R\to P$ are as in 
Remark \ref{rem:josepaper}. Then the map (\ref{eq:bijproj}) is natural with respect to morphisms
$\alpha:R\to R'$ in $\mathcal{C}$, since $\alpha\circ\iota_R=\iota_{R'}$ and
$\pi_{R'}\circ\alpha=\pi_R$. Since the deformation functors are continuous, it suffices to show that the
map (\ref{eq:bijproj}) is bijective for all $R\in\mathrm{Ob}(\mathcal{C})$ to complete the proof of Lemma \ref{lem:addproj}. 

Suppose first that $(M,\phi)$ and $(M',\phi')$ are two lifts of $V$ over $R$ such that
there exists an $R\Lambda$-module isomorphism 
$f=\left(\begin{array}{cc}f_{11}&f_{12}\\f_{21}&f_{22}\end{array}\right): M\oplus P_R\to M'\oplus P_R$ with
$(\phi'\oplus \pi_{R,P})\circ(k\otimes f) = \phi\oplus \pi_{R,P}$. In particular, $\phi'\circ(k\otimes f_{11})=\phi$
and $k\otimes f_{22}$ is the identity morphism on $P$. By Nakayama's Lemma, this implies that $f_{11}$ and $f_{22}$ 
are $R\Lambda$-module isomorphisms, since $M$ and $M'$ are free $R$-modules of the same finite rank.
Therefore, $[M,\phi]=[M',\phi']$ and the map $(\ref{eq:bijproj})$ is injective.

We now show that the map $(\ref{eq:bijproj})$ is surjective. Let $(T,\tau)$ be a lift of $V\oplus P$ over $R$. Since
$P_R$ is a projective $R\Lambda$-module, there exists an $R\Lambda$-module homomorphism $g$ which makes
the following diagram commute
\begin{equation}
\label{eq:need1}
\xymatrix{ P_R\ar[r]^{g} \ar[d]_{\footnotesize{\left(\begin{array}{c}0\\\mathrm{pr}\end{array}\right)}}& T\ar[d]\\
V\oplus P \ar[r]^{\tau^{-1}}&k\otimes_R T}
\end{equation}
where $\mathrm{pr}$ is obtained by tensoring $P_R$ with
$k$ over $R$ and using the $\Lambda$-module isomorphism 
$\pi_{R,P}:k\otimes_{R,\pi_R} P_R\to P$. Then $g$ induces an injective $\Lambda$-module homomorphism
$g'$ and a commutative diagram of $\Lambda$-modules 
\begin{equation}
\label{eq:need2}
\xymatrix{P_R/m_R\,P_R\ar[r]^{g'} \ar[d]_{\pi_{R,P}}& T/m_R T\ar@{=}[dd]\\
P\ar[d]_{\iota_P}&\\
V\oplus P&k\otimes_R T\ar[l]^{\tau}}\\
\end{equation}
where $\iota_P: P \to V\oplus P$ is the natural injection and, 
as before, $m_R$ denotes the maximal ideal of $R$. Using Nakayama's Lemma, it follows that $g$
is injective. Letting $C$ be the cokernel of $g$, we obtain a commutative diagram (\ref{eq:whatever}). Since
$P_R$ and $T$ are free $R$-modules and since $g$ and $g'$ (and hence $\overline{g}=g'\circ \pi_{R,P}^{-1}$) are injective, an
elementary Nakayama's Lemma argument shows that $C$ is also free as an $R$-module. 
Therefore, by Remark \ref{rem:josepaper}(iii), the top row of (\ref{eq:whatever}) splits as a sequence of 
$R\Lambda$-modules. Let $j:C\to T$ be an $R\Lambda$-module splitting of $h$. By tensoring with $k$ over $R$,
we obtain a $\Lambda$-module splitting $\overline{j}:k\otimes_RC \to k\otimes_R T$ of $\overline{h}$.
Consider the $R\Lambda$-module isomorphism
$$(j,g):\quad  C\oplus P_R \to T\,.$$
Since $\tau\circ\overline{g}=\tau\circ g'\circ \pi_{R,P}^{-1}=\iota_P$ by (\ref{eq:need2}),
there exist a $\Lambda$-module isomorphism $\xi:k\otimes_R C\to V$ such that
$p_V\circ\tau = \xi\circ \overline{h}$, where $p_V:V\oplus P \to V$ is the natural projection onto $V$. 
Letting $p_P:V\oplus P \to P$ is the natural projection onto $P$, we have
\begin{eqnarray}
\label{eq:needthis4}
p_P\circ\tau\circ(k\otimes_R g) &=& p_P\circ\tau \circ g' \;= \;p_P\circ\iota_P\circ\pi_{P,R}\;=\;\pi_{P,R},\\
p_V\circ\tau\circ(k\otimes_R g) &=& p_V\circ\tau\circ g'\;=\;p_V\circ\iota_P\circ\pi_{P,R}\;=\;0,\nonumber\\
p_V\circ\tau\circ(k\otimes_R j) &=& p_V\circ\tau\circ\overline{j} \;=\;\xi\circ\overline{h}\circ\overline{j}\;=\; \xi,\mbox{ and}\nonumber\\
p_P\circ\tau\circ(k\otimes_R j) &=& p_P\circ\tau\circ\overline{j} \,.\nonumber
\end{eqnarray}
By Remark \ref{rem:josepaper}(i), there exists an $R\Lambda$-module homomorphism $\lambda:C\to P_R$ such that
$$k\otimes_R\lambda=\pi_{R,P}^{-1}\circ p_P\circ\tau\circ\overline{j}\;.$$
Letting $j_1=j-g\circ \lambda$ shows that $j_1$ is also an $R\Lambda$-module splitting of $h$ and
$\overline{j_1}=k\otimes_R j_1$ is also a $\Lambda$-module splitting of $\overline{h}$. Hence, on replacing $j$ by
$j_1$ and using the equations (\ref{eq:needthis4}), we see that
$p_V\circ\tau\circ(k\otimes_R j_1) = \xi$ and
$$p_P\circ\tau\circ(k\otimes_R j_1) = p_P\circ\tau\circ\overline{j}-p_P\circ\tau\circ (k\otimes_Rg)\circ(k\otimes_R\lambda)=0\,.$$
This means that $(j_1,g):  C\oplus P_R \to T$ provides an isomorphism between the lifts
$(C\oplus P_R,\xi\oplus \pi_{P,R})$ and $(T,\tau)$ of $V\oplus P$ over $R$. 
Hence the map (\ref{eq:bijproj}) is bijective for all $R\in\mathrm{Ob}(\mathcal{C})$, which 
completes the proof of Lemma \ref{lem:addproj}. 
\end{proof}

The following definition of stable equivalence of Morita type goes back to Brou\'{e} \cite{broue1}.

\begin{dfn}
\label{def:stabeq} 
Suppose $\Lambda$ and $\Gamma$ are self-injective finite dimensional $k$-algebras.
Let $X$ be a $\Gamma$-$\Lambda$-bimodule and let $Y$ be a $\Lambda$-$\Gamma$-bimodule. 
We say $X$ and $Y$ induce a \emph{stable equivalence of Morita type} between $\Lambda$ and 
$\Gamma$, if $X$ and $Y$ are projective both as left and as right modules, and if 
\begin{eqnarray}
\label{eq:stab}
Y\otimes_{\Gamma}X&\cong& \Lambda\oplus P \quad\mbox{ as $\Lambda$-$\Lambda$-bimodules, and} \\ 
\nonumber
X\otimes_{\Lambda}Y&\cong& \Gamma\oplus Q \quad\mbox{ as $\Gamma$-$\Gamma$-bimodules}, 
\end{eqnarray}
where $P$ is a projective $\Lambda$-$\Lambda$-bimodule, and $Q$ is a projective 
$\Gamma$-$\Gamma$-bimodule.
In particular, $X\otimes_{\Lambda}-$ and $Y\otimes_{\Gamma}-$ induce mutually inverse 
equivalences between the stable module categories $\Lambda\mbox{-\underline{mod}}$ and 
$\Gamma\mbox{-\underline{mod}}$.
\end{dfn}

\begin{rem}
\label{rem:rickardagain}
It follows from a result by Rickard (see \cite[Cor. 5.5]{rickard1} and \cite[Prop. 6.3.8]{kozim}) that a derived 
equivalence between $D^b(\Lambda\mbox{-{mod}})$ and $D^b(\Gamma\mbox{-{mod}})$ induces
a stable equivalence of Morita type between $\Lambda$ and $\Gamma$.

More precisely,
let $K^{b}(\Lambda\mbox{-proj})$ be the full subcategory of $D^b(\Lambda\mbox{-{mod}})$
consisting of all objects isomorphic to bounded complexes of finitely generated projective
$\Lambda$-modules. Then $K^{b}(\Lambda\mbox{-proj})$ is a thick subcategory of
$D^b(\Lambda\mbox{-{mod}})$ and we can build the Verdier quotient
$$D^b(\Lambda\mbox{-{mod}})/K^{b}(\Lambda\mbox{-proj})$$
(see, for example, \cite[Sect. 4.6]{krause2008} for the construction of this quotient).
Rickard proved in \cite[Thm. 2.1]{rickardJPAA1989} that this Verdier quotient is equivalent
as a triangulated category to the stable module category $\Lambda\mbox{-\underline{mod}}$.

Suppose now that there is a derived equivalence between $D^b(\Lambda\mbox{-{mod}})$ and 
$D^b(\Gamma\mbox{-{mod}})$. As in Section \ref{s:derivedeq}, there exists 
a nice two-sided tilting complex $P^\bullet$ (see Definition \ref{def:nice2sidedtilting}
in the case where $R=k$). Following the proof of \cite[Cor. 5.5]{rickard1}, let $T^\bullet$ be a projective 
$\Gamma$-$\Lambda$-bimodule resolution of $P^\bullet$ such that all terms of $T^\bullet$ are
finitely generated projective $\Gamma$-$\Lambda$-bimodules. For large $n>0$, we can truncate
$T^\bullet$ to obtain a bounded complex
$$S^\bullet:\quad \cdots \to 0\to S^{-n}\to  T^{-n+1}\to  T^{-n+2}\to \cdots$$
which is isomorphic to $P^\bullet$ in $D^b((\Gamma\otimes\Lambda^{op})\mbox{-{mod}})$, 
where all terms but $S^{-n}$ are projective $\Gamma$-$\Lambda$-bimodules and 
$S^{-n}$ is projective as a left $\Gamma$-module and as a right $\Lambda$-module.
If we let $X=\Omega_{\Gamma\Lambda}^{-n}(S^{-n})$, the
$(-n)^{\mathrm{th}}$ syzygy as a $\Gamma$-$\Lambda$-bimodule, then $S^\bullet$ is isomorphic
to the one-term complex $X$ concentrated in degree 0 in 
$D^b((\Gamma\otimes_k\Lambda^{op})\mbox{-{mod}})/
K^{b}(\Gamma\otimes_k\Lambda^{op})\mbox{-proj})$.
Moreover, we have that $\widetilde{S}^\bullet=\mathrm{Hom}_\Gamma(S^\bullet,\Gamma)$ has the form
$$\widetilde{S}^\bullet:\quad \cdots \to
\widetilde{T}^{-n+2}\to \widetilde{T}^{-n+1}\to \widetilde{S}^{-n}\to 0 \to\cdots$$
where $\widetilde{T}^i=\mathrm{Hom}_\Gamma(T^i,\Gamma)$ for $i>-n$ and 
$\widetilde{S}^{-n}=\mathrm{Hom}_\Gamma(S^{-n},\Gamma)$. 
If we let $Y=\Omega_{\Lambda\Gamma}^{n}(\widetilde{S}^{-n})$, the
$n^{\mathrm{th}}$ syzygy as a $\Lambda$-$\Gamma$-bimodule, then $\widetilde{S}^\bullet$ is isomorphic
to the one-term complex $Y$ concentrated in degree 0 in 
$D^b((\Lambda\otimes_k\Gamma^{op})\mbox{-{mod}})/K^{b}(\Lambda\otimes_k\Gamma^{op})\mbox{-proj})$.
Using (\ref{eq:2tiltbetter}), we obtain
\begin{eqnarray*}
Y\otimes_\Gamma X &\cong& \widetilde{S}^\bullet\otimes_{\Gamma}S^\bullet\;\cong\;
\widetilde{P}^\bullet \otimes_{\Gamma}P^\bullet \;\cong \;
\Lambda \qquad\,
\mbox{in $D^b((\Lambda\otimes_k\Lambda^{op})\mbox{-mod})/K^{b}((\Lambda\otimes_k\Lambda^{op})\mbox{-proj})$, and}\\
X\otimes_\Lambda Y &\cong& S^\bullet\otimes_{\Gamma}\widetilde{S}^\bullet\;\cong\;
P^\bullet \otimes_{\Lambda}\widetilde{P}^\bullet\;\cong\;\Gamma \qquad
\mbox{in $D^b((\Gamma\otimes_k\Gamma^{op})\mbox{-mod})/K^{b}((\Gamma\otimes_k\Gamma^{op})\mbox{-proj})$.}
\end{eqnarray*}
Using that $D^b((\Lambda\otimes_k\Lambda^{op})\mbox{-{mod}})/
K^{b}((\Lambda\otimes_k\Lambda^{op})\mbox{-proj})$ is equivalent as a triangulated
category to $(\Lambda\otimes_k\Lambda^{op})\mbox{-\underline{mod}}$ and that
$D^b((\Gamma\otimes_k\Gamma^{op})\mbox{-{mod}})/
K^{b}((\Gamma\otimes_k\Gamma^{op})\mbox{-proj})$ is equivalent as a triangulated
category to $(\Gamma\otimes_k\Gamma^{op})\mbox{-\underline{mod}}$, we obtain
that $X$ and $Y$ satisfy (\ref{eq:stab}). In other words, $X$ and $Y$ induce  a stable equivalence 
of Morita type between $\Lambda$ and  $\Gamma$.
\end{rem}

The following result is proved using Theorem \ref{thm:deformations} and
Lemma \ref{lem:addproj}.

\begin{prop}
\label{prop:stable1}
Suppose $\Lambda$ and $\Gamma$ are finite dimensional \textbf{self-injective} $k$-algebras.
Let $P^\bullet$ be a nice two-sided tilting complex in 
$D^b((\Gamma\otimes_k\Lambda^{op})\mbox{-$\mathrm{mod}$})$ such that $P^\bullet$ is isomorphic
in $D^b((\Gamma\otimes_k\Lambda^{op})\mbox{-$\mathrm{mod}$})/
K^{b}((\Gamma\otimes_k\Lambda^{op})\mbox{-$\mathrm{proj}$})$
to a one-term complex $X$ concentrated in degree 0, as in Remark $\ref{rem:rickardagain}$.
Let $V$ be a finitely generated $\Lambda$-module, and let ${V'} = X\otimes_\Lambda V$, so ${V'}$
is a finitely generated $\Gamma$-module. Then 
$R(\Lambda,V)$ and $R(\Gamma,{V'})$ are isomorphic in $\hat{\mathcal{C}}$.
\end{prop}

\begin{proof}
If we view $V$ as a one-term complex concentrated in degree 0, then it follows from
Proposition \ref{prop:modulecase} and Theorem \ref{thm:deformations} that $R(\Lambda,V)\cong
R(\Gamma,P^\bullet\otimes_\Lambda V)$ in $\hat{\mathcal{C}}$.
We have that 
$$P^\bullet\otimes_\Lambda V\cong S^\bullet\otimes_\Lambda V\cong X\otimes_\Lambda V$$
in $D^b(\Gamma\mbox{-{mod}})/K^{b}(\Gamma\mbox{-$\mathrm{proj}$})$, where $S^\bullet$ is as in
Remark \ref{rem:rickardagain}. By  \cite[Thm. 2.1]{rickardJPAA1989},
$$D^b(\Gamma\mbox{-{mod}})/K^{b}(\Gamma\mbox{-$\mathrm{proj}$})\cong \Gamma\mbox{-\underline{mod}}$$
as triangulated categories. 
By Definition \ref{def:nice2sidedtilting}(b) and by (\ref{eq:stab}), we have
\begin{eqnarray*}
\widetilde{P}^\bullet \otimes_{\Gamma}(P^\bullet\otimes_\Lambda V) &\cong& V\qquad\qquad\qquad\;\,
\mbox{in $D^-(\Lambda)$, and}\\
Y\otimes_\Gamma (X\otimes_\Lambda V) &\cong& V\oplus( P\otimes_\Lambda V)\quad
\mbox{in $\Lambda$-mod.}
\end{eqnarray*}
Since moreover $R(\Lambda,V)\cong R(\Lambda, V\oplus (P\otimes_\Lambda V))$ by Lemma 
\ref{lem:addproj}, we obtain that
$$R(\Gamma,P^\bullet\otimes_\Lambda V)\cong R(\Gamma,X\otimes_\Lambda V)$$
in $\hat{\mathcal{C}}$.
Therefore,
$R(\Lambda,V)\cong R(\Gamma,P^\bullet\otimes_\Lambda V)\cong R(\Gamma,X\otimes_\Lambda V)$
in $\hat{\mathcal{C}}$.
\end{proof}

We next show that a stable equivalence of Morita type between self-injective algebras
preserves versal deformation rings.
The arguments are very similar to those used in \cite[Sect. 2.2]{3sim}. 

\begin{prop}
\label{prop:stabmordef}
Suppose $\Lambda$ and $\Gamma$ are self-injective finite dimensional $k$-algebras.
Suppose $X$ is a $\Gamma$-$\Lambda$-bimodule and $Y$ is a $\Lambda$-$\Gamma$-bimodule
which induce a stable equivalence of Morita type between $\Lambda$ and $\Gamma$. Let $V$ be a 
finitely generated $\Lambda$-module, and define $V'=X\otimes_{\Lambda}V$. Then 
$R(\Lambda,V)$ is isomorphic to $R(\Gamma,V')$ in $\hat{\mathcal{C}}$.
\end{prop}

\begin{proof}
Let $R\in\mathrm{Ob}(\mathcal{C})$ be Artinian. Then $X_R=R\otimes_kX$ is projective as left 
$R\Gamma$-module and as right $R\Lambda$-module, and $Y_R=R\otimes_k Y$ is projective as left 
$R\Lambda$-module and as right $R\Gamma$-module.
Since $X_R\otimes_{R\Lambda} (Y_R) \cong R\otimes_k(X\otimes_\Lambda Y)$, we have, using 
(\ref{eq:stab}),
\begin{eqnarray*}
Y_R\otimes_{R\Gamma} X_R&\cong& R\Lambda \oplus P_R \quad\mbox{ as 
$R\Lambda$-$R\Lambda$-bimodules, and}\\
X_R\otimes_{R\Lambda} Y_R&\cong &R\Gamma\oplus Q_R \quad\mbox{ as 
$R\Gamma$-$R\Gamma$-bimodules},
\end{eqnarray*}
where $P_R=R\otimes_k P$ is a projective $R\Lambda$-$R\Lambda$-bimodule and
$Q_R=R\otimes_k Q$ is a projective $R\Gamma$-$R\Gamma$-bimodule.

Since $P$  is a projective $\Lambda$-$\Lambda$-bimodule, it follows that $P\otimes_\Lambda V$
is a projective left $\Lambda$-module. By Lemma \ref{lem:addproj} it follows that 
$P\otimes_\Lambda V$ has a universal deformation ring, which is isomorphic to $k$.
In particular, every lift of $P\otimes_\Lambda V$ over $R$ is isomorphic to 
$\left(R\otimes_k (P\otimes_\Lambda V), \pi_{R,P\otimes_\Lambda V}\right)$, where
$\pi_{R,P\otimes_\Lambda V}:k\otimes_R \left(R\otimes_k (P\otimes_\Lambda V)\right)
\to P\otimes_\Lambda V$ is the natural isomorphism of $\Lambda$-modules.

Let now $(M,\phi)$ be a lift of $V$ over $R$. Then $M$ is a finitely generated $R\Lambda$-module.
Define $M'=X_R\otimes_{R\Lambda}M$.
Since $X_R$ is a finitely generated projective right $R\Lambda$-module and since $M$ is a finitely 
generated abstractly free $R$-module, it follows that $M'$ is a finitely generated projective, 
and hence abstractly free, $R$-module. 

Next we note that we can view $\Lambda\mbox{-mod}$ as the full subcategory of $R\Lambda\mbox{-mod}$
consisting of all finitely generated $R\Lambda$-modules on which the maximal ideal $m_R$ of $R$ acts 
trivially. Moreover, on $\Lambda\mbox{-mod}$ the functor $X_R\otimes_{R\Lambda} -$
coincides with the functor $X\otimes_{\Lambda} -$.
Define $\phi'=X_R\otimes_{R\Lambda}\phi$, so $\phi'$ is a $\Gamma$-module isomorphism, since
$X_R$ is projective as a right $R\Lambda$-module. 
Then
\begin{equation}
\label{eq:lala}
M'\otimes_Rk=(X_R\otimes_{R\Lambda}M)\otimes_Rk = X_R\otimes_{R\Lambda}(M\otimes_Rk)
\xrightarrow{\phi'} X_R\otimes_{R\Lambda}V =V'
\end{equation}
which means $(M',\phi')$ is a lift of $V'$ over $R$. We therefore obtain for all $R\in\mathrm{Ob}(\mathcal{C})$ a well-defined map 
\begin{eqnarray*}
\tau_R:\mathrm{Def}_\Lambda(V,R)&\to&\mathrm{Def}_\Gamma(V',R)\\
{[M,\phi]}&\mapsto&[M',\phi']=[X_R\otimes_{R\Lambda}M,X_R\otimes_{R\Lambda}\phi]\,.
\end{eqnarray*}

We need to show that $\tau_R$ is bijective.
Arguing as in (\ref{eq:lala}), we see that $(Y_R\otimes_{R\Gamma}M',Y_R\otimes_{R\Gamma}\phi')$
is a lift of $Y\otimes_\Gamma V'\cong V \oplus (P\otimes_\Lambda V)$ over $R$. Moreover, 
\begin{eqnarray}
\label{eq:olala}
(Y_R\otimes_{R\Gamma}M',Y_R\otimes_{R\Gamma}\phi')
&\cong&((R\Lambda\oplus P_R)\otimes_{R\Lambda}M, (R\Lambda\oplus P_R)\otimes_{R\Lambda}\phi) \\
&\cong&(M\oplus (P_R\otimes_{R\Lambda}M), \phi\oplus(P_R\otimes_{R\Lambda}\phi)). \nonumber
\end{eqnarray}
Since $(P_R\otimes_{R\Lambda}M,P_R\otimes_{R\Lambda}\phi)$ is
a lift of the projective $\Lambda$-module $P\otimes_\Lambda V$ over $R$,
it follows from Lemma \ref{lem:addproj} that $\tau_R$ is injective.

Now let $(L,\psi)$ be a lift of $V'=X\otimes_{\Lambda}V$ over $R$.  Then 
$(L',\psi')= (Y_R\otimes_{R\Gamma}L,Y_R\otimes_{R\Gamma}\psi)$ is a lift of 
$V''=Y\otimes_{\Gamma}V'\cong V \oplus (P\otimes_\Lambda V)$ over $R$. By Lemma \ref{lem:addproj}, there exists a lift 
$(M,\phi)$ of $V$ over $R$ such that $(L',\psi')$ is isomorphic to the lift
$\left(M\oplus (R\otimes_k (P\otimes_\Lambda V)),\phi\oplus\pi_{R,P\otimes_\Lambda V}\right)$.
Arguing similarly as in (\ref{eq:olala}), we then have that $(L',\psi')$ is isomorphic to $(M'',\phi'')=
(Y_R\otimes_{R\Gamma}M',Y_R\otimes_{R\Gamma}\phi')$ where $(M',\phi')=(X_R \otimes_{R\Lambda}M,X_R\otimes_{R\Lambda}\phi)$. Therefore, $(X_R\otimes_{R\Lambda}L' ,X_R\otimes_{R\Lambda}\psi')\cong (X_R\otimes_{R\Lambda}M'',X_R\otimes_{R\Lambda}\phi'')$.
Arguing again similarly as in (\ref{eq:olala}) and using Lemma \ref{lem:addproj}, we have 
\begin{eqnarray*}
(X_R\otimes_{R\Lambda}L' ,X_R\otimes_{R\Lambda}\psi') 
&\cong&\left(L\oplus (R\otimes_k (Q\otimes_\Gamma V')),\psi\oplus \pi_{R,Q\otimes_\Gamma V'}\right),
\mbox{ and }\\
(X_R\otimes_{R\Lambda}M'',X_R\otimes_{R\Lambda}\phi'')
&\cong&\left(M' \oplus (R\otimes_k (Q\otimes_\Gamma V')),\phi'\oplus \pi_{R,Q\otimes_\Gamma V'}\right).
\end{eqnarray*}
Thus by Lemma \ref{lem:addproj}, it follows that $(L,\psi)\cong (M',\phi')$, i.e. $\tau_R$ is surjective.

To show that the maps $\tau_R$ are natural with respect to morphisms $\alpha:R\to R'$ in 
$\mathcal{C}$, consider $(M,\phi)$ and $(M',\phi')$ as above.
Since $X_R$ is a projective right $R\Lambda$-module and
$M$ is a free $R$-module, there exists a natural isomorphism
$$f\;:\quad R' \otimes_{R,\alpha}M'= 
R'\otimes_{R,\alpha}(X_R\otimes_{R\Lambda}M)
\to X_{R'}\otimes_{R'\Lambda}(R'\otimes_{R,\alpha} M)$$
of $R'\Gamma$-modules. It is straightforward to see that $f$ provides an isomorphism between the
lifts $(R'\otimes_{R,\alpha}M',(\phi')_\alpha)$ and 
$(X_{R'} \otimes_{R'\Lambda} (R'\otimes_{R,\alpha}M),X_{R'}\otimes_{R'\Lambda}(\phi_\alpha))$
of $V'$ over $R'$.

Since the deformation functors $\hat{F}_V$ and $\hat{F}_{V'}$ are continuous, this implies that they are naturally isomorphic. Hence the versal deformation rings $R(\Lambda,V)$ and $R(\Gamma,V')$ are isomorphic in $\hat{\mathcal{C}}$.
\end{proof}

\begin{rem}
\label{rem:stabmor}
Using the notation of Proposition \ref{prop:stabmordef}, suppose that the stable endomorphism ring 
$\underline{\mathrm{End}}_{\Lambda}(V)$ is isomorphic to $k$. Then it follows that also $\underline{\mathrm{End}}_{\Gamma}(V')\cong k$. 
Moreover, there exists a non-projective indecomposable $\Gamma$-module $V'_0$ (unique up to isomorphism) which is a 
direct summand of  $V'$ with $\underline{\mathrm{End}}_{\Gamma}(V'_0)\cong k$ and $R(\Gamma,V')\cong R(\Gamma,V'_0)$
(see Lemma \ref{lem:addproj}). 
It then follows that $R(\Lambda,V)\cong R(\Gamma,V'_0)$. \end{rem}

\begin{rem}
\label{rem:stablegood}
Let $\Lambda$ be a finite dimensional $k$-algebra, and denote by $\mathrm{mod}_{\mathcal{P}}(\Lambda)$ the full subcategory of $\Lambda\mbox{-mod}$ whose objects are the modules which have no non-zero projective summands.
Suppose $\Gamma$ is another finite dimensional $k$-algebra, and 
$G:\Lambda\mbox{-\underline{mod}}\to \Gamma\mbox{-\underline{mod}}$ is a stable equivalence. 
Let
$$0\to A \xrightarrow{\genfrac{(}{)}{0pt}{}{\mbox{\tiny $f$}}{\mbox{\tiny $s$}}} B \oplus P \xrightarrow{(g,t)} C\to 0$$
be an almost split sequence in $\Lambda\mbox{-mod}$ where $A,B,C$ are in $\mathrm{mod}_{\mathcal{P}}(\Lambda)$, $B$ is non-zero 
and $P$ is projective. Then, by \cite[Prop. X.1.6]{ars}, for any morphism $g':G(B)\to G(C)$ with $G(g)=g'$ in 
$\Gamma\mbox{-\underline{mod}}$,  there is an almost split sequence
$$0\to G(A) \xrightarrow{\genfrac{(}{)}{0pt}{}{\mbox{\tiny $f'$}}{\mbox{\tiny $u$}}} G(B)  \oplus P' \xrightarrow{(g',v)} G(C)\to 0$$
in $\Gamma\mbox{-mod}$ where $P'$ is projective and $G(f)=f'$ in $\Gamma\mbox{-\underline{mod}}$.

Moreover, by \cite[Cor. X.1.9 and Prop. X.1.12]{ars}, if $\Lambda$ and $\Gamma$ are self-injective with no blocks of Loewy length $2$, 
then the stable Auslander-Reiten quivers of $\Lambda$ and $\Gamma$ are isomorphic stable translation quivers, and $G$ commutes with the syzygy functors $\Omega$.
\end{rem}

\section{A family of examples}
\label{s:examples}
\setcounter{equation}{0}

We use the notation from Section \ref{s:derivedequivalences}. Moreover, we assume that $k$ is an algebraically closed field.
In this section, we consider the derived equivalence classes of the family of symmetric $k$-algebras
$D(3\mathcal{R})$ introduced by Erdmann in \cite{erd}.  In Section \ref{s:family}, we describe these derived equivalence classes,
which were obtained by Holm in \cite[Sect. 3.2]{holm}. In Section \ref{s:exampleresults}, we apply the results from Section
\ref{s:derivedequivalences} together with the results in \cite{blehervelez} and \cite{velez2015} to
obtain universal deformation rings of $\Lambda$-modules for
other algebras $\Lambda$ of dihedral type that are derived equivalent to $D(3\mathcal{R})$.

\subsection{The derived equivalence classes of the algebras $D(3\mathcal{R})$}
\label{s:family}

In \cite{erd}, Erdmann introduced the symmetric $k$-algebras of dihedral type
$D(3\mathcal{R})^{a,b,c,d}$, for integers $a\ge 1$, $b,c,d\ge 2$, see Figure \ref{fig:algebra3R}.
\begin{figure}[ht] \hrule \caption{\label{fig:algebra3R} The family $D(3\mathcal{R})^{a,b,c,d}=k[3\mathcal{R}]/I_{3\mathcal{R},a,b,c,d}$
for $r\ge 1$; $s,t,u\ge 2$.}
$$3\mathcal{R}=\vcenter{\xymatrix @R=-.1pc {
0&&1\\
\ar@(ul,dl)_{\alpha} \bullet \ar[rr]^{\beta}  &&\bullet\ar[ldddddddd]^{\delta} \ar@(ur,dr)^{\rho} 
\\&&\\&&\\&&\\&&\\&&\\&&\\&2&\\ 
&
\bullet\ar[uuuuuuuul]^{\lambda}
\ar@(ld,rd)_{\xi} & }}$$
$$I_{3\mathcal{R},a,b,c,d}=\langle \alpha\lambda,\lambda\xi,
\xi\delta,\delta\rho,\rho\beta,\beta\alpha,
\alpha^b-(\lambda\delta\beta)^a,\rho^c-(\beta\lambda\delta)^a,\xi^d-(\delta\beta\lambda)^a\rangle.$$
\vspace{1ex}
\hrule
\end{figure}

In \cite[Sect. 3.2]{holm}, Holm determined all algebras of dihedral type with precisely three isomorphism
classes of simple modules which are derived equivalent to $D(3\mathcal{R})^{a,b,c,d}$ for various $a,b\ge  1$; $c,d\ge 2$. 
Note that for $a\ge 1$; $c,d\ge 2$, we define the algebra $D(3\mathcal{R})^{a,1,c,d}$ as in Figure \ref{fig:algebra3R'}.
In particular, $D(3\mathcal{R})^{a,1,c,d}\cong D(3\mathcal{R})^{a,b,c,d}/\langle \alpha - (\lambda\delta\beta)^a\rangle$ for all $b\ge 2$.
\begin{figure}[ht] \hrule \caption{\label{fig:algebra3R'} The family $D(3\mathcal{R})^{a,1,c,d}=k[3\mathcal{R'}]/I_{3\mathcal{R}',a,c,d}$
for $a\ge 1$; $c,d\ge 2$.}
$$3\mathcal{R}'=\vcenter{\xymatrix @R=-.1pc {
0&&1\\
\bullet \ar[rr]^{\beta}  &&\bullet\ar[ldddddddd]^{\delta} \ar@(ur,dr)^{\rho} 
\\&&\\&&\\&&\\&&\\&&\\&&\\&2&\\ 
&
\bullet\ar[uuuuuuuul]^{\lambda}
\ar@(ld,rd)_{\xi} & }}$$
$$I_{3\mathcal{R}',a,c,d}= \langle \lambda\xi,\xi\delta,\delta\rho,\rho\beta,
\rho^c-(\beta\lambda\delta)^a,\xi^d-(\delta\beta\lambda)^a\rangle.$$
\vspace{1ex}
\hrule
\end{figure}

Holm showed in \cite{holm} that no block of a group algebra with dihedral defect groups is derived equivalent to $D(3\mathcal{R})^{a,b,c,d}$
for any $a,b\ge 1$; $c,d\ge 2$. Note that up to derived equivalence, we can order $1\le a\le b\le c\le d$; $2\le c$ in $D(3\mathcal{R})^{a,b,c,d}$.
By \cite{erd}, there are precisely five additional families of Morita equivalence classes of algebras of dihedral type 
that are derived equivalent to the algebras in the family $D(3\mathcal{R})^{a,b,c,d}$, $a,b\ge 1$; $c,d\ge 2$. 
We list these Morita equivalence classes in Figure \ref{fig:derived3R}. 
\begin{figure}[ht] \hrule \caption{\label{fig:derived3R} Morita equivalence classes of algebras of dihedral type that are derived
equivalent to $D(3\mathcal{R})^{a,b,c,d}$ for various $a,b\ge 1$; $c,d\ge 2$.}
\begin{enumerate}
\item[(A)] The family $D(3\mathcal{Q})^{b,c,d}=k[3\mathcal{Q}]/I_{3\mathcal{Q},b,c,d}$,
$b\ge 1$; $c,d\ge 2$, which is derived equivalent to $D(3\mathcal{R})^{1,b,c,d}\;$:
$$3\mathcal{Q}=\vcenter{\xymatrix @R=-.1pc {
0&&1\\
\ar@(ul,dl)_{\alpha} \bullet \ar[rr]^{\beta}  &&\bullet\ar[ldddddddd]^{\delta} \ar@(ur,dr)^{\rho} 
\\&&\\&&\\&&\\&&\\&&\\&&\\&&\\ 
&
\bullet\ar[uuuuuuuul]^{\lambda}\\
&2&}}$$
$$I_{3\mathcal{Q},b,c,d}= \langle \alpha\lambda,\delta\rho,\rho\beta,\beta\alpha,
\alpha^c-(\lambda\delta\beta)^b,\rho^d-(\beta\lambda\delta)^b\rangle.$$
\vspace{2ex}
\item[(B)]
The family $D(3\mathcal{L})^{c,d}=k[3\mathcal{L}]/I_{3\mathcal{L},c,d}$,
$c,d\ge 2$, which is derived equivalent to $D(3\mathcal{R})^{1,1,c,d}\;$:
$$3\mathcal{L}=\vcenter{\xymatrix @R=-.1pc {
0&&1\\
\ar@(ul,dl)_{\alpha} \bullet \ar[rr]^{\beta}  &&\bullet\ar[ldddddddd]^{\delta} 
\\&&\\&&\\&&\\&&\\&&\\&&\\&&\\ 
&
\bullet\ar[uuuuuuuul]^{\lambda}\\
&2&}}$$
$$I_{3\mathcal{L},c,d}=\langle \alpha\lambda,\beta\alpha,
\alpha^d-(\lambda\delta\beta)^c,\delta(\beta\lambda\delta)^c\rangle.$$
\vspace{2ex}
\item[(C)] 
The family $D(3\mathcal{A})_2^{c,d}=k[3\mathcal{A}]/I_{(3\mathcal{A})_2,c,d}$,
$c\ge d\ge 2$, which is derived equivalent to $D(3\mathcal{R})^{1,1,c,d}\;$:
$$3\mathcal{A}=\vcenter{\xymatrix @R=-.1pc {
&0&\\
1\; \bullet \ar@<.8ex>[r]^(.56){\beta} \ar@<1ex>[r];[]^(.44){\gamma}
& \bullet \ar@<.8ex>[r]^(.44){\delta} \ar@<1ex>[r];[]^(.56){\eta} & \bullet\; 2}}$$
$$I_{(3\mathcal{A})_2,c,d}=\langle \gamma\eta,\delta\beta,(\beta\gamma)^c-(\eta\delta)^d\rangle.$$
\vspace{2ex}
\item[(D)] 
The family $D(3\mathcal{B})_2^{b,c,d}=k[3\mathcal{B}]/I_{(3\mathcal{B})_2,b,c,d}$,
$b,c\ge 1$ $(b+c> 2)$; $d\ge 2$ which is derived equivalent to $D(3\mathcal{R})^{1,b,c,d}$ if $c\ge 2$ and to
$D(3\mathcal{R})^{1,c,b,d}$ if $c=1$ (and hence $b\ge 2$)\;:
$$3\mathcal{B}=\vcenter{\xymatrix @R=-.1pc {
&1&0&\\
&\ar@(ul,dl)_{\alpha} \bullet \ar@<.8ex>[r]^{\beta} \ar@<.9ex>[r];[]^{\gamma}
& \bullet \ar@<.8ex>[r]^(.46){\delta} \ar@<.9ex>[r];[]^(.54){\eta} & \bullet\;2}}$$
$$I_{(3\mathcal{B})_2,b,c,d}=\langle \alpha\gamma,\beta\alpha,\gamma\eta,\delta\beta,
\alpha^d-(\gamma\beta)^b,(\beta\gamma)^b-(\eta\delta)^c\rangle.$$
\vspace{2ex}
\item[(E)] 
The family $D(3\mathcal{D})_2^{a,b,c,d}=k[3\mathcal{D}]/I_{(3\mathcal{D})_2,a,b,c,d}$,
$a,b\ge 1$; $c,d\ge 2$ which is derived equivalent to $D(3\mathcal{R})^{a,b,c,d}$:
$$3\mathcal{D}=\vcenter{\xymatrix @R=-.1pc {
&1&0&2\\
&\ar@(ul,dl)_{\alpha} \bullet \ar@<.8ex>[r]^{\beta} \ar@<.9ex>[r];[]^{\gamma}
& \bullet \ar@<.8ex>[r]^(.46){\delta} \ar@<.9ex>[r];[]^(.54){\eta} & \bullet \ar@(ur,dr)^{\xi}}}$$
$$I_{(3\mathcal{D})_2,a,b,c,d}=\langle \alpha\gamma,\beta\alpha,\gamma\eta,\delta\beta,
\eta\xi,\xi\delta,\alpha^c-(\gamma\beta)^a,(\beta\gamma)^a-(\eta\delta)^b,\xi^d-(\delta\eta)^b\rangle.$$
\end{enumerate}
\vspace{1ex}
\hrule
\end{figure}

\subsection{Universal deformation rings for algebras in the derived equivalence class of $D(3\mathcal{R})$}
\label{s:exampleresults}

In \cite{blehertalbott}, \cite{blehervelez} and \cite{velez2015}, the universal deformation rings of certain modules of
$D(3\mathcal{R})^{a,b,c,d}$ were determined for various $a,b\ge 1$; $c,d\ge 2$.

\begin{rem}
\label{rem:derivedex1}
In \cite{blehertalbott}, the first author and S. Talbott studied the case $D(3\mathcal{R})^{1,1,2,2}$, which is of polynomial growth, 
and determined all indecomposable $D(3\mathcal{R})^{1,1,2,2}$-modules whose {\bf stable} endomorphism rings are isomorphic to $k$,
together with their universal deformation rings.

Using Figure \ref{fig:derived3R}, we obtain the following derived equivalence class of algebras of dihedral type that are derived equivalent to
$D(3\mathcal{R})^{1,1,2,2}$:
$$D(3\mathcal{R})^{1,1,2,2} \sim D(3\mathcal{Q})^{1,2,2} \sim D(3\mathcal{L})^{2,2} \sim
D(3\mathcal{A})_2^{2,2}\sim D(3\mathcal{B})_2^{1,2,2}\sim D(3\mathcal{B})_2^{2,1,2}\sim D(3\mathcal{D})_2^{1,1,2,2}\;.$$
By Proposition \ref{prop:stable1}, it follows 
that if $\Lambda$ is any of the algebras in this derived equivalence class and $V$ is an indecomposable $\Lambda$-module
with $\underline{\mathrm{End}}_\Lambda(V)\cong k$, then $R(\Lambda,V)\cong R(D(3\mathcal{R})^{1,1,2,2},V')$ if $V$
corresponds to $V'$ under the stable  equivalence of Morita type between $\Lambda$ and $D(3\mathcal{R})^{1,1,2,2}$ (which is
induced by the derived equivalence). By 
Remark \ref{rem:stablegood}, we see additionally that the stable Auslander-Reiten quiver components of $V$ and $V'$
match up, including the relative positions of $V$ and $V'$ in these components. This reaffirms the results in \cite[Thm. 1.1]{blehertalbott}
(see also  \cite[Props. 3.1--3.3]{blehertalbott}) for these algebras.
\end{rem}

In \cite[Sect. 3]{blehervelez}, the authors studied the algebra $D(3\mathcal{R})^{1,2,2,2}$  and determined all
indecomposable $D(3\mathcal{R})^{1,2,2,2}$-modules whose {\bf stable} endomorphism rings are isomorphic to $k$,
together with their universal deformation rings. 

Using Proposition \ref{prop:stable1}, Remark \ref{rem:stablegood} and 
\cite[Thm. 1.2]{blehervelez} (see also \cite[Thm. 3.8, Props. 3.9--3.11]{blehervelez}), 
we obtain the following result for all algebras of dihedral type  in the derived equivalence class of $D(3\mathcal{R})^{1,2,2,2}$.

\begin{thm}
\label{thm:derivedex2}
Let $\Lambda$ be one of the following algebras:
\begin{equation}
\label{eq:1222}
D(3\mathcal{R})^{1,2,2,2}, D(3\mathcal{Q})^{2,2,2}, D(3\mathcal{B})_2^{2,2,2}, D(3\mathcal{D})_2^{1,2,2,2}\,.
\end{equation}
Suppose $\mathfrak{C}$ is a component of the stable Auslander-Reiten quiver $\Gamma_s(\Lambda)$.
\begin{itemize}
\item[(i)] If $\mathfrak{C}$ is one of the two $3$-tubes, then $\Omega(\mathfrak{C})$ is the 
other $3$-tube. There are exactly three $\Omega^2$-orbits of modules
in $\mathfrak{C}$ whose stable endomorphism rings are isomorphic to $k$. 
If $U_0$ is a module that belongs to the boundary of $\mathfrak{C}$, then these three 
$\Omega^2$-orbits are represented by $U_0$, by a successor $U_1$ of $U_0$, and by a 
successor $U_2$ of $U_1$ that does not lie in the $\Omega^2$-orbit of $U_0$. The universal
deformation rings are 
$$R(\Lambda,U_0)\cong R(\Lambda,U_1)\cong k,\quad R(\Lambda,U_2)\cong k[[t]].$$

\item[(ii)] There are infinitely many components of $\Gamma_s(\Lambda)$ of type $\mathbb{Z}
A_\infty^\infty$ that each contain a module whose stable endomorphism ring is isomorphic to $k$.
If $\mathfrak{C}$ is such a component, then $\mathfrak{C}=\Omega(\mathfrak{C})$ and there
are exactly six  $\Omega^2$-orbits $($resp. exactly three $\Omega$-orbits$)$ of modules 
in $\mathfrak{C}$ whose stable endomorphism rings are isomorphic to $k$. These three $\Omega$-orbits are 
represented by a module $V_0$, by a successor $V_1$ of $V_0$ that does not lie in the $\Omega$-orbit of $V_0$, 
and by a successor $V_2$ of $V_1$ that does not lie in the $\Omega^2$-orbit of $V_0$. 
The universal deformation rings are 
$$R(\Lambda,V_0)\cong k[[t]]/(t^2),\quad R(\Lambda,V_1)\cong k,\quad R(\Lambda,V_2)\cong k[[t]].$$

\item[(iii)] There are infinitely many $1$-tubes of $\Gamma_s(\Lambda)$ that each contain a module 
whose stable endomorphism ring is isomorphic to $k$. If $\mathfrak{C}$ is such a component, 
then there is exactly one $\Omega^2$-orbit of modules in $\mathfrak{C}$ whose stable 
endomorphism ring is isomorphic to $k$, represented by a module $W_0$ belonging to the 
boundary of $\mathfrak{C}$. The universal deformation ring of $W_0$ is 
$$R(\Lambda_0,W_0)\cong k[[t]].$$
\end{itemize}
\end{thm}

\begin{rem}
\label{rem:moredetail2}
Suppose $\Lambda$ is one of the algebras in (\ref{eq:1222}), and suppose $V$ is a $\Lambda$-module with
$\mathrm{End}_\Lambda(V)\cong k$. 
\begin{enumerate}
\item[(i)] If $\Lambda=D(3\mathcal{R})^{1,2,2,2}$, then $V$ is one of the modules
$$S_0,S_1,S_2,\begin{array}{c}0\\1\end{array},\begin{array}{c}1\\2\end{array}, \begin{array}{c}2\\0\end{array},
\begin{array}{c}0\\1\\2\end{array}, \begin{array}{c}1\\2\\0\end{array}, \begin{array}{c}2\\0\\1\end{array}\,.$$
Moreover, $R(\Lambda,V)\cong k$ if  $V$ has composition series length 2 or 3, and
$R(\Lambda,V)\cong k[[t]]/(t^2)$ if $V$ is simple.

\item[(ii)] If $\Lambda=D(3\mathcal{Q})^{2,2,2}$, then $V$ is one of the modules
$$S_0,S_1,S_2,\begin{array}{c}0\\1\end{array},\begin{array}{c}1\\2\end{array}, \begin{array}{c}2\\0\end{array},
\begin{array}{c}0\\1\\2\end{array}, \begin{array}{c}1\\2\\0\end{array}, \begin{array}{c}2\\0\\1\end{array}\,.$$
Moreover, $R(\Lambda,V)\cong k$ if $V=S_2$ or $V$ has composition series length 2, and
$R(\Lambda,V)\cong k[[t]]/(t^2)$ if $V\in\{S_0, S_1\}$ or $V$ has composition series length 3.

\item[(iii)] If $\Lambda=D(3\mathcal{B})_2^{2,2,2}$, then $V$ is one of the modules
$$S_0,S_1,S_2,\begin{array}{c}0\\1\end{array},\begin{array}{c}1\\0\end{array}, \begin{array}{c}0\\2\end{array}, \begin{array}{c}2\\0\end{array},
\begin{array}{cc}\multicolumn{2}{c}{0}\\1&2\end{array}, \begin{array}{cc}1&2\\ \multicolumn{2}{c}{0}\end{array}\,.$$
Moreover, $R(\Lambda,V)\cong k$ if $V\in\{S_0,S_2\}$ or $V$ has composition series length 3, and
$R(\Lambda,V)\cong k[[t]]/(t^2)$ if $V=S_1$ or $V$ has composition series length 2.

\item[(iv)] If $\Lambda=D(3\mathcal{D})_2^{1,2,2,2}$, then $V$ is one of the modules
$$S_0,S_1,S_2,\begin{array}{c}0\\1\end{array},\begin{array}{c}1\\0\end{array}, \begin{array}{c}0\\2\end{array}, \begin{array}{c}2\\0\end{array},
\begin{array}{cc}\multicolumn{2}{c}{0}\\1&2\end{array}, \begin{array}{cc}1&2\\ \multicolumn{2}{c}{0}\end{array}\,.$$
Moreover, $R(\Lambda,V)\cong k$ if $V\in\left\{S_0,\begin{array}{c}0\\1\end{array},\begin{array}{c}1\\0\end{array}\right\}$ or $V$ has 
composition series length 3, and
$R(\Lambda,V)\cong k[[t]]/(t^2)$ if $V\in\left\{S_1,S_2,\begin{array}{c}0\\2\end{array},\begin{array}{c}2\\0\end{array}\right\}$.
\end{enumerate}
\end{rem}

In \cite{velez2015}, the second author considered all possible $a\ge 1$, $b,c,d\ge 2$
and determined all indecomposable $D(3\mathcal{R})^{a,b,c,d}$-modules whose (usual) endomorphism rings are isomorphic to $k$. Moreover,
he looked at their components in the stable Auslander-Reiten quiver of $D(3\mathcal{R})^{a,b,c,d}$ and determined all
modules in these components whose stable endomorphism rings are isomorphic to $k$, together with their universal deformation rings. 

We use \cite[Thm. 1.1(iv)]{velez2015} to obtain a result concerning 3-tubes for all allowed parameters
$a,b,c,d$. By using similar arguments as in the proof of \cite[Prop. 4.4]{velez2015}, we can 
include the case of $D(3\mathcal{R})^{a,1,c,d}$ for $a\ge 1$ and $c,d\ge 2$.

\begin{thm}
\label{thm:derivedex3}
Let $\Lambda$ be one of the following algebras:
\begin{equation}
\label{eq:rstu}
D(3\mathcal{R})^{a,b,c,d}, D(3\mathcal{Q})^{b,c,d}, D(3\mathcal{L})^{c,d}, D(3\mathcal{A})_2^{c,d}, D(3\mathcal{B})_2^{b,c,d}, D(3\mathcal{B})_2^{c,b,d}, D(3\mathcal{D})_2^{a,b,c,d}\,,
\end{equation}
where $a,b\ge 1$, $c,d\ge 2$ are allowed parameters according to Figure $\ref{fig:derived3R}$.
Suppose $\mathfrak{T}$ is one of the two $3$-tubes of the stable Auslander-Reiten quiver $\Gamma_s(\Lambda)$.
Then $\Omega(\mathfrak{T})$ is the other $3$-tube, and there are precisely three $\Omega$-orbits of modules in 
$\mathfrak{T}\cup \Omega(\mathfrak{T})$ whose stable endomorphism rings are isomorphic to $k$. 
If $U_0$ is a module that belongs to the boundary of $\mathfrak{T}$, then these three 
$\Omega$-orbits are represented by $U_0$, by a successor $U_1$ of $U_0$, and by a 
successor $U_2$ of $U_1$ with $U_0\not\cong\Omega^2(U_2)$. The universal
deformation rings are 
$$R(\Lambda,U_0)\cong R(\Lambda,U_1)\cong k,\quad R(\Lambda,U_2)\cong k[[t]].$$
\end{thm}

Since
in \cite{velez2015} only those components of the stable Auslander-Reiten quiver of $D(3\mathcal{R})^{a,b,c,d}$,
for all $a\ge 1$, $b,c,d\ge 2$, were studied that contain modules whose (usual) endomorphism rings are isomorphic to $k$,
we can only say something about finitely many components of the stable Auslander-Reiten quiver of type
$\mathbb{Z}A_\infty^\infty$, as far as universal deformation rings are concerned. 
Using \cite[Thm. 1.1(i)-(iii)]{velez2015}, we obtain the following result for all allowed parameters
$a,b,c,d$. By using similar arguments as in the proof of \cite[Props. 4.1-4.3]{velez2015}, we can 
include the case of $D(3\mathcal{R})^{a,1,c,d}$ for $a\ge 1$ and $c,d\ge 2$.

\begin{prop}
\label{prop:derivedexvelez}
Let $\Lambda$ be one of the algebras in $(\ref{eq:rstu})$, where $a,b\ge 1$, $c,d\ge 2$ are allowed parameters according to Figure 
$\ref{fig:derived3R}$. Let $\mathfrak{C}$ be a component of $\Gamma_s(\Lambda)$ of type $\mathbb{Z}A_\infty^\infty$ 
containing a module whose stable endomorphism ring is $k$.
\begin{enumerate}
\item[(i)] Suppose $a=1=b$. Then there is at least one component $\mathfrak{C}$ such that the following is true:
There are precisely three $\Omega$-orbits of modules in $\mathfrak{C}\cup\Omega(\mathfrak{C})$ whose stable endomorphism rings are 
isomorphic to $k$, represented by  $V_0,V_1,V_2$ such that $V_1$ is a successor of $V_0$ that does not lie in the $\Omega$-orbit of $V_0$, 
$V_2$ is a successor of $V_1$ that does not lie in the $\Omega^2$-orbit of $V_0$, and
$$R(\Lambda,V_0)\cong k[[t]]/(t^c), \quad R(\Lambda,V_1)\cong k, \quad R(\Lambda,V_2)\cong k[[t]]/(t^d).$$
Moreover, $\mathfrak{C}=\Omega(\mathfrak{C})$ if and only if $c=2$ or $d=2$.
\item[(ii)] Suppose $a=1$ and $b\ge 2$. Then there are at least three components 
$\mathfrak{C}_{1,b}$, $\mathfrak{C}_{2,c}$, $\mathfrak{C}_{3,d}$,
such that the following is true for $\mathfrak{C}_{i,j}$:
There are precisely three $\Omega$-orbits of modules in $\mathfrak{C}_{i,j}\cup\Omega(\mathfrak{C}_{i,j})$ whose stable endomorphism rings 
are isomorphic to $k$, represented by $V_{i,j,0},V_{i,j,1},V_{i,j,2}$ such that $V_{i,j,1}$ is a successor of $V_{i,j,0}$ that does not lie in the 
$\Omega$-orbit of $V_{i,j,0}$, $V_{i,j,2}$ is a successor of $V_{i,j,1}$ that does not lie in the $\Omega^2$-orbit of $V_{i,j,0}$, and
$$R(\Lambda,V_{i,j,0})\cong k[[t]]/(t^j), \quad R(\Lambda,V_{i,j,1})\cong k, \quad R(\Lambda,V_{i,j,2})\cong k[[t]].$$
Moreover, $\mathfrak{C}_{i,j}=\Omega(\mathfrak{C}_{i,j})$ if and only if $j=2$.
\item[(iii)] Suppose $a\ge 2$ and $b=1$. Then there are at least four components 
$\mathfrak{C}_{1,a}$, $\mathfrak{C}_{2,a}$, $\mathfrak{C}_{3,c}$, $\mathfrak{C}_{4,d}$,
such that the following is true for $\mathfrak{C}_{i,j}$:
There are precisely three $\Omega$-orbits of modules in $\mathfrak{C}_{i,j}\cup\Omega(\mathfrak{C}_{i,j})$ whose stable endomorphism rings 
are isomorphic to $k$, represented by $V_{i,j,0},V_{i,j,1},V_{i,j,2}$ such that $V_{i,j,1}$ is a successor of $V_{i,j,0}$ that does not lie in the 
$\Omega$-orbit of $V_{i,j,0}$, $V_{i,j,2}$ is a successor of $V_{i,j,1}$ that does not lie in the $\Omega^2$-orbit of $V_{i,j,0}$, and
$$R(\Lambda,V_{i,j,0})\cong k[[t]]/(t^j), \quad R(\Lambda,V_{i,j,1})\cong k, \quad R(\Lambda,V_{i,j,2})\cong k[[t]].$$
Moreover, $\mathfrak{C}_{i,j}=\Omega(\mathfrak{C}_{i,j})$ if and only if $j=2$.
\item[(iv)] Suppose $a,b\ge 2$. Then there are at least nine components 
$\mathfrak{C}_{1,a}$, $\mathfrak{C}_{2,a}$, $\mathfrak{C}_{3,a}$, $\mathfrak{C}_{4,b}$, $\mathfrak{C}_{5,c}$, $\mathfrak{C}_{6,d}$,
$\mathfrak{C}_{7,\infty}$, $\mathfrak{C}_{8,\infty}$, $\mathfrak{C}_{9,\infty}$, 
such that the following is true for $\mathfrak{C}_{i,j}$:
There are precisely three $\Omega$-orbits of modules in $\mathfrak{C}_{i,j}\cup\Omega(\mathfrak{C}_{i,j})$ whose stable endomorphism rings 
are isomorphic to $k$, represented by $V_{i,j,0},V_{i,j,1},V_{i,j,2}$ such that $V_{i,j,1}$ is a successor of $V_{i,j,0}$ that does not lie in the 
$\Omega$-orbit of $V_{i,j,0}$, $V_{i,j,2}$ is a successor of $V_{i,j,1}$ that does not lie in the $\Omega^2$-orbit of $V_{i,j,0}$, and
$$\qquad \,\quad R(\Lambda,V_{i,j,0})\cong \left\{\begin{array}{c@{\quad:\quad}l}k[[t]]/(t^j)&j\neq\infty\\k[[t]]&j=\infty\end{array}\right\}, \quad R(\Lambda,V_{i,j,1})\cong k, \quad R(\Lambda,V_{i,j,2})\cong k[[t]].$$
Moreover, $\mathfrak{C}_{i,j}=\Omega(\mathfrak{C}_{i,j})$ if and only if $j=2$.
\end{enumerate}
\end{prop}

\begin{rem}
\label{rem:infinite}
Suppose $\Lambda$ is one of the algebras in $(\ref{eq:rstu})$, where $a,b\ge 1$, $c,d\ge 2$ are allowed parameters according to 
Figure $\ref{fig:derived3R}$. Moreover assume $(a,b,c,d)\neq (1,1,2,2)$. 
Note that by \cite[Lemma 3.15]{holm}, $D(3\mathcal{R})^{a,b,c,d}$ and $D(3\mathcal{R})^{b,a,c,d}$ are derived equivalent for all
$a,b\ge 1$; $c,d\ge 2$. In view of Theorem \ref{thm:derivedex3}(ii), it seems plausible that there are usually infinitely many components of 
the stable Auslander-Reiten quiver of $\Lambda$ of type $\mathbb{Z}A_\infty^\infty$ that contain modules whose stable endomorphism 
rings are isomorphic to $k$.
\end{rem}

\appendix
\section{Continuity of deformation functors}
\label{s:continuity}
\setcounter{equation}{0}

In this appendix, we revisit the proof of the continuity of the deformation functor defined in 
\cite{bcderived},
which also has some bearing on the proof of the continuity of the deformation functor defined in 
Definition \ref{def:functordef} (see Step 4 of Section \ref{s:proofbigthm}).
The main point is that in the proof of the injectivity of the map $\Xi_{\mathcal{D}}$ defined in
the proof of \cite[Prop. 7.2]{bcderived} an assumption was made to arrive at a morphism 
$f_i$ as in \cite[Eq. (7.5)]{bcderived}, which needs more explanation. We will provide the necessary 
arguments in this section. We make the same assumptions as in \cite{bcderived}, which are as follows.

\begin{hypo}
\label{hypo:derivedpaper}
Suppose $G$ is a profinite group with finite pseudocompact cohomology, as defined in \cite[Def. 2.13]{bcderived}. Assume $V^\bullet $ is a
complex in $D^-([[kG]])$ which has  only finitely many non-zero cohomology
groups, all of which have finite $k$-dimension. 
Let $\hat{F}_{\mathcal{D}}=\hat{F}_{\mathcal{D},V^\bullet}:\hat{\mathcal{C}} \to 
\mathrm{Sets}$ be the deformation functor defined in \cite[Def. 2.10]{bcderived}.

More precisely, let $\hat{F} = \hat{F}_{V^\bullet}:\hat{\mathcal{C}} \to \mathrm{Sets}$
(resp. $\hat{F}^{\mathrm{f\/l}} = \hat{F}^{\mathrm{f\/l}}_{V^\bullet}:\hat{\mathcal{C}} \to \mathrm{Sets}$)
be the functor which sends an object $R$ of $\hat{\mathcal{C}}$ to the set
$\hat{F}(R)$ (resp. $\hat{F}^{\mathrm{f\/l}}(R)$) of all deformations (resp. all proflat deformations)
of $V^\bullet$ over $R$, and which sends
a morphism $\alpha:R\to R'$ in $\hat{\mathcal{C}}$ to the set map
$\hat{F}(R)\to \hat{F}(R')$ (resp. $\hat{F}^{\mathrm{f\/l}}(R)\to \hat{F}^{\mathrm{f\/l}}(R')$)
induced by $M^\bullet \mapsto R'\hat{\otimes}_{R,\alpha}^{\mathbf{L}}
M^\bullet$. Let  $\mathcal{D}$  denote the empty
condition in the case of the functor $\hat{F}$, and the condition of having
topologically free cohomology groups
in the case of the functor $\hat{F}^{\mathrm{f\/l}}$, and use the notation $\hat{F}_{\mathcal{D}}$
to refer to both $\hat{F}$ and $\hat{F}^{\mathrm{f\/l}}$.
\end{hypo}

\begin{secprop}
\label{prop:continuityagain}
Assume Hypothesis $\ref{hypo:derivedpaper}$. The functor $\hat{F}_{\mathcal{D}}:\hat{\mathcal{C}} \to 
\mathrm{Sets}$ is continuous.
\end{secprop}

\begin{proof}
By \cite[Prop. 2.12 and Cor. 3.6(i)]{bcderived} we may assume, without loss of generality,
that there is a closed normal subgroup $\Delta$ of finite index in $G$ such that
$V^\bullet$ is a bounded above complex of abstractly free finitely generated 
$[k(G/\Delta)]$-modules.
Let $R$ be an object in $\hat{\mathcal{C}}$ with maximal ideal $m_R$, and consider the natural map
\begin{equation}
\label{eq:themap}
\Xi_{\mathcal{D}}: \hat{F}_{\mathcal{D}}(R) \to \lim_{\stackrel{\longleftarrow}{i}}
\hat{F}_{\mathcal{D}}(R/m_R^i)
\end{equation}
defined by 
$$\Xi_{\mathcal{D}}((M^\bullet,\phi))= \{(\,(R/m_R^i) \hat{\otimes}^{\mathbf{L}}_R
M^\bullet,(R/m_R^i) \hat{\otimes}^{\mathbf{L}}_R\phi \;)\}_{i=1}^\infty\;.$$
The proof that $\Xi_{\mathcal{D}}$ is surjective is the same as in the proof of \cite[Prop. 7.2]{bcderived}.
Hence it suffices to show that $\Xi_{\mathcal{D}}$ is injective. Since $\hat{F}^{\mathrm{f\/l}}$
is a subfunctor of $\hat{F}$ by \cite[Prop. 2.12]{bcderived}, it is enough
to show that $\Xi_{\mathcal{D}}$ is injective in case $\hat{F}_{\mathcal{D}}
=\hat{F}$. We abbreviate $\Xi_{\mathcal{D}}$ by $\Xi$ in this case.

\medskip

\noindent
{\sc Claim 1.} Let $(M^\bullet, \phi)\in \hat{F}(R)$. For all positive integers $i$, define
$$Z_i=\{\zeta_i\in\mathrm{End}_{D^-([[(R/m_R^i)G]])}((R/m_R^i)\hat{\otimes}^{\mathbf{L}}_R {M}^\bullet)\;
|\;k\hat{\otimes}^{\mathbf{L}}_{R/m_R^{i}}\zeta_i=0\mbox{ in } D^-([[kG]])\}\,.$$
Then $Z_i$ is a finitely generated nilpotent $(R/m_R^i)$-module.

\medskip

\noindent
\textit{Proof of Claim $1$.} 
By \cite[Lemma 4.2]{bcderived}, we can assume without loss of generality that $M^\bullet$ is a bounded
above complex of topologically free pseudocompact $[[RG]]$-modules. Hence
\begin{equation}
\label{eq:nicerzi}
Z_i=\{\zeta_i\in\mathrm{End}_{K^-([[(R/m_R^i)G]])}((R/m_R^i)\hat{\otimes}_R {M}^\bullet)\;
|\;k\hat{\otimes}_{R/m_R^{i}}\zeta_i=0\mbox{ in } K^-([[kG]])\}\,.
\end{equation}
It is obvious that $Z_i$ is an $(R/m_R^i)$-module. Suppose ${\mathrm{H}}^j(V^\bullet)=0$ for $j<n_1$ and
$j>n_2$. By \cite[Thm. 2.10]{obstructions}, it follows that also 
${\mathrm{H}}^j((R/m_R^i)\hat{\otimes}_R {M}^\bullet)=0$ for $j<n_1$ and $j>n_2$.
By \cite[Cor. 3.6(i)]{bcderived},  there exists a closed normal subgroup $\Delta_i$ of finite index in $G$ 
and a bounded above complex $N_i^\bullet$ of abstractly free finitely generated 
$[(R/m_R^i)(G/\Delta_i)]$-modules such that there is an isomorphism
$\delta: (R/m_R^i)\hat{\otimes}_R {M}^\bullet\to \mathrm{Inf}_{G/\Delta_i}^G(N_i^\bullet)$ in 
$K^{-}_{\mathrm{f\/in}}([[(R/m_r^i)G]])$. Hence we can truncate $N_i^\bullet$ at $n_1$ and $n_2$ to be able to 
assume that $N_i^j=0$ for $j<n_1$ and $j>n_2$ and that $N_i^j$ is an abstractly free finitely generated 
$(R/m_R^i)$-module for $n_1\le j\le n_2$. We obtain that $Z_i$ is an $(R/m_R^i)$-submodule of
$$\mathrm{End}_{K^-(R/m_R^i)}((R/m_R^i)\hat{\otimes}_R {M}^\bullet)\cong
\mathrm{End}_{K^-(R/m_R^i)}(N_i^\bullet)\,,$$
which is isomorphic to a quotient module of $\mathrm{End}_{C^-(R/m_R^i)}(N_i^\bullet)$. This, in
turn, is a submodule of $\prod_{j=n_1}^{n_2}\mathrm{End}_{R/m_R^i}(N_i^j)$, which is a free
$(R/m_R^i)$-module of finite rank. Since $R/m_R^i$ is Noetherian, it follows that
$Z_i$ is a finitely generated $(R/m_R^i)$-module. 
Since $(\zeta_i)^i=0$ in $K^-([[(R/m_R^i)G]])$ for all $\zeta_i\in Z_i$, Claim 1 follows.

\medskip

\noindent
{\sc Claim 2.} Let $(M^\bullet,\phi)$ and $Z_i$ be as in Claim 1. 
Then for all positive integers $i$ and for all $j\ge i$, let $\tau^j_i: Z_j\to Z_i$ be the map 
that sends $\zeta_j$ to $(R/m_R^i)\hat{\otimes}^{\mathbf{L}}_{R/m_R^j}\zeta_j$. Then
there exists $N\ge i$ such that for all $j\ge N$, $\tau^j_i(Z_j)=\tau^N_i(Z_N)$. In other words,
the inverse system $\{Z_i, \tau^j_i\}$ satisfies the Mittag-Leffler condition.

\medskip

\noindent
\textit{Proof of Claim $2$.}
By Claim 1, $Z_i$ is a finitely generated module over the Artinian ring $R/m_R^i$. In particular,
$Z_i$ is Artinian. Consider the descending chain of submodules
$$Z_i= \tau^i_i(Z_i)\supseteq \tau^{i+1}_i(Z_{i+1})\supseteq \tau^{i+2}_i(Z_{i+2})\supseteq \ldots$$
Since $Z_i$ is Artinian, this must stabilize after finitely many steps, proving Claim 2.

\medskip

\noindent
{\sc Claim 3.} Suppose $(M^\bullet,\phi), (\widetilde{M}^\bullet,\widetilde{\phi})
\in \hat{F}(R)$ are such that $\Xi((M^\bullet,\phi))=
\Xi((\widetilde{M}^\bullet,\widetilde{\phi}))$. Then for all $i$, there are isomorphisms
$f_i: {(R/m_R^i)\hat{\otimes}^{\mathbf{L}}_R {M}^\bullet} \to 
{(R/m_R^i)\hat{\otimes}^{\mathbf{L}}_R \widetilde{M}^\bullet}$ in $D^-([[(R/m_R^i)G]])$ such that
$(R/m_R^i)\hat{\otimes}^{\mathbf{L}}_{R/m_R^{i}}f_{i+1}= f_i$ in $D^-([[(R/m_R^i)G]])$ and 
$\widetilde{\phi}\circ(k\hat{\otimes}^{\mathbf{L}}_{R/m_R^{i}}f_{i})=\phi$ in $D^-([[kG]])$.

\medskip

\noindent
\textit{Proof of Claim $3$.}
As in the proof of Claim 1, we can assume without loss of generality that $M^\bullet$ and 
$\widetilde{M}^\bullet$ are bounded
above complexes of topologically free pseudocompact $[[RG]]$-modules. In particular, $Z_i$ is
given as in $(\ref{eq:nicerzi})$. For each positive integer $i$, define 
$$S_i=\bigcap_{j\ge i} \tau_i^j(Z_j)\,,$$
which is a subset of $Z_i$, and define $\sigma_i:S_{i+1}\to S_i$ by the restriction of
$\tau_i^{i+1}$ to $S_i$. Then $\sigma_i$ is a well-defined surjecive map for all $i\ge 1$. 
By Claim 2, for each $i\ge 1$ there exists a positive integer $N_i$ such that
$S_i=\tau_i^{N_i}(Z_{N_i})$.

By assumption, $\Xi((M^\bullet,\phi))=\Xi((\widetilde{M}^\bullet,\widetilde{\phi}))$, which means
that for each positive integer $i$, there exists an isomorphism
$$\gamma_i:{(R/m_R^i)\hat{\otimes}_R {M}^\bullet} \to {(R/m_R^i)\hat{\otimes}_R \widetilde{M}^\bullet}$$
in $K^-([[(R/m_R^i)G]])$ such that
$\widetilde{\phi}\circ(k\hat{\otimes}_{R/m_R^{i}}\gamma_{i})=\phi$ in $K^-([[kG]])$. 

For each positive integer $i$, let $\mathrm{Id}_i$ be the identity cochain map
of $(R/m_R^i)\hat{\otimes}_RM^\bullet$.
Using that $S_i=\tau_i^{N_i}(Z_{N_i})$, we define 
\begin{equation}
\label{eq:fi}
\tilde{f}_i=(R/m_R^i)\hat{\otimes}_{R/m_R^{N_i+1}}\,\gamma_{N_i+1}
\end{equation}
and
\begin{equation}
\label{eq:xi}
\zeta_i=\tilde{f}_i^{-1}\circ\left[(R/m_R^i)\hat{\otimes}_{R/m_R^{i+1}} \tilde{f}_{i+1}\right]-\mathrm{Id}_i\,.
\end{equation}
Then $\zeta_i=(R/m_R^i)\hat{\otimes}_{R/m_R^{N_i+1}}\left(
\gamma_{N_i+1}^{-1}\circ [(R/m_R^{N_i+1})\hat{\otimes}_{R/m_R^{N_i+1}} \gamma_{N_i+2}]-
\mathrm{Id}_{N_i+1}\right)\in \tau_i^{N_i}(Z_{N_i})=S_i$, since
$(k\hat{\otimes}_{R/m_R^{i}}\gamma_{i})=\widetilde{\phi}^{-1}\circ\phi$
in $K^-([[kG]])$ for all $i\ge 1$. 

We define $f_1=\tilde{f}_1$ and $f_2=\tilde{f}_2$. Then
$$(R/m_R)\hat{\otimes}_{R/m_R^{2}}f_2=k\hat{\otimes}_{R/m_R^{N_2+1}}\,\gamma_{N_2+1}
=\widetilde{\phi}^{-1}\circ\phi=k\hat{\otimes}_{R/m_R^{N_1+1}}\,\gamma_{N_1+1}=f_1\,.$$
Let $j\ge 3$, and let $2\le i< j$. We have that 
$\sigma_i\circ\sigma_{i+1}\circ\cdots\circ\sigma_{j-1}:S_j\to S_i$ is surjective. Hence we can
find an element $\zeta_i^{(j)}\in S_j$ which is a preimage of $\zeta_i$ such that
for all $i< t\le j$, $\left(\sigma_t\circ\sigma_{i+1}\circ\cdots\sigma_{j-1}\right)(\zeta_i^{(j)})=\zeta_i^{(t)}$. We define
$$f_j=\tilde{f}_j\circ (\mathrm{Id}_j+\zeta_{j-1}^{(j)})^{-1}\circ (\mathrm{Id}_j+\zeta_{j-2}^{(j)})^{-1}\circ\cdots\circ
(\mathrm{Id}_j+\zeta_2^{(j)})^{-1}$$
where we use Claim 1 to see that,  for all $2\le i\le j$, $\zeta_i^{(j)}\in S_j\subseteq Z_j$ is nilpotent, 
and hence $(\mathrm{Id}_j+\zeta_{i}^{(j)})$ is invertible in $K^-([[(R/m_R^j)G]])$.
We obtain
\begin{eqnarray*}
(R/m_R^{j-1})\hat{\otimes}_{R/m_R^{j}}f_j&=&
\left[(R/m_R^{j-1})\hat{\otimes}_{R/m_R^{j}}\tilde{f}_j\right]\circ 
(\mathrm{Id}_{j-1}+\zeta_{j-1}^{(j-1)})^{-1}\circ \\
&&\qquad(\mathrm{Id}_{j-1}+\zeta_{j-2}^{(j-1)})^{-1}\circ\cdots\circ(\mathrm{Id}_{j-1}+\zeta_2^{(j-1)})^{-1}\\
&=&\tilde{f}_{j-1}\circ (\mathrm{Id}_{j-1}+\zeta_{j-2}^{(j-1)})^{-1}\circ \cdots\circ(\mathrm{Id}_{j-1}+\zeta_2^{(j-1)})^{-1}\\
&=&f_{j-1}
\end{eqnarray*}
in $K^-([[(R/m_R^{j-1})G]])$, where the second to last equality follows from $(\ref{eq:xi})$. Moreover, we have 
\begin{eqnarray*}
\widetilde{\phi}\circ(k\hat{\otimes}_{R/m_R^{j}}f_{j})&=&
\widetilde{\phi}\circ(k\hat{\otimes}_{R/m_R^{j}}\tilde{f}_j)\circ 
(\mathrm{Id}_{1}+k\hat{\otimes}_{R/m_R^{j-1}}\zeta_{j-1})^{-1}\circ \cdots\circ(\mathrm{Id}_{1}+k\hat{\otimes}_{R/m_R^{2}}\zeta_2)^{-1}\\
&=&\widetilde{\phi}\circ(k\hat{\otimes}_{R/m_R^{j}}\tilde{f}_j)\;=\;
\widetilde{\phi}\circ(k\hat{\otimes}_{R/m_R^{N_j+1}}\,\gamma_{N_j+1})\;=\;\phi
\end{eqnarray*}
in $K^-([[kG]])$, where the second equation follows since all $\zeta_i\in S_i\subseteq Z_i$. 
This proves Claim 4.

\medskip

Using $f_i$ from Claim 3 instead of the morphism $f_i$ in \cite[Eq. (7.5)]{bcderived}, the remainder of the proof of the injectivity of $\Xi_\mathcal{D}$ follows as in the proof of
\cite[Prop. 7.2]{bcderived}.
\end{proof}

\begin{thebibliography}{88}

\bibitem{ars} M.~Auslander, I.~Reiten and S.~O.~Smal\/{\o}. Representation theory of Artin
algebras. Cambridge studies in advanced mathematics 36, Cambridge University Press, Cambridge, 1995.

\bibitem{derivedeq} F.~M.~Bleher, Deformations and derived equivalences. 
Proc. Amer. Math. Soc. 134 (2006), 2503--2510. 

\bibitem{3sim} F.~M.~Bleher, Universal deformation rings and dihedral defect groups.
	Trans. Amer. Math. Soc. 361 (2009), 3661--3705.

\bibitem{comptes} F.~M.~Bleher and T.~Chinburg, Deformations
                and derived categories. C. R. Acad. Sci. Paris Ser. I Math.
        334 (2002), 97--100.

\bibitem{bcderived} F.~M.~Bleher and T.~Chinburg, Deformations
                and derived categories. Ann. Inst. Fourier (Grenoble) 55 (2005), 2285--2359.
                
\bibitem{obstructions} F.~M.~Bleher and T.~Chinburg, Obstructions for
deformations of complexes. Ann. Inst. Fourier (Grenoble) 63 (2013), 613--654. 
	     
\bibitem{blehertalbott} F.~M.~Bleher and S.~Talbott, Universal deformation rings of modules for algebras of dihedral type of 
	polynomial growth. Algebr. Represent. Theory 17 (2014), 289--303.
	
\bibitem{blehervelez} F.~M.~Bleher and J.~A.~V\'{e}lez-Marulanda,  
	Universal deformation rings of modules over Frobenius algebras.  J. Algebra 367 (2012), 176--202.
	
\bibitem{broue} M.~Brou\'{e}, Isom\'{e}tries parfaites, types de blocs, cat\'{e}gories d\'{e}riv\'{e}es.
Ast\'{e}risque, 181--182 (1990), 61--92.

\bibitem{broue1} M.~Brou\'{e}, Equivalences of blocks of group algebras. In: Finite dimensional algebras and related topics (Ottawa, ON, 1992), NATO Adv. Sci. Inst. Ser. C Math. Phys. Sci., vol. 424, Kluwer Acad. Publ., Dordrecht, 1994, pp. 1--26.
                        
\bibitem{brumer} A.~Brumer, Pseudocompact algebras, profinite groups and class formations.
	J. Algebra 4 (1966), 442--470.

\bibitem{erd} K.~Erdmann, Blocks of tame representation type and related algebras. 
Lecture Notes in Mathematics, vol. 1428, Springer-Verlag, Berlin-Heidelberg-New York, 1990.

\bibitem{ga1} P.~Gabriel, Des cat\'{e}gories ab\'{e}liennes. Bull. Soc. Math. France 90 
(1962), 323--448.
	
\bibitem{ga2} P.~Gabriel, \'{E}tude infinitesimale des sch\'{e}mas en groupes.
	In: A.~Grothendieck, SGA 3 (with M.~Demazure), Sch\'{e}mas en groupes I, II, III,
	Lecture Notes in Math. 151, Springer Verlag,
         Heidelberg, 1970, pp. 476-- 562.

\bibitem{heller} A.~Heller, The loop-space functor in homological algebra. Trans. Amer. Math. Soc. 96 (1960), 382--394.

\bibitem{holm} T.~Holm, Derived equivalence classification of algebras of
dihedral, semidihedral, and quaternion type. J. Algebra 211 (1999), 159--205.

\bibitem{kozim} S.~K\"onig and A.~Zimmermann, Derived equivalences for group rings. With contributions by Bernhard Keller, Markus Linckelmann, Jeremy Rickard and Rapha\"{e}l Rouquier. Lecture Notes in Mathematics, vol. 1685. Springer-Verlag, Berlin, 1998.

\bibitem{krause2008} H.~Krause, Localization theory for triangulated categories. In: Triangulated categories, London Math. Soc. Lecture Note Ser., 375, Cambridge Univ. Press, Cambridge, 2010, pp. 161--235.

\bibitem{Maz} B.~Mazur, Deformation theory of Galois representations. In:
        Modular Forms
        and Fermat's Last Theorem (Boston, MA, 1995), Springer Verlag,
        Berlin-Heidelberg-New York, 1997, pp. 243--311.
        
\bibitem{milne} J.~S.~Milne, \'{E}tale cohomology. Princeton Univ. Press, Princeton, 1980.

\bibitem{rickardJPAA1989} J.~Rickard, Derived categories and stable equivalence. J. 
	Pure Appl. Algebra 61 (1989), 303--317.

\bibitem{rickard1} J.~Rickard, Derived equivalences as derived  functors.
        J. London Math. Soc. 43 (1991), 37--48.
        
\bibitem{rickard2} J.~Rickard, Splendid equivalences: derived categories and permutation
        modules. Proc. London Math. Soc. 72 (1996), 331--358.
        
\bibitem{rickardICM} J.~Rickard, The abelian defect group conjecture. Proceedings of the International Congress
of Mathematicians, Vol. II (Berlin, 1998), Doc. Math., Extra Vol. II (1998), 121--128.
        	
\bibitem{Sch} M.~Schlessinger, Functors of Artin Rings. Trans. Amer. Math. Soc.
        130, 1968, 208--222.
        
 \bibitem{velez2015}  J.~A.~V\'{e}lez-Marulanda,  Universal deformation rings of string modules over a certain 
 	symmetric special biserial algebra. Beitr. Algebra Geom. 56 (2015), 129--146.

\bibitem{Weibel} C.~A.~Weibel, An introduction to homological algebra. Cambridge studies
in advanced mathematics 38, Cambridge University Press, Cambridge, 1994.

\end{thebibliography}

\end{document}


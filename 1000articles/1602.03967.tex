\documentclass{article}
\usepackage[T2A]{fontenc}
\usepackage[cp1251]{inputenc}
\usepackage[english,russian]{babel}
\usepackage[tbtags]{amsmath}
\usepackage{amsfonts,amssymb,mathrsfs,amscd}
\usepackage[hyper]{msb-a}
\JournalName{ÌÀÒÅÌÀÒÈ×ÅÑÊÈÉ ÑÁÎÐÍÈÊ}
\numberwithin{equation}{section}
\theoremstyle{plain}
\newtheorem{theorem}{Òåîðåìà}
\newtheorem{lemma}{Ëåììà}[section]
\newtheorem{propos}{Ïðåäëîæåíèå}
\newtheorem{corollary}{Ñëåäñòâèå}
\theoremstyle{definition}
\newtheorem{definition}{Îïðåäåëåíèå}
\newtheorem{proof}{Äîêàçàòåëüñòâî}
\newtheorem{remark}{Çàìå÷àíèå}

\begin{document}

\title{Ýêñïîíåíöèàëüíûé ðîñò êîðàçìåðíîñòåé òîæäåñòâ àëãåáð ñ åäèíèöåé}
\author[M.\,V.~Zaicev]{Ì.\,Â.~Çàéöåâ}
\address{ÌÃÓ èìåíè~Ì.\,Â.~Ëîìîíîñîâà}
\email{zaicevmv@mail.ru}
\author[D.~Repov\v s]{Ä.~Ðåïîâø}
\address{Óíèâåðñèòåò Ëþáëÿíû, Ñëîâåíèÿ}
\email{dusan.repovs@guest.arnes.si}

\date{21.08.2014}
\udk{512.572}

\maketitle

\begin{fulltext}

\begin{abstract}
 ðàáîòå èçó÷àåòñÿ àñèìïòîòè÷åñêîå ïîâåäåíèå ýêñïîíåíöèàëüíî îãðàíè÷åííûõ ïîñëåäîâàòåëüíîñòåé
êîðàçìåðíîñòåé òîæäåñòâ àëãåáð ñ åäèíèöåé. Ïîñòðîåíà ñåðèÿ àëãåáð, ó êîòîðûõ îñíîâàíèå
ýêñïîíåíòû óâåëè÷èâàåòñÿ ðîâíî íà $1$ ïðè ïðèñîåäèíåíèè ê èñõîäíîé àëãåáðå âíåøíåé
åäèíèöû. Ïîêàçàíî, ÷òî PI-ýêñïîíåíòû óíèòàðíûõ àëãåáð ìîãóò ïðèíèìàòü ëþáîå çíà÷åíèå
áîëüøå äâóõ, à ýêñïîíåíòû êîíå÷íîìåðíûõ óíèòàðíûõ àëãåáð îáðàçóþò âñþäó ïëîòíîå
ïîäìíîæåñòâî â îáëàñòè $[2,\infty)$.

Áèáëèîãðàôèÿ: 33 íàçâàíèÿ.
\end{abstract}

\begin{keywords} 
òîæäåñòâà, êîðàçìåðíîñòè, ýêñïîíåíöèàëüíûé ðîñò. 
\end{keywords}

\markright{Ðîñò êîðàçìåðíîñòåé àëãåáð ñ åäèíèöåé}

\footnotetext[0]{Ðàáîòà âûïîëíåíà ïðè ïîääåðæêå ÐÔÔÈ (ãðàíò \No~13-01-00234a) è
ÑÏÀ (ãðàíòû P1-0292-0101, J1-5435-0101 è J1-6721-0101).}

\section{Ââåäåíèå}\label{s1}

\subsection{}\label{s1.1}
 ñòàòüå èçó÷àþòñÿ ôóíêöèè, õàðàêòåðèçóþùèå êîëè÷åñòâî òîæäåñòâåííûõ ñîîòíîøåíèé,
âûïîëíÿþùèõñÿ â òîé èëè èíîé àëãåáðå. Êàæäîé àëãåáðå $A$ íàä ïîëåì $F$ íóëåâîé
õàðàêòåðèñòèêè ìîæíî ñîïîñòàâèòü öåëî÷èñëåííóþ ïîñëåäîâàòåëüíîñòü $\{c_n(A)\}$,
$n=1,2,\ldots$, ïîñòðîåííóþ ïî åå ïîëèëèíåéíûì òîæäåñòâàì.  àñèìïòîòè÷åñêîì ïîâåäåíèè
ýòîé ïîñëåäîâàòåëüíîñòè çàëîæåíà îïðåäåëåííàÿ èíôîðìàöèÿ î ñòðîåíèè ñàìîé àëãåáðû
$A$. Íàïðèìåð, åñëè $A$ --- àññîöèàòèâíàÿ àëãåáðà, òî $c_n(A)=1$ äëÿ âñåõ $n$
òîãäà è òîëüêî òîãäà, êîãäà $A$ --- êîììóòàòèâíàÿ íåíèëüïîòåíòíàÿ àëãåáðà. Åñëè æå
$c_n(A)=0$ äëÿ íåêîòîðîãî $n>1$, òî $A$ íèëüïîòåíòíà, $A^n=0$ (è íàîáîðîò). Íåäàâíî 
áûëî ïîêàçàíî, ÷òî $\{c_n(A)\}$ àñèìïòîòè÷åñêè âîçðàñòàåò, ò.å. ñóùåñòâóåò òàêîå
íàòóðàëüíîå $t$, ÷òî $c_{t+j}\le c_{t+j+1}$ äëÿ âñåõ $j=0,1,\ldots$. Åñëè 
$c_{m-1}> c_m$, òî ýòî çíà÷åíèå $t$ òåñíî ñâÿçàíî ñî ñòóïåíüþ íèëüïîòåíòíîñòè 
ðàäèêàëà Äæåêîáñîíà àëãåáðû $A$ (ðåçóëüòàò àíîíñèðîâàí â \cite{GZ1}, ïîëíîå
äîêàçàòåëüñòâî îïóáëèêîâàíî â \cite{GZ2}). Åñëè ïîëå $F$ àëãåáðàè÷åñêè çàìêíóòî,
à $A$ ïðîñòà, òî $c_n(A) \sim d^n$, ãäå $d=\dim A$ (\cite{R}). Çäåñü ñîîòíîøåíèå
$c_n(A) \sim d^n$ îçíà÷àåò, ÷òî
$$
\lim_{n\to\infty} \sqrt[n]{c_n(A)}=d.
$$

Òàêîé æå ýôôåêò íàáëþäàåòñÿ è â ñëó÷àå àëãåáð Ëè \cite{Z}, éîðäàíîâûõ àëãåáð,
àëüòåðíàòèâíûõ àëãåáð è ðÿäà äðóãèõ êëàññîâ \cite{GSZ}. Äëÿ àëãåáð Ëè õîðîøî èçâåñòíà 
îòêðûòàÿ ïðîáëåìà êëàññèôèêàöèè áåñêîíå÷íîìåðíûõ ïðîñòûõ àëãåáð Ëè.  íàñòîÿùåå
âðåìÿ ýòà ïðîáëåìà, âèäèìî, äàëåêà îò ñâîåãî ðåøåíèÿ, îäíàêî îïðåäåëåííóþ
èíôîðìàöèþ î ñòðîåíèè òàêîé àëãåáðû $L$ ìîæíî ïîëó÷èòü, åñëè $\{c_n(L)\}$ èìååò
ýêñïîíåíöèàëüíûé ðîñò \cite{Rzm}.

\subsection{}\label{s1.2}
Íàëè÷èå èëè îòñóòñòâèå åäèíèöû â àëãåáðå ñóùåñòâåííî ñêàçûâàåòñÿ íà ñòðóêòóðå åå
òîæäåñòâ. Íàïðèìåð, åñëè $A$ --- àññîöèàòèâíàÿ àëãåáðà ñ åäèíèöåé, òî ñîâîêóïíîñòü
âñåõ åå òîæäåñòâ ïîëíîñòüþ îïðåäåëÿåòñÿ ñèñòåìîé òàê íàçûâàåìûõ ñîáñòâåííûõ
òîæäåñòâ \cite{Sp}. Åñëè, êðîìå òîãî, $A$ óäîâëåòâîðÿåò âñåì òîæäåñòâàì ìàòðè÷íîé 
àëãåáðû $2\times 2$, òî àñèìïòîòè÷åñêè äëÿ åå Ò-èäåàëà ñóùåñòâóåò ëèøü ñ÷åòíîå ÷èñëî
ÿâíî îïèñûâàåìûõ âàðèàíòîâ \cite{K}. Åñëè $\{c_n(A)\}$ ðàñòåò ïîëèíîìèàëüíî, òî
$c_n(A)= qn^k + O(n^{k-1})$. Äëÿ íåêîòîðîãî öåëîãî $k$ è ïîëîæèòåëüíîãî ðàöèîíàëüíîãî
$q$ \cite{D}. Ïîçäíåå áûëî ïîêàçàíî, ÷òî ïðè ôèêñèðîâàííîì $k$ äëÿ ëþáîãî
$q\in\mathbb{Q}, q>0$, ìîæíî ïîäîáðàòü ïîäõîäÿùóþ àëãåáðó \cite{DR}. È â òîé æå ðàáîòå
áûëî äîêàçàíî, ÷òî åñëè $A$ --- óíèòàðíàÿ àëãåáðà, òî
$$
\frac{1}{k!}\le q \le \sum_{i=2}^k \frac{(-1)^k}{i!} \simeq \frac{1}{e}.
$$

Åùå îäèí ïîëîæèòåëüíûé ýôôåêò íàëè÷èÿ åäèíèöû ïðîÿâèëñÿ â äîêàçàòåëüñòâå ñëåäóþùåé
ãèïîòåçû.  À.~Ðåãåâ â êà÷åñòâå óòî÷íåíèÿ ãèïîòåçû Àìèöóðà ïðåäïîëîæèë, ÷òî
$$
c_n(A)\simeq C n^\frac{t}{2}d^n
$$
äëÿ ëþáîé àññîöèàòèâíîé PI-àëãåáðû, ãäå $t$ è $n$ --- öåëûå, $C=const$. Ïîñëå ñåðèè
÷àñòíûõ ðåçóëüòàòîâ â 2008 ã. ãèïîòåçà Ðåãåâà áûëà ïîäòâåðæäåíà äëÿ àëãåáð ñ 1
\cite{BR}, \cite{Be}. È òîëüêî íåäàâíî áûëà äîêàçàíà ñïðàâåäëèâîñòü
ýòîé ãèïîòåçû â îáùåì ñëó÷àå \cite{GZ1}, \cite{GZ2}.

 ðàáîòå \cite{GMZ} äëÿ âñåõ âåùåñòâåííûõ $\gamma >1$ áûëè ïîñòðîåíû ïðèìåðû êîíå÷íîìåðíûõ
àëãåáð ñ ýêñïîíåíöèàëüíûì ðîñòîì êîðàçìåðíîñòåé $c_n\sim\gamma'\approx \gamma$. Êàê ïîêàçàíî
â \cite{Z1},  äëÿ êîíå÷íîìåðíûõ àëãåáð ñ 1 ýêñïîíåöèàëüíûé ðîñò íå ìîæåò áûòü ìåäëåííåå
÷åì $2^n$.

 ðàáîòå \cite{GZ3} áûëî îòìå÷åíî, ÷òî åñëè $A$ --- àññîöèàòèâíàÿ PI-àëãåáðà, à
$A^\#$ --- àëãåáðà, ïîëó÷åííàÿ èç $A$ ïóòåì ïðèñîåäèíåíèÿ âíåøíåé åäèíèöû, òî
$exp(A^\#)=exp(A)$ èëè $exp(A)+1$. Ýòî íåñëîæíîå óòâåðæäåíèå âûòåêàåò èç
ðåçóëüòàòîâ \cite{GZ4}, \cite{GZ5}, ãäå íå òîëüêî áûëî äîêàçàíî ñóùåñòâîâàíèå 
ïðåäåëà
$$
exp(A)=\lim_{n\to \infty} \sqrt[n]{c_n(A)}
$$
äëÿ ëþáîé àññîöèàòèâíîé PI-àëãåáðû $A$, íî è ïðåäëîæåíà ïðîöåäóðà âû÷èñëåíèÿ ýòîé
âåëè÷èíû. Òåì íå ìåíåå, ýòî íàáëþäåíèå ïîçâîëèëî âûäâèíóòü ãèïîòåçó, ÷òî
$exp(A^\#$) âñåãäà ðàâíÿåòñÿ $exp(A)$ èëè $exp(A)+1$. Ïåðâûé íåòðèâèàëüíûé ïðèìåð,
ïîäòâåðæäàþùèé ýòó ãèïîòåçó, áûë ïîñòðîåí â \cite{Z1}, åùå îäèí ïðèìåð ïðåäëîæåí
â \cite{BBZ}, à â \cite{RZ} ïðèâåäåíà óæå ñåðèÿ ïðèìåðîâ, â êîòîðûõ äëÿ ëþáîé àëãåáðû
$A$ èç ðàáîòû \cite{GMZ} ñ $exp(A)=\gamma\in \mathbb R, 1\le\gamma\le 2$, åå
ðàñøèðåíèå $A^\#$ èìååò ýêñïîíåíòó $exp(A^\#)=\gamma +1\in [2,3]$. Çàìåòèì òàêæå,
÷òî â ðàáîòå \cite{Ra1} àâòîðîì áûëà ïðåäëîæåíà êîíñòðóêöèÿ ïîñòðîåíèÿ ïî àëãåáðå Ëè $L$
íàä ïîëåì $F$ àëãåáðû Ïóàññîíà, ðàâíîé $L\oplus F$ êàê âåêòîðíîå ïðîñòðàíñòâî è
ñîäåðæèò $L$ â êà÷åñòâå ïîäàëãåáðû Ëè êîðàçìåðíîñòè 1. Àëãåáðó $L\oplus F$
 ìîæíî ñ÷èòàòü åñòåñòâåííîé ìîäèôèêàöèåé àëãåáðû $L^\#$. Íåñêîëüêî ïîçæå òîò æå àâòîð
 äîêàçàë, ÷òî $exp(L\oplus F)=exp(L)+1$ \cite{Ra2}.
 
 
\subsection{}\label{s1.3}
Îñíîâíîé öåëüþ äàííîé ðàáîòû ÿâëÿåòñÿ ïîñòðîåíèå ñåìåéñòâà àëãåáð $A_\gamma$, 
$\gamma\in\mathbb R$, $\gamma >1$, äëÿ êîòîðûõ $exp(A_\gamma)=\gamma$ (òåîðåìà \ref{t1}),
$exp(A_\gamma^\#)=\gamma +1$ (òåîðåìà \ref{t2}). Îòìåòèì, ÷òî ïðè ïîñòðîåíèè 
ýòèõ ïðèìåðîâ èñïîëüçîâàëèñü áåñêîíå÷íûå ïåðèîäè÷åñêèå ñëîâà è ñëîâà Øòóðìà, 
êîìáèíàòîðíûå ñâîéñòâà êîòîðûõ èñïîëüçîâàëèñü ïðè ïîëó÷åíèè àñèìïòîòè÷åñêèõ
îöåíîê.

Êðîìå åùå îäíîãî ïîäòâåðæäåíèÿ óïîëÿíóòîé ãèïîòåçû, ýòè ðåçóëüòàòû ïîêàçûâàþò, ÷òî
ëþáîå âåùåñòâåííîå ÷èñëî $\gamma \ge 2$ ìîæåò áûòü ðåàëèçîâàíî êàê PI-ýêñïîíåíòà
óíèòàðíîé àëãåáðû (ñëåäñòâèå \ref{c1}). Êðîìå òîãî, èç òåîðåìû \ref{t2} è ðÿäà 
êîìáèíàòîðíûõ ñâîéñòâ áåñêîíå÷íûõ ñëîâ ñëåäóåò, ÷òî PI-ýêñïîíåíòû êîíå÷íîìåðíûõ
óíèòàðíûõ àëãåáð îáðàçóþò âñþäó ïëîòíîå ïîäìíîæåñòâî â îáëàñòè $[2,\infty)$.

Ñ îñíîâàìè òåîðèè òîæäåñòâåííûõ ñîîòíîøåíèé è êîëè÷åñòâåííîé PI-òåîðèè ìîæíî
ïîçíàêîìèòüñÿ â ìîíîãðàôèÿõ \cite{B}, \cite{Dren}, \cite{GZbook}.

 

\section{Îñíîâíûå ïîíÿòèÿ è êîíñòðóêöèè}\label{s2}

\subsection{}\label{s2.1}
Ïóñòü $A$ --- àëãåáðà íàä ïîëåì $F$, à $F\{X\}$ --- àáñîëþòíî ñâîáîäíàÿ $F$-àãåáðà
ñ áåñêîíå÷íûì ìíîæåñòâîì ïîðîæäàþùèõ $X$. Ïîëèíîì $f=f(x_1,\ldots, x_n)\in F\{X\}$,
$x_1,\ldots, x_n \in X$,  íàçûâàåòñÿ òîæäåñòâîì $A$, åñëè $f(a_1,\ldots, a_n)=0$
äëÿ ëþáûõ $a_1,\ldots,a_n \in A$. Ìíîæåñòâî âñåõ òîæäåñòâ $Id(A)$ àëãåáðû $A$ îáðàçóåò 
èäåàë â $F\{X\}$. Îáîçíà÷èì ÷åðåç $P_n$ ïîäïðîñòðàíñòâî âñåõ ïîëèëèíåéíûõ ìíîãî÷ëåíîâ îò
$x_1,\ldots, x_n$ â $F\{X\}$. Òîãäà $P_n\cap Id(A)$ --- ìíîæåñòâî âñåõ ïîëèëèíåéíûõ 
òîæäåñòâ ñòåïåíè $n$ àëãåáðû $A$. Õîðîøî èçâåñòíî, ÷òî â ñëó÷àå íóëåâîé õàðàêòåðèñòèêè 
îñíîâíîãî ïîëÿ èäåàë $Id(A)$ ïîëíîñòüþ îïðåäåëÿåòñÿ íàáîðîì ïîäïðîñòðàíñòâ
$\{P_n\cap Id(A)\}, n=1,2,\ldots $. Îáîçíà÷èì ÷åðåç $P_n(A)$ ôàêòîðïðîñòðàíñòâî
$$
P_n(A)=\frac{P_n}{P_n\cap Id(A)},
$$
à ÷åðåç $c_n(A)$ --- åãî ðàçìåðíîñòü:
$$
c_n(A)=\dim P_n(A).
$$
Âåëè÷èíà $c_n(A)$ íàçûâàåòñÿ $n$-é êîðàçìåðíîñòüþ òîæäåñòâ àëãåáðû $A$ (èëè ïðîñòî
$n$-é êîðàçìåðíîñòüþ $A$) è ÿâëÿåòñÿ îäíîé èç êîëè÷åñòâåííûõ õàðàêòåðèñòèê
ñîâîêóïíîñòè òîæäåñòâåííûõ ñîîòíîøåíèé àëãåáðû $A$. Èññëåäîâàíèå àñèìïòîòè÷åñêîãî ïîâåäåíèÿ 
ïîñëåäîâàòåëüíîñòè $\{c_n(A)\}$ --- îäíà èç êëþ÷åâûõ çàäà÷ êîëè÷åñòâåííîé PI-òåîðèè.

 îáùåì ñëó÷àå $\{c_n(A)\}$ ìîæåò èìåòü ñâåðõýêñïîíåíöèàëüíûé ðîñòü. Íàïðèìåð,
åñëè $A=F\{X\}$, òî
$$
c_n(A)=\frac{1}{n}C_{2n-2}^{n-1} n!,
$$
åñëè $A$ --- ñâîáîäíàÿ àññîöèàòèâíàÿ àëãåáðà, òî $c_n(A)=n!$, à åñëè $A$ ---
ñâîáîäíàÿ àëãåáðà Ëè, òî $c_n(A)=(n-1)!$. Îäíàêî âî ìíîãèõ ñëó÷àÿõ ðîñò ïîñëåäîâàòåëüíîñòè 
$\{c_n(A)\}$ îãðàíè÷åí ýêñïîíåíöèàëüíîé ôóíêöèåé. Êëàññ àëãåáð ñ ýêñïîíåíöèàëüíî îãðàíè÷åííûì
ðîñòîì êîðàçìåðíîñòåé âêëþ÷àåò ñåáÿ âñå àññîöèàòèâíûå PI-àëãåáðû \cite{R1}, âñå 
êîíå÷íîìåðíûå àëãåáðû \cite{BD} ëþáîé ñèãíàòóðû, àëãåáðû Êàöà-Ìóäè \cite{Z2}, 
áåñêîíå÷íîìåðíûå ïðîñòûå àëãåáðû Ëè êàðòàíîâñêîãî òèïà \cite{M} è öåëûé ðÿä äðóãèõ.
 ýòîì ñëó÷àå îïðåäåëåíû âåðõíèé è íèæíèé ïðåäåëû
$$
\overline{exp}(A) =\overline{\lim_{n\to\infty}}\sqrt[n]{c_n(A)}\, ,\quad
\underline{exp}(A)= \underline{\lim}_{n\to\infty}\sqrt[n]{c_n(A)},
$$
êîòîðûå íàçûâàþòñÿ âåðõíåé è íèæíåé PI-ýêñïîíåíòàìè $A$. Åñëè ñóùåñòâóåò îáû÷íûé
ïðåäåë, ò.å. $\overline{exp}(A)=\underline{exp}(A)$, òî åãî íàçûâàþò (îáû÷íîé)
PI-ýêñïîíåíòîé.

\subsection{}\label{s2.2}
Ïðè èçó÷åíèè àñèìïòîòèêè ðîñòà $\{c_n(A)\}$ ïîëåçíûì èíñòðóìåíòîì ñëóæèò òåîðèÿ
ïðåäñòàâëåíèé ñèììåòðè÷åñêèõ ãðóïï. Ãðóïïà $S_n$ åñòåñòâåííûì îáðàçîì äåéñòâóåò íà $P_n$:
$$
\sigma f(x_1,\ldots, x_n) = f(x_{\sigma(1)},\ldots, x_{\sigma(n)}).
$$
Ïðè ýòîì ïîäïðîñòðàíñòâî $P_n\cap Id(A)$ èíâàðèàíòíî îòíîñèòåëüíî ýòîãî äåéñòâèÿ,
è ïîýòîìó $P_n(A)$ òàêæå íàäåëÿåòñÿ ñòðóêòóðîé $F[S_n]$-ìîäóëÿ. Âñå íåîáõîäèìûå ñâåäåíèÿ
ïî òåîðèè ïðåäñòàâëåíèé ñèììåòðè÷åñêèõ ãðóïï è åå ïðèìåíåíèþ ïðè èññëåäîâàíèè òîæäåñòâåííûõ
ñîîòíîøåíèé ìîæíî íàéòè â \cite{J}, \cite{B}, \cite{Dren}, \cite{GZbook}. Â ñèëó
ïîëíîé ïðèâîäèìîñòè ïðåäñòàâëåíèé ãðóïïû $S_n$  ìîäóëü $P_n(A)$ ðàñêëàäûâàåòñÿ â
ïðÿìóþ ñóììó íåïðèâîäèìûõ $F[S_n]$-ìîäóëåé, ÷òî óäîáíî çàïèñûâàòü íà ÿçûêå òåîðèè
õàðàêòåðîâ. Õàðàêòåð $\chi(P_n(A))$ íàçûâàåòñÿ $n$-ì êîõàðàêòåðîì $A$ è îáîçíà÷àåòñÿ êàê
$\chi_n(A)$. Ðàçëîæåíèå $P_n(A)$ íà íåïðèâîäèìûå êîìïîíåíòû çàïèñûâàåòñÿ êàê
ðàçëîæåíèå $\chi_n(A)$ â ñóììó íåïðèâîäèìûõ õàðàêòåðîâ:
\begin{equation}
\label{e1}
\chi_n(A)=\sum_{\lambda\vdash n}m_\lambda \chi_\lambda,
\end{equation}
ãäå $\chi_\lambda$ --- õàðàêòåð íåïðèâîäèìîãî ïðåäñòàâëåíèÿ $S_n$, ñîîòâåòñòâóþùåãî
ðàçáèåíèþ $\lambda$ ÷èñëà $n$, à íåîòðèöàòåëüíîå öåëîå ÷èñëî $m_\lambda$ --- åãî
êðàòíîñòü. Ñîîòíîøåíèå (\ref{e1}) â ÷àñòíîñòè îçíà÷àåò, ÷òî
\begin{equation}
\label{e2}
c_n(A)=\sum_{\lambda\vdash n}m_\lambda d_\lambda,
\end{equation}
ãäå $d_\lambda=\deg \chi_\lambda$ --- ðàçìåðíîñòü íåïðèâîäèìîãî ïðåäñòàâëåíèÿ ãðóïïû
$S_n$, ñîîòâåòñòâóþùåãî ðàçáèåíèþ $\lambda$. Äëÿ ïîëó÷åíèÿ îöåíîê ðîñòà êîðàçìåðíîñòåé íàì ïîòðåáóåòñÿ åùå îäíà âåëè÷èíà, íàçûâàåìàÿ $n$-é êîäëèíîé àëãåáðû $A$, îïðåäåëÿåìàÿ êàê
$$
l_n(A)=\sum_{\lambda\vdash n} m_\lambda,
$$
ãäå $m_\lambda$ --- êîýôôèöèåíòû èç ïðàâîé ÷àñòè (\ref{e2}). Î÷åâèäíî, ÷òî
\begin{equation}
\label{e2a}
c_n(A) \le l_n(A)\max\{d_\lambda\vert \lambda\vdash n, m_\lambda\ne 0\}.
\end{equation}

Íàì ïîòðåáóåòñÿ áîëåå äåòàëüíàÿ èíôîðìàöèÿ î ñòðîåíèè íåïðèâîäèìûõ $F[S_n]$-ìîäóëåé.
Íàïîìíèì, ÷òî ðàçáèåíèåì $\lambda$ ÷èñëà $n$ íàçûâàåòñÿ óïîðÿäî÷åííûé íàáîð öåëûõ ÷èñåë
$\lambda=(\lambda_1,\ldots,\lambda_k)$, òàêîé, ÷òî $\lambda_1\ge\ldots\ge\lambda_k>0$,
$\lambda_1+\ldots+\lambda_k=n$. ×èñëî $h(\lambda)=k$ íàçûâàåòñÿ âûñîòîé $\lambda$. Ïî
ðàçáèåíèþ $\lambda$ ñòðîèòñÿ òàáëèöà èç $n$ êëåòîê, íàçûâàåìàÿ äèàãðàììîé Þíãà $D_\lambda$.
Îíà ñîñòîèò èç $k$ ñòðîê è ñîäåðæèò $\lambda_j$ êëåòîê â $j$-é ñòðîêå äëÿ êàæäîãî
$j=1,\ldots, k$. Åñëè â êëåòêè äèàãðàììû $D_\lambda$ çàïèñàíû ÷èñëà $1,\ldots,n$, òî
ïîëó÷åííàÿ êîíñòðóêöèÿ íàçûâàåòñÿ òàáëèöåé Þíãà $T_\lambda$. Èçâåñòíî, ÷òî ëþáîé 
íåïðèâîäèìûé $F[S_n]$-ìîäóëü èçîìîðôåí ìèíèìàëüíîìó ëåâîìó èäåàëó $F[S_n]e_{T_\lambda}$
ãðóïïîâîãî êîëüöà ãðóïïû $S_n$, ãäå ýëåìåíò $e_{T_\lambda}$ ñòðîèòñÿ ñëåäóþùèì îáðàçîì.

Îáîçíà÷èì ÷åðåç $R_{T_\lambda}$ ïîäãðóïïó âñåõ ïîäñòàíîâîê, ïåðåñòàâëÿþùèõ ÷èñëà
$1,\ldots, n$ òîëüêî â ïðåäåëàõ ñòðîê òàáëèöû $T_\lambda$. ßñíî, ÷òî
$R_{T_\lambda}\simeq S_{\lambda_1}\times\cdots\times S_{\lambda_k}$. Àíàëîãè÷íî
îïðåäåëÿåòñÿ ïîäãðóïïà $C_{T_\lambda}$, ýëåìåíòû êîòîðîé íå âûâîäÿò êàæäîå ÷èñëî çà 
ïðåäåëû ñòîëáöà $T_\lambda$. Ïîëîæèì
$$
R(T_\lambda)=\sum_{\sigma\in R_{T_\lambda}} \sigma\, ,\quad
C(T_\lambda)=\sum_{\tau\in C_{T_\lambda}}({\rm sgn}\,\tau)\tau
$$
è
$$
e_{T_\lambda}= R(T_\lambda) C(T_\lambda).
$$
Õàðàêòåð ýòîãî ìîäóëÿ è íàçûâàåòñÿ íåïðèâîäèìûì õàðàêòåðîì $\chi_\lambda$. Ýëåìåíò 
$e_{T_\lambda}$ íàçûâàåòñÿ ñèììåòðèçàòîðîì Þíãà è ÿâëÿåòñÿ êâàçèèäåìïîòåíòîì êîëüöà
$F[S_n]$, ò.å. $e_{T_\lambda}^2=\gamma e_{T_\lambda}$, ãäå $\gamma$ --- íåíóëåâîé 
ñêàëÿð. Îòñþäà â ÷àñòíîñòè ñëåäóåò, ÷òî ýëåìåíò $C(T_\lambda)e_{T_\lambda}$ íå ðàâåí 
íóëþ è ïîðîæäàåò òîò æå ñàìûé ìèíèìàëüíûé ëåâûé èäåàë $F[S_n]e_{T_\lambda}$. Â
êîíòåêñòå äåéñòâèÿ $S_n$ íà ïðîñòðàíñòâå ïîëèëèíåéíûõ ìíîãî÷ëåíîâ $P_n$ ýòî
ïîçâîëÿåò ñäåëàòü íåñëîæíûé, íî âàæíûé âûâîä.
\begin{remark}\label{r1}
Ïóñòü $M$ --- íåïðèâîäèìûé $F[S_n]$-ïîäìîäóëü â $P_n$. Òîãäà $M$ ïîðîæäàåòñÿ êàê
$F[S_n]$-ìîäóëü ïîëèëèíåéíûì ìíîãî÷ëåíîì ñî ñëåäóþùèìè ñâîéñòâàìè:
\begin{itemize}
\item
ìíîæåñòâî ïåðåìåííûõ, âõîäÿùèõ â $f$, ðàñïàäàåòñÿ â îáúåäèíåíèå íåïåðåñåêàþùèñÿ
ïîäìíîæåñòâ
$$
\{x_1,\ldots,x_n\}=X_1\cup\ldots\cup X_t,
$$
ãäå $t=\lambda_1$ --- äëèíà ïåðâîé ñòðîêè $D_\lambda$, $|X_j|$ --- âûñîòà $j$-ãî ñòîëáöà
$D_\lambda, j=1,\ldots,k$;
\item
ïîëèíîì $f$ êîñîñèììåòðè÷åí ïî êàæäîìó èç íàáîðîâ $X_1,\ldots, X_t$.
\end{itemize}
\end{remark}

\subsection{}\label{s2.3}
Äëÿ îöåíîê ðàçìåðíîñòåé íåïðèâîäèìûõ ïðåäñòàâëåíèé $S_n$ óäîáíî ïîëüçîâàòüñÿ ôóêöèåé
$\Phi(\lambda)$, çàäàâàåìîé íà ðàçáèåíèÿõ ñëåäóþùèì îáðàçîì.

Ïóñòü ñíà÷àëà $0\le x_1,\ldots, x_d\le 1$ --- ëþáûå âåùåñòâåííûå ÷èñëà, òàêèå, ÷òî
$x_1+\cdots+x_d=1$, à $d\ge 2$. Ïîëîæèì
\begin{equation}
\label{e3}
\Phi(x_1,\ldots, x_d)=\frac{1}{x_1^{x_1}\ldots x_d^{x_d}}.
\end{equation}
Ìû áóäåì ïîëüçîâàòüñÿ íåïðåðûâíîñòüþ ôóíêöèè $\Phi$ è òåì ñâîéñòâîì, ÷òî åñëè
çàôèêñèðîâàòü çíà÷åíèÿ âñåõ ïåðåìåííûõ, êðîìå $x_i, x_j$, òî ìàêñèìóì $\Phi$ äîñòèãàåòñÿ
ïðè $x_i=x_j$. Áîëåå òîãî, åñëè  $x_i >x_j$, òî $\Phi(x_i-\varepsilon, x_j+\varepsilon)$
ðàñòåò ñ ðîñòîì $\varepsilon$ îò $0$ äî $\frac{1}{2}(x_i-x_j)$. Åñëè æå çàôèêñèðîâàòü îäíó èç ïåðåìåííûõ, íàïðèìåð, $x_d=\gamma$, òî ìàêñèìóì äîñòèãàåòñÿ ïðè $x_1=\ldots=x_{d-1}$, ò.å.
$$
\max\,\Phi=\Phi(\theta,\ldots,\theta,\gamma), \quad \hbox{ãäå} 
\quad (d-1)\theta+\gamma=1. 
$$
Ìû áóäåì èñïîëüçîâàòü îáîçíà÷åíèå
\begin{equation}
\label{e4}
\Phi_{d-1}(\gamma)=\Phi(\underbrace{\theta,\ldots,\theta}_{d-1},\gamma),\quad (d-1)\theta+\gamma=1.
\end{equation}

Ïóñòü òåïåðü $\lambda=(\lambda_1,\ldots,\lambda_t)\vdash n$ è $d\ge t$. Ìû áóäåì çàïèñûâàòü
$\lambda$ â âèäå $\lambda=(\lambda_1,\ldots,\lambda_d)$ äàæå åñëè $t<d$, ïîëàãàÿ
$\lambda_{t+1}=\ldots=\lambda_d=0$. Òîãäà
$$
\Phi(\lambda)=\Phi(\frac{\lambda_1}{n},\ldots,\frac{\lambda_d}{n}).
$$
 Î÷åâèäíî, ÷òî çíà÷åíèå $\Phi(\lambda)$ íå çàâèñèò 
îò $d\ge t$, åñëè èñïîëüçîâàòü ñîãëàøåíèå $0^0=1$.

Çíà÷åíèå $\Phi(\lambda)$ è ñòåïåíü õàðàêòåðà $d_\lambda=\deg\chi_\lambda$ ñâÿçàíû ñëåäóþùèì
ñîîòíîøåíèåì.
\begin{lemma}\label{L1}\cite[ëåììà 1]{GZ6}
Ïóñòü $\lambda=(\lambda_1,\ldots,\lambda_t)\vdash n$ --- ðàçáèåíèå $n$ íà $t\le d$
êîìïîíåíò è $n\ge 100$. Òîãäà
$$
\frac{\Phi(\lambda)^n}{n^{d^2+d}}\le d_\lambda\le n\Phi(\lambda)^n.
$$
\end{lemma}

Íàì ïîòðåáóåòñÿ ñëåäóþùåå ñâîéñòâî ôóíêöèè $\Phi$. Ïóñòü $\lambda=(\lambda_1,\ldots,\lambda_q)$, $\mu=(\mu_1,\ldots,\mu_q)$ --- äâà ðàçáèåíèÿ ÷èñëà $n$, 
$\lambda_q, \mu_q >0$. Ìû áóäåì ãîâîðèòü, ÷òî äèàãðàììà Þíãà $D_\mu$ ïîëó÷åíà èç äèàãðàììû
 $D_\lambda$ âûòàëêèâàíèåì âíèç îäíîé êëåòêè, åñëè ñóùåñòâóþò òàêèå $1\le i<j \le q$,
 ÷òî $\mu_i= \lambda_i-1, \mu_j=\lambda_j+1$ è $\mu_k=\lambda_k$ äëÿ âñåõ îñòàëüíûõ
 $1\le k \le q$. Åñëè æå $\lambda=(\lambda_1,\ldots,\lambda_q)$, $\lambda_q>0$,
$\mu=(\mu_1,\ldots,\mu_q,1)\vdash n$, òî $D_\mu$ ïîëó÷åíà âûòàëêèâàíèåì âíèç îäíîé
êëåòêè èç $D_\lambda$, åñëè îäíà èç ñòðîê $D_\mu$ íà îäíó êëåòêó êîðî÷å, ÷åì ó
$D_\lambda$,  à âñå îñòàëüíûå, êðîìå ïîñëåäíåé, èìåþò òó æå äëèíó.

\begin{lemma}\label{L2}\cite[ëåììà 3]{GZ6}, \cite[ëåììà 2]{ZR}
Ïóñòü $D_\mu$ ïîëó÷åíà èç $D_\lambda$ âûòàëêèâàíèåì âíèç îäíîé êëåòêè. Òîãäà
$\Phi(\mu) \ge \Phi(\lambda)$.
\end{lemma}

Ìû òàêæå áóäåì èñïîëüçîâàòü è òàêîå ñâîéñòâî ôóíêöèè $\Phi(x_1,\ldots,x_d)$.

\begin{lemma}\label{L3}\cite[ëåììà 2]{RZ}
Ïóñòü $\Phi(x_1,\ldots,x_d)$ çàäàíà ôîðìóëîé (\ref{e3}) è ïóñòü
$\Phi(z_1,\ldots,z_d)=a$ äëÿ íåêîòîðûõ ôèêñèðîâàííûõ çíà÷åíèé $z_1,\ldots,z_d$. Òîãäà
$$
\max_{0\le t \le 1}\{\Phi(y_1,\ldots,y_d,1-t)|y_1=tz_1,\ldots,y_d=tz_d \}=a+1,
$$
ïðè÷åì ìàêñèìóì äîñòèãàåòñÿ ïðè $t=\frac{a}{a+1}$.
\end{lemma}

Ëåììà \ref{L3} ôàêòè÷åñêè îçíà÷àåò, ÷òî ïðè äîáàâëåíèè ê äèàãðàììå $D_\lambda$ îäíîé
äîïîëíèòåëüíîé ñòðîêè çíà÷åíèå ôóíêöèè $\Phi(\lambda)$ óâåëè÷èâàåòñÿ íå áîëåå
÷åì íà åäèíèöó.

 
 \subsection{}\label{s2.4}
 Äëÿ ïîñòðîåíèÿ ïðèìåðîâ àëãåáð ñ çàäàííûì õàðàêòåðîì ïîâåäåíèÿ $\{c_n(A)\}$ ìû
 âîñïîëüçóåìñÿ ïîäõîäîì, âïåðâûå ïðåäëîæåííîì â ðàáîòå \cite{GMZ} è áàçèðóþùåìñÿ íà êîìáèíàòîðíûõ ñâîéñòâàõ áåñêîíå÷íûõ äâîè÷íûõ ñëîâ. Äëÿ ýòîãî íàïîìíèì íåêîòîðûå
 ïîíÿòèÿ.
 
 Ïóñòü $w=w_1w_2\ldots$ --- áåñêîíå÷íîå ñëîâî â äâîè÷íîì àëôàâèòå, ò.å. âñå $w_i$ ðàâíû
 $0$ èëè $1$. Ñëîæíîñòüþ ñëîâà $w$ íàçûâàåòñÿ ôóíêöèÿ íàòóðàëüíîãî àðãóìåíòà $Comp_w(n)$,
 ðàâíàÿ êîëè÷åñòâó ðàçëè÷íûõ ïîäñëîâ â $w$ äëèíû $n$. Åñëè ñëîâî $w$ ïåðèîäè÷åñêîå, òî
 $Comp_w(n)=const=T$ äëÿ âñåõ $n\ge T$, ãäå $T$ --- ïåðèîä $w$. Èçâåñòíî òàêæå, ÷òî åñëè $w$
 íå ÿâëÿåòñÿ ïåðèîäè÷åñêèì, òî $Comp_w(n) \ge n+1$ äëÿ âñåõ $n\ge 1$ \cite{L}. Ñóììó
 $w_{k+1}+\ldots+w_{k+m}$ êîíå÷íîãî ïîäñëîâà $u=w_{k+1}\ldots w_{k+m}$ ïðèíÿòî îáîçíà÷àòü êàê
 $h(u)$, à äëèíó êàê $|u|$.
 
Äëÿ çàäàííîãî ñëîâà $w$ âåëè÷èíà
\begin{equation} \label{e5}
\pi(w)=\lim_{n\to\infty} \frac{h(w_1\ldots w_n)}{n}
\end{equation}
íàçûâàåòñÿ íàêëîíîì ñëîâà $w$, åñëè ïðåäåë â ïðàâîé ÷àñòè (\ref{e5}) ñóùåñòâóåò.

Åñëè $comp_w(n)=n+1$ äëÿ âñåõ $n\ge 1$, ò ñëîâî $w$ íàçûâàåòñÿ ñëîâîì Øòóðìà. Ñëîâà Øòóðìà îáëàäàþò ñëåäóþùèìè 
ñâîéñòâàìè \cite{L}.

\begin{propos}\label{p1}
Ïóñòü $w$ --- ïåðèîäè÷åñêîå ñëîâî èëè ñëîâî Øòóðìà. Òîãäà ñóùåñòâóåò òàêàÿ êîíñòàíòà $C$, ÷òî
\begin{itemize}
\item[(1)]
$|h(x)-h(y)| \le C$ äëÿ ëþáûõ êîíå÷íûõ ïîäñëîâ $x$ è $y$ îäèíàêîâîé äëèíû;
\item[(2)]
íàêëîí $\pi(w)$ âñåãäà ñóùåñòâóåò;
\item[(3)]
äëÿ ëþáîãî êîíå÷íîãî ïîäñëîâà $u$ â $w$
$$
\vert\frac{h(u)}{|u|} - \pi(w) \vert \le \frac{C}{|u|};
$$
\item[(4)]
äëÿ ëþáîãî âåùåñòâåííîãî $\alpha\in (0;1)$ ñóùåñòâóåò $w$ ñ $\pi(w)=\alpha$ è $w$ ÿâëÿåòñÿ
ïåðèîäè÷åñêèì, åñëè $\alpha$ --- ðàöèîíàëüíîå ÷èñëî, ëèáî ñëîâîì Øòóðìà, åñëè $\alpha$ ---
èððàöèîíàëüíîå. Áîëåå òîãî, ìîæíî âçÿòü $C=1$, åñëè $w$ --- ñëîâî Øòóðìà, ëèáî $C=T$, åñëè
$w$ --- ïåðèîäè÷åñêîå ñëîâî ñ ïåðèîäîì $T$, è òîãäà
$$
\pi(w)=\frac{h(w_1\ldots w_T)}{T}.
$$
\end{itemize}
\end{propos}

 äàëüíåéøåì ìû áóäåì òàêæå ñ÷èòàòü ñëîâà èç îäíèõ íóëåé èëè èç îäíèõ åäèíèö ïåðèîäè÷åñêèìè,
è òîãäà ïðåäëîæåíèå \ref{p1} ðàñïðîñòðàíÿåòñÿ è íà ñëó÷àè $\alpha=0,\alpha=1$.

\section{Ñëîâà Øòóðìà è íåàññîöèàòèâíûå àëãåáðû}\label{s3}

 äàííîì ïàðàãðàôå ìû ïîñòðîèì ñåìåéñòâî íåàññîöèàòèâíûõ àëãåáð, PI-ýêñïîíåíòû êîòîðûõ 
ïðèíèìàþò ëþáûå âåùåñòâåííûå çíà÷åíèÿ èç îáëàñòè $[2;\infty)$. Èäåÿ ïîòðîåíèÿ àëãåáð ñ 
çàäàííûì ðîñòîì êîðàçìåðíîñòåé íà áàçå ñëîâ Øòóðìà âïåðâûå áûëà ïðåäëîæåíà è ðåàëèçîâàíà 
â ðàáîòàõ \cite{GMZ0}, \cite{GMZ}, ãäå äëÿ ëþáîãî âåùåñòâåííîãî $1\le \alpha \le 2$ áûëà ïîñòðîåíà
àëãåáðà $A_\alpha$ ñ $exp(A_\alpha)=\alpha$. Â íåäàâíåé ðàáîòå \cite{RZ} áûëî äîêàçàíî,
÷òî åñëè ê $A_\alpha$ ïðèñîåäèíèòü âíåøíþþ åäèíèöó, òî ó ïîëó÷åííîé àëãåáðû 
$A_\alpha^\#$ ýêñïîíåíòà ñóùåñòâóåò è ðàâíà $\alpha+1$. Ïîñòðîåííàÿ íèæå ñåðèÿ
àëãåáð îáîáùàåò êîíñòðóêöèþ, ïðåäëîæåííóþ â \cite{GMZ}. Ñëåäóåò îòìåòèòü, ÷òî ïðèìåðû àëãåáð
ñ ïðîèçâîëüíîé PI-ýêñïîíåíòîé $\alpha\ge 2$ òàêæå áûëè ïðèâåäåíû â \cite{GMZ}, îäíàêî
ïîïûòêè èõ èñïîëüçîâàíèÿ äëÿ ïîñòðîåíèÿ óíèòàðíûõ àëãåáð ñ ýêñïîíåíòàìè áîëøå òðåõ íå
ïðèâåëè ê óñïåõó. Ýòî è âûçâàëî íåîáõîäèìîñòü ïîñòðîåíèÿ íîâûõ ïðèìåðîâ.

\subsection{}\label{s3.1}
Ïóñòü $m$ è $d$ --- íàòóðàëüíûå ÷èñëà, $m\ge 2, d\le m-1$, è $w=w_1 w_2\ldots$ --- 
áåñêîíå÷íîå ñëîâî â äâîè÷íîì àëôàâèòå $\{0;1\}$. Ðàññìîòðèì áåñêîíå÷íóþ
ïîñëåäîâàòåëüíîñòü $(m_1,m_2,\ldots)$, â êîòîðîé $m_j=m+w_j$ äëÿ âñåõ $j\ge 1$. 
Àëãåáðà $A(m,d,w)$ çàäàåòñÿ ñâîèì áàçèñîì
$$
\{ a_i, b, z_{jk}^i|1\le i \le d, 1\le j \le m_k, k=1,2,\ldots\}
$$
è òàáëèöåé óìíîæåíèÿ
$$
z^i_{jk} a_i=
\left\{
  \begin{array}{rcl}
     z^i_{j+1,k}, &\quad \hbox{åñëè} \quad & j<m_k  \\
    0, &\quad \hbox{åñëè} \quad & j=m_k\, ,
           \end{array}
\right.
$$ 
$$
z^i_{m_k,k} b=
\left\{
  \begin{array}{rcl}
     z^{i+1}_{1k}, &\quad \hbox{åñëè} \quad & i<d  \\
    z_{1,k+1}^1, &\quad \hbox{åñëè} \quad & i=d\, .
           \end{array}
\right.
$$ 
Âñå îñòàëüíûå ïðîèçâåäåíèÿ áàçèñíûõ ýëåìåíòîâ ðàâíû íóëþ. Îòìåòèì íåêîòîðûå ñâîéñòâà àëãåáðû
$A(m,d,w)$;
\begin{itemize}
\item
àëãåáðà $A(m,d,w)$ óäîâëåòâîðÿåò òîæäåñòâó $x_1(x_2x_3)\equiv 0$,
\item
ëèíåéíàÿ îáîëî÷êà $<z^i_{jk}|1\le i\le d, 1\le j \le m_k, k \ge 1 >$ ÿâëÿåòñÿ èäåàëîì â
$A(m,d,w)$ ñ íóëåâûì óìíîæåíèåì êîðàçìåðíîñòè $d+1$,
\item
åñëè $f=f(x_1,\ldots, x_n)$ --- ïîëèëèíåéíûé ìíîãî÷ëåí ñòåïåíè $n\ge d+3$, êîñîñèììåòðè÷íûé 
ïî $x_1,\ldots, x_{d+3}$, òî $f\equiv 0$ --- òîæäåñòâî â $A(m,d,w)$,
\item
åñëè $f=f(x_1,\ldots, x_n)$ --- ïîëèëèíåéíûé ìíîãî÷ëåí ñòåïåíè $n\ge 2d+4$, êîñîñèììåòðè÷íûé 
ïî $x_1,\ldots, x_{d+2}$ è ïî $x_{d+3},\ldots, x_{2d+4}$, òî $f\equiv 0$ --- òîæäåñòâî 
â $A(m,d,w)$.
\end{itemize}

Çàìå÷àíèå \ref{r1} èç ïðåäûäóùåãî ïàðàãðàôà ñðàçó æå ïðèâîäèò ê òàêîìó ðåçóëüòàòó.

\begin{lemma}\label{L4}
Ïóñòü $A(m,d,w)$ --- àëãåáðà, çàäàííàÿ áåñêîíå÷íûì ñëîâîì $w$ è öåëî÷èñëåííûìè ïàðàìåòðàìè
$m\ge 2$ è $1\le d \le m-1$. Åñëè

\begin{equation}\label{e6}
\chi_n(A)=\sum_{\lambda\vdash n} m_\lambda \chi_\lambda
\end{equation}
--- $n$-é êîõàðàêòåð àëãåáðû $A$, òî $m_\lambda\ne 0$ â (\ref{e6}) òîëüêî ïðè 
$h(\lambda)\le d+2$, ãäå $h(\lambda)$ --- âûñîòà $\lambda$, ò.å. ÷èñëî ñòðîê â äèàãðàììå
$D_\lambda$. Êðîìå òîãî, åñëè $\lambda=(\lambda_1,\ldots,\lambda_{d+2})$ è $m_\lambda\ne 0$,
òî $\lambda_{d+2}\le 1$.
\end{lemma}

 

\subsection{}\label{s3.2}
Äëÿ ïîëó÷åíèÿ âåðõíåé îöåíêè íà ðîñò $\{c_n(A(m,d,w)) \}$ íàì íåîáõîäèìî ñíà÷àëà îãðàíè÷èòü
ðîñò êîäëèíû $\{l_n(A(m,d,w)) \}$.

Ïóñòü ñíà÷àëà $A$ --- ïðîèçâîëüíàÿ àëãåáðà. Îáîçíà÷èì ÷åðåç $R=R(y_1,y_2,\ldots)$
îòíîñèòåëüíî ñâîáîäíóþ àëãåáðó ìíîãîîáðàçèÿ $var(A)$, ïîðîæäåííîãî àëãåáðîé $A$, à ÷åðåç
$$
W_n^{(p)}(A)=Span\{y_{i_1}\ldots y_{i_n}|1\le i_1,\ldots,i_n\le p\}
$$
ëèíåéíóþ îáîëî÷êó âñåõ îäíî÷ëåíîâ ñòåïåíè $n$ îò $y_{1},\ldots, y_{p}$ ñî âñåâîçìîæíûìè
ðàññòàíîâêàìè ñêîáîê, ò.å. âñåõ îäíîðîäíûõ ñòåïåíè $n$ ïîëèíîìîâ îò $y_{1},\ldots, y_{p}$
â $R$.

\begin{lemma}\label{L5} \cite[ëåììà 4.1]{GMZ}
Ïóñòü $A$ --- àëãåáðà ñ $n$-ì êîõàðàêòåðîì 
$\chi_n(A)=\sum_{\lambda\vdash n} m_\lambda\chi_\lambda$. Òîãäà äëÿ ëþáîãî $\lambda\vdash n$
ñ $h(\lambda)\le p$ âûïîëíÿåòñÿ íåðàâåíñòâî
\begin{equation}\label{e7}
m_\lambda\le \dim W_n^{(p)}(A).
\end{equation}
\end{lemma}

Âñþäó  â äàëüíåéøåì ìû áóäåì îïóñêàòü ñêîáêè â ëåâîíîðìèðîâàííîì ïðîèçâåäåíèè, ò.å.
çàïèñûâàòü $(zt)v$ êàê $ztv$. Ýòî ñîãëàøåíèå îñîáåííî óäîáíî ïðè ðàáîòå ñ àëãåáðàìè
$A(m,d,w)$, ïîñêîëüêó âñå íåíóëåâûå ïðîèçâåäåíèÿ â íèõ ëåâîíîðìèðîâàíû â ñèëó òîæäåñòâà 
$x_1(x_2x_3)\equiv 0$.

\begin{lemma}\label{L6} 
Ïóñòü $A=A(m,d,w)$ çàäàíà $m,d$ è áåñêîíå÷íûì ñëîâîì $w$. Òîãäà
$$
\dim W_n^{(p)}(A) \le d(m+1)n^{(d+1)p} Comp_w(n).
$$ 
\end{lemma}

\begin{proof}
Îáîçíà÷èì ÷åðåç $W$ ëèíåéíóþ îáîëî÷êó îäíî÷ëåíîâ âèäà $ty_{i_1}\ldots y_{i_{n-1}}$, ãäå
$t=y_{p+1}$, $1\le i_1,\ldots, y_{n-1}\le p$. Òîãäà
$$
\dim W_n^{(p)}(A) \le p\dim W.
$$

Ïóñòü $y$ --- íåêîòîðûé ýëåìåíò èç $W$. ßñíî, ÷òî $y$ --- íåíóëåâîé òîãäà è òîëüêî òîãäà,
êîãäà ñóùåñòâóåò ãîìîìîðôèçì $\sigma: R\to A$, ïðè êîòîðîì $\sigma(y)\ne 0$.

×òîáû ïîëó÷èòü îöåíêó íà ðàçìåðíîñòü $W$ ðàññìîòðèì ñëåäóþùóþ êîíñòðóêöèþ. Ïóñòü
$F<a_1,\ldots, a_d,b>$ --- câîáîäíàÿ àññîöèàòèâíàÿ àëãåáðà ñ ïîðîæäàþùèìè
$a_1,\ldots, a_d,b$ è $M$ --- ñâîáîäíûé ïðàâûé $F<a_1,\ldots, a_d,b>$-ìîäóëü ñ
îäíèì ïîðîæäàþùèì $x$. Òîãäà ëþáîé ýëåìåíò èç $M$ ìîæíî çàïèñàòü â âèäå ëèíåéíîé êîìáèíàöèè ýëåìåíòîâ âèäà $xf(a_1,\ldots, a_d,b)$, ãäå $f(a_1,\ldots, a_d,b)$ --- îäíî÷ëåí îò
$a_1,\ldots, a_d,b$. 

Ïóñòü òåïåðü $\sigma$ --- ãîìîìîðôèçì èç $R$ â $A$. ßñíî, ÷òî óñëîâèå $\sigma(y)=0$, $y\in W$,
äîñòàòî÷íî ïðîâåðèòü òîëüêî äëÿ âñåõ ãîìîìîðôèçìîâ âèäà
$$
\sigma(t)= z^i_{jk}, \sigma(y_s)=\alpha_1^sa_1+\cdots+\alpha_d^s a_d+\beta^s b,
1\le s \le p,
$$
ãäå $\alpha_r^s, \beta^s$ --- ëþáûå ñêàëÿðû èç $F$.
 
Ðàññìîòðèì êîëüöî ìíîãî÷ëåíîâ $F[\alpha_r^s, \beta^s]$, $1\le s\le p, 1\le r \le d$, â
êîòîðîì $\alpha_r^s, \beta^s$  óæå ðàññìàòðèâàþòñÿ êàê ïåðåìåííûå. Äëÿ êðàòêîñòè ìû áóäåì
îáîçíà÷àòü åãî êàê $F[\alpha, \beta]$. Ïóñòü $\psi: W\to M\otimes F[\alpha,\beta]$ --- 
ëèíåéíîå îòîáðàæåíèå, çàäàííîå êàê
\begin{equation}\label{*e}
\psi(ty_{i_1}\ldots y_{i_{n-1}})=
x(\alpha_1^{i_1}a_1+\ldots+\alpha_d^{i_1}a_d+\beta^{i_1} b)\ldots
(\alpha_1^{i_{n-1}}a_1+\ldots+\alpha_d^{i_{n-1}}a_d+\beta^{i_{n-1}} b),
\end{equation}
Çàìåòèì, ÷òî åñëè
$$
h=\sum \lambda_{i_1\ldots i_{n-1}}ty_{i_1}\ldots y_{i_{n-1}},
$$
òî $\psi(h)=0$ òîëüêî åñëè $h\equiv 0$ --- òîæäåñòâî â $A$, ò.å. $h$ --- íóëåâîé ýëåìåíò
îòíîñèòåëüíî ñâîáîäíîé àëãåáðû $R(y_1,y_2,\ldots$)~. Ýòî îçíà÷àåò, ÷òî $(\ref{*e})$ êîððåêòíî
 îïðåäåëÿåò $\psi$ è ÷òî $\psi$ --- âëîæåíèå $W$  â $M\otimes F[\alpha,\beta]$.

Îáîçíà÷èì ÷åðåç $\varphi^i_{jk}$ ëèíåéíîå îòîáðàæåíèå èç  $M$ â $A$,  äëÿ êîòîðîãî
\begin{equation}\label{*ee}
\varphi^i_{jk}(xf(a_1,\ldots, a_d,b))=z^i_{jk}f(a_1,\ldots, a_d,b),
\end{equation}
ãäå ìíîãî÷ëåí â ïðàâîé ÷àñòè (\ref{*ee}) èíòåðïðåòèðóåòñÿ êàê ìíîãî÷ëåí îò ïðàâûõ
óìíîæåíèé íà $a_1,\ldots, a_d,b$  â àëãåáðå $A$. Ïîëîæèì
$$
I=\bigcap_{i,j,k} \ker\varphi^i_{jk}.
$$
Åñëè  $y\in M/I\otimes F[\alpha,\beta]$, òî ïðè ëþáîé ñïåöèôèêàöèè ïåðåìåíûõ
$\{\alpha_r^s,\beta^s\}$ â ïîëå $F$ è ïðè ëþáîé ïîäñòàíîâêå $\varphi^i_{jk}: M\to A$
ýëåìåíò $y$ ïåðåõîäèò â íîëü. Ýòî îçíà÷àåò, ÷òî $W$ âëîæåíî â
$M/I\otimes F[\alpha,\beta]$. Áîëåå òîãî, $W$ âëîæåíî â
$M/I\otimes F[\alpha,\beta]^{(n-1)}$, ãäå $F[\alpha,\beta]^{(n-1)}$ --- ïîäïðîñòðàíñòâî
îäíîðîäíûõ ìíîãî÷ëåíîâ ñòåïåíè $n-1$ â $F[\alpha,\beta]$.  ÷àñòíîñòè,
$$
\dim W \le \dim F[\alpha,\beta]^{(n-1)} \cdot \dim M/I.
$$
Î÷åâèäíî, ÷òî
$$
\dim F[\alpha,\beta]^{(n-1)} \le (n-1)^{dp+p} \le n^{(d+1)p}.
$$

Òåïåðü îöåíèì ñâåðõó ðàçìåðíîñòü $M/I$.
Çàôèêñèðóåì èíäåêñû $i,j,k$. Çàìåòèì ñíà÷àëà, ÷òî èç ïðàâèë óìíîæåíèÿ áàçèñíûõ ýëåìåíòîâ â 
$A$ ñëåäóåò, ÷òî ñóùåñòâóåò ðîâíî îäèí îäíî÷ëåí $f^i_{j,k}$, íå ëåæàùèé â ÿäðå
 $\varphi^i_{j,k}$:
$$
f^i_{j,k}=x\underbrace{a_i\ldots a_i}_{m_k-j}b 
\underbrace{a_{i+1}\ldots a_{i+1}}_{p_1}b\ldots b \underbrace{a_{i+r}\ldots a_{i+r}}_{p_r}b
\underbrace{a_{i+r+1}\ldots a_{i+r+1}}_{s},
$$
ãäå èíäåêñû ó $a_{i+1}\ldots a_{i+r+1}$ âû÷èñëÿþòñÿ ïî ìîäóëþ $d$,
$m_k-j+p_1+\ldots+p_r+s+r+1=n-1$, $s\le d$, à âñå $p_1,\ldots, p_r$ ðàâíû îäíîìó èç
$m_k,m_{k+1},\ldots$ è îïðåäåëÿþòñÿ îäíîçíà÷íî ïîäñëîâîì $w(k, k+n-1) =
(w_k,w_{k+1},\ldots ,w_{k+n-1})$ äëèíû $n$ ñëîâà $w$.  ÷àñòíîñòè, $f^i_{j,k}=f^i_{j,l}$ è
$\ker\varphi^i_{j,k}= \ker\varphi^i_{j,l}$, åñëè $w(k, k+n-1)=w(l, l+n-1)$ â ñëîâå $w$.
Òàê êàê $1\le i \le d, 1\le j \le m+1$, òî ÷èñëî ðàçëè÷íûõ ÿäåð $\ker\varphi^i_{j,k}$ íå
ïðåâîñõîäèò $d(m+1)Comp_w(n)$. Ñëåäîâàòåëüíî,
$$
\dim\frac{M}{I} \le d(m+1)Comp_w(n),\quad \dim W_n^{(p)}(A) \le d(m+1)n^{(d+1)p} Comp_w(n),
$$
è ëåììà äîêàçàíà.
\end{proof}

 êà÷åñòâå ñëåäñòâèÿ ìû ïîëó÷àåì îöåíêó ðîñòà êîäëèíû äëÿ àëãåáðû, çàäàííîé ñëîâîì Øòóðìà èëè 
áåñêîíå÷íûì ïåðèîäè÷åñêèì ñëîâîì.

\begin{propos}\label{p2}
Ïóñòü $A=A(m,d,w)$, ãäå $w$ ---  ñëîâî Øòóðìà èëè áåñêîíå÷íîå ïåðèîäè÷åñêîå ñëîâî. 
Òîãäà
$$
l_n(A) \le 2d^2(m+1)n^{(d+1)(d+3)}(n+1).
$$
\end{propos}
\begin{proof}
Ñîãëàñíî ëåììå \ref{L4} ìû èìååì: $h(\lambda)\le d+2, \lambda_{d+2}\le 1$ äëÿ ëþáîãî
ðàçáèåíèÿ $\lambda\vdash n$ ñ íåíóëåâîé êðàòíîñòüþ $m_\lambda$. Êîëè÷åñòâî òàêèõ
ðàçáèåíèé íà ïðåâîñõîäèò $2d n^{d+1}$. Ïîýòîìó ëåììû \ref{L5} è \ref{L6} äàþò
òðåáóåìóþ îöåíêó.
\end{proof}

\subsection{}\label{s3.3}
Òåïåðü ìû ìîæåì ïðèñòóïèòü ê ïîëó÷åíèþ âåðõíèõ îöåíîê PI-ýêñïîíåíò.

Ïóñòü $A=A(m,d,w)$ --- àëãåáðà, ïîñòðîåííàÿ ïî áåñêîíå÷íîìó ñëîâó $w$. ãäå 
$w$ --- ïåðèîäè÷åñêîå ñëîâî èëè ñëîâî Øòóðìà. Åñëè $f=f(z^i_{jk},a_1,\ldots,a_d,b)$ ---
àññîöèàòèâíîå ñëîâî â àëôàâèòå $\{z^i_{jk},a_1,\ldots,a_d,b\}$, òî ìîæíî ãîâîðèòü î åãî
ñòåïåíÿõ $\deg_b f,\deg_{a_i}f,\deg_{z^i_{jk}}f$ ïî ïåðåìåííûì, îá îáùåé ñòåïåíè $\deg f$,
à òàêæå î çíà÷åíèè $f$ â $A$, åñëè ðàññìàòðèâàòü åãî êàê ëåâîíîðìèðîâàííîå ïðîèçâåäåíèå
áàçèñíûõ ýëåìåíòîâ.

Íàì ïîíàäîáèòñÿ îäíî äîñòàòî÷íîí óñëîâèå òîãî, ÷òî $f\ne 0$.

\begin{lemma}\label{A0}
Äëÿ çàäàííûõ $m,d,w$ íàéäåòñÿ òàêàÿ ïîñëåäîâàòåëüíîñòü $\{\varepsilon_n>0\}, n=1,2,\ldots$, 
÷òî åñëè $f=f(z^i_{jk},a_1,\ldots,a_d,b)$ --- îäíî÷ëåí ñòåïåíè $n$, íå ðàâíûé íóëþ â
$A(m,d,w)$, òî
$$
\frac{\deg_b f}{n} \le\frac{1}{m+\alpha}+\varepsilon_n,
$$
ãäå $\alpha=\pi(w)$ --- íàêëîí ñëîâà $w$. Ïðè ýòîì $\varepsilon_n\to 0$, åñëè $n\to\infty$.
\end{lemma}
\begin{proof}
Ñëîâî $f$ ìîæíî çàïèñàòü â âèäå $f=ZPQ$, ãäå $Z$ --- ïðîèçâåäåíèå áàçèñíûõ ýëåìåíòîâ 
$\{z^i_{jk},a_\alpha, b\}$ ñòåïåíè $\deg Z\le (m+1)d, Q=Q(a_1,\ldots,a_d,b), \deg Q \le 
(m+1)d$, à 
$$
P=a_1^{m_k-1}b\ldots a_d^{m_k-1}b\ldots a_1^{m_{k+t-1}-1}b\ldots a_d^{m_{k+t-1}-1}b.
$$
Òîãäà $\deg_b P=td$ è
$$
\deg_{a_i}P=(m_k-1)+\ldots+(m_{k+t-1}-1)=m_k+\ldots+m_{k+t-1}-t=
(m-1)t+w_k+\ldots+w_{k+t-1}
$$
äëÿ ëþáîãî $i=1,\ldots,d$. Êàê îòìå÷åíî â ïðåäëîæåíèè \ref{p1} äëÿ ñëîâà $w$ ñóùåñòâóåò òàêàÿ
 êîíñòàíòà $C$, ÷òî $|w_k+\ldots+w_{k+t-1}-\alpha t| \le C$. Ïîýòîìó
$$
\deg P=dmt+d(w_k+\ldots+w_{k+t-1})\ge dt(m+\alpha-\frac{C}{t})
$$
è $n=\deg f\ge \deg P$, à $\deg_b f \le td+2d=(t+2)d$. Ñëåäîâàòåëüíî,
$$
\frac{\deg_b f}{n} \le \frac{1+\frac{2d}{t}}{m+\alpha-\frac{C}{t}}.
$$
Ïîñêîëüêó $n\le d(m_k+\ldots+m_{k+t-1})+2(m+1)d \le d(m+1)t+2(m+1)d$, òî
$$
t\ge \frac{n}{d(m+1)} - 2
$$
è $t$ ðàñòåò ëèíåéíî ñ ðîñòîì $n$. Ñëåäîâàòåëüíî,
$$
\lim_{n\to\infty} \frac{\deg_b f}{n} = \frac{1}{m+\alpha},
$$
îòêóäà ñëåäóåò óòâåðæäåíèå ëåììû.

\end{proof}

Òåïåðü ìû ïîëó÷èì îöåíêó ñâåðõó íà ðîñò êîðàçìåðíîñòåé àëãåáðû $A(m,d,w)$.

\begin{lemma}\label{A1}
Ïóñòü $A=A(m,d,w)$, ãäå $w$ --- áåñêîíå÷íîå ïåðèîäè÷åñêîå ñëîâî èëè  ñëîâî Øòóðìà ñ
íàêëîíîì $\alpha=\pi(w)$. Òîãäà
$$
\overline{exp}(A)\le\Phi_d(\frac{1}{m+\alpha}),
$$
ãäå ôóíêöèÿ $\Phi_d$ çàäàíà ôîðìóëîé (\ref{e4}).
\end{lemma}
\begin{proof}
Çàôèêñèðóåì ïðîèçâîëüíîå ìàëîå $\varepsilon>0$ è ïîêàæåì, ÷òî äëÿ íåãî ñóùåñòâóåò òàêîå
$N$, ÷òî åñëè $n\ge N, \lambda\vdash n$ è $m_\lambda\ne 0$ â (\ref{e6}), òî
$$
\Phi(\lambda) \le \Phi_d\left(\frac{1}{m+\alpha}+\varepsilon\right).
$$

Ïóñòü ñíà÷àëà $\lambda_{d+1}=0$, ò.å. $\lambda=(\lambda_1,\ldots,\lambda_d,0,0)$. Òîãäà
$$
\Phi(\lambda) \le\Phi\left(\frac{1}{d},\ldots,\frac{1}{d},0,0\right) \le 
\Phi\left(\underbrace{\theta,\ldots,\theta}_d,\frac{1}{m+\alpha}\right) =
\Phi_d\left(\frac{1}{m+\alpha} \right).
$$

Ïóñòü òåïåðü $\lambda_{d+1}\ne 0$. Òîãäà â ñèëó çàìå÷àíèÿ \ref{r1} ñóùåñòâóåò ïîëèëèíåéíûé
ìíîãî÷ëåí $h=h(x_1,\ldots,x_n)$ êîñîñèììåòðè÷íûé ïî $\lambda_1$ íàáîðàì ïåðåìåííûõ
$X_{1},\ldots, X_{\lambda_1}$, ïðè÷åì $|X_1|=d+1$ èëè $d+2$ â çàâèñèìîñòè îò çíà÷åíèÿ
$\lambda_{d+2}$ (0 èëè 1), à $|X_2|=\ldots= |X_{\lambda_{d+1}}|=d+1$, íå ÿâëÿþùèéñÿ
òîæäåñòâîì $A$. Ñëåäîâàòåëüíî, ñóùåñòâóåò òàêàÿ ïîäñòàíîâêà $\varphi: X\to\{a_r,b,z^i_{jk}\}$,
÷òî $f=\varphi(h)=f(z^i_{jk},a_1,\ldots, a_d,b)$ --- íåíóëåâîé îäíî÷ëåí â $A$. Òîãäà
$\deg_b f\ge \lambda_{d+1}$, è ïî ëåììå \ref{A0}
$$
\frac{\lambda_{d+1}}{n} \le \frac{\deg_b f}{n} \le \frac{1}{m+\alpha} + \varepsilon_n.
$$
Åñëè $\lambda_{d+2}=0$, òî
$$
\Phi(\lambda) \le \Phi(\underbrace{\theta,\ldots,\theta}_d,\frac{1}{m+\alpha}+\varepsilon_n,0)
= \Phi_d(\frac{1}{m+\alpha}+\varepsilon_n) \le \Phi_d(\frac{1}{m+\alpha}+\varepsilon)
$$
ïðè âñåõ áîëüøèõ $n$, ïîñêîëüêó $\varepsilon_n\to 0$  ñ ðîñòîì $n$, à ôóíêöèÿ
$\Phi_d(\frac{1}{m+\alpha}+x)$ âîçðàñòàåò  ïðè óâåëè÷åíèè $x$. Åñëè æå $\lambda_{d+2}=1$,
$$
\Phi(\lambda) \le 
\Phi(\underbrace{\theta,\ldots,\theta}_d,\frac{1}{m+\alpha}+\varepsilon_n,\frac{1}{n}).
$$

Ïîñêîëüêó $\varepsilon_n\to 0$ è $\frac{1}{n}\to 0$ ïðè $n\to\infty$, òî 
$$
\lim_{n\to\infty}\Phi(\theta,\ldots,\theta,\frac{1}{m+\alpha}+\varepsilon_n,\frac{1}{n})
=\Phi(\bar\theta,\ldots,\bar\theta,\frac{1}{m+\alpha},0),
$$
ãäå $\bar\theta d+\frac{1}{m+\alpha}=1$. Ñëåäîâàòåëüíî, íàéäåòñÿ òàêîå $n$, ÷òî
$$
\Phi(\theta,\ldots,\theta,\frac{1}{m+\alpha}+\varepsilon_n,\frac{1}{n}) \le 
\Phi(\theta',\ldots,\theta',\frac{1}{m+\alpha}+\varepsilon,0)
$$
Ñëåäîâàòåëüíî,
$$
\Phi(\lambda) \le 
\Phi\left(\theta,\ldots,\theta,\frac{1}{m+\alpha}+\varepsilon_n,\frac{1}{n}\right) \le 
\Phi\left(\theta',\ldots,\theta',\frac{1}{m+\alpha}+\varepsilon,0\right)
$$
$$ 
= \Phi_d\left(\frac{1}{m+\alpha}+\varepsilon\right),
$$
ãäå $\theta'd+ \frac{1}{m+\alpha}+\varepsilon=1$ è $\theta' \ge \theta$.
Ïîñêîëüêó
$$
c_n(A)=\sum m_\lambda d_\lambda \le l_n(A) \max\{d_\lambda| \lambda\vdash n, m_\lambda\ne 0\},
$$
òî èç ëåììû \ref{L1} è ïðåäëîæåíèÿ \ref{p2} ñëåäóåò, ÷òî
$$
\overline{\lim_{n\to\infty}}\sqrt[n]{c_n(A)} \le 
\Phi_d\left(\frac{1}{m+\alpha}+\varepsilon\right)
$$
äëÿ ëþáîãî ôèêñèðîâàííîãî $\varepsilon >0$. Ñëåäîâàòåëüíî, 
$$
\overline{exp}(A) \le \Phi_d\left(\frac{1}{m+\alpha}\right),
$$
è ëåììà äîêàçàíà.

\end{proof}

Òåïåðü ïåðåéäåì ê íèæíåé îöåíêå ðîñòà êîðàçìåðíîñòåé àëãåáðû $A(m,d,w)$.

\begin{lemma}\label{A2}
Ïóñòü $A(m,d,w)$ --- àëãåáðà èç ëåììû \ref{A1}. Òîãäà $\underline{exp}(A)\ge
\Phi_d\left(\frac{1}{m+\alpha}\right)$, ãäå $\alpha=\pi(w)$ --- íàêëîí ñëîâà $w$.
\end{lemma}

\begin{proof}
Ðàññìîòðèì îäíî÷ëåí
$$
h_1=zx^1_1x^1_2\ldots x^1_py^1_1\ldots x^d_1x^d_2\ldots x^d_py^1_d
$$
â ñâîáîäíîé àëãåáðå $F\{X\}$ ñòåïåíè $(p+1)d+1$, ãäå $p=m_1-1 \ge m-1\ge d$. Ïóñòü 
$Alt^1_1:P_{(p+1)d+1}\to P_{(p+1)d+1}$ --- îïåðàòîð àëüòåðíèðîâàíèÿ ïî
$z,x^1_1, x^2_1,\ldots, x^d_1,y^1_1$, à $Alt^1_i$ --- îïåðàòîð àëüòåðíèðîâàíèÿ ïî
$x^1_i, x^2_i,\ldots, x^d_i,y^1_i$ äëÿ âñåõ $2\le i \le d$. Åñëè $p>d$, òî îáîçíà÷èì 
òàêæå ÷åðåç $Alt^1_{d+j}$ àëüòåðíèðîâàíèå ïî $x^1_{d+j}, x^2_{d+j},\ldots, x^d_{d+j}$
äëÿ âñåõ $1\le j \le p-d$. Ïîëîæèì $f_1= Alt^1_1\ldots Alt^1_p(h_1)$.

Ðàññìîòðèì ïîäñòàíîâêó $\varphi: X\to A$, ïðè êîòîðîé
$$
\varphi(z)=z^1_{11},\varphi(x^1_1)=\ldots =\varphi(x^1_p)=a_1,\ldots,
\varphi(x^d_1)=\ldots =\varphi(x^d_p)=a_d,
$$
$$
\varphi(y^1_1)=\ldots =\varphi(y^1_d)=b.
$$
Òîãäà
$$
\varphi(f_1)= z^1_{11}\underbrace{a_1\ldots a_1}_{m_1-1}b\ldots
\underbrace{a_d\ldots a_d}_{m_1-1}b= z^1_{12}.
$$
Çàìåòèì, ÷òî ðåçóëüòàò ïîäñòàíîâêè $\varphi$ íå èçìåíèòñÿ (ñ òî÷íîñòüþ äî íåíóëåâîãî
 ìíîæèòåëÿ), åñëè ïðèìåíèòü åå íå ê ñàìîìó
ýëåìåíòó $f_1$, à ê åãî ñèììåòðèçàöèè $Sym\, f_1$, ãäå $Sym$ îçíà÷àåò ñèììåòðèçàöèþ 
ïî íàáîðàì
$\{x^1_1,\ldots, x^1_p\}$, $\ldots$, $\{x^d_1,\ldots, x^d_p\}$, $\{y^1_1,\ldots, y^1_d\}$. 
Òîãäà ìíîãî÷ëåí $Sym\, f_1$ ïîðîæäàåò â $P_{(p+1)d+1}$ íåïðèâîäèìûé $F[S_{(p+1)d+1}]$-ìîäóëü,
ñîîòâåòñòâóþùèé ðàçáèåíèþ $\lambda =(\lambda_1,\ldots,\lambda_{d+2})$, ãäå
$\lambda_1=\ldots=\lambda_d=p=m_1-1$, $\lambda_{d+1}=d, \lambda_{d+2}=1$, à óñëîâèå
$\varphi(Sym\, f_1)\ne 0$ îçíà÷àåò, ÷òî êðàòíîñòü $m_\lambda$ â ðàçëîæåíèèè (\ref{e1}) íå
ðàâíà íóëþ.

Îáîçíà÷èì $p_1=p$. Äàëåå äëÿ  âñåõ $j=2,3,\ldots$ ñòðîèì ìíîãî÷ëåíû $f_2,f_3,\ldots$ ñëåäóþùèì
îáðàçîì. Åñëè $f_1,\ldots, f_{j-1}$ óæå ïîñòðîåíû, òî áåðåì
$$
h_j=f_{j-1}x^1_{q+1}\ldots x^1_{q+p_j}y_1^j\ldots
x^d_{q+1}\ldots x^d_{q+p_j}y_d^j,
$$
ãäå $q=p_1+\ldots +p_{j-1},\, p_j=m_j-1$ è îïðåäåëÿåì $f_j$ êàê
$$
f_j=Alt^j_1\ldots Alt^j_{p_j}(h_j),
$$
ãäå $Alt^j_1,\ldots ,Alt^j_d$ ---  àëüòåðíèðîâàíèÿ ïî íàáîðàì 
$\{x^1_{q+1},\ldots,x^d_{q+1}, y^j_1\}$, $\ldots$,\\ $\{x^1_{q+d},\ldots,x^d_{q+d}, y^j_d\}$ 
ñîîòâåòñòâåííî. Åñëè æå $p_j>d$, òî $Alt^j_{d+i}$ --- àëüòåðíèðîâàíèå ïî
$\{x^1_{q+d+i},\ldots,x^d_{q+d+i}\}$, $1\le i \le p_j-d$. Ðàñøèðèì äåéñòâèå ïîäñòàíîâêè
$\varphi: X\to A$, ïîñòðîåííîé íà $(j-1)$-ì øàãå, ïîëàãàÿ
$$
\varphi(x^1_{q+1})=\ldots= \varphi(x^1_{q+p_j})=a_1,\ldots,
\varphi(x^d_{q+1})=\ldots= \varphi(x^d_{q+p_j})=a_d,
$$
$$
\varphi(y^j_{1})\ldots= \varphi(y^j_d)=b.
$$
Òîãäà, êàê è ïðåæäå,
$$
\varphi(Sym\, f_j)=\gamma z^1_{1,j+1} \ne 0,
$$
ãäå ñèìåòðèçàöèÿ $Sym$ ïðîâîäèòñÿ ïî íàáîðàì
$$
\{x^1_1,x^1_2,\ldots,x^1_{q+p_j}\},\ldots,\{x^d_1,x^d_2,\ldots,x^d_{q+p_j}\},
\{y^1_1,\ldots,y^1_d,\ldots,y^j_1,\ldots, y^j_d\}.
$$
Òîãäà, êàê è ïðè $j=1$, $Sym\, f_j$ ïîðîæäàåò íåïðèâîäèìûé ìîäóëü ñ õàðàêòåðîì $\chi_\lambda$,
ãäå $\lambda=(\lambda_1,\ldots,\lambda_{d+2})$, $\lambda_1=\ldots= \lambda_d=m_1+\ldots+m_j-j,
\lambda_{d+1}=jd, \lambda_{d+2}=1$, è $m_\lambda\ne 0$ â (\ref{e1}).

Òàêèì îáðàçîì, äëÿ êàæäîãî íàòóðàëüíîãî $t$ ìû ïîñòðîèëè íå ÿâëÿþùèéñÿ òîæäåñòâîì 
ìíîãî÷ëåí $f_t$ ñòåïåíè
$$
n=n(t)=(m_1+\ldots+m_t)d+1=tmd+d(w_1+\ldots+w_t)+1.
$$
Ïðè ýòîì íåíóëåâîå çíà÷åíèå $f_t$ ïðèíèìàåò ïðè ïîäñòàíîâêå $\varphi: X\to A$, êîãäà
ýëåìåíò $b$ ïîäñòàâëÿåòñÿ $td$ ðàç. Òîãäà ïî ëåììå \ref{A0}
$$
\frac{td}{n} \le \frac{1}{m+\alpha}+\varepsilon_n,
$$
ãäå $\alpha=\pi(w)$ --- íàêëîí $w$, à $\varepsilon_n\to 0$  ïðè $n\to\infty$. Êðîìå òîãî,
ñèììåòðèçàöèÿ $Sym\, f_t$ òîæå íå ÿâëÿåòñÿ òîæäåñòâîì â $A$, $\varphi(Sym\, f_t)=
K\cdot\varphi(f_t), K\ne 0$, è ïîðîæäàåò â $P_n$ íåïðèâîäèìûé $F[S_n]$-ìîäóëü ñ õàðàêòåðîì
$\chi_{\lambda^{(n)}}$, ãäå
$$
\lambda^{(n)}=(\lambda_1,\ldots,\lambda_{d+2}),\,
\lambda_1=\ldots=\lambda_d=m_1+\ldots+m_t-1,\, \lambda_{d+1}=td,\, \lambda_{d+2}=1.
$$
Ñëåäîâàòåëüíî,
$$
\frac{\lambda_{d+1}}{n}=\frac{1}{m+\frac{w_1+\ldots+w_t}{t}+\frac{1}{td}}=\beta
$$
è
$$
\Phi(\lambda^{(n)})=
\Phi\left(\underbrace{\frac{\lambda_1}{n},\ldots,\frac{\lambda_1}{n}}_{d},\beta,\frac{1}{n}\right).
$$

×òîáû ïîëó÷èòü îöåíêó ñíèçó íà $\Phi(\lambda^{(n)})$, âîñïîëüçóåìñÿ ñâîéñòâàìè ïåðèîäè÷åñêèõ 
ñëîâ è ñëîâ Øòóðìà. Ñîãëàñíî ïðåäëîæåíèþ \ref{p1}
$$
\lim_{t\to\infty} \frac{w_1+\cdots+w_t}{t}=\alpha,
$$
à ïîñêîëüêó $mtd \le n \le (m+1)td$, òî âåëè÷èíó $\frac{w_1+\cdots+w_t}{t}$ ìîæíî ñäåëàòü
ñêîëü óãîäíî áëèçêîé ê $\alpha$ äëÿ âñåõ äîñòàòî÷íî áîëüøèõ $n$. Ñëåäîâàòåëüíî, äëÿ
ëþáîãî $\varepsilon>0$ íàéäåòñÿ òàêîå $N$, ÷òî
$$
\beta=\frac{1}{m+\frac{w_1+\ldots+w_t}{t}+\frac{1}{td}} \ge\frac{1}{m+\alpha}-\varepsilon
$$
ïðè âñåõ $n\ge N$. Òîãäà èç ñâîéñòâ ôóíêöèè $\Phi$ ìû ïîëó÷àåì
$$
\Phi(\lambda^{(n)})\ge
\Phi\left(\underbrace{\theta,\ldots,\theta}_{d},\frac{1}{m+\alpha}-\varepsilon,0\right)=
\Phi_d\left(\frac{1}{m+\alpha}-\varepsilon\right),
$$
ãäå $\theta d+ \frac{1}{m+\alpha}-\varepsilon=1$.

Òàê êàê 
$$
c_n(A)\ge d_{\lambda^{(n)}} \ge \frac{1}{n^{(d+2)^2+d+2}}\Phi\left(\lambda^{(n)} \right)^n
$$
â ñèëó ëåììû \ref{L1}, à $\varepsilon >0$ âûáðàíî ïðîèçâîëüíî, òî
$$
\underline{\lim}_{n(t)\to\infty} \sqrt[n(t)]{c_{n(t)}(A)} \ge 
\Phi_d\left(\frac{1}{m+\alpha}\right).
$$
Îñòàëîñü çàìåòèòü, ÷òî $c_n(A)$ --- íåóáûâàþùàÿ ïîñëåäîâàòåëüíîñòü è ÷òî 
$n(t+1)-n(t) \le (m+1)d$, îòêóäà ñëåäóåò ðàâåíñòâî
$$
\underline{exp}(A)=\underline{\lim}_{n\to\infty} \sqrt[n]{c_{n(t)}(A)} \ge 
\Phi_d\left(\frac{1}{m+\alpha}\right).
$$
Ëåììà äîêàçàíà.
\end{proof}

Ëåììû \ref{A1} è \ref{A2} ñðàçó æå äàþò íàì îñíîâíîé ðåçóëüòàò äàííîãî ïàðàãðàôà.

\begin{theorem}\label{t1}
Ïóñòü $m$ è $d$ --- öåëûå ÷èñëà, $m\ge 2, 1\le d \le m-1$, à $w$ --- áåñêîíå÷íîå ïåðèîäè÷åñêîå 
ñëîâî èëè ñëîâî Øòóðìà ñ  íàêëîíîì $\alpha$. Òîãäà PI-ýêñïîíåíòà àëãåáðû $A(m,d,w)$ ñóùåñòâóåò è ðàâíà
$$
exp(A)=\Phi_d\left(\frac{1}{m+\alpha} \right)=
\Phi\left(\underbrace{\frac{m+\alpha-1}{d(m+\alpha)},\ldots,
\frac{m+\alpha-1}{d(m+\alpha)}}_d,\frac{1}{m+\alpha} \right).
$$
\end{theorem}
 

\section{Ýêñïîíåíòû àëãåáð ñ ïðèñîåäèíåííîé åäèíèöåé}\label{s4}

\subsection{}\label{s4.1}
Íàïîìíèì, ÷òî åñëè ê àëãåáðå $A$ ïðèñîåäèíÿåòñÿ âíåøíèì îáðàçîì åäèíèöà, òî ïîëó÷åííóþ 
â ðåçóëüòàòå àëãåáðó ìû îáîçíà÷àåì êàê $A^\#$. Ìû áóäåì ïðèñîåäèíÿòü åäèíèöû ê àëãåáðàì
$A(m,d,w)$, ðàññìîòðåííûì â ïðåäûäóùåì ïàðàãðàôå.

Íàì ïîíàäîáèòñÿ òåõíè÷åñêèé ðåçóëüòàò èç ðàáîòû \cite{RZ}.

Íàïîìíèì, ÷òî äëÿ çàäàííîé àëãåáðû $B$ ÷åðåç $W_n^{(p)}(B)$ îáîçíà÷àåòñÿ ïîäïðîñòðàíñòâî âñåõ îäíîðîäíûõ ñòåïåíè $n$ ìíîãî÷ëåíîâ îò $y_1,\ldots, y_p$ â îòíîñèòåëüíî  ñâîáîäíîé àëãåáðå
$R(y_1,y_2,\ldots)$ ìíîãîîáðàçèÿ $var(B)$ ñî ñâîáîäíûìè ïîðîæäàþùèìè $y_1,y_2,\ldots$~.

\begin{lemma}\label{LL1}\cite[ëåììà 6]{RZ}
Ïóñòü $B$ --- ïðîèçâîëüíàÿ àëãåáðà è ïóñòü $\dim W_n^{(p)}(B) \le\alpha n^T$ äëÿ íåêîòîðûõ
$\alpha\in\mathbb{R}$ è $T\in \mathbb{N}$. Òîãäà 
$\dim W_n^{(p)}(B^\#) \le\alpha(n+1)^{T+p+1}$ 
\end{lemma}

Ñíà÷àëà ìû îöåíèì ñâåðõó ðîñò êîäëèíû.

\begin{lemma}\label{LL2}
Ïóñòü $A=A(m,d,w)$ --- àëãåáðà èç ïðåäûäóùåãî ïàðàãðàôà, ãäå $m\ge 2, d\le m-1$, $w$ --- 
ñëîâî Øòóðìà èëè áåñêîíå÷íîå ïåðèîäè÷åñêîå ñëîâî. Òîãäà
$$
l_n(A^\#)\le (n+1)^{3(d+3)^2}
$$
äëÿ âñåõ äîñòàòî÷íî áîëüøèõ $n$.
\end{lemma}

\begin{proof}

Ïî ëåììå \ref{L6}
$$
\dim W_n^{(d+3)}(A) \le d(m+1)n^{(d+1)(d+3)}Comp_w(n).
$$

Òàê êàê ñëîæíîñòü ïåðèîäè÷åñêîãî ñëîâà --- êîíñòàíòà, à ó ñëîâà Øòóðìà îíà ðàâíà $n+1$, òî
$$
\dim W_n^{(d+3)}(A) \le n^{(d+3)^2}
$$
äëÿ âñåõ äîñòàòî÷íî áîëüøèõ $n$. Ïîýòîìó
$$
\dim W_n^{(d+3)}(A^\#) \le (n+1)^{2(d+3)^2}
$$
ïî ëåììå  \ref{LL1}. Èç çàìå÷àíèÿ \ref{r1} âûòåêàåò, ÷òî
$$
\chi_n(A^\#)=\sum_{{\lambda\vdash n\atop h(\lambda)\le d+3}} m_\lambda\chi_\lambda,
$$
à $m_\lambda\le \dim W_n^{(d+3)}(A^\#) \le (n+1)^{2(d+3)^2}$. È ïîñêîëüêó ÷èñëî ðàçáèåíèé
$\lambda\vdash n$ ñ $h(\lambda) \le d+3$ íå ïðåâîñõîäèò $(n+1)^{d+3}$, òî
$$
l_n(A^\#) \le (n+1)^{3(d+3)^2}.
$$ 
 \end{proof}

Ëåììà \ref{LL2} ïîòðåáóåòñÿ íàì äëÿ âåðõíåé îöåíêè PI-ýêñïîíåíòû àëãåáðû $A(m,d,w)^\#$.
Íî ñíà÷àëà ìû îöåíèì ðîñò åå êîðàçìåðíîñòåé ñíèçó.

\begin{lemma}\label{LL3}
Ïóñòü $A=A(m,d,w)$ çàäàíà ïàðàìåòðàìè $m\ge 2, d\le m-1$ è $w$. Òîãäà
$$
\underline{exp}(A^\#) \ge exp(A)+1.
$$
\end{lemma}

\begin{proof}
Ïðè äîêàçàòåëüñòâå ëåììû \ref{A2} äëÿ ëþáîãî $\delta >0$ áûëà âûáðàíà âîçðàñòàþùàÿ ïîñëåäîâàòåëüíîñòü $n=n(t), t=t_0,t_0+1,\ldots$, ñåìåéñòâî ðàçáèåíèé 
$\lambda^{(n)}\vdash n(t)$ è íàáîð ïîëèíîìîâ $f_t, t\ge t_0$, ñî ñëåäóþùèìè ñâîéñòâàìè:
\begin{itemize}
\item
ðàçáèåíèå $\lambda$ èìååò âèä $\lambda=(\lambda_1,\ldots,\lambda_{d+2})$, 
$\lambda_1=\ldots= \lambda_d=m_1+\cdots+m_t-t, \lambda_{d+1}=td, \lambda_{d+2}=1$,
\item
$\Phi(\lambda^{(n)})\ge \Phi_d(\frac{1}{m+\alpha}-\delta)$, ãäå $\alpha$ --- íàêëîí $w$,
\item
$n(t+1)-n(t) \le d(m+1)$ äëÿ âñåõ $t\ge t_0$,
\item
ñèììåòðèçàöèÿ $f_t$ íå ÿâëÿåòñÿ òîæäåñòâîì $A$  è ïîðîæäàåò íåïðèâîäèìûé $F[S_n]$-ìîäóëü
ñ õàðàêòåðîì $\chi_\lambda$,
\item
$f_t$ êîñîñèììåòðè÷åí ïî $\lambda_1$ íàáîðàì ïåðåìåííûõ: îäèí ðàçìåðà $d+2$, $td-1$ --- ðàçìåðà
$d+1$ è $\lambda_1-\lambda_{d+1}$ --- ðàçìåðà $d$.
\end{itemize}
Êðîìå òîãî, $exp(A)=\Phi_d(\frac{1}{m+\alpha})$.

Îáîçíà÷èì ÷åðåç $\widetilde h_{t,k}$ ïðîèçâåäåíèå
$$
\widetilde h_{t,k}=f_tz_1\ldots z_k, \, k \ge 1.
$$
Ðàññìîòðèì òó æå ïîäñòàíîâêó $\varphi$, êîòîðàÿ äàâàëà íåíóëåâîå çíà÷åíèå äëÿ $f_t$
è $Sym\, f_t$ è ðàñøèðèì åå äåéñòâèå íà $\widetilde h_{t,k}$, ïîëîæèâ
$\varphi(z_1)=\ldots=\varphi(z_k)=1$. Òîãäà, î÷åâèäíî,
$$
\varphi(\widetilde h_{t,k})=\varphi(f_t)\ne 0.
$$
Áîëåå òîãî, åñëè $k\le td$, òî ìû ìîæåì âêëþ÷èòü $z_1,\ldots,z_k$ â ïåðâûå $k$ êîñîñèììåòðè÷íûõ
íàáîðà ó $f_t$ è ïðîâåñòè äîïîëíèòåëüíîå àëüòåðíèðîâàíèå ïî ðàñøèðåííûì íàáîðàì. Ïðè ýòîì èç 
ïðàâèë óìíîæåíèÿ áàçèñíûõ ýëåìåíòîâ $A$ ñëåäóåò, ÷òî
$$
\varphi(Alt(\widetilde h_{t,k}))=\gamma\varphi(\widetilde h_{t,k}),
$$
ãäå $\gamma$ --- íåíóëåâîé öåëî÷èñëåííûé êîýôôèöèåíò. Ó ïîëèíîìà $f_{t,k}=
Alt(\widetilde h_{t,k})$ ïåðåìåíííûå òîæå ðàñïðåäåëåíû ïî $\lambda_1$ êîñîñèììåòðè÷íûì
íàáîðàì: îäèí ðàçìåðà $d+3$, $k-1$ --- ðàçìåðà $d+2$, $td-k$ --- ðàçìåðà ðàçìåðà $d+1$
è  $\lambda_1-td$ --- ðàçìåðà $d$.
Áîëåå òîãî, åñëè ïðîâåñòè åãî ñèììåòðèçàöèþ ïî òåì æå ïåðåìåííûì, ÷òî è äëÿ $f_t$ ïëþñ
ñèììåòðèçàöèþ ïî $z_1,\ldots,z_k$, òî çíà÷åíèå $\varphi(Sym\,(f_{t,k}))$ òîæå ïðîïîðöèîíàëüíî
$\varphi(f_{t})$  ñ íåíóëåâûì êîýôôèöèåíòîì. Òî åñòü ïîëèíîì $Sym\,(f_{t,k})$ ïîðîæäàåò
íåïðèâîäèìûé $F[S_{n+k}]$-ìîäóëü ñ õàðàêòåðîì $\chi_\mu$, ãäå
$$
\mu = (\mu _1,\ldots,\mu_{d+3}), \mu_1=\lambda_1,\ldots, \mu_d=\lambda_d,
\mu_{d+1}=\lambda_{d+1}, \mu_{d+2}=k, \mu_{d+3}=1.
$$
Àíàëîãè÷íî äîêàçûâàåòñÿ, ÷òî âñå ðàçáèåíèÿ âèäà
$$
\mu=(\lambda_1,\ldots,\lambda_d,k,\lambda_{d+1},1),\,
\mu=(k,\lambda_1,\ldots,\lambda_{d+2})
$$
èìåþò íåíóëåâûå êðàòíîñòè â õàðàêòåðå $\chi_{n+k}(A^\#)$. Äðóãèìè ñëîâàìè, ìû ìîæåì äîáàâèòü 
ê äèàãðàììå $D_\lambda$ ëþáóþ ñòðîêó (1-þ, $d+1$-þ ëèáî $d+2$-þ) è ïîëó÷èòü äèàãðàììó 
$D_\mu$, ñîîòâåòñòâóþùóþ ðàçáèåíèþ $\mu\vdash n+k$ ñ íåíóëåâîé êðàòíîñòüþ.

Îöåíèì ñíèçó ìàêñèìàëüíîå çíà÷åíèå $\Phi(\mu)$ è ñîîòâåòñòâóþùåå ýòîìó ìàêñèìàëüíîìó çíà÷åíèþ $k$. Îáîçíà÷èì $\frac{\lambda_1}{n}=u_1,
\ldots,\frac{\lambda_{d+2}}{n}=u_{d+2}, \beta=\Phi(\lambda)$. Òîãäà ïî ëåììå \ref{L3}
\begin{equation}\label{n0}
\Phi\left(\theta u_1,\ldots,\theta u_{d+2},1-\theta\right)=1+\Phi(\lambda)
\end{equation}
--- ìàêñèìàëüíîå çíà÷åíèå, êîòîðîå ìîæåò ïðèíèìàòü $\Phi(\mu)$, ãäå $\theta=
\frac{\beta}{\beta+1}$. Ýòî îçíà÷àåò, ÷òî åñëè $k$ óäîâëåòâîðÿåò äâóì íåðàâåíñòâàì
\begin{equation}\label{n1}
\frac{k}{n+k}\le 1-\theta=\frac{1}{\beta+1}\le\frac{k+1}{n+k+1},
\end{equation}
òî ìàêñèìóì $\Phi(\mu)$ äîñòèãàåòñÿ ëèáî ïðè ýòîì $k$, ëèáî ïðè $k+1$. Ñîîòíîøåíèå (\ref{n1})
ðàâíîñèëüíî äâîéíîìó íåðàâåíñòâó
\begin{equation}\label{n2}
\frac{n}{\beta}-1\le k \le \frac{n}{\beta}.
\end{equation}

Íàïîìíèì, ÷òî $n$ è $k$ çàâèñÿò îò $t: n=n(t), k=k(t)$. Ó÷èòûâàÿ (\ref{n2}) è âûáîð $n(t)$, ìû ïîëó÷àåì
\begin{equation}\label{n3}
n(t+1)+k(t+1)-n(t)-k(t) \le\frac{\beta+1}{\beta} d(m+1).
\end{equation}

Îáîçíà÷èì $r=r(t)=n(t)+k(t)$, à ÷åðåç $\mu^{(r)}$ --- ðàçáèåíèå $r(t)$ ñ ìàêñèìàëüíûì
çíà÷åíèåì $\Phi(\mu^{(r)})$. Òàê êàê ñ ðîñòîì $n$ âåëè÷èíó $\frac{1}{\beta+1}$ âñå áîëåå
òî÷íî àïïðîêñèìèðóåòñÿ äðîáüþ âèäà $\frac{k}{n+k}$, òî ìîæíî ñ ó÷åòîì (\ref{n1}) ñ÷èòàòü, ÷òî
$$
\Phi(\mu^{(r)}) \ge \Phi(\lambda^{(n)})+1-\delta'
$$
ïðè âñåõ äîñòàòî÷íî áîëüøèõ $n$, ãäå $\delta' >0$ --- ëþáàÿ çàðàíåå çàäàííàÿ âåëè÷èíà,
$n=n(t), r=r(t)$. Òîãäà ñ ó÷åòîì ëåììû \ref{L1} ìû èìååì
\begin{equation}\label{n4}
c_{r(t)}(A^\#) \ge \frac{\Phi\left(\mu^{(r(t))} \right)^n }{n^{(d+2)^2+d+3}} 
\ge  \frac{\left(\Phi(\lambda^{(n)})+1-\delta' \right)^n}{n^{(d+2)^2+d+3}} \ge 
\frac{\left(\Phi_d(\frac{1}{m+\alpha}-\delta)+1-\delta' \right)^n}{n^{(d+2)^2+d+3}}.
\end{equation}

Ïîñêîëüêó âñå ðàçíîñòè $r(t+1)-r(t)$ îãðàíè÷åíû îáùåé êîíñòàíòîé (ñì. (\ref{n3})), à
ïîñëåäîâàòåëüíîñòü $\{c_n(A^\#)\}$ --- íåóáûâàþùàÿ, òî èç (\ref{n4}) ñëåäóåò, ÷òî
$$
\underline{\lim}_{n\to\infty}\sqrt[n]{c_n(A^\#)} \ge 
\Phi_d(\frac{1}{m+\alpha}-\delta)+1-\delta'.
$$
Íàêîíåö, òàê êàê $\delta$ è $\delta'$ --- ïðîèçâîëüíûå ñêîëü óãîäíî ìàëûå âåëè÷èíû, ìû ïîëó÷àåì
$$
\underline{exp}(A^\#) \ge exp(A)+1,
$$
è ëåììà äîêàçàíà.

\end{proof}

\subsection{}\label{s4.2}
Òåïåðü ìû ïîëó÷èì âåðõíþþ îöåíêó íà $\overline{exp}(A)^\#$.

\begin{lemma}\label{LL4}
$$
\overline{exp}(A^\#) \le exp(A)+1.
$$
\end{lemma}
 \begin{proof}

Òàê êàê êîäëèíà $l_n(A^\#)$ ïîëèíîìèàëüíî îãðàíè÷åíà ñîãëàñíî ëåììå \ref{LL2}, òî
äîñòàòî÷íî äîêàçàòü, ÷òî
$$
\Phi(\lambda) \le \Phi_d(\frac{1}{m+\alpha})+1=exp(A)+1,
$$
äëÿ ëþáîãî $\lambda\vdash n$ ñ $m_\lambda\ne 0$ â $\chi_n(A)$, êàê ïîêàçûâàåò ñîîòíîøåíèå
(\ref{e2a}).

Ïóñòü $h=h(x_1,\ldots,x_n)$ --- ïîëèëèíåéíûé ìíîãî÷ëåí, íå ÿâëÿþùèéñÿ òîæäåñòâîì $A^\#$,
ïîðîæäàþùèé â $P_n$ íåïðèâîäèìûé $F[S_n]$-ìîäóëü ñ õàðàêòåðîì $\chi_\lambda$. Êàê îòìå÷àëîñü
ðàíåå, ìîæíî ñ÷èòàòü $h$ êîñîñèììåòðè÷íûì ïî $\lambda_1$ íàáîðàì ïåðåìåííûõ, ïðè÷åì
$\lambda_{d+2}$ èç íèõ èìåþò ðàçìåð íå ìåíüøå $\lambda_{d+2}$. Åñëè $\lambda_{d+2}=0$, òî
$$
\Phi(\lambda)\le \Phi(\underbrace{\frac{1}{d+1},\ldots,\frac{1}{d+1}}_{d+1},0,0)
= d+1 <1+\Phi_d(\frac{1}{m+\alpha}).
$$

Ïóñòü $\lambda_{d+2}\ne 0$. Çàôèêñèðóåì ïðîèçâîëüíîå $\varepsilon >0$. Ïîñêîëüêó
$h\not\in Id(A^\#)$, òî ñóùåñòâóåò ïîäñòàíîâêà $\varphi$ áàçèñíûõ ýëåìåíòîâ $A$ è 1 âìåñòî
ïåðåìåííûõ $x_1,\ldots,x_n$, ïðè êîòîðîé
$$
\varphi(h)=f(z^i_{jk},a_1,\ldots, a_d,b)=f
$$
--- íåíóëåâîé îäíî÷ëåí ñòåïåíè $n'$ îò $\{z^i_{jk},a_1,\ldots, a_d,b \} $, ãäå $n'=n-n_1$, à
$n_1$ --- êîëè÷åñòâî åäèíèö èç $A^\#$, ïîäñòàâëåííûõ âìåñòî $x_1,\ldots,x_n$. 

Çàìåòèì, ÷òî $\lambda_{d+3}$ ìîæåò ïðèíèìàòü òîëüêî äâà çíà÷åíèÿ --- $0$ èëè $1$. Ïóñòü
ñíà÷àëà $\lambda_{d+3}=1$.  ýòîì ñëó÷àå â $h$ åñòü $r=\lambda_{d+2}$ êîñîñèììåòðè÷íûõ
íàáîðà ïåðåìåííûõ ðàçìåðà íå ìåíüøå $d+2$ è $\lambda_{d+1}-\lambda_{d+2}$ êîñîñèììåòðè÷íûõ
íàáîðà ïåðåìåííûõ ðàçìåðà  $d+1$. Â êàæäûé èç ýòèõ ïîñëåäíèõ íàáîðîâ ïîäñòàâëåí îäèí
èç ýëåìåíòîâ $\{1,b\}$. Ïóñòü $b$ ïîäñòàâëåí ðîâíî â $k$ íàáîðîâ, è
$\lambda_{d+1}-\lambda_{d+2}=k+t$. Òîãäà $n_1\ge r+t$ è
$$
n''=n-r-t\ge n-n_1=n'=\deg_b f \ge r+k.
$$
Åñëè $\lambda_{d+1}>\lambda_{d+2}$, òî ïåðåáðàñûâàÿ êëåòêè èç $(d+1)$-é ñòðîêè äèàãðàììû
$D_\lambda$ â $(d+2)$-þ, ìîæíî ïîëó÷èòü ðàçáèåíèå $\lambda'\vdash n$, ó êîòîðîãî ëèáî
$(d+2)$-ÿ ñòðîêà $D_{\lambda'}$ èìååò äëèíó $r+k$ (åñëè $k\le t$), ëèáî $(d+1)$-ÿ èìååò
äëèíó $r+k$ (åñëè $k> t$). Âû÷åðêíóâ ýòó ñòðîêó, ìû ïîëó÷àåì ðàçáèåíèå $\mu\vdash n''$,
ó êîòîðîãî $\mu_{d+1}=r+k$. Òîãäà
\begin{equation}\label{n5}
\frac{\mu_{d+1}}{n''} 
\le \frac{\deg_b f}{n'} \le \frac{1}{m+\alpha}+\varepsilon_{n'}
\end{equation}
â ñèëó ëåììû \ref{A0}, ãäå $n'=n-n_1$, à $n_1$ ---- êîëè÷åñòâî åäèíèö, ïîäñòàâëåííûõ â
$h$ âìåñòî $x_1,\ldots, x_n$.

Òåïåðü ìû ïîëó÷èì íåðàâåíñòâî àíàëîãè÷íîå (\ref{n5}) ïðè $\lambda_{d+3}=0$.  ýòîì ñëó÷àå
$h(x_1,\ldots, x_n)$ çàâèñèò îò $r=\lambda_{d+2}$ êîñîñèììåòðè÷íûõ íàáîðîâ ðàçìåðà
$d+2$. Åñëè â êàæäûé èç íèõ áûëè ïîäñòàâëåíû îáà ýëåìåíòà $1\in A^\#, b\in A$, òî òå
æå ðàññóæäåíèÿ, ÷òî è âûøå, äàþò íàì ñîîòíîøåíèå (\ref{n5}).  ïðîòèâíîì ñëó÷àå ìû ëèáî ïîäñòàâëÿåì â $r-1$ èç íèõ åäèíèöó è âî âñå $r$ ýëåìåíò $b$, ëèáî íàîáîðîò ---
â $r$ åäèíèöó è â $r-1$ --- áàçèñíûé ýëåìåíò $b$. Ïåðåáðàñûâàÿ, åñëè íåîáõîäèìî, êëåòêè
èç $(d+1)$-é ñòðîêè $D_\lambda$ â $(d+2)$-þ (êàê è ïðè $\lambda_{d+3}=1$) è âû÷åðêèâàÿ ñòðîêó
äëèíû $r+t$ ($k$ è $t$ îïðåäåëÿþòñÿ òàêæå êàê â ñëó÷àå $\lambda_{d+3}=1$), ìû ïîëó÷àåì ðàçáèåíèå
$\mu\vdash n''=n-r-t$ ñ $\mu_{d+1}=r+k$. Ïðè ýòîì â ïåðâîì èç ñëó÷àåâ ìû ïîëó÷àåì
íåðàâåíñòâà
$$
\mu_{d+1}\le\deg_b f\,;\quad n''\ge n-n_1+1\ge n-n_1=n',
$$
à âî âòîðîì ñëó÷àå --- íåðàâåíñòâà
$$
\mu_{d+1}\le\deg_b f+1\,;\quad n''\ge n-n_1=n'.
$$
Î÷åâèäíî, ÷òî â ïåðâîì ñëó÷àå ðàçáèåíèå $\mu$ óäîâëåòâîðÿåò óñëîâèþ (\ref{n5}), à âî
âòîðîì --- óñëîâèþ
\begin{equation}\label{n6}
\frac{\mu_{d+1}}{n''} \le \frac{\deg_b f}{n'} \le \frac{1}{m+\alpha}+\varepsilon_{n'}
+\frac{1}{n'}.
\end{equation}
Ïîñêîëüêó (\ref{n5}) --- áîëåå ñèëüíîå îãðàíè÷åíèå, ÷åì (\ref{n6}), ìîæíî ñ÷èòàòü, ÷òî
$\mu=(\mu_1,\ldots,\mu_{d+1})\vdash n''$ âñåãäà óäîâëåòâîðÿåò íåðàâåíñòâó (\ref{n6}), â
êîòîðîì $n''\le n$ è $n''\to\infty$ ïðè $n\to\infty$, à $n'=n-n_1$, ãäå $n_1$ --- 
êîëè÷åñòâî åäèíèö, ïîäñòàâëåííûõ â $h(x_1,\ldots,x_n)$, ÷òîáû ïîëó÷èòü íåíóëåâîå çíà÷åíèå.

Çàìåòèì ñíà÷àëà, ÷òî $\lambda_1 \ge n_1$ â ñèëó êîñîñèììåòðè÷íîñòè $h$ ïî $\lambda_1$
íàáîðàì ïåðåìåííûõ. Îáîçíà÷èì $x=\frac{\lambda_1}{n}$. Òîãäà
$$
\Phi(\lambda) \le 
\Phi\left(x,\underbrace{\frac{1-x}{d+2},\ldots,\frac{1-x}{d+2} }_{d+2}\right)=H(x)
$$
Ïðåäåë ôóíêöèè $H(x)$ ïðè $x\to 1$ ðàâåí $1$. Ýòî, â ÷àñòíîñòè, îçíà÷àåò, ÷òî
ñóùåñòâóåò òàêîå öåëîå $q$, ÷òî åñëè $\lambda_1\ge\frac{q-1}{q}n$, òî $\Phi(\lambda) <d$ äëÿ âñåõ äîñòàòî÷íî áîëüøèõ $n\ge N$.

Ðàçäåëèì òåïåðü âñå ðàçáèåíèÿ $\lambda\vdash n\ge N$ íà äâå ãðóïïû --- ãäå 
$\lambda_1 >\frac{q-1}{q}n$ è ãäå $\lambda_1 \le\frac{q-1}{q}n$. Äëÿ âñåõ ðàçáèåíèé
ïåðâîé ãðóïïû íåðàâåíñòâî
$$
\Phi(\lambda) < d < \Phi_d(\frac{1}{m+\alpha}+\varepsilon)
$$
âûïîëíÿåòñÿ â ñèëó âûáîðà $q$ è $n$. Äëÿ ðàçáèåíèé èç âòîðîé ãðóïïû âîñïîëüçóåìñÿ ñîîòíîøåíèåì
(\ref{n6}). Äèàãðàììà $D_\mu$ ïîëó÷åíà èç $D_{\lambda'}$ âû÷åðêèâàíèåì ñòðîêè, à $D_{\lambda'}$ ïîëó÷åíà èç $D_{\lambda}$ ïåðåíîñîì âíèç
íåñêîëüêèõ êëåòîê. Ïîýòîìó ïî ëåììàì \ref{L2} è \ref{L3}
$\Phi(\lambda) \le \Phi(\lambda') \le \Phi(\mu)+1$. Òîãäà èç (\ref{n6}) ñëåäóåò, ÷òî
$$
\Phi(\lambda) \le  \Phi(\mu)+1 \le 
\Phi(\theta,\ldots,\theta,\frac{1}{m+\alpha}+\varepsilon_{n'}+\frac{1}{n'},\frac{1}{n''}),
$$
èñïîëüçóÿ ñâîéñòâà $\Phi$, ãäå
$$
d\theta+\frac{1}{m+\alpha}+\varepsilon_{n'}+\frac{1}{n'}+\frac{1}{n''}=1.
$$

Òàê êàê $\lambda_1 \le\frac{q-1}{q}n$, òî $n'=n-n_1\ge n-\lambda_1\ge \frac{n}{q}$.
Ïîýòîìó $n'\to\infty$ òàêæå êàê è $n''$ ñ ðîñòîì $n$ è $\varepsilon_{n'}\to 0$. 
Êàê è â äîêàçàòåëüñòâå ëåììû \ref{A0}, ïîëó÷àåì, ÷òî
$$
\Phi(\lambda) \le 1+\Phi(\theta',\ldots,\theta',\frac{1}{m+\alpha}+\varepsilon,0)=
1+\Phi_d(\frac{1}{m+\alpha}+\varepsilon)
$$
äëÿ âñåõ äîñòàòî÷íî áîëüøèõ $n$. Ïîñêîëüêó $\varepsilon >0$ âûáðàíî ïðîèçâîëüíî, ìû ïîëó÷àåì
$$
\overline{exp}(A^\#) \le 1+\Phi_d(\frac{1}{m+\alpha})= exp(A)+1.
$$

 \end{proof}

Êîìáèíàöèÿ ëåìì \ref{LL3} è \ref{LL4} ñðàçó äàåò ñëåäóþùèé ðåçóëüòàò.

\begin{theorem}\label{t2}
Ïóñòü $m$ è $d$ --- öåëûå ÷èñëà, $m\ge 2, m-1\ge d$, à $w$ --- áåñêîíå÷íîå ïåðèîäè÷åñêîå ñëîâî
èëè ñëîâî Øòóðìà. Åñëè $A=A(m,d,w)$ è $A^\#$ ïîëó÷åíà èç $A$ ïðèñîåäèíåíèåì  åäèíèöû, òî
$exp(A^\#)$ ñóùåñòâóåò, ïðè÷åì $exp(A^\#)=exp(A)+1$
\end{theorem}

\begin{corollary}\label{c1}
Äëÿ ëþáîãî âåùåñòâåííîãî ÷èñëà $\gamma\ge 2$ ñóùåñòâóåò 
(â îáùåì ñëó÷àå íåàññîöèàòèâíàÿ) àëãåáðà  $A_\gamma$ ñ
åäèíèöåé ñ PI-ýêñïîíåíòîé $exp(A_\gamma)=\gamma$.
\end{corollary}

\begin{proof}
Ïðè çàäàííîì $d$ ñîâîêóïíîñòü çíà÷åíèé
$$
\left\lbrace \Phi_d(\frac{1}{m+\alpha})=exp(A(m,d,w)) \vert 0\le\alpha\le 1,
\quad m=d+1, d+2,\ldots\right\rbrace
$$
ïîêðûâàåò âåñü ïðîìåæóòîê $(d,d+1]$. Ñëåäîâàòåëüíî, ëþáîå âåùåñòâåííîå ÷èñëî $\gamma>2$
ðåàëèçóåòñÿ êàê ýêñïîíåíòà $exp(A^\#)$, ãäå $A=A(m,d,w)$ äëÿ ïîäõîäÿùèõ $m,d$ è $w$.
Äëÿ $\gamma=2$ åñòü ìíîãî ðåàëèçàöèé äàæå â àññîöèàòèâíîì ñëó÷àå. Íàïðèìåð, äëÿ áåñêîíå÷íîìåðíîé àëãåáðû Ãðàññìàíà $G$ ñ åäèíèöåé $c_n(G)=2^{n-1}$ (\cite{KR} èëè
 \cite[òåîðåìà 4.1.8]{GZbook}). Ïîýòîìó $exp(G)=2$.

\end{proof}

Îòäåëüíûé èíòåðåñ ïðåäñòàâëÿåò âîïðîñ î ìíîæåñòâå çíà÷åíèé PI-ýêñïîíåò êîíå÷íîìåðíûõ
àëãåáð. ßñíî, ÷òî åñëè ïîëå $F$ ñ÷åòíî, òî è ýòî ìíîæåñòâî ñ÷åòíî.  ðàáîòå \cite{GMZ}
ïîêàçàíî, ÷òî ìíîæåñòâî $\{exp(A)|\dim A <\infty\}$ âñþäó ïëîòíî â $[1;\infty)$, à â ðàáîòå
\cite{Z} äîêàçàíî, ÷òî äëÿ êîíå÷íîìåðíîé óíèòàðíîé àëãåáðû $A$ ðîñò $\{c_n(A)\}$ ëèáî
ïîëèíîìèàëåí, ëèáî îãðàíè÷åí ñíèçó ïîêàçàòåëüíîé ôóíêöèåé $2^n$.

Åùå îäíèì ñëåäñòâèåì òåîðåì \ref{t1} è \ref{t2} ÿâëÿåòñÿ òîò ôàêò, ÷òî ñîâîêóïíîñòü 
PI-ýêñïîíåíò êîíå÷íîìåðíûõ àëãåáð ñ åäèíèöåé ÿâëÿåòñÿ âñþäó ïëîòíûì ïîäìíîæåñòâîì â îáëàñòè
 $[2;\infty)\subset \mathbb{R}$.
 
\begin{corollary}\label{c2}
Äëÿ ëþáûõ âåùåñòâåííûõ $2\le\alpha <\beta$ ñóùåñòâóåò êîíå÷íîìåðíàÿ
(â îáùåì ñëó÷àå íåàññîöèàòèâíàÿ)  àëãåáðà $B$ ñ åäèíèöåé, òàêàÿ, ÷òî
$$
\alpha \le exp(B) \le \beta.
$$
\end{corollary}

\begin{proof}
Ðàññìîòðèì àëãåáðó $A(m,d,w)$, ãäå $w$ --- áåñêîíå÷íîå ïåðèîäè÷åñêîå ñëîâî ñ ïåðèîäîì $T$,
è âìåñòå ñ íåé --- êîíå÷íîìåðíóþ àëãåáðó $B=B(m,d,w)$  ñ áàçèñîì
$$
\left\lbrace a_1,\ldots,a_d,b, z^i_{jk}\vert
1\le i \le d,\,  1\le j \le m+w_j,\, 1\le k \le T \right\rbrace
$$
è òàáëèöåé óìíîæåíèÿ
$$
z^i_{jk} a_i=
\left\{
  \begin{array}{rcl}
     z^i_{j+1,k}, &\quad \hbox{åñëè} \quad & j<m+w_k  \\
    0, &\quad \hbox{åñëè} \quad & j=m+w_k\, ,
           \end{array}
\right.
$$ 
$$
z^i_{m+w_k,k} b=
\left\{
  \begin{array}{rcl}
     z^{i+1}_{1k}, &\quad \hbox{åñëè} \quad & i<d  \\
    z_{1,k+1}^1, &\quad \hbox{åñëè} \quad & i=d,\, k<T \\
    z_{11}^1, &\quad \hbox{åñëè} \quad & i=d,\, k=T\, .
           \end{array}
\right.
$$ 
Ëåãêî çàìåòèòü, ÷òî àëãåáðû $A(m,d,w)$ è $B(m,d,w)$ PI-ýêâèâàëåíòíû, ò.å. èìåþò îäíè è òå 
æå òîæäåñòâà. Íî òîãäà è àëãåáðû $A(m,d,w)^\#$ è $B(m,d,w)^\#$ òîæå PI-ýêâèâàëåíòíû. Ïîýòîìó
$exp(A(m,d,w)^\#)=exp(B(m,d,w)^\#)$.  ÷àñòíîñòè, $exp(B(m,d,w)^\#) = exp(A(m,d,w))+1$.

Ïî ïðåäëîæåíèþ \ref{p1} äëÿ ëþáîãî ðàöèîíàëüíîãî $q\in(0;1)$ ñóùåñòâóåò ïåðèîäè÷åñêîå ñëîâî 
$w$ ñ íàêëîíîì $\pi(w)=q$. Íî òîãäà
$$
exp(B(m,d,w)^\#)=\Phi_d(\frac{1}{m+q})+1
$$
â ñèëó òåîðåìû \ref{t2}. Ïîýòîìó ìîæíî ïîäîáðàòü òàêîå ðàöèîíàëüíîå ïîëîæèòåëüíîå $q < 1$,
÷òî $\alpha \le exp(B(m,d,w)^\#) \le \beta$.

\end{proof}

\end{fulltext}

\begin{thebibliography}{99}

\RBibitem{GZ1}
\by À.~Äæàìáðóíî, Ì.\,Â.~Çàéöåâ
\paper Àñèìïòîòè÷åñêîå âîçðàñòàíèå ïîñëåäîâàòåëüíîñòåé êîðàçìåðíîñòåé òîæäåñòâ àññîöèàòèâíûõ àëãåáð
\jour  Âåñòíèê Ìîñê. óí-òà. Ñåð. 1, Ìàòåìàòèêà. Ìåõàíèêà.
\issue 3
\pages 54--56
\yr 2014

\RBibitem{GZ2}
\by A.~Giambruno, M.~Zaicev
\paper Growth of polynomial identities: is the sequence of codimensions eventually 
non-decreasing?
\jour Bull. London Math. Soc.
\vol 46
\issue 4
\pages 771–-778
\yr 2014

\Bibitem{R}
\by A.~Regev
\paper Codimensions and trace codimensions of matrices are asymptotically equal
\jour Israel J. Math.
\vol 47
\issue 2-3
\pages 246--250
\yr 1984

\RBibitem{Z}
\by A.~Giambruno, A.~Regev,  M.~Zaicev
\paper Simple and semisimple Lie algebras and codimension growth
\jour Trans. Amer. Math. Soc.
\vol 352
\issue 4
\pages 1935–-1946
\yr 2000

\Bibitem{GSZ}
\by A.~Giambruno, I.~Shestakov, M.~Zaicev
\paper Finite dimensional nonassociative algebras and codimension growth
\jour Adv. Appl. Math.
\vol 47
\pages 125--139
\yr 2011

\RBibitem{Rzm}
\by Þ.\,Ï.~Ðàçìûñëîâ
\paper Ïðîñòûå àëãåáðû Ëè â ìíîãîîáðàçèÿõ, ïîðîæäåííûõ àëãåáðàìè Ëè êàðòàíîâñêîãî òèïà
\jour Èçâ. ÀÍ ÑÑÑÐ. Ñåð. ìàòåì.
\vol 51
\issue 6
\pages 1228–-1264
\yr 1987

\Bibitem{Sp}
\by W.~Specht
\paper Gesetze in Ringen. I. (German) 
\jour Math. Z. 
\vol 52
\issue 2
\pages 557-–589
\yr 1950

\RBibitem{K}
\by À.\,Ð.~Êåìåð
\paper Îá àñèìïòîòè÷åñêîì áàçèñå òîæäåñòâ àëãåáð ñ åäèíèöåé èç ìíîãîîáðàçèÿ $Var(M_2(F))$
\jour Èçâ. âóçîâ. Ìàòåì.
\issue 6
\pages 71–-76
\yr 1989

\Bibitem{D}
\by V.~Drensky
\paper  Relations for the cocharacter sequences of T-ideals. 
Proceedings of the International Conference on Algebra, Part 2 
 (Novosibirsk, 1989)
\jour , Contemp. Math.
\vol 131
\issue Part 2
\pages 285–300
\yr 1992

 
 
 

\Bibitem{DR}
\by V.~Drensky, A.~Regev
\paper Exact asymptotic behaviour of the codimensions of some P.I. algebras
\jour Israel J. Math.
\vol 96
\pages 231–-242
\yr 1996

\Bibitem{BR}
\by A.~Berele, A.~Regev
\paper Asymptotic behaviour of codimensions of p. i. algebras satisfying Capelli identities
\jour Trans. Amer. Math. Soc.
\vol 360
\issue 10
\pages 5155–-5172
\yr 2008

\Bibitem{Be}
\by A.~Berele
\paper Properties of hook Schur functions with applications to p.i. algebras
\jour Adv. in Appl. Math.
\vol 41
\issue 1
\pages 52–-75
\yr 2008

\Bibitem{GMZ}
\by A.~Giambruno, S.~Mishchenko, M.~Zaicev
\paper Codimensions of Algebras and Growth Functions
\jour Adv. Math. 
\vol 217
\issue 3
\pages 1027--1052
\yr 2008

\RBibitem{Z1}
\by Ì.\,Â.~Çàéöåâ
\paper Òîæäåñòâà êîíå÷íîìåðíûõ óíèòàðíûõ àëãåáð
\jour Àëãåáðà è ëîãèêà
\vol 50
\issue 5
\pages 563--594.
\yr 2011

\Bibitem{GZ3}
\by A.~Giambruno, M.~Zaicev
\paper Proper identities, Lie identities and exponential codimension growth
\jour J. Algebra
\vol 320
\issue 5
\pages 1933--1962
\yr 2008

\Bibitem{GZ4}
\by A.~Giambruno, M.~Zaicev 
\paper On codimension growth of finitely generated associative algebras
\jour Adv. Math. 
\vol 140
\pages 145--155
\yr 1998

\Bibitem{GZ5}
\by A.~Giambruno, M.~Zaicev 
\paper Exponential codimension growth of P.I. algebras: an exact estimate
\jour Adv. Math. 
\vol 142
\pages 221--243
\yr 1999

\RBibitem{BBZ}
\by Î.\,Å.~Áåçóùàê, À.\,À.~Áåëÿåâ, Ì.\,Â.~Çàéöåâ
\paper Êîðàçìåðíîñòè òîæäåñòâ àëãåáð ñ ïðèñîåäèí¸ííîé åäèíèöåé. 
\jour Ôóíäàì. è ïðèêë. ìàòåì.
\vol 18
\issue 3
\pages 11--26
\yr 2013

\Bibitem{RZ}
\by D.~Repov\v s, M.~Zaicev
\paper Numerical invariants of identities of unital algebras
\jour Comm. Alg.
\vol 43
\issue 9 
\pages 3823--3839
\yr 2015

\RBibitem{Ra1}
\by Ñ.\,Ì.~Ðàöååâ 
\paper Àëãåáðû Ïóàññîíà ïîëèíîìèàëüíîãî ðîñòà
\jour Ñèá. ìàò. æ.
\vol 54
\issue 3
\pages 700--711
\yr 2013

\RBibitem{Ra2}
\by Ñ.\,Ì.~Ðàöååâ 
\paper Âçàèìîñâÿçü àëãåáð Ïóàññîíà è àëãåáð Ëè íà ÿçûêå òîæäåñòâ
\jour Ìàòåì. çàìåòêè
\vol 96
\issue 4
\pages 567--577
\yr 2014

\RBibitem{B}
\by Þ.\,À.~Áàõòóðèí
\book Òîæäåñòâà â àëãåáðàõ Ëè 
\publaddr Ì.
\publ Íàóêà
\yr 1985

\Bibitem{Dren}
\by V.~Drensky
\book Free algebras and PI-algebras. Graduate course in algebra 
\publaddr Singapore
\publ Springer-Verlag
\yr 2000

\Bibitem{GZbook}
\by A.~Giambruno, M.~Zaicev
\book Polynomial identities and asymptotic methods
\bookinfo Mathematical Surveys and Monographs, 122
\publ American Mathematical Society
\publaddr Providence, RI
\yr 2005

\Bibitem{R1}
\by A.~Regev
\paper Existence of identities in $A\otimes B$
\jour Israel J. Math.
\vol 11
\pages 131-–152
\yr 1972

\Bibitem{BD}
\by Yu.~Bahturin, V.~Drensky
\paper Graded polynomial identities of matrices 
\jour Linear Algebra Appl. 
\vol 357
\pages 15–-34
\yr 2002

\RBibitem{Z2}
\by Ì.\,Â.~Çàéöåâ
\paper Ìíîãîîáðàçèÿ àôôèííûõ àëãåáð Êàöà-Ìóäè
\jour Ìàòåì. çàìåòêè
\vol 62
\issue 1
\pages 95--102
\yr 1997

\RBibitem{M}
\by Ñ.\,Ï.~Ìèùåíêî
\paper Ðîñò ìíîãîîáðàçèé àëãåáð Ëè
\jour ÓÌÍ
\vol 45
\issue 6
\pages 25--45
\yr 1990

\RBibitem{J}
\by Ã.~Äæåéìñ 
\book Òåîðèÿ ïðåäñòàâëåíèé ñèììåòðè÷åñêèõ ãðóïï   
\publaddr Ì.
\publ Íàóêà
\yr 1982

\Bibitem{GZ6}
\by A.~Giambruno,  M.~Zaicev
\paper On codimension growth of finite dimensional Lie superalgebras
\jour J. London Math. Soc.
\vol 95
\pages 534--548.
\yr 2012

\RBibitem{ZR}
\by Ì.\,~Çàéöåâ, Ä.~Ðåïîâø 
\paper ×åòûðåõìåðíàÿ ïðîñòàÿ àëãåáðà ñ äðîáíîé PI-ýêñïîíåíòîé
\jour Ìàòåì. çàìåòêè
\vol 95
\issue 4
\pages 538--553
\yr 2014

\Bibitem{L}
\by M.~Lothaire
\book Algebraic Combinatorics on Words 
\bookinfo Encyclopedia Math. Appl., vol. 90
\publaddr Cambridge
\publ Cambridge University Press
\yr 2002

\Bibitem{GMZ0}
\by A.~Giambruno, S.~Mishchenko, M.~Zaicev
\paper Algebras with intermediate growth of the codimensions
\jour Adv. in Appl. Math. 
\vol 37
\issue 3
\pages 360–-377
\yr 2006

\Bibitem{KR}
\by D.~Krakowski, A.~Regev
\paper The polynomial identities of the Grassmann algebra
\jour Trans. Amer. Math. Soc.
\vol 181
\pages 429--438
\yr 1973

\end{thebibliography}
\end{document}


\documentclass{amsart}[12pt]

\newtheorem{theorem}{Theorem}[section]
\newtheorem{lemma}[theorem]{Lemma}
\newtheorem{proposition}[theorem]{Proposition}
\newtheorem{corollary}[theorem]{Corollary}
\newtheorem{question}[theorem]{Question}
\newtheorem{problem}[theorem]{Problem}

\theoremstyle{definition}
\newtheorem{definition}[theorem]{Definition}
\newtheorem{example}[theorem]{Example}
\newtheorem{examples}[theorem]{Examples}
\newtheorem{remark}[theorem]{Remark}
\newtheorem{remarks}[theorem]{Remarks}
\newtheorem{algorithm}[theorem]{Algorithm}





\usepackage{amscd,amssymb}

\begin{document}

\title[Varieties of bicommutative algebras]
{Varieties of bicommutative algebras}
\author{Vesselin Drensky}
\date{}
\address{Institute of Mathematics and Informatics,
Bulgarian Academy of Sciences,
Acad. G. Bonchev Str., Block 8,
1113 Sofia, Bulgaria}
\email{drensky@math.bas.bg}
\subjclass[2010]
{17A30, 17A50, 20C30.}
\keywords{Free bicommutative algebras, varieties of bicommutative algebras, codimension sequence, codimension growth, two-dimensional algebras.}
\thanks
{Partially supported by Grant I02/18
``Computational and Combinatorial Methods
in Algebra and Applications''
of the Bulgarian National Science Fund.}
\maketitle

\begin{abstract}
Bicommutative algebras are nonassociative algebras satisfying the polynomial identities
of right- and left-commutativity $(x_1x_2)x_3=(x_1x_3)x_2$ and $x_1(x_2x_3)=x_2(x_1x_3)$.
Let $\mathfrak B$ be the variety of all bicommutative algebras over a field $K$ of characteristic 0 and let $F({\mathfrak B})$ be the free
algebra of countable rank in $\mathfrak B$. We prove that if $\mathfrak V$ is a subvariety of $\mathfrak B$
satisfying a polynomial identity $f=0$ of degree $k$, where $0\not=f\in F({\mathfrak B})$, then the codimension sequence
$c_n({\mathfrak V})$, $n=1,2,\ldots$,
is bounded by a polynomial in $n$ of degree $k-1$. Since $c_n({\mathfrak B})=2^n-2$ for $n\geq 2$, and $\exp({\mathfrak B})=2$,
this gives that $\exp({\mathfrak V})\leq 1$, i.e., $\mathfrak B$ is minimal with respect to the codimension growth.
Up to isomorphism there are three one-generated two-dimensional bicommutative algebras $A$ which are nonassociative
and with the property $\dim A^2=1$. We present bases of their polynomial identities and show that one of these algebras
generates the whole variety $\mathfrak B$.
\end{abstract}

\section{Introduction}

Bicommutative algebras are nonassociative algebras over a field $K$ satisfying the polynomial identities
of right- and left-commutativity
\begin{equation}\label{identities of bicommutativity}
(x_1x_2)x_3=(x_1x_3)x_2,\quad x_1(x_2x_3)=x_2(x_1x_3).
\end{equation}
In the sequel we consider algebras over a field $K$ of characteristic 0 only.
One-sided commutative algebras appeared first in the paper by Cayley \cite{Ca} in 1857.
In the modern language this is the right-symmetric Witt algebra $W_1^{\text{rsym}}$ in one variable.
Maybe the most important examples of one-side commutative algebras are Novikov algebras which are
left-commutative and right-symmetric. The latter means that the algebras satisfy the polynomial identity
$(x_1,x_2,x_3)=(x_1,x_3,x_2)$ for the associator $(x_1,x_2,x_3)=(x_1x_2)x_3-x_1(x_2x_3)$.
The motivation to study Novikov algebras comes from the needs of the Hamiltonian
operator in mechanics and the equations of hydrodynamics, see \cite{DIT} and \cite{DZ} for details.
Examples of bicommutative algebras are the two-dimensional algebras $A_{\pi,\varrho}$, $\pi,\varrho\in K$,
generated by an element $r$ and with multiplication rules
\begin{equation}\label{two-dimensional algebras}
r\cdot r^2=\pi r^2,r^2\cdot r=\varrho r^2,r^2\cdot r^2=\pi\varrho r^2.
\end{equation}
In is easy to see that up to isomorphism the algebras $A_{\pi,\varrho}$ coincide with one of the algebras
$A_{0,0},A_{1,1},A_{0,1},A_{1,0}$, and $A_{1,-1}$. The first one is nilpotent of class 3, the second one is associative-commutative,
and the latter three algebras are nonassociative.

The structure of the free bicommutative algebra and the most important numerical invariants of the T-ideal of the polynomial identities
were described by Dzhumadil'daev, Ismailov, and Tulenbaev \cite{DIT}, see also the announcement \cite{DT}.
In \cite{DZ}, jointly with Zhakhayev, we proved that finitely generated bicommutative algebras are weakly noetherian, i.e., satisfy the ascending chain
condition for two-sided ideals, and answer into affirmative the finite basis problem for varieties of bicommutative algebras
over a field of arbitrary characteristic.

One of the most important measures for the complexity of the polynomial identities of a variety $\mathfrak V$ of $K$-algebras
is the codimension sequence $c_n({\mathfrak V})$,  $n=1,2,\ldots$, where
$c_n({\mathfrak V})$ is the dimension of the multilinear polynomials of degree $n$ in the free algebra $F_n({\mathfrak V})$ of rank $n$.
As a first approximation to the more precise estimate
of the growth of the codimensions one studies the behaviour of $\displaystyle \sqrt[n]{c_n({\mathfrak V})}$.
In the special case when
\[
\exp({\mathfrak V})=\lim_{n\to\infty}\sqrt[n]{c_n({\mathfrak V})}
\]
exists it is called the {\it exponent} of $\mathfrak V$, see Giambruno and Zaicev \cite{GZ1, GZ2}
who proved that for associative PI-algebras the exponent always exists and is an integer.
Following \cite{GZ3} the variety $\mathfrak V$ is {\it minimal of a given exponent} if
$\exp({\mathfrak W})<\exp({\mathfrak V})$ for all proper subvarieties $\mathfrak W$ of $\mathfrak V$.
(In \cite{D1} such varieties were called {\it extremal}.)

It was shown in \cite{DIT} that for the variety $\mathfrak B$ of all bicommutative algebras
\[
c_1({\mathfrak B})=1 \text{ and } c_n({\mathfrak B})=2^n-2,\quad n=2,3,\ldots.
\]
Hence $\exp({\mathfrak B})=2$. Our first main result is that the variety $\mathfrak B$ is minimal of exponent 2.
More precisely we show that if $\mathfrak V$ is a subvariety of $\mathfrak B$
satisfying a polynomial identity $f=0$ of degree $k$, where $0\not=f\in F({\mathfrak B})=F_{\infty}({\mathfrak B})$, then the codimension sequence
$c_n({\mathfrak V})$, $n=1,2,\ldots$,
is bounded by a polynomial in $n$ of degree $k-1$. The results of \cite{DIT} give that the variety $\mathfrak B$ is generated
by the free algebra $F_2({\mathfrak B})$ of rank 2. As a consequence we slightly improve this and show that $\mathfrak B$ is generated
by the free algebra $F_1({\mathfrak B})$ of rank 1. As a byproduct of our approach, starting with the basis of  $F({\mathfrak B})$ in \cite{DIT}
we give a new proof of the description of the cocharacter sequence $\chi_n({\mathfrak B})$, $n=1,2,\ldots$.
Finally we study the polynomial identities of the two-dimensional algebras $A_{\pi,\varrho}$
with multiplication defined by (\ref{two-dimensional algebras}).
We show that the algebra $A_{1,-1}$ generates the whole variety $\mathfrak B$. The varieties
$\text{var}(A_{0,1})$ and $\text{var}(A_{1,0})$ generated by the algebras $A_{0,1}$ and $A_{1,0}$
are defined as subvarieties of $\mathfrak B$ by the polynomial identities $x_1(x_2x_3)=0$ and $(x_1x_2)x_3=0$,
i.e., they are equal, respectively, to the varieties of left-nilpotent and right-nilpotent of class 3 bicommutative algebras.

\section{Preliminaries}

We fix a field $K$ of characteristic 0. All algebras, vector spaces, and tensor products will be over $K$.
Traditionally, one states the results on polynomial identities and cocharacter sequences in the language of representation theory
of the symmetric group $S_n$. Instead, we shall work with representation theory
of the general linear group $GL_d=GL_d(K)$. Then using the approach developed by Berele \cite{B} and the author \cite{D0}
we shall translate easily the results in terms of representations of $S_n$.
We start with the necessary background on representation theory
of $GL_d$ acting canonically on the $d$-dimensional vector space $KX_d$
with basis $X_d=\{x_1,\ldots,x_d\}$.
We refer, e.g., to \cite{M} for general facts and to \cite{D2}  for applications in the spirit of the problems
considered here.
All $GL_d$-modules which appear in this paper are
completely reducible and are direct sums of irreducible polynomial
modules. The irreducible polynomial representations of $GL_d$ are
indexed by partitions $\lambda=(\lambda_1,\ldots,\lambda_d)$,
$\lambda_1\geq \cdots\geq \lambda_d\geq 0$. We denote by
$W(\lambda)=W_d(\lambda)$ the corresponding irreducible $GL_d$-module.
The action of $GL_d$ on $KX_d$ is extended diagonally on the tensor algebra of $KX_d$
and, up to isomorphism, all $W(\lambda)$ can be found there. The tensor algebra of $KX_d$
is isomorphic, also as a $GL_d$-module, to the free associative algebra $K\langle X_d\rangle=K\langle x_1,\ldots,x_d\rangle$.
Since the diagonal action of $GL_d$ on the tensor algebra is not affected on the parentheses, we may work also in the absolutely free
algebra $K\{X_d\}$ and in the relatively free algebra $F_d({\mathfrak V})$ of any variety $\mathfrak V$.

The module $W(\lambda)\subset K\{X_d\}$ is generated by a
unique, up to a multiplicative constant,
multihomogeneous element $w_{\lambda}$ of degree $\lambda=(\lambda_1,\ldots,\lambda_d)$, i.e.,
homogeneous of degree $\lambda_k$ with respect to each variable $x_k$,
called the {\it highest weight vector} of $W(\lambda)$.
In order to state the characterization of the highest weight vectors we recall that for an algebra $R$
the linear operator $\delta:R\to R$ is a derivation
if $\delta(r_1r_2)=\delta(r_1)r_2+r_1\delta(r_2)$ for all $r_1,r_2\in R$.
If $\delta:KX_d\to KX_d$ is any linear operator of the $d$-dimensional vector space, then $\delta$ can be extended
in a unique way to a derivation of $K\langle X_d\rangle, K\{X_d\}$, and of any relatively algebra $F_d({\mathfrak V})$.
The following lemma is a partial case of a result by
De Concini, Eisenbud, and Procesi \cite{DEP}, see also
Almkvist, Dicks, and Formanek \cite{ADF}.
In the version which we need, the first part of the lemma was established
by Koshlukov \cite{K}.

\begin{lemma}\label{criterion for hwv} {\rm (see, e.g., \cite{BD})}
Let $1\leq i<j\leq d$ and let
$\Delta_{x_j\to x_i}$ be the derivation of $K\{X_d\}$
defined by $\Delta_{x_j\to x_i}(x_j)=x_i$, $\Delta_{x_j\to x_i}(x_k)=0$, $k\not=j$.
If $w(X_d)=w(x_1,\ldots,x_d) \in K\{X_d\}$
is multihomogeneous of degree $\lambda_k$ with respect to $x_k$,
then $w(X_d)$ is a highest weight vector for some
$W(\lambda)$ if and only if $\Delta_{x_j\to x_i}(w(X_d))=0$
for all $i<j$. Equivalently, $w(X_d)$ is a highest weight vector
for $W(\lambda)$ if and only if
\[
g_{ij}(w(X_d))=w(X_d),\quad 1\leq i<j\leq d,
\]
where $g_{ij}$ is the linear operator of the $KX_d$ which
sends $x_j$ to $x_i+x_j$ and fixes the other $x_k$.
\end{lemma}

If $W_i$, $i=1,\ldots,m$,
are $m$ isomorphic copies of the $GL_d$-module $W(\lambda)$
and $w_i\in W_i$ are highest weight
vectors, then the highest weight vector of any submodule $W(\lambda)$
of the direct sum $W_1\oplus\cdots\oplus W_m$ has the form
$\xi_1w_1+\cdots+\xi_mw_m$ for some $\xi_i\in K$.
Any $m$ linearly independent highest weight vectors can serve
as a set of generators of the $GL_d$-module $W_1\oplus\cdots\oplus W_m$.
The algebra $F_d({\mathfrak V})$ decomposes as a $GL_d$-module as
\begin{equation}\label{GL-decomposition of free algebras}
F_d({\mathfrak V})=\bigoplus_{\lambda}m_{\lambda}({\mathfrak V})W(\lambda),
\end{equation}
where the summation runs on all partitions $\lambda$ in not more than $d$ parts and the nonnegative integer $m_{\lambda}({\mathfrak V})$
is the multiplicity of $W(\lambda)$ in the decomposition of $F_d({\mathfrak V})$.
The canonical multigrading of $F_d({\mathfrak V})$ which counts the degree of each variable in $X_d$
agrees with the action of $GL_d$ in the following way. Let
\[
H(F_d({\mathfrak V}),T_d)=H(F_d({\mathfrak V}),t_1,\ldots,t_d)
\]
\[
=\sum_{n_i\geq 0}\dim F_d^{(n)}({\mathfrak V})T_d^n
=\sum_{n_i\geq 0}\dim F_d^{(n)}({\mathfrak V})t_1^{n_1}\cdots t_d^{n_d}
\]
be the Hilbert series of $F_d({\mathfrak V})$ as a multigraded vector space, where $F_d^{(n)}({\mathfrak V})$ is the multihomogeneous
component of $F_d({\mathfrak V})$ of degree $n=(n_1,\ldots,n_d)$.
Then
\[
H(F_d({\mathfrak V}),T_d)=\sum_{\lambda}m_{\lambda}({\mathfrak V})s_{\lambda}(T_d)=\sum_{\lambda}m_{\lambda}({\mathfrak V})s_{\lambda}(t_1,\ldots,t_d),
\]
where $s_{\lambda}(T_d)$ is the Schur function corresponding to the partition $\lambda$.

There is another group action which is important for the theory of algebras with polynomial identities.
The symmetric group $S_n$ acts on the vector space $P_n({\mathfrak V})$ of the multilinear polynomials of degree $n$ in $F_n({\mathfrak V})$ by
\[
\sigma(f(x_1,\ldots,x_n))=f(x_{\sigma(1)},\ldots,x_{\sigma(n)}),\quad \sigma\in S_n,f\in P_n({\mathfrak V}).
\]
The $S_n$-character of $P_n({\mathfrak V})$ is called the $S_n$-{\it cocharacter} of $\mathfrak V$.
It is known that the decomposition of the $n$-th cocharacter
\begin{equation}\label{Sn-decomposition for cocharacters}
\chi_n({\mathfrak V})=\sum_{\lambda\vdash n}m_{\lambda}({\mathfrak V})\chi_{\lambda},
\end{equation}
where $\chi_{\lambda}$ is the irreducible $S_n$-character indexed with the partition $\lambda$ of $n$, is determined by the Hilbert series of
$F_n({\mathfrak V})$. The multiplicities $m_{\lambda}({\mathfrak V})$ are the same for $F_n({\mathfrak V})$ in (\ref{GL-decomposition of free algebras})
and for $\chi_n({\mathfrak V})$ in (\ref{Sn-decomposition for cocharacters}).
Finally, we recall a special case of the Young rule (and of the Littlewood-Richardson rule)
for the product of two Schur functions $s_{(p)}(T_d)$ and $s_{(q)}(T_d)$
(and also for the tensor product $W(p)\otimes W(q)$ of the $GL_d$-modules $W(p)$ and $W(q)$). We assume that $p\geq q$. The case $p<q$
is similar.
\begin{equation}\label{Young rule}
\begin{split}
s_{(p)}(T_d)s_{(q)}(T_d)=\sum_{k=0}^qs_{(p+q-k,k)}(T_d),\\
W(p)\otimes W(q)\cong \bigoplus_{k=0}^qW(p+q-k,k).
\end{split}
\end{equation}
We shall need also estimates for the degree of the irreducible $S_n$-characters.

\begin{lemma}\label{degree for Sn-characters}
The degree $d_{\lambda}$ of the irreducible $S_n$-character $\chi_{\lambda}$, $\lambda=(\lambda_1,\lambda_2)\vdash n$,
is a polynomial in $n$ of degree $\lambda_2$.
\end{lemma}

\begin{proof} By the hook formula
\[
d_{\lambda}=\frac{n!}{\prod h_{ij}},
\]
where $h_{ij}$ is the length of the hook at the $(i,j)$-position of the Young diagram of $\lambda$.
For $\lambda=(\lambda_1,\lambda_2)\vdash n$ the lengths of the hooks of the first row are equal, reading them from right to left, to
\[
1,2,\ldots,n-2\lambda_2,n-2\lambda_2+2,\ldots,n-\lambda_2+1
\]
and those of the second row are $1,2,\ldots,\lambda_2$. Hence
\[
d_{\lambda}=\frac{n(n-1)\cdots (n-\lambda_2+1)(n-2\lambda_2+1)}{\lambda_2!},
\]
which is a polynomial of degree $\lambda_2$ in $n$.
\end{proof}

Let $\mathfrak B$ be the variety of all bicommutative algebras.
We assume that the free bicommutative algebras $F=F({\mathfrak B})$ and $F_d=F_d({\mathfrak B})$ are freely generated, respectively,
by the sets $X=\{x_1,x_2,\ldots\}$ and $X_d=\{x_1,\ldots,x_d\}$.
By \cite{DIT} the basis of the square $F^2_d$ of the algebra $F_d$ as a $K$-vector space consists of the following polynomials:
\begin{equation}\label{basis of F^2}
u_{i,j}=x_{i_1}(\cdots(x_{i_{p-1}}((\cdots((x_{i_p} x_{j_1})x_{j_2})\cdots)x_{j_q}))\cdots),
\end{equation}
where $p,q\geq 1$, $1\leq i_1\leq\cdots\leq i_{p-1}\leq i_p\leq d$, $1\leq j_1\leq j_2\leq\cdots\leq j_q\leq d$.
For any permutations $\sigma\in S_p$ and $\tau\in S_q$ the element $u_{i,j}$ from (\ref{basis of F^2}) satisfy the equality
\begin{equation}\label{action of Sm x Sn on the square}
u_{i,j}=x_{i_{\sigma(1)}}(\cdots(x_{i_{\sigma(p-1)}}((\cdots((x_{i_{\sigma(p)}} x_{j_{\tau(1)}})x_{j_{\tau(2)}})\cdots)x_{j_{\tau(q)}}))\cdots).
\end{equation}
The properties and the multiplication rules of $F_d({\mathfrak B})$ from \cite{DIT} are restated in \cite{DZ} in the following way.
We consider the polynomial algebra $K[Y_d,Z_d]=K[y_1,\ldots,y_d,z_1,\ldots,z_d]$ in $2d$ commutative and associative variables.
We associate to each monomial $u_{i,j}$ in (\ref{basis of F^2}) the monomial
\[
\psi(u_{i,j})=y_{i_1}\cdots y_{i_{p-1}}y_{i_p} z_{j_1}z_{j_2}\cdots z_{j_q}\in K[Y_d,Z_d]
\]
and extend $\psi$ by linearity to a linear map $\psi:F_d^2\to K[Y_d,Z_d]$. The image $\psi(F_d^2)$ is spanned by all monomials
\[
Y_d^{\alpha}Z_d^{\beta}=y_1^{\alpha_1}\cdots y_d^{\alpha_d}z_1^{\beta_1}\cdots z_d^{\beta_d},\quad
\vert\alpha\vert=\sum_{i=1}^d\alpha_i>0,\vert\beta\vert=\sum_{j=1}^d\beta_j>0.
\]
Then we define an algebra $G_d$ generated by $X_d$ with basis
\[
X_d\cup\{Y_d^{\alpha}Z_d^{\beta}\mid\vert\alpha\vert,\vert\beta\vert>0\}
\]
and multiplication rules
\begin{equation}\label{multiplication in G_d}
\begin{array}{c}
x_ix_j=y_iz_j,\\
\\
x_i(Y_d^{\alpha}Z_d^{\beta})=y_iY_d^{\alpha}Z_d^{\beta},\\
\\
(Y_d^{\alpha}Z_d^{\beta})x_j=Y_d^{\alpha}Z_d^{\beta}z_j,\\
\\
(Y_d^{\alpha}Z_d^{\beta})(Y_d^{\gamma}Z_d^{\delta})=Y_d^{\alpha+\gamma}Z_d^{\beta+\delta}.\\
\end{array}
\end{equation}
The algebras $F_d$ and $G_d$ are isomorphic both as algebras and as multigraded vector spaces with isomorphism $\psi:F_d\to G_d$
which sends $x_i\in F_d$ to $x_i\in G_d$ and acts on $F_d^2$ in the same way as the linear map
$\psi:F_d^2\to K[Y_d,Z_d]$ defined above.

\section{Free bicommutative algebras}
In this section we give an alternative proof of the formula for the cocharacter sequence of $\mathfrak B$ given in \cite{DIT}
and describe the highest weight vectors of the irreducible $GL_d$-submodules of $F_d=F_d({\mathfrak B})$.

\begin{lemma}\label{the graded structure of square of F}
As a multigraded vector space the square $F_d^2$ of the free bicommutative algebra $F_d$ is isomorphic to the tensor product
$\omega(K[Y_d])\otimes\omega(K[Z_d])$, where $\omega$ is the augmentation ideal of the polynomial algebra, i.e., the ideal of polynomials
without constant term. As a $GL_d$-module $F_d^2$ is isomorphic to the direct sum of tensor products
\begin{equation}\label{GL-structure of the square of F}
\bigoplus_{p,q\geq 1}W(p)\otimes W(q).
\end{equation}
\end{lemma}

\begin{proof}
We identify the monomial $Y_d^{\alpha}Z_d^{\beta}\in K[Y_d,Z_d]$ with $Y_d^{\alpha}\otimes Z_d^{\beta}\in K[Y_d]\otimes K[Z_d]$.
Then the first part of the lemma is simply a restatement of the fact that the image of $F_d^2$ under the action of $\psi$ has a basis
$\{Y_d^{\alpha}Z_d^{\beta}\mid\vert\alpha\vert,\vert\beta\vert>0\}$.  The second part of the lemma holds because
the $GL_d$-module $K[Y_d]^{(p)}$ of the homogeneous polynomials of degree $p$ in $K[Y_d]$ is isomorphic to $W(p)$ and similarly for $K[Z_d]^{(q)}$.
\end{proof}

\begin{proposition}\label{the n-th cocharacter of B}\cite{DIT}
The cocharacter sequence of the variety $\mathfrak B$ of all bicommutative algebras is
\[
\chi_n({\mathfrak B})=\sum_{(\lambda_1,\lambda_2)\vdash n}m_{(\lambda_1,\lambda_2)}({\mathfrak B})\chi_{(\lambda_1,\lambda_2)},
\]
where
\begin{equation}\label{multiplicities of cocharacters of B}
\begin{split}
m_{(1)}({\mathfrak B})=1,\\
m_{(n)}({\mathfrak B})=n-1,\quad n>1,\\
m_{(\lambda_1,\lambda_2)}({\mathfrak B})=n-2\lambda_2+1,\quad \lambda_2>0.
\end{split}
\end{equation}
\end{proposition}

\begin{proof}
The multiplicities of the irreducible $S_n$-characters in the cocharacter sequence (\ref{Sn-decomposition for cocharacters})
and of the irreducible $GL_d$-modules of the homogeneous component $F_d^{(n)}$ of degree $n$ of the free algebra $F_d$ in (\ref{GL-decomposition of free algebras})
are the same for $d\geq n$. Hence we may work in $F_d$ instead of with $\chi_n({\mathfrak B})$. Since the case $n=1$ is trivial, we shall assume that $n>1$.
By Lemma \ref{the graded structure of square of F} and the Young rule (\ref{Young rule}) we derive that the only nontrivial multiplicities
$m_{\lambda}({\mathfrak B})$ are for $\lambda=(\lambda_1,\lambda_2)$. Then $m_{\lambda}({\mathfrak B})$ is equal to the number of tensor products
$W(p)\otimes W(q)$ which contain an isomorphic copy of $W(\lambda)$ as a submodule. For $\lambda=(n)$ there are $n-1$ possibilities
\[
W(1)\otimes W(n-1),W(2)\otimes W(n-2),\ldots,W(n-1)\otimes W(1),
\]
i.e., $m_{(n)}({\mathfrak B})=n-1$. For $\lambda=(\lambda_1,\lambda_2)$ with $\lambda_2>0$ the possibilities are
\[
W(\lambda_2)\otimes W(n-\lambda_2),W(\lambda_2+1)\otimes W(n-\lambda_2-1),\ldots,W(n-\lambda_2)\otimes W(\lambda_2),
\]
which gives $m_{(\lambda_1,\lambda_2)}({\mathfrak B})=n-2\lambda_2+1$.
\end{proof}

The action of $GL_d$ on the $d$-dimensional vector space $KX_d$ induces a similar action on $KY_d$ and $KZ_d$ which is extended diagonally
on the square $G_d^2$ of the algebra $G_d$.

\begin{lemma}\label{hwv of G}
The following polynomials $w_{\lambda}^{(k)}$ form a maximal linearly independent system of highest weight vectors of the $GL_d$-submodules $W(\lambda)$ in $G_d^2$:
\begin{equation}\label{hwv of W(lambda)}
\begin{split}
w_{(n)}^{(j)}=y_1^jz_1^{n-j},&\quad j=1,2,\ldots,n-1,\\
w_{\lambda}^{(j)}=y_1^j(y_1z_2-y_2z_1)^{\lambda_2}z_1^{\lambda_1-\lambda_2-j}, &\quad j=0,1,\ldots,\lambda_1-\lambda_2,\text{ if }\lambda_2>0.
\end{split}
\end{equation}
\end{lemma}

\begin{proof}
For a fixed $\lambda$ the elements (\ref{hwv of W(lambda)}) are linearly independent because are nonzero and of pairwise different degree in $y_1$.
They are of degree $\lambda_1$ with respect to $y_1,z_1$ and of degree $\lambda_2$ with respect to $y_2,z_2$.
By Proposition \ref{the n-th cocharacter of B} the multiplicities of $W(n)$ and
$W(\lambda)$, $\lambda=(\lambda_1,\lambda_2)\vdash n$, in $G_d^2$
are, respectively,
\[
m_{(n)}({\mathfrak B})=n-1\text{ and }m_{(\lambda_1,\lambda_2)}({\mathfrak B})=n-2\lambda_2+1=\lambda_1-\lambda_2+1.
\]
Hence their number coincides with the number of polynomials in (\ref{hwv of W(lambda)}).
Now, it is sufficient to show that all $w_{\lambda}^{(j)}$ are highest weight vectors. Applying Lemma \ref{criterion for hwv},
this is obvious for $w_{(n)}^{(j)}$. Let $\lambda_2>0$. The analogue $\Delta_{y_2\to y_1,z_2\to z_1}$ of the derivation $\Delta_{x_2\to x_1}$ acting on $K[Y_d,Z_d]$
sends $y_1,z_1$ to 0 and $y_2,z_2$ to $y_1,z_1$, respectively.
Obviously
\[
\Delta_{y_2\to y_1,z_2\to z_1}(w_{\lambda}^{(j)})
=\lambda_2y_1^j(y_1z_2-y_2z_1)^{\lambda_2-1}z_1^{\lambda_1-\lambda_2-j}\Delta_{y_2\to y_1,z_2\to z_1}(y_1z_2-y_2z_1)=0
\]
and all $w_{\lambda}^{(j)}$ are highest weight vectors.
\end{proof}

\section{Subvarieties}

In this section we assume that $\mathfrak V$ is a proper subvariety of $\mathfrak B$ and $\mathfrak V$ satisfies a nontrivial polynomial identity
$f=0$ of degree $k$, where $0\not=f(X_d)\in F=F({\mathfrak B})$. Since the case $k=1$ is trivial we shall assume that $k\geq 2$.
In the sequel we shall work mainly in the isomorphic copies $G$ and $G_d$ of the algebras $F$ and $F_d$ instead of in $F$ and $F_d$.
Identifying $F$ and $F_d$ with their isomorphic copies, we shall denote the corresponding elements with the same symbols.
In particular, if $f(X_d)\in F_d^2$ we shall write $f(Y_d,Z_d)\in G_d^2$ and vise versa. 
Since the $GL_d$-module generated by $f$ contains an irreducible submodule $W(\lambda)$,
there exists a highest weight vector $w_{\lambda}$ such that the polynomial identity $w_{\lambda}=0$ follows from the polynomial identity $f=0$.
Hence we may assume that $\mathfrak V$ satisfies some polynomial identity $w_{\lambda}(x_1,x_2)=0$ for $\lambda\vdash k$.
Then $w_{\lambda}(Y_2,Z_2)\in G_2$ is a linear combination of the highest weight vectors in (\ref{hwv of W(lambda)}) and for some $\xi_j\in K$
\begin{equation}\label{hwv for subvariety}
\begin{split}
w_{(k)}=\sum_{j=1}^{k-1}\xi_jy_1^jz_1^{k-j},&\text{ for } \lambda=(k),\\
w_{(k)}=(y_1z_2-y_2z_1)^{\lambda_2}\sum_{j=0}^{\lambda_1-\lambda_2}\xi_jy_1^jz_1^{\lambda_1-\lambda_2-j},
&\text{ for } \lambda=(\lambda_1,\lambda_2),\lambda_2>0.
\end{split}
\end{equation}
If $f(X_d)\in F_d$ is multihomogeneous then its partial linearization
$\text{lin}_{x_i}f(X_d)$ in $x_i$ is the component of degree $\deg_{x_i}-1$ with respect to $x_i$ of the polynomial
$f(x_1,\ldots,x_i+x_{d+1},\ldots,x_d)\in F_{d+1}$. If $\Delta_{x_i\to x_{d+1}}$ is the derivation of $F_{d+1}$
which sends $x_i$ to $x_{d+1}$ and the other $x_j$ to 0,
then
\[
\text{lin}_{x_i}f(X_d)=(\text{lin}_{x_i}f)(X_{d+1})=\Delta_{x_i\to x_{d+1}}(f(X_d)).
\]
If $u\in F$ then $(\text{lin}_{x_i}f)(x_1,\ldots,x_d,u)$ can be expressed in terms of derivations as
\[
(\text{lin}_{x_i}f)(x_1,\ldots,x_d,u)=\Delta_{x_i\to u}(f(X_d)),
\]
where $\Delta_{x_i\to u}$ is the derivation of $F$ sending $x_i$ to $u$ and all other generators $x_j$ to 0.
The action of the analogue of $\Delta_{x_i\to x_{d+1}}$ on $G^2$ is clear: It sends $y_i$ and $z_i$, respectively, to $y_{d+1}$ and $z_{d+1}$ and all other
variables $y_j$ and $z_j$ to 0. We denote this derivation by $\Delta_{y_i\to y_{d+1},z_i\to z_{d+1}}$.
Now we shall translate the action of $\delta_{x_i\to u}$ on $F^2$, $u\in F^2$,
in the language of $G$ and the usual partial derivatives.

\begin{lemma}\label{action of delta}
Let $u\in K[Y,Z]$ be in the image $G^2\subset K[Y,Z]$ of $F^2$.
Let $\Delta_{y_i,z_i\to u}$ be the derivation of $K[Y,Z]$ which sends the variables $y_i,z_i$ to $u$ and the other variables to $0$.
If $f(X_d)\in F^2$ is multihomogeneous, then the image of
$(\text{lin}_{x_i}f)(x_1,\ldots,x_d,u)$ in $G^2$ is
\[
\Delta_{y_i,z_i\to u}(f)=\left(\frac{\partial f}{\partial y_i}+\frac{\partial f}{\partial z_i}\right)u.
\]
\end{lemma}

\begin{proof}
It is sufficient to consider the case when $f$ and $u$ are monomials and $i=1$:
\[
f=(y_1^{\alpha_1}z_1^{\beta_1})v_1v_2, v_1=y_2^{\alpha_2}\cdots y_d^{\alpha_d},v_2=z_2^{\beta_2}\cdots z_d^{\beta_d},
\alpha_1+\beta_1\geq 1,\vert\alpha\vert>0,\vert\beta\vert>0,
\]
\[
u=Y_d^{\gamma}Z_d^{\delta},\vert\gamma\vert>0,\vert\delta\vert>0.
\]
Then
\[
\Delta_{y_1\to y_{d+1},z_1\to z_{d+1}}(f)=(\alpha_1y_1^{\alpha_1-1}y_{d+1}z_1^{\beta_1}+\beta_1y_1^{\alpha_1}z_1^{\beta_1-1}z_{d+1})v_1v_2
\]
\[
=\frac{\partial f}{\partial y_1}y_{d+1}+\frac{\partial f}{\partial z_1}z_{d+1},
\]
In virtue of (\ref{action of Sm x Sn on the square}) we may assume that the preimage
$\displaystyle \psi^{-1}\left(\frac{\partial f}{\partial y_1}y_{d+1}\right)$ in $F_{d+1}^2$ is of the form
$\alpha_1(\cdots(x_{d+1}x_{j_1})\cdots)$, where the dots before and after $(x_{d+1}x_{j_1})$ correspond to the beginning and the end of the element in the form
(\ref{basis of F^2}). Since $u\in F^2$ we obtain that
\[
\psi(\alpha_1(\cdots(ux_{j_1})\cdots))=\alpha_1(\cdots (Y_d^{\gamma}Z_d^{\delta}z_{j_1})\cdots)=\frac{\partial f}{\partial y_1}u.
\]
Similarly
\[
\psi(\beta_1(\cdots(x_{i_p}u)\cdots))=\beta_1(\cdots (y_{i_p}Y_d^{\gamma}Z_d^{\delta})\cdots)=\frac{\partial f}{\partial z_1}u.
\]
\end{proof}

\begin{lemma}\label{consequences of w(lambda)}
If $0\not=f\in W(\lambda_1,\lambda_2)\subset F({\mathfrak B})$, then all polynomial identities $w_{(\mu_1,\mu_2)}^{(j)}=0$ with $\mu_2\geq \lambda_1$
are consequences of the polynomial identity $f=0$.
\end{lemma}

\begin{proof}
As commented in the beginning of the section, we may assume that $f=w_{\lambda}$ is a highest weight vector in $W(\lambda_1,\lambda_2)\subset F_2$.
Hence, working in $G_2$ instead of in $F_2$, $w_{\lambda}$ has the form (\ref{hwv for subvariety}), i.e.,
\[
w_{\lambda}=\sum_{j\geq p}\xi_j w_{\lambda}^{(j)}=(y_1z_2-y_2z_1)^{\lambda_2}\sum_{j\geq p}\xi_jy_1^jz_1^{\lambda_1-\lambda_2-j},\xi_p\not=0.
\]
First, let $p>0$, i.e., $w_{\lambda}$ is divisible by $y_1^p$.
The partial linearizations of the identity $w_{\lambda}=0$ are its consequences. Hence $\Delta_{y_1\to y_2,z_1\to z_2}(w_{\lambda})=0$ which has the form
\[
(y_1z_2-y_2z_1)^{\lambda_2}\sum_{j\geq p}\xi_j(jy_1^{j-1}y_2z_1^{\lambda_1-\lambda_2-j}+(\lambda_1-\lambda_2-j)y_1^jz_1^{\lambda_1-\lambda_2-j-1}z_2)=0
\]
is also a consequence of $w_{\lambda}=0$ and the same holds for
\[
w_{(\lambda_1,\lambda_2+1)}=\Delta_{y_1\to y_2,z_1\to z_2}(w_{\lambda})z_1-(\lambda_1-\lambda_2)w_{\lambda}z_2
\]
\[
=-(y_1z_2-y_2z_1)^{\lambda_2+1}\sum_{j\geq p-1}(j+1)\xi_{j+1}y_1^jz_1^{\lambda_1-\lambda_2-j-1}=0.
\]
We obtained that $w_{(\lambda_1,\lambda_2+1)}=0$ is a consequence of $w_{(\lambda_1,\lambda_2)}=0$.
It is divisible by $y_1^{p-1}$ but is not divisible by $y_1^p$.
Continuing in this way we shall reach a consequence
\[
w_{(\lambda_1,\lambda_2+p)}=(y_1z_2-y_2z_1)^{\lambda_2+p}j!\xi_pz_1^{\lambda_1-\lambda_2-p}=0.
\]
Now the consequence
\[
y_1\Delta_{y_1\to y_2,z_1\to z_2}(w_{(\lambda_1,\lambda_2+p)})-y_2(\lambda_1-\lambda_2+p)w_{(\lambda_1,\lambda_2+p)}=0
\]
is of the form $w_{(\lambda_1,\lambda_2+p+1)}=0$ and is divisible by $z_1^{\lambda_1-\lambda_2-p-1}$ only. Continuing the process we shall obtain
as a consequence
\[
w_{(\lambda_1,\lambda_1)}=(y_1z_2-y_2z_1)^{\lambda_1}=w_{(\lambda_1,\lambda_1)}^{(0)}.
\]
Since all $w_{(\mu_1,\mu_2)}^{(j)}$ with $\mu_2\geq\lambda_1$ are divisible by $w_{(\lambda_1,\lambda_1)}^{(0)}$
and hence are its consequences, we complete the proof.
\end{proof}

\begin{corollary}\label{consequences of identity of degree k}
If $0\not=f\in F$ is of degree $k$ then all identities $w_{(\mu_1,\mu_2)}^{(j)}=0$ with $\mu_2\geq k$ follow from the identity $f=0$.
\end{corollary}

\begin{proof}
The statement follows immediately from Lemma \ref{consequences of w(lambda)} because if $(\lambda_1,\lambda_2)\vdash k$, then $\lambda_1\leq k$.
\end{proof}

\begin{lemma}\label{consequences in one variable}
The polynomial identity $w_{(k,k)}^{(0)}=(y_1z_2-y_2z_1)^k=0$ has as consequences all identities
\[
(y_1z_1)^k(y_1-z_1)^kw_{\mu}^{(j)}=0
\]
for all $\mu=(\mu_1,\mu_2)$ and all $j=0,1,\ldots,\mu_1-\mu_2$.
\end{lemma}

\begin{proof}
We apply the derivation $\Delta_{y_2,z_2\to y_1z_1}$ and obtain as a consequence of the identity $w_{(k,k)}^{(0)}=0$ the identity
\[
\Delta_{y_2,z_2\to y_1z_1}(w_{(k,k)}^{(0)})=y_1z_1\left(\frac{\partial}{\partial y_2}+\frac{\partial}{\partial z_2}\right)(y_1z_2-y_2z_1)^k
\]
\[
=ky_1z_1(y_1z_2-y_2z_1)^{k-1}\left(\frac{\partial}{\partial y_2}+\frac{\partial}{\partial z_2}\right)(y_1z_2-y_2z_1)
\]
\[
=ky_1z_1(y_1-z_1)(y_1z_2-y_2z_1)^{k-1}=0.
\]
Continuing in this way we obtain
\[
\Delta_{y_2,z_2\to y_1z_1}^k(w_{(k,k)}^{(0)})=k!(y_1z_1)^k(y_1-z_1)^k=0
\]
which gives that $(y_1z_1)^k(y_1-z_1)^kw_{\mu}^{(j)}=0$ for all $\mu$ and all $j$.
\end{proof}

\begin{corollary}\label{generating by free algebra of rank 1}
The variety $\mathfrak B$ is generated by its one-generated free algebra $F_1({\mathfrak B})$.
\end{corollary}

\begin{proof}
If $\text{var}(F_1({\mathfrak B}))\not=\mathfrak B$, then by Lemma \ref{consequences of w(lambda)} the algebra
$F_1({\mathfrak B})$ satisfies some identity $w_{(k,k)}^{(0)}$ and by Lemma \ref{consequences in one variable} satisfies the identity
$(y_1z_1)^k(y_1-z_1)^k=0$ in one variable. This means that $(y_1z_1)^k(y_1-z_1)^k=0$ in $F_1({\mathfrak B})$
which is impossible.
\end{proof}

The following theorem is the first main result of our paper.


\begin{theorem}\label{main theorem}
If $\mathfrak V$ is a proper subvariety of the variety $\mathfrak B$ of all bicommutative algebras such that
$\mathfrak V$ satisfies a polynomial identity $f=0$ of degree $k$, $0\not=f\in F({\mathfrak B})$, then
$c_n({\mathfrak V})$ is bounded by a polynomial of degree $k-1$.
\end{theorem}

\begin{proof}
Let
\begin{equation}\label{cocharacters of V}
\chi_n({\mathfrak V})=\sum_{\lambda\vdash n}m_{\lambda}({\mathfrak V})\chi_{\lambda},\quad n=1,2,\ldots,
\end{equation}
be the cocharacter sequence of $\mathfrak V$. By Proposition \ref{the n-th cocharacter of B} the summation in (\ref{cocharacters of V}) runs on
$\lambda=(\lambda_1,\lambda_2)\vdash n$. By Corollary \ref{consequences of identity of degree k} we obtain that
$m_{(\lambda_1,\lambda_2)}({\mathfrak V})=0$ for $\lambda_2\geq k$. If $\lambda_1-\lambda_2\leq 3k-1$, then
\[
m_{(\lambda_1,\lambda_2)}({\mathfrak V})\leq m_{(\lambda_1,\lambda_2)}({\mathfrak B}) \leq \lambda_1-\lambda_2+1\leq 3k.
\]
Now, let $\lambda_1-\lambda_2\geq 3k$. By Lemma \ref{consequences in one variable}, the variety $\mathfrak V$ satisfies the identities
\[
w_j=(y_1z_1)^k(y_1-z_1)^k(y_1z_2-y_2z_1)^{\lambda_2}y_1^jz_1^{\lambda_1-\lambda_2-3k-j}=0,\quad
j=0,1,\ldots,\lambda_1-\lambda_2-3k.
\]
All $w_j$, $j=0,1,\ldots,\lambda_1-\lambda_2-3k$, are linearly independent in $F_2({\mathfrak B})$
and are highest weight vectors for $GL_2$-submodules of $F_2({\mathfrak B})$.
Hence the multiplicity $m_{\lambda}({\mathfrak V})$ satisfies the inequality
\[
m_{\lambda}({\mathfrak V})\leq m_{\lambda}({\mathfrak B})-(\lambda_1-\lambda_2-3k+1)=(\lambda_1-\lambda_2+1)-(\lambda_1-\lambda_2-3k+1)=3k.
\]
Hence (\ref{cocharacters of V}) satisfies the inequality
\[
\chi_n({\mathfrak V})=\sum_{(\lambda_1,\lambda_2)\vdash n\atop \lambda_2<k}m_{\lambda}({\mathfrak V})\chi_{(\lambda_1,\lambda_2)}
\leq \sum_{j=0}^{k-1}3k\chi_{(n-j,j)}.
\]
We obtain that the codimension sequence $c_n({\mathfrak B})$, $n=1,2,\ldots$, satisfies
\[
c_n({\mathfrak B})\leq\sum_{j=0}^{k-1}3kd_{(n-j,j)}
\]
which by Lemma \ref{degree for Sn-characters} is a polynomial of degree $k-1$.
\end{proof}

\begin{remark}\label{better bound}
We may precise Corollary \ref{consequences of identity of degree k}: If ${\mathfrak V}\subset{\mathfrak B}$
satisfies an identity $w_{\lambda}=0$ of degree $k$ and $\lambda_2>0$ in $\lambda=(\lambda_1,\lambda_2)\vdash k$,
then $\lambda_1\leq k-1$ and $\mathfrak V$
satisfies all identities $w_{\mu}^{(j)}=0$ for $\mu=(\mu_1,\mu_2)$ and $\mu_2\geq k-1$. Hence in this case $c_n({\mathfrak V})$
is bounded by a polynomial of degree $k-2$.
\end{remark}

\begin{example} The bound by a polynomial of degree $k-2$ in Remark \ref{better bound} is sharp.
Let $\mathfrak V$ be the subvariety of $\mathfrak B$ defined by the polynomial identity of right nilpotency $(\cdots((x_1x_2)x_3)\cdots )x_k=0$.
It is easy to see that the image in $G$ of the T-ideal $T({\mathfrak V})$ of the identities of $\mathfrak V$
is generated as an ordinary two-sided ideal by the products $y_{i_1}z_{i_2}\cdots z_{i_k}$. Hence if $\mu_2\geq k-1$, then
all $w_{(\mu_1,\mu_2)}^{(j)}$ belong to this T-ideal and
\[
w_{(n-k+2,k-2)}^{(n-2k+4)}=y_1^{n-2k+4}(y_1z_2-y_2z_1)^{k-2}
\]
does not belong to this ideal. Hence
$c_n({\mathfrak V})\geq d_{(n-k+2,k-2)}$ which is a polynomial of degree $k-2$. We do not know whether there exists
a variety ${\mathfrak V}\subset\mathfrak B$ satisfying a polynomial identity
in one variable of degree $k$ such that $c_n({\mathfrak V})$ grows as a polynomial of degree $k-1$.
\end{example}

\section{Two-dimensional algebras}

The classification of all two-dimensional algebras can be traced back to the two-dimensional part of the classification project
in the seminal book by B. Peirce \cite{Pe} published lithographically in 1870
in a small number of copies for distribution among his friends
and then reprinted posthumously in 1881 with addenda of his son C.S. Peirce.
The complete classification over any field was finished by Petersson \cite{P} in 2000. The paper \cite{P} contains also the history of the classification.
See as well \cite{GG} for the contributions of Peirce and \cite{GR, RTS}. Concerning the polynomial identities of two-dimensional algebras,
Giambruno, Mishchenko, and Zaicev \cite{GMZ} proved that the growth of the codimension sequence $c_n(A)$ of such an algebra $A$
over a field of characteristic 0 is either linear (and bounded by $n+1$) or grows exponentially as $2^n$. In this section we shall study the polynomial identities
of the bicommutative algebras $A_{\pi,\varrho}$ with multiplication given in (\ref{two-dimensional algebras}).

\begin{proposition}\label{classification of two-dimensional algebras}
There are only five nonisomorphic algebras $A_{\pi,\varrho}$:
\[
A_{0,0},A_{1,1},A_{0,1},A_{1,0},A_{1,-1}.
\]
\end{proposition}

\begin{proof}
Let $\pi=0$, $\varrho\not=0$. If we replace the generator $r$ by $r_1=\varrho r$, then
\[
r^2\cdot r=\varrho r^2,\quad \varrho^3r_1^2\cdot r_1=\varrho \varrho^2r_1^2,\quad r_1^2\cdot r_1=r_1^2,
\]
i.e., $A_{0,\varrho}\cong A_{0,1}$. Similarly, $A_{\pi,0}\cong A_{1,0}$. If $\pi=\varrho\not=0$, then the change of the generator $r$ with
$r_1=\pi r$ gives that
\[
r^2\cdot r=\pi r^2,\quad \pi^3r_1^2\cdot r_1=\pi \pi^2r_1^2,\quad r_1^2\cdot r_1=r_1^2,\quad r_1\cdot r_1^2=r_1^2,
\]
and$A_{\pi,\pi}\cong A_{1,1}$. Finally, let $\pi\not=\varrho$ be different from 0. We fix solutions $\xi$ and $\eta$ of the linear system
\[
\pi(\xi+\varrho\eta)=1,\quad \varrho(\xi+\pi\eta)=-1.
\]
Then $r_1=\xi r+\eta r^2$ satisfies the conditions
\[
r_1^2=(\xi+\pi\eta)(\xi+\varrho\eta)r^2=-\frac{1}{\pi\varrho}r^2,
\]
\[
r_1\cdot r_1^2=-\frac{1}{\pi\varrho}(\xi r+\eta r^2)r^2=-\frac{1}{\pi\varrho}\pi(\xi+\eta\varrho)r^2=\pi(\xi+\eta\varrho)r_1^2=r_1^2,
\]
\[
r_1^2\cdot r_1=\varrho(\xi+\pi\eta)r_1^2=-r_1^2,
\]
i.e., $A_{\pi,\varrho}\cong A_{1,-1}$.
Obviously, the five algebras $A_{0,0},A_{1,1},A_{0,1},A_{1,0},A_{1,-1}$ are pairwise nonisomorphic:
$A_{0,0}$ is nilpotent of class 3, $A_{1,1}$ is associative-commutative, $A_{0,1}$ is left-nilpotent of class 3 but not right-nilpotent,
$A_{1,0}$ is right-nilpotent of class 3 but not left-nilpotent, $A_{1,-1}$ is noncommutative and neither left- nor right-nilpotent.
\end{proof}

The algebra $A_{0,0}$ satisfies the identities
\[
x_1x_2=x_2x_1,\quad (x_1x_2)x_3=0.
\]
Obviously they form a basis of its polynomial identities and the cocharacter sequence of $A_{0,0}$ is
\[
\chi_1(A_{0,0})=\chi_{(1)},\quad\chi_2(A_{0,0})=\chi_{(2)},\quad \chi_n(A_{0,0})=0,\quad n=3,4,\ldots.
\]
Similarly, one basis of the polynomial identities of the algebra $A_{1,1}$ consists of
\[
x_1x_2=x_2x_1,\quad (x_1x_2)x_3=x_1(x_2x_3)
\]
and the cocharacter sequence is
\[
c_n(A_{1,1})=\chi_{(n)},\quad n=1,2,\ldots .
\]
The next theorem gives bases for the polynomial identities and the cocharacter sequences of the other three algebras.

\begin{theorem}\label{PIs of two-dimensional algebras}
{\rm (i)} As subvarieties of the variety $\mathfrak B$ of all bicommutative algebras the varieties $\text{\rm var}(A_{0,1})$ and $\text{\rm var}(A_{1,0})$ generated
by the algebras $A_{0,1}$ and $A_{1,0}$ are defined by the identities of left-nilpotency $x_1(x_2x_3)=0$ and right-nilpotency $(x_1x_2)x_3=0$, respectively.
Their cocharacter and codimension sequences coincide and are
\[
\chi_1(A_{0,1})=\chi_1(A_{1,0})=\chi_{(1)},\chi_n(A_{0,1})=\chi_{(n)}+\chi_{(n-1,1)},\quad n=2,3,\ldots,
\]
\[
c_n(A_{0,1})=c_n(A_{1,0})=n,\quad n=1,2,\ldots.
\]
 {\rm (ii)} The algebra $A_{1,-1}$ generates the whole variety $\mathfrak B$.
\end{theorem}

\begin{proof}
(i) Clearly the algebra $A_{0,1}$ satisfies the polynomial identity $x_1(x_2x_3)=0$. The origins in $F=F({\mathfrak B})$ of the polynomials
$w_{\lambda}^{(j)}$ from (\ref{hwv of W(lambda)}) have the form
\[
w_{(n)}^{(j)}(x_1)=\underbrace{x_1(\cdots (x_1(((x_1}_{j\text{ times}}\underbrace{x_1)\cdots )x_1}_{n-j\text{ times}}))\cdots),
\]
\[
w_{(\lambda_1,\lambda_2)}^{(j)}(x_1,x_2)
=\underbrace{x_1(\cdots x_1}_{j\text{ times}}((\cdots((x_1x_2-x_2x_1)^{\lambda_2}\underbrace{x_1)\cdots )x_1}_{\lambda_1-\lambda_2-j\text{ times}})\cdots).
\]
Obviously $w_{(\lambda_1,\lambda_2)}^{(j)}$ follows from $x_1(x_2x_3)=0$ for $\lambda=(n)$, $j=2,\ldots,n-1$, $n\geq 3$,
for $\lambda=(n-1,1)$, $j=1,\ldots,n-2$, and for $\lambda=(\lambda_1,\lambda_2)$, $\lambda_2\geq 2$.
On the other hand $w_{(n)}^{(1)}(r)=r^2\not=0$, $w_{(n-1,1)}^{(0)}(r,r^2)=-r^2\not=0$. This shows that the identities of $A_{0,1}$ follow from $x_1(x_2x_3)=0$,
$\chi_1(A_{0,1})=\chi_{(1)}$, $\chi_n(A_{0,1})=\chi_{(n)}+\chi_{(n-1,1)}$, $n=2,3,\ldots$, and $c_n(A_{0,1})=n$, $n=1,2,\ldots$. The proof for $A_{1,0}$ is similar.

(ii) By Corollary \ref{generating by free algebra of rank 1} it is sufficient to show that the algebra $A_{1,-1}$ does not satisfy an identity in one variable.
Let
\[
w_{(n)}(y_1,z_1)=\sum_{j=1}^{n-1}\xi_jw_{(n)}^{(j)}(y_1,z_1),\quad \xi_j\in K,
\]
be a polynomial in $G$ which corresponds to a homogeneous polynomial identity $f(x_1)=0$ in one variable and of degree $n\geq 2$,
$0\not=f(x_1)\in F({\mathfrak B})$. We shall evaluate $f(x_1)$ on all $\gamma r+\delta r^2\in A_{1,-1}$, $\gamma,\delta\in K$.
Since
\[
(\gamma r+\delta r^2)^2=(\gamma^2-\delta^2)r^2,
\]
\[
(\gamma r+\delta r^2)\cdot(\gamma r+\delta r^2)^2=(\gamma-\delta)(\gamma^2-\delta^2)r^2,
\]
\[
(\gamma r+\delta r^2)^2\cdot(\gamma r+\delta r^2)=-(\gamma+\delta)(\gamma^2-\delta^2)r^2,
\]
we obtain that the evaluation of the proimage of $w_{(n)}^{(j)}(y_1,z_1)$ on $\gamma r+\delta r^2$ is equal to
\[
(-1)^{n-j-1}(\gamma^2-\delta^2)(\gamma-\delta)^{j-1}(\gamma+\delta)^{n-j-1}r^2=(-1)^{n-1}(\delta-\gamma)^j(\delta+\gamma)^{n-j}.
\]
Hence
\[
f(\gamma r+\delta r^2)=(-1)^{n-1}w_{(n)}(\delta-\gamma,\delta+\gamma)r^2=0.
\]
When $\gamma$ and $\delta$ run on the whole field $K$ the same holds for $\delta-\gamma$ and $\delta+\gamma$. Therefore the polynomial
$w_{(n)}(y_1,z_1)$ vanishes evaluated on the infinite field $K$ and hence is identically equal to 0. This means that
$A_{1,-1}$ does not satisfy any polynomial identity in one variable and hence generates the whole variety $\mathfrak B$.
\end{proof}

The following easy lemma gives an upper bound for the codimensions of a finite dimensional algebra.
It was established in a more general form for graded algebras in \cite{BaD}. We include the proof for self-containedness of the exposition
and to correct the misprint $c_n(A)\leq \dim^n(A)$ instead of $c_n(A)\leq \dim^{n+1}(A)$.

\begin{lemma}\label{growth of codimensions}
If $A$ is a finite dimensional algebra then
\[
c_n(A)\leq \dim^{n+1}(A),\quad n=1,2,\ldots .
\]
\end{lemma}

\begin{proof}
Let $\dim(A)=m$ and let $A$ have a basis $\{r_1,\ldots,r_m\}$. We consider the multilinear identity
\[
f(x_1,\ldots,x_n)=\sum_{(\sigma)}\xi_{(\sigma)}(x_{\sigma(1)}\cdots)(\cdots x_{\sigma(n)})=0,\quad \xi_{(\sigma)}\in K,
\]
where the summation runs on all permutations $\sigma\in S_n$ and all possible bracket decompositions. Clearly, $f(x_1,\ldots,x_n)=0$
is a polynomial identity for $A$ if and only if $f(r_{i_1},\ldots,r_{i_n})=0$ for all possible choices of the basis elements $r_{i_1},\ldots,r_{i_n}$.
Let
\[
f(r_{i_1},\ldots,r_{i_n})=\sum_{j=1}^mf_j(r_{i_1},\ldots,r_{i_n})r_j,
\]
where $f_j(r_{i_1},\ldots,r_{i_n})\in K$ are linear functions in the coefficients $\xi_{(\sigma)}$.
Considering $\xi_{(\sigma)}$ as unknowns, we obtain the linear homogeneous system
\begin{equation}\label{linear system}
f_j(r_{i_1},\ldots,r_{i_n})=0,\quad r_{i_1},\ldots,r_{i_n}\in\{r_1,\ldots,r_m\},j=1,\ldots,m.
\end{equation}
The system has $n!C_n$ unknowns, where $C_n$ is the $n$-th Catalan number (equal to the number of the bracket decompositions).
Since the codimension $c_n(A)$ is equal to the rank of the system and the system has $m^{n+1}$ equations, its rank is less or equal to $m^{n+1}$
and the same holds for the $n$-th codimension $c_n(A)$.
\end{proof}

\begin{remark}
It was shown in \cite{GMZ} that if the two-dimensional algebra $A$ has a one-dimensional nilpotent ideal, then
$c_n(A)\leq n+1$. The algebras $A_{0,1}$ and $A_{1,0}$ satisfy this condition and Theorem \ref{PIs of two-dimensional algebras} (i) shows that
their codimensions are very close to the upper bound. For the algebra $A_{1,-1}$ the results in \cite{GMZ} give that
\[
\frac{2^n}{n^2}\leq c_n(A_{1,-1})\leq 2^{n+1}.
\]
The bound $c_n(A_{1,-1})\leq 2^{n+1}$ can be improved if we consider two-dimensional algebras $A$ with the property that $\dim(A^2)=1$.
In the proof of Lemma \ref{growth of codimensions}, the algebra $A_{1,-1}$ has a basis $\{r,r^2\}$
and the values of $f(r^{i_1},\ldots,r^{i_n})$, $i_k=1,2$, belong to the ideal $A_{1,-1}^2$.
Hence, solving the linear system (\ref{linear system}) we have to follow only the coefficient of $r^2$.
Since the number of the equations is $2^n$, we obtain $c_n(A_{1,-1})\leq 2^n$.
By \cite{DIT} and Theorem \ref{PIs of two-dimensional algebras} (ii) we have that $c_n(A_{1,-1})=c_n({\mathfrak B})=2^n-2$.
Again, this is very close to the upper bound $2^n$.
\end{remark}

\begin{thebibliography}{99}

\bibitem{ADF}
G. Almkvist, W. Dicks, E. Formanek,
{\it Hilbert series of fixed free algebras and noncommutative classical invariant theory},
J. Algebra {\bf 93} (1985), 189-214.

\bibitem{BaD}
Yu. Bahturin, V. Drensky,
{\it Graded polynomial identities of matrices},
Linear Algebra Appl. {\bf 357} (2002), 15-34.

\bibitem{BD}
F. Benanti, V. Drensky,
{\it Defining relations of minimal degree of the trace algebra of $3 \times 3$ matrices},
J. Algebra {\bf 320} (2008), No. 2, 756-782.

\bibitem{B}
A. Berele,
{\it Homogeneous polynomial identities}
Israel J. Math.  {\bf 42}  (1982), No. 3, 258-272.

\bibitem{Ca}
A. Cayley,
{\it On the theory of analytical forms called trees},
Phil. Mag. {\bf 13} (1857), 19-30.
Collected Math. Papers, University Press, Cambridge, Vol. 3, 1890, 242-246.

\bibitem{DEP}
C. De Concini, D. Eisenbud, C. Procesi,
{\it Young diagrams and determinantal varieties}, Invent. Math.
{\bf 56} (1980), 129-165.

\bibitem{D0}
V.S. Drenski,
{\it Representations of the symmetric group and varieties of linear algebras} (Russian),
Matem. Sb. {\bf 115} (1981), 98-115.
Translation: Math. USSR Sb. {\bf 43} (1981), 85-101.

\bibitem{D1}
V. Drensky,
{\it Extremal varieties of algebras. I} (Russian),
Serdica {\bf 13} (1987), 320-332.

\bibitem{D2}
V. Drensky,
{\it Free Algebras and PI-Algebras. Graduate Course in Algebra},
Springer-Verlag, Singapore, 2000.

\bibitem{DZ}
V. Drensky, B.K. Zhakhayev,
{\it Noetherianity and Specht problem
for varieties of bicommutative algebras},
arXiv:1706.02529v1 [math.RA].

\bibitem{DIT}
A.S. Dzhumadil'daev, N.A. Ismailov, K.M. Tulenbaev,
{\it Free bicommutative algebras},
Serdica Math. J. {\bf 37} (2011), No. 1, 25-44.

\bibitem{DT}
A.S. Dzhumadil'daev, K.M. Tulenbaev,
{\it Bicommutative algebras} (Russian),
Usp. Mat. Nauk {\bf 58} (2003), No. 6, 149-150.
Translation: Russ. Math. Surv. {\bf 58} (2003), No. 6, 1196-1197.

\bibitem{GMZ}
A. Giambruno, S. Mishchenko, M. Zaicev,
{\it Codimension growth of two-dimensional non-associative algebras},
Proc. Am. Math. Soc. {\bf 135} (2007), No. 11, 3405-3415.

\bibitem{GZ1}
A. Giambruno, M. Zaicev,
{\it On codimension growth of finitely generated associative algebras},
Adv. Math. {\bf 140} (1998), No. 2, 145-155.

\bibitem{GZ2}
A. Giambruno, M. Zaicev,
{\it On codimension growth of finitely generated associative algebras},
Adv. Math. {\bf 142} (1999), No. 2, 221-243.

\bibitem{GZ3}
A. Giambruno, M. Zaicev,
{\it Codimension growth and minimal superalgebras},
Trans. Amer. Math. Soc. {\bf 355} (2003), No. 12, 5091-5117.

\bibitem{GR}
M. Goze, E. Remm,
{\it 2-dimensional algebras},
Afr. J. Math. Phys. {\bf 10} (2011), No. 1, 81-91.

\bibitem{GG}
I. Grattan-Guinness,
{\it Benjamin Peirce's `Associative Algebra'} (1870){\it : New light on its preparation and `publication'},
Ann. Sci. {\bf 54} (1997), No. 6, 597-606.

\bibitem{K}
P. Koshlukov,
{\it Polynomial identities for a family of simple Jordan algebras},
Commun. Algebra {\bf 16} (1988), 1325-1371.

\bibitem{M}
I.G. Macdonald,
{\it Symmetric Functions and Hall Polynomials},
Oxford Univ. Press (Clarendon), Oxford, 1979. Second Edition, 1995.

\bibitem{Pe}
B. Peirce,
{\it Linear Associative Algebra}, Washington, 1870.
Reprinted: Am. J. Math. {\bf 4} (1881), 99-215, addenda: 215-229 (pp. 225-229 by C.S. Peirce).

\bibitem{P}
H.P. Petersson,
{\it The classification of two-dimensional nonassociative algebras},
Result. Math. {\bf 37} (2000), Nos 1-2, 120-154.

\bibitem{RTS}
M. Rausch de Traubenberg, M. Slupinski,
{\it Polynomial invariants and moduli of generic two-dimensional commutative algebras},
arXiv:1612.05737v1 [math.AC].

\end{thebibliography}

\end{document}




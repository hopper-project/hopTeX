\documentclass[11 pt]{amsart}
\usepackage{epsfig,graphics,amsmath,amssymb,amsthm}
\newtheorem{Theorem}{Theorem}[section]

\newtheorem{Corollary}[Theorem]{Corollary}
\newtheorem{Proposition}[Theorem]{Proposition}
\newtheorem{Lemma}[Theorem]{Lemma}

\theoremstyle{definition}
\newtheorem{Definition}[Theorem]{Definition}
\newtheorem{Example}[Theorem]{Example}
\newtheorem{Remark}[Theorem]{Remark}

\begin{document}

\title[An Algebraic Characterization of a Dehn Twist for Nonorientable
Surfaces] {An algebraic characterization of a Dehn twist for
nonorientable surfaces}

\author{Ferihe Atalan}
\address{Department of Mathematics, Atilim University,
06836 \newline Ankara, TURKEY} \email{fatalan@atilim.edu.tr}
\date{\today}
\thanks{The author is supported by TUBITAK-110T665.}
\subjclass[2010]{20F38, 57N05}\keywords{Mapping class groups, Dehn
twists, nonorientable surfaces} \pagenumbering{arabic}

\begin{abstract} Let $N_g^k$ be a nonorientable surface of
genus \ $g\geq 5$ \ with \ $k$-punctures. In this note, we will give
an algebraic characterization of a Dehn twist about a simple closed
curve on $N_g^k$. Along the way, we will fill some little gaps in
the proofs of some theorems in \cite{A} and \cite{I1} giving
algebraic characterizations of Dehn twists about separating simple
closed curves. Indeed, our results will give an algebraic
characterization for the topological type of Dehn twists about
separating simple closed curves.
\end{abstract}

\maketitle
\section{Introduction}

In this note, $N_{g,r}^k$  will denote the  nonorientable surface of
genus $g$ with $r$ boundary components and $k$ punctures (or
distinguished points). The mapping class group of $N_{g,r}^k$, the
group of isotopy classes of all diffeomorphisms of $N_{g,r}^k$,
where diffeomorphisms and isotopies fix each point on the boundary
is denoted by ${\rm Mod}(N_{g,r}^k)$. If we restrict ourselves to
the diffeomorphisms and isotopies to those which do not permute the
punctures then we obtain the pure mapping class group ${\rm
PMod}(N_{g,r}^k)$. The subgroup ${\rm PMod^{+}}(N_{g,r}^k)$ of the
pure mapping class group consists of the pure mapping classes that
preserve the local orientation around each puncture. Also the twist
subgroup of ${\rm Mod}(N_{g,r}^k)$, generated by Dehn twists about
two-sided simple closed curves is denoted by $\mathcal{T}$.

An algebraic characterization of a Dehn twist plays important role
in the computation of the outer automorphism group of the mapping
class group of a surface (orientable or nonorientable, see \cite{A},
\cite{ASzep} and \cite{I1}). Moreover, it is one of the main tools
in the proof of the fact that any injective endomorphism of the
mapping class group of an orientable surface must be an isomorphism
proved by Ivanov and McCarthy (\cite{I3}). We note that it is also
used in the proof of the fact that any isomorphism between two
finite index subgroups of the extended mapping class group of an
orientable surface is the restriction of an inner automorphism of
this group (\cite{I2}, \cite{K1}). Using an algebraic
characterization of Dehn twists, we show that any automorphism of
the mapping class group of a surface takes Dehn twists to  Dehn
twists.

For orientable surfaces, N. V. Ivanov gave an algebraic
characterization of Dehn twists in \cite{I1}. December in 2012, E.
Irmak reported that the proofs of Ivanov's theorem on algebraic
characterization of Dehn twists about separating simple closed
curves have some gaps (see counter example in Section\,2). For
closed nonorientable surfaces, we gave an algebraic characterization
of Dehn twists in \cite{A}, closely following Ivanov's work
\cite{I1}. Therefore, the algebraic characterization in \cite{A} has
also gaps. In this paper, we will not only give an algebraic
characterization of a Dehn twist about a simple closed curve on \
$N_g^k$ \ but also an algebraic characterization of the topological
type of the curve the Dehn twist is about. In particular, this paper
will fill the gaps, mentioned above, in both \cite{A} and \cite{I1}.

The organization of the paper is as follows: In Section 2 we will
state and prove an algebraic characterization for Dehn twists about
nonseparating curves with nonorientable complements
(Theorem~\ref{Chr-1}). The statement and its proof  are slightly
different than the ones in \cite{A,ASzep} and its proof also
provides a proof for Corollary~\ref{Cor-1.1}. In Section 3, after
proving some preliminary results we will first characterize the Dehn
twists about characteristic curves on a nonorientable surface of
even genus. Then Lemma~\ref{LemChr-2.2} will lead to an  algebraic
characterization of Dehn twists about separating curves
(Theorem~\ref{Chr-2.1}). Moreover, this algebraic characterization
will encode the topological type of the separating curve the Dehn
twist is about.

\section{Preliminaries}\label{Prelim}

Let $S$ denote the surface  $N_{g}^k$ and let $a$ be a simple closed
curve on $S$. If a regular neighborhood of $a$ is an annulus or a
M\"obius strip, then we call $a$ a two-sided or a one-sided simple
closed curve, respectively. The curve $a$ will be called trivial, if
it bounds a disc with at most one puncture or a M\"obius band on $S$
(or if it is isotopic to a boundary component). Otherwise, it is
called nontrivial.

We will denote by  $S^a$ the result of cutting of $S$ along the
simple closed curve $a$. The simple closed curve $a$ is called
nonseparating if $S^a$ is connected. Otherwise, it is called
separating.

Let $H$ be a group. If $G\leq H$ is a subgroup and $h\in H$ is an
element of $H$, then the center of $H$, the centralizer of $G$ in
$H$ and the centralizer of $h$ in $G$ will be denoted by $C(H)$,
$C_H(G)$ and $C_G(h)$, respectively.

Let \ $P: \Sigma_{g-1}^{2k}\rightarrow N_{g}^k$ \ be the orientation double
covering of  \ $N_{g}^k$ \ and \ $\tau:\Sigma_{g-1}^{2k}\rightarrow\Sigma_{g-1}^{2k}$ \ the
Deck transformation, which is an involution.  It is well known that any diffeomorphism \
$f:N_g^k\rightarrow N_g^k$ \ has exactly two lifts to the orientation double
covering and exactly one of them is orientation preserving. Moreover, we can regard
\ ${\rm Mod}(N_g^k)$ \ as the
subgroup \ ${\rm Mod}(\Sigma_{g-1}^{2k})^\tau$, the subgroup of mapping classes which
are invariant under the action of the deck transformation (see also \cite{A}).

Let \ $\Gamma(m)$, where \ $m \in \mathbb{Z}$, $m>1$, be the kernel of
the natural homomorphism $${\rm Mod}(\Sigma_{g-1}^{2k})
\rightarrow {\rm Aut} \,(H_{1}(\Sigma_{g-1}^{2k}, \mathbb{Z} / m
\mathbb{Z})).$$ Then \ $\Gamma(m)$ \ is a subgroup of finite index in \
${\rm Mod}(\Sigma_{g-1}^{2k})$. Let \ $\Gamma'(m)=\Gamma(m) \cap
{\rm Mod}(N_{g,n}^k)$, regarding \ ${\rm Mod}(N_{g,n}^k)$ \ as a
subgroup of \ ${\rm Mod}(\Sigma_{g-1}^{2k})$ \ as described above.

It is well known that \ $\Gamma(m)$ and hence $\Gamma'(m)$ \ consist of
pure elements only provided that \ $m\geq 3$, (\cite{I1}). In this
paper, we will always assume that \ $\Gamma(m)$ for only $m\geq 3$.

Throughout the paper, \ $\Gamma'$ \ will denote a
subgroup of finite index in \ $\Gamma'(m)$.

First, we give the following algebraic characterization of Dehn
twists about nonseparating curves with nonorientable complement.

\begin{Theorem}\label{Chr-1}
Let $S=N_g^k$ be a connected nonorientable surface of genus $g\geq
5$ with $k$ punctures. Let ${\rm M}(S)$ be any of three groups ${\rm
PMod}^+(S)$, ${\rm PMod}(S)$ and ${\rm Mod}(S)$ and let $\Gamma'$ be
as above. An element $f \in {\rm M}(S)$ is a Dehn twist about a
nonseparating simple closed curve with nonorientable complement if
and only if the following conditions are satisfied:\\
\begin{itemize}
\item[i)] $C(C_{\Gamma'}(f^{n})) \cong \mathbb{Z}$, for any integer $n \neq 0$
such that $f^{n} \in \Gamma'$.

\item[ii)] If $g$ is odd (even) there is an abelian
subgroup $K$ of rank $\displaystyle\frac{3g-7}{2} + k$
(respectively, of maximal rank $\displaystyle\frac{3g-8}{2} + k$)
generated by $f$ and its conjugates freely.

\item[iii)] $f$ is a primitive element of $C_{{\rm M}(S)}(K)$.
\end{itemize}
\end{Theorem}

\begin{proof}
We will prove the odd genus case mainly following the proof of
Ivanov in \cite{I1}. The even genus case is basically the same with
minor changes, which we will emphasize whenever needed.

Assume that the above conditions are satisfied, then we have to show
that $f$ is a Dehn twist about a nonseparating circle.

Choose any integer $n\neq 0$ such that $f^{n}\in \Gamma'$. If
$f^{n}$ is equal to the identity, then clearly $C_{\Gamma'}(f^{n}) =
\Gamma'$ and hence by the first condition, $C(\Gamma')$ is
isomorphic to $\mathbb{Z}$. However, this is a contradiction to the
condition $(ii)$. Therefore, $f^{n}$ is not the identity element.

Let $\mathcal{C'}$ be the minimal reduction system for $f^n$. Let
$G$ denote the subgroup generated by the twists about the two-sided
circles in $\mathcal{C'}$. Set $G' = G \cap \Gamma'$. Then $G$ and
$G'$ are free abelian groups. Firstly, we will show that $G' \subset
C(C_{\Gamma'}(f^{n}))$. Let $g \in C_{\Gamma'}(f^{n})$. Since $g$
commutes with  $f^{n}$, it preserves the canonical reduction system
$\mathcal{C'}$. Because $g$ is pure, it fixes each circle of
$\mathcal{C'}$ and also preserves orientation of a regular
neighbourhood of each two-sided circle of $\mathcal{C'}$. It follows
that $g$ commutes with each generator $G$, hence $G \subseteq
C_{{\rm M} (S)}(C_{\Gamma'}(f^{n}))$. So, $G' \subset
C_{\Gamma'}(C_{\Gamma'}(f^{n}))= C(C_{\Gamma'}(f^{n}))$. For the
last equality, we note that the inclusion
$C(C_{\Gamma'}(f^{n}))\subseteq C_{\Gamma'}(C_{\Gamma'}(f^{n}))$ is
clear. For the other inclusion, let $g \in
C_{\Gamma'}(C_{\Gamma'}(f^{n}))$.  Thus, $[g,h]=1$, for all $h \in
C_{\Gamma'}(f^{n})$.  Since $f^{n} \in \Gamma'$ we have $f^n \in
C_{\Gamma'}(f^{n})$ and hence $[g,f^n]=1$. This yields that $g \in
C_{\Gamma'}(f^{n})$. Now it follows that $g \in
C_{\Gamma'}(C_{\Gamma'}(f^{n})) \cap
C_{\Gamma'}(f^{n})=C(C_{\Gamma'}(f^{n}))$.

The assumption that $C(C_{\Gamma'}(f^{n}))=\mathbb{Z}$ yields that
$\mathcal{C'}$ has at most one two-sided circle. Now assume that
$\mathcal{C'}$ has no two-sided circle, so that $\mathcal{C'} =
\{c_{1}, \cdots, c_{l} \}$, where each $c_{i}$ is a one-sided
circle. Then $S^{\mathcal{C'}}$ is connected and the restriction
$f^{n}_{\mid S^{\mathcal{C'}}}$ is either the identity or
pseudo-Anosov.

Let us denote the number $\displaystyle\frac{3g-7}{2}+k$
($\displaystyle\frac{3g-8}{2}+k$, in the even genus case) by $s$. If
$f^{n}_{\mid S^{\mathcal{C'}}}$ is the identity, then $f^{n}$ must
be a product of Dehn twists about some circles in $\mathcal{C'}$
which is not possible since each $c_{i}$ is one-sided. Therefore,
$f^{n}_{\mid S^{\mathcal{C'}}}$ and hence $f_{\mid
S^{\mathcal{C'}}}$ is a pseudo-Anosov diffeomorphism. However, in
this case, the maximal abelian group containing $f$ has rank one. By
the assumption $g\geq 5$ and thus $s\geq 4 > 1$. This is clearly a
contradiction. (Similarly, in the even genus case, we have $s\geq 5
> 1$.) Hence, $\mathcal{C'}$ has exactly one two-sided circle and
$\mathcal{C'}= \{c_{1}, \cdots, c_l, a\}$, where $a$ is a
two-sided circle and each $c_i$ is one-sided.

Let  $D$ be the subgroup generated by $f^{n}$ and the twist about
$a$ and denote the intersection $D \cap \Gamma'$ by $D'$. Hence, $D'
\subset C(C_{\Gamma'}(f^{n}))$ and hence $D'$ is isomorphic to
$\mathbb{Z}$. Now it follows that $f^{n}$ is a power of the Dehn
twist $t_{a}$. In other words, $f^{n} = t_{a}^{m}$ for some integer
$m$.

Next we will show that the circle $a$ is nonseparating. Assume on
the contrary that $a$ is indeed a separating circle. Then, $S^{a} =
S_{1} \cup S_{2}$ such that $\chi_{j}=\chi(S_{j})$ and
$\chi_{1}+\chi_{2}=\chi(S)$. Since $a$ is nontrivial we see that
$\chi_{j} < 0$ for $j=1,2$. Without loss of generality, we can
assume that $\chi_{1} \geq \chi_{2}$.

Let $f_{1}, \cdots, f_s$ denote the elements that are conjugate to
$f$ generating the abelian subgroup $K$ in the statement of the
theorem. In particular, each $f^{n}_{i} = t_{a_{i}}^{m}$, where
$a_{i}$ is a two-sided circle. Since $f_i$ and $f$ are conjugate,
each $a_{i}$ separates the surface into two components, say
$S_{1}^{i}$ and $S_{2}^{i}$, which are diffeomorphic to $S_1$ and
$S_2$, respectively. By the structure of abelian groups, we know
that the circles $a_{i}$ can be chosen to be disjoint and pairwise
nonisotopic. Furthermore, each $S_{1}^{i}$ is diffeomorphic to
$S_{1}^{j}$ and $a_{i}$ is not isotopic to $a_{j}$, for all $i \neq
j$. Hence, $S_1^{i}$ cannot be contained inside the surface
$S_{1}^{j}$.  Therefore, for any $i\neq j$, we have $S_{1}^{i} \cap
S_{1}^{j} = \emptyset$; in other words, the surfaces $S_{1}^{i}$ are
all disjoint. Let $S^{0} = Cl(S \setminus (S_{1}^{1} \cup \cdots
\cup S_{1}^s))$. The fact that $a_{i}$ are being pairwise
nonisotopic yields that $S^{0}$ cannot be an annulus.  Furthermore,
$S^{0}$ is not a sphere, a torus, a Klein bottle, a projective
plane, a disk or a M\"obius band. Thus, $\chi(S^{0}) < 0$.

Now, $2-g-k = \chi(S)= \chi(S^{0}) + \chi(S_{1}^{1}) + \cdots +
\chi(S_{1}^s)$ and hence, $2-g-k < s \chi(S_{1}^{1})$. Then, $g-2+k
> s \mid \chi(S_{1}^{1}) \mid$. Hence, we get $2g+2k-4 >
3g+2k-7$, which implies $3 > g$. This is a clear contradiction.
Thus, $a$ must be a nonseparating circle. (For even genus, in a
similar fashion, we would get $2g+2k-4 > 3g+2k-8$, which would yield
$4 > g$, a contradiction to the assumption that $g\geq 6$.)

Next, we take an element $\psi \in C_{{\rm M}(S)}(K)$ and choose a
diffeomorphism $\Psi : S \to S$ in the isotopy class $\psi$. The
group $K$ is generated by $f_{1}, \cdots, f_s$, which are all
conjugate to $f$ and hence $f^n_{i} = t_{a_{i}}^m$ as above. Since
$\psi \in C_{{\rm M}(S)}(K)$, $\psi f_{i} \psi^{-1} = f_{i}$ and
thus $\psi f_{i}^{n} \psi^{-1} = f_{i}^{n}$. In other words, $\psi
t_{a_{i}}^m \psi^{-1} = t_{a_{i}}^m$ and thus $t_{\Psi(a_{i})}^m =
t_{a_{i}}^m$. Hence, replacing $\Psi$ by an isotopy we may assume
that $\Psi(a_{i})=a_{i}$. Let $\mathcal{\bar{C}} = \{a_{1}, \cdots,
a_s \}$.

Since $s$ is the maximal number of pairwise nonisotopic disjoint
two-sided circles on $S$, we see that every component of the surface
$S^{\mathcal{\bar{C}}}$ is either a once-punctured annulus or a
once-punctured M\"obius band. The total number of the boundary
components of the components of $S^{\mathcal{\bar{C}}}$ is $2s$ and
hence all the components of $S^{\mathcal{\bar{C}}}$ except one must
be pair of pants.

In the even genus case, some extra care is needed. One can easily
see that in the even genus case, there are abelian subgroups of rank
$\displaystyle\frac{3g-8}{2} + k$ and $\displaystyle\frac{3g-6}{2} +
k$, which are maximal in the sense that they are not contained in
abelian subgroups of higher ranks (compare with Lemma 2.7 of
\cite{A}). By requiring that the subgroup $K$ of rank
$\displaystyle\frac{3g-8}{2} + k$ \ to be maximal in this sense we
ensure that all but two of the components of $S^{\mathcal{\bar{C}}}$
are pair of pants and the remaining two components are one-punctured
M\"obius bands. This can be seen easily using an Euler
characteristic argument (see also Remark~\ref{EvenGenusMaxCurv}).

Now suppose that $\Psi$ sends one of these components to another
one. Since $\Psi$ fixes the circles (boundary components) of
$S^\mathcal{\bar{C}}$, we see that $S= \bar P \cup \bar R$ where
$\Psi(P)=R$. However, this is a contradiction to the assumption that
$S$ is a nonorientable surface of odd genus. Hence, $\Psi$ does not
permute the components. It follows that, $\Psi$ is a composition of
twists about the circles $a_{i}$. (Similarly, in the even genus
case, $\Psi$ cannot interchange the components since by assumption
$g\geq 6$.)

Note that $f \in C_{{\rm M}(S)}(K)$ because $f \in K$ and $K$ is
abelian. Without loss of generality, assume that $f=f_{1}$. Then,
$f^{n} = f_{1}^{n} = t_{a_{1}}^{m}$. Now by the above paragraph, we
deduce that $f$ is a power of $t_{a_{1}}$; $f=t_{a_{1}}^{l}$, for
some $l$, because $K$ is free abelian. Therefore,
$f_{i}=t_{a_i}^{l}$ and $K$ is generated by $t_{a_{i}}^{l}$. In
particular, $t_{a}$ is in $C_{{\rm M}(S)}(K)$. Finally, by the last
condition, $f$ is primitive and hence, $f = t_{a_1}=t_a$. This
finishes the proof for surfaces of odd genera.

In the even genus case, one has to prove also that the complement of
$a$ is nonorientable.  Assume on the contrary that the complement of
$a$ is orientable. Since each $t_{a_i}$ is conjugate to $t_a$ the
complement of each $a_i$ is also orientable. It follows that each
$a_i$ represents the ${\mathbb Z}_2$-homology cycle Poinc\'are dual
to the first Stiefel-Whitney class $\omega_1$ of the surface $S$. In
particular, for any $1\leq i,j \leq s$, $i\neq j$, the circles $a_i$
and $a_j$ form the boundary of a subsurface $S_{i,j}$ of $S$.
Moreover, since the complement of each $a_i$ is orientable,
$S_{i,j}$ is an orientable surface of genus at least one with at
least two boundary components. By relabeling $a_i$'s if necessary,
we may assume that the interior of each subsurface $S_{i,i+1}$
(where we set $S_{s,s+1}=S_{s,1}$) does not intersect any $a_k$.
Hence, $S_{i,i+1}\cap S_{j,j+1}=\emptyset$, if and only if
$|i-j|>1$.  Now, $S=\cup_{1\geq i \geq s}S_{i,i+1}$, and thus,
computing the Euler characteristics of both sides, we see that
$$2-g-k=\chi(S)=\sum_{i=1}^s \chi(S_{i,i+1})\leq -2s=-3g+8-2k ,$$
which yields the inequality $6\geq 2g+k $, a contradiction since
$g\geq 6$ by the assumption. Hence, we conclude that the complement of $a$ must be
nonorientable.

The other direction of the theorem is straight forward and left to
the reader (see also \cite{I1}).
\end{proof}

\begin{Remark}\label{EvenGenusMaxCurv} In the mapping class groups of
even genus nonorientable surfaces there are  abelian subgroups of
different maximal ranks. This with one example is illustrated in
\cite{ASzep}.
\end{Remark}

Before we go on characterizing other Dehn twists algebraically, we
state a corollary of the proof the above theorem.

\begin{Corollary}\label{Cor-1.1} Let $S$ be the surface in the above theorem and
$f \in {\rm M}(S)$ an element. Then there are integers $m>0$, $N$
and a simple closed curve $c$ so that $f^N=t_c^m$ if and only if the
following conditions are satisfied:

\begin{itemize}

\item[i)] $C(C_{\Gamma'}(f^{n})) \cong \mathbb{Z}$, for any integer
$n \neq 0$ such that $f^{n} \in \Gamma'$.

\item[ii)] There exists a free abelian subgroup $K$
of ${\rm M}(S)$ generated by $f$ and $\displaystyle\frac{3g-9}{2}
+k$, when $g$ is odd, (respectively, $\displaystyle\frac{3g-10}{2}
+k$, when $g$ is even), twists about two-sided nonseparating simple
closed curves such that $rank \ (K)=\displaystyle\frac{3g-7}{2} +
k$, when $g$ is odd, (respectively, $\displaystyle\frac{3g-8}{2}
+k$, when $g$ is even).
\end{itemize}
\end{Corollary}

Hence by the above theorem and the corollary, we have algebraically
characterized, in all the three groups ${\rm PMod}^+(S)$, ${\rm
PMod}(S)$ and ${\rm Mod}(S)$ of odd genus surfaces, the Dehn twists
about nonseparating simple closed curves (respectively in the even
genus case, the Dehn twists about nonseparating simple closed curves
with nonorientable complements) and those elements $f \in {\rm
M}(S)$, whose some power is a positive power of a Dehn twist about a
nontrivial two-sided simple closed curve.

\subsection{A Counter Example.} The situation for separating curves is more involved.
First we remark that Theorem 3.2 and Theorem 3.3 of \cite{A} are not
correct as they are stated. E. Irmak informed us about the following
counter example constructed by L. Paris: Let \ $c$ \ be a separating
curve in a surface $S$ so that one of the two components of $S^c$ is
a torus with one boundary component $\Sigma_{1,1}$ as in the
Figure~\ref{oneholedsurfaces}. Consider the mapping class
$\tau=(t_at_b)^3$, where $a$ and $b$ are the curves given in the
same figure.  Let $K$ be an abelian subgroup as in the statement of
Theorem 3.2 or Theorem 3.3 of \cite{A} for the Dehn twist $t_c$.
Since $\tau$ \ commutes with both $t_a$ and $t_b$, $\tau$ is in
$C_{{\rm M}(S)}(K)$. However, $t_c=\tau^2$ \ and hence $t_c$ is not
a primitive element in $C_{{\rm M}(S)}(K)$. Note that this example
exists since the center of the mapping class group of $\Sigma_{1,1}$
is not trivial.  Indeed it is isomorphic to the infinite cyclic
group generated by $\tau$. There are two more cases that might cause
similar problem. The first one is $\Sigma_{1,1}^1$, \ whose mapping
class group is isomorphic to that of $\Sigma_{1,1}$. The final
problematic case is the surface $N_{1,1}^1$ \ whose mapping class
group is the infinite cyclic group generated by the class of $v$,
the class of the puncture slide diffeomorphism
(Figure~\ref{oneholedsurfaces}). In this case, $v^2$ \ is the Dehn
twist about the boundary curve $c$.

\begin{Remark}\label{IvnovThm2.2}
The example we have just described above, also provides a counter
example for Theorem 2.2 of \cite{I1}, a result for algebraic
characterization for Dehn twists (mainly, about separating curves)
in orientable surfaces. The methods we will provide here, which work
for both orientable and nonorientable surfaces, not only provide an
algebraic characterization for Dehn twists about separating curves
but also for the topological type of the separating curve the Dehn
twist is about (see Remark~\ref{Rem-GeneralDehnTwistCh}).
\end{Remark}

\begin{figure}[hbt]
 \begin{center}
 \includegraphics[width=11cm]{oneholedsurfaces.eps}
\caption {} \label{oneholedsurfaces}
\end{center}
\end{figure}

\section{Preparation for the algebraic characterization of Dehn
twist about separating curves}

We start with the following technical result which we will use
later.

\begin{Proposition}\label{Prop-MaxAbelSub}
Let $S=N_g^k$ be a nonorientable surface of even genus. Then for any
integer $s=0,\cdots, \frac{g-2}{2}$, the group ${\rm M}(S)$ has an
abelian subgroup of rank $\displaystyle\frac{3g-6-2s}{2} +
k$, which is freely generated by Dehn twists about pairwise
nonisotopic nonseparating circles, and so that no abelian subgroup
containing this subgroup has bigger rank. Moreover, when we cut the surface
along these circles, the resulting surface is a disjoint union of
$g+k-2s-2$ many pair of pants and $2s$ many two holed real
projective planes.
\end{Proposition}

\begin{proof}
Let $r=\displaystyle\frac{3g-6-2s}{2} + k$. By the
maximality condition of the proposition, all the components of
the surface cut along these \ $r$ \
circles have Euler characteristic $-1$.  In other words, each
component is either a pair of pants or a two holed real projective
plane.  Assume that there are $l$ many pair of pants and $m$ many
two holed real projective planes. Hence, considering Euler
characteristics we obtain the equation
$$2-g-k=\chi (S)=\chi (\coprod_l N_0^3 \cup \coprod_m N_1^2) =-l-m \ .$$
On the other hand, counting the number of punctures, we obtain
$$3g-6-2s+3k= 3l+2m \ .$$ These two equations yield $m=2s$ and
$l=g+k-2s-2$ as desired. Finally, the existence of such subgroups is
readily seen by inspection.
\end{proof}

\bigskip

A sequence of Dehn twists $t_{a_1},\cdots,t_{a_n},$ is called a
chain if the following geometric intersection $i(a_i,a_{i+1})=1$,
for $i=1,\cdots,n-1$. The integer $n\geq 1$ is called the length of
the chain. Note that if a chain has length more than one then each
$a_i$ must be nonseparating (and has nonorientable complement, if
$S$ is nonorientable of even genus). For a tree or chain of Dehn
twists we always fix an orientation for a tubular neighborhood of
the union of simple closed curves (which is always orientable) and
consider Dehn twists using this orientation. It is known that two
Dehn twists $t_a, \ t_b$ satisfy the braid relation \
$t_at_bt_a=t_bt_at_b$ \ if and only if $i(a,b)=1$ on nonorientable
surfaces (see \cite{SM1}). Hence, by the above results any
automorphism $\Psi:{\rm M}(S)\rightarrow {\rm M}(S)$ maps a chain of
Dehn twists of length at least two to another chain of Dehn twists
of the same length.  In this note, unless we state otherwise a chain
or a tree in a nonorientable surface $S$ will mean a chain or a tree
of Dehn twists about nonseparating two-sided simple closed curves
with nonorientable complement.

Below we will give a generalization of Lemma 3.7 of \cite{A} to
punctured surfaces. We will include the proof since the one in
\cite{A} has a gap, indicated by B. Szepietowski.

\begin{Lemma}\label{Lem-SepChains}
Let $S=N_g^k$ be a nonorientable surface of genus $g\geq 5$. Then
the image of a disc separating chain under an automorphsim of ${\rm
M}(S)$ is again a chain which separates a disc.
\end{Lemma}

\begin{proof} If the genus of the surface is odd then a chain is
separating if and only if it is maximal in ${\rm M}(S)$.  However,
being maximal is clearly preserved under an automorphism. Now Lemma
3.5 of \cite{A} finishes the proof.

Now assume that the genus $g\geq 6$ is an even integer. Further
assume that $c_1,\cdots, c_{2l+1}$ is a disc separating chain in
${\rm M}(S)$. Hence, when we delete a tubular neighborhood of the
chain from the surface we obtain a disjoint union of a disc and a
nonorientable surface, call $S_0$, of genus $g-2l$ with $k$
punctures and one boundary component. By Euler characteristic
calculation and inspection we see that the group ${\rm M}(S_0)$ has
a abelian subgroup $K$ of rank $\displaystyle\frac{3(g-2l)-6}{2} +
k+2$, contained in each $C_{{\rm M}(S)}(t_{c_i})$, which is freely
generated by Dehn twists about pairwise nonisotopic circles.

Now suppose on the contrary that the image $d_1,\cdots, d_{2l+1}$ of
the chain $c_1,\cdots, c_{2l+1}$ under an automorphism is not
separating. So by Lemma 3.5 of \cite{A} the complement of a tubular
neighborhood of the chain $d_1,\cdots, d_{2l+1}$ in $S$ is an
orientable surface of genus $\displaystyle\frac{g-2l}{2}-1$ with $k$
punctures and two boundary components, say $c_1$ and $c_2$. Let us
call this surface $S_1$. (Since the surface $S$ is nonorientable the
circles $c_1$ and $c_2$ are both characteristic in the surface $S$.)
The image of the abelian subgroup $K$ under the same automorphism is
again a maximal subgroup in ${\rm M}(S)$ and it lies in each
$C_{{\rm M}(S)}(t_{d_i})$, $i=1,\cdots, 2l+1$. However, the
orientable subsurface $S_1$ can support an abelian subgroup $K_0$ in
${\rm M}(S)$, which lies in each $C_{{\rm M}(S)}(t_{d_i})$, of rank
at most $\mbox{rank}(K)-1$. Some of the generators of both groups
are Dehn twists about characteristic or separating curves. However,
by Corollary~\ref{Cor-1.1} some powers of these generators are
preserved under automorphisms. This finishes the proof.
\end{proof}

Now we will characterize algebraically a separating pair of Dehn
twists about some two-sided simple closed curves, each of which is
nonseparating with nonorientable complement (so that together they
separate the surface).

\begin{Lemma}\label{Separating Pair Characterization}
Let $S=N_g^k$ be a nonorientable surface of genus $g\geq 5$ and
$a_1$ and $a_2$ be disjoint, nonisotopic, nonseparating two-sided
simple closed curves with nonorientable complements. Then $a_1$ and
$a_2$ together separate the surface if and only if the following
conditions are satisfied:
\begin{enumerate}
\item For any Dehn twist $t_b$ satisfying the braid relation
$t_{a_1}t_bt_{a_1}=t_bt_{a_1}t_b$ we have $t_b\not\in C_{{\rm
M}(S)}(t_{a_2})$;
\item In the even genus case, the twists $t_{a_1}$ and $t_{a_2}$ are
contained in a free generating set, whose elements are all
nonseparating circles with nonorientable complement, of a maximal
abelian subgroup $K$ in ${\rm M}(S)$, of rank
$r=\frac{3g-6-2s}{2}+k$, where $s=1$ or $s=2$.

Moreover, if the conditions of lemma are satisfied, then $s=2$ if
both components of the surface $S$ cut along the curves $a_1$ and
$a_2$ are nonorientable of even genus, and $s=1$ otherwise.
\end{enumerate}
\end{Lemma}

\begin{proof}
Suppose first that $a_1$ and $a_2$ separate the surface. In
particular, the homology classes $[a_1]$ and $[a_2]$ are equal in
$H_1(S,{\mathbb Z}_2)$ and thus the first condition is trivially
satisfied. It is easy to construct the required free abelian
subgroup $K$, in case the genus is even.

For the other direction assume that the conditions of the lemma are
satisfied, but on the contrary suppose that the surface cut along
the curves $a_1$ and $a_2$ is connected.  Hence, the surface $S$ cut
along only $a_2$, say $S_0$, is connected.

First we will treat the case where the genus $g$ is an odd integer.
Hence, the surface $S_0$ is a connected nonorientable surface of
genus at least 3. Moreover, the curve $a_1$ is still nonseparating
in $S_0$. Hence, there is two-sided curve $b$ in $S_0$, whose
geometric intersection with $a_1$ is one. This is a contradiction to
the first condition of the assumption.

Now let us consider the even genus case.  By
Proposition~\ref{Prop-MaxAbelSub} the surface cut along the $r$ many
curves, about which the Dehn twists generate the subgroup $K$, has
nonorientable components. Thus the surface $S_0$ is a connected
nonorientable surface of genus at least four, and the curve $a_1$ is
neither separating nor characteristic in $S_0$. Hence, as above
there is a two-sided curve $b$ in $S_0$, whose geometric
intersection with $a_1$ is one. This finishes the proof for the even
genus case.

The final part of the lemma is an immediate consequence of
Proposition~\ref{Prop-MaxAbelSub}.
\end{proof}

Let $a_1,a_2,a_3$ be distinct, nonisotopic, nonseparating two-sided
simple closed curves with nonorientable complements. We say that
they form a triangle if each geometric intersection $i(a_i,a_j)=1$,
for all $i\neq j$ (see \cite{ASzep}).  A triangle is called
orientation preserving if there are Dehn twists about these circles,
naturally denoted by $t_{a_1}$, $t_{a_2}$ and $t_{a_3}$, so that any
two satisfy the braid relation. (Note that on a nonorientable
surface we have exactly two Dehn twists about any two-sided circle
$a$, which we may denote by $t_a$ and $t_a^{-1}$.) Otherwise, we
call the triangle orientation reversing.

The following result is proved in \cite{ASzep}.

\begin{Lemma}\label{Punctured Annulus}
The above triangle of Dehn twists is orientation preserving if and
only if the union of these curves has an orientable regular
neighborhood. Moreover, a regular neighborhood of a nonorientable
triangle is $N_{4,1}$, a genus four nonorientable surface with one
boundary component.
\end{Lemma}

We will call a separating pair of Dehn twists $\{t_{a_1},t_{a_2}\}$
is of type I if there is no orientation reversing triangle in the
intersection of centralizers $$C_{{\rm M}(S)}(t_{a_1})\cap C_{{\rm
M}(S)}(t_{a_2}) \ .$$ Otherwise, we call it a type II pair. Now we
prove as a consequence of what we have proved so far, an algebraic
characterization of type I pairs of Dehn twists on nonorientable
surfaces.

\begin{Corollary}\label{Characterization of Type I}
Let $S$ be a nonorientable surface of genus at least five and
$\{t_{a_1},t_{a_2}\}$ be a separating pair of Dehn twists on $S$
about disjoint, nonisotopic, nonseparating two-sided simple closed
curves with nonorientable complements and  $\Psi:{\rm
M}(S)\rightarrow {\rm M}(S)$ be any automorphism. If this pair is of
type I then so is $\{\Psi(t_{a_1}),\Psi(t_{a_2})\}$.
\end{Corollary}

\begin{proof}  Assume on the contrary that the pair
$\{\Psi(t_{a_1}),\Psi(t_{a_2})\}$ is of type II. Hence, there is
orientation reversing triangle contained in $C_{{\rm
M}(S)}(t_{a_1})\cap C_{{\rm M}(S)}(t_{a_2}) \ $ so that its image
under $\Psi$ is also orientation reversing. It follows that there
are two Dehn twists about two-sided nonseparating simple closed
curves (necessarily meeting transversally at one point), say $t_x$
and $t_y$, so that we have both $t_xt_yt_x=t_yt_xt_y$ and
$t^{-1}_xt_yt^{-1}_x=t_yt^{-1}_xt_y$. However, this is not possible.
Indeed, multiplying these two relations side by side we get
$t_xt_y^2t^{-1}_x=t_yt_xt_y^2t^{-1}_xt_y$. Letting
$t_z=t_xt_yt^{-1}_x=t_{t_x(y)}$, we obtain $t_z^2=t_yt_z^2t_y$,
where the curve $z=t_x(y)$ intersects $y$ transversally at one
point. We may as well assume that the curves $y$ and $z$ are lying
in a torus. Therefore, we have the relation $(t_z^2t_y)^4=1$ (cf.
page 82 of \cite{F-M}). Combining this with the relation
$t_z^2=t_yt_z^2t_y$ we get,
$$1=(t_z^2t_y)^4=t_z^2(t_yt_z^2t_y)t_z^2(t_yt_z^2t_y)=t_z^8 \ ,$$ a
clear contradiction. This finishes the proof.
\end{proof}

Let $t_a$  be a Dehn twist about a separating simple closed curve
$a$ in a nonorientable surface $S$. Maximal chains contained in the
centralizer $C_{{\rm M}(S)}(t_a)$ correspond to maximal chains of
Dehn twists about two-sided simple closed curves (nonseparating with
nonorientable complement) lying in different components of $S^a$.
The lengths of these maximal chains determine the topological type
the curve $a$ up to a great extend, however fail to characterize its
topological type completely. A more powerful tool is to use maximal
trees contained in the centralizers. By a maximal tree of Dehn
twists in a surface nonorientable $S$ we will mean a connected
maximal tree of Dehn twists about pairwise nonisotopic nonseparating
two-sided simple closed curves with nonorientable complements. If
$S$ is orientable we will only require that the curves in the tree
to be nonseparating.

\begin{Remark}\label{Remark-MaxChainInTree vs In Surface}
As it is seen in the figure below a maximal chain in a maximal tree
need not to be a maximal chain in the surface. Note that the chain
of circles $1,2,\cdots,7$ is maximal in the tree but not in the
surface, which contains the longer chain $1,2,\cdots,7,8,9$.

\begin{figure}[hbt]
 \begin{center}
 \includegraphics[width=6cm]{maxtree.eps}
\caption {} \label{maxtree}
\end{center}
\end{figure}
\end{Remark}

The trees below will be useful for the rest of the paper:
$T_{2g+1,1}^k$,  $T_{2g+2,1}^k$ and $OT_{g,1}^k$.

\begin{figure}[hbt]
 \begin{center}
 \includegraphics[width=8cm]{oneholedtree1.eps}
\caption {} \label{oneholedtree1}
\end{center}
\end{figure}

\begin{figure}[hbt]
 \begin{center}
 \includegraphics[width=8cm]{oneholedtree2.eps}
\caption {} \label{oneholedtree2}
\end{center}
\end{figure}

\begin{figure}[hbt]
 \begin{center}
 \includegraphics[width=8cm]{oneholedtree.eps}
\caption {} \label{oneholedtree}
\end{center}
\end{figure}

We may endow an embedding of one of these trees into the group ${\rm
M}(S)$, where $S$ is a nonorientable surface of genus at least five,
with a coloring of its vertices.  For example, the tree
$T_{2g+1,1}^k$ in Figure~\ref{oneholedtree1}  has the vertices
$a_{4g-3}$ and $a_{4g}$ colored. This will mean that any orientation
reversing triangle in ${\rm M}(S)$, whose vertices commute with the
colored vertices also commute with all the vertices of the tree.
Equivalently, any orientation reversing triangle which lies in
$$C_{{\rm M}(S)}(t_{a_{4g-3}})\cap C_{{\rm M}(S)}(t_{a_{4g}}) \ ,$$
also lies in $C_{{\rm M}(S)}(t_{a})$, for all vertices $a$ of the
tree.

For nonorientable surfaces of even genera, first we present an
algebraic characterization of a Dehn twist about a two-sided simple
closed curve, whose complement is orientable.

\begin{Lemma}\label{Lem-compOrien}
Let $g\geq 2, \ k\geq 0$ be integers and $c$ be a nontrivial
two-sided simple closed curve in $S=N_{2g+2}^k$. Then $S^c$ is
orientable if and only if the colored tree $NT_{2g+2}^k$ (see
Figure\,\ref{tree}) can be embedded in the centralizer $C_{{\rm
M}(S)}(t_c)$ as a maximal tree, where
\begin{itemize}
\item[1)] Each maximal chain in the tree is a maximal chain in ${\rm M}(S)$;

\item[2)] The tree $T$ has a chain with length larger than or equal to any
chain in the centralizer $C_{{\rm M}(S)}(t_c)$;

\item[3)] Any two vertices connected to $a_{4g-3}$, both different than
$a_{4g-4}$, form a separating pair.
\end{itemize}

\end{Lemma}

\begin{proof}
One direction is clear. Now assume that $NT_{2g+2}^{k}$ lies in the
centralizer $C_{{\rm M}(S)}(t_c)$. We claim that the surface $S$ cut
along the curves $a_1$, $a_2$ and $a_3$ has two components one of
which is an orientable surface of genus $g-1$ with $k$ punctures and
one boundary component: To see this, consider a tubular neighborhood
of the tree $NT_{2g+2}^k$ with the vertex $a_0$ deleted. This is an
orientable surface of genus $g$ with $1+k+2(g-1)$ boundary
components.  By the maximality of the tree, $2g-2$ many of these
components must bound discs. To illustrate this, consider, for
example, the maximal chain $a_1,a_2,a_4,a_6,a_5$. The boundary
component corresponding to this chain should bound a disc, because
otherwise the tree would not be maximal (we could attach another
two-sided simple closed curve to $a_2$). Note also that by the
condition ({\it 3}) of the hypothesis of the lemma the $k$ pairs
$(t_{a_{4g-2}},t_{a_{4g-1}})$, $(t_{a_{4g-1}},t_{a_{4g}})$,
$\cdots$, $(t_{a_{4g+k-3}},t_{a_{4g+k-2}})$ on the right corner of
the tree are all bounding. By maximality and the coloring they must
all bound punctured annuli. This finishes the proof of the claim.

Now, if we attach a tubular neighborhood of the circle $a_0$ to this
subsurface we obtain another subsurface, call $S_0$, of genus $g$
with $k$ punctures and two boundary components. Note that the two
boundary components of $S_0$ should be glued so that the resulting
surface would be nonorientable of genus $2g+2$ with $k$ punctures.
Finally, $c$ being disjoint from each vertex of the tree implies
that $c$ lies in $S-\cup_{i\geq1} a_i$, \ which is a disjoint union
of $2g$ discs, $k$ punctured annuli and a one holed Klein bottle. In
particular, up to isotopy, $c$ is the unique nontrivial two-sided
simple closed curve in this punctured Klein bottle.  This finishes
the proof.

\begin{figure}[hbt]
 \begin{center}
 \includegraphics[width=8cm]{tree.eps}
\caption {} \label{tree}
\end{center}
\end{figure}
\end{proof}

We note that the above lemma, Corollary~\ref{Cor-1.1} and the ideas
in Theorem~\ref{Chr-1} yield the following algebraic
characterization for Dehn twists about nonseparating simple closed
curves with orientable complements on nonorientable surfaces of even
genus.

\begin{Corollary}\label{Cor-CompOrient}
Let $g\geq 2, \ k\geq 0$ be integers $f$ an mapping class in ${\rm
M}(S)$ such that $f^N=t_c^m$ for some integers $m>0$, $N$ and a
nontrivial simple closed curve $c$ on $S=N_{2g+2}^k$. Then $c$ is a
characteristic curve (i.e., its complement $S^c$ is orientable) and
$f=t_c$ if and only if $f$ is a primitive element (as in the
statement of Theorem~\ref{Chr-1}) and the colored tree $NT_{2g+2}^k$
(see Figure\,\ref{tree}) can be embedded in the centralizer $C_{{\rm
M}(S)}(t_c)$ as a maximal tree, where
\begin{itemize}
\item[1)] Each maximal chain in the tree is a maximal chain in ${\rm M}(S)$;

\item[2)] The tree $T$ has a chain with length larger than or equal to any
chain in the centralizer $C_{{\rm M}(S)}(t_c)$;

\item[3)] Any two vertices connected to $a_{4g-3}$, both different than
$a_{4g-4}$, form a separating pair.
\end{itemize}

\end{Corollary}

\bigskip
\noindent

For separating Dehn twists characterization we will make use of the
following lemma.

\begin{Lemma}\label{LemChr-2.2} Let $g$ be a positive integer and
$T$ be one of the colored trees $T_{2g+1,1}^k$, $T_{2g+2,1}^k$ or
$OT_{g,1}^k$, embedded in the group ${\rm M}(S)$, where $S$ is a
nonorientable surface of genus at least five. Suppose that $c$ is a
nontrivial separating simple closed curve in $S$ and the tree $T$
lies in the centralizer $C_{{\rm M}(S)}(t_c)$ as a maximal tree.
Moreover, assume the followings:

\begin{itemize}
\item[1)] Each maximal chain in the tree is a maximal chain in ${\rm M}(S)$;

\item[2)] If the tree $T$ is \ $T_{2g+1,1}^k$ \ or \ $T_{2g+2,1}^k$, then it
has a chain with length larger than or equal to any chain in the
centralizer $C_{{\rm M}(S)}(t_c)$.

\item[3)] Any two vertices connected to $a_{4g-2}$, except $a_{4g-4}$, form a
separating pair.

\item[4a)] If $T=T_{2g+1,1}^k$ then $g\geq 2$, and if $T=T_{2g+2,1}^k$ then
$g\geq 1$. Moreover, in both cases, there is an orientation
reversing triangle in $C_{{\rm M}(S)}(t_c)$, which is not contained
in $$\bigcap_{a_i\in V(T)}C_{{\rm M}(S)}(t_{a_i}),$$ where $V(T)$ is
the set of vertices of $T$;

\item[4b)] If $T=OT_{g,1}^k$ then $g\geq 2$, and any orientation reversing
triangle in ${\rm M}(S)$ also lies in
$$C_{{\rm M}(S)}(t_{a_{4g-3}})\cap C_{{\rm M}(S)}(t_{a_{4g-1}}) \
.$$
\end{itemize}

\noindent Then $S^c$ has a component homeomorphic to $N_{2g+1,1}^k$,
$N_{2g+2,1}^k$ or $\Sigma_{g,1}^k$, respectively.
\end{Lemma}

\begin{proof}
\noindent {\it Case 1: $T=T_{2g+1,1}^k$}. First we assume that $S$
is of odd genus. For each Dehn twist belonging the tree choose a
two-sided nonseparating simple closed curve $a_i$ with nonorientable
complement.  Let $S_0$ be the closure of a tubular neighborhood of
the tree of the curves $a_i$. Then $S_0$ is an orientable subsurface
of $S$ with Euler characteristic $\chi(S_0)=2-4g-k$ and with $2g+k$
boundary components. So $S_0$ is an orientable surface of genus $g$
with $2g+k$ boundary components. By the definition of maximal tree
each maximal chain in $T$ is a maximal chain in the surface $S$.
Moreover, $S$ has odd genus and thus each maximal chain contained in
$T$ separates the surface (this is the only place we use the
assumption that $S$ is of odd genus). Hence, each boundary component
bounds either a disc, a once punctured disc, an annulus or a
M\"obius band. Again by maximality of the tree, $2g-2$ many of these
components must bound discs. To illustrate this, consider for
example, the maximal chain $a_1,a_2,a_4,a_6,a_5$. The boundary
component corresponding to this chain should bound a disc, because
otherwise the tree would not be maximal (we could attach another
two-sided simple closed curve to $a_2$).

Note that by the condition ({\it 3}) of the hypothesis of the lemma
the $k+1$ pairs $(t_{a_{4g-3}},t_{a_{4g}})$,
$(t_{a_{4g}},t_{a_{4g+1}})$, $\cdots$,
$(t_{a_{4g+k-1}},t_{a_{4g-1}})$ on the right corner of the tree are
all bounding.

The condition ({\it 2}) of the hypothesis of the lemma implies that
at most two of these pairs may bound a projective plane with two
boundary components. On the other hand, the condition (4a) implies
that at least one of them bounds a projective plane with two
boundary components.  Finally, the coloring of the vertices implies
that boundary components corresponding to the chain
$a_{4g-3},a_{4g}$ is the only pair that bounds a projective plane
with two boundary components. Hence, the other $k$ bounding pairs
must bound punctured annuli.

By attaching $2g-2$ discs and $k$ punctured discs and a M\"obius
band to $S_0$ we get a nonorientable surface, say $S_1$, of genus
$2g+1$ with one boundary component. $S_1$ is contained as a
subsurface in one of the two components of $S^c=S_2 \cup S_3$, say
in $S_2$. Since $S_2$ has only one boundary component, the boundary
component corresponding to the chain \ $a_1,a_2,a_3$ \ must be
parallel to the boundary of $S_2$. Hence, we are done in the odd
genus case.

Now let us consider the case where $S$ has even genus. If each
maximal chain in $T$ is separating in $S$ then the above proof works
in this case as well. Now we will show that any maximal chain in $T$
is indeed separating in $S$. To prove this, assume that there is a
maximal chain $t_{c_1},\cdots,t_{c_{2l+1}}$ in $T$, and thus in
${\rm M}(S)$, so that the chain of two-sided simple closed curves
$c_1,\cdots,c_{2l+1}$ is not separating in $S$. A tubular
neighborhood $\nu$ of the union of these curves is an orientable
surface of genus $l$ with two boundary components. Since the chain
is maximal in ${\rm M}(S)$ we see that the surface $S-Int(\nu)$ is
an annulus, possibly with punctures, so that when the orientable
surfaces $\nu$ and $S-Int(\nu)$ are glued along the two boundary
components, the resulting surface $S$ is nonorientable (see
Figure\,\ref{2holedsurface}) with genus $2l+2$. Since the separating
curve $c$ is disjoint from the tree and thus from the chain, we see
that $c$ lies in the annulus and bounds a (at least twice) punctured
disc.

\begin{figure}[hbt]
 \begin{center}
 \includegraphics[width=8cm]{2holedsurface.eps}
\caption {} \label{2holedsurface}
\end{center}
\end{figure}

So the topological type of $c$ is determined up to the number of
punctures of $S$ contained in either sides of $c$. On the other
hand, since the surface $S$ has even genus, exactly two of the
bounding pairs  \ $(t_{a_{4g-3}},t_{a_{4g}})$,
$(t_{a_{4g}},t_{a_{4g+1}})$, $\cdots$,
$(t_{a_{4g+k-1}},t_{a_{4g-1}})$ bound a projective plane with two
boundary components. On the other hand, the coloring of the vertices
implies that $a_{4g-3},a_{4g}$ is the only pair bound a projective
plane with two boundary components. This gives the desired
contradiction. Hence the proof finishes in the case where
$T=T_{2g+1,1}^k$.
\bigskip

\noindent {\it Case 2: $T=T_{2g+2,1}^k$}. Now first assume that the
surface $S$ has odd genus. We proceed analogous to the previous case
and arrive at the point, where the $k+2$ pairs
$(t_{a_{4g-3}},t_{a_{4g}})$, $(t_{a_{4g}},t_{a_{4g+1}})$, $\cdots$,
$(t_{a_{4g+k}},t_{a_{4g-1}})$ on the right corner of the tree are
all bounding. Again the condition ({\it 2}) of the hypothesis of the
lemma implies that at most two of these pairs bound a projective
plane with two boundary components and the others will bound
punctured annuli. Then condition (4a) and the coloring of the
vertices imply that exactly the two pairs
$(t_{a_{4g-3}},t_{a_{4g}})$ and $(t_{a_{4g}},t_{a_{4g+1}})$ bound
projective plane with two boundary components and all other pairs
bound punctured annuli. This finishes the proof for the odd genus
case.

The even genus case is again analogous to that in Case $1$. We just
need to show that any maximal chain in the tree is separating. We
proceed as in case $T=T_{2g+1,1}^k$. Without loss of generality we
may assume that $c_{2l+1}$ belongs to the set
$$\{a_{4g-3},a_{4g},a_{4g+1},\cdots,a_{4g+k},a_{4g-1}\} \ .$$
The condition ({\it 2}) of the hypothesis of the lemma implies that
the $k+2$ pairs $(t_{a_{4g-3}},t_{a_{4g}})$,
$(t_{a_{4g}},t_{a_{4g+1}})$, $\cdots$, $(t_{a_{4g+k}},t_{a_{4g-1}})$
on the right corner of the tree are all bounding punctured annuli.
However, this contradicts to the maximality of the tree, since in
this case we may add two more vertices to the tree as in
Figure\,\ref{oneholedtree2}. So we are done in this case too.

\bigskip
\noindent{\it Case 3: $T=OT_{g,1}^k$}. This case is easy now,
because by condition (4b) all the bounding pairs on the right side
of the tree will bound punctured annuli (see the paragraph below
Remark~\ref{Remark-MaxChainInTree vs In Surface}). It follows that
the surface $S_1$, in this case, will be an orientable surface of
genus $g$ with one boundary component and with $k$ punctures
(compare with the surface $S_1$ we obtained in the case
$T=T_{2g+1,1}^k$ above).
\end{proof}

\begin{Remark} {\bf 1)} One needs to be careful when using the
above lemma to characterize the topological type of the curve (or
the Dehn twist $t_c$) algebraically: Namely, if one of the
components of $S^c$, the surface $S$ cut along the curve $c$, is
orientable of genus at least two then we can embed $T=OT_{g,1}^k$
into the centralizer $C_{{\rm M}(S)}(t_c)$ satisfying the conditions
of the lemma so that the topological type of the curve (or the Dehn
twist $t_c$) is algebraically characterized. If the orientable
component is of genus one, then the intersection of $C_{{\rm
M}(S)}(t_c)$ with the centralizer of the tree (the intersection of
all the centralizers of the vertices of $T$) contains a pair of Dehn
twists $t_a$ and $t_b$, about nonseparating circles with
nonorientable components so that $t_a$ and $t_b$ satisfy the Braid
relation.

On the other hand, if both components of $S^c$ are nonorientable,
then we have to choose the component of $S^c$ so that Condition (2)
of the lemma is satisfied, which is always possible if $g\geq 7$
(see also the next theorem). In other words, we need to choose the
component which contains the longer chain.

{\bf 2)} For odd genus surfaces $S$ the condition ({\it 3}) of the
hypothesis of the lemma is void. For example consider  the pair
$(t_{a_{4g-3}},t_{a_{4g}})$ in the first tree $T_{2g+1,1}^k$. The
maximal chain  $t_{a_{4g-3}},t_{a_{4g-2}},t_{a_{4g}}$ in $T$ is
maximal in the surface and thus is separating. Hence, a tubular
neighborhood of the chain $t_{a_{4g-3}},t_{a_{4g-2}},t_{a_{4g}}$ is
a torus with two boundary components, each of which is a separating
curve. This implies that the pair $(t_{a_{4g-3}},t_{a_{4g}})$ is
separating.
\end{Remark}

\section{Completing the proof of the algebraic characterization of
Dehn twist about separating curves}

If $c$ is a separating circle on a nonorientable surface $S=N_g^k$
of genus $g\geq 7$ then one of the two components of $S^c$ is either
a nonorientable surface of genus at least genus four or an
orientable surface of genus at least two. Hence, the above lemma,
the remark following it, together with Corollary~\ref{Cor-1.1} and
Corollary~\ref{Cor-CompOrient} give an algebraic characterization of
Dehn twists about separating curves on nonorientable surfaces
provided that $g\geq 7$.

\begin{Theorem}\label{Chr-2.1} Let $g\geq 5, \ k\geq 0$ be integers $f$ an
mapping class in ${\rm M}(S)$ such that $f^N=t_c^m$ for some
integers $m>0$, $N$ and a nontrivial separating simple closed curve
$c$ on $S=N_g^k$. If Lemma~\ref{LemChr-2.2} is applicable, which is
always the case if $g\geq 7$, then $f=t_c$ if and only if
\begin{itemize}
\item[a)] $f$ is a primitive element of $C_{{\rm M}(N_{g}^{k})}(K)$ if $c$ does not separate
$\Sigma_{1,1}$ \ $\Sigma_{1,1}^1$ \ or $N_{1,1}^1$;
\item[b)] $f=t_c=(t_at_b)^6$ if $c$ separates
$\Sigma_{1,1}$ or $\Sigma_{1,1}^1$ (see
Figure\,\ref{oneholedsurfaces});
\item[c)] $f=t_c=v^2$ if $c$ separates
$N_{1,1}^1$ (see Figure\,\ref{oneholedsurfaces}).
\end{itemize}
Moreover, the topological type of $c$ is determined completely via
Lemma~\ref{LemChr-2.2}.

If $Lemma~\ref{LemChr-2.2}$ is not applicable then $f=t_c$ if and
only if $f$ primitive.  If further $g=6$ then each component of
$S^c$ is a nonorientable surface of genus three and the number
punctures is $r-2$, where $r$ is the biggest integer such that the
tree below can be embedded in $C_{{\rm M}(S)}(t_c)$. Finally, if
$g=5$ then one of the components is again a nonorientable surface of
genus three. The other component is orientable if and only if
$C_{{\rm M}(S)}(t_c)$ contains a Dehn twist about a characteristic
curve. The number punctures in each component can be determined the
same way as in genus six.
\end{Theorem}

\begin{proof}
As we mentioned in the paragraph above the theorem, if $g\geq 7$ we
are done. If $g=5$ or $6$ and the hypothesis of
Lemma~\ref{LemChr-2.2} is still satisfied then again there is
nothing to do. Hence, we are left with the cases $g=5$ or $6$ and
one of the components of $S^c$ is nonorientable of genus three.  In
other words, we need to characterize these two cases algebraically.
In case $g=6$, via Euler characteristic considerations, the maximal
trees below contained in $C_{{\rm M}(S)}(t_c)$ determines the number
of punctures in each component.  Indeed, it is $r-2$ if $r$ is as in
the theorem.

\begin{figure}[hbt]
 \begin{center}
 \includegraphics[width=4cm]{daisy.eps}
\caption {} \label{daisy}
\end{center}
\end{figure}

If $g=5$ and $C_{{\rm M}(S)}(t_c)$ does not contain a Dehn twist
about a characteristic curve then the second component is a
punctured Klein bottle. We know that any two two-sided circles in
the second component intersect geometrically in at least two points.
In other words, this component does not contain any chain of length
larger than one. However, the genus three component clearly contains
a chain of length three. Hence, we can distinguish the topological
components of $S^c$ algebraically. The number of punctures in the
genus three component is determined exactly as in the genus six case
(so it is $r-2$).  Finally, the number of punctures in the other
component is then $k-r+2$.
\end{proof}

\begin{Remark}\label{Rem-GeneralDehnTwistCh}
Theorem 2.1 of \cite{I1} characterizes algebraically the Dehn twists
about nonseparating curves in orientable surfaces of genus $g>0$
(except closed surfaces of genus two). Clearly, some appropriate
versions of Lemma~\ref{LemChr-2.2} and Theorem~\ref{Chr-2.1} for
orientable surfaces provide an algebraic characterization we
mentioned in Remark~\ref{IvnovThm2.2}. In fact, the versions for
orientable surfaces will be even easier since a large portion of our
efforts has been spent to distinguish algebraically an annulus with
one puncture from a M\"obius band.
\end{Remark}

\section{Some Applications}

The subgroup $\mathcal{T}$ of the mapping class group generated by
all Dehn twists about two-sided simple closed curves, is called the
twist subgroup. It is known that this subgroup is of index $2^{k+1} k!$
in ${\rm Mod}(N_{g}^{k})$, provided that $g \geq 3$ (\cite{K},  \cite{SM2}).
Now, we state the following corollary of Theorem~\ref{Chr-2.1}.

\begin{Corollary}\label{Cor-AlgChr}
For $g\geq 5$, let $\Phi:{\rm M}(N_g^k)\rightarrow{\rm M}(N_g^k)$ be
an automorphism and $\mathcal{T}\leq {\rm M}(N_g^k)$ be the twist
subgroup. If $t_c\in \mathcal{T}$ is a Dehn twist then so is
$\Phi(t_c)$.  Moreover, the Dehn twists $t_c$ and $\Phi(t_c)$ are
topologically equivalent. In other words, there is a homeomorphism
$f:N_g^k\rightarrow N_g^k$ \ such that $\Phi(t_c)=t_{f(c)}$. In
particular, the twist subgroup is a characteristic subgroup of ${\rm
M}(N_g^k)$.
\end{Corollary}

The subgroup ${\rm PMod}^{+}(N_{g}^{k})$ has index $2^k$ in ${\rm
PMod}(N_{g}^{k})$ and contains the twist subgroup $\mathcal{T}$ as a
subgroup of index two (\cite{SM2}) provided that $g\geq 3$.

\begin{Lemma}\label{Lem-Chr-PMPlus}
The subgroup ${\rm PMod}^{+}(N_{g}^{k})$ is characteristic in ${\rm
Mod}(N_{g}^{k})$ and ${\rm PMod}(N_{g}^{k})$, provided that $g\geq
5$.
\end{Lemma}

\begin{proof}
We know that the twist subgroup is characteristic in all the three
groups in the statement of the lemma. On the other hand, ${\rm
PMod}^{+}(N_g^k)$ is generated by the twist subgroup and the set of
all $Y$-homeomorphisms, each of which is supported inside Klein
bottles with one boundary component, so that both components of the
surface cut along this boundary curve are nonorientable. It is easy
to see that this set is also characteristic (cf. Theorem 3.9 and
Theorem 3.10 of \cite{A}). Indeed, one can see this directly as
follows: Note that, in the Klein bottle with the boundary circle
$e$, a $Y$-homeomorphism represents a mapping class, say $\tau$,
which is characterized as a mapping class that is not a Dehn twist
but $\tau^2=t_e$. It follows that ${\rm PMod}^{+}(N_g^k)$ is
characteristic in ${\rm Mod}(N_g^k)$ and ${\rm PMod}(N_g^k)$.
\end{proof}

\section*{acknowledgement}

I would like to thank B. Szepietowski for his numerous valuable
comments on the earlier version of this paper. I would like to thank
also  Y. Ozan for reviewing and several suggestions.

\bigskip
\providecommand{\bysame}{\leavevmode\hboxto3em{\hrulefill}\thinspace}
\begin{thebibliography}{1}

\bibitem{A} F. Atalan, Outer automorphisms of mapping class groups of
nonorientable surfaces, Internat. J. Algebra Comput., {\bf 20}(3)
(2010) 437-456.

\bibitem{ASzep} F. Atalan, B. Szepietowski, Automorphisms of the mapping class group of a
nonorientable surface, arXiv:1403.2774v2 (2014).

\bibitem{F-M} B. Farb and D. Margalit, A primer on mapping class groups,
Princeton University Press, New Jersey  (2012).

\bibitem{I1} N. V. Ivanov, Automorphisms of Teichmuller modular
groups, in Lecture Notes in Math. {\bf 1346} (Springer, Berlin,
1988) pp. 199-270.

\bibitem{I2} N. V. Ivanov, Automorphisms of complexes of curves and of Teichm\"uller spaces,
Internat. Math. Res. Notices  (1997), 651-666.

\bibitem{I3} N. V. Ivanov and J. D. McCarthy, On injective homomorphisms between
 Teichm\"uller modular groups I, Invent. Math. {\bf 135} (1999), 425–486.

\bibitem{K} M. Korkmaz, Mapping class groups of nonorientable surfaces, Geom.
Dedicata, {\bf 89} (2002) 109-133.

\bibitem{K1} M. Korkmaz, Automorphisms of complexes of curves on
punctured spheres and on punctured tori, Topology and its
Applications, {\bf 95} (1999) 85-111.

\bibitem{SM1} M. Stukow, Dehn twists on nonorientable surfaces, Fund. Math.
{\bf 189} (2006), 117-147.

\bibitem{SM2} M. Stukow, The twist subgroup of the mapping class group of
a nonorientable surface, Osaka J. Math. {\bf 46} (2009) 717-738.

\end{thebibliography}
\end{document}


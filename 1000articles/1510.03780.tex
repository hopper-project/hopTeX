\documentclass[12pt]{amsart}
\usepackage[utf8]{inputenc}
\usepackage{amssymb}
\usepackage{hyperref}
\usepackage[final]{showkeys} 

\xyoption{all}

\theoremstyle{definition}
\newtheorem{mydef}{Definition}[section]
\newtheorem{lem}[mydef]{Lemma}
\newtheorem{thm}[mydef]{Theorem}
\newtheorem{conjecture}[mydef]{Conjecture}
\newtheorem{cor}[mydef]{Corollary}
\newtheorem{claim}[mydef]{Claim}
\newtheorem{question}[mydef]{Question}
\newtheorem{hypothesis}[mydef]{Hypothesis}
\newtheorem{prop}[mydef]{Proposition}
\newtheorem{defin}[mydef]{Definition}
\newtheorem{example}[mydef]{Example}
\newtheorem{remark}[mydef]{Remark}
\newtheorem{notation}[mydef]{Notation}
\newtheorem{fact}[mydef]{Fact}

\newbox\noforkbox \newdimen\forklinewidth
\forklinewidth=0.3pt \setbox0\hbox{$\textstyle\smile$}
\setbox1\hbox to \wd0{\hfil\vrule width \forklinewidth depth-2pt
 height 10pt \hfil}
\wd1=0 cm \setbox\noforkbox\hbox{\lower 2pt\box1\lower
2pt\box0\relax}

\setbox0\hbox{$\textstyle\smile$}
\setbox1\hbox to \wd0{\hfil{\sl /\/}\hfil} \setbox2\hbox to
\wd0{\hfil\vrule height 10pt depth -2pt width
               \forklinewidth\hfil}
\wd1=0 cm \wd2=0 cm
\newbox\doesforkbox
\setbox\doesforkbox\hbox{\lower 0pt\box1 \lower
2pt\box2\lower2pt\box0\relax}

\def\1nf{{\mathop{\copy\noforkbox}\limits}^{(1)}}

\def\2nf{{\mathop{\copy\noforkbox}\limits}^{(2)}}
\def\3nf{{\mathop{\copy\noforkbox}\limits}^{(3)}}

\title[Downward categoricity in tame AECs]{A downward categoricity transfer for tame abstract elementary classes}
\date{\today\\
AMS 2010 Subject Classification: Primary 03C48. Secondary: 03C45, 03C52, 03C55, 03C75, 03E55.}
\keywords{Abstract elementary classes; Categoricity; Good frames; Classification theory; Tameness; Orthogonality}

\parindent 0pt
\parskip 5pt

\setcounter{tocdepth}{1}

\author{Sebastien Vasey}
\email{sebv@cmu.edu}
\urladdr{http://math.cmu.edu/\textasciitilde svasey/}
\address{Department of Mathematical Sciences, Carnegie Mellon University, Pittsburgh, Pennsylvania, USA}
\thanks{The author is supported by the Swiss National Science Foundation.}

\begin{document}

\begin{abstract}
  We prove a downward transfer from categoricity in a successor in tame abstract elementary classes (AECs). This complements the upward transfer of Grossberg and VanDieren and improves the Hanf number in Shelah's downward transfer (provided the class is tame).
 
  \begin{thm}\label{main-thm-abstract}
  Let ${\mathcal{K}}$ be an AEC with amalgamation. If ${\mathcal{K}}$ is ${\text{LS}} ({\mathcal{K}})$-weakly tame and categorical in a successor $\lambda \ge \beth_{\left(2^{{\text{LS}} ({\mathcal{K}})}\right)^+}$, then ${\mathcal{K}}$ is categorical in all $\lambda' \ge \beth_{\left(2^{{\text{LS}} ({\mathcal{K}})}\right)^+}$.
  \end{thm}

  The argument uses orthogonality calculus and gives alternate proofs to both the Shelah and the Grossberg-VanDieren transfers. 

  We deduce Shelah's categoricity conjecture in universal classes with amalgamation:

  \begin{thm}\label{abstract-thm-2}
    Let ${\mathcal{K}}$ be a universal class with amalgamation. If ${\mathcal{K}}$ is categorical in some $\lambda \ge \beth_{\left(2^{|L (K)| + \aleph_0}\right)^+}$, then ${\mathcal{K}}$ is categorical in all $\lambda' \ge \beth_{\left(2^{|L (K)| + \aleph_0}\right)^+}$.
  \end{thm}
\end{abstract}

\maketitle

\tableofcontents

\section{Introduction}

In his milestone paper on AECs with amalgamation \cite{sh394}, Shelah proves:

\begin{fact}\label{shelah-downward-categ}
  Let ${\mathcal{K}}$ be an AEC with amalgamation\footnote{See Remark \ref{nomax-rmk} for why joint embedding or no maximal models need not be assumed.}. If ${\mathcal{K}}$ is categorical in a successor $\lambda \ge {h ({{h ({{\text{LS}} ({\mathcal{K}})})}})}$, then ${\mathcal{K}}$ is categorical in all $\lambda' \in [{h ({{h ({{\text{LS}} ({\mathcal{K}})})}})}, \lambda]$.
\end{fact}

where we have set:

\begin{notation}\label{hanf-notation}
  For $\lambda$ an infinite cardinal, ${h ({\lambda})} := \beth_{(2^{\lambda})^+}$.
\end{notation}

It has been asked by Baldwin \cite[Problem D.1.(5)]{baldwinbook09} whether the bound in Fact \ref{shelah-downward-categ} can be lowered to ${h ({{\text{LS}} ({\mathcal{K}})})}$. Indeed an earlier result of Makkai and Shelah \cite{makkaishelah} is that if ${\mathcal{K}}$ is a class of models of an $L_{\kappa, \omega}$ sentence, for $\kappa$ a strongly compact cardinal, then one can reduce the Hanf number above to ${h ({{\text{LS}} ({\mathcal{K}}) + \kappa})}$. In fact, an upward transfer also holds:

\begin{fact}\label{makkaishelah-fact}
  If ${\mathcal{K}}$ is the class of models of an $L_{\kappa, \omega}$-sentence for $\kappa$ a strongly compact cardinal and ${\mathcal{K}}$ is categorical in a successor $\lambda \ge {h ({|L| + \kappa})}$, then ${\mathcal{K}}$ is categorical in every $\lambda' \ge {h ({|L| + \kappa})}$.
\end{fact}

The upward part was later generalized by Grossberg and VanDieren \cite{tamenesstwo, tamenessthree} to:

\begin{fact}\label{gv-upward-transfer}
  Let ${\mathcal{K}}$ be an AEC with amalgamation and no maximal models. Let $\lambda > {\text{LS}} ({\mathcal{K}})$. If ${\mathcal{K}}$ is $\lambda$-tame and categorical in $\lambda^+$, then ${\mathcal{K}}$ is categorical in all $\lambda' \ge \lambda^+$.
\end{fact} 

Recall that tameness is a locality property for Galois types, also introduced by Grossberg and VanDieren \cite{tamenessone}: we say that an AEC ${\mathcal{K}}$ is \emph{$(<\kappa)$}-tame if every Galois types is determined by its restrictions to models of size less than $\kappa$. \emph{$\lambda$-tame} means $(<\lambda^+)$-tame. See \cite{lieberman2011} for an equivalent definition in terms of a natural topology on Galois types being Hausdorff.

Why is Fact \ref{gv-upward-transfer} a generalization of the upward part of Makkai and Shelah? Because Boney \cite{tamelc-jsl} proved that if ${\mathcal{K}}$ is an AEC with amalgamation and $\kappa > {\text{LS}} ({\mathcal{K}})$ is strongly compact, then ${\mathcal{K}}$ is $(<\kappa)$-tame\footnote{Boney also proves the same conclusion if ${\mathcal{K}}$ is the class of models of an $L_{\kappa, \omega}$-sentence and $\kappa \ge {\text{LS}} ({\mathcal{K}})$.}. A common theme in the study of tame AECs is that many results that hold above a strongly compact already hold assuming tameness. In this vein, many results proven in \cite{makkaishelah} have been shown to hold in tame AECs. See for example \cite{b-k-vd-spectrum, bg-v9, ss-tame-toappear-v3, bv-sat-v3, sv-infinitary-stability-v3, indep-aec-v5}. While it is known that the statement ``all AECs are tame'' is equivalent to the existence of a proper class of almost strongly compact cardinals \cite{lc-tame-v2}, many examples of tame AECs are known (see the introduction to \cite{tamenessone}).

Here we generalize the downward part of Fact \ref{makkaishelah-fact} to any tame AEC:

\textbf{Main Theorem \ref{main-thm}.}
Let ${\mathcal{K}}$ be a ${\text{LS}} ({\mathcal{K}})$-tame AEC with amalgamation. If ${\mathcal{K}}$ is categorical in a successor $\lambda \ge {h ({{\text{LS}} ({\mathcal{K}})})}$, then ${\mathcal{K}}$ is categorical in all $\lambda' \ge {h ({{\text{LS}} ({\mathcal{K}})})}$.

This answers the aforementioned question of Baldwin in case the AEC is ${\text{LS}} ({\mathcal{K}})$-tame. In fact, Theorem \ref{main-thm} can be generalized to \emph{weakly tame} AECs (that is, we only ask that Galois types over \emph{saturated} models be determined by their small restrictions, see Definition \ref{weak-tameness-def}). Since in \cite[Main Claim II.2.3]{sh394}, Shelah proves that an AEC (with amalgamation and no maximal models) categorical in a successor $\lambda$ is ${h ({{\text{LS}} ({\mathcal{K}})})}$-weakly tame below $\lambda$, we obtain an alternate\footnote{Note that our alternate proof is not self-contained and still uses much material from \cite{sh394} and later work of Shelah and others.} proof of Fact \ref{shelah-downward-categ}. The details are in Corollary \ref{shelah-alternate}. We similarly get an alternate proof of a special case of Fact \ref{gv-upward-transfer} (see Corollary \ref{gv-alternate}), where we require that the categoricity happens above ${h ({{\text{LS}} ({\mathcal{K}})})}$\footnote{That the result holds under only \emph{weak} tameness is implicit in Grossberg and VanDieren's proof and stated explicitly in \cite[Theorem 15.11.(2)]{baldwinbook09}.}.

To prove Theorem \ref{main-thm}, we first use the methods of \cite{ss-tame-toappear-v3} to build a global good frame (a forking-like notion for types of elements over models). The local notion for models of size $\lambda$, good $\lambda$-frames, are the main notion in Shelah's book on AEC \cite{shelahaecbook}. We then develop orthogonality calculus in this setup (versions of some of our results have been independently derived by Villaveces and Zambrano \cite{viza}), heavily inspired from Shelah's development of orthogonality calculus in successful good $\lambda$-frames \cite[Section III.6]{shelahaecbook}, and use it to define a notion of unidimensionality similar to what is defined in \cite[Section III.2]{shelahaecbook}. We show unidimensionality in $\lambda$ is equivalent to categoricity in $\lambda^+$ and use orthogonality calculus to transfer unidimensionality across cardinals. While we work in a more global setup than Shelah's, we do not assume that the good frames we work with are successful \cite[Definition III.1.1]{shelahaecbook}, so we do \emph{not} assume that the nonforking relation is defined for types of models (it is only defined for types of elements). To get around this difficulty, we use the theory of \emph{independent sequences} introduced by Shelah to good $\lambda$-frames in \cite[Section III.5]{shelahaecbook} and developed in \cite{tame-frames-revisited-v4} for global good frames. 

An interesting methodological point compared to the proof of the categoricity transfer of \cite{sh394} is that we do not use models of set theory to prove the transfer of ``no Vaughtian pair'' (see $(\ast)_9$ in the proof of \cite[Theorem II.2.7]{sh394} or \cite[Theorem 14.12]{baldwinbook09}). As outlined above, the method of proof of Theorem \ref{main-thm} is much more local, and this is what allows us to lower the Hanf number.

The reader may ask how this work differs from the categoricity transfers in \cite{ap-universal-v8, categ-primes-v3}. There we do \emph{not} assume that the categoricity cardinal is a successor but assume the existence of prime models over sets of the form $M \cup \{a\}$. Moreover the only purpose of orthogonality calculus there is to prove a technical lemma saying that a certain class of models omitting\footnote{Or really, models over which a certain type has a unique extension.} a type has a good frame. Here we use orthogonality globally and develop more of its properties, without assuming existence of prime models. At the end of Section \ref{main-thm-sec}, we use our methods to give improvements on several results in \cite{indep-aec-v5,ap-universal-v8,categ-primes-v3}, including Theorem \ref{abstract-thm-2} from the abstract.

This paper was written while working on a Ph.D.\ thesis under the direction of Rami Grossberg at Carnegie Mellon University and I would like to thank Professor Grossberg for his guidance and assistance in my research in general and in this work specifically. I also thank John Baldwin and Monica VanDieren for helpful feedback on an earlier draft of this paper.

\section{Background}

We give some background on superstability and independence that will be used in the next sections. We assume familiarity with the basics of AECs as laid out in e.g.\ \cite{baldwinbook09} or the forthcoming \cite{grossbergbook}. We will use the notation from the preliminaries of \cite{sv-infinitary-stability-v3}. Everywhere below, ${\mathcal{K}}$ is an AEC.

The definition of superstability below is already implicit in \cite{shvi635} and has since then been studied in several papers, e.g.\ \cite{vandierennomax, gvv-toappear-v1_2, indep-aec-v5, bv-sat-v3, gv-superstability-v2, vv-symmetry-transfer-v2}. We will use the definition from \cite[Definition 10.1]{indep-aec-v5}.

\begin{defin}
  For $M_0 {\le} M$ in ${\mathcal{K}}$, $M$ is \emph{universal over $M_0$} if for any $N \in {\mathcal{K}}_{\|M_0\|}$ with $M_0 {\le} N$, there exists $f: N \xrightarrow[M_0]{} M$.
\end{defin}

\begin{defin}\label{ss assm}
  ${\mathcal{K}}$ is \emph{$\mu$-superstable} (or \emph{superstable in $\mu$}) if:

  \begin{enumerate}
    \item $\mu \ge {\text{LS}} ({\mathcal{K}})$.
    \item ${\mathcal{K}}_\mu$ is nonempty, has amalgamation, joint embedding, and no maximal models.
    \item ${\mathcal{K}}$ is stable in $\mu$\footnote{That is, $|{\text{gS}} (M)| \le \mu$ for all $M \in {\mathcal{K}}_\mu$. Some authors call this ``Galois-stable''. Here we omit the ``Galois'' prefix, also in the naming of other concepts such as saturated models.}, and:
    \item\label{split assm} $\mu$-splitting in ${\mathcal{K}}$ satisfies the following
  locality property: for all limit ordinal $\delta < \mu^+$ and every increasing continuous sequence ${\langle {M_i : i \le \delta} \rangle}$ in ${\mathcal{K}}_\mu$ with $M_{i + 1}$ universal over $M_i$ for all $i < \delta$, if $p \in {\text{gS}} (M_\delta)$, then there exists $i < \delta$ so that $p$ does not $\mu$-split over $M_i$.
  \end{enumerate}
\end{defin}

In our setup, superstability follows from categoricity. If (as will be the case in this paper) the AEC is categorical in a successor, this is due to Shelah and appears as \cite[Lemma 6.3]{sh394}. The heart of the proof in the general case appears as \cite[Theorem 2.2.1]{shvi635} and the result is stated for classes with amalgamation in \cite[Theorem 6.3]{gv-superstability-v2}.

\begin{fact}[The Shelah-Villaveces theorem]\label{shvi} 
  Let $\mu \ge {\text{LS}} ({\mathcal{K}})$. If ${\mathcal{K}}$ is has amalgamation, no maximal models, and is categorical in a $\lambda > \mu$, then ${\mathcal{K}}$ is $\mu$-superstable.
\end{fact}

Recall the definition of a limit model (see \cite{gvv-toappear-v1_2} for history and motivation). Here we give a global definition, where we permit the limit model and the base to have different sizes.

\begin{defin}\label{limit-def}
  Let $M_0 {\le} M$ be models in ${\mathcal{K}}_{\ge {\text{LS}} ({\mathcal{K}})}$. $M$ is \emph{limit over $M_0$} if there exists a limit ordinal $\delta$ and a strictly increasing continuous sequence ${\langle {N_i : i \le \delta} \rangle}$ such that:

  \begin{enumerate}
    \item $N_0 = M_0$.
    \item $N_\delta = M$.
    \item For all $i < \delta$, $N_{i + 1}$ is universal over $N_i$.
  \end{enumerate}

  We say that $M$ is \emph{limit} if it is limit over some $M' {\le} M$.
\end{defin}

\begin{defin}\label{ksat-def}
  Assume that ${\mathcal{K}}$ has amalgamation. 

  \begin{enumerate}
    \item For $\lambda > {\text{LS}} ({\mathcal{K}})$, ${{{{\mathcal{K}}}^{{{\lambda}}\text{-sat}}}}$ is the class of $\lambda$-saturated models in ${\mathcal{K}}_{\ge \lambda}$. We order it with the strong substructure relation inherited from ${\mathcal{K}}$.
    \item We also define ${{{{\mathcal{K}}}^{{{{\text{LS}} ({\mathcal{K}})}}\text{-sat}}}}$ to be the class of models $M \in {\mathcal{K}}_{\ge {\text{LS}} ({\mathcal{K}})}$ such that for all $A \subseteq |M|$ with $|A| \le {\text{LS}} ({\mathcal{K}})$, there exists a limit model $M_0 {\le} M$ with $M_0 \in {\mathcal{K}}_{{\text{LS}} ({\mathcal{K}})}$ and $A \subseteq |M_0|$. We order ${{{{\mathcal{K}}}^{{{{\text{LS}} ({\mathcal{K}})}}\text{-sat}}}}$ with the strong substructure relation inherited from ${\mathcal{K}}$.
  \end{enumerate}
\end{defin}
\begin{remark}
  If ${\mathcal{K}}$ has amalgamation and is stable in ${\text{LS}} ({\mathcal{K}})$, then ${{{{\mathcal{K}}}^{{{{\text{LS}} ({\mathcal{K}})}}\text{-sat}}}}_{{\text{LS}} ({\mathcal{K}})}$ is the class of limit models in ${\mathcal{K}}_{{\text{LS}} ({\mathcal{K}})}$.
\end{remark}

Together with superstability, a powerful tool is the symmetry property for splitting, first isolated by VanDieren \cite{vandieren-symmetry-v1}:
\begin{defin}\label{sym defn}
Let $\mu \ge {\text{LS}} ({\mathcal{K}})$ and assume that ${\mathcal{K}}$ has amalgamation in $\mu$. ${\mathcal{K}}$ exhibits \emph{symmetry for $\mu$-splitting} (or \emph{$\mu$-symmetry} for short) if  whenever models $M,M_0,N\in{\mathcal{K}}_\mu$ and elements $a$ and $b$  satisfy the conditions \ref{limit sym cond}-\ref{last} below, then there exists  $M^b$  a limit model over $M_0$, containing $b$, so that ${\text{gtp}}(a/M^b)$ does not $\mu$-split over $N$.
\begin{enumerate} 
\item\label{limit sym cond} $M$ is universal over $M_0$ and $M_0$ is a limit model over $N$.
\item\label{a cond}  $a\in M\backslash M_0$.
\item\label{a non-split} ${\text{gtp}}(a/M_0)$ is non-algebraic and does not $\mu$-split over $N$.
\item\label{last} ${\text{gtp}}(b/M)$ is non-algebraic and does not $\mu$-split over $M_0$. 
\end{enumerate}
\end{defin}

We will use that symmetry follows from categoricity in a successor cardinal. The same results holds when ${\mathcal{K}}$ is categorical in a high-enough cardinal, see \cite{vv-structure-categ-v2}.

\begin{fact}[Corollary 5.2 in \cite{vv-symmetry-transfer-v2}]\label{sym-from-categ}
  Let $\mu \ge {\text{LS}} ({\mathcal{K}})$. Assume that ${\mathcal{K}}$ has amalgamation, no maximal models, and is categorical in a cardinal $\lambda$ with ${\text{cf} ({\lambda})} > \mu$. Then ${\mathcal{K}}$ is $\mu$-superstable and has $\mu$-symmetry.
\end{fact}

Let us also recall the definition of weak tameness. We use the notation from \cite[Definition 11.6]{baldwinbook09}

\begin{defin}\label{weak-tameness-def}
  Let $\chi, \mu$ be cardinals with ${\text{LS}} ({\mathcal{K}}) \le \chi \le \mu$. Assume that ${\mathcal{K}}_{[\chi, \mu]}$ has amalgamation. ${\mathcal{K}}$ is \emph{$(\chi, \mu)$-weakly tame} if for any saturated $M \in {\mathcal{K}}_\mu$, any $p, q \in {\text{gS}} (M)$, if $p \neq q$, there exists $M_0 \in {\mathcal{K}}_{\chi}$ with $M_0 {\le} M$ and $p {\upharpoonright} M_0 \neq q {\upharpoonright} M_0$. For $\theta \ge \mu$, ${\mathcal{K}}$ is \emph{$(\chi, <\theta)$-weakly tame} if it is $(\chi, \mu)$-weakly tame for every $\mu \in [\chi, \theta)$. $(\chi, \le \theta)$-weakly tame means $(\chi, <\theta^+)$-weakly tame. Finally, ${\mathcal{K}}$ is \emph{$\chi$-weakly tame} if it is $(\chi, \mu)$-weakly tame for every $\mu \ge \chi$.
\end{defin}

\subsection{Good frames}

In \cite[Definition II.2.1]{shelahaecbook}\footnote{The definition here is simpler and more general than the original: We will \emph{not} use Shelah's axiom (B) requiring the existence of a superlimit model of size $\lambda$. Several papers (e.g.\ \cite{jrsh875}) define good frames without this assumption.}, Shelah introduces \emph{good frames}, a local notion of independence for AECs. This is the central concept of his book and has seen many other applications, such as a proof of Shelah's categoricity conjecture for universal classes \cite{ap-universal-v8}. A \emph{good $\lambda$-frame} is a triple ${\mathfrak{s}} = ({\mathcal{K}}_\lambda, {\mathop{\copy\noforkbox}\limits}, {{\text{gS}}^\text{bs}})$ where:

\begin{enumerate}
  \item ${\mathcal{K}}$ is a nonempty AEC which has amalgamation in $\lambda$, joint embedding in $\lambda$, no maximal models in $\lambda$, and is stable in $\lambda$.
  \item For each $M \in {\mathcal{K}}_\lambda$, ${{\text{gS}}^\text{bs}} (M)$ (called the set of \emph{basic types} over $M$) is a set of nonalgebraic Galois types over $M$ satisfying (among others) the \emph{density property}: if $M {<} N$ are in ${\mathcal{K}}_\lambda$, there exists $a \in |N| \backslash |M|$ such that ${\text{gtp}} (a / M; N) \in {{\text{gS}}^\text{bs}} (M)$.
  \item ${\mathop{\copy\noforkbox}\limits}$ is an (abstract) independence relation on types of length one over models in ${\mathcal{K}}_\lambda$ satisfying the basic properties of first-order forking in a superstable theory: invariance, monotonicity, extension, uniqueness, transitivity, local character, and symmetry (see \cite[Definition II.2.1]{shelahaecbook}).
\end{enumerate}

As in \cite[Definition II.6.35]{shelahaecbook}, we say that a good $\lambda$-frame ${\mathfrak{s}}$ is \emph{type-full} if for each $M \in {\mathcal{K}}_\lambda$, ${{\text{gS}}^\text{bs}} (M)$ consists of \emph{all} the nonalgebraic types over $M$. We focus on type-full good frames in this paper and hence just write ${\mathfrak{s}} = ({\mathcal{K}}_\lambda, {\mathop{\copy\noforkbox}\limits})$. For notational simplicity, we extend nonforking to algebraic types by specifying that algebraic types do not fork over their domain. Given a type-full good $\mu$-frame ${\mathfrak{s}} = ({\mathcal{K}}_\lambda, {\mathop{\copy\noforkbox}\limits})$ and $M_0 {\le} M$ both in ${\mathcal{K}}_\lambda$, we say that a nonalgebraic type $p \in {\text{gS}} (M)$ \emph{does not ${\mathfrak{s}}$-fork over $M_0$} if it does not fork over $M_0$ according to the abstract independence relation ${\mathop{\copy\noforkbox}\limits}$ of ${\mathfrak{s}}$. When ${\mathfrak{s}}$ is clear from context, we omit it and just say that \emph{$p$ does not fork over $M_0$.} We say that a good $\mu$-frame ${\mathfrak{s}}$ is \emph{on ${\mathcal{K}}_\lambda$} if its underlying class is ${\mathcal{K}}_\lambda$. 

We will use the following without comments. See \cite[Fact 3.6]{vv-symmetry-transfer-v2} for a proof.

\begin{fact}
  If ${\mathfrak{s}}$ is a type-full good $\lambda$-frame on ${\mathcal{K}}_\lambda$, then ${\mathcal{K}}$ is $\lambda$-superstable.
\end{fact}

It was pointed out in \cite{ss-tame-toappear-v3} (and further improvements in \cite[Section 10]{indep-aec-v5} or \cite[Theorem 6.12]{vv-symmetry-transfer-v2}) that tameness can be combined with superstability to build a good frame. This can also be done using only weak tameness:

\begin{fact}[Theorem 5.1 in \cite{vv-structure-categ-v2}]\label{good-frame-weak-tameness}
  Let $\lambda > \mu \ge {\text{LS}} ({\mathcal{K}})$. Assume that ${\mathcal{K}}$ is superstable in every $\chi \in [\mu, \lambda]$ and has $\lambda$-symmetry.
  
  If ${\mathcal{K}}$ is $(\mu, \lambda)$-weakly tame, then there exists a type-full good $\lambda$-frame with underlying class ${{{{\mathcal{K}}}^{{{\lambda}}\text{-sat}}}}_\lambda$ (so in particular, ${{{{\mathcal{K}}}^{{{\lambda}}\text{-sat}}}}_\lambda$ is the initial segment of an AEC).
\end{fact}

Once we have a good $\lambda$-frame, we can enlarge it so that the nonforking relation works over larger models. For ${\mathcal{F}} = [\lambda, \theta)$ an interval with $\theta \ge \lambda$ a cardinal or $\infty$, we define a \emph{type-full good ${\mathcal{F}}$-frame} similarly to a type-full good $\lambda$-frame but require forking to be defined over models in ${\mathcal{K}}_{\mathcal{F}}$ (similarly, the good properties hold of the class ${\mathcal{K}}_{\mathcal{F}}$, e.g.\ ${\mathcal{K}}$ is stable in every $\mu \in {\mathcal{F}}$). See \cite[Definition 2.21]{ss-tame-toappear-v3} for a precise definition. For a type-full good ${\mathcal{F}}$-frame ${\mathfrak{s}} = ({\mathcal{K}}_{\mathcal{F}}, {\mathop{\copy\noforkbox}\limits})$ and ${\mathcal{K}}'$ a subclass of ${\mathcal{K}}_{\mathcal{F}}$, we define the restriction ${\mathfrak{s}} {\upharpoonright} {\mathcal{K}}'$ of ${\mathfrak{s}}$ to ${\mathcal{K}}'$ in the natural way (see \cite[Definition 3.15.(2)]{indep-aec-v5}).

\begin{fact}[Corollary 6.9 in \cite{tame-frames-revisited-v4}]\label{frame-upward-transfer}
  Let $\theta > \lambda \ge {\text{LS}} ({\mathcal{K}})$. Let ${\mathcal{F}} := [\lambda, \theta)$. Assume that ${\mathcal{K}}_{\mathcal{F}}$ has amalgamation. Let ${\mathfrak{s}}$ be a type-full good $\lambda$-frame on ${\mathcal{K}}_{\lambda}$. If ${\mathcal{K}}$ is $(\lambda, <\theta)$-tame\footnote{This is defined as in Definition \ref{weak-tameness-def}, i.e.\ for every $M \in {\mathcal{K}}_{[\lambda, \theta)}$ and every $p, q \in {\text{gS}} (M)$, if $p \neq q$, there exists $M_0 {\le} M$ so that $M_0 \in {\mathcal{K}}_\lambda$ and $p {\upharpoonright} M_0 \neq q {\upharpoonright} M_0$.}, then there exists a type-full good ${\mathcal{F}}$-frame ${\mathfrak{s}}'$ extending ${\mathfrak{s}}$: ${\mathfrak{s}}' {\upharpoonright} {\mathcal{K}}_\lambda = {\mathfrak{s}}$.
\end{fact}

We obtain:

\begin{prop}\label{frame-existence}
  Let ${\mathcal{K}}$ be an AEC with amalgamation and no maximal models. Assume that ${\mathcal{K}}$ is categorical in a successor $\lambda > {\text{LS}} ({\mathcal{K}})^+$. 

  \begin{enumerate}
    \item If $\mu \in ({\text{LS}} ({\mathcal{K}}), \lambda)$ is such that ${\mathcal{K}}$ is $({\text{LS}} ({\mathcal{K}}), \mu)$-weakly tame, then there exists a type-full good $\mu$-frame with underlying class ${{{{\mathcal{K}}}^{{{\mu}}\text{-sat}}}}_\mu$.
    \item Let $\theta > {\text{LS}} ({\mathcal{K}})^+$. If ${\mathcal{K}}$ is $({\text{LS}} ({\mathcal{K}}), <\theta)$-tame, then there exists a type-full good $[{\text{LS}} ({\mathcal{K}})^+, \theta)$-frame with underlying class ${{{{\mathcal{K}}}^{{{{\text{LS}} ({\mathcal{K}})^+}}\text{-sat}}}}_{[{\text{LS}} ({\mathcal{K}})^+, \theta)}$.
  \end{enumerate}
\end{prop}
\begin{proof} 
  By Fact \ref{sym-from-categ}, ${\mathcal{K}}$ is $\mu$-superstable and has $\mu$-symmetry for every $\mu \in [{\text{LS}} ({\mathcal{K}}), \lambda)$. Now:
  \begin{enumerate}
    \item By Fact \ref{good-frame-weak-tameness}.
    \item Use the previous part with $\mu = {\text{LS}} ({\mathcal{K}})^+$, then apply Fact \ref{frame-upward-transfer}. Note that by (the proof of) \cite[Theorem 6.8]{vv-symmetry-transfer-v2}, ${{{{\mathcal{K}}}^{{{{\text{LS}} ({\mathcal{K}})^+}}\text{-sat}}}}_{[{\text{LS}} ({\mathcal{K}})^+, \theta)}$ is the initial segment of an AEC.
  \end{enumerate}
\end{proof}

Assuming only weak tameness, we can show that if ${\mathfrak{s}}$ is a (type-full) good $\mu$-frame and ${\mathfrak{s}}'$ is a good $\lambda$-frame with $\mu < \lambda$ and the underlying class of ${\mathfrak{s}}'$ is the saturated models in the underlying class of ${\mathfrak{s}}$, then the nonforking in ${\mathfrak{s}}'$ can be described in terms of ${\mathfrak{s}}$. This is proven as Theorem \ref{frame-canon-weak-tameness} in the appendix and is used to replace tameness by weak tameness in the main theorem.

\section{Local orthogonality}

In this section, we assume:

\begin{hypothesis}
  ${\mathfrak{s}} = ({\mathcal{K}}_\lambda, {\mathop{\copy\noforkbox}\limits})$ is a type-full good $\lambda$-frame.
\end{hypothesis}

Our aim is to develop some orthogonality calculus as in \cite[Section III.6]{shelahaecbook}. There Shelah works in a good $\lambda$-frame that is \emph{successful} (see \cite[Definition III.1.1]{shelahaecbook}). Note that by \cite[Claim III.9.6]{shelahaecbook} such a good frame can be extended to a type-full one. Thus the framework we work in is more general. In fact, it is strictly more general: Hart and Shelah \cite{hs-example} have shown that for an arbitrary $n < \omega$ there exists a sentence $\phi \in L_{\omega_1, \omega}$ which is categorical in every $\aleph_m$, $m \le n$ but not in $\aleph_{n + 1}$. By \cite[Theorem 10.2]{ext-frame-jml}, when $n \ge 1$, there exists a type-full good $\aleph_{n - 1}$-frame ${\mathfrak{t}}$ on ${\mathcal{K}}_{\aleph_{n - 1}}$ (where ${\mathcal{K}}$ is the class of models of $\psi$ ordered by an appropriate fragment), but the frame cannot be extended to a good $\aleph_n$-frame. Therefore by \cite[II.9.1]{shelahaecbook} ${\mathfrak{t}}$ is not successful.

One of the main component of the definition of successful is the existence property for uniqueness triples (see \cite[Definition III.1.1]{shelahaecbook} or \cite[Definition 4.1.(5)]{jrsh875}). We showed in \cite[Lemma 11.7]{indep-aec-v5} that this property is equivalent to a version of domination assuming the existence of a global independence relation. We start by developing a replacement for uniqueness triples in our setup.

Crucial in this section is the uniqueness of limit models, first proven by Shelah in \cite[Claim II.4.8]{shelahaecbook} (see also \cite[Theorem 9.2]{ext-frame-jml}).

\begin{fact}\label{uq-limit}
  Let $M_0, M_1, M_2 \in {\mathcal{K}}_\lambda$. 

  \begin{enumerate}
    \item If $M_1$ and $M_2$ are limit models, then $M_1 \cong M_2$.
    \item If in addition $M_1$ and $M_2$ are both limit over $M_0$, then $M_1 \cong_{M_0} M_2$.
  \end{enumerate}
\end{fact}

The following appears in \cite[Claim 1.21]{shelahaecbook}. It is stated for $M, N$ superlimit but the proof goes through if $M$ and $N$ are limit models.

\begin{fact}[The conjugation property]\label{conj-prop}
  If $M {\le} N$ are limit models in ${\mathcal{K}}_{\lambda}$ and $p \in {\text{gS}} (N)$ does not fork over $M$, then there exists $f: N \cong M$ so that $f (p) = p {\upharpoonright} M$.
\end{fact}

The next definition is a replacement for a global definition of forking. It already plays a role in \cite{makkaishelah} (see Lemma 4.17 there) and \cite{bv-sat-v3} (see Definition 3.10 there). A similar notion is called ``smooth independence'' in \cite{viza}.

\begin{defin}\label{1-forking-def}
  For $M \in {\mathcal{K}}_{\mathcal{F}}$ and $p \in {\text{gS}}^{<\infty} (M)$, we say that $p$ \emph{does not 1-${\mathfrak{s}}$-fork} over $M_0$ if $M_0 {\le} M$ and for $I \subseteq \ell (p)$ with $|I| = 1$, we have that $p^I$ does not ${\mathfrak{s}}$-fork over $M_0$.
\end{defin}

\begin{notation}
  We write ${{[A]^1} {\mathop{\copy\noforkbox}\limits}_{{M_0}}^{{N}} {M}}$ if for some (any) enumeration ${\bar{a}}$ of $A$, ${\text{gtp}} ({\bar{a}} / M; N)$ does not $1$-${\mathfrak{s}}$-fork over $M_0$. That is, ${{a} {\mathop{\copy\noforkbox}\limits}_{{M_0}}^{{N}} {M}}$ for all $a \in A$.
\end{notation}
\begin{remark}[Disjointness]\label{disj-rmk}
  Because nonforking extensions of nonalgebraic types are nonalgebraic, if ${{[A]^1} {\mathop{\copy\noforkbox}\limits}_{{M_0}}^{{N}} {M}}$, then $|M| \cap A \subseteq |M_0|$.
\end{remark}

The next definition is modeled on \cite[Definition 11.5]{indep-aec-v5} but uses 1-forking instead of a global independence relation.  

\begin{defin}
  $(a, M, N)$ is a \emph{weak domination triple in ${\mathfrak{s}}$} if $M, N \in K_\lambda$, $M {\le} N$, $a \in |N| \backslash |M|$ and for any $N' {\ge} N$ and $M' {\le} N$ with $M {\le} M'$ and $M', N' \in K_\lambda$, if ${{a} {\mathop{\copy\noforkbox}\limits}_{{M}}^{{N'}} {M'}}$, then ${{[N]^1} {\mathop{\copy\noforkbox}\limits}_{{M}}^{{N'}} {M'}}$.
\end{defin}

We now want to show the existence property for weak domination triples: For any type $p \in {\text{gS}} (M)$, there exists a weak domination triple $(a, M, N)$ with $p = {\text{gtp}} (a / M; N)$. We manage to do it for limit models. The proof is a more local version of \cite[Lemma 11.12]{indep-aec-v5} (which adapted \cite[Proposition 4.22]{makkaishelah}). First, we prove that some local character holds for 1-forking:

\begin{lem}\label{lc-lem}
  Let ${\langle {M_i : i \le \lambda^+} \rangle}$ be increasing continuous with $M_i \in {\mathcal{K}}_\lambda$ for all $i < \lambda^+$. Let $N {\ge} M_{\lambda^+}$ and let $A \subseteq |N|$ be such that $|A| \le \lambda$. Let $N_i {\le} N$ be in ${\mathcal{K}}_\lambda$ and contain $A \cup |M_i|$. Then there exists $i < \lambda^+$ such that ${{[A]^1} {\mathop{\copy\noforkbox}\limits}_{{M_i}}^{{N_j}} {M_j}}$ for all $j \ge i$.
\end{lem}
\begin{proof}
  We first show:

  \paragraph{\underline{Claim}} For each $a \in A$ there exists $i_a < \lambda^+$ so that ${{a} {\mathop{\copy\noforkbox}\limits}_{{M_{i_a}}}^{{N_j}} {M_j}}$ for all $j \ge i_a$.

  This is enough because then we can take $i := \sup_{a \in A} i_a$.

  \paragraph{\underline{Proof of claim}} By local character, for each limit $j < \lambda^+$, there exists $i_j < j$ so that ${{a} {\mathop{\copy\noforkbox}\limits}_{{M_{i_j}}}^{{N_j}} {M_j}}$. By Fodor's lemma, there exists $i_a < \lambda^+$ such that for unboundedly many $j < \lambda^+$, $i_j = i_a$. By monotonicity of forking, $i_a$ is as desired.
\end{proof}

\begin{lem}\label{seq-lc}
  Assume ${\langle {M_i : i < \lambda^+} \rangle}$, ${\langle {N_i : i < \lambda^+} \rangle}$ are increasing continuous in ${\mathcal{K}}_\lambda$ such that $M_i {\le} N_i$ for all $i < \lambda^+$. Then there exists $i < \lambda^+$ such that ${{[N_i]^1} {\mathop{\copy\noforkbox}\limits}_{{M_i}}^{{N_{j}}} {M_{j}}}$ for all $j \ge i$.
\end{lem}
\begin{proof}
  By Lemma \ref{lc-lem}, for each $i < \lambda^+$, there exists $j_i < \lambda^+$ such that ${{[N_i]^1} {\mathop{\copy\noforkbox}\limits}_{{M_{j_i}}}^{{N_j}} {M_j}}$ for all $j \ge j_i$. Let $i^\ast < \lambda^+$ be such that $j_i < i^\ast$ for all $i < i^\ast$. Then it is easy to check that ${{[N_{i^\ast}]^1} {\mathop{\copy\noforkbox}\limits}_{{M_{i^\ast}}}^{{N_j}} {M_j}}$ for all $j \ge i^\ast$, which is as needed.
\end{proof}

\begin{thm}\label{weak-dom-existence}
  Let $M \in K_\lambda$ be a limit model. For each nonalgebraic $p \in {\text{gS}} (M)$, there exists a weak domination triple $(a, M, N)$ such that $p = {\text{gtp}} (a / M; N)$.
\end{thm}
\begin{proof}
  Assume not. We know that $M$ is a limit model. Therefore by local character there exists $M^\ast \in K_\lambda$ such that $M$ is limit over $M^\ast$ and $p$ does not fork over $M^\ast$.

  \paragraph{\underline{Claim}} For any limit $M' {\ge} M$ with $M' \in {\mathcal{K}}_{\lambda}$, if $q \in {\text{gS}} (M')$ is the nonforking extension of $p$, then there is no weak domination triple $(b, M', N')$ such that $q = {\text{gtp}} (b / M'; N')$. 

  \paragraph{\underline{Proof of claim}} By the conjugation property (Fact \ref{conj-prop}), there exists $f: M' \cong M$ such that $f (q) = p$. Now use that weak domination triples are invariant under isomorphisms. $\dagger_{\text{Claim}}$
  
  We construct ${\langle {M_i : i < \lambda^+} \rangle}$, ${\langle {N_i : i < \lambda^+} \rangle}$ increasing continuous such that for all $i < \lambda^+$:

  \begin{enumerate}
    \item $M_0 = M$.
    \item $M_i {\le} N_i$ are both in ${\mathcal{K}}_\lambda$.
    \item $M_{i + 1}$ is limit over $M_i$ and $N_{i + 1}$ is limit over $N_i$.
    \item ${\text{gtp}} (a / M_i; N_i)$ is the nonforking extension of $p$. In particular, ${{a} {\mathop{\copy\noforkbox}\limits}_{{M_0}}^{{N_i}} {M_i}}$.
    \item ${{[N_i]^1} {\mathop{\copy\doesforkbox}\limits}_{{M_i}}^{{N_{i + 1}}} {M_{i + 1}}}$.
  \end{enumerate}

  This is enough, since then we get a contradiction to Lemma \ref{seq-lc}. This is possible: If $i = 0$, let $N_0 \in {\mathcal{K}}_\lambda$ be such that $p = {\text{gtp}} (a / M; N_0)$. At limits, take unions. Now assume everything up to $i$ has been constructed. By the claim, $(a, M_i, N_i)$ cannot be a weak domination triple. This means there exists $M_i' {\ge} M_i$ and $N_i' {\ge} N_i$ all in ${\mathcal{K}}_\lambda$ such that ${{a} {\mathop{\copy\noforkbox}\limits}_{{M_i}}^{{N_i'}} {M_i'}}$ but ${{[N_i]^1} {\mathop{\copy\doesforkbox}\limits}_{{M_i}}^{{N_i'}} {M_i'}}$. By the extension property of forking, pick $M_{i + 1} \in {\mathcal{K}}_\lambda$ limit over $M_i$ containing $M_i'$ and $N_{i + 1} {\ge} N_i'$ such that $N_{i + 1}$ is limit over $N_i$ and ${{a} {\mathop{\copy\noforkbox}\limits}_{{M_i}}^{{N_{i + 1}}} {M_{i + 1}}}$. 
\end{proof}

We now give a definition of orthogonality in terms of independent sequences. 

\begin{defin}[Independent sequence, III.5.2 in \cite{shelahaecbook}]
Let $\alpha$ be an ordinal.

  \begin{enumerate}
  \item ${\langle {a_i:i < \alpha} \rangle} \smallfrown {\langle {M_i : i \leq \alpha} \rangle}$ is said to be \emph{independent in $(M, M', N)$} when:

    \begin{enumerate}
    \item $(M_i)_{i \leq \alpha}$ is increasing continuous in ${\mathcal{K}}_\lambda$.
    \item $M {\le} M' {\le} M_0$ and $M, M' \in {\mathcal{K}}_\lambda$.
    \item $M_\alpha {\le} N$ and $N \in {\mathcal{K}}_\lambda$.
    \item For every $i < \alpha$ , ${{a_i} {\mathop{\copy\noforkbox}\limits}_{{M}}^{{M_{i+1}}} {M_i}}$.
    \end{enumerate}

    ${\langle {a_i : i < \alpha} \rangle} \smallfrown {\langle {M_i: i \leq \alpha} \rangle}$ is said to be \emph{independent over $M$} when it is independent in $(M, M_0, M_\alpha)$.
  \item ${\bar{a}} := {\langle {a_i : i < \alpha} \rangle}$ is said to be \emph{independent in $(M, M_0, N)$} when $M {\le} M_0 {\le} N$, $\bar{a} \in {{}^{{\alpha}}{|N|}}$, and for some ${\langle {M_i : i \leq \alpha} \rangle}$ and a model $N^+$ such that $M_{\alpha} {\le} N^+$, $N {\le} N^+$, and ${\langle {a_i : i < \alpha} \rangle} \smallfrown {\langle {M_i : i \leq \alpha} \rangle}$ is independent over $M$. When $M = M_0$, we omit it and just say that ${\bar{a}}$ is independent in $(M, N)$.
  
  \end{enumerate}
\end{defin}
\begin{remark}
We will use the definition above when $\alpha = 2$. In this case, we have that ${\langle {ab} \rangle}$ is independent in $(M, N)$ if and only if ${{a} {\mathop{\copy\noforkbox}\limits}_{{M}}^{{N}} {b}}$ (technically, the right hand side of the ${\mathop{\copy\noforkbox}\limits}$ relation must be a model but we can remedy this by extending the nonforking relation in the natural way, as in the definition of the minimal closure in \cite[Definition 3.4]{bgkv-v2}). 
\end{remark}

\begin{defin}\label{perp-def}
  Let $M \in {\mathcal{K}}_\lambda$ and let $p, q \in {\text{gS}} (M)$ be nonalgebraic. We say that $p$ is \emph{weakly orthogonal to $q$} and write $p {{{\underset{{{\text{wk}}}}{\overset{{{{\mathfrak{s}}}}}{{\perp}}}}}} q$ (or just $p {{{{\underset{{{\text{wk}}}}{\overset{{{{}}}}{{\perp}}}}}}} q$ if ${\mathfrak{s}}$ is clear from context) if for all $N {\ge} M$ and all $a, b \in |N|$ such that ${\text{gtp}} (a / M; N) = p$ and ${\text{gtp}} (b / M; N) = q$, ${\langle {a b} \rangle}$ is independent in $(M, N)$.

  We say that \emph{$p$ is orthogonal to $q$} (written $p {{\underset{{{}}}{\overset{{{\mathfrak{s}}}}{{\perp}}}}} q$, or just $p \perp q$ if ${\mathfrak{s}}$ is clear from context) if for every $N \in {\mathcal{K}}_\lambda$ with $N {\ge} M$, $p' {{{{\underset{{{\text{wk}}}}{\overset{{{{}}}}{{\perp}}}}}}} q'$, where $p'$, $q'$ are the nonforking extensions to $N$ of $p$ and $q$ respectively.
\end{defin}
\begin{remark}
  Definition \ref{perp-def} is equivalent to Shelah's (\cite[Definition III.6.2]{shelahaecbook}), see \cite[Claim III.6.4.(2)]{shelahaecbook} assuming that ${\mathfrak{s}}$ is \emph{successful}. By a similar proof (and assuming that ${\mathfrak{s}}$ has primes), it is also equivalent to the definition in terms of primes in \cite[Definition 2.2]{categ-primes-v3}.
\end{remark}

We will use the following consequence of symmetry:

\begin{fact}[Theorem 4.2 in \cite{jasi}]\label{sym-indep}
  For any $M_0 {\le} M {\le} N$ all in ${\mathcal{K}}_\lambda$, if $a, b \in |N| \backslash |M_0|$, then ${\langle {ab} \rangle}$ is independent in $(M_0, M, N)$ if and only if ${\langle {ba} \rangle}$ is independent in $(M_0, M, N)$.
\end{fact}

\begin{lem}\label{perp-fund}
  Let $M \in {\mathcal{K}}_\lambda$. Let $p, q \in {\text{gS}} (M)$ be nonalgebraic. 

  \begin{enumerate}
  \item\label{perp-fund-1} If $M$ is limit, then $p \perp q$ if and only if $p {{{{\underset{{{\text{wk}}}}{\overset{{{{}}}}{{\perp}}}}}}} q$.
  \item\label{perp-fund-2} $p {{{{\underset{{{\text{wk}}}}{\overset{{{{}}}}{{\perp}}}}}}} q$ if and only if $q {{{{\underset{{{\text{wk}}}}{\overset{{{{}}}}{{\perp}}}}}}} p$.
  \item\label{perp-fund-3} If $p {{{{\underset{{{\text{wk}}}}{\overset{{{{}}}}{{\perp}}}}}}} q$, then whenever $(a, M, N)$ is a weak domination triple representing $q$, $p$ is omitted in $N$. In particular, if $M$ is limit, there exists $N \in {\mathcal{K}}_\lambda$ with $M {<} N$ so that $p$ is omitted in $N$.
  \end{enumerate}
\end{lem}
\begin{proof} \
  \begin{enumerate}
    \item By the conjugation property (Fact \ref{conj-prop}). See the proof of \cite[Lemma 2.6]{categ-primes-v3}.
    \item By Fact \ref{sym-indep}.
    \item Let $N' {\ge} N$ be in ${\mathcal{K}}_\lambda$ and let $b \in |N'|$ realize $p$. We have that ${\langle {ab} \rangle}$ is independent in $(M, N')$. Therefore there exists $N'' {\ge} N$ in ${\mathcal{K}}_\lambda$, $M' \in {\mathcal{K}}_\lambda$ so that $M {\le} M' {\le} N''$, $b \in |M'|$, and ${{a} {\mathop{\copy\noforkbox}\limits}_{{M}}^{{N''}} {M'}}$. By domination, ${{[N]^1} {\mathop{\copy\noforkbox}\limits}_{{M}}^{{N''}} {M'}}$, so by disjointness (Remark \ref{disj-rmk}), $b \notin |N|$. The last sentence follows from the existence property for weak domination triple (Theorem \ref{weak-dom-existence}).
  \end{enumerate}
\end{proof}

\section{Unidimensionality}

\begin{hypothesis}
  ${\mathfrak{s}} = ({\mathcal{K}}_\lambda, {\mathop{\copy\noforkbox}\limits})$ is a type-full good $\lambda$-frame and ${\mathcal{K}}$ is categorical in $\lambda$.
\end{hypothesis}

In this section we give a definition of unidimensionality similar to the ones in \cite[Definition V.2.2]{shelahfobook} or \cite[Section III.2]{shelahaecbook}. We show that ${\mathfrak{s}}$ is unidimensional if and only if ${\mathcal{K}}$ is categorical in $\lambda^+$ (this uses categoricity in $\lambda$). In the next section, we will show how to transfer unidimensionality across cardinals, hence getting the promised categoricity transfer. In \cite[Section III.2]{shelahaecbook}, Shelah gives several different definitions of unidimensionality and also shows (see \cite[III.2.3, III.2.9]{shelahaecbook}) that the so-called ``weak-unidimensionality'' is equivalent to categoricity in $\lambda^+$ (hence our definition is equivalent to Shelah's weak unidimensionality) but it is unclear how to transfer it across cardinals without assuming that the frame is successful.

Note that the hypothesis of categoricity in $\lambda$ implies that the model of size $\lambda$ is limit, hence weak orthogonality and orthogonality coincide, see Lemma \ref{perp-fund}.

Rather than defining what it means to be unidimensional, we find it clearer to define what it means to \emph{not} be unidimensional:

\begin{defin}\label{unidim-def}
  ${\mathfrak{s}}$ is \emph{unidimensional} if the following is \emph{false}: for every $M \in {\mathcal{K}}_\lambda$ and every nonalgebraic $p \in {\text{gS}} (M)$, there exists $M' \in {\mathcal{K}}_{\lambda}$ with $M' {\ge} M$ and nonalgebraic $p', q \in {\text{gS}} (M')$ so that $p'$ extends $p$ and $p' \perp q$.
\end{defin}

We first give an equivalent definition using minimal types:

\begin{defin}
  For $M \in {\mathcal{K}}_\lambda$, a type $p \in {\text{gS}} (M)$ is \emph{minimal} if for every $M' {\ge} M$ with $M' \in {\mathcal{K}}_\lambda$, $p$ has a unique nonalgebraic extension to ${\text{gS}} (M')$.
\end{defin}

\begin{remark}\label{min-fork}
  If $M {\le} N$ are in ${\mathcal{K}}_\lambda$ and $p \in {\text{gS}} (N)$ is nonalgebraic such that $p {\upharpoonright} M$ is minimal, then $p$ does not fork over $M$ (because the nonforking extension of $p {\upharpoonright} M$ has to be $p$).
\end{remark}

By the proof of $(\ast)_5$ in \cite[Theorem II.2.7]{sh394}:

\begin{fact}[Density of minimal types]
  For any $M \in {\mathcal{K}}_\lambda$ and nonalgebraic $p \in {\text{gS}} (M)$, there exists $M' \in {\mathcal{K}}_\lambda$ and $p' \in {\mathcal{K}}_\lambda$ such that $M {\le} M'$, $p'$ extends $p$, and $p'$ is minimal.
\end{fact}

\begin{lem}\label{technical-multidim-equiv}
  The following are equivalent:

  \begin{enumerate}
    \item\label{equiv-1} ${\mathfrak{s}}$ is not unidimensional.
    \item\label{equiv-2} For every $M \in {\mathcal{K}}_{\lambda}$ and every minimal $p \in {\text{gS}} (M)$, there exists $M' {\ge} M$ with $M' \in {\mathcal{K}}_{\lambda}$ and $p', q \in {\text{gS}} (M')$ nonalgebraic so that $p' \perp q$.
    \item\label{equiv-3} For every $M \in {\mathcal{K}}_{\lambda}$ and every minimal $p \in {\text{gS}} (M)$, there exists a nonalgebraic $q \in {\text{gS}} (M)$ with $p \perp q$.
  \end{enumerate}
\end{lem}
\begin{proof}
  (\ref{equiv-1}) implies (\ref{equiv-2}) because (\ref{equiv-2}) is a special case of (\ref{equiv-1}). Conversely, (\ref{equiv-2}) implies (\ref{equiv-1}): given $M \in {\mathcal{K}}_\lambda$ and $p \in {\text{gS}} (M)$, first use density of minimal types to extend $p$ to a minimal $p' \in {\text{gS}} (M')$ (so $M' \in {\mathcal{K}}_\lambda$ $M {\le} M'$). Then apply (\ref{equiv-2}).

  Also, if (\ref{equiv-3}) holds, then (\ref{equiv-2}) holds with $M = M'$. Conversely, assume that (\ref{equiv-2}) holds. Let $p \in {\text{gS}} (M)$ be minimal and let $p', q, M'$ witness (\ref{equiv-2}), i.e.\ $p', q \in {\text{gS}} (M')$, $p'$ extends $p$ and $p' \perp q$. By Remark \ref{min-fork}, $p'$ does not fork over $M$. By the conjugation property (Fact \ref{conj-prop}), there exists $f: M' \cong M$ so that $f (p') = p$. Thus $p \perp f (q)$, hence (\ref{equiv-2}) holds.
\end{proof}

We use the characterization to show that unidimensionality implies categoricity in $\lambda^+$. This is similar to \cite[Proposition 4.25]{makkaishelah} but the proof is slightly more involved since our definition of unidimensionality is weaker. We start with a version of density of minimal types inside a fixed model. We will use the following fact, whose proof is a straightforward direct limit argument: 

\begin{fact}[Theorem 11.1 in \cite{baldwinbook09}]\label{limit-type}
  Let ${\langle {M_i : i \le \omega} \rangle}$ be an increasing continuous chain in ${\mathcal{K}}_\lambda$ and for each $i < \omega$, let $p_i \in {\text{gS}} (M_i)$ be such that $j < i$ implies $p_i {\upharpoonright} M_j = p_j$. Then there exists $p \in {\text{gS}} (M_\omega)$ so that $p {\upharpoonright} M_i = p_i$ for all $i < \omega$.
\end{fact}

\begin{lem}\label{minimal-ext}
  Let $M_0 {\le} M$ with $M_0 \in {\mathcal{K}}_\lambda$ and $M \in {\mathcal{K}}_{> \lambda}$. Let $p \in {\text{gS}} (M_0)$. Then there exists $M_1 \in {\mathcal{K}}_\lambda$ with $M_0 {\le} M_1 {\le} M$ and $q \in {\text{gS}} (M_1)$ so that $q$ extends $p$ and for all $M' {\le} M$ with $M' \in {\mathcal{K}}_\lambda$, $M_1 {\le} M'$, any extension of $q$ to ${\text{gS}} (M')$ does not fork over $M_1$.
\end{lem}
\begin{proof}
  Suppose not. Build ${\langle {N_i : i < \omega} \rangle}$ increasing in ${\mathcal{K}}_\lambda$ and ${\langle {q_i : i < \omega} \rangle}$ such that for all $i < \omega$:

  \begin{enumerate}
    \item $N_0 = M_0$, $q_0 = p$.
    \item $N_i {\le} M$.
    \item $q_i \in {\text{gS}} (N_i)$ and $q_{i + 1}$ extends $q_i$.
    \item $q_{i + 1}$ forks over $N_i$.
  \end{enumerate}

  This is possible since we assumed that the lemma failed. This is enough: let $N_\omega := \bigcup_{i < \omega} N_i$. Let $q \in {\text{gS}} (N_\omega)$ extend each $q_i$ (exists by Fact \ref{limit-type}). By local character, there exists $i < \omega$ such that $q$ does not fork over $N_i$, so $q {\upharpoonright} N_{i + 1} = q_{i + 1}$ does not fork over $N_i$, contradiction.
\end{proof}

\begin{lem}\label{unidim-categ-0}
  If ${\mathfrak{s}}$ is unidimensional, then ${\mathcal{K}}$ is categorical in $\lambda^+$.
\end{lem}
\begin{proof}
  Assume that ${\mathcal{K}}$ is \emph{not} categorical in $\lambda^+$. We show that (\ref{equiv-2}) of Lemma \ref{technical-multidim-equiv} holds so ${\mathfrak{s}}$ is not unidimensional. Let $M_0 \in {\mathcal{K}}_\lambda$ and let $p \in {\text{gS}} (M_0)$ be minimal. We consider two cases:

  \paragraph{\underline{Case 1}} There exists $M \in {\mathcal{K}}_{\lambda^+}$, $M_1 \in {\mathcal{K}}_{\lambda}$ with $M_0 {\le} M_1 {\le} M$ and an extension $p' \in {\text{gS}} (M_1)$ of $p$ so that $p'$ is omitted in $M$.

  Let $c \in |M| \backslash |M_1|$. Fix $M' {\le} M$ in ${\mathcal{K}}_\lambda$ containing $c$ so that $M_1 {\le} M'$ and let $q := {\text{gtp}} (c / M_1; M')$. We claim that $q {{{{\underset{{{\text{wk}}}}{\overset{{{{}}}}{{\perp}}}}}}} p'$ (and so by Lemma \ref{perp-fund}, $p' \perp q$, as needed). Let $N \in {\mathcal{K}}_\lambda$ be such that $N {\ge} M_1$ and let $a, b \in |N|$ be such that $p' = {\text{gtp}} (b / M_1; N)$, $q = {\text{gtp}} (a / M_1; N)$. We want to see that ${\langle {ba} \rangle}$ is independent in $(M_1, N)$. We have that ${\text{gtp}} (a / M_1; N) = {\text{gtp}} (c / M_1; M')$, so let $N' \in {\mathcal{K}}_\lambda$ with $M' {\le} N'$ and $f: N \xrightarrow[M_1]{} N'$ witness it, i.e.\ $f (a) = c$. Let $b' := f (b)$. We have that ${\text{gtp}} (b' / M'; N')$ extends $p'$, and $b' \notin |M'|$ since $p'$ is omitted in $M$, hence by minimality ${\text{gtp}} (b' /M'; N')$ does not fork over $M_1$. In particular, ${\langle {c b'} \rangle}$ is independent in $(M_1, N')$. By invariance and monotonicity, ${\langle {b a} \rangle}$ is independent in $(M_1, N)$.

  \paragraph{\underline{Case 2}} Not Case 1: For every $M \in {\mathcal{K}}_{\lambda^+}$, every $M_1 \in {\mathcal{K}}_{\lambda}$ with $M_0 {\le} M_1 {\le} M$, every extension $p' \in {\text{gS}} (M_1)$ of $p$ is realized in $M$.

  By categoricity in $\lambda$ and non-categoricity in $\lambda^+$, we can find $M \in {\mathcal{K}}_{\lambda^+}$ with $M_0 {\le} M$ and $q_0 \in {\text{gS}} (M_0)$ omitted in $M$. Let $M_1 \in {\mathcal{K}}_\lambda$, $M_0 {\le} M_1 {\le} M$ and $q \in {\text{gS}} (M_1)$ extend $q_0$ so that any extension of $q$ to a model $M' {\le} M$ in ${\mathcal{K}}_\lambda$ does not fork over $M_1$ (this exists by Lemma \ref{minimal-ext}). Let $p' \in {\text{gS}} (M_1)$ be a nonalgebraic extension of $p$. By assumption, $p$ is realized by some $c \in |M|$. Now by the same argument as above (reversing the roles of $p'$ and $q$), $p' {{{{\underset{{{\text{wk}}}}{\overset{{{{}}}}{{\perp}}}}}}} q$, hence $p' \perp q$, as desired.
\end{proof}

For the converse of Lemma \ref{unidim-categ-0}, we will use:

\begin{fact}[Theorem 6.1 in \cite{tamenessthree}]\label{no-vp-fact}
  Assume that ${\mathcal{K}}$ is categorical in $\lambda^+$. Then there exists $M \in {\mathcal{K}}_\lambda$ and a minimal type $p \in {\text{gS}} (M)$ which is realized in every $N \in {\mathcal{K}}_\lambda$ with $M {<} N$.
\end{fact}
\begin{remark}
  The proof of Fact \ref{no-vp-fact} uses categoricity in $\lambda$ in a strong way (it uses that the union of an increasing chain of limit models is limit).
\end{remark}

\begin{lem}\label{unidim-categ-1}
  If ${\mathcal{K}}$ is categorical $\lambda^+$, then ${\mathfrak{s}}$ is unidimensional.
\end{lem}
\begin{proof}
  By Fact \ref{no-vp-fact}, there exists $M \in {\mathcal{K}}_{\lambda}$ and a minimal $p \in {\text{gS}} (M)$ so that $p$ is realized in every $N {>} M$. Now assume for a contradiction that ${\mathcal{K}}$ is \emph{not} unidimensional. Then by Lemma \ref{technical-multidim-equiv}, there exists a nonalgebraic $q \in {\text{gS}} (M)$ such that $p \perp q$. By Lemma \ref{perp-fund}.(\ref{perp-fund-3}) (note that $M$ is limit by categoricity in $\lambda$), there exists $N \in {\mathcal{K}}_\lambda$ with $N {>} M$ so that $p$ is omitted in $N$, a contradiction to the choice of $p$.
\end{proof}

\begin{thm}\label{unidim-categ}
  ${\mathfrak{s}}$ is unidimensional if and only if ${\mathcal{K}}$ is categorical in $\lambda^+$.
\end{thm}
\begin{proof}
  By Lemmas \ref{unidim-categ-0} and \ref{unidim-categ-1}.
\end{proof}

\section{Global orthogonality}

\begin{hypothesis}\label{sec-5-hyp} \
  \begin{enumerate}
    \item ${\mathcal{K}}$ is an AEC.
    \item $\theta > {\text{LS}} ({\mathcal{K}})$ is a cardinal or $\infty$. We set ${\mathcal{F}} := [{\text{LS}} ({\mathcal{K}}), \theta)$.
    \item ${\mathfrak{s}} = ({\mathcal{K}}_{\mathcal{F}}, {\mathop{\copy\noforkbox}\limits})$ is a type-full good ${\mathcal{F}}$-frame.
  \end{enumerate}
\end{hypothesis}

We start developing the theory of orthogonality and unidimensionality in a more global context (with no real loss, the reader can think of $\theta = \infty$ as being the main case). The main problem is to show that for $M$ sufficiently saturated, if $p, q \in {\text{gS}} (M)$ do not fork over $M_0$, then $p \perp q$ if and only if $p {\upharpoonright} M_0 \perp q {\upharpoonright} M_0$. This can be done with the conjugation property in case $\|M_0\| = \|M\|$ but in general one needs to use more tools from the study of independent sequences. We start by recalling a few facts that we will use without further mention:

\begin{fact}\label{satfact-1}
  For any $\lambda \in {\mathcal{F}}$ with $\lambda > {\text{LS}} ({\mathcal{K}})$, ${{{{\mathcal{K}}}^{{{\lambda}}\text{-sat}}}}_{[\lambda, \theta)}$ is the initial segment of an AEC with Löwenheim-Skolem number $\lambda$.
\end{fact}
\begin{proof}
  By uniqueness of limit models \ref{uq-limit} and \cite[Corollary 3]{vandieren-sat-v1}, see also the proof of \cite[Theorem 6.6]{vv-symmetry-transfer-v2}.
\end{proof}

\begin{fact}\label{satfact-2}
  For $\lambda \in {\mathcal{F}}$, $M \in {{{{\mathcal{K}}}^{{{\lambda}}\text{-sat}}}}_\lambda$ if and only if $M$ is limit.
\end{fact}
\begin{proof}
  This is trivial if $\lambda = {\text{LS}} ({\mathcal{K}})$, so assume that $\lambda > {\text{LS}} ({\mathcal{K}})$. If $M$ is limit, then by uniqueness of limit models, $M$ is saturated. Conversely if $M$ is saturated, then it must be unique, hence isomorphic to a limit model.
\end{proof}

We will also use a few more facts about independent sequences:

\begin{fact}[Corollary 5.6 in \cite{tame-frames-revisited-v4}]\label{indep-facts}
  Independent sequences of length two satisfy the axioms of a good ${\mathcal{F}}$-frame. For example:
  \begin{enumerate}
    \item Monotonicity: If ${\langle {ab} \rangle}$ is independent in $(M_0, M, N)$ and $M_0 {\le} M_0' {\le} M' {\le} M {\le} N {\le} N'$, then ${\langle {ab} \rangle}$ is independent in $(M_0', M', N')$.
    \item Continuity: If ${\langle {M_i : i \le \delta} \rangle}$ is increasing continuous, $M_\delta {\le} N$, and ${\langle {a b} \rangle}$ is independent in $(M_0, M_i, N)$ for all $i < \delta$, then ${\langle {ab} \rangle}$ is independent in $(M_0, M_\delta, N)$.
  \end{enumerate}
\end{fact}
\begin{remark}
  The reader may ask inside which frame we work when we say that ${\langle {ab} \rangle}$ is independent (the global frame ${\mathfrak{s}}$ or its restriction?). By monotonicity, the answer does not matter, i.e.\ the independent sequences are the same either way. Similarly, if $\lambda > {\text{LS}} ({\mathcal{K}})$ and $M_0, M, N \in {{{{\mathcal{K}}}^{{{\lambda}}\text{-sat}}}}_{[\lambda, \theta)}$, then ${\langle {ab} \rangle}$ is independent in $(M_0, M, N)$ with respect to ${\mathfrak{s}}$ if and only if it is independent in $(M_0, N)$ with respect to ${\mathfrak{s}} {\upharpoonright} {{{{\mathcal{K}}}^{{{\lambda}}\text{-sat}}}}_{[\lambda, \theta)}$ (i.e.\ we can require the models witnessing the independence to be saturated). This is a simple consequence of the extension property.
\end{remark}

We now define global orthogonality. 

\begin{defin}
  Let $M \in {\mathcal{K}}_{\mathcal{F}}$. For $p, q \in {\text{gS}} (M)$ nonalgebraic, we write $p \perp q$ for $p {{\underset{{{}}}{\overset{{{{\mathfrak{s}} {\upharpoonright} {\mathcal{K}}_{\|M\|}}}}{{\perp}}}}} q$, and $p {{{{\underset{{{\text{wk}}}}{\overset{{{{}}}}{{\perp}}}}}}} q$ for $p {{{\underset{{{\text{wk}}}}{\overset{{{{{\mathfrak{s}} {\upharpoonright} {\mathcal{K}}_{\|M\|}}}}}{{\perp}}}}}} q$ (recall Definition \ref{perp-def}).
\end{defin}

Note that a priori we need not have that if $p \perp q$ and $p', q'$ are nonforking extensions of $p$ and $q$ to big models, then $p' \perp q'$. This will be proven first.

\begin{lem}\label{wkperp-cont}
  Let $\delta$ be a limit ordinal. Let ${\langle {M_i : i \le \delta} \rangle}$ be increasing continuous in ${\mathcal{K}}_{\mathcal{F}}$. Let $p, q \in {\text{gS}} (M_\delta)$ be nonalgebraic and assume that $p {\upharpoonright} M_i {{{{\underset{{{\text{wk}}}}{\overset{{{{}}}}{{\perp}}}}}}} q {\upharpoonright} M_i$ for all $i < \delta$. Then $p {{{{\underset{{{\text{wk}}}}{\overset{{{{}}}}{{\perp}}}}}}} q$.
\end{lem}
\begin{proof}
  By the continuity property of independent sequences (Fact \ref{indep-facts}).
\end{proof}

\begin{lem}\label{perp-cont}
  Let $\delta$ be a limit ordinal. Let ${\langle {M_i : i \le \delta} \rangle}$ be increasing continuous in ${\mathcal{K}}_{\mathcal{F}}$. Let $p, q \in {\text{gS}} (M_\delta)$ be nonalgebraic and assume that $p {\upharpoonright} M_i \perp q {\upharpoonright} M_i$ for all $i < \delta$. Then $p \perp q$.
\end{lem}
\begin{proof}
  By local character, there exists $i < \delta$ so that both $p$ and $q$ do not fork over $M_i$. Without loss of generality, $i = 0$. Let $\lambda := \|M_\delta\|$. If there exists $i < \delta$ so that $\lambda = \|M_i\|$, then the result follows from the definition of orthogonality. So assume that $\|M_i\| < \lambda$ for all $i < \delta$. Let $M' {\ge} M_\delta$ be in ${\mathcal{K}}_\lambda$ and let $p', q'$ be the nonforking extensions to $M'$ of $p$, $q$ respectively. We want to see that $p' {{{{\underset{{{\text{wk}}}}{\overset{{{{}}}}{{\perp}}}}}}} q'$. Let ${\langle {M_i' : i \le \delta} \rangle}$ be an increasing continuous resolution of $M'$ such that $M_i {\le} M_i'$ and $\|M_i'\| = \|M_i\|$ for all $i < \delta$. We know that $p' {\upharpoonright} M_i'$ does not fork over $M_0$, hence over $M_i$ and similarly $q' {\upharpoonright} M_i'$ does not fork over $M_i$. Therefore by definition of orthogonality, $p' {\upharpoonright} M_i' {{{{\underset{{{\text{wk}}}}{\overset{{{{}}}}{{\perp}}}}}}} q' {\upharpoonright} M_i'$. By Lemma \ref{wkperp-cont}, $p' {{{{\underset{{{\text{wk}}}}{\overset{{{{}}}}{{\perp}}}}}}} q'$.
\end{proof}

\begin{lem}\label{perp-nf-1}
  Let $M_0 {\le} M$ be both in ${\mathcal{K}}_{\mathcal{F}}$. Let $p, q \in {\text{gS}} (M)$ be nonalgebraic so that both do not fork over $M_0$. If $p {\upharpoonright} M_0 \perp q {\upharpoonright} M_0$, then $p \perp q$.
\end{lem}
\begin{proof}
  Let $\delta := {\text{cf} ({\|M\|})}$. Build ${\langle {N_i : i \le \delta} \rangle}$ increasing continuous such that $N_0 = M_0$, $N_\delta = M$, and $p {\upharpoonright} N_i \perp q {\upharpoonright} N_i$ for all $i \le \delta$. This is easy: at successor steps, we require $\|N_i\| = \|N_{i + 1}\|$ and use the definition of orthogonality. At limit steps, we use Lemma \ref{perp-cont}. Then $p {\upharpoonright} N_\delta \perp q {\upharpoonright} N_\delta$, but $N_\delta = M$ so $p \perp q$.
\end{proof}

\begin{question}\label{perp-nf-question}
Is the converse true? That is if $M_0 {\le} M$ are in ${\mathcal{K}}_{\mathcal{F}}$, $p, q \in {\text{gS}} (M)$ do not fork over $M_0$ and $p \perp q$, do we have that $p {\upharpoonright} M_0 \perp q {\upharpoonright} M_0$?
\end{question}

An answer to this question would be useful in order to transfer unidimensionality up in a more conceptual way than below. With a very mild additional hypothesis, we give a positive answer in Theorem \ref{perp-global} of the appendix.

We now go back to studying unidimensionality. 

\begin{defin}
  For $\lambda \in {\mathcal{F}}$, we say that ${\mathfrak{s}}$ is \emph{$\lambda$-unidimensional} if the following is \emph{false}: for every limit $M \in {\mathcal{K}}_\lambda$ and every nonalgebraic $p \in {\text{gS}} (M)$, there exists a limit $M' {\ge} M$ in ${\mathcal{K}}_\lambda$ and $p', q \in {\text{gS}} (M')$ so that $p'$ extends $p$ and $p' \perp q$.
\end{defin}
\begin{remark}
  When $\lambda > {\text{LS}} ({\mathcal{K}})$, ${\mathfrak{s}}$ is $\lambda$-unidimensional if and only if ${\mathfrak{s}} {\upharpoonright} {{{{\mathcal{K}}}^{{{\lambda}}\text{-sat}}}}_\lambda$ is unidimensional (see Definition \ref{unidim-def}). If ${\mathcal{K}}$ is categorical in ${\text{LS}} ({\mathcal{K}})$, this also holds when $\lambda = {\text{LS}} ({\mathcal{K}})$ (if ${\mathcal{K}}$ is not categorical in ${\text{LS}} ({\mathcal{K}})$, we do not know that ${{{{\mathcal{K}}}^{{{{\text{LS}} ({\mathcal{K}})}}\text{-sat}}}}$ is an AEC).
\end{remark}

Our next goal is to prove (assuming categoricity in ${\text{LS}} ({\mathcal{K}})$) that $\lambda$-unidimensionality is equivalent to $\mu$-unidimensionality for every $\lambda, \mu \in {\mathcal{F}}$. We will use another characterization of $\lambda$-unidimensionality when $\lambda > {\text{LS}} ({\mathcal{K}})$:

\begin{lem}\label{min-unidim}
  Let $\lambda > {\text{LS}} ({\mathcal{K}})$ be in ${\mathcal{F}}$. The following are equivalent:
  
  \begin{enumerate}
    \item\label{min-unidim-1} ${\mathfrak{s}}$ is \emph{not} $\lambda$-unidimensional.
    \item\label{min-unidim-2} There exists a saturated $M \in {\mathcal{K}}_\lambda$ and nonalgebraic types $p, q \in {\text{gS}} (M)$ such that $p$ is minimal and $p \perp q$.
  \end{enumerate}
\end{lem}
\begin{proof} \
  \begin{itemize}
    \item (\ref{min-unidim-1}) implies (\ref{min-unidim-2}): Assume that ${\mathfrak{s}}$ is not $\lambda$-unidimensional. Let $M \in {{{{\mathcal{K}}}^{{{\lambda}}\text{-sat}}}}_\lambda$ and let $p \in {\text{gS}} (M)$ be minimal (exists by density of minimal types and uniqueness of saturated models). By Lemma \ref{technical-multidim-equiv}, there exists $q \in {\text{gS}} (M)$ so that $p \perp q$.
    \item (\ref{min-unidim-2}) implies (\ref{min-unidim-1}): Let $M \in {{{{\mathcal{K}}}^{{{\lambda}}\text{-sat}}}}_\lambda$ and let $p, q \in {\text{gS}} (M)$ be nonalgebraic so that $p$ is minimal and $p \perp q$. We show that ${{{{\mathcal{K}}}^{{{\lambda}}\text{-sat}}}}$ is \emph{not} categorical in $\lambda^+$, which is enough by Theorem \ref{unidim-categ}. Fix $N {\le} M$ in ${\mathcal{K}}_{{\text{LS}} ({\mathcal{K}})}$ so that $p$ does not fork over $N$. Build a strictly increasing continuous chain ${\langle {M_i : i \le \lambda^+} \rangle}$ such that for all $i < \lambda^+$:

  \begin{enumerate}
    \item $M_i \in {{{{\mathcal{K}}}^{{{\lambda}}\text{-sat}}}}_\lambda$.
    \item $M_0 = M$.
    \item $p$ is omitted in $M_i$.
  \end{enumerate}

  This is enough, since then $p$ is omitted in $M_{\lambda^+}$ so $M_{\lambda^+} \in {\mathcal{K}}_{\lambda^+}$ cannot be saturated. This is possible: at limits we take unions and for $i = 0$ we set $M_0 := M$. Now let $i = j + 1$ be given. Let $p' \in {\text{gS}} (M_j)$ be the nonforking extension of $p$. By uniqueness of saturated models, there exists $f: M_j \cong_{N} M_0$. By uniqueness of nonforking extension, $f (p') = p$. By Lemma \ref{perp-fund}.(\ref{perp-fund-3}), there exists $M' {\ge} M_0$ in ${{{{\mathcal{K}}}^{{{\lambda}}\text{-sat}}}}_\lambda$ so that $p$ is omitted in $M'$. Let $M_{j + 1} := f^{-1}[M']$. Then $p'$ is omitted in $M_{j + 1}$. Since $p$ is minimal, $p$ is omitted in $|M_{j + 1}| \backslash |M_j|$, and hence by induction in $M_{j + 1}$.
  \end{itemize}
\end{proof}

An issue in transferring unidimensionality up is that we do not have a converse to Lemma \ref{perp-nf-1} (see Question \ref{perp-nf-question}), so we will ``cheat''and use the following transfer which follows from the proof of \cite[Theorem 6.3]{tamenessthree}:

\begin{fact}\label{upward-transfer-2}
  If ${\mathcal{K}}$ is categorical in ${\text{LS}} ({\mathcal{K}})$ and ${\text{LS}} ({\mathcal{K}})^+$, then ${\mathcal{K}}$ is categorical in all $\mu \in [{\text{LS}} ({\mathcal{K}}), \theta]$.
\end{fact}

\begin{thm}\label{unidim-transfer}
  Assume that ${\mathcal{K}}$ is categorical in ${\text{LS}} ({\mathcal{K}})$. Let $\lambda$ and $\mu$ both be in ${\mathcal{F}}$. Then ${\mathfrak{s}}$ is $\lambda$-unidimensional if and only if ${\mathfrak{s}}$ is $\mu$-unidimensional. 
\end{thm}
\begin{proof}
  Without loss of generality, $\mu < \lambda$. We first show that if ${\mathfrak{s}}$ is \emph{not} $\mu$-unidimensional, then ${\mathfrak{s}}$ is \emph{not} $\lambda$-unidimensional. Assume that ${\mathfrak{s}}$ is not $\mu$-unidimensional. Let $M_0 \in {{{{\mathcal{K}}}^{{{\mu}}\text{-sat}}}}_\mu$ and let $p \in {\text{gS}} (M_0)$ be minimal (exists by density of minimal types). By definition (and the proof of Lemma \ref{technical-multidim-equiv}), there exists $q \in {\text{gS}} (M_0)$ so that $p \perp q$. Now let $M \in {{{{\mathcal{K}}}^{{{\lambda}}\text{-sat}}}}_\lambda$ be such that $M_0 {\le} M$. Let $p', q'$ be the nonforking extensions to $M$ of $p$ and $q$ respectively. By Lemma \ref{perp-nf-1}, $p' \perp q'$. By Lemma \ref{min-unidim}, ${\mathfrak{s}}$ is not $\lambda$-unidimensional.

  Conversely, assume that ${\mathfrak{s}}$ is $\mu$-unidimensional. By the first part, ${\mathfrak{s}}$ is ${\text{LS}} ({\mathcal{K}})$-unidimensional. By Theorem \ref{unidim-categ}, ${\mathcal{K}}$ is categorical in ${\text{LS}} ({\mathcal{K}})^+$. By Fact \ref{upward-transfer-2}, ${\mathcal{K}}$, and hence ${{{{\mathcal{K}}}^{{{\lambda}}\text{-sat}}}}$, is categorical in $\lambda^+$. By Theorem \ref{unidim-categ} again, ${\mathfrak{s}}$ is $\lambda$-unidimensional.
\end{proof}

We obtain the promised categoricity transfer. Note that it suffices to assume that ${{{{\mathcal{K}}}^{{{\lambda}}\text{-sat}}}}$ (not ${\mathcal{K}}$) is categorical in $\lambda^+$.

\begin{cor}\label{good-categ-transfer}
  Assume that ${\mathcal{K}}$ is categorical in ${\text{LS}} ({\mathcal{K}})$ and let $\lambda \in {\mathcal{F}}$. If ${{{{\mathcal{K}}}^{{{\lambda}}\text{-sat}}}}$ is categorical in $\lambda^+$, then ${\mathcal{K}}$ is categorical in every $\mu \in [{\text{LS}} ({\mathcal{K}}), \theta]$.
\end{cor}
\begin{proof}
  Assume that ${{{{\mathcal{K}}}^{{{\lambda}}\text{-sat}}}}$ is categorical in $\lambda^+$. We prove by induction on $\mu \in [{\text{LS}} ({\mathcal{K}}), \theta]$ that ${\mathcal{K}}$ is categorical in $\mu$. By assumption, ${\mathcal{K}}$ is categorical in ${\text{LS}} ({\mathcal{K}})$. Now let $\mu \in ({\text{LS}} ({\mathcal{K}}), \theta]$ and assume that ${\mathcal{K}}$ is categorical in every $\mu_0 \in [{\text{LS}} ({\mathcal{K}}), \mu)$. If $\mu$ is limit, then it is easy to see that every model of size $\mu$ must be saturated, hence ${\mathcal{K}}$ is categorical in $\mu$. Now assume that $\mu$ is a successor, say $\mu = \mu_0^+$ for $\mu_0 \in {\mathcal{F}}$. By assumption, ${{{{\mathcal{K}}}^{{{\lambda}}\text{-sat}}}}$ is categorical in $\lambda^+$. By Theorem \ref{unidim-categ}, ${\mathfrak{s}}$ is $\lambda$-unidimensional. By Theorem \ref{unidim-transfer}, ${\mathfrak{s}}$ is $\mu_0$-unidimensional. By Theorem \ref{unidim-categ}, ${{{{\mathcal{K}}}^{{{\mu_0}}\text{-sat}}}}$ is categorical in $\mu_0^+$. By the induction hypothesis, ${\mathcal{K}}$ is categorical in $\mu_0$, hence ${{{{\mathcal{K}}}^{{{\mu_0}}\text{-sat}}}} = {\mathcal{K}}_{\ge \mu_0}$, so ${\mathcal{K}}$ is categorical in $\mu_0^+ = \mu$, as desired.
\end{proof}

We finish this section by combining our results with the categoricity transfer in tame AECs with primes of \cite{categ-primes-v3}. As there, we say that ${\mathcal{K}}$ \emph{has primes} if it has primes over sets of the form $M \cup \{a\}$. That is (see \cite[Section II.3]{shelahaecbook}), if $M {\le} N$ are in ${\mathcal{K}}$ and $a \in |N| \backslash |M|$, there exists $N_0 {\le} N$ so that $M {\le} N_0$, $a \in |N_0|$, and whenever $N' \in {\mathcal{K}}$, $b \in |N'|$ are such that ${\text{gtp}} (b / M; N') = {\text{gtp}} (a / M; N)$, there exists $f: N_0 \xrightarrow[M]{} N'$ with $f (a) = b$. We define localizations such as ``${\mathcal{K}}_{\mathcal{F}}$ has primes'' in the natural way.

The value of AECs with primes is that a categoricity transfer from categoricity in a \emph{limit} cardinal holds. The next fact is a local version of \cite[Theorem 2.16]{categ-primes-v3} (which follows directly from its proof):

\begin{fact}\label{prime-transfer-frame}
  Assume that ${\mathcal{K}}$ is categorical in ${\text{LS}} ({\mathcal{K}})$ and ${\mathcal{K}}_{\mathcal{F}}$ has primes. If ${\mathcal{K}}$ is categorical in \emph{some} $\lambda \in [{\text{LS}} ({\mathcal{K}})^+, \theta]$, then ${\mathcal{K}}$ is categorical in \emph{every} $\mu \in [{\text{LS}} ({\mathcal{K}})^+, \theta]$.
\end{fact}

Combining Fact \ref{prime-transfer-frame} with Corollary \ref{good-categ-transfer}, we obtain:

\begin{thm}\label{prime-transfer-frame-2}
  Assume that ${\mathcal{K}}$ is categorical in ${\text{LS}} ({\mathcal{K}})$ and let $\lambda_0 \in {\mathcal{F}}$. If there exists $\lambda \in [\lambda_0^+, \theta]$ such that:
  \begin{enumerate}
    \item ${{{{\mathcal{K}}}^{{{\lambda_0}}\text{-sat}}}}_{[\lambda_0, \lambda)}$ has primes. 
    \item ${{{{\mathcal{K}}}^{{{\lambda_0}}\text{-sat}}}}$ is categorical in $\lambda$.
  \end{enumerate}

  Then ${\mathcal{K}}$ is categorical in \emph{every} $\mu \in [{\text{LS}} ({\mathcal{K}}), \theta]$.
\end{thm}
\begin{proof}
  By Fact \ref{prime-transfer-frame} (where ${\mathcal{K}}$, ${\text{LS}} ({\mathcal{K}}), \theta$ there stand for ${{{{\mathcal{K}}}^{{{\lambda_0}}\text{-sat}}}}, \lambda_0, \lambda$ here), ${{{{\mathcal{K}}}^{{{\lambda_0}}\text{-sat}}}}$ is categorical in $\lambda_0^+$. By Corollary \ref{good-categ-transfer}, ${\mathcal{K}}$ is categorical in every $\mu \in [{\text{LS}} ({\mathcal{K}}), \theta]$.
\end{proof}
\begin{remark}
  This shows that it is enough to assume existence of primes \emph{below} the categoricity cardinal to get categoricity everywhere.
\end{remark}

\section{The main theorem}\label{main-thm-sec}

We prove the main results of this paper. All throughout, we assume:

\begin{hypothesis}\label{ap-hyp}
  ${\mathcal{K}}$ is an AEC with amalgamation and no maximal models. 
\end{hypothesis}
\begin{remark}\label{nomax-rmk}
  Let ${\mathcal{K}}$ be an AEC with amalgamation. Then we can write ${\mathcal{K}}$ as $\bigcup_{i \in I} K^i$, where the ${\mathcal{K}}^i$s are disjoint AECs with ${\text{LS}} ({\mathcal{K}}^i) = {\text{LS}} ({\mathcal{K}})$, and each $K^i$ has amalgamation and joint embedding (see the notion of a diagram in \cite[Definition I.2.2]{shelahaecbook} or \cite[Lemma 16.14]{baldwinbook09}). Assume in addition that $\lambda \ge {\text{LS}} ({\mathcal{K}})$ is such that ${\mathcal{K}}_\lambda$ has joint embedding (e.g.\ ${\mathcal{K}}$ is categorical in $\lambda$). Then using amalgamation ${\mathcal{K}}_{\ge \lambda}$ has joint embedding. So there is a unique $i^\ast \in I$ so that $(K^{i^\ast})_{\ge \lambda} = {\mathcal{K}}$, so when $i \neq i^\ast$, $(K^i)_{\ge \lambda} = \emptyset$. Moreover if ${\mathcal{K}}$ has arbitrarily large models (e.g.\ it has a model of size ${h ({{\text{LS}} ({\mathcal{K}})})}$), then ${\mathcal{K}}^{i^\ast}$ has no maximal models and there exists $\chi < {h ({{\text{LS}} ({\mathcal{K}})})}$ so that for $i \neq i^\ast$, $(K^i)_{\ge \chi} = \emptyset$. Therefore for the purpose of proving a conclusion about ${\mathcal{K}}_{\ge {h ({{\text{LS}} ({\mathcal{K}})})}}$ (such as ``${\mathcal{K}}$ is categorical in all $\lambda' \ge H_1$''), we can restrict ourselves to $K^{i^\ast}$, i.e.\ assume without loss of generality that ${\mathcal{K}}$ has joint embedding and no maximal models. In particular, the ``no maximal models'' hypothesis is not needed for the last sentence of Theorem \ref{main-thm}, Corollary \ref{shelah-alternate}, Theorem \ref{improved-prime-categ}, and Corollary \ref{improved-univ-categ}.
\end{remark}

We will use the notation from Chapter 14 of \cite{baldwinbook09} by writing $H_1 := {h ({{\text{LS}} ({\mathcal{K}})})}$ and $H_2 := {h ({H_1})}$ (recall Notation \ref{hanf-notation}). The main lemma collects both the case of categoricity in a successor and the case of categoricity in a limit (but then we assume that the AEC has primes).

\begin{lem}[Main lemma]\label{main-lem}
  Let $\theta \ge \lambda > {\text{LS}} ({\mathcal{K}})^+$ be such that ${\mathcal{K}}$ is $({\text{LS}} ({\mathcal{K}}), <\theta)$-weakly tame. Assume that ${\mathcal{K}}$ is categorical in $\lambda$. Then ${{{{\mathcal{K}}}^{{{{\text{LS}} ({\mathcal{K}})^+}}\text{-sat}}}}$ is categorical in all $\mu \in [{\text{LS}} ({\mathcal{K}})^+, \theta]$ provided that one of the following holds:
  
  \begin{enumerate}
    \item $\lambda$ is a successor.
    \item ${\mathcal{K}}$ is $({\text{LS}} ({\mathcal{K}})^+, < \lambda)$-tame, the model of size $\lambda$ is saturated, and there exists $\lambda_0 \in [{\text{LS}} ({\mathcal{K}})^+, \lambda)$ so that ${{{{\mathcal{K}}}^{{{\lambda_0}}\text{-sat}}}}_{[\lambda_0, \lambda)}$ has primes.
  \end{enumerate}
\end{lem}
\begin{proof}
  We prove the conclusion assuming the first condition. The proof assuming the second condition is similar (use Theorem \ref{prime-transfer-frame-2}). We first prove the result assuming that ${\mathcal{K}}$ is $({\text{LS}} ({\mathcal{K}}), <\theta)$-tame (not just weakly). We will then explain how to modify the proof so that it works also for weak tameness.

  By Proposition \ref{frame-existence}, there exists a type-full good $[{\text{LS}} ({\mathcal{K}})^+, \theta)$-frame ${\mathfrak{s}}$ with underlying class ${\mathcal{K}}_{\mathfrak{s}} := {{{{\mathcal{K}}}^{{{{\text{LS}} ({\mathcal{K}})^+}}\text{-sat}}}}_{[{\text{LS}} ({\mathcal{K}})^+, \theta)}$. Moreover, ${\mathcal{K}}_{\mathfrak{s}}$ is categorical in ${\text{LS}} ({\mathcal{K}})^+$. Thus we can apply Corollary \ref{good-categ-transfer} where ${\mathcal{K}}, {\text{LS}} ({\mathcal{K}})$ there stand for ${{{{\mathcal{K}}}^{{{{\text{LS}} ({\mathcal{K}})^+}}\text{-sat}}}}, {\text{LS}} ({\mathcal{K}})^+$ here.

  Now let us argue that $({\text{LS}} ({\mathcal{K}}), <\theta)$-weak tameness suffices. So assume that ${\mathcal{K}}$ is $({\text{LS}} ({\mathcal{K}}), <\theta)$-weakly tame. For the upward part, by Fact \ref{gv-upward-transfer}, ${{{{\mathcal{K}}}^{{{{\text{LS}} ({\mathcal{K}})^+}}\text{-sat}}}}$ is categorical in all $\mu \in [\lambda^+, \theta]$ (that the tameness assumption can be weakened to only $({\text{LS}} ({\mathcal{K}}), <\theta)$-weak tameness is implicit in Grossberg and VanDieren's paper and stated explicitly in Chapter 13 of \cite{baldwinbook09}). 

  It remains to show the downward part, i.e.\ that ${{{{\mathcal{K}}}^{{{{\text{LS}} ({\mathcal{K}})^+}}\text{-sat}}}}$ is categorical in all $\mu \in [{\text{LS}} ({\mathcal{K}})^+, \lambda^+)$. By Proposition \ref{frame-existence}, for each $\mu \in [{\text{LS}} ({\mathcal{K}})^+, \lambda^+)$, there exists a type-full good $\mu$-frame with underlying class ${{{{\mathcal{K}}}^{{{\mu}}\text{-sat}}}}_\mu$. By Theorem \ref{frame-canon-weak-tameness}, the frames agree with each other, i.e.\ for $\mu < \mu'$ both in $[{\text{LS}} ({\mathcal{K}})^+, \lambda^+)$, forking in the good $\mu'$-frame can be described naturally in terms of forking in the good $\mu$-frame. This is enough to make the proof of Corollary \ref{good-categ-transfer} go through (${\mathcal{K}}, {\text{LS}} ({\mathcal{K}}), \theta)$ there now stand for ${{{{\mathcal{K}}}^{{{{\text{LS}} ({\mathcal{K}})^+}}\text{-sat}}}}, {\text{LS}} ({\mathcal{K}})^+, \lambda^+$ here), we just have to make sure that anytime a resolution is taken, all the components are saturated.
\end{proof}

To deduce the main theorem of this paper, we will use:

\begin{fact}[The AEC omitting type theorem, II.1.10 in \cite{sh394}]\label{omitting-type}
  Let $\lambda > {\text{LS}} ({\mathcal{K}})$. If ${\mathcal{K}}$ is categorical in $\lambda$ and the model of size $\lambda$ is ${\text{LS}} ({\mathcal{K}})^+$-saturated, then every model in ${\mathcal{K}}_{\ge H_1}$ is ${\text{LS}} ({\mathcal{K}})^+$-saturated.
\end{fact}

\begin{thm}\label{main-thm}
  Let $\theta \ge \lambda > {\text{LS}} ({\mathcal{K}})^+$ be such that ${\mathcal{K}}$ is $({\text{LS}} ({\mathcal{K}}), <\theta)$-weakly tame and $\lambda$ is a successor. If ${\mathcal{K}}$ is categorical in $\lambda$, then ${\mathcal{K}}$ is categorical in all $\mu \in [\min (H_1, \lambda), \theta]$.

  In particular, if ${\mathcal{K}}$ is ${\text{LS}} ({\mathcal{K}})$-weakly tame and categorical in a successor $\lambda \ge H_1$, then ${\mathcal{K}}$ is categorical in all $\mu \ge H_1$.
\end{thm}
\begin{proof}
  By Lemma \ref{main-lem}, ${{{{\mathcal{K}}}^{{{{\text{LS}} ({\mathcal{K}})^+}}\text{-sat}}}}$ is categorical in all $\mu \in [{\text{LS}} ({\mathcal{K}})^+, \theta]$. Since $\lambda$ is regular, the model of size $\lambda$ is saturated hence ${\text{LS}} ({\mathcal{K}})^+$-saturated, so every model in ${\mathcal{K}}_{\ge \lambda}$ is ${\text{LS}} ({\mathcal{K}})^+$-saturated. By Fact \ref{omitting-type}, every model in ${\mathcal{K}}_{\ge H_1}$ is ${\text{LS}} ({\mathcal{K}})^+$-saturated. Thus every model in ${\mathcal{K}}_{\ge \min (H_1, \lambda)}$ is ${\text{LS}} ({\mathcal{K}})^+$-saturated, that is:

  $$
  {{{{\mathcal{K}}}^{{{{\text{LS}} ({\mathcal{K}})^+}}\text{-sat}}}}_{\ge \min (H_1, \lambda)} = {\mathcal{K}}_{\ge \min (H_1, \lambda)}
  $$

  The result follows.
\end{proof}

Next, we deduce Shelah's downward categoricity transfer \cite{sh394}. We will use the following fact which appears as \cite[Main Claim II.2.3]{sh394} (a simplified and improved argument is in \cite[Theorem 11.15]{baldwinbook09}):

\begin{fact}\label{weak-tameness-from-categ-fact}
  Let $\lambda > \mu \ge H_1$. Assume that ${\mathcal{K}}$ is categorical in $\lambda$, and the model of cardinality $\lambda$ is $\mu^+$-saturated. Then there exists $\chi < H_1$ such that ${\mathcal{K}}$ is $(\chi, \mu)$-weakly tame.
\end{fact}

\begin{cor}\label{shelah-alternate}
  If ${\mathcal{K}}$ is categorical in a successor $\lambda > H_2$, then ${\mathcal{K}}$ is categorical in all $\mu \in [H_2, \lambda]$.
\end{cor}
\begin{proof}
  Again, we can use Remark \ref{nomax-rmk} to assume without loss of generality that ${\mathcal{K}}$ has no maximal models. Since $\lambda$ is a successor, the model of size $\lambda$ is saturated. By Fact \ref{weak-tameness-from-categ-fact}, ${\mathcal{K}}$ is $(H_1, <\lambda)$-weakly tame. By Theorem \ref{main-thm} (where ${\mathcal{K}}, {\text{LS}} ({\mathcal{K}}), \lambda, \theta$ there stand for ${\mathcal{K}}_{\ge H_1}, H_1, \lambda, \lambda$ here), ${\mathcal{K}}$ is categorical in all $\mu \in [H_2, \lambda]$.
\end{proof}

An alternate proof of a special case of the upward transfer of Grossberg and VanDieren \cite{tamenesstwo, tamenessthree} can also be obtained, see Corollary \ref{gv-alternate} in the appendix. 
We can similarly deduce several consequences on tame AECs with primes. One of the main result of \cite{categ-primes-v3} was (the point compared to Shelah's downward categoricity transfer \cite{sh394} is that $\lambda$ need \emph{not} be a successor): 

\begin{fact}[Theorem 3.8 in \cite{categ-primes-v3}]\label{prime-fact}
  Assume that ${\mathcal{K}}$ is $H_2$-tame and ${\mathcal{K}}_{\ge H_2}$ has primes. If ${\mathcal{K}}$ is categorical in some $\lambda > H_2$, then it is categorical in all $\lambda' \ge H_2$.
\end{fact}

Using the methods of this paper, we can obtain categoricity in more cardinals provided that ${\mathcal{K}}$ has more tameness:

\begin{thm}\label{improved-prime-categ}
  Assume that ${\mathcal{K}}$ is ${\text{LS}} ({\mathcal{K}})$-tame and ${\mathcal{K}}_{\ge H_2}$ has primes. If ${\mathcal{K}}$ is categorical in some $\lambda > H_2$, then ${\mathcal{K}}$ is categorical in all $\lambda' \ge H_1$.
\end{thm}
\begin{proof}
  By Fact \ref{prime-fact}, ${\mathcal{K}}$ is categorical in $H_2^+$. By Theorem \ref{main-thm}, ${\mathcal{K}}$ is categorical in all $\lambda' \ge H_1$.
\end{proof}
\begin{remark}
  We could also use Lemma \ref{main-lem}. The key is that by \cite[II.1.6]{sh394} (see also \cite[Theorem 14.9]{baldwinbook09}), ${\mathcal{K}}$ is categorical in $H_2$, hence ${{{{\mathcal{K}}}^{{{H_2}}\text{-sat}}}} = {\mathcal{K}}_{\ge H_2}$. From the statement of Lemma \ref{main-lem}, we deduce that it is enough to assume that ${\mathcal{K}}_{[H_2, \lambda)}$ (rather than ${\mathcal{K}}_{\ge H_2}$) has primes. Similarly, we can assume only ${\text{LS}} ({\mathcal{K}})$-weak tameness, $(H_2, <\lambda)$-tameness, and that the model of size $\lambda$ is saturated.
\end{remark}

Specializing to universal classes, recall that the eventual categoricity conjecture holds there with a Hanf number of $\beth_{H_1}$. More precisely:

\begin{fact}[Theorem 6.8 in \cite{ap-universal-v8}]\label{univ-fact}
  Let ${\mathcal{K}}$ be a universal class (not necessarily with amalgamation). If ${\mathcal{K}}$ is categorical in some $\lambda \ge H_1$, then ${\mathcal{K}}$ is categorical in all $\lambda' \ge \min (\beth_{H_1}, \lambda)$.
\end{fact}

We can use our main theorem to obtain the full categoricity conjecture (i.e.\ the Hanf number is $H_1$) \emph{assuming amalgamation}.

\begin{cor}\label{improved-univ-categ}
  Let ${\mathcal{K}}$ be a universal class with amalgamation\footnote{See Hypothesis \ref{ap-hyp} and the remark following it.}. If ${\mathcal{K}}$ is categorical in a $\lambda \ge H_1$, then ${\mathcal{K}}$ is categorical in all $\lambda' \ge H_1$.
\end{cor}
\begin{proof}
  By Fact \ref{univ-fact}, ${\mathcal{K}}$ is categorical in $\lambda^+$. By \cite[Corollary 3.8]{ap-universal-v8}, ${\mathcal{K}}$ is ${\text{LS}} ({\mathcal{K}})$-tame. Now apply Theorem \ref{main-thm}.
\end{proof}

We finish with an improvement on the Hanf number of \cite[Theorem 1.6]{indep-aec-v5}. There we showed that assuming an unpublished claim of Shelah and weak GCH\footnote{Weak GCH is the statement that $2^\lambda < 2^{\lambda^+}$ for every cardinal $\lambda$.}, Shelah's eventual categoricity conjecture holds in fully tame and short\footnote{An AEC ${\mathcal{K}}$ is \emph{$(<\kappa)$-fully tame and short} if for every distinct $p, q \in {\text{gS}}^{\alpha} (M)$ there exists $I \subseteq \alpha$ and $A \subseteq |M|$ with $|I| + |A| < \kappa$ and $p^I {\upharpoonright} A \neq q^I {\upharpoonright} A$. See \cite[Definitions 3.1,3.3]{tamelc-jsl}.} AECs with amalgamation. We restate Shelah's unpublished claim here (see the introduction of \cite{indep-aec-v5} for a discussion\footnote{Note also that Shelah claims the same result in \cite[Theorem IV.7.12]{shelahaecbook} assuming only amalgamation. We are unable to verify Shelah's proof (the statement also contains an error that contradicts Morley's categoricity theorem). The Hanf numbers are also higher than what we prove below.}). This stems from \cite[Discussion III.12.40]{shelahaecbook}. A proof should appear in \cite{sh842}. For the rest of this section, we drop Hypothesis \ref{ap-hyp}.

\begin{claim}\label{claim-xxx}
  Assume the weak GCH. Let ${\mathfrak{s}}$ be an $\omega$-successful (see \cite[Definition III.1.12]{shelahaecbook}) good $\lambda$-frame with underlying class ${\mathcal{K}}_\lambda$. If ${\mathcal{K}}$ is categorical in $\lambda$ and in a $\mu > \lambda^{+\omega}$, then ${\mathcal{K}}$ is categorical in all $\mu' > \lambda^{+\omega}$.
\end{claim}

In fully tame and short superstable AECs, $\omega$-successful good frames can be built. The first construction appears in \cite[Section 11]{indep-aec-v5} and the Hanf numbers are improved in \cite[Lemma A.14]{ap-universal-v8}: We improve the Hanf numbers further here by taking into account recent improvements on when the union of a chain of saturated models is saturated.

\begin{fact}\label{omega-successful-constr}
Let ${\mathcal{K}}$ be an AEC with amalgamation. Assume that ${\mathcal{K}}$ is fully $(<\kappa)$-tame and short with $\kappa \le {\text{LS}} ({\mathcal{K}})^+$. Assume further that ${\mathcal{K}}$ is ${\text{LS}} ({\mathcal{K}})$-superstable. Let $\lambda > {\text{LS}} ({\mathcal{K}})$ be such that $\lambda = \lambda^{<\kappa}$. Then there exists an $\omega$-successful good $\lambda^+$-frame with underlying class ${{{{\mathcal{K}}}^{{{\lambda^+}}\text{-sat}}}}_{\lambda^+}$.
\end{fact}

We can now prove the promised categoricity transfer:

\begin{thm}\label{fully-tame-short-categ}
  Assume Claim \ref{claim-xxx} and the weak GCH.
  
  Let ${\mathcal{K}}$ be an AEC with amalgamation and no maximal models. Assume that ${\mathcal{K}}$ is fully $(<\kappa)$-tame and short with $\kappa \le {\text{LS}} ({\mathcal{K}})^+$ and $\kappa$ regular. If ${\mathcal{K}}$ is categorical in a $\lambda > \left({\text{LS}} ({\mathcal{K}})^{<\kappa}\right)^{+\omega}$, then ${\mathcal{K}}$ is categorical in all $\lambda' \ge \min (\lambda, H_1)$.
\end{thm}
\begin{proof}
  By Fact \ref{shvi}, ${\mathcal{K}}$ is ${\text{LS}} ({\mathcal{K}})$-superstable. Let $\mu := ({\text{LS}} ({\mathcal{K}})^{<\kappa})^+$. By Fact \ref{omega-successful-constr}, There exists an $\omega$-successful good $\mu$-frame with underlying class ${{{{\mathcal{K}}}^{{{\mu}}\text{-sat}}}}_{\mu}$. By Claim \ref{claim-xxx}, ${{{{\mathcal{K}}}^{{{\mu}}\text{-sat}}}}$ is categorical in every $\lambda' > \mu^{+\omega}$. Now ${{{{\mathcal{K}}}^{{{\mu}}\text{-sat}}}}_{\ge \lambda} = {\mathcal{K}}_{\ge \lambda}$. If $\lambda \le H_1$, the result follows. If $\lambda > H_1$, we have in particular that ${\mathcal{K}}$ is categorical in $\lambda^+$. By Theorem \ref{main-thm}, ${\mathcal{K}}$ is categorical in all $\lambda' \ge H_1$.
\end{proof}

We obtain Shelah's categoricity conjecture for certain fully tame and short AECs with amalgamation: 

\begin{cor}
  Assume Claim \ref{claim-xxx} and the weak GCH.
  
  Let ${\mathcal{K}}$ be an AEC with amalgamation which is:

  \begin{enumerate}
    \item ${\text{LS}} ({\mathcal{K}})$-weakly tame.
    \item Fully $\chi$-tame and short, for some $\chi < H_1$.
  \end{enumerate}

  If ${\mathcal{K}}$ is categorical in some $\lambda \ge H_1$, then ${\mathcal{K}}$ is categorical in all $\lambda' \ge H_1$. 
\end{cor}
\begin{proof}
  By Remark \ref{nomax-rmk}, without loss of generality ${\mathcal{K}}$ has no maximal models. By Theorem \ref{fully-tame-short-categ} (where ${\mathcal{K}}, \kappa$ there stand for ${\mathcal{K}}_{\ge \chi}$, $\chi^+$ here, note that $\left(2^{\chi}\right)^{+\omega} < H_1 \le \lambda$), ${\mathcal{K}}$ is categorical in all $\lambda' \ge {h ({\chi})}$. In particular, ${\mathcal{K}}$ is categorical in ${h ({\chi})}^+$. By Theorem \ref{main-thm}, ${\mathcal{K}}$ is categorical in all $\lambda' \ge H_1$.
\end{proof}

We can also state a transfer assuming the existence of large cardinals:

\begin{thm}
  Assume Claim \ref{claim-xxx} and the weak GCH.

  Let ${\mathcal{K}}$ be an AEC and let $\kappa > {\text{LS}} ({\mathcal{K}})$ be a strongly compact cardinal. If ${\mathcal{K}}$ is categorical in a $\lambda \ge {h ({\kappa})}$, then ${\mathcal{K}}$ is categorical in all $\lambda' \ge {h ({\kappa})}$.
\end{thm}
\begin{proof}
  By the methods of \cite{makkaishelah} (see the discussion before \cite[Proposition 7.3]{tamelc-jsl}), ${\mathcal{K}}_{\ge \kappa}$ has amalgamation and no maximal models. By \cite[Theorem 4.5]{tamelc-jsl}, ${\mathcal{K}}$ is fully $(<\kappa)$-tame and short. Now apply Theorem \ref{fully-tame-short-categ} to ${\mathcal{K}}_{\ge \kappa}$.
\end{proof}

\appendix

\section{More on canonicity}

Here, we prove: 

\begin{thm}\label{frame-canon-weak-tameness}
  Let ${\mathcal{K}}$ be an abstract elementary class and let $\theta > {\text{LS}} ({\mathcal{K}})$ be such that ${\mathcal{K}}_{{\text{LS}} ({\mathcal{K}}), \theta)}$ has amalgamation and ${\mathcal{K}}$ is $({\text{LS}} ({\mathcal{K}}), \theta)$-weakly tame. If ${\mathfrak{s}}$ is a type-full good ${\text{LS}} ({\mathcal{K}})$-frame with underlying class ${\mathcal{K}}_{{\text{LS}} ({\mathcal{K}})}$ and ${\mathfrak{s}}'$ is a type-full good $\theta$-frame with underlying class ${{{{\mathcal{K}}}^{{{\theta}}\text{-sat}}}}_\theta$, then for every $M {\le} N$ in ${{{{\mathcal{K}}}^{{{\theta}}\text{-sat}}}}_\theta$ and $p \in {\text{gS}} (N)$, $p$ does not ${\mathfrak{s}}'$-fork over $M$ if and only if there exists $M_0 {\le} M$ with $M_0 \in {\mathcal{K}}_{{\text{LS}} ({\mathcal{K}})}$ so that $p {\upharpoonright} N_0$ does not ${\mathfrak{s}}$-fork over $M_0$ for every $N_0 \in {\mathcal{K}}_{{\text{LS}} ({\mathcal{K}})}$ with $M_0 {\le} N_0 {\le} N$.
\end{thm}

Intuitively, this says that forking in ${\mathfrak{s}}'$ can be described by forking in ${\mathfrak{s}}$ in a canonical way (i.e.\ using Shelah's description of the extended frame, see \cite[Section II.2]{shelahaecbook}). We will use the following result, which gives an explicit description of forking in any categorical good frame:

\begin{fact}[The canonicity theorem, 9.6 in \cite{indep-aec-v5}]\label{canon-fact}
  Let ${\mathfrak{s}}$ be a type-full good $\lambda$-frame with underlying class ${\mathcal{K}}_\lambda$.
  If $M {\le} N$ are limit models in ${\mathcal{K}}_{\lambda}$, then for any $p \in {\text{gS}} (N)$, $p$ does not ${\mathfrak{s}}$-fork over $M$ if and only if there exists $M' \in {\mathcal{K}}_{\lambda}$ such that $M$ is limit over $M'$ and $p$ does not $\lambda$-split over $M'$.
\end{fact}

\begin{proof}[Proof of Theorem \ref{frame-canon-weak-tameness}]
  Note that by uniqueness of limit models, every model in ${{{{\mathcal{K}}}^{{{\theta}}\text{-sat}}}}_\theta$ is limit. 

  For $M, N \in {{{{\mathcal{K}}}^{{{\theta}}\text{-sat}}}}_\theta$ with $M {\le} N$, let us say that $p \in {\text{gS}} (N)$ \emph{does not ($\ge {\mathfrak{s}})$-fork over $M$} if it satisfies the condition in the statement of the theorem, namely there exists $M_0 {\le} M$ with $M_0 \in {\mathcal{K}}_{{\text{LS}} ({\mathcal{K}})}$ so that $p {\upharpoonright} N_0$ does not ${\mathfrak{s}}$-fork over $M_0$ for every $N_0 \in {\mathcal{K}}_{{\text{LS}} ({\mathcal{K}})}$ with $M_0 {\le} N_0 {\le} N$. Let us say that $p$ \emph{does not $\theta$-fork over $M$} if it satisfies the description of the canonicity theorem, namely there exists $M' \in {{{{\mathcal{K}}}^{{{\theta}}\text{-sat}}}}_{\theta}$ such that $M$ is limit over $M'$ and $p$ does not $\theta$-split over $M'$. Notice that by the canonicity theorem (Fact \ref{canon-fact}), $p$ does not ${\mathfrak{s}}'$-fork over $M$ if and only if $p$ does not $\theta$-fork over $M$. Thus it is enough to show that $p$ does not $(\ge {\mathfrak{s}})$-fork over $M$ if and only if $p$ does not $\theta$-fork over $M$. We first show one direction:
  
  \paragraph{\underline{Claim}} Let $M {\le} N$ both be in ${{{{\mathcal{K}}}^{{{\theta}}\text{-sat}}}}_\theta$ and let $p \in {\text{gS}} (N)$. If $p$ does not $(\ge {\mathfrak{s}})$-fork over $M$, then $p$ does not $\theta$-fork over $M$.
  
  \paragraph{\underline{Proof of Claim}} We know that $M$ is limit, so let ${\langle {M_i : i < \delta} \rangle}$ witness it, i.e.\ $\delta$ is limit, for all $i < \delta$, $M_i \in {{{{\mathcal{K}}}^{{{\theta}}\text{-sat}}}}_{\theta}$, $M_i {{<}_{\text{univ}}} M_{i + 1}$, and $\bigcup_{i < \delta} M_i = M$. By \cite[Claim II.2.11.(5)]{shelahaecbook}, there exists $i < \delta$ such that $p {\upharpoonright} M$ does not $(\ge {\mathfrak{s}})$-fork over $M_i$. By \cite[Claim II.2.11.(4)]{shelahaecbook}, $p$ does not $(\ge {\mathfrak{s}})$-fork over $M_i$. By weak tameness and the uniqueness property of ${\mathfrak{s}}$, $(\ge {\mathfrak{s}})$-forking has the uniqueness property (see the proof of \cite[Theorem 3.2]{ext-frame-jml}). By \cite[Lemma 4.2]{bgkv-v2}, $(\ge {\mathfrak{s}})$-nonforking must be extended by $\theta$-nonsplitting, so $p$ does not $\theta$-split over $M_i$. Therefore $p$ does not $\theta$-fork over $M$, as desired. $\dagger_{\text{Claim}}$.

  Now as observed above, $(\ge {\mathfrak{s}})$-forking has the uniqueness property. Also, $\theta$-forking has the extension property (as ${\mathfrak{s}}'$-forking has it). The claim tells us that $\theta$-nonforking extends $(\ge {\mathfrak{s}})$-forking and hence by \cite[Lemma 4.1]{bgkv-v2}, they are the same.
\end{proof}

\section{Superstability for long types}

We generalize Definition \ref{ss assm} and use it to prove the extension property for $1$-forking (recall Definition \ref{1-forking-def}). This is used to give a converse to Lemma \ref{perp-nf-1} in the next appendix. Everywhere below, ${\mathcal{K}}$ is an AEC.

\begin{defin}\label{ss-parametrized}
  Let $\alpha \le \omega$ be a cardinal. ${\mathcal{K}}$ is \emph{$(<\alpha, \mu)$-superstable} (or \emph{$(<\alpha)$-superstable in $\mu$}) if it satisfies Definition \ref{ss-parametrized} except that in addition in condition (\ref{split assm}) there we allow $p \in {\text{gS}}^{<\alpha} (M_\delta)$ rather than just $p \in {\text{gS}} (M_\delta)$ (that is, $p$ need not have length one). \emph{$(\le \alpha, \mu)$-superstable} means $(<\alpha^+, \mu)$-superstable. When $\alpha = 2$, we omit it (that is, $\mu$-superstable means $(< 2, \mu)$-superstable which is the same as $(\le 1, \mu)$-superstable).
\end{defin}

While not formally equivalent to Definition \ref{ss assm}, Definition \ref{ss-parametrized} is very close. For example, the proof of Fact \ref{shvi} also gives:

\begin{fact}\label{shvi2} 
  Let $\mu \ge {\text{LS}} ({\mathcal{K}})$. If ${\mathcal{K}}$ is has amalgamation, no maximal models, and is categorical in a $\lambda > \mu$, then ${\mathcal{K}}$ is $(<\omega, \mu)$-superstable.
\end{fact}

Even without categoricity, we can obtain eventual $(<\omega)$-superstability from just $(\le 1)$-superstability and tameness. This uses another equivalent definition of superstability: solvability:

\begin{thm}\label{tame-long-ss}
  Assume ${\mathcal{K}}$ has amalgamation, no maximal models, and is $(<{\text{LS}} ({\mathcal{K}}))$-tame. If ${\mathcal{K}}$ is ${\text{LS}} ({\mathcal{K}})$-superstable, then there exists $\mu_0 < H_1$ such that ${\mathcal{K}}$ is $(<\omega)$-superstable in every $\mu \ge \mu_0$.
\end{thm}
\begin{proof}[Proof sketch]
  By \cite[Theorem 5.43]{gv-superstability-v2}, there exists $\mu_0 < H_1$ such that ${\mathcal{K}}$ is $(\mu_0, \mu)$-solvable for every $\mu \ge \mu_0$. This means (\cite[Definition IV.1.4.(1)]{shelahaecbook}) that for every $\mu \ge \mu_0$, there exists an EM Blueprint $\Phi$ so that $\text{EM} (I, \Phi)$ is a superlimit in ${\mathcal{K}}$ for every linear order $I$ of size $\mu$. Intuitively, it gives a weak version of categoricity in $\mu$. As observed in \cite[Section 6]{gv-superstability-v2}, this weak version is enough for the proof of the Shelah-Villaveces theorem to go through, hence by Fact \ref{shvi2}, ${\mathcal{K}}$ is $(<\omega)$-superstable in $\mu$ for every $\mu \ge \mu_0$.
\end{proof}
\begin{remark}
  If ${\mathcal{K}}$ has amalgamation, is ${\text{LS}} ({\mathcal{K}})$-tame for types of length less than $\omega$, and is $(<\omega, {\text{LS}} ({\mathcal{K}}))$-superstable, then (by the proof of \cite[Proposition 10.10]{indep-aec-v5}) ${\mathcal{K}}$ is $(<\omega)$-superstable in every $\mu \ge {\text{LS}} ({\mathcal{K}})$. However here we want to stick to using regular tameness (i.e.\ tameness for types of length one).
\end{remark}

To prove the extension property for $1$-forking, we will use:

\begin{fact}[Extension property for splitting, 2.12.(3) in \cite{vv-structure-categ-v2}]\label{ext-splitting}
  Let ${\mathcal{K}}$ be an AEC, $\theta > {\text{LS}} ({\mathcal{K}})$. Let $\alpha \le \omega$ be a cardinal and assume that ${\mathcal{K}}$ is $(<\alpha)$-superstable in every $\mu \in [{\text{LS}} ({\mathcal{K}}), \theta)$. Let $M_0 {\le} M {\le} N$ be in ${\mathcal{K}}_{[{\text{LS}} ({\mathcal{K}}), \theta)}$, with $M$ limit over $M_0$. Let $p \in {\text{gS}}^{<\alpha} (M)$ be such that $p$ does not ${\text{LS}} ({\mathcal{K}})$-split over $M_0$. Then there exists an extension $q \in {\text{gS}}^{<\alpha} (N)$ of $p$ which does not ${\text{LS}} ({\mathcal{K}})$-split over $M_0$.
\end{fact}

\begin{thm}\label{piecewise-ext}
  Let $\theta > {\text{LS}} ({\mathcal{K}})$. Write ${\mathcal{F}} := [{\text{LS}} ({\mathcal{K}}), \theta)$. Let ${\mathfrak{s}}$ be a type-full good ${\mathcal{F}}$-frame with underlying class ${\mathcal{K}}_{\mathcal{F}}$. Let $\alpha \le \omega$ be a cardinal and assume that ${\mathcal{K}}$ is $(<\alpha, \mu)$-superstable for every $\mu \in {\mathcal{F}}$. Let $M {\le} N$ be in ${\mathcal{K}}_{\mathcal{F}}$ with $M$ a limit model. Let $p \in {\text{gS}}^{<\alpha} (M)$. Then there exists $q \in {\text{gS}}^{<\alpha} (N)$ that extends $p$ so that $q$ does not $1$-${\mathfrak{s}}$-fork over $M$ (recall Definition \ref{1-forking-def}).
\end{thm}
\begin{proof}
  Without loss of generality, $N$ is a limit model. Let $\mu := \|M\|$. By $(<\alpha)$-superstability, there exists $M_0 \in {\mathcal{K}}_{\mu}$ such that $M$ is limit over $M_0$ and $p$ does not $\mu$-split over $M_0$. By Fact \ref{ext-splitting}, there exists $q \in {\text{gS}} (N)$ extending $p$ so that $q$ does not $\mu$-split over $M_0$. We claim that $q$ does not $1$-${\mathfrak{s}}$-fork over $M$. Let $I \subseteq \ell (p)$ have size one. By monotonicity of splitting, $q^I$ does not $\mu$-split over $M_0$. By local character, let $N_0 {\le} N$ be such that $M {\le} N_0$, $N_0 \in {\mathcal{K}}_\mu$, $N_0$ is limit, and $q^I$ does not ${\mathfrak{s}}$-fork over $N_0$. By monotonicity of splitting again, $q^I {\upharpoonright} N_0$ does not $\mu$-split over $M_0$. By the canonicity theorem (Fact \ref{canon-fact}) applied to the frame ${\mathfrak{s}} {\upharpoonright} {\mathcal{K}}_\mu$, $q^I {\upharpoonright} N_0$ does not ${\mathfrak{s}}$-fork over $M$. By transitivity, $q^I$ does not ${\mathfrak{s}}$-fork over $M$, as desired.
\end{proof}

\section{More on global orthogonality}

Assuming superstability for types of length two, we prove a converse to Lemma \ref{perp-nf-1}, partially answering Question \ref{perp-nf-question}. We then prove a few more facts about global orthogonality and derive an alternative proof of a special case of the upward categoricity transfer of Grossberg and VanDieren \cite{tamenessthree}.

\begin{hypothesis}\label{appendix-3-hyp} \
  \begin{enumerate}
    \item ${\mathcal{K}}$ is an AEC.
    \item $\theta > {\text{LS}} ({\mathcal{K}})$ is a cardinal or $\infty$. We set ${\mathcal{F}} := [{\text{LS}} ({\mathcal{K}}), \theta)$.
    \item ${\mathfrak{s}} = ({\mathcal{K}}_{\mathcal{F}}, {\mathop{\copy\noforkbox}\limits})$ is a type-full good ${\mathcal{F}}$-frame.
    \item ${\mathcal{K}}$ is $(\le 2)$-superstable in every $\mu \in {\mathcal{F}}$.
  \end{enumerate}
\end{hypothesis}
\begin{remark}
  Compared to Hypothesis \ref{sec-5-hyp}, we have added $(\le 2)$-superstability. Note that this would follow automatically if ${\mathfrak{s}}$ was a type-full good frame for types of length two, hence it is quite a minor addition. It also holds if ${\mathcal{K}}$ is categorical above ${\mathcal{F}}$ (Fact \ref{shvi2}) or even if it is just tame (Theorem \ref{tame-long-ss}).
\end{remark}

\begin{lem}\label{perp-nf-2}
  Let $M_0 {\le} M$ be both in ${\mathcal{K}}_{\mathcal{F}}$ with $M_0 \in {\mathcal{K}}_{{\text{LS}} ({\mathcal{K}})}$ limit. Let $p, q \in {\text{gS}} (M)$ be nonalgebraic so that both do not fork over $M_0$. If $p {{{{\underset{{{\text{wk}}}}{\overset{{{{}}}}{{\perp}}}}}}} q$, then $p {\upharpoonright} M_0 {{{{\underset{{{\text{wk}}}}{\overset{{{{}}}}{{\perp}}}}}}} q {\upharpoonright} M_0$.
\end{lem}
\begin{proof}
  Assume that $p {\upharpoonright} M_0 \not {{{{\underset{{{\text{wk}}}}{\overset{{{{}}}}{{\perp}}}}}}} q {\upharpoonright} M_0$. We show that $p \not {{{{\underset{{{\text{wk}}}}{\overset{{{{}}}}{{\perp}}}}}}} q$. Let $N {\ge} M_0$, $a, b \in |N|$ realize in $N$ $p {\upharpoonright} M_0$ and $q {\upharpoonright} M_0$ respectively and such that ${\langle {ab} \rangle}$ is not independent in $(M_0, N)$. Let $r := {\text{gtp}} (ab / M_0; N)$. By Theorem \ref{piecewise-ext}, there exists $r' \in {\text{gS}}^2 (M)$ that extends $r$ and so that $r'$ does not $1$-${\mathfrak{s}}$-fork over $M_0$ (recall Definition \ref{1-forking-def}). Let $N' {\ge} M$ and let ${\langle {a'b'} \rangle}$ realize $r'$ in $N'$. Then ${\text{gtp}} (a' / M; N')$ does not fork over $M_0$ and extends $p {\upharpoonright} M_0$, hence $a'$ must realize $p$ in $N'$. Similarly, $b'$ realizes $q$. We claim that ${\langle {a'b'} \rangle}$ is \emph{not} independent in $(M, N')$, hence $p \not {{{{\underset{{{\text{wk}}}}{\overset{{{{}}}}{{\perp}}}}}}} q$. If ${\langle {a'b'} \rangle}$ were independent in $(M, N')$, there would exist $N'' {\ge} N'$ and $M' {\le} N''$ so that $M {\le} M'$, $b \in |M'|$, and ${\text{gtp}} (a' / M'; N'')$ does not fork over $M$. By transitivity, ${\text{gtp}} (a' / M'; N'')$ does not fork over $M_0$. This shows that ${\langle {a' b'} \rangle}$ is independent in $(M_0, N'')$, so since ${\text{gtp}} (a' b' / M_0; N'') = {\text{gtp}} (a b / M_0; N)$, we must have that ${\langle {ab} \rangle}$ is independent in $(M_0, N)$, a contradiction.
\end{proof}

We obtain:

\begin{thm}\label{perp-global}
  Let $M \in {\mathcal{K}}_{\mathcal{F}}$ and $p, q \in {\text{gS}} (M)$. Then:

  \begin{enumerate}
    \item If $M \in {{{{\mathcal{K}}}^{{{{\text{LS}} ({\mathcal{K}})}}\text{-sat}}}}_{\mathcal{F}}$, then $p \perp q$ if and only if $p {{{{\underset{{{\text{wk}}}}{\overset{{{{}}}}{{\perp}}}}}}} q$.
    \item If $M \in {{{{\mathcal{K}}}^{{{{\text{LS}} ({\mathcal{K}})}}\text{-sat}}}}_{\mathcal{F}}$, then $p \perp q$ if and only if $q \perp p$.
    \item If $M_0 \in {{{{\mathcal{K}}}^{{{{\text{LS}} ({\mathcal{K}})}}\text{-sat}}}}_{\mathcal{F}}$ is such that $M_0 {\le} M$ and both $p$ and $q$ do not fork over $M_0$, then $p \perp q$ if and only if $p {\upharpoonright} M_0 \perp q {\upharpoonright} M_0$.
  \end{enumerate}
\end{thm}
\begin{proof} \
  \begin{enumerate}
  \item If $p \perp q$, then $p {{{{\underset{{{\text{wk}}}}{\overset{{{{}}}}{{\perp}}}}}}} q$ by definition. Conversely, assume that $p {{{{\underset{{{\text{wk}}}}{\overset{{{{}}}}{{\perp}}}}}}} q$. Fix a limit $M_0 \in {\mathcal{K}}_{{\text{LS}} ({\mathcal{K}})}$ such that $M_0 {\le} M$ and both $p$ and $q$ do not fork over $M_0$. By Lemma \ref{perp-nf-2}, $p {\upharpoonright} M_0 {{{{\underset{{{\text{wk}}}}{\overset{{{{}}}}{{\perp}}}}}}} q {\upharpoonright} M_0$. By Lemma \ref{perp-fund}.(\ref{perp-fund-1}), $p {\upharpoonright} M_0 \perp q {\upharpoonright} M_0$. By Lemma \ref{perp-nf-1}, $p \perp q$.
  \item A similar proof, using (\ref{perp-fund-2}) instead of (\ref{perp-fund-1}) in Lemma \ref{perp-fund}.
  \item By local character and transitivity, we can fix a limit $M_0' \in {\mathcal{K}}_{{\text{LS}} ({\mathcal{K}})}$ such that $M_0' {\le} M_0$ and both $p$ and $q$ do not fork over $M_0'$. Now by what has been proven above and Lemmas \ref{perp-nf-1} and \ref{perp-nf-2}, $p \perp q$ if and only if $p {\upharpoonright} M_0' \perp q {\upharpoonright} M_0'$ if and only if $p {\upharpoonright} M_0 \perp q {\upharpoonright} M_0$.
  \end{enumerate}
\end{proof}

We can now give another proof of the upward transfer of unidimensionality (the second part of the proof of Theorem \ref{unidim-transfer}). This does not use Fact \ref{upward-transfer-2}.

\begin{lem}\label{unidim-upward}
  Let $\mu < \lambda$ be in ${\mathcal{F}}$. If ${\mathfrak{s}}$ is $\mu$-unidimensional, then ${\mathfrak{s}}$ is $\lambda$-unidimensional.
\end{lem}
\begin{proof}
  Assume that ${\mathfrak{s}}$ is \emph{not} $\lambda$-unidimensional. Let $M_0 \in {\mathcal{K}}_{\mu}$ be limit and let $p_0 \in {\text{gS}} (M_0)$ be minimal. We show that there exists a limit $M_0' \in {\mathcal{K}}_{\mu}$, $p_0', q_0' \in {\text{gS}} (M_0')$ such that $p_0'$ extends $p_0$ and $p_0' \perp q_0'$. This will show that ${\mathcal{K}}$ is not $\mu$-unidimensional by Lemma \ref{technical-multidim-equiv}. Let $M \in {\mathcal{K}}_\lambda$ be saturated such that $M_0 {\le} M$ and let $p \in {\text{gS}} (M)$ be the nonforking extension of $p_0$. By non-$\lambda$-unidimensionality (and Lemma \ref{technical-multidim-equiv}), there exists $q \in {\text{gS}} (M)$ so that $p \perp q$. Let $M_0' \in {\mathcal{K}}_\mu$ be limit such that $M_0 {\le} M_0' {\le} M$ and $q$ does not fork over $M_0'$. Let $p_0' := p {\upharpoonright} M_0'$, $q_0' := q {\upharpoonright} M_0'$. By Theorem \ref{perp-global}, $p_0' \perp q_0'$, as desired.
\end{proof}

We obtain the promised alternate proof to Grossberg-VanDieren. Note however that the hypotheses we have are stronger than in \cite{tamenessthree}: we ask for more tameness and categoricity above the Hanf number. Intuitively, this is because we are using a lot of room to setup the abstract machinery of good frames and obtain $(<\omega)$-superstability.

For this corollary, we drop Hypothesis \ref{appendix-3-hyp}. 

\begin{cor}\label{gv-alternate}
  Let ${\mathcal{K}}$ be an AEC with amalgamation. If ${\mathcal{K}}$ is $(<{\text{LS}} ({\mathcal{K}}))$-tame and categorical in a successor $\lambda \ge H_1$, then ${\mathcal{K}}$ is categorical in all $\mu \ge \lambda$.
\end{cor}
\begin{proof}
  By Remark \ref{nomax-rmk}, we can assume without loss of generality that ${\mathcal{K}}$ has no maximal models. By Fact \ref{shvi}, ${\mathcal{K}}$ is ${\text{LS}} ({\mathcal{K}})$-superstable. By Theorem \ref{tame-long-ss}, ${\mathcal{K}}$ is $(<\omega)$-superstable in every $\mu \ge H_1$. Now say $\lambda = \lambda_0^+$. By Proposition \ref{frame-existence}, there exists a type-full good $(\ge \lambda_0)$-frame ${\mathfrak{s}}$ with underlying class ${{{{\mathcal{K}}}^{{{{\text{LS}} ({\mathcal{K}})^+}}\text{-sat}}}}_{\ge H_1}$. By Fact \ref{satfact-1}, we can restrict the frame further to have underlying class ${{{{\mathcal{K}}}^{{{\lambda_0}}\text{-sat}}}}_{\ge \lambda_0}$. By Corollary \ref{good-categ-transfer} (using Lemma \ref{unidim-upward} to transfer unidimensionality up), ${{{{\mathcal{K}}}^{{{\lambda_0}}\text{-sat}}}}$ is categorical in every $\mu \ge \lambda_0$. Now ${{{{\mathcal{K}}}^{{{\lambda_0}}\text{-sat}}}}_{\ge \lambda} = {\mathcal{K}}_{\ge \lambda}$ (by categoricity in $\lambda$), so the result follows.
\end{proof}

\bibliographystyle{amsalpha}
\bibliography{categ-good-aecs}

\end{document}


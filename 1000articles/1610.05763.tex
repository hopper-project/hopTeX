\documentclass[12pt]{amsart}
\usepackage{amssymb}
\usepackage{amssymb,amscd,amsmath,amsthm}

\setlength{\topmargin}{-0.5in}
\setlength{\oddsidemargin}{-0.25in}
\setlength{\evensidemargin}{-0.25in}
\setlength{\textwidth}{7.0in}
\setlength{\textheight}{9.0in}

\newtheorem{thm}{Theorem}
\newtheorem{cor}[thm]{Corollary}
\newtheorem{prop}[thm]{Proposition}
\newtheorem{lemma}[thm]{Lemma}
\newtheorem{remark}[thm]{Remark}
\newtheorem{example}[thm]{Example}
\newtheorem{definition}[thm]{Definition}
\newtheorem{conj}[thm]{Conjecture}

\begin{document}

\title[Non-homogeneous Fourier matrices]{Classification of non-homogeneous Fourier matrices associated with modular data up to rank 5}

\author[G. Singh]{Gurmail Singh}
\address{Department of Mathematics and Statistics, University of Regina, Regina, Canada, S4S 0A2}
\email{Gurmail.Singh@uregina.ca}

\date{}

\keywords{Integral modular data, Fusion rings, $C$-algebra, Reality-based algebra.}

\subjclass[2000]{Primary 05E30, Secondary , 05E99, 81R05}

\begin{abstract}
Modular data is an important topic of study in rational conformal field theory, \cite{TG}. A modular datum defines finite dimensional representations of the modular group $\mbox{SL}_2({{\mathbb Z}})$. For every Fourier matrix in a modular datum there exists an Allen matrix obtained from the Fourier matrix after dividing each its row with the first entry of that row. In this paper, we classify the non-homogenous Fourier matrices and non-homogenous Allen matrices up to rank $5$. The methods developed here are useful for the classification of the matrices of higher ranks. Also, we establish some results that are helpful in recognizing the $C$-algebras not arising from Allen matrices by just looking at the character table of the $C$-algebra, in particular, the first row of the character table.
\end{abstract}

\maketitle

\section{Introduction}

\medskip
A modular datum defines finite dimensional representations of the modular group $\mbox{SL}_2({{\mathbb Z}})$. For every Fourier matrix $S$ in a modular datum there exists a fusion algebra defined by the structure constants of the Fourier matrix
, see \cite{MC1} and \cite{TG}. But we shall apply two step scaling on the Fourier matrices and get a $C$-algebra that has nice properties. The first scaling is applied on the rows of the Fourier matrix $S$ by dividing the each row by its first entry and obtain the \emph{Allen matrix} $s$. The second scaling is applied on the columns of the Allen matrix $s$ by multiplying  each column of $s$ by its first entry and we obtain the matrix $P$. The matrix $P$ is the character table of the $C$-algebra generated by the columns of $P$ under componentwise multiplication and we call the algebra a $C$-algebra arising from an Allen matrix, whereas the columns of the Allen matrix $s$ generate the fusion algebra but not the $C$-algebra. In this article, we shall use the interesting properties of the $C$-algebras arising from Allen matrices that comes from the interrelationship degrees and the norms of the rows of the Allen matrix, consequently, we get the strong results.

\smallskip

Let $A$ be a finite dimensional and commutative algebra over ${{\mathbb C}}$ with  distinguished basis  $\mathbf{B} = \{b_0, b_1, \hdots, b_{r-1}\}$, and an ${{\mathbb R}}$-linear and ${{\mathbb C}}$-conjugate linear involution $*:A \rightarrow A$. Let $\delta: A \rightarrow {{\mathbb C}}$ be an algebra homomorphism. Then the triple $(A,{{\mathbf B}},\delta)$ is called a \emph{$C$-algebra} if  satisfies the following properties:

\begin{enumerate}
\item $b_0 =b_0^*= 1_A \in \mathbf{B}$,

\item for all $b_i \in \mathbf{B}$, $(b_i)^* = b_{i^*} \in \mathbf{B}$,
\item the structure constants of $A$ with respect to the basis $\mathbf{B}$ are real numbers, i.e.~for all $b_i, b_j \in \mathbf{B}$, we have
$$b_ib_j = \sum_{b_k \in \mathbf{B}}^{}\lambda_{ijk}b_k, \mbox{ for some } \lambda_{ijk} \in \mathbb{R},$$
\item  for all $b_i, b_j \in \mathbf{B}, \lambda_{ij0} \neq 0 \iff j = i^*$,
\item for all $b_i \in \mathbf{B}, \lambda_{ii^*0} = \lambda_{i^*i0} > 0$.
\item for all $b_i\in {{\mathbf B}}$, $\delta(b_i)=\delta(b_{i^*})>0$.
\end{enumerate}

Let $(A,{{\mathbf B}},\delta)$ be a $C$-algebra.  Then $A$ is called \emph{symmetric} if $b_{i^*}=b_i$ for all $i$. The algebra homomorphism $\delta$ is called a \emph{degree map}, and $\delta(b_i)$, for all $b_i\in {{\mathbf B}}$, are  called the degrees. For each $i\neq 0$, $\delta(b_i)$ is called a nontrivial degree and $\delta(b_0)=1$ is called a trivial degree. If in addition, $\delta(b_i)=\lambda_{ii^*0}$ for all $b_i\in {{\mathbf B}}$ then we call the basis ${{\mathbf B}}$ standard basis. The algebra $A$ is a $C^*$-algebra whose involution satisfies $\alpha^* = \sum_i \bar{\alpha}b_{i^*}$, for all $\alpha = \sum_i \alpha_i b_i \in A$.   In particular, $A$ is a $r$-dimensional semisimple algebra, see \cite{HSnew1}, \cite{G1}.

Group algebras of finite abelian groups, Bose-Mesner algebras of finite commutative association schemes, and commutative fusion rule algebras are all special types of $C$-algebras.  For the detailed discussion on $C$-algebras and its non-commutative generalizations, see \cite{AFM}, \cite{Hig87},  \cite{HSnew1}, \cite{G1}, and references therein. Every $C$-algebra with positive degree map has a unique standard basis, which can be arranged by rescaling each basis element by $\delta(b_i)/\lambda_{ii^*0}$.

\smallskip

Let $(A,\mathbf{B}, \delta)$ be $C$-algebra. Then, we define its {\it order} to be
$$ n := \delta(\mathbf{B}^+) = \sum_{i=0}^d \delta(b_i), $$
and its {\it standard feasible trace} to be $\rho: A \rightarrow \mathbb{C}$ with $\rho(\alpha) = n \alpha_0$ for all $\alpha = \sum_i \alpha_i b_i \in A$.  Note that $\rho$ really is a trace function on $A$ because $\rho(\alpha\beta)=\rho(\beta\alpha)$ for all $\alpha, \beta \in A$.    Positivity of the degree map makes $A$ into a Frobenius algebra with nondegenerate hermitian form
$$ \langle \alpha, \beta \rangle = \rho(\alpha \beta^*), \quad \mbox{ for all } \alpha,\beta \in A. $$
Being a nonsingular trace function on the finite-dimensional Frobenius algebra $A$, $\rho$ can be expressed as a  linear combination of the irreducible characters of $A$, see \cite{Hig87}.  Let $Irr(A)$ denote the set of irreducible characters of $A$. The coefficients $m_{\chi}$ in this linear combination $\rho = \sum_{\chi} m_{\chi} \chi$ are the {\it multiplicities} of a $C$-algebra, where sum runs over $\chi \in Irr(A)$. The multiplicities are always positive real numbers, see \cite{HIB2}.
The standard feasible trace is a character, known as \emph{standard character} precisely when all the multiplicities are positive rational integers, see \cite{JA}, \cite{HIB2} and  \cite{HS2}. The standard feasible trace is a \emph{pseudo-standard character} if all the multiplicities are not integers but rational numbers, see \cite{HSnew2}.

\smallskip

Let $(A,\mathbf{B},\delta)$ be a $C$-algebra with order $n$.  Higman's character formula
$$ e_{\chi} = \frac{m_{\chi}}{n} \sum_i \frac{\chi(b_i^*)}{\lambda_{ii^*0}}b_i $$
expresses the centrally primitive idempotents of $A$ in terms of the standardized basis $\mathbf{B}$, see  \cite{Hig87}.
Since $\chi(b_{i^*})=\overline{\chi(b_i)}$ for all $b_i \in \mathbf{B}$, we have
$$\chi(e_{\chi}) = \chi(b_0) = \frac{m_{\chi}}{n}\sum_i \frac{\overline{\chi(b_i)}\chi(b_i)}{\lambda_{ii^*0}}.$$
$\chi(b_0)$ is the degree of $\chi$ so it is always a positive integer.  When the degrees $\delta(b_i)$ are all positive and real, then $n$ is positive and real, so the multipicity $m_{\chi}$ is positive and real, see \cite{HIB1}.   It follows then that $e_{\chi}^* = e_{\chi}$.
From the above expression for central idempotents $e_\chi$ we obtain the   orthogonality relation \cite{Hig87}
$$\sum\limits_{k=0}^d \frac{\chi_i(b_{k^*})}{\lambda_{kk^*0}}\chi_j(b_k)=\delta_{ij}\frac {\chi_i(b_0)\delta({{\mathbf B}}^+)}{m_{\chi_i}}.$$
Though we remark that the above orthogonality relation is also true for the noncommutative generalizations of $C$-algebras, see \cite{AFM},  \cite{HSnew1}, \cite{Hig87} and \cite{G1}.

\medskip

In Section 2, we give the definition of  Allen matrices and construct the $C$-algebras with  bases consist of the columns of the Allen matrices, the matrices derived from Fourier matrices, such that  the componentwise multiplication on the columns generate integral structure constants. In Section 3, we summarize the results that are useful to recognize the $C$-algebras not arising from Allen matrices. In Section 4, first we prove an interesting theorem that is useful for the other  remaining results. Then we classify the non-homogenous Fourier matrices and the non-homogenous Allen matrices with the condition that one of the nontrivial degrees is equal to $1$. It is not hard recover the Allen matrices and Fourier matrices from the character tables of the $C$-algebras arising from Allen matrices. Therefore, in this section we give the character tables of the $C$-algebras arising from Allen matrices. In Section 5, we classify the integral Fourier matrices up to rank $5$. Also, in the previous section we point out that there is no integral Fourier matrix of rank $3$ and there is just one integral Fourier matrix of rank $2$.

\section{Construction of $C$-algebras arising from Allen matrices}

In this section, we give definitions of the Fourier matrices and construct $C$-algebras from them. The Fourier matrices are commonly studied in investigations of modular data and fusion algebras arising from them, see \cite{MC} and  \cite{MC1}. The special cases include Kac-Peterson matrices, Hadamard matrices, and the matrices  corresponding to the Grothendieck rings of finite groups, see \cite{MC}, \cite{MC1} and \cite{TG}. Modular data and quasi-modular data are defined differently in the literature. To keep the generality, we assume the structure constants to be integers instead of nonnegative integers, see \cite{BI}, \cite{MC}, \cite{MC1} and \cite{TG}.

\begin{definition}\label{ModularDef} Let $r\in {{\mathbb Z}}^+$ and  $I$ be an $r \times r$ identity matrix. A pair $(S,T)$ of $r\times r$ complex matrices is called modular datum  (quasi-modular datum, respectively) if
\begin{enumerate}
\item $S$ is a unitary and symmetric matrix, so $S\bar{S}^T = 1, S= S^T$,
\item  $T$ is diagonal matrix and of finite order,
\item $S_{i0}>0$ for $0\leq i\leq r-1$,
\item $(ST)^3=S^2$ $\big($$(ST)^3=I$, respectively$\big)$,
\item $N_{ijk}= \sum_l{S_{li}S_{lj}\bar S_{lk}}{S^{-1}_{l0}} \in {{\mathbb Z}}$, for all $0\leq i,j,k\leq r-1$.
\end{enumerate}
 \end{definition}

\begin{definition}
A matrix $S$ satisfying the axioms $(i), (iii)$ and $(v)$ of Definition \ref{ModularDef} is called a \emph{Fourier matrix}.
\end{definition}

\begin{definition} Let $S$ be a Fourier matrix. We call a matrix $s =[s_{ij}]$ an \emph{Allen matrix} if $s_{ij}= {S_{ij}}{S^{-1}_{i0}}$ for all $i,j$.
\end{definition}

The entries of an Allen matrix are algebraic integers, see \cite{MC1} and \cite[Proposition 2.17]{HIB2}. Therefore, in fact, an Allen matrix with rational entries has rational integer entries. Since $s_{ij}= {S_{ij}}{S^{-1}_{i0}}$ for all $i,j$, $S \in {{\mathbb R}}^{r\times r}$ if and only if $s \in {{\mathbb R}}^{r\times r}$. Thus we have the following definition.

\begin{definition}
Let $S$ be a Fourier matrix and $s$ be the corresponding Allen matrix. If $S \in {{\mathbb R}}^{r\times r}$ $(S \in {{\mathbb Q}}^{r\times r},$ resp.$)$ then we call the corresponding Allen matrix $s$ a real (integral, resp.) Allen matrix.
\end{definition}

Note that for an Allen matrix $s$, $s_{i0} = 1$ for all $0\leq i \leq r-1$ and  $s\bar s^T$ is a diagonal matrix. If $s \bar s^T=$diag$(d_0,d_1,\hdots, d_{r-1})$, where $d_i=\sum_js_{ij}\bar s_{ij}$ then  $N_{ijk}= \sum_l{s_{li}s_{lj}\bar s_{lk}}{d^{-1}_l}$ for all $0\leq i,j,k\leq r-1$. Also,  $d_i|s_{ji}|^2 = d_j|s_{ij}|^2$,  for all $0\leq i,j\leq r-1$.
If an Allen matrix $s$ and an associated diagonal matrix $T$ are integral matrices then the modular datum $(s,T)$ is called an \emph{integral modular datum} and the Allen matrix $s$ is called an \emph{integral Fourier matrix}. The study of integral modular datum is of independent interest. The following definitions of integral modular datum and integral Fourier matrix are from Cuntz's paper \cite{MC1}.

\begin{definition}\label{IntModularDef} Let $r\in {{\mathbb Z}}^+$. A pair $(s,T)$ of $r\times r$ integral matrices is called an integral modular datum if
\begin{enumerate}
\item $s_{i0} = 1$ for $0\leq i\leq r-1$, det$(s)\neq 0$,
\item  $ss^T=$diag$(d_0,d_1, \hdots, d_{d})$, where $d_i=\sum_js^2_{ij}$,
\item $\sqrt{d_j}s_{ij}=\sqrt{d_i}s_{ji}$ for $0\leq i,j\leq r-1$, so $s$ is symmetrizable,
\item $N_{ijk}= \sum_l {s_{li}s_{lj}s_{lk}}{d^{-1}_l} \in {{\mathbb Z}}$, for all $0\leq i,j,k\leq r-1$,
\item $T$ is diagonal,
\item $S^2 = (ST)^3$, where $S_{ij}=s_{ij}/\sqrt{d_i}$.
\end{enumerate}
 \end{definition}

\begin{definition}
A matrix $s$ satisfying the axioms $(i), (iii)$ and $(v)$ of Definition \ref{IntModularDef}  is called an \emph{integral Fourier matrix}.
\end{definition}

Thus an integral Fourier matrix is in fact an integral Allen matrix. Therefore, every result that is true for Allen matrices is also true for integral Fourier matrices.
Also, the real Allen matrices of a specific class also become integral Fourier matrices, see \cite[Theorem 25]{G1}. But an Allen matrix need not be an integral Fourier matrix, for example, the character table of cyclic group of order $4$ is an Allen matrix but not an integral Fourier matrix.

\begin{definition}
Let $s$ an Allen matrix (integral Fourier matrix) and $s \bar s^T=$diag$(d_0,d_1,\hdots, d_{r-1})$, where $d_i=\sum_js_{ij}\bar s_{ij}$ for all $0\leq i,j\leq r-1$. The magnitudes of the row vectors of $s$, $d_0,d_1,\hdots,d_{r-1}$ are called norms. We call $d_0 (=\delta({{\mathbf B}}^+))$ the principal norm and $d_1,d_2,\hdots,d_{r-1}$  non-principal norms. The principal norm $d_0$ is also known as the \emph{size of a modular datum}.
\end{definition}

There is an interesting row-and-column operation procedure that can be applied to a Fourier matrix $S$ that results in a first eigenmatrix $P$, the character table, of a self-dual $C$-algebra. The self-duality property comes from the symmetry of the Fourier matrix $S$.
At the first step, we scale the rows of a Fourier matrix $S$ to get an Allen matrix $s$ by dividing the each row by its first entry. At the second step, if we further scale the columns of the Allen matrix $s$ by multiplying each column of $s$ by its first entry, we get the matrix $P$ with entries $p_{ij}=s_{ij}s_{0j}$ for all $0\leq i,j\leq r-1$. The matrix $P$ obtained in such a way is a character table of the self-dual $C$-algebra with standard basis consists of the columns of $P$ under componentwise multiplication, see \cite[Theorem 11]{G1}. Thus we call the matrix $P$ a first eigenmatrix.
We can reverse the process to get the Fourier matrix $S$ from the first eigenmatrix $P$. The Allen matrix $s$ is retrieved by dividing each column of $P$ by the square root of its first entry: $s_{ij} = \dfrac{p_{ij}}{\sqrt{p_{0j}}}$. Further, to retrieve the Fourier matrix $S$ we divide the each row of the Allen matrix $s$ by its norm: $S_{ij}=\dfrac{s_{ij}}{\sqrt{d_i}}$, where $d_{i}=\sum_js_{ij}\bar s_{ij}$. For the detailed discussion on the correspondence of matrices $S$, $s$ and $P$ see \cite{G1}. Thus it not hard to obtain an Allen matrix from a Fourier matrix and a Fourier matrix from an Allen matrix. Also, both Allen matrix $s$ and Fourier matrix  $S$ can be recovered  from the first eigenmatrix $P$ and $P$ can be  obtained from the matrices $s$  and $S$. Therefore, it is enough to classify either one of these three matrices. In Section 4, we shall give the first eigenmatrices of rank $2$ to $5$.

\begin{remark}\rm Throughout of this paper we use the following notations. Let $S(=[S_{ij}])$ be a Fourier matrix. Then  $s(=[s_{ij}])$ and $P(=[P_{ij}])$ denote the Allen matrix and the first eigenmatrix, respectively. We label the rows and columns with $0,1,2, \dots$.
The set of the columns of matrices $P$ and $s$ are denoted by ${{\mathbf B}}=\{b_0,b_1,\hdots, b_{r-1}\}$ and $\tilde {{\mathbf B}}=\{\tilde b_0, \tilde b_1,\hdots, \tilde b_{r-1}\}$, respectively. The structure constants generated by the columns, with the componentwise multiplication, of $P$ and $s$ are denoted by $\lambda_{ijk}$ and $N_{ijk}$, respectively. For any matrix $B$  the transpose of $B$ is denoted by $B^T$. A primitive $n$th root of unity is denoted by $\zeta_n$.
\end{remark}

For the remaining part of this section we collect some results and definitions about $C$-algebras arising from Allen matrices that are used frequently in this paper. Although the proofs of these results can be found in the author's another paper, see \cite{G1}, but to show that how the construction of $C$-algebra from Allen matrices works we include the proof of the one result here.

\smallskip

Let $S$ be a Fourier matrix. Let ${{\mathbf B}}$ and $\tilde{{\mathbf B}}$ be the sets of the columns of matrices $P$ and $s$, respectively. Let ${{\mathbb C}}{{\mathbf B}}$ and ${{\mathbb C}}\tilde{{\mathbf B}}$ be linear span of ${{\mathbf B}}$ and $\tilde{{\mathbf B}}$ over the field of  complex numbers. The ${{\mathbb C}}$-conjugate linear involution $*$ on columns of $S$ is given by the involution on elements of $S$,  defined as $(S_{ij})^*=S_{ij^*}=\bar S_{ij}$ for all $i,j$. If $S_j$ denotes the $j$th column of a Fourier matrix $S$ then the involution on $S_j$ is defined as $(S_j)^*=S_{j^*}=[S_{0j^*}, S_{1j^*}, \hdots, S_{(r-1),j^*}]^T$. Since $b_i=s_{0i}\tilde b_{i}$, the structure constants generated by the basis ${{\mathbf B}}$ are given by $\lambda_{ijk}={N_{ijk}s_{0i}s_{0j}}{s^{-1}_{0k}}$, for all $i,j, k$.  As $S$ is a unitary and symmetric matrix, therefore, $N_{ij0}= \sum_l{S_{li}S_{lj}\bar S_{l0}}{S^{-1}_{l0}}\neq 0 \iff j=i^*$ and $N_{ii^*0}=1>0$ for all $i,j$. Hence, $\lambda_{ij0}\neq 0 \iff j=i^*$  and $\lambda_{ii^*0}>0$ for all $i,j$. Also, the first column each of the matrices $P$ and $s$ is an identity element. Therefore, tthe vector spaces ${{\mathbb C}}{{\mathbf B}}$ and ${{\mathbb C}}\tilde{{\mathbf B}}$ satisfy the first five axioms of the definition of a $C$-algebra.

In the next theorem we show that corresponding to every Allen matrix there exists a $C$-algebra.

\begin{thm}Let $S$ be a Fourier matrix. Let ${{\mathbf B}}$ the set of the columns of matrices $P$. Prove that the vector space ${{\mathbb C}}{{\mathbf B}}$ is a $C$-algebra and ${{\mathbf B}}$ is the standard basis.
\end{thm}

\begin{proof}We just need to check the last axiom of the definition of a $C$-algebra. Let $\tilde {{\mathbf B}}$ and ${{\mathbf B}}$ denote the columns of $s$ and $P$ respectively. Define a ${{\mathbb C}}$-conjugate linear map $\delta:A\longrightarrow {{\mathbb C}}$  as  $\delta(\sum_ia_i\tilde{b_i})=\sum_i\bar a_is_{0i}$. Thus $\delta(b_i)= \delta(s_{0i}\tilde b_{i}) = s^2_{0i}$, hence $\delta$ is positive valued. Since $b_{i^*}= \bar s_{0i}\tilde{b}_{i^*}=s_{0i}\tilde{b}_{i^*}$, $(b_ib_{i^*})_0=s_{0i}s_{0i}(\tilde b_i \tilde b_{i^*})_0= s^2_{0i}=\delta(b_i)$. The map $\delta$ is an algebra homomorphism, to see:
$$\begin{array}{rcl}\delta(b_i b_j)&=&
s_{0i}s_{0j}\sum\limits^{}_{k}N_{ijk}\delta(\tilde{b_k})
\\&=&s_{0i}s_{0j}[\dfrac{s_{0i}s_{0j}}{d_0} \sum\limits^{r-1}_{k=0}\bar s_{0k}s_{0k}]+s_{0i}s_{0j}[\sum\limits^{r-1}_{l=1}\dfrac{s_{li}s_{lj}}{d_l}
\sum\limits^{r-1}_{k=0}\bar s_{lk}s_{0k}]
\\&=&s_{0i}s_{0j}[\dfrac{s_{0i}s_{0j}}{d_0} d_0]+s_{0i}s_{0j}[0]
\\&=&s^2_{0i}s^2_{0j}=\delta(b_i)\delta(b_j)\end{array}$$
In the third last equality we use the fact that $ \sum\limits^{n}_{k=1} s_{0k}\bar  s_{0k} =d_0$ and the rows of $s$ are orthogonal.
\end{proof}

Let $\tilde {{\mathbf B}}$ and ${{\mathbf B}}$ denote the columns of $s$ and $P$ respectively.  Then the vector space ${{\mathbb C}}\tilde{{\mathbf B}}$ not necessarily always has a degree map. But the vector space ${{\mathbb C}}\tilde{{\mathbf B}}$ is a fusion algebra. By the above theorem, the vector space ${{\mathbb C}}{{\mathbf B}}$ has a degree map. In this paper we make use of this rich property of the $C$-algebras and we get the nice results.

\begin{definition} Let ${{\mathbf B}}=\{b_0,b_1,\hdots, b_{r-1}\}$ denote the columns of a matrix $P$ obtained from an Allen matrix (integral Fourier matrix, respectively) $s$. Then the $C$-algebra $(A,{{\mathbf B}}, \delta)$ with basis ${{\mathbf B}}$ is called a $C$-algebra arising from an Allen matrix (integral Fourier matrix, respectively) $s$.
\end{definition}

\begin{definition}
Let $(A,{{\mathbf B}}, \delta)$ be a $C$-algebra arising from an Allen matrix $s$ with standard basis ${{\mathbf B}}=\{b_0,b_1,\hdots, b_{r-1}\}$. Let $k\in {{\mathbb Z}}^+$ and $\delta(b_i)=k$ for all $1\leq i\leq r-1$. Then $A$ is called a \emph{homogenous $C$-algebra} with \emph{homogeneity degree} $k$, and the associated Fourier matrix $S$ (Allen matrix $s$, respectively) is called a \emph{homogeneous Fourier matrix} (\emph{homogeneous Allen matrix}, respectively). If a Fourier matrix $S$ (Allen matrix $s$, respectively) is not a homogenous matrix then it is called  \emph{non-homogeneous Fourier matrix $S$} (\emph{non-homogeneous Allen matrix $s$}, respectively).
\end{definition}

The following proposition states that the degrees (multiplicities) of a $C$-algebra  arising from an Allen matrix divide the order of the $C$-algebra. In addition, it states that not only the list of multiplicities matches with the list of degrees but also their indices match.

\begin{prop}\label{SymmetrizingPropertyProp}
Let $(A,\mathbf{B})$ be a $C$-algebra arising from Allen matrix $s$.
\begin{enumerate}
 \item  The multiplicities (degrees) divide the order of $A$, Also, $m_j= d_0/d_j$ for all $j$.

  \item The degrees of $A$ exactly match with the multiplicities of $A$, that is, $m_j=\delta(b_j)$ for all $j$.
\end{enumerate}
\end{prop}

\begin{definition}\label{AllenIntDef} Let $(A,{{\mathbf B}}, \delta)$ be a $C$-algebra with standard basis ${{\mathbf B}}=\{b_0,b_1,\hdots, b_{r-1}\}$ with a character table $P$. We say that $A$  satisfy the \emph{Allen integrality condition} if ${\lambda_{ijk}\sqrt{\delta(b_k)}}/{\sqrt{\delta(b_i)\delta(b_j)}}\in {{\mathbb Z}}$ for all $i,j,k$.
\end{definition}

If a $C$-algebra is arising from an Allen matrix then it is self-dual. In general, the converse is not true. In the next theorem we states that the converse is also true for $C$-algebras satisfying the Allen integrality condition.

\begin{thm}\label{IntegralCAlgebrasAndFourierMatrices}
Let $(A,{{\mathbf B}}, \delta)$ be a $C$-algebra with standard basis ${{\mathbf B}}=\{b_0,b_1, \hdots, b_{r-1}\}$. Then $A$ is self-dual and satisfies the Allen integrality condition if and only if $A$ arises from an Allen matrix.
\end{thm}

\smallskip

\section{General results on non-homogeneous $C$-algebras arising from Allen matrices}

\medskip

In this section, we prove the results that are helpful in recognizing some $C$-algebras that are not arising from Allen matrices by mere looking at the character tables of the $C$-algebras, in particular, the first row of the character tables. Note that an adjacency algebra of a commutative association scheme is a $C$-algebra. All the character tables of the association schemes used here are produced by Hanaki and Miyamoto, see \cite{HM}. In this section, after the proof of each proposition we will give only few examples of the character tables of the association schemes just to demonstrate the application of the proposition.

\smallskip

In the following lemma we prove that it does not require that all the structure constants to be nonnegative for the subset of basis elements with degree $1$ to become a group.

\begin{lemma}\label{ElementaryGroupLemma}
Let $(A,{{\mathbf B}}, \delta)$ be a $C$-algebra arising from an  Allen matrix $s$ with standard basis ${{\mathbf B}}=\{b_0,b_1,\hdots, b_{r-1}\}$. Let the structure constants $\lambda_{ii^*j}\geq 0$ for all $i,j$. If $L=\{b\in {{\mathbf B}}: \delta(b)=1\}$ then $L$ is an abelian group.
\end{lemma}

\begin{proof}
As $\delta(b_0)=1$, $b_0\in L$. Also $b_ib_{i^*}=\lambda_{ii^*0}b_0+\sum_j\lambda_{ii^*j}b_j$ for all $i$. Since $\lambda_{ii^*j}\geq 0$ and  $\delta(b_i)=1$,  $b_ib_{i^*}=b_0$ for all $i$.  Thus, $b^{-1}_i=b_{i^*}$.  Therefore, $L$ is a group. Since $A$ is commutative, $L$ is an  abelian group.
\end{proof}

In the next theorem, we collect the conditions under which a $C$-algebra arising from an Allen matrix cannot have just one degree different from $1$.

\begin{thm}\label{OneDegreeDifferentThm} Let $(A,{{\mathbf B}}, \delta)$ be a $C$-algebra arising from a real Allen matrix $s$ with standard basis ${{\mathbf B}}=\{b_0,b_1,\hdots, b_{r-1}\}$. Let the structure constants $\lambda_{iij}\geq 0$ for all $i,j$.
\begin{enumerate}
  \item If the rank of $A$ is an even integer but greater than $2$, then $A$ cannot have only one degree different from $1$.
  \item If rank of $A$ is greater than $3$, and  $|p_{ij}|\leq p_{0j}$ for all $i,j$. Then $A$ cannot have only one degree greater or equal to $r$ and the remaining degrees equal to $1$.
\end{enumerate}
\end{thm}

\begin{proof}  For $(i)$. By Lemma \ref{ElementaryGroupLemma}, the elements of the basis ${{\mathbf B}}$ with degree $1$ form an elementary abelian group, the order of the group is an even integer. The assertion follows from the order considerations.

\smallskip

For $(ii)$. Let the character table $P$ of $A$ have just one degree different from $1$. Let the degree pattern be  $[1,1, \hdots,1,k]$, where $k\geq r$. Therefore, first row of the character table is $[1,1, \hdots,1,k]$. Since $m_0=m_1=\hdots=m_{r-2}$, $d_0=d_1=\hdots=d_{r-2}$. As $|p_{ij}|\leq p_{0j}$, therefore the only possible entries of row 2, row 3, $\hdots$, row $r-2$ of $P$ are  $[1,1, \hdots,1,-k]$, which is not possible as  $P$  is nonsingular.
\end{proof}

The Part $(i)$ of the above theorem is also true if the order of the algebra is greater than $2$, see Theorem \ref{RankTwoThm}. The character table of the association scheme {\tt as4(2)} is an example where Part $(ii)$ of the above theorem fails for the rank $3$. In the next proposition, we find a condition on the order of a $C$-algebra that forces all degrees to be $1$. 

\begin{lemma}\label{OrderRankComparisionLemma}Let $(A,{{\mathbf B}}, \delta)$ be a $C$-algebra arising from an Allen matrix $s$ with standard  basis ${{\mathbf B}}=\{b_0,b_1,\hdots, b_{r-1}\}$.  Let $p$ be smallest prime among all divisors of $r$. If the order of $A$ is less than or equal to $p(r-1)+1$, then  $\delta(b_i)=1$ for all $i$.
\end{lemma}

\begin{proof}
By Proposition \ref{SymmetrizingPropertyProp}, every degree divides the order of $A$. Therefore $\delta(b_i)\geq p$ for all $i>0$.  Since there are $r-1$ nontrivial degrees,  the assertion is true.
\end{proof}

Moreover, the result is also true if we replace $p$ by the smallest degree of the algebra. The next proposition helps to recognize the $C$-algebras not arising from Allen matrices.

\begin{prop}Let $(A,{{\mathbf B}}, \delta)$ be a $C$-algebra arising from an Allen matrix with standard  basis ${{\mathbf B}}=\{b_0,b_1,\hdots, b_{r-1}\}$. Let $\delta(b_i)=k_i$ for all $i\geq 1$.  Let the number of $i$'s such that  $\delta(b_i)=1$ be $t$ and the  remaining $r-t$ degrees $\delta(b_j)=k$. Then the possible values of $k$ are the divisors of $t$.
\end{prop}

\begin{proof}
By \cite[Proposition 15]{G1}, $\delta(b_l)\in {{\mathbb Z}}$ for all $l$. Therefore, by Proposition \ref{SymmetrizingPropertyProp},  ${d_0}{k^{-1}}\in {{\mathbb Z}}$. Thus $d_0=t+(d_0-t)k$ implies $tk^{-1} \in {{\mathbb Z}}$. Hence $k$ is in the subset of the divisors of $t$.
\end{proof}

For example, from the character table of the association schemes {\tt as9(3), as9(8), as10(6), as16(20), as16(21)} and {\tt as16(62)} etcetera  we conclude these are not the character tables of the association scheme arising from Allen matrices. By the above theorem we also conclude that any $C$-algebra of rank $2$ and order greater than $2$ cannot arise from an Allen matrix, also see Theorem \ref{RankTwoThm}.

In the next proposition, we generalize the Cuntz's result about integral Fourier matrices with nonnegative structure constants, see \cite{MC1}. This proposition is for the real Allen matrices and it does not require all the structure constants to be nonnegative. It helps us to recognize the $C$-algebras not arising from the Allen matrices.

\begin{prop}
Let $(A,{{\mathbf B}}, \delta)$ be a $C$-algebra arising from a real Allen matrix with standard basis ${{\mathbf B}}=\{b_0,b_1, \hdots,b_{r-1}\}$. Let  the structure constants $\lambda_{iij}\geq 0$ for all $i,j$. If all the degrees of $A$ different from $1$ are all equal then the nontrivial degree might be power of $2$. Note: the algebra $A$  need not be homogeneous.
\end{prop}

\begin{proof} Let $\delta(b_i) =k$ for some $1\leq i\leq r-1$. By Lemma \ref{ElementaryGroupLemma}, the elements of ${{\mathbf B}}$ with degree $1$ form an elementary abelian group. Thus the number of linear elements of ${{\mathbf B}}$ is a power of $2$, say $2^m$, $m\in {{\mathbb Z}}^+$. If the number of nonlinear degrees, the degrees not equal to $1$, is equal to $n$ then $d_0=2^m+nk$. By Proposition \ref{SymmetrizingPropertyProp}, $k$ divides $d_0$ implies $k$ divides $2^m$.
\end{proof}

By the above proposition,  from the character table of the association schemes {\tt as12(9), as14(4), as16(10), as16(20), as16(21)} and {\tt as16(62)} etcetera we conclude that these are not the character tables of the $C$-algebras arising from Allen matrices.

In the next proposition we give a criteria to predict the possible number of occurrence of a degree if it is one of the degrees, divide the bigger degrees and  divisible by smaller degrees.

\begin{prop}\label{PredictionOfDegreesAllenCaseProp}Let $(A,{{\mathbf B}}, \delta)$ be a $C$-algebra arising from an Allen matrix, with standard  basis ${{\mathbf B}}=\{b_0,b_1,\hdots, b_{r-1}\}$. Let $\delta(b_i)=k_i$ for all $i\geq 1$.  If for a given $j$, $k_j(\neq 1)$ divides all $k_i(\neq 1)$ then number of degrees equal to $1$ is  multiple of $k_j$, $\alpha k_j$ (say).
In general, let $k_t$ be a degree divisible by all the smaller degrees and divides all the bigger degrees. Let $k_s$ be the largest degree among all the degrees strictly less than $k_t$. If the sum of the smaller degrees less than $k_s$ is $\lambda k_s$ then the number of degrees equal to $k_s$ is  $(nk_t-\lambda k_s){k^{-1}_s}$, where $n\in {{\mathbb Z}}^+$. 
\end{prop}

\begin{proof}
Since $A$ is self dual, $k_i=m_i$ for all $i$. By Proposition \ref{SymmetrizingPropertyProp},  ${d_0}{m^{-1}_i}=d_i$ for each $i$. Now $d_0=1+\sum\limits^{r-1}_{i=1}k_i$, thus ${d_0}{k^{-1}_j} \in {{\mathbb Z}}$ implies $ ({1+\sum\limits^{r-1}_{i=1}k_i}){k^{-1}_j}
=({1+\sum\limits^{}_{k_i=1}k_i+\sum\limits^{}_{k_i\geq k_j}k_i}){k^{-1}_j}
=({1+\sum\limits^{}_{k_i\geq k_j}k_i}){k^{-1}_j}+ \alpha
 \in {{\mathbb Z}}$, where $\alpha\in {{\mathbb Z}}$. Thus the number of degrees equal to $1$ are multiple of $k_j$.

Similarly,  ${d_0}{k^{-1}_t} \in {{\mathbb Z}}$ implies $ ({1+\sum\limits^{r-1}_{i=1}k_i}){k^{-1}_t}
=({1+\sum\limits^{}_{k_i<k_s}k_i+\sum\limits^{}_{k_i=k_s}k_i
+\sum\limits^{}_{k_i\geq k_s}k_i}){k^{-1}_t}
=(\lambda k_s +\sum\limits^{}_{k_i=k_s}k_i){k^{-1}_t}+ \beta
 \in {{\mathbb Z}}$, where $\beta\in {{\mathbb Z}}$. Thus the number of degrees equal to $k_s$ is  $(nk_t-\lambda k_s){k^{-1}_s}$, where $n\in {{\mathbb Z}}^+$.
\end{proof}

For example, the the character tables association schemes {\tt as6(3),  as7(2), as8(3), as8(5), as8(6)} etcetera and all association schemes of rank $2$ and order greater than $2$ violates the conditions of the above proposition so they are not character tables of the association schemes  arising from Allen matrices, for the character tables see \cite{HM}.

\begin{lemma}Let $(A,{{\mathbf B}}, \delta)$ be a $C$-algebra arising from an integral Fourier matrix of odd rank and odd order, with standard  basis ${{\mathbf B}}=\{b_0,b_1,\hdots, b_{r-1}\}$. Let the odd degree among all the degrees of $A$ is maximum. Then rank of $A$ must be at least $10$.
\end{lemma}

\begin{proof}
Let $d_0=\delta(b_i)a_i$. By \cite[Lemma 3.7]{MC1}, $d_0$ is an odd square, thus $a_i$ is a square.
Let $\delta(b_1)$ be an odd integer and $\delta(b_1)\geq \delta(b_i)$ for each $i$. Therefore, $d_0\geq 9\delta(b_1)$, and $ d_0\leq  1+d\delta(b_1)$. Thus $ 9\delta(b_1)\leq 1+d\delta(b_1)$ implies  $\delta(b_1)(9-d)\leq 1$ implies $\delta(b_1) \in {{\mathbb Z}}^+$ only if $d\geq9$.
\end{proof}

\section{Non-homogeneous $C$-algebras of rank $2$ to $5$}

\medskip

In this section we classify the non-homogenous Allen matrices and non-homogenous Fourier matrices of rank $2$ to $5$. In fact, we find the first eigenmatrices, the character tables, as the associated Allen matrices and the Fourier matrices can be retrieved by scaling columns of $P$ matrices and then rows of the Allen matrices, respectively, see Section 2. For the classification of the homogenous Allen matrices and the homogenous Fourier matrices see \cite{G1}. All the theorems, propositions and  methods developed in this section are interesting and they can be used and extended for the classification of the Allen matrices and the Fourier matrices of higher ranks.

\smallskip

In the next theorem, we prove that any row or column permutation of first eigenmatrix $P$ will lead to the simultaneous row and column permutation of $P$. This theorem is useful for the classification of Allen matrices and Fourier matrices as we do not need to consider all the row and column permutation of a given first eigenmatrix $P$ of a $C$-algebra that we assume is arising from an Allen matrix and it is the consequence of the Proposition \ref{SymmetrizingPropertyProp}. Note: if a $C$-algebra is arising from an Allen matrix then it is self dual, thus we can always get a character table $P$ of the $C$-algebra such that $P\bar P=nI$, where $n$ is the order of the algebra and $I$ is the identity matrix.

\begin{thm}\label{SimulataneousPermThm} Let $(A,{{\mathbf B}}, \delta)$ be a $C$-algebra of rank $r$ with standard basis ${{\mathbf B}}$. Let $n$ be the order of the algebra and $P$ be the character table  such that $P\bar P=nI$.
\begin{enumerate}
  \item Let $Q$ be the matrix obtained from $P$ by simultaneous rows and columns permutation. Then $Q\bar Q=nI$.
\item Let $S$ be a symmetric and unitary matrix. Then any simultaneous row and column permutation of $S$ is again a symmetric and unitary matrix.
  \item Let $(A,{{\mathbf B}}, \delta)$ be a $C$-algebra arising from an Allen matrix $s$. Let $Q$ be the character table of $A$ not necessarily be obtained by scaling the columns of Allen matrix $s$. If we consider it to be obtained from the Allen matrix $s$ then $Q\bar Q=nI$. That is, any row or column permutation of $P$ will lead to the simultaneous row and column permutation of $P$.
       
\end{enumerate}
\end{thm}

\begin{proof}
For $(i)$. Let $P=[p_{ij}]$ be the character table of $A$. Since $P\bar P=nI$, for all $0\leq i, j\leq r-1$, $\sum p_{ij}\bar p_{ji}=n$ and $\sum p_{ij}\bar p_{jk}=0$, $k\neq i$. Let $Q=[q_{ij}]$ be the matrix obtained from the matrix $P$ after simultaneous permutation of any number of rows and columns of the matrix $P$.
Therefore, for all $0\leq i, j\leq r-1$, $\sum q_{ij}\bar q_{ji}=\sum p_{ij}\bar p_{ji}=n$ and $\sum q_{ij}\bar q_{jk}=\sum p_{ij}\bar p_{jk}=0$, $k\neq i$. Hence  $Q\bar Q=nI$.

\smallskip

For $(ii)$. Proof is similar to the part $(i)$ above.

\smallskip

For $(iii)$. Here we consider the case of row permutations and the proof for column permutations is similar to the proof for row permutations. Let $P$ be the character table of a $C$-algebra arising from an Allen matrix $s$,  such that $P\bar P=nI$. Let the rows $i_1,i_2, \hdots, i_l$ moves to the positions of the rows $j_1,j_2, \hdots, j_l$. Let $s$ be the Allen matrix obtained from $P$ by scaling the columns with the square roots of the degrees. Thus the row $i_k$ of $s$ moves to the row $j_k$. By Proposition \ref{SymmetrizingPropertyProp}, $d_0m_i=d_i$, where $m_i (=\delta(b_i))$ is the multiplicity of $A$ and this fact will force column $i_k$ to move to column $j_k$. Hence the assertion is true by part $(i)$.
\end{proof}

The next theorem classifies the Allen matrices and Fourier matrices of rank $2$.

\begin{thm}\label{RankTwoThm}
Let $(A,{{\mathbf B}}, \delta)$ be a $C$-algebra of rank two arising from an Allen matrix $s$.. Then $S$ ($s$, respectively) is a unique Fourier matrix (integral Fourier matrix).
\end{thm}

\begin{proof}
Let  $P$ be character table for the $C$-algebra arising from an Allen matrix $s$. Therefore,
$$P=\left[
       \begin{array}{cc}
         1 & k \\
         1 & -1 \\
       \end{array}
     \right],~ s = \left[
       \begin{array}{cc}
         1 & \sqrt{k} \\
         1 & -1/\sqrt{k} \\
       \end{array}
     \right] \mbox{ and } S= \dfrac1{\sqrt{1+k}}\left[
       \begin{array}{cc}
         1 &  \sqrt{k}  \\
          \sqrt{k}  & -1 \\
       \end{array}
     \right].$$
Since $N_{111} =({k-1})/{\sqrt{k}}$ is an integer, $k=1$.
\end{proof}

The above theorem also confirms the decision of the results of the previoous section that conclude that an association scheme of rank $2$ and order greater than $2$ cannot be the association scheme associated an Allen matrix.

Let $(S,T)$ be a modular datum. Then $(S,\zeta_3T)$ is also a modular datum. Hence, in a modular datum, for a given Fourier matrix there might be at least three different torsion  diagonal matrices.
For the Fourier matrix $S=\dfrac{1}{\sqrt2}\left[
       \begin{array}{cc}
         1 & 1 \\
         1 & -1 \\
       \end{array}
     \right]$ and integral Fourier matrix $s=\left[
       \begin{array}{cc}
         1 & 1 \\
         1 & -1 \\
       \end{array}
     \right]$ of the above theorem there exists a torsion matrix $T=$diag$ (\xi ,\mu)$, so that $(S,T)$ and $(s,T)$ are Modular datum and integral Modular datum respectively, see \cite[Example 1]{MC1}. For the complete list of the diagonal matrices associated with the Fourier matrix $S$ see \cite[Example 36]{G1}.

\medskip

Now we classify the Allen matrices (Fourier matrices) of rank $3$.
Let $P$ be the character table for a symmetric $C$-algebra with standard basis ${{\mathbf B}}=\{b_0,b_1,b_2\}$. Let $b_1b_2= ub_1+vb_2$.
Then
$$P=\left[
   \begin{array}{ccc}
     1 & k & l \\
     1 & \phi_1 & \phi_2\\
     1 & \psi_1 & \psi_2 \\
   \end{array}
 \right],
$$ where
$\phi_1= ({v-u-1+\sqrt{(u-v-1)^2+4u}})/{2}$,
$\quad \phi_2= ({u-v-1-\sqrt{(u-v-1)^2+4u}})/{2}$,

$\psi_1=({v-u-1-\sqrt{(u-v-1)^2+4u}})/{2}$ and
$\quad \psi_2= ({u-v-1+\sqrt{(u-v-1)^2+4u}})/{2}$.

\noindent Therefore, $$s=\left[
   \begin{array}{ccc}
     1 & \sqrt{k} & \sqrt{l} \\
     1 & \dfrac{\phi_1}{\sqrt{k}} & \dfrac{\phi_2}{\sqrt{l}}\\
     1 & \dfrac{\psi_1}{\sqrt{k}} & \dfrac{\psi_2}{\sqrt{l}} \\
   \end{array}
 \right].
 $$
Thus
$d_0= 1+k+l=n$, $d_1=1+\dfrac{|\phi_1|^2}{k}+\dfrac{|\phi_2|^2}{l}=\dfrac{n}{k}$,
$d_2=1+\dfrac{|\psi_1|^2}{k}+\dfrac{|\psi_2|^2}{l}=\dfrac{n}{l}$.

\begin{thm}\label{t2}
There is no symmetric homogenous $C$-algebra of rank 3 arising from an Allen matrix.
\end{thm}

\begin{proof} Let $(A,{{\mathbf B}}, \delta)$ be a symmetric homogenous $C$-algebra arising from an Allen matrix $s$. Since $S^T=S$, $ \phi_2k= \psi_1l$ implies $u=v$. The structure constant $N_{210}=0$ implies $k=2u$. Thus  $\phi_1=({-1+\sqrt{1+2k}})/{2}$ and $\phi_2=({-1-\sqrt{1+2k}})/{2}$.
Since $A$ is homogeneous, all the degrees are equal to $1$, see \cite[Thorem 25]{G1}.
 
But the structure constants
$$N_{112}=\dfrac{1}{(2k+1)\sqrt{k}}[k^2+ \dfrac{k}{2}] \mbox{ and } N_{222}
=\dfrac{1}{(2k+1)\sqrt{k}}[k^2-1-\dfrac32k]$$
are not integers for $k=1$.
\end{proof}

\begin{thm}\label{p3}
Let $(A,{{\mathbf B}}, \delta)$ be a symmetric $C$-algebra of rank $3$ arising from an Allen matrix $s$. Then the only possible degree pattern of $A$ is $[1,1,2]$ up to permutations and the corresponding matrices $P$, $s$ and $S$ are as follows.
$$P=\left[
   \begin{array}{ccc}
     1 & 1 & 2 \\
     1 & 1 & -2\\
     1 & -1 & 0 \\
   \end{array}
 \right],~~
s=\left[
   \begin{array}{ccc}
     1 & 1 & \sqrt2 \\
     1 & 1 & -\sqrt2\\
     1 & -1 & 0 \\
   \end{array}
 \right] \mbox{ and } S=\left[
   \begin{array}{ccc}
     1/2 & 1/2 & 1/\sqrt2 \\
     1/2 & 1/2 & -1/\sqrt2\\
     1/\sqrt2 & -1/\sqrt2 & 0 \\
   \end{array}
 \right].
$$
\end{thm}

\begin{proof}
Let $1$, $k$ and $l$ be the degrees of $A$. Therefore, by Proposition \ref{SymmetrizingPropertyProp}, $k$ divides $1+l$, and $l$ divides $1+k$. By Theorem \ref{t2}, $A$ cannot be homogenous, thus the only possible  degree pattern of $A$ are $[1,1,2]$ and $[1,2,3]$, up to permutations.
Since $N_{012}=0$, $1-\dfrac{v}{k}-\dfrac{u}{l}=0$. Therefore, the degree pattern $[1,1,2]$ and $[1,2,3]$ implies $v=1-\dfrac{u}{2}$ and $v=2-\dfrac{2u}{3}$, respectively.

\smallskip

Case $1$. Let the degree pattern be $[1,1,2]$, that is, $k=1$ and $l=2$.

Therefore, $v=1-\dfrac{u}2 $ implies
$$\phi_1=\big({-\dfrac{3u}2+\sqrt{(\dfrac{3u}2-2)^2+4u}}\big)/{2} \mbox{ and } \psi_1=({-\dfrac{3u}2-\sqrt{\big(\dfrac{3u}2-2)^2+4u}}\big)/{2}.$$
Since $N_{011}=1$, $u^3(u-1)=0$. Thus $u=0$, or  $u=1$. Hence $(u,v)=(0,1)$, or $(u,v)=(1,1/2)$.

\smallskip

Subcase 1. If $(u,v)=(0,1)$.

Then $\phi_1=1$, $\psi_1=-1$, $\phi_2=-2$ and $\psi_2=0$. Therefore,
$$P=\left[
   \begin{array}{ccc}
     1 & 1 & 2 \\
     1 & 1 & -2\\
     1 & -1 & 0 \\
   \end{array}
 \right],
s=\left[
   \begin{array}{ccc}
     1 & 1 & \sqrt2 \\
     1 & 1 & -\sqrt2\\
     1 & -1 & 0 \\
   \end{array}
 \right] \mbox{ and } S=\left[
   \begin{array}{ccc}
     1/2 & 1/2 & 1/\sqrt2 \\
     1/2 & 1/2 & -1/\sqrt2\\
     1/\sqrt2 & -1/\sqrt2 & 0 \\
   \end{array}
 \right].
$$

Subcase 2. If $(u,v)=(1,\dfrac12)$.

Then $\phi_1=\dfrac{-3+\sqrt{17}}{4}$, $\psi_1=\dfrac{-3-\sqrt{17}}{4}$, $\phi_2=\dfrac{-1-\sqrt{17}}{4}$ and $\psi_2=\dfrac{-1+\sqrt{17}}{4}$.
Therefore,
$$P=\left[
   \begin{array}{ccc}
     1 & 1 & 2 \\
     1 & \dfrac{-3+\sqrt{17}}{4} & \dfrac{-1-\sqrt{17}}{4}\\
     1 & \dfrac{-3-\sqrt{17}}{4} & \dfrac{-1+\sqrt{17}}{4} \\
   \end{array}
 \right]\mbox{ and } s=\left[
   \begin{array}{ccc}
    1 & 1 & \sqrt2 \\
     1 & \dfrac{-3+\sqrt{17}}{4} & \dfrac{-1-\sqrt{17}}{4\sqrt2}\\
     1 & \dfrac{-3-\sqrt{17}}{4} & \dfrac{-1+\sqrt{17}}{4\sqrt2} \\
   \end{array}
 \right].$$
Notice that $m_0=m_1$, but $d_0=4$ and $d_1=1+\dfrac{1}{16}[39-3\sqrt{17}]$, which is a contradiction to the Proposition \ref{SymmetrizingPropertyProp}. Hence $u=1$ and $v=\dfrac12$ is not a possible case.

\medskip

Case $2$. Let the degree pattern be $[1,2,3]$, that is, $k=2$ and $l=3$.

\noindent Therefore, $v=2-\dfrac{2u}3 $, implies
$$\phi_1=\big({-\dfrac{5u}3+1+\sqrt{(\dfrac{5u}3-3)^2+4u}}\big)/{2} \mbox{ and } \psi_1=\big({-\dfrac{5u}3+1-\sqrt{(\dfrac{5u}3-3)^2+4u}}\big)/{2}.$$
Since $N_{011}=1$,
$625u^4-1850u^3+2520u^2-1296u+243=0.$ But it has no real roots, see \cite{EL}.
Thus we rule out $[1,2,3]$ degree pattern, because an Allen matrix associated with a symmetric $C$-algebra might be a real matrix, see \cite[Proposition 21]{G1}.
\end{proof}

\begin{remark}
In fact, the character table $P$ above is the character table of an association scheme {\tt as4(2)}. By Theorem \ref{p3}, there is no  symmetric $C$-algebra of rank $3$ arising from an integral Allen matrix (integral Fourier matrix).
\end{remark}

In the next theorem we prove that there only one asymmetric $C$-algebra of rank 3 arising from an Allen matrix. Moreover, the following theorem shows that for rank $3$ it is not necessary to assume $|s_{ij}|\leq s_{0j}$ for all $i,j$ to prove that the homogeneous $C$-algebra arising from an Allen matrix is a group algebra, see \cite[Theorem 25]{G1}.

\begin{thm}\label{t9}
Let $(A,{{\mathbf B}}, \delta)$ be a asymmetric $C$-algebra arising from an Allen matrix $s$ of rank $3$. Then $s$ is the character table of the cyclic group of order $3$.
\end{thm}

\begin{proof}
The character table $P$ for asymmetric table algebra is as follows, see \cite[\S 4]{HRB}.
$$P=\left[
   \begin{array}{ccc}
     1 & k & k \\
      1 & \alpha & \bar\alpha \\
     1 & \bar\alpha & \alpha\\
    \end{array}
 \right],$$
where $\alpha={(-1+i\sqrt{1+2k})}/{2}.$
By \cite[Theorem 3.3]{HRB}, the structure constants are given by $\lambda_{112} =(k+1)/2$,
$\lambda_{121}= (k-1)/2$. The corresponding $s$ matrix is:
$$s=\left[
   \begin{array}{ccc}
     1 & \sqrt{k} & \sqrt{k} \\
      1 & {\alpha}/{\sqrt{k}} &  {\bar\alpha}/{\sqrt{k}} \\
     1 &  {\bar\alpha}/{\sqrt{k}} &  {\alpha}/{\sqrt{k}}\\
    \end{array}
 \right].
$$
Now $\lambda_{ijk}=  {s^{-1}_{0k}}{s_{0i}s_{0j}N_{ijk}}$ implies $N_{ijk}= {\lambda_{ijk}s_{0k}}{s^{-1}_{0i}s^{-1}_{0j}}$.
Therefore, $N_{112}= ({k+1})/{2\sqrt{k}}$ and
$N_{121}= ({k-1})/{2\sqrt{k}}$. Since $N_{112}$ and $N_{121}$ are positive integers,  $k=1$. Then $\alpha=(-1+i\sqrt{3})/2=\zeta_3$, and
$$s(=P)=\left[
   \begin{array}{ccc}
     1 & 1 & 1 \\
      1 & \zeta_3 & \zeta^2_3 \\
     1 & \zeta^2_3 & \zeta_3\\
    \end{array}
 \right]\mbox{ and }S=\dfrac{1}{\sqrt3}\left[
   \begin{array}{ccc}
     1 & 1 & 1 \\
      1 & \zeta_3 & \zeta^2_3 \\
     1 & \zeta^2_3 & \zeta_3\\
    \end{array}
 \right]
.$$
\end{proof}

Now we classify the Allen matrix with one nontrivial degree equal to $1$. By Theorem \ref{SimulataneousPermThm}, we assume without loss of generality $\delta(b_1)=1$.

\smallskip

In the next lemma we show that if the degree of a basis element of a $C$-algebra arising from an Allen matrix is equal to $1$ then elements of the corresponding row of the character table of the algebra is completely determined.

\begin{lemma}\label{EqualityOfKandLLemmaRank5}Let $(A,{{\mathbf B}}, \delta)$ be a $C$-algebra arising from an Allen matrix $s$ of rank $r$ with standard  basis ${{\mathbf B}}=\{b_0,b_1,\hdots, b_{r-1}\}$. Let $P$ be the character table of $A$ and  $|s_{ij}|\leq s_{0j}$ for all $j$. Let $\delta(b_j)=k_j(=p_{0j})$ be the degree of $b_j$ for all $j$ such that $k_0=k_i=1$ for some $i$. Then $p_{ij}=\pm k_j$ for all $j$.
\end{lemma}

\begin{proof}
Since $\delta(b_i)=1$,  $d_i= d_0$. The first and $i$th rows of $s$ are $[1,1,\sqrt{k_2},\sqrt{k_3},\hdots, \sqrt{k_{r-1}}]$ and $[1,p_{i1},p_{i2}/\sqrt{k_2},p_{i3}/\sqrt{k_3},\hdots, p_{i,r-1}/\sqrt{k_{r-1}}]$, respectively. Since $d_0=d_i$,
$p_{i1}=\pm 1, p_{i2}/\sqrt{k_2}=\pm \sqrt{k_2}, p_{13}/\sqrt{k_3}=\pm \sqrt{k_3}, \hdots, p_{i,r-1}/\sqrt{k_{r-1}}=\pm \sqrt{k_{r-1}}$. Hence $p_{ij}=\pm k_j$ for all $j$.
\end{proof}

\begin{lemma}\label{EqualityOfKandLLemmaRank4}Let $(A,{{\mathbf B}}, \delta)$ be a $C$-algebra, of rank $4$, arising from a real Allen matrix with standard  basis ${{\mathbf B}}=\{b_0,b_1,\hdots, b_3\}$. Let $|s_{ij}|\leq s_{0j}$ for all $j$. If $\delta(b_1)=1$, $\delta(b_2)=k_2$ and $\delta(b_3) =k_3$, then either $k_3=k_2$ or $k_3=k_2+2$.
\end{lemma}

\begin{proof}
$\delta(b_1)=1$ implies $m_1 =1$ implies $d_1= d_0$. The entries of the first and second rows of $s$ are $[1,1,\sqrt{k_2},\sqrt{k_3}]$ and $[1, s_{11}, \dfrac{p_{12}}{\sqrt{k_2}},\dfrac{p_{13}}{\sqrt{k_3}}]$, respectively. Since $d_0=d_1$,
$\dfrac{p_{12}}{\sqrt{k_2}}=\pm \sqrt{k_2}$,  and $\dfrac{p_{13}}{\sqrt{k_3}}=\pm\sqrt{k_3}$, that is, $p_{12} = \pm k_2, p_{13}=\pm k_3$. Therefore, if $p_{11} = -1$, $k_3=k_2$, and if $p_{11} = 1$, then $k_3=k_2+2$.
\end{proof}

For a homogeneous $C$-algebra arising from Allen matrix $s$ with $|s_{ij}|\leq s_{0j} $ for all $j$, the authors have already proved the Allen matrix is a character table of an abelian group, see. The following proposition classify the non-homogeneous Allen matrices of rank $4$ with two degrees equal to $1$.

\begin{prop}\label{CharacterTableExpreesionWithDifferentNormPropRank4} Let $(A,{{\mathbf B}}, \delta)$ be a non-homogeneous $C$-algebra arising from an Allen matrix $s$ of rank $4$ with standard  basis ${{\mathbf B}}=\{b_0,b_1,\hdots, b_3\}$. If $|s_{ij}|\leq s_{0j} $ for all $j$.  If $\delta(b_i)=1$ for one $i>0$ and $\delta(b_j)=k_j$ for all $j \neq i$.  Then the character table $P$ has the following expressions up to simultaneous permutation of rows and columns. (Though we remark that each the matrix in the proof is an Allen matrix even with permutations.)
$$\left[
   \begin{array}{cccc}
     1 & 1 & 2 &2\\
     1 & -1 &    2 &  -2\\
       1 &  1 &  -1&  -1\\
         1 & -1 &  -1&  1\\
   \end{array}\right], ~~\left[
   \begin{array}{cccc}
     1 & 1 & 2 &4\\
     1 &  1 &    2 &  -4\\
       1 &  1 &  -2&  0\\
         1 & -1 &  0& 0\\
   \end{array}\right]\mbox{ and } \left[
   \begin{array}{cccc}
     1 & 1 & 4 &6\\
     1 &  1 &   4 &  -6\\
       1 &  1 &  -2&  0\\
         1 & -1 &  0& 0\\
   \end{array}\right].$$
\end{prop}

\begin{proof} Without loss of generality, let $k_1=1$, see Theorem \ref{SimulataneousPermThm}. Thus the character table
$$P=\left[
   \begin{array}{cccc}
     1 & 1 & k_2 & k_3\\
     1 & p_{11} &   p_{12} & p_{13}\\
       1 & p_{21} &  p_{22}&  p_{23}\\
         1 & p_{31} &  p_{32}&  p_{33}\\
   \end{array}
 \right].$$

Case 1. If $p_{11},  p_{12}$ and  $p_{13}$ are not rational integers.

\smallskip

Since $d_0=d_1$, $|p_{11}|=1, |p_{12}|=k_2$ and $|p_{13}|=k_3$. Thus $p_{11},  p_{12}$ and  $p_{13}$ cannot be irrational, hence they can be non-real algebraic integers. As complex conjugate of an irreducible character is an irreducible character, without loss of generality we assume the third irreducible character is a complex conjugate of the second character, thus we have $k_2=1$, see Theorem \ref{SimulataneousPermThm}. Therefore, the entries of the fourth row of $P$ might be rational integers and as $A$ is non-homogeneous, by Proposition \ref{SymmetrizingPropertyProp}, $d_0=d_im_i$  implies $k_3=3$ and $d_3=2$. Hence there might be two zeros in the fourth row. But each entry of the rows $1, 2$ and $3$ is non-zero. As $S$ is symmetric, we get a contradiction.

\medskip

Case 2. If $p_{11},  p_{12}$ and  $p_{13}$ are rational integers.

\smallskip

By Lemma \ref{EqualityOfKandLLemmaRank4}, the second row of $P$ is either $[1,-1,k_2,-k_2]$ or $[1,1,k_2,-(k_2+2)]$.

\medskip

Subcase 1. Let the second row of $P$ be $[1,-1,k_2,-k_2]$.

\smallskip

Then, the first line of the character table is $[1,1,k_2,k_2]$.
Using symmetry of the matrix $S$ and orthogonality of characters, we have
$$P=\left[
   \begin{array}{cccc}
     1 & 1 & k_2 &k_2\\
     1 & -1 &    k_2 &  -k_2\\
      1 &  1 &  -1&  -1\\
         1 & -1 &  -1&  1\\
   \end{array}
 \right].$$
Since $A$ is non-homogenous, $k_2\neq1$. As $k_2$ divides $d_0$, therefore $k_2=2$. Hence
$$P=\left[
   \begin{array}{cccc}
     1 & 1 & 2 &2\\
     1 & -1 &    2 &  -2\\
       1 &  1 &  -1&  -1\\
         1 & -1 &  -1&  1\\
   \end{array}\right]$$

\medskip

Subcase 2. Let the second row of $P$ be $[1, 1,k_2,-(k_2+2)]$.

\smallskip

Then, the first line of the character table is $[1,1,k_2,k_2+2]$.
The row sum of the character table is zero for each row, using symmetry of the matrix $S$ and orthogonality of characters, we have
$$P=\left[
   \begin{array}{cccc}
     1 & 1 & k_2 &k_2+2\\
     1 & 1 &    k_2 &  -(k_2+2)\\
     1 & 1 &  -2&  0\\
     1 &-1 &  0&  0\\
   \end{array}
 \right].$$
Since $A$ is non-homogenous, $k_2\neq1$. As $k_2$ divides $d_0$, therefore $k_2=2$ or $4$. Hence
$$ P=\left[
   \begin{array}{cccc}
     1 & 1 & 2 &4\\
     1 &  1 &    2 &  -4\\
       1 &  1 &  -2&  0\\
         1 & -1 &  0& 0\\
   \end{array}\right]\mbox{ or } P=\left[
   \begin{array}{cccc}
     1 & 1 & 4 &6\\
     1 &  1 &   4 &  -6\\
       1 &  1 &  -2&  0\\
         1 & -1 &  0& 0\\
   \end{array}\right].$$

\medskip

Subcase 3. Let the second row of $P$ be $[1, 1,-k_2,k_2-2)]$.

\smallskip

Then, the first line of the character table is $[1,1,k_2,k_2-2]$.
The row sum of the character table is zero for each row, using symmetry of the matrix $S$ and orthogonality of characters, we have
$$P=\left[
   \begin{array}{cccc}
      1 & 1 & k_2 &k_2-2\\
     1 & 1 &    -k_2 &  k_2-2\\
      1 & -1 &  p_{22}&  0\\
         1 & 1 & 0&  -2\\
   \end{array}
 \right].$$
Since $(k_2-2)\big|(k_2+2)$, $k_2\in\{3,4,6\}$.  But $k_2=3$ is not possible. For $k_2\in\{4,6\}$, $d=2$ implies $p_{22}=0$. Thus the following are the only possible values of $P$.
$$\left[
   \begin{array}{cccc}
      1 & 1 & 4 &2\\
     1 & 1 &    -4 &  2\\
      1 & -1 &  0&  0\\
         1 & 1 & 0&  -2\\
   \end{array}
 \right],~~\left[
   \begin{array}{cccc}
      1 & 1 & 6 &4\\
     1 & 1 &    -6 &  4\\
      1 & -1 &  0&  0\\
         1 & 1 & 0&  -2\\
   \end{array}
 \right].$$
Note that corresponding to each of the above $P$ matrix the Allen matrix $s$ has nonnegative integral structure constants $N_{ijk}$. But this can be   verified through the Allen integrality condition, see Definition \ref{AllenIntDef} and Theorem \ref{IntegralCAlgebrasAndFourierMatrices}.
\end{proof}

\begin{remark} We note that all the Allen matrices in the above proposition are not integral Fourier matrices as the multiplicities are not square integers, see \cite[Lemma 12]{G1}. In the above case the first character table is exactly same as the character table for the association schemes {\tt as6(5)}  up to simultaneous permutation of rows $3,4$. The remaining two are the character tables of the association schemes {\tt as8(4)} and {\tt as12(8)}.
\end{remark}

Since the row sum for the character table of a $C$-algebra is zero, if $\delta(b_i)=1$ we can determine the $i$th row of the character table $P$. In the next proposition we assume $r-1$ entries of the second row and adjust the $r$th entry of the second row using the fact that row sum is zero and then determine whether there is an Allen matrix of rank $5$ under the two conditions.

\begin{prop}\label{CharacterTableExpreesionWithDifferentNormPropRank5}
Let $(A,{{\mathbf B}}, \delta)$ be a non-homogeneous $C$-algebra, of rank $5$, arising from an Allen matrix with standard  basis ${{\mathbf B}}=\{b_0,b_1,\hdots, b_4\}$. If $|s_{ij}|\leq s_{0j} $ for all $j$.  If $\delta(b_i)=1$ for one $i>0$ and $\delta(b_j)=k_j$ for all $j \neq i$. Then the character table has the following expression up to simultaneous permutation of row and columns.
$$\left[
   \begin{array}{ccccc}
     1 & 1 & 2 &2&2\\
     1 & 1 &  2 &  -2&-2\\
      1 & 1 & -2& 0&0\\
     1 &  -1 & 0&   \sqrt{2} &-\sqrt{2}\\
     1 & -1 & 0&  -\sqrt{2} & \sqrt{2}\\
   \end{array}\right], ~~\left[
   \begin{array}{ccccc}
     1 & 1 & 2 &4&8\\
     1 & 1 &   2 & 4&-8\\
      1 &  1 & 2&  -4& 0\\
     1 & 1 &  -2&  0& 0\\
     1 &  -1 &  0&  0& 0\\
\end{array}\right],
   \left[
   \begin{array}{ccccc}
     1 & 1 & 4 &3&3\\
     1 & 1 &  4 &  -3&-3\\
      1 & 1 & -2& 0&0\\
     1 &  -1 & 0&   \sqrt{3} &-\sqrt{3}\\
     1 & -1 & 0&  -\sqrt{3} & \sqrt{3}\\
   \end{array}\right].
$$
\end{prop}

\begin{proof}

Without loss of generality, let $k_1=1$, see Theorem \ref{SimulataneousPermThm}. Thus the character table
$$P=\left[
   \begin{array}{ccccc}
     1 & 1 & k_2 & k_3&k_4\\
     1 & p_{11} &   p_{12} & p_{13}&p_{14}\\
     1 &  p_{21} &  p_{22}&  p_{23}& -(1+p_{21} +  p_{22}+  p_{23})\\
     1 &  p_{31} &  p_{32}&  p_{33}& -(1+p_{31} +  p_{32}+  p_{33})\\
     1 &  p_{41} &  p_{42}&  p_{43}& -(1+p_{41} +  p_{42}+  p_{43})\\
   \end{array}
 \right].$$

Case 1. If $p_{11},  p_{12}, p_{13}$ and  $p_{14}$ are not all rational integers.

\smallskip

Since $d_0=d_1$, $|p_{11}|=1, |p_{12}|=k_2, |p_{13}|=k_3$ and $|p_{14}|=k_4$. Since the row sum is zero, at least two of these $p_{11},  p_{12}, p_{13}$ and  $p_{14}$ can be non-real algebraic integers. As complex conjugate of an irreducible character is an irreducible character, without loss of generality, we assume that the third irreducible character is a complex conjugate of the second character, thus we have $k_2=1$, see Theorem \ref{SimulataneousPermThm}. Without loss of generality, let $k_3\leq k_4$.  By Proposition \ref{SymmetrizingPropertyProp}, $d_0=d_im_i$  implies $(k_3,k_4) \in \{(1, 2),
(1, 4),
(2, 5),
(3, 6),
(6, 9)
\}$.

If $k_3=1$ then $d_3=d_0=d_1=d_2$ implies all the entries of the row $1,2,3$ and $4$ are non zero. Since $k_3\neq k_4$, the entries of the fifth row  are rational integers as the complex conjugate of an irreducible character is an irreducible character. Thus there might be at least two zero entries in the fifth row of $P$. But the matrix $S$ is symmetric. Hence we get a contradiction.

For $(k_3,k_4) \in \{(2, 5),
(3, 6),
(6, 9)
\}$ as $k_3\neq k_4$, the entries of the row $5$ are rational integers because $k_2=k_1=k_0=1$ and the complex conjugate of an irreducible character is an irreducible character. Each entry of the row $1, 2$ and $3$ of $P$ is non-zero. But for each the above pair there are exactly $3$ zeros in the fifth row. Since $S$ is a symmetric matrix, we get a contradiction.

\medskip

Case 2. If $p_{11},  p_{12}, p_{13}$ and  $p_{14}$ are all rational integers.

\smallskip

By Lemma \ref{EqualityOfKandLLemmaRank5} and Theorem \ref{SimulataneousPermThm} the only possible degree patterns are:

$[1, -1, k_2, k_3, -(k_2+k_3)]$, $[1, 1, k_2, k_3, -(k_2+k_3+2)]$, $[1, ~ 1,~     k_2,   ~-k_3,~ -(k_2-k_3+2)]$,

$[1, ~ -1,~     k_2,   ~-k_3,~ -(k_2-k_3)]$, $[1, ~ 1,~     -k_2,   ~-k_3,~ k_2+k_3-2]$, $[1, ~ -1,~     -k_2,   ~-k_3,~ k_2+k_3]$.

\medskip

Subcase 1. Let the second row of $P$ be $[1, -1, k_2, k_3, -(k_2+k_3)]$.

\smallskip

Then the first row of the character table is $[1, 1, k_2, k_3, k_2+k_3]$ and $(k_2+k_3)\big|(2+k_2+k_3)$. Thus $(k_2,k_3)=(1,1)$.
Therefore, by the orthogonality of characters, we have
$$P=\left[
   \begin{array}{ccccc}
     1 & 1 & 1 &1&2\\
     1 & -1 &    1 &  1&-2\\
     1 &  1 &  p_{22}&  p_{23}& -1\\
     1 & 1 &  p_{32}&  p_{33}& -1\\
     1 &  -1 & p_{42}&  p_{43}& 1\\
   \end{array}\right].
$$
As $m_2=m_3=1$, therefore $d_2=d_3=6$. But each of $|p_{22}|$, $|p_{23}|$, $|p_{32}|$ and $|p_{33}|$ can be at most $1$, thus both $d_2$ and $d_3$ are strictly less than  $6$, a contradiction. Hence this case is not possible.

\medskip

Subcase 2. Let the second row of $P$ be $[1, 1, k_2, k_3, -(k_2+k_3+2)]$.

\smallskip

Then the first row of the character table is $[1, 1, k_2, k_3, k_2+k_3+2]$.
Therefore, by the orthogonality of the irreducible characters and symmetry of the matrix $S$, we have
$$P=\left[
   \begin{array}{ccccc}
     1 & 1 & k_2 &k_3&k_2+k_3+2\\
     1 & 1 &   k_2 &  k_3&-(k_2+k_3+2)\\
      1 &  1 &  p_{22}&  -2-p_{22}& 0\\
     1 & 1 &  p_{32}&  -2-p_{32}& 0\\
     1 &  -1 &  0&  0& 0\\
   \end{array}\right].
$$
Without loss of generality, let $k_2\leq k_3$. Since $k_2\big|(2k_3+4)$ and $k_3\big|2k_2+4$, we have

$(k_2,k_3)\in \{(1, 2),
(1, 3),
(1, 6),
(2, 4),
(2, 8),
(3, 10),
(4, 6),
(4, 12),
(6, 16),
(8, 10),
(12, 28)\}
$.

Note that $k_2\neq k_3$, thus $d_3\neq d_4$. But the  conjugate irreducible characters have same norm. Thus $p_{22},  p_{23}, p_{32}$ and $p_{33}$
are rational integers. Therefore, the det$(P) \in{{\mathbb Z}}$ and $(\mbox{det}P)^2=n$. Thus $n=2(k_2+k_3+2)$ need to be a square. But the only two pairs $(2, 4),
(4, 12)$ do not fail this test.

For $(k_2,k_3)=(2, 4)$, $d_2=8=1+1+\big(\dfrac{p_{22}}{\sqrt2}\big)^2+\big(\dfrac{-2-p_{22}}{\sqrt4}\big)^2$. Since the entries of $P$ are algebraic integers, we have $p_{22}=2$.
Similarly, $d_3=4$ implies $p_{32}=-2$.
Thus the character table
$$P=\left[
   \begin{array}{ccccc}
     1 & 1 & 2 &4&8\\
     1 & 1 &   2 & 4&-8\\
      1 &  1 & 2&  -4& 0\\
     1 & 1 &  -2&  0& 0\\
     1 &  -1 &  0&  0& 0\\
\end{array}\right].$$

For $(k_2,k_3)=(4, 12)$, $d_2=9$ and  $d_3=3$. Therefore, we have  $p_{22}=4,-5$ and  $p_{32}=1, -2$.

\smallskip

Thus the possible character tables are:
$$P_1=\left[
   \begin{array}{ccccc}
     1 & 1 & 4 &12&18\\
     1 & 1 &   4 & 12&-18\\
      1 &  1 & 4&  -6& 0\\
     1 & 1 &   1&  -3& 0\\
     1 &  -1 &  0&  0& 0\\
   \end{array}\right],\mbox{  } P_2=\left[
   \begin{array}{ccccc}
     1 & 1 & 4 &12&18\\
     1 & 1 &   4 & 12&-18\\
      1 &  1 & -5&  3& 0\\
     1 & 1 &   1&  -3& 0\\
     1 &  -1 &  0&  0& 0\\
   \end{array}\right],$$
$$P_3=\left[
   \begin{array}{ccccc}
     1 & 1 & 4 &12&18\\
     1 & 1 &   4 & 12&-18\\
      1 &  1 & 4&  -6& 0\\
     1 & 1 &  -2&  0& 0\\
     1 &  -1 &  0&  0& 0\\
   \end{array}\right],\mbox{   } P_4=\left[
   \begin{array}{ccccc}
     1 & 1 & 4 &12&18\\
     1 & 1 &   4 & 12&-18\\
      1 &  1 & -5&  3& 0\\
      1 & 1 &  -2&  0& 0\\
     1 &  -1 &  0&  0& 0\\
   \end{array}\right].$$
Since an Allen matrix have algebraic integer entries, the above matrices are not the character tables of the $C$-algebras arising from Allen matrices, \cite[Proposition 15]{G1}.

\medskip

Subcase 3. Let the second row of $P$ be $[1, ~ 1,~     k_2,   ~-k_3,~ -(k_2-k_3+2)]$.

\smallskip

Then, the first row of the character table is $[1, 1, k_2, k_3, k_2-k_3+2]$.
Therefore, by the orthogonality of characters, symmetry of the matrix $S$ and $P\bar P=nI$, we have
$$P=\left[
   \begin{array}{ccccc}
     1 & 1 & k_2 &k_3&k_2-k_3+2\\
     1 & 1 &   k_2 &  -k_3&-(k_2-k_3+2)\\
      1 & 1 & -2& 0&0\\
     1 &  -1 & 0&  p_{33}& -p_{33}\\
     1 & -1 & 0&  p_{43}& -p_{43}\\
   \end{array}\right].
$$
Therefore $k_2\big|4$, $k_3\big|2k_2+4$,  $(k_2-k_3+2)\big|(k_2+k_3+2)$ and $k_2-k_3+2>0$.
Hence $(k_2,k_3)\in \{(1, 1),
(1, 2),
(2, 2),
(4, 2),
(4, 3),
(4, 4)\}$.
Since $|s_{ij}|\leq s_{0j}$, $(k_2,k_3) \not \in \{(1,1),(1,2)\}$.
For $(k_2,k_3)=(2,2)$, $d_3=d_4=4$. Thus $p_{33}\bar p_{33}=2$, $p_{43}\bar p_{43}=2$. But the integrality of the structure constants and orthogonality of characters forces $p_{33}=  \pm\sqrt{2}$  and $p_{43}=  \mp\sqrt{2}$. Therefore, up to simultaneous permutation of row $4$ and row $5$, and column $4$ and column $5$, we have
$$P=\left[
   \begin{array}{ccccc}
     1 & 1 & 2 &2&2\\
     1 & 1 &  2 &  -2&-2\\
      1 & 1 & -2& 0&0\\
     1 &  -1 & 0&   \sqrt{2} &-\sqrt{2}\\
     1 & -1 & 0&  -\sqrt{2} & \sqrt{2}\\
   \end{array}\right].
$$
\medskip

For $(k_2,k_3)=(4,2)$, $d_3=6$
implies $|p_{33}|=\dfrac{4}{\sqrt3}>2$, a contradiction.

For $(k_2,k_3)=(4,3)$, $k_4=3$, $d_3=4$ and $d_4=4$
implies $|p_{33}|= \sqrt3$ and $|p_{43}|= \sqrt3$. But the integrality of structure constants and orthogonality of characters forces $p_{33}=  \pm\sqrt{3}$  and $p_{43}=  \mp\sqrt{3}$. Therefore, up to simultaneous permutation of row $4$ and row $5$, and column $4$ and column $5$, we have
$$P=\left[
   \begin{array}{ccccc}
     1 & 1 & 4 &3&3\\
     1 & 1 &  4 &  -3&-3\\
      1 & 1 & -2& 0&0\\
    1 &  -1 & 0&   \sqrt{3} &-\sqrt{3}\\
     1 & -1 & 0&  -\sqrt{3} & \sqrt{3}\\
   \end{array}\right].
$$
Although the structure constants are not all integers, for example $\lambda_{342}=3/2$, but it satisfies the Allen integrality condition, see Definition \ref{AllenIntDef}. Hence the $P$ is a character table of a $C$-algebra arising from an Allen matrix.

\medskip

Subcase 4. Let the second row of $P$ be $[1, ~ -1,~     k_2,   ~-k_3,~ -(k_2-k_3)]$.

\smallskip

Then, the first row of the character table is $[1,~ 1, ~k_2, ~k_3, ~ k_2-k_3]$.
Therefore, by the orthogonality of  the characters and symmetry of the matrix $S$, we have
$$P=\left[
   \begin{array}{ccccc}
     1 & 1 & k_2 &k_3&k_2-k_3\\
     1 & -1 &    k_2 &  -k_3&-(k_2-k_3)\\
     1 &  1 &  0&  p_{23}& -(2+ p_{23})\\
     1 & -1 &  p_{32}&  p_{33}& -(p_{32}+  p_{33})\\
     1 & -1 &  p_{42}&  p_{43}& -(p_{42}+  p_{43})\\
   \end{array}\right].
$$
Since $P\bar P=nI$, from row $1$, $2$ and column $3$, we get $k_2=0$, a contradiction. Hence this case is not possible.

\medskip

Subcase 5. Let the second row of $P$ be $[1, ~ 1,~     -k_2,   ~-k_3,~ k_2+k_3-2]$.

\smallskip

Then the first row of the character table is $[1, ~1, ~k_2, ~k_3, k_2+k_3-2]$.
Therefore, by the symmetry of the matrix $S$ and $P\bar P=nI$, we have
$$P=\left[
   \begin{array}{ccccc}
     1 & 1 & k_2 &k_3&k_2+k_3-2\\
     1 & 1 &   -k_2 &  -k_3& k_2+k_3-2\\
      1 &  -1  &  p_{22}&  p_{23}& -(p_{22}+  p_{23})\\
     1 &  -1 &  p_{32}&  p_{33}& -(p_{32}+  p_{33})\\
     1 &  1 &  p_{42}&  p_{43}& -(2 +  p_{42}+  p_{43})\\
   \end{array}\right].
$$
Now from orthogonality of characters and symmetry of the matrix $S$, we have
$$P=\left[
   \begin{array}{ccccc}
     1 & 1 & k_2 &k_3&k_2+k_3-2\\
     1 & 1 &   -k_2 &  -k_3& k_2+k_3-2\\
      1 & -1 &  p_{22}&  -p_{22}&0\\
     1 & -1 &  p_{32}&  -p_{32}& 0\\
     1 &  1 &  0&  0& -2\\
   \end{array}\right].
$$
Therefore $k_2\big|2k_3$, $k_3\big|2k_2$,  $(k_2+k_3-2)\big|(k_2+k_3+2)$ and $k_2+k_3-2>0$.
Without loss of generality, let $k_2\leq k_3$. Hence $(k_2,k_3)\in \{(1, 2),
(2, 2)\}$.
Since $|s_{ij}|\leq s_{0j}$, $(k_2,k_3)\neq (1,2)$.

\medskip

For $(k_2,k_3)=(2,2)$, $d_2=d_3=4$. Thus $p_{22}\bar p_{22}=2$, $p_{32}\bar p_{32}=2$. But the integrality of structure constants and orthogonality of characters forces $p_{22}=  \pm\sqrt{2}$  and $p_{32}=  \mp\sqrt{2}$.
Therefore, up to simultaneous permutation of rows and columns, 
we have
$$P=\left[
   \begin{array}{ccccc}
     1 & 1 & 2 &2&2\\
     1 & 1 &  -2 &  -2&2\\
       1 & -1 &  \sqrt{2} &-\sqrt{2}&0\\
     1 & -1 & -\sqrt{2} & \sqrt{2}& 0\\
     1 &  1 &  0&  0& -2\\
   \end{array}\right].
$$

\medskip

Subcase 6. Let the second row of $P$ be $[1, ~ -1,~     -k_2,   ~-k_3,~ k_2+k_3]$.

\smallskip

Then the first row of the character table is $[1,~ 1,~ k_2,~ k_3, ~k_2+k_3]$.
Therefore, by the symmetry of the matrix $S$ and orthogonality of the characters, we have
$$P=\left[
   \begin{array}{ccccc}
     1 & 1 & k_2 &k_3&k_2+k_3\\
     1 & -1 &    -k_2 &  -k_3&k_2+k_3\\
      1 &  -1 &  p_{22}&  p_{23}& -1\\
     1 &  -1 &  p_{32}&  p_{33}& -1\\
     1 &  1 &  p_{42}&  p_{43}& 0\\
   \end{array}\right].
$$
As in Subcase 1, $k_2=k_3=1$ implies $d_2=d_3=6$, and we get a contradiction.
\end{proof}

\begin{remark}\rm
We note that all the above Allen matrices in the above proposition are not integral Fourier matrices as the multiplicities are not square integers. In the above proposition, the first two matrices are the character table for {\tt as08(10), as16(24)}, respectively. The third matrix of order $12$ is not the character table of  an association scheme as it has negative structure constants.
\end{remark}

\section{$C$-algebras arising from integral Fourier matrices of rank $4$ and $5$}
\begin{thm}\label{FourDifferentDegreesThm4}Let $(A,{{\mathbf B}}, \delta)$ be a non-homogenous $C$-algebra arising from an Allen matrix $s$  of rank $4$ with standard  basis ${{\mathbf B}}=\{b_0,b_1,\hdots, b_3\}$. Then the Allen matrix cannot be integral Fourier matrix. In other words, there is no $C$-algebra of rank $4$ arising from integral Fourier matrix.
\end{thm}

\begin{proof} Suppose the Allen matrix is an integral Fourier matrix. Let $\delta(b_1)= k_1$, $\delta(b_2)=k_2$  and $\delta(b_3)=k_3$. Since $s$ is an integral Fourier matrix, $k_1, k_2$ and $k_3$ are square integers, see \cite[Lemma 12]{G1}. As $d_0=1+k_1+k_2+k_3$ and
$k_1, k_2$ and $k_3$ divide $d_0$. Therefore, $k_2+k_3\equiv-1\mod k_1$, $k_1+k_3\equiv-1\mod k_2$ and $k_1+k_2\equiv-1\mod k_3$.

Claim: $k_1=k_2=k_3=1$. Suppose $k_1\geq k_2, k_3$. If $k_1$ is an even integer then as $k_1,k_2$ and $k_3$ are squares, we have $k_1\equiv 0 \mod4$ and $d_0\not\equiv 0 \mod4$, a contradiction. Therefore $k_1, k_2$ and $k_3$ are odd integer.
On the other hand if all $k_1,k_2$ and $k_3$ are odd integers then
$k_1,k_2,k_3\equiv 1 \mod4$. But $d_0\equiv 0\mod4$, implies $d_0=k_1a$, $a\geq 4$. Therefore $k_1(a-1)=1+k_2+k_3$ implies $3k_1\leq 1+k_2+k_3=k_1(a-1)$ implies $k_2$ or $k_3>k_2$, again a contradiction.
\end{proof}

\begin{thm}\label{FourDifferentDegreesThm5}Let $(A,{{\mathbf B}}, \delta)$ be a non-homogenous $C$-algebra, of rank $5$, arising from an Allen matrix with standard  basis ${{\mathbf B}}=\{b_0,b_1,\hdots, b_4\}$. Then the Allen matrix cannot be integral Fourier matrix. In other words, there is no $C$-algebra of rank $5$ arising from integral Fourier matrix.
\end{thm}

\begin{proof}Suppose the Allen matrix is an integral Fourier matrix. Let $\delta(b_1)=k_1$, $\delta(b_2)= k_2$, $\delta(b_3)=k_3$  and $\delta(b_4)=k_4$. Since $s$ is an integral Fourier matrix, $k_1,k_2,k_3$ and $k_4$ are squares and integers, see \cite[Lemma 12]{G1}. By \cite[Lemma 3.7]{MC1}, $d_0$ is a square. Therefore, $d_0 \equiv 0,1 \mod4$.
Let $d_0=k_4a, d_0=k_3b,d_0=k_2c,d_0=k_1d$, each of $a,b,c$ and $d$ is greater than $1$ and a square integer as $A$ is non-homogeneous and $d_0$ is a square.

\smallskip

Case 1. Each of $k_1,k_2,k_3,k_4$ is an odd integer.

Then $d_0$ is odd, $k_1,k_2,k_3,k_4$ are odd implies $a,b,c,d$ are odd and greater than $9$. Without loss of generality, let $k_4\geq k_1,k_2,k_3$. Therefore, $d_0\geq 9k_4, d_0=1+k_1+k_2+k_3+k_4\leq 1+4k_4$, a contradiction.

\smallskip

Case 2. Three of $k_1,k_2,k_3,k_4$ are  odd and one is even.

Then $d_0$ is even. Without loss of generality, let $k_4$ is even. Thus $d_0=k_4a$ implies $a\geq 4$.

\smallskip

Subcase 1. If $k_4>k_1,k_2,k_3$.

Then $d_0\geq 4k_4$, $d_0\leq 1+(k_4-1)+(k_4-1)+(k_4-1)+k_4=4k_4-2$, a contradiction.

\smallskip

Subcase 2. If $k_4<$ one of $k_1,k_2,k_3$, say $k_3$, so $k_1,k_2\leq k_3$.

Then $d_0\geq 4k_3$ and $b$ is an even square. Thus $d_0\leq 1+k_3+k_3+k_3+(k_3-1)=4k_3$ implies $k_1=k_2=l$ and $k_4=k_3-1$, $d_0=4k_3$. Now $d_0=4x^2$, as $d_0$ is a square and an even integer. Hence $k_1=k_2=k_3=x^2$ and $k_4=x^2-1$. As $x^2-1$ divides $4x^2$ and $x$ is an odd integer thus  $x^2-1$ divides $4$, we get a contradiction.

\smallskip

Case 3. Two of $k_1,k_2,k_3,k_4$ are  odd and two are even.

Then $d_0\equiv 3 \mod 4$, a contradiction. 

\smallskip

Case 4. One of $k_1,k_2,k_3,k_4$ is odd and three are even.

Then $d_0\equiv 2 \mod 4$, a contradiction. 
\end{proof}

\begin{thebibliography}{10}

\bibitem{AFM} Z.~Arad, E.~Fisman, and M.~Muzychuk, Generalized table algebras, {\it Israel J. Math.}, {\bf 114} (1999), 29-60.

\bibitem{HM} A.~Hanaki and I.~Miyamoto, Classification of Small Association Schemes. (math.shinshu-u.ac.jp/~hanaki/as/).

 \bibitem{EB} E. Bannai, Association Schemes and Fusion Algebras, Journal of Algebraic Combinatorics 2 (1993), 327-344.

 \bibitem{BI} E. Bannai and E. Bannai, Spin Models on Finite Cyclic Groups, Journal of Algebraic Combinatorics 3 (1994), 243-259.

\bibitem{JA} Javad Bagherian and Amir Rahnamai Barghi, Standard character condition for C-algebras, arXiv:0803.2423 [math.RT].

\bibitem{HIB2} Harvey I. Blau, Table algebras, {\it European J. Combin.}, {\bf 30} (2009), no.~6, 1426-1455.

\bibitem{HIB1} Harvey I. Blau, Quotient structures in $C$-algebras, {\it J. Algebra}, {\bf 175} (1995), no.~1, 24-64; Erratum: {\bf 177} (1995), no.~1, 297-337.

\bibitem{MC2} Michael Cuntz, Fusion algebras for imprimitive complex
reflection groups, Journal of Algebra {\bf 311} (2007) 251-267.

\bibitem{MC} Michael Cuntz, Fusion algebras with negative structure constants, Journal of Algebra {\bf 319} (2008) 4536-4558.

\bibitem{MC1} Michael Cuntz, Integral modular data and congruences, J Algebr Comb {\bf29}, (2009), 357-387.

\bibitem{TG} Terry Gannon, Modular data: the algebraic combinatorics of conformal field theory. J. Algebraic Combin. \textbf{22}(2), 211-250 (2005).

\bibitem{HSnew1} Allen Herman and Gurmail Singh, Central torsion units of integral reality-based algebras with positive degree map, International Electronic Journal of Algebra, accepted.

\bibitem{HS1} Allen Herman and Gurmail Singh, On the Torsion Units of Integral Adjacency Algebras of Finite Association Schemes, {{\it Algebra}, Vol. 2014, 2014, Article ID 842378, 5 pages.

\bibitem{HS2} Allen Herman and Gurmail Singh, Torsion units of integral C-algebras, {\it JP Journal of Algebra, Number Theory, and Applications}, 36 (2), 2015, 141-155.

\bibitem{Hig87} D. G. Higman, Coherent algebras, {\it Linear Algebra Appl.}, {\bf 93} (1987), 209-239.

\bibitem{HRB} A. Hosseini and A. Rahnamai Barghi, Table algebras of rank 3 and its applications to strongly regular graphs, {\it Journal of Algebra and Its Applications}, {\bf 12} (5) (2013), 125-141.

\bibitem{EL}E. L. Rees, Graphical Discussion of the Roots of a Quartic Equation, {\it The American Mathematical Monthly}, Vol. 29, No. 2 (Feb., 1922), pp. 51-55

\bibitem{HSnew2} Gurmail Singh and Allen Herman, Orders of torsion units of integral reality-based algebras with positive degree map and rational multiplicities, submitted.

\bibitem{G1} Gurmail Singh, Classification of homogeneous Fourier matrices associated with modular data, arXiv:[math.RA].

}
\end{thebibliography}

\end{document}


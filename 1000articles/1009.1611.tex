\documentclass[12pt,a4paper,leqno]{amsart}
\usepackage{latexsym,amssymb,amsthm,amscd}
\usepackage{eepic}

\setlength{\evensidemargin}{6mm}
\setlength{\oddsidemargin}{6mm}
\setlength{\textheight}{230mm}
\setlength{\textwidth}{150mm}
\allowdisplaybreaks[4]

\catcode`\@=12
\makeatletter

\makeatother
\newtheorem{thm}{Theorem}[section]
\newtheorem{cor}[thm]{Corollary}
\newtheorem{lem}[thm]{Lemma}
\newtheorem{prop}[thm]{Proposition}
\theoremstyle{definition}
\newtheorem{defn}[thm]{Definition}
\newtheorem{rem}[thm]{Remark}
\newtheorem{exam}[thm]{Example}
\newtheorem{prob}[thm]{Problem}
\newtheorem{ques}[thm]{Question}

\title{Motivic invariant of real polynomial functions and Newton polyhedron}
\author{Goulwen Fichou and Toshizumi Fukui}
\thanks{The authors has been supported by Saitama University, Rennes 1
  University and the ANR project ANR-08-JCJC-0118-01.}
\address{IRMAR (UMR 6625), Universit\'e de Rennes 1, Campus de
  Beaulieu, 35042 Rennes Cedex, France and 
Department of Mathematics, Faculty of Science, Saitama University,
Shimo-Okubo, Urawa 338-8570, Japan}

\subjclass[2000]{14P20 (14B05 14P25 32S15)}
\keywords{Zeta functions, virtual Betti numbers, blow-Nash equivalence}

\begin{document}
\begin{abstract} We propose a computation of real motivic zeta
  functions for real polynomial functions, using Newton polyhedron.
As a consequence we show that the weights are blow-Nash invariants of
convenient weighted homogeneous polynomials in three variables.
\end{abstract}
\maketitle

In Singularity Theory, one aim to classify singular objects with
respect to a given equivalence relation. We focus on the singularities of
function germs. We are mainly interested in the case of weighted
homogeneous polynomial functions, that is polynomial functions that
become homogeneous by assigning a particular weight to each of the
variables. Concerning such functions, considered as germs at the
origin, we tackle the question of the invariance of the weights under
a given equivalence relation between germs.

Concerning complex analytic function germs, the first result is this
direction is due to K. Saito \cite{Saito} who proved in 1971 that the weights
are local analytic invariants of the pair $(\mathbb C^n,f^{-1}(0))$ at
the origin, for $f$ a weighted homogeneous polynomial. Concerning the
topological equivalence, E. Yoshinaga and M. Suzuki \cite{YoSu} in
1979 (and later T. Nishimura \cite{Ni} in 1986) proved the topological
invariance of the weights in dimension two, whereas O. Saeki
\cite{Saeki} in 1988 treated the three dimensional case.

In this paper, we are concerned with the real counterpart of this
question, considering equivalence relation on real analytic function germs. If the topological
equivalence is by far too weak in the real setting, the most relevant
equivalence relation to consider is the blow-analytic equivalence introduced by
T.-C. Kuo (cf. \cite{Kuo}, and also \cite{FKK,FP} for surveys). Real analytic function germs
$f,g:(\mathbb R^n,0)\longrightarrow (\mathbb R,0)$ are said to be
blow-analytically equivalent in the sense of \cite{Kuo} if
there exist real modifications $\beta_f:M_f \longrightarrow \mathbb R^n$
and $\beta_g:M_g \longrightarrow \mathbb R^n$ and an analytic
isomorphism $\Phi :(M_f, \beta_f^{-1}(0)) \longrightarrow (M_g,
\beta_g^{-1}(0))$ which induces a homeomorphism $\phi:(\mathbb R^n,0)
\longrightarrow  (\mathbb R^n,0)$ such that $f=g \circ \phi$. 

For polynomial functions, or more generally Nash functions
(i.e. real analytic functions with semi-algebraic graph), a natural
counterpart exists, called blow-Nash equivalence, that takes into
account the algebraic nature of Nash functions. This equivalence
relation have been proved to have nice properties
(cf. \cite{FKK,Fichou,simple}). 

The question of the invariance of the weights for weighted homogeneous
polynomial functions under blow-analytic equivalence already appeared
as a conjecture in \cite{Fukui} and as a question in \cite{KP}. A
positive answer has been given by O. M.~Abderrahmane \cite{Ould} for the two
variables case, using two invariants of the blow-analytic equivalence:
the Fukui Invariants \cite{Fukui} and zeta functions \cite{KP}
constructed by S. Koike and A. Parusi\'nski using Motivic Integration \cite{DL} with Euler
characteristic with compact supports as a measure. 

We prove as theorem \ref{w3} that the result holds true
in the three variables case for weighted homogeneous polynomial
functions that are convenient, under the blow-Nash equivalence. To
prove this, we investigate the zeta function introduced in
\cite{Fichou} as an invariant of the blow-Nash equivalence, using as a measure the virtual Poincar\'e polynomial
\cite{MCP}. This polynomial is an additive and multiplicative
invariant for real algebraic sets, which degree is equal to the
dimension of the variety. 

The main tool for the proof of theorem
\ref{w3} is to estimate the degrees of the coefficient of the zeta
functions in terms of the Newton polyhedron of a given polynomial
function.
Zeta functions in Motivic Integration have already been computed in
terms of Newton polyhedron \cite{DK,DL-modif,Guibert}, and our theorem
\ref{Thm1} is just an adaptation in order to focus on the question of
degrees. The main result in this paper, theorem \ref{ThDegBnd}, gives a bound for the degree of the
coefficient of the zeta function, which leads to the notion of
\textit{leading exponent} in section \ref{Le}. In the case of
convenient weighted homogeneous polynomial functions, this leading
exponent enable to recover precious informations on the weights. These
informations will be sufficient to conclude for theorem \ref{w3}.

\section{Motivic measure for arc space}

In this section we recall briefly how we can mesure arc spaces in the
context of real geometry, using
the more general theory of motivic integration as developped by Denef
\& Loeser \cite{DL}.

The measure takes its value in the Grothendieck ring of real algebraic
varieties \cite{MCP}. It is defined as the free abelian group $K_0$ generated by isomorphism
classes $[X]$ of real algebraic varieties modulo the subgroup generated by the
relation $[X]=[Y]+[X\setminus Y]$ for $Y\subset X$ a closed
subvariety. The ring structure comes from cartesian product of varieties.

\subsection{Motivic zeta functions}\label{arc-space}

Let $M$ be a real analytic manifold and $S$ a subset of $M$.
Consider the space of formal arcs
$$
\mathcal L(M,S):=\{
\alpha:({\mathbb{R}},0)\to (M,S):\
\alpha \textrm{~~formal}\}.
$$
We set $\mathcal L(M,x)$ when $S=\{x\}$ is reduced to one point, and let $L_k(M,x)$ is the set of $k$-jets of elements of $\mathcal L(M,x)$.
We set $L_k=L_k({\mathbb{R}}^n,0)$. 

Let $p_m:\mathcal{L}({\mathbb{R}}^n,0)\to L_m$ denote the map defined by 
taking $m$-jet. 
For a so-called constructible subset $\mathcal A$ of $\mathcal L({\mathbb{R}}^n,0)$, we define 
$$
[\mathcal A]=\lim_{m\to\infty}\frac{[p_m(\mathcal A)]}{{\mathbb{L}}^{mn}},
$$
where $[p_m(\mathcal A)]$ is the measure of $p_m(\mathcal A)$ in $K_0$
and ${\mathbb{L}}$ the measure of ${\mathbb{R}}$,
when the limit exists as an element of
$K_0({\mathop{\mathrm{Var}}\nolimits}_{\mathbb{R}})[[{\mathbb{L}}^{-1}]]$. This is for instance the case when $A$ is
the preimage under a truncation map $p_m$ of a Zariski constructible
subset of $L_k(M,x)$ (cf. \cite{DL}).
The subsets of the arc space we will
consider in this paper will all be constructible.

As an important example, we focus on the measure of the space of arcs
with a specified order.

\begin{exam}\label{ex1}
For $a=(a_1\dots,a_n)\in{\mathbb{N}}^n$, $a_i\ge 0$, we consider the set
$\mathcal{L}_a$ of arcs in $\mathbb R^n$ whose $i$-th component
vanishes if $a_i=0$ or is of order $a_i$ otherwise. Namely
$$
\mathcal{L}_a=\{\alpha\in\mathcal L({\mathbb{R}}^n,0):
{\mathop{\mathrm{ord}}\nolimits} x_i{\text{\scriptsize$\circ$}}\alpha=a_i\ (i\in I(a)),\ 
x_i{\text{\scriptsize$\circ$}}\alpha\equiv 0\ (i\not\in I(a))\}
$$
where $I(a)=\{i:a_i>0\}$. 
If $m$ is greater than the maximal value of $a_i, i=1,\ldots,n$, then 
$$
[p_m(\mathcal L_a)]=
{\mathbb{L}}^{m|I(a)|-{\mathop{\textrm{\scriptsize $\mathop\sum$}}}_{i}a_i}({\mathbb{L}}-1)^{|I(a)|}
=({\mathbb{L}}-1)^{|I(a)|}{\mathbb{L}}^{m|I(a)|-s(a)}
$$ 
where 
$s(a)={\mathop{\textrm{\scriptsize $\mathop\sum$}}}_{i=1}^na_i$. Therefore 
$$
[\mathcal L_a]=
\lim_{m\to\infty}
\frac{[p_m(\mathcal L_a)]}{{\mathbb{L}}^{mn}}=
\begin{cases}
({\mathbb{L}}-1)^n{\mathbb{L}}^{-s(a)}, \textrm{~~if~~} |I(a)|=n,\\
0, \textrm{~~if~~} |I(a)|<n.
\end{cases}
$$
\end{exam}

In other words, arcs with some components equal to zero can be seen,
as a finite order, as being the image under truncation of arcs with bigger order. We will use this description in order to compute in
section \ref{poly} the arc spaces associated to a given real analytic
function germ as follows. 

Let $f:({\mathbb{R}}^n,0)\to ({\mathbb{R}},0)$ be a real analytic
function germ. For $k\in \mathbb N$, we define the arc space $\mathcal
A_k(f)$ by
$$
\mathcal A_k(f)=
\{\alpha\in\mathcal L({\mathbb{R}}^n,0):f{\text{\scriptsize$\circ$}}\alpha(t)=ct^k+\cdots,\ c\ne0\}.$$
Similarly, we defined arc spaces with signs $\mathcal A^\pm_k(f)$ by
$$\mathcal A^\pm_k(f)=\{\alpha\in\mathcal L({\mathbb{R}}^n,0):f{\text{\scriptsize$\circ$}}\alpha(t)=ct^k+\cdots,\ c=\pm1\}.$$

Since the $k$-jet of $\alpha\in\mathcal L({\mathbb{R}}^n,0)$ 
determines the $k$-jet of $f{\text{\scriptsize$\circ$}}\alpha$, we obtain  
$$
[\mathcal A_k(f)]=\frac{[p_m(\mathcal A_k(f))]}{{\mathbb{L}}^{mn}},\textrm{~~and~~}
[\mathcal A^\pm_k(f)]=\frac{[p_m(\mathcal A^\pm_k(f))]}{{\mathbb{L}}^{mn}}.
$$
for $m\ge k$. The associated zeta function, and zeta functions with signs, are defined by 
$$
Z(f)={\mathop{\textrm{\scriptsize $\mathop\sum$}}}_{k=1}^\infty[\mathcal A_k(f)]t^k,\textrm{~~and~~}
Z^\pm(f)={\mathop{\textrm{\scriptsize $\mathop\sum$}}}_{k=1}^\infty[\mathcal A^\pm_k(f)]t^k. 
$$

\begin{exam} Consider the one variable polynomial function given by
  $f(x)=x^d$. Then
$$
[\mathcal A_k(f)]=
\begin{cases}
({\mathbb{L}}-1){\mathbb{L}}^{-a}&(k=ad)\\
0& (d\not| k)
\end{cases}$$
so that 
$$Z(f)=\frac{({\mathbb{L}}-1)t^d/{\mathbb{L}}}{1-t^d/{\mathbb{L}}}.$$
\end{exam}

\subsection{Virtual Poincar\'e polynomial}\label{virt}

For real algebraic varieties, the best realization known of the
Grothendieck ring is given by the virtual Poincar\'e polynomial
\cite{MCP}. It assigns to a Zariski constructible set a polynomial
with integer coefficients in such a way that the coefficients coincide
with the Betti numbers with ${\mathbb{Z}}_2$-coefficients for proper regular
real algebraic sets. Denoting by $u$ the indeterminacy, the virtual
Poincar\'e polynomial of a $n$-dimensional sphere is equal to $1+u^n$,
and a consequence of the addtivity the virtual Poincar\'e polynomial
of an affine $n$-dimensional space is $u^n$.
Moreover, the virtual Poincar\'e polynomial
specializes to the Euler characteristic with compact supports when it
is evaluated at $u=-1$.

A crucial porperty of the virtual Poincar\'e polynomial is that its
degree is equal to the dimension of the set. In particular, and
contrary to the Euler characteristics with compact supports, the
virtual Poincar\'e polynomial cannot be zero for a non-empty set. In particular
we will be interested in the degree of $[\mathcal A_k(f)]$ in section \ref{estim-deg}.

\section{Arc spaces and Newton polyhedron}
In this section we are interested in expressing the mesure of the arc
spaces associated to a germ in terms of its Newton polyhedron. Similar
results are already used in \cite{DK,DL-modif,Guibert}. Here we focus mainly on a
formula that will enable us to estimate efficiently the degrees of the
virtual Poincar\'e polynomial of the arc spaces in terms of the Newton polyhedron.

We begin by introducing some standart notations for Newton polyhedron.

\subsection{Newton polyhedron}\label{Np}
Let $f:{\mathbb{R}}^n\to{\mathbb{R}}$ denote a polynomial function.
Consider its Taylor expansion at $0$:
$$
f(x)={\mathop{\textrm{\scriptsize $\mathop\sum$}}}_{\nu \in {\mathbb{N}}^n}c_\nu x^\nu,\qquad
x^\nu={x_1}^{\nu_1}\cdots x_n^{\nu_n},\quad\nu=(\nu_1,\dots,\nu_n) \in {\mathbb{N}}^n. 
$$
Let $\Gamma_+(f)$ denote the \textbf{Newton polyhedron} of $f$,
defined as the convex hull of the set
$$
{\mathop{\textrm{\small $\mathop\bigcup$}}}(\{\nu\}+{\mathbb{R}}_\ge^n:c_\nu\ne0). 
$$
The \textbf{Newton boundary} $\Gamma(f)$ of $f$ is the union of the
compact faces of $\Gamma_+(f)$. We denote by $\gamma < \Gamma_+(F)$
the belonging of the face $\gamma$ to $\Gamma_+(f)$.

For $a=(a_1,\dots,a_n)\in \mathbb R^n$, $\nu=(\nu_1,\dots,\nu_n)\in \mathbb R^n$, we set 
$\langle a,\nu\rangle=a_1\nu_1+\cdots+a_n\nu_n$, and define the
multiplicity $m_f(a)$ of $a$ along $f$ by
$$
m_f(a)=\min\{\langle a,\nu\rangle :\nu\in\Gamma_+(f)\}$$
and the face $\gamma_f(a)$ of the Newton polyhedron associated to $a$
by
$$
\gamma_f(a)=\{\nu\in\Gamma_+(f):\langle a,\nu\rangle=m(a)\}.$$
Define also 
$$f_S(x)={\mathop{\textrm{\scriptsize $\mathop\sum$}}}_{\nu\in S}c_\nu x^\nu$$
for a subset $S$ of ${\mathbb{R}}^n$. 

We define an equivalence relation in ${\mathbb{R}}_\ge^n$ by 
$$
a\sim b \quad\Longleftrightarrow\quad \gamma_f(a)=\gamma_f(b)
$$
for $a=(a_1,\dots,a_n)$, $b=(b_1,\dots,b_n)\in{\mathbb{R}}^n_\ge$. 
We call the quotient space ${\mathbb{R}}^n_+/\sim$ by the 
\textbf{dual Newton polyhedron} 
and denote it by $\Gamma^*(f)$. This is often identified with a cone
subdivision of ${\mathbb{R}}_\ge^n$.  

Let $\Gamma^{(1)}(f)$ denote the set of primitive generators of the
1-cones of $\Gamma^*(f)$. We denote by $\Gamma^{(1)}_+(f)$ the set of $a\in\Gamma^{(1)}(f)$ 
with $m_f(a)>0$. 

\subsection{Motivic invariant of polynomial function}\label{poly}
We want to express the measure of the arc spaces associated to a
polynomial function in terms of its Newton polyhedron. The set of
exponents $k$ for which the arc spaces $\mathcal A_k(f)$ are not empty
have already been studied in the context of blow-analytic
equivalence (they are the so-called Fukui Invariants, cf
\cite{Fukui}). This set coincides with the set of exponents that do appear in the
zeta function with non-zero coefficients since there measure under the
virtual Poincar\'e polynomial cannot be equal to zero
(cf. section \ref{virt}). Actually, define subsets $A(f)$ and $A^\pm(f)$ of ${\mathbb{N}}$ by 
$$
A(f)=\{k:\mathcal A_k(f)\ne\emptyset\}$$
and
$$
A^\pm(f)=\{k:\mathcal A_k^\pm(f)\ne\emptyset\}. 
$$ 
Set 
$$m_0(f)=\min\{m_f(a):f_{\gamma(a)} \textrm{ is not definite}\},$$
and
\begin{align*}
T(f)=&\{m\in{\mathbb{N}}:m\ge m_0(f)\},\\
S(f)=&\{m_f(a):\exists c\in({\mathbb{R}}^*)^n, f_{\gamma(a)}(c)\ne0\},\\
S^\pm(f)=&\{m_f(a):\exists c\in({\mathbb{R}}^*)^n, f_{\gamma(a)}(c)=\pm1\}.
\end{align*}
Then 
$$
A(f)=S(f)\cup T(f),\qquad
A^\pm(f)=S^\pm(f)\cup T(f).
$$

We will now be interested in computing the mesure
of the non-empty arc spaces in terms of the Newton polyhedron of
$f$. Define algebraic subsets $X_\gamma$ and $X^\pm_\gamma$ of
$({\mathbb{R}}^*)^n$ associated to a face $\gamma$ of the Newton polyhedron by
\begin{align*}
X_\gamma=&\{c\in({\mathbb{R}}^*)^n:f_\gamma(c)=0\},\\
X^{\pm}_\gamma=&\{c\in({\mathbb{R}}^*)^n:f_\gamma(c)=\pm1\}.
\end{align*}

\begin{rem} The measure of $X_\gamma$ and $X^\pm_\gamma$
  contain $({\mathbb{L}}-1)^{n-\dim\gamma}$ as a factor in the sense that there exist varieties 
$\widehat{X}_\gamma$ , $\widehat{X}^{\pm}_\gamma$ 
in $({\mathbb{R}}^*)^{\dim\gamma}$ so that 
$$
X_\gamma=({\mathbb{R}}^*)^{n-\dim\gamma}\times\widehat{X}_\gamma$$
and
$$
X^{\pm}_\gamma=({\mathbb{R}}^*)^{n-\dim\gamma}\times\widehat{X}^{\pm}_\gamma. 
$$
Looking at measure, we therefore obtain  
$$
[X_\gamma]=({\mathbb{L}}-1)^{n-\dim\gamma}[\widehat{X}_\gamma]$$
and
$$
[X^{\pm}_\gamma]=({\mathbb{L}}-1)^{n-\dim\gamma}[\widehat{X}^{\pm}_\gamma]. 
$$
\end{rem}

We say $f$ is \textbf{non-degenerate} if all singular points of
$f_\gamma$ are concentrated in the hyperplane coordinates for all are
compact faces $\gamma$ of $\Gamma_+(f)$, namely
$$
\biggl(
{\frac{\partial{f_\gamma}}{\partial{x_1}}}(c),\dots,{\frac{\partial{f_\gamma}}{\partial{x_n}}}(c)
\biggr)\ne(0,\dots,0)
$$
for each $c\in ({\mathbb{R}}^*)^n$ with $f_\gamma(c)=0$, where $\gamma$ runs
along the compact faces of $\Gamma_+(f)$.

If $f$ is non-degenerate, 
then $X_\gamma$ (resp. $\widehat{X}_\gamma$) is a non-singular 
submanifold of $({\mathbb{R}}^*)^n$ (resp. $({\mathbb{R}}^*)^{I(\gamma)}$) 
of codimension 1, whenever it is not empty.

Next lemma computes the measure of the arc spaces associated to $f$
for arcs with a specified order $a\in \mathbb N^n$. As motivated in
section \ref{arc-space}, it is sufficient to consider strictly positive orders.

\begin{lem}\label{KeyLem}
Take $a>0$ in $\mathbb N^n$.
If $f$ is non-degenerate and $\gamma=\gamma(a)$, then we have 
\begin{align*}
[\mathcal L_a\cap \mathcal A_k(f)]=&
\begin{cases}
0&\textrm{~~if~~}m_f(a)>k\\
\bigl(({\mathbb{L}}-1)^n-[X_{\gamma}]\bigr)\, 
{\mathbb{L}}^{-s(a)}&\textrm{~~if~~}m_f(a)=k,\\
({\mathbb{L}}-1)\,[X_{\gamma}]\,{\mathbb{L}}^{-s(a)-k+m_f(a)}&\textrm{~~if~~}m_f(a)<k.
\end{cases}
\end{align*}
In the case with signs
\begin{align*}
[\mathcal L_a\cap \mathcal A^\pm_k(f)]=&
\begin{cases}
0&\textrm{~~if~~}m_f(a)>k\\
[X^\pm_{\gamma}]\,{\mathbb{L}}^{-s(a)}&\textrm{~~if~~}m_f(a)=k,\\
[X_{\gamma}]\,{\mathbb{L}}^{-s(a)-k+m_f(a)}&\textrm{~~if~~}m_f(a)<k.
\end{cases}
\end{align*}
\end{lem}

\begin{proof}
Take $\alpha\in\mathcal{L}_a$ and 
define $\phi(t)=(\phi_1(t),\dots,\phi_n(t))$ by 
$$
\alpha(t)=(t^{a_1}\phi_1(t),\dots,t^{a_n}\phi_n(t)).
$$
We remark that $\phi_i(0)\ne0$ for $i=1,\dots,n$. 
Then we have
$$
f(\alpha(t))
={\mathop{\textrm{\scriptsize $\mathop\sum$}}}_{\nu \in {\mathbb{N}}^n} c_\nu
(t^{a_1}\phi_1(t))^{\nu_1}\cdots
(t^{a_n}\phi_n(t))^{\nu_n}\\
={\mathop{\textrm{\scriptsize $\mathop\sum$}}}_{\nu \in {\mathbb{N}}^n} c_\nu
\phi(t)^{\nu}t^{\langle a,\nu\rangle}.
$$
Setting $F(t)=t^{-m_f(a)}f(\alpha(t))$, we obtain therefore 
\begin{equation}\label{A}
F(t)=f_a(\phi(t))+R(t)
\end{equation}
where $R(t)$ is a function, depending on 
$t$ and the coefficient of $\phi(t)$,  with $R(t)\to0$ ($t\to0$). 
This implies the assertion in case $m_f(a)=k$. 

We focus now on the case $m_f(a)<k$. By differentiating \eqref{A} by $t$, we obtain 
$$
F'(t)={\mathop{\textrm{\scriptsize $\mathop\sum$}}}_{i=1}^n{\frac{\partial{f_a}}{\partial{x_i}}}(\phi(t))(x_i{\text{\scriptsize$\circ$}}\phi)'(t)+R'(t).
$$
Differentiating again, the result is
$$
F''(t)=
{\mathop{\textrm{\scriptsize $\mathop\sum$}}}_{i=1}^n{\frac{\partial{f_a}}{\partial{x_i}}}(\phi(t))(x_i{\text{\scriptsize$\circ$}}\phi)''(t)
+{\mathop{\textrm{\scriptsize $\mathop\sum$}}}_{i=1}^n{\mathop{\textrm{\scriptsize $\mathop\sum$}}}_{j=1}^n{\frac{\partial{{}^2f_a}}{\partial{x_ix_j}}}(\phi(t))
(x_i{\text{\scriptsize$\circ$}}\phi)'(t)
(x_j{\text{\scriptsize$\circ$}}\phi)'(t)
+R''(t).
$$
Repeating differentiation $\ell$ times, we obtain 
$$
F^{(\ell)}(t)=
{\mathop{\textrm{\scriptsize $\mathop\sum$}}}_{i=1}^n{\frac{\partial{f_a}}{\partial{x_i}}}(\phi(t))(x_i{\text{\scriptsize$\circ$}}\phi)^{(\ell)}(t)
+\cdots
+R^{(\ell)}(t).
$$
We remark that $(x_i{\text{\scriptsize$\circ$}}\phi)^{(\ell)}(0)$ concerns only on the first
 term of the equation when we evaluate $t$ at $t=0$. 
We consider the condition 
\begin{equation}\label{M}
F'(0)=F''(0)=\cdots=F^{(k-m(a)-1)}(0)=0.
\end{equation}
Since $f$ is non-degenerate, one of the partial derivatives of 
$f_a$ is non-zero on the zero locus of $f_a$, and 
\eqref{M} determines the coefficient of $\phi(t)$, inductively. 
The equation 
$$
F^{(k-m(a))}(0)\ne0\qquad(\textrm{resp. }\ F^{(k-m(a))}(0)=\pm1)
$$ 
imposes that a coefficient of $\phi(t)$ should be non-zero (resp. determines 
a coefficient of $\phi(t)$). 
So this implies the assertion in case $m_f(a)<k$. 
\end{proof}

As a consequence we obtain a nice description of the mesure of the arc
spaces associated to $f$ in terms of the geometry of $f$ and of the
combinatrics of its Newton polyhedron. For a face $\gamma$ of $\Gamma_+(f)$ and $k \in \mathbb N$, set
\begin{align*}
P_k(\gamma)=&
{\mathop{\textrm{\scriptsize $\mathop\sum$}}}_{a:\ a>0,~\gamma(a)=\gamma,\ m_f(a)=k}\,{\mathbb{L}}^{-s(a)}
\end{align*}
and
\begin{align*}
Q_k(\gamma)=&
{\mathop{\textrm{\scriptsize $\mathop\sum$}}}_{a:\ a>0,~\gamma(a)=\gamma,\ m_f(a)<k}\,{\mathbb{L}}^{-k+m_f(a)-s(a)}.
\end{align*}
Note that, even if the summation may be infinite (we consider it in the
completed Grothendieck ring \cite{DL}), $P_k(\gamma)$ and $Q_k(\gamma)$ are element of the
Grothendieck ring of varieties localised at ${\mathbb{L}}$ since it measure
arcs with a fixed order along $f$.

\begin{thm}\label{Thm1}
If $f$ is a non-degenerate polynomial, the measure of the arcs spaces
associated to $f$ can be decomposed as:
$$[\mathcal A_k(f)]
=
{\mathop{\textrm{\scriptsize $\mathop\sum$}}}_{\gamma < \Gamma_+(f)}\,\bigl(({\mathbb{L}}-1)^{n}-[X_{\gamma}]\bigr)\,P_k(\gamma)
+({\mathbb{L}}-1){\mathop{\textrm{\scriptsize $\mathop\sum$}}}_{\gamma < \Gamma_+(f)}\,[X_{\gamma}]\,Q_k(\gamma)$$
In the case with signs, we obtain similarly:
$$[\mathcal A_k^\pm(f)]
={\mathop{\textrm{\scriptsize $\mathop\sum$}}}_{\gamma < \Gamma_+(f)}\,[X_\gamma^\pm]\ P_k(\gamma)
+{\mathop{\textrm{\scriptsize $\mathop\sum$}}}_{\gamma < \Gamma_+(f)}\,[X_\gamma]\ Q_k(\gamma).$$
\end{thm}

\begin{rem}
That decomposition of the measure of the arc spaces into two
sums is motivated by the difference between arcs of order $a$ that
directly contribute to the coefficient of $t^{m_f(a)}$ of the zeta
  functions and arcs that contribute to coefficient for bigger orders
  than $m_f(a)$. To understand both contribution will be the main step
  in section \ref{weight} in order to recover the weights of a weighted
  homogeneous polynomial.
\end{rem}

\begin{rem}\label{levelh} We can rewrite $P_k(\gamma)$ as
$$P_k(\gamma)={\mathop{\textrm{\scriptsize $\mathop\sum$}}}_{a:\ a>0,~\gamma(a)=\gamma,\ m_f(a)=k}\,{\mathbb{L}}^{-k+m_f(a)-s(a)}$$
since $m_f(a)=k$. Therefore the difference between $P_k(\gamma)$ and
$Q_k(\gamma)$ lies in the value of $m_f$.
In particular, we are lead to focus on the levels of
the piecewise linear function $m_f-s$ defined on the dual of the
Newton polyhedron of $f$, and more precesely on the subsets defined by
$m_f=k$ and $m_f<k$ for a given integer $k\in \mathbb N$. In the
sequel, we denote by
$h$ the function $h=m_f-s$.
\end{rem}

\textit{Proof of theorem \ref{Thm1}}
By Lemma \ref{KeyLem}, we obtain
\begin{align*}
[\mathcal A_k(f)]
=&
{\mathop{\textrm{\scriptsize $\mathop\sum$}}}_{a:\ a>0}\,
[\mathcal L_a\cap\mathcal A_k(f)]\\
=&
{\mathop{\textrm{\scriptsize $\mathop\sum$}}}_{\gamma < \Gamma_+(f)}\,
\bigl(({\mathbb{L}}-1)^{n}-[X_{\gamma}]\bigr)
{\mathop{\textrm{\scriptsize $\mathop\sum$}}}_{a:\ a>0, ~\gamma(a)=\gamma,\ m_f(a)=k}\,{\mathbb{L}}^{-s(a)}\\
&+
({\mathbb{L}}-1)
{\mathop{\textrm{\scriptsize $\mathop\sum$}}}_{\gamma < \Gamma_+(f)}\,[X_{\gamma}]\,
{\mathop{\textrm{\scriptsize $\mathop\sum$}}}_{a:\ a>0,~\gamma(a)=\gamma,\ m_f(a)<k}
{\mathbb{L}}^{-k+m_f(a)-s(a)}. 
\end{align*}
Similarly, we have 
\begin{align*}
[\mathcal A_k^\pm(f)]
=&
{\mathop{\textrm{\scriptsize $\mathop\sum$}}}_{a:\ a>0}\,
[\mathcal L_a\cap\mathcal A_k^\pm(f)]\\
=&
{\mathop{\textrm{\scriptsize $\mathop\sum$}}}_{\gamma < \Gamma_+(f)}\,
[X_{\gamma}^\pm]
{\mathop{\textrm{\scriptsize $\mathop\sum$}}}_{a:\ a>0,~\gamma(a)=\gamma,\ m_f(a)=k}\, 
{\mathbb{L}}^{-s(a)}\\
&+
{\mathop{\textrm{\scriptsize $\mathop\sum$}}}_{\gamma < \Gamma_+(f)}\,
[X_{\gamma}]\ 
{\mathop{\textrm{\scriptsize $\mathop\sum$}}}_{a:\ a>0,~\gamma(a)=\gamma,\ m_f(a)<k}
{\mathbb{L}}^{-k+m_f(a)-s(a)}. 
\end{align*}
\begin{flushright} $\Box$ \end{flushright}

\begin{rem}\label{closed}
Note that it is possible to write down $[\mathcal A_k(f)]$ in terms of
closed algebraic subsets $\overline{\widehat{X}_\gamma}$. This form
will be useful in section \ref{reco}. To this aim, we remark that there are integers 
$\sigma$
are faces of $\Gamma_+(f)$, so that 
$$
[\widehat{X}_\gamma]
={\mathop{\textrm{\scriptsize $\mathop\sum$}}}_{\sigma\subset\gamma}m_{\gamma,\sigma}[\overline{\widehat{X}_\sigma}]
$$
This is based on the following
equalities:$[\overline{\widehat{X}_\gamma}]={\mathop{\textrm{\scriptsize $\mathop\sum$}}}_{\tau:
  \textrm{face of $\gamma$}}[\widehat{X}_\tau].$

For instance, if $\dim\gamma=1$, we clearly have
$[\widehat{X}_\gamma]=[\overline{\widehat{X}_\gamma}].$ whereas if  $\dim\gamma=2$, we have
$[\overline{\widehat{X}_\gamma}]=
[\widehat{X}_\gamma]+
\sum_{\tau: \textrm{1-face of $\gamma$}}[\widehat{X}_\tau]$,
and thus 
$$
[\widehat{X}_\gamma]=[\overline{\widehat{X}_\gamma}]-
{\mathop{\textrm{\scriptsize $\mathop\sum$}}}_{\tau: \textrm{1-face of $\gamma$}}[\overline{\widehat{X}_\tau}]. 
$$

More generally we obtain a decomposition of the form
$$[\mathcal A_k(f)]
=({\mathbb{L}}-1)^n{\mathop{\textrm{\scriptsize $\mathop\sum$}}}_{\gamma}P_k(\gamma)
-{\mathop{\textrm{\scriptsize $\mathop\sum$}}}_{\sigma}({\mathbb{L}}-1)^{n-\dim\sigma}
[\overline{\widehat{X}_\sigma}]\overline{P}_k(\sigma)
+{\mathop{\textrm{\scriptsize $\mathop\sum$}}}_{\sigma}({\mathbb{L}}-1)^{n-\dim\sigma+1}
[\overline{\widehat{X}_\sigma}]\overline{Q}_k(\sigma)$$
where 
\begin{align*}
\overline{P}_k(\sigma)=&
{\mathop{\textrm{\scriptsize $\mathop\sum$}}}_{a:\ a>0}\,
m_{\gamma(a),\sigma}({\mathbb{L}}-1)^{\dim\gamma(a)-\dim\sigma}
{\mathbb{L}}^{-s(a)}:\gamma(a)\supset\sigma,\ m_f(a)=k,\\
\overline{Q}_k(\sigma)=&
{\mathop{\textrm{\scriptsize $\mathop\sum$}}}_{a:\ a>0}\,
m_{\gamma(a),\sigma}({\mathbb{L}}-1)^{\dim\gamma(a)-\dim\sigma}
{\mathbb{L}}^{-s(a)-k+m_f(a)}:
\gamma(a)\supset\sigma,\ m_f(a)<k.
\end{align*}
\end{rem}

\begin{rem}
Let $\Sigma$ be a nonsingular subdivision of $\Gamma^*(f)$. 
Let $\Sigma^{(k)}$ denote the set of $k$-cones of $\Sigma$. 
We identifies $\Sigma^{(1)}$ with the set of primitive vectors 
which generate 1-cones.  
We set 
\begin{align*}
N_k(m_1,\dots,m_p)=&
\{(u_1,\dots,u_p)\in{\mathbb{Z}}^p:
u_1m_1+\cdots+u_pm_p=k,\ u_1>0,\dots,u_p>0\}
\\
N_{<k}(m_1,\dots,m_p)=&
\{(u_1,\dots,u_p)\in{\mathbb{Z}}^p:
u_1m_1+\cdots+u_pm_p<k,\ u_1>0,\dots,u_p>0\}
\end{align*}
for positive integers $k$, $m_1$, \dots, $m_p$. 
Then we have 
\begin{align*}
P_k(\gamma)=&
\sum_{\begin{matrix}
\sigma\in\Sigma\\
\sigma=\langle a^{j_1}, \dots, a^{j_p}\rangle_{{\mathbb{R}}_\ge}\\
a^{j_1},\dots,a^{j_p}\in\Sigma^{(1)}\\
\gamma(a^{j_l})\supset\gamma\ (1\le l\le p)
\end{matrix}}
\sum_{(u_1,\dots,u_p)\in N_k(m(a^{j_1}),\dots,m(a^{j_p}))}
{\mathbb{L}}^{-u_1s(a^{j_1})-\dots-u_ps(a^{j_p})},
\\
Q_k(\gamma)=&
\sum_{\sigma\in\Sigma,\textrm{\scriptsize
$\begin{matrix}
\sigma\in\Sigma\\
\sigma=\langle a^{j_1}, \dots, a^{j_p}\rangle_{{\mathbb{R}}_\ge}\\
a^{j_1},\dots,a^{j_p}\in\Sigma^{(1)}\\
\gamma(a^{j_l})\supset\gamma\ (1\le l\le p)
\end{matrix}$} }
\sum_{(u_1,\dots,u_p)\in N_{<k}(m(a^{j_1}),\dots,m(a^{j_p}))}
{\mathbb{L}}^{-u_1s(a^{j_1})-\dots-u_ps(a^{j_p})}.\\
\end{align*}
We will use this description in the proof of proposition \ref{prop-2} below.
\end{rem}

\subsection{Two variables case}
Theorem \ref{Thm1} enables to give a precise formula for the zeta
functions (without using resolution of singularities). We will give in
this section a complete description of the naive zeta functions for
two variables convenient polynomials.

We consider a polynomial function $f(x_1,x_2)$ in two variables. As
noticed in remark \ref{levelh}, we investigate first the level of the
piecewise linear function $h=m_f-s$ on the dual of the Newton
polyhedron of $f$.

\begin{lem}
For any $a\in {\mathbb{N}}^2$ we have $h(a)\ge0$. Moreover, 
\begin{itemize}
\item $h(a)=0$ if and only if $(a;m_f(a))=(1,1;2)$. 
\item $h(a)=1$ if and only if 
$$
(a;m_f(a))=(1,1;3), (1,2;4), (2,1;4), (2,3;6), (3,2;6).
$$ 
\end{itemize}
\end{lem}
\begin{proof}
Let $p$ and $q$ be positive coprime integers.
Let $(p_0+qg,q_0)$ and $(p_0,q_0+pg)$ be two successive vertices of the Newton polyhedron $\Gamma_+(f)$.
The vector $(p,q)$ supports the face connecting these two points. 
Since $m_f(p,q)=p_0p+q_0q+pqg$, we have 
$$
m_f(p,q)-s(p,q)=(p_0-1)p+(q_0-1)q+pqg.
$$
\begin{itemize}
\item 
If $p_0\ge1$ and $q_0\ge1$, then  
$$
m_f(p,q)-s(p,q)=(p_0-1)p+(q_0-1)q+pqg\ge1.
$$  
The equality holds if and only if $p_0=q_0=p=q=g=1$.
\item 
If $p_0\ge1$ and $q_0=0$, then 
$$
m_f(p,q)-s(p,q)=(p_0-1)p-q+pqg=(p_0-1)p+(pg-1)q\ge0.
$$
\begin{itemize}
\item The equality holds if and only if $p_0=p=g=1$. 
\item If $pg\ge2$, then $m_f(p,q)-s(p,q)\ge1$, 
and the equality holds only if $pg=2$ and $q=1$. 
\end{itemize}
\item 
If $p_0=q_0=0$, then $m_f(p,q)-s(p,q)=pqg-p-q$. 
\begin{itemize}
\item 
If $g\ge2$, $(p,q)=(1,1)$, then  
$m_f(p,q)-s(p,q)=g-2\ge0$.
\item 
If $g\ge2$, then  
$$
m_f(p,q)-s(p,q)=pqg-p-q=(g-2)pq+p(q-1)+q(p-1)\ge0.
$$ 
The equality holds if and only if $g=2$ and $p=q=1$. 
\item 
If $g\ge2$ and $pq\ge2$, then 
$$
m_f(p,q)-s(p,a)=(g-2)pq+(p-1)q+(q-1)p\ge1.
$$  
The equality holds if and only if $g=2$ and $(p,q)=(1,2)$ or $(2,1)$. 
\item If $g=1$ and $(p,q)\ne(1,1)$, then 
$$
m_f(p,q)-s(p,q)-1=pq-p-q-1=(p-1)(q-1)-2\ge0. 
$$
The equality holds if and only if $(p,q)=(2,3)$ or $(3,2)$. 
\end{itemize}
\end{itemize}
\end{proof}

\begin{exam}
Let $f(x_1,x_2)={x_1}^{d_1}+{x_2}^{d_2}$,
where $d_1$ and $d_2$ are even integers. 
Set $(d_1,d_2)$ the greatest common divisor of $d_1$ and $d_2$
and $d'_1=d_1/(d_1,d_2)$, $d'_2=d_2/(d_1,d_2)$. 
If $k=d'_1d'_2(d_1,d_2)$, then we have
\begin{align*}
\frac{[\mathcal A_k(f)]}{({\mathbb{L}}-1)^2}
=&
{\mathbb{L}}^{-d'_1-d'_2}
+{\mathop{\textrm{\scriptsize $\mathop\sum$}}}_{a_2>d_1'}{\mathbb{L}}^{-d'_2-a_2}
+{\mathop{\textrm{\scriptsize $\mathop\sum$}}}_{a_1>d'_2}{\mathbb{L}}^{-a_1-d'_1}
\\
=&
{\mathbb{L}}^{-d'_1-d'_2}
+2{\mathop{\textrm{\scriptsize $\mathop\sum$}}}_{s\ge1}{\mathbb{L}}^{-d'_1-d'_2-s}. 
\end{align*}
\end{exam}

Let $f(x_1,x_2)$ be a polynomial with non-degenerate Newton principal
part.
Choose primitive vectors $a^j$, for $j=0,1,\dots,q$, so that 
$$
\Gamma^{(1)}(f)\subset\{a^0,a^1,\dots,a^q\},\quad \det(a^j a^{j+1})=1. 
$$
We assume that $f$ is convenient. 
Then $a^0=(1,0)$ and $a^q=(0,1)$, and the vectors $a^0$, $a^1$, \dots,
$a^{q}$ define a nonsingular subdivison of $\Gamma^*(f)$.  

\begin{prop}\label{prop-2} Let $f(x_1,x_2)$ be a convenient polynomial with
  non-degenerate Newton principal part. With the notations introduced
  upstairs, we have
\begin{align*}
\frac{[\mathcal A_k(f)]}{{\mathbb{L}}-1}
=&
{\mathop{\textrm{\scriptsize $\mathop\sum$}}}_{m_f(a^1)|k} {\mathbb{L}}^{-\tfrac{ks(a^1)}{m_f(a^1)}}
+({\mathbb{L}}-1){\mathop{\textrm{\scriptsize $\mathop\sum$}}}_{j=1}^{q-1}
{\mathop{\textrm{\scriptsize $\mathop\sum$}}}_{(u,v)\in N_k(m_f(a^j),m_f(a^{j+1}))}{\mathbb{L}}^{-us(a^j)-vs(a^{j+1})}\\
&+{\mathop{\textrm{\scriptsize $\mathop\sum$}}}_{m_f(a^q)|k} {\mathbb{L}}^{-\tfrac{ks(a^{q-1})}{m_f(a^{q-1})}}
+\sum_{j=1}^{q-1}(({\mathbb{L}}-1)-[\widehat{X}_{\gamma(a^j)}])
{\mathop{\textrm{\scriptsize $\mathop\sum$}}}_{m(a^j)|k}{\mathbb{L}}^{-\tfrac{ks(a^j)}{m(a^j)}}\\
&+({\mathbb{L}}-1)\sum_{j=1}^{q-1}[\widehat{X}_{\gamma(a^j)}]
\frac{{\mathbb{L}}^{-k-h(a^j)}(1-{\mathbb{L}}^{-\lfloor\frac{k-1}{m_f(a^j))}\rfloor h(a^j)})}
{1-{\mathbb{L}}^{-h(a^j)}}. 
\end{align*}
\end{prop}
\begin{proof}
By theorem \ref{Thm1}, we have the following description of the
measure of $\mathcal A_k(f)$: 
\begin{align*}
\frac{[\mathcal A_k(f)]}{{\mathbb{L}}-1}
=&
({\mathbb{L}}-1){\mathop{\textrm{\scriptsize $\mathop\sum$}}}_{a:\ a>0,~m_f(a)=k, \ \dim\gamma(a)=0}
{\mathbb{L}}^{-s(a)}\\
&+
{\mathop{\textrm{\scriptsize $\mathop\sum$}}}_{\gamma:\dim\gamma=1}
\bigl(({\mathbb{L}}-1)-[\widehat{X}_{\gamma}]\bigr)
{\mathop{\textrm{\scriptsize $\mathop\sum$}}}_{a:~a>0, ~\gamma(a)=\gamma,~m_f(a)=k}
{\mathbb{L}}^{-s(a)}
\\
&+({\mathbb{L}}-1)
{\mathop{\textrm{\scriptsize $\mathop\sum$}}}_{\gamma:\dim\gamma=1}
[\widehat{X}_{\gamma}]
{\mathop{\textrm{\scriptsize $\mathop\sum$}}}_{a:\ a>0, ~\gamma(a)=\gamma,~m_f(a)<k}
{\mathbb{L}}^{-s(a)-k+m_f(a)}
\end{align*}

Then

{\small
\begin{align*}
&\frac{[\mathcal A_k(f)]}{{\mathbb{L}}-1}
\\
=&
({\mathbb{L}}-1)
\biggl[{\mathop{\textrm{\scriptsize $\mathop\sum$}}}_{m_f(a^1)|k} {\mathbb{L}}^{-u-\tfrac{ks(a^1)}{m_f(a^1)}}
+{\mathop{\textrm{\scriptsize $\mathop\sum$}}}_{j=1}^{q-1}
{\mathop{\textrm{\scriptsize $\mathop\sum$}}}_{(u,v)\in N_k(m_f(a^j),m_f(a^{j+1}))}{\mathbb{L}}^{-us(a^j)-vs(a^{j+1})}
\\
&+{\mathop{\textrm{\scriptsize $\mathop\sum$}}}_{m_f(a^q)|k} {\mathbb{L}}^{-\tfrac{ks(a^{q-1})}{m_f(a^{q-1})}-v}\biggr]
+{\mathop{\textrm{\scriptsize $\mathop\sum$}}}_{j=1}^{q-1}(({\mathbb{L}}-1)-[\widehat{X}_{\gamma(a^j)}])
{\mathop{\textrm{\scriptsize $\mathop\sum$}}}_{m(a^j)|k}{\mathbb{L}}^{-\tfrac{ks(a^j)}{m(a^j)}}\\
&+({\mathbb{L}}-1)\sum_{j=1}^{q-1}[\widehat{X}_{\gamma(a^j)}]
{\mathop{\textrm{\scriptsize $\mathop\sum$}}}_{m(a^j)|l}{\mathbb{L}}^{-k+l-\tfrac{ls(a^j)}{m(a^j)}}: l<k\\
=&
({\mathbb{L}}-1)\biggl[
{\mathop{\textrm{\scriptsize $\mathop\sum$}}}_{m_f(a^1)|k} {\mathbb{L}}^{-\tfrac{ks(a^1)}{m_f(a^1)}}
\frac{{\mathbb{L}}^{-1}}{1-{\mathbb{L}}^{-1}}
+{\mathop{\textrm{\scriptsize $\mathop\sum$}}}_{j=1}^{q-1}
{\mathop{\textrm{\scriptsize $\mathop\sum$}}}_{(u,v)\in N_k(m_f(a^j),m_f(a^{j+1}))}{\mathbb{L}}^{-us(a^j)-vs(a^{j+1})}\\
&+{\mathop{\textrm{\scriptsize $\mathop\sum$}}}_{m_f(a^q)|k} {\mathbb{L}}^{-\tfrac{ks(a^{q-1})}{m_f(a^{q-1})}}
\frac{{\mathbb{L}}^{-1}}{1-{\mathbb{L}}^{-1}}\biggr]
+\sum_{j=1}^{q-1}(({\mathbb{L}}-1)-[\widehat{X}_{\gamma(a^j)}])
{\mathop{\textrm{\scriptsize $\mathop\sum$}}}_{m(a^j)|k}{\mathbb{L}}^{-\tfrac{ks(a^j)}{m(a^j)}}\\
&+({\mathbb{L}}-1)\sum_{j=1}^{q-1}[\widehat{X}_{\gamma(a^j)}]
{\mathop{\textrm{\scriptsize $\mathop\sum$}}}_{m(a^j)|l}{\mathbb{L}}^{-k+l(1-\tfrac{s(a^j)}{m(a^j)})}: l<k
\\
=&
{\mathop{\textrm{\scriptsize $\mathop\sum$}}}_{m_f(a^1)|k} {\mathbb{L}}^{-\tfrac{ks(a^1)}{m_f(a^1)}}
+({\mathbb{L}}-1){\mathop{\textrm{\scriptsize $\mathop\sum$}}}_{j=1}^{q-1}
{\mathop{\textrm{\scriptsize $\mathop\sum$}}}_{(u,v)\in N_k(m_f(a^j),m_f(a^{j+1}))}{\mathbb{L}}^{-us(a^j)-vs(a^{j+1})}\\
&+{\mathop{\textrm{\scriptsize $\mathop\sum$}}}_{m_f(a^q)|k} {\mathbb{L}}^{-\tfrac{ks(a^{q-1})}{m_f(a^{q-1})}}
+\sum_{j=1}^{q-1}(({\mathbb{L}}-1)-[\widehat{X}_{\gamma(a^j)}])
{\mathop{\textrm{\scriptsize $\mathop\sum$}}}_{m(a^j)|k}{\mathbb{L}}^{-\tfrac{ks(a^j)}{m(a^j)}}\\
&+({\mathbb{L}}-1)\sum_{j=1}^{q-1}[\widehat{X}_{\gamma(a^j)}]
{\mathop{\textrm{\scriptsize $\mathop\sum$}}}_{m(a^j)|l}{\mathbb{L}}^{-k}
{\mathop{\textrm{\scriptsize $\mathop\sum$}}}_{u=1}^{\lfloor\frac{k-1}{m_f(a^j)}\rfloor}
{\mathbb{L}}^{-u h(a^j)}:u=\frac{l}{m_f(a^j)}\\
=&
{\mathop{\textrm{\scriptsize $\mathop\sum$}}}_{m_f(a^1)|k} {\mathbb{L}}^{-\tfrac{ks(a^1)}{m_f(a^1)}}
+({\mathbb{L}}-1){\mathop{\textrm{\scriptsize $\mathop\sum$}}}_{j=1}^{q-1}
{\mathop{\textrm{\scriptsize $\mathop\sum$}}}_{(u,v)\in N_k(m_f(a^j),m_f(a^{j+1}))}{\mathbb{L}}^{-us(a^j)-vs(a^{j+1})}\\
&+{\mathop{\textrm{\scriptsize $\mathop\sum$}}}_{m_f(a^q)|k} {\mathbb{L}}^{-\tfrac{ks(a^{q-1})}{m_f(a^{q-1})}}
+\sum_{j=1}^{q-1}(({\mathbb{L}}-1)-[\widehat{X}_{\gamma(a^j)}])
{\mathop{\textrm{\scriptsize $\mathop\sum$}}}_{m(a^j)|k}{\mathbb{L}}^{-\tfrac{ks(a^j)}{m(a^j)}}\\
&+({\mathbb{L}}-1)\sum_{j=1}^{q-1}[\widehat{X}_{\gamma(a^j)}]
\frac{{\mathbb{L}}^{-k-h(a^j)}(1-{\mathbb{L}}^{-\lfloor\frac{k-1}{m_f(a^j))}\rfloor h(a^j)})}
{1-{\mathbb{L}}^{-h(a^j)}}
\end{align*}
}and we complete the proof.
\end{proof}

\section{Estimate of degrees}\label{estim-deg}

This section is the heart of the paper. We prove a linear bound for
the degree of the virtual Poincar\'e polynomial of the arc spaces. We
show that this bound is sharp. As a consequence, we can recover from
the zeta function of a polynomial germs a \textit{leading tengent} from a
\textit{leading exponent}, denoted by $L_e$. We show that this leading
exponent gives back information about the weights for a weighted
homogeneous polynomial with is non-degenerate and convenient.

Recall that $u$ stands for the Poincar\'e polynomial of the affine line.

\subsection{Leading exponent}\label{Le}
Let $f$ be a non-degenerate polynomial. We keep the notations of
section \ref{Np} concerning the Newton polyhedron of $f$.
We define the leading exponent of $f$ to be
$$
L_e(f)
=\sup\Bigl\{
0,1-\frac{s(a)}{m_f(a)}:a\in\Gamma_+^{(1)}(f)\Bigr\}. 
$$
The sign of the leading exponent will be of major importance in the
sequel. The following statements are direct consequences of the definitions: 
\begin{itemize}
\item If there is $a\in\Gamma^{(1)}_+(f)$ with $h(a)>0$, then $L_e(f)>0$, 
\item If $h(a)\le0$ for any $a\in\Gamma^{(1)}_+(f)$, then $L_e(f)=0$. 
\end{itemize}

We set  
$\Gamma^{(1)}_{\max}(f)
=\bigl\{a\in\Gamma^{(1)}(f):1-\frac{s(a)}{m_f(a)}=L_e(f)\bigr\}$. 

\begin{prop}\label{1111} \
\begin{itemize}
\item If $(1,\dots,1)\not\in\Gamma_+(f)$, then $L_e(f)>0$. 
\item If $(1,\dots,1)\in\Gamma(f)$, then $L_e(f)=0$ 
and $\Gamma^{(1)}_{\max}(f)\ne\emptyset$. 
\item If $(1,\dots,1)\in{\mathop{\mathrm{Int}}\nolimits}\Gamma_+(f)$, then $L_e(f)=0$ 
and $\Gamma^{(1)}_{\max}(f)=\emptyset$. 
\end{itemize}
\end{prop}

\begin{proof}
If $(1,\dots,1)\in\Gamma_+(f)$, then we have $h(a)\le0$ 
for any $a>0$. In fact, we have 
$$
m_f(a)=\min\{\langle a,\nu\rangle:\nu\in\Gamma_+(f)\}
\le\langle a,(1,\dots,1)\rangle=s(a).
$$
Since $\frac{h(a)}{m_f(a)}\le0$, we obtain $L_e(f)=0$. 
In this case $(1,\dots,1)\in\Gamma(f)$ if and only if 
there exist $a\in\Gamma^{(1)}(f)$ with $(1,\dots,1)\in\gamma(a)$, 
which implies $m_f(a)=s(a)$, $h(a)=0$ and 
$\Gamma^{(1)}_{\max}(f)\ne\emptyset$.
So $(1,\dots,1)\in{\mathop{\mathrm{Int}}\nolimits}\Gamma_+(f)$ is equivalent to the fact that 
$h(a)=m_f(a)-s(a)$ is strictly negative for any $a>0$.

If $(1,\dots,1)\not\in\Gamma_+(f)$, 
then $h(a)>0$ for some $a>0$, and we obtain $L_e(f)>0$.
\end{proof}

Set $R_k=\{a=(a_1,\dots,a_n)\in{\mathbb{R}}_{\ge}^n, 0\le m(a)\le k\}$.

\begin{lem}
$L_e(f)=\sup\{h(a):a\in R_1\}$.
\end{lem}
\begin{proof}
We first remark that there exist a finite polyhedral partition $\{P_i\}$ 
of $R_1$ so that $m_f(a)$ is linear on each $P_i$.  
Since $\sup h|_{R_1}=\max_i\{\sup {h|_{P_i}}\}$, it is enough to consider
 $\sup h|_{P_i}$. 
So the supremum of $h(a)$ on $P_i$ is attained by a vertex of $P_i$. 
We remark that multiple of $a\in\Gamma^{(1)}(f)$ with $m_{f}(a)=0$ 
cannot attain the maximum, since $h(a)=-s(a)<0=h(0)$.   
So the possible vertices of $P_i$ are $0$, or $\frac{k}{m_f(a)}a$
 for $a\in\Gamma^{(1)}_+(f)$.
The values of $h$ at these points are 
$0$ or $1-\frac{s(a)}{m_f(a)}$, and we obtain the result. 
\end{proof}

The importance of the leading exponent lies in the fact that it gives rise to a bound for the degree of the
virtual Poincar\'e polynomial of the arc spaces.

\begin{thm}\label{ThDegBnd}
We have the following inequality:
\begin{equation}\label{DegBnd}
\deg[\mathcal A_k(f)]\le n-k+k L_e(f).
\end{equation}
\begin{itemize}
\item If $(1,\dots,1)\not\in{\mathop{\mathrm{Int}}\nolimits}\Gamma_+(f)$,  
then there are arbitrary big $k$ so that 
the equality in \eqref{DegBnd} holds. 
\item 
If $(1,\dots,1)\in{\mathop{\mathrm{Int}}\nolimits}\Gamma_+(f)$, 
then $h(a)<0$ for all $a\in\Gamma^{(1)}_{\max}(f)$ and 
the equality in \eqref{DegBnd}
does not hold. In that case the degree of $[\mathcal A_k(f)]$ is 
$$
n-k+\sup\{h(a):a\in R_k\cap{\mathbb{Z}}^n, a>0\}.
$$
\end{itemize}
\end{thm}

\begin{proof}[Proof of Theorem \ref{ThDegBnd}]
First, note that 
$$
\sup\{h(a):a\in R_k\}=k L_e(f).
$$
Then
\begin{align*}
L_e(f)
=&\sup\Bigl\{0,1-\frac{s(la)}{m_f(la)}:a\in\Gamma_+^{(1)}(f),\ l>0\Bigr\}\\ 
=&\sup\{0,1-s(la):a\in\Gamma_+^{(1)}(f),\ m_f(la)=1\}\\ 
=&\sup\{0,1-s(la):m_f(a)>0,\ m_f(la)=1\}
\end{align*}
by linearity of the function $\frac{s(a)}{m_f(a)}$ on 
$P_i\cap\{a:m_f(a)=1\}$. Finally
$$
L_e(f)
=\sup\Bigl\{0,1-\frac{s(a)}{m_f(a)}:m_f(a)>0\Bigr\}.
$$
The leading monomial of 
$\sum_\gamma((u-1)^n-[X_\gamma])P_k(\gamma)$ 
is attained by {$a^0$} with 
$$
h(\textrm{{$a^0$}})=\sup\{h(a):a_i\ge0,\ m_f(a)=k\}.
$$
Thus we have 
\begin{align*}
\deg{\mathop{\textrm{\scriptsize $\mathop\sum$}}}_\gamma((u-1)^n-[X_\gamma])P_k(\gamma)
=&n+\max\{-s(a):m_f(a)=k\}\\
\le&n+kL_e(f)-k.
\end{align*} 
The leading monomial of $\sum_\gamma [X_\gamma]
Q_k(\gamma)$ is attained by {$a^0$} with 
$$
h(\textrm{{$a^0$}})
=\sup\{h(a):a\ne0,\ a_i\ge0, \ 0<m_f(a)<k,\ X_{\gamma(a)}\ne\emptyset\}.
$$
Thus we have 
\begin{align*}
\deg{\mathop{\textrm{\scriptsize $\mathop\sum$}}}_\gamma[X_\gamma]Q_k(\gamma)
=&-k+\max\{\dim X_{\gamma(a)}+m_f(a)-s(a):0<m_f(a)<k\}\\
\le&n+kL_e(f)-k-1.
\end{align*} 
By Theorem \ref{Thm1}, 
we obtain the result. 
Since 
$$m(ta+(1-t)b)\ge tm_f(a)+(1-t)m(b),\quad 
s(ta+(1-t)b)=ts(a)+(1-t)s(b)
$$
for $t\in[0,1]$, we obtain 
$$
h(ta+(1-t)b)\ge th(a)+(1-t)h(b). 
$$
If $a$ and $b$ attain {the} maximum of $h$ on $R_k$, 
then $h(ta+(1-t)b)$ should be the maximum whenever $ta+(1-t)b\in R_k$
($t\in[0,1]$).
\end{proof}

\begin{rem} We can give a precise description of those $k$ that give
  equality in \eqref{DegBnd} in case
  $(1,\dots,1)\not\in{\mathop{\mathrm{Int}}\nolimits}\Gamma_+(f)$. Actually, it follows from the
  proof of theorem \ref{Thm1} that for 
$a^1, \dots, a^p\in\Gamma^{(1)}_{\max}(f)$ and $b_j>0$ $(j=1,\dots,p)$ so that 
$$
\gamma(a^1)\cap\dots\cap\gamma(a^p)\ne\emptyset \textrm{~~and ~~}
{\mathop{\textrm{\scriptsize $\mathop\sum$}}}_{j=1}^pb_ja^j\in{\mathbb{Z}}^n,
$$
then the equality in \eqref{DegBnd} holds for $k={\mathop{\textrm{\scriptsize $\mathop\sum$}}}_{j=1}^pb_jm_f(a^j)$. 
Conversely, if the equality in \eqref{DegBnd} holds,  
then there exists $a={\mathop{\textrm{\scriptsize $\mathop\sum$}}}_{j=1}^pb_ja^j\in R_k\cap{\mathbb{Z}}^n$ $(a>0)$ with 
above conditions satisfied. 
\end{rem}

We focus now on how we can compute the leading exponent of $f$ from its zeta
function. Define $\alpha_0(f)$ by
$$
\alpha_0(f)=
\sup\Bigl\{
\alpha:\lim_{u\to\infty}\frac{Z_f(u^\alpha t)}{u^n}=0
\Bigr\}.
$$
\begin{prop}\label{propLe}
\begin{itemize}
\item
We have $L_e(f)=1-\alpha_0(f)$. 
\item 
If $\Gamma^{(1)}_{\max}(f)=\{a\}$ with $h(a)>0$, then  
$$
\lim_{u\to\infty}\frac{Z_f(u^{\alpha_0(f)}t)}{u^n}=
\frac{t^{m_f(a)}}{1-t^{m_f(a)}}.
$$
\end{itemize}
\end{prop}
 
\begin{proof}
First we remark that 
$$
\frac{Z_f(u^\alpha t)}{u^n}
=\sum_{k=1}^\infty\frac{\deg[\mathcal A_k(f)]u^{\alpha k}}{u^n}t^k. 
$$
Since $\deg[\mathcal A_k(f)]\le n-k+k L_e(f)$, 
we can write 
$$
\frac{[\mathcal A_k(f)]}{u^n}
=c_ku^{(L_e(f)-1)k}+\textrm{(lower order terms)}
$$
where 
$$
c_k=\#\biggl\{(b_a)_{a\in\Gamma^{(1)}_{\max}(f)}:
a={\mathop{\textrm{\scriptsize $\mathop\sum$}}}_{a\in\Gamma^{(1)}_{\max}(f)}b_aa\in{\mathbb{Z}}^n\cap R_k,\ m_f(a)=k,\ b_a\ge0
\biggr\}. 
$$
This implies that 
$$\frac{[\mathcal A_k(f)]u^{\alpha k}}{u^n}
=c_ku^{(L_e(f)-1+\alpha)k}+\textrm{(lower
order terms,)}$$
so that tending $u$ to $\infty$ implies
$$
\frac{[\mathcal A_k(f)]u^{\alpha k}}{u^n}
\to
\begin{cases}
0& \textrm{if } \alpha<1-L_e(f),\\
c_k& \textrm{if } \alpha=1-L_e(f),\\
\infty& \textrm{if } \alpha>1-L_e(f).
\end{cases}
$$
If $\Gamma^{(1)}_{\max}(f)=\{a\}$ and $h(a)>0$, then $c_k=1$ when $m_f(a)\mid k$. 
We thus have 
$$
\lim_{u\to\infty}\frac{Z_f(u^{\alpha_0(f)}t)}{u^n}
={\mathop{\textrm{\scriptsize $\mathop\sum$}}}_{k=1}^{\infty} t^{k m_f(a)}
=\frac{t^{m_f(a)}}{1-t^{m_f(a)}},
$$
and the proof is achieved. 
\end{proof}

\subsection{Contribution of facets}
Let $\gamma$ be an ($n-1$)-dimensional face of $\Gamma_+(f)$. 
Take $v$ with $\gamma(v)=\gamma$. 
\begin{lem}
Then we obtain 
\begin{align*}
\deg P_k(\gamma)=&
\begin{cases}
-\infty&\textrm{~~if~~}m_f(v)\not|\ k,\\
-k\frac{s(v)}{m_f(v)}&\textrm{~~if~~}m_f(v)\ |\ k
\end{cases}
\end{align*}
and
\begin{align*}
\\
\deg Q_k(\gamma)=&
\begin{cases}
-\infty&\textrm{~~if~~}k\le m_f(v),\\
-k+\bigl\lfloor\frac{k-1}{m_f(v)}\bigr\rfloor (m_f(v)-s(v))&\textrm{~~if~~} k>m_f(v),\quad m_f(v)>s(v),\\
-k+m_f(v)-s(v)&\textrm{~~if~~} k>m_f(v),\quad m_f(v)\le s(v).
\end{cases}
\end{align*}
\end{lem}
\begin{proof}
These are consequences of the following equalities: 
\begin{align*}
\deg P_k(\gamma)=&\max\{-s(a):a=mv\ (m\in {\mathbb{Z}}_+),\ m_f(a)=k\}\\
\deg Q_k(\gamma)=&-k+\max\{m_f(a)-s(a):a=mv\ (m\in{\mathbb{Z}}_+), \ m_f(mv)\le k-1\}\end{align*}
which follow from the previous discussion.  
\end{proof}
\begin{cor}
When $m_f(v)|k$, we have 
\begin{itemize}
\item If $h(v)>0$, then $\deg P_k(\gamma)>\deg Q_k(\gamma)$, 
\item If $h(v)=0$, then $\deg P_k(\gamma)=\deg Q_k(\gamma)$, 
\item If $h(v)<0$, then $\deg P_k(\gamma)<\deg Q_k(\gamma)$. 
\end{itemize}
\end{cor}
\begin{proof}
If $m_f(v)-s(v)>0$ and $m_f(v)|k$, we obtain 
\begin{align*}
\deg P_k(\gamma)-\deg Q_k(\gamma)
=&
-k\frac{s(v)}{m_f(v)}+k-
\Bigl\lfloor\frac{k-1}{m_f(v)}\Bigr\rfloor(m_f(v)-s(v))
\\
=&
\biggl(\frac{k}{m_f(v)}-\Bigl\lfloor\frac{k-1}{m_f(v)}\Bigr\rfloor\biggr)(m_f(v)-s(v))
\\
=&
(m_f(v)-s(v))>0,
\end{align*}
which shows the first statement. 
If $m_f(v)-s(v)\le0$ and $m_f(v)|k$, we obtain 
\begin{align*}
\deg P_k(\gamma)-\deg Q_k(\gamma)
=&
-k\frac{s(v)}{m_f(v)}+k-(m_f(v)-s(v))
\\
=&
\biggl(\frac{k}{m_f(v)}-1\biggr)(m_f(v)-s(v)),
\end{align*}
and this implies the remaining statement.
\end{proof}
A contribution of facet to $Z_f(t)$ can be seen as follows. 
\begin{lem}\label{lemfacet}
We have 
$$
\lim_{u\to1}
\frac{[\mathcal A_k(f)]}{u-1}=-{\mathop{\textrm{\scriptsize $\mathop\sum$}}}_{a>0, ~m(a)=k, \ \dim\gamma(a)=n-1}[X_{\gamma(a)}]_{u=1}. 
$$
\end{lem}
We prepare some notations for the proof of lemma \ref{lemfacet}.
For $I\subset\{1,2,\dots,n\}$, 
we set 
$$
f_I(x)={\mathop{\textrm{\scriptsize $\mathop\sum$}}}_{\nu:~\nu_i=0, \ i\not\in I}c_\nu x^\nu .
$$
For $a=(a_1,\dots,a_n)$, 
we define 
\begin{align*}
m_I(a)=&\min\{\langle a,\nu\rangle :\nu\in\Gamma_+(f_{I(a)})\},\\
\gamma_I(a)=&\{\nu\in\Gamma_+(f_{I(a)}):\langle a,\nu\rangle=m_I(a)\}.
\end{align*}
We remark that $\mathcal L_a\cap\mathcal A_k(f)=
\mathcal L_a\cap\mathcal A_k(f_{I(a)})$. 
Since $[\mathcal L_{I(a)}]=0$ for $I\subsetneq \{1,\dots,n\}$ by
example \ref{ex1}, 
we obtain
$$
[\mathcal A_k(f)]={\mathop{\textrm{\scriptsize $\mathop\sum$}}}_{a:~I(a)=\{1,\dots,n\}}
[\mathcal L_a\cap\mathcal A_k(f)]. 
$$
\begin{proof}
We decompose the set of arcs $\mathcal A_k(f)$ with respect to the
value under $s$ of the order $a$ of the arcs. By additivity of the
virtual Poincar\'e polynomial we obtain
\begin{align*}
&[p_m(\mathcal A_k(f))]=
{\mathop{\textrm{\scriptsize $\mathop\sum$}}}_{a:~s(a)\le m}
[p_m(\mathcal L_a\cap\mathcal A_k(f_{I(a)}))]
\\
&=
{\mathop{\textrm{\scriptsize $\mathop\sum$}}}_{a:~s(a)\le m,~m_I(a)=k}
(u-1)^{|I(a)|-\dim\gamma_I(a)}((u-1)^{\dim\gamma_I(a)}
-[\widehat{X}_{\gamma_I(a)}])u^{|I(a)|m-s(a)}\\
&\qquad+
{\mathop{\textrm{\scriptsize $\mathop\sum$}}}_{a:~s(a)\le m,~m_I(a)<k}
(u-1)^{|I(a)|-\dim\gamma_I(a)+1}
[\widehat{X}_{\gamma_I(a)}]u^{|I(a)|m-s(a)-k+m_I(a)}
\end{align*}
thus we obtain 
\begin{align*}
&\lim_{m\to\infty}
\frac{[p_m(\mathcal A_k(f))]}{u^{mn}(u-1)}
\\
=&
{\mathop{\textrm{\scriptsize $\mathop\sum$}}}_{a:~a>0,~m_f(a)=k}
(u-1)^{n-\dim\gamma(a)-1}((u-1)^{\dim\gamma(a)}
-[\widehat{X}_{\gamma(a)}])u^{-s(a)}\\
&+
{\mathop{\textrm{\scriptsize $\mathop\sum$}}}_{a:~a>0,~m_f(a)<k}
(u-1)^{n-\dim\gamma(a)}
[\widehat{X}_{\gamma(a)}]u^{-s(a)-k+m_f(a)}.
\end{align*}
We obtain the result tending $u$ to $1$. 
\end{proof}

\subsection{Weighted homogeneous polynomials}\label{weight}

Let $f\in {\mathbb{R}}[x_1,\ldots,x_n]$ be a weighted homogeneous polynomial, namely there exists weights
$w_1,\ldots,w_n\in {\mathbb{N}}$ relatively prime and a weighted degree $d\in
{\mathbb{N}}$ such that $f(x_1^{w_1},\ldots,x_n^{w_n})$ is homogeneous of
degree $d$. 

In that case $\Gamma_+ (f)$ has a unique facet $\gamma_f$,
with the associated $1$-cone generated by the primitive vector $v={\mathop{\mathrm{lcm}}\nolimits}
(w_1,\ldots,w_n)(\frac{1}{w_1},\ldots, \frac{1}{w_n})$, and then 
$$m_f(v)={\mathop{\mathrm{lcm}}\nolimits} (w_1,\ldots,w_n)$$
and 
$$h(v)={\mathop{\mathrm{lcm}}\nolimits}
(w_1,\ldots,w_n)(1-\sum_{i=1}^n \frac{1}{w_i}).$$ 
Moreover 
$$L_e(f)=\sup
\{0, 1-\sum_{i=1}^n \frac{1}{w_i})\}.$$
In particular, if we
are able to compute $h(v)$ and $m_f(v)$ from the zeta function, then we can recover the sum of
the inverse of the weights of $f$.

Let us assume that the Newton polyhedron of $f$ is
convenient, that is the monomials $x_i^{\frac{d}{w_i}}$ do appear in
the expression $f$ with non-zero coefficient, and that $f$ is nondegenerate with
respect to its Newton polyhedron. 

By proposition \ref{propLe} we
recover $L_e(f)$ from the zeta function of
$f$, and recover even $m_f(v)$ if $h(v)>0$. We
focus now on recovering $h(v)$. Next lemma is a direct consequence of proposition \ref{propLe}.

\begin{lem} $L_e(f)>0$ if and only if $h(v)>0$, and more precisely $h(v)=m_f(v)L_e(f)$.
\end{lem}
  
In case $L_e(f)=0$, the situation is more difficult to handle. Note
that generically, the degree of $[\mathcal A_k(f)]$ is given by $n+\max \{\deg
P_k(\gamma_f),\deg Q_k(\gamma_f)\}$ --it may be different if some
$X_{\gamma}$ are empty). The degree of $P_k(\gamma_f)$ and
$P_k(\gamma_f)$ may be express as
\begin{align*}
\deg P_k(\gamma)
=&\max \{-s(a): m_f(a)=k\} \\ 
=&\max \{-k+m_f(a)-s(a):
m_f(a)=k\}\\
=&-k+\max \{h(a):m_f(a)=k\}
\end{align*}
whereas 
\begin{align*}
\deg Q_k(\gamma)
=&\max \{-k+m_f(a)-s(a):m_f(a)<k\}\\ 
=&-k+\max \{h(a):m_f(a)<k\}.
\end{align*}
Therefore we are lead to understand the levels of the
function $h=m_f-s$ on $\mathbb N^n$, and more precisely on the subsets
of $\mathbb N^n$ defined by $\{m_f(a)=k\}$ and $\{m_f(a)<k\}$.

 
To begin with, let us forget that we are interested in integral points
and describe its
levels on $\{a\in \mathbb R^n, a_i \geq 0 \}$. The function $h$ is linear on each
cone of $\Gamma^*(f)$, therefore its levels are completely described
by its value on $v$ and on the canonical basis $\{e_1,\ldots,e_n\}$ of
$\mathbb R^n$. Note that $h(e_i)=m_f(e_i)-1\geq -1$, with equality in case the
Newton polyhedron is convenient.

In particular, 
\begin{itemize}
\item if $h(v)=0$ there are only negative levels
that are cylinder parallel to
the line generated by $v$, 
\item if $h(v)<0$ they are (bounded) simplices with a
vertex on the line generated by $v$ and other vertices on the
positive coordinates axis, whereas 
\item if $h(v)>0$, they are unbounded simplices with a
vertex on the line generated by $v$.
\end{itemize}
Coming back to the computation of the degree of  $P_k(\gamma)$ and $
Q_k(\gamma)$, we need to investigate the integral points on these
levels.

\begin{lem} Assume $L_e(f)=0$ and $[X_{\gamma_f}]\neq 0$. 
\begin{itemize}
\item $\deg [\mathcal A_k(f)]\leq n-k$ with equality 
for $k$ big enough if and only if $h(v)=0$,
\item $\deg [\mathcal A_k(f)]<n-k$ if and only if $h(v)<0$. In that
  case $\deg [\mathcal A_k(f)]=n-k+\textrm{{$\max$}}\{h(a):a>0,~[X_{\gamma
    (a)}]\neq 0 \}$.
\end{itemize}
\end{lem}

In particular in the case $h(v)<0$, we can only recover the sign of
$h(v)$, but not its value.

\begin{proof} It suffices to compute the maximum of $h$ on
  $\{m_f(a)=k\}$ and on $\{m_f(a)<k\}$. In the convenient case the
  levels of $m_f$ are given by translation of the positive part
  of the hyperplane
  coordinate along the line generated by $v$. Then if $h(v)=0$
  the maximum on $\{m_f(a)=k\}$ as a real number is attained on the
  line generated by $v$. In particular, if $k>m_f(v)$, this maximum is
  attained in $v$.

In case $h(v)<0$, the levels of $h$ decreases along the line generated
by $v$, therefore for $k>m_f(v)$, this maximum is bigger than $h(v)$
and strictly negative.
\end{proof}

\begin{rem} It may happen that $\max \{h(a):a>0\}$ is stricly bigger
  then $h(v)$ in the case $h(v)<0$. Consider for example
  $f(x_1,x_2,x_3)=x_1^2+x_2^2+x_3^m$ with $p$ odd. A direct
  computation gives that $v=(2,2,m)$
  and $h(v)=-2$ whereas $h(1,1,1)=-1$.
\end{rem}

Assume $L_e(f)=0$ and $[X_{\gamma_f}]= 0$. Then  $[X_{\gamma}]= 0$ for
all face $\gamma$ of $\Gamma (f)$, therefore the degree of $[\mathcal A_k(f)]$ is only
given by $P_k(\gamma)$ and the discussion is simplified. 

\begin{lem} Assume $L_e(f)=0$ and $[X_{\gamma_f}]= 0$. 
\begin{itemize}
\item $\deg [\mathcal A_k(f)]\leq n-k$ with equality for infinitely
  many $k$ if and only if $h(v)=0$,
\item $\deg [\mathcal A_k(f)]<n-k$ if and only if $h(v)<0$.
\end{itemize}
\end{lem}

\begin{proof} As a consequence of the proof of theorem \ref{Thm1}, we
  have
$$\deg P_k(\gamma)=\max\{-s(a): m_f(a)=k\} \leq
-s(\frac{k}{m_f(v)}v),$$
with equality when $m_f(v)$ divides $k$. Now if $h(v)=0$ then
$s(\frac{k}{m_f(v)}v)=k$ whereas $s(\frac{k}{m_f(v)}v)>k$ if $h(v)<0$.
\end{proof}

Therefore we are able to recognize the sign of $h(v)$ from the zeta
function of $f$, and its value if it is positive or zero.

\begin{prop}\label{signh} Assume $f$ is a convenient weighted homogeneous
  polynomial non degenerate with respect to its Newton
  polyhedron. Denote by $v$ the primitive vector associated to $f$.
\begin{itemize}
\item $h(v)>0$ if and only if $L_e(f)>0$, and more precisely $h(v)=m_f(v)L_e(f)$.
\item $h(v)=0$ if and only if $\deg [\mathcal A_k(f)]\leq n-k$, with equality for infinitely
  many $k$.
\item $h(v)<0$ if and only if $\deg [\mathcal A_k(f)]<n-k$.
\end{itemize}
\end{prop}

\begin{rem}\label{2-case} In \cite{Ould}, it is shown that the weights of two
  variables non-degenerate weighted homogeneous polynomial are
  invariants under blow-analytic equivalence, using the zeta function
  defined with the Euler characteristic with compact support. Because
  of the properties of the virtual Poincar\'e polynomials, we can
  recover easily the same result, in the setting of blow-Nash
  equivalence. Actually the first exponent of the zeta function
  combined with the leading exponent $L_e(f)$ give the weights.
\end{rem}

\section{Recovering the weights}\label{reco}

We prove that we can recover the weights of a convenient non-degenerate weighted
homogeneous polynomials in three variables. Similarly to the two
variables case (cf. remark \ref{2-case}), it easy to recover the multiplicity whereas the
inverse of the sum of the weights is obtained by proposition
\ref{signh}. We prove below that in the three variables case, we are
able to recover the ultimate weight.

Let $f(x_1,x_2,x_3)$ be a weighted homogeneous polynomial whose Newton polyhedron is convenient.
Let $(p_1,0,0)$, $(0,p_2,0)$ and $(0,0,p_3)$ denote the vetices  of
$\Gamma_+(f)$. Assume $p_1 \leq p_2 \leq p_3$ without lost of
generality.

Let $\gamma$ denote the compact 2-dimensional face of $\Gamma_+(f)$.
Set $\gamma_i=\gamma\cap\{\nu_i=0\}$. 
Set $p_{ij}={\mathop{\mathrm{LCM}}\nolimits}(p_i,p_j)$ and $p_{123}={\mathop{\mathrm{LCM}}\nolimits}(p_1,p_2,p_3)$. As a
consequence of theorem \ref{Thm1} and remark \ref{closed}, we can
describe completely the zeta function of $f$.
\begin{prop}
$$
\frac{[\mathcal A_k(f)]}{u-1}
=
\frac{P_k}
{u^{
\lfloor\frac{k}{p_1}\rfloor
+\lfloor\frac{k}{p_2}\rfloor
+\lfloor\frac{k}{p_3}\rfloor}
}
+
{\mathop{\textrm{\scriptsize $\mathop\sum$}}}_{1\le l<k}
\frac{(u-1)Q_l}
{u^{k-l
+\lfloor\frac{l}{p_1}\rfloor
+\lfloor\frac{l}{p_2}\rfloor
+\lfloor\frac{l}{p_3}\rfloor
}}
$$

where 
\begin{align*}
P_k=&
\begin{cases}
0&p_i\nmid k\ (i=1,2,3)\\
1&p_i|k,\ p_{ij}\nmid k\ (i\ne j)\\
1+u-[\widehat{X}_{\gamma_s}]&p_{ij}\mid k, \ p_{123} \nmid k, \ \{i,j,s\}=\{1,2,3\}\\
1+u+u^2-[\overline{\widehat{X}_{\gamma}}]&p_{123}\mid k
\end{cases}
\end{align*}
and
\begin{align*}
Q_l=&
\begin{cases}
0 & p_i\nmid l\ (i=1,2,3) \textrm{ or }p_i\mid l,\ p_{ij}\nmid l\ (i\ne j)\\
[\widehat{X}_{\gamma_s}]&p_{ij}\mid l, \ p_{123} \nmid l, \ \{i,j,s\}=\{1,2,3\}\\
[\widehat{X}_{\gamma_1}]
+[\widehat{X}_{\gamma_2}]
+[\widehat{X}_{\gamma_3}]
+[\widehat{X}_{\gamma}]u^{-1}&p_{123}\mid l .
\end{cases}
\end{align*}
\end{prop}

\begin{cor}
the following inequality holds:
$$
\deg[\mathcal A_k(f)]\le3-\frac{k}{p_1}-\frac{k}{p_2}-\frac{k}{p_3}.
$$
\end{cor}

\begin{rem} Note in particular that
\begin{itemize}
\item  
if $p_1<p_2\le p_3$, we have $[\mathcal A_{p_1}(f)]=\frac{u-1}{u}$
\item
if $p_1=p_2<p_3$, we have 
$[\mathcal A_{p_1}(f)]=\frac{(u-1)(1+u-[\widehat{X}_{\gamma_3}])}{u^2}$. 
\item 
if $p_1=p_2=p_3$, we have 
$[\mathcal A_{p_1}(f)]
=\frac{(u-1)(1+u+u^2-[\overline{\widehat{X}_{\gamma}}])}{u^3}$. 
\end{itemize}
\end{rem}

However, in order to recover the weights, it will be enough to
concentrate the study on some specific part of the zeta function. Actually, it is enough to recover the integers $p_1,p_2$ and $p_3$ from
the zeta function of $f$. Note that we already recover the
multiplicity of $f$, that is $p_1$, as the order of
the zeta function. Moreover we know from proposition \ref{signh} how
to recover the sign of $h(v)$. In the particular case of $h(v)<0$ the function
$f$ has only \textit{simple} singularities in the sense of Arnold \cite{Arnold}
and we already know how to recover the
weights from \cite{simple}. In the general situation, if $h(v)\geq 0$ we obtain $L_e(f)$ by
proposition \ref{propLe} which is equal to
$1-\frac{1}{p_1}-\frac{1}{p_2}-\frac{1}{p_3}$. It is therefore sufficient
to find $p_2$ in order to recover all the weights.

The idea is to recover $p_2$ in the zeta function as the first
contribution that does not come from the smallest face of the Newton
polyhedron $(p_1,0,0)$. Denote by $A(f)$ the set of Fukui invariants $A(f)=\{k:[\mathcal
A_l]\neq 0 \}$.

We treat the cases $p_1$ even and $p_1$ odd separately.
In case $p_1$ is odd, note that $[X_{\gamma_3}]\neq 0$, and
therefore $A(f)\cap \mathbb N_{\geq p_2}=
\mathbb N_{\geq p_2} $. 

Set 
$$\alpha=\min \{l\in \mathbb N:A(f)\cap \mathbb N_{\geq l}=
\mathbb N_{\geq l}\},$$
$$\beta=\min \{l\in \mathbb N: [\mathcal A_l]
\neq 0,~~p_1 \nmid l\}.$$
As $p_1$ is odd, note that $\alpha \leq p_{12}$.
\begin{lem}
Assume $p_1$ is odd. 
\begin{itemize}
\item If $p_1 \nmid \beta -1$, then $p_2=\beta$.
\item If $p_1 \mid \beta -1$ and $\beta -1< \alpha$, then
  $p_2=\beta$.
\item  If $p_1 \mid \beta -1$ and $\beta -1= \alpha$, then either
$p_2=\beta -1$ or $p_2=p_3=\beta$.
\end{itemize}
\end{lem}

\begin{proof} If $p_1$ divides $p_2$, then $\alpha=p_2$ et $\beta=\alpha +1$, so that
  $p_1$ divides $\beta-1$. As $\beta=p_2$ if $p_1$ does not divide
  $p_2$, we obtain the first point.

Now, if $p_1$ divides $\beta-1$, then either $p_1$ divides $p_2=\beta-1$ and
$\alpha=p_2$ or  $p_1$ divides $p_2-1$ and $p_2=\beta$.
\end{proof}

In particular, if $p_1 \mid \beta -1$ and $\beta = \alpha +1$, we obtain
two possibilities for the value $p_2$. We show below that only
one of these possibilities gives the correct value for the sum
$\frac{1}{p_1}+\frac{1}{p_2}+\frac{1}{p_3}$, except in one particular
case that we need to treat separately.

\begin{lem}
Assume $p_1 \mid \beta -1$ and $\beta = \alpha +1$. Assume moreover
that the value of $\frac{1}{p_1}+\frac{1}{p_2}+\frac{1}{p_3}$ is
given. Then we can decide whether
$p_2=\alpha$ or $p_2=\beta$, except in the cases $(p_1,p_2,p_3)=(3,4,4)$ or $(3,3,6)$.
\end{lem}

\begin{proof}
Assume  $(p_2,p_3)=(\beta,\beta)$
and $(p'_2,p'_3)=(\alpha,l)$ with $l\geq \alpha$ satisfing
$$\frac{1}{p_2}+\frac{1}{p_3}=\frac{1}{p'_2}+\frac{1}{p'_3}.$$
Then $l=\frac{\alpha(\alpha+1)}{\alpha-1}$ should be integer therefore
$\alpha=3$ and $l=6$. 
\end{proof}

In case $p_1$ is even, it may arrive that $\alpha=\infty$ (if
$[X_{\gamma_f}]=0$), and even $\beta=\infty$ (if $p_1$ divides $p_2$
and $p_3$). Therefore we need to take care also about the coefficients of
the zeta function. 

Set
$$\delta=\min \{l: -1 \textrm{ is a root of } [\mathcal A_l] \}.$$ 
\begin{lem}
Assume $p_1$ is even. Then $p_2=\min \{\alpha,\beta,\delta\}$ except if
$\alpha = \beta -1$ and $[\mathcal
A_{\alpha}]=u ^{-\frac{\alpha}{p_1}}(u-1)$. In that case $p_2=\beta$.
\end{lem}

\begin{proof}
If $\alpha =\infty$, then $[\mathcal A_{p_2}]=$ if $p_2<p_3$ whereas
$[\mathcal A_{p_2}]=$ in case $p_2=p_3$. Therefore $p_2=\delta \leq \beta$.

If $p_1$ does not divide $p_2$ and $p_2<p_3$, then $p_2=\beta
\leq \delta$. We claim that necessarily $\beta \leq
\alpha$. Actually if $\alpha<\beta$ then $\alpha=\beta-1$ and so $p_1$
divides $p_2-1$. In order to obtain $p_2+1$ and $p_2+2$ in $A(f)\cap
\mathbb N_{\geq \alpha}$, we then have $p_1=2$ and $p_3=p_2+2$. But in
that case $(p_1,p_2,p_3)=(2,3,5)$ and $h(v)<0$.

If $p_1$ does not divide $p_2$ and $p_2=p_3$, then
$p_2=\beta \leq \min \{\alpha,\delta\}$ unless $p_1$ divides $p_2-1$ and
$[X_{\gamma_1}]\neq 0$ (and then $\beta=\alpha+1\leq \delta$). In that case $[\mathcal
A_{\alpha}]=u ^{-\frac{\alpha}{p_1}}(u-1)$.

If $p_1$ divides $p_2$, assume first that $p_2<p
_3$. Then $p_2=\delta \leq \min \{\alpha, \beta\}$ if
$[X_{\gamma_3}]= 0$ whereas $p_2=\alpha \leq \min\{\beta, \delta\}$ if
$[X_{\gamma_3}]\neq 0$. In the latter case, note that $[\mathcal
A_{\alpha}]=u ^{-\frac{\alpha}{p_1}-1}(u^2-1-(u-1)[\widehat
X_{\gamma_3}])$ with $[\widehat X_{\gamma_3}]$ even (indeed $[\widehat
X_{\gamma_3}]$ is the number of real solutions of a real
polynomial of even degree not vanishing at zero, with only simple real
roots because $f$ has isolated singularities). 

Finally in the case $p_1$ divides $p_2=p_3$, then $p_2=\alpha \leq \{\beta,
\delta\}$ if $\alpha \neq \infty$.

\end{proof}

\begin{thm}\label{w3}  Convenient weighted
  homogeneous polynomials which share the same zeta functions have the
  same weights.
\end{thm}

\begin{proof} If $h(v)<0$ we refer to \cite{simple}. Otherwise, by the preceeding lemmas and by proposition \ref{propLe}
  we know how to recover $p_1$, $p_2$ and
  $\frac{1}{p_1}+\frac{1}{p_2}+\frac{1}{p_3}$ except in the particular
  case where $p_1=3$ and
  $\frac{1}{p_1}+\frac{1}{p_2}+\frac{1}{p_3}=\frac{5}{6}$. Therefore
  it suffices to be able to distinguish the cases $(p_1,p_2,p_3)=(3,4,4)$ and
  $(3,3,6)$. 

A direct computation shows that $[\mathcal A_{4}]=u ^{-3}(u-1)^2$
if $(p_1,p_2,p_3)=(3,3,6)$ whereas  $[\mathcal A_{4}]=u ^{-3}(u^2-1-(u-1)[\widehat X_{\gamma_1}])$
if $(p_1,p_2,p_3)=(3,4,4)$, so that the spaces of arcs of level $4$ are different, except when
$[\widehat X_{\gamma_1}]=2$. However $[\mathcal A_{5}]=u ^{-4}(u-1)^2$ if
  $(p_1,p_2,p_3)=(3,3,6)$ whereas $[\mathcal A_{5}]=u
  ^{-4}(u-1)[\widehat X_{\gamma_1}]$ if $(p_1,p_2,p_3)=(3,4,4)$,
  so at the level $5$ the spaces of arcs are different in that case.
\end{proof}

\begin{thebibliography}{99}

\bibitem{Ould}
O. M.~Abderrahmane,
Weighted homogeneous polynomials and blow-analytic equivalence,
Singularity theory and its applications, 333--345, 
Adv.~Stud.~Pure Math., 43, Math.~Soc.~Japan, Tokyo, 2006.

\bibitem{Arnold} V. Arnold, S. Goussein-Zad\'e, A. Varchenko,
Singularity of differentiable maps, 
Birkhauser, Boston, 1985

\bibitem{DK} J. Denef, K. Hoornaert, 
Newton polyhedra and Igusa's local zeta function,
J. Number Theory  89  (2001),  no. 1, 31--64

\bibitem{DL} J. Denef, F. Loeser,
Germs of arcs on singular algebraic varieties and motivic integration,
Invent. Math.  135  (1999),  no. 1, 201--232

\bibitem{DL-modif} J. Denef, F. Loeser,
Caract\'eristiques d'Euler-Poincar\'e, fonctions z\^eta locales et
modifications analytiques,
J. Amer. Math. Soc. 5 (1995), 705--720

\bibitem{Fichou} G. Fichou,
Motivic invariants of Arc-Symmetric sets and Blow-Nash Equivalence, 
Compositio Math. 141 (2005) 655--688 

\bibitem{simple} G. Fichou,
Blow-Nash type of simple singularities,
J. Math. Soc. Japan, 60 no. 2 (2008), 445-470 

\bibitem{Fukui} T. Fukui, 
Seeking invariants for blow-analytic equivalence, 
Compositio Math.  105  (1997),  no. 1, 95--108

 \bibitem{FKK} T. Fukui, S. Koike, T.-C. Kuo, 
Blow-analytic equisingularities, properties, problems and progress, 
Real Analytic and Algebraic Singularities (T. Fukuda, T. Fukui,
S. Izumiya and S. Koike, ed), Pitman Research Notes in Mathematics Series, 381 (1998), pp. 8-29

\bibitem{FP} T. Fukui, L. Paunescu, 
On blow-analytic equivalence, 
Arc spaces and additive invariants in real algebraic geometry,
Panoramas et Synth\`eses, SMF 26 (2008) 87-125

\bibitem{Guibert} G. Guibert, 
Espaces d'arcs et invariants d'Alexander, 
Comment. Math. Helv.  77  (2002),  no. 4, 783--820

\bibitem{KP} S. Koike, A. Parusi\'nski, 
Motivic-type invariants of blow-analytic equivalence,  
Ann. Inst. Fourier (Grenoble)  53  (2003),  no. 7, 2061--2104

\bibitem{Kuo} T.-C. Kuo, 
On classification of real singularities,
Invent. Math. 82 (1985), 257-262

\bibitem{MCP} C. McCrory, A. Parusi\'nski,
Virtual Betti numbers of real algebraic varieties,  
C. R. Math. Acad. Sci. Paris  336  (2003),  no. 9, 763--768

\bibitem{Ni} T. Nishimura,
Topological invariance of weights for weighted homogeneous
singularities, 
Kodai Math. J.  9  (1986),  no. 2, 188--190

\bibitem{Saeki} O. Saeki,
Topological Invariance of Weights for Weighted Homogeneous Isolated Singularities in $C^3$ 
Proceedings of the American Mathematical Society, Vol. 103, No. 3 (Jul., 1988), 905--909 

\bibitem{Saito} K. Saito, 
Quasihomogene isolierte Singularitaten von Hyperflachen,
Invent. Math.  14  (1971), 123--142

\bibitem{Saito87}
K. Saito,
Regular system of weights and associated singularities,
Complex analytic singularities,
479--526, Adv. Stud. Pure Math., 8, North-Holland, Amsterdam, 1987.

\bibitem{XuYau}
Y.-J. Xu, S. S.-T.~Yau, 
Classification of topological types of isolated quasi-homogeneous 
two-dimensional hypersurface singularities, 
Manuscripta Math. 64 (1989), 445--469. 

\bibitem{YoSu} E. Yoshinaga, M. Suzuki,
Topological types of quasihomogeneous singularities in $C^{2}$,
Topology  18  (1979), no. 2, 113--116

\end{thebibliography}
\end{document}


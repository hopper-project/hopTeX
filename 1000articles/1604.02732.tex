\documentclass[a4paper,11pt,reqno]{amsart}
\usepackage[english]{babel}
\usepackage{color,amsmath, amssymb, latexsym,mathptmx, amsfonts}

\usepackage[matrix,arrow,curve,frame]{xy}    

\xymatrixcolsep{1.9pc}                          
\xymatrixrowsep{1.9pc}
\newdir{ >}{{}*!/-5pt/\dir{>}}                  

\addtolength{\textwidth}{2.3cm} \calclayout

\footskip30pt

\newtheorem{theorem}{Theorem}[section]

\newtheorem*{theorem*}{Theorem~\ref{ZM_fr_and_LM_fr}}

\newtheorem{lemma}[theorem]{Lemma}
\newtheorem{proposition}[theorem]{Proposition}
\newtheorem{corollary}[theorem]{Corollary}
\newtheorem{definition}[theorem]{Definition}
\newtheorem{notation}[theorem]{Notation}
\newtheorem{remark}[theorem]{Remark}

\begin{document}
\sloppy

\title{Framed motives of relative motivic spheres}

\author{Grigory Garkusha}
\address{Department of Mathematics, Swansea University, Singleton Park, Swansea SA2 8PP, United Kingdom}
\email{g.garkusha@swansea.ac.uk}

\author{Alexander Neshitov}
\address{Department of Mathematics and Statistics, 585 King Edward Avenue, Ottawa ON K1N 6N5, Canada}
\email{alexander.neshitov@gmail.com}

\author{Ivan Panin}
\address{St. Petersburg Branch of V. A. Steklov Mathematical Institute,
Fontanka 27, 191023 St. Petersburg, Russia}

\address{St. Petersburg State University, Department of Mathematics and Mechanics, Universitetsky prospekt, 28, 198504,
Peterhof, St. Petersburg, Russia}

\email{paniniv@gmail.com}

\thanks{This paper was partly written during the visit of the second author to
Swansea University. He would like to thank the University for the
kind hospitality. The third author thanks for the support the
Russian Science Foundation (grant no. 14-21-00035).}

\keywords{Motivic homotopy theory, framed motives, motivic spheres}

\subjclass[2010]{14F42, 55P42}

\begin{abstract}
The category of framed correspondences ${\operatorname{Fr}}_*(k)$, framed presheaves
and framed sheaves were invented by Voevodsky in his unpublished
notes~\cite{V2}. Based on the theory, framed motives are introduced
and studied in~\cite{GP1}. The aim of this paper is to prove the
following results stated in~\cite[9.3]{GP1}: for any $k$-smooth
scheme $X$ and any $n{\geqslant} 1$ the map of simplicial pointed sheaves
$(-,{\mathbb{A}}^1\rfloor\mathbb G_m)^{\wedge n}_+\to T^n$ induces a Nisnevich
local level weak equivalence of $S^1$-spectra
   $$M_{fr}(X\times ({\mathbb{A}}^1\rfloor \mathbb G_m)^{\wedge n})\to M_{fr}(X\times T^n)$$
and the sequence of $S^1$-spectra
   $$M_{fr}(X \times T^n \times \mathbb G_m) \to M_{fr}(X \times T^n \times{\mathbb} A^1) \to M_{fr}(X \times T^{n+1})$$
is locally a homotopy cofiber sequence in the Nisnevich topology.
\end{abstract}

\maketitle

\thispagestyle{empty} \pagestyle{plain}

\tableofcontents

\section{Introduction}

Based on Voevodsky's theory of framed correspondences~\cite{V2}, the
machinery of framed motives has been developed in~\cite{GP1}. One of
the main purposes of the machinery is to compute an explicit fibrant
resolution of the suspension ${\mathbb} P^1$-spectrum $\Sigma^\infty_{{\mathbb}
P^1}X_+$ of a $k$-smooth algebraic variety $X\in
Sm/k$~\cite[9.5]{GP1}.

As it is shown in~\cite{GP1}, the category of framed correspondences
of level zero ${\operatorname{Fr}}_0(k)$ has an action by pointed sets $X\otimes
K:=\bigsqcup_{K\setminus *}X$ with $X\in Sm/k$ and $K$ a pointed
set. The cone of $X$ is the simplicial object $X\otimes I$ in
${\operatorname{Fr}}_0(k)$, where $(I,1)$ is the pointed simplicial set $\Delta[1]$
with basepoint 1. There is a natural morphism $i_0:X\to X\otimes I$
in $\Delta^{\textrm{op}}{\operatorname{Fr}}_0(k)$. Given an inclusion of smooth
schemes $X\hookrightarrow Y$, denote by $Y\rfloor X$ a simplicial
object in ${\operatorname{Fr}}_0(k)$ which is obtained from the pushout in
$\Delta^{\textrm{op}}{\operatorname{Fr}}_0(k)$ of the diagram
   $$Y\hookleftarrow X{\buildrel {i_0}\over}\hookrightarrow X\otimes I.$$
We can think of $Y\rfloor X$ as a cone of the inclusion
$X\hookrightarrow Y$. It is the underlying simplicial scheme for the
ordinary pointed motivic space $(-,Y\rfloor X)_+$. Proceeding
inductively in this way, we define a $n$-multisimplicial scheme
$(Y\rfloor X)^{\wedge n}$, which is also the successive cone of the
standard $n$-dimensional cube $(X\hookrightarrow Y)^{\times n}$ in
${\operatorname{Fr}}_0(k)$. The diagonal of $(Y\rfloor X)^{\wedge n}$ is the
underlying simplicial scheme for the motivic space $(-,Y\rfloor
X)_+^{\wedge n}:=(-,Y\rfloor X)_+\wedge{\buildrel n\over}\cdots\wedge(-,Y\rfloor
X)_+$.

The computation of an explicit fibrant resolution of
$\Sigma^\infty_{{\mathbb} P^1}X_+$ requires the following theorem (we
refer the reader to~\cite[Def.~9.1]{GP1} for the relevant
definitions of framed motives of simplicial Nisnevich sheaves).

\begin{theorem}\label{cone}
Let $k$ be an infinite perfect field. For any $k$-smooth scheme
$X\in Sm/k$ and any $n{\geqslant} 1$, the map of simplicial pointed sheaves
$(-,{\mathbb{A}}^1\rfloor\mathbb G_m)^{\wedge n}_+\to T^n$ induces a level
Nisnevich local weak equivalence of $S^1$-spectra
   $$M_{fr}(X\times ({\mathbb{A}}^1\rfloor \mathbb G_m)^{\wedge n})\to M_{fr}(X\times T^n).$$
Moreover, the sequence of $S^1$-spectra
   $$M_{fr}(X \times T^n \times \mathbb G_m) \to M_{fr}(X \times T^n \times{\mathbb} A^1) \to M_{fr}(X \times T^{n+1})$$
is locally a homotopy cofiber sequence in the Nisnevich topology.
\end{theorem}

The main goal of the paper is to prove this theorem. In particular,
it computes locally the stable homotopy type of the framed motive
$M_{fr}(X\times T^n)$ of the relative motivic sphere $X_+\wedge
T^n=X\times {\mathbb} A^n/(X\times {\mathbb} A^n-X\times 0)$ as the framed
motive $M_{fr}(X\times ({\mathbb{A}}^1\rfloor \mathbb G_m)^{\wedge n})$ of the
multisimplicial scheme $X\times({\mathbb{A}}^1\rfloor\mathbb G_m)^{\wedge n}$.
Another consequence of the theorem says that for every $n{\geqslant} 0$ the
natural morphism
   \begin{equation}\label{eq:main_l_w_equivalence}
   M_{fr}(X\times T^n\times({\mathbb{A}}^1\rfloor \mathbb G_m))\to M_{fr}(X\times T^{n+1})
   \end{equation}
is locally a level weak equivalence of $S^1$-spectra in the
Nisnevich topology (see Corollary~\ref{cormain}). We should stress
that the proof of~\cite[9.5]{GP1} depends on the local
equivalence~\eqref{eq:main_l_w_equivalence}.

It is worth to mention that the framed motive functor defined
in~\cite[Def.~9.1]{GP1} converts motivic equivalences between
pointed projectively cofibrant motivic spaces to Nisnevich local
stable equivalences of $S^1$-spectra. However, it is an interesting
open question whether the functor converts motivic equivalences
between {\it any\/} pointed motivic spaces to Nisnevich local stable
equivalences. Theorem~\ref{cone} gives an affirmative answer
to this question for the motivic equivalence
$X_+\wedge(-,{\mathbb{A}}^1\rfloor\mathbb G_m)^{\wedge n}_+\to X_+\wedge T^n$
(its proof is in Section~\ref{sec:cone}).

The following theorem (see Theorem~\ref{ZM_fr_and_LM_fr}) is {\it crucial\/} in our analysis and
allows to reduce many computations for framed motives of algebraic varieties to analogous
computations for complexes of linear framed presheaves, which are normally much simpler.
This is the case, in particular, with Theorem~\ref{cone}. It reduces to Theorem~\ref{th:Main}.
It is as well worth to mention another similar application of this kind. In~\cite{AGP}
the Cancellation Theorem for framed motives of algebraic varieties is proved
by reducing it to complexes of linear framed presheaves.

\begin{theorem}\label{ZM_fr_and_LM_fr}
For any integer $m{\geqslant} 0$,
the natural morphism of framed $S^1$-spectra
   $$\lambda_{X_+\wedge T^m}:{\mathbb} Z{\operatorname{Fr}}_*^{S^1}(X_+\wedge T^m)\to EM({\operatorname{\mathbb{Z}F}}_*(-,X_+\wedge T^m))$$
is a schemewise stable equivalence. Moreover,
the natural morphism of framed $S^1$-spectra
   $$l_{X\times T^m}: {\mathbb} ZM_{fr}(X\times T^m)\to LM_{fr}(X\times T^m)$$
is a schemewise stable equivalence. In particular, for any $U\in
Sm/k$ one has
   $$\pi_*({\mathbb} ZM_{fr}(X\times T^m)(U))=H_*({\operatorname{\mathbb{Z}F}}(\Delta^\bullet\times U,X_+\wedge T^m))=H_*(C_*{\mathbb} Z{\operatorname{F}}(U,X_+\wedge T^m)).$$
\end{theorem}

Notation used in Theorem \ref{ZM_fr_and_LM_fr} is explained in Section~\ref{sec:cone}.
The theorem itself is proved in Appendix B. Thus in order to prove Theorem~\ref{cone},
it is {\it sufficient\/} to prove the following result.

\begin{theorem}\label{th:Main}
Let $k$ be an infinite perfect field. For any $k$-smooth scheme $X$
and any $n{\geqslant} 0$ the natural map of complexes of linear framed
presheaves
   $$C_*{\operatorname{\mathbb{Z}F}}(X_+\wedge T^n \wedge {\mathbb{A}}^1_+)/C_*{\operatorname{\mathbb{Z}F}}(X_+\wedge T^n \wedge {{\mathbb{G}_m}}_+)\to C_*{\operatorname{\mathbb{Z}F}}(X_+\wedge T^{n+1})$$
is a Nisnevich local equivalence.
\end{theorem}

Our proof of Theorem~\ref{th:Main} splits in two steps, each of
which is of independent interest. Firstly we introduce a linear
framed sub-presheaf ${\operatorname{\mathbb{Z}F}}^{qf}(X_+\wedge T^{n+1})$ of the linear
framed presheaf ${\operatorname{\mathbb{Z}F}}(X_+\wedge T^{n+1})$ and prove the following

\begin{theorem}\label{p:main}
For any $k$-smooth scheme $X$ and any $n{\geqslant} 0$, the natural
morphism
   $${\operatorname{\mathbb{Z}F}}(X_+\wedge T^n\wedge{\mathbb{A}}^1_+)/{\operatorname{\mathbb{Z}F}}(X_+\wedge T^n\wedge {{\mathbb{G}_m}}_+)\to {\operatorname{\mathbb{Z}F}}^{qf}(X_+\wedge T^{n+1})$$
of linear framed presheaves is an isomorphism locally in the
Nisnevich topology.
\end{theorem}

Applying the singular complex construction and the Resolution
Theorem of~\cite{GP1}, one can show that the natural morphism of
complexes of linear framed presheaves
   \begin{equation}\label{eq:factor_complex}
    C_*{\operatorname{\mathbb{Z}F}}(X_+\wedge T^n\wedge{\mathbb{A}}^1_+)/C_*{\operatorname{\mathbb{Z}F}}(X_+\wedge T^n\wedge{{\mathbb{G}_m}}_+)\to C_*{\operatorname{\mathbb{Z}F}}^{qf}(X_+\wedge T^{n+1})
   \end{equation}
is locally a quasi-isomorphism in the Nisnevich topology (see
Proposition~\ref{p:factor_complex} for details).

Secondly using a moving lemma discussed in Section~\ref{mlemma}, we
then prove the following

\begin{theorem}\label{p:moving}
The inclusion of complexes of linear framed presheaves
$$C_*{\operatorname{\mathbb{Z}F}}^{qf}(X_+\wedge T^{n+1}) \hookrightarrow C_*{\operatorname{\mathbb{Z}F}}(X_+\wedge T^{n+1})$$
is locally a quasi-isomorphism in the Zarisky topology.
\end{theorem}

Clearly, Theorem~\ref{p:moving} together with the Nisnevich local
quasi-isomorphism~\eqref{eq:factor_complex} imply
Theorem~\ref{th:Main}.

Throughout the paper we denote by $Sm/k$ the category of smooth
separated schemes of finite type over the base field $k$. The
subcategory of affine smooth $k$-varieties is denoted by $AffSm/k$.
Given a scheme $W$ and a family of regular function
${\varphi}_1,\ldots,{\varphi}_m$ on $W$, we write $Z({\varphi}_1,\ldots,{\varphi}_{m})$
to denote a closed subset in $W$ which is the common vanishing locus
of the family ${\varphi}_1,\ldots,{\varphi}_m$. Whenever we speak about $\Gamma$-spaces
we follow the terminology of Bousfield--Friedlander~\cite{BF}.

\section{Framed presheaves ${\operatorname{Fr}}(-,Y/(Y-S))$ and ${\operatorname{\mathbb{Z}F}}(-,Y/(Y-S))$}

\begin{definition}\label{def:FrY/Y-S}{\rm
(I) Let $Y$ be a $k$-smooth scheme and $S\subset Y$ be a closed
subset and let $U\in Sm/k$. An {\it explicit framed correspondence
of level $m{\geqslant} 0$ from $U$ to $Y/(Y-S)$} consists of the following
tuples:
   $$(Z,W,{\varphi}_1,\ldots,{\varphi}_{m};g:W\to Y),$$
where $Z$ is a closed subset of $U\times{\mathbb} A^m$, finite over $U$,
$W$ is an \'{e}tale neighborhood of $Z$ in $U\times{\mathbb} A^m$,
${\varphi}_1,\ldots,{\varphi}_{m}$ are regular functions on $W$, $g$ is a
regular map such that $Z=Z({\varphi}_1,\ldots,{\varphi}_{m})\cap g^{-1}(S)$.
The set $Z$ is called the {\it support\/} of the explicit framed
correspondence. We shall also write quadruples $\Phi = (Z,W,{\varphi};g)$
to denote explicit framed correspondences.

(II) Two explicit framed correspondences $(Z,W,{\varphi};g)$ and
$(Z',W',{\varphi}';g')$ of level $m$ are said to be {\it equivalent\/} if
$Z=Z'$ and there exists an \'{e}tale neighborhood $W''$ of $Z$ in
$W\times_{{\mathbb} A^m_U}W'$ such that ${\varphi}\circ pr$ agrees with
${\varphi}'\circ pr'$ and the morphism $g\circ pr$ agrees with $g'\circ
pr'$ on $W''$.

(III) A {\it framed correspondence of level $m$ from $U$ to
$Y/(Y-S)$\/} is the equivalence class of an explicit framed
correspondence of level $m$ from $U$ to $Y/(Y-S)$. We write
${\operatorname{Fr}}_m(U,Y/(Y-S))$ to denote the set of framed correspondences of
level $m$ from $U$ to $Y/(Y-S)$. We regard it as a pointed set whose
distinguished point is the class $0_{Y/(Y-S),m}$ of the explicit
correspondence $(Z,W,{\varphi};g)$ with $W=\emptyset$.

(IV) If $S=Y$ then the set ${\operatorname{Fr}}_m(U,Y/(Y-S))$ is denoted by
${\operatorname{Fr}}_m(U,Y)$ and is called the {\it set of framed correspondences of
level $m$ from $U$ to $Y$}.

(V) Following Voevodsky~\cite{V2}, the {\it category of framed
correspondences\/} ${\operatorname{Fr}}_*(k)$ has objects those of $Sm/k$ and its
morphisms are the sets ${\operatorname{Fr}}_*(U,Y):=\bigsqcup_{m{\geqslant} 0}{\operatorname{Fr}}_m(U,Y)$,
$U,Y\in Sm/k$.

(VI) A {\it framed presheaf\/} is just a contravariant functor from
the category ${\operatorname{Fr}}_*(k)$ to sets.

}\end{definition}

Let $X,Y$ and $S$ be $k$-smooth schemes and let
\begin{gather*}
\Psi=(Z',{\mathbb{A}}^k\times V\xleftarrow{(\alpha,\pi')}
W',\psi_1,\psi_2,\dots,\psi_k;g:W'\to U)\in {\operatorname{Fr}}_k(V,U)
\end{gather*}
be an explicit correspondence of level $k$ from $V$ to $U$ and let
\begin{gather*}
\Phi=(Z,{\mathbb{A}}^m\times U\xleftarrow{(\beta,\pi)}
W,\varphi_1,\varphi_2,\dots,\varphi_m;g':W\to Y)\in {\operatorname{Fr}}_m(U,Y/(Y-S))
\end{gather*}
be an explicit correspondence of level $m$ from $U$ to $Y/(Y-S)$. We
define $\Psi^{*}(\Phi)$ as an explicit correspondence of level $k+m$
from $V$ to $Y/(Y-S)$ as
\[
(Z\times_U Z',{\mathbb{A}}^{k+m}\times V\xleftarrow{(\alpha,\beta,\pi')} W'\times_U W,\psi_1,\psi_2,\dots,\psi_k,\varphi_1,\varphi_2,\dots,\varphi_m,,g'\circ pr_W)\in
{\operatorname{Fr}}_{k+m}(V,Y/(Y-S)).
\]
Clearly, the pull-back operation $(\Psi,\Phi)\mapsto \Psi^{*}(\Phi)$
of explicit correspondences respects the equivalence relation on
them. We get a pairing
   \begin{equation}\label{eq:compos}
    {\operatorname{Fr}}_k(V,U)\times {\operatorname{Fr}}_m(U,Y/(Y-S))\to {\operatorname{Fr}}_{k+m}(V,Y/(Y-S))
   \end{equation}
making ${\operatorname{Fr}}_*(U,Y/(Y-S)):=\bigsqcup_{m{\geqslant} 0} {\operatorname{Fr}}_m(-,Y/(Y-S))$ a
${\operatorname{Fr}}_*(k)$-presheaf.

Let $X,Y,S$ and $T$ be smooth schemes. There is an \textit{external
product}
 \begin{equation}\label{eq:ext_product}
{\operatorname{Fr}}_m(U,Y/(Y-S))\times {\operatorname{Fr}}_n({{\rm pt}},{{\rm pt}}) \xrightarrow{-\boxtimes -}
{\operatorname{Fr}}_{m+n}(U,Y/(Y-S))
 \end{equation}
given by
$$((Z,W,\varphi_1,\varphi_2,\dots,\varphi_m;g),(Z',W',\psi_1,...,\psi_n))\mapsto
(Z\times Z',W\times
W',\varphi_1,\varphi_2,\dots,\varphi_m,\psi_1,...,\psi_n;g).$$

Set $\sigma:= (\{0\},{\mathbb{A}}^1,id : A^1 \to {\mathbb{A}}^1,const : {\mathbb{A}}^1 \to pt)\in {\operatorname{Fr}}_1(pt,pt)$.
Denote by
$$\Sigma: {\operatorname{Fr}}_m(U,Y/(Y-S))\to {\operatorname{Fr}}_{m+1}(U,Y/(Y-S))$$
the map $\Phi \mapsto \Phi\boxtimes \sigma$. Following
Voevodsky~\cite{V2} we give the following

\begin{definition}{\rm
We shall refer to the set
\[
{\operatorname{Fr}}(U,Y/(Y-S)):=\operatorname{colim}({\operatorname{Fr}}_0(U,Y/(Y-S))\xrightarrow{\Sigma}{\operatorname{Fr}}_1(U,Y/(Y-S))
\xrightarrow{\Sigma} {\operatorname{Fr}}_2(U,Y/(Y-S)) \dots)
\]
as the \textit{set stable framed correspondences from $U$ to
$Y/(Y-S)$}. Clearly, ${\operatorname{Fr}}(-,Y/(Y-S))$ is a framed presheaf of
pointed sets with the empty framed correspondence being the
distinguished point.

Clearly, ${\operatorname{Fr}}(-,Y/(Y-S))$ is even a framed functor in the sense of
Voevodsky \cite{V2} meaning that ${\operatorname{Fr}}(\emptyset)=*$ and
${\operatorname{Fr}}(U_1\sqcup U_2,Y/(Y-S))={\operatorname{Fr}}(U_1,Y/(Y-S))\times
{\operatorname{Fr}}(U_2,Y/(Y-S))$.

}\end{definition}

\begin{definition}[cf.~\cite{GP1,GP3}]\label{stab}{\rm
 Let $Y\in Sm/k$ and $S\subset Y$ be as in Definition~\ref{def:FrY/Y-S}. Let $U$ be a $k$-smooth scheme. Denote by
\begin{itemize}
\item[$\diamond$]
${\operatorname{\mathbb{Z}Fr}}_m(U,Y/(Y-S)):=\widetilde{\mathbb{Z}}[{\operatorname{Fr}}_m(U,Y/(Y-S))]=\mathbb{Z}[{\operatorname{Fr}}_m(U,Y/(Y-S))]/\mathbb{Z}\cdot 0_{Y/(Y-S),m}$,
i.e the free abelian group generated by the set ${\operatorname{Fr}}_m(U,Y/(Y-S))$
modulo $\mathbb{Z}\cdot 0_{Y/(Y-S),m}$;

\item[$\diamond$]
${\operatorname{\mathbb{Z}F}}_m(U,Y/(Y-S)):={\operatorname{\mathbb{Z}Fr}}_m(U,Y/(Y-S))/A$, where $A$ is the subgroup
generated by the elements
\begin{multline*}
(Z\sqcup Z', W,(\varphi_1,\varphi_2,\dots,\varphi_m);g) - \\
-(Z, W\setminus
Z',(\varphi_1,\varphi_2,\dots,\varphi_m)|_{W\setminus
Z'};g|_{W\setminus Z'}) - (Z',{W\setminus
Z},(\varphi_1,\varphi_2,\dots,\varphi_m)|_{W\setminus
Z};g|_{W\setminus Z}).
\end{multline*}
\end{itemize}
The elements of ${\operatorname{\mathbb{Z}F}}_m(U,Y/(Y-S))$ are called {\it linear framed
correspondences from $U$ to $Y/(Y-S)$ of level $m$}.
}
\end{definition}

\begin{definition}\label{F_m_U_Y/(Y-S)}{\rm
Define ${\operatorname{F}}_m(U,Y/(Y-S))\subset {\operatorname{Fr}}_m(U,Y/(Y-S))$ as a subset consisting of
$(Z,W,{\varphi};g)\in {\operatorname{Fr}}_m(U,Y/(Y-S))$ such that $Z$ is connected.

Clearly, the set ${\operatorname{F}}_m(U,Y/(Y-S))-0_m$ is a free basis of the abelian group
${\operatorname{\mathbb{Z}F}}_m(U,Y/(Y-S))$. However, the assignment $U\mapsto {\operatorname{F}}_m(U,Y/(Y-S))$
is not a presheaf even on the category $Sm/k$.
}
\end{definition}

The category of {\it linear framed correspondences ${\operatorname{\mathbb{Z}F}}_*(k)$\/} is
defined in~\cite{GP1}. We shall also refer to contravariant functors
from the category ${\operatorname{\mathbb{Z}F}}_*(k)$ to Abelian groups as {\it linear framed
presheaves}.

Set ${\operatorname{\mathbb{Z}F}}_*(U,Y/(Y-S))=\bigoplus_{m{\geqslant} 0}{\operatorname{\mathbb{Z}F}}_m(U,Y/(Y-S))$. The
pairing~\eqref{eq:compos} induces in a natural way a bilinear
pairing
\begin{equation}\label{eq:linear_comp}
{\operatorname{\mathbb{Z}F}}_k(V,U)\times {\operatorname{\mathbb{Z}F}}_m(U,Y/(Y-S))\to {\operatorname{\mathbb{Z}F}}_{k+m}(U,Y/(Y-S)).
\end{equation}
The latter pairing makes ${\operatorname{\mathbb{Z}F}}_*(-,Y/(Y-S))$ a linear framed
presheaf.

The external product~\eqref{eq:ext_product} induces in a natural way
an external product of the form
\begin{equation}\label{eq:linear_ext_product}
{\operatorname{\mathbb{Z}F}}_m(U,Y/(Y-S)))\times {\operatorname{\mathbb{Z}F}}_n({{\rm pt}},{{\rm pt}}) \xrightarrow{-\boxtimes -}
{\operatorname{\mathbb{Z}F}}_{m+n}(U,Y/(Y-S))
\end{equation}

Let $Y\in Sm/k$ and $S\subset Y$ be as in
Definition~\ref{def:FrY/Y-S}. {\it One of the main linear framed
presheaves of this paper\/} we are interested in is defined as
$${\operatorname{\mathbb{Z}F}}(-,Y/(Y-S))=\operatorname{colim}({\operatorname{\mathbb{Z}F}}_0(-,Y/(Y-S))\xrightarrow{\Sigma}{\operatorname{\mathbb{Z}F}}_1(-,Y/(Y-S))\xrightarrow{\Sigma} {\operatorname{\mathbb{Z}F}}_2(-,Y/(Y-S))\xrightarrow{\Sigma}\dots).$$

\section{Presheaves ${\operatorname{Fr}}^{qf}(-,Y/(Y-S)\wedge T)$ and ${\operatorname{\mathbb{Z}F}}^{qf}(-,Y/(Y-S)\wedge T)$}\label{sec:Fr_qf_and_ZF_qf}

Let $Y$ be a $k$-smooth variety, $S\subset Y$ be a closed subset. Then the Nisnevich sheaf
$Y/(Y-S)\wedge T$ equals to $Y\times {\mathbb{A}}^1/(Y\times {\mathbb{A}}^1-S\times \{0\})$.
Thus we have framed presheaves
\begin{multline*}
{\operatorname{Fr}}_*(-,Y/(Y-S)\wedge T):={\operatorname{Fr}}_*(-,Y\times {\mathbb{A}}^1/(Y\times {\mathbb{A}}^1-S\times \{0\})) \quad \text{and} \\
\quad {\operatorname{Fr}}(-,Y/(Y-S)\wedge T):={\operatorname{Fr}}(-,Y\times {\mathbb{A}}^1/(Y\times {\mathbb{A}}^1-S\times \{0\}))
\end{multline*}
and linear framed presheaves
\begin{multline*}
{\operatorname{\mathbb{Z}F}}_*(-,Y/(Y-S)\wedge T):={\operatorname{\mathbb{Z}F}}_*(-,Y\times {\mathbb{A}}^1/(Y\times {\mathbb{A}}^1-S\times \{0\})) \quad \text{and} \\
\quad   {\operatorname{\mathbb{Z}F}}(-,Y/(Y-S)\wedge T):={\operatorname{\mathbb{Z}F}}(-,Y\times {\mathbb{A}}^1/(Y\times {\mathbb{A}}^1-S\times \{0\})).
\end{multline*}

Specifying definitions of the previous section, a section of
${\operatorname{Fr}}_m(-,Y/(Y-S)\wedge T)$ on $U\in Sm/k$ is given by a tuple
   $$(Z,W,{\varphi}_1,\ldots,{\varphi}_{m};g:W\to Y;f:W\to {\mathbb{A}}^1),$$
where $Z$ is a closed subset of $U\times{\mathbb} A^m$ finite over $U$,
$W$ is an \'{e}tale neighborhood of $Z$ in $U\times{\mathbb} A^m$,
${\varphi}_1,\ldots,{\varphi}_{m},f$ are regular functions on $W$, $g$ is a
regular map such that
   $$Z=Z({\varphi}_1,\ldots,{\varphi}_{m},f)\cap g^{-1}(S).$$
The suspension map
   $$\Sigma: {\operatorname{Fr}}_m(-,Y/(Y-S)\wedge T)\to{\operatorname{Fr}}_{m+1}(-,Y/(Y-S)\wedge T)$$
sends $(Z,W,{\varphi}_1,\ldots,{\varphi}_{m};g;f)$ to $(Z\times \{0\},W\times
{\mathbb{A}}^1,{\varphi}_1\circ pr_W,\ldots,{\varphi}_{m}\circ pr_W,pr_{{\mathbb{A}}^1};g\circ
pr_W;f\circ pr_W)$. For brevity, we write $(Z\times \{0\},W\times
{\mathbb{A}}^1,{\varphi},t;g;f)$ for the latter framed correspondence.

Let ${\operatorname{Fr}}^{qf}_m(U,Y/(Y-S)\wedge T)\subset {\operatorname{Fr}}_m(U,Y/(Y-S)\wedge T)$
be a subset consisting of those elements $c$ for which there is an
explicit framed correspondence $(Z,W,{\varphi}_1,\ldots,{\varphi}_{m};g;f)$
representing $c$ such that the closed subset
$Z({\varphi}_1,\ldots,{\varphi}_{m})\cap g^{-1}(S)\subset W$ is {\it
quasi-finite} over $U$.

Set ${\operatorname{Fr}}^{qf}_*(U,Y/(Y-S)\wedge T):=\bigsqcup_{m{\geqslant}
0}{\operatorname{Fr}}^{qf}_m(U,Y/(Y-S)\wedge T)$. Clearly,
${\operatorname{Fr}}^{qf}_*(-,Y/(Y-S)\wedge T)$ is a framed subpresheaf of the
framed presheaf ${\operatorname{Fr}}_*(-,Y/(Y-S)\wedge T)$. Also, it is clear that
the suspension $\Sigma$ takes ${\operatorname{Fr}}^{qf}_m(U,Y/(Y-S)\wedge T)$ to
${\operatorname{Fr}}^{qf}_{m+1}(U,Y/(Y-S)\wedge T)$. Set
   $${\operatorname{Fr}}^{qf}(-,Y/(Y-S)\wedge T):=\operatorname{colim}({\operatorname{Fr}}^{qf}_0(-,Y/(Y-S)\wedge T)\xrightarrow{\Sigma}{\operatorname{Fr}}^{qf}_1(-,Y/(Y-S)\wedge T)\xrightarrow{\Sigma}\dots)$$
By the very construction, ${\operatorname{Fr}}^{qf}(-,Y/(Y-S)\wedge T)$ is a pointed
framed subpresheaf of the pointed framed presheaf
${\operatorname{Fr}}(-,Y/(Y-S)\wedge T)$.

Let $(Z,W,\varphi;g;f)\in {\operatorname{Fr}}_m(U,Y/(Y-S))$ be such that $Z=Z_1\sqcup Z_2$. Then
$(Z,W,\varphi;g;f)\in {\operatorname{Fr}}^{qf}_m(U,Y/(Y-S))$ if and only if for $i,j=1,2$ and $j\neq i$ one has
   $$(Z_i, W\setminus Z_j,\varphi|_{W\setminus Z_j};g|_{W\setminus Z_j};f|_{W\setminus Z_j})\in {\operatorname{Fr}}^{qf}_m(U,Y/(Y-S)).$$

This observation leads to the following

\begin{definition}\label{def:ZF_m_qf}{\rm
Let $Y\in Sm/k$ and $S\subset Y$ be as in
Definition~\ref{def:FrY/Y-S}. Let $U$ be a $k$-smooth scheme. Set
   $${\operatorname{\mathbb{Z}F}}^{qf}_m(U,Y/(Y-S)\wedge T):={\mathbb} Z[{\operatorname{Fr}}^{qf}_m(U,Y/(Y-S)\wedge T)]/A,$$
where $A$ is the subgroup generated by the elements
$$(Z\sqcup Z', W,\varphi;g;f)
-(Z, W\setminus Z',\varphi|_{W\setminus Z'};g|_{W\setminus
Z'};f|_{W\setminus Z'}) - (Z',{W\setminus Z},\varphi|_{W\setminus
Z};g|_{W\setminus Z};g|_{W\setminus Z}).$$

}\end{definition}

Set ${\operatorname{\mathbb{Z}F}}^{qf}_*(U,Y/(Y-S)\wedge T)=\bigoplus_{m{\geqslant}
0}{\operatorname{\mathbb{Z}F}}^{qf}_m(U,Y/(Y-S)\wedge T)$. The pairing~\eqref{eq:linear_comp}
gives rise to a natural pairing ${\operatorname{\mathbb{Z}F}}_k(V,U)\times
{\operatorname{\mathbb{Z}F}}^{qf}_m(U,Y/(Y-S))\to {\operatorname{\mathbb{Z}F}}^{qf}_{k+m}(U,Y/(Y-S))$. The latter
pairing makes ${\operatorname{\mathbb{Z}F}}^{qf}_*(-,Y/(Y-S))$ a linear framed presheaf.

The external product~\eqref{eq:linear_ext_product} gives rise to an
external product
\begin{equation*}\label{potom}
{\operatorname{\mathbb{Z}F}}^{qf}_m(U,Y/(Y-S))\wedge T)\times {\operatorname{\mathbb{Z}F}}_n({{\rm pt}},{{\rm pt}}) \xrightarrow{-\boxtimes -}
{\operatorname{\mathbb{Z}F}}^{qf}_{m+n}(U,Y/(Y-S)\wedge T)
\end{equation*}

\begin{definition}\label{def:ZF_qf}{\rm
Set,
   $${\operatorname{\mathbb{Z}F}}^{qf}(-,Y/(Y-S)\wedge T)=\operatorname{colim}({\operatorname{\mathbb{Z}F}}^{qf}_0(-,Y/(Y-S)\wedge T)\xrightarrow{\Sigma}{\operatorname{\mathbb{Z}F}}^{qf}_1(-,Y/(Y-S)\wedge T)\xrightarrow{\Sigma}\dots)$$
By the very construction ${\operatorname{\mathbb{Z}F}}^{qf}(-,Y/(Y-S)\wedge T)$ is a linear
framed presheaf.

}\end{definition}

\begin{definition}\label{def:F_m_and_F_m_qf}{\rm
For $U\in Sm/k$ let ${\operatorname{F}}_m(U,Y/(Y-S)\wedge T)$ be a subset of ${\operatorname{Fr}}_m(U,Y/(Y-S)\wedge T)$ consisting of elements
$$(Z,W,{\varphi}_1,\ldots,{\varphi}_{m};g;f) \in {\operatorname{Fr}}_m(U,Y/(Y-S)\wedge T)$$
such that $Z$ is connected. Clearly, this definition is consistent with Definition~\ref{F_m_U_Y/(Y-S)}. Set,
   $${\operatorname{F}}^{qf}_m(U,Y/(Y-S)\wedge T)={\operatorname{F}}_m(U,Y/(Y-S)\wedge T)\cap {\operatorname{Fr}}^{qf}_m(U,Y/(Y-S)\wedge T).$$
${\operatorname{F}}_m(U,Y/(Y-S)\wedge T)\setminus {*}$ is plainly a free basis of the free abelian group
${\operatorname{\mathbb{Z}F}}_m(U,Y/(Y-S)\wedge T)$.

}\end{definition}

The following lemma is obvious.

\begin{lemma}\label{l:ZF_qf_m_and_ZF_m}
$(1)$ The set $({\operatorname{F}}^{qf}_m(U,Y/(Y-S)\wedge T)\setminus{*})$ is a free
basis of ${\operatorname{\mathbb{Z}F}}^{qf}_m(U,Y/(Y-S)\wedge T)$.

$(2)$ The natural map ${\operatorname{\mathbb{Z}F}}^{qf}_m(U,Y/(Y-S)\wedge T) \to
{\operatorname{\mathbb{Z}F}}_m(U,Y/(Y-S)\wedge T)$ is injective and identifies
${\operatorname{\mathbb{Z}F}}^{qf}_m(U,Y/(Y-S)\wedge T)$ with a direct summand of
${\operatorname{\mathbb{Z}F}}_m(U,Y/(Y-S)\wedge T)$. Therefore ${\operatorname{\mathbb{Z}F}}^{qf}(-,Y/(Y-S)\wedge T)$
is a framed subpresheaf of the framed presheaf ${\operatorname{\mathbb{Z}F}}(-,Y/(Y-S)\wedge
T)$.
\end{lemma}

\section{Useful lemmas}

In this section we prove a couple of useful lemmas used in the
following sections. We start with a useful remark.

\begin{remark}\label{r:henselian_properties}{\rm
Let $W$ be a local regular scheme with a closed point $w$. Let
$S\subset W$ be a closed subset. Then $S$ is a local scheme which is
connected and $w$ is the only closed point of $S$.

Furthermore, let $U$ be a henzelian local regular scheme with the
closed point $u$. Let $\pi: W\to U$ be a morphism such that
$\pi(w)=u$ and $\pi|_S: S\to U$ is quasi-finite. By~\cite[Theorem
I.4.2]{Mi} $S$ is finite over $U$. Regarding $S$ as a closed
subscheme with the reduced structure, it is connected and local.

Let $S$ be a henzelian local scheme and let $Z$ be its closed
subset. Let $S^h_Z$ be the henzelization of $S$ at $Z$. Then the
canonical morphism $can_{S,Z}:S^h_Z \to S$ is an isomorphism.
Indeed, $(S,id_S,i:Z\hookrightarrow S)$ is the initial object in the
category of \'{e}tale neighborhoods of $Z$ in $S$.

}\end{remark}

\begin{lemma}\label{l:closed-embedding}
Let $V$ be an affine scheme, $Z\subset V$ be a closed connected
subset, $can=can_{V,Z}:V^h_Z \to V$ be the henzelization of $V$ at
$Z$, and let $s:Z\to V^h_Z$ be the section of $can$ over $Z$. Let
$U$ be a regular local henzelian scheme, $q:V\to U$ be a smooth
morphism such that the morphism $q|_Z:Z\to U$ is finite.
Furthermore, suppose $Y\subset V^h_Z$ is a closed subset containing
$s(Z)$, which is quasi-finite over $U$. Then $can|_Y: Y\to V$ is a
closed embedding, $can(Y)$ contains $Z$ and $can^{-1}(can(Y))=Y$.
\end{lemma}

\begin{proof}
Since $Z$ is finite over the local henzelian $U$ and $Z$ is
connected, it is local and henzelian. Since $Z$ is local, then so is
the scheme $W=V^h_Z$. By Remark~\ref{r:henselian_properties} the
scheme $Y$ is local, connected and finite over $U$. Hence
$Y_1=can(Y)$ is closed in $V$ and finite over $U$ and contains $Z$.
So, $Y_1$ is a local henzelian scheme.

The scheme $can^{-1}(Y_1)$ is the henzelization of $Y_1$ at $Z$. By
Remark~\ref{r:henselian_properties} the morphism
$p_1=can|_{can^{-1}(Y_1)}:can^{-1}(Y_1)\to Y_1$ is an isomorphism.
Clearly, $Y$ is a closed subset of $can^{-1}(Y_1)$ and let $i:Y \to
can^{-1}(Y_1)$ be the inclusion. Then the ring map $(p_1\circ i)^*:
\Gamma(Y_1,\mathcal O_{Y_1})\to \Gamma(Y,\mathcal O_{Y})$ is
surjective. On the other hand, this ring morphism is injective,
because both schemes are reduced affine and the morphism $p_1\circ
i$ is surjective. Hence $p_1\circ i:Y\to Y_1$ is an isomorphism.
Thus $can|_Y:Y\to V$ is a closed embedding.
\end{proof}

\begin{lemma}\label{l:lift}
Under the assumptions of Lemma~\ref{l:closed-embedding} let
$Y\subset V$ be a closed connected subset containing $Z$ and be
finite over $U$. Then there is a unique section $t:Y\to V^h_Z$ of
the morphism $can:V^h_Z\to V$ and $t(Y)=can^{-1}(Y)$ contains
$s(Z)$.
\end{lemma}

\begin{proof}
Since $U$ is local henzelian, then so is the scheme $Y$. The scheme
$can^{-1}(Y)$ is the henzelization of $Y$ at $Z$. By
Remark~\ref{r:henselian_properties} the morphism
$p=can|_{can^{-1}(Y)}:can^{-1}(Y)\to Y$ is an isomorphism. Set
$t=in\circ p^{-1}:Y\to V^h_Z$. Clearly, $t$ is a section of $can$
over $Y$. If $t':Y\to V^h_Z$ is another section of $can$, then
$t'=t$.
\end{proof}

\section{The linear framed presheaf ${\operatorname{\mathbb{Z}F}}^{qf}(X_+\wedge T^{n+1})$}\label{s:Fr(X_T_n}

Let $X$ be a $k$-smooth variety and let $Y=X\times {\mathbb{A}}^n$ and
$S=X\times \{0\}$ one has an equality of the Nisnevich sheaves
$Y/(Y-S)=X_+\wedge T^n$. Thus there are framed presheaves
   $${\operatorname{Fr}}^{qf}(-,X_+\wedge T^{n+1}):={\operatorname{Fr}}^{qf}(-,Y/(Y-S)\wedge T) \quad \text{and} \quad {\operatorname{Fr}}(-,X_+\wedge T^{n+1}):={\operatorname{Fr}}(-,Y/(Y-S)\wedge T),$$
$${\operatorname{\mathbb{Z}F}}^{qf}(-,X_+\wedge T^{n+1}):={\operatorname{\mathbb{Z}F}}^{qf}(-,Y/(Y-S)\wedge T) \quad \text{and} \quad {\operatorname{\mathbb{Z}F}}(-,X_+\wedge T^{n+1}):={\operatorname{\mathbb{Z}F}}(-,Y/(Y-S)\wedge T).$$

There are also framed presheaves:
\begin{itemize}
\item[$\diamond$] ${\operatorname{Fr}}^{qf}_*(-,X_+\wedge T^{n+1}):={\operatorname{Fr}}^{qf}_*(-,Y/(Y-S)\wedge T)$,
\item[$\diamond$] ${\operatorname{Fr}}_*(-,X_+\wedge T^{n+1}):={\operatorname{Fr}}_*(-,Y/(Y-S)\wedge T)$,
\item[$\diamond$] ${\operatorname{\mathbb{Z}F}}^{qf}_*(-,X_+\wedge T^{n+1}):={\operatorname{\mathbb{Z}F}}^{qf}_*(-,Y/(Y-S)\wedge T)$,
\item[$\diamond$] ${\operatorname{\mathbb{Z}F}}_*(-,X_+\wedge T^{n+1}):={\operatorname{\mathbb{Z}F}}_*(-,Y/(Y-S)\wedge T)$,
\end{itemize}
and sets
\begin{itemize}
\item[$\diamond$] ${\operatorname{F}}_m(U,X_+\wedge T^{n+1}):={\operatorname{F}}_m(U,Y/(Y-S)\wedge
T)$,
\item[$\diamond$] ${\operatorname{F}}^{qf}_m(U,X_+\wedge T^{n+1}):={\operatorname{F}}^{qf}_m(U,Y/(Y-S)\wedge T)$.
\end{itemize}
Specifying definitions of Section~\ref{sec:Fr_qf_and_ZF_qf}, a
section of ${\operatorname{Fr}}_m(-,X_+\wedge T^{n+1})$ on $U\in Sm/k$ is a tuple
   $$c=(Z,W,{\varphi}_1,\ldots,{\varphi}_{m};h:W\to X\times {\mathbb{A}}^1;f:W\to {\mathbb{A}}^1),$$
where $Z$ is a closed subset of $U\times{\mathbb} A^m$ finite over $U$,
$W$ is an \'{e}tale neighborhood of $Z$ in ${\mathbb{A}}^m\times U$,
${\varphi}_1,\ldots,{\varphi}_{m},f$ are regular functions on $W$,
$h=(g,{\varphi}_{m+1},\ldots,{\varphi}_{m+n}):W\to X\times {\mathbb{A}}^n$ is a regular
map such that
   $$Z=Z({\varphi}_1,\ldots,{\varphi}_{m},f)\cap h^{-1}(X\times \{0\})=Z({\varphi}_1,\ldots,{\varphi}_{m},f,{\varphi}_{m+1},\ldots,{\varphi}_{m+n}).$$
The section $c$ is in ${\operatorname{Fr}}^{qf}_m(U,X_+\wedge T^{n+1})$ if and only
if the vanishing locus
$Z({\varphi}_1,\ldots,{\varphi}_{m},{\varphi}_{m+1},\ldots,{\varphi}_{m+n})$ is {\it
quasi-finite\/} over $U$.

The section $c$ is in ${\operatorname{F}}_m(U,X_+\wedge T^{n+1})$ if and only if the
set $Z$ is {\it connected}. The section $c$ is in
${\operatorname{F}}^{qf}_m(U,X_+\wedge T^{n+1})$ if and only if the set $Z$ is {\it
connected\/} and the vanishing locus
$Z({\varphi}_1,\ldots,{\varphi}_{m},{\varphi}_{m+1},\ldots,{\varphi}_{m+n})$ is {\it
quasi-finite} over $U$.

The suspension map $\Sigma: {\operatorname{Fr}}_m(-,X_+\wedge T^{n+1}) \to
{\operatorname{Fr}}_{m+1}(-,X_+\wedge T^{n+1})$ sends
$(Z,W,{\varphi}_1,\ldots,{\varphi}_{m};g;{\varphi}_{m+1},\ldots,{\varphi}_{m+n};f)$ to
$(Z\times \{0\},W\times
{\mathbb{A}}^1,{\varphi}_1,\ldots,{\varphi}_{m},t;g;{\varphi}_{m+1},\ldots,{\varphi}_{m+n};f)$.

\begin{notation}{\rm
For the convenience of computations we shall write
$(Z,W,{\varphi}_1,\ldots,{\varphi}_{m},{\varphi}_{m+1},\ldots,{\varphi}_{m+n};f:W\to
{\mathbb{A}}^1,g:W\to X)$ for $(Z,W,{\varphi}_1,\ldots,{\varphi}_{m};g:W\to
X;{\varphi}_{m+1},\ldots,{\varphi}_{m+n};f:W\to {\mathbb{A}}^1)$ in the rest of the
paper.

}\end{notation}

The canonical morphism ${\mathbb{A}}^1\to T$ induces a morphism of framed
presheaves
   $$p:{\operatorname{Fr}}_*(-,X_+\wedge T^{n}\wedge {\mathbb{A}}^1_+) \to {\operatorname{Fr}}_*(-,X_+\wedge T^{n+1}),$$
sending
$(Z,W,{\varphi}_1,\ldots,{\varphi}_{m},{\varphi}_{m+1},\ldots,{\varphi}_{m+n};f;g)$ to
$(Z',W,{\varphi}_1,\ldots,{\varphi}_{m},{\varphi}_{m+1},\ldots,{\varphi}_{m+n},{\varphi}_{m+n+1};g),$
where $Z'=Z\cap Z(f)$, ${\varphi}_{m+n+1}=f$.

\begin{lemma}\label{l:pushout}
If $U$ is essentially $k$-smooth local henzelian, then the image of
the map $p:{\operatorname{Fr}}_*(U,X_+\wedge T^{n}\wedge {\mathbb{A}}^1_+) \to
{\operatorname{Fr}}_*(U,X_+\wedge T^{n+1})$ is contained in ${\operatorname{Fr}}^{qf}_*(U,X_+\wedge
T^{n+1})$. Moreover, for any $m{\geqslant} 0$ the map $p$ sends
${\operatorname{F}}_m(U,X_+\wedge T^{n}\wedge {\mathbb{A}}^1_+)$ to the set
${\operatorname{F}}^{qf}_m(U,X_+\wedge T^{n+1})$. Finally, the square of pointed
sets
$$\xymatrix{{\operatorname{F}}_m(U,X_+\wedge T^{n}\wedge {{\mathbb{G}_m}}_{,+}) \ar[r]^{i}\ar[d]_{}& {\operatorname{F}}_m(U,X_+\wedge T^{n}\wedge {\mathbb{A}}^1_+) \ar^{p}[d]\\
               {*} \ar[r]^{}&  {\operatorname{F}}^{qf}_m(U,X_+\wedge T^{n+1}), }$$
is a pushout square. Here ${*} \in {\operatorname{F}}_m(U,X_+\wedge T^{n+1})$ is the
empty framed correspondence.
\end{lemma}

\begin{proof}
The first assertion is obvious. To prove the second one, take an
element
   $$c=(Y,W,{\varphi}_1,\ldots,{\varphi}_{m+n},f:W\to {\mathbb{A}}^1;g:W\to X) \in {\operatorname{F}}_m(U,X_+\wedge T^{n}\wedge {\mathbb{A}}^1_+)$$
with $W=({\mathbb{A}}^m\times U)^h_Y$. Then
$p(c)=(Z,W,{\varphi}_1,\ldots,{\varphi}_{m+n+1};g)$, where $Z=Y\cap Z(f)$,
${\varphi}_{m+n+1}=f$.

Since $Y$ is connected and finite over the henzelian $U$, then $Y$
is henzelian and local. Hence $W$ is local. By
Remark~\ref{r:henselian_properties} the closed subset $Z$ is
connected, whence the second assertion of the lemma.

To prove the third assertion, it sufficient to construct a section
   $$s:{\operatorname{F}}^{qf}_m(U,X_+\wedge T^{n+1})\setminus * \to {\operatorname{F}}_m(U,X_+\wedge T^{n}\wedge{\mathbb{A}}^1_+)\setminus {\operatorname{F}}_m(U,X_+\wedge T^{n}\wedge{{\mathbb{G}_m}}_+)$$
of $p$ and check that the map $p$ is injective on the complement of
${\operatorname{F}}_m(U,X_+\wedge T^{n}\wedge{{\mathbb{G}_m}}_+)$.

We construct $s$ as follows. Take an element $c=(Z,{\mathbb{A}}^m\times
U\xleftarrow{can} W,{\varphi}_1,\ldots,{\varphi}_{m+n+1};g)$ in
${\operatorname{F}}^{qf}_m(U,X_+\wedge T^{n+1})$ with a non-empty $Z$ and with
$W=({\mathbb{A}}^m\times U)^h_Z$. Since $Z$ is connected and finite over the
local henzelian $U$, the scheme $W$ is local. Set
   $$s(c)=(can(Y),{\mathbb{A}}^m\times U\xleftarrow{can} W,{\varphi}_1,\ldots,{\varphi}_{m+n};{\varphi}_{m+n+1}:W\to {\mathbb{A}}^1;g:W\to X),$$
where $Y=Z({\varphi}_1,\ldots,{\varphi}_{m+n})\subset W$. The set $Y$ is
quasi-finite over $U$, because $c\in {\operatorname{F}}^{qf}_m(U,X_+\wedge
T^{n+1})$. By Remark~\ref{r:henselian_properties} the set $Y$ is
finite over $U$ and connected and local. By
Lemma~\ref{l:closed-embedding} the morphism $can|_{Y}:Y\to
{\mathbb{A}}^m\times U$ is a closed embedding and $can^{-1}(can(Y))=Y$. Thus
$s(c)\in {\operatorname{F}}_m(U,X_+\wedge T^{n+1}\wedge {\mathbb{A}}^1_+).$ Clearly,
$p(s(c))=c$.

Now check the required injectivity for $p$. Let
$c'=(Y',W',{\varphi}'_1,\ldots,{\varphi}'_{m+n},f':W'\to {\mathbb{A}}^1;g':W'\to X)$ and
$c''=(Y'',W'',{\varphi}''_1,\ldots,{\varphi}''_{m+n},f'':W''\to
{\mathbb{A}}^1;g'':W''\to X)$ in ${\operatorname{F}}_m(U,X_+\wedge T^{n}\wedge {\mathbb{A}}^1_+)$ be two
elements with $W'=({\mathbb{A}}^m\times U)^h_{Y'}$ and $W''=({\mathbb{A}}^m\times
U)^h_{Y''}$. Let $can':W'\to {\mathbb{A}}^m\times U$ be the canonical morphism
and let $s':Y'\to W'$ be the section of $can'$.

Suppose that $p(c)=p(c')$ and the support $Z=Y'\cap Z(f')=Y''\cap
Z(f'')$ is non-empty. We must check that $c'=c''$. The element
$p(c')$ is of the form
   $$(Z,{\mathbb{A}}^m\times U\xleftarrow{can} W,{\varphi}_1,\ldots,{\varphi}_{m+n+1};g),$$
where $Z=Y'\cap Z(f')$, $W=({\mathbb{A}}^m\times U)^h_Z$,
${\varphi}_{i}={\varphi}_i|_{W}$,$g=g|_W$. Consider the canonical morphism
$can_1: W\to W'$. It is the henzelization of $W'$ at $s'(Y')$. Note
that $s'(Y')\subset W'$ is a closed subset containing $s'(Z)$.
Moreover, $Y'$ is finite over the henzelian $U$ and connected. By
Lemma~\ref{l:lift} there is a unique section $t':Y'\to W$ of the
morphism $can_1$, and $t'(Y')=can^{-1}_1(Y')$ contains $s(Z)$, where
$s:Z\to W$ is the section of $can_1$ (the morphism $s$ is also the
section of $can=can'\circ can_1$). These arguments imply an equality
   $$c'=(Y',W,{\varphi}'_1|_W,\ldots,{\varphi}'_{m+n}|_W,f'|_W:W\to {\mathbb{A}}^1;g'|_W:W\to X) \in {\operatorname{F}}_m(U,X_+\wedge T^{n}\wedge {\mathbb{A}}^1_+).$$
By the same reason one has an equality
   $$c''=(Y'',W,{\varphi}''_1|_W,\ldots,{\varphi}''_{m+n}|_W,f''|_W:W\to {\mathbb{A}}^1;g''|_W:W\to X) \in {\operatorname{F}}_m(U,X_+\wedge T^{n}\wedge {\mathbb{A}}^1_+).$$
Since $W$ is the henzelization of ${\mathbb{A}}^m\times U$ at $Z$ and
$p(c')=p(c'')$, one has equalities ${\varphi}'_i|_W={\varphi}''_i|_W$ for
$i=1,...,m+n$, $f'|_W=f''_W$ and $g'|_W=g''|_W$. Hence $Y'=Y''$ and,
moreover, $c'=c''$ in ${\operatorname{F}}_m(U,X_+\wedge T^{n}\wedge {\mathbb{A}}^1_+)$. The
desired injectivity is proved. The section $s$ is constructed above
and our lemma follows.
\end{proof}

\begin{corollary}
\label{cor:pushout}
For any integer $n{\geqslant} 0$ the natural morphism
$$\alpha_*: {\operatorname{\mathbb{Z}F}}_*(X_+\wedge T^n \wedge {\mathbb{A}}^1_+)/{\operatorname{\mathbb{Z}F}}_*(X_+\wedge T^n \wedge {{\mathbb{G}_m}}_+)\to {\operatorname{\mathbb{Z}F}}^{qf}_*(X_+\wedge T^{n+1})$$
of ${\operatorname{\mathbb{Z}F}}_*(k)$-presheaves is an isomorphism locally for the Nisnevich
topology. As a consequence, the natural morphism
   $$\alpha: {\operatorname{\mathbb{Z}F}}(X_+\wedge T^n \wedge {\mathbb{A}}^1_+)/{\operatorname{\mathbb{Z}F}}(X_+\wedge T^n \wedge {{\mathbb{G}_m}}_+)\to {\operatorname{\mathbb{Z}F}}^{qf}(X_+\wedge T^{n+1})$$
of ${\operatorname{\mathbb{Z}F}}_*(k)$-presheaves is an isomorphism locally for the Nisnevich topology.
\end{corollary}

We are now in a position to prove Theorem~\ref{p:main}.

\begin{proof}[Proof of Theorem~\ref{p:main}]
The theorem is implied by Definitions~\ref{def:ZF_qf},
\ref{def:F_m_and_F_m_qf} and Corollary~\ref{cor:pushout}.
\end{proof}

\begin{lemma}\label{l:weak-eq_and_C_*}
Let $k$ be an infinite perfect field. Let $A$ and $B$ be linear
framed presheaves such that the cohomology presheaves of the
complexes $C_*(A)$ and $C_*(B)$ are quasi-stable
(see~\cite[Def.~2.6]{GP1} for the definition of quasi-stability).
Let $\alpha: A\to B$ be a morphism of linear framed presheaves,
which is an isomorphism locally in the Nisnevich topology. Then the
morphism
   $$C_*(\alpha): C_*(A)\to C_*(B)$$
is a quasi-isomorphism locally in the Nisnevich topology.
\end{lemma}

\begin{proof}
By assumption the map of the Eilenberg--Mac~Lane spectra
$\alpha:EM(A)\to EM(B)$ is a local weak equivalence. Hence the
induced map $\alpha:EM(C_*(A))\to EM(C_*(B))$ is a motivic weak
equivalence of $S^1$-spectra. By assumption the stable homotopy
groups of the spectra $EM(C_*(A)), EM(C_*(B))$ are radditive,
quasi-stable and ${\mathbb} A^1$-invariant (see~\cite[Introduction]{GP2}
for the definition of radditivity). The Resolution Theorem
of~\cite{GP1} implies $\alpha:EM(C_*(A))\to EM(C_*(B))$ is a
Nisnevich local weak equivalence of $S^1$-spectra. Thus the morphism
of complexes $C_*(\alpha): C_*(A)\to C_*(B)$ is a quasi-isomorphism
locally in the Nisnevich topology.
\end{proof}

We are now in a position to prove a statement which is necessary for
the proof of Theorem~\ref{th:Main}.

\begin{proposition}\label{p:factor_complex}
Let $k$ be an infinite perfect field. Then the morphism
   $$C_*(\alpha): C_*{\operatorname{\mathbb{Z}F}}(X_+\wedge T^n \wedge {\mathbb{A}}^1_+)/C_*{\operatorname{\mathbb{Z}F}}(X_+\wedge T^n \wedge {{\mathbb{G}_m}}_+)\to C_*{\operatorname{\mathbb{Z}F}}^{qf}(X_+\wedge T^{n+1})$$
is a quasi-isomorphism locally in the Nisnevich topology.
\end{proposition}

\begin{proof}
The morphism $\alpha: {\operatorname{\mathbb{Z}F}}(X_+\wedge T^n \wedge {\mathbb{A}}^1_+)/{\operatorname{\mathbb{Z}F}}(X_+\wedge
T^n \wedge {{\mathbb{G}_m}}_+)\to {\operatorname{\mathbb{Z}F}}^{qf}(X_+\wedge T^{n+1})$ is a morphism of
linear framed presheaves. Set $A={\operatorname{\mathbb{Z}F}}(X_+\wedge T^n \wedge
{\mathbb{A}}^1_+)/{\operatorname{\mathbb{Z}F}}(X_+\wedge T^n \wedge {{\mathbb{G}_m}}_+)$ and $B={\operatorname{\mathbb{Z}F}}^{qf}(X_+\wedge
T^{n+1})$. Then the cohomology presheaves of the complexes $C_*(A)$
and $C_*(B)$ are quasi-stable by the construction of $A$ and $B$.
Now Theorem~\ref{p:main} and Lemma~\ref{l:weak-eq_and_C_*} imply the
claim.
\end{proof}

Thus we have computed locally the complex $C_*{\operatorname{\mathbb{Z}F}}(X_+\wedge T^n
\wedge {\mathbb{A}}^1_+)/C_*{\operatorname{\mathbb{Z}F}}(X_+\wedge T^n \wedge {{\mathbb{G}_m}}_+)$ as the complex
$C_*{\operatorname{\mathbb{Z}F}}^{qf}(X_+\wedge T^{n+1})$. Our next goal is to show that the
latter complex is quasi-isomorphic to $C_*{\operatorname{\mathbb{Z}F}}(X_+\wedge T^{n+1})$
locally in the Zarisky topology verifying Theorem~\ref{p:moving}.
The next two sections are dedicated to this theorem.

\section{A filtration on ${\operatorname{\mathbb{Z}F}}_n(-,X_+\wedge T^{n+1})$}

We start with the following

\begin{definition}\label{d:defining_set}{\rm
Let $U\in Sm/k$ be an affine variety and let $c=(Z,W,{\varphi};g)\in
{\operatorname{Fr}}_m(U,X_+\wedge T^{n+1})$ be a framed correspondence. A finite set
of polynomials $F_1,\ldots, F_r \in k[{\mathbb{A}}^{m+n+1}\times U]$ is said
to be {\it $c$-defining\/} if for every point $u\in U$ there is
$i\in \{1,2...,r\}$ such that
\begin{itemize}
\item[$\diamond$] the polynomial $F_i(-,u) \in k(u)[{\mathbb{A}}^{m+n+1}]$ is
nonzero,
\item[$\diamond$] ${\varphi}(W_u)\subseteq Z(F_i(-,u))$ in
${\mathbb{A}}^{m+n+1}_u$.
\end{itemize}
Note that if a finite set of polinomials $F_1,\ldots, F_r \in
k[{\mathbb{A}}^{m+n+1}\times U]$ is $c$-defining, where
$c=(Z,W,{\varphi};g)=(Z_1\sqcup Z_2,W,{\varphi};g)$, then the same collection
of polynomials is $(Z_1,W-Z_2,{\varphi};g)$-defining and is
$(Z_2,W-Z_1,{\varphi};g)$-defining respectively.

}\end{definition}

The following lemma is crucial in our analysis.

\begin{lemma}
Let $m,n{\geqslant} 0$ and $Y$ an affine (possibly non-irreducible)
$k$-variety. Let $W\to {\mathbb{A}}^m\times Y$ be an \'{e}tale morphism and
$\psi: W\to {\mathbb{A}}^{m+n+1}\times Y$ be a morphism of $Y$-schemes. Then
there is a finite set of polynomials $F_1,\ldots, F_r \in
k[{\mathbb{A}}^{m+n+1}\times Y]$ such that for every point $y\in Y$ there is
$i\in \{1,2...,r\}$ such that
\begin{itemize}
\item[$\diamond$] the polynomial $F_i(-,y) \in k(y)[{\mathbb{A}}^{m+n+1}]$ is nonzero and
\item[$\diamond$] ${\varphi}(W_y)\subseteq Z(F_i(-,u))$ in ${\mathbb{A}}^{m+n+1}_u$.
\end{itemize}
\end{lemma}

\begin{proof}
We proceed by induction in the dimension of $Y$. If $dim(Y)=0$ then
there is nothing to prove. Now suppose $dim(Y)>0$. Let $Y_1,...,Y_l$
be all irreducible components of $Y$. For an index $i$ from
$\{1,...,l\}$ take the restriction of the map $\psi|_{W_{Y_i}}\colon
W_{Y_i}\to{\mathbb{A}}^{m+n+1}\times Y_i$ to $Y_i$. The dimension of $W_{Y_i}$
is $m+dim(Y_i)$. Thus the closure of its image is contained in the
zero locus $Z(\bar F_{(i)})$ of a non-zero polynomial $\bar
F_{(i)}\in k[{\mathbb{A}}^{m+n+1}\times Y_i]$. Since $Y_i$ is closed in the
affine variety $Y$, the polynomial $\bar F_i$ can be extended to a
polynomial $F_i\in k[{\mathbb{A}}^{m+n+1}\times Y]$. Let $V\subset Y$ be an
open subset consisting of those $y\in Y$ such that there is $i$ with
$0\neq F_i(-,y)$ in $k(y)[{\mathbb{A}}^{m+n+1}]$. Let $Y'=Y-V$.

By construction, $V$ has a non-empty intersection with every
irreducible component of $Y$. Thus $dim(Y')<dim(Y)$. By the
inductive assumption there are polynomials $\bar F_{l+1},...,\bar
F_{r}$ in $k[{\mathbb{A}}^{m+n+1}\times Y']$ such that for every point $y\in
Y'$ there is $j\in \{l+1,...,r\}$ with each polynomial $\bar
F_j(-,y) \in k(y)[{\mathbb{A}}^{m+n+1}]$ nonzero and $\psi(W_y)\subseteq
Z(\bar F_j(-,y))$ in ${\mathbb{A}}^{m+n+1}_y$. Since $Y'$ is closed in the
affine $Y$ for any $j$, the polynomial $\bar F_j$ can be extended to
a polynomial $F_j$ in $k[{\mathbb{A}}^{m+n+1}\times Y]$. Clearly, the set of
polynomials $F_i$, where $i\in \{1,2,...,r\}$, are the desired
polynomials for $Y$.
\end{proof}

The preceding lemma has the following

\begin{corollary}\label{cor:existdef}
Let $U\in Sm/k$ be an affine variety and let $c=(Z,W,{\varphi};g)\in
{\operatorname{Fr}}_m(U,X_+\wedge T^{n+1})$ be a framed correspondence. Then there
exists a $c$-defining set of polynomials $F_1,\ldots, F_r \in
k[{\mathbb{A}}^{m+n+1}\times U]$.

Moreover, if $f: V\to U$ is a morphism of
$k$-smooth affine varieties and $F_1,\ldots, F_r \in
k[{\mathbb{A}}^{m+n+1}\times U]$ is a $c$-defining set, then $f^*(F_1),\ldots,
f^*(F_r) \in k[{\mathbb{A}}^{m+n+1}\times V]$ is a $f^*(c)$-defining set.
\end{corollary}

Let $U\in Sm/k$ be an affine variety. Let $d>0$. Define
${\operatorname{Fr}}_m^{<d}(U,X_+\wedge T^{n+1})$ as a subset of ${\operatorname{Fr}}_m(U,X_+\wedge
T^{n+1})$ consisting of those $c=(Z,W,{\varphi};g)\in {\operatorname{Fr}}_m(U,X_+\wedge
T^{n+1})$ for which there exists a $c$-defining set $F_1,\ldots F_r
\in k[{\mathbb{A}}^{m+n+1}\times U]$ with $\deg F_i<d$ for all $i=1,\ldots,r$.
Set,
   $$({\operatorname{Fr}}^{qf}_m)^{<d}(U,X_+\wedge T^{n+1}):= {\operatorname{Fr}}^{qf}_m(U,X_+\wedge T^{n+1}) \cap{\operatorname{Fr}}_m^{<d}(U,X_+\wedge T^{n+1}),$$

Corollary~\ref{cor:existdef} has the following obvious application.

\begin{lemma}\label{l:exhauting_1}
For any integers $m,n{\geqslant} 0$ and any integer $d>0$ the following
statements are true:
\begin{itemize}
\item[(i)]
${\operatorname{Fr}}_m^{<d}(-,X_+\wedge T^{n+1})$ is a subpresheaf on $AffSm/k$ of
the presheaf ${\operatorname{Fr}}_m(-,X_+\wedge T^{n+1})$;
\item[(ii)]
the increasing filtration of the presheaf ${\operatorname{Fr}}_m(-,X_+\wedge
T^{n+1})|_{AffSm/k}$ by subpresheaves ${\operatorname{Fr}}^{<d}_m(-,X_+\wedge
T^{n+1})$ is exhauting;
\item[(iii)]
$({\operatorname{Fr}}^{qf}_m)^{<d}(-,X_+\wedge T^{n+1})$ is a subpresheaf on
$AffSm/k$ of the presheaf ${\operatorname{Fr}}^{qf}_m(-,X_+\wedge T^{n+1})$;
\item[(iv)]
the increasing filtration of the presheaf ${\operatorname{Fr}}^{qf}_m(-,X_+\wedge
T^{n+1})|_{AffSm/k}$ by subpresheaves $({\operatorname{Fr}}^{qf}_m)^{<d}(-,X_+\wedge
T^{n+1})$ is exhauting.
\end{itemize}
\end{lemma}

\begin{definition}\label{d:filtration_on_ZF_qf_m}{\rm
For an affine $k$-smooth $U$ we define ${\operatorname{\mathbb{Z}F}}_m^{<d}(U,X_+\wedge T^{n+1})$ as
$$\mathbb Z[{\operatorname{Fr}}_m^{<d}(U,X_+\wedge T^{n+1})]/
\langle(Z_1\sqcup Z_2,W,{\varphi};g)-(Z_1,W_2,{\varphi}|_{W_2};g|_{W_2})
-(Z_2,W_1,{\varphi}|_{W_1};g|_{W_1})\rangle,$$ where $W_i=W-Z_i$ for
$i=1,2$. By Lemma~\ref{l:exhauting_1} the assignment $U\mapsto
{\operatorname{\mathbb{Z}F}}_m^{<d}(U,X_+\wedge T^{n+1})$ is a presheaf on $AffSm/k$.

Likewise, for an affine $k$-smooth $U$ define
$({\operatorname{\mathbb{Z}F}}^{qf}_m)^{<d}(U,X_+\wedge T^{n+1})$ as
$$\mathbb Z[({\operatorname{Fr}}^{qf}_m)^{<d}(U,X_+\wedge T^{n+1})]/
\langle(Z_1\sqcup
Z_2,W,{\varphi};g)-(Z_1,W_2,{\varphi}|_{W_2};g|_{W_2})-(Z_2,W_1,{\varphi}|_{W_1};g|_{W_1})\rangle,$$
where $W_i=W-Z_i$ for $i=1,2.$ By Lemma~\ref{l:exhauting_1} the
assignment $U\mapsto ({\operatorname{\mathbb{Z}F}}^{qf}_m)^{<d}(U,X_+\wedge T^{n+1})$ is a
presheaf on $AffSm/k$.

Recall that ${\operatorname{\mathbb{Z}F}}_m(-,X_+\wedge T^{n+1})$ is a presheaf on $Sm/k$
(see Definition~\ref{stab}) given by
   $$\mathbb Z[{\operatorname{Fr}}_m(U,X_+\wedge T^{n+1})]/\langle(Z_1\sqcup Z_2,W,{\varphi};g)-(Z_1,W_2,{\varphi}|_{W_2};g|_{W_2})-(Z_2,W_1,{\varphi}|_{W_1};g|_{W_1})\rangle,$$
where $W_i=W-Z_i$ for $i=1,2$. In turn, ${\operatorname{\mathbb{Z}F}}^{qf}_m(-,X_+\wedge
T^{n+1})$ is a presheaf on $Sm/k$ (see Definition~\ref{def:ZF_m_qf})
given by
   $$\mathbb Z[{\operatorname{Fr}}^{qf}_m(U,X_+\wedge T^{n+1})]/\langle(Z_1\sqcup Z_2,W,{\varphi};g)-(Z_1,W_2,{\varphi}|_{W_2};g|_{W_2})-(Z_2,W_1,{\varphi}|_{W_1};g|_{W_1})\rangle,$$
where $W_i=W-Z_i$ for $i=1,2.$

}\end{definition}

For any positive integers $d < d'$ the inclusion
${\operatorname{Fr}}_m^{<d}(-,X_+\wedge T^{n+1})\subset {\operatorname{Fr}}_m^{<d}(-,X_+\wedge
T^{n+1})$ induces a morphism of Abelian presheaves on $AffSm/k$
   $${\operatorname{\mathbb{Z}F}}_m^{<d}(U,X_+\wedge T^{n+1})\to {\operatorname{\mathbb{Z}F}}_m^{<d'}(U,X_+\wedge T^{n+1}).$$
Likewise, for any positive $d < d'$ the inclusion
$({\operatorname{Fr}}^{qf}_m)^{<d}(-,X_+\wedge T^{n+1})\subset
({\operatorname{Fr}}^{qf}_m)^{<d'}(-,X_+\wedge T^{n+1})$ induces a morphism of
Abelian presheaves on $AffSm/k$
   $$({\operatorname{\mathbb{Z}F}}^{qf}_m)^{<d}(U,X_+\wedge T^{n+1})\to ({\operatorname{\mathbb{Z}F}}^{qf}_m)^{<d'}(U,X_+\wedge T^{n+1}).$$

The next statement is a consequence of Lemma~\ref{l:exhauting_1}.

\begin{corollary}\label{cor:exhauting_2}
One has two equalities of presheaves on $AffSm/k$
   $$\operatorname{colim}_{d} {\operatorname{\mathbb{Z}F}}_m^{<d}(-,X_+\wedge T^{n+1})={\operatorname{\mathbb{Z}F}}_m(-,X_+\wedge T^{n+1})$$
and
   $$\operatorname{colim}_{d} ({\operatorname{\mathbb{Z}F}}^{qf}_m)^{<d}(-,X_+\wedge T^{n+1})={\operatorname{\mathbb{Z}F}}^{qf}_m(-,X_+\wedge T^{n+1}).$$
\end{corollary}

\begin{proof}
This is straightforward.
\end{proof}

\section{Moving lemma}\label{mlemma}

\begin{lemma}\label{lem:tnonzero}\footnote{We thank A.~Ananyevskiy for suggesting Lemma~\ref{lem:tnonzero}
in its present form.} Let $L$ be a field and $F\in L[x_1,\ldots
x_{r+1}]$ a nonzero polynomial such that $\deg F\leqslant (d-1)$.
Then the polynomials $F(t^{d^r},t^{d^{r-1}},\ldots, t^d,t)$ and
$F(t^d,\ldots,t^{d^{r-1}},t^{d^r},t)$ are both non-zero in $L[t]$.
Moreover, for any non-zero $s \in L$ the polynomials
$F(st^{d^r},st^{d^{r-1}},\ldots, st^d,t)$ and
$F(st^d,\ldots,st^{d^{r-1}},st^{d^r},t)$ are both non-zero.
\end{lemma}

\begin{proof}
Let us prove that the first polynomial is non-zero. If
$F=\sum_{I=(i_1,\ldots, i_{r+1})} a_I x^I$ then
   \begin{equation}\label{zvezda}
    F(t^{d^r},t^{d^{r-1}},\ldots, t^r,t)=\sum a_I t^{i_1d^r+\ldots+ i_nr+\ldots +i_{n+1}}
   \end{equation}
Let us check that if $I$ and $J$ are two different multi-indices,
then $I\cdot (d^r,\ldots,d,1)\neq J\cdot (d^r,\ldots,d,1)$. Indeed,
if these are equal, then
   \[d^r(i_1-j_1)+\cdots+d(i_n-j_n)+(i_{n+1}-j_{n+1})=0.\]
It follows that $i_{n+1}-j_{n+1}$ is divisible by $d$, but
$|i_{n+1}-j_{n+1}|\leqslant d-1$, hence $i_{n+1}=j_{n+1}$. Then
$(i_n-j_n)$ is divisible by $d$, hence zero and $(i_1-j_1)$ is zero
by induction. Thus all powers of $t$ in the sum~\eqref{zvezda} are
distinct. So if some $a_I\neq 0$ then the right hand side
of~\eqref{zvezda} is nonzero. The second polynomial is obtained from
the first one by permuting powers of $t$. Thus it is non-zero. For
any multi-index $I$ the coefficient at $t^{i_1d^r+\ldots+
i_nr+\ldots +i_{n+1}}$ in the polynomial
$F(st^{d^r},st^{d^{r-1}},\ldots, st^d,t)$ is obtained from $a_I$ by
multiplying a power of $s$. We see that the polynomial
$F(st^{d^r},st^{d^{r-1}},\ldots, st^d,t)$ is non-zero, and hence so
is $F(st^d,\ldots,st^{d^{r-1}},st^{d^r},t)$.
\end{proof}

Let $U\in Sm/k$ be an affine variety and let $c=(Z,W,{\varphi})\in
{\operatorname{Fr}}_m(U,X_+\wedge T^{n+1};g)$ be a framed correspondence. Let $d>0$
be an integer. Set
   $$t_d(c)=(Z,W,{\varphi}_1-{\varphi}_{m+n+1}^d,{\varphi}_2-{\varphi}_{m+n+1}^{d^2},\ldots,{\varphi}_{m+n}-{\varphi}_{m+n+1}^{d^{m+n}},{\varphi}_{m+n+1};g).$$
Note that
$Z({\varphi}_1-{\varphi}_{m+n+1}^d,{\varphi}_2-{\varphi}_{m+n+1}^{d^2},\ldots,{\varphi}_{m+n}-{\varphi}_{m+n+1}^{d^{m+n}},{\varphi}_{m+n+1})=Z({\varphi}_1,\ldots,{\varphi}_{m+n},{\varphi}_{m+n+1})$
in $W$. Thus the tuple $t_d(c)$ is an element in ${\operatorname{Fr}}_n(U,X_+\wedge
T^{n+1})$ such that its support $Z$ is the same with that of $c$. If
$s$ is a variable, we put
   $$h^d(c)=(Z\times {\mathbb{A}}^1,W\times {\mathbb{A}}^1,{\varphi}_1-s{\varphi}_{m+n+1}^d,{\varphi}_2-s{\varphi}_{m+n+1}^{d^2},\ldots,{\varphi}_{m+n}-s{\varphi}_{m+n+1}^{d^{m+n}},{\varphi}_{m+n+1};g).$$
Note that
$Z({\varphi}_1-s{\varphi}_{m+n+1}^d,\ldots,{\varphi}_{m+n}-s{\varphi}_{m+n+1}^{d^{m+n}},{\varphi}_{m+n+1})=Z\times
{\mathbb{A}}^1$ in $W\times{\mathbb} A^1$. Thus the tuple $h^d(c)$ is an element in
${\operatorname{Fr}}_n(U\times{\mathbb} A^1,X_+\wedge T^{n+1})$ whose support equals
$Z\times {\mathbb{A}}^1$.

\begin{remark}\label{r:quasi-finite}{\rm
Let $c=(Z,W,{\varphi};g)\in {\operatorname{Fr}}_n(U,X_+\wedge T^{n+1})$ and let $\pi:
W\to U$ be the composite map $U\xleftarrow{pr_U} U\times
{\mathbb{A}}^{m+n+1}\leftarrow W$. Then the map $({\varphi},\pi)\colon
W\to{\mathbb{A}}^{m+n+1}\times U$ is quasi-finite over $0\times U$. Hence it
is quasi-finite over some Zariski open neighborhood $V$ of $0\times
U$. Then $W'=({\varphi},\pi)^{-1}(V)$ is a Zariski open neighborhood of
$Z$ in $W$. Replacing $W$ by $W'$ if necessary, we may and shall
always assume in what follows that $({\varphi},\pi): W\to
{\mathbb{A}}^{m+n+1}\times U$ is quasi-finite.

}\end{remark}

\begin{lemma}\label{l:homotopy_properties}
If $U\in AffSm/k$ then the following statements are true:
\begin{itemize}
\item [(i)] if $c \in {\operatorname{Fr}}^{<d}_m(U,X_+\wedge T^{n+1})$, then $t_d(c) \in {\operatorname{Fr}}^{qf}_m(U,X_+\wedge T^{n+1})$;
\item [(ii)] if $c \in ({\operatorname{Fr}}^{qf}_m)^{<d}(U,X_+\wedge T^{n+1})$, then $h^d_s(c) \in {\operatorname{Fr}}^{qf}_m(U\times {\mathbb{A}}^1,X_+\wedge
T^{n+1})$.
\end{itemize}
\end{lemma}

\begin{proof}
Prove the first assertion. Let $c=(Z,W,{\varphi};g)\in {\operatorname{Fr}}_m(U,X_+\wedge
T^{n+1}) \in {\operatorname{Fr}}^{<d}_m(U,X_+\wedge T^{n+1})$. Let $F_1,\ldots F_r
\in k[{\mathbb{A}}^{m+n+1}_U]$  be a $c$-defining set with $\deg F_i<d$ for
all $i=1,\ldots,r$. We must check that $t_d(c)$ is in
${\operatorname{Fr}}^{qf}_m(U,X_+\wedge T^{n+1})$. So, take
$$Y=Z({\varphi}_1-{\varphi}_{m+n+1}^d,{\varphi}_2-{\varphi}_{m+n+1}^{d^2},\ldots,{\varphi}_{m+n}-{\varphi}_{m+n+1}^{d^{m+n}})\subset W.$$
We must check that for any point $u\in U$ the fiber $Y(u)$ of $Y$
over $u$ is finite. Let $\theta: {\mathbb{A}}^1\to {\mathbb{A}}^{m+n+1}$ be a morphism
taking a point $t$ to the point $(t^d,t^{d^2},...,t^{d^{m+n}},t)$.
It is a closed embedding with the image $C=\theta({\mathbb{A}}^1)$.

By Remark \ref{r:quasi-finite} the morphism $\psi=({\varphi},\pi):W\to
{\mathbb{A}}^{m+n+1}\times U$ is quasi-finite. Given a point $u\in U$ there is
a polynomial $F$ from the $c$-difining set such that $F(-,u)$ is
non-zero and its vanishing locus $Z(F(-,u))$ in ${\mathbb{A}}^{m+n+1}_u$
contains $\psi(W(u))$. Clearly, $Y(u)$ is contained in the set
   $$\psi^{-1}(Z(F(-,u))\cap C).$$
The set $Z(F(-,u))\cap C$ is in a bijection with the vanishing locus
of the polynomial $F(t^d,t^{d^2},...,t^{d^{m+n}},t)$ on the line
${\mathbb{A}}^1$ with the coordinate $t$. Thus by Lemma~\ref{lem:tnonzero} the
set $Z(F(-,u))\cap C$ is finite, and hence so is $Y(u)$. The first
assertion is proved.

Let us verify the second assertion. Let $c=(Z,W,{\varphi};g) \in
({\operatorname{Fr}}^{qf}_m)^{<d}(U,X\times T^{n+1})$ and let $F_1,\ldots F_r \in
k[{\mathbb{A}}^{m+n+1}_U]$  be a $c$-defining set with $\deg F_i<d$ for all
$i=1,\ldots,r$. We must check that $h^d(c)$ is in
${\operatorname{Fr}}^{qf}_m(U\times {\mathbb{A}}^1,X\times T^{n+1})$. So, take
   $$Y_s=Z({\varphi}_1-s{\varphi}_{m+n+1}^d,{\varphi}_2-s{\varphi}_{m+n+1}^{d^2},\ldots,{\varphi}_{m+n}-s{\varphi}_{m+n+1}^{d^{m+n}})\subset W\times {\mathbb{A}}^1,$$
where $s$ is the coordinate on the additional factor ${\mathbb{A}}^1$. We must
check that for any point $v\in U\times {\mathbb{A}}^1$ the fiber $Y_s(v)$ of
$Y_s$ over $v$ is finite. Replacing the base field $k$ by its
algebraic closure $\bar k$, we may assume that any point $v\in
U\times {\mathbb{A}}^1$ is of the form $(u,a)$ with $a\in \bar k$. So, we must
check that for any point $(u,a)\in U\times {\mathbb{A}}^1$ the fiber $Y_a(u))$
of $Y_s$ over $(u,a)$ is finite.

Given $0\neq a\in \bar k$, let $\theta_a: {\mathbb{A}}^1\to {\mathbb{A}}^{m+n+1}$ be a
morphism taking a point $t$ to the point
$(at^d,at^{d^2},...,at^{d^{m+n}},t)$. It is a closed embedding with
the image $C_a=\theta_a({\mathbb{A}}^1)$. For $a=0$ let $\theta_0: {\mathbb{A}}^1\to
{\mathbb{A}}^{m+n+1}$ be the morphism taking a point $t$ to $(0,0,...,0,t)$.
It is a closed embedding with the image $C_0=\theta_0({\mathbb{A}}^1)$. It is
the last coordinate line ${\mathbb{A}}^1_{m+n+1}$ in ${\mathbb{A}}^{m+n+1}$.

By Remark~\ref{r:quasi-finite} the morphism $\psi=({\varphi},\pi):W\to
{\mathbb{A}}^{m+n+1}\times U$ is quasi-finite. Given a point $u\in U$, there
is a polynomial $F$ from the $c$-defining set such that $F(-,u)$ is
non-zero and its vanishing locus $Z(F(-,u))$ in ${\mathbb{A}}^{m+n+1}_u$
contains $\psi(W(u))$. For a given $0\neq a\in \bar k$ the set
$Y_a(u)$ is contained in the set
   $$\psi^{-1}(Z(F(-,u))\cap C_a).$$
The set $Z(F(-,u))\cap C_a)$ is in a bijection with the vanishing
locus of the polynomial $F(at^d,at^{d^2},...,at^{d^{m+n}},t)$ on the
line ${\mathbb{A}}^1$ with the coordinate $t$. Thus by
Lemma~\ref{lem:tnonzero} the set $Z(F(-,u))\cap C_a)$ is finite in
this case, and hence so is $Y_a(u)$.

For $a=0$, the set $Y_0(u)$ coincides with the closed subset
$Z({\varphi}_1,...,{\varphi}_{m+n})$ in $W$. It is quasi-finite over $U$,
because $c=(Z,W,{\varphi};g) \in ({\operatorname{Fr}}^{qf}_m)^{<d}(U,X_+\wedge T^{n+1})$.
The second assertion is proved.
\end{proof}

Lemma~\ref{l:homotopy_properties} implies that the assignment
$c\mapsto t_d(c)$ gives a morphism of presheaves of pointed sets
   $$t_d: {\operatorname{Fr}}^{<d}_m(-,X_+\wedge T^{n+1}) \to {\operatorname{Fr}}^{qf}_m(-,X_+\wedge T^{n+1})$$
on $AffSm/k$. It also implies that the assignment $c\mapsto h^d(c)$
gives a morphism of presheaves of pointed sets
   $$h_d: {\operatorname{Fr}}^{<d}_m(-,X_+\wedge T^{n+1}) \to {\operatorname{Fr}}^{qf}_m(-\times {\mathbb{A}}^1,X_+\wedge T^{n+1})$$
on $AffSm/k$. Finally, the assignment $c\mapsto h_d(c)$ gives a
morphism of presheaves of pointed sets
   $$h_{d}^{qf}:({\operatorname{Fr}}^{qf}_m)^{<d}(-,X_+\wedge T^{n+1}) \to {\operatorname{Fr}}^{qf}_m(-\times {\mathbb{A}}^1,X_+\wedge T^{n+1})$$
on $AffSm/k$.

Consider a diagram
$$\xymatrix{{\operatorname{Fr}}^{<d}_m(-,X_+\wedge T^{n+1}) \ar[rrr]^(.4){i_d}  \ar[rrrd]^{t_d} &&& {\operatorname{Fr}}_m(-,X_+\wedge T^{n+1}) \\
              ({\operatorname{Fr}}^{qf}_m)^{<d}(-,X_+\wedge T^{n+1}) \ar[u]^{in_d}  \ar[rrr]^(.35){j_d} &&& {\operatorname{Fr}}^{qf}_m(-,X_+\wedge T^{n+1}) \ar[u]^{in} }$$
of presheaves of pointed sets on $AffSm/k$.
Lemma~\ref{l:homotopy_properties} shows that $h_d:
{\operatorname{Fr}}^{<d}_m(-,X_+\wedge T^{n+1}) \to
\underline\operatorname{Hom}({\mathbb{A}}^1,{\operatorname{Fr}}_m(-,X_+\wedge T^{n+1}))$ is an
${\mathbb{A}}^1$-homotopy between the morphisms $i\circ t_d$ and $i_d$. It
also shows that $h_{d}^{qf}: ({\operatorname{Fr}}^{qf}_m)^{<d}(-,X_+\wedge T^{n+1})
\to \underline\operatorname{Hom}({\mathbb{A}}^1,{\operatorname{Fr}}^{qf}_m(-,X_+\wedge T^{n+1}))$ is an
${\mathbb{A}}^1$-homotopy between the morphisms $t_d\circ in_d$ and $j_d$.

Applying the free abelian group functor to the morphisms
$t_d$,$i_d$,$j_d$,$in_d$ and $in$, we get certain morphisms between
presheaves of abelian groups as well as two ${\mathbb{A}}^1$-homotopies
(namely, $\mathbb Z [t_d]$, $\mathbb Z [i_d]$, $\mathbb Z [j_d]$,
$\mathbb Z [in_d]$, $\mathbb Z [in]$, $\mathbb Z [h_d]$ and $\mathbb
Z [h_d^{qf}]$). Note that these morphisms and these two homotopies
respect the additivity relations. Thus following
Definition~\ref{d:filtration_on_ZF_qf_m}, we finally get morphisms
$I_d$,$J_d$,$In_d$,$In$ and a morphism of presheaves
   $$T_d: {\operatorname{\mathbb{Z}F}}^{<d}_m(-,X_+\wedge T^{n+1}) \to {\operatorname{\mathbb{Z}F}}^{qf}_m(-,X_+\wedge T^{n+1}),$$
and two ${\mathbb{A}}^1$-homotopies $H_d$, $H_d^{qf}$. In this way we get a
diagram
   $$\xymatrix{{\operatorname{\mathbb{Z}F}}^{<d}_m(-,X_+\wedge T^{n+1}) \ar[rrr]^(.4){I_d}  \ar[rrrd]^{T_d} &&& {\operatorname{\mathbb{Z}F}}_m(-,X_+\wedge T^{n+1}) \\
              ({\operatorname{\mathbb{Z}F}}^{qf}_m)^{<d}(-,X_+\wedge T^{n+1}) \ar[u]^{In_d}  \ar[rrr]^(.4){J_d} &&& {\operatorname{\mathbb{Z}F}}^{qf}_m(-,X_+\wedge T^{n+1}) \ar[u]^{In} }$$
of presheaves of abelian groups on $AffSm/k$.

We document these arguments as follows.

\begin{lemma}\label{l:two_homotopies}
The ${\mathbb{A}}^1$-homotopy $h_d$ yields an ${\mathbb{A}}^1$-homotopy
$$H_d: {\operatorname{\mathbb{Z}F}}^{<d}_m(-,X_+\wedge T^{n+1}) \to \underline\operatorname{Hom}({\mathbb{A}}^1,{\operatorname{\mathbb{Z}F}}_m(-,X_+\wedge T^{n+1}))$$
between $I\circ T_d$ and $I_d$. The ${\mathbb{A}}^1$-homotopy $h_{d}^{qf}$ yields an ${\mathbb{A}}^1$-homotopy
   $$H_{d}^{qf}: ({\operatorname{\mathbb{Z}F}}^{qf}_m)^{<d}(-,X_+\wedge T^{n+1}) \to \underline\operatorname{Hom}({\mathbb{A}}^1,{\operatorname{\mathbb{Z}F}}^{qf}_m(-,X_+\wedge T^{n+1}))$$
between $T_d\circ In_d$ and $J_d$.
\end{lemma}

\begin{proposition}\label{ZF_qf_and_ZF}
For any integers $m,n{\geqslant} 0$ and for any $d>0$ the morphism
$$In: C_*{\operatorname{\mathbb{Z}F}}^{qf}_m(-,X_+\wedge T^{n+1}) \to C_*{\operatorname{\mathbb{Z}F}}_m(-,X_+\wedge T^{n+1})$$
of complexes of presheaves of abelian groups is a section-wise quisi-isomorphism
on the category $AffSm/k$.
\end{proposition}

\begin{proof}
The functor $C_*$ converts ${\mathbb{A}}^1$-homotopies to simplicial homotopies.
Now the proposition follows from Lemma \ref{l:two_homotopies}
and Corollary \ref{cor:exhauting_2}.
\end{proof}

We are now in a position to prove Theorem~\ref{p:moving}

\begin{proof}[Proof of Theorem~\ref{p:moving}]
By Definitions~\ref{def:ZF_qf} and \ref{stab} the complexes
$C_*{\operatorname{\mathbb{Z}F}}^{qf}(-,X_+\wedge T^{n+1}))$ and $C_*{\operatorname{\mathbb{Z}F}}(-,X_+\wedge
T^{n+1}))$ are the colimits of complexes $C_*{\operatorname{\mathbb{Z}F}}^{qf}_m(-,X_+\wedge
T^{n+1}))$ and $C_*{\operatorname{\mathbb{Z}F}}_m(-,X_+\wedge T^{n+1}))$ over the suspension
morphisms $\Sigma$. The morphisms $In$ commute with the suspension
morphisms $\Sigma$, i.e. $\Sigma \circ In=In\circ \Sigma:
C_*{\operatorname{\mathbb{Z}F}}^{qf}_m(-,X_+\wedge T^{n+1})\to C_*{\operatorname{\mathbb{Z}F}}_{m+1}(-,X_+\wedge
T^{n+1})$. Proposition~\ref{ZF_qf_and_ZF} now completes the proof.
\end{proof}

\section{Proof of Theorem~\ref{cone}}\label{sec:cone}

To finish the paper, it remains to prove Theorem~\ref{cone}. To this
end we need some definitions and Theorem~\ref{ZM_fr_and_LM_fr} stated in the Introduction.
Also, we follow notation of Section~\ref{s:Fr(X_T_n}.

Given $X\in Sm/k$, $n{\geqslant} 0$ and a finite pointed set $(K,*)$, put
$(X_+\wedge T^n)\otimes K:=(X\otimes K)_+\wedge T^n$.
Set $K'=K-*$. Clearly,
$(X_+\wedge T^n)\otimes K=(X\times K'\times {\mathbb{A}}^m)/(X\times K'\times {\mathbb{A}}^m-X\times K'\times \{0\})$.
For a $k$-smooth scheme $U$ and a finite pointed set $(K,*)$ set
\begin{equation}\label{eq:Fr_n_U_X_times_T_m_otimes_K}
{\operatorname{Fr}}_n(U,(X_+\wedge T^m)\otimes K):={\operatorname{Fr}}_n(U,(X\times K'\times {\mathbb{A}}^m)/(X\times K'\times {\mathbb{A}}^m-X\times K'\times \{0\})),
\end{equation}
where the right hand side is from Definition~\ref{def:FrY/Y-S}(III).
Also, set
\begin{equation*}\label{eq:ZF_n_U_X_times_T_m_otimes_K}
{\operatorname{\mathbb{Z}F}}_n(U,(X_+\wedge T^m)\otimes K):={\operatorname{\mathbb{Z}F}}_n(U,(X\times K'\times {\mathbb{A}}^m)/(X\times K'\times {\mathbb{A}}^m-X\times K'\times \{0\})),
\end{equation*}
where the right hand side is from Definition~\ref{stab}. Finally set
\begin{equation}\label{eq:F_n_U_X_times_T_m_otimes_K}
{\operatorname{F}}_n(U,(X_+\wedge T^m)\otimes K):={\operatorname{F}}_n(U,(X\times K'\times {\mathbb{A}}^m)/(X\times K'\times {\mathbb{A}}^m-X\times K'\times \{0\})),
\end{equation}
where the right hand side is from Definition~\ref{F_m_U_Y/(Y-S)}.
By Definition~\ref{F_m_U_Y/(Y-S)} the set ${\operatorname{F}}_n(U,(X_+\wedge T^m)\otimes K)-0_n$
is a free basis of the abelian group ${\operatorname{\mathbb{Z}F}}_n(U,(X_+\wedge T^m)\otimes K)$.

Denote by ${\mathbb}
Z{\operatorname{Fr}}_*^{S^1}(X_+\wedge T^n)$ the Segal $S^1$-spectrum
   $$({\operatorname{\mathbb{Z}Fr}}_*(-,X_+\wedge T^n),{\operatorname{\mathbb{Z}Fr}}_*(-,(X_+\wedge T^n)\otimes S^1),\ldots).$$
Also, let $EM({\operatorname{\mathbb{Z}F}}_*(-,X_+\wedge T^n))$ be the Segal $S^1$-spectrum
   $$({\operatorname{\mathbb{Z}F}}_*(-,X_+\wedge T^n),{\operatorname{\mathbb{Z}F}}_*(-,(X_+\wedge T^n)\otimes S^1),\ldots).$$
The equality ${\operatorname{\mathbb{Z}F}}_*(-,(X\sqcup X')_+\wedge T^n)={\operatorname{\mathbb{Z}F}}_*(-,X_+\wedge
T^n)\oplus {\operatorname{\mathbb{Z}F}}_*(-,X'_+\wedge T^n)$ shows that the $\Gamma$-space
$(K,*)\mapsto {\operatorname{\mathbb{Z}F}}_*(U,(X_+\wedge T^n)\otimes K)$ is fully determined
by the abelian group ${\operatorname{\mathbb{Z}F}}_*(U,X_+\wedge T^n)$. Hence
$EM({\operatorname{\mathbb{Z}F}}_*(-,X_+\wedge T^n))$ is the Eilenberg--Mac~Lane spectrum for
${\operatorname{\mathbb{Z}F}}_*(-,X_+\wedge T^n)$. The $\Gamma$-space morphism $[(K,*)\mapsto
{\operatorname{\mathbb{Z}Fr}}_*(-,(X_+\wedge T^n)\otimes K)]\to [(K,*)\mapsto
{\operatorname{\mathbb{Z}F}}_*(-,(X_+\wedge T^n)\otimes K)]$ induces a morphism of framed
$S^1$-spectra
   $$\lambda_{X_+\wedge T^n}:{\mathbb} Z{\operatorname{Fr}}_*^{S^1}(X_+\wedge T^n)\to EM({\operatorname{\mathbb{Z}F}}_*(-,X_+\wedge T^n)).$$
Also, denote by ${\mathbb} ZM_{fr}(X\times T^n)$, $X\in Sm/k$, the Segal
$S^1$-spectrum
   \begin{equation}\label{eq:ZMFr}
   (C_*{\operatorname{\mathbb{Z}Fr}}(-,X_+\wedge T^n),C_*{\operatorname{\mathbb{Z}Fr}}(-,(X_+\wedge T^n)\otimes S^1),\ldots).
   \end{equation}
Let $LM_{fr}(X\times T^n)$ be the Segal $S^1$-spectrum
   $$EM({\operatorname{\mathbb{Z}F}}(\Delta^\bullet\times-,X_+\wedge T^n))=({\operatorname{\mathbb{Z}F}}(\Delta^\bullet\times-,X_+\wedge T^n),{\operatorname{\mathbb{Z}F}}(\Delta^\bullet\times-,(X_+\wedge T^n)\otimes S^1),\ldots).$$
The above arguments show that $LM_{fr}(X\times T^n)$ is the
Eilenberg--Mac~Lane spectrum associated with the complex
${\operatorname{\mathbb{Z}F}}(\Delta^\bullet\times-,X_+\wedge T^n)$.
The $\Gamma$-space morphism 
$$[(K,*)\mapsto
{\operatorname{\mathbb{Z}Fr}}(\Delta^\bullet\times-,(X_+\wedge T^n)\otimes K)]\to [(K,*)\mapsto
{\operatorname{\mathbb{Z}F}}(\Delta^\bullet\times-,(X_+\wedge T^n)\otimes K)]$$
induces a morphism of framed
$S^1$-spectra 
   $$l_{X\times T^n}: {\mathbb} ZM_{fr}(X\times T^n)\to LM_{fr}(X\times T^n).$$
Note that stable homotopy groups of $LM_{fr}(X\times
T^n)=EM({\operatorname{\mathbb{Z}F}}(\Delta^\bullet\times-,X_+\wedge T^n))$ are equal to
homology groups of the complex ${\operatorname{\mathbb{Z}F}}(\Delta^\bullet\times-,X_+\wedge
T^n)$. By~\cite[\S II.6.2]{Sch} homotopy groups $\pi_*({\mathbb}
ZM_{fr}(X\times T^n)(U))$ of ${\mathbb} ZM_{fr}(X\times T^n)$ evaluated at
$U\in Sm/k$ are homology groups $H_*(M_{fr}(X\times T^n)(U))$ of
$M_{fr}(X\times T^n)(U)$.

For the convenience of the reader we recall Theorem~\ref{ZM_fr_and_LM_fr}
proved in Appendix~B. It computes, in particular, homology of
the framed motives $M_{fr}(X\times T^n)$ of the relative motivic spheres $X_+\wedge T^n$.

\begin{theorem*}
For any integer $m{\geqslant} 0$, the natural morphism of framed $S^1$-spectra
   $$\lambda_{X_+\wedge T^m}:{\mathbb} Z{\operatorname{Fr}}_*^{S^1}(X_+\wedge T^m)\to EM({\operatorname{\mathbb{Z}F}}_*(-,X_+\wedge T^m))$$
is a schemewise stable equivalence. Moreover,
the natural morphism of framed $S^1$-spectra
   $$l_{X\times T^m}: {\mathbb} ZM_{fr}(X\times T^m)\to LM_{fr}(X\times T^m)$$
is a schemewise stable equivalence. In particular, for any $U\in
Sm/k$ one has
   $$\pi_*({\mathbb} ZM_{fr}(X\times T^m)(U))=H_*({\operatorname{\mathbb{Z}F}}(\Delta^\bullet\times U,X_+\wedge T^m))=H_*(C_*{\mathbb} Z{\operatorname{F}}(U,X_+\wedge T^m)).$$
\end{theorem*}

By~\cite[Def.~9.1]{GP1} the framed motive of the pointed sheaf
$X_+\wedge T^n$ is the Segal $S^1$-spectum
   $$M_{fr}(X\times T^n)=(C_*{\operatorname{Fr}}_{\bullet}(-,X_+\wedge T^n), C_*{\operatorname{Fr}}_{\bullet}(-,(X_+\wedge T^n)\otimes S^1),C_*{\operatorname{Fr}}_{\bullet}(-,(X_+\wedge T^n)\otimes S^2),...).$$
As explained in the proof of \cite[9.2]{GP1} there is a natural
bijection
   $${\operatorname{Fr}}_m(U,X_+\wedge T^n)\to {\operatorname{Fr}}_{m,\bullet}(U,X_+\wedge T^n):=\operatorname{Hom}_{Shv^{nis}_\bullet(Sm/k)}(U_+\wedge({\mathbb} P^1,\infty)^{\wedge m},X_+\wedge T^n\wedge T^m).$$
Thus $M_{fr}(X\times T^n)$ is the Segal $S^1$-spectrum of the form
   $$(C_*{\operatorname{Fr}}_{}(-,X_+\wedge T^n), C_*{\operatorname{Fr}}_{}(-,(X_+\wedge T^n)\otimes S^1),C_*{\operatorname{Fr}}_{}(-,(X_+\wedge T^n)\otimes S^2),...)$$
with the presheaves ${\operatorname{Fr}}_{}(-,X_+\wedge T^n)$ defined in Section
\ref{s:Fr(X_T_n}. In particular, ${\mathbb} ZM_{fr}(X\times T^n)$ is
exactly the Segal $S^1$-spectum (\ref{eq:ZMFr}) defined above in the
present section.

We are now in a position to prove the remaining Theorem~\ref{cone}.

\begin{proof}[Proof of Theorem~\ref{cone}]
By Theorem~\ref{th:Main} the map of complexes of presheaves of
abelian groups
\begin{equation}\label{eq: C_*_A1/Gm}
C_*{\mathbb} Z{\operatorname{F}}(X_+\wedge T^{n}\wedge {\mathbb{A}}^1_+)/C_*{\mathbb} Z{\operatorname{F}}(X_+\wedge
T^{n}\times  {{\mathbb{G}_m}}_+) \to C_*{\mathbb} Z{\operatorname{F}}(X_+\wedge T^{n+1})
\end{equation}
is a local quasi-isomorphism. The $S^1$-spectra $LM_{fr}(X\times
T^{n}\times {\mathbb{A}}^1)$, $LM_{fr}(X\times T^{n}\times  {{\mathbb{G}_m}})$ and
$LM_{fr}(X\times T^{n+1})$ are the Eilenberg-Maclane $S^1$-spectra
of the complexes $C_*{\mathbb} Z{\operatorname{F}}(X_+\wedge T^{n}\wedge {\mathbb{A}}^1_+)$, $C_*{\mathbb}
Z{\operatorname{F}}(X_+\wedge T^{n}\wedge  {{\mathbb{G}_m}}_+)$ and $C_*{\mathbb} Z{\operatorname{F}}(X_+\wedge
T^{n+1})$ respectively. Thus the map
\begin{equation*}\label{eq: LM_fr_A1/Gm}
LM_{fr}(X\times T^{n}\times {\mathbb{A}}^1)/LM_{fr}(X\times T^{n}\times  {{\mathbb{G}_m}})
\to LM_{fr}(X\times T^{n+1}),
\end{equation*}
induced by~\eqref{eq: C_*_A1/Gm}, is a local stable weak
equivalence, and hence so is the map
   $${\mathbb} ZM_{fr}(X\times T^{n}\times {\mathbb{A}}^1)/{\mathbb} ZM_{fr}(X\times T^{n}\times  {{\mathbb{G}_m}}) \to {\mathbb} ZM_{fr}(X\times T^{n+1})$$
by Theorem~\ref{ZM_fr_and_LM_fr}. The $S^1$-spectra $M_{fr}(X\times T^{n}\times
{\mathbb{A}}^1)$, $M_{fr}(X\times T^{n}\times {{\mathbb{G}_m}})$, $M_{fr}(X\times T^{n+1})$
are connected. Now the stable Whitehead theorem~\cite[II.6.30]{Sch}
implies the map
\begin{equation}\label{eq: M_fr_A1/Gm}
M_{fr}(X\times T^{n}\times {\mathbb{A}}^1)/M_{fr}(X\times T^{n}\times  {{\mathbb{G}_m}}) \to M_{fr}(X\times T^{n+1})
\end{equation}
is a local stable weak equivalence. Consider the following sequence
of natural maps
\begin{multline*}
M_{fr}(X\times ({\mathbb{A}}^1\rfloor \mathbb G_m)^{\wedge (n+1)})=M_{fr}(X\times ({\mathbb{A}}^1\rfloor \mathbb G_m)^{\wedge n}\times ({\mathbb{A}}^1\rfloor \mathbb G_m))\xrightarrow{(1)} \\
Cone[M_{fr}(X\times ({\mathbb{A}}^1\rfloor \mathbb G_m)^{\wedge n}\times {{\mathbb{G}_m}} ) \to M_{fr}(X\times ({\mathbb{A}}^1\rfloor \mathbb G_m)^{\wedge n}\times {\mathbb{A}}^1)] \xrightarrow{(2)} \\
Cone[M_{fr}(X\times T^{n}\times {{\mathbb{G}_m}}) \to M_{fr}(X\times T^{n}\times {\mathbb{A}}^1)] \xrightarrow{(3)} \\
M_{fr}(X\times T^{n}\times {\mathbb{A}}^1)/M_{fr}(X\times T^{n}\times  {{\mathbb{G}_m}}) \xrightarrow{(4)}
M_{fr}(X\times T^{n+1}).
\end{multline*}
The arrows $(1)$ and $(3)$ are sectionwise stable weak equivalences
by standard reasons. The arrow $(2)$ is a local stable weak
equivalence by induction. The arrow $(4)$ is exactly the
map~\eqref{eq: M_fr_A1/Gm}, and so it is a local stable weak
equivalence.

Hence for any $\ell{\geqslant} 1$ the canonical morphism
\begin{equation}\label{eq:cone_and_T}
M_{fr}(X\times ({\mathbb{A}}^1\rfloor \mathbb G_m)^{\wedge\ell})\to
M_{fr}(X\times T^{\ell})
\end{equation}
is a local stable weak equivalence. By~\cite[9.2]{GP1} the
simplicial $S^1$-spectrum $M_{fr}(X\times T^{\ell})(U)$ with $U$ a
local Henselian smooth scheme and $X$ any smooth scheme is a
connected $\Omega$-spectrum. The proof of~\cite[9.2]{GP1} works
equally for the $\Gamma$-space $K \to C_*{\operatorname{Fr}}(U,(X\times ({\mathbb{A}}^1\rfloor
\mathbb G_m)^{\wedge \ell}) \otimes K)$. Thus the latter
$\Gamma$-space is special. By Lemma~\ref{l:connectivity} the zero
space $C_*{\operatorname{Fr}}(-,X\times ({\mathbb{A}}^1\rfloor\mathbb G_m)^{\wedge \ell})$ of
the Segal $S^1$-spectrum $M_{fr}(X\times ({\mathbb{A}}^1\rfloor \mathbb
G_m)^{\wedge\ell})(U)$ is connected. Thus the Segal $S^1$-spectrum
$M_{fr}(X\times ({\mathbb{A}}^1\rfloor \mathbb G_m)^{\wedge\ell})(U)$ is a
connected $\Omega$-spectrum. Hence the
morphism~\eqref{eq:cone_and_T} is a local level weak equivalence.

Finally, using Theorem~\ref{ZM_fr_and_LM_fr}, the assertion that the sequence
of $S^1$-spectra
   $$M_{fr}(X \times T^n \times \mathbb G_m) \to M_{fr}(X \times T^n \times{\mathbb} A^1) \to M_{fr}(X \times T^{n+1})$$
is locally a homotopy cofiber sequence in the Nisnevich topology
reduces to the assertion that the sequence of complexes of linear
framed presheaves
   $$C_*{\operatorname{\mathbb{Z}F}}(X_+\wedge T^n \wedge {{\mathbb{G}_m}}_+)\to C_*{\operatorname{\mathbb{Z}F}}(X_+\wedge T^n \wedge {\mathbb{A}}^1_+)\to C_*{\operatorname{\mathbb{Z}F}}(X_+\wedge T^{n+1})$$
is locally a homotopy cofiber sequence. The latter follows from
Theorem~\ref{th:Main}.
\end{proof}

\begin{corollary}\label{cormain}
For every $n{\geqslant} 0$, the natural morphism
$$M_{fr}(X\times T^n\times({\mathbb{A}}^1\rfloor \mathbb G_m))\to M_{fr}(X\times T^{n+1})$$
is locally a level weak equivalence of $S^1$-spectra in the Nisnevich topology.
\end{corollary}

\begin{proof}
Consider a commutative diagram
   $$\xymatrix{M_{fr}(X\times({\mathbb{A}}^1\rfloor\mathbb G_m)^{\wedge n+1})\ar[d]\ar[r]&M_{fr}(X\times T^{n+1})\ar@{=}[d]\\
               M_{fr}(X\times T^{n}\times({\mathbb{A}}^1\rfloor\mathbb G_m))\ar[r]&M_{fr}(X\times T^{n+1})}$$
The left vertical and upper horizontal arrows are locally level weak
equivalences of $S^1$-spectra by Theorem~\ref{cone}, and hence so is
the lower horizontal arrow.
\end{proof}

\appendix\section{}

In this section we prove the following useful

\begin{lemma}\label{l:connectivity}
For any $X\in Sm/k$ and any $n>0$ the simplicial pointed presheaves
$C_*{\operatorname{Fr}}(-,X\times ({\mathbb{A}}^1\rfloor\mathbb G_m)^{\wedge n})$ and
$C_*{\operatorname{Fr}}(-,X_+\wedge T^n)$ are locally connected in the Nisnevich
topology.
\end{lemma}

\begin{proof}
Firstly check the Nisnevich local connectivity of $C_*{\operatorname{Fr}}(-,X\times
({\mathbb{A}}^1\rfloor\mathbb G_m))$. Clearly, the map $\pi_0(C_*{\operatorname{Fr}}(-,X\times
{\mathbb{A}}^1))\to\pi_0(C_*{\operatorname{Fr}}(-,X\times({\mathbb{A}}^1\rfloor\mathbb G_m)))$ is
surjective. On the other hand the composite map of pointed sets
   $$\pi_0(C_*{\operatorname{Fr}}(-,X\times {{\mathbb{G}_m}}))\to\pi_0(C_*{\operatorname{Fr}}(-,X\times{\mathbb{A}}^1))\to\pi_0(C_*{\operatorname{Fr}}(-,X\times({\mathbb{A}}^1\rfloor\mathbb G_m)))$$
is constant, because it factors through the pointed set
$\pi_0(C_*{\operatorname{Fr}}(-,(X\times {{\mathbb{G}_m}})\otimes I))=*$. Thus it is sufficient
to check that for any local essentially $k$-smooth Henselian $U$ the
map
\begin{equation}\label{eq:connectivity}
\pi_0(C_*{\operatorname{Fr}}(U,X\times {{\mathbb{G}_m}}))  \to  \pi_0(C_*{\operatorname{Fr}}(U,X\times {\mathbb{A}}^1))
\end{equation}
is surjective. Take a framed correspondence
   $$c_0=(Z,W, {\varphi}; (f,g):W\to X\times {\mathbb{A}}^1)\in {\operatorname{Fr}}_n(U,X\times {\mathbb{A}}^1).$$
We may and will assume that $W=({\mathbb{A}}^n_U)^h_Z$ is the henselization of
${\mathbb{A}}^n_U$ at the closed subset $Z$. We want to find $h_t\in
{\operatorname{Fr}}_n({\mathbb{A}}^1_U,{\mathbb{A}}^1_k)$ and $c_1\in {\operatorname{Fr}}_n(U,X\times {{\mathbb{G}_m}}_{,k})$ such
that $h_0=c_0$, $h_{1}=j \circ c_1$, where $j:X\times {{\mathbb{G}_m}}
\hookrightarrow X\times {\mathbb{A}}^1_k$  is the open embedding. {\it
Firstly, suppose the residue field of $U$ at its closed point $u$ is
finite over the ground field $k$}. Let $Z_u$ be the fiber of $Z$
over $u$. Since $Z$ is finite over $U$  the scheme $Z_u$ is finite
over $pt:=\operatorname{Spec}(k)$.

Hence one can take a $k$-rational point $a$ from the open subset
${\mathbb{A}}^1_k-g(Z_u)- \{0\}$ of the affine line ${\mathbb{A}}^1_k$. Then $(g-a)(Z_u)
\subset {{\mathbb{G}_m}}_{,k}$ and $(g-a)(W)\subset {{\mathbb{G}_m}}_{,k}$, hence
$(f,(g-a))(W)\subset X\times {{\mathbb{G}_m}}_{,k}$ as well. Thus
$c_a:=(Z,W,{\varphi}; (f,(g - a)))\in {\operatorname{Fr}}_n({{\rm pt}},X\times {{\mathbb{G}_m}}_{,k})$. Set
$$h_t=(Z\times {\mathbb{A}}^1,W\times {\mathbb{A}}^1,{\varphi};  (f,(g - t)))\in {\operatorname{Fr}}_n({\mathbb{A}}^1,X\times {\mathbb{A}}^1_k).$$
Clearly, $h_0=c_0$, $h_{a}=j \circ c_a$. Hence the arrow
(\ref{eq:connectivity}) is surjective provided that the field $k(u)$
is finite over the ground field $k$.

Note that for any field extension $K/k$, any $k$-smooth scheme $X$
and any $K$-smooth scheme $V$ one has an adjunction
${\operatorname{Fr}}(V,X)={\operatorname{Fr}}(V,X_K)$. Using this adjunction, we may always assume
that the residue field $k(u)$ is finite over the ground field. Thus
the arrow (\ref{eq:connectivity}) is surjective and
$C_*{\operatorname{Fr}}(-,X\times ({\mathbb{A}}^1\rfloor\mathbb G_m))$ is locally connected.

By induction, suppose $C_*{\operatorname{Fr}}(-,X\times ({\mathbb{A}}^1\rfloor\mathbb
G_m)^{\wedge n})$ is locally connected. Then $C_*{\operatorname{Fr}}(-,X\times
({\mathbb{A}}^1\rfloor\mathbb G_m)^{\wedge(n+1)})$ is the realization of a
simplicial space of the form
   $$[r]\mapsto C_*{\operatorname{Fr}}(-,Y_r\times({\mathbb{A}}^1\rfloor\mathbb G_m)^{\wedge n})$$
with $Y_r\in Sm/k$. Since each $C_*{\operatorname{Fr}}(-,Y_r\times
({\mathbb{A}}^1\rfloor\mathbb G_m)^{\wedge n})$ is locally connected by the
induction hypothesis, then $C_*{\operatorname{Fr}}(-,X\times ({\mathbb{A}}^1\rfloor\mathbb
G_m)^{\wedge(n+1)})$ is locally connected as well.

Now let us prove the Nisnevich local connectivity of
$C_*{\operatorname{Fr}}(-,X_+\wedge T^n)$. We give a proof for $n=1$. The general
case is treated similarly. For any local essentially $k$-smooth
Henselian $U$ consider a framed correspondence
   $$b_0=(Z,W, {\varphi}_1,...,{\varphi}_n,{\varphi}_{n+1}; f:W\to X)\in {\operatorname{Fr}}_n(U,X_+\wedge T).$$
It is sufficient to find $h_t\in {\operatorname{Fr}}_n({\mathbb{A}}^1_U,X_+\wedge T)$ such
that $h_0=b_0$ and $h_{1}=0_n$ is the empty framed correspondence.
We may and will assume that $W=({\mathbb{A}}^n_U)^h_Z$ is the henselization of
${\mathbb{A}}^n_U$ at the closed subset $Z$. As above first suppose the
residue field of $U$ at its closed point $u$ is finite over the
ground field $k$. Let $Z_u$ be the fiber of $Z$ over $u$. Take a
$k$-rational point $a$ from the open subset ${\mathbb{A}}^1_k-{\varphi}_{n+1}(Z_u)-
\{0\}$ of the affine line ${\mathbb{A}}^1_k$. Then $({\varphi}_{n+1}-a)(Z_u)
\subset {{\mathbb{G}_m}}_{,k}$. Set
   $$h_t=(Z\times {\mathbb{A}}^1,W\times {\mathbb{A}}^1,{\varphi}_1,...,{\varphi}_{n},{\varphi}_{n+1}-t;f:W\to X)\in {\operatorname{Fr}}_n({\mathbb{A}}^1_U,X_+\wedge T).$$
Clearly, $h_0=b_0$, $h_{a}=0_n$.

Using the above adjunction ${\operatorname{Fr}}(V,Y)={\operatorname{Fr}}(V,Y_K)$, we may always
assume that the residue field $k(u)$ is finite over the ground
field. Thus $C_*{\operatorname{Fr}}(-,X_+\wedge T)$ is locally connected. The same
is true for $C_*{\operatorname{Fr}}(-,X_+\wedge T^n)$, hence the lemma.
\end{proof}

\section{}

The main goal of this section is to prove Theorem~\ref{ZM_fr_and_LM_fr}.
It will be proved at the end of the section.

Let ${\mathbb} S$ be the sphere $S^1$-spectrum. Let $* \subset {\mathbb} S$ be
its trivial $S^1$-subspectum corresponding to the basepoint. Let $\mathcal A$ be a pointed set
with a distinguished point $*$. Denote by ${\mathbb} S_{\mathcal A}$ the $S^1$-spectrum
$\prod_{(\mathcal A-*)}{\mathbb} S$. Let ${\mathbb} S'_{\mathcal A}$ be the
$S^1$-subspectrum $\vee_{(\mathcal A-*)}{\mathbb} S$ in ${\mathbb} S_{\mathcal A}$.

Given a finite pointed subset $A\subset \mathcal A$, let ${\mathbb} S_{A}\subset {\mathbb}
S_{\mathcal A}$ be an $S^1$-subspectrum of the form $\prod_{\mathcal
(A-*)}E_a$, where $E_a={\mathbb} S$ if $a\in A-*$ and $E_a=*$ if $a\in \mathcal
A-A$. If $a\in A-*$ we shall write ${\mathbb} S_a$ to denote ${\mathbb} S_{\{a,*\}}$, where
$\{a,*\}\subset A$ is the two elements subset of $A$. Let ${\mathbb}
S'_A\subset {\mathbb} S_A$ be the $S^1$-subspectrum $\vee_{a\in (A-*)} {\mathbb}
S_a$ in ${\mathbb} S_A$. Clearly, the inclusion ${\mathbb} S'_A\subset {\mathbb} S_A$
is a stable equivalence of $S^1$-spectra.
Set ${\mathbb} S^{f}_{\mathcal A}:=\cup_{A\subset \mathcal A}{\mathbb} S_A\subset {\mathbb} S_{\mathcal A}$,
where the union is taken over the set of all finite pointed subsets $A$ of the pointed set
$\mathcal A$.

The following lemma is straightforward and the proof is left to the reader.

\begin{lemma}\label{l:cup_S_A_and_cup_S'_A}
Let $\mathcal A$ be a pointed set and $I$ be a set such that
for any $i\in I$ there is a finite pointed subset $A(i)\subset \mathcal A$.
We have two $S^1$-subspectra $\cup_{i\in I}{\mathbb} S'_{A(i)}$,
$\cup_{i\in I}{\mathbb} S_{A(i)}$ of the spectrum ${\mathbb} S^f_{\mathcal A}$.
Suppose $\cup_{i\in I}A(i)=\mathcal A$. Then
$\vee_{(\mathcal A-*)}{\mathbb} S_a=\cup_{i\in I}{\mathbb} S'_{A(i)}$ and the inclusion
   $$\vee_{(\mathcal A-*)}{\mathbb} S_a=\cup_{i\in I}{\mathbb} S'_{A(i)} \hookrightarrow \cup_{i\in I}{\mathbb} S_{A(i)}$$
is a stable equivalence of $S^1$-spectra.
\end{lemma}

{\it An application of this lemma is given below in this section.}
For a finite pointed set $(K,*)$ consider a set
$Map^f_\bullet(\mathcal A,K)$
of those maps $\rho$ of pointed sets such that the set $\rho^{-1}(K-*)$ is finite.
Consider a $\Gamma$-space $\Gamma^f_{\mathcal A}$ defined by
$\Gamma^f_{\mathcal A}(K,*)=Map^f_\bullet(\mathcal A,K)$.
For a finite pointed subset $A\subset \mathcal A$ consider a subset
$Map^A_\bullet(\mathcal A,K)\subset Map^f_\bullet(\mathcal A,K)$
consisting of all maps $\rho$ such that $\rho^{-1}(K-*)\subset A$.
Consider a $\Gamma$-space $\Gamma_{A}$ defined by
$\Gamma_{A}(K,*)=Map^A_\bullet(\mathcal A,K)$.

Clearly, for any inclusion of finite pointed subsets $A'\subset A$ of $\mathcal A$ one has inclusions
$Map^{A'}_\bullet(\mathcal A,K)\subset Map^A_\bullet(\mathcal A,K)$
and $\Gamma_{A'}\subset \Gamma_{A}$. Moreover, one has
   $$\cup_{A\subset \mathcal A}Map^A_\bullet(\mathcal A,K)=Map^f_\bullet(\mathcal A,K) \ \text{and} \
\cup_{A\subset \mathcal A}\Gamma_{A}=\Gamma^f_{\mathcal A},$$
where the union is taken over the set of all pointed finite subsets $A$ in the pointed set
$\mathcal A$.

Let $A\subset \mathcal A$ be a finite pointed subset. For any element
$a\in A-*$ set $Map^a_\bullet(\mathcal A,K)=Map^{a\sqcup *}_\bullet(\mathcal A,K)$,
where $a\sqcup *$ stands for the two elements pointed subset of $\mathcal A$.
Set $\Gamma_a=\Gamma_{a\sqcup *}$. That is
$\Gamma_{a}(K,*)=Map^{a\sqcup *}_\bullet(\mathcal A,K)$.
Let $Map^{A,s}_\bullet(\mathcal A,K)\subset Map^A_\bullet(\mathcal A,K)$
consist of maps $\rho$
such that the subset $\rho^{-1}(K-*)\subset A$ either has one element or is the empty set.
Consider a $\Gamma$-subspace $\Gamma'_{A}\subset \Gamma_{A}$ such that
$\Gamma'_{A}(K,*)=Map^{A,s}_\bullet(\mathcal A,K)$.

The $\Gamma$-space $\Gamma_{A}$ is isomorphic to the $\Gamma$-space
$\prod_{a\in (A-*)}\Gamma_a$. The $\Gamma$-space $\Gamma'_{A}$ is isomorphic to the $\Gamma$-space $\vee_{a\in (A-*)}\Gamma'_a$. Moreover these isomorphisms are consistent with the inclusion $\vee_{a\in (A-*)}\Gamma_a \subset \prod_{a\in (A-*)}\Gamma_a$.

\begin{lemma}\label{l:cup_G_A_and_cup_G'_A}
Let $\mathcal A$, $I$ and $A(i)$ be as in Lemma
\ref{l:cup_S_A_and_cup_S'_A}.
There are two $\Gamma$-subspaces $\cup_{i\in I}\Gamma'_{A(i)}$,
$\cup_{i\in I}\Gamma_{A(i)}$ of the $\Gamma$-space $\Gamma^f_{\mathcal A}$.
Suppose $\cup_{i\in I}A(i)=\mathcal A$. Then
$\vee_{a \in (\mathcal A-*)}\Gamma_a=\cup_{i\in I}\Gamma'_{A(i)}$ and we have natural inclusions
$$\vee_{a\in (\mathcal A-*)}\Gamma_a=\cup_{i\in I}\Gamma'_{A(i)} \hookrightarrow \cup_{i\in I}\Gamma_{A(i)}$$
of $\Gamma$-spaces.
\end{lemma}

Let $Seg: \Gamma-spaces \to S^1-spectra$ be the functor associating the Segal $S^1$-spectrum
to a $\Gamma$-space. Then $Seg(\Gamma^f_{\mathcal A})={\mathbb} S^f_{\mathcal A}$.
Given a non-distinguished element $a\in \mathcal A$ one has $Seg(\Gamma_a)={\mathbb} S$
and the functor $Seg$ converts the inclusion
$\Gamma_a\subset \Gamma^f_{\mathcal A}$ to the inclusion
${\mathbb} S_a\subset {\mathbb} S^f_{\mathcal A}$.
For any finite pointed subset $A$ in $\mathcal A$, the functor $Seg$ converts the inclusion
$\Gamma_A\subset \Gamma^f_{\mathcal A}$
to the inclusion ${\mathbb} S_A\subset {\mathbb} S^f_{\mathcal A}$.
It also converts the inclusion $\Gamma'_A\subset \Gamma_A$
to the inclusion ${\mathbb} S'_A\subset {\mathbb} S_A$ as well as the inclusion
$\vee_{a \in (\mathcal A-*)}\Gamma_a\subset \Gamma^f_{\mathcal A}$
to the inclusion
$\vee_{a \in (\mathcal A-*)}{\mathbb} S_a\subset {\mathbb} S^f_{\mathcal A}$.

The above arguments together with Lemmas \ref{l:cup_S_A_and_cup_S'_A} and \ref{l:cup_G_A_and_cup_G'_A}
prove the following

\begin{lemma}\label{l:cup_Seg_A_and_cup_Seg'_A}
Let $\mathcal A$, $I$ and $A(i)$ be as in Lemma
\ref{l:cup_S_A_and_cup_S'_A}.
There are two $S^1$-subspectra $\cup_{i\in I}Seg(\Gamma'_{A(i)})$,
$\cup_{i\in I}Seg(\Gamma_{A(i)})$ of the $S^1$-spectrum $Seg(\Gamma^f_{\mathcal A})$.
Suppose $\cup_{i\in I}A(i)=\mathcal A$. Then
$\vee_{a \in (\mathcal A-*)}Seg(\Gamma_a)=\cup_{i\in I}Seg(\Gamma'_{A(i)})$ and the inclusion
$$\vee_{a\in (\mathcal A-*)}Seg(\Gamma_a)=\cup_{i\in I}Seg(\Gamma'_{A(i)}) \hookrightarrow \cup_{i\in I}Seg(\Gamma_{A(i)})$$
is a stable equivalence of $S^1$-spectra.
\end{lemma}

\begin{notation}\label{n:A_and_I_specific}{\rm
Let
$U,X\in Sm/k$ and let $m,n{\geqslant} 0$ be integers. Set $\mathcal A={\operatorname{F}}_m(U,X_+\wedge T^n)$
and regard it as a pointed set pointed by the empty framed correspondence $0_m$.
Set $I={\operatorname{Fr}}_m(U,X_+\wedge T^n)-0_m$.
}\end{notation}

In the remaining part of this section we use notation from Section \ref{s:Fr(X_T_n}.

\begin{definition}\label{def:A(Phi)}{\rm
Given $\Phi=(Z,W,{\varphi};g),\Phi'=(Z',W',{\varphi}';g')\in {\operatorname{Fr}}_m(U,X_+\wedge
T^n)$, we write $\Phi'{\leqslant} \Phi$ if there is a closed subset $Z''$
in ${\mathbb{A}}^m\times U$ such that $Z=Z'\sqcup Z''$ and
   $$(Z',W',{\varphi}';g')=(Z',W-Z'',{\varphi}|_{W-Z''};g|_{W-Z''}) \in {\operatorname{Fr}}_m(U,X_+\wedge T^n).$$
For any $\Phi\in {\operatorname{Fr}}_m(U,X_+\wedge T^n)$ set
$A(\Phi)=\{\Psi\in {\operatorname{F}}_m(U,X_+\wedge T^n): \Psi {\leqslant} \Phi \} \subset \mathcal A$.
Clearly,
\begin{equation*}\label{eq:A_and_A_Phi}
\mathcal A=\cup_{\Phi\in I} A(\Phi).
\end{equation*}
}
\end{definition}

For a finite pointed set $(K,*)$ the pointed set ${\operatorname{Fr}}_m(U,(X_+\wedge
T^n)\otimes K)$ is defined by the
formula~\eqref{eq:Fr_n_U_X_times_T_m_otimes_K}. Let $K'=K-*$. By
Definition~\ref{def:FrY/Y-S}(III) the set ${\operatorname{Fr}}_m(U,(X_+\wedge
T^n)\otimes K)$ consists of equivalence classes of tuples
$(Z,W,{\varphi};g;f)$, where $Z$ is a closed subset of $U\times{\mathbb} A^m$,
finite over $U$, $W$ is an \'{e}tale neighborhood of $Z$ in
$U\times{\mathbb} A^m$,
${\varphi}_1,\ldots,{\varphi}_{m},{\varphi}_{m+1},\ldots,{\varphi}_{m+n}$ are regular
functions on $W$, $(g,f): W\to X\times K'$ is a regular map such
that $Z=Z({\varphi}_1,\ldots,{\varphi}_{m+n})$. Notice that regular maps from
$W$ to $X\otimes K$ are in one-to-one correspondence with couples of
regular maps $(W\to X,W\to K')$.

For a finite pointed set $(K,*)$, the pointed set ${\operatorname{F}}_m(U,(X_+\wedge
T^n)\otimes K)$ is defined by the
formula~\eqref{eq:F_n_U_X_times_T_m_otimes_K}. By
Definition~\ref{F_m_U_Y/(Y-S)}
it consists of those elements
$(Z,W,{\varphi};g;f)\in {\operatorname{Fr}}_m(U,(X_+\wedge T^n)\otimes K)$ such that the
closed subset $Z$ of $U\times{\mathbb} A^m$ is connected.

\begin{definition}\label{def:Gamma_m_and_Gamma'_m}{\rm
Denote by $\Gamma_m(U,X_+\wedge T^n)$ the $\Gamma$-space $(K,*)\mapsto
{\operatorname{Fr}}_m(U,(X_+\wedge T^n)\otimes K)$. \\
Similarly, $\Gamma'_m(U,X_+\wedge T^n)$ stands
for the $\Gamma$-space $(K,*)\mapsto {\operatorname{F}}_m(U,(X_+\wedge T^n)\otimes
K)$.

Given $\Phi\in I$ define $\Gamma_m(U,X_+\wedge T^n)_{\Phi}$ as a
$\Gamma$-subspace of the $\Gamma$-space $\Gamma_m(U,X_+\wedge T^n)$
such that for a finite pointed set $(K,*)$
   $$\Gamma_m(U,X_+\wedge T^n)_{\Phi}(K)=\{(Z,W,{\varphi};g;f)\in {\operatorname{Fr}}_m(U,(X_+\wedge T^n)\otimes K)\mid (Z,W,{\varphi};g){\leqslant} \Phi \in {\operatorname{Fr}}_m(U,X_+\wedge T^n)\}.$$
Define $\Gamma'_m(U,X_+\wedge T^n)_{\Phi}$ as a $\Gamma$-subspace of
the $\Gamma$-space $\Gamma'_m(U,X_+\wedge T^n)$ such that for a finite
pointed set $(K,*)$
   $$\Gamma'_m(U,X_+\wedge T^n)_{\Phi}(K)=\{(Z,W,{\varphi};g;f)\in {\operatorname{F}}_n(U,(X_+\wedge T^n)\otimes K)\mid (Z,W,{\varphi};g){\leqslant} \Phi \in {\operatorname{F}}_m(U,X_+\wedge T^n)\}.$$

}\end{definition}

\begin{definition}{\rm
For a finite pointed set $(K,*)$ put $K'=K-*$ and consider a pointed
set map
   $$inc_{K}: {\operatorname{Fr}}_m(U,(X_+\wedge T^n)\otimes K)\to Map^f_{Sets_{\bullet}}(\mathcal A,K),$$
which is defined as follows. Let $\Psi=(Z,W,{\varphi};g;f)\in
{\operatorname{Fr}}_m(U,(X_+\wedge T^n)\otimes K)$ and $a=(Z_a,W_a,{\varphi}_a;g_a)\in
\mathcal A={\operatorname{F}}_m(U,X_+\wedge T^n)$. If the element $a$ is in
$\mathcal A-A((Z,W{\varphi};g))$, then the map $inc_{K}(\Psi)$ takes the
element $a$ to the distinguished point $*$ of the set $K$. If $a\in
A((Z,W,{\varphi};g))-0_m$, then the map $inc_{K}(\Psi)$ takes the element
$a$ to $f(Z_a)\in K'\subset K$. Finally, the map $inc_{K}(\Psi)$
sends $0_m$ to the distinguished point $*$ of the set $K$.

Recall that $Z_a$ is connected and if $a\in A((Z,W,{\varphi};g))$, then
$Z=Z_a\sqcup Z''$ for some $Z''$. Define a $\Gamma$-space morphism
$$inc_m: \Gamma_m(U,X_+\wedge T^n)\to \Gamma^f_{\mathcal A}$$
sending a finite pointed set $(K,*)$ to the pointed set map
$inc_{K}$. It is straightforward to check that it is indeed a
$\Gamma$-space morphism.

}\end{definition}

The following lemma is crucial.

\begin{lemma}\label{l:key}
The $\Gamma$-space morphism $inc_m$ is injective. Moreover, using
this inclusion the following identifications hold:

\begin{enumerate}
\item for any $\Phi\in I$, one has $\Gamma_m(U,X_+\wedge
T^n)_{\Phi}=\Gamma_{A(\Phi)}$ and $\cup_{\Phi\in
I}\Gamma_m(U,X_+\wedge T^n)_{\Phi}= \cup_{\Phi\in I}\Gamma_{A(\Phi)}$;
\item for any $\Phi\in I$, one has $\Gamma'_m(U,X_+\wedge
T^n)_{\Phi}=\Gamma'_{A(\Phi)}$ and $\cup_{\Phi\in
I}\Gamma'_m(U,X_+\wedge T^n)_{\Phi}=\cup_{\Phi\in
I}\Gamma'_{A(\Phi)}$;
\item for any $a\in \mathcal A-0_m={\operatorname{F}}_m(U,X_+\wedge T^n)-0_m$, one has $\Gamma_m(U,X_+\wedge T^n)_a=\Gamma_a$;
\item $\vee_{a\in (\mathcal A-*)}\Gamma_m(U,X_+\wedge T^n)_a=\vee_{a\in (\mathcal A-*)}\Gamma_a$.
\end{enumerate}
\end{lemma}

Applying the Segal functor $Seg$, we see that
Lemmas~\ref{l:cup_G_A_and_cup_G'_A}
and~\ref{l:cup_Seg_A_and_cup_Seg'_A} imply the following

\begin{proposition}
Let $\mathcal A={\operatorname{F}}_m(U,X_+\wedge T^n)$, $I={\operatorname{Fr}}_m(U,X_+\wedge
T^n)-0_m$ be as Notation~\ref{n:A_and_I_specific} and for $\Phi\in
I$ let the subset $A(\Phi)\subset \mathcal A$ be as in
Definition~\ref{def:A(Phi)}. There are two $S^1$-subspectra
$\cup_{i\in I}Seg(\Gamma'_m(U,X_+\wedge T^n)_{\Phi})$, $\cup_{i\in
I}Seg(\Gamma_m(U,X_+\wedge T^n)_{\Phi})$ of the $S^1$-spectrum
$Seg(\Gamma^f_{\mathcal A})$. One has an equality of the
$S^1$-subspectra
   $$\vee_{a \in (\mathcal A-*)}Seg(\Gamma_m(U,X_+\wedge T^n)_a)=\cup_{\Phi\in I}Seg(\Gamma'_m(U,X_+\wedge T^n)_{\Phi})$$
and the inclusion
\begin{multline*}
\vee_{a\in (\mathcal A-*)}Seg(\Gamma_m(U,X_+\wedge T^n)_a)=\cup_{\Phi\in I}Seg(\Gamma'_m(U,X_+\wedge T^n)_{\Phi})=Seg(\Gamma'_m(U,X_+\wedge T^n)) \hookrightarrow \\
\hookrightarrow \cup_{\Phi\in I}Seg(\Gamma_m(U,X_+\wedge T^n)_{A(\Phi)})=Seg(\Gamma_m(U,X_+\wedge T^n))
\end{multline*}
is a stable equivalence of the $S^1$-spectra.
\end{proposition}

Set,
   $${\operatorname{Fr}}^{S^1}_m(U,X_+\wedge T^n)=Seg(\Gamma_m(U,X_+\wedge T^n)) \ \text{and} \ {\operatorname{Fr}}^{S^1}_m(U,X_+\wedge T^n)_{\Phi}=Seg(\Gamma_m(U,X_+\wedge T^n)_{\Phi}),$$
   $${\operatorname{F}}^{S^1}_m(U,X_+\wedge T^n)=Seg(\Gamma'_m(U,X_+\wedge T^n)) \ \text{and} \  {\operatorname{F}}^{S^1}_m(U,X_+\wedge T^n)_{\Phi}=Seg(\Gamma'_m(U,X_+\wedge T^n)_{\Phi}).$$

Under this notation the preceding proposition implies the following

\begin{theorem}\label{p:Fr_and_F}
Let $U,X\in Sm/k$ and let $m,n{\geqslant} 0$ be integers. One has an
equality of $S^1$-subspectra
$$\vee_{\Psi \in ({\operatorname{F}}_m(U,X_+\wedge T^n)-0_m)}{\operatorname{F}}^{S^1}_m(U,X_+\wedge T^n)_{\Psi}=\cup_{\Phi\in ({\operatorname{Fr}}_m(U,X_+\wedge T^n)-0_m)}{\operatorname{F}}^{S^1}_m(U,X_+\wedge T^n)_{\Phi}$$
of the spectra ${\operatorname{Fr}}^{S^1}_m(U,X_+\wedge T^n)$ and the inclusion
\begin{equation}\label{eq:F_S1_and_Fr_S1}
{\operatorname{F}}^{S^1}_m(U,X_+\wedge T^n)\subset {\operatorname{Fr}}^{S^1}_m(U,X_+\wedge T^n)
\end{equation}
is a stable equivalence of the $S^1$-spectra.
\end{theorem}

We are now in a position to prove Theorem~\ref{ZM_fr_and_LM_fr}.

\begin{proof}[Proof of Theorem~\ref{ZM_fr_and_LM_fr}]
This follows from Theorem~\ref{p:Fr_and_F}.
Indeed, consider the composite morphism of $S^1$-spectra
   $${\operatorname{\mathbb{Z}F}}^{S^1}_m(U,X_+\wedge T^n)\to {\operatorname{\mathbb{Z}Fr}}^{S^1}_m(U,X_+\wedge T^n)\xrightarrow{\lambda_{X_+\wedge T^n}} {\operatorname{\mathbb{Z}F}}^{S^1}_m(U,X_+\wedge T^n),$$
where the left arrow is induced by the arrow~\eqref{eq:F_S1_and_Fr_S1}.
Within Definitions~\ref{F_m_U_Y/(Y-S)} and~\ref{def:Gamma_m_and_Gamma'_m}, Theorem \ref{p:Fr_and_F} implies the left arrow is a
stable equivalence of $S^1$-spectra. Note that the composite morphism is the identity map.
Thus the morphism $\lambda_{X_+\wedge T^n}$ is a stable equivalence of $S^1$-spectra.
This finishes the proof.
\end{proof}

\begin{thebibliography}{XXXX}

\bibitem[AGP]{AGP}
A. Ananyevskiy, G.~Garkusha, I.~Panin, Cancellation theorem for
framed motives of algebraic varieties, preprint arXiv:1601.06642.

\bibitem[BF]{BF}
A. K. Bousfield, E. M. Friedlander, Homotopy theory of $\Gamma$-spaces, spectra, and bisimplicial sets, 
In Geometric applications of homotopy theory II, Lecture Notes in Math. 658, Springer Verlag, 1978, pp.~80-130.

\bibitem[GP1]{GP1}
G.~Garkusha, I.~Panin, Framed motives of algebraic varieties (after
V. Voevodsky), preprint arXiv:1409.4372.

\bibitem[GP2]{GP2}
G.~Garkusha, I.~Panin, Homotopy invariant presheaves with framed transfers,
preprint,2015, arXiv:1504.00884.

\bibitem[GP3]{GP3}
G.~Garkusha, I.~Panin, Linear framed motives of algebraic varieties,
preprint, 2016.

\bibitem[Mi]{Mi}
J.~Milne. \'Etale cohomology, Princeton Mathematical Series 33,
Princeton University Press, 1980.

\bibitem[Sch]{Sch} S. Schwede, {An untitled book project about symmetric spectra},
         available at www.math.uni-bonn.de/people/schwede/SymSpec-v3.pdf (version April 2012).

\bibitem[Voe]{V2}
V.~Voevodsky, Notes on framed correspondences, unpublished, 2001.
Also available at math.ias.edu/vladimir/files/framed.pdf

\end{thebibliography}
\end{document}


\documentclass{amsart}
\usepackage{graphicx}

\newtheorem{theorem}{Theorem}[section]
\newtheorem{lemma}[theorem]{Lemma}
\newtheorem{corollary}[theorem]{Corollary}
\newtheorem{proposition}[theorem]{Proposition}
\newtheorem{observation}[theorem]{Observation}

\theoremstyle{definition}
\newtheorem{definition}[theorem]{Definition}
\newtheorem{example}[theorem]{Example}
\newtheorem{xca}[theorem]{Exercise}
\newtheorem{claim}[theorem]{Claim}
\newtheorem{conjecture}[theorem]{Conjecture}
\newtheorem{problem}[theorem]{Problem}

\theoremstyle{remark}
\newtheorem{remark}[theorem]{Remark}

\numberwithin{equation}{section}

\begin{document}

\title{Edge number of knots and links}

\author{Makoto Ozawa}
\address{Department of Natural Sciences, Faculty of Arts and Sciences, Komazawa University, 1-23-1 Komazawa, Setagaya-ku, Tokyo, 154-8525, Japan}
\email{w3c@komazawa-u.ac.jp}

\subjclass{Primary 57M25; Secondary 57Q35}

\keywords{knot, link, diagram, bridge number, edge number}

\begin{abstract}
We introduce a new numerical invariant of knots and links made from the partitioned diagrams.
It measures the complexity of knots and links.
\end{abstract}

\maketitle

\section{Why does not a knot come loose?}

There are a lot of answers to a question ``Why does not a knot come loose?''.
The next example gives one of the answers.

In Figure \ref{trefoil}, the trefoil knot diagram is partitioned by 3 vertices into 3 edges $e_1, e_2$ and $e_3$.
The edge $e_1$ is over $e_2$, $e_2$ is over $e_3$ and $e_3$ is over $e_1$.
This relation is known to be a three-way deadlock.
It seems that this is one of the factors that why a knot cannot come loose.

\begin{figure}[htbp]
	\begin{center}
		\includegraphics[trim=0mm 0mm 0mm 0mm, width=.4\linewidth]{trefoil.eps}
	\end{center}
	\caption{a 3-edge presentation}
	\label{trefoil}
\end{figure}

\section{Introduction to the definition of edge number}

Throughout this paper we work in the piecewise linear category.
We shall study knots and links in the three-dimensional Euclidean space $\Bbb{R}^{3}$.
For the standard definitions and results of knots and links, we refer to \cite{A}, \cite{BZ}, \cite{C}, \cite{Kaw}, \cite{Lic2}, \cite{L}, \cite{M} and \cite{R}.

Let $K$ be a knot or link and $D$ be a diagram of $K$.
We partition $D$ by $n$ vertices into $n$ edges $e_1,\ldots, e_n$.
We assume that each component of $K$ has at least one vertex.
We call such a diagram $D$ an {\em $n$-partitioned diagram}.

\begin{definition}
An $n$-partitioned diagram $D$ is an {\em $n$-cycle presentation} of $K$ if
\begin{enumerate}
\item Each edge of $D$ has no self-crossing.
\item For any pair of two edges $e_i$ and $e_j$, exactly one of the following holds.
	\begin{enumerate}
	\item $e_i$ is over $e_j$ at every crossing of $e_i$ and $e_j$.
	\item $e_i$ is under $e_j$ at every crossing of $e_i$ and $e_j$.
	\item There is no crossing of $e_i$ and $e_j$.
	\end{enumerate}
\end{enumerate}

We define an {\em edge number} $e(K)$ of $K$ as the minimum number of $n$ where $n$ is taken over all $n$-cycle presentation of $K$.

We can make a directed graph $G(D)$ from an $n$-cycle presentation $D$ as follows.
\begin{enumerate}
\item For each edge $e_i$, we assign a vertex $v_i$ to $G(D)$.
\item For two edges $e_i$ and $e_j$ which have at least one crossing, if $e_i$ is over (resp. under) $e_j$, then we assign an oriented edge from $e_i$ to $e_j$ (resp. from $e_j$ to $e_i$) to $G(D)$.
\end{enumerate}
\end{definition}

Figure \ref{digraph} shows a digraph $G(D)$ obtained from a 3-edge presentation $D$ in Figure \ref{trefoil}.

\begin{figure}[htbp]
	\begin{center}
		\includegraphics[trim=0mm 0mm 0mm 0mm, width=.4\linewidth]{digraph.eps}
	\end{center}
	\caption{the digraph of a 3-edge presentation in Figure \ref{trefoil}}
	\label{digraph}
\end{figure}

\begin{remark}
If we exclude the condition (2) in the definition of an $n$-cycle presentation, then any knot has a 2-cycle presentation.
Figure \ref{trefoil2} will help you to show this.
\begin{figure}[htbp]
	\begin{center}
		\includegraphics[trim=0mm 0mm 0mm 0mm, width=.6\linewidth]{trefoil2.eps}
	\end{center}
	\caption{2-cycle presentation without the condition (2)}
	\label{trefoil2}
\end{figure}
\end{remark}

\section{Results and Conjecture}

\begin{proposition}\label{non-trivial}
A knot $K$ is non-trivial if and only if $e(K)\ge 3$.
\end{proposition}

\begin{proposition}\label{3-cycle}
Let $K$ be a knot with $e(K)=3$ and $D$ a 3-cycle presentation of $K$.
Then $G(D)$ is an oriented 3-cycle.
\end{proposition}

\begin{theorem}\label{main}
For any minimal $n$-cycle presentation $D$,
\begin{enumerate}
\item $G(D)$ is connected.
\item $G(D)$ is not a path.
\end{enumerate}
\end{theorem}

\begin{conjecture}
For a non-trivial knot $K$ and any $n$-cycle presentation $D$ of $K$, $G(D)$ contains at least one oriented cycle.
\end{conjecture}

\begin{proposition}\label{exist}
For any $n\ge 3$ $(n\equiv 0, 1, 2, 3, 4\mod 6)$, there exists a knot such that $c(K)=n$ and $e(K)=3$.
\end{proposition}

\begin{conjecture}\label{bridge}
$b(K)\le e(K)$.
\end{conjecture}

\begin{remark}
$e(K)\le 2b(K)$.
\end{remark}

\begin{problem}
Find a knot $K$ with $e(K)=4$.
We expect that $5_1$ is a candidate.
\end{problem}

\section{Proofs}

\begin{proof}(of Proposition \ref{non-trivial})

($\Rightarrow$) Suppose that $e(K)=1$ and let $D$ be a 1-cycle presentation of $K$.
Then by the definition (1) of an $n$-cycle presentation, $D$ has no crossing.
Hence $K$ is trivial.
Next suppose that $e(K)=2$ and let $D$ be a 2-cycle presentation of $K$.
$D$ has two edges $e_1$ and $e_2$ and without loss of generality we may assume that $e_1$ is over $e_2$ at every crossings of $e_1$ and $e_2$.
Then by the definition (1) of an $n$-cycle presentation, $D$ is a 1-bridge presentation with an over-bridge $e_1$ and an under-bridge $e_2$.
Hence $K$ is trivial.

($\Leftarrow$) Suppose that $K$ is trivial.
Then $K$ has a 1-cycle presentation.
Hence $e(K)=1$.
\end{proof}

\begin{proof}(of Proposition \ref{3-cycle})
Let $D$ be a 3-cycle presentation of $K$ with three edges $e_1, e_2$ and $e_3$.
Suppose that $G(D)$ does not form an oriented 3-cycle.
Without loss of generality, we may assume that $e_1$ is over $e_2$ at every crossings of $e_1$ and $e_2$, $e_2$ is over $e_3$ at every crossings of $e_2$ and $e_3$, and $e_1$ is over $e_3$ at every crossings of $e_1$ and $e_3$.
Then $D$ is a descending diagram by specifying an orientation in the order $e_1,e_2,e_3$.
Therefore $K$ is trivial and this contradicts $e(K)=3$ and Proposition \ref{non-trivial}.
Hence $G(D)$ forms an oriented 3-cycle.
\end{proof}

\begin{lemma}\label{neighbourhood}
Let $K$ be a knot with $e(K)=n$ and $D$ an $n$-cycle presentation of $K$.
Then 
\begin{enumerate}
\item For any successive vertices $v_i, v_{i+1}$ of $G(D)$, both of them can not be a source nor sink in $G(D)-v_iv_{i+1}$.
\item For any successive non-adjacent vertices $v_i, v_{i+1}$ of $G(D)$, there exists a vertex $v_k$ in $N(v_i)\cap N(v_{i+1})$ such that $v_iv_kv_{i+1}$ is an oriented path,
\end{enumerate}
where $N(v)$ denotes the set of vertices which is adjacent to $v$.
\end{lemma}

\begin{proof}(of Lemma \ref{neighbourhood})
Let $K$ be a knot with $e(K)=n$ and $D$ an $n$-cycle presentation of $K$.

(1) Suppose without loss of generality that there exist successive vertices $v_i, v_{i+1}$ of $G(D)$ such that both of them are source in $G(D)-v_iv_{i+1}$.
Since the subarc $e_i\cup e_{i+1}$ is over all other edges $e_j\ (j\ne i, i+1)$, there is an isotopy of $D$ such that $e_i\cup e_{i+1}$ has no self-crossing.
Then we can regard the subarc $e_i\cup e_{i+1}$ as a single edge $e_i'$, and the new edge $e_i'$ satisfies the definition (1) and (2) of an $n$-cycle presentation.
See Figure \ref{reduce1}.
Hence we obtain an $(n-1)$-cycle presentation of $K$ and this contradicts $e(K)=n$.

\begin{figure}[htbp]
	\begin{center}
	\includegraphics[trim=0mm 0mm 0mm 0mm, width=.7\linewidth]{reduce1.eps}
	\end{center}
	\caption{reducing $e_i\cup e_{i+1}$}
	\label{reduce1}
\end{figure}

(2) Suppose without loss of generality that there exist successive non-adjacent vertices $v_i,v_{i+1}$ such that for any vertex $v_k$ in $N(v_i)\cap N(v_{i+1})$, the edge $v_iv_k$ (resp. $v_{i+1}v_k$) has an orientation from $v_i$ to $v_k$ (resp. from $v_{i+1}$ to $v_k$).
We regard the subarc $e_i\cup e_{i+1}$ as a single edge $e_i'$.
Then $e_i'$ satisfies the definition (1) and (2) of an $n$-cycle presentation.
See Figure \ref{reduce2}.
Hence we obtain an $(n-1)$-cycle presentation of $K$ and this contradicts $e(K)=n$.
\begin{figure}[htbp]
	\begin{center}
	\includegraphics[trim=0mm 0mm 0mm 0mm, width=.7\linewidth]{reduce2.eps}
	\end{center}
	\caption{reducing $e_i\cup e_{i+1}$}
	\label{reduce2}
\end{figure}
\end{proof}

\begin{remark}\label{distance}
By Lemma \ref{neighbourhood} (2), for any successive vertices $v_i, v_{i+1}$, $d(v_i,v_{i+1})\le 2$, where $d(v,v')$ denotes the distance between $v$ and $v'$ in the graph.
\end{remark}

\begin{proof}(of Theorem \ref{main})
Let $D$ be a minimal $n$-cycle presentation.

(1) Suppose that $G(D)$ is disconnected.
Then there exist successive vertices $v_i,v_{i+1}$ such that $v_i$ is not connected to $v_{i+1}$ in $G(D)$.
This contradicts Remark \ref{distance}.

(2) Suppose that $G(D)$ is a path.
Then by Remark \ref{distance}, $G(D)$ has length two.
This contradicts Proposition \ref{3-cycle}.
\end{proof}

\begin{proof}(of Proposition \ref{exist})
The demanded knots are displayed in Figure \ref{6m}.
By \cite{Kau}, \cite{Mur} and \cite{Thi}, these diagrams are minimal crossing since they are alternating diagrams.
\begin{figure}[htbp]
	\begin{center}
	\begin{tabular}{ccc}
	\includegraphics[trim=0mm 0mm 0mm 0mm, width=.2\linewidth]{6m_0.eps}&
	\includegraphics[trim=0mm 0mm 0mm 0mm, width=.23\linewidth]{6m_1.eps}&
	\includegraphics[trim=0mm 0mm 0mm 0mm, width=.29\linewidth]{6m_2.eps}\\
	$n\equiv 0\mod 6$ & $n\equiv 1\mod 6$ & $n\equiv 2\mod 6$ \\
	\includegraphics[trim=0mm 0mm 0mm 0mm, width=.24\linewidth]{6m_3.eps}&
	\includegraphics[trim=0mm 0mm 0mm 0mm, width=.24\linewidth]{6m_4.eps}&
	\\
	$n\equiv 3\mod 6$ & $n\equiv 4\mod 6$ & \\
	\end{tabular}
	\end{center}
	\caption{knots with $e(K)=3$}
	\label{6m}
\end{figure}
\end{proof}

\noindent{\em "Proof" of Conjecture \ref{bridge}.}
Let $K$ be a knot with $e(K)=n$ and $D$ an $n$-cycle presentation of $K$ on the 2-sphere $S^2$ dividing the 3-sphere $S^3$ into two 3-balls $B_+$ and $B_-$.
We push all subarcs of $D$, which are regular neighbourhoods of vertices, into $B_-$, and pull the rest of $D$ into $B_+$.
Then $K$ intersects $S^2$ in $2n$-points, and $K\cap B_{\pm}$ consists of properly embedded $n$-arcs in $B_{\pm}$.
Conjecture \ref{bridge} will be proved if we can show that the $n$-string tangle $(B_+,K\cap B_+)$ is trivial.

\bibliographystyle{amsplain}
\begin{thebibliography}{10}

\bibitem{A} C. Adams, {\em The Knot Book}, American Mathematical Society, 2004.

\bibitem{BZ} G. Burde and H. Zieschang, {\em Knots}, Walter de Gruyter, 2002.

\bibitem{C} P. Cromwell, {\em Knots and Links}, Cambridge University Press, 2004.

\bibitem{Kau} L. H. Kauffman, {\em New invariants in the theory of knots}, Amer. Math. Mon. {\bf 95} (1988) 195-242.

\bibitem{Kaw}A. Kawauchi, {\em Survey on Knot Theory}, Birkh\"{a}user Verlag, 1996.

\bibitem{Lic2} W. B. R. Lickorish, {\em An Introduction to Knot Theory}, Springer, 1997.

\bibitem{L} C. Livingston, {\em Knot Theory}, The Mathematical Association of America, 1996.

\bibitem{Mur} K. Murasugi, {\em Jones Polynomials and Classical Conjectures in Knot Theory}, Topology {\bf 26} (1987) 297-307.

\bibitem{M} K. Murasugi, {\em Knot Theory and Its Applications}, translated by B. Kurpita, Birkh\"{a}user Boston, 1996.

\bibitem{O} M. Ozawa, {\em Ascending number and edge number of knots and links}, a talk in a conference ``Tohoku Knot Seminor'' at Yamagata University, 2000.

\bibitem{O2} M. Ozawa, {\em Ascending number of knots and links}, preprint at arXiv:0705.3337, 2007.

\bibitem{R} D. Rolfsen, {\em Knots and Links}, AMS Chelsea Publishing, 2003.

\bibitem{Thi} M. Thistlethwaite, {\em A Spanning Tree Expansion for the Jones Polynomial}, Topology {\bf 26} (1987) 297-309.

\end{thebibliography}

\end{document}


\documentclass{amsart}

\usepackage{amssymb}
\usepackage{latexsym}
\usepackage{amsmath}
\usepackage{euscript}
\usepackage{graphics}
\usepackage{tikz}

      
      
      
      
      
      
      
      
   

      
      
      
      
      
      
      
      
   

      
      
      
      
      
      
      
      
   

  

\def\bm\chi{\mbox{\boldmath$\chi$}}

\let\xker={{\xker\,}} 

\unitlength=1mm

\newtheorem{thm}{Theorem}[section]
\newtheorem{cor}[thm]{Corollary}
\newtheorem{lem}[thm]{Lemma}
\newtheorem{prop}[thm]{Proposition}

\newtheorem{theorem}{Theorem}[section]
\newtheorem{proposition}[theorem]{Proposition}
\newtheorem{corollary}[theorem]{Corollary}
\newtheorem{lemma}[theorem]{Lemma}
\newtheorem{definition}[theorem]{Definition}
\theoremstyle{remark}
\newtheorem{example}[theorem]{Example}
\newtheorem{remark}[theorem]{Remark}

\numberwithin{equation}{section}

\author{Alexander I. Aptekarev}
\address{
Alexander I. Aptekarev\\
Keldysh Institute for Applied Mathematics\\
Russian Academy of Sciences\\
Miusskaya pl. 4\\
125047 Moscow, RUSSIA}

\author{Maxim Derevyagin}
\address{
Maxim Derevyagin\\
University of Mississippi\\
Department of Mathematics\\
Hume Hall 305 \\
P. O. Box 1848 \\
University, MS 38677-1848, USA }
\email{derevyagin.m@gmail.com}

\author{Walter Van Assche}
\address{
Walter Van Assche\\
KU Leuven\\
Department of Mathematics\\
Celestijnenlaan 200B box 2400\\
BE-3001 Leuven}

\date{\today}

 \subjclass{Primary 42C05, 37K10; Secondary 37K60, 39A12, 39A14, 65Q10.}
\keywords{Multiple orthogonal polynomials, discrete integrable system, discrete zero curvature condition, ordinary and partial difference equations, two-dimensional Schur-Euclid algorithm, two-dimensional continued fractions, recurrence relations}

\begin{document}

\title[Discrete integrable systems generated by HP approximants]{Discrete integrable systems generated by
Hermite-Pad\'e approximants}

\begin{abstract}
We consider {Her\-mite-Pad\'e}{} approximants in the framework of
discrete integrable systems defined on the lattice ${{\mathbb Z}}^2$. We show that the concept of multiple orthogonality is intimately related to the Lax representations for the entries of the nearest neighbor recurrence relations and it thus gives rise to discrete integrable systems. In particular, we show how to construct  classes of global solutions of the discrete integrable systems in question.
\end{abstract}

\maketitle

\section{Introduction}\label{sec:1}

Nowadays modern technologies allow us to handle an enormous amount of information. As a consequence of this development, it is in many instances more advantageous to face the analysis of discrete data rather than continuous data. For this reason we are witnessing that the world in this century
requires more and more the understanding of discrete models and that is why we decided to concentrate our attention on studying discrete models that take their origin in \textit{orthogonality}, one of the basic mathematical concepts.
Mathematically speaking, the discrete models we are going to consider are \textit{systems of difference equations}. Recent advances in a number of mathematical fields reveal that discrete systems are in many respects even more fundamental than continuous ones (for instance see \cite{Nov},  \cite{S2010}).

In this paper we follow the streamline of \textit{discrete integrable systems} (see \cite{B2004}, \cite{BS2002}) and our main interest is in discrete systems on ${{\mathbb Z}}^2$ represented by a field of square  invertible  matrices
\begin{equation}\label{0 DIS}
L_{n,m}\,, \,\, M_{n,m}\,\in \,{{\mathbb C}}^{d\times d}\, , \qquad n,m\in{{\mathbb Z}},
\end{equation}
which satisfy the \textit{discrete zero curvature} (or form \textit{a Lax pair}) condition on ${{\mathbb Z}}^2$:
\begin{equation}\label{0 ZCC}
L_{n,m+1}M_{n,m}-M_{n+1,m}L_{n,m}=0\,.
\end{equation}
The elements of the discrete system \eqref{0 DIS} are transition matrices which define the evolution of the \textit{wave function} $\Psi_{n,m}$
\begin{equation}\label{0 WF}
\Psi_{n+1,m}(z)=L_{n,m}(z)\Psi_{n,m}(z), \quad \Psi_{n,m+1}(z)=M_{n,m}(z)\Psi_{n,m}(z),
\end{equation}
and the condition \eqref{0 ZCC} describes the consistency or integrability of the equations  \eqref{0 WF}. In turn, the relations \eqref{0 ZCC} represent a nonlinear system of difference equations.

Our findings are mainly inspired by the connection between discrete integrable systems, \textit{orthogonal polynomials}, 
and \textit{Pad\'e approximants}
(see \cite{PGR1995}, \cite{SNderK2011}). For example, the discrete dynamics
\begin{equation}\label{0 DTT}
x^k\, d\mu(x), \qquad k\in{{\mathbb Z}}_+
\end{equation}
of the measure $d\mu$ supported on $[0,+\infty)$ generates a family of orthogonal polynomials $\{P_n^{(k)}(x)\}, \,\, \mbox{deg }P_n^{(k)}=n$. These polynomials also appear in the Pad\'e table and some nearest neighbours of this table are related
$$
P_{n+1}^{(k)}(x)  =  xP_{n}^{(k+1)}(x)-V_{n}^{(k)}P_{n}^{(k)},  \qquad
P_{n+1}^{(k)}(x)  =  xP_{n}^{(k+2)}(x)-W_{n}^{(k)}P_{n}^{(k+1)},
$$
where the coefficients of the relations are expressed by means of the Hankel determinants
\begin{equation*}
V_{n}^{(k)}  =  \frac{S_{n+1}^{(k+1)}S_{n}^{(k)}}{S_{n}^{(k+1)}S_{n+1}^{(k)}} ,\quad
W_{n}^{(k)}  =  \frac{S_{n+1}^{(k+1)}S_{n}^{(k+1)}}{S_{n+1}^{(k)}S_{n}^{(k+2)}} ,\quad
S_{n+1}^{(k)}=
\left|\begin{array}{cccc}
s_{k}           & \ldots      &  s_{n+k}  \\
\vdots          & {\begin{picture}(2,2)
\multiput(0,0)(1.5,1){3}{.}\end{picture}}            &  \vdots  \\
s_{n+k}      & \ldots     &  s_{2n+k}
\end{array}\right|,
\end{equation*}
and $s_j=\int_{0}^{\infty}x^j\, d\mu(x)$.
Finally, the consistency of these relations gives the discrete zero curvature condition \eqref{0 ZCC} for the discrete system \eqref{0 DIS} of $2 \times 2$ matrices
\begin{equation*}
L_{n,k}=\left( \begin{array}{cc}
-V_{n}^{(k)} & x \\
-V_{n}^{(k)} & x + W_{n}^{(k)} -V_{n}^{(k+1)}
\end{array} \right), \quad
M_{n,k}=\frac{1}{x}\left( \begin{array}{cc}
0 & x \\
- V_{n}^{(k)} & x+ W_{n}^{(k)}
\end{array} \right).
\end{equation*}
Recall that in the theory of Pad\'e approximants this discrete system becomes
 the quotient-difference algorithm, and in integrable systems theory it leads to the discrete-time Toda equation
(see, e.g., \cite{Suris}).

\medskip

In the present paper we introduce a new
 class of discrete integrable systems of $3 \times 3$ matrices  \eqref{0 DIS}--\eqref{0 ZCC}. The construction  of these systems is based on the  theory of \textit{Hermite-Pad\'e rational approximants}, which were introduced by Hermite \cite{Her} in
connection to his outstanding proof of the transcendence of $e$. These days this theory is known to play an important role in various fields ranging from number theory \cite{Aper}, \cite{Apt2011}, \cite{vanA2001} to random matrix theory \cite{K2010}, \cite{AptKu}.

To proceed, let us briefly consider the concept of Hermite-Pad\'e rational approximants (for details, see the surveys \cite{Apt}, \cite{vanA1999}). Let $\vec{f} = (f_1,f_2)$ be a vector of Laurent series at infinity
\begin{equation}  \label{0 f}
   f_j(z) = \sum_{k=0}^\infty \frac{s_{j,k}}{z^{k+1}},\qquad j=1,2.
\end{equation}
The \textit{{Her\-mite-Pad\'e}{} rational approximants} (of type II)
\[  \pi_{\vec{n}} = \left( \frac{Q_{\vec{n}}^{(1)}}{P_{\vec{n}}},
    \frac{Q_{\vec{n}}^{(2)}}{P_{\vec{n}}} \right) \]
for the vector $\vec{f}$ and multi-index $\vec{n} = (n_1,  n_2) \in \mathbb{N}^2$
are defined by
\[  {\operatorname{deg}} P_{\vec{n}} \leq |\vec{n}| = n_1 + n_2, \]
\begin{equation}   \label{0 HP}
    f_j(z)P_{\vec{n}}(z) - Q_{\vec{n}}^{(j)}(z) =:
    R_{\vec{n}}^{(j)}(z) = \mathcal{O}\left(\frac{1}{z^{n_j+1}}\right), \qquad z \to \infty,
\end{equation}
where the $Q_{\vec{n}}^{(j)}$ are polynomials,
for $j=1, 2$.
This definition is equivalent to a homogeneous linear system of equations
for the coefficients of the polynomial $P_{n_1,n_2}$.
This system always has a solution, but the solution is not necessarily unique.
In the case of uniqueness (up to a multiplicative constant) and in
case any non-trivial solution has full degree $ \mbox{deg}\,P_{n_1,n_2}= n_1+n_2$,
the multi-index $(n_1,n_2)$ is called \textit{normal} and the polynomial $P_{\vec{n}}$ can
be normalized to be monic.

Clearly these polynomials can be put in a table $\{P_{n,m}\}$. If all indices of this table are normal, then the system of functions \eqref{0 f} is called a \textit{perfect system}. The notion of perfect systems was introduced by Mahler \cite{Mah}. For perfect systems, the {Her\-mite-Pad\'e}{} polynomials \eqref{0 HP} satisfy a system of recurrence
relations
\begin{equation} \label{0 RR}
\begin{cases}
    \,\,\,P_{n+1,m}(x) = (x-c_{n,m})P_{n,m}(x) - a_{n,m} P_{n-1,m}(x) - b_{n,m} P_{n,m-1}(x),  \\
    \,\,\,P_{n,m+1}(x) = (x-d_{n,m})P_{n,m}(x) - a_{n,m} P_{n-1,m}(x) - b_{n,m} P_{n,m-1}(x),
\end{cases}
\end{equation}
with $a_{0,m}=b_{n,0}=0$ for all $n,m \geq 0$ and $a_{n,m},b_{n,m}\neq 0$ for all other indices $(n,\,m)$.
As we will see, the consistency of these relations also leads to Lax pair representations, where the corresponding matrices $L_{n,m}$ and $M_{n,m}$ have the forms
\begin{equation} \label{Int_eq:2.8}
    L_{\cdot,\cdot}  = \begin{pmatrix}
                  x+ \alpha_1& \alpha_2 & \alpha_3 \\
                  \alpha_4 & 0 &  0 \\
                  \alpha_5 & 0 &  1
                 \end{pmatrix},\qquad
    M_{\cdot,\cdot}  = \begin{pmatrix}
                  x+\beta_1 & \beta_2 & \beta_3 \\
                  \beta_4 & 1 &  0 \\
                  \beta_5 & 0 &  0
                 \end{pmatrix},
\end{equation}
with $(\alpha_2)_{0,m}=(\alpha_4)_{0,m}=(\beta_3)_{n,0}=(\beta_5)_{n,0}=0$ for all $n,m \geq 0$ and for the rest of the indices $(n,\,m)$ we have $(\alpha_j)_{n,m},\,(\beta_j)_{n,m}\neq 0,\,\, j=2,\cdots,\,5$.
Both sets of coefficients  of the relations \eqref{0 RR} and of entries of the matrices \eqref{Int_eq:2.8} can be represented by  the power series coefficients  of the perfect system of functions \eqref{0 f}
\begin{equation} \label{0 a-s-alpha}
(a,\,b,\,c,\,d)_{n,m}\quad \longleftarrow \quad
\{s_{j,k}\}_{j=1,2} \quad \longrightarrow \quad \binom{\alpha_1,\,\cdots,\,\alpha_5} {\beta_1,\,\cdots,\,\beta_5}_{n,m}
\end{equation}
(details will be given below).
Our main result is the following.
\begin{theorem} \label{T0 1}
The zero curvature condition \eqref{0 ZCC}
holds for a family of $3\times 3$ transition matrices $L_{n,m}$ and $M_{n,m}$ of the form \eqref{Int_eq:2.8} if and only if there is a perfect system of two functions \eqref{0 f} such that  $P_{n,m}$ are the {Her\-mite-Pad\'e}{} polynomials with the coefficients of the recurrence relations \eqref{0 RR} corresponding to \eqref{0 a-s-alpha}.
\end{theorem}
To give more details on the equivalence of discrete integrable system \eqref{Int_eq:2.8} and recurrence relations for the {Her\-mite-Pad\'e}{} polynomials \eqref{0 RR} we also highlight the following.
\begin{theorem} \label{T0 2}
 The discrete Lax pair equations \eqref{0 ZCC} for the matrices $L_{n,m}$, $M_{n,m}$ of the form \eqref{Int_eq:2.8} are equivalent to the nonlinear system of difference equations for the coefficients of the recurrence relations \eqref{0 RR}
\begin{equation}  \label{0 DiffEq}
\begin{cases}
    \,\,\,c_{n,m+1}\,=\, c_{n,m}\,+\,\displaystyle\frac{(a+b)_{n+1,m}\,-\,(a+b)_{n,m+1}}{(c-d)_{n,m}}  \\
    \,\,\,d_{n,m+1}\,=\, d_{n,m}\,+\,\displaystyle\frac{(a+b)_{n+1,m}\,-\,(a+b)_{n,m+1}}{(c-d)_{n,m}}  \\
    \,\,\,a_{n,m+1}\,=\,a_{n,m} \,\displaystyle\frac{(c-d)_{n,m}}{(c-d)_{n-1,m}} \\
    \,\,\,b_{n+1,m}\,=\,b_{n,m} \,\displaystyle\frac{(c-d)_{n,m}}{(c-d)_{n,m-1}}
\end{cases},\qquad n,m \geq 0\,,
\end{equation}
with initial $(c,\,a)_{n,0}$, $(d,\,b)_{0,m}$   and  boundary conditions $a_{0,m} = 0 = b_{n,0}$.
\end{theorem}

The {Her\-mite-Pad\'e}{} approximants are intimately related to the notion of multiple orthogonal polynomials.
If the coefficients of the Laurent series (\ref{0 f}) are the moments of positive measures $\mu_1$ and  $\mu_2$
supported on $\mathbb{R}$
\begin{equation}   \label{0 fMOP}
   f_j(z) = \sum_{k=0}^\infty \frac{s_{j,k}}{z^{k+1}} = \int_{\mathbb{R}} \frac{d\mu_j(x)}{z-x}, \qquad
  s_{j,k} = \int_{\mathbb{R}} x^k\, d\mu_j(x),
\end{equation}
then the {Her\-mite-Pad\'e}{} denominators $P_{n_1,n_2}$ from \eqref{0 HP} satisfy
\begin{equation}  \label{0 MOP}
\int P_{n_1,n_2}(x) x^k\, d\mu_j(x) = 0, \qquad k=0,1,\ldots,n_j-1,\qquad  j=1,2.
\end{equation}
Polynomials  defined by the system of orthogonality relations \eqref{0 MOP} are called \textit{multiple orthogonal polynomials.} The idea of this concept is the following: given two measures $(\mu_1, \mu_2)$, we split the orthogonality relation between
these measures and aim to find a monic polynomial $P_{n_1,n_2}$ of degree $ \mbox{deg}\,P_{n_1,n_2}=n_1+n_2$ that is orthogonal to the $n_1$ first monomials with respect to one measure and to the $n_2$ first monomials with respect to the other one.

The multiple orthogonal polynomials (i.e., the  {Her\-mite-Pad\'e}{} polynomials) inherit all the remarkable properties for Hermite-Pad\'e approximants, like existence  of the monic polynomials of full degree for the normal indices and the recurrence relations \eqref{0 RR}, which were obtained for the first time in 
\cite{vanA2011} for the multiple orthogonal polynomials. In the context of our paper we use these polynomials to generate a general class of perfect systems for which the corresponding tables of multiple orthogonal polynomials exist entirely.

\bigskip

\noindent{\bf Structure of the paper.} The following two sections serve as an introduction to the topic. In particular we give in Section~\ref{sec:2}  more explanations about general discrete integrable systems. Then, in Section~\ref{sec:3}, we consider some known $2 \times 2$ matrix relations from the theory of orthogonal polynomials and continued fractions, i.e.,
the theory that concerns classical diagonal Pad\'e approximants. A generalization of these relations to the $3 \times 3$ matrix case for {Her\-mite-Pad\'e}{} approximants and multiple orthogonal polynomials, which plays a decisive role for establishing the connection to discrete integrable systems represented by $3 \times 3$ matrices, is presented in Section~\ref{sec:4}. Particularly, in that section we give and prove several propositions, which lead to a proof of Theorems~\ref{T0 1}, \ref{T0 2} (Subsections 4.1, 4.2) and provide the reader with a generic class of perfect systems such as Angelesco and Nikishin systems (Subsection 4.3).

\bigskip

\noindent{\bf Acknowledgements.} A.I. Aptekarev was supported by grant RScF-14-21-00025. M. Derevyagin thanks the hospitality of Department of Mathematics of KU Leuven, where his part of the research was mainly done while he was a postdoc there. M. Derevyagin and W. Van Assche gratefully acknowledge the support of FWO Flanders project G.0934.13, KU Leuven research grant OT/12/073 and the Belgian Interuniversity Attraction Poles 
programme P07/18.

\section{The generic Lax representations}\label{sec:2}

Here we recall some basic notions in the theory of discrete integrable systems following \cite{Adler2001}, \cite{BS2002}
(see also \cite{PGR1995}, \cite{SNderK2011}).

Let us consider a regular square lattice ${{\mathbb Z}}^2$, that is the set of all pairs $(n,m)$ of integer numbers $n$ and $m$.
The main object of our study are {\it wave functions} $\Psi_{n,m}$ defined on all the vertices $(n,m)$ of ${{\mathbb Z}}^2$
and having their values in ${{\mathbb C}}^{k\times k}$ (for simplicity, we restrict ourselves here to the cases $k=2$ and $k=3$).
The wave function $\Psi_{n,m}$ depends on a complex parameter $z$, which is interpreted as the spectral
parameter. We assume that for any oriented edge the values of the wave function at the vertices that this edge connects are related via
{\it transition matrices} $L_{n,m}$ and $M_{n,m}$ as follows
\[
\Psi_{n+1,m}(z)=L_{n,m}(z)\Psi_{n,m}(z), \quad \Psi_{n,m+1}(z)=M_{n,m}(z)\Psi_{n,m}(z).
\]
We always require that the transition matrices are  invertible and therefore one has
 \[
\Psi_{n,m}(z)=L_{n,m}^{-1}(z)\Psi_{n+1,m}(z), \quad \Psi_{n,m}(z)=M_{n,m}^{-1}(z)\Psi_{n,m+1}(z).
\]
It is clear that the value of the wave function must not depend on the path one takes to get to the corresponding vertex. Thus, in order that the wave function $\Psi_{n,m}$ is well defined, the following {\it zero curvature condition}
must be satisfied
\begin{equation}\label{Lax}
L_{n,m+1}M_{n,m}-M_{n+1,m}L_{n,m}=0, \qquad n,m\in{{\mathbb Z}}.
\end{equation}
As is known, the zero curvature condition is equivalent to  integrability. Thus,
a discrete system that admits the representation \eqref{Lax} is called integrable \cite{Adler2001}, \cite{B2004}, \cite{BS2002}.

Before going further, let us take a careful look at the zero curvature condition. Observe at first that
one can rewrite \eqref{Lax} as
\[
L_{n,m}^{-1}M_{n+1,m}^{-1}L_{n,m+1}M_{n,m}=I.
\]
Next, from the following picture
\begin{center}
\begin{tikzpicture}
\draw[step=2cm,gray,very thin] (-0.9,-0.9) grid (2.9,2.9);
\draw[very thick] (0,0) rectangle (2,2);
\filldraw[black] (0,0) circle (2pt) node[anchor=north east] {(n,m)};
\draw[very thick, ->] (0,1)node[anchor=north east] {$M_{n,m}$};
\filldraw[black] (0,2) circle (2pt) node[anchor=south east] {(n,m+1)};
\draw[very thick, ->] (0.9,2)--(1,2) node[anchor=south] {$L_{n,m+1}$};
\filldraw[black] (2,2) circle (2pt) node[anchor=south west] {(n+1,m+1)};
\draw[very thick, <-] (2,0.9)--(2,1) node[anchor=north west] {$M_{n+1,m}^{-1}$};
\filldraw[black] (2,0) circle (2pt) node[anchor=north west] {(n+1,m)};
\draw[very thick, ->] (1.1,0)--(1,0) node[anchor=north west] {$L_{n,m}^{-1}$};
\end{tikzpicture}
\end{center}
we see that the zero curvature condition implies that the product of the transition matrices along
the oriented simple square path on ${{\mathbb Z}}^2$ beginning at the vertex $(n,m)$ is the identity matrix.
This observation can be immediately extended to the case of domino paths by reducing them to the just considered simplest case.
\begin{center}
\begin{tikzpicture}
\draw[step=1cm,gray,very thin] (-1.9,-1.9) grid (2.9,2.9);
\draw[ultra thick] (0,0) rectangle (1,2);
\draw[dashed, very thick] (0,1)--(1,1);
\end{tikzpicture}
\end{center}
Now it is clear how to generalize this statement to the case of any closed oriented path on ${{\mathbb Z}}^2$.
Thus the condition \eqref{Lax} means that if one fixes a closed oriented path on the lattice ${{\mathbb Z}}^2$, then
the product of the transition matrices in the order they appear along the path must be the identity matrix.
Note that this property resembles the Cauchy theorem for holomorphic functions and, therefore, it can be considered
as its noncommutative multiplicative analogue for functions on ${{\mathbb Z}}^2$.
The relation \eqref{Lax} is also called the Lax representation and this is one of the ways to say that the underlying
discrete system is integrable.

It turns out that different types of wave functions appear in the theory of orthogonal polynomials and they  are very useful to achieve a big variety of goals. However, it has to be pointed out that the discrete integrable systems and wave functions, that have their origin in orthogonality, are naturally defined on ${{\mathbb N}}^2$, where ${{\mathbb N}}=\{1,2,3, \dots\}$.
However, one can appropriately extend them to ${{\mathbb Z}}_+^2$ or even to ${{\mathbb Z}}^2$ depending on the needs.

\section{Orthogonal polynomials via $2\times 2$ matrix polynomials}\label{sec:3}

In this section we show how wave functions naturally originate in the classical theory of orthogonal polynomials and continued fractions. The classical concepts of this section will be generalized in Section~\ref{sec:4} in order to get new  discrete integrable systems, whose Lax pairs are expressed via $3\times 3$ matrices.

\subsection{The Schur-Euclid algorithm}\label{sec:31}

Suppose we are given a nontrivial Borel measure $d\mu$ on the real line ${{\mathbb R}}$. Assume also that all the moments of the measure $d\mu$
are finite. Then the Schur algorithm, which is a straightforward generalization of Euclid's algorithm, leads
to the following continued fraction
\[
\varphi(z)=\int_{{\mathbb R}}\frac{d\mu(t)}{t-z}\sim
-\frac{1}{\displaystyle{z-a_0-\frac{b_0^2}{\displaystyle{z-a_1-\frac{b_1^2}{\ddots}}}}},
\]
where $b_j^2>0$ and $a_j\in{{\mathbb R}}$ for $j=0,1,\dots$. This continued fraction is called a $J$-fraction.

It is natural to consider continued fractions as infinite sequences of linear fractional transformations. In particular,
in the case of the $J$-fraction, one has the following sequence
\begin{equation*}
    \varphi_j(z)=T_j\left(\varphi_{j+1}(z)\right)=-\frac{1}{z-a_j+
    b_j^2\varphi_{j+1}(z)}, \qquad j\in{{\mathbb Z}}_+,
\end{equation*}
with the initial condition $\varphi_0=\varphi$.
Also, it is well known that a linear fractional transformation can be represented as a $2\times 2$ matrix, i.e.
\begin{equation*}
T_j\mapsto{{\mathcal W}}_j(z)=\begin{pmatrix}0 & -\frac{1}{b_j}\\
                            b_j &  \frac{z-a_j}{b_j}
                            \end{pmatrix},\qquad j\in{{\mathbb Z}}_+.
\end{equation*}
Let us now introduce matrices corresponding to the approximants for the $J$-fraction, that is, the finite truncations
of the continued fraction:
\begin{equation}\label{helpW}
{{\mathcal W}}_{[n,0]}(z)={{\mathcal W}}_0(z){{\mathcal W}}_1(z)\dots{{\mathcal W}}_n(z), \qquad n\in{{\mathbb Z}}_+.
\end{equation}
Before showing how to construct a set of a wave functions and transition matrices on ${{\mathbb Z}}^2$ let us see what the elements
of the matrix polynomial ${{\mathcal W}}_{[n,0]}$ are. To this end, let us put
\begin{equation*}
   \left(\begin{array}{c}
  -Q_0 \\
  P_0 \\
\end{array}\right):=\left(\begin{array}{c}
  0 \\
  1 \\
\end{array}\right),\quad
\left(\begin{array}{c}
 -Q_{j+1}(z) \\
 P_{j+1}(z) \\
\end{array}\right):={{\mathcal W}}_{[j,0]}(z)\left(\begin{array}{c}
  0 \\
  1 \\
\end{array}\right), \quad j\in{{\mathbb Z}}_+ .
\end{equation*}
Then taking into account the relation ${{\mathcal W}}_{[j,0]}(z)={{\mathcal W}}_{[j-1,0]}(z){{\mathcal W}}_{j}(z)$ we also have that
\begin{equation*}
{{\mathcal W}}_{[j,0]}(z)\left(\begin{array}{c}
  1 \\
  0 \\
\end{array}\right)=
{{\mathcal W}}_{[j-1,0]}(z)\left(\begin{array}{c}
  0 \\
  b_j \\
\end{array}\right)=
\left(\begin{array}{c}
  -b_j Q_j(z) \\
  b_j P_j(z) \\
\end{array}\right), \quad j\in{{\mathbb N}}.
\end{equation*}
So, the matrix ${{\mathcal W}}_{[j,0]}$ has the following form
\begin{equation*}
{{\mathcal W}}_{[j,0]}(z)=\left(\begin{array}{cc}
  -b_j Q_j(z) & -Q_{j+1}(z) \\
  b_j P_j(z) & P_{j+1}(z) \\
\end{array}\right), \quad j\in{{\mathbb Z}}_+ .
\end{equation*}
Furthermore, rewriting the relation entrywise
\[
\left(\begin{array}{c}
 -Q_{j+1}(z) \\
 P_{j+1}(z) \\
\end{array}\right)={{\mathcal W}}_{[j-1,0]}(z){{\mathcal W}}_{j}(z)\left(\begin{array}{c}
  0 \\
  1 \\
\end{array}\right)=
\frac{1}{b_j}{{\mathcal W}}_{[j-1,0]}(z)\left(\begin{array}{c}
  -1 \\
  z-a_j \\
\end{array}\right), \quad j\in{{\mathbb N}},
\]
we see that the polynomials $P_{j}$, $Q_{j}$ are solutions of the following three-term recurrence relation
\begin{equation}\label{3termOP}
b_{j-1}u_{j-1}(z)+a_ju_{j}(z)+b_{j}u_{j+1}(z)=zu_j(z),\qquad j\in{{\mathbb N}},
\end{equation}
with the initial conditions
\begin{equation*}\label{DiffEq}
 \begin{split}
    P_0(z)&=1,\quad P_1(z)=\frac{z-a_0}{b_0},\\
     Q_0(z)&=0,\quad Q_1(z)=\frac{1}{b_0}.
\end{split}
\end{equation*}
Thus the entries of the matrix ${{\mathcal W}}_{[n,0]}$ are orthogonal polynomials of the first and second kind and the corresponding
orthogonality measure is $d\mu$. It is worth mentioning that such $2\times 2$ matrix polynomials are extensively used in the theory of moment problems \cite{Ach1961} and  also show that this theory is a particular case of the theory of canonical systems \cite[Chapter 8]{Sakh1997} (see also \cite{Sakh1999}).

To proceed with the wave function, first of all note that we can define it as follows
\[
\Psi_{n,0}(z)={{\mathcal W}}_{[n,0]}(z), \qquad n\in{{\mathbb Z}}_+.
\]
Next, we can extend it to ${{\mathbb Z}}^2_+$ by the following rule
\[
\Psi_{n,m}(z)={{\mathcal W}}_{[n+m,0]}(z), \qquad n,m\in{{\mathbb Z}}_+,
\]
and, consequently, the transition matrices are
\[
L_{n,m}={{\mathcal W}}_{n+m},\quad M_{n,m}={{\mathcal W}}_{n+m}, \qquad n,m\in{{\mathbb Z}}_+.
\]
Finally, it remains to extend the wave function to the entire lattice ${{\mathbb Z}}^2$ by the symmetry
\begin{equation}\label{SymW}
\Psi_{-n,m}=\Psi_{n,m}, \quad \Psi_{n,-m}=\Psi_{n,m},\quad \Psi_{-n,-m}=\Psi_{n,m}, \qquad n,m\in{{\mathbb Z}}_+.
\end{equation}
In this case, the corresponding Lax representations become a trivial identity and thus they do lead to
trivial discrete integrable systems. However, it is very useful to keep this observation in mind in order to see what happens for various generalizations of orthogonal polynomials.

\subsection{Riemann-Hilbert problems}\label{sec:32}

Here we consider a different interpretation of the Schur-Euclid algorithm in the context of Riemann-Hilbert problems,
which turned out to be quite efficient for asymptotic analysis. Recall that in \cite{FIK1992} a fascinating characterization of
orthogonal polynomial in terms of a Riemann-Hilbert problem was found.
We will explain this characterization here briefly. To this end, let us consider a weight function $w$ on ${{\mathbb R}}$ that is smooth and has sufficient decay at $\pm\infty$ so that
all the moments $\int_\dRx^kw(x)\,dx$ exist. Then the Riemann-Hilbert problem (RHP) consists of the following:
find a $2\times 2$ matrix valued function $Y_n(z)=Y(z)$ such that
\begin{enumerate}
    \item[(i)]
        $Y(z)$ is  analytic for $z\in{{\mathbb C}} \setminus {{\mathbb R}}$.
    \item[(ii)]
        $Y$ possesses continuous boundary values for $x\in{{\mathbb R}}$
        denoted by $Y_{+}(x)$ and $Y_{-}(x)$, where $Y_{+}(x)$ and $Y_{-}(x)$
        are the limiting values of $Y(z')$ as $z'$ approaches $x$ from
        above and below, respectively, and
        \begin{equation}\label{RHPYb}
            Y_+(x) = Y_-(x)
            \begin{pmatrix}
                1 & w(x) \\
                0 & 1
            \end{pmatrix},
            \qquad x\in{{\mathbb R}}.
        \end{equation}
    \item[(iii)]
        $Y(z)$ has the following asymptotic behavior at infinity:
        \begin{equation} \label{RHPYc}
            Y(z)= \left(I+ \mathcal{O} \left( \frac{1}{z} \right)\right)
            \begin{pmatrix}
                z^{n} & 0 \\
                0 & z^{-n}
            \end{pmatrix}, \qquad z \to\infty.
        \end{equation}
\end{enumerate}
Before giving the solution of this RHP for $Y$, let us recall that the monic orthogonal polynomials
$\pi_n(z)=z^n+\dots$ satisfy the following three term recurrence relation:
\[
z\pi_j(z)=\pi_{j+1}(z)+a_j\pi_{j}(z)+b_{j-1}^2\pi_{j-1}(z),\qquad j\in{{\mathbb Z}}_+.
\]
According to \cite{FIK1992}, the matrix valued function $Y(z)$ given by
    \begin{equation} \label{RHPYsolution}
        Y(z) =
        \begin{pmatrix}
            \pi_n(z) & \frac{1}{2\pi i} \int_{{\mathbb R}}  \frac{\pi_n(x) w(x)}{x-z}\,dx \\[2ex]
            -2\pi i \gamma_{n-1}^2 \pi_{n-1}(z) & -\gamma_{n-1}^2 \int_{{\mathbb R}} \frac{\pi_{n-1}(x)w(x)}{x-z}\,dx
        \end{pmatrix}
    \end{equation}
is the unique solution of the RHP for $Y$. Here $\gamma_n$ is the leading coefficient of the corresponding orthonormal polynomial.
Now, as in the previous section, one can define a function $\Psi_{n,m}$ on ${{\mathbb Z}}^2$. To see that
this function is a wave function, we need to be able to construct transition matrices.
To this end, let us observe that $\det Y$ is an analytic function in $\mathbb{C} \setminus \mathbb{R}$
which has no jump on the real axis. Therefore, $\det Y$ is an entire function. Its behavior near infinity
is $\det Y(z) = 1 + \mathcal{O}(1/z)$. Thus by Liouville's theorem we find that $\det Y = 1$. Consequently, one can consider
the matrix
\[
  L_{n,0} = Y_{n+1} Y_{n}^{-1}.
\]
Clearly $L_{n,0}$ is an analytic function on ${{\mathbb C}} \setminus {{\mathbb R}}$, and since $Y_{n}$
and $Y_{n+1}$ have the same jump matrix on $\mathbb{R}$ we see that $L_{n,0}$ has no jump on $\mathbb{R}$.
Hence $L_{n,0}$ is an entire matrix function. We write the asymptotic condition in the following form
\[
Y_{n}(z) = \left( I + \frac{A(n)}{z} + \mathcal{O}(1/z^2) \right)
		\begin{pmatrix} z^{n} & 0  \\ 0 & z^{-n} \end{pmatrix},
\]
where $A(n,m)$ is the $2\times 2$ matrix coefficient of $1/z$ in the $\mathcal{O}(1/z)$ term.
After some calculations and using Liouville's theorem, we find that
\begin{equation}\label{OP_transM}
L_{n,0} = \begin{pmatrix}
                  z+A_{1,1}(n+1)-A_{1,1}(n)  & -A_{1,2}(n)   \\
                  A_{2,1}(n+1)  & 0
                 \end{pmatrix}, \qquad n\in{{\mathbb Z}}_+.
\end{equation}

\begin{remark}\label{RfactorRH}
In fact we haven't fully used the Riemann-Hilbert problem to recover the wave function
and transition matrices in this case. What we actually exploited is the fact that the solution admits the following factorization
\[
Y_n(z)=R_n(z)\begin{pmatrix}
                1 & \int_{{\mathbb R}}  \frac{w(x)\,dx}{x-z}  \\
                0 & 1
            \end{pmatrix},
\]
where $R_n(z)$ is a matrix polynomial that has the form
\[
R_n=L_{n-1,0}\dots L_{0,0}.
\]
Basically, $R_n$ has a structure similar to that of ${{\mathcal W}}_{[n,0]}$ (see formula \eqref{helpW}).
Moreover, $R_n$ coincides with ${{\mathcal W}}_{[n,0]}$ up to a constant factor and the inversion.
Now, we can clearly see that what we really need here is the Cauchy transform $\int_{{\mathbb R}}  \frac{w(x)dx}{x-z}$ and its asymptotic 
behavior at infinity
in order to have \eqref{RHPYc}. Therefore, it is clear that one can repeat all the steps for any Borel measure with finite moments of all orders.
In other words, we have arrived at the Schur-Euclid algorithm:
\begin{enumerate}
    \item[(i)]
        we start with the function
        \[
       Y(z)=Y_0(z)=\begin{pmatrix}
                1 & \int_{{\mathbb R}}  \frac{d\mu(x)}{x-z}  \\
                0 & 1
            \end{pmatrix},
         \]
         where $d\mu$ is a probability Borel measure with finite moments of all orders;
    \item[(ii)]
        having constructed $Y_n$, we look for the transition matrix $L_{n,0}$ of the form \eqref{OP_transM} such that
        the function
        \[
        Y_{n+1}=L_{n,0}Y_n
        \]
        obeys the asymptotic condition \eqref{RHPYc}.
\end{enumerate}
Let us emphasize that the transition matrix in step (ii) is uniquely determined due to the construction.

We will see in the next section that Riemann-Hilbert problems
admit generalizations in higher dimensions. Thus, they can serve as a tool to develop the multidimensional
Schur-Euclid algorithm.
\end{remark}

\section{{Her\-mite-Pad\'e}{} and Multiple orthogonal polynomials}\label{sec:4}

Here we present a discrete integrable system associated with a family of {Her\-mite-Pad\'e}{} approximants  and multiple orthogonal polynomials.

\subsection{Two-dimensional recurrence relations}\label{sec:41}

We begin by recalling a generalization of orthogonal polynomials to {Her\-mite-Pad\'e}{} polynomials $P_{n,m}$ for two functions $(f_1,f_2)$, which are analytic in a neighbourhood of infinity. It follows from the Cauchy theorem applied to \eqref{0 HP} that {Her\-mite-Pad\'e}{} polynomials satisfy the orthogonality relations
\begin{equation}  \label{4 HPOR}
\oint_{\Gamma}P_{n_1,n_2}(z) \,z^k\,f_j(z)\, dz = 0, \qquad k=0,1,\ldots,n_j-1,\quad  j=1,2,       
\end{equation}
where the contour $\Gamma:=\partial\Omega$ is the boundary of a domain $\Omega\ni\infty$ in which the functions $f_j\in H(\Omega),\,\,j=1,2$ have holomorphic (analytic and single-valued) continuations. We note that the orthogonality relations \eqref{4 HPOR} are non-Hermitian. They actually become Hermitian when the functions $(f_1,f_2)$ are the Cauchy transforms \eqref{0 MOP} of positive measures $d\mu_1(x),\,d\mu_2(x)$ with compact support on the real line. In this case, the coefficients of the Laurent series \eqref{0 f} for $(f_1,f_2)$ can be considered as the moments of $d\mu_1(x),\,d\mu_2(x)$:
\[
s_k^{(j)} = \oint_{\Gamma} z^k\,f_j(z)\, dz \qquad \longrightarrow \qquad s_k^{(j)} = \int x^k\, d\mu_j(x),\quad j=1,2.
\]

Using the determinant of the coefficients $s_k^{(j)}$
\begin{equation} \label{eq:2.12}
   S_{n,m} = \left| \begin{matrix}
                   s_0^{(1)} & s_1^{(1)} & \cdots & s_{n-1}^{(1)} \\
                   s_1^{(1)} & s_2^{(1)} & \cdots & s_{n}^{(1)} \\
	            \vdots & \vdots & \cdots & \vdots \\
                   s_{n+m-1}^{(1)} & s_{n+m}^{(1)} & \cdots & s_{2n+m-2}^{(1)} \end{matrix}		
           \begin{matrix}
                   s_0^{(2)} & s_1^{(2)} & \cdots & s_{m-1}^{(2)} \\
                   s_1^{(2)} & s_2^{(2)} & \cdots & s_{m}^{(2)} \\
	            \vdots & \vdots & \cdots & \vdots \\
                   s_{n+m-1}^{(2)} & s_{n+m}^{(2)} & \cdots & s_{n+2m-2}^{(2)} 		
                  \end{matrix} \right|,
\end{equation}
we can write a formula for the {Her\-mite-Pad\'e}{} polynomials
\small
\[    P_{n,m}(x) = \frac{1}{S_{n,m}}
          \left|\begin{matrix}
                   s_0^{(1)} & s_1^{(1)} & \cdots & s_{n-1}^{(1)} \\
                   s_1^{(1)} & s_2^{(1)} & \cdots & s_{n}^{(1)} \\
	            \vdots & \vdots & \cdots & \vdots \\
                   s_{n+m}^{(1)} & s_{n+m+1}^{(1)} & \cdots & s_{2n+m-1}^{(1)} \end{matrix}		
           \begin{matrix}
                   s_0^{(2)} & s_1^{(2)} & \cdots & s_{m-1}^{(2)} \\
                   s_1^{(2)} & s_2^{(2)} & \cdots & s_{m}^{(2)} \\
	            \vdots & \vdots & \cdots & \vdots \\
                   s_{n+m}^{(2)} & s_{n+m+1}^{(2)} & \cdots & s_{n+2m-1}^{(2)} 		
                  \end{matrix}
          \begin{matrix} 1 \\ x \\ \vdots \\ x^{n+m} \end{matrix} \right|
\]
\normalsize
provided that $S_{n,m}$ is nonvanishing. The latter case is a criterion of normality of the index $(n,m)$.
In this paper we assume that
all multi-indices are normal and we investigate the nearest-neighbor recurrence relations.

In \cite{WVAGerKui} a matrix Riemann-Hilbert problem formulation for multiple orthogonal was proposed. Here we slightly generalize this approach for the case of {Her\-mite-Pad\'e}{} polynomials. We can formulate the following Riemann-Hilbert problem:
find a $3\times 3$ matrix function $Y$ such that
\begin{enumerate}
\item[(i)] $Y$ is analytic on $\mathbb{C} \setminus \Gamma,\,\,\,$(i.e. $Y\in H(\Omega)$ and $Y\in H(\mathbb{C} \setminus \overline{\Omega})$ ),
\item[(ii)] the continuous limits $Y_{+}(x): = \displaystyle\lim_{\Omega\ni\,\xi\rightarrow x\in \,\Gamma } Y(\xi)$,
$Y_{-}(x): = \displaystyle\lim_{\{\mathbb{C} \setminus \overline{\Omega}\}\ni\,\xi\rightarrow x\in \,\Gamma } Y(\xi)$ exist and
\begin{equation} \label{eq:2.1}
      Y_+(x) = Y_-(x) \begin{pmatrix}
                        1 & f_1(x) & f_2(x) \\
                        0 &  1  & 0  \\
                        0 & 0 & 1
                        \end{pmatrix}, \qquad x \in \Gamma,
\end{equation}
\item[(iii)] for $z \to \infty$ one has
\begin{equation} \label{eq:2.2}
     Y(z) = \big( I + \mathcal{O}(1/z) \big) \begin{pmatrix} z^{n+m} & 0 & 0 \\ 0 & z^{-n} & 0 \\ 0 & 0 & z^{-m} \end{pmatrix}.
\end{equation}
\end{enumerate}
Following \cite{FIK1992}, \cite{WVAGerKui} it is easy to show that this Riemann-Hilbert problem
has a unique solution in terms of  the {Her\-mite-Pad\'e}{} when $(n,m)$, $(n-1,m)$ and $(n,m-1)$
are normal indices, i.e.,
\begin{equation}  \label{eq:2.3}
  Y = \begin{pmatrix}
      P_{n,m} & C_1(P_{n,m}) & C_2(P_{n,m}) \\
      c_1(n,m) P_{n-1,m} & c_1 C_1(P_{n-1,m}) & c_1 C_2(P_{n-1,m}) \\
      c_2(n,m) P_{n,m-1} & c_2 C_1(P_{n,m-1}) & c_2 C_2(P_{n,m-1})
      \end{pmatrix}
\end{equation}
where the Cauchy transform is used
\[
C_j(P) = \frac{1}{2\pi i} \oint_{\Gamma} \frac{P(x)f_j(x)}{x-z}\, dx, \qquad j=1,2,
\]
and the constants $c_1$ and $c_2$ are given by 
$$\frac{-2\pi i}{c_1(n,m)} =  \oint_{\Gamma} P_{n-1,m}(x) x^{n-1}f_1(x)\, dx,  \,\,
\frac{-2\pi i}{c_2(n,m)} =  \oint_{\Gamma}P_{n,m-1}(x) x^{m-1}f_2(x)\, dx. $$

One of the natural outcomes of representing the {Her\-mite-Pad\'e}{} polynomials in the form of Riemann-Hilbert problems
is that the nearest-neighbor recurrence relations come out in a natural way.

\begin{proposition} \label{thm:1}
Suppose all multi-indices $(n,m) \in{{\mathbb Z}}_+^2$ are normal.
Then the {Her\-mite-Pad\'e}{}  polynomials satisfy the system of recurrence
relations \eqref{0 RR}.
\end{proposition}
\begin{proof} As we already mentioned in the introduction, the recurrence relations \eqref{0 RR} for multiple orthogonal polynomials were obtained in \cite{vanA2011}. Here, we will follow the same approach.
Actually, the proof presented in \cite{vanA2011} uses the Riemann-Hilbert problem but, as we see later, the main ingredient of that proof
is a factorization similar to the one revealed in Remark \ref{RfactorRH}. Basically,
the proof goes along the same lines as the construction of the wave function from the Riemann-Hilbert
problem in the case of orthogonal polynomials (see Section 3.2).

Let us start by making the standard observation that $\det Y$ is an analytic function in $\mathbb{C} \setminus \mathbb{R}$
with no jump on the contour $\Gamma$. Hence $\det Y$ is an entire function and its behavior near infinity
is $\det Y(z) = 1 + \mathcal{O}(1/z)$. Thus, by Liouville's theorem we find that $\det Y = 1$.   We can therefore consider
the matrix
\[
L_{n,m} = Y_{n+1,m} Y_{n,m}^{-1},
\]
where the subscript $(n,m)$ is used for the solution \eqref{eq:2.3} of the Riemann-Hilbert problem with
the polynomial $P_{n,m}$ in the entry of the first row and the first column of $Y_{n,m}$.
Clearly $L_{n,m}$ is an analytic function on $\mathbb{C} \setminus \mathbb{R}$, and since $Y_{n,m}$
and $Y_{n+1,m}$ have the same jump matrix on $\mathbb{R}$ we see that $L_{n,m}$ has no jump on $\mathbb{R}$.
Hence $R_1$ is an entire matrix function. If we write the asymptotic condition \eqref{eq:2.2} as
\[    Y_{n,m}(z) = \left( I + \frac{A(n,m)}{z} + \mathcal{O}(1/z^2) \right)
		\begin{pmatrix} z^{n+m} & 0 & 0 \\ 0 & z^{-n} & 0 \\ 0 & 0 & z^{-m} \end{pmatrix}, \]
where $A(n,m)$ is the $3\times 3$ matrix coefficient of $1/z$ in the $\mathcal{O}(1/z)$ term of \eqref{eq:2.2}, then after some calculus
and in view of Liouville's theorem we find
\begin{equation} \label{eq:2.8}
    L_{n,m}  = \begin{pmatrix}
                  z+A_{1,1}(n+1,m)-A_{1,1}(n,m) & -A_{1,2}(n,m) & -A_{1,3}(n,m) \\
                  A_{2,1}(n+1,m) & 0 &  0 \\
                  A_{3,1}(n+1,m) & 0 &  1
                 \end{pmatrix},
\end{equation}
where $A_{i,j}(n,m)$ is the entry on row $i$ and column $j$ of $A(n,m)$.
We can therefore write
\begin{equation} \label{eq:2.9}
      Y_{n+1,m} =L_{n,m}  Y_{n,m}.
\end{equation}
In a similar way we also have
\begin{equation} \label{eq:2.10}
      Y_{n,m+1} = M_{n,m}  Y_{n,m},
\end{equation}
with
\begin{equation} \label{eq:2.11}
    M_{n,m}  = \begin{pmatrix}
                  z+A_{1,1}(n,m+1)-A_{1,1}(n,m) & -A_{1,2}(n,m) & -A_{1,3}(n,m) \\
                  A_{2,1}(n,m+1) & 1 &  0 \\
                  A_{3,1}(n,m+1) & 0 &  0
                 \end{pmatrix}.
\end{equation}
Now, introducing
\begin{equation} \label{cd}
    c_{n,m} = A_{1,1}(n,m)-A_{1,1}(n+1,m), \quad
    d_{n,m} = A_{1,1}(n,m)-A_{1,1}(n,m+1)
\end{equation}
and
\begin{equation} \label{ab}   a_{n,m} = c_1(n,m)A_{1,2}(n,m), \quad b_{n,m} = c_2(n,m)A_{1,3}(n,m)\end{equation}
we see that the $(1,1)$-entry of \eqref{eq:2.9} gives the first relation in \eqref{0 RR}
$$ P_{n+1,m}(x) = (x-c_{n,m})P_{n,m}(x) - a_{n,m} P_{n-1,m}(x) - b_{n,m} P_{n,m-1}(x),$$
and \eqref{eq:2.10} gives the second relation in \eqref{0 RR}
$$P_{n,m+1}(x) = (x-d_{n,m})P_{n,m}(x) - a_{n,m} P_{n-1,m}(x) - b_{n,m} P_{n,m-1}(x). $$
\end{proof}

\subsection{Discrete integrable systems represented by $3\times 3$ matrices}\label{sec:42}
Another immediate consequence of the reformulation of {Her\-mite-Pad\'e}{} approximation in terms of Riemann-Hilbert problems is a bridge between the corresponding recurrence relations and discrete integrable system whose transition matrices are $3\times 3$ matrices.
\begin{proposition}\label{Pr2}
Let $(f_1,f_2)$ be a perfect system., i.e., all the multi-indices $(n,m)$ are normal. 
Then there exists a wave function \eqref{eq:2.3} on ${{\mathbb Z}}_+^2$ and its
transition matrices are given by \eqref{Int_eq:2.8}:
\begin{equation*}
    L_{n,m}  = \begin{pmatrix}
                  x+ \alpha_{n,m}^{(1)}& \alpha_{n,m}^{(2)} & \alpha_{n,m}^{(3)} \\
                  \alpha_{n+1,m}^{(4)} & 0 &  0 \\
                  \alpha_{n+1,m}^{(5)} & 0 &  1
                 \end{pmatrix},\qquad
    M_{n,m}  = \begin{pmatrix}
                  x+\beta_{n,m}^{(1)} & \alpha_{n,m}^{(2)} & \alpha_{n,m}^{(3)} \\
                  \alpha_{n,m+1}^{(4)} & 1 &  0 \\
                  \alpha_{n,m+1}^{(5)} & 0 &  0
                 \end{pmatrix},
\end{equation*}
whose entries are related to the coefficients of the recurrence relations \eqref{0 RR} for the {Her\-mite-Pad\'e}{} polynomials of the
functions $f_1$ and $f_2$ as follows:
\begin{equation} \label{abcd}
    c_{n,m} = -\alpha_{n,m}^{(1)}, \quad
    d_{n,m} = -\beta_{n,m}^{(1)}, \quad
\   a_{n,m} = -\alpha_{n,m}^{(4)}\alpha_{n,m}^{(2)}, \quad b_{n,m} = -\alpha_{n,m}^{(5)}\alpha_{n,m}^{(3)}.
\end{equation}
\end{proposition}

\begin{proof}
We take a function $Y$ of the form \eqref{eq:2.3}, then \eqref{eq:2.9} and \eqref{eq:2.10} give us
transition matrices $L_{n,m},\,M_{n,m}$ of the form  \eqref{eq:2.8} and \eqref{eq:2.11}.
Taking into account that the normalizing factors in \eqref{ab} are
$$ c_1(n,m) = A_{2,1}(n,m)\quad \mbox{and} \quad c_2(n,m) = A_{3,1}(n,m),$$
the relations \eqref{ab} and \eqref{cd} give \eqref{abcd}.
Finally, we notice that to prove \eqref{eq:2.9} and \eqref{eq:2.10} we only used the fact that $Y$
admits the following factorization
\[
Y(z)=R(z)\begin{pmatrix}
                        1 &\oint_{\Gamma} \frac{f_1(x)}{x-z}\,dx& \oint_{\Gamma} \frac{f_2(x)}{x-z}\,dx\\
                        0 &  1  & 0  \\
                        0 & 0 & 1
                        \end{pmatrix},
\]
where $R$ is a matrix polynomial.

\end{proof}

\begin{remark}
Amazingly, the algorithm behind this provides us with the tool for factorizing
matrix polynomials.
At the same time, it should be stressed that the described scheme to find transition matrices can be considered as a two-dimensional generalization of the Schur-Euclid algorithm that can be used to define two-dimen\-sional continued fractions:
\begin{enumerate}
    \item[(i)]
        we start with the function
        \[
       Y(z)=Y_0(z)=\begin{pmatrix}
                        1 &\oint_{\Gamma} \frac{f_1(x)}{x-z}\,dx& \oint_{\Gamma} \frac{f_2(x)}{x-z}\,dx\\
                        0 &  1  & 0  \\
                        0 & 0 & 1
                        \end{pmatrix},
         \]
         where $f_1$ and $f_2$ are Laurent series \eqref{0 f};
    \item[(ii)]
        having constructed $Y_{n,m}$, we look for the transition matrices $L_{n,m}$ and $M_{n,m}$ of the
        form \eqref{eq:2.8} and \eqref{eq:2.11}, such that
        the functions
        \[
        Y_{n+1,m} =L_{n,m}  Y_{n,m},\quad
        Y_{n,m+1} = M_{n,m}  Y_{n,m},
        \]
        obey the corresponding asymptotic condition \eqref{eq:2.2}.
\end{enumerate}
Note that the transition matrices in step (ii) are uniquely determined due to the construction.
\end{remark}

Now we simplify the zero curvature condition
\[
0 = L_{n,m+1} M_{n,m} -  M_{n+1,m} L_{n,m},
\]
to the form of \eqref{0 DiffEq}.
\begin{proof}[Proof of Theorem~\ref{T0 2}] In \cite{vanA2011} the consistency condition for the  recurrence coefficients of \eqref{0 RR} was obtained in the following form:
\smallskip

\begin{equation} \label{eq:3.1}\begin{cases}
    d_{n+1,m}-d_{n,m} = c_{n,m+1} - c_{n,m} \\
    b_{n+1,m}-b_{n,m+1} + a_{n+1,m}-a_{n,m+1} = \det \begin{pmatrix} d_{n+1,m} & d_{n,m} \\
	                                                             c_{n,m+1} & c_{n,m} \end{pmatrix}
  \\
   \displaystyle \frac{a_{n,m+1}}{a_{n,m}} = \frac{c_{n,m}-d_{n,m}}{c_{n-1,m}-d_{n-1,m}}  \\
   \displaystyle \frac{b_{n+1,m}}{b_{n,m}} = \frac{c_{n,m}-d_{n,m}}{c_{n,m-1}-d_{n,m-1}}
\end{cases}.\end{equation}
\smallskip

Using the first equation in \eqref{eq:3.1}, we subtract the columns of the determinant of the second equation in \eqref{eq:3.1}. We thus obtain the first two equations of \eqref{0 DiffEq}. The third and fourth equations of \eqref{0 DiffEq} and \eqref{eq:3.1} are the same.
\end{proof}

\begin{remark}
There are other systems related to the orthogonality concepts for which the consistency leads to non-trivial zero curvature conditions  \cite{PGR1995}, \cite{SNderK2011}, \cite{SpZh}.
\end{remark}

It turns out that the consistency conditions \eqref{eq:3.1} (or, equivalently, the zero curvature condition) are also sufficient for a sequence of {Her\-mite-Pad\'e}{} polynomials
to exist and correspond to a perfect system of functions. To complete the proof of  Theorem~\ref{T0 2} it remains to prove the following result.

\begin{proposition}
Suppose that the zero curvature condition
\[
0 = L_{n,m+1} M_{n,m} -  M_{n+1,m} L_{n,m},
\]
holds for a family of invertible matrices $L_{n,m}$ and $M_{n,m}$ of the form \eqref{eq:2.8} and \eqref{eq:2.11}.
Then there are two functions $f_1$ and $f_2$ such that the polynomials $P_{n,m}$ satisfying the corresponding
relations \eqref{0 RR} are the {Her\-mite-Pad\'e}{} polynomials for $f_1$ and $f_2$.
\end{proposition}

\begin{proof}
To determine the functions we first construct the polynomials $P_{n,0}$ and $P_{0,m}$. This can be done since they satisfy ordinary three-term recurrence relations.
So these polynomials are orthogonal polynomials due to the Favard theorem. Let $f_1$ be the function corresponding to $P_{n,0}$
and let $f_2$ be the function for $P_{0,m}$. Next, the  consistency conditions \eqref{eq:3.1}
allow us to define $Y_{n,m}$ in a unique way for all pairs $(n,m)\in{{\mathbb Z}}_+^2$. Due to the asymptotic condition \eqref{eq:2.2},
the first column of $Y_{n,m}$ consists of {Her\-mite-Pad\'e}{} polynomials. At the same time, these polynomials
coincide with $P_{n,m}$. Some more details on how to reconstruct the sequence $P_{n,m}$ from the marginal orthogonal
polynomials are given in \cite{FHVanA}.
\end{proof}

To conclude this subsection, note that we are again dealing with the wave function $\Psi_{n,m}$ that
coincides with $Y_{n,m}$ for $(n,m)\in{{\mathbb Z}}_+^2$ and is extended to ${{\mathbb Z}}^2$ by the symmetry \eqref{SymW}.

\subsection{Systems of measures that generate discrete integrable systems}\label{sec:43}

One of the main obstacles to construct a table of multiple orthogonal polynomials is to ensure that each index is normal, that is, the corresponding determinant $S_{n,m}$ is non-zero. This issue was addressed for the first time by K. Mahler \cite{Mah}, who coined the notion of perfect systems. 
To be more precise, a system of two measures is called perfect if each index in the corresponding table is normal, i.e. $S_{n,m}\ne 0$ for all $n,m\in{{\mathbb Z}}_+$. In this section we give two rather general classes of perfect systems. In turn, these systems give rise to an infinite number of examples of discrete integrable systems.

\subsubsection{Angelesco systems.} A. Angelesco considered in \cite{Ang} the following systems of measures. Let $\Delta_1$ and $\Delta_2$ be disjoint bounded intervals on the real line and $\mu_1$ and $\mu_1$ be a system of measures such that
${{\rm supp\,}} \mu_j = \Delta_j$.

Fix $\vec{n} \in {{\mathbb Z}}_+^{2}$ and consider the multiple orthogonal polynomials of the so called Angelesco system $(\widehat{\mu}_1,\widehat{\mu}_2)$ relative to $\vec{n}$; here  $\widehat{\mu}$ denotes the Cauchy transform of $\mu$:
\[
\widehat{\mu}(z) = \int \frac{d\mu(x)}{z-x}.
\]
By construction, we have that
\[ \int x^{\nu} P_{\vec{n}}(x) \, d\mu_j(x) =0, \qquad \nu = 0,\ldots,n_j -1,\quad j=1,2.
\]
Therefore, $P_{\vec{n}}$ has $n_j$ simple zeros in the interior (with respect to the Euclidean topology of ${{\mathbb R}}$) of $\Delta_j$. As a consequence, since the intervals $\Delta_j$ are disjoint, ${\operatorname{deg}} P_{\vec{n}} = |\vec{n}|$ and
\textit{Angelesco systems are perfect}.

Unfortunately, Angelesco's paper received little attention and such systems  reappeared many years later in \cite{Nik1} where E.M. Nikishin  deduced some of their formal properties.

\subsubsection{Nikishin systems}

Another class of systems for which the perfectness was proved only recently was introduced by E.M. Nikishin  \cite{Nik}.

To get an idea about these systems let us consider two non intersecting bounded intervals $\Delta_{1}, \Delta_{1}$ on the real line. Suppose we are given two measures $\sigma_{1}$ and $\sigma_{1}$ supported
on $\Delta_{1}$ and $\Delta_{1}$, respectively. With these two measures we define a third one in the following way
\[ d\left<\sigma_{1},\sigma_{2}\right>(x) = \widehat{\sigma}_{2}(x)\, d\sigma_{1}(x);
\]
that is, one multiplies the first measure by a weight formed by the Cauchy transform of the second measure. Thus, we have arrived to the notion of a Nikishin system. A system of two measures $(\mu_1, \mu_2)$ of the form
\[
d\mu_1(x)=d\sigma_{1}(x), \qquad d\mu_2(x)=\int \frac{d\sigma_1(t)}{x-t}\, d\mu_1(x)
=d\left<\sigma_{1},\sigma_{2}\right>
\]
is called a Nikishin system.
Let us emphasize here that the measures from a Nikishin system have the same support,
which is a totally different situation than in the case of an Angelesco system.
The notion of Nikishin systems can be generalized to any finite number of measures \cite{AptLopRo}.
Let us mention here that it was a long standing problem to prove that \textit{a general Nikishin system
$(\mu_1, \mu_2, \dots, \mu_p)$ is perfect for $p\ge 2$}. This fact was finally proved
in the remarkable paper \cite{FL}.

\begin{thebibliography}{99}

\bibitem{Adler2001}
V. E. Adler, 
\textit{Discrete equations on planar graphs. Symmetries and integrability of difference equations (Tokyo, 2000)},
J. Phys. A \textbf{34} (2001), no.~48, 10453--10460.

\bibitem{Ach1961}
N. I. Akhiezer, 
\textit{The Classical Moment Problem and Some Related Problems in Analysis},
Hafner publishing Co., New York, 1965.

\bibitem{Ang}
A. Angelesco, 
\textit{Sur deux extensions des fractions continues alg\'ebriques}, 
C.R. Acad. Sci. Paris \textbf{18} (1919), 262--263.

\bibitem{Aper} 
R. Ap\'ery, 
\textit{Irrationalit\'e de $\zeta(2)$ et $\zeta(3)$}, 
Ast\'erisque \textbf{61} (1979), 11--13.

\bibitem{Apt}  
A.~I.~Aptekarev,
\textit{Multiple orthogonal polynomials},
J.\ Comput.\ Appl.\ Math.\ \textbf{99} (1998), no.~1--1, 423--448.

\bibitem{Apt2011}
A.~I.~Aptekarev,
\textit{Rational approximants for Euler's constant and recurrence relations},
Proceedings of the Steklov Institute of Mathematics \textbf{272}  (2011), 138--141.

\bibitem{AptKu}
A. I. Aptekarev, A. B. J.  Kuijlaars, 
\textit{Hermite-Pad\'e approximations and multiple orthogonal polynomial ensembles},  
Russ. Math. Surv. \textbf{66} (2011), no.~6, 1133--1200.

\bibitem{AptLopRo}
A. I. Aptekarev, G. L. L\'opez, I. A. Rocha, 
\textit{Ratio asymptotics of Hermite-Pad\'e polynomials for Nikishin systems}, 
Mat. Sb.  \textbf{196}  (2005),  no.~8, 3--20 (in Russian);  translation in  Sb. Math.  \textbf{196}  (2005),  no.~7--8, 1089--1107. 

\bibitem{B2004}
A. I. Bobenko, 
\textit{Discrete differential geometry. Integrability as consistency}, 
Discrete integrable systems, 85--110, Lecture Notes in Phys. \textbf{644}, Springer, Berlin, 2004.

\bibitem{BS2002}
A. I. Bobenko, Yu. B. Suris, 
\textit{Integrable systems on quad-graphs}, 
Int. Math. Res. Not. 2002, no. 11, 573--611.

\bibitem{FL}
U. Fidalgo Prieto, G. L\'opez Lagomasino, 
\textit{Nikishin systems are perfect}, 
Constr. Approx. \textbf{34} (2011), no.~3, 297--356.

\bibitem{FHVanA}
G. Filipuk, M. Haneczok, W. Van Assche, 
\textit{Computing recurrence coefficients of multiple orthogonal polynomials},
arXiv:1406.0364.

\bibitem{FIK1992}
A.S. Fokas, A.R. Its, A.V. Kitaev, 
\textit{The isomonodromy approach to matrix models in 2D quantum gravity},
Comm. Math. Phys. \textbf{147} (1992), no.~2, 395--430.

\bibitem{Her} 
C.~Hermite,
\textit{Sur la fonction exponentielle},
C.R. Acad.\ Sci.\ Paris \textbf{77} (1873), 18--24; 74--79; 226--233.

\bibitem{Ismail} 
M.E.H. Ismail,
\textit{Classical and Quantum Orthogonal Polynomials in One Variable},
Encyclopedia of Mathematics and its Applications \textbf{98},
Cambridge University Press, 2005.

\bibitem{K2010}
A. B. J. Kuijlaars, 
\textit{Multiple orthogonal polynomials in random matrix theory}, 
Proceedings of the International Congress of Mathematicians. Volume III, 1417--1432, 
Hindustan Book Agency, New Delhi, 2010.

\bibitem{PGR1995}
V. Papageorgiou, B. Grammaticos, A. Ramani, 
\textit{Orthogonal polynomial approach to discrete Lax pairs for initial-boundary value problems of the QD algorithm}, 
Lett. Math. Phys. \textbf{34} (1995), no. 2, 91--101.

\bibitem{Mah} 
K. Mahler,
\textit{Perfect systems},
Compos. Math. \textbf{19} (1968), no.~2,  95--166.

\bibitem{Nik1}
E. M. Nikishin, 
\textit{A system of Markov functions}, 
Vestnik Moskov. Univ. Ser. I Mat. Mekh (1979):4, 60--63 (in Russian); translation in Moscow Univ. Math. Bull. \textbf{34} (1979), 63--66.

\bibitem{Nik}
E. M. Nikishin, 
\textit{On simultaneous Pad\'e approximants}, 
Matem. Sb. \textbf{113} (1980), 499--519 (in Russian); translation in Math. USSR Sb.  \textbf{41} (1982), 409--425.

\bibitem{Nov}
S. P. Novikov,
\textit{Four Lectures: Discretization and Integrability. Discrete Spectral Symmetries}, 
in ``Integrability'',
Lecture Notes in Physics \textbf{767}, Springer, Berlin, 2009, pp.~119--138.

\bibitem{Sakh1997}
L. A. Sakhnovich, 
\textit{Interpolation theory and its applications}, 
Mathematics and its Applications \textbf{428},
Kluwer Academic Publishers, Dordrecht, 1997.

\bibitem{Sakh1999}
L. A. Sakhnovich, 
\textit{Spectral theory of canonical differential systems. Method of operator identities},
Operator Theory: Advances and Applications \textbf{107}, Birkh\"auser Verlag, Basel, 1999.

\bibitem{S2010}
S. Smirnov, 
\textit{Discrete complex analysis and probability}, 
Proceedings of the International Congress of Mathematicians. Volume I, 595--621, 
Hindustan Book Agency, New Delhi, 2010.

\bibitem{SNderK2011}
P. E. Spicer, F. W. Nijhoff, P. H. van der Kamp,
\textit{Higher analogues of the discrete-time Toda equation and the quotient-difference algorithm},
Nonlinearity \textbf{24} (2011), no. 8, 2229--2263.

\bibitem{SpZh}
V. Spiridonov, A. Zhedanov, 
\textit{Spectral transformation chains and some new biorthogonal rational functions}, 
Comm. Math. Phys. \textbf{210} (2000), no. 1, 49--83.

\bibitem{Suris}
Y. B. Suris, 
\textit{Bi-Hamiltonian structure of the $qd$ algorithm and new discretizations of the Toda lattice},
Phys. Lett. A \textbf{206} (1995), no.~3--4, 153--161.

\bibitem{vanA1999}
 W. Van Assche, 
\textit{Multiple orthogonal polynomials, irrationality and  transcendence}, 
in ``Continued fractions: from analytic number theory to constructive approximation'',
Contemporary Mathematics \textbf{236} (1999), 325-342.

\bibitem{vanA2001}
W. Van Assche,
\textit{ Little $q$-Legendre polynomials and irrationality of certain Lambert series}, 
Ramanujan J. \textbf{5} (2001), 295--310.

\bibitem{vanA2011}
W. Van Assche, 
\textit{Nearest neighbor recurrence relations for multiple orthogonal polynomials},
J. Approx. Theory \textbf{163} (2011), no. 10, 1427--1448.

\bibitem{WVAGerKui} 
W. Van Assche, J. S. Geronimo, A. B. J. Kuijlaars,
\textit{Riemann-Hilbert problems for multiple orthogonal polynomials},
Special functions 2000: current perspective and future directions (Tempe, AZ),
NATO Sci. Ser. II Math.\ Phys.\ Chem.\ \textbf{30}, Kluwer Acad. Publ., Dordrecht, 2001, pp.~23--59.
\end{thebibliography}
\end{document}


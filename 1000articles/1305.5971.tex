\documentclass[10pt]{amsart}

\usepackage{amsfonts,amssymb,amscd,amstext}
\usepackage[a4paper,hmargin=3.5cm,vmargin=4cm]{geometry}
\usepackage{hyperref}
\usepackage{graphicx}
\usepackage[utf8]{inputenc}
\usepackage{esint}
\usepackage{enumerate}

\newtheorem{theorem}{Theorem}[section]
\newtheorem{proposition}[theorem]{Proposition}
\newtheorem{lemma}[theorem]{Lemma}
\newtheorem{corollary}[theorem]{Corollary}

\theoremstyle{definition}

\newtheorem{example}[theorem]{Example}

\newtheorem{problem}{Problem}

\newtheorem{definition}{Definition} 

\theoremstyle{remark}

\newtheorem{remark}[theorem]{Remark}

\newenvironment{enum}\begin{enumerate}\end{enumerate}

\numberwithin{equation}{section}

\setcounter{tocdepth}{1}

\begin{document}

\title[Area-stationary and stable surfaces in Sol]{On the classification of complete area-stationary and stable surfaces in the sub-Riemannian Sol manifold}

\author[M.~Galli]{Matteo Galli} \address{Departamento de
Geometr\'{\i}a y Topolog\'{\i}a \\
Universidad de Granada \\ E--18071 Granada \\ Espa\~na}
\email{galli@ugr.es}

\date{\today}

\thanks{Research supported by  MCyT-Feder grant MTM2010-21206-C02-01 and J. A. grant P09-FMQ-5088}
\subjclass[2000]{53C17, 49Q20} 
\keywords{sub-Riemannian geometry, area-stationary surfaces, stable surfaces, pseudo-hermitian manifolds, Sol geometry}

\begin{abstract}
We study the classification of area-stationary and stable $C^2$ regular surfaces in the space of the rigid motions of the Minkowski plane ${E(1,1)}$, equipped with its sub-Riemannian structure. We construct examples of area-stationary surfaces that are not foliated by sub-Riemannian geodesics. We also prove that there exist an infinite number of $C^2$ area-stationary surfaces with a singular curve. Finally we show the stability of $C^2$ area-stationary surfaces foliated by sub-Riemannian geodesics. 
\end{abstract}

\maketitle

\thispagestyle{empty}

\bibliographystyle{amsplain}

\tableofcontents

\section{Introduction}

The study of the sub-Riemannian area functional in three-dimensional pseudo-hermitian manifolds and in other sub-Riemannian spaces has been largely investigated in the last years, see \cite{Am-SC-Vi, MR2831583, BA-SC-Vi, MR2600502, MR2583494,  ChengHwang2nd,  CJHMY2,  CJHMY, CHY, DGNAV, DGNnotable,  Da-Ga-Nh-Pa, MR3044134, MR2979606, Ga-Nh,  Hl-Pa2, Ri-Ro-Hu,  Hu-Ro, Hu-Ro2, Ri1,  Ri-Ro, Ro,      Sh}, among others. 

One of the more interesting questions concerning the sub-Riemannian area functional is:

\begin{problem}\label{domanda1}
\emph{Which are the area-minimizing surfaces in a given three-dimensional contact sub-Riemannian manifold?}
\end{problem}

A surface ${\Sigma}$ is \emph{area-minimizing} if ${A}({\Sigma}){\leqslant} {A}(\tilde{\Sigma})$, for any compact deformation $\tilde{\Sigma}$ of ${\Sigma}$. To answer the previous question, a natural preliminary step is the study of the \emph{area-stationary} surfaces, the critical points of the area functional.

\begin{problem}\label{domanda2}
\emph{Which are the area-stationary surfaces in a given three-dimensional contact sub-Riemannian manifold?}
\end{problem}

We will consider these questions in the class of $C^2$ regular surfaces. For a general introduction about the study of the area functional in sub-Riemannian spaces, we refer the interested reader to \cite{CDPT} and \cite{Gaphd}, that treat the case of ${{\mathbb{H}}}^n$ and  the contact sub-Riemannian manifolds respectively.

In Sasakian space forms, the classification of $C^2$ area stationary surfaces was given in \cite{Ri-Ro-Hu} in the case of the Heisenberg group ${{\mathbb{H}}}^n$ and in  \cite{Ro} for the Sasakian structures of  $S^3$ and $\widetilde{SL}_2({{\mathbb{R}}})$.   In the case of pseudo-hermitian three-manifolds that are not Sasakian, the only known results concerning Problem \ref{domanda1}  and Problem \ref{domanda2} are given in \cite{MR3044134}, where the group of the rigid motions of the Euclidean plane $E(2)$ is studied. 

Concerning the three-dimensional pseudo-hermitian manifolds, we have the following classification result, \cite[Theorem~3.1]{Pe}, in terms of the Webster scalar curvature $W$ and of the pseudo-hermitian torsion $\tau$

\begin{proposition}\label{classification} Let  $M$ be a simply connected contact 3-manifold, homogeneous in the sense of Boothby and Wang, \cite{MR0112160}. Then $M$ is one of the following Lie group:
	\begin{itemize}
	\item [(1)] if $M$ is unimodular 
	         \begin{itemize}
	         \item the first Heisenberg group $\mathbb{H}^1$ when $W=|\tau|=0$;
	         \item the three-sphere group $SU(2)$ when $W> 2|\tau|$;
	         \item the group $\widetilde{SL(2,\mathbb{R})}$ when $-2|\tau|\neq W<2|\tau|$;
	         \item the group $\widetilde{E(2)}$, universal cover of the group of rigid motions of the Euclidean plane, when $W=2|\tau|>0$;
	         \item the group $E(1,1)$ of rigid motions of Minkowski 2-space, when $W=-2|\tau|<0$;
	         \end{itemize}
	\item [(2)] if $M$ is non-unimodular, the Lie algebra is given by
	        \[
	        [X,Y]=\alpha Y+2T ,\quad  [X,T]=\gamma Y, \quad  [Y,T]=0, \quad \alpha\neq 0,
	        \]
	        where $\{X,Y\}$ is an orthonormal basis of ${\mathcal{H}}$, $J(X)=Y$ and $T$ is the Reeb vector field. In this case $W<2|\tau|$ and when $\gamma=0$ the structure is Sasakian and $W=-\alpha^2$.
	\end{itemize}
\end{proposition}

About the models of the unimodular case, Problem \ref{domanda1} and Problem \ref{domanda2} are not investigated only for the case of the Sol geometry, modeling by the space ${E(1,1)}$, and its study is the aim of this work. 

After some preliminaries, the paper is organized as follows. 

In Section \ref{sec:carcurves}, we compute explicitly the coordinates of the characteristic curves with given initial conditions. These curves play an important role in the study of area-stationary surfaces, since the regular part ${\Sigma}-{\Sigma}_0$ of a surface ${\Sigma}$ is foliated by characteristic curves, that are not in general sub-Riemannian geodesics, since ${E(1,1)}$ is characterized by a non-vanishing pseudo-hermitian torsion. 

Section \ref{sec:stationary} is the core of the paper. We first characterize the $C^2$ complete, area-stationary surfaces immersed in ${E(1,1)}$ with singular points or singular curves that are sub-Riemannian geodesics. On the other hand, for the first time in the three-dimensional pseudo-hermitian setting, we also find examples of area-stationary surfaces that are not foliated by sub-Riemannian geodesics. We stress that these examples form an infinite family, i.e., given an horizontal curve ${\Gamma}$, we can construct an area-stationary surface having ${\Gamma}$ as singular set ${\Sigma}_0$. 

Finally in Section \ref{sec:minimizing} we prove that complete area-stationary surfaces with non-empty singular set, whose characteristic curves are sub-Riemannian geodesics, are stable. We also find three families of non-singular planes that are area-minimizing, using a calibration argument. 

We remark that Section \ref{sec:minimizing} opens two interesting questions. Is a stable complete area-stationary  surface in ${E(1,1)}$ with a singular curve  always foliated by sub-Riemannian geodesics in ${\Sigma}-{\Sigma}_0$? Do some  other complete stable area-stationary surfaces in ${E(1,1)}$ with empty singular set exist? 

\section{Preliminaries}
\label{sec:preliminaries}

\subsection{The group $E(1,1)$ of rigid motions of the Minkowski plane}
We consider the group of rigid motions of the Minkowski plane $E(1,1)$, that is a unimodular Lie group with a natural sub-Riemannian structure.  As a model of $E(1,1)$ we choose as underlying manifold ${{\mathbb{R}}}^3$ with the following orthonormal basis of left-invariant vector fields
\begin{equation}\label{def:basis}
\begin{split}
X&=\frac{\partial}{\partial z}\\
Y&=\frac{1}{\sqrt{2}}\bigg(-e^{z}\frac{\partial}{\partial x}+e^{-z}\frac{\partial}{\partial y}\bigg)\\
T&=\frac{1}{\sqrt{2}}\bigg(e^{z}\frac{\partial}{\partial x}+e^{-z}\frac{\partial}{\partial y}\bigg).
\end{split}
\end{equation} 
We have that $\{X,Y\}$ is a orthonormal basis of the horizontal distribution ${\mathcal{H}}$ and $T$ is the Reeb vector field. The scalar product of two vector fields $W$ and $V$ with respect to the metric induced by the basis $\{X,Y,T\}$ will be often denoted by ${\big<{W,V}\big>}$. This structure of $E(1,1)$ is characterized by the following Lie brackets, \cite{MR0425012},
\begin{equation}\label{liebrakets}
\begin{split}
[X,Y]=-T \qquad [X,T]=-Y \qquad [Y,T]=0.
\end{split}
\end{equation} 
In fact, applying \cite[eq.~9.1 and eq.~9.3]{MR3044134} we obtain that the Webster scalar curvature is $W=-1/2$ and the matrix of the pseudo-hermitian torsion $\tau$ in the ${X,Y,T}$ basis is 
\[
\left( \begin{array}{ccc} 
0 & -\frac{1}{2}& 0 \\
-\frac{1}{2} & 0& 0\\
0 & 0 & 0 \end{array} \right).
\] 
The following derivatives can be easily computed
\begin{equation}\label{Gchistoffel}
\begin{split}
&{\nabla}_{X}X=0, \quad  {\nabla}_{Y}X=0,  \quad    {\nabla}_{T}X=\frac{1}{2}Y,    \\
&  {\nabla}_{X}Y=0, \quad    {\nabla}_{Y}Y=0, \quad    {\nabla}_{T}Y=-\frac{1}{2}X,
\end{split}
\end{equation}
where ${\nabla}$ denotes the pseudo-hermitian connection, \cite{Dr-To}.
Furthermore we have the characterization $-2|\tau|^2=W<0$ peculiar of $E(1,1)$, \cite{Pe}. We also define the involution $J$, the so-called complex structure, on ${\mathcal{H}}$ by $J(X)=Y$ and $J(Y)=-X$.

\subsection{The geometry of regular surfaces in $E(1,1)$}
We consider a $C^1$ surface ${\Sigma}$ immersed in $E(1,1)$. We define the \emph{sub-Riemannian area} of ${\Sigma}$ as
\[
{A}({\Sigma})=\int\limits_{\Sigma}{|N_{H}|}\, d{\Sigma},
\]
where $N_h$ denotes the projection of the Riemannian unit normal $N$ to ${\mathcal{H}}$ and $d{\Sigma}$ denotes the Riemannian area element on ${\Sigma}$. In the sequel we always denote by $N$ the inner unit normal. The singular set ${\Sigma}_0$ is composed by the points in which $T{\Sigma}$ coincides with ${\mathcal{H}}$. Outside ${\Sigma}_0$, we can define the \emph{horizontal unit normal} 
\[
\nu_h:=\frac{N_h}{|N_h|}
\]
and the \emph{characteristic vector field} as $Z:=J(\nu_h)$. It is straightforward to verify that $\{Z,S\}$ is an orthonormal basis of $T{\Sigma}$ outside ${\Sigma}_0$, where
\[
S:={{\big<{N,T}\big>}} \nu_h-{|N_{H}|} T.
\]
Finally, outside ${\Sigma}_0$, we define the \emph{mean curvature} of ${\Sigma}$ by
\begin{equation}\label{def:H}
H:=-{\big<{{\nabla}_Z \nu_h,Z}\big>}.
\end{equation}
Given a surface ${\Sigma}$ as zero level set of a function $u:\Omega\subset {E(1,1)}\rightarrow {{\mathbb{R}}}$, we can express
\begin{equation}\label{eq:nuhexpression}
\nu_h=-\frac{u_z X+\frac{1}{\sqrt{2}}(-e^z u_x+e^{-z}u_y)Y}{\sqrt{u_z^2+\frac{1}{2}(-e^z u_x+e^{-z}u_y)^2}}
\end{equation}
and
\begin{equation}\label{eq:Zexpression}
Z=\frac{\frac{1}{\sqrt{2}}(-e^z u_x+e^{-z}u_y) X-u_z Y}{\sqrt{u_z^2+\frac{1}{2}(-e^z u_x+e^{-z}u_y)^2}}. 
\end{equation}
We define a \emph{minimal surface} as a surface with vanishing mean curvature $H$. 

\begin{proposition} Let ${\Sigma}$ be a minimal surface defined as the zero level set of a $C^2$ function $u:\Omega\subset{E(1,1)} \rightarrow {{\mathbb{R}}}$. Then $u$ satisfies the equation
\begin{equation}\label{eq:minimalsurface}
\begin{split}
u_{zz}(-e^z u_x+e^{-z}u_y)^2+u_z^2 (-e^{2z}u_{xx}-2u_{xy}+e^{-2z}u_{yy})\\
-u_z (-e^z u_x+e^{-z}u_y) (-2e^z u_{xz}-e^z u_x +2e^{-z}u_{yz}-e^{-z}u_y)=0
\end{split}
\end{equation}
 on $\Omega$.
\end{proposition}

\begin{proof} From \eqref{def:H}, \eqref{eq:nuhexpression} and \eqref{eq:Zexpression} we can find that $u$ has to satisfy
\begin{equation}\label{eq:minimalsurfacehalf}
Y(u)^2 X(X(u))-Y(u)X(u)Y(X(u))-Y(u)X(u)X(Y(u))+X(u)^2Y(Y(u))=0
\end{equation}
on $\Omega$. Now, using \eqref{def:basis}, we can transform \eqref{eq:minimalsurfacehalf} into \eqref{eq:minimalsurface}.
\end{proof}
We will call \eqref{eq:minimalsurface} the \emph{minimal surface equation}. 

\begin{remark}\label{immediateminimaremark} From \eqref{eq:minimalsurfacehalf}, it  is immediate to note that a surface ${\Sigma}$ satisfying $u_z\equiv 0$ or $-e^z u_x+e^{-z}u_y\equiv 0$ is always minimal. 
\end{remark}

In the following Lemma, we compute some important quantities related to the torsion and the geometry of a surface. It follows from \cite[eq.~9.8]{MR3044134},

\begin{lemma}\label{lem:calcolotausullasuperficie}
Let ${\Sigma}$ be a $C^1$ surface in ${E(1,1)}$, then we have
\begin{equation*}
\begin{split}
{\big<{\tau(Z),Z}\big>}&=-{\big<{Z,X}\big>}{\big<{Z,Y}\big>}={\big<{\nu_h,X}\big>}{\big<{\nu_h,Y}\big>}=-{\big<{\tau(\nu_h),\nu_h}\big>},\\
{\big<{\tau(Z),\nu_h}\big>}&=\frac{1}{2}({\big<{Z,Y}\big>}^2- {\big<{Z,X}\big>}^2).
\end{split}
\end{equation*}
\end{lemma}

\section{Characteristic curves in ${E(1,1)}$}\label{sec:carcurves}

In this section we will study the equation of the integral curves of $Z$ on ${\Sigma}$, that are known as \emph{characteristic curves}. It is well-known that a surface with constant mean curvature $H$ is foliated by characteristic curves in ${\Sigma}-{\Sigma}_0$. In general, a \emph{characteristic curve} is an arc-length parametrized horizontal curve ${\gamma}$ in ${E(1,1)}$, that satisfies the equation 
\begin{equation}\label{eq:carcurves}
{\nabla}_{\dot{\gamma}} \dot{\gamma}+HJ(\dot{\gamma})=0,
\end{equation}
where $\dot{\gamma}$ denotes the tangent vector along ${\gamma}$ and $H$ is the (constant) curvature of ${\gamma}$. 
We stress that a curve ${\gamma}$ satisfying  \eqref{eq:carcurves} is not a sub-Riemannian geodesic. In fact a characteristic curve ${\gamma}$ is a sub-Riemannian geodesic if and only if $H=0$ and $\dot{\gamma}$ satisfies the additional equation
\begin{equation}
       {\big<{\tau(\dot{\gamma}),\dot{\gamma}}\big>}=0,
\end{equation}
see \cite[Proposition~15]{Ru}, that forces ${\gamma}$ to be an integral curve of $X$ or $Y$ by Lemma \ref{lem:calcolotausullasuperficie}.

\begin{proposition}\label{lemma:carcurves} Let ${\gamma}$ be a characteristic curve in ${E(1,1)}$ with curvature $H=0$. Then ${\gamma}$ belongs to the family of curves
\begin{equation}
{\gamma}(t)=(x_0+\dot{x}_0 t, y_0+\dot{y}_0 t, z_0)
\end{equation}
or to the family
\begin{equation}
{\gamma}(t)=\bigg(x_0+\frac{\dot{x}_0}{\dot{z}_0}(e^{\dot{z}_0t}-1), y_0-\frac{\dot{y}_0}{\dot{z}_0}(e^{-\dot{z}_0t}-1),
z_0+\dot{z}_0t\bigg),
\end{equation}
where ${\gamma}(0)=(x_0,y_0,z_0)$ and $\dot{\gamma}(0)=(\dot{x}_0,\dot{y}_0,\dot{z}_0)$.
\end{proposition}

\begin{proof} We consider the curve ${\gamma}:I\rightarrow {\Sigma}$, where $I$ denotes an interval. We express ${\gamma}(t)=(x(t),y(t),z(t))$ and we get
\begin{equation}\label{eq:gammadot}
\begin{split}
\dot{\gamma}(t)&=\dot{x} \frac{\partial}{\partial x}+\dot{y} \frac{\partial}{\partial y}+\dot{z} \frac{\partial}{\partial z}\\
&=\dot{z} X+\frac{1}{\sqrt{2}}(\dot{y}e^z-\dot{x}e^{-z})Y+\frac{1}{\sqrt{2}}(\dot{y}e^z+\dot{x}e^{-z})T,
\end{split}
\end{equation}
since 
\begin{equation*}
\begin{split}
\frac{\partial}{\partial x}&=\frac{1}{\sqrt{2}}e^{-z}(T-Y)\\
\frac{\partial}{\partial y}&=\frac{1}{\sqrt{2}}e^{z}(Y+T).
\end{split}
\end{equation*}
From \eqref{eq:gammadot} and the fact that ${\gamma}$ is horizontal, we have
\begin{equation}\label{eq:horcondgamma}
\dot{y}e^z+\dot{x}e^{-z}=0.
\end{equation}
Now ${\nabla}_{\dot{\gamma}}\dot{\gamma}=0$ is equivalent to the system
\begin{equation}\label{eq:nablagaga}
 \begin{cases}
\dot{z}=\dot{z}_0 \\
\dot{y}e^z-\dot{x}e^{-z}=c_0
\end{cases},
\end{equation}
where $\dot{z}_0$ and $c_0$ are constants. We distinguish two cases. The first one corresponds to $\dot{z}_0=0$. This means that $z=z_0$, with $z_0\in{{\mathbb{R}}}$, and so \eqref{eq:horcondgamma} and \eqref{eq:nablagaga} are reduced to
\begin{equation}\label{sistemadaintegrare}
\begin{cases}
2\dot{y}=e^{-z_0}c_0 \\
2\dot{x}=-e^{z_0}c_0
\end{cases},
\end{equation}
that implies ${\gamma}(t)=(x_0-e^{z_0}(c_0/2) t, y_0+e^{-z_0}(c_0/2) t, z_0)$, where $c_0\neq 0$ and $x_0,y_0\in{{\mathbb{R}}}$. 

The second possibility is $\dot{z_0}\neq 0$, that implies $z(t)=z_0+\dot{z}_0 t$, with $z_0\in{{\mathbb{R}}}$. In this case integrating \eqref{sistemadaintegrare} we obtain
\[
{\gamma}(t)=(x_0+\frac{c_0 e^{z_0}}{2\dot{z}_0}-\frac{c_0}{2\dot{z}_0}e^{z_0+\dot{z}_0t}, y_0+\frac{c_0 e^{-z_0}}{2\dot{z}_0}-\frac{c_0}{2\dot{z}_0}e^{-(z_0+\dot{z}_0)t},
z_0+\dot{z}_0t),
\]
where ${\gamma}(0)=(x_0,y_0,z_0)$. Finally, to conclude the result, we note that 
\[
\frac{c_0}{2}=\dot{y}_0e^{z_0}=-\dot{x}_0e^{-z_0}.
\]
\end{proof}

\section{Complete area-stationary surfaces with non-empty singular set in ${E(1,1)}$}\label{sec:stationary}

\subsection{Complete area-stationary surfaces containing isolated singular points}  

The local structure of a $C^1$ surface ${\Sigma}$ with prescribed mean curvature $H\in C$, in a neighborhood of an isolated singular point, is well understood, \cite[Theorem~D and Corollary~E]{CJHMY2}. In the less general case of a bounded mean curvature surface of class $C^2$, applying  \cite[Theorem~B and Section~7]{CJHMY}, we have

\begin{lemma}\label{singularsetthpoint} Let ${\Sigma}$ be a ${C}^2$ oriented immersed surface with constant mean curvature $H$ in ${E(1,1)}$. If $p\in{\Sigma}_{0}$ is an isolated singular point,  then, there exists $r > 0$ and $\lambda\in 
\mathbb{R}$ such that the set described as 
\begin{equation*}
D_{r} (p) = \{\gamma^{H}_{p,v}  (s)| v \in T_{p}{\Sigma}, |v| = 1, s \in [0, r)\}, 
\end{equation*}
is an open neighborhood of $p$ in ${\Sigma}$, where $\gamma^{H}_{p,v}$ denote the characteristic curve starting from $p$ in the direction $v$ with curvature $H$.
\end{lemma}
 
First we construct the unique example, up to contact isometries, of a minimal surface with isolated singular points.  

\begin{proposition}\label{prop:minimalstationarysingularpoint} Let ${\Sigma}$ be a $C^2$ complete, area-stationary surface immersed in ${E(1,1)}$ with $H=0$ and with an isolated singular point $p_0=(x_0,y_0,z_0)$. Then ${\Sigma}=\{(x,y,z)\in{E(1,1)} : e^{z-z_0}(y-y_0)+x-x_0=0\}$.
\end{proposition}

\begin{proof} By Lemma \ref{singularsetthpoint}, the only possible way to construct a complete area-stationary surface, with a singular point $p_0$, is consider the union of all characteristic curves ${\gamma}$ of curvature $0$ with initial conditions ${\gamma}(0)=p_0$ and $\dot{\gamma}(0)\in T_{p_0}{\Sigma}={\mathcal{H}}_{p_0}$, $|\dot{\gamma}(0)|=1$. We can suppose $p_0=0$, since ${E(1,1)}$ is homogeneous. 

We consider the initial velocities
\begin{equation*}
\begin{split}
\dot{\gamma}_\alpha(0)&=\cos(\alpha)X(0)+\sin(\alpha)Y(0)\\
&=\cos(\alpha)\frac{\partial}{\partial z}(0)+\frac{\sin(\alpha)}{\sqrt{2}}\bigg(-\frac{\partial}{\partial x}(0)+\frac{\partial}{\partial y}(0)\bigg),
\end{split}
\end{equation*}
for $\alpha\in[0,2\pi[$. In this way we obtain as characteristic curves
\begin{equation}
{\gamma}_\alpha(t)=\bigg(-\frac{\sin(\alpha)}{\sqrt{2}\cos(\alpha)}(e^{\cos(\alpha)t}-1), -\frac{\sin(\alpha)}{\sqrt{2}\cos(\alpha)}(e^{-\cos(\alpha)t}-1),
\cos(\alpha)t\bigg),
\end{equation}
for $\alpha\in]0,2\pi[$ and ${\gamma}_0(t)=(0,0,t)$ for $\alpha=0$. At this point it is easy show that ${\Sigma}$ is the zero level set of the function $e^zy+x=0$ (or equivalently $e^{-z}x+y=0$), that satisfies \eqref{eq:minimalsurface}.

\end{proof}

\subsection{Complete area-stationary surfaces containing singular curves}
From \cite[Corollary~5.4]{MR3044134} we have 

\begin{lemma}\label{charmeetsingular} Let ${\Sigma}$ be a ${C}^2$  minimal surface with non-empty singular set ${\Sigma}_{0}$ immersed in ${E(1,1)}$. Then ${\Sigma}$ is area stationary if and only if the characteristic curves meet the singular curves orthogonally with respect the metric ${\big<{\, ,\, }\big>}$, induced by the orthonormal basis \eqref{def:basis}. 
\end{lemma}

In the following lemma, we prove that a minimal area-stationary surface can not contain more than a singular curve.

\begin{lemma} Let ${\Sigma}$ be a $C^2$ complete, minimal, area-stationary surface, containing a singular curve ${\Gamma}$, immersed in ${E(1,1)}$. Then ${\Sigma}$ cannot contain more singular curves. 
\end{lemma}

\begin{proof} We consider a singular curve ${\Gamma}({\varepsilon})=(x({\varepsilon}),y({\varepsilon}),z({\varepsilon}))$ in ${\Sigma}$. Then, as ${\Sigma}$ is foliated by characteristic curves, we can parametrize it by the map 
\[
F({\varepsilon},t)={\gamma}_{\varepsilon}(t)=(x({\varepsilon},t),y({\varepsilon},t),z({\varepsilon},t)),
\]
where ${\gamma}_{\varepsilon}(t)$ is the characteristic curves with initial data ${\gamma}_{\varepsilon}(0)={\Gamma}({\varepsilon})$ and 
\begin{equation}
\begin{split}
\dot{\gamma}_{\varepsilon}(0)&=J(\dot{\Gamma}({\varepsilon}))=\dot{z}({\varepsilon})J(X)+\frac{1}{\sqrt{2}}(\dot{y}({\varepsilon})e^{z({\varepsilon}})-\dot{x}({\varepsilon})e^{-z({\varepsilon})})J(Y)\\
&=  \frac{1}{\sqrt{2}}(-\dot{z}({\varepsilon})e^{z({\varepsilon})},\dot{z}({\varepsilon})e^{-z({\varepsilon})},\dot{x}({\varepsilon})e^{-z({\varepsilon})}-\dot{y}({\varepsilon})e^{z({\varepsilon})}).
\end{split}
\end{equation}
We define the function $V_{\varepsilon}(t):=(\partial F/\partial {\varepsilon})(t,{\varepsilon})$ that is a smooth Jacobi-like vector field along ${\gamma}_{\varepsilon}(t)$, \cite[Section~4]{MR3044134}. We have that, in a singular point $({\varepsilon},t)$, the vertical component of $V_{\varepsilon}$ 
\[
{\big<{V_{\varepsilon},T}\big>}({\varepsilon},t)=\frac{\partial x}{\partial{\varepsilon}}({\varepsilon},t)e^{-z({\varepsilon},t)}+\frac{\partial y}{\partial{\varepsilon}}({\varepsilon},t)e^{z({\varepsilon},t)}
\]
vanishes. We suppose that ${\Gamma}$ is not an integral curve of $X$ or $Y$. Then, from the following expression of the component of $F({\varepsilon},t)$
\begin{equation}\label{eq:parametrizzazionecurvaorizzontale}
\begin{split}
x({\varepsilon},t)&=x({\varepsilon})+\frac{\dot{z}({\varepsilon})e^{z({\varepsilon})}}{\dot{x}({\varepsilon})e^{-z({\varepsilon})}-\dot{y}({\varepsilon})e^{z({\varepsilon})}}(e^{(\dot{x}({\varepsilon})e^{-z({\varepsilon})}-\dot{y}({\varepsilon})e^{z({\varepsilon})})t/\sqrt{2}}-1)\\
y({\varepsilon},t)&=y({\varepsilon})-\frac{\dot{z}({\varepsilon})e^{-z({\varepsilon})}}{\dot{x}({\varepsilon})e^{-z({\varepsilon})}-\dot{y}({\varepsilon})e^{z({\varepsilon})}}(e^{-(\dot{x}({\varepsilon})e^{-z({\varepsilon})}-\dot{y}({\varepsilon})e^{z({\varepsilon})})t/\sqrt{2}}-1)\\
z({\varepsilon},t)&=z({\varepsilon})+\frac{\dot{x}({\varepsilon})e^{-z({\varepsilon})}-\dot{y}({\varepsilon})e^{z({\varepsilon})}}{\sqrt{2}}t,
\end{split}
\end{equation}
we have 
\begin{equation}
\begin{split}
{\big<{V_{\varepsilon},T}\big>}({\varepsilon},t)&=\bigg \{\dot{x}({\varepsilon})e^{-z({\varepsilon})}+\frac{\ddot{z}({\varepsilon})}{\dot{x}({\varepsilon})e^{-z({\varepsilon})}-\dot{y}({\varepsilon})e^{z({\varepsilon})}}-            
\frac{\dot{z}({\varepsilon})\frac{\partial}{\partial{\varepsilon}}(\dot{x}({\varepsilon})e^{-z({\varepsilon})}-\dot{y}({\varepsilon})e^{z({\varepsilon})})}{(\dot{x}({\varepsilon})e^{-z({\varepsilon})}-\dot{y}({\varepsilon})e^{z({\varepsilon})})^2}
\bigg\}\cdot\\
&\hspace{2cm}\cdot(e^{-(\dot{x}({\varepsilon})e^{-z({\varepsilon})}-\dot{y}({\varepsilon})e^{z({\varepsilon})})t/\sqrt{2}}-e^{(\dot{x}({\varepsilon})e^{-z({\varepsilon})}-\dot{y}({\varepsilon})e^{z({\varepsilon})})t/\sqrt{2}})\\
&+\frac{\dot{z}({\varepsilon})^2}{(\dot{x}({\varepsilon})e^{-z({\varepsilon})}-\dot{y}({\varepsilon})e^{z({\varepsilon})})^2}(e^{-(\dot{x}({\varepsilon})e^{-z({\varepsilon})}-\dot{y}({\varepsilon})e^{z({\varepsilon})})t/\sqrt{2}}+e^{(\dot{x}({\varepsilon})e^{-z({\varepsilon})}-\dot{y}({\varepsilon})e^{z({\varepsilon})})t/\sqrt{2}}-2),
\end{split}
\end{equation}
that vanishes only for the values $({\varepsilon},0)$, for positive values of $t$. On the other hand, if ${\Gamma}$ is an integral curve of $Y$, we get 
\begin{equation}
\begin{split}
x({\varepsilon},t)&=x({\varepsilon})\\
y({\varepsilon},t)&=y({\varepsilon})\\
z({\varepsilon},t)&=z({\varepsilon})+\frac{\dot{x}({\varepsilon})e^{-z({\varepsilon})}-\dot{y}({\varepsilon})e^{z({\varepsilon})}}{\sqrt{2}}t;
\end{split}
\end{equation}
if ${\Gamma}$ is an integral curve of $X$, we have
\begin{equation}
\begin{split}
x({\varepsilon},t)&=x({\varepsilon})-\frac{\dot{z}({\varepsilon})e^{z({\varepsilon})}}{\sqrt{2}}t\\
y({\varepsilon},t)&=y({\varepsilon})+\frac{\dot{z}({\varepsilon})e^{-z({\varepsilon})}}{\sqrt{2}}t\\
z({\varepsilon},t)&=z({\varepsilon}).
\end{split}
\end{equation}
In both cases, the singular set is only the curve ${\Gamma}({\varepsilon})$.

\end{proof}

The vertical component of $V_{\varepsilon}$ can be computed more directly using \cite[Proposition~4.3]{MR3044134}, since $H=0$. On the other hand, the explicit computation of the components of the parametrization $F({\varepsilon},t)$ allows us to characterize all the $C^2$ area-stationary complete surfaces with a singular curve that is a characteristic curve of curvature $0$. We stress that, when the characteristic curves are sub-Riemannian geodesics, these examples can also be constructed from Remark \ref{immediateminimaremark}.

\begin{proposition} Let ${\Sigma}$ be an area-stationary surface with $H=0$, with a singular curve ${\Gamma}$ that is a characteristic curve of curvature $0$. Then, if ${\Gamma}$ is a sub-Riemannian geodesic, ${\Sigma}$ belongs to one of the families
\begin{itemize}
\item[(i)] $\{ a x+b y+ c=0: (x,y,z)\in{E(1,1)}, a,b,c\in {{\mathbb{R}}}\}$;
\item[(ii)] $\{e^{z-z_0} (y-y_0)+e^{z_0-z}(x-x_0)=0: (x,y,z)\in {E(1,1)}, x_0,y_0,z_0\in {{\mathbb{R}}}\}$.
\end{itemize}
Otherwise, we suppose that ${\Gamma}$ is a characteristic curve passing through $(x_0,y_0,z_0)$ with velocity $(\dot{x}_0,\dot{y}_0,\dot{z}_0)$, $\dot{x}_0,\dot{y}_0,\dot{z}_0\neq 0$. We can parametrize ${\Sigma}$ by $F:{{\mathbb{R}}}^2\rightarrow {E(1,1)}$, with $F({\varepsilon},t)=(x({\varepsilon},t),y({\varepsilon},t),z({\varepsilon},t))$ and 
\begin{equation}\label{eq:parametrizationsingularcurvenotgeo}
\begin{split}
x({\varepsilon},t)&=x_0+\frac{\dot{x}_0}{\dot{z}_0}(e^{\dot{z}_0{\varepsilon}}-1)+\frac{\dot{z}_0e^{z_0+\dot{z}_0{\varepsilon}}}{\dot{x}_0e^{-z_0}-\dot{y}_0e^{z_0}}(e^{(\dot{x}_0e^{-z_0}-\dot{y}_0e^{z_0})t/\sqrt{2}}-1)\\
y({\varepsilon},t)&=y_0-\frac{\dot{y}_0}{\dot{z}_0}(e^{-\dot{z}_0{\varepsilon}}-1)-\frac{\dot{z}_0e^{-z_0-\dot{z}{\varepsilon}}}{\dot{x}_0e^{-z_0}-\dot{y}_0e^{z_0}}(e^{-(\dot{x}_0e^{-z_0}-\dot{y}_0e^{z_0})t/\sqrt{2}}-1)\\
z({\varepsilon},t)&=z_0+\dot{z}_0{\varepsilon}+\frac{\dot{x}_0e^{-z_0}-\dot{y}_0e^{z_0}}{\sqrt{2}}t.
\end{split}
\end{equation}

\end{proposition}

\begin{remark} We note that the surfaces parametrized by \eqref{eq:parametrizationsingularcurvenotgeo} are the first examples of area-stationary surfaces  that are not foliated by sub-Riemannian geodesics in three-dimensional contact sub-Riemannian manifolds, up to our knowledge. In fact this phenomena do not appear in the roto-translation group, \cite[Lemma~10.4]{MR3044134}, even if its pseudo-hermitian torsion is non-vanishing.  In that case, the presence of two singular curves force the the surface to be foliated by sub-Riemannian geodesics or to be not area-stationary. On the other hand, it is well-known that a minimal surface is foliated by sub-Riemannian geodesics in any three-dimensional Sasakian manifold.

\end{remark}

\begin{remark} Given any horizontal curve ${\Gamma}=(x({\varepsilon}),y({\varepsilon}),z({\varepsilon}))$ in ${E(1,1)}$, we stress that \eqref{eq:parametrizzazionecurvaorizzontale} provide a parametrization $F({\varepsilon},t):{{\mathbb{R}}}^2\rightarrow {\Sigma} \subset {E(1,1)}$ of a complete area-stationary surface ${\Sigma}$ with ${\Sigma}_0={\Gamma}$.

\end{remark}

\section{Complete area-minimizing surfaces in ${E(1,1)}$}\label{sec:minimizing}

\subsection{Complete area-minimizing surfaces with empty singular set} 

In \cite[Proposition~9.8]{MR3044134} is shown a general necessary condition for the stability of a non-singular surface in pseudo-hermitian Lie groups. This condition states that the quantity
\[
W-{\big<{\tau(Z),\nu_h}\big>}={\big<{\nu_h,Y}\big>}^2-1={\big<{Z,X}\big>}^2-1
\]
has to be always non-positive.  This condition is trivial in ${E(1,1)}$ due to the negativity of the Webster scalar curvature. On the other hand it has been used strongly in the classification of the stable, area-stationary surfaces without singular points in the manifolds ${{\mathbb{H}}}^1$, $SU(2)$ and $\widetilde{E(2)}$, see \cite{MR3044134,Ri-Ro-Hu, Ro}. In any way, we can prove the following

\begin{proposition}\label{lemmapianiminimizzanti} The families of planes
\begin{itemize}
\item[(i)] $\{ x+c=0: (x,y,z)\in{E(1,1)}, c\in{{\mathbb{R}}}\}$;
\item[(ii)] $\{ y+c=0: (x,y,z)\in{E(1,1)}, c\in{{\mathbb{R}}}\}$;
\item[(iii)] $\{ z+c=0: (x,y,z)\in{E(1,1)}, c\in{{\mathbb{R}}}\}$;
\end{itemize}
are area-stationary, foliated by sub-Riemannian geodesics, and area-minimizing. 
\end{proposition}

\begin{proof}
We prove the result for ${\Sigma}=\{ x=0: (x,y,z)\in{E(1,1)}\}$, since all the cases are similar. In this case, from \eqref{eq:nuhexpression} and \eqref{eq:Zexpression}, we have
\[
\nu_h=Y \qquad Z=-X.
\]
So the integral curves of $Z$ are sub-Riemannian geodesics and ${\Sigma}_0=\emptyset$. Now Remark \ref{immediateminimaremark} implies that ${\Sigma}$ is area-stationary. Finally we can foliate a neighborhood of ${\Sigma}$ in ${E(1,1)}$ simply translating ${\Sigma}$. We obtain a foliation by area-stationary surfaces and a standard calibration argument imply that ${\Sigma}$ is area-minimizing, see for example \cite{BA-SC-Vi}, \cite{Ri1} or \cite[\S~5]{Ri-Ro}.
\end{proof}

\begin{remark}
The family of planes $\{ ax+by+cz+d=0: (x,y,z)\in{E(1,1)}, a,b,c,d \in{{\mathbb{R}}}\}$ are not minimal, since they do not satisfy \eqref{eq:minimalsurface}.
\end{remark}

A very natural question is: are the planes in Proposition \ref{lemmapianiminimizzanti} the unique complete area-minimizing surfaces with empty singular set in ${E(1,1)}$? 

We have only been able to find the following sufficient condition

\begin{lemma}\label{lem:sufficientstabilityempty}  Let ${\Sigma}$ be a $C^2$ complete oriented minimal surface immersed in {E(1,1)}, with empty singular set ${\Sigma}_0$. If on ${\Sigma}$ there holds ${\big<{N,T}\big>}{\leqslant} 0$, then ${\Sigma}$ is stable.
\end{lemma}

\begin{proof} Taking into account the expression for stability operator for non-singular surfaces in \cite[Lemma~8.3]{MR3044134}, we only need to show that 
\[
2Z\bigg({\frac{{\big<{N,T}\big>}}{|N_h|}} \bigg)+{\bigg({\frac{{\big<{N,T}\big>}}{|N_h|}}\bigg)^2}{\leqslant} 0
\]
on ${\Sigma}$. Given a point $p$ in ${\Sigma}$, let $I$ an open interval containing the origin and $\alpha:I\rightarrow {\Sigma}$ a piece of the integral curve of $S$ passing through $p$. Consider the characteristic curve $\gamma_{\varepsilon}(s)$ of ${\Sigma}$ with $\gamma_{\varepsilon}(0)=\alpha({\varepsilon})$. We define the map $F:I\times{{\mathbb{R}}}\rightarrow {\Sigma}$ given by $F({\varepsilon},s)=\gamma_{\varepsilon}(s)$ and denote $V(s):=(\partial F/\partial {\varepsilon})(0,s)$ which is a Jacobi-like vector field along $\gamma_0$, see \cite[Proposition~4.3]{MR3044134}. Denoting by $'$ the derivatives of functions depending on $s$, and the covariant derivative along $\gamma_0$ respect to ${\nabla}$ and $\dot{\gamma}_0$ by $Z$. Using \cite[Lemma~3.1, Eq.~4.4 and Eq.~4.5]{MR3044134} we get
\begin{equation}\label{vt0s}
{\big<{V,T}\big>}(0)=-{|N_{H}|},
\end{equation}
\begin{equation}\label{vt1s}
{\big<{V,T}\big>}'(0)=-{{\big<{N,T}\big>}},
\end{equation}
\begin{equation}\label{vt2s}
{\big<{V,T}\big>}''(0)=-{|N_{H}|} \bigg( Z\bigg({\frac{{\big<{N,T}\big>}}{|N_h|}} \bigg)+{\bigg({\frac{{\big<{N,T}\big>}}{|N_h|}}\bigg)^2}   \bigg)
\end{equation}
It is easy to show that $g(V,T)$ never vanishes along $\gamma_0$ as ${\Sigma}_0$ is empty, see \cite[Proof of Lemma~9.5]{MR3044134}. On the other hand, by \cite[Proposition~4.3]{MR3044134} and Lemma \ref{lem:calcolotausullasuperficie}, we have that ${\big<{V,T}\big>}$ satisfies the ordinary differential equation
\[
{\big<{V,T}\big>}'''(s)-{\big<{Z,X}\big>}^2{\big<{V,T}\big>}'(s)=0
\]
along ${\gamma}_0$. We suppose that ${\big<{Z,X}\big>}\neq 0$. Taking into account the initial conditions \eqref{vt0s}, \eqref{vt1s} and \eqref{vt2s}, we obtain
\[
{\big<{V,T}\big>}(s)=a \cosh(|{\big<{Z,X}\big>}|s)+ b \sinh(|{\big<{Z,X}\big>}|s)+c,
\] 
where 
\[
a=\frac{{|N_{H}|} \bigg( Z\bigg({\frac{{\big<{N,T}\big>}}{|N_h|}} \bigg)+{\bigg({\frac{{\big<{N,T}\big>}}{|N_h|}}\bigg)^2}   \bigg)}{{\big<{X,Z}\big>}^2}, \quad b= -\frac{{\big<{N,T}\big>}}{|{\big<{Z,X}\big>}|}
\]
and
\[ 
c= -{|N_{H}|} -\frac{{|N_{H}|} \bigg( Z\bigg({\frac{{\big<{N,T}\big>}}{|N_h|}} \bigg)+{\bigg({\frac{{\big<{N,T}\big>}}{|N_h|}}\bigg)^2}   \bigg)}{{\big<{X,Z}\big>}^2}.
\]
We have that ${\big<{V,T}\big>}(s)\neq 0$ implies 
\[
a+b=\frac{{|N_{H}|} \bigg( Z\bigg({\frac{{\big<{N,T}\big>}}{|N_h|}} \bigg)+{\bigg({\frac{{\big<{N,T}\big>}}{|N_h|}}\bigg)^2}   \bigg)}{{\big<{X,Z}\big>}^2} -\frac{{\big<{N,T}\big>}}{|{\big<{Z,X}\big>}|}{\leqslant} 0.
\] 
Then we can conclude 
\begin{equation*}
\begin{split}
2&Z\bigg({\frac{{\big<{N,T}\big>}}{|N_h|}} \bigg)+{\bigg({\frac{{\big<{N,T}\big>}}{|N_h|}}\bigg)^2}{\leqslant}2 \{Z\bigg({\frac{{\big<{N,T}\big>}}{|N_h|}} \bigg)+{\bigg({\frac{{\big<{N,T}\big>}}{|N_h|}}\bigg)^2}\}{\leqslant} 2 |{\big<{Z,X}\big>}|{\frac{{\big<{N,T}\big>}}{|N_h|}}{\leqslant} 0
\end{split}
\end{equation*}
on ${\gamma}_0$. Now since the choice of $p$ is arbitrary, we get the statement. 

If ${\big<{Z,X}\big>}=0$, we conclude that ${\Sigma}$ is stable if and only if ${{\big<{N,T}\big>}}=0$, by \cite[Proposition~9.8]{MR3044134}.

\end{proof}

\begin{remark} We note that the surfaces described in the points (i), (ii), (iii) of Proposition \ref{lemmapianiminimizzanti}, are characterized by ${{\big<{N,T}\big>}}=-e^z/\sqrt{2}$, ${{\big<{N,T}\big>}}=-e^z/\sqrt{2}$ and ${{\big<{N,T}\big>}}\equiv 0$ respectively, where $N$ denotes the inward unit normal on ${\Sigma}$. In the third family the planes are vertical surfaces and they satisfy $W-{\big<{\tau(Z),\nu_h}\big>}\equiv 0$. 
\end{remark}

Taking into account the geometric invariants of ${E(1,1)}$, we expect the existence of other examples of complete oriented minimal surface with empty singular set.

\subsection{Complete area-minimizing surfaces with non-empty singular set} 

We consider the stability operator constructed in \cite[Theorem~8.6]{MR3044134}

\begin{lemma}\label{Stability operator II} Let ${\Sigma}$ be a $C^2$ oriented minimal surface immersed in {E(1,1)}, with singular set ${\Sigma}_0$ and $\partial{\Sigma}=\emptyset$. If ${\Sigma}$ is stable then, for any function $u\in C^1_0({\Sigma})$ such that $Z(u)=0$ in a tubular neighborhood of a singular curve and constant in a tubular neighborhood of an isolated singular point, we have $Q(u){\geqslant} 0$, where
\begin{equation*}
\begin{split}
Q(u):=&\int\limits_{\Sigma}\{|N_h|^{-1}Z(u)^2+{|N_{H}|}( (1+{\big<{Z,Y}\big>}^2)-({|N_{H}|} (1/2-{\big<{Z,Y}\big>}^2)-{\big<{{\nabla}_S \nu_h,Z}\big>}   )^2  )u^2\}d{\Sigma}\\
&+4\int\limits_{({\Sigma}_0)_c}{{\big<{N,T}\big>}}{\big<{Z,Y}\big>}^2{\big<{Z,\nu}\big>}u^2d({\Sigma}_0)_c+\int\limits_{({\Sigma}_0)_c} S(u)^2  d({\Sigma}_0)_c.
\end{split}
\end{equation*}
Here $d({\Sigma}_0)_c$ is the Riemannian length measure on $({\Sigma}_0)_c$ and $\nu$ is the external unit normal to $({\Sigma}_0)_c$.
\end{lemma}

\begin{corollary}
Let ${\Sigma}$ be a plane in the family $\{ a x+b y+ c=0: (x,y,z)\in{E(1,1)}, a,b,c\in {{\mathbb{R}}}\}$. Then ${\Sigma}$ is stable.
\end{corollary}

\begin{proof} We know that ${\Sigma}$ is area-stationary with a singular line, obtained intersecting ${\Sigma}$ with the plane $z=\log\sqrt{b/a}$.  From \eqref{eq:Zexpression} we get
\[
Z=\frac{-b e^z+ a e^{-z}}{|-b e^z+ a e^{-z}|}X, 
\]
which is orthogonal to the singular line. Since 
\[
{\big<{{\nabla}_S \nu_h,Z}\big>}={\big<{{\nabla}_S Y,X}\big>}=\frac{|N_{H}|}{2},
\]
we have that the stability operator
\[
Q(u)=\int\limits_{\Sigma}\{|N_h|^{-1}Z(u)^2+{|N_{H}|}{{\big<{N,T}\big>}}^2 u^2\}d{\Sigma}+\int\limits_{{\Sigma}_0} S(u)^2  d{\Sigma}_0
\]
is always non-negative for any admisible test function $u$.
\end{proof}

\begin{remark} The planes $\{ a x+b y+ c=0: (x,y,z)\in{E(1,1)}, a,b,c\in {{\mathbb{R}}}\}$ are also area-minimzing, by calibration arguments. 

\end{remark}

\begin{corollary} We consider the surface ${\Sigma}=\{e^z y+e^{-z}x=0: (x,y,z)\in {E(1,1)}\}$. Then ${\Sigma}$ is stable. 

\end{corollary}

\begin{proof} From \eqref{eq:Zexpression} we get
\[
Z=-\frac{(e^z y-e^{-z}x)Y}{|e^z y-e^{-z}x|}
\]
and ${\Sigma}_0=\{(0,0,z):(x,y,z)\in {E(1,1)}\}$. From \eqref{Gchistoffel} we have
\[
{\big<{{\nabla}_S \nu_h,Z}\big>}={\big<{{\nabla}_S Y,X}\big>}=-\frac{|N_{H}|}{2},
\]
which implies
\[
Q(u)=\int\limits_{\Sigma}\{|N_h|^{-1}Z(u)^2+2{|N_{H}|}^2 u^2\}d{\Sigma}+\int\limits_{{\Sigma}_0} S(u)^2  d{\Sigma}_0+ 4\int\limits_{{\Sigma}_0} u^2  d{\Sigma}_0{\geqslant} 0,
\]
for all admissible $u$.
 \end{proof}
 
 
 
 \begin{corollary} The surfaces defined in Proposition \ref{prop:minimalstationarysingularpoint} are stable. 
\end{corollary}

\begin{proof} For simplicity we will prove the statement in the case of $x_0=y_0=z_0=0$. We note that, since ${\Sigma}_0=(0,0,0)$, the argument in the proof of Lemma \ref{lem:sufficientstabilityempty} works and  the condition ${\big<{N,T}\big>}=-(1+e^z)/\sqrt{2}{\leqslant} 0$ is a sufficient condition for the stability in the complementary of any tubular neighborhood  of  ${\Sigma}_0$. Finally we observe that the stability operator in Lemma \ref{Stability operator II} does not give contributions of the singular set in the case of isolated singular points. 
 \end{proof}

\providecommand{\bysame}{\leavevmode\hbox to3em{\hrulefill}\thinspace}
\providecommand{\MR}{\relax\ifhmode\unskip\space\fi MR }
\providecommand{\MRhref}[2]{  \href{http://www.ams.org/mathscinet-getitem?mr=#1}{#2}
}
\providecommand{\href}[2]{#2}
\begin{thebibliography}{10}

\bibitem{Am-SC-Vi}
Luigi Ambrosio, Francesco Serra~Cassano, and Davide Vittone, \emph{Intrinsic
  regular hypersurfaces in {H}eisenberg groups}, J. Geom. Anal. \textbf{16}
  (2006), no.~2, 187--232. \MR{MR2223801 (2007g:49072)}

\bibitem{MR2831583}
D.~Barbieri and G.~Citti, \emph{Regularity of minimal intrinsic graphs in
  3-dimensional sub-{R}iemannian structures of step 2}, J. Math. Pures Appl.
  (9) \textbf{96} (2011), no.~3, 279--306. \MR{2831583 (2012g:53050)}

\bibitem{BA-SC-Vi}
Vittorio Barone~Adesi, Francesco Serra~Cassano, and Davide Vittone, \emph{The
  {B}ernstein problem for intrinsic graphs in {H}eisenberg groups and
  calibrations}, Calc. Var. Partial Differential Equations \textbf{30} (2007),
  no.~1, 17--49. \MR{MR2333095 (2009c:35044)}

\bibitem{MR2600502}
Francesco Bigolin and Francesco~Serra Cassano, \emph{Distributional solutions
  of {B}urgers' equation and intrinsic regular graphs in {H}eisenberg groups},
  J. Math. Anal. Appl. \textbf{366} (2010), no.~2, 561--568. \MR{2600502}

\bibitem{MR0112160}
W.~M. Boothby and H.~C. Wang, \emph{On contact manifolds}, Ann. of Math. (2)
  \textbf{68} (1958), 721--734. \MR{0112160 (22 \#3015)}

\bibitem{MR2583494}
Luca Capogna, Giovanna Citti, and Maria Manfredini, \emph{Regularity of
  non-characteristic minimal graphs in the {H}eisenberg group {$\Bbb H^1$}},
  Indiana Univ. Math. J. \textbf{58} (2009), no.~5, 2115--2160. \MR{2583494
  (2010j:58032)}

\bibitem{CDPT}
Luca Capogna, Donatella Danielli, Scott~D. Pauls, and Jeremy~T. Tyson, \emph{An
  introduction to the {H}eisenberg group and the sub-{R}iemannian isoperimetric
  problem}, Progress in Mathematics, vol. 259, Birkh\"auser Verlag, Basel,
  2007. \MR{MR2312336}

\bibitem{ChengHwang2nd}
Jih-Hsin Cheng and Jenn-Fang Hwang, \emph{Variations of generalized area
  functionals and p-area minimizers of bounded variation in the {H}eisenberg
  group}, Bulletin of the Inst. of Math., Academia Sinica (New Series), 5
  (2010), 369-412.

\bibitem{CJHMY}
Jih-Hsin Cheng, Jenn-Fang Hwang, Andrea Malchiodi, and Paul Yang, \emph{Minimal
  surfaces in pseudohermitian geometry}, Ann. Sc. Norm. Super. Pisa Cl. Sci.
  (5) \textbf{4} (2005), no.~1, 129--177. \MR{2165405 (2006f:53008)}

\bibitem{CJHMY2}
\bysame, \emph{A {C}odazzi-like equation and the singular set for {$C^1$}
  smooth surfaces in the {H}eisenberg group}, J. Reine Angew. Math.
  \textbf{671} (2012), 131--198. \MR{2983199}

\bibitem{CHY}
Jih-Hsin Cheng, Jenn-Fang Hwang, and Paul Yang, \emph{Existence and uniqueness
  for {$p$}-area minimizers in the {H}eisenberg group}, Math. Ann. \textbf{337}
  (2007), no.~2, 253--293. \MR{MR2262784 (2009h:35120)}

\bibitem{DGNAV}
D.~Danielli, N.~Garofalo, and D.~M. Nhieu, \emph{Sub-{R}iemannian calculus on
  hypersurfaces in {C}arnot groups}, Adv. Math. \textbf{215} (2007), no.~1,
  292--378. \MR{2354992 (2009h:53061)}

\bibitem{DGNnotable}
\bysame, \emph{A notable family of entire intrinsic minimal graphs in the
  {H}eisenberg group which are not perimeter minimizing}, Amer. J. Math.
  \textbf{130} (2008), no.~2, 317--339. \MR{2405158 (2009b:49102)}

\bibitem{Da-Ga-Nh-Pa}
D.~Danielli, N.~Garofalo, D.~M. Nhieu, and S.~D. Pauls, \emph{Instability of
  graphical strips and a positive answer to the {B}ernstein problem in the
  {H}eisenberg group {$\Bbb H^1$}}, J. Differential Geom. \textbf{81} (2009),
  no.~2, 251--295. \MR{2472175 (2010e:53007)}

\bibitem{Dr-To}
Sorin Dragomir and Giuseppe Tomassini, \emph{Differential geometry and analysis
  on {CR} manifolds}, Progress in Mathematics, vol. 246, Birkh\"auser Boston
  Inc., Boston, MA, 2006. \MR{MR2214654 (2007b:32056)}

\bibitem{Gaphd}
Matteo Galli, \emph{Area-stationary surfaces in contact sub-{R}iemannian
  manifolds}, Ph.D. thesis, Universidad de Granada, 2012.

\bibitem{MR3044134}
\bysame, \emph{First and second variation formulae for the sub-{R}iemannian
  area in three-dimensional pseudo-hermitian manifolds}, Calc. Var. Partial
  Differential Equations \textbf{47} (2013), no.~1-2, 117--157. \MR{3044134}

\bibitem{MR2979606}
Matteo Galli and Manuel Ritor{\'e}, \emph{Existence of isoperimetric regions in
  contact sub-{R}iemannian manifolds}, J. Math. Anal. Appl. \textbf{397}
  (2013), no.~2, 697--714. \MR{2979606}

\bibitem{Ga-Nh}
Nhieu~Den Garofalo~Nicola, \emph{Isoperimetric and sobolev inequalities for
  carnot-carath\'eodory spaces and the existence of minimal surfaces}, Comm.
  Pure Appl. Math. \textbf{49} (1996), no.~10, 1081--1144.

\bibitem{Hl-Pa2}
Robert~K. Hladky and Scott~D. Pauls, \emph{Constant mean curvature surfaces in
  sub-{R}iemannian geometry}, Journal of Differential Geometry \textbf{79}
  (2008), no.~1, 111--139.

\bibitem{Ri-Ro-Hu}
Ana Hurtado, Manuel Ritor{\'e}, and C{\'e}sar Rosales, \emph{The classification
  of complete stable area-stationary surfaces in the {H}eisenberg group {$\Bbb
  H^1$}}, Adv. Math. \textbf{224} (2010), no.~2, 561--600. \MR{2609016}

\bibitem{Hu-Ro}
Ana Hurtado and C{\'e}sar Rosales, \emph{Area-stationary surfaces inside the
  sub-{R}iemannian three-sphere}, Math. Ann. \textbf{340} (2008), no.~3,
  675--708. \MR{MR2358000 (2008i:53038)}

\bibitem{Hu-Ro2}
\bysame, \emph{Stables surfaces inside the sub-{R}iemannian three-sphere}, in
  preparation (2010).

\bibitem{MR0425012}
John Milnor, \emph{Curvatures of left invariant metrics on {L}ie groups},
  Advances in Math. \textbf{21} (1976), no.~3, 293--329. \MR{0425012 (54
  \#12970)}

\bibitem{Pe}
Domenico Perrone, \emph{Homogeneous contact {R}iemannian three-manifolds},
  Illinois J. Math. \textbf{42} (1998), no.~2, 243--256. \MR{MR1612747
  (99a:53067)}

\bibitem{Ri1}
Manuel Ritor{\'e}, \emph{Examples of area-minimizing surfaces in the
  sub-{R}iemannian {H}eisenberg group {$\Bbb H^1$} with low regularity}, Calc.
  Var. Partial Differential Equations \textbf{34} (2009), no.~2, 179--192.
  \MR{MR2448649 (2009h:53062)}

\bibitem{Ri-Ro}
Manuel Ritor{\'e} and C{\'e}sar Rosales, \emph{Area-stationary surfaces in the
  {H}eisenberg group {$\Bbb H^1$}}, Adv. Math. \textbf{219} (2008), no.~2,
  633--671. \MR{MR2435652 (2009h:49075)}

\bibitem{Ro}
C{\'e}sar Rosales, \emph{Complete stable {CMC} surfaces with empty singular set
  in {S}asakian sub-{R}iemannian 3-manifolds}, Calc. Var. Partial Differential
  Equations \textbf{43} (2012), no.~3-4, 311--345. \MR{2875642}

\bibitem{Ru}
Michel Rumin, \emph{Formes diff\'erentielles sur les vari\'et\'es de contact},
  J. Differential Geom. \textbf{39} (1994), no.~2, 281--330. \MR{MR1267892
  (95g:58221)}

\bibitem{Sh}
Nataliya Shcherbakova, \emph{Minimal surfaces in sub-{R}iemannian manifolds and
  structure of their singular sets in the {$(2,3)$} case}, ESAIM Control Optim.
  Calc. Var. \textbf{15} (2009), no.~4, 839--862. \MR{MR2567248}

\end{thebibliography}

\end{document}

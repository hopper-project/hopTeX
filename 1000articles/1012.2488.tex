\documentclass{amsart}
\let\englishhyphenmins\undefined
\usepackage{amssymb}

\textwidth	490pt
\textheight	690pt
\oddsidemargin	-10pt
\evensidemargin -10pt
\topmargin	-20pt

\usepackage{xy}
\xyoption{all}

\newtheorem{lem}{Lemma}[section]
\newtheorem{theorem}[lem]{Theorem}
\newtheorem{proposition}[lem]{Proposition}
\newtheorem{cons}[lem]{Corollary}
\newtheorem{claim}[lem]{Claim}
\newtheorem{corollary}[lem]{Corollary}
\newtheorem{problem}[lem]{Problem}
\newtheorem{question}[lem]{Question}
\newtheorem{lemma}[lem]{Lemma}
\theoremstyle{definition}
\newtheorem{exam}[lem]{Examples}
\newtheorem{definition}[lem]{Definition}
\newtheorem{rem}[lem]{Remark}

\newtheorem{remark}[lem]{Remark}

\title{Algebra in the superextensions of semilattices}
\author{Taras Banakh and Volodymyr Gavrylkiv}
\address{Ivan Franko National University of Lviv, Ukraine and\newline
Uniwersytet Humanistyczno-Przyrodniczy Jana Kochanowskiego, Kielce, Poland}
\email{t.o.banakh@gmail.com}
\address{Vasyl Stefanyk Precarpathian National University,
Ivano-Frankivsk, Ukraine}
\email{vgavrylkiv@yahoo.com}

\begin{document}
\begin{abstract} Given a semilattice $X$ we study the algebraic properties of the semigroup ${\upsilon}(X)$ of upfamilies on $X$. The semigroup ${\upsilon}(X)$ contains the Stone-\v Cech extension $\beta(X)$, the superextension $\lambda(X)$, and the space of filters $\varphi(X)$ on $X$ as closed subsemigroups. We prove that ${\upsilon}(X)$ is a semilattice iff $\lambda(X)$ is a semilattice iff $\varphi(X)$ is a semilattice iff the semilattice $X$ is  finite and linearly ordered. We prove that the semigroup $\beta(X)$ is a band if and only if $X$ has no infinite antichains, and the semigroup $\lambda(X)$ is commutative if and only if $X$ is a bush with finite branches.
\end{abstract}
\subjclass{06A12, 20M10}
\keywords{semilattice, band, commutative semigroup,  the space of upfamilies, the space of filters, the space of maximal linked systems, superextension}
\maketitle

\section*{Introduction}

One of powerful tools in the modern Combinatorics of Numbers is
the method of ultrafilters based on the fact that each
(associative) binary operation $*:X\times X\to X$ defined on a
discrete topological space $X$ extends to a
right-topological (associative) operation $*:\beta (X)\times \beta
(X)\to\beta (X)$ on the Stone-\v Cech compactification $\beta (X)$
of $X$, see \cite{HS}, \cite{P}.
 The Stone-\v Cech extension $\beta (X)$ is the space
of ultrafilters on $X$. The extension of the operation from $X$ to
$\beta(X)$ can be defined by the simple formula:
\begin{equation}\label{extension}{\mathcal U}\ast{\mathcal V}=\big{\langle}\bigcup_{x\in U}x{*}V_x:U\in{\mathcal U},\;\;(V_x)_{x\in U}\in {\mathcal V}^U\big{\rangle},
\end{equation}
where ${\langle} \mathcal B{\rangle}=\{A\subset X:\exists B\in\mathcal B\;\;B\subset A\}$ is the upper closure of a family $\mathcal B$. In this case $\mathcal B$ is called a {\em base} of ${\langle}\mathcal B{\rangle}$.

Endowed with the so-extended operation, the Stone-\v Cech
compactification $\beta(X)$ becomes a compact right-topological
semigroup. The algebraic properties of this semigroup (for
example, the existence of idempotents or minimal left ideals) have
important consequences in combinatorics of numbers, see \cite{HS},
\cite{P}.

In \cite{G2} it was observed that the binary operation $*$ extends
not only to $\beta (X)$ but also  to the space $\upsilon(X)$ of all
upfamilies on $X$. By definition, a family ${\mathcal {F}}$ of
non-empty subsets of a discrete space $X$ is called an {\em
upfamily} if for any sets $A\subset B\subset X$ the inclusion $A\in{\mathcal {F}}$ implies $B\in{\mathcal {F}}$. The space ${\upsilon}(X)$ is a closed subspace of the double power-set $\mathcal P(\mathcal P(X))$ endowed with the compact Hausdorff topology of the Tychonoff power $\{0,1\}^{\mathcal P(X)}$. In the papers \cite{G1}, \cite{G2}, \cite{BGN}--\cite{BG4} the space $\upsilon(X)$ was denoted by $G(X)$ and its elements were called inclusion hyperspaces\footnote{We decided to change the terminology and notation after discovering the paper \cite[2.7.4]{SS} that discusses monadic properties of the up-set functor ${\upsilon}$.}. The extension of a binary
operation $\ast$ from $X$ to $\upsilon(X)$ can be defined in the same way
as for ultrafilters, i.e., by the formula~(\ref{extension})
applied to any two upfamilies ${\mathcal U},{\mathcal V}\in \upsilon(X)$. If $X$
is a semigroup, then $\upsilon(X)$ is a compact Hausdorff
right-topological semigroup containing $\beta (X)$ as closed
subsemigroups. The algebraic properties of this semigroups were
studied in details in \cite{G2}.

The space ${\upsilon}(X)$ of upfamilies over a discrete space $X$ contains many
interesting subspaces. First we recall some definitions. An upfamily ${\mathcal A}\in {\upsilon}(X)$ is defined to be
\begin{itemize}
\item {\em a filter} if
$A_1\cap A_2\in{\mathcal A}$ for all sets $A_1,A_2\in{\mathcal A}$;
\item {\em an ultrafilter} if ${\mathcal A}={\mathcal A}'$ for any filter ${\mathcal A}'\in {\upsilon}(X)$ containing ${\mathcal A}$;
\item {\em linked} if $A\cap B\ne\emptyset$ for any sets $A,B\in{\mathcal A}$;
\item {\em maximal linked} if ${\mathcal A}={\mathcal A}'$ for any linked
upfamily ${\mathcal A}'\in {\upsilon}(X)$ containing ${\mathcal A}$.
\end{itemize}

By ${\varphi}(X)$, $\beta(X)$, $N_2(X)$, and $\lambda(X)$  we denote the subspaces of ${\upsilon}(X)$ consisting of filter, ultrafilters, linked upfamilies, and maximal linked upfamilies, respectively. The space $\lambda(X)$ is
called {\em the superextension} of $X$, see \cite{vM}, \cite{Ve}. In \cite{G2} it was
observed that for a discrete semigroup $X$ the subspaces ${\varphi}(X)$, $\beta(X)$, $N_2(X)$, $\lambda(X)$ are closed subsemigroups of
the semigroup ${\upsilon}(X)$.
The following diagram describes the inclusion relations between
these subspaces of ${\upsilon}(X)$ (an arrow $A\to B$ indicates that $A$ is a subset of $B$).
$$
\xymatrix{
\beta(X)\ar[d]\ar[r]&\lambda(X)\ar[d]\\
\varphi(X)\ar[r]&N_{2}(X)\ar[r]&{\upsilon}(X)
}$$

In \cite{G2}, \cite{BGN} --- \cite{BG4} we studied the properties of the compact right-topological semigroup $\upsilon(X)$ and its subsemigroups for groups $X$. In this paper we shall study the algebraic structure of the semigroups $\lambda(X)$, $\varphi(X)$, $N_2(X)$, and $\upsilon(X)$ for semilattices $X$.

Let us recall that a {\em semilattice} is a commutative idempotent semigroup.
Idempotent semigroups are called {\em bands}. So, in a band each element $x$ is an {\em idempotent}, which means that $xx=x$.
A semigroup $S$ is {\em linear} if $xy\in\{x,y\}$ for any elements $x,y\in X$. It follows that each linear semigroup $S$ is a band. Each (linear) semilattice is  partially (linearly) ordered by the relation $\le$ defined by $x\le y$ iff $xy=x$.

A semigroup $S$ is {\em cancellative} if for each element $a\in S$ the left shift $l_a:S\to S$, $l_a:x\mapsto ax$, and the right shift $r_a:S\to S$, $r_a:x\mapsto xa$, are injective.
A semigroup $S$ is called {\em Clifford} (resp. {\em sub-Clifford}) if $S$ is a union of groups (resp. of cancellative semigroups). Observe that a subsemigroup of a sub-Clifford semigroup is sub-Clifford and a finite semigroup $S$ is Clifford if and only if it is sub-Clifford. It is easy to see that a semigroup $S$ is sub-Clifford if and only if for every natural numbers $n,m$ it is {\em $(n,m)$-Clifford} in the sense that for any element $x\in S$ the equality $x^{n+1}=x^{m+1}$ implies $x^n=x^m$.

A semigroup $S$ is called {\em a regular semigroup} if $a\in aSa$ for any $a\in S$. Such a semigroup
$S$ is called {\em an inverse semigroup} if $ab=ba$ for any
idempotents $a,b\in S$. Observe that each band is a Clifford
semigroup and every Clifford semigroup is sub-Clifford and regular. An inverse semigroup with a unique idempotent is a group.

These algebraic properties relate as follows:
$$\xymatrix{
&&\mbox{sub-Clifford semigroup}\ar[r]&\mbox{(1,2)-Clifford semigroup}\\
\mbox{semilattice}\ar[r]\ar[d]&\mbox{band}\ar[r]&\mbox{Clifford semigroup}\ar[u]\ar[r]&\mbox{regular semigroup}\\
\mbox{commutative inverse semigroup}\ar[rr]&&\mbox{Clifford inverse semigroup}\ar[r]\ar[u]&\mbox{inverse semigroup}\ar[u]\\
\mbox{commutative group}\ar[rr]\ar[u]&&\mbox{group}\ar[u]
}
$$

In this paper we shall characterize semigroups $X$ whose extensions ${\upsilon}(X)$, $\lambda(X)$, $\varphi(X)$ or $N_2(X)$ are bands, linear semigroups, commutative semigroups, or semilattices. In Section~\ref{s:lattice} we shall characterize lattices $X$ whose extensions ${\upsilon}(X)$, $\lambda(X)$, $\varphi(X)$ are lattices. The results obtained in this paper will be applied in the paper \cite{BG5} devoted to the superextensions of inverse semigroups. 

\section{Semigroups whose extensions are bands}

In this section we shall characterize semigroups $X$ whose extensions ${\upsilon}(X)$, $\lambda(X)$ or $\varphi(X)$ are bands.
Let us recall that a semigroup $S$ is a (linear) band if $xx=x$ for all $x\in X$ (and $xy\in\{x,y\}$ for all $x,y\in X$).

Let us recall that an element $a$ of a semigroup $S$ is {\em regular} in $S$ if $a\in aSa$. It is clear that each idempotent is a regular element.

\begin{theorem}\label{t1.1} For a semigroup $X$ the following conditions are equivalent:
\begin{enumerate}
\item $X$ is linear;
\item ${\upsilon}(X)$ is a band;
\item $\varphi(X)$ is a band;
\item $\lambda(X)$ is a band.
\end{enumerate}
\end{theorem}

\begin{proof} $(1){\Rightarrow}(2)$ Assume that the semigroup $X$ is linear. To show that ${\upsilon}(X)$ is a band, we should check that $\A*{\mathcal A}={\mathcal A}$ for any upfamily  ${\mathcal A}\in\upsilon(X)$. Since $X$ is linear, for any $A\in{\mathcal A}$ we get $A=A*A\in\A*{\mathcal A}$ and hence ${\mathcal A}\subset\A*{\mathcal A}$.

To show that ${\mathcal A}\supset\A*{\mathcal A}$, fix any basic subset $B=\bigcup\limits_{x\in A}x{*}A_x\in\A*{\mathcal A}$ where $A\in{\mathcal A}$ and $A_x\in{\mathcal A}$ for all $x\in A$.

Now we consider two cases.

(i) There is $x\in A$ such that $xa=a$ for all $a\in A_x$. In this case ${\mathcal A}\ni A_x=x{*}A_x\subset B$ and thus $B\in{\mathcal A}$.

(ii) For every $x\in A$ there is $a\in A_x$ such that $xa\ne a$ and hence $xa=x$ (as $X$ is linear). In this case ${\mathcal A}\ni A\subset\bigcup_{x\in A}x*A_x=B$ and hence $B\in{\mathcal A}$.
\smallskip

The implications $(2){\Rightarrow}(3,4)$ are trivial.
\smallskip

$(3){\Rightarrow}(1)$ Assume that $\varphi(X)$ is a band. Then $X$, being a subsemigroup of $\varphi(X)$, also is a band. To show that $X$ is linear, take any two points $x,y\in X$ and consider the filter ${\mathcal {F}}={\langle}\{x,y\}{\rangle}\in\varphi(X)$. Being an idempotent, the filter ${\mathcal {F}}$ is regular in ${\upsilon}(X)$. Consequently, we can find an upfamily ${\mathcal A}\in\upsilon(X)$ such that $\F*\A*{\mathcal {F}}={\mathcal {F}}$. It follows that there are sets $A_x,A_y\in{\mathcal A}$ such that $(xA_x\cup yA_y)\cdot\{x,y\}\subset\{x,y\}$. In particular, for every $a_x\in A_x$ we get $xa_xy\in\{x,y\}$. If $xa_xy=x$, then $xy=xa_xyy=xa_xy=x$. If $xa_xy=y$, then $xy=xxa_xy=xa_xy=y$, witnessing that the band $X$ is linear.

$(4){\Rightarrow}(1)$ Assume that $\lambda(X)$ is a band. Then $X$, being a subsemigroup of $\lambda(X)$, is a band as well. Assuming that the band $X$ is not linear, we can find two points $x,y\in X$ such that $xy\notin\{x,y\}$.
We claim that the maximal linked system ${\mathcal L}={\langle}\{x,y\},\{x,xy\},\{y,xy\}{\rangle}\in\lambda(X)$ is not an idempotent. We shall prove more: the element ${\mathcal L}$ is not regular in the semigroup ${\upsilon}(X)$.
Assuming the converse, we can find an upfamily ${\mathcal A}\in{\upsilon}(X)$ such that $\LL*\A*{\mathcal L}={\mathcal L}$. It follows from $\{x,y\}\in{\mathcal L}=\LL*\A*{\mathcal L}$ that $\{x,y\}\supset \bigcup_{u\in L}u*B_u$ for some set $L\in{\mathcal L}$ and some sets $B_u\in\A*{\mathcal L}$, $u\in L$. The linked property of family ${\mathcal L}$ implies that the intersection $L\cap\{x,xy\}$ contains some point $u$. Now for the set $B_u\in\A*{\mathcal L}$ find a set $A\in{\mathcal A}$ and a family $(L_a)_{a\in A}\in{\mathcal L}^A$ such that $B_u\supset\bigcup_{a\in A}a*L_a$. Fix any point $a\in A$ and a point $v\in L_a\cap\{y,xy\}$. Then $uav\in uaL_a\subset uB_u\subset \{x,y\}$.
Since $u\in\{x,xy\}$ and $v\in \{y,xy\}$, the element $uav$ is equal to $xby$ for some element $b\in\{a,ya,ax,yax\}$. So, $xby\in\{x,y\}$. If $xby=x$, then $xy=xbyy=xby=x\in\{x,y\}$. If $xby=y$, then $xy=xxby=xby=y\in\{x,y\}$. In both cases we obtain a contradiction with the choice of the points $x,y\notin\{x,y\}$.
\end{proof}

 Observe that the proof of Theorem~\ref{t1.1} yields a bit more, namely:

\begin{proposition} For a band $X$ the following conditions are equivalent:
\begin{enumerate}
\item $X$ is linear;
\item each element of $\varphi(X)$ is regular in ${\upsilon}(X)$;
\item each element of $\lambda(X)$ is regular in ${\upsilon}(X)$.
\end{enumerate}
\end{proposition}

The linearity of a semilattice $X$ can be also characterized via the $(1,2)$-Clifford property of the semigroups $\varphi(X)$ and $\lambda(X)$.

\begin{theorem}\label{t(1,2)} For a semilattice $X$ the following conditions are equivalent:
\begin{enumerate}
\item $X$ is linear;
\item $\varphi(X)$ is $(1,2)$-Clifford;
\item $\lambda(X)$ is $(1,2)$-Clifford.
\end{enumerate}
\end{theorem}

\begin{proof} The implications $(1){\Rightarrow}(2,3)$ follow from Theorem~\ref{t1.1} because each band is a $(1,2)$-Clifford semigroup.
\smallskip

$(2,3){\Rightarrow}(1)$ Assume that the semilattice $X$ is not linear. Then $X$ contains two elements $x,y\in X$ such that $yx=xy\notin\{x,y\}$.

Consider the filter ${\mathcal {F}}={\langle}\{x,y\}{\rangle}$ and observe that ${\mathcal {F}}\ne{\mathcal {F}}\cdot{\mathcal {F}}={\langle}\{x,xy,y\}{\rangle}={\mathcal {F}}\cdot{\mathcal {F}}\cdot{\mathcal {F}}$, which means that the semigroup $\varphi(X)$ is not $(1,2)$-Clifford.

To see that $\lambda(X)$ is not $(1,2)$-Clifford, consider the maximal linked system
${\mathcal L}={\langle}\{x,y\},\{x,xy\},\{y,xy\}{\rangle}\in\lambda(X)$ and observe that ${\mathcal L}\ne{\mathcal L}\cdot{\mathcal L}={\langle}\{xy\}{\rangle}={\mathcal L}\cdot{\mathcal L}\cdot{\mathcal L}$.
\end{proof}

Next we characterize semigroups $X$ whose Stone-\v Cech extension $\beta(X)$ is a band. A sequence $(x_n)_{n\in{\omega}}$ of points of some set $X$ is called {\em injective} if $x_n\ne x_m$ for any distinct numbers $n,m\in{\omega}$.

\begin{theorem}
For a band  $X$ the semigroup $\beta(X)$ is a band if and
only if for each injective sequence $(x_n)_{n\in{\omega}}$ in $X$ there are numbers $n<m$ such that $x_nx_m\in\{x_n,x_m\}$.
\end{theorem}

\begin{proof} To prove the ``only if'' part, assume that $(x_n)_{n\in{\omega}}$ is an injective sequence in $X$ such that $x_nx_m\notin\{x_n,x_m\}$ for all $n<m$. We claim that there is an infinite subset $\Omega\subset {\omega}$ such that $x_nx_m\ne x_k$ for any numbers $n,m,k\in\Omega$ with $n<m$.
For this we shall apply the famous Ramsey Theorem. Consider the 4-coloring $\chi:[{\omega}]^3\to 4=\{0,1,2,3\}$ of the set $[{\omega}]^3=\{(k,n,m)\in{\omega}^3:k<n<m\}$, defined by
$$\chi(k,n,m)=\begin{cases}
1 &\mbox{if $x_kx_n=x_m$},\\
2 &\mbox{if $x_kx_m=x_n$},\\
3 &\mbox{if $x_nx_m=x_k$},\\
0 &\mbox{otherwise}.
\end{cases}
$$By the Ramsey Theorem \cite[5.1]{P}, there is an infinite set $\Omega\subset{\omega}$ such that $\chi(\Omega^3\cap[{\omega}]^3)$ is a singleton. It follows from the definition of the coloring $\Omega$ that this singleton is $\{0\}$, which means that for any numbers $k,n,m\in\Omega$ with $n<m$ and $k\notin\{n,m\}$ we get $x_nx_m\ne x_k$. Since $x_nx_m\notin\{x_n,x_m\}$ for any numbers $n<m$, we conclude that $x_nx_m\ne x_k$ for any numbers $k,n,m\in\Omega$ with $n<m$.

Now take any free ultrafilter ${\mathcal A}$ that contains the set $A=\{x_n\}_{n\in\Omega}$. Then for every $n\in{\omega}$ the set $A_{>n}=\{x_m:n<m\in\Omega\}$ belongs to the ultrafilter ${\mathcal A}$. The choice of the sequence $A=\{x_n\}_{n\in\Omega}$ guarantees that $A\cap \bigcup_{n\in\Omega}x_n*A_{>n}=\emptyset$, which implies that ${\mathcal A}\ne\A*{\mathcal A}$ and hence the ultrafilter ${\mathcal A}$ is not an idempotent in $\beta(X)$.
\smallskip

To prove the ``if'' part, assume that $\beta(X)$ is not a band and find an ultrafilter ${\mathcal {F}}\in\beta(X)$ with $\F*{\mathcal {F}}\neq{\mathcal {F}}$. In particular,
$\F*{\mathcal {F}}\nsubseteq{\mathcal {F}}$. This implies that for some $A\in{\mathcal {F}}$ and
$\{A_x\}_{x\in A}\subset{\mathcal {F}}$ the set $\bigcup_{x\in
A}x{*}A_x\notin{\mathcal {F}}$.

Consider the set $X^{\uparrow}_{\mathcal {F}}=\{x\in X: {\uparrow}x\in{\mathcal {F}}\}$ where ${\uparrow}x=\{y\in X:xy=x\}$.
We claim that $X^{\uparrow}_{\mathcal {F}}\notin{\mathcal {F}}$. Assuming that
$X^{\uparrow}_{\mathcal {F}}\in{\mathcal {F}}$, we conclude that $A\cap X^{\uparrow}_{\mathcal {F}}\in{\mathcal {F}}$.
This implies that ${\uparrow}a\in{\mathcal {F}}$ and ${\uparrow}a\cap
A_a\in{\mathcal {F}}$ for any $a\in A\cap X^{\uparrow}_{\mathcal {F}}$. Therefore
$a*({\uparrow}a\cap A_a)=\{a\}$ and hence $$\bigcup_{x\in
A}x*A_x\supset\bigcup_{x\in A\cap
X^{\uparrow}_{\mathcal {F}}}x*({\uparrow}x\cap A_x)=\bigcup_{x\in A\cap
X^{\uparrow}_{\mathcal {F}}}\{x\}=A\cap X^{\uparrow}_{\mathcal {F}}\in{\mathcal {F}}.$$ Thus
$\bigcup_{x\in A}x{*}A_x\in{\mathcal {F}}$. This contradiction shows that $X^{\uparrow}_{\mathcal {F}}\notin{\mathcal {F}}$.

Next, consider the set $X^{\downarrow}_{\mathcal {F}}=\{x\in X:
{\downarrow}x\in{\mathcal {F}}\}$ where ${\downarrow}x=\{y\in X:xy=y\}$. We claim that
$X^{\downarrow}_{\mathcal {F}}\notin{\mathcal {F}}$. Assume that
$X^{\downarrow}_{\mathcal {F}}\in{\mathcal {F}}$. Then $A\cap X^{\downarrow}_{\mathcal {F}}\in{\mathcal {F}}$.
This implies that ${\downarrow}a\in{\mathcal {F}}$ and ${\downarrow}a\cap
A_a\in{\mathcal {F}}$ for any $a\in A\cap X^{\downarrow}_{\mathcal {F}}$. Therefore
$${\downarrow}a\cap A_a\subset a*({\downarrow}a\cap A_a)\subset a*A_a\subset   \bigcup_{x\in A}x*A_x.$$ Thus
$\bigcup_{x\in A}x{*}A_x\in{\mathcal {F}}$. This contradiction shows that $X^{\downarrow}_{\mathcal {F}}\notin{\mathcal {F}}$.

Since ${\mathcal {F}}$ is an ultrafilter, $X^{\uparrow}_{\mathcal {F}}\cup X^{\downarrow}_{\mathcal {F}}\notin{\mathcal {F}}$ and
$Z_{\mathcal {F}}=X\setminus(X^{\uparrow}_{\mathcal {F}}\cup
X^{\downarrow}_{\mathcal {F}})\in{\mathcal {F}}$. Let $x_0\in Z_{\mathcal {F}}$ be arbitrary and by induction, for every $n\in{\omega}$ choose a point
$x_{n+1}\in Z_{\mathcal {F}}\setminus\bigcup_{i\leq
n}({\uparrow}x_i\cup{\downarrow}x_i)\in{\mathcal {F}}$.
 Then the injective sequence $(x_n)_{n\in{\omega}}$ has the required property:
$x_nx_m\notin\{x_n,x_m\}$ for $n<m$ (which follows from $x_m\notin {\downarrow}x_n\cup{\uparrow}x_n$).
\end{proof}

A subset $A$ of a semigroup $X$ is called an {\em antichain} if $ab\notin\{a,b\}$ for any distinct points $a,b\in A$.
Theorem implies the following characterization:

\begin{corollary} For a semilattice  $X$ the semigroup $\beta(X)$ is a band if and only if each antichain in $X$ is finite.
\end{corollary}

\section{Semilattices whose extensions are commutative}

In this section we recognize the structure of semilattices $X$ whose extensions ${\upsilon}(X)$, $N_2(X)$ or $\lambda(X)$ are commutative.

Commutative semigroups of ultrafilters were characterized in \cite[4.27]{HS} as follows:

\begin{theorem}\label{t2.4} The Stone-\v Cech extension $\beta(X)$ of a semigroup $S$ is not commutative if and only if there are sequences $(x_n)_{n\in{\omega}}$ and $(y_n)_{n\in{\omega}}$ in $X$ such that $\{x_ky_n:k<n\}\cap\{y_kx_n:k<n\}=\emptyset$.
\end{theorem}

This characterization implies the following (well-known) fact:

\begin{corollary}\label{c2.5} If the Stone-\v Cech extension $\beta(X)$ of a semilattice $X$ is commutative, then each linear subsemigroup in $X$ in finite.
\end{corollary}

\begin{proof} Assume conversely that $X$ contains an infinite linear subsemilattice $L$. Being linear, $L$ is linearly ordered by the order $\le$ defined by $x\le y$ iff $xy=x$. Since $L$ is infinite, we can apply Ramsey Theorem in order to find an injective sequence $(z_n)_{n\in{\omega}}$ in $L$,  which is either strictly increasing or strictly decreasing. Put $x_n=z_{2n}$ and $y_n=z_{2n+1}$ for $n\in{\omega}$. Applying Theorem~\ref{t2.4} to the sequences $(x_n)_{n\in{\omega}}$ and $(y_n)_{n\in{\omega}}$ we conclude that the semigroup $\beta(L)$ is not commutative. Then $\beta(X)$ is not commutative neither.
\end{proof}

In spite of Theorem~\ref{t2.4} the following problem seems to be open.

\begin{problem} Describe the structure of a semilattice $X$ whose Stone-\v Cech extension $\beta(X)$ is commutative.
\end{problem}

A similar problem on commutativity of semigroups ${\upsilon}(X)$ also is open:

\begin{problem} Characterize semigroups $X$ whose extension ${\upsilon}(X)$ is commutative.\newline {\rm (It can be shown that if ${\upsilon}(X)$ is commutative, then $X$ is a commutative semigroup with finite linear idempotent band $E=\{x\in X:xx=x\}$ and $x^3=x^4$ for all $x\in X$).}
\end{problem}

We shall resolve this problem for bands. First we prove a useful result on multiplication of upfamilies on linear semigroups.

For a semigroup $X$ denote by ${\upsilon^\bullet}(X)$ the subsemigroup of ${\upsilon}(X)$ consisting of all upfamilies ${\mathcal A}\in{\upsilon}(X)$ such that for each set $A\in{\mathcal A}$ there is a finite subset $F\in{\mathcal A}$ with $F\subset A$.

For a semigroup $X$ and two upfamilies ${\mathcal A},\mathcal B\in{\upsilon}(X)$ let $${\mathcal A}\otimes\mathcal B={\langle} A*B:A\in{\mathcal A},\;B\in\mathcal B{\rangle}.$$
It is clear that ${\mathcal A}\otimes\mathcal B\subset\A*\mathcal B$. In the following theorem we show that for finite linear semigroups the converse inclusion also holds.

\begin{theorem}\label{t2.1} If $X$ is a linear semigroup, then $\A*\mathcal B={\mathcal A}\otimes\mathcal B$ for any upfamilies ${\mathcal A}\in{\upsilon^\bullet}(X)$ and $\mathcal B\in\upsilon(X)$.
\end{theorem}

\begin{proof} On the semigroup $X$ consider the relation $\le$ defined by:  $x\le y$ iff $yx=x$. This relation is reflexive and transitive. For a subsets $A\subset X$ and a point $x\in X$ we write $A\le x$ if $a\le x$ for all $a\in A$. It follows from the definition of the semigroup operation $*$ on ${\upsilon}(X)$ that ${\mathcal A}\otimes\mathcal B\subset\A*\mathcal B$. To prove the reverse inclusion, fix any basic set $C=\bigcup_{a\in A}a{*}B_a\in\A*\mathcal B$ where $A\in{\mathcal A}$ and $B_a\in\mathcal B$ for all $a\in A$. Since ${\mathcal A}\in{\upsilon^\bullet}(X)$, we can assume that the set $A$ is finite and hence can be enumerated as $A=\{a_1,\dots,a_n\}$ so that $a_i\le a_{i+1}$ for all $i<n$.
Now let us consider two cases.

1. For some $i\le n$ we get $B_{a_i}\le a_i$, which means that $a_ib=b$ for all $b\in B_{a_i}$ and hence $a_i*B_{a_i}=B_{a_i}$. For every $j\ge i$ the inequality $B_{a_i}\le a_i\le a_j$ implies $a_j*B_{a_i}=B_{a_i}$. Consequently, $A*B_{a_i}\subset\{a_1,\dots,a_{i-1}\}\cup B_{a_i}$.

We can assume that $i$ is the smallest number such that $B_{a_i}\le a_i$. 
In this case the minimality of $i$ implies that $B_{a_j}\not\le a_j$ for all $j<i$. This means $b_j\not\le a_j$ for some $b_j\in B_{a_j}$ and hence $a_jb_j=a_j$ (as $a_jb_j\in\{a_j,b_j\}$ and $a_jb_j\ne b_j$). Then $a_j{*}B_{a_j}\ni a_jb_j=a_j$ and thus $A*B_{a_i}\subset\{a_1,\dots,a_{i-1}\}\cup B_{a_i}\subset\bigcup_{j=1}^na_jB_{a_j}$, which implies that $C\in{\mathcal A}\otimes\mathcal B$.

2. $B_{a_i}\not\le a_i$ for all $i\le n$. In this case $a_i\in a_i*B_{a_i}$ for all $i$.
Observe that for any $b\in B_{a_n}$ and $i\le n$ we get $a_ib\in\{a_i,b\}$ by the linearity of $X$.
If $a_ib\ne a_i$, then $a_ib=b$ and $a_ib=b=a_na_ib=a_nb=\in a_nB_{a_n}$. So,
$${\mathcal A}\otimes \mathcal B\ni A*B_{a_n}\subset \{a_1,\dots,a_n\}\cup a_nB_{a_n}\subset \bigcup_{i=1}^na_iB_{a_i}=C$$ and hence  $C\in{\mathcal A}\otimes\mathcal B$.
\end{proof}

Now we are able to characterize bands $X$ with commutative extensions ${\upsilon}(X)$ and $N_2(X)$.

\begin{theorem}\label{t2.2} For a band $X$ the following conditions are equivalent:
\begin{enumerate}
\item $X$ is a finite linear semilattice;
\item the semigroup $\upsilon(X)$ is commutative;
\item the semigroup $N_2(X)$ is commutative;
\item the semigroup $\lambda(X)$ is commutative and $(1,2)$-Clifford.
\end{enumerate}
\end{theorem}

\begin{proof} The implication $(1){\Rightarrow}(2)$ follows from Theorem~\ref{t2.1} as
$\A*\mathcal B={\mathcal A}\otimes\mathcal B=\mathcal B\otimes {\mathcal A}=\mathcal B*{\mathcal A}$ for every ${\mathcal A},\mathcal B\in{\upsilon}^\bullet(X)={\upsilon}(X)$.
\smallskip

The implication $(2){\Rightarrow}(3)$ is trivial.
\smallskip

$(3){\Rightarrow}(1)$ Assume that the semigroup $N_2(X)$ is commutative. Then so is the semigroup $X$. Being a commutative band, the semigroup $X$ is a semilattice. Assuming that $X$ is not linear, we can find two points $x,y\in X$ with $xy\notin\{x,y\}$. It can be shown that
the linked upfamilies ${\mathcal A}=\langle \{x,y\}\rangle$ and
$\mathcal B=\langle \{x,xy\}, \{y,xy\}\rangle\in N_k(X)$ do not commute because $\{xy\}\in{\mathcal A}{*}\mathcal B\setminus\mathcal B{*}{\mathcal A}$.
Therefore, $X$ is a linear semilattice. Since $\beta(X)\subset{\upsilon}(X)$ is commutative, Corollary~\ref{c2.5} implies that the linear semilattice $X$ is finite.
\smallskip

$(1)\Leftrightarrow(4)$ If $X$ is a finite linear semilattice, then $\lambda(X)$ is commutative by the  implication $(1){\Rightarrow}(2)$ of this theorem and is $(1,2)$-Clifford by Theorem~\ref{t(1,2)}.

If the semigroup $\lambda(X)$ is commutative and $(1,2)$-Clifford, then the semigroup  $X\subset\lambda(X)$ is commutative and by Theorem~\ref{t(1,2)}, $X$ is linear. By Corollary~\ref{c2.5}, the linear semilattice $X$ is finite.
\end{proof}

Now we shall characterize semilattices $X$ with commutative superextension $\lambda(X)$. A semilattice $X$ is called a {\em bush} if for any maximal linear subsemilattices $A,B\subset X$ the product $A*B$ is the singleton $\{\min X\}$ containing the smallest element $\min X$ of $X$. This definition implies that $A\cap B=A*B=\{\min X\}$. By a {\em branch} of a bush $X$ we understand a maximal linear subsemilattice of $X$.

\begin{theorem}\label{t2.6} A semilattice $X$ has commutative superextension $\lambda(X)$ if and only if $X$ is a bush with finite branches.
\end{theorem}

\begin{proof} First assume that $X$ is a bush with finite branches, and take any two maximal linked systems ${\mathcal A},\mathcal B\in\lambda(X)$. Since the products $\A*\mathcal B$ and $\mathcal B*{\mathcal A}$ are maximal linked upfamilies, the equality $\A*\mathcal B=\mathcal B*{\mathcal A}$ will follow as soon as we check that any two basic sets $C_{AB}=\bigcup_{a\in A}a{*}B_a\in\A*\mathcal B$ and $C_{BA}=\bigcup_{b\in B}b{*}A_b\in\mathcal B*{\mathcal A}$ have non-empty intersection. Here $A\in{\mathcal A}$, $(B_a)_{a\in A}\in \mathcal B^A$, $B\in\mathcal B$, and $(A_b)_{b\in B}\in{\mathcal A}^B$.  Assume conversely that $C_{AB}\cap C_{BA}=\emptyset$. Then either $\min X\notin C_{AB}$ or $\min X\notin C_{BA}$.

Without loss of generality, $\min X\notin C_{AB}$. Then $\min X\notin A$ and for each $a\in A$ the set $\{a\}\cup B_a$ lies in a branch of $X$. Since branches of $X$ meet only at the point $\min X$, all the sets $\{a\}\cup B_a$, $a\in A$, lie in the same (finite) branch. Repeating the argument of Theorem~\ref{t2.1}, we can show that $C_{AB}\supset AB'$ for some set $B'\in\mathcal B$. Since $\mathcal B$ is linked, there is a point $b\in B\cap B'$. By the same reason, there is a point $a\in A\cap A_b$. Then $ab=ba\in AB'\cap bA_b\subset C_{AB}\cap C_{BA}$ and we are done.
\smallskip

Now assume that $X$ is a semilattice with commutative superextension  $\lambda(X)$. Corollary~\ref{c2.5} implies that all branches of $X$ are finite. We claim that for every $z\in X$ the lower set ${\downarrow}z=\{x\in X:xz=x\}$ is linear. Assuming the converse, find two points $x,y\in{\downarrow}z$ such that $xy\notin\{x,y\}$. It follows that the points $x,y,z,xy$ are pairwise distinct. It is easy to check that the maximal linked upfamilies ${\mathcal A}={\langle}\{x,y\},\{x,z\},\{y,z\}{\rangle}$ and $\mathcal B={\langle}\{x,y\},\{x,xy\},\{y,xy\}{\rangle}$ do not commute because $\{x,y\}\in\mathcal B*{\mathcal A}\setminus\A*\mathcal B$. Thus ${\downarrow}z$ is linear for every $z\in X$, which means that $X$ is a tree.

Assuming that the tree $X$ is not a bush, we can find two points $x,y\in X$ such that $xy\notin\{x,y,z\}$ where $z=\min X$. Now consider the maximal linked systems ${\mathcal A}={\langle} \{x,y\},\{x,z\},\{y,z\}{\rangle}$ and $\mathcal B={\langle}\{x,y\},\{x,xy\},\{y,xy\}{\rangle}$ and observe that they do not commute as $\{xy\}\in \A*\mathcal B$ misses the set $\{x,y,z\}\in\mathcal B*{\mathcal A}$.
\end{proof}

\section{Semigroups whose extensions are semilattices}

In this section we shall characterize semigroups $X$ whose extensions ${\upsilon}(X)$, $\lambda(X)$, $\varphi(X)$, or $N_2(X)$ are semilattices.

\begin{theorem}\label{t3.1} For a semigroup $X$ the following conditions are equivalent:
\begin{enumerate}
\item $X$ is finite linear semilattice;
\item $\upsilon(X)$ is a semilattice;
\item $\lambda(X)$ is a semilattice;
\item $\varphi(X)$ is a semilattice.
\end{enumerate}
\end{theorem}

\begin{proof} $(1){\Rightarrow}(2)$ If $X$ is a finite linear semilattice, then $\upsilon(X)$ is a semilattice (=commutative band) by Theorems~\ref{t1.1} and \ref{t2.2}.
\smallskip

The implications $(2){\Rightarrow}(3,4)$ are trivial.
\smallskip

The implication $(3){\Rightarrow}(1)$ follows from Theorems~\ref{t1.1} and \ref{t2.6}.
\smallskip

$(4){\Rightarrow}(1)$ Assume that $\varphi(X)$ is a semilattice. Then $X$, being a subsemigroup of the commutative semigroup $\varphi(X)$ is commutative. Since $\varphi(X)$ is a band, $X$ is a linear semigroup by Theorem~\ref{t1.1}. Thus $X$, being a commutative linear semigroup, is a linear semilattice.
Since the subsemigroup $\beta(X)\subset\lambda(X)$ is commutative, the linear semilattice $X$ is finite by Corollary~\ref{c2.5}.
\end{proof}

\section{Semigroups whose extensions are linear}

In this section we characterize semigroups $X$ whose extensions ${\upsilon}(X)$, $\lambda(X)$ or $\varphi(X)$ are linear semigroups.

A semigroup $S$ is called a {\em semigroup of left} ({\em right}) {\em zeros} if $xy=x$ (resp. $xy=y$) for all $x,y\in X$.

\begin{theorem}\label{t4.1} For a semigroup $X$ the semigroup ${\upsilon}(X)$ is linear if and only if $X$ is either a semigroup of right zeros or a semigroup of left zeros.
\end{theorem}

\begin{proof} If $X$ is a semigroup of left zeros, then for any upfamilies ${\mathcal A},\mathcal B\in{\upsilon}(X)$ and any basic element $\bigcup_{x\in A}xB_x\in\A*\mathcal B$ we get $\bigcup_{x\in A}xB_x=\bigcup_{x\in A}\{x\}=A$ and thus $\A*\mathcal B\subset{\mathcal A}$. On the other hand, each $A\in{\mathcal A}$ belongs to $\mathcal A*\mathcal B$ as $A=A*B\in\A*\mathcal B$ for any $B\in\mathcal B$.

 Assume that the semigroup ${\upsilon}(X)$ is linear. Then $X$, being a subsemigroup of ${\upsilon}(X)$, also is linear. Let $x,y$ be any two distinct elements of $X$. First we prove that $xy\ne yx$.
Assume conversely that $xy=yx$. Then $xy=yx\in\{x,y\}$ and we lose no generality assuming that $xy=x$. Now consider two upfamilies ${\mathcal A}={\langle}\{x,y\}{\rangle}$ and $\mathcal B={\langle}\{x\},\{y\}{\rangle}$ and observe that
$$\mathcal B*{\mathcal A}={\langle}\{xx,xy\},\{yx,yy\}{\rangle}={\langle}\{x\},\{x,y\}{\rangle}={\langle}\{x\}{\rangle}\notin\{{\mathcal A},\mathcal B\},$$ so $\upsilon(X)$ is not linear and this is a required contradiction.

Thus $xy\ne yx$ for all distinct points $x,y\in X$. We call a pair $(x,y)\in X^2$ {\em left} if $xy=x$ and $yx=y$ and {\em right} if $xy=y$ and $yx=x$. Since $X$ is linear, each pair $(x,y)\in X^2$ is either left or right. We claim that either all pairs $(x,y)\in X^2$ are left or else all such pairs are right. Assuming the opposite, find pairs $(x,y),(a,b)\in X^2$ such that $(x,y)$ is not left and $(a,b)$ is not right. Then $x\ne y$, $a\ne b$ and the pair $(x,y)$ is right while $(a,b)$ is left. Consider the filters ${\mathcal A}={\langle}\{x,a\}{\rangle}$ and $\mathcal B={\langle}\{y,b\}{\rangle}$ and observe that $\A*\mathcal B={\langle}\{xy,xb,ay,ab\}{\rangle}={\langle}\{y,xb,ay,a\}{\rangle}$. Since ${\upsilon}(X)$ is linear, either $\A*\mathcal B={\mathcal A}$ or $\A*\mathcal B=\mathcal B$. In the first case $\{x,a\}\supset\{y,xb,ay,a\} \supset\{y,a\}$ and hence $y=a$. In the second case, $\{y,a\}\subset\{y,b\}$ and thus $a=y$. Now consider the filters $\mathcal C={\langle} \{x,b\}{\rangle}$ and $\mathcal D={\langle}\{a\}{\rangle}$ and observe that $\mathcal C*\mathcal D={\langle}\{xa,ba\}{\rangle}={\langle}\{xy,b\}{\rangle}={\langle}\{y,b\}{\rangle}={\langle}\{a,b\}{\rangle}\notin\{\mathcal C,\mathcal D\}$, which contradicts the linearity of ${\upsilon}(X)$.

Therefore either each pair $(x,y)\in X^2$ is left and then $X$ is a semigroup of left zeros or else each pair $(x,y)\in X^2$ is right and then $X$ is a semigroup of right zeros.
\end{proof}

\begin{theorem}\label{t4.2} For a semigroup $X$ the following conditions are equivalent:
\begin{enumerate}
\item the semigroup $\varphi(X)$ is linear;
\item the semigroup $N_2(X)$ is linear;
\item either $X$ is a semigroup of left zeros or $X$ is a semigroup of right zeros or else $X$ is a semilattice of order $|X|\le 2$.
\end{enumerate}
\end{theorem}

\begin{proof} $(3){\Rightarrow}(2)$ If $|X|=1$, then $N_2(X)$ is a singleton and hence is a
linear semigroup. If $X$ is a semilattice of order $|X|=2$, then $X=\{0,1\}$ for some elements $0,1$ with $0\cdot 1=1\cdot 0=0$. In this case $N_2(X)=\varphi(X)$ is a 3-element linear semilattice ordered as:
$${\langle} \{0\}{\rangle}\le {\langle}\{0,1\}{\rangle}\le{\langle}\{1\}{\rangle}.$$

If $X$ is a semigroup of left or right zeros, then the semigroup $\upsilon(X)$ is linear by Theorem~\ref{t4.1} and so is its subsemigroup $N_2(X)$.
\smallskip

$(2){\Rightarrow}(1)$ Is the semigroup $N_2(X)$ is linear, then so is its subsemigroup $\varphi(X)$.
\smallskip

$(1){\Rightarrow}(3)$ Assume that the semigroup $\varphi(X)$ is linear.
Then $X$, being a subsemigroup of $\varphi(X)$, is linear as well.
If $|X|\le 2$, then either $X$ is a linear semilattice or a semigroup of left or right zeros. So, we assume that $|X|\ge 3$. We claim that distinct elements $x,y\in X$ do not commute. Assume conversely that $xy=yx$ for some distinct elements $x,y\in X$. Since $xy=yx\in\{x,y\}$ we lose no generality assuming that $xy=yx=x$. Fix any element $z\in X\setminus\{x,y\}$. Now consider 3 cases:

1. $zx=z$. In this case we can consider the filters ${\mathcal A}={\langle}\{z,y\}{\rangle}$ and $\mathcal B={\langle}\{x,y\}{\rangle}$ and observe that $\A*\mathcal B={\langle}\{zx,yx,zy,yy\}{\rangle}={\langle} \{z,x,zy,y\}{\rangle}\notin\{{\mathcal A},\mathcal B\}$, which contradicts the linearity of $\varphi(X)$.

2. $zx=x$ and $zy=z$.  In this case we can consider the filters ${\mathcal A}={\langle}\{z,y\}{\rangle}$ and $\mathcal B={\langle}\{x,y\}{\rangle}$ and observe that $\A*\mathcal B={\langle}\{zx,yx,zy,yy\}{\rangle}={\langle} \{x,x,z,y\}{\rangle}\notin\{{\mathcal A},\mathcal B\}$, which contradicts the linearity of $\varphi(X)$.

3. $zx=x$ and $zy=y$.  In this case we can consider the filters ${\mathcal A}={\langle}\{x,z\}{\rangle}$ and $\mathcal B={\langle}\{y,z\}{\rangle}$ and observe that $\A*\mathcal B={\langle}\{xy,xz,zy,zz\}\}{\rangle}={\langle} \{x,xz,y,z\}{\rangle}\notin\{{\mathcal A},\mathcal B\}$, which again contradicts the linearity of $\varphi(X)$.

Those contradictions show that distinct elements of $X$ do not commute. Continuing as in the proof of Theorem~\ref{t4.1}, we can show that $X$ is a semigroup of right or left zeros.
\end{proof}

Finally, we characterize commutative semigroups with linear superextensions.

\begin{theorem}\label{t4.3} For a commutative semigroup $X$ the semigroup $\lambda(X)$ is linear if and only if $X$ is a linear semilattice of order $|X|\le 3$.
\end{theorem}

\begin{proof} If $X$ is a linear semilattice of order $|X|\le 2$, then the semigroup $\lambda(X)=X$ is linear.

If $X$ is a linear semilattice of order $|X|=3$, then $X$ can be identified with the set $3=\{0,1,2\}$ endowed with the operation $xy=\min\{x,y\}$.
The semigroup $\lambda(X)$ contains 4 elements: $0,1,2$ and $\Delta=\{A\subset 3:|A|\ge 2\}$. One can check that $\lambda(3)$ is a linear semilattice ordered as follows:
$$0\le \Delta\le 1\le 2.$$

This proves the ``if'' part of the theorem. To prove the ``only if'' part we first shall analyze the structure of the superextension $\lambda(4)$ of the semilattice $4=\{0,1,2,3\}$ endowed with the operation $xy=\min\{x,y\}$.
By Theorem~\ref{t3.1}, $\lambda(4)$ is a semilattice. It contains 12 elements: $${\langle} k{\rangle},\;\;\Delta_k={\langle} \{A\subset n:|A|=2,\;k\notin A\}\mbox{ \  and \ $\square_k={\langle}\{n\setminus\{k\},A:A\subset n,\;|A|=2,\;k\in A\}{\rangle}$ \ where $k\in 4$}.$$
The order structure of the semilattice $\lambda(4)$ is described in the following diagram:
$$\xymatrix{
&{\langle} 3{\rangle}\\
&\square_3\ar[u]\\
\Delta_1\ar[ur]&\Delta_2\ar[u]&\Delta_0\ar[ul]\\
\square_0\ar[u]\ar[ur]&\square_2\ar[ul]\ar[ur]&\square_1\ar[u]\ar[ul]\\
&{\langle} 2{\rangle}\ar[u]\\
&\Delta_3\ar[u]\ar[uul]\ar[uur]\\
&{\langle}1{\rangle}\ar[u]\\
&{\langle}0{\rangle}\ar[u]
}
$$
Looking at this diagram we see that the semilattice $\lambda(4)$ is not linear.

Now assume that $X$ is a commutative semigroup whose superextension $\lambda(X)$ is linear. Then $X$ is a linear semilattice. If $|X|>3$, then $\lambda(X)$ is not linear as it contains a subsemigroup isomorphic to the semilattice $\lambda(4)$, which is not linear.
\end{proof}

\section{Lattices whose extensions are lattices}\label{s:lattice}

In this section we characterize lattices whose extensions ${\upsilon}(X)$, $\lambda(X)$ or $\varphi(X)$ are lattices.

A {\em lattice} is a set $X$ endowed with two semilattice operations $\wedge,\vee:X\times X\to X$ such that $(x\wedge y)\vee y=y$ and $(x\vee y)\wedge y=y$ for all $x,y\in X$.

Both operations $\wedge$ and $\vee$ of a lattice $X$ can be extended to right-topological operations $\wedge$ and $\vee$ on the compact Hausdorff space ${\upsilon}(X)$. Is it natural to ask if the triple $({\upsilon}(X),\wedge,\vee)$ is a lattice.

A lattice will be called {\em linear} if $x\wedge y,x\vee y\in\{x,y\}$ for all $x,y\in X$.

\begin{theorem} For a lattice $X$ the following conditions are equivalent:
\begin{enumerate}
\item $X$ is a linear lattice of order $|X|\le 2$.
\item ${\upsilon}(X)$ is a lattice;
\item $\lambda(X)$ is a lattice;
\item $\varphi(X)$ is a lattice.
\end{enumerate}
\end{theorem}

\begin{proof} $(1){\Rightarrow}(2)$ If $X$ is a linear lattice of order $|X|=1$, then ${\upsilon}(X)=X$ is a trivial lattice. If $X$ is a linear lattice of order 2, then $X$ can be identified with the lattice $2=\{0,1\}$ endowed with the operations $x\wedge y=\min\{x,y\}$ and $x\vee y=\max\{x,y\}$. In this case $\lambda(2)=\beta(2)$ coincides with the lattice $2$, $\varphi(2)=\{{\langle}\{0\}{\rangle},{\langle}\{0,1\}{\rangle},{\langle}\{1\}{\rangle}\}$ is a 3-element lattice, isomorphic to the lattice $3=\{0,1,2\}$ endowed with the operations $\min$ and $\max$, and ${\upsilon}(2)=\big\{{\langle}\{0\}{\rangle},{\langle} \{0,1\}{\rangle},{\langle}\{0\},\{1\}{\rangle},{\langle}\{1\}{\rangle}\big\}$ is a 4-element lattice isomorphic to the lattice $\{0,1\}^2$.
\smallskip

The implications $(2){\Rightarrow}(3,4)$ are trivial.
\smallskip

$(3,4){\Rightarrow}(1)$ Assume that $\lambda(X)$ or $\varphi(X)$ is a lattice. By Theorem~\ref{t3.1}, the lattice $X$ is finite and linear.  We claim that $|X|\le 2$. Assuming the converse, we conclude that the lattice $X$ contains a sublattice isomorphic to the lattice $(3,\min,\max)$.

Consider the maximal linked upfamily $\Delta=\{A\subset 3:|A|\ge 2\}$ and observe that $$\max\{\Delta,{\langle} 1{\rangle}\}={\langle} 1{\rangle}=\min\{\Delta,{\langle} 1{\rangle}\},$$ which implies that $\lambda(3)$ is not a lattice and then $\lambda(X)$ also is not a lattice.

Next, consider the filters ${\mathcal A}={\langle}\{0,1,2\}{\rangle}$ and $\mathcal B={\langle}\{0,2\}{\rangle}$ and observe that $$\max\{{\mathcal A},\mathcal B\}={\mathcal A}=\min\{{\mathcal A},\mathcal B\}$$implying that $\varphi(3)$ is not a lattice and then $\varphi(X)$ also cannot be a lattice.
\end{proof}

\begin{thebibliography}{10}

\bibitem{BGN} T.~Banakh, V.~Gavrylkiv, O.~Nykyforchyn, {\em Algebra
in superextensions of groups, I: zeros and commutativity},
Algebra Discrete Math. (2008), No.3, 1--29.

\bibitem{BG2} T.~Banakh, V.~Gavrylkiv. {\em Algebra in superextension of groups, II:
cancelativity and centers}, Algebra Discrete Math. (2008), No.4, 1--14.

\bibitem{BG3} T.~Banakh, V.~Gavrylkiv. {\em Algebra in superextension of groups: the minimal ideal of $\lambda(G)$}, Mat. Stud. {\bf 31} (2009), 142--148.

\bibitem{BG4} T.~Banakh, V.~Gavrylkiv. {\em Algebra in the superextensions of twinic groups}, Dissert. Math. {\bf 473} (2010), 74pp.

\bibitem{BG5} T.~Banakh, V.~Gavrylkiv. {\em The superextensions of inverse semigroups}, preprint.

\bibitem{CP} A.H.~Clifford, G.B.~Preston, { The algebraic theory of semigroups}. Vol. I., Mathematical
Surveys. {\bf 7}. AMS, Providence, RI, 1961.

\bibitem{G1} V.~Gavrylkiv. {\em The spaces of inclusion hyperspaces over noncompact
spaces}, Matem. Studii. {\bf 28:1} (2007), 92--110.

\bibitem{G2} V.~Gavrylkiv, {\em Right-topological semigroup operations on inclusion hyperspaces},
Mat. Stud. {\bf 29}:1 (2008), 18--34.

\bibitem{HS} N.~Hindman, D.~Strauss, {Algebra in the Stone-\v Cech compactification}, de Gruyter, Berlin, New York, 1998.

\bibitem{vM} J.~van Mill,
Supercompactness and Wallman spaces, Math. Centre Tracts. {\bf
85}. Amsterdam: Math. Centrum., 1977.

\bibitem{P} I.~Protasov, Combinatorics of Numbers, VNTL, Lviv, 1997.

\bibitem{SS} C.~Schubert, G.~Seal, {\em Extensions in the theory of Lax algebra}, Theory and Appl. of Categories, {\bf 21}:7 (2008), 118--151.

\bibitem{TZ} A.~Teleiko, M.~Zarichnyi. Categorical Topology of
Compact Hausdoff Spaces, VNTL, Lviv, 1999.

\bibitem{Ve} A.~Verbeek. Superextensions of topological spaces. MC
Tract 41, Amsterdam, 1972.

\end{thebibliography}

\end{document}


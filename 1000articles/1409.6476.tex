\documentclass[a4paper,11pt]{amsart}
\usepackage{color}
\usepackage{cite}
\usepackage{amssymb,amsmath,amscd}
\usepackage[english]{babel}
\usepackage[latin1]{inputenc}

\newtheorem{prop}{Proposition}[section]
\newtheorem{thm}[prop]{Theorem}
\newtheorem{cor}[prop]{Corollary}

\theoremstyle{definition}
\newtheorem{rem}[prop]{Remark}

\theoremstyle{definition}

\theoremstyle{definition}

\begin{document}

\title[On $(p,r)$-null sequences and their relatives]{On $(p,r)$-null sequences and their relatives}

\author{Kati Ain}
\address{Kati Ain, Faculty of Mathematics and Computer Science, Tartu University, J. Liivi 2, 50409 Tartu, Estonia}
\email{kati.ain@ut.ee}
\author{Eve Oja}
\address{Eve Oja, Faculty of Mathematics and Computer Science, Tartu University, J. Liivi 2, 50409 Tartu, Estonia; Estonian Academy of Sciences, Kohtu 6, 10130 Tallinn, Estonia} 
\email{eve.oja@ut.ee}
\dedicatory{Dedicated to Professor Albrecht Pietsch on his eightieth birthday}
\keywords{Banach spaces, ${{(p,r)}}$-null sequences, relatively $(p,r)$-compact sets, $(p,r)$-compact operators, operator ideals.}
\thanks{The research was partially supported by
Estonian Science Foundation Grant 8976 and by institutional research funding IUT20-57 of the Estonian Ministry of Education and Research.}
\subjclass[2010]{Primary: 46B50. Secondary: 46B20, 46B45, 47B07, 47B10, 47L20.}

\begin{abstract}
Let $1\leq p < \infty$ and $1\leq r \leq {p^{\ast}}$, where ${p^{\ast}}$ is the conjugate index of $p$. We prove an omnibus theorem, which provides numerous equivalences for a sequence $(x_n)$ in a Banach space $X$ to be a ${{(p,r)}}$-null sequence. One of them is that $(x_n)$  is ${{(p,r)}}$-null if and only if $(x_n)$ is null and relatively ${{(p,r)}}$-compact. This equivalence is known in the ``limit'' case when $r={p^{\ast}}$, the case of the $p$-null sequence and $p$-compactness. Our approach is more direct and easier than those applied for the proof of the latter result. We apply it also to characterize the unconditional and weak versions of ${{(p,r)}}$-null sequences.
\end{abstract}

\maketitle

\begin{section}{Introduction}
Let $X$ be a Banach space and let $c_0(X)$ denote the space of null sequences in $X$. Recently, Delgado and Pi\~neiro \cite{PD} introduced and studied an interesting class of $p$-null sequences, where $p\geq 1$, which is a linear subspace of $c_0(X)$. In \cite{O-JF}, it was proved that the space of $p$-null sequences in $X$ can be identified with the Chevet--Saphar tensor product $c_0 \hat{\otimes}_{d_p}X$.

On the other hand, there is a strong form of compactness, the $p$-compact\-ness, that has been studied during the last dozen  years in the literature (see, e.g., \cite{ALO, AMR, CK, DOPS, DPS1, GLT, P2, SK1}). The $p$-null sequences can be characterized via the $p$-compactness as follows. (The definitions will be given in Section 2.)

\begin{thm}[Delgado--Pi\~neiro--Oja]\label{DPO}
Let $1\leq p < \infty$. A sequence $(x_n)$ in a Banach space $X$ is $p$-null if and only if $(x_n)$ is null and relatively $p$-compact.
\end{thm}

Theorem \ref{DPO} was discovered in \cite[Proposition 2.6]{PD} and proved in the case of Banach spaces enjoying a version of the approximation property depending on $p$ (by \cite{O-JM}, this version of the approximation property coincides with the classical one for the closed subspaces of $L_p(\mu)$-spaces). For arbitrary Banach spaces, Theorem \ref{DPO} was proved in \cite{O-JF}.

The proof of Theorem \ref{DPO} in \cite{O-JF} relies on the above-mentioned description of the space of $p$-null sequences as a Chevet--Saphar tensor product. Very recently, an alternative natural proof was found by Lassalle and Turco \cite{LT} who rediscovered and applied a powerful theory due to Carl and Stephani \cite{CS} from 1984. Key concepts of the Carl--Stephani theory are ${{\mathcal A}}$-null sequences and ${{\mathcal A}}$-compact sets in Banach spaces, which are defined for an arbitrary operator ideal ${{\mathcal A}}$. Lassalle--Turco's proof in \cite{LT} relies on the following operator ideal version of Theorem \ref{DPO}, deduced from the Carl--Stephani theory in \cite[Proposition 1.4]{LT}.

\begin{thm}[Lassalle--Turco]\label{LT}
Let ${{\mathcal A}}$ be an operator ideal. A sequence $(x_n)$ in a Banach space $X$ is ${{\mathcal A}}$-null if and only if $(x_n)$ is null and  ${{\mathcal A}}$-compact.
\end{thm}

A starting point for the present article was the observation that, in the proof of Theorem \ref{DPO}, Theorem \ref{LT} could be used in a more efficient way than in \cite{LT}. In particular, the technical result \cite[Proposition 1.5]{LT} would not be needed in the proof. Even more, it is obtained for ``free'' as a by-product (see Remark \ref{R3.3}). Moreover, in that way, Theorem \ref{LT} can be applied to prove results similar to Theorem \ref{DPO} also in cases when the method of \cite{O-JF} cannot be applied. One of such cases is, for instance, the one that involves the recent concepts of $(p,r)$-compactness \cite{ALO}  and of $(p,r)$-null sequences \cite{AO1}.

In Section 3, we prove an omnibus theorem, Theorem \ref{omni}, which provides six equivalent properties  for a sequence in a Banach space to be a ${{(p,r)}}$-null sequence. For completeness, let us cite here the part of the omnibus Theorem \ref{omni} which directly corresponds to Theorem \ref{DPO}.

\begin{thm}\label{th1.3}
Let $1 \leq p < \infty$ and $1\leq r \leq {p^{\ast}}$, where ${p^{\ast}}$ denotes the conjugate index of $p$. A sequence $(x_n)$ in a Banach space $X$ is ${{(p,r)}}$-null if and only if $(x_n)$ is null and relatively ${{(p,r)}}$-compact.
\end{thm}

Let us remark that in the ``limit'' case $r={p^{\ast}}$, the $(p,{p^{\ast}})$-null and $(p,{p^{\ast}})$-compactness are precisely the $p$-null and $p$-compactness. This is, in fact, the only special case when Theorem \ref{th1.3} could be proved by the method in \cite{O-JF}. The reason is simple: the method in \cite{O-JF} uses the Hahn--Banach  theorem. But the ${{(p,r)}}$-context provides a suitable norm only if $r= {p^{\ast}}$, and in all other cases merely quasi-norms are available. But, as well known, quasi-normed spaces do not enjoy the Hahn--Banach theorem.

The approach developed in Section 3 is applied in Section 4 to characterize the unconditional and weak versions of ${{(p,r)}}$-null sequences.

Our notation is standard. We consider Banach spaces over the same, either real or complex, field $\mathbb K$. The closed unit ball of a Banach space $X$ is denoted by $B_X$. 

We denote by ${{\mathcal L}}$, ${{\mathcal W}}$, ${{\mathcal K}}$, and $\overline {{\mathcal F}}$, respectively, the operator ideals  of bounded, weakly compact, compact, and approximable linear operators. We refer to Pietsch's book \cite{P} and the survey paper \cite{DJP} by Diestel, Jarchow, and Pietsch for the theory of operator ideals. Let us recall here only the definition of the operator ideal ${{\mathcal A}}^\mathrm{sur}$, the \emph{surjective hull} of an operator ideal ${{\mathcal A}}$ (see \cite[Section~2]{S73} and \cite[4.7.1]{P}). An operator $T\in {{\mathcal L}}(Y,X)$ \emph{belongs to} ${{\mathcal A}}^\mathrm{sur}(Y,X)$ if $Tq \in {{\mathcal A}}(Z,X)$ for some surjection $q\in{{\mathcal L}} (Z,Y)$. Obviously, ${{\mathcal A}} \subset {{\mathcal A}}^\mathrm{sur}$. If ${{\mathcal A}} = {{\mathcal A}} ^\mathrm{sur}$, then ${{\mathcal A}}$ is called \emph{surjective}.

The Banach space of all absolutely $p$-summable sequences in $X$ is denoted by $\ell_p(X)$ and its norm by ${\left\lVert {\cdot} \right\rVert}_p$. By $\ell_p ^w (X)$ we mean the Banach space of weakly $p$-summable sequences in $X$ with the norm ${\left\lVert {\cdot} \right\rVert}_p^w$ (see, e.g., \cite[pp. 32--33]{DJT}). If $1\leq p \leq \infty$, then ${p^{\ast}}$ denotes the conjugate index of $p$ (i.e., $1/p + 1/{p^{\ast}} =1$ with the convention $ 1/ \infty =0$).

To simplify notation, we shall use the symbol $\ell_\infty$ instead of $c_0$ and, more generally, $\ell_\infty (X)$ instead of $c_0(X)$ if $X$ is a Banach space.
\end{section}

\begin{section}{Basic concepts and notation}
{\bf 2.1. The ${{(p,r)}}$-compactness of sets and operators.} Let $X$ be a Banach space. Let $1\leq p \leq \infty$ and $1\leq r \leq {p^{\ast}}$. We define the \emph{$(p,r)$-convex hull} of a sequence $(x_k)\in \ell_p(X)$ by 
\[
(p,r) \mbox{-conv}(x_k)=\left\{ \sum_{k=1}^\infty a_k x_k : (a_k)\in B_{\ell_r} \right\}.
\]

As in \cite{ALO}, we say that a subset $K$ of $X$ is \emph{relatively $(p,r)$-compact} if $K\subset (p,r)\mbox{-conv}(x_n)$ for some $(x_n)\in \ell_p(X)$. According to Grothendieck's criterion, the $(\infty, 1)$-compactness coincides with the usual compactness (because $(\infty, 1)\mbox{-conv}(x_n)$ is precisely the closed absolutely convex hull of $(x_n)$). The $(p,1)$-compactness was occasionally considered in the 1980s by Reinov \cite{R1984} and by Bourgain and Reinov \cite{BR} in the study of approximation properties of order $s\leq 1$. The $(p, {p^{\ast}})$-compactness was introduced in 2002 by Sinha and Karn \cite{SK1} under the name of \emph{$p$-compactness}. Remark that the $1$-compactness was considered already in 1973 by Stephani \cite[Section~4]{S73} under the name of nuclearity (of sets) (see also Remark \ref{rem2.3}).

The notion of $p$-null sequences is due to Delgado and Pi\~neiro \cite{PD}. It was extended in \cite{AO1} in a verbatim way as follows. We call a sequence $(x_n)$ in $X$ \emph{$(p,r)$-null} if for every ${\varepsilon} > 0$ there exist $(z_k)\in{\varepsilon} B_{\ell_p(X)}$ and  $N\in \mathbb N$ such that $x_n \in (p,r)\mbox{-conv}(z_k)$ for all $n \geq N$. The \emph{$p$-null} sequences in \cite{PD} are precisely the $(p,{p^{\ast}})$-null sequences.

A useful way  to look at $(p,r)$-convex hulls is the following. It is well known and easy to see that every $(x_k)\in \ell_p(X)$ defines a compact, even approximable, operator $\Phi _{(x_k)}: \ell_r {\rightarrow} X$ through the equality
\[
\Phi _{(x_k)}(a_k)= \sum _{k=1}^\infty a_k x_k, \; (a_k) \in \ell _r.
\]
Clearly,
\[
(p,r)\mbox{-conv}(x_k)= \Phi_{(x_k)}(B_{\ell_r}).
\]

In \cite{ALO}, ${{(p,r)}}$-compact operators were introduced in an obvious way: a linear operator $T: Y {\rightarrow} X $ is {\it $(p,r)$-compact} if $T(B_Y)$ is a relatively $(p,r)$-compact subset of $X$. 
Let ${{\mathcal K}}_{(p,r)}$ denote the class of all $(p,r)$-compact operators acting between arbitrary Banach spaces. Then ${{\mathcal K}}_{(p, {p^{\ast}})}= {{\mathcal K}}_p$, the class of \emph{$p$-compact operators in the sense of Sinha--Karn} \cite{SK1}. And ${{\mathcal K}}_{(p,1)}$ is the class of  \emph{$p$-compact operators in the Bourgain--Reinov sense} (cf. \cite{BR, R1984}).

Properties of ${{\mathcal K}} _p$ were studied in \cite{SK1} and, for instance, in the recent papers \cite{DPS1, DPS2, SK2}. In \cite{ALO}, an alternative approach, which is direct and easier than in these articles, was developed to study the (quasi-Banach) operator ideal structure of ${{\mathcal K}}_{{(p,r)}}$, among others, encompassing and clarifying main results on ${{\mathcal K}}_p = {{\mathcal K}}_{(p, {p^{\ast}})}$. (Remark that in the latter case the same approach was independently developed by Pietsch \cite{P2} yielding an important far-reaching theory of the (Banach) operator ideal ${{\mathcal K}}_p$.)

The approach in \cite{ALO} starts as follows. One observes that ${{\mathcal K}}_{{(p,r)}}$ is a surjective operator ideal (an easy straightforward verification). Another immediate observation is that 
\[
\Phi_{(x_n)}\in {{\mathcal N}} _{(p,1,{r^{\ast}})}(\ell_r, X),
\]
the space of $(p,1,{r^{\ast}})$-nuclear operators (for the definition of ${{\mathcal N}}_{(t,u,v)}$, see \cite[18.1.1]{P}). But then, by the definition of the surjective hull, the injective associate of $\Phi_{(x_n)}$ belongs to ${{\mathcal N}}_{(p,1,{r^{\ast}})}^{\mathrm{sur}}$. Let us denote it by $\overline\Phi_{(x_n)}$. Observing that any $T \in{{\mathcal K}}_{{(p,r)}} (Y,X)$ can be factorized as $T= \overline \Phi _{(x_n)}S$, one easily obtains that 
\[
{{\mathcal K}}_{(p,r)}={{\mathcal N}}_{(p,1,{r^{\ast}} )}^{\mathrm{sur}}
\]
as operator ideals (see \cite[Theorem 3.2]{ALO}).

\bigskip

{\bf 2.2. Some classes of bounded sets.} Let us introduce some useful notation which is inspired by \cite{S3}, but seems to be more suggestive than the notation in \cite{S3}.

Let ${\boldsymbol{{b}}}$ denote the class of all bounded subsets of all Banach spaces, and let ${\boldsymbol{{g}}}$ be a subclass of ${\boldsymbol{{b}}}$. Let $X$ be a Banach space. Following \cite[Definition~1.1]{S3}, we denote by ${\boldsymbol{{g}}}(X)$ the family of subsets of $X$ which are of type ${\boldsymbol{{g}}}$. For instance, ${\boldsymbol{{b}}}(X)$ is the family of all bounded subsets of $X$.

We denote by ${\boldsymbol{w}}$ and ${\boldsymbol{k}}$, respectively, the classes of all relatively weakly compact and relatively compact subsets of all Banach spaces. It is convenient to denote by ${\boldsymbol{{k}}}_{{(p,r)}}$ the class of all relatively ${{(p,r)}}$-compact sets in all Banach spaces. In particular, ${\boldsymbol{k}}= {\boldsymbol{k}} _{(\infty, 1)}$ and ${\boldsymbol{k}} _p := {\boldsymbol{k}}_{(p,{p^{\ast}})}$, the class of all relatively $p$-compact sets.

Let ${{\mathcal A}}$ be an operator ideal. Denote by ${{\mathcal A}}({\boldsymbol{{g}}})$ the subclass of ${\boldsymbol{{b}}}$, which is given as
\[
{{\mathcal A}} ({\boldsymbol{{g}}})(X)= \{ E \subset X : E\subset T(F) \textrm{ for some } F\in {\boldsymbol{{g}}}(Y) \textrm{ and } T\in {{\mathcal A}}(Y, X) \}
\]
where $X$ is an arbitrary Banach space (in \cite{S3}, the notation ${{\mathcal A}} \circ {\boldsymbol{{g}}}$ is used).

In this notation, Grothendieck's criterion of compactness reads as follows.
\begin{prop}[Grothendieck]\label{G}
One has ${\boldsymbol{k}} = \overline {{\mathcal F}} ({\boldsymbol{b}} ) = {{\mathcal K}} ({\boldsymbol{b}})$.
\end{prop} 

\begin{proof}
Let $X$ be a Banach space and let $K \in {\boldsymbol{k}} (X)$. Grothendieck's criterion gives us a sequence $(x_n)\in c_0(X)$ such that $K\subset \Phi _{(x_n)}(B_{\ell_1})$. Since $\Phi _{(x_n)}\in \overline {{\mathcal F}}(\ell_1 , X)$, it is clear that $K$ is of type $\overline {{\mathcal F}} ({\boldsymbol{b}})$. But $\overline {{\mathcal F}} ({\boldsymbol{b}}) \subset {{\mathcal K}} ({\boldsymbol{b}})$ because $\overline {{\mathcal F}} \subset {{\mathcal K}}$. Finally, if $K$ is of type ${{\mathcal K}} ({\boldsymbol{b}})$, then it is relatively compact. 
\end{proof}

Proposition \ref{G} says, in particular, that ${\boldsymbol{k}} _{(\infty, 1)}= {{\mathcal K}}_{(\infty, 1)}( {\boldsymbol{b}})$. Using the definitions of ${\boldsymbol{k}} _{{(p,r)}}$ and ${{\mathcal K}}_{{(p,r)}}$ together with the observation (see Section 2.1) that $\Phi_{(x_n)}$ belongs to the operator ideal ${{\mathcal N}}_{(p,1,{r^{\ast}})}$, the above proof yields also the general case.

\begin{prop}\label{G-gen}
Let $1\leq p \leq \infty$ and $1\leq r \leq {p^{\ast}}$. Then ${\boldsymbol{k}} _{{(p,r)}} = {{\mathcal N}}_{(p,1,{r^{\ast}})}({\boldsymbol{b}})= {{\mathcal K}}_{{(p,r)}} ({\boldsymbol{b}})$.
\end{prop}

\begin{rem}\label{rem2.3}
Using the notion of ideal system of sets (see \cite{S73}), the equalities ${\boldsymbol{k}} = {{\mathcal K}}({\boldsymbol{b}})$ and ${\boldsymbol{w}} = {{\mathcal W}} ({\boldsymbol{b}})$ were observed in \cite{S3}. In the special case $p=1$, $r=\infty$, the left-hand equality ${\boldsymbol{k}}_1= {\boldsymbol{k}}_{(1, \infty)}= {{\mathcal N}} ({\boldsymbol{b}})$ of Proposition \ref{G-gen} was proved in \cite{S73}; here ${{\mathcal N}}= {{\mathcal N}}_{(1,1,1)}$ denotes, as usual, the operator ideal of (classical) nuclear operators. 
\end{rem}

\bigskip

{\bf 2.3. ${{\mathcal A}}$-null sequences and ${{\mathcal A}}$-compact sets.} Let us now describe the relevant notions (cf. Theorem \ref{LT}) from the Carl--Stephani theory \cite{CS}, which is based on earlier work by Stephani \cite{S72, S73, S3}.

Let ${{\mathcal A}}$ be an operator ideal.

Following \cite[Lemma~1.2]{CS}, a sequence $(x_n)$ in a Banach space $X$ is called \emph{${{\mathcal A}}$-null} if there exist a Banach space $Y$, a null sequence $(y_n)$ in $Y$, and $T\in{{\mathcal A}}(Y, X)$ such that $x_n =T y_n$ for all $n \in {{\mathbb N}}$.

Using the notation of Section 2.2 and following \cite[Theorem 1.2]{CS}, we say (as in \cite{LT}) that a subset $K$ of a Banach space $X$ is \emph{${{\mathcal A}}$-compact} if $K$ is of type ${{\mathcal A}}({\boldsymbol{k}})$, i.e. $K\in {{\mathcal A}}({\boldsymbol{k}}) (X)$. 

Using Proposition \ref{G} and \ref{G-gen} we shall see now that the relatively ${{(p,r)}}$-compact sets, ${{\mathcal N}}_{(p,1,{r^{\ast}})}$-compact sets, and ${{\mathcal K}}_{{(p,r)}}$-compact sets are all the same.
\begin{prop}\label{prop3.1}
Let $1\leq p \leq \infty$ and $1\leq r \leq {p^{\ast}}$. Then ${\boldsymbol{k}} _{{(p,r)}} = {{\mathcal N}}_{(p,1,{r^{\ast}})}({\boldsymbol{k}})= {{\mathcal K}}_{{(p,r)}} ({\boldsymbol{k}})$.
\end{prop}
\begin{proof}
We know that ${{\mathcal N}}_{(p,1,{r^{\ast}})}$ is a minimal operator ideal (see \cite[18.1.4]{P}). This means that ${{\mathcal N}}_{(p,1,{r^{\ast}})}= \overline {{\mathcal F}} \circ {{\mathcal N}}_{(p,1,{r^{\ast}})} \circ \overline {{\mathcal F}}$ (see \cite[4.8.6]{P}). Hence, using Propositions \ref{G-gen} and \ref{G}, we have
\begin{eqnarray*}
{{\mathcal K}}_{{(p,r)}} ({\boldsymbol{k}})& \subset & {{\mathcal K}}_{{(p,r)}} ({\boldsymbol{b}}) = {\boldsymbol{k}}_{{(p,r)}} = {{\mathcal N}}_{(p,1,{r^{\ast}})}({\boldsymbol{b}})= (\overline {{\mathcal F}} \circ {{\mathcal N}}_{(p,1,{r^{\ast}})})(\overline {{\mathcal F}} ({\boldsymbol{b}}))\\
&=& \overline {{\mathcal F}} \circ {{\mathcal N}}_{(p,1,{r^{\ast}})}({\boldsymbol{k}})\subset {{\mathcal N}}_{(p,1,{r^{\ast}})}({\boldsymbol{k}}) \subset {{\mathcal K}}_{{(p,r)}} ({\boldsymbol{k}}).
\end{eqnarray*}
This shows that ${\boldsymbol{k}} _{{(p,r)}} = {{\mathcal N}}_{(p,1,{r^{\ast}})}({\boldsymbol{k}})= {{\mathcal K}}_{{(p,r)}} ({\boldsymbol{k}})$.
\end{proof}

\begin{rem}
The second equality in Proposition \ref{prop3.1} also follows from the general Carl--Stephani theory. Indeed, for any operator ideal ${{\mathcal A}}$, it is known (see \cite[p.~79]{CS}) that a subset is ${{\mathcal A}}$-compact if and only if it is ${{\mathcal A}}^ \mathrm{sur}$-compact. And (see Section 2.1) ${{\mathcal N}}_{(p,1,{r^{\ast}})}^\mathrm{sur}={{\mathcal K}}_{{(p,r)}}$.
\end{rem}
\end{section}

\begin{section}{An omnibus characterization of $(p,r)$-null sequences}
Theorem \ref{omni} below is an omnibus theorem, which provides six equivalent properties for a sequence in a Banach space to be a ${{(p,r)}}$-null sequence. One of these properties is to be a uniformly ${{(p,r)}}$-null sequence, which is a natural (formal) strengthening of a ${{(p,r)}}$-null sequence.

Let $1 \leq p < \infty$ and $1\leq r \leq {p^{\ast}}$. We call a sequence $(x_n)$ in a Banach space $X$ \emph{uniformly ${{(p,r)}}$-null} if there exists $(z_k) \in B_{\ell_p(X)}$ with the following property: for every ${\varepsilon} >0$ there exists $N \in {{\mathbb N}}$ such that $x_n \in {\varepsilon} \, (p,r)\textnormal{-conv}(z_k)$ for all $n\geq N$. 

We say that $(x_n)$ is \emph{uniformly $p$-null} if it is uniformly $(p,{p^{\ast}})$-null. The latter property was implicitly used in a result by Lassalle and Turco asserting (in the above terminology) that the $p$-null sequences are always uniformly $p$-null (concerning the proof (and its simple alternative), see Remark \ref{R3.3}).

\begin{thm}\label{omni}
Let $1 \leq p < \infty$ and $1 \leq r \leq {p^{\ast}}$. For a sequence $(x_n)$ in a Banach space $X$ the following statements are equivalent:
\begin{enumerate}
\item $(x_n)$ is $(p,r)$-null,
\item $(x_n)$ is null and relatively $(p,r)$-compact,
\item $(x_n)$ is null and ${{\mathcal N}}_{(p,1,{r^{\ast}})}$-compact,
\item $(x_n)$ is null and ${{\mathcal K}}_{(p,r)}$-compact,
\item $(x_n)$ is ${{\mathcal N}}_{(p,1,{r^{\ast}})}$-null,
\item $(x_n)$ is ${{\mathcal K}}_{(p,r)}$-null,
\item $(x_n)$ is uniformly $(p,r)$-null.
\end{enumerate}
\end{thm}

\begin{proof}
An easy verification of (a)$\Rightarrow $(b) can be found in \cite[Proposition~2]{AO1}. For completeness and easy reference, let us present it here. 

Since  $(x_n)$ is $(p,r)$-null, for every ${\varepsilon} >0$ there are $N \in {{\mathbb N}}$ and $(z_k)\in \ell_p(X)$, $\Vert (z_k) \Vert _p \leq {\varepsilon}$, such that $x_n =\sum _{k=1}^\infty a_k^n z_k $, where $(a_k ^n)_{k=1}^\infty \in B_{\ell _r}$, for all $n \geq N$. Hence, for all $n\geq N$,
\[
\Vert x_n  \Vert\leq \sum _{k=1}^\infty \Vert a_k^n z_k\Vert \leq \Vert (a_k^n)_k \Vert _{p^{\ast}} \Vert (z_k)\Vert _p \leq \Vert (a_k^n)_k\Vert _r \Vert (z_k)\Vert _p \leq {\varepsilon}, \; 
\]  
and therefore $x_n{\rightarrow} 0$.

Since $\{x_N, x_{N+1}, ...\}\subset(p,r)$-conv$(z_k)$ and $(z_k)\in \ell_p(X)$, 
the sequence
\[
y_k=\begin{cases}
x_k& \text{if } k< N, \\
z_{k-N+1}  & \text{if } k \geq N,
\end{cases}
\]
is in $\ell_p(X)$ and $x_n \in (p,r)$-conv$(y_k)$ for all $n \in {{\mathbb N}}$. This means that $(x_n)$ is relatively ${{(p,r)}}$-compact.

Implications (b)$ \Leftrightarrow $(c)$\Leftrightarrow$(d) are immediate from Proposition \ref{prop3.1}.

Implications (c)$\Leftrightarrow$(e) and (d)$\Leftrightarrow$(f) are immediate from Theorem \ref{LT}.

To prove that (f)$\Rightarrow$(g), let $(x_n)$ be a ${{\mathcal K}}_{(p,r)}$-null sequence. Then there are a null sequence $(y_n)$ in a Banach space $Y$ and an operator $T \in {{\mathcal K}}_{{(p,r)}} (Y,X)$ such that $x_n= Ty_n$ for all $n\in {{\mathbb N}}$. The ${{(p,r)}}$-compactness of $T$ gives us a sequence $(w_k) \in \ell_p(X)$ such that $T(B_Y)\subset {{(p,r)}}$-conv$(w_k)$.
Now $(z_k):=\left (\frac{w_k}{{\left\lVert {(w_k)} \right\rVert}_p }\right )\in B_{\ell_p(X)}$, and let ${\varepsilon}>0$. As $(y_n)$ is null in $Y$, for ${\varepsilon}_0:=\frac{\varepsilon}{{\left\lVert {(w_k)} \right\rVert}_p}$ there exists $N\in {{\mathbb N}}$ such that $Ty_n \in {\varepsilon}_0 T(B_Y)$ for all $n\geq N$. Hence,
\[
x_n\in {\varepsilon}_0{{(p,r)}}\textnormal{-conv}(w_k) ={\varepsilon}_0 {\left\lVert {(w_k)} \right\rVert}_p {{(p,r)}} \textnormal{-conv}(z_k)= {\varepsilon} {{(p,r)}}\textnormal{-conv}(z_k)
\]
for all $n \geq N$, as desired.

The implication (g)$\Rightarrow$(a) is clear from the definitions, because if $(z_k)\in B_{\ell_p(X)}$, then $({\varepsilon} z_k)\in {\varepsilon} B_{\ell_p(X)}$ and ${{(p,r)}} \textnormal{-conv}({\varepsilon} z_k)= {\varepsilon} {{(p,r)}}\textnormal{-conv}(z_k)$.
\end{proof}

\begin{rem}\label{R3.3}
In the special case when $r={p^{\ast}}$, Theorem \ref{omni} contains Theorem \ref{DPO}, complementing it and providing for it a somewhat easier proof than in \cite{LT}. In fact, the technical Lassalle--Turco result \cite[Proposition 1.5]{LT} (inspired by \cite[Theorem 1]{AMR}) is not needed. Even more, this technical result appears as a simple by-product of our proof: it is precisely the special case of the implication (a)$\Rightarrow$(g) when $r= {p^{\ast}}$.
\end{rem}

Let ${{\mathcal A}}$ be an operator ideal. Let $K$ be an ${{\mathcal A}}$-compact set and let $(x_n)$ be an ${{\mathcal A}}$-null sequence. If ${{\mathcal B}}$ is a larger operator ideal than ${{\mathcal A}}$, i.e. ${{\mathcal A}} \subset {{\mathcal B}}$, then, by definitions, clearly, $K$ is also ${{\mathcal B}}$-compact and $(x_n)$ is ${{\mathcal B}}$-null. In \cite[Proposition~4.7]{ALO}, it was proved  that 
\[
{{\mathcal K}}_{{(p,r)}} = {{\mathcal I}}_{(p,1,{r^{\ast}})}^\mathrm{sur} \circ {{\mathcal K}},
\]
where ${{\mathcal I}}_{(p,1, {r^{\ast}})}$ is the operator ideal of \emph{$(p,1,{r^{\ast}})$-integral operators} (for the definition of these general integral operators, see \cite[19.1.1]{P}). This equality enables us to extend characterizations (d) and (f) of ${{(p,r)}}$-null sequences of Theorem \ref{omni} to even more larger operator ideal than ${{\mathcal K}}_{{(p,r)}}$, namely to ${{\mathcal I}}_{(p,1,{r^{\ast}})}^\mathrm{sur}$.

\begin{prop}\label{prop3.3}
Let $1 \leq p < \infty$ and $1 \leq r \leq {p^{\ast}}$. For a sequence $(x_n)$ in a Banach space $X$ the following statements are equivalent:
\begin{enumerate}
\item $(x_n)$ is $(p,r)$-null,
\item $(x_n)$ is null and ${{\mathcal I}}_{(p,1,{r^{\ast}})}^\mathrm{sur}$-compact,
\item $(x_n)$ is ${{\mathcal I}}_{(p,1,{r^{\ast}})}^\mathrm{sur}$-null.
\end{enumerate}
\end{prop}

\begin{proof}
As was mentioned, ${{\mathcal K}}_{{(p,r)}} = {{\mathcal I}} _{(p,1, {r^{\ast}})}^\mathrm{sur}\circ {{\mathcal K}}$. Hence, using Propositions \ref{G-gen} and \ref{G}, we have
\[
{\boldsymbol{k}} _{{(p,r)}} = {{\mathcal K}}_{{(p,r)}} ({\boldsymbol{b}})= {{\mathcal I}}_{(p,1,{r^{\ast}})}^\mathrm{sur}({{\mathcal K}}({\boldsymbol{b}})) = {{\mathcal I}}_{(p,1,{r^{\ast}})}^\mathrm{sur} ({\boldsymbol{k}}).
\]
This shows that relatively ${{(p,r)}}$-compact sets are exactly ${{\mathcal I}}_{(p,1,{r^{\ast}})}^\mathrm{sur}$-compact sets. The claim now follows from Theorems \ref{omni} and \ref{LT}.
\end{proof}

Concerning the special case when $r={p^{\ast}}$, i.e., ${r^{\ast}}=p$, by definition, the operator ideal of \emph{right $p$-nuclear operators} ${{\mathcal N}}^p= {{\mathcal N}}_{(p,1,p)}$ (cf. \cite[18.1.1]{P} and, e.g., \cite[p.~140]{Ry}). Also, let ${{\mathcal P}}_p$ denote the operator ideal of \emph{absolutely $p$-summing operators} ($p$-summing operators in \cite{DJT}). It was noted in \cite[p.~157]{ALO} that ${{\mathcal P}}_p^\mathrm{dual}= {{\mathcal I}}_{(p,1,p)}^\mathrm{sur}$. Therefore we can spell out, from Theorem \ref{omni} and Proposition \ref{prop3.3}, the following omnibus characterization of $p$-null  sequences.

\begin{cor}
Let $1 \leq p < \infty$. For a sequence $(x_n)$ in a Banach space $X$ the following statements are equivalent:
\begin{enumerate}
\item $(x_n)$ is $p$-null,
\item $(x_n)$ is null and relatively $p$-compact,
\item $(x_n)$ is null and ${{\mathcal N}}^p$-compact,
\item $(x_n)$ is null and ${{\mathcal K}}_p$-compact,
\item $(x_n)$ is null and ${{\mathcal P}}_p ^\mathrm{dual}$-compact,
\item $(x_n)$ is ${{\mathcal N}}^p$-null,
\item $(x_n)$ is ${{\mathcal K}}_p$-null,
\item $(x_n)$ is ${{\mathcal P}}_p^\mathrm{dual}$-null,
\item $(x_n)$ is uniformly $p$-null.
\end{enumerate}
\end{cor}

\end{section}

\begin{section}{Unconditionally and weakly ${{(p,r)}}$-null sequences}
{\bf 4.1. Unconditional and weak ${{(p,r)}}$-compactnesses.} The (uniformly) ${{(p,r)}}$-null sequences and ${{(p,r)}}$-compactness in a Banach space $X$ are defined in terms of ${{(p,r)}}$-convex hulls using the space $\ell_p(X)$ of absolutely $p$-summable sequences in $X$. In general, ${{(p,r)}}$-convex hulls can be defined using the space $\ell_p ^w(X)$ of weakly $p$-summable sequences in $X$. This is a pretty old idea, going back at least to the paper \cite[p.~51]{CSa} by Castillo and Sanchez in 1993. In \cite{CSa}, the $(p,{p^{\ast}})$-convex hull of $(x_n)\in \ell_p^w(X)$ was considered under the name of ${p^{\ast}}$-convex hull of $(x_n)$. In 2002, Sinha and Karn \cite{SK1} developed some of their theory of $p$-compactness in a more general context of weak $p$-compactness. In \cite{SK1}, also the $(p,{p^{\ast}})$-convex hull of $(x_n)\in \ell_p^w(X)$ was used but under the name of $p$-convex hull of $(x_n)\in \ell_p^w(X)$.

Let $1\leq p <\infty$ and $1 \leq r \leq {p^{\ast}}$. In the present Section 4, we shall assume that  the definition of the \emph{${{(p,r)}}$-convex hull} ${{(p,r)}}$-conv$(x_n)$ (see Section 2.1) is extended to $(x_n)\in \ell_p^w(X)$. In this case, the operator $\Phi _{(x_n)}: \ell_r {\rightarrow} X$ is also well defined and 
\[
(p,r)\textnormal{-conv}(x_n)=\Phi _{(x_n)}(B_{\ell_r}).
\]
But $\Phi_{(x_n)}$ need not be a compact operator any more (see, e.g., Section 4.3).

``Between'' absolutely $p$-summable sequences $\ell_p(X)$ and weakly $p$-summable sequences $\ell_p^w(X)$, there is the Banach space $\ell_p^u(X)$ of \emph{unconditionally $p$-summable sequences} (see, e.g., \cite[8.2,~8.3]{DF}; we follow \cite{BCFP} in our terminology). The space $\ell_p^u(X)$ is defined as the (closed) subspace of $\ell_p^w(X)$, formed by the $(x_n)\in \ell_p^w(X)$ satisfying $(x_n)= \lim _{N{\rightarrow} \infty}(x_1, ... , x_N, 0,0,...)$ in $\ell_p^w(X)$. The space $\ell_p^u(X)$ was introduced and thoroughly studied by Fourie and Swart \cite{FS1} in 1979. In particular, it follows from \cite[Theorem~1.4]{FS1} that $\Phi _{(x_n)}$ is compact whenever $(x_n)\in \ell_p^u(X)$. In fact, $\Phi _{(x_n)}: \ell_{p^{\ast}} {\rightarrow} X$ is compact if and only if $(x_n)\in \ell_p^u(X)$ (see \cite[Theorem~1.4]{FS1} or, e.g., ~\cite[8.2]{DF}). 

It is rather easy to see that our approach in Sections 2 and 3 goes through if $\ell_p(X)$ is replaced with the larger space $\ell_p^u(X)$. Let us start by fixing the relevant terminology and notation.

We define \emph{relatively unconditionally} (respectively, \emph{weakly}) \emph{${{(p,r)}}$-compact} sets in $X$ by replacing $\ell_p(X)$ with $\ell_p^u(X)$ (respectively, with $\ell_p^w (X)$) in the definition of relatively ${{(p,r)}}$-compact sets. The classes of corresponding sets in all Banach spaces are denoted, respectively, by ${\boldsymbol{u}}_{{(p,r)}}$ and ${\boldsymbol{w}} _{{(p,r)}}$. So that ${\boldsymbol{k}} _{{(p,r)}} \subset {\boldsymbol{u}} _{{(p,r)}} \subset {\boldsymbol{w}} _{{(p,r)}} $ and ${\boldsymbol{u}} _{{(p,r)}} \subset {\boldsymbol{k}}$. 

A linear operator $T: Y{\rightarrow} X$ is \emph{unconditionally} (respectively, \emph{weakly}) \emph{${{(p,r)}}$-compact} if $T(B_Y)$ is a relatively unconditionally (respectively, weakly) ${{(p,r)}}$-compact subset of $X$. Let ${{\mathcal U}}_{{(p,r)}}$ and ${{\mathcal W}}_{{(p,r)}}$ denote the classes of all unconditionally and weakly ${{(p,r)}}$-compact operators acting between arbitrary Banach spaces, so that ${{\mathcal K}}_{{(p,r)}} \subset {{\mathcal U}}_{{(p,r)}} \subset {{\mathcal W}}_{{(p,r)}}$ and ${{\mathcal U}}_{{(p,r)}} \subset {{\mathcal K}}$. It is clear from the definitions that ${\boldsymbol{u}}_{{(p,r)}} = {{\mathcal U}}_{{(p,r)}} ({\boldsymbol{b}})$ and ${\boldsymbol{w}}_{{(p,r)}} ={{\mathcal W}}_{{(p,r)}} ({\boldsymbol{b}})$. An easy straightforward verification, as in the case of ${{\mathcal K}}_{{(p,r)}}$ (cf. \cite[Propositions~2.1~and~2.2]{ALO}), shows that ${{\mathcal U}}_{{(p,r)}}$ and ${{\mathcal W}}_{{(p,r)}}$ are surjective operator ideals. 

Note that ${{\mathcal W}}_{(p,{p^{\ast}})} = {{\mathcal W}}_p$, the class of \emph{weakly $p$-compact operators}, studied in \cite{SK1}. Similarly, in all cases, we shall write ``$p$-'' instead of ``$(p,{p^{\ast}})$-'', and speak, for instance, about the operator ideal ${{\mathcal U}}_p$ of unconditionally $p$-compact operators.
 
\medskip

{\bf 4.2. Unconditionally ${{(p,r)}}$-null sequences.} We define \emph{(uniformly) unconditionally ${{(p,r)}}$-null} sequences in $X$ by replacing $\ell_p(X)$ with $\ell_p^u(X)$ in the corresponding definitions of ${{(p,r)}}$-null and uniformly ${{(p,r)}}$-null sequences. The definition of the weak versions of these concepts will be given in Section 4.3;  it turns out to be unreasonably restrictive to define the weak versions just by replacing $\ell_p(X)$ with $\ell_p^w(X)$.

Let $(x_n)\in \ell_p^u(X)$. Then (see \cite[Lemma~1.2]{FS1}) $x_n=\delta_n y_n$ for some $(\delta _n)\in c_0$ and $(y_n)\in \ell_p^w(X)$. Since, clearly, 
\[
\Phi _{(x_n)}= \sum_{n=1}^\infty e_n \otimes x_n = \sum _{n=1}^\infty \delta _n e_n \otimes y_n
\]
(where $e_n\in \ell_r ^\ast $ are the unit vectors) and (as well known and easy to verify) $(e_n)\in B_{\ell_r^w (\ell_r ^\ast)}$, we have, by the definition of $(t,u,v)$-nuclear operators \cite[18.1.1]{P},
\[
\Phi _{(x_n)}\in {{\mathcal N}}_{(\infty , {p^{\ast}}, {r^{\ast}})}(\ell_r, X).
\]
Similarly, as in Section 2.1, we get that
\[
{{\mathcal U}}_{{(p,r)}} = {{\mathcal N}}^{\mathrm{sur}}_{(\infty, {p^{\ast}}, {r^{\ast}})}.
\]
This implies that 
\[
{{\mathcal U}}_{{(p,r)}} = {{\mathcal K}}\circ {{\mathcal U}}_{{(p,r)}} \circ {{\mathcal K}} .
\]
Indeed, as in the proof of Proposition \ref{prop3.1}, ${{\mathcal N}}_{(\infty, {p^{\ast}}, {r^{\ast}})}= \overline {{\mathcal F}} \circ {{\mathcal N}}_{(\infty, {p^{\ast}}, {r^{\ast}})} \circ \overline {{\mathcal F}}$, and therefore 
\[
{{\mathcal U}}_{{(p,r)}} =(\overline {{\mathcal F}} \circ {{\mathcal N}}_{(\infty, {p^{\ast}}, {r^{\ast}})}\circ \overline {{\mathcal F}})^\mathrm{sur}\subset \overline {{\mathcal F}} ^\mathrm{sur} \circ {{\mathcal N}}_{(\infty, {p^{\ast}}, {r^{\ast}})}^\mathrm{sur} \circ \overline {{\mathcal F}} ^\mathrm{sur} = {{\mathcal K}} \circ {{\mathcal U}}_{{(p,r)}} \circ {{\mathcal K}},
\]
because $\overline {{\mathcal F}} ^\mathrm{sur}={{\mathcal K}}$ (see, e.g., \cite[4.7.13]{P}).

Further, similarly to Proposition \ref{G-gen}, we have ${\boldsymbol{u}}_{{(p,r)}} ={{\mathcal N}}_{(\infty, {p^{\ast}}, {r^{\ast}})}({\boldsymbol{b}}) = {{\mathcal U}}_{{(p,r)}} ({\boldsymbol{b}})$, which implies (cf. Proposition \ref{prop3.1} and its proof) that ${\boldsymbol{u}}_{{(p,r)}} = {{\mathcal N}}_{(\infty, {p^{\ast}}, {r^{\ast}})} ({\boldsymbol{k}})={{\mathcal U}}_{{(p,r)}} ({\boldsymbol{k}})$. Using the above facts and proceeding as in the proof of Theorem \ref{omni}, we come to the omnibus characterization of unconditionally ${{(p,r)}}$-null sequences.

\begin{thm}\label{omni_uncond}
Let $1 \leq p < \infty$ and $1 \leq r \leq {p^{\ast}}$. For a sequence $(x_n)$ in a Banach space $X$ the following statements are equivalent:
\begin{enumerate}
\item $(x_n)$ is unconditionally $(p,r)$-null,
\item $(x_n)$ is null and relatively unconditionally $(p,r)$-compact,
\item $(x_n)$ is null and ${{\mathcal N}}_{(\infty, {p^{\ast}}, {r^{\ast}})}$-compact,
\item $(x_n)$ is null and ${{\mathcal U}}_{(p,r)}$-compact,
\item $(x_n)$ is ${{\mathcal N}}_{(\infty, {p^{\ast}}, {r^{\ast}} )}$-null,
\item $(x_n)$ is ${{\mathcal U}}_{(p,r)}$-null,
\item $(x_n)$ is uniformly unconditionally $(p,r)$-null.
\end{enumerate}
\end{thm}
\begin{proof}
It is mostly the verbatim version of the proof of Theorem \ref{omni}. Only the claim that $(x_n)$ is null whenever $(x_n)$ is unconditionally ${{(p,r)}}$-null (see the implication (a)$\Rightarrow $(b)) needs to be commented (also for an easy reference in Section 4.3 below).

So, let $(x_n)$ be unconditionally ${{(p,r)}}$-null. Then, as in the proof of (a)$\Rightarrow$(b) in Theorem \ref{omni}, for every ${\varepsilon}>0$ there are $N\in {{\mathbb N}}$ and $(z_k)\in \ell_p^u(X)$, ${\left\lVert {(z_k)} \right\rVert}_p^w \leq {\varepsilon}$, such that $x_n= \sum _{k=1}^\infty a_k^n z_k$, where $(a_k^n)_{k=1}^\infty\in B_{\ell_r}$, for all $n \geq N$. Hence, 
\[
{\left\lVert {x_n} \right\rVert}=\sup_{{x^{\ast}} \in B_{X^{\ast}}} \vert {x^{\ast}} (x_n)\vert \leq \sup_{{x^{\ast}} \in B_{X^{\ast}}} \sum _{k=1}^\infty \vert a_k^n  {x^{\ast}} (z_k) \vert \leq {\left\lVert {(a_k^n)_k} \right\rVert}_r {\left\lVert {(z_k)} \right\rVert}_p^w \leq {\varepsilon},
\]
for all $n\geq N$, and therefore $x_n{\rightarrow} 0$.
\end{proof}

Recall (see \cite[Theorem~2.5]{FS2} or, e.g., \cite[18.3.2]{P}) that ${{\mathcal N}}_{(\infty, p, {p^{\ast}})}$ coincides with the operator ideal $\mathrm K_p$ of \emph{classical} $p$-compact operators. Following Fourie and Swart \cite{FS1} or Pietsch \cite[18.3.1~and~18.3.2]{P}, a linear operator $T: Y {\rightarrow} X$ is called \emph{$p$-compact}, i.e., $T\in \mathrm K_p(Y,X)$, if there exist $A\in {{\mathcal K}}(Y, \ell_p)$ and $B\in {{\mathcal K}}(\ell_p, X)$ such that $T=BA$. Remark (see \cite{O-JM} and \cite{P2}) that ${{\mathcal K}}_p$ and $\mathrm K_p$ are notably different as operator ideals. 

Since ${{\mathcal U}}_{p^{\ast}}= {{\mathcal U}}_{({p^{\ast}}, p)}= {{\mathcal N}}_{(\infty, p, {p^{\ast}})}^{\mathrm{sur}}$, we get that $ \mathrm K_{p}^{\mathrm{sur}} = {{\mathcal U}}_{p^{\ast}}$ as a description of the surjective hull of $\mathrm K_p$.

Let us spell out, from Theorem \ref{omni_uncond}, an omnibus characterization of unconditionally $p$-null (i.e., $(p,{p^{\ast}})$-null) sequences.

\begin{cor}
Let $1 \leq p < \infty$. For a sequence $(x_n)$ in a Banach space $X$ the following statements are equivalent:
\begin{enumerate}
\item $(x_n)$ is unconditionally $p$-null,
\item $(x_n)$ is null and relatively unconditionally $p$-compact,
\item $(x_n)$ is null and $\mathrm K_{p^{\ast}}$-compact,
\item $(x_n)$ is null and ${{\mathcal U}}_p$-compact,
\item $(x_n)$ is $\mathrm K_{p^{\ast}}$-null,
\item $(x_n)$ is ${{\mathcal U}}_p$-null,
\item $(x_n)$ is uniformly unconditionally $p$-null.
\end{enumerate}
\end{cor}

\medskip
{\bf 4.3. Weakly ${{(p,r)}}$-null sequences and weakly ${{\mathcal A}}$-null sequences.} Let $1\leq p <\infty$ and $1\leq r \leq {p^{\ast}}$, as before. What about the weakly ${{(p,r)}}$-null sequences? It would be natural to expect that they would form a subclass of weakly null sequences, but not a subclass of null sequences as in the case of ${{(p,r)}}$-null sequences (which might be called also absolutely ${{(p,r)}}$-null sequences) or unconditionally ${{(p,r)}}$-null sequences. This means that we cannot employ the ``verbatim'' definition: replacing $\ell_p(X)$ with $\ell_p^w(X)$. 

Indeed (see the proof of Theorem \ref{omni_uncond}), such a ``weakly'' ${{(p,r)}}$-null sequence would always be a null sequence. And, for instance, looking at $X=\ell_{p^{\ast}}$, every null sequence $(x_n)$ in $X$ would be uniformly ``weakly'' $(p,{p^{\ast}})$-null, because the unit vector basis $(e_k)$ of $X$ belongs to $B_{\ell_p^w(X)}$ and, since $\Phi_{(e_k)}=I_X$, we have $x_n= \Phi_{(e_k)}x_n \in {\left\lVert {x_n} \right\rVert} p \textnormal{-conv}(e_k)$.

To motivate a definition for weakly ${{(p,r)}}$-null sequences, let us make the following observation from Theorem \ref{omni}, yielding two more characterizations of ${{(p,r)}}$-null sequences.

\begin{prop}\label{prop4.3}
Let $1 \leq p < \infty$ and $1 \leq r \leq {p^{\ast}}$. For a sequence $(x_n)$ in a Banach space $X$ the following statements are equivalent:
\begin{enumerate}

\item $(x_n)$ is $(p,r)$-null,
\item for every ${\varepsilon} >0$ there exist $(z_k)\in \ell_p(X)$ and $N\in {{\mathbb N}}$ such that ${\left\lVert {x_n} \right\rVert}\leq {\varepsilon}$ and $x_n\in {{(p,r)}} \textnormal{-conv}(z_k)$ for all $n\geq N$,
\item there exists $(z_k)\in \ell_p(X)$ with the following property: for every ${\varepsilon}>0$ there exists $N\in {{\mathbb N}}$ such that ${\left\lVert {x_n} \right\rVert}\leq {\varepsilon}$ and $x_n \in {{(p,r)}} \textnormal{-conv}(z_k)$ for all $n\geq N$.
\end{enumerate}
\end{prop}

\begin{proof}
The implication (i)$\Rightarrow$(ii) is clear from the proof of Theorem \ref{omni}, the first part of (a)$\Rightarrow$(b). 

From (ii), it is clear that $x_n{\rightarrow} 0$, and also (fixing, e.g., ${\varepsilon}=1$ and looking at the proof of Theorem \ref{omni}, the second part of (a)$\Rightarrow$(b)) that $(x_n)$ is relatively ${{(p,r)}}$-compact. By Theorem \ref{omni}, (b)$\Rightarrow $(a), $(x_n)$ is ${{(p,r)}}$-null, meaning that (ii)$\Rightarrow $(i). By Theorem \ref{omni}, (b)$\Rightarrow$(g), $(x_n)$ is uniformly ${{(p,r)}}$-null. Hence, assuming that ${\varepsilon}\leq 1$, condition (iii) holds (similarly to the implication (i)$\Rightarrow$(ii) above).

Finally, (iii)$\Rightarrow$(ii) is more than obvious, and we saw above that (ii)$\Leftrightarrow $(i).
\end{proof}

Looking at Proposition \ref{prop4.3}, it seems to be natural to make the following definitions.

Let $(x_n)$ be a sequence in a Banach space $X$. We call $(x_n)$ \emph{weakly ${{(p,r)}}$-null} if for every ${x^{\ast}} \in{X^{\ast}}$ and every ${\varepsilon}>0$ there exist $(z_k)\in \ell_p^w(X)$ and $N\in {{\mathbb N}}$ such that $\vert {x^{\ast}}(x_n) \vert \leq {\varepsilon}$ and $x_n\in {{(p,r)}} \textnormal{-conv}(z_k)$ for all $n\geq N$. We call $(x_n)$ \emph{uniformly weakly ${{(p,r)}}$-null} if there exists $(z_k)\in \ell_p^w(X)$ with the following property: for every ${x^{\ast}} \in {X^{\ast}}$ and every ${\varepsilon}>0$ there exists $N\in {{\mathbb N}}$ such that $\vert {x^{\ast}} (x_n)\vert\leq {\varepsilon}$ and $x_n \in {{(p,r)}} \textnormal{-conv}(z_k)$ for all $n\geq N$.

Let ${{\mathcal A}}$ be an operator ideal. In the present context, it would be natural to complement the Carl--Stephani theory with the concepts of weakly ${{\mathcal A}}$-null sequences and weakly ${{\mathcal A}}$-compact sets as follows. 

We call a sequence $(x_n)$ in a Banach space $X$ \emph{weakly ${{\mathcal A}}$-null} if there exist a Banach space $Y$, a weakly null sequence $(y_n)$ in $Y$, and $T\in{{\mathcal A}}(Y, X)$ such that $x_n =T y_n$ for all $n \in {{\mathbb N}}$. We say that a subset $K$ of $X$ is \emph{weakly ${{\mathcal A}}$-compact} if $K$ is of type ${{\mathcal A}}({\boldsymbol{w}})$, i.e., $K \in {{\mathcal A}}({\boldsymbol{w}})(X)$. (Recall that ${\boldsymbol{w}}$ denotes the class of all relatively weakly compact sets.)

Two basic facts in the Carl--Stephani theory \cite{CS} are that the classes of ${{\mathcal A}}$-null and ${{\mathcal A}}^\mathrm{sur}$-null sequences coincide, and so also do ${{\mathcal A}}$-compact and ${{\mathcal A}}^\mathrm{sur}$-compact sets. The ``weak'' versions of these results do not hold. 

Indeed, let ${{\mathcal V}}$ denote the operator ideal of \emph{completely continuous} operators, i.e., of operators who take weakly null sequences to null sequences. Then ${{\mathcal V}} ^\mathrm{sur}={{\mathcal L}}$ (see, e.g., \cite[4.7.13]{P}). Consequently, the weakly ${{\mathcal V}}$-null sequences are (precisely, because null sequences are ${{\mathcal K}}$-null, hence ${{\mathcal V}}$-null) the null sequences, but the weakly ${{\mathcal V}}^\mathrm{sur}$-null sequences are precisely the weakly null sequences. Similarly, the weakly ${{\mathcal V}}$-compact sets are precisely relatively compact:
\[
{{\mathcal V}}({\boldsymbol{w}})={{\mathcal V}} ( {{\mathcal W}} ( {\boldsymbol{b}}))= ({{\mathcal V}} \circ {{\mathcal W}} )( {\boldsymbol{b}}) = {{\mathcal K}}({\boldsymbol{b}})= {\boldsymbol{k}} 
\]
(see Remark \ref{rem2.3} for the equality ${\boldsymbol{w}}= {{\mathcal W}}({\boldsymbol{b}})$ and, e.g., \cite[3.1.3]{P} for the equality ${{\mathcal V}}\circ {{\mathcal W}}= {{\mathcal K}}$). But ${{\mathcal V}} ^\mathrm{sur}({\boldsymbol{w}})= {\boldsymbol{w}}$.

However, for our purposes, the following analogue of the Lassalle--Turco Theorem \ref{LT}, characterizing \emph{weakly} ${{\mathcal A}}$-null sequences, will be sufficient.

\begin{prop}\label{prop4.4}
Let ${{\mathcal A}}$ be an operator ideal and let $(x_n)$ be a sequence in a Banach space $X$.
\begin{enumerate}
\item If $(x_n)$ is weakly ${{\mathcal A}}$-null, then $(x_n)$ is weakly null and weakly ${{\mathcal A}}$-compact.
\item If $(x_n)$ is weakly null and weakly ${{\mathcal A}}$-compact, then $(x_n)$ is weakly ${{\mathcal A}}^\mathrm{sur}$-null.
\end{enumerate}

In particular, if ${{\mathcal A}}$ is surjective, then $(x_n)$ is weakly ${{\mathcal A}}$-null if and only if $(x_n)$ is weakly null and weakly ${{\mathcal A}}$-compact.
\end{prop}

\begin{proof}
(a) We have $x_n= Ty_n$ for some $T \in {{\mathcal A}}(Y,X)$ and weakly null sequence $(y_n)$ in $Y$. Hence $(x_n)$ is weakly null. Since $(y_n)$ is relatively weakly compact in $Y$, $(x_n)$ is weakly ${{\mathcal A}}$-compact.

(b) We know that $(x_n)\subset T(K)$ for some $T\in {{\mathcal A}}(Y,X)$ and weakly compact subset $K$ of $Y$. We may and shall assume that $0 \in K$. Denote by $\overline T$ the injective associate of $T$. Then $T=\overline T q$, where $q: Y{\rightarrow} Z:=Y/\ker T$ is the quotient mapping, and $\overline T \in {{\mathcal A}}^\mathrm{sur}(Z, X)$ (by the definition of ${{\mathcal A}}^\mathrm{sur}$). 

If $q(K)$ and $\overline T(q(K))=T(K)$ are endowed with their weak topologies from $Z$ and $X$, respectively, then $\overline T: q(K){\rightarrow} T(K)$ is a continuous bijection, hence a homeomorphism. Let $x_n=Tk_n=\overline Tqk_n$ for some $k_n\in K$ and let $z_n=qk_n$. Then $z_n=\overline T ^{-1}x_n {\rightarrow} \overline T^{-1}(0)=0$ weakly (recall that $0\in K$ and $(x_n)$ is weakly null by the assumption). Since $x_n=\overline T z_n$ for all $n\in {{\mathbb N}}$, $(x_n)$ is weakly ${{\mathcal A}}^\mathrm{sur}$-null.
\end{proof}

We saw (in Sections 2.2, 2.3, 4.1, 4.2) that ${\boldsymbol{k}} _{{(p,r)}} = {{\mathcal K}}_{{(p,r)}} ({\boldsymbol{b}})= {{\mathcal K}}_{{(p,r)}} ({\boldsymbol{k}})$ and, similarly, ${\boldsymbol{u}} _{{(p,r)}} = {{\mathcal U}}_{{(p,r)}} ({\boldsymbol{b}}) = {{\mathcal U}}_{{(p,r)}} ({\boldsymbol{k}})$. Also ${\boldsymbol{w}}_{{(p,r)}} ={{\mathcal W}}_{{(p,r)}} ({\boldsymbol{b}})$ (see Section 4.1). In general, ${{\mathcal W}}_{{(p,r)}} ({\boldsymbol{b}}) \neq {{\mathcal W}}_{{(p,r)}} ({\boldsymbol{k}})$. Indeed, as was mentioned in the beginning of Section 4.3, for $X=\ell_{p^{\ast}}$, one has $\Phi _{(e_k)}=I_X$. Hence, ${{\mathcal W}}_p(X,X)={{\mathcal L}}(X,X)$ and therefore ${{\mathcal W}}_p({\boldsymbol{b}})(X)={\boldsymbol{b}}(X)$, but ${{\mathcal W}}_p({\boldsymbol{k}})(X)={\boldsymbol{k}}(X)$. We shall need the fact that in many cases ${{\mathcal W}} _{{(p,r)}} ({\boldsymbol{b}})= {{\mathcal W}}_{{(p,r)}} ({\boldsymbol{w}})$.

\begin{prop}\label{prop4.5}
Let $1\leq p < \infty$ and $1 < r \leq {p^{\ast}}$ with $r< \infty$ if $p=1$. Then 
\[
{{\mathcal W}}_{{(p,r)}} = {{\mathcal W}}_{{(p,r)}} \circ {{\mathcal W}} \text{ and } {\boldsymbol{w}}_{{(p,r)}} ={{\mathcal W}}_{{(p,r)}} ({\boldsymbol{w}}).
\]
\end{prop}

\begin{proof}
Let $X$ and $Y$ be Banach spaces and $T \in {{\mathcal W}}_{{(p,r)}}(Y,X)$. As in  the case of ${{\mathcal W}}_p$ in \cite[pp.~20--21]{SK1} and of ${{\mathcal K}}_{{(p,r)}}$ (see Section 2.1), we get a natural factorization $T=\overline \Phi _{(x_n)} S$ with $(x_n)\in \ell_p^w(X)$, where $\overline \Phi_{(x_n)}$ is the injective associate of $\Phi_{(x_n)}$ and $S \in {{\mathcal L}}(Y,Z)$, where $Z:= \ell_r / \ker \Phi_{(x_n)}$. Since $\Phi_{(x_n)} \in {{\mathcal W}}_{{(p,r)}} (\ell_r, X)$, we have $\overline \Phi _{(x_n)}\in {{\mathcal W}}_{{(p,r)}} ^\mathrm{sur}(Z,X) = {{\mathcal W}}_{{(p,r)}} (Z,X)$, because ${{\mathcal W}}_{{(p,r)}}$ is surjective. Since $\ell_r$ is reflexive, also $Z$ is, and therefore $S \in {{\mathcal W}}(Y,Z)$. This proves that ${{\mathcal W}}_{{(p,r)}} ={{\mathcal W}}_{{(p,r)}} \circ {{\mathcal W}}$. Now, using this, we have
\[
{\boldsymbol{w}}_{{(p,r)}} = {{\mathcal W}}_{{(p,r)}} ({\boldsymbol{b}})= ({{\mathcal W}}_{{(p,r)}} \circ {{\mathcal W}})({\boldsymbol{b}})= {{\mathcal W}}_{{(p,r)}} ({{\mathcal W}} ({\boldsymbol{b}}))={{\mathcal W}}_{{(p,r)}} ({\boldsymbol{w}}). \qedhere
\]
\end{proof}

\begin{rem}\label{rem4.6}
We do not know whether Proposition \ref{prop4.5} holds in the ``limit'' case $r=1$, i.e., for ${{\mathcal W}}_{(p,1)}$. It does not hold in the other ``limit'' case $p=1$, $r=\infty$, i.e., for ${{\mathcal W}}_1= {{\mathcal W}}_{(1, \infty)}$. Indeed, as we saw above, ${{\mathcal W}}_1(c_0, c_0)= {{\mathcal L}}(c_0, c_0)$, and hence 
\[
{\boldsymbol{w}} _1(c_0)= {{\mathcal W}}_1 ({\boldsymbol{b}})(c_0)= {\boldsymbol{b}} (c_0) \neq {\boldsymbol{w}} (c_0)= {{\mathcal W}}_1({\boldsymbol{w}}).
\]
In particular, ${{\mathcal W}}_{(1,\infty)} \not \subset {{\mathcal W}}$. In all other cases ${{\mathcal W}}_{{(p,r)}} \subset {{\mathcal W}}$. For $r\neq 1$, this is clear from Proposition \ref{prop4.5}. But ${{\mathcal W}}_{(p,1)} \subset {{\mathcal W}}_{{(p,r)}}$ (by the definition of ${{\mathcal W}}_{(p,\cdot)}$, because $B_{\ell_1}\subset B_{\ell_r}$).
\end{rem}

\begin{rem}\label{rem4.7}
In the case $p=1$, $1\leq r \leq {p^{\ast}}$, including also the case $p=1$, $r=\infty$ (cf. Remark \ref{rem4.6}), Proposition \ref{prop4.5} holds in a strong form for a large class of Banach spaces $X$. Namely, for $X$ that does not contain $c_0$ isomorphically. In this case (and only in this case), $\ell_1 ^w (X)=\ell_1^u(X)$, by the classical Bessaga--Pe{\l}czy\'{n}ski theorem \cite[Theorem~5]{BP} (see, e.g., \cite[8.3]{DF}). Therefore (see Section 4.2),
\[
{{\mathcal W}}_{(1,r)}(Y,X)={{\mathcal U}}_{(1,r)}(Y,X)=({{\mathcal K}}\circ {{\mathcal U}}_{(1,r)}\circ {{\mathcal K}})(Y,X)
\]
for all Banach spaces $Y$, and
\[
{\boldsymbol{w}}_{(1,r)}(X)= {\boldsymbol{u}}_{(1,r)}(X) ={{\mathcal U}}_{(1,r)}({\boldsymbol{k}})(X)= {{\mathcal N}}_{(\infty, \infty, {r^{\ast}})}({\boldsymbol{k}})(X).
\]
\end{rem}

Keeping in mind that the operator ideal ${{\mathcal W}}_{{(p,r)}}$ is surjective (see Section 4.1) we come to an omnibus characterization of weakly ${{(p,r)}}$-null sequences.

\begin{thm}\label{thm4.8}
Let $1\leq p < \infty$ and $1 < r \leq {p^{\ast}}$ with $r< \infty$ if $p=1$. For a sequence $(x_n)$ in a Banach space $X$ the following statements are equivalent:
\begin{enumerate}
\item $(x_n)$ is weakly $(p,r)$-null,
\item $(x_n)$ is weakly null and relatively weakly $(p,r)$-compact,
\item $(x_n)$ is weakly null and weakly ${{\mathcal W}}_{{(p,r)}}$-compact,
\item $(x_n)$ is weakly ${{\mathcal W}}_{{(p,r)}}$-null,
\item $(x_n)$ is uniformly weakly $(p,r)$-null.
\end{enumerate}
\end{thm}

\begin{proof}
(a)$\Rightarrow$(b) It is clear from the definition that $x_n{\rightarrow} 0$ weakly. Also, by the definition, we have (fixing, e.g., ${\varepsilon}=1$) $N\in {{\mathbb N}}$ and $(z_k)\in \ell_p^w(X)$ such that $\{x_N, x_{N+1}, ...\}\subset(p,r)$-conv$(z_k)$. Continuing verbatim to the proof of Theorem \ref{omni}, the second part of (a)$\Rightarrow $(b), we see that $(x_n)$ is relatively weakly ${{(p,r)}}$-compact.

Implications (b)$\Leftrightarrow $(c) and (c)$\Leftrightarrow$(d) are immediate from Propositions \ref{prop4.5} and \ref{prop4.4}, respectively.

To prove that (d)$\Rightarrow $(e), let $(x_n)$ be a weakly ${{\mathcal W}}_{(p,r)}$-null sequence. Then there are a weakly null sequence $(y_n)$ in a Banach space $Y$ and an operator $T \in {{\mathcal W}}_{{(p,r)}} (Y,X)$ such that $x_n= Ty_n$ for all $n\in {{\mathbb N}}$. The weak ${{(p,r)}}$-compactness of $T$ gives us a sequence $(w_k) \in \ell_p^w(X)$ such that $T(B_Y)\subset {{(p,r)}}$-conv$(w_k)$. We also have an $M>0$ such that ${\left\lVert {y_n} \right\rVert} \leq M$ for all $n\in {{\mathbb N}}$. Now $(z_k):=(M w_k )\in \ell_p(X)$ and $x_n\in {{(p,r)}}\textnormal{-conv}(z_k)$ for all $n\in {{\mathbb N}}$. As $(x_n)$ is weakly null in $X$, for every ${x^{\ast}} \in{X^{\ast}}$ and ${\varepsilon}>0$ there exists $N\in {{\mathbb N}}$ such that $\vert {x^{\ast}}(x_n)\vert \leq {\varepsilon}$ for all $n\geq N$. Hence, $(x_n)$ is uniformly weakly ${{(p,r)}}$-null.

The implication (e)$\Rightarrow$(a) is clear from the definitions.
\end{proof}

\begin{rem}
As we saw, all implications of Theorem \ref{thm4.8}, except (b)$\Rightarrow $(c), also hold in the ``limit'' cases $r=1$ and $p=1$, $r=\infty$. In the proof, we used that the implication (b)$\Rightarrow $(c) is immediate from Proposition \ref{prop4.5} (see also Remark \ref{rem4.6}). We do not know whether Theorem \ref{thm4.8} holds in these cases. If $p=1$ and $1\leq r \leq {p^{\ast}}$, Theorem \ref{thm4.8} holds in a stronger form for those Banach spaces $X$ that do not contain $c_0$ isomorphically. Indeed, by Remark \ref{rem4.7}, in condition (b), ``weakly $(1,r)$-compact'' is the same as ``unconditionally $(1,r)$-compact'' and in condition (c) ``weakly ${{\mathcal W}}_{(1,r)}$-compact'' is the same as ``${{\mathcal U}}_{(1,r)}$-compact'' and also the same as ``${{\mathcal N}}_{(\infty, \infty, {r^{\ast}})}$-compact''. In condition (d), ``weakly ${{\mathcal W}}_{(1,r)}$-null'' is the same as ``weakly ${{\mathcal U}}_{(1,r)}\circ {{\mathcal K}}$-null'', which is the same as ``${{\mathcal U}}_{(1,r)}$-null'', since compact operators take weakly null sequences to null sequences, i.e., ${{\mathcal K}}\subset {{\mathcal V}}$ (see, e.g., \cite[1.11.4]{P}). This shows that in the special case when $p=1$, $1\leq r \leq {p^{\ast}}$, and $X$ does not contain $c_0$ isomorphically, all conditions of Theorem \ref{omni_uncond} are equivalent to the conditions of Theorem \ref{thm4.8}.
\end{rem}
\end{section}

\bigskip

\begin{thebibliography}{DFJP}
\bibliographystyle{alpha}

\bibitem{ALO} K. Ain, R. Lillemets, E. Oja, {\em Compact operators which are defined by $\ell_p$-spaces}, Quaest. Math. {\bf 35} (2012) 145--159.

\bibitem{AO1} K. Ain, E. Oja, {\em A description of relatively $(p,r)$-compact sets}, Acta Comment. Univ. Tartu. Math. {\bf 16} (2012)  227--232.

\bibitem{AMR} R. Aron, M. Maestre, P. Rueda, {\em $p$-Compact holomorphic mappings}, RACSAM {\bf 104} (2010) 353--364.

\bibitem{BP} C. Bessaga, A. Pe{\l}czy{\'n}ski, {\em On bases and unconditional convergence of series in {B}anach spaces}, Studia Math. {\bf 17} (1958) 151--164.

\bibitem{BCFP} G. Botelho, D. Cariello, V.V. F{\'a}varo, D. Pellegrino, {\em Maximal spaceability in sequence spaces}, Linear Algebra Appl. {\bf 437} (2012) 2978--2985.

\bibitem{BR} J. Bourgain, O. Reinov, {\em On the approximation properties for the space $H^\infty$}, Math. Nachr. {\bf 122} (1985) 19--27.

\bibitem{CS} B. Carl, I. Stephani, {\em On $A$-compact operators, generalized entropy numbers and entropy ideals}, Math. Nachr. {\bf 199} (1984) 77--95.

\bibitem{CSa} J.M.F. Castillo, F. Sanchez, {\em Dunford--{P}ettis-like properties of continuous vector function spaces}, Rev. Mat. Univ. Complut. Madrid {\bf 6} (1993) 43--59.

\bibitem{CK} Y.S. Choi, J.M. Kim, {\em The dual space of $({{\mathcal L}}(X,Y ), \tau_p)$ and the $p$-approximation property}, J. Funct. Anal. {\bf 259} (2010) 2437--2454.

\bibitem{DF} A. Defant, K. Floret, {\em Tensor Norms and Operator Ideals}, North-Holland Publishing Co., Amsterdam, 1993.

\bibitem{DOPS} J.M. Delgado, E. Oja, C. Pi\~{n}eiro, E. Serrano, {\em The $p$-approximation property in terms of density of finite rank operators}, J. Math. Anal. Appl. {\bf 354} (2009) 159--164.

\bibitem{DPS1} J.M. Delgado, C. Pi\~{n}eiro, E. Serrano, {\em Operators whose adjoints are quasi $p$-nuclear}, Studia Math. {\bf 197} (2010) 291--304.

\bibitem{DPS2} J.M. Delgado, C. Pi\~{n}eiro, E. Serrano, {\em Density of finite rank operators in the Banach space of $p$-compact operators}, J. Math. Anal. Appl. {\bf 370} (2010) 498--505.

\bibitem{DJP} J. Diestel, H. Jarchow, A. Pietsch, {\em Operator ideals},  in: W.B.~Johnson and J.~Lindenstrauss (eds.), Handbook of the Geometry of Banach Spaces, Vol. 1, pp. 437--496, Elsevier, Amsterdam, 2001. 

\bibitem{DJT} J. Diestel, H. Jarchow, A. Tonge, {\em Absolutely Summing Operators}, Cambridge Univ. Press, Cambridge, 1995.

\bibitem{FS1} J. Fourie, J. Swart, {\em Banach ideals of $p$-compact operators}, Manuscripta Math. {\bf 26} (1979) 349--362.

\bibitem{FS2} J. Fourie, J. Swart, {\em Tensor products and {B}anach ideals of {$p$}-compact operators}, Manuscripta Math. {\bf 35} (1981) 343--351.

\bibitem{GLT} D. Galicer, S. Lassalle, P. Turco, {\em The ideal of $p$-compact operators: a tensor product approach}, Studia Math. {\bf 211} (2012) 269--286.

\bibitem{LT} S. Lassalle, P. Turco, {\em The Banach ideal of ${{\mathcal A}}$-compact operators and related approximation properties}, J. Funct. Anal. {\bf 265} (2013) 2452--2464.

\bibitem{O-JM} E. Oja, {\em A remark on the approximation of $p$-compact operators by finite-rank operators}, J. Math. Anal. Appl. {\bf 387} (2012) 949--952.

\bibitem{O-JF} E. Oja, {\em Grothendieck's nuclear operator theorem revisited with an application to $p$-null sequences}, J. Funct. Anal. {\bf 263} (2012) 2876--2892.

\bibitem{P} A. Pietsch, {\em Operator Ideals}, Deutsch. Verlag Wiss., Berlin, 1978; North-Holland Publishing Company, Amsterdam-New York-Oxford, 1980.

\bibitem{P2} A. Pietsch, {\em The ideal of $p$-compact operators and its maximal hull}, Proc. Amer. Math. Soc. {\bf 142} (2014) 519--530.

\bibitem{PD} C. Pi\~{n}eiro, J.M. Delgado, {\em $p$-Convergent sequences and Banach spaces in which $p$-compact sets are $q$-compact}, Proc. Amer. Math. Soc. {\bf 139} (2011) 957--967.

\bibitem{R1984} O. Reinov, {\em A survey of some results in connection with Grothendieck approximation property}, Math. Nachr. {\bf 119} (1984) 257--264.

\bibitem{Ry} R.A. Ryan, {\em Introduction to Tensor Products of Banach Spaces}, Springer Monographs in Mathematics, Springer-Verlag, London, 2002.

\bibitem{SK1} D.P. Sinha, A.K. Karn, {\em Compact operators whose adjoints factor through subspaces of $\ell_p$}, Studia Math. {\bf 150} (2002) 17--33.

\bibitem{SK2} D.P. Sinha, A.K. Karn, {\em Compact operators which factor through subspaces of $\ell_p$}, Math. Nachr. {\bf 281} (2008) 412--423.

\bibitem{S72} I. Stephani, {\em Surjektive Operatorenideale {\"u}ber der Gesamtheit aller Banachr{\"a}ume}, Wiss. Z. Friedrich-Schiller-Univ. Jena {\bf 21} (1972) 187--206.

\bibitem{S73} I. Stephani, {\em Surjektive Operatorenideale {\"u}ber der Gesamtheit aller Banachr{\"a}ume und ihre Erzeugung}, Beitr{\"a}ge Anal. {\bf 5} (1973) 75--89.

\bibitem{S3} I. Stephani, {\em Generating systems of sets and quotients of surjective operator ideals}, Math. Nachr.  {\bf 99}  (1980), 13--27.

\end{thebibliography}

\end{document}


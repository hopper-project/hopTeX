
\documentclass[a4paper]{amsart}

\usepackage{amsthm,amsfonts,amsmath, amssymb}

\usepackage{cmbright}
\usepackage[T1]{fontenc}

\usepackage{geometry}                
\geometry{}                   
\usepackage{anysize}
\marginsize{3cm}{3cm}{3cm}{3cm}

\setlength{\footnotesep}{{1.25}\footnotesep}
\setlength{\parindent}{0cm}
\setlength{\parskip}{2ex}

\usepackage{graphicx,float,enumerate,subfigure,tikz}
\usepackage[ruled,boxed,commentsnumbered,norelsize]{algorithm2e}
\usepackage{verbatim}
\usepackage{url}
\usepackage{epstopdf}
\usepackage[capitalise]{cleveref}

\newtheorem{theorem}{Theorem}
\newtheorem{corollary}[theorem]{Corollary}
\newtheorem{lemma}[theorem]{Lemma}
\newtheorem{proposition}[theorem]{Proposition}
\newtheorem{problem}[theorem]{Problem}
\newtheorem{claim}{Claim}

\newtheorem{conjecture}[theorem]{Conjecture}
\newtheorem{question}[theorem]{Question}

\theoremstyle{definition}
\newtheorem{dfn}{Definition}[section]

\theoremstyle{remark}
\newtheorem{remark}[theorem]{Remark}

\crefname{remark}{Remark}{Remarks}

\usepackage{mathtools}
\DeclarePairedDelimiter\ceil{\lceil}{\rceil}
\DeclarePairedDelimiter\floor{\lfloor}{\rfloor}
 
 
 

 

\crefname{rmk}{Remark}{Remarks}
\crefname{problem}{Problem}{Problems}
\date{\today}
\title{On polytopes close to being simple}

\author{Guillermo Pineda-Villavicencio}
\author{Julien Ugon}
\author{David Yost}
\address{Centre for Informatics and Applied Optimisation, Federation University Australia}
\email{\texttt{\{g.pinedavillavicencio,j.ugon,d.yost\}@federation.edu.au}}

\thanks{Thanks}
\keywords{Reconstruction, simple polytope, $k$-skeleton}
\subjclass[2010]{Primary 52B05; Secondary 52B12}

\begin{document}
\begin{abstract} The reconstruction of simple polytopes from their graphs was first established by Blind and Mani, and later by Kalai. This result was then generalised to polytopes with at most two nonsimple vertices by Doolittle,  and by Nevo and the present three authors, independently. A vertex is \emph{nonsimple} if its number of incident edges is greater than the polytope dimension $d$; it is {\it simple} otherwise. These two recent papers measure deviation from simple polytopes by counting the number of nonsimple vertices. In this paper we continue studying this notion by focusing on polytopes with small number of vertices. We prove  that reconstruction from graphs also hold for polytopes with $d+3$ vertices and at most $d-1$ nonsimple vertices, polytopes with $d+4$ vertices and at most $d-2$ nonsimple vertices, and polytopes with $d+k$ ($k\ge 5$) vertices and at most  $d-k+3$ nonsimple vertices. The assertions about polytopes with $d+k$ vertices and $k=3,4,5$ are best possible.

We also study another measure of deviation from being a simple polytope, the {\it excess}, defined as $\xi(P):=\sum_{v\in \operatorname{vert} P} (\deg(v)-d)$, where $\deg$ denote the number of edges incident to the vertex $v$ in the polytope $P$. Simple polytopes have excess zero. We prove that polytopes with excess at most $d-1$ are reconstructible from their graphs and this is best possible.
  
In obtaining the previous assertions, we produce a number of intermediate results of independent interest, such as that polytopes with $d+k$ vertices and at most $d-1$ nonsimple vertices are $(d-k)$-fold pyramids.  
\end{abstract}
\maketitle

\section{Introduction}

The $k$-dimensional {\it skeleton} of a polytope $P$, denoted $k$-$skel(P)$, is the set of all its faces of dimension $\le k$. The 1-skeleton of $P$ is the {\it graph} $G(P)$ of $P$. Reconstructing  a polytope from its $k$-skeleton amounts to giving the combinatorial structure\footnote{The {\it combinatorial structure} of a polytope is given by partially ordering its faces by inclusion.} of the polytope by solely querying the $k$-skeleton. It however suffices to reconstruct the facets of $P$, since the combinatorial structure of  a polytope is determined by the vertex-facet incidence graph, where a facet is adjacent to a vertex if it contains the vertex \cite[Sec.~16.1.1]{GooORo04}. Every $d$-polytope is reconstructible from its $(d-2)$-skeleton \cite[Thm.~12.3.1]{Gru03}, and there are combinatorially inequivalent $d$-polytopes with the same $(d-3)$-skeleton: take, for instance, a bipyramid over a $(d-1)$-simplex and a pyramid over a bipyramid over a $(d-2)$-simplex. Throughout the paper we let $d:=\dim P$.

The graph of a polytope  only gives partial information on the structure of the polytope. It does not even determine the dimension of the polytope, as exemplified by the cyclic $d$-polytope on $n$ vertices \cite[p.~11]{Zie95} and the $n$-simplex $n>d$. Hence,  we always assume the dimension is given; see \cite[Notes of Ch.~3]{Zie95}. In this case, we say that the dimension is not reconstructible from the graph. More generally, we say that a parameter or a property of a polytope is {\it reconstructible} from the graph if it can be recovered from the graph; otherwise it is {\it nonreconstructible}. For instance, being simple is reconstructible from the graph, while the number $f_{d-1}$ of facets ($f_{d-1}\ge d+3$) is in general nonreconstructible; a $(d-3)$-fold pyramid over a bipyramid over a triangle has the same number of facets,  $d+3$, and the same graph as a bipyramid over a $(d-1)$-simplex. 

Furthermore, not even the graph together with the {\it $f$-vector}, a vector with the number of faces of each dimension, determine the combinatorial structure of the polytope. Indeed, all neighbourly simplicial $d$-polytopes share the same graph and $f$-vector as the cyclic $d$-polytope \cite[Sec.~8.3-8.4]{Zie95}.
   

For some classes of polytopes, the graph however determines their combinatorial structure. Polytopes with dimension at most 3 are all reconstructible from their graphs, with the case of dimension 3 following from Steinitz's Theorem \cite[Ch.~4]{Zie95} and Whitney's Theorem \cite[Sec.~4.1]{Zie95}. For higher dimensions, Blind and Mani \cite{BliMan87}, and later Kalai \cite{Kal88}, showed that a simple polytope is determined by its graph \cite[Sec.~3.2]{Zie95}. Nevo and the three authors of this paper \cite{NevPinUgo17}, and independently Doolittle \cite{Doo17}, showed that polytopes wit at most two nonsimple vertices are reconstructible from their graphs. The paper \cite{NevPinUgo17} also exhibited for each $d\ge 4$ a pair of $d$-polytopes $Q^1_d$ and $Q^2_d$ with $2d$ vertices, exactly $d-1$ nonsimple vertices and the same $(d-3)$-skeleton, proving that the previous result is best possible for $d=4$. See \cref{fig:nonrecontructible-3nonsimple} for Schlegel diagrams of some of these polytopes $Q^1_d$ and $Q^2_d$. Other reconstruction results can be found in \cite[Sec.~20.5]{GooORo04}.

\begin{figure}
\begin{center}
\includegraphics{2-skel-Construction}
\end{center}
\caption{Examples of polytopes $Q_d^1$ and $Q_d^2$. The missing 2-face of a bipyramid over a simplex is highlighted in (d), while in (c) this bipyramid is split into two simplices.}
\label{fig:nonrecontructible-3nonsimple}
\end{figure}
In this paper we keep studying the structure and reconstruction of polytopes which are ``close'' to being simple. For a $d$-polytope, we call a vertex of degree\footnote{The {\it degree} of a vertex is the number of edges incident with the vertex.}  $d$ {\it simple}, otherwise we call it {\it nonsimple}. There are many approaches for a $d$-polytope to get close to being simple;  we are interested in those that guarantee reconstructibility from graphs.  

\begin{description}
\item[Approach 1] A $d$-polytope with ``few'' nonsimple vertices; this is the approach taken in \cite{Doo17,NevPinUgo17}. 
\item[Approach 2] A $d$-polytope with small {\it excess}, where the {\it excess} $\xi$ of a polytope is defined as $\sum_{u\in \operatorname{vert} P} (\deg u-d)$; simple polytopes have excess 0. 

\item[Approach 3] A $d$-polytope in which every pair of nondisjoint facets intersect at a ridge. 
\end{description}
In Approach 1 the polytopes $Q^1_d$ and $Q^2_d$ show that in general by ``few'' we must mean at most $d-2$ nonsimple vertices. With regard to Approach 2, the aforementioned  pair of a bipyramid over a $(d-1)$-simple and a pyramid over a bipyramid over a $(d-2)$-simplex have excess exactly $d$. So by small excess we must mean excess at most $d-1$. But then the excess theorem \cite[Thm.~3.4]{PinUgoYos16a} states that the smallest values of the excess of a $d$-polytope are 0 and $d-2$. The results of this paper, described next, revolve around Approaches 1 and 2. We want to highlight that some of the results of the paper came about after the authors test a number of hypotheses on  \texttt{polymake} \cite{GawJos00} .
 
{\bf Theorem.} {\it Let $P$ be a $d$-polytope with excess at most $d-1$. Then the graph of $P$ determines the entire combinatorial structure of $P$. This result is best possible in the sense that there are $d$-polytopes with excess $d$ which are not reconstructible from their graphs.
}

The next results allow us to decide whether a polytope with small number of nonsimple vertices is pyramidal or not. A polytope is an {\it $r$-fold pyramid} if it is a pyramid whose basis is an $(r-1)$-fold pyramid, and any polytope is a 0-fold pyramid. If a vertex $u$ is an apex of a pyramid $P$, we say that $P$ is {\it pyramidal} at $u$.

{\bf Theorem.} {\it Let $P$ be a $d$-polytope with at most $d-1$ nonsimple vertices and a vertex $u$ adjacent to every other vertex in the polytope. Then the polytope is pyramidal at $u$, and it is reconstructible from its $k$-skeleton if and only if the basis is reconstructible from its $k$-skeleton. 

Furthermore, these statements are best possible as there are pyramidal and nonpyramidal $d$-polytopes with exactly $d$ nonsimple vertices and the same graph.}

The next two propositions give some interesting structural results.

{\bf Proposition.}{ \it Let $P$ be a $d$-polytope with at most $d+k$ vertices and at most $d-1$ nonsimple vertices. Then $P$ is a $(d-k)$-fold pyramid.}

The previous proposition says nothing about $d=k$; this is fixed  next.

{\bf Proposition.} {\it Let $P$ be a $d$-polytope with $2d$ vertices and at most $d-2$ nonsimple vertices. Then $P$ is either a simplicial $d$-prism or a pyramid.

Furthermore, this is best possible as there are $4$-polytopes with 8 vertices and 3 nonsimple vertices which are neither simplicial $4$-prisms nor pyramids (cf.~Polytopes $Q_4^1$ and $Q_4^2$).}

With the use of the above results at hand, we prove our reconstruction results involving polytopes with a small number of vertices.

\begin{table}
\begin{tabular}{c c c c c}
{Facet}&{Polytope 1}&{Polytope 2}&{Polytope 3}&Polytope 4\\
\hline
 0:&\{2 3 4 5 6\}&\{2 3 4 5 6\}&\{1 2 3 4 5 6\}&\{2 3 4 5 6\}\\
1:&\{1 3 4 5 6\}&\{1 3 4 5 6\}&\{0 3 4 5 6\}&\{1 3 4 5 6\}\\
2:&\{1 2 5 6\}&\{0 1 2 5 6\}&\{0 2 5 6\}&\{0 1 2 5 6\}\\
3:&\{1 2 4 6\}&\{1 2 4 6\}&\{0 2 4 6\}&\{0 1 2 3 4\}\\
4:&\{1 2 3 5\}&\{0 2 3 5\}&\{0 2 3 5\}&\{1 2 4 6\}\\
5:&\{0 2 3 4\}&\{0 1 3 5\}&\{0 1 3 4\}&\{0 2 3 5\}\\
6:&\{0 1 3 4\}&\{1 2 3 4\}&\{0 1 2 4\}&\{0 1 3 5\}\\
7:&\{0 1 2 4\}&\{0 1 2 3\}&\{0 1 2 3\}\\
8:&\{0 1 2 3\}\\
\hline
\end{tabular}
\caption{Vertex-facet incidences of all nonpyramidal $4$-polytopes with seven vertices, four nonsimple vertices and  and the same graph. The common graph has degree sequence $(4,4,4,5,5,6,6)$. These four polytopes can be obtained from the catalogues  in \cite{FukMiyMor13a} or the Gale diagrams of all 4-polytopes with seven vertices presented in \cite[Fig.~5]{Gru70}.}
\label{tab:4Polytopes7Vertices4Nonsimple}
\end{table}
	
	
{\bf Theorem.}{ \it Let $P$ be a $d$-polytope with $d+3$ vertices and  at most $d-1$ nonsimple vertices. Then the polytope is either a  $(d-3)$-fold  pyramid over  a simplicial $3$-prism,  a $(d-3)$-fold  pyramid over  $Q^1_3$,  or a $(d-2)$-fold pyramid over a pentagon. As a consequence, the graph of $P$ determines its entire combinatorial structure. 

These results are best possible in the sense that there are nonpyramidal  $d$-polytopes with $d+3$ vertices and exactly $d$ nonsimple vertices which are not reconstructible from their graphs (cf.~\cref{tab:4Polytopes7Vertices4Nonsimple}).
}

We can do a bit more for polytopes with $d+3$ vertices. Define the  {\it join} of two graphs $G_1$ and $G_2$ with disjoint vertex sets $V_1$ and $V_2$ and edge sets $E_1$ and $E_2$, respectively, as the graph with vertex set $V_1\cup V_2$ and edge set $E_1\cup E_2$ plus all the edges joining a vertex in $V_1$ to a vertex  in $V_2$.

\begin{figure}
\begin{center}
\includegraphics{dPlus3}
\end{center}
\caption{Structure of $d$-polytopes with $d+3$ vertices.}
\label{fig:dPlus3}
\end{figure}

{\bf Proposition.} {\it Let $P$ be a $d$-polytope with $d+3$ vertices. Then the following hold.
\begin{enumerate}[(i)]
\item If $P$ has exactly $d$ nonsimple vertices, then the graph of $P$ is the join of the complete graph on $d-2$ vertices and the graph in \cref{fig:dPlus3} (b). 

\item If $P$ has exactly $d+1$ nonsimple vertices, then $P$ is either a $(d-3)$-fold pyramid over the polytope in \cref{fig:dPlus3} (a) or the graph of $P$ is the join of the complete graph on $d-2$ vertices and the graph in \cref{fig:dPlus3} (c).
\end{enumerate}}

{\bf Theorem.} {\it Let $P$ be a $d$-polytope with $d+4$ vertices. If  $P$ has at most $d-2$ nonsimple vertices, then the graph of $P$ determines its entire combinatorial structure. Furthermore, in the case of $P$ having at most $d-1$ nonsimple vertices, for $d=3$ there are exactly five such polytopes, and for every $d\ge4$ there are exactly nine such polytopes in which case $P$ is either a simplicial $4$-prism or a $(d-4)$-fold pyramid, and it is reconstructible from its 2-skeleton. 

These results are best possible in the sense that there are  nonpyramidal $4$-polytopes with eight vertices and three nonsimple vertices which are not reconstructible from their graphs  (cf.~Polytopes $Q^1_4$ and $Q^2_4$).}

The previous two results can be pushed a bit further. 

{\bf Theorem.} {\it  Let $k\ge 5$ and let $P$ be a $d$-polytope with $d+k$ vertices and  at most $d-k+3$ nonsimple vertices. Then the graph of $P$ determines the entire combinatorial structure of $P$. The particular case of $k=5$ is best possible as shown by the pair of polytopes $Q^1_5$  and $Q^2_5$.}

In a similar vein, we wonder if the following is true for every $k\ge 1$. 

\begin{problem}\label{prob:smallVertRec} Let $k\ge 1$ and let $P$ be a $d$-polytope with $d+k$ vertices and  at most $d-2$ nonsimple vertices. Is it true that for $d\ge k+1$  the graph of $P$ determines its entire combinatorial structure? \end{problem}

A more general version of \cref{prob:smallVertRec} was posed in \cite{NevPinUgo17}. From the previous results it ensues that \cref{prob:smallVertRec} is true for any $d$ and $k=1,2,3,4,5$.  Note that \cref{prop:d+k} reduces \cref{prob:smallVertRec} to the following subproblem.

\begin{problem}\label{prob:2dVert} Let $P$ be a $d$-polytope with $2d$ vertices and  at most $d-2$ nonsimple vertices. Does the graph of $P$ determine its entire combinatorial structure? \end{problem}

\section{Polytopes with small excess} Recall that the excess $\xi$ of a $d$-polytope $P$ is $\xi(P)=\sum_{u\in \operatorname{vert} P} (\deg u-d)$. Polytopes with small excess $\xi=d-2,d-1$ were first studied in \cite{PinUgoYos16a}, where the following structural results appeared.

\begin{theorem}[Excess theorem, {\cite[Thm.~3.3]{PinUgoYos16a}}]\label{thm:excess} The smallest values of the excess of a polytope are $0$ and $d-2$.
\end{theorem}

\begin{theorem}[Structure of polytopes with excess $d-2$, {\cite[Thm.~4.7]{PinUgoYos16a}}]\label{thm:excess-d-2-full-story} Any $d$-polytope $P$  with excess exactly $d-2$ either

\begin{enumerate}[(i)]
\item has a unique nonsimple vertex, which is the intersection of two facets, or
\item has $d-2$ vertices of excess degree one, which form a simplex $(d-3)$-face which is the intersection of four facets.
\end{enumerate}  

In either case, $P$ has a facet with excess $d-3$.
\end{theorem}
\begin{theorem}[Structure of polytopes with excess $d-1$, {\cite[Thm.~4.16]{PinUgoYos16a}}]\label{thm:excess-d-1-full-story} Let $P$ be $d$-polytope with  excess degree $d-1$, where  $d>3$. Then $d=5$ and either

\begin{enumerate}[(i)]
\item there is a single vertex with excess four, which is the intersection of three facets,

\item there are two vertices with excess two, and the edge joining them is the intersection of two facets, or

\item there are four vertices each with excess one, which form a quadrilateral 2-face which is the intersection of two facets.
\end{enumerate}
\end{theorem}

We also need to borrow the following definitions and results from \cite{NevPinUgo17}.

Call an acyclic orientation  of $G(P)$ {\it good } if for every nonempty face $F$ of $P$ the graph $G(F)$ of $F$ has a unique sink\footnote{ A {\it sink} is a vertex with no directed edges going out.}. Actually, we only need that the acyclic orientation has a unique sink in every facet, so for us this possibly larger set represents the good orientations. An acyclic orientation of $G(P)$  induces a partial ordering of the vertices  of $G(P)$, and those orientations arising from shellings of the dual polytope are good orientations.

Define an {\it initial} set of a graph $G(P)$ with respect to some orientation as a set such that no edge is directed from a vertex not in the set to a vertex in the set.

Define a {\it $k$-frame} as a subgraph of $G(P)$ isomorphic to the star $K_{1,k}$, where the vertex of degree $k$ is called the {\it root} of the frame. If the root of a frame is a simple vertex, we say that the frame is {\it simple}. Any induced $(d-1)$-connected subgraph of $G$ whose simple vertices have each degree $d-1$ and each nonsimple vertex has degree $\ge d-1$ is called a {\it feasible}  subgraph. We say that a $k$-frame with root $x$ is {\it valid} if there is a facet containing $x$ and precisely the edges of the frame. If $x$ is a simple vertex any of its $(d-1)$-frames is valid.

By considering different shelling orders of the dual polytope of a polytope $P$, the paper \cite{NevPinUgo17} established the following.  
\begin{corollary}[{ \cite[Cor.~4.3]{NevPinUgo17}}] \label{cor:Orientation-F-Initial}For every facet of a polytope the following orientations exist.
\begin{enumerate}[(i)]
\item A good orientation of $G(P)$ such that the vertices of the facet are initial, and within the facet any two vertices can be chosen to be the sink and source.

\item A good orientation of $G(P)$ such that the vertices of the facet are initial, and within the facet a face $R$ is initial and a vertex not in $R$ is a sink in the facet.
\end{enumerate}
\end{corollary}

\begin{lemma}[{ \cite[Cor.~4.4]{NevPinUgo17}}] \label{lem:feasible-subgraphs}  Let $P$ be a $d$-polytope, and let $H$ be a feasible subgraph of $G(P)$ containing at most $d-2$ nonsimple vertices. If the graph $G(F)$ of some facet $F$ is contained in $H$, then $H\simeq G(F)$.
\end{lemma}

We are now ready to prove \cref{thm:ExcessRec}.

\begin{theorem}\label{thm:ExcessRec} Let $P$ be a $d$-polytope with excess at most $d-1$. Then the graph of $P$ determines the entire combinatorial structure of $P$. This result is best possible in the sense that there are $d$-polytopes with excess $d$ which are not reconstructible from their graphs.
\end{theorem}

\begin{proof} We prove the theorem for $d\ge 4$. We first consider the case of $\xi=d-2$. In view of \cref{thm:excess-d-2-full-story} and of the reconstruction results of \cite{NevPinUgo17}, namely \cite[Thms.~4.5-4.6]{NevPinUgo17}, we can assume that 
the polytope $P$ has $d-2$ vertices of excess degree one, which form a $(d-3)$-simplex $R$ which is the intersection of four facets.

Denote by $\mathcal{H}_R$ the set of feasible subgraphs which contain the complete graph $G(R)$ on the vertices of $R$, and by  $\mathcal{A}_R$  the set of all acyclic orientations of $G(P)$ in which (1)  a subgraph $H_R$ in $\mathcal{H}_R$ is initial, and (2) within $H_R$ the subgraph $G(R)$ is initial. It follows that $H_R$ has a sink which is a simple vertex.

{\bf Claim 1.} A feasible subgraph $H_R$  of $G$  is the graph of a facet containing $R$ iff (1) it contains $R$,  (2) is initial with respect to a good orientation $O$ in $\mathcal{A}_R$, and (3) has a unique sink which is a simple vertex. 
\begin{proof}

We reason as in the proof of Claim 1 of \cite[Thm.~4.6]{NevPinUgo17}.
First consider a facet $F_R$ containing $R$. \cref{cor:Orientation-F-Initial}(ii) ensures the existence of a good orientation of $G(P)$ in which the vertices of $F_R$ are initial, that the vertices of $R$ are initial within $F_R$, and that a simple vertex $x$ is a sink in the facet.  This proves the ``only if'' part of the claim. 

 Let $O\in \mathcal{A}_R$ and let $h_k^O$ denote the number of simple vertices of $G$ with indegree $k$. Define\[f^O_{R}:=h^O_{d-1}+dh_d^O.\] 
The function $f^O_{R}$ counts the number of pairs $(F,w)$, where $F$ is a facet of $P$ and $w$ is a simple sink in $F$ of an orientation in $\mathcal{A}_R$. 

Let $O$ be any acyclic orientation, let $H_R$ be a feasible subgraph in $\mathcal{H}_R$, and let $x$ be the simple sink in $H_R$ with respect to $O$. Suppose $H_R$ does not represent the facet $F_R$ containing $x$. Then, in view of \cref{lem:feasible-subgraphs}, there are vertices of  $F_R$ outside $H_R$. Since $H_R$ is initial with respect to $O$, the facet $F_R$ would contain two sinks, one of them being $x$. 

Consequently, given that there is a good orientation in  $ \mathcal{A}_R$ and a subgraph $H_R$ representing a facet, we have that \[\min f^O_{R}=f_{d-1}.\]  Also, an orientation of $\mathcal{A}_R$ minimising $f_{R}^O$ must be a good orientation. 

 Let $x$ be the simple sink in $H_R$ with respect to $O$, then $x$ defines a unique facet $F_R$ of $P$. Therefore, all the other vertices of $F_R$ are smaller than $x$ with respect to the ordering induced by $O$. Since $H_R$ is an initial set in $O$ and since there is directed path in $G(F_R)$ from any other vertex of $G(F_R)$ to $x$,  we must have $\operatorname{vert} F_R \subseteq V(H_R)$ and we are at home by \cref{lem:feasible-subgraphs}. 
\end{proof}
	
Running through all the good orientations in  $\mathcal{A}_R$, we recognise all the graphs of facets containing $R$; say that its set is $\mathcal{F}_{R}$. By \cite[Lem.~4.5]{PinUgoYos16a} there are exactly four such facets, and in each such facet the nonsimple vertices have degree $d-1$. Consequently, for each nonsimple vertex in $R$ we also have the valid frames in each of these facets. 

We now turn our attention to facets nondisjoint from $R$. In virtue of \cite[Lem.~4.5]{PinUgoYos16a}, every facet $F$ in $P$ intersecting $R$ but not containing it misses exactly one vertex of $R$, and every vertex of $R$  has degree $d$ in $F$. Thus, for each nonsimple vertex $u$ in $R$ the facet containing $u$ and missing some vertex $v$ in $R$ are given by the $d$-frame of $u$ which misses $v$. This in turn implies that we know all the valid frames of each nonsimple vertex in $P$, and of course, of each simple vertex. Thus, the result follows from \cite[Thm.~2.3]{Jos00}, which states that a polytope can be reconstructed from its graph if the valid frames of each vertex are known.  

For the case of $\xi=d-1$, thanks to \cref{thm:excess-d-1-full-story} and the results on polytopes with one or two nonsimple vertices presented in \cite[Thms.~4.5-4.6]{NevPinUgo17}, we can assume that the polytope has dimension five and the four vertices of excess one are contained in a quadrilateral 2-face $R$. Then the proof proceed mutatis mutandis as in the case of $\xi=d-2$, just replacing the $(d-3)$-simplex by the 2-face. 

A bipyramid over a $(d-1)$-simplex and a pyramid over a bipyramid over a $(d-2)$-simplex give a pair  of nonreconstructible $d$-polytopes with excess $d$, and exactly $d$ vertices of degree $d+1$ for $d\ge 4$.

This completes the proof of the theorem.  
\end{proof}

\section{Pyramidal or not pyramidal} In this section, our $d$-polytopes have at most $d-1$ nonsimple vertices and a vertex $u$ adjacent to every other vertex.

\begin{theorem} \label{thm:pyramidRec} Let $P$ be a $d$-polytope with at most $d-1$ nonsimple vertices and a vertex $u$ adjacent to every other vertex in the polytope. Then the polytope is pyramidal at $u$, and it is reconstructible from its $k$-skeleton if and only if the basis is reconstructible from its $k$-skeleton. 

Furthermore, these statements are best possible as there are pyramidal and nonpyramidal $d$-polytopes with exactly $d$ nonsimple vertices and the same graph.\end{theorem}

\begin{proof} We only prove the reconstruction statement for graphs (1-skeletons), but the result extends to $k$-skeletons for $k\ge 2$. 

Let $G$ be the graph of the polytope $P$, and let $H:=G-u$ be the subgraph obtained from deleting the vertex $u$. If $H$ were not feasible then it must fail to be $(d-1)$-connected by \cref{lem:feasible-subgraphs}, and it would be at most $(d-2)$-connected. Thus removing the vertices disconnecting $H$ plus $u$ would disconnect  $G$, contradicting Balinski's theorem \cite[Sec.~3.4]{Zie95}. Thus $H$ must be feasible.

In this case, since $H$ contains the graph of a facet and at most $d-2$ nonsimple vertices, it coincides with the graph of the unique facet $F$ which does not contain $u$ by \cref{lem:feasible-subgraphs}. Thus $P$ is pyramidal with basis $F$. If $P$ is pyramidal, then $H$ is inevitably feasible. 

If $F$ is reconstructible, then we can obtain the $(d-2)$-faces of $F$, and from them, the remaining facets of $P$. Otherwise $P$ is not reconstructible. 

For examples of pyramidal and nonpyramidal $d$-polytopes with exactly $d$ nonsimple vertices and the same graph, look no further than a bipyramid over a $(d-1)$-simplex and a pyramid over a bipyramid over a $(d-2)$-simplex. .
\end{proof}

	
\section{Polytopes with a small number of vertices} 
We start this section by stating a few result which will prove useful, including a characterisation of $d$-polytopes with $d+2$ vertices which is an easy consequence of \cite[Sec.~6.1]{Gru03}.

\begin{proposition}[{\cite[Lem.~10(ii)-(iii)]{PrzYos16}}]\label{prop:simplePolytopes}  Up to combinatorial equivalence, the $d$-simplex and the simplicial $d$-prism are the only  simple $d$-polytopes with at most $2d$ vertices. 
\end{proposition}

\begin{remark}\label{rmk:simpleVertexOutside}
Let $P$ be a polytope, $F$ a facet of $P$ and $u$ a nonsimple vertex of $P$ which is contained in $F$. If $u$ is adjacent to a simple vertex $x$ of $P$ in $P\setminus F$, then $u$ must be adjacent to another vertex in $P\setminus F$, other than $x$.
\end{remark}

\begin{proposition}\label{prop:d+k} Let $P$ be a $d$-polytope with at most $d+k$ vertices and at most $d-1$ nonsimple vertices. Then $P$ is a $(d-k)$-fold pyramid.
\end{proposition}
\begin{proof} 

We proceed by induction on $d$ for all $k\ge 1$.  The basis case $d=3$ and all $k\ge1$ only requires to look at $k=1,2$, in which cases, $P$ must be either a simplex or a pyramid over a quadrilateral \cite[Fig.~2]{BriDun73}, respectively.   Now assume that the claim is true for $3,\ldots,d-1$ and all $ k\ge 1$. If $k=1$ the $P$ is a simplex and the result immediately follows. So assume $k\ge 2$.  Furthermore, if $k\le d$ there is nothing to prove. So assume $d\ge k+1$. 

Observe that if $P$ is pyramid, we are done. Since in this case, the basis would have $d-1+k$ vertices and at most $d-1-1$ nonsimple vertices, and would therefore be a $(d-1-k)$-fold pyramid. Hence $P$ would be $(d-k)$-fold pyramid. \cref{thm:pyramidRec} then ensures that the maximum degree  ${\Delta}$ in the polytope graph is at most $d+k-2$, otherwise $P$ would be a pyramid again. Equally we are done if $P$ is a simple $d$-polytope, since in the case of $d\ge k+1$, $P$ would have at most $2d-1$ vertices and be a simplex.  

  
Suppose by way of contradiction that the $d$-polytope $P$ is nonsimple and nonpyramidal. Let $u$ be a vertex of degree  ${\Delta}$. Then $d+1\le {\Delta}\le d+k-2$.  The vertex figure $P/u$ of $P$   at $u$ has exactly ${\Delta}\le d-1+k-1$ vertices. Moreover,  $P/u$  has at most $d-1-1$ nonsimple vertices: for a vertex to be nonsimple in $P/u$ must arise from an edge between $u$ and another nonsimple vertex. Therefore, by the induction hypothesis $P/u$ is a $(d-k)$-fold pyramid, and consequently, the facet $F_u$ associated with $u$ in the dual $P^*$ of $P$, which is a dual of $P/u$ \cite[Thm.~11.5]{Bro83}, is also a $(d-k)$-fold pyramid. 

In the facet $F_u$ there is exactly one ridge which does not contain an apex $v^*$ of $F_u$; dually there is a facet $F$ in $P$ containing $u$, the one associated with $v^*$,  from which $u$ sends exactly one edge out. Since $P$ is nonpyramidal there are at least two vertices outside $F$, one of which must be adjacent to $u$ and nonsimple by \cref{rmk:simpleVertexOutside}. Therefore, the facet $F$ has at most $d-1+k-1$ vertices, of which at most $d-1-1$ are nonsimple vertices. Thus, $F$ is a $(d-k)$-fold pyramid by the induction hypothesis. An apex of $F$ has degree exactly ${\Delta}-1$ in $F$ and can be assumed to be $u$ by \cref{thm:pyramidRec}. The other facet $F'$ containing the basis $R$ of $F$ has at most $d-1+k-1$ vertices, and it is therefore a $(d-k)$-fold pyramid by the induction hypothesis, since the apex $u$ of $F$, which is nonsimple, is not in $F'$. As a result, $F'$ also has exactly ${\Delta}$ vertices and the vertex $u'$ in $F'\setminus F$ has degree ${\Delta}-1$ in $F'$ and by \cref{thm:pyramidRec} is an apex in $F'$.  Observe that there must be a vertex, say $z$, outside $F\cup F'$, otherwise $P$ would be pyramidal. Hence removing the nonsimple vertices in $F\cup F'$, at most $d-1$ vertices altogether, would disconnect $z$ from the simple vertices in $F\cup F'$, a contradiction. Notice that $R$ contains at most $d-3$ nonsimple vertices, and thus, at least two simple vertices.  This contradiction completes the proof of the proposition. 
    \end{proof}
  
The following proposition next deals with the case $d=k$. But first we need \cref{lem:pyrOrSimpleVert}.  

\begin{lemma}\label{lem:pyrOrSimpleVert}
Let $P$ be a $d$-polytope. If $P$ contains at most $d-2$ nonsimple vertices, then either $P$ is a pyramid or each nonsimple vertex is not adjacent to at least two simple vertices. 
\end{lemma}

\begin{proof}
Assume that $P$ is not a pyramid, and suppose by contradiction that a nonsimple vertex $u$ is not adjacent to at most one simple vertex $v$.

Removing all the nonsimple vertices plus possibly $v$ cannot disconnect the graph, according to Balinski's theorem. Therefore, the set $S$ of simple vertices which are neighbours of $u$ is connected. Let $x$ be a simple vertex in $S$. Then, there is a facet $F_x$ containing $x$ but not $u$. This facet also contains all neighbours of $x$ other than $u$, and in turn their neighbours if they are simple, and so on. Therefore $F_x$ contains all the  vertices in $S$. Since $P$ is not pyramidal, there must be a vertex $y\ne u$ that is not in this facet, meaning that $y$ has at most one simple neighbour. But then $y$ would have degree at most $d-2$ in the polytope, a contradiction 
\end{proof}
      
\begin{proposition}
\label{prop:d+d} Let $P$ be a $d$-polytope with $2d$ vertices and at most $d-2$ nonsimple vertices. Then $P$ is either a simplicial $d$-prism or a pyramid.

Furthermore, this is best possible as there are $4$-polytopes with 8 vertices and 3 nonsimple vertices which are neither simplicial $4$-prisms nor pyramids (cf.~Polytopes $Q_4^1$ and $Q_4^2$).
\end{proposition}

\begin{proof} The proof idea of \cref{prop:d+k} also proves this proposition. Suppose that $P$ is nonpyramidal. If $P$ is a simple polytope, then it is a simplicial $d$-prism (cf.~\cref{prop:simplePolytopes}). So assume that $P$ is a nonsimple polytope. Then, from \cref{lem:pyrOrSimpleVert}, it ensues that the maximum degree $\Delta$ of a vertex in $P$ is $2d-3$. 

As in the proof of \cref{prop:d+k}, we proceed by induction on $d$.  In the  basis case $d=3$ we have that $P$ is either a simplicial 3-prism or a pyramid over a quadrilateral, as desired. See \cite[Fig.~2]{BriDun73}.   Now assume that the claim is true for $3,\ldots,d-1$. 

Let $u$ be a vertex of degree  ${\Delta}$. Then $d+1\le {\Delta}\le 2d-3=d-1+d-2$.  The vertex figure $P/u$ of $P$ at $u$ has exactly ${\Delta}\le d-1+d-2$ vertices. Moreover,  $P/u$  has at most $d-2-1$ nonsimple vertices: for a vertex to be nonsimple in $P/u$ must arise from an edge between $u$ and another nonsimple vertex. Therefore, by \cref{prop:d+k}, $P/u$ is a pyramid, and consequently, the facet $F_u$ associated with $u$ in the dual $P^*$ of $P$ is also a pyramid. 

Since in the facet $F_u$ there is exactly one ridge which does not contain the apex $v^*$ of $F_u$, there is  dually a facet $F$ in $P$ containing $u$, the one associated with $v^*$,  from which $u$ sends exactly one edge out.

 Since $P$ is nonpyramidal there are at least two vertices outside $F$, one of which must be adjacent to $u$ and nonsimple by \cref{rmk:simpleVertexOutside}.   
 
 Therefore, the facet $F$ has at most $2(d-1)$ vertices, of which at most $d-2-1$ are nonsimple vertices. Thus, $F$ is either a pyramid or a simplicial $(d-1)$-prism by the induction hypothesis. Next we rule these two cases out.
 
In the case of $F$ being a pyramid, an apex of it has degree exactly ${\Delta}-1$ in $F$ and can be assumed to be $u$ by \cref{thm:pyramidRec}. We now establish that the other facet $F'$ containing the basis $R$ of $F$ is also a pyramid. Indeed, if  $F'$ has less than $2d-2$ vertices, \cref{prop:d+k} gives that $F'$ is pyramidal. If $F'$ has $2(d-1)$ vertices, the induction hypothesis gives that it is either a simplicial $(d-1)$-prism or a pyramid. If $F'$ were a simplicial $(d-1)$-prism then $R$ is either a simplicial $(d-2)$-prism or a $(d-2)$-simplex. In the case of $R$ being a  $(d-2)$-simplex, $\Delta=d$, a contradiction. In the case of $R$ being a simplicial $(d-2)$-prism, $\Delta=2d-3$. Let $z$ be the unique vertex of $P$ outside $F\cup F'$. But then removing  $u$, the two vertices in $F'\setminus R$,  and the nonsimple vertices in $R$, at most $d-1$ vertices altogether, would disconnect $z$ from the simple vertices in $R$, a contradiction. So $F'$ is a pyramid as claimed, with apex say $u'\in F'\setminus F$ and basis $R$. As before, there must be a  vertex outside $F\cup F'$, denote it $z$ again.  Analogously, by removing  $u$, $u'$  and the nonsimple vertices in $R$, at most $d-2$ vertices altogether, would disconnect $z$ from the simple vertices in $R$, a contradiction. So the case of $F$ being a pyramid is ruled out.
   
In the case of $F$ being a simplicial $(d-1)$-prism, $u$ would have degree $d-1$ in $F$, implying that ${\Delta}=d$, a contradiction. So this case is also ruled out, and thereby completing the proof of the proposition.\end{proof} 

We are now in a position to prove our structural and reconstruction results for polytopes with small number of vertices.

\begin{theorem}\label{thm:dPlus3VertRec}	Let $P$ be a $d$-polytope with $d+3$ vertices and  at most $d-1$ nonsimple vertices. Then the polytope is either a  $(d-3)$-fold  pyramid over  a simplicial $3$-prism,  a $(d-3)$-fold  pyramid over  $Q^1_3$,  or a $(d-2)$-fold pyramid over a pentagon. As a consequence, the graph of $P$ determines its entire combinatorial structure. 

These results are best possible in the sense that there are nonpyramidal  $d$-polytopes with $d+3$ vertices and exactly $d$ nonsimple vertices which are not reconstructible from their graphs (cf.~\cref{tab:4Polytopes7Vertices4Nonsimple}).

\end{theorem}

\begin{proof}Let $P$ be a $d$-polytope with $d+3$ vertices and  at most $d-1$ nonsimple vertices. If  $d\le 3$, the polytope is reconstructible from its graph, and so is the case if $P$ is a simple polytope. So suppose $P$ is not a simple polytope. \cref{prop:d+k} gives that $P$ is a $(d-3)$-fold pyramid. So the reconstruction results now follows from induction on $d$, with the basis case being $d=3$. Hence the polytope is reconstructible by \cref{thm:pyramidRec}. 

Once we have that the polytope is a $(d-3)$-fold pyramid, to obtain all such polytopes we only need to look for 3-polytopes with six vertices and at most two nonsimple vertices, which can be found in \cite[Fig.~3]{BriDun73}.

For examples of nonpyramidal $d$-polytopes with $d+3$ vertices and $d$ nonsimple consider, see \cref{tab:4Polytopes7Vertices4Nonsimple}.
\end{proof}

The next proposition gives structural results for polytopes with $d+3$ vertices and exactly $d$ or $d+1$ nonsimple vertices. But we first need a lemma on polytopes with $d+2$ vertices.

\begin{lemma}\label{lem:dPlus2VertRec} A $d$-polytope with $d+2$ vertices has either 
\begin{enumerate}[(i)]
\item exactly $d-2$ nonsimple vertices in which case it is a $(d-2)$-fold pyramid over a quadrilateral; or
\item at least $d$ nonsimple vertices.
\end{enumerate}

In the former case the polytope is reconstructible from its graph, while in the latter case there are pairs of inequivalent $d$-polytopes with the same graph.
\end{lemma}

\begin{proposition}
\label{prop:d+3-d-d+1}  Let $P$ be a $d$-polytope with $d+3$ vertices. Then the following hold.
\begin{enumerate}[(i)]
\item If $P$ has exactly $d$ nonsimple vertices, then the graph of $P$ is the join of the complete graph on $d-2$ vertices and the graph in \cref{fig:dPlus3} (b). 

\item If $P$ has exactly $d+1$ nonsimple vertices, then $P$ is either a $(d-3)$-fold pyramid over the polytope in \cref{fig:dPlus3} (a) or the graph of $P$ is the join of the complete graph on $d-2$ vertices and the graph in \cref{fig:dPlus3} (c).
\end{enumerate}

\end{proposition}

\begin{proof}  If the polytope is pyramidal, then the statements would follow from looking at the basis. So  assume $P$ is nonpyramidal.  We also suppose that $P$ is not a $(d-3)$-fold pyramid over the polytope in \cref{fig:dPlus3} (a). We first establish that the following claim.

{\bf Claim 1.} For $d\ge4$ a $d$-polytope with $d+3$ vertices and either $d$ or $d+1$ nonsimple vertices has a vertex $u$ adjacent to every other vertex.

\begin{proof} The statement is true for $d=4$ as shown by the catalogue of all 4-polytopes with seven vertices presented  in \cite{FukMiyMor13a}  or in \cite[Fig.~5]{Gru70}. So assume $d\ge 5$ and by contradiction that every nonsimple vertex has degree $d+1$.

 Consider a facet $F$ containing a simple vertex $x$  but not some other simple vertex $y$. Then $F$ is either a simplex or has $d+1$ vertices.

In the case of $F$ being a simplex, the three vertices outside $F$ send at least $3d-5$ edges into $F$, while the vertices of $F$ send exactly $1+2(d-1)$ edges outside $F$. So if $d-4\ge1$ there must be a vertex in $F$ with degree $d+2$. 

In the case of $F$ having $d+1$ vertices. We distinguish two cases based on \cref{lem:dPlus2VertRec}. If $F$ has $d-1$ nonsimple vertices, the two vertices outside $F$ send at least $2d-2$ edges into $F$ and the vertices of $F$ send at least $2+d-1$ edges outside $F$. So if $d-3\ge1$ there is a vertex in $F$ with degree $d+2$.  If instead $F$ is a $(d-3)$-fold pyramid over a quadrilateral, let $u$ be an apex and $R_u$ the ridge not containing $u$. Then, by \cref{rmk:simpleVertexOutside} , the vertex $u$ must be adjacent to 
the nonsimple vertex outside $F$, denote it by $v$. Since the vertices outside $F$ must be adjacent and since  the other facet containing $R_u$ must be a $(d-3)$-fold pyramid over a quadrilateral with $v$ as an apex, we get that $v$ has degree $d+2$. 
\end{proof}
	  
Our proof continues by proving the proposition statements by induction on $d$, with the basis case $d=3$. following from \cite[Fig.~3]{BriDun73}. Actually, the proposition immediately follows from the statement below.
\begin{quote}
(*) In the polytope $P$ there are exactly two vertices with degree $d+1$, and they are adjacent. 
\end{quote}

Note that, thanks to Balinski's theorem, this fully determines the graphs of $d$-polytopes with $d+3$ vertices  and either $d$ or $d+1$ nonsimple vertices. So this is our induction hypothesis. Let $u$ be a vertex of degree $d+2$ whose existence is guaranteed by Claim 1.

The vertex figure $P/u$ of $P$ at $u$ is a $(d-1)$-polytope with $d-1+3$ vertices. If $P/u$ is a pyramid, then  the basis of the pyramid induces a facet $F$ which contains $u$ and has $d-1+3$ vertices. So $P$ is a pyramid, whose basis $F$, by the induction hypothesis, satisfies the condition. Therefore, $P$ satisfies the condition.

If $P/u$ is nonpyramidal, then it has either $d-1$ or $d$ nonsimple vertices and satisfies the induction hypothesis. Every edge in the vertex figure corresponds to a 2-face containing $u$, which is a triangle. Hence, if two vertices in the vertex figure are adjacent, the two corresponding vertices of $P$, which are adjacent to $u$,  are also adjacent. So the graph of $P/u$ is a subgraph of $G(P)$. As a consequence, the $d-3$ vertices of $P/u$ with degree $d+1$ in $P/u$ give rise to $d-3$ vertices of degree $d+2$ in $P$. So $P$ has at least $d-2$ vertices of degree $d+2$, including the vertex $u$. If $P$ has exactly $d$ nonsimple vertices, then $P/u$ has exactly two vertices of degree $d$, which are adjacent. These two vertices in turn  give rise to the adjacent vertices of $P$ with degree $d+1$, since $P$ has three simple vertices. In the case of $P$ having exactly $d+1$ nonsimple vertices, the vertices of $P/u$ with degree $d+1$ in $P/u$ give rise to vertices of degree $d+2$ in $P$ and the two adjacent vertices of $P/u$ with degree $d$ in $P/u$ give rise to two adjacent vertices in $P$ with degree $d+1$. So if   $G(P/u)$ has exactly $d$ nonsimple vertices, by the induction hypothesis, removing the $d-1$ nonsimple vertices with degree $d+2$ from $G(P)$ leaves a subgraph which includes the graph in \cref{fig:dPlus3}(c), and we are at home. If instead $G(P/u)$ has exactly $d-1$ nonsimple vertices, removing the $d-2$ nonsimple vertices with degree $d+2$ from $G(P)$ leaves a subgraph which includes the graph in \cref{fig:dPlus3}(b),  contradicting the existence of exactly two simple vertices in $P$. Hence the graph of $P/u$ is the subgraph of $G(P)$ obtained by removing $u$. This proves the result.
\end{proof}

We next deal with $d$-polytopes with at least $d+4$ vertices.
	
\begin{theorem}\label{thm:dPlus4VertRec} Let $P$ be a $d$-polytope with $d+4$ vertices. If  $P$ has at most $d-2$ nonsimple vertices, then the graph of $P$ determines its entire combinatorial structure. Furthermore, in the case of $P$ having at most $d-1$ nonsimple vertices, for $d=3$ there are exactly five such polytopes, and for every $d\ge4$ there are exactly nine such polytopes in which case $P$ is either a simplicial $4$-prism or a $(d-4)$-fold pyramid, and it is reconstructible from its 2-skeleton. 

These results are best possible in the sense that there are  nonpyramidal $4$-polytopes with eight vertices and three nonsimple vertices which are not reconstructible from their graphs  (cf.~Polytopes $Q^1_4$ and $Q^2_4$).
\end{theorem}
	
\begin{proof} If $d\le 3$ the polytope is reconstructible from its graph. In the case of $P$ being a simple polytope (cf.~\cref{prop:simplePolytopes}), $P$ is reconstructible from its graph.  So suppose $P$ is not  a simple polytope. The statement that $P$ is a $(d-4)$-fold pyramid follows at once from \cref{prop:d+k}.  By \cref{thm:pyramidRec} the reconstruction statement now reduces to proving that a 4-polytope with eight vertices and at most two nonsimple vertices is reconstructible from its graph, which follows from \cite[Thm.~4.6]{NevPinUgo17}. 

For $d=3$ the 3-polytopes with at most two nonsimple vertices can be found in \cite[Fig.~4]{BriDun73}. For $d\ge 4$, once we have that the polytope is a $(d-4)$-fold pyramid, we only need to look for 4-polytopes with eight vertices and at most three nonsimple vertices, which can be found in the catalogues \cite{FukMiyMor13a}.

Finally, note that reconstructing from the 2-skeleton reduces to showing that a 4-polytope with eight vertices and at most three nonsimple vertices is reconstructible from its 2-skeleton, which follows from \cite[Thm.~12.3.1]{Gru03}  
\end{proof}

	
\begin{theorem}\label{thm:d+kVert}
Let $k\ge 5$ and let $P$ be a $d$-polytope with $d+k$ vertices and  at most $d-k+3$ nonsimple vertices. Then the graph of $P$ determines the entire combinatorial structure of $P$. The particular case of $k=5$ is best possible as shown by the pair of polytopes $Q^1_5$  and $Q^2_5$.
\end{theorem}
	
\begin{proof} If $d< k$ then the polytope is reconstructible from its graph by \cite[Thm.~4.6]{NevPinUgo17}. If $P$ is a simple polytope, it is reconstructible from its graph.  In the case of $d=k$ and $P$ being nonsimple, since $d-k+3\le d-2$, \cref{prop:d+d} gives that $P$ is a pyramid, and the reconstruction follows from  \cref{thm:pyramidRec}  and \cite[Thm.~4.6]{NevPinUgo17} since the basis of the pyramid would have at most 2 nonsimple vertices. So assume $d\ge k+1$. From \cref{prop:d+k} it ensues that $P$ is a $(d-k)$-fold pyramid .  By \cref{thm:pyramidRec} the reconstruction statement now reduces to proving that a $k$-polytope $Q$ with $2k$ vertices and at most 3 nonsimple vertices is reconstructible from its graph. By \cref{prop:d+d} $Q$ is either a simplicial $k$-prism, in which case is reconstructible from its graph, or a pyramid over a $(k-1)$-polytope with 9 vertices and at most 2 nonsimple vertices, in which case $Q$ is also reconstructible since the reconstruction of the basis follows from \cite[Thm.~4.6]{NevPinUgo17} .  
\end{proof}

\section{Acknowledgments}
Guillermo Pineda would like to thank Michael Joswig for the hospitality at the Technical University of Berlin and for suggesting looking at the reconstruction problem for polytopes with small number of vertices.

\begin{thebibliography}{10}

\bibitem{BliMan87}
R.~Blind and P.~Mani-Levitska, \emph{Puzzles and polytope isomorphisms},
  Aequationes Math. \textbf{34} (1987), no.~2-3, 287--297. \MR{921106
  (89b:52008)}

\bibitem{BriDun73}
D.~Britton and J.~D. Dunitz, \emph{A complete catalogue of polyhedra with eight
  or fewer vertices}, Acta Crystallographica Section A \textbf{29} (1973),
  no.~4, 362--371.

\bibitem{Bro83}
A.~Br{\o}ndsted, \emph{An introduction to convex polytopes}, Graduate Texts in
  Mathematics, vol.~90, Springer-Verlag, New York, 1983. \MR{683612
  (84d:52009)}

\bibitem{Doo17}
J.~Doolittle, \emph{Reconstructing nearly simple polytopes from their graph},
  arXiv 1701.08334, 2017.

\bibitem{FukMiyMor13a}
K.~Fukuda, H.~Miyata, and S.~Moriyama, \emph{Classification of oriented
  matroids},
  \url{http://www-imai.is.s.u-tokyo.ac.jp/~hmiyata/oriented_matroids/}, 2013.

\bibitem{GooORo04}
J.~E. Goodman and J.~O'Rourke (eds.), \emph{Handbook of discrete and
  computational geometry}, 2nd ed., Discrete Mathematics and its Applications
  (Boca Raton), Chapman \& Hall/CRC, Boca Raton, FL, 2004. \MR{2082993
  (2005j:52001)}

\bibitem{GawJos00}
E.~Gawrilow and M.~Joswig, \emph{polymake: a framework for analyzing convex
  polytopes}, Polytopes---combinatorics and computation ({O}berwolfach, 1997),
  DMV Sem., vol.~29, Birkh\"auser, Basel, 2000, pp.~43--73. \MR{1785292}

\bibitem{Gru70}
B.~Gr{\"u}nbaum, \emph{Polytopes, graphs, and complexes}, Bull. Amer. Math.
  Soc. \textbf{76} (1970), 1131--1201. \MR{0266050 (42 \#959)}

\bibitem{Gru03}
B.~Gr{{\"u}}nbaum, \emph{Convex polytopes}, 2nd ed., Graduate Texts in
  Mathematics, vol. 221, Springer-Verlag, New York, 2003, Prepared and with a
  preface by V. Kaibel, V. Klee and G. M. Ziegler. \MR{1976856 (2004b:52001)}

\bibitem{Jos00}
M.~Joswig, \emph{Reconstructing a non-simple polytope from its graph},
  Polytopes---combinatorics and computation ({O}berwolfach, 1997), DMV Sem.,
  vol.~29, Birkh\"auser, Basel, 2000, pp.~167--176. \MR{1785298 (2001f:52023)}

\bibitem{Kal88}
G.~Kalai, \emph{A simple way to tell a simple polytope from its graph}, Journal
  of Combinatorial Theory, Series A \textbf{49} (1988), no.~2, 381 -- 383.

\bibitem{Kle74}
Victor Klee, \emph{Polytope pairs and their relationship to linear
  programming}, Acta Math. \textbf{133} (1974), 1--25. \MR{0345000 (49 \#9739)}

\bibitem{NevPinUgo17}
E.~Nevo, G.~Pineda-Villavicencio, J.~Ugon, and D.~Yost, \emph{On the
  reconstruction of polytopes}, arXiv 1702.08739, 2017.

\bibitem{PinUgoYos16a}
G.~Pineda-Villavicencio, J.~Ugon, and D.~Yost, \emph{The excess degree of a
  polytope}, 2017.

\bibitem{PrzYos16}
K.~Przes{\l}awski and D.~Yost, \emph{More indecomposable polytopes}, Extracta
  Mathematicae \textbf{31} (2016), 169--188.

\bibitem{Zie95}
G.~M. Ziegler, \emph{Lectures on polytopes}, Graduate Texts in Mathematics,
  vol. 152, Springer-Verlag, New York, 1995. \MR{1311028 (96a:52011)}

\end{thebibliography}

\end{document}  

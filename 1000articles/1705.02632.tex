\documentclass[reqno]{amsart}
\usepackage{amsmath}
\usepackage{amscd, amssymb, amsfonts, amsbsy}
\usepackage{enumerate}
\usepackage{hyperref}
\usepackage{verbatim}

\numberwithin{equation}{section}

\theoremstyle{plain}
\newtheorem{theorem}{Theorem}[section]
\newtheorem{lemma}[theorem]{Lemma}

\theoremstyle{definition}
\newtheorem{assumption}[theorem]{Assumption}

\newtheorem*{acknowledgment}{Acknowledgment}

\theoremstyle{remark}
\newtheorem{remark}[theorem]{Remark}

\makeatletter

\makeatother

\begin{document}
\title[$L_{p,q}$-estimates for elliptic and parabolic systems]{Weighted $L_{p,q}$-estimates for higher order elliptic and parabolic systems  with  BMO coefficients on Reifenberg flat domains}

\author[J. Choi]{Jongkeun Choi}
\address[J. Choi]{Department of Mathematics, Korea University, 145 Anam-ro, Seongbuk-gu, Seoul, 02841, Republic of Korea}
\email{jongkeun\_choi@korea.ac.kr}

\author[D. Kim]{Doyoon Kim}
\address[D. Kim]{Department of Mathematics, Korea University, 145 Anam-ro, Seongbuk-gu, Seoul, 02841, Republic of Korea}
\email{doyoon\_kim@korea.ac.kr}
\thanks{D. Kim was supported by Basic Science Research Program through the National Research Foundation of Korea (NRF) funded by the Ministry of Education (2014R1A1A2054865).}

\subjclass[2010]{35B45, 35J48, 35K41, 35R05}
\keywords{higher order elliptic and parabolic systems; weighted $L_{p,q}$-estimates;  BMO coefficients}

\begin{abstract}
We prove weighted $L_{p,q}$-estimates for divergence type higher order elliptic and parabolic systems with irregular coefficients on Reifenberg flat domains.
In particular, in the parabolic case the coefficients do not have any regularity assumptions in the time variable.
As functions of the spatial variables, the leading coefficients are permitted to have small mean oscillations.
The weights are in the class of Muckenhoupt weights $A_p$.
We also prove the solvability in weighted Sobolev spaces for the systems in the whole space, on a half space, and on  bounded Reifenberg flat domains.  
\end{abstract}

\maketitle

\section{Introduction}
We study weighted $L_{p,q}$-estimates and the solvability of divergence type higher order parabolic systems
\begin{equation}		\label{problem}
{\boldsymbol{u}}_t+(-1)^m{\mathcal{L}}{\boldsymbol{u}}=\sum_{|\alpha|\le m} D^\alpha {\boldsymbol{f}}_\alpha 
\end{equation}
in $\Omega_T=(-\infty,T)\times \Omega$, where $T\in (-\infty,\infty]$ and $\Omega$ is either a bounded or unbounded domain in ${\mathbb{R}}^d$.
The domain $\Omega$ can be ${\mathbb{R}}^d$ as well.
The differential operator ${\mathcal{L}}$ is in divergence form of order $2m$ acting on column vector valued functions ${\boldsymbol{u}}=(u^1,\ldots,u^n)^{\operatorname{tr}}$ defined on $\Omega_T$ as follows:
\[
{\mathcal{L}}{\boldsymbol{u}}=\sum_{|\alpha|
\le m,\, |\beta| \le m}D^{\alpha}(A^{\alpha\beta}D^\beta{\boldsymbol{u}}).
\] 
Here, $\alpha=(\alpha_1,\ldots,\alpha_d)$, $\beta=(\beta_1,\ldots,\beta_d)$ are multi-indices,  and we write $D^\alpha {\boldsymbol{u}}=D_x^\alpha {\boldsymbol{u}}$ for the spatial derivative of ${\boldsymbol{u}}$;
\[
D^\alpha{\boldsymbol{u}}=D^{\alpha_1}_1\ldots D^{\alpha_d}_d{\boldsymbol{u}}.
\]
All the coefficients $A^{\alpha\beta}=A^{\alpha\beta}(t,x)$ are  $n\times n$ complex-valued matrices whose entries $A^{\alpha\beta}_{ij}(t,x)$ are bounded measurable functions defined on the entire space ${\mathbb{R}}^{d+1}$.
We also consider elliptic systems with operators as ${\mathcal{L}}$ in \eqref{problem}.
In this case all the coefficients and functions involved are independent of the time variable.

Throughout this paper,  the leading coefficients $A^{\alpha\beta}$, $|\alpha|=|\beta|=m$, satisfy the Legendre-Hadamard ellipticity condition, which is more general than the uniform ellipticity condition.
We assume that the coefficients of the parabolic systems are merely measurable in the time variable and have small mean oscillations with respect to the spatial variables $({\mathrm{BMO}}_x)$.
As mentioned in \cite{MR1828321},  such type of coefficients with no regularity assumption in the time variable are necessary in the study of filtering theory.

In this paper,  we prove a priori weighted (mixed) norm estimates for divergence type higher order parabolic systems \eqref{problem} with ${\mathrm{BMO}}_x$ coefficients in  $\Omega_T$, where $\Omega$ is a Reifenberg flat domain (possibly unbounded).
More precisely, we first establish a priori weighted $L_p$-estimates with a Muckenhoupt weight $w\in A_p$:
$$
\sum_{|\alpha|\le m}\|D^\alpha {\boldsymbol{u}}\|_{L_{p,w}(\Omega_T)}\le N\sum_{|\alpha|\le m}\|{\boldsymbol{f}}_\alpha\|_{L_{p,w}(\Omega_T)}, \quad p\in (1,\infty).
$$
The weight $w$ is  defined on a space of homogeneous type ${\mathcal{X}}$ in ${\mathbb{R}}^{d+1}$ such that $\Omega_T \subset {\mathcal{X}}$. 
For  more precise definition of weights defined on spaces of homogeneous type, see Section \ref{161006@sec1}.
With regard to $L_p$-estimates with Muckenhoupt weights defined in the whole space, see \cite{MR3225808, MR3467697, arXiv:1410.6394v2, MR2286441, MR1980981}.
Then, we obtain a priori weighted $L_{p,q}$-estimates with a mixed weight:
\begin{equation}		\label{161002@eq1}
\sum_{|\alpha|\le m}\|D^\alpha {\boldsymbol{u}}\|_{L_{p,q,w}(\Omega_T)}\le N\sum_{|\alpha|\le m}\|{\boldsymbol{f}}_\alpha\|_{L_{p,q,w}(\Omega_T)}, \quad p,\,q\in (1,\infty).
\end{equation}
The $L_{p,q}$-norm with the mixed weight $w=w_1w_2$ is defined as 
\begin{equation}		\label{161007@eq1}
\|f\|_{L_{p,q,w}(\Omega_T)}=\left(\int_{{\mathcal{X}}_2}\left(\int_{{\mathcal{X}}_1}|f|^pI_{\Omega_T}w_1 \right)^{q/p}w_2 \right)^{1/q},
\end{equation}
where  $w_1$ is an $A_p$ weight  on a space of homogeneous type ${\mathcal{X}}_1\subset {\mathbb{R}}^{d_1}$ and $w_2$ is an $A_q$ weight on a space of homogeneous type ${\mathcal{X}}_2\subset {\mathbb{R}}\times {\mathbb{R}}^{d_2}$, $d_1+d_2=d$. 
Here, $\Omega_T\subset {\mathcal{X}}_1\times {\mathcal{X}}_2$ and  the mixed weight $w$ is defined on ${\mathcal{X}}_1\times {\mathcal{X}}_2 \subset {\mathbb{R}}^{d+1}$.
For more discussions of mixed weights, see Section \ref{161006@sec1} in this paper and \cite{arXiv:1603.07844v1}.
We also discuss the solvability in weighted Sobolev spaces for higher order parabolic systems \eqref{problem} with ${\mathrm{BMO}}_x$ coefficients when the domain is either the whole space, a half space, or  bounded Reifenberg flat domains.
The corresponding results for divergence type higher order elliptic systems with ${\mathrm{BMO}}$ (bounded mean oscillations) coefficients are also addressed.

In the study of divergence type higher order parabolic systems,
to the best of the authors' knowledge, our results  are completely new  in the sense that Sobolev spaces with $A_p$ weights are considered, the coefficients are merely measurable in time, the domains are Reifenberg flat, and the solvability is explicitly addressed.
Even in the case of  $L_p$-estimates (with unweighted and unmixed-norm) for higher order systems on Reifenberg flat domains, there are no such results in the literature. 
The only existing literature is a recent paper \cite{MR2771670}, where the authors prove $L_p$-estimates for higher order systems with ${\mathrm{BMO}}_x$ coefficients  in the whole space, on a half space, and on bounded Lipschitz domains.
Restricted to fourth-order parabolic systems with ${\mathrm{BMO}}_x$ coefficients, unweighted $L_p$-estimates on a bounded Reifenberg flat domain are obtained in \cite{MR2460025}.
With regard to the study of unweighted $L_p$-estimates for second order parabolic systems with ${\mathrm{BMO}}_x$ and ${\mathrm{VMO}}$ (vanishing mean oscillations) coefficients, see, for instance, \cite{MR2328932, MR2650802} and references therein.
Note that  the domains in this paper are not necessarily spaces of homogeneous type.
In recent paper \cite{arXiv:1603.07844v1}, the authors proved weighted $L_{p,q}$-estimates for non-divergence type higher order systems with ${\mathrm{BMO}}_x$ coefficients in the whole space and on half spaces.
As is well known, the whole space and half spaces are homogeneous type.
See also \cite{MR2286441} for weighted $L_{p,q}$-estimates of non-divergence type higher order systems with time independent ${\mathrm{VMO}}_x$ coefficients on the whole space.
The authors in \cite{arXiv:1603.07844v1} also obtained weighted $L_{p,q}$-estimates for divergence type higher order systems with partially ${\mathrm{BMO}}$ coefficients (measurable in one spatial direction and having small BMO semi norms in the other variables including the time variable) on bounded Reifenberg flat domains which are spaces of homogeneous type.
See \cite[Section 7]{arXiv:1603.07844v1} for more details.

Our argument on establishing weighted $L_p$-estimates with unmixed-norm is based on  techniques, we call it \emph{level set argument}, used in \cite{MR2835999}, where the authors proved unweighted $L_p$-estimates for divergence type higher order systems with partially ${\mathrm{BMO}}$ coefficients on a Reifenberg flat domain.
In \cite{MR2835999}, the key for obtaining $L_p$-estimates lies in mean oscillation estimates, $L_\infty$-estimates, and level set estimates of solutions combined with  the measure theory on the ``crawling of ink spots" which can be found in \cite{MR563790}.
In this paper, to establish weighted $L_{p}$-estimates for arbitrary $p\in (1,\infty)$, we refine the measure theory  to  spaces of homogeneous type with weighted measures.
Moreover, we generalize the level set estimates from \cite{MR2835999}, roughly speaking, to control the weighted measure of  level sets of $|D^\alpha {\boldsymbol{u}}|$ by those of ${\mathcal{M}}(|{\boldsymbol{f}}_\alpha|^q)^{1/q}$ with not only $q=2$ but also for any $q\in (1,\infty)$.
Here, ${\mathcal{M}}$ is the Hardy-Littlewood maximal function operator.
This type of level set estimate with $q=2$ was used, for instance, in \cite{MR3225808, MR3467697} to obtain weighted $L_p$-estimates for divergence type second order systems.
A noteworthy difference is that in this paper, $L_p$-estimates are established with $A_p$ weights, whereas in \cite{MR3225808, MR3467697}  $L_p$-estimates are obtained with $A_{p/2}$ weights, the collection of which is strictly smaller than $A_p$.

Using aforementioned weighted $L_p$-estimates (unmixed norm), we prove weighted $L_{p,q}$-estimates  \eqref{161002@eq1} (mixed norm) for the systems.
The key ingredient  is a refined version of Rubio de Francia extrapolation theorem used in \cite{arXiv:1603.07844v1}.
We remark that such a refinement of the well-known extrapolation theorem (see, for instance, \cite{MR2797562}) is necessary in our setting where the coefficients and the boundaries are very rough.
For more discussion, see the  paragraph above Theorem \ref{1017@@thm1} in this paper.
We also refer to \cite{MR2286441}, where the well-known version of the extrapolation theorem  was employed to obtain the $L_{p,q}$-estimates for non-divergence type parabolic systems with  time-independent ${\mathrm{VMO}}$ coefficients in the whole space.

With unweighted Sobolev spaces, the solvability of the systems with irregular coefficients can be obtained from a priori estimates and the method of continuity because the solvability of systems with simple coefficients  is known.
In \cite{MR1980981}, the authors also used the method of continuity for the solvability of higher order elliptic systems with VMO coefficients in weighted Sobolev spaces.
However, we are not able to find any literature explicitly dealing with solvability in Sobolev spaces with weighted and mixed-norm for higher order parabolic systems (with simple coefficients). 
Here, we present the solvability results by using  a priori weighted $L_{p,q}$-estimates and the solvability in unweighted Sobolev spaces. 
For more details, see the proofs of Theorems \ref{160621@thm1} and \ref{160621@thm2}.
From the results on parabolic systems, we immediately obtain the corresponding results for higher order elliptic systems in divergence form with ${\mathrm{BMO}}$ coefficients. 

Mixed-norm estimates for second order parabolic equations in divergence and non-divergence type with ${\mathrm{VMO}}_x$ coefficients were studied by Krylov.
In \cite{MR2352490}, he proved $L_{p,q}$-estimates with the norm \eqref{161007@eq1}, where  ${\mathcal{X}}_1={\mathbb{R}}^d$ and  ${\mathcal{X}}_2={\mathbb{R}}$, namely
$$
\|f\|_{L_{p,q}({\mathbb{R}}^{d+1})}=\left(\int_{\mathbb{R}}\left(\int_{{\mathbb{R}}^d}|f|^p\,dx\right)^{q/p}\,dt\right)^{1/q}. 
$$
We note that in the non-divergence case, the $L_{p,q}$-estimates  were proved only for $q\ge p$.
Recently, this result was generalized by Dong-Kim \cite{arXiv:1603.07844v1} to the mixed-norm estimates with arbitrary $p,\,q\in (1,\infty)$ for non-divergence type higher order systems with ${\mathrm{BMO}}_x$ coefficients.
For $L_{p,q}$-estimates of systems with ${\mathrm{VMO}}_x$ coefficients that are independent of $t$, we refer the reader to \cite{MR2286441}.
See also \cite{arXiv:1510.07643v2, arXiv:1410.6394v2,MR2982717} for $L_{p,q}$-regularity of the  evolution equation.
In the divergence case, Dong-Kim \cite{MR2764911} proved $L_{p,q}$-estimates, $p,\,q\in (1,\infty)$, for second order parabolic systems with  either ${\mathrm{BMO}}_x$ or partially ${\mathrm{BMO}}$ coefficients.
When parabolic systems are defined on a bounded domain, they consider either the Dirichlet boundary condition or the conormal derivative boundary condition. 
We point out that $L_{p,q}$-estimates for such boundary value problems play a significant role in the study  of Green functions.
In \cite{MR3261109}, the authors obtained  global Gaussian estimates of Green functions for second order parabolic systems satisfying a Robin-type boundary condition by using the $L_{p,q}$-estimates for the systems; see the proof of \cite[Theorem 3.3]{MR3261109} for more details.
Based on our results regarding $L_{p,q}$-estimates for higher order systems, the corresponding Green functions are discussed elsewhere in the future.

The remainder of this paper is organized as follows.
Section 2 contains  some notation and definitions. 
In Section 3, we state our main theorems including the results of higher order elliptic systems.
In Section 4, we present  some auxiliary results, while in Section 5, we establish interior and boundary estimates for derivatives of solutions.
Section 6 is devoted to the level set argument, and based on the results in Sections 4-6, we provide the proofs of the main theorems in Section 7.
In the appendix, we provide the proofs of some technical lemmas.

\section{Preliminaries}		\label{sec2}

\subsection{Basic notation}

We use $X=(t,x)$ to denote a point in ${\mathbb{R}}^{d+1}$; $x=(x_1,\ldots,x_d)$ is a point in ${\mathbb{R}}^d$.
We also write $Y=(s,y)$ and $X_0=(t_0,x_0)$, etc.
Let ${\mathbb{R}}^d_+=\{x\in {\mathbb{R}}^d:x_1>0\}$ and ${\mathbb{R}}^{d+1}_+={\mathbb{R}}\times {\mathbb{R}}^d_+$.
We  use $\Omega$ to denote an open set in ${\mathbb{R}}^d$ and $\Omega_T=(-\infty,T)\times \Omega$.
We use the following notion for cylinders in ${\mathbb{R}}^{d+1}$:
\begin{align*}
&Q_r(X)=(t-r^{2m},t)\times B_r(x), \\
&Q_r^+(X)=Q_r(X)\cap {\mathbb{R}}^{d+1}_+,\\
&{\mathcal{B}}_r(X)=(t-r^{2m},t+r^{2m})\times B_r(x),
\end{align*}
where $B_r(x)$ is the usual Euclidean ball of radius $r$ centered at $x\in {\mathbb{R}}^d$.
Here, if we define the parabolic distance between the points $X$ and $Y$ in ${\mathbb{R}}^{d+1}$ as 
\begin{equation}							\label{eq0215_01}
\rho(X,Y)=\max \big\{|x-y|, |t-s|^{\frac{1}{2m}}\big\},
\end{equation}
then 
$$
{\mathcal{B}}_r(X)=\{Y\in {\mathbb{R}}^{d+1}:\rho(X,Y)<r\},
$$
that is, ${\mathcal{B}}_r(X)$ is an open ball in ${\mathbb{R}}^{d+1}$ equipped with the parabolic distance $\rho$.
We abbreviate $Q_r=Q_r(0)$ and $B_r=B_r(0)$, etc.
For a function $f$ on $Q$, we use  $\left( f \right)_Q$ to denote the average of $f$ in $Q$, that is,
\[
(f)_Q=\frac{1}{|Q|}\int_Q f={\operatorname{\,\,\text{\bf--}\kern-.98em\DOTSI\intop\ilimits@\!\!}}_Q f.
\]

\subsection{Weights on a space of homogeneous type}		\label{161006@sec1}

Let ${\mathcal{X}}$ be a set. 
A nonnegative symmetric function $\rho$ on ${\mathcal{X}}\times {\mathcal{X}}$ is called a quasi-metric on ${\mathcal{X}}$ if there exists a positive constant $K_1$ such that
$$
\rho(x,x)=0\quad \text{and}\quad \rho(x,y)\le K_1\left(\rho(x,z)+\rho(z,y) \right)
$$
for any $x,y,z\in {\mathcal{X}}$.
We denote balls in ${\mathcal{X}}$ by
\begin{equation}
							\label{eq0624_01}
B_r^{\mathcal{X}}(x)=\{y\in {\mathcal{X}}:\rho(x,y)<r\}, \quad \forall x\in {\mathcal{X}}, \quad \forall r>0.
\end{equation}
We say that $({\mathcal{X}},\rho,\mu)$ is a space of homogeneous type if $\rho$ is a quasi-metric on ${\mathcal{X}}$, $\mu$ is a Borel measure defined on a $\sigma$-algebra on ${\mathcal{X}}$ which contains all the balls in ${\mathcal{X}}$, and the following doubling property holds: there exists a constant $K_2$ such that for any $x\in {\mathcal{X}}$ and $r>0$, 
$$
0<\mu(B_{2r}^{\mathcal{X}}(x))\le K_2\mu(B_r^{\mathcal{X}}(x))<\infty.
$$
Without loss of generality, we assume that balls $B_r^{\mathcal{X}}(x)$ are open in ${\mathcal{X}}$.

For any $p\in (1,\infty)$ and a space of homogeneous type $({\mathcal{X}},\rho,\mu)$, the space $A_p({\mathcal{X}})$ denotes the set of all nonnegative functions $w(x)$ on ${\mathcal{X}}$ such that 
$$
[w]_{A_p}:=\sup_{\substack{x\in {\mathcal{X}} \\ r>0}}\left({\operatorname{\,\,\text{\bf--}\kern-.98em\DOTSI\intop\ilimits@\!\!}}_{B_r^{\mathcal{X}}(x)}w\,d\mu \right)\left({\operatorname{\,\,\text{\bf--}\kern-.98em\DOTSI\intop\ilimits@\!\!}}_{B_r^{\mathcal{X}}(x)}w^{-\frac{1}{p-1}}\,d\mu\right)^{p-1}<\infty.
$$
One can easily check that  $[w]_{A_p} \ge 1$ for all $w \in A_p({\mathcal{X}})$ and $[w]_{A_p}=1$ when $w\equiv1$.
We denote
\[
w(E)=\int_E w \, d\mu.
\]

Throughout this paper, whenever ${\mathcal{X}}$ is said to be a space of homogeneous type in ${\mathbb{R}}^k$ for some positive integer $k$, we mean the triple $({\mathcal{X}}, \rho, \mu)$, where ${\mathcal{X}}$ is an open set in ${\mathbb{R}}^k$, the metric $\rho$ is the usual Euclidean distance, and $\mu$ is the Lebesgue measure in ${\mathbb{R}}^k$.
If ${\mathcal{X}}$ is assumed to be a space of homogeneous type in ${\mathbb{R}} \times {\mathbb{R}}^k$ (or ${\mathbb{R}}^{k+1}$), then ${\mathcal{X}}$ is an open set in ${\mathbb{R}}^{k+1}$, the metric $\rho$ is the parabolic distance defined in \eqref{eq0215_01}, and $\mu$ is the Lebesgue measure in ${\mathbb{R}}^{k+1}$.
Thus, for example, when we consider weights of the type $w(t,x) = w_1(x') w_2(t,x'')$ in the mixed norm case for parabolic equations/systems,  where $w_1$ is a weight on a space of homogeneous type ${\mathcal{X}}_1\subset {\mathbb{R}}^{d_1}$ and $w_2$ is a weight on a space of homogeneous type ${\mathcal{X}}_2\subset {\mathbb{R}} \times {\mathbb{R}}^{d_2}$, $d_1+ d_2 = d$,  ${\mathcal{X}}_1$  is equipped with the usual Euclidean distance and the $d_1$-dimensional Lebesgue measure, and ${\mathcal{X}}_2$  is equipped with  the parabolic distance $\rho$ and the $(d_2+1)$-dimensional Lebesgue measure.

Since we consider only the Lebesgue measures and the parabolic or Euclidean distances, the constant $K_1$ is always $1$ in our case. However, the doubling constant $K_2$ may vary depending on the choice of ${\mathcal{X}}$.
For example, the doubling constants of the whole spaces ${\mathbb{R}}^d$ and ${\mathbb{R}}^{d+1}$ are $2^d$ and $2^{d+2m}$, respectively.
When ${\mathcal{X}}\subset {\mathbb{R}}^{d+1}$ and there exists a constant $\varepsilon>0$ such that $|{\mathcal{B}}_r(X)\cap {\mathcal{X}} |\ge \varepsilon |{\mathcal{B}}_r(X)|$ for any $X\in {\mathcal{X}}$ and $r>0$, ${\mathcal{X}}$ is a space of homogeneous type with a doubling constant $K_2=K_2(d,m,\varepsilon)$.
If ${\mathcal{X}}={\mathbb{R}}\times \Omega$, where $\Omega$ is a bounded Reifenberg flat domain in ${\mathbb{R}}^d$, then the doubling constant of ${\mathcal{X}}$ is determined by $d$, $m$, $|\Omega|$, $R_0$, and $\gamma\in (0,1/4)$, where $R_0$ and $\gamma$ are constants in Assumption \ref{0923.ass1}; see \cite[Remark 7.3]{arXiv:1603.07844v1}.
Moreover, if ${\mathcal{X}}$ is assumed to be ${\mathcal{X}}_1\times {\mathcal{X}}_2$, where ${\mathcal{X}}_1$ and ${\mathcal{X}}_2$ are spaces of homogeneous type with doubling constants $K_2'$ and $K_2''$ in ${\mathbb{R}}^{d_1}$ and ${\mathbb{R}}\times {\mathbb{R}}^{d_2}$, $d_1+d_2=d$, respectively, 
then ${\mathcal{X}}$ is a space of homogeneous type in ${\mathbb{R}}^{d+1}$ with a doubling constant $K_2$, where $K_2$ is determined by $K_2'K_2''$; see Lemma \ref{160629@lem1}.

\subsection{Function spaces}

Let $p,\,q\in (1,\infty)$, $-\infty\le S<T\le \infty$, $\Omega$ be an open set in ${\mathbb{R}}^d$, and $(S,T)\times \Omega\subseteq {\mathcal{X}}_1\times {\mathcal{X}}_2$, where 
 ${\mathcal{X}}_1$ and ${\mathcal{X}}_2$ are  spaces of homogeneous type in  ${\mathbb{R}}^{d_1}$ and ${\mathbb{R}}\times {\mathbb{R}}^{d_2}$, $d_1+d_2=d$, respectively.
Let 
\[
w(t,x)=w(t,x',x'')=w_1(x')w_2(t,x''), \quad x'\in {\mathcal{X}}_1, \quad (t,x'')\in {\mathcal{X}}_2,
\]
where $w_1\in A_p({\mathcal{X}}_1)$ and $w_2\in A_q({\mathcal{X}}_2)$.
For such $w$, we define $L_{p,q,w}((S,T)\times \Omega)$ as the set of all measurable functions $u$ on $(S,T)\times \Omega$ having a finite norm
\[
\|u\|_{L_{p,q,w}((S,T)\times \Omega)}=\left(\int_{{\mathcal{X}}_2}\left(\int_{{\mathcal{X}}_1} |u|^pI_{(S,T)\times \Omega} w_1(x')\,dx'\right)^{q/p}w_2(t,x'')\,dx''\,dt\right)^{1/q}.
\]
We use 
\[
W^{1,m}_{p,q,w}((S,T)\times \Omega)=\{ u: u, Du, \ldots, D^m u, u_t\in L_{p,q,w}((S,T)\times \Omega)\}
\]
equipped with the norm
\[
\|u\|_{W^{1,m}_{p,q,w}((S,T)\times \Omega)}=\|u_t\|_{L_{p,q,w}((S,T)\times \Omega)}+\sum_{|\alpha| \le m}\|D^\alpha u\|_{L_{p,q,w}((S,T)\times \Omega)}.
\]
We also set 
\[
{\mathbb{H}}^{-m}_{p,q,w}((S,T)\times \Omega)=\left\{f:f=\sum_{|\alpha| \le m}D^\alpha f_\alpha, \quad f_\alpha \in L_{p,q,w}((S,T)\times \Omega)\right\},
\]
\[
\|f\|_{{\mathbb{H}}^{-m}_{p,q,w}((S,T)\times \Omega)}=\inf\left\{\sum_{|\alpha| \le m}\|f_\alpha\|_{L_{p,q,w}((S,T)\times \Omega)}: f=\sum_{|\alpha| \le m}D^\alpha f_\alpha \right\},
\]
and 
\begin{multline*}
{\mathcal{H}}^m_{p,q,w}((S,T)\times \Omega)\\
=\left\{ u: u_t\in {\mathbb{H}}^{-m}_{p,q,w}((S,T)\times \Omega), \,\, D^\alpha u\in L_{p,q,w}((S,T)\times \Omega), \,\, |\alpha| \le m\right\},
\end{multline*}
\[
\|u\|_{{\mathcal{H}}^m_{p,q,w}((S,T)\times \Omega)}=\|u_t\|_{{\mathbb{H}}^{-m}_{p,q,w}((S,T)\times \Omega)}+\sum_{|\alpha| \le m}\|D^\alpha u\|_{L_{p,q,w}((S,T)\times \Omega)}.
\]
We denote by $\mathring{\mathcal{H}}^m_{p,q,w}((S,T)\times \Omega)$ the closure of $C^\infty_0([S,T]\times \Omega)$ in ${\mathcal{H}}^m_{p,q,w}((S,T)\times \Omega)$, where $C^\infty_0([S,T]\times \Omega)$ is the set of all infinitely differentiable functions defined on $[S,T]\times \Omega$ with a compact support in $[S,T]\times \Omega$.
We abbreviate $L_{p,q,w}((S,T)\times \Omega)^n=L_{p,q,w}((S,T)\times \Omega)$, $L_{p,p,w}((S,T)\times \Omega)=L_{p,w}((S,T)\times \Omega)$, and $L_{p,q,1}((S,T)\times \Omega)=L_{p,q}((S,T)\times \Omega)$, etc.

For the elliptic case, we assume that $\Omega\subset {\mathcal{X}}_1\times {\mathcal{X}}_2$, where ${\mathcal{X}}_1$ and ${\mathcal{X}}_2$ are spaces of homogeneous type in ${\mathbb{R}}^{d_1}$ and ${\mathbb{R}}^{d_2}$, $d_1+d_2=d$, respectively.
Let 
\[
w(x)=w_1(x')w_2(x''), \quad x'\in {\mathcal{X}}_1, \quad x''\in {\mathcal{X}}_2,
\]
where $w_1\in A_p({\mathcal{X}}_1)$ and $w_2\in A_q( {\mathcal{X}}_2)$.
For such $w$, we define $L_{p,q,w}( \Omega)$ as the set consisting of all measurable functions $u$ defined on $\Omega$ having a finite norm
$$
\|u\|_{L_{p,q,w}(\Omega)}=\left(\int_{ {\mathcal{X}}_2}\left(\int_{{\mathcal{X}}_1} |u|^pI_\Omega w_1(x')\,dx'\right)^{q/p}w_2(x'')\,dx''\right)^{1/q}.
$$
We also set 
$$
W^m_{p,q,w}(\Omega)=\left\{u: D^\alpha u\in L_{p,q,w}(\Omega), \, |\alpha|\le m\right\},
$$
$$
\|u\|_{W^m_{p,q,w}(\Omega)}=\sum_{|\alpha|\le m}\| D^\alpha u\|_{L_{p,q,w}(\Omega)}.
$$
We denote by $\mathring{W}^m_{p,q,w}(\Omega)$ the closure of $C_0^\infty(\Omega)$ in $W^m_{p,q,w}(\Omega)$.

\section{Main results}		\label{sec3}

Throughout this section, we assume that the coefficients of ${\mathcal{L}}$ are bounded:
\begin{equation}		\label{boundedness}
\big|A^{\alpha\beta}_{ij}\big| \le 
\left\{
\begin{aligned}
\delta^{-1}&\quad \text{if }\, |\alpha|=|\beta|=m,\\
K &\quad \text{if }\, |\alpha|<m \,\text{ or } \, |\beta|<m,
\end{aligned}
\right.
\end{equation}
and the leading coefficients satisfy the  Legendre-Hadamard ellipticity condition:
\begin{equation}		\label{LH}
\Re \left(\sum_{|\alpha|=|\beta|=m} \sum_{i,j=1}^n A^{\alpha\beta}_{ij}(X)\xi^\alpha \xi^\beta \overline{\eta}_i\eta_j \right) \ge \delta|\xi|^{2m}|\eta|^2  
\end{equation}
for any $X\in {\mathbb{R}}^{d+1}$, $\xi\in {\mathbb{R}}^d$, and $\eta\in {\mathbb{C}}^n$.
Here, we use the notation $\Re(f)$ to denote the real part of $f$.

To state our regularity assumption on the leading coefficients, we introduce the following notation. 
For a function ${\boldsymbol{g}}=( g^1,\ldots, g^n)^{\operatorname{tr}}$ on ${\mathbb{R}}^{d+1}$, 
we define the mean oscillation of ${\boldsymbol{g}}$ in $Q_R(X_0)$ with respect to $x$ as
\begin{equation*}		
({\boldsymbol{g}})^{x,\sharp}_R(X_0)={\operatorname{\,\,\text{\bf--}\kern-.98em\DOTSI\intop\ilimits@\!\!}}_{Q_R(X_0)}\big|{\boldsymbol{g}}(s,y)-{\operatorname{\,\,\text{\bf--}\kern-.98em\DOTSI\intop\ilimits@\!\!}}_{B_R(x_0)}{\boldsymbol{g}}(s,z) \, dz  \big| \,dy \, ds.
\end{equation*}

\begin{assumption}[$\gamma$]		\label{0923.ass1}
There exists $R_0\in (0,1]$ such that the following hold.
\begin{enumerate}[(i)]
\item
For any $X_0=(t_0,x_0)\in{\mathbb{R}}\times \overline{\Omega}$ and $R\in(0,R_0]$ such that either $B_R(x_0)\subset \Omega$ or $x_0\in \partial \Omega$, we have
$$
\sum_{|\alpha|=|\beta|=m}(A^{\alpha\beta})^{x,\sharp}_R(X_0)\le \gamma.
$$
\item
For any $X_0=(t_0,x_0)\in {\mathbb{R}}\times \partial \Omega$ and $R\in (0,R_0]$, 
there is a spatial coordinate system depending on $x_0$ and $R$ such that in this new coordinate system, we have
\begin{equation*}		
\{y:{x_0}_1+\gamma R<y_1\}\cap B_R(x_0)\subset \Omega_R(x_0)\subset \{y:{x_0}_1-\gamma R<y_1\}\cap B_R(x_0),
\end{equation*}
where ${x_0}_1$ is the first coordinate of $x_0$ in the new coordinate system.
\end{enumerate}
\end{assumption}

\begin{theorem}		\label{1008@thm1}
Let $T\in (-\infty,\infty]$, $\Omega$  be a domain in ${\mathbb{R}}^d$, and $\Omega_T\subseteq {\mathcal{X}}$, where  ${\mathcal{X}}$ is  a space of homogeneous type in ${\mathbb{R}}^{d+1}$ with a doubling constant $K_2$.
Let $p\in (1,\infty)$, $K_0\ge 1$, $w\in A_p({\mathcal{X}})$, and $[w]_{A_p}\le K_0$.
Then there exist constants 
\begin{align*}
&\gamma=\gamma(d,m,n,\delta,p,K_0,K_2)\in (0,1/6),\\
&\lambda_0=\lambda_0(d,m,n,\delta,p, K_0,K_2, R_0,K)>0
\end{align*}
such that, under Assumption \ref{0923.ass1} $(\gamma)$, for ${\boldsymbol{u}}\in \mathring{\mathcal{H}}^m_{p,w}(\Omega_T)$ satisfying 
\begin{equation}	\label{160608@eq5}
{\boldsymbol{u}}_t+(-1)^m{\mathcal{L}}{\boldsymbol{u}}+\lambda{\boldsymbol{u}}=\sum_{|\alpha| \le m}D^\alpha {\boldsymbol{f}}_\alpha \quad \text{in }\, \Omega_T, 
\end{equation}
where $\lambda\ge \lambda_0$ and ${\boldsymbol{f}}_\alpha\in L_{p,w}(\Omega_T)$, we have 
\begin{equation}		\label{1017@e3a}
\sum_{|\alpha| \le m}\lambda^{1-\frac{|\alpha|}{2m}}\|D^\alpha {\boldsymbol{u}}\|_{L_{p,w}(\Omega_T)}\le N\sum_{|\alpha| \le m}\lambda^{\frac{|\alpha|}{2m}}\|{\boldsymbol{f}}_\alpha\|_{L_{p,w}(\Omega_T)},
\end{equation}
where $N=N(d,m,n,\delta,p,K_0,K_2)$.
\end{theorem}

The next theorem is about the solvability of the system \eqref{160608@eq5} in the weighted Sobolev space $\mathring{\mathcal{H}}^m_{p,w}( \Omega_T)$.

\begin{theorem}		\label{160621@thm1}
Let $T\in (-\infty,\infty]$, $\Omega$  be a domain in ${\mathbb{R}}^d$, and $\Omega_T\subseteq {\mathcal{X}}$, where ${\mathcal{X}}$ is  a space of homogeneous type in ${\mathbb{R}}^{d+1}$ with a doubling constant $K_2$.
Let $p\in (1,\infty)$, $K_0\ge 1$, $w\in A_p({\mathcal{X}})$, and $[w]_{A_p}\le K_0$.
Assume that $|\Omega|<\infty$, $\Omega={\mathbb{R}}^d$, or $\Omega={\mathbb{R}}^d_+$.
Then there exist constants 
\begin{align*}
&\bar\gamma=\bar\gamma(d,m,n,\delta,p,K_0,K_2)\in (0,1/6),\\
&\bar\lambda_0=\bar\lambda_0(d,m,n,\delta,p, K_0, K_2, R_0,K)>0
\end{align*}
such that, under Assumption \ref{0923.ass1} $(\bar\gamma)$, for any $\lambda\ge \bar\lambda_0$ and ${\boldsymbol{f}}_\alpha\in L_{p,w}(\Omega_T)$, $|\alpha|\le m$, there exists a unique  ${\boldsymbol{u}}\in \mathring{\mathcal{H}}_{p,w}^m(\Omega_T)$ satisfying \eqref{160608@eq5}.
\end{theorem}

\begin{remark}
In the case when $\Omega={\mathbb{R}}^d$,  
Assumption \ref{0923.ass1} $(\gamma)$ in Theorem \ref{160621@thm1} is to be understood as Assumption \ref{0923.ass1} $(\gamma)$ $(i)$ because $\Omega$ has no boundary.
\end{remark}

We prove the following weighted $L_{p,q}$-estimates (mixed norms) for parabolic systems.

\begin{theorem}		\label{1016@thm1}
Let $T\in (-\infty,\infty]$, $\Omega$  be a domain in ${\mathbb{R}}^d$, and $ \Omega_T\subseteq {\mathcal{X}}_1\times {\mathcal{X}}_2$, where ${\mathcal{X}}_1$ and ${\mathcal{X}}_2$ are  spaces of homogeneous type with  doubling constants $K_2'$ and $K_2''$ in ${\mathbb{R}}^{d_1}$ and ${\mathbb{R}}\times {\mathbb{R}}^{d_2}$, $d_1+d_2=d$, respectively.
Let $p,q\in (1,\infty)$, $K_0\ge 1$, and 
$$
w(t,x)=w_1(x')w_2(t,x''), \quad x'\in {\mathcal{X}}_1, \quad (t,x'')\in {\mathcal{X}}_2,
$$
where $w_1\in A_p({\mathcal{X}}_1)$ with $[w_1]_{A_p}\le K_0$ and $w_2\in A_q({\mathcal{X}}_2)$ with $[w_2]_{A_q}\le K_0$.
Then there exist constants
\begin{align*}
&\gamma=\gamma(d,m,n,\delta,p,q,K_0, K_2', K_2'')\in (0,1/6),\\
&\lambda_0=\lambda_0(d,m,n,\delta,p,q, K_0,K_2',K_2'',R_0,K)>0,
\end{align*}
such that, under Assumption \ref{0923.ass1} $(\gamma)$, for ${\boldsymbol{u}}\in \mathring{\mathcal{H}}^m_{p,q,w}(\Omega_T)$ satisfying 
\begin{equation}		\label{160621@eq1}
{\boldsymbol{u}}_t+(-1)^m{\mathcal{L}}{\boldsymbol{u}}+\lambda{\boldsymbol{u}}=\sum_{|\alpha| \le m}D^\alpha {\boldsymbol{f}}_\alpha \quad \text{in }\, \Omega_T, 
\end{equation}
where $\lambda\ge \lambda_0$ and ${\boldsymbol{f}}_\alpha\in L_{p,q,w}(\Omega_T)$, we have 
\begin{equation}		\label{1017@e4}
\sum_{|\alpha| \le m}\lambda^{1-\frac{|\alpha|}{2m}}\|D^\alpha {\boldsymbol{u}}\|_{L_{p,q,w}(\Omega_T)}\le N\sum_{|\alpha| \le m}\lambda^{\frac{|\alpha|}{2m}}\|{\boldsymbol{f}}_\alpha\|_{L_{p,q,w}(\Omega_T)},
\end{equation}
where $N=N(d,m,n,\delta,p,q,K_0, K_2', K_2'',d_1,d_2)$.
\end{theorem}

We obtain the solvability of the system \eqref{160621@eq1} in $\mathring{\mathcal{H}}^m_{p,q,w}(\Omega_T)$ as follows.

\begin{theorem}		\label{160621@thm2}
Let $T\in (-\infty,\infty]$, $\Omega$  be a domain in ${\mathbb{R}}^d$, and $\Omega_T\subseteq {\mathcal{X}}_1\times {\mathcal{X}}_2$, where ${\mathcal{X}}_1$ and ${\mathcal{X}}_2$ are  spaces of homogeneous type with  doubling constants $K_2'$ and $K_2''$ in ${\mathbb{R}}^{d_1}$ and ${\mathbb{R}}\times {\mathbb{R}}^{d_2}$, $d_1+d_2=d$, respectively.
Let $p,q\in (1,\infty)$, $K_0\ge 1$, and 
$$
w(t,x)=w_1(x')w_2(t,x''), \quad x'\in {\mathcal{X}}_1, \quad (t,x'')\in {\mathcal{X}}_2,
$$
where $w_1\in A_p({\mathcal{X}}_1)$ with $[w_1]_{A_p}\le K_0$ and $w_2\in A_q({\mathcal{X}}_2)$ with $[w_2]_{A_q}\le K_0$.
Assume that $|\Omega|<\infty$, $\Omega={\mathbb{R}}^d$, or $\Omega={\mathbb{R}}^d_+$.
Then there exist constants 
\begin{align*}
&\bar{\gamma}=\bar{\gamma}(d,m,n,\delta,p,q,d_1,d_2,K_0, K_2', K_2'')\in (0,1/6),\\
&\bar{\lambda}_0=\bar{\lambda}_0(d,m,n,\delta,p,q,d_1,d_2, K_0, K_2',K_2'',R_0,K)>0,
\end{align*}
such that, under Assumption \ref{0923.ass1} $(\bar{\gamma})$, for any $\lambda\ge \bar{\lambda}_0$ and ${\boldsymbol{f}}_\alpha\in L_{p,q,w}(\Omega_T)$, $|\alpha|\le m$, there exists a unique
${\boldsymbol{u}}\in \mathring{\mathcal{H}}^m_{p,q,w}(\Omega_T)$ satisfying \eqref{160621@eq1}.
\end{theorem}

If unmixed norms are considered, the elliptic case as in the theorem below is covered by  \cite{MR2835999}.
Here, we present  the mixed norm case for elliptic systems, which follows easily from Theorem \ref{1016@thm1} and a standard argument  in the proof of  \cite[Theorem 2.6]{MR2650802}.

\begin{theorem}		\label{160622@thm1}
Let $\Omega$ be a domain in ${\mathbb{R}}^d$  and $ \Omega\subseteq {\mathcal{X}}_1 \times {\mathcal{X}}_2$, where ${\mathcal{X}}_1$ and $ {\mathcal{X}}_2$ are  spaces of homogeneous type with  doubling constants $K_2'$ and $K_2''$ in ${\mathbb{R}}^{d_1}$ and ${\mathbb{R}}^{d_2}$, $d_1+d_2=d$, respectively.
Let $p,q\in (1,\infty)$, $K_0\ge 1$, and 
$$
w(x)=w_1(x')w_2(x''), \quad x'\in {\mathcal{X}}_1, \quad x''\in {\mathcal{X}}_2,
$$
where $w_1\in A_p({\mathcal{X}}_1)$ with $[w_1]_{A_p}\le K_0$ and $w_2\in A_q( {\mathcal{X}}_2)$ with $[w_2]_{A_q}\le K_0$.
Then there exist constants
\begin{align*}
&\gamma=\gamma(d,m,n,\delta,p,q,K_0, K_2', K_2'')\in (0,1/6),\\
&\lambda_0=\lambda_0(d,m,n,\delta,p,q,K_0, K_2',K_2'',R_0,K)>0,
\end{align*}
such that, under Assumption \ref{0923.ass1} $(\gamma)$, for ${\boldsymbol{u}}\in \mathring{W}^m_{p,q,w}(\Omega)$ satisfying 
\begin{equation}		\label{160622@@eq1}
(-1)^m{\mathcal{L}}{\boldsymbol{u}}+\lambda{\boldsymbol{u}}=\sum_{|\alpha| \le m}D^\alpha {\boldsymbol{f}}_\alpha \quad \text{in }\, \Omega, 
\end{equation}
where $\lambda\ge \lambda_0$ and ${\boldsymbol{f}}_\alpha\in L_{p,q,w}(\Omega)$, we have 
$$
\sum_{|\alpha| \le m}\lambda^{1-\frac{|\alpha|}{2m}}\|D^\alpha {\boldsymbol{u}}\|_{L_{p,q,w}(\Omega)}\le N\sum_{|\alpha| \le m}\lambda^{\frac{|\alpha|}{2m}}\|{\boldsymbol{f}}_\alpha\|_{L_{p,q,w}(\Omega)},
$$
where $N=N(d,m,n,\delta,p,q,K_0, K_2', K_2'',d_1,d_2)$.
\end{theorem}

The following theorem is derived from Theorem \ref{160622@thm1} in the same manner as Theorem \ref{160621@thm1} is derived from Theorem \ref{1008@thm1}.

\begin{theorem}		\label{160622@thm5}
Let $\Omega$ be an open set in ${\mathbb{R}}^d$  and $\Omega\subseteq {\mathcal{X}}_1\times {\mathcal{X}}_2$, where ${\mathcal{X}}_1$ and ${\mathcal{X}}_2$ are  spaces of homogeneous type with  doubling constants $K_2'$ and $K_2''$ in ${\mathbb{R}}^{d_1}$ and ${\mathbb{R}}^{d_2}$, $d_1+d_2=d$, respectively.
Let $p,q\in (1,\infty)$, $K_0\ge 1$, and 
$$
w(x)=w_1(x')w_2(x''), \quad x'\in {\mathcal{X}}_1, \quad x''\in {\mathcal{X}}_2,
$$
where $w_1\in A_p({\mathcal{X}}_1)$ with $[w_1]_{A_p}\le K_0$ and $w_2\in A_q({\mathcal{X}}_2)$ with $[w_2]_{A_q}\le K_0$.
Assume that $|\Omega|<\infty$, $\Omega={\mathbb{R}}^d$, or $\Omega={\mathbb{R}}^d_+$.
Then there exist constants 
\begin{align*}
&\bar{\gamma}=\bar{\gamma}(d,m,n,\delta,p,q,d_1,d_2,K_0, K_2', K_2'')\in (0,1/6),\\
&\bar{\lambda}_0=\bar{\lambda}_0(d,m,n,\delta,p,q,d_1,d_2,K_0,K_2'
,K_2'',R_0,K)>0,
\end{align*}
such that, under Assumption \ref{0923.ass1} $(\bar{\gamma})$, 
for any $\lambda\ge \bar{\lambda}_0$ and ${\boldsymbol{f}}_\alpha\in L_{p,q,w}(\Omega)$, $|\alpha|\le m$, there exists a unique
${\boldsymbol{u}}\in \mathring{W}^m_{p,q,w}(\Omega)$ satisfying \eqref{160622@@eq1}.
\end{theorem}

\section{Some auxiliary results}		

The results in this section can be found, for instance, in \cite{MR3243734, MR1232192}.
Here, we present those results in a form convenient for later use along with some of their proofs.
In particular, we specify the parameters on which the constants $N$ and $\mu$ in the results below depend.
For example, we assume that $[w]_{A_p} \le K_0$, $K_0 \ge 1$, and show that the constants $N$ in the inequalities depend on $K_0$ rather than $[w]_{A_q}$.

In this section, we assume that ${\mathcal{X}}$ is a space of homogeneous type in ${\mathbb{R}}^{d+1}$ (resp. ${\mathbb{R}}^d$) with the distance $\rho$ in \eqref{eq0215_01} (resp. the usual Euclidean distance), the Lebesgue measure, and a doubling constant $K_2$.
In the case that ${\mathcal{X}} \subset {\mathbb{R}}^{d+1}$, as we recall, ${\mathcal{B}}_r^{\mathcal{X}}(X)$ is a ball in ${\mathcal{X}}$ defined by 
$$
{\mathcal{B}}_r^{\mathcal{X}}(X)=\{Y\in {\mathcal{X}}:\rho(X,Y)<r\}.
$$

\begin{lemma}[Reverse H\"older's inequality]		\label{1016@lem1}
Let $p\in (1,\infty)$, $K_0 \ge 1$, $w\in A_p({\mathcal{X}})$, and  $[w]_{A_p}\le K_0$.
Then there exist constants $\mu_0>1$ and $N>0$, depending only on $p$, $K_0$, and $K_2$, such that 
$$
\left({\operatorname{\,\,\text{\bf--}\kern-.98em\DOTSI\intop\ilimits@\!\!}}_{{\mathcal{B}}_r^{\mathcal{X}}(X)} w^{\mu_0}\,dY\right)^{\frac{1}{\mu_0}}\le N{\operatorname{\,\,\text{\bf--}\kern-.98em\DOTSI\intop\ilimits@\!\!}}_{{\mathcal{B}}_r^{\mathcal{X}}(X)}w\,dY
$$
for any $X\in {\mathcal{X}}$ and $r>0$.
\end{lemma}

\begin{proof}
See \cite[Theorem 7.3.3]{MR3243734} or \cite[Theorem 3, p. 212]{MR1232192}.
\end{proof}

\begin{lemma}		\label{1016@lem2}
Let $p\in (1,\infty)$, $K_0 \ge 1$, $w\in A_p({\mathcal{X}})$, and  $[w]_{A_p}\le K_0$.
Then there exist constants $p_0\in (1,p)$ and $N>0$, depending only on $p$, $K_0$, and $K_2$, such that $w\in A_{p_0}({\mathcal{X}})$ and 
\begin{equation}		\label{160614@eq2}
\left({\operatorname{\,\,\text{\bf--}\kern-.98em\DOTSI\intop\ilimits@\!\!}}_{{\mathcal{B}}_r^{\mathcal{X}}(X)}w^{-\frac{1}{p_0-1}}\,dY\right)^{p_0-1}\le N \left({\operatorname{\,\,\text{\bf--}\kern-.98em\DOTSI\intop\ilimits@\!\!}}_{{\mathcal{B}}_r^{\mathcal{X}}(X)}w^{-\frac{1}{p-1}}\,dY\right)^{p-1}
\end{equation}
for any $X\in {\mathcal{X}}$ and $r>0$.
\end{lemma}

\begin{proof}
Let $p'=\frac{p}{p-1}$.
Since $w\in A_p({\mathcal{X}})$, we obtain that   $v=w^{-\frac{1}{p-1}}\in A_{p'}({\mathcal{X}})$.
Indeed, for any $X\in {\mathcal{X}}$ and $r>0$, we have 
$$
\left({\operatorname{\,\,\text{\bf--}\kern-.98em\DOTSI\intop\ilimits@\!\!}}_{{\mathcal{B}}^{\mathcal{X}}_r(X)}v\,dY\right)\left({\operatorname{\,\,\text{\bf--}\kern-.98em\DOTSI\intop\ilimits@\!\!}}_{{\mathcal{B}}^{\mathcal{X}}_r(X)}v^{-\frac{1}{p'-1}}\,dY\right)^{p'-1}
$$
$$
=\left({\operatorname{\,\,\text{\bf--}\kern-.98em\DOTSI\intop\ilimits@\!\!}}_{{\mathcal{B}}^{\mathcal{X}}_r(X)}w^{-\frac{1}{p-1}}\,dY\right)\left({\operatorname{\,\,\text{\bf--}\kern-.98em\DOTSI\intop\ilimits@\!\!}}_{{\mathcal{B}}^{\mathcal{X}}_r(X)}w\,dY\right)^{\frac{1}{p-1}}
\le [w]_{A_p}^{\frac{1}{p-1}}\le K_0^{\frac{1}{p-1}}.
$$
Therefore, by Lemma \ref{1016@lem1}, we have 
\[
\left({\operatorname{\,\,\text{\bf--}\kern-.98em\DOTSI\intop\ilimits@\!\!}}_{{\mathcal{B}}^{\mathcal{X}}_r(X)}v^{\mu_0}\,dY\right)^{\frac{1}{\mu_0}}\le N{\operatorname{\,\,\text{\bf--}\kern-.98em\DOTSI\intop\ilimits@\!\!}}_{{\mathcal{B}}^{\mathcal{X}}_r(X)}v\,dY,
\]
where $(N,\mu_0)=(N,\mu_0)(p, K_0, K_2)$.
By taking $p_0=\frac{p-1}{\mu_0}+1\in (1,p)$ in the above inequality, we obtain \eqref{160614@eq2}.
\end{proof}

\begin{lemma}		\label{1015@lem3}
Let $p\in (1,\infty)$, $K_0 \ge 1$, $w\in A_p({\mathcal{X}})$, and  $[w]_{A_p}\le K_0$. 
Then there exist  constants $\mu_1\in (0,1)$ and  $N_1>0$, depending only on $p$, $K_0$, and $K_2$, such that for any measurable set $E\subset {\mathcal{X}}$, we have 
\[
\frac{1}{K_0}\left(\frac{|E\cap {\mathcal{B}}_r^{\mathcal{X}}(X)|}{|{\mathcal{B}}_r^{\mathcal{X}}(X)|}\right)^p\le \frac{w(E\cap {\mathcal{B}}_r^{\mathcal{X}}(X))}{w({\mathcal{B}}_r^{\mathcal{X}}(X))}\le N_1\left(\frac{|E\cap {\mathcal{B}}_r^{\mathcal{X}}(X)|}{|{\mathcal{B}}_r^{\mathcal{X}}(X)|}\right)^{\mu_1}
\]
for any $X\in {\mathcal{X}}$ and $r>0$.
\end{lemma}

\begin{proof}
From H\"older's inequality and the definition of $A_p$, it follows that 
\begin{align*}
|E\cap {\mathcal{B}}_r^{\mathcal{X}}(X)|&=\int_{E\cap {\mathcal{B}}_r^{\mathcal{X}}(X)}w^{\frac{1}{p}}w^{-\frac{1}{p}}\,dY\\
&\le w(E\cap {\mathcal{B}}_r^{\mathcal{X}}(X))^{\frac{1}{p}}\left({\operatorname{\,\,\text{\bf--}\kern-.98em\DOTSI\intop\ilimits@\!\!}}_{{\mathcal{B}}_r^{\mathcal{X}}(X)}w^{-\frac{1}{p-1}}\,dY\right)^{\frac{p-1}{p}}|{\mathcal{B}}_r^{\mathcal{X}}(X)|^{\frac{p-1}{p}}\\
& \le [w]_{A_p}^{\frac{1}{p}} w(E\cap {\mathcal{B}}_r^{\mathcal{X}}(X))^{\frac{1}{p}}\left({\operatorname{\,\,\text{\bf--}\kern-.98em\DOTSI\intop\ilimits@\!\!}}_{{\mathcal{B}}_r^{\mathcal{X}}(X)}w\,dY\right)^{-\frac{1}{p}}|{\mathcal{B}}_r^{\mathcal{X}}(X)|^{\frac{p-1}{p}}\\
&=[w]_{A_p}^{\frac{1}{p}}\left(\frac{w(E\cap {\mathcal{B}}_r^{\mathcal{X}}(X))}{w({\mathcal{B}}_r^{\mathcal{X}}(X))}\right)^{\frac{1}{p}}|{\mathcal{B}}_r^{\mathcal{X}}(X)|,
\end{align*}
which gives  the first inequality.
For the second inequality, we observe that H\"older's inequality and Lemma \ref{1016@lem1} imply that
\begin{align*}
w(E\cap {\mathcal{B}}_r^{\mathcal{X}}(X))&\le |E\cap {\mathcal{B}}_r^{\mathcal{X}}(X)|^{\frac{\mu_0-1}{\mu_0}}|{\mathcal{B}}_r^{\mathcal{X}}(X)|^{\frac{1}{\mu_0}}\left({\operatorname{\,\,\text{\bf--}\kern-.98em\DOTSI\intop\ilimits@\!\!}}_{{\mathcal{B}}_r^{\mathcal{X}}(X)} w^{\mu_0}\, dY\right)^{\frac{1}{\mu_0}}\\
&\le N|E\cap {\mathcal{B}}_r^{\mathcal{X}}(X)|^{\frac{\mu_0-1}{\mu_0}}|{\mathcal{B}}_r^{\mathcal{X}}(X)|^{\frac{1}{\mu_0}}\left({\operatorname{\,\,\text{\bf--}\kern-.98em\DOTSI\intop\ilimits@\!\!}}_{{\mathcal{B}}_r^{\mathcal{X}}(X)} w\, dY\right)\\
&=N\left(\frac{|E\cap {\mathcal{B}}_r^{\mathcal{X}}(X)|}{|{\mathcal{B}}_r^{\mathcal{X}}(X)|}\right)^{\frac{\mu_0-1}{\mu_0}}w({\mathcal{B}}_r^{\mathcal{X}}(X)),
\end{align*}
where $(N,\mu_0)=(N,\mu_0)(p,K_0,K_2)$.
The lemma is proved.
\end{proof}

The following Hardy-Littlewood maximal function theorem with $A_p$ weights was obtained in \cite{MR0740173}.
Below, we denote the maximal function of $f$ defined on ${\mathcal{X}}$ by 
\begin{equation}		\label{160617@eq1}
{\mathcal{M}} f(X)=\sup_{\substack{Z\in {\mathcal{X}}, r>0\\ X\in {\mathcal{B}}_r^{\mathcal{X}}(Z)}}{\operatorname{\,\,\text{\bf--}\kern-.98em\DOTSI\intop\ilimits@\!\!}}_{{\mathcal{B}}_r^{\mathcal{X}}(Z)}|f(Y)|\, dY.
\end{equation}

\begin{theorem}		\label{1008@thm5}
Let $p\in (1,\infty)$, $K_0 \ge 1$, $w\in A_p({\mathcal{X}})$, and $[w]_{A_p}\le K_0$.
Then for any $f\in L_{p,w}({\mathcal{X}})$, we have 
\[
\|{\mathcal{M}} f\|_{L_{p,w}({\mathcal{X}})}\le N\|f\|_{L_{p,w}({\mathcal{X}})},
\]
where $N=N(p,K_0, K_2)>0$.
\end{theorem}

\section{Interior and boundary estimates}		

In this section, we denote
\begin{equation*}
{\mathcal{L}}_0{\boldsymbol{u}}=\sum_{|\alpha|=|\beta|=m}D^\alpha\big({A}^{\alpha\beta}_0D^\beta{\boldsymbol{u}}\big),
\end{equation*}
where $A_0^{\alpha\beta}=A_0^{\alpha\beta}(t)$ satisfy \eqref{boundedness} and \eqref{LH}.

In the lemma below, we provide $L_\infty$-estimates not only for a weak solution ${\boldsymbol{u}}$ but also for its derivatives $D^m {\boldsymbol{u}}$.
In fact, the results in the lemma is proved by the standard iteration argument along with the Sobolev embedding theorem and the known $L_p$-estimates for  systems.
Precisely, since the coefficients $A_0^{\alpha\beta}$ are independent of the spatial variables, we view the operator ${\mathcal{L}}_0$ as a non-divergence type operator and use the $L_p$-estimates for non-divergence type systems proved in \cite{MR2771670}.
The proof is mostly standard, so we only describe the major steps.

For a given constant $\lambda \ge 0$ and functions ${\boldsymbol{u}}$ and ${\boldsymbol{f}}_\alpha$, $|\alpha| \le m$, we write
\begin{equation}
							\label{eq0223_01}		
U=\big(\lambda^{\frac{1}{2}-\frac{|\alpha|}{2m}}D^\alpha {\boldsymbol{u}}\big)_{|\alpha| \le m} \quad \text{and}\quad F=\big(\lambda^{\frac{|\alpha|}{2m}-\frac{1}{2}}{\boldsymbol{f}}_\alpha\big)_{|\alpha| \le m},
\end{equation}
where $f_\alpha \equiv 0$ for $|\alpha|<m$ whenever $\lambda = 0$.

\begin{lemma}		\label{0929.lem1}
Let $\lambda\ge 0$ and $q\in (1,\infty)$.
\begin{enumerate}[$(a)$]
\item
If 
${\boldsymbol{u}}\in C^\infty_{\operatorname{loc}}((-\infty,0] \times {\mathbb{R}}^d)$
satisfies
\begin{equation*}				
{\boldsymbol{u}}_t+(-1)^m{\mathcal{L}}_0{\boldsymbol{u}}+\lambda{\boldsymbol{u}}=0 \quad \text{in }\, Q_2,
\end{equation*}
then we have 
\begin{equation}		\label{1002.eq1}
\|U\|_{L_\infty(Q_1)}\le N\|U\|_{L_q(Q_2)},
\end{equation}
where $N=N(d,m,n,\delta,q)$.
\item
If 
${\boldsymbol{u}}\in C^\infty_{\operatorname{loc}}((-\infty,0] \times \overline{{\mathbb{R}}^d_+})$ satisfies
\begin{equation*}		
\left\{
\begin{aligned}
{\boldsymbol{u}}_t+(-1)^m{\mathcal{L}}_0{\boldsymbol{u}}+\lambda {\boldsymbol{u}}=0 &\quad \text{in }\, Q_2^+,\\
|{\boldsymbol{u}}| =\cdots=|D^{m-1}_1{\boldsymbol{u}}|=0 &\quad \text{on }\, Q_2 \cap \{X \in {\mathbb{R}}^{d+1}: x_1 = 0\},
\end{aligned}
\right.
\end{equation*}
then we have
$$
\|U\|_{L_\infty(Q_1^+)}\le N\|U\|_{L_q(Q_2^+)},
$$
where $N=N(d,m,n,\delta,q)$.

\end{enumerate}
\end{lemma}

\begin{proof}
We first prove the assertion (a) with $\lambda=0$.
As mentioned above, owing to the coefficients being independent of the spatial variables, ${\boldsymbol{u}} \in C_{\operatorname{loc}}^\infty((-\infty,0] \times {\mathbb{R}}^d)$ satisfies the following non-divergence type system
\begin{equation}
							\label{eq0222_04}
{\boldsymbol{u}}_t+(-1)^m \sum_{|\alpha|=|\beta|=m} A_0^{\alpha\beta} D^\alpha D^\beta {\boldsymbol{u}} = 0 \quad \text{in }\, Q_2.
\end{equation}
By the $L_p$-estimate for non-divergence type systems in \cite[Theorem 2]{MR2771670} and the localization argument as in the proof of \cite[Lemma 1]{MR2771670}, we obtain
$$
\|{\boldsymbol{u}}\|_{W_p^{1,2m}(Q_r)}\le N\|{\boldsymbol{u}}\|_{L_p(Q_R)}
$$
for any $p \in (1,\infty)$ and $1 \le r < R \le 2$, where $N=N(d,m,n,\delta,p,r,R)$.
From the above inequality, the standard iteration argument, and Sobolev embedding type results (see Lemmas \ref{0922.lem2} and \ref{0922.lem4}), we have 
\begin{equation}				\label{eq0222_03}
\|{\boldsymbol{u}}\|_{L_\infty(Q_1)} \le N\|{\boldsymbol{u}}\|_{W^{1,2m}_{q_1}(Q_1)}\le N\|{\boldsymbol{u}}\|_{L_q(Q_2)},
\end{equation} 
where $q_1 \in [q,\infty)$ is sufficiently large so that $2m > (d+2m)/q_1$ (see Lemma \ref{0922.lem2}). 
Since $D^m {\boldsymbol{u}}$ also satisfies \eqref{eq0222_04}, we obtain \eqref{eq0222_03} with $D^m {\boldsymbol{u}}$ in place of ${\boldsymbol{u}}$, which is  \eqref{1002.eq1} with $\lambda = 0$.

Now we prove the assertion (b) for $\lambda = 0$. We repeat the above argument by using $L_p$-estimates for systems defined on a half space with the Dirichlet boundary condition. 
Precisely, using \cite[Theorem 4]{MR2771670} instead of \cite[Theorem 2]{MR2771670}, we arrive at
\begin{equation}				\label{eq0222_05}
\|D^m {\boldsymbol{u}}\|_{L_\infty(Q_1^+)} \le N\|{\boldsymbol{u}}\|_{W^{1,2m}_{q_1}(Q_1^+)}\le N\|{\boldsymbol{u}}\|_{L_q(Q_2^+)},
\end{equation}
where $q_1 \in [q,\infty)$ and $m > (d+2m)/q_1$ (again, see Lemma \ref{0922.lem2}). 
Then, one can bound the last term in \eqref{eq0222_05} by $N \|D^m{\boldsymbol{u}}\|_{L_q(Q_2^+)}$ by repeatedly using the Poincar\'e inequality (see, for instance, \cite[Theorem 10.2.5]{MR2435520}), i.e.,
for $u \in W_p^1(B_1^+)$, we have
$$
\int_{B_1^+}|u(x)|^p \, dx=\int_{B_1}|\bar{u}(x)|^p \, dx
=\frac{1}{2|B_1|}\int_{B_1}\int_{B_1 \setminus B_1^+}|\bar{u}(x)-\bar{u}(y)|^p \, dy \, dx
$$
$$
\le \frac{1}{2|B_1|}\int_{B_1}\int_{B_1}|\bar{u}(x)-\bar{u}(y)|^p \, dy \, dx
\le 2^d \int_{B_1}|\nabla \bar{u}(x)|^p \, dx,
$$
where $\bar{u} \in W_p^1(B_1)$ is an extension of $u$ to $B_1$ so that $u \equiv 0$ on $B_1 \setminus B_1^+$.

For a general $\lambda>0$, we only prove (a). The other case is entirely analogous. We use an idea by S. Agmon.
Let $\eta=\eta_\lambda(\tau)$ be a smooth function on ${\mathbb{R}}$ defined by 
\[
\eta(\tau)=\cos(\lambda^{\frac{1}{2m}}\tau)+\sin (\lambda^{\frac{1}{2m}}\tau ).
\]
Note that 
\[
(-1)^m D^{2m}_\tau\eta=\lambda \eta \quad \text{and}\quad |D^j_\tau \eta(0)|=\lambda^{\frac{j}{2m}}, \quad \forall j=0,1,\ldots.
\]
By setting 
\[
\hat{{\boldsymbol{u}}}(t,x,\tau)={\boldsymbol{u}}(t,x)\eta(\tau) \quad \text{and}\quad \widehat{Q}_r=(-r^{2m},0)\times \{(x,\tau)\in {\mathbb{R}}^{d+1}:|(x,\tau)|<r\},
\]
we see that $\hat{{\boldsymbol{u}}}$ satisfies 
\[
\hat{{\boldsymbol{u}}}_t+(-1)^m {\mathcal{L}}_0 \hat{{\boldsymbol{u}}}+(-1)^mD_\tau^{2m}\hat{{\boldsymbol{u}}}=0 \quad \text{in }\, \widehat{Q}_2.
\]
By applying the  result for $\lambda = 0$ to $\hat{{\boldsymbol{u}}}$, we obtain 
\begin{equation}		\label{0922.eq1d}
\|D^m_x {\boldsymbol{u}}\|_{L_\infty(Q_1)}\le\|D^m_{(x,\tau)} \hat{{\boldsymbol{u}}}\|_{L_\infty(\widehat{Q}_1)}\le N\|D^m_{(x,\tau)}\hat{{\boldsymbol{u}}}\|_{L_q(\widehat{Q}_2)}.
\end{equation}
We also obtain from \eqref{eq0222_03} that 
\begin{equation}		\label{0922.eq1e}
\|{\boldsymbol{u}}\|_{L_\infty(Q_1)}\le \|\hat{{\boldsymbol{u}}}\|_{L_\infty(\widehat{Q}_1)}\le N\|\hat{{\boldsymbol{u}}}\|_{L_q(\widehat{Q}_2)}\le N\|{\boldsymbol{u}}\|_{L_q(Q_2)}.
\end{equation}
Notice that $D^m_{(x,\tau)}\hat{{\boldsymbol{u}}}$ is a linear combination of 
\[
\lambda^{\frac{1}{2}-\frac{k}{2m}}\cos (\lambda^{\frac{1}{2m}}\tau)D^k_x{\boldsymbol{u}} \quad \text{and}\quad \lambda^{\frac{1}{2}-\frac{k}{2m}}\sin (\lambda^{\frac{1}{2m}}\tau)D^k_x{\boldsymbol{u}}, \quad k=0,1,\ldots,m.
\]
Therefore, by combining \eqref{0922.eq1d} and \eqref{0922.eq1e}, and then, using the interpolation inequalities, we conclude \eqref{1002.eq1}.
\end{proof}

In the following lemma,  we consider the operator ${\mathcal{L}}$ without lower order terms, i.e., 
\[
{\mathcal{L}} {\boldsymbol{u}}=\sum_{|\alpha|=|\beta|=m}D^\alpha(A^{\alpha\beta}D^\beta {\boldsymbol{u}}).
\]

\begin{lemma}		\label{0929-lem1}
Let $T\in [0,\infty]$,  $\lambda\ge 0$, $q\in (1,\infty)$, $\nu \in (1,\infty)$, $\nu' = \nu/(\nu-1)$, and $\Omega$ be a domain in ${\mathbb{R}}^d$.
Assume ${\boldsymbol{u}}\in C^\infty_0((-\infty,T]\times \Omega)$ satisfies
\begin{equation}		\label{1004.eq1}
{\boldsymbol{u}}_t+(-1)^m{\mathcal{L}}{\boldsymbol{u}}+\lambda {\boldsymbol{u}}=\sum_{|\alpha| \le m}D^\alpha {\boldsymbol{f}}_\alpha \quad \text{in }\ \Omega_T, 
\end{equation}
where ${\boldsymbol{f}}_\alpha\in L_{q,\operatorname{loc}}((-\infty,T] \times \overline{\Omega})$, $|\alpha| \le m$.
\begin{enumerate}[$(a)$]
\item
Suppose that Assumption \ref{0923.ass1} $(\gamma)$ {\rm (i)} holds at $0\in \Omega$ with $\gamma>0$. 
Then for $R$ such that $0 < R \le \min(R_0,\operatorname{dist} (0,\partial \Omega))$,  ${\boldsymbol{u}}$ admits a decomposition 
\[
{\boldsymbol{u}}={\boldsymbol{v}}+{\boldsymbol{w}} \quad \text{in }\, Q_R
\]
satisfying
\begin{align}
\label{0923.eq2}
(|W|^q)^{\frac{1}{q}}_{Q_R}&\le N\gamma^{\frac{1}{q\nu'}}(|U|^{q\nu} )^{\frac{1}{q\nu}}_{Q_R}+N(|F|^q)^{\frac{1}{q}}_{Q_R},\\
\label{0923.eq2a}
\|V\|_{L_\infty(Q_{R/4})}&\le N\gamma^{\frac{1}{q\nu'}}(|U|^{q\nu})^{\frac{1}{q\nu}}_{Q_R}+N(|F|^q)^{\frac{1}{q}}_{Q_R}+N(|U|^q)^{\frac{1}{q}}_{Q_R}.
\end{align}

\item
Suppose that Assumption \ref{0923.ass1} $(\gamma)$ holds at $0\in \partial \Omega$ with $\gamma\in \big(0,\frac{1}{6}\big)$.
Then for $R\in (0,R_0]$,  ${\boldsymbol{u}}$ admits a decomposition 
\[
{\boldsymbol{u}}={\boldsymbol{v}}+{\boldsymbol{w}} \quad \text{in }\, {\mathcal{C}}_R:=Q_R\cap \Omega_T
\]
satisfying
\begin{align}
\label{0923.eq2b}
(|W|^q)^{\frac{1}{q}}_{{\mathcal{C}}_R}&\le N\gamma^{\frac{1}{q\nu'}}(|U|^{q\nu} )^{\frac{1}{q\nu}}_{{\mathcal{C}}_R}+N(|F|^q)^{\frac{1}{q}}_{{\mathcal{C}}_R},\\
\label{0923.eq2c}
\|V\|_{L_\infty({\mathcal{C}}_{R/6})}&\le N\gamma^{\frac{1}{q\nu'}}(|U|^{q\nu})^{\frac{1}{q\nu}}_{{\mathcal{C}}_R}+N(|F|^q)^{\frac{1}{q}}_{{\mathcal{C}}_R}+N(U^q)^{\frac{1}{q}}_{{\mathcal{C}}_R}.
\end{align}
\end{enumerate}
Here, the constant $N$ depends on $d$, $m$, $n$, $\delta$, $\nu$, and $q$, and
$V$ and $W$ are defined in the same way as $U$ in \eqref{eq0223_01} with ${\boldsymbol{u}}$ replaced by ${\boldsymbol{v}}$ and ${\boldsymbol{w}}$, respectively.
\end{lemma}

\begin{proof}
The proof is an adaptation of that of \cite[Lemma 8.3]{MR2835999}.
We may assume that $A^{\alpha\beta}$ and ${\boldsymbol{f}}_\alpha$ are infinitely differentiable.
If not, we take the standard mollifications and prove the estimates for the mollifications.
Then we can pass to the limit because the constants $N$ in the estimates are independent of the regularity of $A^{\alpha\beta}$ and ${\boldsymbol{f}}_\alpha$.
We further assume $\lambda>0$.
Otherwise, we add the term $\varepsilon {\boldsymbol{u}}$, $\varepsilon>0$, to both sides of \eqref{1004.eq1} and obtain the estimates for the modified system.
Then we let $\varepsilon \to 0^+$.

To prove the assertion $(a)$, we define 
\[
{\mathcal{L}}_0{\boldsymbol{u}}=\sum_{|\alpha|=|\beta|=m}D^\alpha(A^{\alpha\beta}_0D^\beta {\boldsymbol{u}}), 
\]
where 
\[
A^{\alpha\beta}_0(t)={\operatorname{\,\,\text{\bf--}\kern-.98em\DOTSI\intop\ilimits@\!\!}}_{B_R}A^{\alpha\beta}(t,y)\,dy.
\]
Let $\varphi$ be a smooth function on ${\mathbb{R}}^{d+1}$ satisfying
$$
0\le \varphi\le 1, \quad \operatorname{supp} \varphi\subset {\mathcal{B}}_R, \quad \text{and}\quad \varphi\equiv 1\text{ on } Q_{R/2}.
$$
By \cite[Theorem 1]{MR2771670}, there exists a unique ${\boldsymbol{w}}\in {\mathcal{H}}^m_q({\mathbb{R}}^d_{0})$ satisfying
$$
{\boldsymbol{w}}_t+(-1)^m{\mathcal{L}}_0 {\boldsymbol{w}}+\lambda {\boldsymbol{w}}=(-1)^m\sum_{|\alpha|=|\beta|=m}D^\alpha\big(\varphi(A^{\alpha\beta}_0-A^{\alpha\beta})D^\beta {\boldsymbol{u}}\big)+\sum_{|\alpha| \le m}D^\alpha(\varphi {\boldsymbol{f}}_\alpha)
$$
in $ {\mathbb{R}}^d_0$, where as we recall ${\mathbb{R}}^d_0 = (-\infty,0) \times {\mathbb{R}}^d$, and 
$$		
\|W\|_{L_q({\mathbb{R}}^d_0)}\le N\sum_{|\alpha|=|\beta|=m}\|(A^{\alpha\beta}_0-A^{\alpha\beta})D^\beta{\boldsymbol{u}}\|_{L_q(Q_R)}
$$
$$
+N\sum_{|\alpha| \le m}\lambda^{\frac{|\alpha|}{2m}-\frac{1}{2}}\|{\boldsymbol{f}}_\alpha\|_{L_q(Q_R)},
$$
where $N=N(d,m,n,\delta,q)$.
This together with H\"older's inequality gives \eqref{0923.eq2}.
Since all functions and coefficients involved are infinitely differentiable, by the classical parabolic theory, ${\boldsymbol{w}}$ is infinitely differentiable.
Therefore, the function   ${\boldsymbol{v}}={\boldsymbol{u}}-{\boldsymbol{w}}$ is also infinitely differentiable, and it satisfies 
\[
{\boldsymbol{v}}_t+(-1)^m{\mathcal{L}}_0 {\boldsymbol{v}}+\lambda {\boldsymbol{v}}=0 \quad \text{in }\, Q_{R/2}.
\]
By Lemma \ref{0929.lem1} $(a)$ with scaling,  we obtain  
\begin{equation*}	
\|V\|_{L_\infty(Q_{R/4})}\le N(|V|^q)^{\frac{1}{q}}_{Q_R}\le N(|U|^q)^{\frac{1}{q}}_{Q_R}+N(|W|^q)^{\frac{1}{q}}_{Q_R}.
\end{equation*}
Thus, we obtain \eqref{0923.eq2a} by using the above inequality and \eqref{0923.eq2}.

Next, we prove the assertion $(b)$.
Without loss of generality, we may assume that Assumption \ref{0923.ass1} $(\gamma)$ holds at $0$ in the original $(t,x)$-coordinates.
Define ${\mathcal{L}}_0$ and $\varphi$ as above.
Consider a smooth function $\chi=\chi_R$ defined on ${\mathbb{R}}$ such that 
$$
\chi(x_1)\equiv 0 \text{ for } x_1\le \gamma R, \quad \chi(x_1)\equiv 1 \text{ for } x_1\ge 2\gamma R,
$$
$$
|D^k\chi| \le N(\gamma R)^{-k} \text{ for } k=1,\ldots, m.
$$
Then, $\hat{{\boldsymbol{u}}}(X)=\chi(x_1){\boldsymbol{u}}(X)$ along with all its derivatives vanishes on $Q_R\cap \{x_1\le \gamma R\}$ and satisfies in $Q_R^{\gamma+}:=Q_R\cap \{ x_1>\gamma R\}$,
$$
\hat{{\boldsymbol{u}}}_t+(-1)^m{\mathcal{L}}_0\hat{{\boldsymbol{u}}}+\lambda\hat{{\boldsymbol{u}}}=(-1)^m \sum_{|\alpha|=|\beta|=m}D^\alpha\big(\big(A_0^{\alpha\beta}-A^{\alpha\beta})D^\beta{\boldsymbol{u}}\big)
$$
$$
+\sum_{|\alpha| \le m}\chi D^\alpha{\boldsymbol{f}}_\alpha+(-1)^m{\boldsymbol{g}}+(-1)^m {\boldsymbol{h}},
$$
where we set
\[
{\boldsymbol{g}}={\mathcal{L}}_0((\chi-1){\boldsymbol{u}}) \quad \text{and}\quad {\boldsymbol{h}}=(1-\chi){\mathcal{L}}{\boldsymbol{u}}.
\]
Let $\hat{{\boldsymbol{w}}}$ be the unique $\mathring{\mathcal{H}}^m_q((-\infty,T) \times \{x:x_1>\gamma R\})$  solution of the problem  (see \cite[Theorem 3]{MR2771670}).
\begin{multline*}		
\hat{{\boldsymbol{w}}}_t+(-1)^m {\mathcal{L}}_0\hat{{\boldsymbol{w}}}+\lambda \hat{{\boldsymbol{w}}}=(-1)^m\sum_{|\alpha|=|\beta|=m}D^\alpha\big(\varphi(A_0^{\alpha\beta}-A^{\alpha\beta})D^\beta{\boldsymbol{u}}\big)\\
+\sum_{|\alpha| \le m}\chi D^\alpha(\varphi {\boldsymbol{f}}_\alpha)+(-1)^m\hat{{\boldsymbol{g}}}+(-1)^m\hat{{\boldsymbol{h}}}
\end{multline*}
in $(-\infty,T)\times \{x:x_1>\gamma R\}$, where 
\begin{align*}
&\hat{{\boldsymbol{g}}}=\sum_{|\alpha|=|\beta|=m}D^\alpha\big(A_0^{\alpha\beta}\varphi D^\beta((\chi-1){\boldsymbol{u}})\big),\\
&\hat{{\boldsymbol{h}}}=(1-\chi)\sum_{|\alpha|=|\beta|=m}D^\alpha(A^{\alpha\beta}\varphi D^\beta {\boldsymbol{u}}).
\end{align*}
By using the argument as in \cite[Lemma A.1]{arXiv:1603.07844v1}, we obtain 
\begin{equation}		\label{1007@eq1}
\sum_{k=0}^m \lambda^{\frac{1}{2}-\frac{k}{2m}}\big(I_{Q^{\gamma+}_R}|D^k\hat{{\boldsymbol{w}}}|^q\big)^{\frac{1}{q}}_{{\mathcal{C}}_R}\le N\gamma^{\frac{1}{q\nu'}}\left(|U|^{q\nu}\right)^{\frac{1}{q\nu}}_{{\mathcal{C}}_R}+N(|F|^q)^{\frac{1}{q}}_{{\mathcal{C}}_R}.
\end{equation}
We extend $\hat{{\boldsymbol{w}}}$ to be zero in ${\mathcal{C}}_R \setminus Q^{\gamma+}_R$, so that $\hat{{\boldsymbol{w}}}\in {\mathcal{H}}^m_q({\mathcal{C}}_R)$.
Set
\[
{\boldsymbol{w}}=\hat{{\boldsymbol{w}}}+(1-\chi){\boldsymbol{u}} \quad \text{and}\quad {\boldsymbol{v}}={\boldsymbol{u}}-{\boldsymbol{w}}.
\]
Then similar to (7.19) in \cite{MR2835999}, we deduce \eqref{0923.eq2b} from \eqref{1007@eq1}.
Moreover, we find that
${\boldsymbol{v}}\equiv 0$ in ${\mathcal{C}}_R\setminus Q^{\gamma+}_R$ and ${\boldsymbol{v}}$ satisfies
$$
\left\{
\begin{aligned}
{\boldsymbol{v}}_t+(-1)^m{\mathcal{L}}_0{\boldsymbol{v}}+\lambda{\boldsymbol{v}}=0 &\quad \text{in }\, Q_{R/2}\cap \{x_1>\gamma R\},\\
|{\boldsymbol{v}}|=\cdots=|D^{m-1}_1{\boldsymbol{v}}|=0 &\quad \text{on }\, Q_{R/2}\cap \{x_1=\gamma R\}.
\end{aligned}
\right.
$$
We write $X_0=(0,x_0)\in {\mathbb{R}}^{d+1}$, where $x_0=(\gamma R, 0,\ldots,0)\in {\mathbb{R}}^d$. 
Then we have 
\begin{align*}
(Q_{R/6}\cap \{x_1>\gamma R\}) &\subset (Q_{R/6}(X_0)\cap \{x_1>\gamma R\})\\
&\subset (Q_{R/3}(X_0)\cap \{x_1>\gamma R\})\subset (Q_{R/2}\cap \{x_1>\gamma R\}).
\end{align*}
Therefore, by applying Lemma \ref{0929.lem1} $(b)$ with scaling, we obtain 
\begin{align*}
\|V\|_{L_\infty({\mathcal{C}}_{R/6})}&=\|V\|_{L_\infty(Q_{R/6}\cap \{x_1>\gamma R\})}\\
&\le \|V\|_{L_\infty(Q_{R/6}(X_0)\cap \{x_1>\gamma R\})}\\
&\le N(|V|^q)^{1/q}_{Q_{R/3}(X_0)\cap \{x_1>\gamma R\}}\le N(|V|^q)^{1/q}_{{\mathcal{C}}_{R/2}},
\end{align*}
which together with \eqref{0923.eq2b} gives \eqref{0923.eq2c}.
\end{proof}

\section{Level set argument}		

In this section, we consider the operator ${\mathcal{L}}$ without lower order terms, i.e.,
\[
{\mathcal{L}} {\boldsymbol{u}}=\sum_{|\alpha|=|\beta|=m}D^\alpha(A^{\alpha\beta}D^\beta {\boldsymbol{u}}).
\]
We denote 
\begin{equation}		\label{160617@eq2}
{\mathcal{C}}_r(X)=Q_r(X)\cap \Omega_T.
\end{equation}
If ${\mathcal{X}}$ is a space of homogeneous type in ${\mathbb{R}}^{d+1}$, then since ${\mathcal{X}}$ is open in ${\mathbb{R}}^{d+1}$ and we use the parabolic distance, we see that
$$
{\mathcal{B}}_r^{\mathcal{X}}(X)={\mathcal{B}}_r(X)\cap {\mathcal{X}},
$$
where, as we recall, ${\mathcal{B}}_r^{\mathcal{X}}(X)$ is a ball in ${\mathcal{X}}$ defined in \eqref{eq0624_01}.

For a function $f$ on ${\mathcal{X}}$, we define its maximal function ${\mathcal{M}} f$ by \eqref{160617@eq1}.
We also denote for $s>0$,  $\nu\in (1,\infty)$, $\nu'=\nu/(\nu-1)$, and $q\in (1,\infty)$ that 
\begin{equation}		\label{1008@eq1}
\begin{aligned}
{\mathcal{E}}_1(s)&=\{X\in \Omega_T: |U|(X)>s\},\\
{\mathcal{E}}_2(s)&=\big\{X\in  \Omega_T:\gamma^{-\frac{1}{q\nu'}}({\mathcal{M}}(I_{\Omega_T}|F|^{q})(X))^{\frac{1}{q}}+({\mathcal{M}}(I_{\Omega_T}|U|^{q\nu})(X))^{\frac{1}{q\nu}}>s\big\},
\end{aligned}
\end{equation}
where $U$ and $F$ are as in \eqref{eq0223_01}.

\begin{lemma}		\label{1015@lem1}
Let $T \in (-\infty,\infty]$, $\nu \in (1,\infty)$, $\nu' = \nu/(\nu-1)$, $\Omega$ be a domain in ${\mathbb{R}}^d$,  and $\Omega_T\subseteq {\mathcal{X}}$, where ${\mathcal{X}}$ is  a space of homogeneous type in ${\mathbb{R}}^{d+1}$ with a doubling constant $K_2$.
Let 
$p_0,q\in (1,\infty)$, $\hat{K}_0\ge 1$, $w\in A_{p_0}({\mathcal{X}})$, and $[w]_{A_{p_0}}\le \hat{K}_0$.
Suppose that  Assumption \ref{0923.ass1} $(\gamma)$ holds with $\gamma\in \big(0,\frac{1}{6}\big)$, and ${\boldsymbol{u}}\in C^\infty_0((-\infty,T]\times \Omega)$ satisfies
\begin{equation*}		
{\boldsymbol{u}}_t+(-1)^m{\mathcal{L}}{\boldsymbol{u}}+\lambda{\boldsymbol{u}}=\sum_{|\alpha| \le m}D^\alpha {\boldsymbol{f}}_\alpha \quad \text{in }\,  \Omega_T,
\end{equation*}
where $\lambda>0$ and ${\boldsymbol{f}}_\alpha\in L_{q,\operatorname{loc}}((-\infty,T] \times \overline{\Omega})$, $|\alpha|\le m$.
Then there exists a constant $\kappa=\kappa(d,m,n,\delta,\nu,q)>1$ such that the following holds: for $X\in (-\infty,T]\times\overline{\Omega}$, $R\in (0,R_0]$, and $s>0$,
if
\begin{equation*}		
N_1\gamma^{\frac{\mu_1}{\nu'}}\le\frac{w\big({\mathcal{B}}_{R/64}^{\mathcal{X}}(X)\cap {\mathcal{E}}_1(\kappa s)\big)}{w\big({\mathcal{B}}_{R/64}^{\mathcal{X}}(X)\big)},
\end{equation*}
where $(N_1,\mu_1)=(N_1,\mu_1)(p_0,\hat{K}_0,K_2)$ are from Lemma \ref{1015@lem3}, then we have 
\begin{equation}		\label{1015@eq1a}
{\mathcal{C}}_{R/64}(X) \subset {\mathcal{E}}_2(s).
\end{equation}
\end{lemma}

\begin{proof}
Let  $X=(t,x)\in (-\infty,T]\times \overline{\Omega}$, $R\in (0,R_0]$, and $s>0$.
Owing to Lemma \ref{1015@lem3}, we have 
$$
\frac{w\big({\mathcal{B}}_{R/64}^{\mathcal{X}}(X)\cap {\mathcal{E}}_1(\kappa s)\big)}{w\big({\mathcal{B}}_{R/64}^{\mathcal{X}}(X)\big)}\le N_1\left(\frac{|{\mathcal{B}}_{R/64}^{\mathcal{X}}(X) \cap {\mathcal{E}}_1(\kappa s)|}{|{\mathcal{B}}_{R/64}^{\mathcal{X}}(X)|}\right)^{\mu_1}.
$$
Therefore, it suffices to claim that \eqref{1015@eq1a} holds, provided that 
\begin{equation}		\label{1015@eq1}		
\gamma^{\frac{1}{\nu'}}<\frac{|{\mathcal{B}}_{R/64}^{\mathcal{X}}(X)\cap {\mathcal{E}}_1(\kappa s)|}{|{\mathcal{B}}_{R/64}^{\mathcal{X}}(X)|}.
\end{equation}
By dividing ${\boldsymbol{u}}$ and ${\boldsymbol{f}}_\alpha$ by $s$, we may assume $s=1$.
We prove the claim by contradiction.
Suppose that 
\[
\gamma^{-\frac{1}{q\nu'}}({\mathcal{M}}(I_{\Omega_T}|F|^{q})(Z))^{\frac{1}{q}}+({\mathcal{M}}(I_{\Omega_T}|U|^{q\nu})(Z))^{\frac{1}{q\nu}}\le 1
\]
for some $Z\in {\mathcal{C}}_{R/64}(X)$.
Set 
$$
T^*=\min \big(t+(R/64)^{2m},T\big) \quad \text{and}\quad X^*=(T^*,x)\in (-\infty,T]\times \overline{\Omega}.
$$
If $\operatorname{dist} (x,\partial \Omega)\ge R/8$,  we have 
\[
Z\in {\mathcal{C}}_{R/64}(X)\subset Q_{R/32}(X^*)\subset Q_{R/8}(X^*)\subset\Omega_T.
\]
By Lemma \ref{0929-lem1} (a), ${\boldsymbol{u}}$ admits a decomposition ${\boldsymbol{u}}={\boldsymbol{v}}+{\boldsymbol{w}}$ in $Q_{R/8}(X^*)$ with the estimates
\[
(|W|^{q})_{Q_{R/8}(X^*)}\le N_2\gamma^{\frac{1}{\nu'}} \quad \text{and}\quad \|V\|_{L_\infty(Q_{R/32}(X^*))}\le N_2,
\]
where $N_2=N_2(d,m,n,\delta,\nu,q)$.
From this together with Chebyshev's inequality, it follows that 
\begin{align}
\nonumber
|{\mathcal{B}}_{R/64}^{\mathcal{X}}(X)\cap {\mathcal{E}}_1(\kappa)|&\le \{Y\in Q_{R/32}(X^*):|U|(Y)>\kappa\}\\
\nonumber
&\le \{Y\in Q_{R/32}(X^*):|W|(Y)>\kappa-N_2\}\\
\nonumber
&\le \int_{Q_{R/32}(X^*)}\left|\frac{W}{\kappa-N_2}\right|^{q} \ dY\le \frac{N_2\gamma^{\frac{1}{\nu'}}|Q_{R/8}(X^*)|}{|\kappa-N_2|^{q}}.\\
\label{160804@eq3}
&\le \frac{N_2'\gamma^{\frac{1}{\nu'}}|{\mathcal{B}}_{R/64}^{\mathcal{X}} (X)|}{|\kappa-N_2|^{q}},
\end{align}
where $N_2'=N_2'(d,m,n,\delta,\nu,q)$
and the last inequality is due to 
$$
|Q_{R/8}(X^*)|\le N(d)|Q_{R/64}(X)|\le N(d)|{\mathcal{B}}^{\mathcal{X}}_{R/64}(X)|.
$$
The estimate \eqref{160804@eq3} contradicts with \eqref{1015@eq1} if we choose a sufficiently large $\kappa$.

On the other hand, if $\operatorname{dist} (x, \partial \Omega)<R/8$,  we take $x_0\in \partial \Omega$ such that $\operatorname{dist} (x,\partial \Omega)=|x-x_0|$.
Note that
\[
Z\in {\mathcal{C}}_{R/64}(X)\subset {\mathcal{C}}_{R/6}(X_0^*)\subset {\mathcal{C}}_R(X_0^*), \quad X_0^*=(T^*,x_0).
\]
By Lemma \ref{0929-lem1} (b), ${\boldsymbol{u}}$ admits a decomposition ${\boldsymbol{u}}={\boldsymbol{v}}+{\boldsymbol{w}}$ in ${\mathcal{C}}_R(X_0^*)$ with the estimates
\[
(|W|^{q})_{{\mathcal{C}}_R(X_0^*)}\le N_3\gamma^{\frac{1}{\nu'}} \quad \text{and}\quad \|V\|_{{\mathcal{C}}_{R/6}(X_0^*)}\le N_3,
\]
where $N_3=N_3(d,m,n,\delta,\nu,q)$.
Therefore, we obtain 
\begin{align}
\nonumber
|{\mathcal{B}}_{R/64}^{\mathcal{X}}(X)\cap {\mathcal{E}}_1(\kappa)|&\le\{Y\in {\mathcal{C}}_{R/6}(X^*_0):|U|(Y)>\kappa\}\\
\nonumber
&\le \{Y\in {\mathcal{C}}_{R/6}(X^*_0):|W|(Y)>\kappa-N_3\}\\
\label{160804@eq6}
&\le \int_{{\mathcal{C}}_{R/6}(X_0^*)}\left|\frac{W}{\kappa-N_3}\right|^q\ dY\le \frac{N_3\gamma^{\frac{1}{\nu'}}|{\mathcal{C}}_R(X_0^*)|}{|\kappa-N_3|^q}.
\end{align}
Using the fact that 
\begin{equation}		\label{160804@eq1}
|\Omega\cap B_r(y)|\ge N(d)r^d, \quad \forall y\in \overline{\Omega}, \quad \forall r\in (0,R_0],
\end{equation}
we have 
$$
|{\mathcal{C}}_R(X^*_0)|\le N(d)R^{d+2m} \le N(d)|{\mathcal{B}}_{R/64}^{\mathcal{X}}(X)|.
$$
Thus, from \eqref{160804@eq6}, we obtain that 
$$
|{\mathcal{B}}_{R/64}^{\mathcal{X}}(X)\cap {\mathcal{E}}_1(\kappa)|\le \frac{N_3'\gamma^{\frac{1}{\nu'}}|{\mathcal{B}}_{R/64}^{\mathcal{X}}(X)|}{|\kappa-N_3|^q},
$$
where $N_3'=N_3'(d,m,n,\delta,\nu,q)$, which contradicts with \eqref{1015@eq1} if we choose a sufficiently large $\kappa$.
Thus, the claim is proved.
\end{proof}

\begin{lemma}		\label{1016@lem5}
Let $T\in (-\infty,\infty]$, $\Omega$  be a domain  in ${\mathbb{R}}^d$, and $\Omega_T\subseteq {\mathcal{X}}$, where 
${\mathcal{X}}$ is  a space of homogeneous type in ${\mathbb{R}}^{d+1}$ with a doubling constant $K_2$.
Let $p\in (1,\infty)$, $K_0\ge 1$, $w\in A_p({\mathcal{X}})$, and   $[w]_{A_p}\le K_0$.
Then there exists a constant 
\[
\gamma=\gamma(d,m,n,\delta,p, K_0,K_2)\in (0,1/6)
\]
such that, under  Assumption \ref{0923.ass1} $(\gamma)$, the following holds:
if ${\boldsymbol{u}}\in C^\infty_0((-\infty,T]\times \Omega)$ vanishes outside $Q_{\gamma R_0}(X_0)$, where $X_0\in {\mathbb{R}}^{d+1}$, and satisfies 
\[
{\boldsymbol{u}}+(-1)^m {\mathcal{L}}{\boldsymbol{u}}+\lambda{\boldsymbol{u}}=\sum_{|\alpha| \le m}D^\alpha {\boldsymbol{f}}_\alpha \quad \text{in }\, \Omega_T,
\]
where $\lambda>0$ and ${\boldsymbol{f}}_\alpha\in L_{p,w}( \Omega_T)$, then we have 
\begin{equation}		\label{1013@eq2}
\|U\|_{L_{p,w}( \Omega_T)}\le N\|F\|_{L_{p,w}(\Omega_T)},
\end{equation}
where  $N=N(d,m,n,\delta,p,K_0,K_2)$.
\end{lemma}

\begin{proof}
By Lemma \ref{1016@lem2}, we see that 
$$
w\in A_{p_0}({\mathcal{X}}) \quad \text{and}\quad [w]_{A_{p_0}}\le \hat{K}_0
$$
for some constants $p_0\in (1,p)$ and $\hat{K}_0\ge1$, depending only on $p$, $K_0$, and $K_2$.
We denote
$$
q=\frac{p}{p_0}, \quad \nu=\frac{p_0+1}{p_0}, \quad \nu'=\frac{\nu}{\nu-1}=p_0+1,
$$
and  let $(N_1,\mu_1)=(N_1,\mu_1)(p_0,\hat{K}_0,K_2)$ and  $\kappa=\kappa(d,m,n,\delta,\nu,q)$ be constants  in Lemma \ref{1015@lem3} and Lemma \ref{1015@lem1}, respectively.
We recall the notation \eqref{160617@eq2} and \eqref{1008@eq1}, and we remark that ${\boldsymbol{f}}_\alpha\in L_{q,\operatorname{loc}}((-\infty,T]\times \overline{\Omega})$.
Indeed, by H\"older's inequality, we obtain for $X\in {\mathcal{X}}$ and $R>0$ that   
$$
\int_{{\mathcal{B}}_R^{\mathcal{X}}(X)} |{\boldsymbol{f}}_\alpha|^qI_{\Omega_T}\,dX\le \left(\int_{{\mathcal{B}}_R^{\mathcal{X}}(X)}|{\boldsymbol{f}}_\alpha|^pI_{\Omega_T}w\,dX\right)^{\frac{1}{p_0}}\left(\int_{{\mathcal{B}}_R^{\mathcal{X}}(X)}w^{-\frac{1}{p_0-1}}\,dX\right)^{\frac{p_0-1}{p_0}}.
$$
Let  $\gamma\in (0,1/6)$ be a constant satisfying 
$$
N_1\gamma^{\frac{\mu_1}{\nu'}}<1.
$$
Since $\operatorname{supp}{\boldsymbol{u}}\subset Q_{\gamma R_0}(X_0)$, 
it suffices to prove the lemma when 
$$
\operatorname{supp}{\boldsymbol{u}}\subset {\mathcal{B}}_{2\gamma R_0}(X_0), \quad  X_0\in (-\infty,T]\times \overline{\Omega}.
$$
We first claim that for any $X\in \Omega_T$ and $R\ge R_0/64$, we have 
\begin{equation}		\label{1012@e2}
w({\mathcal{E}}_1(\kappa s)\cap {\mathcal{B}}_{R}^{\mathcal{X}}(X))<N_1 \gamma^{\frac{\mu_1}{\nu'}}w({\mathcal{B}}^{\mathcal{X}}_{R}(X)),
\end{equation}
provided that 
$$
s>s_0:= \frac{N_2}{N_1\gamma^{\mu_1/\nu'}\kappa w({\mathcal{B}}_{R_0/3}^{\mathcal{X}}(X_0))^{1/p_0}}\|U\|_{L_{p_0,w}(\Omega_T)}, \quad N_2=N_2(K_2).
$$
Since $\operatorname{supp} {\boldsymbol{u}}\subset {\mathcal{B}}_{2\gamma R_0}(X_0)$, we only need to consider the case when ${\mathcal{B}}_{R}^{\mathcal{X}}(X)\cap {\mathcal{B}}_{2\gamma R_0}(X_0)\neq \emptyset$.
In this case, we have
$$
{\mathcal{B}}_{R_0/3}^{\mathcal{X}}(X_0)\subset {\mathcal{B}}_{45R}^{\mathcal{X}}(X), 
$$
and thus, by H\"older's inequality and the doubling property of ${\mathcal{X}}$, we obtain
\begin{align*}
w({\mathcal{E}}_1(\kappa s)\cap {\mathcal{B}}_{R}^{\mathcal{X}}(X))&\le \frac{1}{\kappa s}\int_{{\mathcal{B}}_{45R}^{\mathcal{X}}(X)}I_{\Omega_T}|U|w\,dY\\
&\le \frac{1}{\kappa s}w({\mathcal{B}}_{45R}^{\mathcal{X}}(X))^{1-\frac{1}{p_0}}\|U\|_{L_{p_0,w}(\Omega_T)}\\
&\le \frac{N_2}{\kappa s}\frac{w({\mathcal{B}}_{R}^{\mathcal{X}}(X))}{w({\mathcal{B}}^{\mathcal{X}}_{R_0/3}(X_0))^{1/p_0}}\|U\|_{L_{p_0,w}(\Omega_T)},
\end{align*}
where $N_2=N_2(K_2)$, which implies \eqref{1012@e2}.
Therefore, by using \eqref{1012@e2}, Lemma \ref{1015@lem1}, and a result from measure theory on the ``crawling of ink spots" which can be found in \cite[Section 2]{MR563790}, we have the following inequality;
\begin{equation}		\label{160804@eq9}
w({\mathcal{E}}_1(\kappa s))\le N\gamma^{\frac{\mu_1}{\nu'}}w({\mathcal{E}}_2(s)), \quad \forall s>s_0,
\end{equation}
where $N=N(d,m,p,K_0,K_2)$.
We provide a detailed proof of \eqref{160804@eq9} in the appendix for the reader's convenience.
In addition, see \cite{MR3467697}.

By \eqref{160804@eq9}, we obtain
\begin{align}
\nonumber
\|U\|_{L_{p,w}(\Omega_T)}^p&=p\int_0^\infty w({\mathcal{E}}_1(s))s^{p-1}\,ds =p\kappa^p\int_0^\infty w({\mathcal{E}}_1(\kappa s))s^{p-1}\,ds\\
\nonumber
&\le N\int_0^{s_0} w({\mathcal{E}}_1(\kappa s))s^{p-1}\,ds+N\gamma^{\frac{\mu_1}{\nu'}}\int_{0}^\infty w({\mathcal{E}}_2( s))s^{p-1}\,ds\\
\label{1013@eq1b}
&:=I_1+I_2,
\end{align}
where $N=N(d,m,n,\delta,\nu,p,K_0, K_2)$.
Notice from Chebyshev's inequality that
$$
w({\mathcal{E}}_1(\kappa s))\le (\kappa s)^{-p_0}\|U\|_{L_{p_0,w}(\Omega_T)}^{p_0}, \quad \forall s>0.
$$
Using this together with H\"older's inequality and Lemma \ref{1015@lem3},  we have 
\begin{align*}
I_1&\le N \left(\int_0^{s_0}s^{p-p_0-1}\,ds\right)\|U\|^{p_0}_{L_{p_0,w}(\Omega_T)}\\
&\le N\gamma^{\frac{\mu_1}{\nu'}(p_0-p)}\left(\frac{w({\mathcal{B}}^{\mathcal{X}}_{2\gamma R_0}(X_0))}{w({\mathcal{B}}^{\mathcal{X}}_{R_0/3}(X_0))}\right)^{\frac{p-p_0}{p_0}}\|U\|_{L_{p,w}(\Omega_T)}^p\\
&\le N\gamma^{\frac{\mu_1}{\nu'}(p_0-p)}\left(\frac{|{\mathcal{B}}^{\mathcal{X}}_{2\gamma R_0}(X_0)|}{|{\mathcal{B}}^{\mathcal{X}}_{R_0/3}(X_0)|}\right)^{\mu_1\frac{p-p_0}{p_0}}\|U\|_{L_{p,w}(\Omega_T)}^p.
\end{align*}
Since (use \eqref{160804@eq1})
$$
\frac{|{\mathcal{B}}^{\mathcal{X}}_{2\gamma R_0}(X_0)|}{|{\mathcal{B}}^{\mathcal{X}}_{R_0/3}(X_0)|}\le \frac{|{\mathcal{B}}_{2\gamma R_0}(X_0)|}{|{\mathcal{C}}_{R_0/3}(X_0)|}\le N\gamma^{d+2m},
$$
we have
$$
I_1\le N\gamma^{\mu_1(p-p_0)\left(\frac{d+2m}{p_0}-\frac{1}{\nu'}\right)}\|U\|^p_{L_{p,w}(\Omega_T)}.
$$
Note that
\[
I_2\le N\gamma^{\frac{\mu_1}{\nu'}}\left(\gamma^{-\frac{p}{q\nu'}}\big\|({\mathcal{M}}(I_{\Omega_T}|F|^q))^{\frac{1}{q}}\big\|^p_{L_{p,w}({\mathcal{X}})}+\big\|({\mathcal{M}}(I_{\Omega_T}|U|^{q\nu})^{\frac{1}{q\nu}}\big\|^p_{L_{p,w}({\mathcal{X}})}\right).
\]
Therefore, we obtain by Hardy-Littlewood maximal function theorem (Theorem \ref{1008@thm5}) that 
\begin{equation}		\label{1016@e1a}
I_2\le N\gamma^{\frac{1}{\nu'}\left(\mu_1-\frac{p}{q}\right)}\|F\|_{L_{p,w}(\Omega_T)}^p+N\gamma^{\frac{\mu_1}{\nu'}}\|U\|_{L_{p,w}(\Omega_T)}^p.
\end{equation}
Finally, by combining \eqref{1013@eq1b}--\eqref{1016@e1a},  and then, choosing a sufficiently small $\gamma$, we conclude \eqref{1013@eq2}.
\end{proof}

\section{Proofs of main theorems}	\label{160825@sec1}

We begin with the proof of Theorem \ref{1008@thm1}, which is about a priori weighted $L_p$-estimates of solutions.

\begin{proof}[Proof of Theorem \ref{1008@thm1}]
By moving all the lower-order terms to the right-hand side of the system, we may assume that  all the lower order coefficients $A^{\alpha\beta}$, $|\alpha|+|\beta|<2m$, are zero.
Then we prove the estimate \eqref{1017@e3a}  using Lemma \ref{1016@lem5} and the standard partition of unity argument.
The  details are omitted.
\end{proof}

We now turn to Theorem \ref{160621@thm1}, which is about the solvability of higher order parabolic systems in  weighted Sobolev spaces $\mathring{\mathcal{H}}^m_{p,w}(\Omega_T)$.
The proof of Theorem \ref{160621@thm1} is based on  a priori  estimates with $A_p$ weights (see Theorem \ref{1008@thm1}) and Lemma \ref{160612@lem1} below, where we show  the solvability in unweighted Sobolev spaces of the system on a Reifenberg flat domain.
We remark that if ${\boldsymbol{u}}$ belongs to $\mathring{\mathcal{H}}^m_{p_1}(\Omega_T)$ with a sufficiently large $p_1$ and has a bounded support, then ${\boldsymbol{u}}$ belongs to a weighted Sobolev space $\mathring{\mathcal{H}}^m_{p,w}(\Omega_T)$. 
Thus, to prove the solvability of the system in $\mathring{\mathcal{H}}^m_{p,w}(\Omega_T)$, we  construct a sequence  of   compactly supported solutions ${\boldsymbol{u}}_k\in \mathring{\mathcal{H}}^m_{p_1}(\Omega_T)$  of the system with bounded right-hand side. 
It follows that ${\boldsymbol{u}}_k\in \mathring{\mathcal{H}}^m_{p,w}(\Omega_T)$ and the sequence converges to the solution of the system in the weighted Sobolev space $\mathring{\mathcal{H}}^m_{p,w}(\Omega_T)$.
See the proof of Theorem \ref{160621@thm1} below.

\begin{lemma}		\label{160612@lem1}
Let $\Omega$ be a domain in ${\mathbb{R}}^d$, $T\in (-\infty,\infty]$, $p_1\in (1,\infty)$, and ${\boldsymbol{f}}_\alpha\in L_{p_1}( \Omega_T)$, $|\alpha|\le m$.
Then there exist constants 
\begin{align*}
&\gamma_1=\gamma_1(d,m,n,\delta,p_1)\in (0,1/4),\\
&\lambda_1=\lambda_1(d,m,n,\delta,p_1, R_0,K)>0
\end{align*}
such that, under Assumption \ref{0923.ass1} $(\gamma_1)$,
for any $\lambda\ge \lambda_1$, there exists a unique  ${\boldsymbol{u}}\in \mathring{\mathcal{H}}_{p_1}^m(\Omega_T)$ satisfying 
\begin{equation}		\label{160612@eq1}
{\boldsymbol{u}}_t+(-1)^m{\mathcal{L}}{\boldsymbol{u}}+\lambda{\boldsymbol{u}}=\sum_{|\alpha| \le m}D^\alpha {\boldsymbol{f}}_\alpha \quad \text{in }\, \Omega_T.
\end{equation}
Moreover, ${\boldsymbol{u}}$ satisfies 
\begin{equation}		\label{160612@eq2}
\sum_{|\alpha| \le m}\lambda^{1-\frac{|\alpha|}{2m}}\|D^\alpha {\boldsymbol{u}}\|_{L_{p_1}(\Omega_T)}\le N\sum_{|\alpha| \le m}\lambda^{\frac{|\alpha|}{2m}}\|{\boldsymbol{f}}_\alpha\|_{L_{p_1}(\Omega_T)},
\end{equation}
where $N=N(d,m,n,\delta,p_1)$.
Furthermore, if ${\boldsymbol{f}}_\alpha\equiv 0$ in $\Omega_S$, where $S\in (-\infty,T)$, then the solution ${\boldsymbol{u}}$ is also zero in $\Omega_S$.
\end{lemma}

\begin{proof}
By using Theorem \ref{1008@thm1} with $w\equiv 1$ and the method of continuity, we easily obtain a unique ${\boldsymbol{u}}\in \mathring{\mathcal{H}}^m_{p_1}(\Omega_T)$ satisfying \eqref{160612@eq1} and \eqref{160612@eq2}.
In the case when ${\boldsymbol{f}}_\alpha \equiv 0$ in $\Omega_S$ with $S\in (-\infty,T)$,  
since ${\boldsymbol{u}}$ satisfies 
$$
{\boldsymbol{u}}_t+(-1)^m{\mathcal{L}}{\boldsymbol{u}}+\lambda {\boldsymbol{u}}=0 \quad \text{in }\, \Omega_S,
$$
by the uniqueness (or \eqref{160612@eq2}), ${\boldsymbol{u}}\equiv 0$ for $t<S$.
\end{proof}

We now prove Theorem \ref{160621@thm1}.

\begin{proof}[Proof of Theorem \ref{160621@thm1}]
{\bf{Case 1}}: $|\Omega|<\infty$.
Owing to Lemma \ref{160621@lem1}, we may assume that ${\boldsymbol{f}}_\alpha\in L_{p,w}(\Omega_T)\cap L_\infty(\Omega_T)$, $|\alpha|\le m$.
We first prove the theorem for $T<\infty$.
Since $|\Omega|<\infty$, by Lemma \ref{1016@lem1}, there exists $\mu_0=\mu_0(p,K_0,K_2)>1$ such that 
\begin{equation}		\label{160703@eq1}
w\in L_{\mu_0}((a,b)\times \Omega) \quad \text{for any }\, a,b\in(-\infty,T], \quad  a<b.
\end{equation}
Denote 
$$
p_1=\frac{p\mu_0}{\mu_0-1}, \quad \bar{\gamma}=\min(\gamma, \gamma_1), \quad \bar{\lambda}_0=\min(\lambda_0,\lambda_1),
$$
where $\gamma$ and $\lambda_0$ are constants in Theorem \ref{1008@thm1}, and $\gamma_1$ and $\lambda_1$ are constants in Lemma \ref{160612@lem1}.
Fix $\lambda\ge \bar{\lambda}_0$. 
Then by Lemma \ref{160612@lem1}, for any  positive integer $k$ satisfying $-k<T$, there exists a unique ${\boldsymbol{u}}_k\in \mathring{\mathcal{H}}^m_{p_1}(\Omega_T)$ satisfying 
\begin{equation}		\label{160804@Eq1}
({\boldsymbol{u}}_k)_t+(-1)^m{\mathcal{L}}{\boldsymbol{u}}_k+\lambda{\boldsymbol{u}}_k=\sum_{|\alpha| \le m}D^\alpha (I_{(-k,T)\times \Omega}{\boldsymbol{f}}_\alpha) \quad \text{in }\,  \Omega_T.
\end{equation}
Moreover, it holds that  ${\boldsymbol{u}}_k\equiv 0$ in $\Omega_{-k}$ and 
$$
\sum_{|\alpha|\le m}\lambda^{1-\frac{|\alpha|}{2m}}\|D^\alpha {\boldsymbol{u}}_k\|_{L_{p_1}(\Omega_T)}=\sum_{|\alpha|\le m}\lambda^{1-\frac{|\alpha|}{2m}}\|D^\alpha {\boldsymbol{u}}_k\|_{L_{p_1}((-k,T)\times \Omega)}
$$
$$
\le N\sum_{|\alpha|\le m}\lambda^{\frac{|\alpha|}{2m}}\|{\boldsymbol{f}}_\alpha\|_{L_{p_1}((-k,T)\times \Omega)}\le N\sum_{|\alpha|\le m}\lambda^{\frac{|\alpha|}{2m}}\|{\boldsymbol{f}}_\alpha\|_{L_{p_1}(\Omega_T)}.
$$
Using this together with \eqref{160703@eq1}, we have 
$$
\sum_{|\alpha|\le m}\|D^\alpha {\boldsymbol{u}}_k\|_{L_{p,w}(\Omega_T)}=\sum_{|\alpha|\le m}\|D^\alpha {\boldsymbol{u}}_k\|_{L_{p,w}((-k,T)\times \Omega)}
$$
\begin{equation}		\label{160918@eq1}
\le \sum_{|\alpha|\le m}\|D^\alpha {\boldsymbol{u}}_k\|_{L_{p_1}((-k,T)\times \Omega)}\|w\|_{L_{\mu_0}((-k,T)\times \Omega)}^{1/p}<\infty.
\end{equation}
Therefore, we obtain by \eqref{160804@Eq1} that $({\boldsymbol{u}}_k)_t$ belongs to ${\mathbb{H}}^{-m}_{p,w}(\Omega_T)$,
which implies that ${\boldsymbol{u}}_k\in \mathring{\mathcal{H}}^m_{p,w}(\Omega_T)$.
Observe that for $k>l$, 
$$
({\boldsymbol{u}}_k-{\boldsymbol{u}}_l)_t+(-1)^m{\mathcal{L}}({\boldsymbol{u}}_k-{\boldsymbol{u}}_l)+\lambda({\boldsymbol{u}}_k-{\boldsymbol{u}}_l)=\sum_{|\alpha|\le m}D^\alpha(I_{(-k,-l)\times \Omega}{\boldsymbol{f}}_\alpha) \quad \text{in }\, \Omega_T.
$$
By Theorem \ref{1008@thm1}, we have 
$$
\sum_{|\alpha|\le m}\|D^\alpha ({\boldsymbol{u}}_k-{\boldsymbol{u}}_l)\|_{L_{p,w}(\Omega_T)}\le N\sum_{|\alpha|\le m}\|{\boldsymbol{f}}_\alpha\|_{L_{p,w}((-k,-l)\times \Omega)}
$$
\begin{equation}		\label{160805@Eq8}
\le N\sum_{|\alpha|\le m}\|{\boldsymbol{f}}_\alpha\|_{L_{p,w}(\Omega_{-l})},
\end{equation}
where $N=N(d,m,n,\delta,p,K_0,K_2,\lambda)$.
Since the right-hand side of \eqref{160805@Eq8} tends to $0$ as $l\to \infty$, $\{{\boldsymbol{u}}_k\}$ is a Cauchy sequence in $\mathring{\mathcal{H}}^m_{p,w}(\Omega_T)$, and hence, there exists a function ${\boldsymbol{u}}\in \mathring{\mathcal{H}}^m_{p,w}(\Omega_T)$ such that 
${\boldsymbol{u}}_k \to {\boldsymbol{u}}$ in $ \mathring{\mathcal{H}}^m_{p,w}(\Omega_T)$ as $k\to \infty$.
This gives the existence of a solution.
The uniqueness of  solution
in the space  $\mathring{\mathcal{H}}^m_{p,w}(\Omega_T)$ is a simple consequence of \eqref{1017@e3a}.

For $T=\infty$, 
let ${\boldsymbol{u}}_{k}$ be the weak solution in $\mathring{\mathcal{H}}^m_{p,w}(\Omega_k)$ of 
$$
({\boldsymbol{u}}_k)_t+(-1)^m{\mathcal{L}} {\boldsymbol{u}}_k+\lambda{\boldsymbol{u}}_k=\sum_{|\alpha|\le m}D^\alpha ({\boldsymbol{f}}_\alpha ) \quad \text{in }\, \Omega_k
$$
and $\eta_k=\eta_k(t)$ be a smooth function on ${\mathbb{R}}$ satisfying
$$
0\le \eta_k\le 1, \quad \eta_k= 1  \, \text{ on } \, (-\infty,k/2), \quad \eta_k= 0 \, \text{ on }\, (k, \infty), \quad |(\eta_k)_t|\le \frac{4}{k}.
$$
Then ${\boldsymbol{v}}_k=\eta_k{\boldsymbol{u}}_k$
belongs to $\mathring{\mathcal{H}}^m_{p,w}(\Omega_\infty)$, and it satisfies 
$$
({\boldsymbol{v}}_k)_t+(-1)^m{\mathcal{L}} {\boldsymbol{v}}_k+\lambda {\boldsymbol{v}}_k=\sum_{|\alpha|\le m}D^\alpha(\eta_k {\boldsymbol{f}}_\alpha)+(\eta_k)_t{\boldsymbol{u}}_k \quad \text{in }\, \Omega_\infty.
$$
By repeating the same argument as in the proof of the case $T<\infty$, it is not difficult to see that $\{{\boldsymbol{v}}_k\}_{k=1}^\infty$ is a Cauchy sequence in $\mathring{\mathcal{H}}^m_{p,w}(\Omega_\infty)$, and its limit ${\boldsymbol{u}}\in \mathring{\mathcal{H}}^m_{p,w}(\Omega_\infty)$ is the solution of 
$$
{\boldsymbol{u}}_t+(-1)^m{\mathcal{L}}{\boldsymbol{u}}+\lambda{\boldsymbol{u}}=\sum_{|\alpha|\le m}D^\alpha f_\alpha \quad \text{in }\, \Omega_\infty.
$$

\noindent
{\bf{Case 2}}: $\Omega={\mathbb{R}}^d$ or $\Omega={\mathbb{R}}^d_+$.
We only consider the case when $\Omega={\mathbb{R}}^d$.
The case of $\Omega={\mathbb{R}}^d_+$ can be treated in a similar way.
Due to the method of continuity and the a priori estimate (see Theorem \ref{1008@thm1}), it suffices to prove the solvability of systems when the leading coefficients $A^{\alpha\beta}$, $|\alpha|=|\beta|=m$, are constants and all the lower order coefficients $A^{\alpha\beta}$, $|\alpha|+|\beta|<2m$, are zero.
For $k=1,2,3,\ldots$, denote
$$
{\mathcal{X}}_k=\{(t,x): (k^{2m}t,kx)\in {\mathcal{X}}\},
$$
$$
w_k(t,x)=w(k^{2m}t,kx), \quad \text{and}\quad  {\boldsymbol{f}}_\alpha^k(t,x)=\frac{k^{2m}}{k^{|\alpha|}}{\boldsymbol{f}}_\alpha(k^{2m}t, kx).
$$
One can easily show that ${\mathcal{X}}_k$ is a space of homogeneous type with the same doubling constant $K_2$ of ${\mathcal{X}}$, and $w_k\in A_p({\mathcal{X}}_k)$ with $[w_k]_{A_p}\le K_0$.

Below, we write $T_k=T/k^{2m}$ if $T<\infty$ and $T_k=\infty$ if $T=\infty$.
Since the ball $B_1$ is Reifenberg flat for any $\gamma>0$,
by using the result of {\bf{Case 1}}, there exists a constant
\begin{align*}
\bar\lambda_0=\bar\lambda_0(d,m,n,\delta,p,R_0,K_0,K_2)>0
\end{align*}
such that for any $\lambda\ge \bar{\lambda}_0$, there exists a unique ${\boldsymbol{v}}_k\in \mathring{\mathcal{H}}^m_{p,w_k}((B_1)_{T_k})$ satisfying
$$
({\boldsymbol{v}}_k)_t+(-1)^m{\mathcal{L}} {\boldsymbol{v}}_k+\lambda_k {\boldsymbol{v}}_k=\sum_{|\alpha|\le m}D^\alpha {\boldsymbol{f}}_\alpha^k \quad \text{in }\, (B_1)_{T_k},
$$
where $\lambda_k=\lambda k^{2m}$.
Moreover, it follows from \eqref{1017@e3a} that 
\begin{equation*}		
\sum_{|\alpha|\le m}\lambda_k^{1-\frac{|\alpha|}{2m}}\|D^\alpha {\boldsymbol{v}}_k\|_{L_{p,w_k}((B_1)_{T_k})}\le N_0\sum_{|\alpha|\le m}\lambda_k^{\frac{|\alpha|}{2m}}\|{\boldsymbol{f}}_\alpha^k\|_{L_{p,w_k}((B_1)_{T_k})},
\end{equation*}
where $N_0=N_0(d,m,n,\delta,p,K_0,K_2)$.
Define $\bar{{\boldsymbol{v}}}_k$ on $(B_k)_T$ by 
$$
\bar{{\boldsymbol{v}}}_k(t,x)={\boldsymbol{v}}_k\left(\frac{t}{k^{2m}}, \frac{x}{k}\right).
$$ 
We see that $\bar{{\boldsymbol{v}}}_k\in \mathring{\mathcal{H}}^m_{p,w}((B_k)_T)$,   
$$
(\bar{{\boldsymbol{v}}}_k)_t+(-1)^m{\mathcal{L}} \bar{{\boldsymbol{v}}}_k+\lambda \bar{{\boldsymbol{v}}}_k=\sum_{|\alpha|\le m}D^\alpha {\boldsymbol{f}}_\alpha \quad \text{in }\, (B_k)_{T},
$$
and
\begin{equation}		\label{160805@Eq1}
 \sum_{|\alpha|\le m}\lambda^{1-\frac{|\alpha|}{2m}}\|D^\alpha \bar{{\boldsymbol{v}}}_k\|_{L_{p,w}((B_k)_T)}
\le N_0\sum_{|\alpha|\le m}\lambda^{\frac{|\alpha|}{2m}}\|{\boldsymbol{f}}_\alpha\|_{L_{p,w}({\mathbb{R}}^{d}_T)}.
\end{equation}

Now let $\zeta_k=\zeta_k(x)$ be a smooth function on ${\mathbb{R}}^d$ satisfying 
$$
0\le \zeta_k\le 1, \quad \zeta\equiv 1 \text{ in }\, B_{k/2}, \quad \operatorname{supp}\zeta_k\subset B_k, \quad |D^\alpha\zeta|\le Nk^{-|\alpha|}.
$$
The function ${\boldsymbol{u}}_k=\zeta_k \bar{{\boldsymbol{v}}}_k$ satisfies 
\begin{equation}		\label{160805@Eq3}
({\boldsymbol{u}}_k)_t+(-1)^m {\mathcal{L}} {\boldsymbol{u}}_k+\lambda {\boldsymbol{u}}_k=\zeta_k\sum_{|\alpha|\le m}D^\alpha {\boldsymbol{f}}_\alpha+(-1)^m{\mathcal{L}}{\boldsymbol{u}}_k-(-1)^m\zeta_k{\mathcal{L}}\bar{{\boldsymbol{v}}}_k
\end{equation}
in ${\mathbb{R}}^d_T$.
We observe that for $k>l$,   
$$
({\boldsymbol{u}}_k-{\boldsymbol{u}}_l)_t+(-1)^m {\mathcal{L}}({\boldsymbol{u}}_k-{\boldsymbol{u}}_l)+\lambda ({\boldsymbol{u}}_k-{\boldsymbol{u}}_l)
$$
$$
=(\zeta_k-\zeta_l)\sum_{|\alpha|\le m}D^{\alpha}{\boldsymbol{f}}_\alpha+(-1)^m({\mathcal{L}} {\boldsymbol{u}}_k- \zeta_k {\mathcal{L}}\bar{{\boldsymbol{v}}}_k)-(-1)^m({\mathcal{L}} {\boldsymbol{u}}_l- \zeta_l {\mathcal{L}}\bar{{\boldsymbol{v}}}_l)
$$
in ${\mathbb{R}}^d_T$.
Therefore by applying Theorem \ref{1008@thm1} and \eqref{160805@Eq1}, we have 
$$
\sum_{|\alpha|\le m}\|D^\alpha ({\boldsymbol{u}}_k-{\boldsymbol{u}}_l)\|_{L_{p,w}({\mathbb{R}}^d_T)}\le N\sum_{|\alpha_1|\le |\alpha|\le m} \|D^{\alpha_1}(\zeta_k-\zeta_l){\boldsymbol{f}}_\alpha\|_{L_{p,w}({\mathbb{R}}^d_T)}
$$
$$
+N\sum_{\substack{|\alpha|\le m\\ 1\le |\alpha_1|\le m}}\|D^{\alpha_1}\zeta_k D^\alpha \bar{{\boldsymbol{v}}}_k\|_{L_{p,w}({\mathbb{R}}^d_T)}
+N\sum_{\substack{|\alpha|\le m\\ 1\le |\alpha_1|\le m}}\|D^{\alpha_1}\zeta_l D^\alpha \bar{{\boldsymbol{v}}}_l\|_{L_{p,w}({\mathbb{R}}^d_T)}
$$
\begin{equation}		\label{160805@Eq2}
\le N\sum_{|\alpha_1|\le |\alpha|\le m} \|D^{\alpha_1}(\zeta_k-\zeta_l){\boldsymbol{f}}_\alpha\|_{L_{p,w}({\mathbb{R}}^d_T)}
+\frac{N}{l}\sum_{|\alpha|\le m}\|{\boldsymbol{f}}_\alpha\|_{L_{p,w}({\mathbb{R}}^d_T)},
\end{equation}
where $N=N(d,m,n,\delta,p,R_0,K_0,K_2, \lambda)$.
Since the right-hand side of \eqref{160805@Eq2} tends to $0$ as $l\to \infty$, 
$\{{\boldsymbol{u}}_k\}_{k=1}^\infty$ is a Cauchy sequence in $\mathring{\mathcal{H}}^m_{p,w}({\mathbb{R}}^d_T)$, and thus, there exists a function ${\boldsymbol{u}}\in \mathring{\mathcal{H}}^m_{p,w}({\mathbb{R}}^d_T)$ such that ${\boldsymbol{u}}_k\to {\boldsymbol{u}}$ in $\mathring{\mathcal{H}}^m_{p,w}(\Omega_T)$.
From \eqref{160805@Eq3}, it is routine to check that ${\boldsymbol{u}}$ satisfies 
$$
{\boldsymbol{u}}_t+(-1)^m{\mathcal{L}}{\boldsymbol{u}}+\lambda {\boldsymbol{u}}=\sum_{|\alpha|\le m}D^{\alpha}{\boldsymbol{f}}_\alpha \quad \text{in }\, {\mathbb{R}}^d_T.
$$
Finally, the uniqueness is a simple consequence of \eqref{1017@e3a}.
The theorem is proved.
\end{proof}

We now turn to the proof of Theorem \ref{1016@thm1}.
The proof is based on the weighted $L_p$-estimates obtained in Theorem \ref{1008@thm1} and the following theorem, which is  a refined version of the extrapolation theorem. 
The well-known version of the theorem (see, for instance, \cite{MR2797562}) requires the inequality \eqref{eq0222_01} to hold for all $w \in A_p$.
However, the theorem below allows us to obtain \eqref{eq0222_02} for a given $w \in A_p$ by only checking the inequality \eqref{eq0222_01} for a subset (determined by $p,q,K_0, K_2$) of $A_p$.
This refinement is needed because the weighted $L_p$-estimate \eqref{1017@e3a} holds only for $w$ satisfying $[w]_{A_p}\le K_0$.
See the proof of Theorem \ref{1016@thm1} below.

\begin{theorem}[Extrapolation theorem]		\label{1017@@thm1}
Let  ${\mathcal{X}}$ be a space of homogeneous type in ${\mathbb{R}}^{d+1}$ or ${\mathbb{R}}^d$ with a doubling
constant $K_2$.
Let $p,\, q\in (1,\infty)$, $K_0 \ge 1$, $w\in A_q({\mathcal{X}})$, and $[w]_{A_q}\le K_0$.
Then there exists a constant ${\mathcal{K}}_0={\mathcal{K}}_0(p,q,K_0,K_2)\ge 1$ such that if
\begin{equation}
							\label{eq0222_01}
\|f\|_{L_{p,\tilde{w}}({\mathcal{X}})}\le N_0\|g\|_{L_{p,\tilde{w}}({\mathcal{X}})}
\end{equation}
for every $\tilde{w}\in A_p({\mathcal{X}})$ satisfying $[\tilde{w}]_{A_p}\le {\mathcal{K}}_0$, then we have 
\begin{equation}
							\label{eq0222_02}
\|f\|_{L_{q,w}({\mathcal{X}})}\le 4N_0\|g\|_{L_{q,w}({\mathcal{X}})}.
\end{equation}
\end{theorem}

\begin{proof}
See \cite[Theorem 2.6]{arXiv:1603.07844v1}.
\end{proof}

\begin{proof}[Proof of Theorem \ref{1016@thm1}]
Assume ${\boldsymbol{u}}\in C^\infty_0((-\infty,T]\times \Omega)$ satisfies
\[
{\boldsymbol{u}}_t+(-1)^m {\mathcal{L}}{\boldsymbol{u}}+\lambda{\boldsymbol{u}}=\sum_{|\alpha| \le m}D^\alpha {\boldsymbol{f}}_\alpha \quad \text{in }\, \Omega_T.
\]
Let $\tilde{w}_2\in A_p({\mathcal{X}}_2)$ with $[\tilde{w}_2]_{A_p}\le {\mathcal{K}}_0$, where  ${\mathcal{K}}_0={\mathcal{K}}_0(p,q,K_0,K_2'')\ge 1$ is the constant in Theorem \ref{1017@@thm1}.
Notice from Lemma \ref{160629@lem1} that  ${\mathcal{X}}_1\times {\mathcal{X}}_2$ is a space of homogeneous type in ${\mathbb{R}}^{d+1}$ with a doubling constant $K_2=K_2(K_2'K_2'')$, and
\[
\tilde{w}=w_1\tilde{w}_2\in A_p({\mathcal{X}}_1\times {\mathcal{X}}_2) \quad \text{with}\quad [\tilde{w}]_{A_p}\le \tilde{K}_0,
\]
where $\tilde{K}_0=\tilde{K}_0(p,q,K_0,K_2',K_2'')$.
By applying Theorem \ref{1008@thm1}, there exist constants 
\begin{align*}
&\gamma=\gamma(d,m,n,\delta,p,q,K_0,K_2',K_2'')\in (0,1/6),\\
&\lambda_0=\lambda_0(d,m,n,\delta,p,q, K_0,K_2',K_2'',R_0,K)>0
\end{align*}
such that, under Assumption \ref{0923.ass1} $(\gamma)$, we have 
\begin{equation}		\label{1017@@eq1}
\sum_{|\alpha| \le m}\lambda^{1-\frac{|\alpha|}{2m}}\|D^\alpha {\boldsymbol{u}}\|_{L_{p,\tilde{w}}(\Omega_T)}\le N_1\sum_{|\alpha| \le m}\lambda^{\frac{|\alpha|}{2m}}\|{\boldsymbol{f}}_\alpha\|_{L_{p,\tilde{w}}(\Omega_T)},
\end{equation}
where $N_1=N_1(d,m,n,\delta,p,q,K_0,K_2',K_2'')$.
Set  
\begin{align*}
&\psi(t,x'')=\sum_{|\alpha|\le m}\lambda^{1-\frac{|\alpha|}{2m}}\| I_{\Omega_T}D^\alpha {\boldsymbol{u}}(t,\cdot,x'')\|_{L_{p,w_1}({\mathcal{X}}_1)},\\
&\phi(t,x'')=\sum_{|\alpha| \le m}\lambda^{\frac{|\alpha|}{2m}}\| I_{\Omega_T}{\boldsymbol{f}}_\alpha(t,\cdot,x'')\|_{L_{p,w_1}({\mathcal{X}}_1)}.
\end{align*}
It follows from \eqref{1017@@eq1} that  
\begin{equation*}
\|\psi\|_{L_{p,\tilde{w}_2}({\mathcal{X}}_2)}\le N_2\|\phi\|_{L_{p,\tilde{w}_2}({\mathcal{X}}_2)},
\end{equation*}
where $N_2=N_2(d,m,n,\delta,p,q,d_1,d_2,K_0,K_2',K_2'')$. 
Since the above inequality is satisfied for any $\tilde{w}_2\in A_p({\mathcal{X}}_2)$ with $[\tilde{w}_2]_{A_p}\le {\mathcal{K}}_0$, by Theorem \ref{1017@@thm1}, we have 
\[
\|\psi\|_{L_{q,w}({\mathcal{X}}_2)}\le 4N_2\|\phi\|_{L_{q,w}({\mathcal{X}}_2)},
\]
which gives the estimate \eqref{1017@e4}.
The theorem is proved.
\end{proof}

\begin{proof}[Proof of Theorem \ref{160621@thm2}]
We only prove the case when $T<\infty$ and $|\Omega|<\infty$.
The other cases are similar to the proof of Theorem \ref{160621@thm1}.
Assume that ${\boldsymbol{f}}_\alpha\in L_{p,q,w}(\Omega_T)\cap L_\infty(\Omega_T)$.
By Lemma \ref{1016@lem1}, there exists $\mu_0=\mu_0(p,q,K_0,K_2',K_2'')>1$ such that 
$$
w \in L_{\mu_0}((a,b)\times \Omega) \quad \text{for any }\, a,\,b\in (-\infty,T],\quad a<b.
$$
Set
$$
p_1:=\max\left(\frac{p\mu_0}{\mu_0-1}, \frac{q\mu_0}{\mu_0-1}\right),
$$
and define
$$
\bar{\gamma}=\min(\gamma,\gamma_1), \quad  \bar{\lambda}_0=\min(\lambda_0,\lambda_1),
$$
where $\gamma$ and $\lambda_0$ are constants in Theorem \ref{1016@thm1}, and $\gamma_1$ and $\lambda_1$ are constants in Lemma \ref{160612@lem1}.
Fix $\lambda\ge \bar{\lambda}_0$.
By Lemma \ref{160612@lem1}, for any positive integer $k$ satisfying $-k<T$, there exists a unique ${\boldsymbol{u}}_k\in \mathring{\mathcal{H}}^m_{p_1}(\Omega_T)$ such that  ${\boldsymbol{u}}_k\equiv 0$ in $\Omega_{-k}$,
$$
({\boldsymbol{u}}_k)_t+(-1)^m{\mathcal{L}}{\boldsymbol{u}}_k+\lambda{\boldsymbol{u}}_k=\sum_{|\alpha| \le m}D^\alpha (I_{(-k,T)\times \Omega}{\boldsymbol{f}}_\alpha) \quad \text{in }\,  \Omega_T,
$$
and 
$$
\sum_{|\alpha|\le m}\lambda^{1-\frac{|\alpha|}{2m}}\|D^\alpha {\boldsymbol{u}}_k\|_{L_{p_1}(\Omega_T)}\le N\sum_{|\alpha|\le m}\lambda^{\frac{|\alpha|}{2m}}\|{\boldsymbol{f}}_\alpha\|_{L_{p_1}(\Omega_T)}.
$$
Similar to \eqref{160918@eq1}, by the above inequality, we obtain
\begin{equation}		\label{161104@eq1}
\sum_{|\alpha|\le m}\|D^\alpha {\boldsymbol{u}}_k\|_{L_{p,q,w}(\Omega_T)}\le  {\mathcal{S}}\|w_1I_{\Omega}\|_{L_{\mu_0}({\mathcal{X}}_1)}^{1/p} \|w_2I_{(-k,T)\times \Omega}\|_{L_{\mu_0}({\mathcal{X}}_2)}^{1/q}<\infty,
\end{equation}
where 
$$
{\mathcal{S}}=\sum_{|\alpha|\le m}\left(\int_{{\mathcal{X}}_2}\left(\int_{{\mathcal{X}}_1}|D^\alpha {\boldsymbol{u}}_k|^{\frac{p\mu_0}{\mu_0-1}}I_{(-k,T)\times \Omega}\,dx'\right)^{q/p}\,dx''\,dt\right)^{\frac{\mu_0-1}{q\mu_0}}.
$$
The estimate \eqref{161104@eq1} implies that  ${\boldsymbol{u}}_k\in \mathring{\mathcal{H}}^m_{p,q,w}(\Omega_T)$.
Moreover, by following the same arguments as in the proof of Theorem \ref{160621@thm1}, one can easily check that $\{{\boldsymbol{u}}_k\}$ is a Cauchy sequence in $\mathring{\mathcal{H}}^m_{p,q,w}(\Omega_T)$ and its limit ${\boldsymbol{u}}\in \mathring{\mathcal{H}}^m_{p,q,w}(\Omega_T)$ is the solution of 
$$
{\boldsymbol{u}}_t+(-1)^m{\mathcal{L}}{\boldsymbol{u}}+\lambda {\boldsymbol{u}}=\sum_{|\alpha|\le m}D^\alpha {\boldsymbol{f}}_\alpha \quad \text{in }\, \Omega_T.
$$
The theorem is proved.
\end{proof}

\section{Appendix}		\label{app}

\begin{lemma}[{\cite[Sec.10.2]{MR519341}}]			\label{0922.lem2}
Let $r\in (0,\infty)$, $1< q\le p <\infty$, and $k=0,1,\ldots,2m-1$.
Assume that 
\[
\frac{1}{q}-\frac{1}{p}\le \frac{2m-k}{d+2m}.
\]
If $u\in W^{1,2m}_q(Q_r)$, then we have $D^ku\in L_p(Q_r)$ and 
$$
\|D^ku\|_{L_p(Q_r)}\le N\|u\|_{W^{1,2m}_q(Q_r)},
$$
where $N=N(d,m,k,p,q,r)$.
The statement remains true, provided that $Q_r$ is replaced by $Q_r^+$.
\end{lemma}

\begin{lemma}[{\cite[Sec.18.12]{MR521808}}]		\label{0922.lem4}
Let $r\in (0,\infty)$, $\mu\in (0,1)$, $1<q<\infty$, and $k=0,1,\ldots,2m-1$.
Assume that
\[
q\ge\frac{d+2m}{2m-k-\mu}.
\]
If $u\in W^{1,2m}_q(Q_r)$, then we have $D^ku\in C^\mu(Q_r)$ and 
\begin{equation*}
\|D^k u\|_{C^\mu(Q_r)}\le N\|u\|_{W^{1,2m}_q(Q_r)},
\end{equation*}
where $N=N(d,m,k,q,r,\mu)$.
The statement remains true, provided that $Q_r$ is replaced by $Q_r^+$.
\end{lemma}

\begin{lemma}		\label{160629@lem1}
Let ${\mathcal{X}}_1$ and ${\mathcal{X}}_2$ are spaces of homogeneous type with doubling constants $K_2'$ and $K_2''$ in ${\mathbb{R}}^{d_1}$ and ${\mathbb{R}}\times {\mathbb{R}}^{d_2}$, $d_1+d_2=d$, respectively.
Then ${\mathcal{X}}_1\times {\mathcal{X}}_2$ is a space of homogeneous type in ${\mathbb{R}}^{d+1}$ with the distance $\rho$ in \eqref{eq0215_01} and  a doubling constant $K_2=K_2(K_2'K_2'')$.
Moreover, if $p\in (1,\infty)$, $K_0',\, K_0''\ge 1$, and 
$$
w(t,x)=w_1(x')w_2(t,x''), \quad x'\in {\mathcal{X}}_1, \quad (t,x'')\in {\mathcal{X}}_2,
$$
where $w_1\in A_p({\mathcal{X}}_1)$ with $[w_1]_{A_p}\le K_0'$ and $w_2\in A_p({\mathcal{X}}_2)$ with $[w_2]_{A_p}\le K_0''$, then 
$w\in A_p({\mathcal{X}}_1\times {\mathcal{X}}_2)$ with $[w]_{A_p}\le K_0=K_0(p,K_0',K_0'',K_2',K_2'')$.
\end{lemma}

\begin{proof}
Denote ${\mathcal{X}}={\mathcal{X}}_1\times {\mathcal{X}}_2$.
Without loss of generality, we assume that $0\in {\mathcal{X}}$.
Since $B_{2r}^{\mathcal{X}}\subset B^{{\mathcal{X}}_1}_{2r}\times B^{{\mathcal{X}}_2}_{2r}$ and $B^{{\mathcal{X}}_1}_{r/2}\times B^{{\mathcal{X}}_2}_{r/2}\subset B_r^{\mathcal{X}}$, by using the doubling property, we have 
$$
|B^{\mathcal{X}}_{2r}|\le |B^{{\mathcal{X}}_1}_{2r}||B^{{\mathcal{X}}_2}_{2r}|\le (K_2'K_2'')^2|B^{{\mathcal{X}}_1}_{r/2}||B^{{\mathcal{X}}_2}_{r/2}|\le (K'_2K_2'')^2|B_r^{\mathcal{X}}|.
$$
Moreover, we obtain
$$
\left({\operatorname{\,\,\text{\bf--}\kern-.98em\DOTSI\intop\ilimits@\!\!}}_{B_r^{\mathcal{X}}}w\,dx\,dt\right)\left({\operatorname{\,\,\text{\bf--}\kern-.98em\DOTSI\intop\ilimits@\!\!}}_{B_r^{\mathcal{X}}}w^{-\frac{1}{p-1}}\,dx\,dt\right)^{p-1}\le \left(\frac{|B_r^{{\mathcal{X}}_1}||B_{r}^{{\mathcal{X}}_2}|}{|B_r^{\mathcal{X}}|}\right)^pK_0'K_0''
$$
$$
\le \left(\frac{|B_r^{{\mathcal{X}}_1}||B_{r}^{{\mathcal{X}}_2}|}{|B_{r/2}^{{\mathcal{X}}_1}||B^{{\mathcal{X}}_2}_{r/2}|}\right)^pK_0'K_0''\le (K_2'K_2'')^pK_0'K_0''.
$$
The lemma is proved.
\end{proof}

The following lemma is used to show the estimate \eqref{160804@eq9}.
We recall the notation \eqref{160617@eq2}, and we  point out that when $\Omega$ is a Reifenberg flat with $\gamma\in \big(0,\frac{1}{2}\big)$, the inequality  \eqref{160804@eq9a} is valid with $N_0=N_0(d)$. 

\begin{lemma}		\label{160804@LEM1}
Let $T\in (-\infty,\infty]$, $\Omega$ be a domain in ${\mathbb{R}}^d$,  and $\Omega_T\subseteq{\mathcal{X}}$, where  ${\mathcal{X}}$ is an open set in ${\mathbb{R}}^{d+1}$ and a space of homogeneous type with a doubling constant $K_2$.
Assume that there exist $R_0\in (0,1]$ and $N_0>0$ such that 
\begin{equation}		\label{160804@eq9a}
|{\mathcal{C}}_R(X)|\ge N_0R^{d+2m}, \quad \forall X\in \Omega_T, \quad \forall R\in (0,R_0].
\end{equation}
Let $p\in (1,\infty)$, $K_0\ge 1$, $w\in A_p({\mathcal{X}})$, and $[w]_{A_p}\le K_0$.
Suppose that  $D_0$ and $D_1$ are Borel sets satisfying $D_0\subset D_1\subset \Omega_T$, and that there exists a constant $\varepsilon\in (0,1)$ such that the following hold:
\begin{enumerate}[$(i)$]
\item
$w(D_0\cap {\mathcal{B}}_{R}^{\mathcal{X}}(X))<\varepsilon w({\mathcal{B}}^{\mathcal{X}}_{R}(X))$ for $X\in \Omega_T$ and $R\ge R_0$.
\item
For any $X\in \Omega_T$ and for all $R\in (0, R_0]$ with $w({\mathcal{B}}^{\mathcal{X}}_R(X)\cap D_0)\ge \varepsilon w({\mathcal{B}}_R^{\mathcal{X}}(X))$, we have ${\mathcal{C}}_R(X)\subset D_1$.
\end{enumerate}
Then we obtain 
\[
w(D_0)\le N\varepsilon w(D_1),
\]
where $N=N(d,p, K_0,K_2,N_0)$.

\end{lemma}

\begin{proof}
We first claim that for almost every $X\in D_0$, there exists  $R_X\in (0, R_0)$ such that 
\begin{equation*}
w(D_0\cap {\mathcal{B}}^{\mathcal{X}}_{R_X}(X))=\varepsilon w({\mathcal{B}}_{R_X}^{\mathcal{X}}(X))
\end{equation*}
and 
\begin{equation}		\label{160804@eq9b}
w(D_0\cap {\mathcal{B}}^{\mathcal{X}}_{R}(X))< \varepsilon w({\mathcal{B}}_R^{\mathcal{X}}(X)), \quad \forall R\in (R_X, R_0].
\end{equation}
We define a function $\rho$ on $[0,R_0]$  by 
\[
\rho(r)=\frac{w(D_0\cap {\mathcal{B}}^{\mathcal{X}}_r(X))}{w({\mathcal{B}}^{\mathcal{X}}_r(X))}=\frac{1}{w({\mathcal{B}}_r^{\mathcal{X}}(X))}\int_{{\mathcal{B}}_r^{\mathcal{X}}(X)}I_{D_0}w\,dY.
\]
By applying the Lebesgue differentiation theorem and using the fact that 
$$
w(X)>0 \quad \text{almost every }\, X\in {\mathcal{X}}, 
$$
we obtain for almost every $X\in D_0$ that 
$$
\rho(0)=\lim_{r\to 0^+}\left(\frac{|{\mathcal{B}}_r^{\mathcal{X}}(X)|}{w({\mathcal{B}}_r^{\mathcal{X}}(X))}\times {\operatorname{\,\,\text{\bf--}\kern-.98em\DOTSI\intop\ilimits@\!\!}}_{{\mathcal{B}}_r^{\mathcal{X}}(X)}I_{D_0}w\,dx\right)=1.
$$
Since $\rho$ is continuous on $[0, R_0]$ and $\rho(R_0)<\varepsilon$, there exists $r_X\in (0,R_0)$ such that $\rho(r_X)=\varepsilon$.
Then we obtain the claim by setting
$$
R_X:=\max\{r_X\in (0,R_0):\rho(r_X)=\varepsilon\}.
$$
Hereafter, we denote 
$$
\Gamma=\big\{B^{\mathcal{X}}_{R_X}(X):X\in D_0'\big\},
$$
where  $D_0'$ is  the set of all points $X\in D_0$ such that $R_X$ exists.
Then by the Vitali lemma, we have a countable subcollection $G$ in $\Gamma$ such that  
\begin{enumerate}[(a)]
\item
$Q\cap Q'=\emptyset$ for any $Q,\, Q'\in G$ satisfying $Q\neq Q'$.
\item
$D_0\subset \bigcup_{{\mathcal{B}}_{R_X}^{\mathcal{X}}(X)\in G}{\mathcal{B}}_{5R_X}^{\mathcal{X}}(X)$.
\end{enumerate}
Indeed, the subcollection $G$ can be constructed as follows.
We write $\Gamma_1=\Gamma$, and  choose a cylinder ${\mathcal{B}}_{R_{X_1}}(X_1)$, denoted by $Q^1$, in  $\Gamma$ such that 
$$
R_{X_1}>\frac{1}{2}\sup_{{\mathcal{B}}^{\mathcal{X}}_{R_X}(X)\in \Gamma_1}R_X,
$$
and split $\Gamma_1=\Gamma_2\cup \Gamma_2^*$, where 
$$
\Gamma_2=\{Q\in \Gamma_1: Q^1\cap Q=\emptyset\}, \quad \Gamma_2^*=\{Q\in \Gamma_1:Q^1\cap Q\neq \emptyset\}.
$$
Assume that $Q^k$ and  $\Gamma_{k+1}$ have been already determined.
If $\Gamma_{k+1}$ is empty, then the process ends.
If not,  we choose a cylinder $Q^{k+1}={\mathcal{B}}^{\mathcal{X}}_{R_{X_{k+1}}}(X_{k+1})$ in $\Gamma_{k+1}$ such that 
$$
R_{X_{k+1}}>\frac{1}{2}\sup_{{\mathcal{B}}_{R_X}^{\mathcal{X}}(X)\in \Gamma_{k+1}}R_X,
$$
and split $\Gamma_{k+1}=\Gamma_{k+2}\cup \Gamma_{k+2}^*$, where
$$
\Gamma_{k+2}=\{Q\in \Gamma_{k+1}: Q^{k+1}\cap Q=\emptyset\}, \quad \Gamma_{k+2}^*=\{Q\in \Gamma_{k+1}:Q^{k+1}\cap Q\neq \emptyset\}.
$$
We define $G=\{Q^k\}_{k\in J}$, where $J\subseteq {\mathbb{N}}$.
Obviously, $G$ satisfies $(a)$.
To see $(b)$, we note that 
$$
Q\subset {\mathcal{B}}^{\mathcal{X}}_{5R_{X_{k}}}(X_k), \quad \forall Q\in \Gamma_{k+1}^*, \quad \forall k\in J,
$$
and 
$$
\Gamma_1=\bigcup_{k\in J}\Gamma_{k+1}^*.
$$
Therefore, we have 
$$
D_0\subset \bigcup_{Q\in \Gamma_1} Q\subset \bigcup_{{\mathcal{B}}^{\mathcal{X}}_{R_X}(X)\in G}{\mathcal{B}}_{5R_X}^{\mathcal{X}}(X),
$$
which implies that $G$ satisfies $(b)$.

Now we are ready to prove the lemma.
From the assumption $(i)$, \eqref{160804@eq9b}, and the doubling property of ${\mathcal{X}}$, it follows that  
$$
w(D_0\cap {\mathcal{B}}_{5R_X}^{\mathcal{X}}(X))<\varepsilon w({\mathcal{B}}_{5R_X}^{\mathcal{X}}(X))\le \varepsilon (K_2)^3w({\mathcal{B}}^{\mathcal{X}}_{R_X}(X)), \quad \forall X\in D_0'.
$$
Using this together with $(b)$, we obtain 
\begin{align*}
w(D_0)&=w\Biggl(\bigcup_{{\mathcal{B}}^{\mathcal{X}}_{R_X}(X)\in G}D_0\cap {\mathcal{B}}_{5R_X}^{\mathcal{X}}(X)\Biggr)\\
&\le \sum_{{\mathcal{B}}_{R_X}^{\mathcal{X}}(X)\in G}w(D_0\cap {\mathcal{B}}_{5R_X}^{\mathcal{X}}(X))\\
&\le \varepsilon (K_2)^3\sum_{{\mathcal{B}}_{R_X}^{\mathcal{X}}(X)\in G} w({\mathcal{B}}_{R_X}^{\mathcal{X}}(X)).
\end{align*}
Observe that Lemma \ref{1015@lem3} and \eqref{160804@eq9a} yield
$$
w({\mathcal{B}}_{R_X}^{\mathcal{X}}(X))\le K_0\left(\frac{|{\mathcal{B}}_{R_X}^{\mathcal{X}}(X)|}{|{\mathcal{C}}_{R_X}(X)|}\right)^pw({\mathcal{C}}_{R_X}(X))\le N w({\mathcal{C}}_{R_X}(X)),
$$
where $N=N(d,K_0,p,N_0)$.
By combining the above two inequalities, and then, using $(a)$ and the assumption $(ii)$, we have 
$$
w(D_0)\le \varepsilon N\sum_{{\mathcal{B}}_{R_X}^{\mathcal{X}}(X)\in G}w({\mathcal{C}}_{R_X}(X))
$$
$$
=\varepsilon Nw\Biggl(\bigcup_{{\mathcal{B}}^{\mathcal{X}}_{R_X}(X)\in G}{\mathcal{C}}_{R_X}(X)\Biggr)\le \varepsilon N w(D_1),
$$
where $N=N(d,K_0,K_2,p,N_0)$.
The lemma is proved.
\end{proof}

In the lemma below, we prove that  functions in weighted $L_p$ spaces can be approximated by bounded functions.

\begin{lemma}		\label{160621@lem1}
Let $T\in(-\infty,\infty]$, $\Omega$ be a domain in ${\mathbb{R}}^d$, and $\Omega_T\subseteq {\mathcal{X}}_1\times {\mathcal{X}}_2$, where ${\mathcal{X}}_1$ and ${\mathcal{X}}_2$ are  spaces of homogeneous type  in ${\mathbb{R}}^{d_1}$ and ${\mathbb{R}}\times {\mathbb{R}}^{d_2}$, $d_1+d_2=d$, respectively.
Let $p,\, q\in (1,\infty)$ and 
$$
w(t,x)=w_1(x')w_2(t,x''), \quad x'\in {\mathcal{X}}_1, \quad (t,x'')\in {\mathcal{X}}_2, 
$$
where $w_1\in A_p({\mathcal{X}}_1)$ and $w_2\in A_q({\mathcal{X}}_2)$.
Then  for given $f\in L_{p,q,w}(\Omega_T)$, there exists a sequence $\{f_k\}_{k=1}^\infty$ in $L_{p,q,w}(\Omega_T)\cap L_\infty(\Omega_T)$ such that $f_k\to f$ in $L_{p,q,w}(\Omega_T)$ as $k\to \infty$.
\end{lemma}

\begin{proof}
Since $fw_1^{1/p}w_2^{1/q}\in L_{p,q}(\Omega_T)$, there exists a sequence $\{g_k\}_{k=1}^\infty$ in $C^\infty_0(\Omega_T)$ such that 
\begin{equation}		\label{160621@Eq1}
g_k\to fw_1^{1/p}w_2^{1/q} \quad \text{in }\, L_{p,q}(\Omega_T).
\end{equation}
Let us define 
$$
f_k=g_kw_1^{-1/p}w_2^{-1/q}I_{M^1_k\times M^2_k}, 
$$
where 
$$
M^1_k=\left\{x'\in {\mathcal{X}}_1:w_1(x')\ge \frac{1}{k}\right\}, \quad M^2_k=\left\{(t,x'')\in {\mathcal{X}}_2:w_2(t,x'')\ge \frac{1}{k}\right\}.
$$
Note that $f_k$ are bounded functions on $\Omega_T$ and 
$$
\int_{{\mathcal{X}}_2}\left(\int_{{\mathcal{X}}_1}|f_k-f|^pI_{\Omega_T}w_1\,dx'\right)^{q/p}w_2\,dx''\,dt
$$
$$
\le \int_{M_k^2}\left(\int_{M_k^1}|f_k-f|^pI_{\Omega_T}w_1\,dx'\right)^{q/p}w_2\,dx''\,dt
$$
$$
+ \int_{{\mathcal{X}}_2}\left(\int_{{\mathcal{X}}_1\setminus M_k^1}|f|^pI_{\Omega_T}w_1\,dx'\right)^{q/p}w_2\,dx''\,dt
$$
$$
+ \int_{{\mathcal{X}}_2\setminus M_k^2}\left(\int_{{\mathcal{X}}_1}|f|^pI_{\Omega_T}w_1\,dx'\right)^{q/p}w_2\,dx''\,dt:=I^1_k+I^2_k+I^3_k.
$$
It follows from \eqref{160621@Eq1} that $I_k^1\to 0$ as $k\to \infty$.
Since $|{\mathcal{X}}_1\setminus M_k^1|\to 0$ as $k\to \infty$, we have 
$$
|f|^pI_{\Omega_T} w_1I_{{\mathcal{X}}_1\setminus M^1_k}\to 0 \quad \text{a.e. in }\, {\mathcal{X}}_1.
$$
Using the dominated convergence theorem, we obtain that  
$$
\int_{{\mathcal{X}}_1\setminus M_k^1}|f|^pI_{\Omega_T}w_1\,dx'\to 0 \quad \text{a.e. in  }{\mathcal{X}}_2,
$$
and thus, by the dominated convergence theorem again, we obtain $I^2_k\to 0$ as $k\to \infty$.
Similarly, we obtain that $I_k^3\to 0$ as $k\to \infty$.
The lemma is proved.
\end{proof}

\bibliographystyle{plain}

\begin{thebibliography}{10}

\bibitem{MR0740173}
Hugo Aimar and Roberto~A. Mac\'ias.
\newblock Weighted norm inequalities for the {H}ardy-{L}ittlewood maximal
  operator on spaces of homogeneous type.
\newblock {\em Proc. Amer. Math. Soc.}, 91(2):213--216, 1984.

\bibitem{MR519341}
Oleg~V. Besov, Valentin~P. Il'in, and Sergey~M. Nikol'ski{\u\i}.
\newblock {\em Integral representations of functions and imbedding theorems.
  {V}ol. {I}}.
\newblock V. H. Winston \& Sons, Washington, D.C., 1978.
\newblock Translated from the Russian, Scripta Series in Mathematics, Edited by
  Mitchell H. Taibleson.

\bibitem{MR521808}
Oleg~V. Besov, Valentin~P. Il'in, and Sergey~M. Nikol'ski{\u\i}.
\newblock {\em Integral representations of functions and imbedding theorems.
  {V}ol. {II}}.
\newblock V. H. Winston \& Sons, Washington, D.C.; Halsted Press [John Wiley
  \&\ Sons], New York-Toronto, Ont.-London, 1979.
\newblock Scripta Series in Mathematics, Edited by Mitchell H. Taibleson.

\bibitem{MR2328932}
Sun-Sig Byun.
\newblock Optimal ${W}^{1,p}$ regularity theory for parabolic equations in
  divergence form.
\newblock {\em J. Evol. Equ.}, 7(3):415--428, 2007.

\bibitem{MR3225808}
Sun-Sig Byun and Dian~K. Palagachev.
\newblock Weighted {$L^p$}-estimates for elliptic equations with measurable
  coefficients in nonsmooth domains.
\newblock {\em Potential Anal.}, 41(1):51--79, 2014.

\bibitem{MR3467697}
Sun-Sig Byun, Dian~K. Palagachev, and Lubomira~G. Softova.
\newblock Global gradient estimates in weighted {L}ebesgue spaces for parabolic
  operators.
\newblock {\em Ann. Acad. Sci. Fenn. Math.}, 41(1):67--83, 2016.

\bibitem{MR2460025}
Sun-Sig Byun and Lihe Wang.
\newblock Fourth-order parabolic equations with weak {BMO} coefficients in
  {R}eifenberg domains.
\newblock {\em J. Differential Equations}, 245(11):3217--3252, 2008.

\bibitem{MR3261109}
Jongkeun Choi and Seick Kim.
\newblock Green's functions for elliptic and parabolic systems with
  {R}obin-type boundary conditions.
\newblock {\em J. Funct. Anal.}, 267(9):3205--3261, 2014.

\bibitem{MR2797562}
David~V. Cruz-Uribe, Jos\'e~Maria Martell, and Carlos P\'erez.
\newblock {\em Weights, extrapolation and the theory of Rubio de Francia},
  volume 215 of {\em Operator Theory: Advances and Applications}.
\newblock Birkh\"auser/Springer Basel AG, Basel, 2011.

\bibitem{arXiv:1603.07844v1}
Hongjie Dong and Doyoon Kim.
\newblock On {$L_p$}-estimates for elliptic and parabolic equations with {$A_p$}
  weights.
\newblock {\em arXiv:1603.07844v1}.

\bibitem{MR2650802}
Hongjie Dong and Doyoon Kim.
\newblock Parabolic and elliptic systems with {VMO} coefficients.
\newblock {\em Methods Appl. Anal.}, 16(3):365--388, 2009.

\bibitem{MR2835999}
Hongjie Dong and Doyoon Kim.
\newblock Higher order elliptic and parabolic systems with variably partially
  {BMO} coefficients in regular and irregular domains.
\newblock {\em J. Funct. Anal.}, 261(11):3279--3327, 2011.

\bibitem{MR2764911}
Hongjie Dong and Doyoon Kim.
\newblock {$L_p$} solvability of divergence type parabolic and elliptic systems
  with partially {BMO} coefficients.
\newblock {\em Calc. Var. Partial Differential Equations}, 40(3-4):357--389,
  2011.

\bibitem{MR2771670}
Hongjie Dong and Doyoon Kim.
\newblock On the {$L_p$}-solvability of higher order parabolic and elliptic
  systems with {BMO} coefficients.
\newblock {\em Arch. Ration. Meth. Anal.}, 199(3):889--941, 2011.

\bibitem{arXiv:1510.07643v2}
Chiara Gallarati and Mark Veraar.
\newblock Evolution families and maximal regularity for systems of parabolic
  equations.
\newblock {\em arXiv:1510.07643v2}.

\bibitem{arXiv:1410.6394v2}
Chiara Gallarati and Mark Veraar.
\newblock Maximal regularity for non-autonomous equations with measurable
  dependence on time.
\newblock {\em arXiv:1410.6394v2, Potential Anal., to appear}.

\bibitem{MR3243734}
Loukas Grafakos.
\newblock {\em Classical Fourier analysis}, volume 249 of {\em Graduate Texts
  in Mathematics}.
\newblock Springer, New York, third edition, 2014.

\bibitem{MR2286441}
Robert Haller-Dintelmann, Horst Heck, and Matthias Hieber.
\newblock {$L^p$}-{$L^q$} estimates for parabolic systems in non-divergence
  form with {VMO} coefficients.
\newblock {\em J. London Math. Soc. (2)}, 74(3):717--736, 2006.

\bibitem{MR1980981}
Horst Heck and Matthias Hieber.
\newblock Maximal ${L}^p$-regularity for elliptic operators with
  {VMO}-coefficients.
\newblock {\em J. Evol. Equ.}, 3(2):332--359, 2003.

\bibitem{MR1828321}
Nicolai~V. Krylov.
\newblock The heat equation in ${L}_q((0,{T}),{L}_p)$-spaces with weights.
\newblock {\em SIAM J. Math. Anal.}, 32(5):1117--1141, 2001.

\bibitem{MR2352490}
Nicolai~V. Krylov.
\newblock Parabolic equations with {VMO} coefficients in {S}obolev spaces with
  mixed norms.
\newblock {\em J. Funct. Anal.}, 250(2):521--558, 2007.

\bibitem{MR2435520}
Nicolai~V. Krylov.
\newblock {\em Lectures on elliptic and parabolic equations in {S}obolev
  spaces}, volume~96 of {\em Graduate Studies in Mathematics}.
\newblock American Mathematical Society, Providence, RI, 2008.

\bibitem{MR563790}
Nicolai~V. Krylov and Mikhail~V. Safonov.
\newblock A property of the solutions of parabolic equations with measurable
  coefficients.
\newblock {\em Izv. Akad. Nauk SSSR Ser. Mat.}, 44(1):161--175, 1980.

\bibitem{MR1232192}
Elias~M. Stein.
\newblock {\em Harmonic analysis: real-variable methods, orthogonality, and
  oscillatory integrals}, volume~43 of {\em Princeton Mathematical Series}.
\newblock Princeton University Press, Princeton, NJ, 1993.

\bibitem{MR2982717}
Jan van Neerven, Mark Veraar, and Lutz Weis.
\newblock Maximal ${L}^p$-regularity for stochastic evolution equations.
\newblock {\em SIAM J. Math. Anal.}, 44(3):1372--1414, 2012.

\end{thebibliography}

\end{document}


\documentclass[11pt]{amsart}
\usepackage[margin=1in]{geometry} 
\usepackage{graphicx} 
\graphicspath{{Figures/}}

\usepackage{rotating}
\usepackage{color}
\usepackage{bm}  
\usepackage{amssymb,amsmath,amsfonts}
\usepackage{amscd}
\usepackage{verbatim}
\usepackage{subfigure}

\DeclareMathAlphabet\mathbold{OML}{cmm}{b}{it}

\definecolor{black}{rgb}{0,0,0}

 

\numberwithin{equation}{section}
 
\newtheorem{theorem}{Theorem}[section]
\newtheorem{lemma}{Lemma}[section]
\newtheorem{remark}{Remark}[section]
\newtheorem{corollary}{Corollary}[section]
\newtheorem{definition}{Definition}[section]
\newtheorem{algorithm}{Algorithm}[section]
\newtheorem{assumption}{Assumption}[section]
\newtheorem{proposition}[theorem]{Proposition}
\theoremstyle{definition}\newtheorem{example}{Example}[section]

 
 
 
 
 

\def\b1{{\mathbf 1}}

 

 
 
 

 

                       
    
 

 
                     
       
              
              

              

 

 
 
\title[Preconditioning of weighted ${{\boldsymbol H}}({\operatorname{div}})$-norm and applications]
{Preconditioning of weighted ${{\boldsymbol H}}({\operatorname{div}})$-norm and applications to numerical simulation of 
highly heterogeneous media}

\author[J.~Kraus, R.~Lazarov, M.~Lymbery, S.~Margenov, L.~Zikatanov]
{Johannes Kraus, Raytcho  Lazarov, Maria Lymbery, Svetozar Margenov, and Ludmil Zikatanov}

\address{Thea-Leymann-Str. 9 45141 Essen, Germany}
\email{johannes.kraus@uni-due.de}

\address{Department of Mathematics, 
Texas A \& M University, College Station, TX 77843, USA and Institute
of Mathematics  and Informatics, Bulgarian Academy of Sciences, Acad. G. Bonchev St., Bl. 8,  1113 - Sofia, Bulgaria}
\email{lazarov@math.tamu.edu}

\address{Institute of Information and Communication Technologies, Bulgarian Academy of Sciences,
Acad. G. Bonchev St., Block 2,  1113 - Sofia, BULGARIA}
\email{mariq@parallel.bas.bg}

\address{Institute of Information and Communication Technologies, Bulgarian Academy of Sciences,
Acad. G. Bonchev St., Block 2,  1113 - Sofia, BULGARIA}
\email{margenov@parallel.bas.bg}

\address{Department of Mathematics, The Pennsylvania State University,
  University Park, PA 16802, USA and Institute of Mathematics and
  Informatics, Bulgarian Academy of Sciences, Acad. G. Bonchev St.,
  Bl. 8, 1113 - Sofia, Bulgaria} \email{ludmil@psu.edu}

\keywords{mixed finite elements, least-squares, high contrast media, robust preconditioners
for weighted ${{\boldsymbol H}}({\operatorname{div}})$-norm, discrete Poincar\'e inequality}

 
\subjclass{65F10, 65N20, 65N30}
 
\date{February 26, 2013--beginning; Today is \today}
 \thanks{}

\begin{document}
 
\begin{abstract}
In this paper we propose and analyze  a preconditioner for a system arising from a finite element
approximation of second order elliptic problems describing processes in 
highly heterogeneous media.
Our approach uses the technique of multilevel methods (see,
e.g. \cite{2008VassilevskiP-aa}) and the recently proposed
preconditioner based on additive Schur complement approximation by
J. Kraus (see, \cite{Kraus_12}). The main results are the design and a 
theoretical and numerical justification of an iterative method for
such problems that is robust with respect to the contrast of the
media, defined as the ratio between the maximum and minimum values of
the coefficient (related to the permeability/conductivity).
\end{abstract}
\maketitle

\section{Introduction}\label{sec:intro}
We consider the following second order elliptic 
boundary value problem written in mixed form 
for the unknown scalar functions $p(x)$ 
and the vector function ${{\mathbf u}}$: 
\begin{subequations}
\begin{alignat}{2} 
\label{equation-1}
 {{\mathbf u}} + K(x) \nabla p &= 0 \qquad && \text{in $\Omega$,}\\
\label{equation-2}
{\operatorname{div}} {{\mathbf u}}  &= f && \text{in $\Omega$},\\
\label{D BC}
p &= 0 && \text{on $\Gamma_D$ , } \\%$\partial \Omega$.}
\label{N BC} 
{{\mathbf u}} \cdot {{\mathbf n}}  &=0 && \text{on $\Gamma_N$. }
\end{alignat}
\end{subequations}
Here, $K: \mathbb{R}^d\mapsto \mathbb{R}^{d\times d}$ is a
function taking values in symmetric, 
positive definite (SPD) $d\times d$ matrices for almost all $x \in \Omega$.
 The given forcing term $f $ is function in $ L^2(\Omega)$,   
 $\Omega \subset \mathbb{R}^d$ ($d=2,3$) is a bounded polyhedral
 domain, and its boundary $\partial \Omega $ is split into two
 non-overlapping parts $\Gamma_D$ and $\Gamma_N$.  In the case pure
 Neumann problem, i.e. $\Gamma_N=\partial \Omega$, we assume that $f$
 satisfies the compatibility condition $\int_\Omega f dx=0$. In such a
 case the solution is determined uniquely by taking $\int_\Omega p ~dx=0$.
 To simplify our presentation, we shall assume that $\Gamma_D$ is an
 non-empty set with strictly positive measure which is also closed
 with respect to $\partial \Omega$, so the above problem
 has unique solution $ p \in H^1(\Omega)$.

This equation is a model used for example in heat and mass
transfer,  flows in porous media, diffusion of passive chemicals, 
electromagnetics,  and other applied areas. 
Here we target applications of equations \eqref{equation-1} - \eqref{N
  BC} to flows in {\it highly heterogeneous} media of high contrast. 
This means that the coefficient $K(x)$ represents media with
multiscale features, such as many inclusions where $K(x)$ has small
values and long connected subdomains where $K(x)$ has large values. An
important characteristic for such media is the contrast $\kappa$,
defined by \eqref{contrast}. In the terminology of flows in porous
media the coefficient $K(x)$ is called permeability and the unknown 
variables $p$ and ${{\mathbf u}}$ are called pressure and 
velocity, respectively.

Flows in porous media appear in many industrial, scientific, engineering, and 
environmental applications. One common characteristic of these diverse areas 
of applications is that porous media are intrinsically multiscale and 
typically display heterogeneities over a wide range of length-scales.
The computer generated permeability coefficient $K(x)$ which exhibits such
features and is used in petroleum engineering simulations as a
benchmark is found in the SPE10 Project~\cite{SPE10_project}.

Depending on the application and its goals, solving the governing equations of flows in porous media might be sought at: 
(a) a coarse scale (e.g., if only the global pressure drop for a given 
flow rate is needed, and no other fine 
scale details of the solution are important), 
or  a coarse scale enriched with some desirable fine scale details, and 
(b) the finest scale  that resolves media heterogeneities, if computationally 
this is possible. 
The numerical solution of latter class of problems is a challenging task that has attracted a 
substantial attention in the scientific and engineering community.

The aim of this paper is finite element approximations of the problem 
\eqref{equation-1} -- \eqref{N BC} on a mesh that resolves the finest scale of
the permeability. This leads to a very large system of algebraic equations 
and its efficient solution is the object of this paper.  We give two finite element
approximations, dual mixed and least-squares FEM, and  develop, study 
and test an optimal (with respect to the contrast ${\kappa}$ and the mesh
size $h$)  preconditioner for the corresponding algebraic problem.

The system arising from the finite element approximation
has symmetric matrix, which is indefinite in the case of a mixed method
and positive definite in the case of a least-squares approximation. In
both cases the preconditioner is a block diagonal matrix, which in
combination with a minimal residual method in the case of mixed method
and conjugate gradient iteration in the case of least-squares FEM,
leads to an optimal preconditioner so that the number of iterations
does not depend neither on the contrast nor on the mesh size.

In the least-squares methods written as a system of linear equations
for the unknown ${{\mathbf u}}$ and $p$, the upper left block (corresponding to
the vector variable ${{\mathbf u}}$) is a symmetric and positive definite matrix,
corresponding to the finite element approximation of the weighted
bilinear form $(K^{-1}{{\mathbf u}}, {{\mathbf v}}) + ({\operatorname{div}} {{\mathbf u}}, {\operatorname{div}} {{\mathbf v}})$ with
${{\boldsymbol H}}({\operatorname{div}})$-conforming finite elements.  For the mixed method this form
plays also essential role (see,
e.g. \cite{Arnold1997preconditioning}).  It is known that the
preconditioner of Arnold, Falk, and Winther,
\cite{Arnold1997preconditioning, Arnold2000MG}, is optimal if $K(x)$
is a constant matrix. However, since $K(x)$ is the permeability of
heterogeneous media of high contrast, the corresponding matrix is very
ill-conditioned and consequently construction of an optimal
preconditioner is a nontrivial task.
The main result and novelty of this paper is design, theoretical justification, and 
experimental  study of a preconditioner for the matrix corresponding 
to the above weighted bilinear form 
that will produce an optimal iterative method  which converges independently
of the contrast $\kappa$.  Such construction is based on the
method developed in~\cite{Kraus_Lymb_Mar_2014}.

The paper is organized as follows. In Section \ref{sec:problem} the dual mixed and 
least-squares variational formulations of the boundary value 
problem~\eqref{equation-1}--\eqref{N BC} are given.  Next we
introduce the weighted ${{\boldsymbol H}}({\operatorname{div}})$ bilinear form and show proper
inf-sup condition using the corresponding weighted norm. We emphasize
that the constant in the inf-sup condition, coercivity and the
boundedness of the corresponding forms do not depend on the contrast
of the media defined by \eqref{contrast}.
In Section~\ref{s:FEM} we introduce finite element approximations for the mixed and least-squared formulations. A key point in this part is Lemma \ref{lem:stab_fin}  
where we establish an inf-sup condition 
on a discrete level with a constant independent on the contrast. The detailed proof of the
Lemma is given in the Appendix. In Section \ref{s:precon}, which is central to the paper,
we describe the preconditioning method for the finite element  systems. 
Starting with the definition of a block-diagonal preconditioner for the dual-mixed formulation
in operator notation, see Section~\ref{sec:b-diag-operator}, a reformulation of the FE problem
in matrix notation is given in Section~\ref{sec:matrix-notation}.
The key issue in designing a contrast-independent Krylov solver becomes the construction
of a robust preconditioner for the weighted $H({\operatorname{div}})$-norm. This is addressed in 
details in Section~\ref{sec:robust-H_div-precond} by, first, discussing an abstract auxiliary
space two-grid method, see Section~\ref{sec:as-2-grid-precond}, and then, defining an
auxiliary space multigrid (ASMG) method, see Section~\ref{sec:asmg}. Two variants of the
algorithm are described in Section~\ref{sec:algorithms} differing only in the choice of the basis
in which smoothing and residual updates are performed. Finally, in Section~\ref{sec:numerics}, we 
present the numerical results for three different examples of porous media in two dimensions
in order to test the robustness of the preconditioner with respect to media contrast and its
optimality with respect to the mesh-size. All numerical results confirm our theoretical findings.

\section{Problem formulation}\label{sec:problem}

\subsection{Notation and preliminaries}

For functions defined on ${{\Omega}}$ we shall use the standard notations for
Sobolev spaces.  Namely, $H^s({{\Omega}})$, $s \ge 0$ an integer, is the
space of functions having their generalized derivatives up to order
$s$ square-integrable on ${{\Omega}}$.  We denote by $(\cdot,\cdot)$ the
$L^2$ and $[L^2]^d$ inner products. 
The standard norms on $H^s$ are denoted by $\|\cdot\|_{s}$. 
For $s=0$ we shall often use $\|\cdot\|$ without subscript. 
When the
norm is weighted with a matrix valued function $\omega(x)$,
with $\omega(x)$ SPD for almost all $x\in \Omega$ we use the
notation:
\[
\|{{\mathbf v}}\|_{0,\omega} := 
\|\omega^{1/2}{{\mathbf v}}\|, \quad
|\phi|_{1,\omega}:= \|\nabla \phi\|_{0,\omega} =\|\omega^{1/2}\nabla \phi\|.
\]
Occasionally, when we consider only a subset of $\Omega$, say,
$T\subset \Omega$ we will indicate this in the notation for the norms
and seminorms, i.e., we have $\|\cdot\|_{s,T}$,
$\|\cdot\|_{s,\omega,T}$, $|\cdot|_{s,T}$, and $|\cdot|_{s,\omega,T}$.

In the following, if $\omega\in \mathbb{R}^{d\times d}$ is symmetric matrix,
the norm $\|\omega\|_{\ell^2}$ is, as usual, the spectral radius of
$\omega$.  
We introduce the number 
\begin{equation}\label{contrast}
\kappa = \max_{x \in \Omega} (\|K(x)\|_{\ell^2}\|K^{-1}(x)\|_{\ell^2}). 
\end{equation}
called the contrast of the media, is assumed that this could be many
orders of magnitude.  For scalar permeability $K(x)$, obviously,
$\kappa = \max_{x \in {{\Omega}}}K(x)/\min_{x \in {{\Omega}}}K(x)$.  
The values of $\kappa$ typically are very large (up to $10^8$). 

The Hilbert space ${{\boldsymbol H}}({\operatorname{div}})$ consists of square-integrable
vector-fields on ${{\Omega}}$ with square-integrable divergence. The inner product in
${{\boldsymbol H}}({\operatorname{div}})$ is given by
\begin{equation}\label{H-div-prod}
\Lambda({{\mathbf u}}, {{\mathbf v}})= ({{\mathbf u}},{{\mathbf v}}) + ({\operatorname{div}} {{\mathbf u}}, {\operatorname{div}} {{\mathbf v}}).
\end{equation}
Further we use the notation of the Sobolev spaces $H^1_D({{\Omega}})$ and ${{\boldsymbol H}}_N({\operatorname{div}})$
\begin{equation}\label{H-D}
H^1_D({{\Omega}}) = \{ \phi \in H^1({{\Omega}}): ~~\phi(x)=0 \quad \text{on} \quad \Gamma_D\}.
\end{equation}
In the case when $\Gamma_D=\emptyset$ we have
\begin{equation}\label{H-D-2}
H^1_D({{\Omega}}) = \{ \phi \in H^1({{\Omega}}): ~~\int_{\Omega}\phi=0\}.
\end{equation}
Note that for $\phi\in H^1_D({{\Omega}})$ the seminorms
$|\phi|_1=\|\nabla \phi\|$ and 
$|\phi|_{1,\omega}=\|\omega^{1/2}\nabla \phi\|$  
are in fact norms on $H^1_D(\Omega)$ and we denote these norms
by 
$\|\phi\|_1$ and $\|\phi\|_{1,\omega}$. 
The
space   ${{\boldsymbol H}_{\hspace{-0.2mm}N}}({\operatorname{div}})$ is defined as follows: 
\begin{equation}\label{H-N}
  {{\boldsymbol H}_{\hspace{-0.2mm}N}}({\operatorname{div}}) 
:= {{\boldsymbol H}_{\hspace{-0.2mm}N}}({\operatorname{div}}; {{\Omega}})= \{ {{\mathbf v}} \in {{\boldsymbol H}}({\operatorname{div}}; {{\Omega}}): ~~{{\mathbf v}}(x) \cdot {{\mathbf n}}=0 \quad \text{on} \quad \Gamma_N\}.
\end{equation}

Together with \eqref{H-div-prod} the weighted inner product in the space ${{\boldsymbol H}}({\operatorname{div}})$ 
will play a fundamental role in our analysis  
\begin{equation}\label{WH-div-prod}
\Lambda_{{\alpha}}({{\mathbf u}}, {{\mathbf v}})= ({{\alpha}}~{{\mathbf u}},{{\mathbf v}}) + ({\operatorname{div}} {{\mathbf u}}, {\operatorname{div}} {{\mathbf v}}), 
\quad \alpha(x) = K^{-1}(x).
\end{equation}
We also denote
\[
\|{{\mathbf v}}\|^2_{\Lambda_\alpha}  = 
\Lambda_{{\alpha}}({{\mathbf v}}, {{\mathbf v}})= 
\|{{\mathbf v}}\|^2_{0,\alpha} + \|{\operatorname{div}} {{\mathbf v}}\|^2.
\]
Note, that a weighted bilinear form of the type $
\Lambda_{\alpha,\beta}({{\mathbf u}}, {{\mathbf v}})= \alpha ({{\mathbf u}}, {{\mathbf v}}) + \beta ({\operatorname{div}}
{{\mathbf u}}, {\operatorname{div}} {{\mathbf v}}) $, with $\alpha > 0$ and $\beta > 0$ constants, was
used by Arnold, Falk, and Winther in \cite{Arnold2000MG} to design
multigrid methods for ${{\boldsymbol H}}({\operatorname{div}})$-systems.  A key moment in their
study was the construction of a multigrid method that converges
uniformly with respect to the parameters $\alpha$ and $\beta$. The
important difference between our bilinear form $\Lambda_\alpha$
compared with $ \Lambda_{\alpha,\beta}$ is that in our form $\alpha$
is a highly heterogeneous function with high contrast.
This makes the proof of a proper {\it inf-sup} condition more delicate
and the construction of an efficient preconditioner more complicated.

The main objective of this work is a derivation and theoretical justification
of numerical methods for solving the problem \eqref{equation-1} --
\eqref{N BC} 
that are robust with respect to the media contrast, or in the ideal case, 
do not depend on the contrast
$\kappa$. In this regard, in the following we write $x \lesssim y$ to
denote the existence of a constant $C$ such that $x\le Cy$. In case
$x\lesssim y$ and $y\lesssim x$ we write $x\eqsim y$. It is important
to remember that in our analysis the constants hidden in $\lesssim$ or
$\eqsim$ are always independent of the contrast and the mesh size.

Note that for  all $\xi\in\mathbb{R}^d$, $x\in \Omega$ and writing $K$
for $K(x)$ we have
\begin{eqnarray*}
\|K^{-1}\|_{\ell^2}(K\xi\cdot\xi)& \ge &
(\xi\cdot\xi)\|K^{-1}\|_{\ell^2}
\inf_{\theta\in \mathbb{R}^d}\frac{(K\theta\cdot \theta)}{(\theta\cdot\theta)}\\
&= &
(\xi\cdot\xi) \, \|K^{-1}\|_{\ell^2}
\inf_{\theta\in
  \mathbb{R}^d}\frac{(\theta\cdot\theta)}{(K^{-1}\theta\cdot\theta)}
=
(\xi\cdot\xi)\|K^{-1}\|_{\ell^2}\frac{1}{\|K^{-1}\|_{\ell^2}}=(\xi\cdot\xi).
\end{eqnarray*}
Based on this, without loss of generality we assume that 
the following inequality holds
\begin{equation}\label{bound_below_K}
(\xi\cdot \xi) \le  (K(x)\xi\cdot\xi), \quad \xi\in \mathbb{R}^d.   
\end{equation}  
As seen from the considerations above, such an assumption is always
fulfilled if we scale the coefficient and work with
\[
K(x)\leftarrow 
K(x)\left\|\|K^{-1}(x)\|_{\ell^2}\right\|_{L^{\infty}(\Omega)}. 
\] 
Clearly, such rescaling does not change the value of $\kappa$. 

We note that the standard elliptic theory ensures existence and
uniqueness of a unique solution $p \in H^1_D({{\Omega}})$. However, since the
coefficient matrix $K(x)$ is piece-wise smooth and may have very large
jumps, the solution $p$ has low regularity, say $H^{1+s}({{\Omega}})$,
where $s>0$ could depend on the contrast $\kappa$ in a subtle (and 
unfavorable) way. We need to take this into account when proving the
stability of the discrete methods with constants independent of
$\kappa$.

\subsection{Weak formulations of the elliptic problem}\label{s:weak}

Now we introduce the dual mixed and least-squares formulations of the 
problem \eqref{equation-1} -- \eqref{N BC}.

\subsubsection{Dual mixed formulation}\label{ss:dual_form}
To present the dual mixed weak form we require the following notation 
$
{{\boldsymbol V}} \equiv {{\boldsymbol H}_{\hspace{-0.2mm}N}}({\operatorname{div}};\Omega)$ 
and 
$
W \equiv \{ q \in L^2(\Omega) ~~\text{and} ~(q,1)=0  ~~~\text{if} ~~~ \Gamma_N=\partial \Omega \}.
$

We multiply the first equation by $K^{-1}(x)$ and 
a test function $ {{\mathbf v}}$,  integrate over ${{\Omega}}$, and perform integration by parts to get
\begin{equation}\label{first}
(K^{-1}(x) {{\mathbf u}}, {{\mathbf v}}) - (p, {\operatorname{div}} {{\mathbf v}}) = 0
\end{equation}
Next, we  multiply the second equation by a test function
$q$ and integrate over ${{\Omega}}$ to get
\begin{equation}\label{div-eqn}
({\operatorname{div}} {{\mathbf u}}, q)=(f,q).
\end{equation}
Then the weak form of  the problem \eqref{equation-1} -- \eqref{N BC}
 is: find  ${{\mathbf u}} \in {{\boldsymbol V}}$ and $p \in W$ such that
\begin{equation}\label{eq:dual_mixed}
{{\mathcal A}}^{DM}({{\mathbf u}},p; {{\mathbf v}},q)= - (f, q),
\quad \mbox{for all}\quad ({{\mathbf v}}, q) \in {{\boldsymbol V}} \times W,
\end{equation}
where the bilinear form 
${{\mathcal A}}^{DM}({{\mathbf u}},p; {{\mathbf v}},q): ({{\boldsymbol V}}, W)  \times ( {{\boldsymbol V}}, W) \to {\mathbb{R}}$ is defined as
\begin{equation}\label{A-form-mixed}
{{\mathcal A}}^{DM}({{\mathbf u}},p; {{\mathbf v}},q):=({{\alpha}} {{\mathbf u}}, {{\mathbf v}})
-(p, {\operatorname{div}} {{\mathbf v}}) - ({\operatorname{div}} {{\mathbf u}}, q). 
\end{equation}

\begin{remark}
It is clear that we could test the equation \eqref{div-eqn} 
with $q={\operatorname{div}} {{\mathbf v}}$, multiply 
by an arbitrary positive constant  $\beta$, and add to 
equation \eqref{first}. This will result in a slightly different bilinear form
$$
{{\mathcal A}}^{DM}_\beta ({{\mathbf u}},p; {{\mathbf v}},q)
:=({{\alpha}} {{\mathbf u}}, {{\mathbf v}}) + \beta ({\operatorname{div}} {{\mathbf u}}, {\operatorname{div}} {{\mathbf v}})-(p, {\operatorname{div}} {{\mathbf v}}) - ({\operatorname{div}} {{\mathbf u}}, q). 
$$
and modified right hand side of \eqref{eq:dual_mixed} equal to $ - (f,
q) + \beta ( f, {\operatorname{div}} {{\mathbf v}})$.  This modified problem will have the same
solution. However, such formulation might be beneficial in the design
of a better preconditioner for the finite element system, see,
e.g. \cite{vassilevski1996preconditioning}.
\end{remark}

\subsubsection{Least-squares formulation} 
In this case we take the 
$L^2$-norm of the second equation \eqref{equation-2} and the weighted $L^2$-norm
with weight $K^{-1/2}$ of the first equations in \eqref{equation-1}
to obtain a quadratic functional, e.g. \cite{pehlivanov1994least}.
Then the minimizer of this least-squares functional satisfies
the following weak form: find ${{\mathbf u}} \in {{\boldsymbol H}_{\hspace{-0.2mm}N}}({\operatorname{div}})$ and $p \in H^1_D({{\Omega}})$ such that 
\begin{equation}\label{eq:LS}
{{\mathcal A}}^{LS}({{\mathbf u}},p; {{\mathbf v}},q)=(f, {\operatorname{div}} {{\mathbf v}}), 
\quad \mbox{for all}\quad ({{\mathbf v}}, q) \in {{\boldsymbol H}_{\hspace{-0.2mm}N}}({\operatorname{div}}) \times H^1_D({{\Omega}}),
\end{equation}
where the bilinear form 
${{\mathcal A}}^{LS}({{\mathbf u}},p; {{\mathbf v}},q): ({{\boldsymbol H}_{\hspace{-0.2mm}N}}({\operatorname{div}}), H^1_D({{\Omega}}) ) \times ({{\boldsymbol H}_{\hspace{-0.2mm}N}}({\operatorname{div}}), H^1_D({{\Omega}}) ) \to {\mathbb{R}}$
is defined as
\begin{equation}\label{A-form-LS}
{{\mathcal A}}^{LS}({{\mathbf u}},p; {{\mathbf v}},q):=({{\alpha}} {{\mathbf u}}, {{\mathbf v}}) +({\operatorname{div}} {{\mathbf u}}, {\operatorname{div}} {{\mathbf v}}) + (p, {\operatorname{div}} {{\mathbf v}}) + ({\operatorname{div}} {{\mathbf u}}, q) + (K \nabla p, \nabla q).
\end{equation}
Clearly, the bilinear 
form $ {{\mathcal A}}^{LS}({{\mathbf u}},p; {{\mathbf v}},q)$ is symmetric and coercive, see, e.g.
\cite{pehlivanov1994least}.  We would like to establish the coercivity with a 
constant independent of the
contrast $\kappa$ and later we provide a proof of this. 

\subsection{Stability of the weak formulations}
Now we study the stability of the discrete problems \eqref{eq:dual_mixed} and 
\eqref{eq:LS}.  We shall use the Poincar\'e inequality
\begin{equation}\label{Poincare}
\text{there is $C_P>0$ such that} ~~\mbox{for all}\quad 
q \in H^1_D(\Omega) \quad
\|q\|^2 \le C_P \| \nabla q\|^2. 
\end{equation}
The constant $C_P$ depends only on the geometry of the domain
$\Omega$ and the splitting of $\partial \Omega$ into $\Gamma_D$ and 
$\Gamma_N$. Moreover, due to \eqref{bound_below_K} on the
coefficient $K(x)$ we also have the inequality
$$
\|q\|^2 \le C_P \| \nabla q\|^2 \le 
C_P \|\nabla q\|^2_{0,K}.
$$

\subsubsection{Stability of the mixed formulation}
First, we consider the mixed form. For this we need continuity  and 
 {\it inf-sup} condition (see, e.g. \cite{1991BrezziF_FortinM-aa})
for the bilinear form  ${{\mathcal A}}^{DM}({{\mathbf u}},p; {{\mathbf v}},q)$
on the spaces $ {{\boldsymbol V}} $ and $L^2(\Omega)$  equipped with  the weighted norm 
$ (\Lambda_{{\alpha}}({{\mathbf v}},{{\mathbf v}}))^{\frac{1}{2}}$ and the standard $L^2$-norm $\|p\|$,
respectively.

\begin{lemma} \label{lem:stab_var}
Let $W=L^2(\Omega)$, ${{\boldsymbol V}}={{\boldsymbol H}_{\hspace{-0.2mm}N}}({\operatorname{div}})$, and $\Vert {{\mathbf v}} \Vert_{\Lambda_{{\alpha}}}:=
(\Lambda_{{\alpha}}({{\mathbf v}},{{\mathbf v}}))^{\frac{1}{2}}$. Then the following inequalities hold 
\begin{enumerate}
\item 
For all ${{\mathbf u}}, {{\mathbf v}}\in{{\boldsymbol V}}$ and  for all  $p, q\in W$
\begin{equation}\label{continuity}
{{\mathcal A}}^{DM}({{\mathbf u}},p; {{\mathbf v}},q) 
\le ( \|{{\mathbf u}}\|^2_{\Lambda_{{\alpha}}} +\|p\|^2)^\frac12( \|{{\mathbf v}}\|^2_{\Lambda_{{\alpha}}} +\|q\|^2)^\frac12;
\end{equation}  
\item There is a constant ${{\alpha}}_0>0$ independent of $\alpha$ such that
\begin{equation}\label{A-infsup}
\sup_{{{\mathbf v}} \in {{\boldsymbol V}}, \, q \in L^2} 
\frac{{{\mathcal A}}^{DM}({{\mathbf u}},p; {{\mathbf v}},q) }{(\|{{\mathbf v}}\|^2_{\Lambda_{{\alpha}}} +\|q\|^2)^\frac12} 
\ge \alpha_0 ( \|{{\mathbf u}}\|^2_{\Lambda_{{\alpha}}} +\|p\|^2 )^\frac12
\end{equation}

\end{enumerate}

\end{lemma}
\begin{proof}
The first inequality follows immediately by applying Schwarz inequality to all three terms and having
in mind that ${{\alpha}} $ is a positive function. Proving the {\it inf-sup} condition \eqref{A-infsup} is equivalent 
to proving the following inequality (see, \cite{Ern-Guermond}):
 \begin{equation}\label{inf_sup_stab_var}
  \inf_{q\in W} \sup_{{{\mathbf v}}\in{{\boldsymbol V}}}\frac{(\nabla\cdot {{\mathbf v}},q)}{\Vert {{\mathbf v}} \Vert_{\Lambda_{{\alpha}}}
\Vert q \Vert}\ge 
\gamma >0, \qquad \mbox{for all}\quad {{\mathbf v}}\in{{\boldsymbol V}},\quad \mbox{for all}\quad q\in W. 
 \end{equation}
 As is well known, if $\gamma $ is independent of the contrast
 $\kappa$, then so is $\alpha_0$ and for more details on the relation
 between the constants $\gamma$ and $\alpha_0$ we refer to
 \cite{Xu2003Zikatanov}. It is clear then that proving a contrast
 independent estimate for $\gamma$, which we show next, automatically
 leads to a contrast independent $\alpha_0$.

Inf-sup condition \eqref{inf_sup_stab_var} for Raviart-Thomas 
finite element spaces  and standard ${{\boldsymbol H}}_N({\operatorname{div}})$-norm is a well known fact.  
However, we have slight modification the ${{\boldsymbol H}}_N({\operatorname{div}})$-norm 
and we would like to show that the constant $\gamma$ does not
depend on the contrast $\kappa$. For this reason we provide a proof.

We first prove that for all $q\in W$ there exists ${{\mathbf w}}\in{{\boldsymbol V}}$ such that 
\begin{equation}\label{lem:stab_var_1}
\|{{\mathbf w}}\|_{0,\alpha}^2\le C_P \, \Vert q \Vert^2.
\end{equation}
For this let $\varphi\in H_D^1(\Omega)$ be the solution to the variational problem
\begin{equation}\label{eq:standard-variational}
( K(x) \nabla \varphi,\nabla\chi ) = (q, \chi), 
\quad \mbox{for all}\quad \chi\in H_D^1(\Omega).
\end{equation}
We then set ${{\mathbf w}} = K \nabla \varphi$. By construction ${{\mathbf w}} \in {{\boldsymbol V}}$ 
with ${\operatorname{div}} {{\mathbf w}}= -q$ which holds in $L^2(\Omega)$. 
Since $\alpha K = I_{d\times d}$ for almost all $x\in \Omega$,  then
\begin{eqnarray*}
\|{{\mathbf w}}\|_{0,\alpha}^2& = & 
(\alpha\;K\nabla\varphi,{{\mathbf w}})= (\nabla\varphi,{{\mathbf w}})
= -(\varphi,{\operatorname{div}} {{\mathbf w}})=(\varphi, q)\\ &\le& 
\|\varphi\|\|q\|
\le \sqrt{C_P}(\nabla\varphi,\nabla\varphi)^\frac12 \|q\|\\
&\le& \sqrt{C_P} \,|\nabla\varphi|_{0,K}\|q\|
= \sqrt{C_P} \|{{\mathbf w}}\|_{0,\alpha}\|q\|.
\end{eqnarray*}
Here, we used the Poincar\'e inequality \eqref{Poincare} and the fact
that from~\eqref{bound_below_K} we have 
$ (K(x)\xi\cdot\xi) \ge (\xi\cdot\xi)$ for $\xi\in \mathbb{R}^d$ and $x\in \Omega$.

Next, for any $q\in L^2(\Omega)$ we have  
\begin{eqnarray*}
\sup_{{{\mathbf v}}\in {{\boldsymbol V}}} \frac{(q,{\operatorname{div}} {{\mathbf v}})}{\|{{\mathbf v}}\|_{\Lambda_\alpha}} &\ge &
\frac{(q,{\operatorname{div}} {{\mathbf w}})}{\|{{\mathbf w}}\|_{\Lambda_\alpha}}=  
\frac{\|q\|^2}{ (\|{{\mathbf w}}\|_{0,\alpha}^2+\|{\operatorname{div}} {{\mathbf w}}\|^2)^\frac12}\\
& = & 
\frac{\|q\|^2}{( \|{{\mathbf w}}\|_{0,\alpha}^2+\|q\|^2)^\frac12}
\ge \frac{\|q\|}{\sqrt{C_P+1}}.
\end{eqnarray*}
This completes the proof.
\end{proof}

\subsubsection{Stability of the least-squares formulation}
For this we require continuity  and  coercivity of the bilinear form
${{\mathcal A}}^{LS}({{\mathbf u}},p; {{\mathbf v}},q)$
on the space $ {{\boldsymbol H}_{\hspace{-0.2mm}N}}({\operatorname{div}}) \times H^1_0(\Omega)\equiv {{\boldsymbol V}} \times W$. 
The   space $ {{\boldsymbol H}_{\hspace{-0.2mm}N}}({\operatorname{div}})$  is  
equipped with  the weighted norm 
$ (\Lambda_{{\alpha}}({{\mathbf v}},{{\mathbf v}}))^{\frac{1}{2}}$ and the  space $H^1_D(\Omega)$ is equipped with the
norm $\|q\|_{1,K}=\|K^{1/2}\nabla q\|$. 

\begin{lemma} \label{lem:stab_var_LS}
Let $W=H^1_D(\Omega)$ and ${{\boldsymbol V}}={{\boldsymbol H}_{\hspace{-0.2mm}N}}({\operatorname{div}})$.
Then the following inequalities hold: 
\begin{enumerate}
\item  For all ${{\mathbf u}}, {{\mathbf v}}\in{{\boldsymbol V}}$ and  for all  $p, q\in W$
\begin{equation}\label{continuity_LS}
{{\mathcal A}}^{LS}({{\mathbf u}},p; {{\mathbf v}},q) 
\le 2( \|{{\mathbf u}}\|^2_{\Lambda_{{\alpha}}} +\|p\|^2_{1,K})^\frac12( \|{{\mathbf v}}\|^2_{\Lambda_{{\alpha}}} +\|q\|^2_{1,K})^\frac12;
\end{equation}  
\item 
For all ${{\mathbf v}}\in{{\boldsymbol V}}$ and $q \in W$ then
\begin{equation}\label{A-coercivity}
{{\mathcal A}}^{LS}({{\mathbf v}},q; {{\mathbf v}},q) 
\ge \frac13 ( \|{{\mathbf v}}\|^2_{\Lambda_{{\alpha}}} +\|q\|^2_{1,K})
\end{equation}
\end{enumerate}

\end{lemma}
\begin{proof}
The bound \eqref{continuity_LS} follows easily by first transforming the terms
$({\operatorname{div}} {{\mathbf u}}, q) $ and $ ({\operatorname{div}} {{\mathbf v}}, p)$ using integration by parts to
$-(\nabla q, {{\mathbf u}}) $ and $ -(\nabla p,  {{\mathbf v}})$ and then using Schwarz inequality.

For the bound \eqref{A-coercivity} we fist note that 
\begin{equation}\label{div_bound}
{{\mathcal A}}^{LS}({{\mathbf v}},q; {{\mathbf v}},q)  = \|{\operatorname{div}} {{\mathbf v}} \|^2 + (\alpha {{\mathbf v}} - \nabla q, {{\mathbf v}} - K \nabla q) 
\ge \|{\operatorname{div}} {{\mathbf v}} \|^2.
\end{equation}
Next, we proceed as follows: using the Schwarz inequality, the
Poincar\'e inequality, $\alpha K = I_{d\times d}$, and
also $2ab\le \frac{1}{\epsilon}a^2+{\epsilon}b^2$ for all
$\epsilon > 0$, $a,b\in \mathbb{R}$ gives
\begin{eqnarray*}
{{\mathcal A}}^{LS}({{\mathbf v}},q; {{\mathbf v}},q)   &&  \ge  (1-\frac{1}{\epsilon})\|{\operatorname{div}} {{\mathbf v}}\|^2  +  
\|{{\mathbf v}}\|_{0,\alpha}^2 + \|q\|^2_{1,K} - \epsilon \|q \|^2 \\
 &&   \ge (1-\frac{1}{\epsilon})\|{\operatorname{div}} {{\mathbf v}} \|^2 
+ \|{{\mathbf v}}\|^2_{0,\alpha} + (1 - \epsilon C_P)\|\nabla q\|_{0,K}, \quad \epsilon >0.
\end{eqnarray*}
Choosing $\epsilon = \frac{1}{2 C_P}$ 
we get
$$
{{\mathcal A}}^{LS}({{\mathbf v}},q; {{\mathbf v}},q) \ge (1-2 C_P)\|{\operatorname{div}} {{\mathbf v}} \|^2 
+ \|{{\mathbf v}}\|_{0,\alpha}^2 + \frac12 \|\nabla q\|_{0,K}^2.
$$
Now multiplying inequality \eqref{div_bound} by a 
constant $\beta >0$ and adding it to this inequality we have
$$
(1+ \beta) {{\mathcal A}}^{LS}({{\mathbf v}},q; {{\mathbf v}},q) \ge ( 1-2 C_P + \beta) \|{\operatorname{div}} {{\mathbf v}} \|^2 
+ \|{{\mathbf v}}\|^2_{0,\alpha} + \frac12 \|\nabla q\|_{0,K}^2.
$$
If $C_P\le \frac12$, then we choose $\beta=\frac12$ and obtain \eqref{A-coercivity}.
If $C_P \ge \frac12$, then
choosing $\beta > 0 $, such that $  1-2 C_P + \beta= \frac12$, we derive
the lower bound for 
${{\mathcal A}}^{LS}({{\mathbf v}},q; {{\mathbf v}},q)$ with $ (4 C_P -1)/(4C_P+ 1)$, which obviously is greater or equal to
$\frac13$. Thus we have  the desired bound \eqref{A-coercivity}.
\end{proof}
\begin{remark}
We stress that, under the assumption \eqref{bound_below_K} 
and the choice of the norms in the spaces ${{\boldsymbol V}}$ and $W$,  
 the coercivity constant of the bilinear form in the least-squares formulation
does not depend on the contrast $\kappa$.
\end{remark}

\section{FEM approximations}\label{s:FEM}

\subsection{Finite element partitioning and spaces}\label{ss:FE_spaces}
We assume that the domain $\Omega$ is connected and is triangulated
with $d$ dimensional simplices. The triangulation is denoted by
$\mathcal{T}_h$ and we assume that the simplices forming
$\mathcal{T}_h$ are shape regular (the ratio between the diameter of a
simplex and the inscribed ball is bounded above). 
Now we consider the finite element approximation of the problem
\eqref{equation-1} -- \eqref{N BC} using the finite dimensional
spaces  ${{\boldsymbol V}_{\hspace{-0.2mm}h}} \subset {{\boldsymbol V}}$ 
and $W_h \subset W$ of piece-wise polynomial functions. 

It is well known that for the vector variable ${{\mathbf u}}$ we can use ${{\boldsymbol H}}({\operatorname{div}})$-conforming 
or Raviart-Thomas space ${{\mathcal{RT}}_{{k}}}$ or Brezzi-Douglas-Marini ${{\mathcal{BDM}}_{{k+1}}}$ finite elements. 
However, since we have a problem with low regularity it is natural to use the lowest order
finite element spaces. For the vector variable ${{\mathbf u}}$ we use the standard
Raviart-Thomas 
$\text{RT}_0$ for simplices and cubes. In the case of simplices we can apply 
also Brezzi-Douglas-Marini
$\text{BDM}_1$  finite elements.

For the two formulations, least-squares and mixed, we use $W$-conforming  finite elements spaces for pressure $p$. Note that for the least-squares formulation $W=H^1_D(\Omega)$ while for the mixed formulation $W=L^2(\Omega)$.
For both cases we  show that the corresponding finite element methods are stable uniformly with 
respect to the contrast $\kappa$.

\subsection{Mixed FEM and its stability}
Thus, 
we take 
\begin{equation}\label{space Vh}
{{\boldsymbol V}_{\hspace{-0.2mm}h}}=\{ {{\mathbf v}} \in {{\boldsymbol V}}: \, {{\mathbf v}} |_T \in {{\mathcal{RT}}_{{0}}} \,\,\, \mbox{for} \,\, T \in {\mathcal T}_h\} 
\end{equation}
and 
\begin{equation}\label{space Wh}
W_h =\{ q \in L^2(\Omega): \, q|_{T} \in {\mathcal P}_0, 
\, \text{i.e. $q$ is a piece-wise constant function on} \,\,  {\mathcal T}_h\}.
\end{equation}
The mixed finite element approximation 
of  the problem \eqref{equation-1} -- \eqref{N BC}
 is: find  ${{\mathbf u}}_h \in {{\boldsymbol V}_{\hspace{-0.2mm}h}}$ and $p_h \in W_h$ such that
\begin{equation}\label{eq:dual_mixed_FEM}
{{\mathcal A}}^{DM}({{\mathbf u}}_h,p_h; {{\mathbf v}},q)=(f,q), 
\quad \mbox{for all}\quad ({{\mathbf v}}, q) \in {{\boldsymbol V}_{\hspace{-0.2mm}h}} \times W_h,
\end{equation}
where the bilinear form $ {{\mathcal A}}^{DM}({{\mathbf u}}_h,p_h; {{\mathbf v}},q)$ is defined by 
\eqref{A-form-mixed}.
Our goal is to prove the discrete variant of the {\it inf-sup} condition.
\begin{lemma}\label{lem:stab_fin}
Let ${{\boldsymbol V}_{\hspace{-0.2mm}h}}$ be the space defined by \eqref{space Vh} and $W_h$ be the space defined by
\eqref{space Wh}.
Then independently of the contrast $\kappa$ and the step-size $h$
the following inequality holds true:
\begin{equation}\label{inf_sup_fin_set}
 \inf_{q_h\in W_h}\sup_{{{\mathbf v}}_h\in{{\boldsymbol V}_{\hspace{-0.2mm}h}}}
\frac{({\operatorname{div}} {{\mathbf v}}_h,q_h)}{\Vert {{\mathbf v}}_h\Vert_{\Lambda_{{\alpha}}} \Vert q_h\Vert}\ge 
\gamma >0 . 
\end{equation}
\end{lemma}
\begin{proof}
This is stated and proved as Theorem~\ref{theorem:inf-sup} in the
Appendix. 
\end{proof}
As a consequence of Lemma \ref{lem:stab_fin} and \eqref{continuity}  we have
\begin{theorem}\label{DM_stability}
The following bounds are valid for all ${{\mathbf u}} \in {{\boldsymbol V}_{\hspace{-0.2mm}h}}$ and $ p \in W_h$:
\begin{equation}\label{DM-Fem_stability}
\alpha_0 ( \|{{\mathbf u}}\|^2_{\Lambda_{{\alpha}}} +\|p\|^2)^\frac12 \le \sup_{{{\mathbf v}} \in{{\boldsymbol V}_{\hspace{-0.2mm}h}}, q\in W_h} 
\frac{{{\mathcal A}}^{DM}({{\mathbf u}},p; {{\mathbf v}},q) }{(\|{{\mathbf v}}\|^2_{\Lambda_{{\alpha}}} +\|q\|^2)^\frac12} \le 
( \|{{\mathbf u}}\|^2_{\Lambda_{{\alpha}}} +\|p\|^2)^\frac12.
\end{equation} 
The constant $\alpha_0$ may depend on the shape regularity of the 
mesh, but is  independent of the contrast $\kappa$ and the mesh-size $h$.
\end{theorem}

\subsection{Least-squares FEM and its stability}
For the least-squares formulation the space $W_h$ for the pressure $p$  
consists of standard conforming  Lagrangian 
finite elements involving linear polynomials  ${\mathcal P}_1$ (on triangles) 
and bilinear polynomials ${\mathcal Q}_1$ (on rectangles).

Since in this case we have conforming spaces, then the results of Lemma 
\ref{lem:stab_var_LS} are valid so that finite element method is stable. Thus we have
\begin{theorem}\label{LS_stability}
The bilinear form $ {{\mathcal A}}^{LS}({{\mathbf u}},p; {{\mathbf v}},q)$ is coercive and bounded on ${{\boldsymbol V}_{\hspace{-0.2mm}h}} \times W_h$
so that the following bounds are valid for all ${{\mathbf v}} \in {{\boldsymbol V}_{\hspace{-0.2mm}h}}$ and $ q \in W_h$:
\begin{equation}\label{LS-Fem_coercivity}
\frac13 ( \|{{\mathbf v}}\|^2_{\Lambda_{{\alpha}}} +\|q\|^2_{1,K})  \le {{\mathcal A}}^{LS}({{\mathbf v}},q; {{\mathbf v}},q)   
\end{equation}
and 
\begin{equation}\label{LS-FEM-continuity}
 {{\mathcal A}}^{LS}({{\mathbf u}},p; {{\mathbf v}},q)  \le 2
(\|{{\mathbf u}}\|^2_{\Lambda_{{\alpha}}} +\|p\|^2_{1,K})^\frac12
( \|{{\mathbf v}}\|^2_{\Lambda_{{\alpha}}} +\|q\|^2_{1,K})^\frac12.
\end{equation} 
\end{theorem}
We note that the coercivity and the stability of the
  least-squares FEM immediately show that we have an efficient preconditioner
  for this problem as long as we know how to solve an weighted ${{\boldsymbol H}}({\operatorname{div}})$
  problem and also a scalar elliptic problem. Moreover, since the
  stiffness matrix resulting from the least-squares FEM
  is an SPD matrix, such a preconditioner can be used
  in the conjugate gradient algorithm to obtain an efficient and optimal
  solution method.  
\section{Preconditioning}\label{s:precon}

\subsection{Block-diagonal preconditioner for the system of the finite element method}\label{sec:b-diag-operator}
Now we consider the problem \eqref{eq:dual_mixed} and for definiteness
we restrict ourselves to the lowest order Raviart-Thomas mixed finite
elements on a rectangular grid. The goal of this section is to develop
and justify a preconditioner for the algebraic problem resulting from
the Galerkin method \eqref{eq:dual_mixed} that is independent of the
contrast of the media.

Then \eqref{eq:dual_mixed} can be written as an operator equation in $X_h={{\boldsymbol V}}_h\times W_h$.
Namely, for ${\bm x}_h =({{\mathbf u}}_h, p_h)$ we have 
\begin{equation}\label{mixed-operator}
 \mathcal{A}_h {\bm x}_h = {\bm f}_h, \quad \mbox{for} \quad
{\bm f}_h =({\bm 0}, f_h) \in X_h,
\end{equation}
where for all
$\bm{y}_h=({{\mathbf v}}_h,q_h)\in X_h$ 
$$
( \mathcal{A}_h {\bm x}_h, {\bm y}_h) = {{\mathcal A}}^{DM}({{\mathbf u}}_h,p_h; {{\mathbf v}}_h,q_h)
\quad
\mbox{or} \quad
 ( \mathcal{A}_h {\bm x}_h, {\bm y}_h) = {{\mathcal A}}^{LS}({{\mathbf u}}_h,p_h; {{\mathbf v}}_h,q_h).
$$
Obviously, the operator $\mathcal{A}_h:X_h\rightarrow X_h^{\star}$ is self-adjoint
on $X_h={{\boldsymbol V}}_h\times W_h$ and indefinite in the case of 
dual mixed formulation and positive definite  in the case of least-squares approximation.

As it has been shown in~\cite{1991BrezziF_FortinM-aa}, the operator norms 
corresponding to the mixed finite element disretization~\eqref{eq:dual_mixed}
\begin{equation}\label{eq:operator_norms}
 \Vert\mathcal{A}_h\Vert_{\mathcal{L}(X_h,X_h^{\star})}\;\; \mbox{and}\;\; 
\Vert\mathcal{A}_h^{-1}\Vert_{\mathcal{L}(X_h^{\star},X_h)} \;\; \mbox{are uniformly bounded}. 
\end{equation}
The same is valid for the operator norms corresponding to the least squares formulation.

Now our goal is to construct a positive definite self-adjoint operator 
$\mathcal{B}_h:X_h\rightarrow X_h^{\star}$ such that all eigenvalues of 
$\mathcal{B}_h^{-1}\mathcal{A}_h$ are bounded 
uniformly independent of $h$ and, what is even more important, independent of the 
contrast $\kappa$. From~\eqref{eq:operator_norms} it follows that 
\begin{equation}\label{eq:precond_norms}
 \Vert\mathcal{B}_h\Vert_{\mathcal{L}(X_h,X_h^{\star})}\;\; \mbox{and}\;\; 
\Vert\mathcal{B}_h^{-1}\Vert_{\mathcal{L}(X_h^{\star},X_h)} \;\; \mbox{being uniformly bounded in} \;\; h \;\; \mbox{and} 
\;\; \kappa
\end{equation}
is sufficient for $\mathcal{B}_h$ to be a uniform and robust
preconditioner for the minimum residual (MinRes) iteration for the
mixed method and conjugate gradient (CG) for the least-squares
method. 

Let the block-diagonal preconditioner $\mathcal{B}_h$ be defined as
\begin{equation}\label{eq:AFW_preconditioner}
 \mathcal{B}_h:=\left[
\begin{array}{cc}
 A_h & 0\\[2ex]
 0 & D_h
\end{array}
\right]
\end{equation}
where $A_h:\,{{\boldsymbol V}}_h\rightarrow {{\boldsymbol V}}_h^{*}$ is given by
$$
(A_h{{\mathbf u}}_h,{{\mathbf v}}_h):=\Lambda_{{\alpha}}({{\mathbf u}}_h,{{\mathbf v}}_h)=
({{\alpha}} \, {{\mathbf u}}_h,{{\mathbf v}}_h)+(\nabla\cdot {{\mathbf u}}_h,\nabla\cdot {{\mathbf v}}_h)
$$
and $D_h: ~W_h \to W_h^*$ is either $I_h$, the identity on $W_h$ (for the mixed method) or
is defined by the relation $(D_h p_h, q_h)=(K \nabla p_h, \nabla q_h)$ for the least-squares
finite element method. 

\begin{remark}\label{iterations}
As easily seen, preconditioning is an equivalence relation,
  
  that is, if
  $Z_1\eqsim Z_2$ and $Z_2\eqsim Z_3$ for some operators $Z_1$,
  $Z_2$ and $Z_3$, then $Z_1\eqsim Z_3$. Therefore, uniform condition
  number estimates for $\mathcal{B}_h$, directly carry over to the
  case when $\mathcal{B}_h$ is replaced by any of its uniform (with
  respect to the contrast $\kappa$ and with respect to the mesh size $h$) preconditioners.  
\end{remark}

Then condition~\eqref{eq:precond_norms} reduces to $\Vert A_h
\Vert_{\mathcal{L}({{\boldsymbol V}}_h,{{\boldsymbol V}}_h^{\star})}$ and $\Vert
A_h^{-1}\Vert_{\mathcal{L}({{\boldsymbol V}}_h^{\star},{{\boldsymbol V}}_h)}$ being uniformly
bounded in $h$ and $\kappa$, which in the case of the mixed method is
sufficient for optimality of the preconditioner,
see~\cite{Arnold1997preconditioning}.  For the case of the least-squares
formulation ~\eqref{eq:precond_norms} we also need uniform boundedness
of $\Vert D_h \Vert_{\mathcal{L}(W_h, W_h^{\star})}$ and
$\Vert D_h^{-1}\Vert_{\mathcal{L}(W_h^{\star},W_h)}$.
Optimal
  preconditioners using multigrid and domain decomposition methods for
  solving the system
$D_h p_h=c_h$
have been developed by many authors
  in the last three decades. One approach for the solution of
$D_h p_h=c_h$
(which is, in some sense, the closest one to our work)
is the multilevel and auxiliary space multigrid methods proposed
in~\cite{Kraus_12,Kraus_Lymb_Mar_2014}. 

Thus, the main task  in this section is the development and study of
a robust and uniformly convergent (with respect to $h$ and $\kappa$)
iterative method for solving systems with
$A_h {{\mathbf u}}_h={{\mathbf b}}_h$. 

\subsection{Reformulation of the FE problem using matrix notation}\label{sec:matrix-notation}
The derivation and the justification of the
preconditioner
will be in the
framework of algebraic multilevel/multigrid methods. As a first step
we rewrite the operator equation \eqref{mixed-operator} in a matrix
form. For doing this, instead of functions ${\bm x}_h =({{\mathbf u}}_h, p_h)
\in {{\boldsymbol V}}_h \times W_h$ we use the vectors consisting of the
degrees of freedom determining ${\bm x}_h$ 
through the nodal basis functions, namely,
$$
{\bm x}=\left[
\begin{array}{c}
 {\bf u} \\
 {\bf p}
\end{array} \right ],
\quad \mbox{where} \quad
\quad {\bf u} \in \mathbb{R}^{|\mathcal{E}_h|},
\quad {\bf p} \in \mathbb{R}^{|\mathcal{T}_h|},
\quad \mbox{are vector columns} 
$$
and $ {|\mathcal{E}_h|}$ is the number of edges in $ \mathcal{E}_h$, excluding those 
on $\Gamma_N$, and $ {|\mathcal{T}_h|}$  is the number of
rectangles of the partition $ \mathcal{T}_h$.  Then $A$, ${B_{\operatorname{div}}}$,
$\widetilde A$, $R$, denote matrices being either square or
rectangular.  As a result of this convention
the equation \eqref{mixed-operator} can be written in a matrix form  
for mixed and least-squares approximations, correspondingly,

\begin{equation}\label{eq:saddle_point_sys}
 \left[
\begin{array}{cc}
{{M}_{{\alpha}}} & -{B_{\operatorname{div}}}^T \\
-{B_{\operatorname{div}}} & 0
\end{array}
\right]
\left[
\begin{array}{c}
 {\bf u} \\
 {\bf p}
\end{array}
\right]=
\left[
\begin{array}{c}
 \bf{0} \\
 \bf{f}
\end{array}
\right], 
\quad
\left[
\begin{array}{cc}
 A & {B_{\operatorname{div}}}^T \\
{B_{\operatorname{div}}} & D
\end{array}
\right]
\left[
\begin{array}{c}
 {\bf u} \\
 {\bf p}
\end{array}
\right]=
\left[
\begin{array}{c}
 \bf{0} \\
 \bf{f}
\end{array}
\right], 
\end{equation}
where 
$
{{\mathbf v}}^T {{M}_{{\alpha}}} {{\mathbf u}} =(\alpha {{\mathbf u}}_h, {{\mathbf v}}_h)$,
${{\mathbf v}}^T {B_{\operatorname{div}}}^T {\bm p}=(p_h, {\operatorname{div}} {{\mathbf v}}_h)$,
${\bm q}^T {B_{\operatorname{div}}} {{\mathbf u}}=({\operatorname{div}} {{\mathbf u}}_h, q_h)$,
$ {{\mathbf v}}^T A{{\mathbf u}} = \Lambda_{{\alpha}}({{\mathbf u}}_h,{{\mathbf v}}_h)$, 
and $  {\bm q}^T D {\bm p}=(K \nabla p_h, \nabla q_h).
$
Our aim now is to derive and study a preconditioner for the algebraic system 
 \eqref{eq:saddle_point_sys}, which due to the above considerations reduces 
to efficient preconditioning of 
the system
\begin{equation}\label{algebraic-hdiv}
A {{\mathbf u}} = {{\mathbf b}}, \quad {{\mathbf u}}, {{\mathbf b}} \in  \mathbb{R}^N, \quad N:={|\mathcal{E}_h|}.
\end{equation}

\subsection{Robust preconditioning of the weighted ${{\boldsymbol H}}({\operatorname{div}})$-norm}\label{sec:robust-H_div-precond}

The preconditioning technique described in this section is based on additive Schur complement
approximation (ASCA), see, \cite{Kraus_12}. For the sake of self-containedness, first we will
recall the auxiliary space two-grid and multigrid methods that have recently been introduced
in~\cite{Kraus_Lymb_Mar_2014}. Then we formulate two variants of the algorithm 
resulting from ASCA that are subject to
numerical testing in Section~\ref{sec:numerics}.

\subsubsection{Auxiliary space two-grid preconditioner}\label{sec:as-2-grid-precond}

The first step in the construction of the preconditioner involves a covering
$\Omega$ by $n$  overlapping subdomains $\Omega_{i}$, i.e.,  
\begin{equation*}
 \overline{\Omega}=\bigcup_{i=1}^{n} \overline{\Omega}_{i}.
\end{equation*}
This overlapping covering of $\Omega$ is to some
extent
arbitrary with generous overlap.
For practical purposes however, we consider the situation shown on Figure \ref{fig:covering}.
In this particular case any finite element in the partition $\mathcal{T}_h$
will belong to no more than four subdomains.
We associate subdomain matrices $A_{i}$,
$i=1,\dots,n$ with the subdomains $\Omega_{i}$ and assume that
they have the following assembling property
$$
A=\sum_{i=1}^{n} R_{{i}}^T A_{{i}} R_{{i}},
$$
where $ R_{i}$ is a rectangular matrix that extends by zero the vector associated with the
degrees of freedom of ${{\mathbf u}}_h$ in $\Omega_i$ to a vector representing the degrees of 
freedom in the whole domain $\Omega$.
Assume further that the set $\mathcal{D}$ of degrees of freedom (DOF) of ${{\mathbf u}}_h$ 
is partitioned into a set $\mathcal{D}_{\rm f}$, fine DOF, and a set
$\mathcal{D}_{\rm c}$, coarse DOF, so that
\begin{equation}\label{eq:partitioning_DOF}
\mathcal{D} = \mathcal{D}_{\rm f} \oplus \mathcal{D}_{\rm c},
\end{equation}
where $N_1:=\vert\mathcal{D}_{\rm f} \vert$ and $N_2:=\vert\mathcal{D}_{\rm c} \vert$
denote the cardinalities of $\mathcal{D}_{\rm f}$ and $\mathcal{D}_{\rm c}$, respectively,
with $N_1 + N_2 =N:=|{\mathcal E}_h|$. Recall that  $ |{\mathcal E}_h|$ 
is the number of edges in the partitioning ${\mathcal T}_h$ with the 
edges on $\Gamma_N$ excluded.
Such splitting is not obvious for the mixed finite element 
method and will be explained in detail later.
The splitting 
\eqref{eq:partitioning_DOF} induces a representation of the matrices $A$ and
$A_{i}$ into two-by-two block form, i.e.,
\begin{equation}\label{eq:A_two_by_two-x}
A=\left[
        \begin{array}{cc}
        A_{11} & A_{12} \\
        A_{21} & A_{22}
        \end{array}
\right],\qquad
 A_{i}=\left[
        \begin{array}{cc}
        A_{i:11} & A_{i:12} \\
        A_{i:21} & A_{i:22}
        \end{array}
\right], \quad i=1,\dots,n.
\end{equation}
We now introduce the following auxiliary domain decomposition matrix
\begin{equation}\label{eq:auxA}
\widetilde{A}=
\left[ \begin{array}{ccccc}
A_{1:11} &&&& A_{1:12} R_{1:2} \\[0.5ex]
& A_{2:11} &&& A_{2:12} R_{2:2} \\[0.5ex]
&& \ddots && \vdots \\[0.5ex]
&&& A_{{n}:11} & A_{{n}:12} R_{n:2} \\[0.5ex]
R_{1:2}^T A_{1:21} & R_{2:2}^T A_{2:21} & \hdots & R^T_{n:2} A_{{n}:21} 
& \displaystyle \sum_{i=1}^{n} R^T_{i:2} A_{i:22} R_{i:2}
                \end{array}
\right] ,
\end{equation}
Setting 
$\widetilde{A}_{11}={\rm diag}\{A_{1:11}, \ldots, A_{{n}:11}\}$,
$\widetilde{A}_{22}=\sum_{i=1}^{n} R^T_{i:2} A_{i:22} R_{i:2}$ we have 
\begin{equation}\label{eq:A_two_by_two}
\widetilde{A}=\left[
        \begin{array}{cc}
        \widetilde{A}_{11} & \widetilde{A}_{12} \\
        \widetilde{A}_{21} & \widetilde{A}_{22}
        \end{array}
\right].
\end{equation}
Note that if $A$ is an SPD matrix then $\widetilde{A}$ is a symmetric and positive semi-definite matrix.
Moreover, $A \in {{\mathbb{R}}^{{N{\times}N}}}$ and
$\widetilde{A} \in {{\mathbb{R}}^{{\widetilde{N}{\times}\widetilde{N}}}}$
are related via
\begin{equation}
A=R \widetilde{A} R^T 
\end{equation}
where
\begin{equation}\label{eq:R}
R=\left[ \begin{array}{cc}
           R_1 & 0 \\ 0 & I_2
           \end{array}
\right] , \qquad
R_1^T=\left[ \begin{array}{c}
           R_{1:1} \\ R_{2:1} \\ \vdots \\ R_{n:1}
           \end{array}
\right] .
\end{equation}
\begin{definition}[cf.~\cite{Kraus_12}]
The additive Schur complement approximation (ASCA) of the exact Schur complement
$S=A_{22} - A_{21} A_{11}^{-1} A_{12}$ is denoted by $Q$ and defined as the
Schur complement of
$\widetilde{A}$, i.e., 
$$Q:= \widetilde{A}_{22} - 
\widetilde{A}_{21} \widetilde{A}_{11}^{-1} \widetilde{A}_{12}
=\sum_{i=1}^{n} R_{{i}:2}^T (A_{{i}:22} - A_{{i}:21} 
A_{{i}:11}^{-1} A_{{i}:12}) R_{{i}:2}.
$$
\end{definition}
\begin{remark}
Note that $\widetilde{A}_{22}=A_{22}$. Thus $\widetilde{N}_2=N_2$ and $\widetilde{N}_1 \ge N_1$.
\end{remark}
Next, let $V={{\mathbb{R}}^{{N}}}$ and $\widetilde{V}={{\mathbb{R}}^{{\widetilde{N}}}}$ and
define the surjective mapping $\Pi_{\widetilde{D}}: \widetilde{V} \rightarrow V$ by 
\begin{equation}\label{eq:Pi}
\Pi_{\widetilde{D}}=(R \widetilde{D} R^T)^{-1} R \widetilde{D}, 
\end{equation}
where $\widetilde{D}$ is a block-diagonal matrix, e.g.,
\begin{equation}\label{eq:tilde_D}
\widetilde{D}=\left[
             \begin{array}{cc}
              \widetilde{D}_{11} & 0 \\
              0 & I
             \end{array}
           \right]
\end{equation}
where $\widetilde{D}_{11}=\widetilde{A}_{11}$ or
$\widetilde{D}_{11}={\rm diag}(\widetilde{A}_{11})$.

Then let us consider the following fictitious-space two-grid preconditioner
$C$ for $A$, which is implicitly defined via its inverse 
\begin{equation}\label{eq:two-grid_1}
C^{-1}=\Pi_{\widetilde{D}} \widetilde{A}^{-1} \Pi^T_{\widetilde{D}}. 
\end{equation}
The method of fictitious space preconditioning has first been  proposed in
  \cite{1985MatsokinA_NepomnyashchikhS-aa,Nepomnyaschikh1991mesh,
    nepomnyaschikh1995fictitious}, and refined
  in~\cite{Xu1996auxiliary} by adding an additional smoother. The
  basis of the  auxiliary space multigrid algorithm, is the following auxiliary
  space two-grid preconditioner $B$, which is implicitly defined via
  its inverse
\begin{equation}\label{eq:two-grid_3}
B^{-1} := \overline{M}^{-1} 
+ (I - M^{-T} A) C^{-1} (I - A M^{-1})  
\end{equation}
where $C$ is defined according to~\eqref{eq:two-grid_1}, the operator $M$
denotes an $A$-norm convergent smoother, i.e., $\Vert I-M^{-1}A\Vert_A
< 1$,  
and $\overline{M}=M(M+M^T-A)^{-1}M^T$ is the corresponding symmetrized
smoother. Examples of such smoothers include the Gauss-Seidel smoother and  
the damped Jacobi smoother and are well known. 

Below in Theorem~\ref{thm:KLM}
we present an estimate for the condition number under the following assumptions:
\begin{itemize}
\item[(i)] the smoother $M$ satisfies
for all ${{\bf{{v}}}} \in V={{\mathbb{R}}^{{N}}}$: 
\begin{equation}\label{eq:eb_1}
\underbar{$c$}\langle{{\bf{{v}}}},{{\bf{{v}}}}\rangle \le \rho_A \langle \overline{M}^{-1}{{\bf{{v}}}},{{\bf{{v}}}}\rangle
\le \bar{c}\langle{{\bf{{v}}}},{{\bf{{v}}}}\rangle, 
\end{equation}
and 
\begin{equation}\label{eq:eb_2}
 \Vert M^{-T} A {{\bf{{v}}}}\Vert^2\le \frac{\eta}{\rho_A}\Vert {{\bf{{v}}}} \Vert_A^2, \quad 
\end{equation}
where $\rho_A=\lambda_{\max}(A)$ denotes the spectral radius of $A$
and $\eta$ is  a non-negative constant;\\
\item[(ii)] the operator $\Pi$ defined by
\begin{equation}\label{eq:two-grid_4}
\Pi:=(I-M^{-T} A)\Pi_{\widetilde{D}}=
(I-M^{-T} A)(R\widetilde{D}R^T)^{-1}R\widetilde{D}
\end{equation}
satisfies the estimate
\begin{equation}\label{eq:eb_3}
\Vert \Pi \tilde{{{\bf{{v}}}}}\Vert_A^2\le c_{\Pi} 
\Vert \tilde{{{\bf{{v}}}}}\Vert_{\widetilde{A}}^2, \quad 
\mbox{for all}\quad \tilde{{{\bf{{v}}}}}\in \widetilde{V}={{\mathbb{R}}^{{\widetilde{N}}}}. 
\end{equation}
\end{itemize}
Note that due to $\Vert\Pi^{\star}\Pi\Vert=\Vert\Pi\Pi^{\star}\Vert$, the bound \eqref{eq:eb_3} 
is equivalent to
$\Vert \Pi^{\star} {{\bf{{v}}}}\Vert_{\widetilde{A}}^2\le c_{\Pi} \Vert {{\bf{{v}}}}\Vert_{A}^2$ 
for all ${{\bf{{v}}}}\in V$, where $\Pi^{\star}=\widetilde{A}^{-1}\Pi^T A$
denotes the operator satisfying the identity
$\langle \Pi \tilde{{{\bf{{u}}}}},{{\bf{{v}}}}\rangle_A = \langle\tilde{{{\bf{{u}}}}},\Pi^{\star}{{\bf{{v}}}} 
\rangle_{\widetilde{A}}$ for all $\tilde{{{\bf{{u}}}}} \in \widetilde{V}, {{\bf{{v}}}}\in V$.

\begin{remark}
While the results stated here provide uniform bounds on the condition
numbers of the preconditioned systems, it is also straightforward to
state similar results for a convergent iterative method. Indeed, 
if, for a large enough scaling parameter $\tau$,  we set 
$
C^{-1}=\tau \Pi_{\widetilde{D}} \widetilde{A}^{-1}
\Pi^T_{\widetilde{D}}$, then 
it holds that the related stationary iterative method is $A$-norm
convergent, namely,  $\Vert I-C^{-1}A\Vert_A < 1$. To keep the
presentation focused on preconditioning, however, we only consider
$\tau=1$. 
\end{remark}

We are now ready to state the theorem (see~\cite{Kraus_Lymb_Mar_2014})
which gives the condition number estimate. 
\begin{theorem}[Condition number estimate~\cite{Kraus_Lymb_Mar_2014}]\label{thm:KLM}
Under the assumptions~\eqref{eq:eb_1}--\eqref{eq:eb_3} the two-grid preconditioner $B$ 
defined in \eqref{eq:two-grid_3} and \eqref{eq:two-grid_1} satisfies
\begin{equation} \label{eq:eb_7}
  \frac{\underbar{c}}{\underbar{c}+\eta} \le    \lambda_{\min}(B^{-1}A)  
\le \lambda(B^{-1}A) \le     \lambda_{\max}(B^{-1}A)\le\bar{c}+c_{\Pi},
\end{equation}
that is, $\kappa(B^{-1}A)\le 
(\bar{c}+c_{\Pi})(\underbar{c}+\eta)/\underbar{c}$.
\end{theorem}

Next, we note that if no
  smoothing is applied, i.e. $B=C$, then the condition number estimate
  provided in Theorem~\ref{thm:KLM} also holds and we have the
  following result.
\begin{corollary} If $B=C$ and under the assumptions of
  Theorem~\ref{thm:KLM} we have the estimate
\begin{equation}\label{eq:pi}
\kappa(B^{-1}A)\le c_{\Pi}=c=\Vert\pi_{\widetilde{D}}
\Vert^2_{\widetilde{A}},
\quad\mbox{where}\quad
\pi_{\widetilde{D}} := R^T \Pi_{\widetilde{D}} .
\end{equation}
\end{corollary}

\subsubsection{Auxiliary space multigrid method}\label{sec:asmg}
Let $k=0,1,\dots,\ell-1$ be the index of mesh refinement where $k=0$ corresponds
to the finest mesh, i.e., $A^{(0)}:=A_h=A$ denotes the fine-grid matrix. Consider
the sequence of auxiliary space stiffness matrices $\widetilde{A}^{(k)}$,
in the two-by-two block factorized form 
\begin{equation}\label{factorizationK}
({\widetilde{A}}^{(k)})^{-1} = 
(\widetilde{L}^{(k)})^T \widetilde{D}^{(k)} \widetilde{L}^{(k)} ,
\end{equation}
where
\begin{equation}\label{factorizationKl}
\widetilde{L}^{(k)} =
\left [
\begin{array}{cc}
I & \\
-\widetilde{A}^{(k)}_{21} (\widetilde{A}^{(k)}_{11})^{-1} & I
\end{array}
\right ] , \quad
\widetilde{D}^{(k)} =
\left [
\begin{array}{cc}
(\widetilde{A}^{(k)}_{11})^{-1} & \\
& {Q^{(k)}}^{-1}
\end{array}
\right ] 
\end{equation}
and 
the additive Schur complement approximation $Q^{(k)}$ defines the next coarser level matrix, i.e.
\begin{equation}\label{factorizationK2}
 A^{(k+1)}:=Q^{(k)}.
\end{equation}

The algebraic multilevel iteration (AMLI)-cycle auxiliary space multigrid (ASMG) preconditioner
$B^{(k)}$
approximating $A^{(k)}$ is defined on all levels $k < \ell$ via the following relations
(see~\cite{Kraus_Lymb_Mar_2014})
 \begin{equation}\label{multigrid_preconditioner}
{B^{(k)}}^{-1} := 
{\overline{M}^{(k)}}^{-1} + (I - {M^{(k)}}^{-T} A^{(k)})
\Pi^{(k)} 
 (\widetilde{L}^{(k)})^T  \overline{D}^{(k)}
\widetilde{L}^{(k)} {\Pi^{(k)}}^T (I - A^{(k)} {M^{(k)}}^{-1}),
 \end{equation}
where
\begin{equation}
\overline{D}^{(k)} :=
\left [
\begin{array}{cc}
(\widetilde{A}^{(k)}_{11})^{-1} & \\
& B_{\nu}^{(k+1)}
\end{array}
\right ]
\end{equation}
and $B_{\nu}^{(k+1)}$
is an approximation of the inverse of the coarse-level matrix,
i.e., $B_{\nu}^{(k+1)} \approx {A^{(k+1)}}^{-1}$,
\begin{equation}
B_{\nu}^{(\ell)} := {A^{(\ell)}}^{-1}.
\end{equation} 

In the linear AMLI-cycle $B_{\nu}^{(k+1)}$ is a polynomial approximation of ${A^{(k+1)}}^{-1}$,
i.e. 
$$
\begin{array}{ccc}
 B_{\nu}^{(k+1)} & := & (I-p^{(k)}({B^{(k+1)}}^{-1}A^{(k+1)})){A^{(k+1)}}^{-1} \\
 & =: & q^{(k)}({B^{(k+1)}}^{-1}A^{(k+1)})){B^{(k+1)}}^{-1}
\end{array}
$$
that requires the action of ${B^{(k+1)}}^{-1}$. The classical choice of $p^{(k)}(t)$
is a scaled and shifted Chebyshev polynomial of degree $\nu_k=\nu$ where
$$
p^{(k)}(0)=1,\qquad q^{(k)}(t):=\frac{1-p^{(k)}(t)}{t}\approx \frac{1}{t}.
$$
Other polynomial approximations are possible, e.g., choosing $q^{(k)}(t)$ to be the
best approximation to $1/t$ in uniform norm, see \cite{Kraus2012polynomial}.

In case of the nonlinear AMLI-cycle ASMG method
$$B_{\nu}^{(k+1)}=B_{\nu}^{(k+1)}[\cdot]$$
is a nonlinear mapping whose action on a vector ${{\bf{{d}}}}$ is realized by $\nu$
iterations of a preconditioned Krylov subspace method.
In the following the generalized conjugate gradient method will serve this
purpose and hence we denote $B_{\nu}^{(k+1)}[\cdot]=B^{(k+1)}_{{\rm GCG,\nu}}[\cdot]$.

An important step in the construction is that performing $B_{\nu}^{(k+1)}[\cdot]$
one applies~\eqref{multigrid_preconditioner} also for preconditioning at level $(k+1)$.
Hence, \eqref{multigrid_preconditioner} becomes a nonlinear operator, too--we will
therefore write
$${B^{(k)}}^{-1}={B^{(k)}}^{-1}[\cdot], \quad \mbox{for all } k<\ell.$$

\subsubsection{Nonlinear ASMG algorithms}\label{sec:algorithms}

In the remainder of this section we will present two variants of nonlinear ASMG algorithms
for preconditioning the SPD matrices arising from discretization of the weighted bilinear form
\eqref{WH-div-prod}
and comment on some details of their implementation in the specific situation of using
lowest-order Raviart-Thomas elements on rectangles. 

On Figure \ref{fig:covering} we give an illustration of the {\it covering of ${\Omega}$} by
overlapping subdomains, these are $9$ staggered subdomains each of size $1/2$ of the 
original domain $\Omega$.

\begin{figure}[ht!]
 \includegraphics[width=0.2\textwidth]{element7}
\caption{Covering of the domain by nine overlapping subdomains}\label{fig:covering}
\end{figure}

Next, let us comment on the {\it partitioning~\eqref{eq:partitioning_DOF} of
the set $\mathcal{D}$ of DOF}.  We illustrate it in case of two grids 
coarse, ${\mathcal T}_H$, and fine, ${\mathcal T}_h$, where $H=2h$.
Then the corresponding sets of the edges are ${\mathcal E}_H$ and ${\mathcal E}_h$.
The following relations are obvious: $4|{\mathcal T}_H|= |{\mathcal T}_h| $
and $2|{\mathcal E}_H|+ 4|{\mathcal T}_H |= |{\mathcal E}_h| $.
Since in the setting of the lowest-order Raviart-Thomas finite elements it is not immediately
clear how to partition $\mathcal{D}$, we perform a preprocessing step which consists
of a {\it compatible two-level basis transformation}, see e.g. \cite{Kraus_12}. 
The global matrix $A$ is transformed 
according to
\begin{equation}\label{eq:A_hat}
 \widehat{A}=J^T A J, \quad  
\widehat{A}, J,  A  \in {\mathbb{R}}^{ |{\mathcal E}_h| \times  |{\mathcal E}_h| },
\end{equation}
where the transformation matrix $J$ is the product of a permutation matrix $P$ and another 
transformation matrix $J_{\pm}$, i.e.
\begin{equation}\label{eq:transformation_J}
 J=P J_{\pm}, \quad P, J_{\pm} \in {\mathbb{R}}^{ |{\mathcal E}_h| \times  |{\mathcal E}_h| }.
\end{equation}

The permutation $P$ allows us to provide 
a two-level numbering of the DOF that splits them into two groups, 
the first one consisting of DOF associated with fine-grid edges that are not part of 
any coarse-grid  edge 
(interior DOF) and the second one keeping all remaining DOF ordered such that any two that are 
on one edge have consecutive numbers. The transformation matrix $J_{\pm}$ in~\eqref{eq:transformation_J} 
is of the form
$$
J_{\pm}=\left[\begin{array}{cc}
    I& \\
    & J_{22}
    \end{array}\right],
\quad \mbox{where} \quad I \in {\mathbb{R}}^{(4 |{\mathcal{T}_H}|) \times (4 |{\mathcal{T}_H}|)},
$$
 $J_{22}$  has the form 
$$
J_{22}=\frac12 \left[\begin{array}{cccccccc}
    1 & -1 & & & & & & \\
    &  &  1 & -1 & & & & \\
    & & & &  \ddots & \ddots & & \\
    & & & & & & -1 & 1 \\
    1 & 1 & & & & & & \\
    &  &  1 & 1 & & & & \\
    & & & &  \ddots & \ddots & & \\
    & & & & & & 1 & 1 
    \end{array}\right],
\quad \mbox{where} \quad J_{22} \in {\mathbb{R}}^{(2|{\mathcal{E}_H}|) \times (2 |{\mathcal{E}_H}|)}.
$$

The analogous transformation is performed on a local level on each subdomain $\Omega_i$, i.e.
\begin{equation}\label{eq:A_i_hat}
\widehat{A}_i=J^T_i A_i J_i,
\end{equation} 
where again $J_i=P_i J_{\pm,i}$ with $P_i$ the permutation explained above but performed
on the degrees of freedom in the subdomain $\Omega_i$ and $ J_{\pm,i}$ has the 
same meaning but restricted to the subdomain $\Omega_i$. 

\begin{definition}
The global and local transformations will be called {\it compatible} if  
\begin{equation}\label{eq:compatibility1}
 \widehat{A}=J^T A J = \sum_{i=1}^{n_{\mathcal G}} \widehat{R}_i^T \widehat{A}_i \widehat{R}_i
\end{equation}
which is equivalent to 
\begin{equation}\label{eq:compatibility2}
 R_i J=J_i \widehat{R}_i.
\end{equation}
\end{definition}

The introduced transformation matrix $J$ defines the splitting of the DOF into coarse and fine, namely 
the FDOF correspond to the set of interior DOF and half differences on the coarse edges while the 
CDOF correspond to the half sums on the coarse edges.

Finally we formulate the following two algorithms (variants of nonlinear ASMG)
for solving linear systems with the matrix $A$ resulting from discretization
of~\eqref{WH-div-prod}.

The first algorithm applies the nonlinear ASMG method in the two-level basis, i.e.,
all matrices $A^{(k)}$ are transformed into two-level basis, i.e.,
$$\widehat{A}^{(k)}={J^{(k)}}^T A^{(k)} J^{(k)}, \quad \mbox{for all } k<\ell$$
and also the smoothing is performed in the two-level basis, i.e., using $\widehat{M}$
instead of $M$. 

\begin{algorithm}{Nonlinear ASMG method--Variant~I: Action of~\eqref{multigrid_preconditioner}
on a vector $\widehat{{{\bf{{d}}}}}$}\label{algorithm1} \\[-1ex]
\hrule \vspace{1ex}
\begin{tabular}{lll}
& Pre-smoothing: & $\widehat{{{\bf{{u}}}}} = (\widehat{M}^{(k)})^{-1} \widehat{{{\bf{{d}}}}}$ \\
& Auxiliary space correction: & $\left\{
\begin{array}{l}
\left(\begin{array}{c} \tilde{{{\bf{{q}}}}}_1 \\ \tilde{{{\bf{{q}}}}}_2 \end{array}\right)
:=\tilde{{{\bf{{q}}}}} = \Pi_{\widetilde{D}^{(k)}}^T (\widehat{{{\bf{{d}}}}} - \widehat{A}^{(k)}
\widehat{{{\bf{{u}}}}}) \\
\tilde{{{\bf{{p}}}}}_1 = (\widetilde{A}^{(k)}_{11})^{-1} \tilde{{{\bf{{q}}}}}_1 \\
\tilde{{{\bf{{p}}}}}_2 ={J^{(k+1)}}
B^{(k+1)}_{{\rm GCG,\nu}}[{J^{(k+1)}}^T (\tilde{{{\bf{{q}}}}}_2 -  \widetilde{A}^{(k)}_{21} \tilde{{{\bf{{p}}}}}_1)] \\
\tilde{{{\bf{{q}}}}}_1 = \tilde{{{\bf{{p}}}}}_1 - (\widetilde{A}^{(k)}_{11})^{-1} \widetilde{A}^{(k)}_{12} \tilde{{{\bf{{p}}}}}_2 \\
\tilde{{{\bf{{q}}}}}_2 = \tilde{{{\bf{{p}}}}}_2 \\
\widehat{{{\bf{{v}}}}} = \widehat{{{\bf{{u}}}}} + \Pi_{\widetilde{D}^{(k)}} \tilde{{{\bf{{q}}}}}
\end{array} \right.$ \\
& Post-smoothing: & $(\widehat{B}^{(k)})^{-1}[\widehat{{{\bf{{d}}}}}] :=
\widehat{{{\bf{{v}}}}} + (\widehat{M}^{(k)})^{-T} (\widehat{{{\bf{{d}}}}} - \widehat{A}^{(k)}
\widehat{{{\bf{{v}}}}})$
\end{tabular}
\\[1ex]
\hrule
\end{algorithm}

The second algorithm applies the nonlinear ASMG method directly in the original basis of
standard Raviart-Thomas basis functions and reads as follows. 
\begin{algorithm}{Nonlinear ASMG method--Variant~II: Action of~\eqref{multigrid_preconditioner}
on a vector ${{\bf{{d}}}}$}\label{algorithm2} \\[-1ex]
\hrule \vspace{1ex}
\begin{tabular}{lll}
& Pre-smoothing: & ${{\bf{{u}}}} = {M^{(k)}}^{-1} {{\bf{{d}}}}$ \\
& Auxiliary space correction: & $\left\{
\begin{array}{l}
\left(\begin{array}{c} \tilde{{{\bf{{q}}}}}_1 \\ \tilde{{{\bf{{q}}}}}_2 \end{array}\right)
:=\tilde{{{\bf{{q}}}}} = \Pi_{\widetilde{D}^{(k)}}^T {J^{(k)}}^T ({{\bf{{d}}}} - A^{(k)} {{\bf{{u}}}}) \\
\tilde{{{\bf{{p}}}}}_1 = (\widetilde{A}^{(k)}_{11})^{-1} \tilde{{{\bf{{q}}}}}_1 \\
\tilde{{{\bf{{p}}}}}_2 = B^{(k+1)}_{{\rm GCG,\nu}}[(\tilde{{{\bf{{q}}}}}_2 -  \widetilde{A}^{(k)}_{21} \tilde{{{\bf{{p}}}}}_1)] \\
\tilde{{{\bf{{q}}}}}_1 = \tilde{{{\bf{{p}}}}}_1 - (\widetilde{A}^{(k)}_{11})^{-1} \widetilde{A}^{(k)}_{12} \tilde{{{\bf{{p}}}}}_2 \\
\tilde{{{\bf{{q}}}}}_2 = \tilde{{{\bf{{p}}}}}_2 \\
{{\bf{{v}}}} = {{\bf{{u}}}} +  J^{(k)} \Pi_{\widetilde{D}^{(k)}} \tilde{{{\bf{{q}}}}}
\end{array} \right.$ \\
& Post-smoothing: & ${B^{(k)}}^{-1}[{{\bf{{d}}}}] := {{\bf{{v}}}} + {M^{(k)}}^{-T} ({{\bf{{d}}}} - A^{(k)} {{\bf{{v}}}})$
\end{tabular}
\\[1ex]
\hrule
\end{algorithm}
\begin{remark}
Note that the matrices $\widetilde{A}^{(k)}_{11}$, $\widetilde{A}^{(k)}_{12}$,
$\widetilde{A}^{(k)}_{21}$ are identical in both algorithms. 
\end{remark}
\begin{remark}
In general the matrices $\widehat{A}^{(k)}$ have slightly
more nonzero entries as compared to $A^{(k)}$ and thus Algorithm~\ref{algorithm2}
increases computational memory requirements.
\end{remark}
\begin{remark}
Considering the two-level preconditioners $\widehat{B}$ and $B$ defined according
to Algorithm~\ref{algorithm1} and Algorithm~\ref{algorithm2}, and assuming that no
smoothing is performed, i.e., $\widehat{M}=M=0$, the corresponding condition number
bounds for the preconditioned operators in two-level basis and standard basis read
$\kappa(\widehat{B}^{-1}\widehat{A})\le \Vert R^T \Pi_{\widetilde{D}} \Vert^2_{\widetilde{A}}$
and
$\kappa(B^{-1}A)\le \Vert R^T J \, \Pi_{\widetilde{D}} \Vert^2_{\widetilde{A}}$.

\end{remark}

\section{Numerical Experiments}\label{sec:numerics}

\subsection{Description of the parameters and the numerical test examples}

Subject to numerical testing are three representative cases of problems characterized by a highly 
varying coefficient ${{\alpha}}(x) = K^{-1}(x)$, namely:
\begin{enumerate}
\item[[a\hspace{-1ex}]] A binary distribution of the coefficient described by islands on which
 ${{\alpha}}=1.0$ against a background where ${{\alpha}}=10^{-q}$, see~Figure~\ref{fig:islands_binary}; 
 \item[[b\hspace{-1ex}]] Inclusions with ${{\alpha}}=1.0$ and a background with a coefficient 
 ${{\alpha}}={{\alpha}}_{T}=10^{-q_{rand}}$ that is constant on each element $\tau \in  {\mathcal T}_h$,
 where the random integer exponent $q_{rand}\in\{0,1,2,\dots,q\}$ is uniformly distributed,
see Figure~\ref{fig:islands_random};
\item[[c\hspace{-1ex}]] Three two-dimensional slices of the SPE10 
(Society of Petroleum Engineers) 
benchmark problem, see~\cite{SPE10_project},
where the contrast $\kappa$ is $10^7$ for slices 44 and 74 and $10^6$ for slice~54,
see~Figure~\ref{fig:spe_10}.
\end{enumerate}

\begin{figure}[hb]
\begin{center}
\subfigure[$32\times 32$ mesh]{
\includegraphics[width=0.21\textwidth]{islands_32}
\label{fig:islands_32}}
\hspace{10mm}
\subfigure[$128\times 128$ mesh]{
\includegraphics[width=0.21\textwidth]{islands_128}
\label{fig:islands_128}}
\hspace{10mm}
\subfigure[$512\times 512$ mesh]{
\includegraphics[width=0.21\textwidth]{islands_512}
\label{fig:islands_512}}
\caption{Binary distribution of the permeability $K(x)$ corresponding to test case [a]}
\label{fig:islands_binary}
 \end{center}
\end{figure}

\begin{figure}[hb]
\begin{center}
\subfigure[$32\times 32$ mesh]{
\includegraphics[width=0.21\textwidth]{islands_32r}
\label{fig:islands_32r}}
\hspace{10mm}
\subfigure[$128\times 128$ mesh]{
\includegraphics[width=0.21\textwidth]{islands_128r}
\label{fig:islands_128r}}
\hspace{10mm}
\subfigure[$512\times 512$ mesh]{
\includegraphics[width=0.21\textwidth]{islands_512r}
\label{fig:islands_512r}}
\caption{Random distribution of  ${{\alpha}}=K^{-1}(x)$ corresponding to test case [b]}
\label{fig:islands_random}
 \end{center}
\end{figure}

\begin{figure}[hb]
\begin{center}
\subfigure[Slice 44]{
\includegraphics[width=0.28\textwidth]{spe10_slice44}
\label{fig:slice44}}
\hspace{3mm}
\subfigure[Slice 54]{
\includegraphics[width=0.28\textwidth]{spe10_slice54}
\label{fig:slice54}}
\hspace{3mm}
\subfigure[Slice 74]{
\includegraphics[width=0.28\textwidth]{spe10_slice74}
\label{fig:slice74}}
\caption{Distributions of the permeability $K(x)$ along planes $x_3=44, 54, 74$ form 
the benchmark SPE10 on a $ 128 \times 128 $ mesh}
\label{fig:spe_10}
 \end{center}
\end{figure}

The numerical experiments are performed over a uniform mesh consisting of $N{\times}N$ elements (squares) 
where $N=4,8,\ldots,512$, i.e. up to $525 312$ velocity DOF and $262 144$ pressure DOF.
We have used a direct method to solve the problems on the coarsest grid.
The iterative process has been initialized with a random vector.
Its convergence has been tested for linear systems with right-hand side zero.
We have used overlapping coverings of the domain as shown in Figure~\ref{fig:covering}, 
where the subdomains are composed of $8\times 8$ elements and overlap with half of their
width/height. 
In the presentation of results we use the following notations:
\begin{itemize}
\item $\ell$ denotes the number of levels;
\item $q=\frac{\log \kappa}{\log 10}$ is the logarithm of the contrast $\kappa$;
\item $n_{ASMG}$ is the number of auxiliary space multigrid iterations;
\item $m\geq 0$ is the number of 
point Gauss-Seidel pre- and post-smoothing steps;
\item $\rho$ is the average convergence factor defined by
\begin{equation}\label{eq:average_factor}
\rho=\Bigg(\frac{\Vert u_{n_{ASMG}} \Vert}{\Vert u_0 \Vert}\Bigg)^{1/n_{ASMG}},
\end{equation}
where 
$u_{n_{ASMG}}$ is the
first iterate (approximate solution of \eqref{algebraic-hdiv}) for which the residual
has decreased by a factor of at least $10^8$.
\end{itemize}
The matrix $\widetilde{D}$ is as in~\eqref{eq:tilde_D} where 
$\widetilde{D}_{11}=\widetilde{A}_{11}$. This choice of $\widetilde{D}$ requires an
additional preconditioner for the iterative solution of linear systems with the matrix
$D=R \widetilde{D} R^T$, which is part of the efficient application of the operator
$\Pi_{\widetilde{D}}$. The systems with $D$ are solved using the preconditioned conjugate
gradient (PCG) method. The stopping criterion for this inner iterative process is a
residual reduction by a factor $10^6$, the number of PCG iterations to reach it--where
reported--is denoted by $n_i$. A robust and uniform preconditioner $B_{ILUE}$ for $D$
can be constructed based on incomplete factorization using exact local factorization (ILUE).
The definition of $B_{ILUE}$ is as follows:
$$
B_{ILUE}:=LU,\qquad U:=\sum_{i=1}^n R_i^T U_i R_i,\qquad L:=U^T {\rm diag}(U)^{-1}, 
$$ 
where 
$$
D_i=L_i U_i,\qquad D=\sum_{i=1}^n R_i^T D_i R_i, \qquad {\rm diag}(L_i)=I,
$$
for details see~\cite{Kraus2009robust}.
Note that as $D_i$ are the local contributions to $D$ related to the subdomains
$\Omega_i$, $i=1,\dots,n$, they are all non-singular.

The next two sections are devoted to the presentation of numerical results.
The experiments fall into two categories.
The first category, presented in Section~\ref{sec:Hdiv}, serves the evaluation of the
performance of the ASMG method on linear systems arising from discretization of the
weighted $H({\operatorname{div}})$ bilinear form~\eqref{WH-div-prod}. All three test cases, [a], [b],
and [c], are considered and both variants of the ASMG method, Algorithm~\ref{algorithm1}
and~\ref{algorithm2}, are compared, testing V- and W-cycle with and without smoothing.

The second category of experiments, discussed in Section~\ref{sec:saddle_point_sys},
addresses the solution of the indefinite linear system~\eqref{eq:saddle_point_sys}
arising from problem~\eqref{eq:dual_mixed} by a preconditioned MinRes method.
The main purposes are, on the one hand, to confirm the robustness of the block-diagonal
preconditioner~\eqref{eq:AFW_preconditioner} with respect to arbitrary multiscale
coefficient variations, and on the other hand, to demonstrate its numerical scalability.

\subsection{Numerical tests for solving the system~\eqref{algebraic-hdiv}}\label{sec:Hdiv}
Here we test the ASMG preconditioner for solving the system~\eqref{algebraic-hdiv} with
a matrix corresponding 
to the discretization
of the form $\Lambda_\alpha({{\mathbf u}},{{\mathbf v}})$. 
\begin{example} \label{ex:1} 
The first set of experiments is for the test cases~[a] and~[b].
In Tables~\ref{table:a_bilinear_alg1_V_m0}--\ref{table:b_bilinear_alg1_W_m1} we report the 
number of outer iterations $n_{ASMG}$ for the $\ell$-level V-cycle and W-cycle ASMG method
defined by
Algorithm~\ref{algorithm1}.
The coarsest mesh is composed of $4\times 4$ squares corresponding to $40$ DOF 
on the ${{\mathcal{RT}}_{{0}}}$ space. 
\end{example}

\begin{table}[ht!]
 \begin{center}
 \begin{tabular}{c| c  c | c  c | c c  | c c | c  c }
 \multicolumn {11}{c}{ASMG V-cycle: bilinear form~\eqref{WH-div-prod}, 
Algorithm~\ref{algorithm1}} \\
\multicolumn{1}{c}{~} & \multicolumn{2}{ c |}{$\ell=3$} & \multicolumn{2}{c|}{$\ell=4$} 
& \multicolumn{2}{c|}{$\ell=5$} & \multicolumn{2}{c|}{$\ell=6$}
& \multicolumn{2}{c}{$\ell=7$} 
\\
\cline{2-11}
& $n_{ASMG}$ & $\rho$ & $n_{ASMG}$ & $\rho$ & $n_{ASMG}$ & $\rho$   &   $n_{ASMG}$ & $\rho$ & $n_{ASMG}$ & $\rho$ \\% & $n_{it}$ & $\rho$ \\
\hline 
$q = 0$   & $4$ & $0.006$  & $6$ & $0.031$  & $8$  & $0.097$  & $11$ & $0.161$  & $12$ & $0.202$ \\% & ~$00$~ & ~$0.000$~ \\ 
$q = 1$   & $4$ & $0.005$  & $6$ & $0.034$  & $9$  & $0.115$  & $11$ & $0.169$  & $13$ & $0.224$ \\% & ~~ & ~~  \\
$q = 2$   & $4$ & $0.004$  & $6$ & $0.032$  & $10$ & $0.151$  & $12$ & $0.199$  & $13$ & $0.235$ \\% & ~~ & ~~  \\
$q = 3$   & $3$ & $0.002$  & $6$ & $0.031$  & $10$ & $0.158$  & $12$ & $0.211$  & $13$ & $0.235$ \\% & ~~ & ~~  \\
$q = 4$   & $4$ & $0.004$  & $6$ & $0.041$  & $11$ & $0.157$  & $13$ & $0.213$  & $14$ & $0.256$ \\% & ~~ & ~~   \\
$q = 5$   & $4$ & $0.006$  & $7$ & $0.053$  & $11$ & $0.185$  & $16$ & $0.314$  & $19$ & $0.366$ \\% & ~~ & ~~   \\
$q = 6$   & $4$ & $0.008$  & $8$ & $0.085$  & $15$ & $0.271$  & $24$ & $0.462$  & $25$ & $0.495$ \\% & ~~ & ~~   \\
\end{tabular} \vspace{2ex}
\caption{Example~\ref{ex:1}: case [a] with $K(x)=10^q$ and no smoothing steps ($m=0$)}\label{table:a_bilinear_alg1_V_m0}
 \end{center}
\end{table}

\begin{table}[h!]
 \begin{center}
 \begin{tabular}{c| c  c | c  c | c c  | c c | c  c }
 \multicolumn {11}{c}{ASMG V-cycle: bilinear form~\eqref{WH-div-prod}, 
Algorithm~\ref{algorithm1}} \\
\multicolumn{1}{c}{~} & \multicolumn{2}{ c |}{$\ell=3$} & \multicolumn{2}{c|}{$\ell=4$} 
& \multicolumn{2}{c|}{$\ell=5$} & \multicolumn{2}{c|}{$\ell=6$}
& \multicolumn{2}{c}{$\ell=7$} 
\\
\cline{2-11}
& $n_{ASMG}$ & $\rho$ & $n_{ASMG}$ & $\rho$ & $n_{ASMG}$ & $\rho$   &   $n_{ASMG}$ & $\rho$ & $n_{ASMG}$ & $\rho$ \\ 
\hline 
$q = 0$   & $4$ & $0.005$  & $5$ & $0.024$  & $6$ & $0.043$  & $8$ & $0.083$  & $8$  & $0.093$ \\% & ~$00$~ & ~$0.000$~ \\ 
$q = 1$   & $3$ & $0.002$  & $5$ & $0.022$  & $7$ & $0.058$  & $8$ & $0.084$  & $9$  & $0.121$ \\% & ~~ & ~~ \\
$q = 2$   & $3$ & $0.002$  & $5$ & $0.019$  & $7$ & $0.068$  & $8$ & $0.091$  & $9$  & $0.121$ \\% & ~~ & ~~ \\
$q = 3$   & $3$ & $0.002$  & $5$ & $0.018$  & $7$ & $0.070$  & $8$ & $0.095$  & $9$  & $0.125$ \\% & ~~ & ~~ \\
$q = 4$   & $3$ & $0.002$  & $5$ & $0.017$  & $7$ & $0.069$  & $8$ & $0.098$  & $10$ & $0.142$ \\% & ~~ & ~~ \\
$q = 5$   & $3$ & $0.002$  & $5$ & $0.017$  & $8$ & $0.082$  & $9$ & $0.118$  & $10$ & $0.145$ \\% & ~~ & ~~ \\
$q = 6$   & $4$ & $0.005$  & $4$ & $0.010$  & $8$ & $0.092$  & $9$ & $0.125$  & $11$ & $0.181$ \\% & ~~ & ~~ \\
\end{tabular} \vspace{2ex}
\caption{Example~\ref{ex:1}: case [a] with $K(x)=10^q$ and two smoothing steps ($m=2$)}\label{table:a_bilinear_alg1_V_m2}
 \end{center}
\end{table}

As we can see by comparing the results summarized in Tables~\ref{table:a_bilinear_alg1_V_m0}
and~\ref{table:a_bilinear_alg1_V_m2} the $V$-cycle multigrid preconditioner gains robustness
with respect to the contrast of a binary distribution of a piecewise constant permeability
coefficient when increasing the number of smoothing steps from zero to two. We further observe
an increase of the number of ASMG iterations for decreasing mesh size $h$ (in average $2$-$3$
times). At the same time, as seen in Table~\ref{table:a_bilinear_alg1_W_m1}, the $W$-cycle
preconditioner with one smoothing step is robust with respect to both the contrast and
the mesh size $h$.

\begin{table}[h!]
 \begin{center}
 \begin{tabular}{c| c  c | c  c | c c  | c c | c  c }
 \multicolumn {11}{c}{ASMG W-cycle: bilinear form~\eqref{WH-div-prod}, 
Algorithm~\ref{algorithm1}} \\
\multicolumn{1}{c}{~} & \multicolumn{2}{ c |}{$\ell=3$} & \multicolumn{2}{c|}{$\ell=4$} 
& \multicolumn{2}{c|}{$\ell=5$} & \multicolumn{2}{c|}{$\ell=6$}
& \multicolumn{2}{c}{$\ell=7$} 
\\
\cline{2-11}
& $n_{ASMG}$ & $\rho$ & $n_{ASMG}$ & $\rho$ & $n_{ASMG}$ & $\rho$   &   $n_{ASMG}$ & $\rho$ & $n_{ASMG}$ & $\rho$ \\ 
\hline 
$q = 0$   & $4$ & $0.005$  & $4$ & $0.007$  & $4$ & $0.006$  & $4$ & $0.005$  & $4$ & $0.005$ \\% & ~$0$~ & ~$0.000$~ \\ 
$q = 1$   & $3$ & $0.002$  & $4$ & $0.007$  & $4$ & $0.007$  & $4$ & $0.005$  & $4$ & $0.004$ \\% & ~~ & ~~ \\
$q = 2$   & $3$ & $0.002$  & $5$ & $0.015$  & $5$ & $0.017$  & $4$ & $0.006$  & $3$ & $0.001$ \\% & ~~ & ~~ \\
$q = 3$   & $3$ & $0.002$  & $5$ & $0.017$  & $5$ & $0.020$  & $4$ & $0.006$  & $3$ & $0.002$ \\% & ~~ & ~~ \\
$q = 4$   & $3$ & $0.002$  & $5$ & $0.016$  & $5$ & $0.019$  & $4$ & $0.005$  & $4$ & $0.005$ \\% & ~~ & ~~ \\
$q = 5$   & $3$ & $0.002$  & $4$ & $0.009$  & $6$ & $0.030$  & $4$ & $0.007$  & $4$ & $0.009$ \\ 
$q = 6$   & $4$ & $0.005$  & $5$ & $0.017$  & $5$ & $0.020$  & $5$ & $0.025$  & $7$ & $0.066$ \\ 
\end{tabular} \vspace{2ex}
\caption{Example~\ref{ex:1}: case [a], one smoothing step ($m=1$)}\label{table:a_bilinear_alg1_W_m1}
 \end{center}
\end{table}

In the next set of numerical experiments we consider the same
distribution of inclusions of low permeability
as before but this time against a background of a randomly distributed piecewise constant
permeability coefficient as shown on Figure~\ref{fig:islands_random}. The results, presented
in Tables~\ref{table:b_bilinear_alg1_V_m0} and~\ref{table:b_bilinear_alg1_V_m2}, are even
better than those obtained for the binary distribution in the sense that here both $V$-
and $W$-cycle are robust with respect to the contrast.

\begin{table}[h!]
 \begin{center}
 \begin{tabular}{c| c  c | c  c | c c  | c c | c  c }
 \multicolumn {11}{c}{ASMG V-cycle: bilinear form~\eqref{WH-div-prod},
Algorithm~\ref{algorithm1}} \\
\multicolumn{1}{c}{~} & \multicolumn{2}{ c |}{$\ell=3$} & \multicolumn{2}{c|}{$\ell=4$} 
& \multicolumn{2}{c|}{$\ell=5$} & \multicolumn{2}{c|}{$\ell=6$}
& \multicolumn{2}{c}{$\ell=7$} 
\\
\cline{2-11}
& $n_{ASMG}$ & $\rho$ & $n_{ASMG}$ & $\rho$ & $n_{ASMG}$ & $\rho$   &   $n_{ASMG}$ & $\rho$ & $n_{ASMG}$ & $\rho$ \\% & $n_{it}$ & $\rho$ \\
\hline 
$q = 0$   & $4$ & $0.007$  & $6$ & $0.027$  & $9$  & $0.102$  & $10$ & $0.156$  & $12$ & $0.210$ \\% & ~$00$~ & ~$0.000$~ \\ 
$q = 1$   & $4$ & $0.006$  & $6$ & $0.035$  & $9$  & $0.103$  & $11$ & $0.171$  & $13$ & $0.224$ \\% & ~~ & ~~ \\
$q = 2$   & $4$ & $0.005$  & $6$ & $0.032$  & $9$  & $0.102$  & $11$ & $0.159$  & $13$ & $0.222$ \\% & ~~ & ~~ \\
$q = 3$   & $4$ & $0.006$  & $6$ & $0.042$  & $9$  & $0.110$  & $11$ & $0.174$  & $13$ & $0.229$ \\% & ~~ & ~~ \\
$q = 4$   & $4$ & $0.006$  & $7$ & $0.043$  & $9$  & $0.127$  & $11$ & $0.183$  & $13$ & $0.233$ \\% & ~~ & ~~ \\
$q = 5$   & $4$ & $0.006$  & $7$ & $0.049$  & $10$ & $0.138$  & $12$ & $0.195$  & $13$ & $0.239$ \\% & ~~ & ~~ \\
$q = 6$   & $4$ & $0.006$  & $7$ & $0.056$  & $10$ & $0.149$  & $12$ & $0.207$  & $14$ & $0.252$ \\% & ~~ & ~~ \\
\end{tabular} \vspace{2ex}
\caption{Example~\ref{ex:1}: case [b], no smoothing steps ($m=0$)}\label{table:b_bilinear_alg1_V_m0}
 \end{center}
\end{table}
\begin{table}[ht!]
 \begin{center}
 \begin{tabular}{c| c  c | c  c | c c  | c c | c  c }
 \multicolumn {11}{c}{ASMG V-cycle: bilinear form~\eqref{WH-div-prod}, Algorithm~\ref{algorithm1}} \\
\multicolumn{1}{c}{~} & \multicolumn{2}{ c |}{$\ell=3$} & \multicolumn{2}{c|}{$\ell=4$} 
& \multicolumn{2}{c|}{$\ell=5$} & \multicolumn{2}{c|}{$\ell=6$}
& \multicolumn{2}{c}{$\ell=7$} 
\\
\cline{2-11}
& $n_{ASMG}$ & $\rho$ & $n_{ASMG}$ & $\rho$ & $n_{ASMG}$ & $\rho$   &   $n_{ASMG}$ & $\rho$ & $n_{ASMG}$ & $\rho$ \\% & $n_{it}$ & $\rho$ \\
\hline 
$q = 0$   & $4$ & $0.005$  & $5$ & $0.024$  & $6$ & $0.046$  & $8$ & $0.083$  & $8$  & $0.091$ \\% & ~$0$~ & ~$0.000$~ \\ 
$q = 1$   & $4$ & $0.005$  & $6$ & $0.033$  & $7$ & $0.060$  & $8$ & $0.091$  & $9$  & $0.124$ \\% & ~~ & ~~ \\
$q = 2$   & $3$ & $0.002$  & $5$ & $0.023$  & $6$ & $0.045$  & $7$ & $0.069$  & $9$  & $0.121$ \\% & ~~ & ~~ \\
$q = 3$   & $3$ & $0.002$  & $5$ & $0.021$  & $6$ & $0.043$  & $7$ & $0.071$  & $8$  & $0.100$ \\% & ~~ & ~~ \\
$q = 4$   & $4$ & $0.005$  & $5$ & $0.023$  & $6$ & $0.044$  & $8$ & $0.089$  & $9$  & $0.122$ \\% & ~~ & ~~ \\
$q = 5$   & $4$ & $0.005$  & $5$ & $0.024$  & $6$ & $0.045$  & $8$ & $0.090$  & $9$  & $0.125$ \\% & ~~ & ~~ \\
$q = 6$   & $4$ & $0.005$  & $6$ & $0.034$  & $6$ & $0.045$  & $8$ & $0.091$  & $10$ & $0.142$ \\% & ~~ & ~~ \\
\end{tabular} \vspace{2ex}
\caption{Example~\ref{ex:1}: case [b], two smoothing steps ($m=2$)}\label{table:b_bilinear_alg1_V_m2}
 \end{center}
\end{table}

\begin{table}[ht!]
 \begin{center}
 \begin{tabular}{c| c  c | c  c | c c  | c c | c  c }
 \multicolumn {11}{c}{ASMG W-cycle: bilinear form~\eqref{WH-div-prod}, Algorithm~\ref{algorithm1}} \\
\multicolumn{1}{c}{~} & \multicolumn{2}{ c |}{$\ell=3$} & \multicolumn{2}{c|}{$\ell=4$} 
& \multicolumn{2}{c|}{$\ell=5$} & \multicolumn{2}{c|}{$\ell=6$}
& \multicolumn{2}{c}{$\ell=7$} 
\\
\cline{2-11}
& $n_{ASMG}$ & $\rho$ & $n_{ASMG}$ & $\rho$ & $n_{ASMG}$ & $\rho$   &   $n_{ASMG}$ & $\rho$ & $n_{ASMG}$ & $\rho$  \\% & $n_{it}$ & $\rho$ \\
\hline 
$q = 0$   & $4$ & $0.005$  & $4$ & $0.007$  & $4$ & $0.006$  & $4$ & $0.005$  & $4$ & $0.005$ \\% & ~$0$~ & ~$0.000$~ \\ 
$q = 1$   & $4$ & $0.006$  & $4$ & $0.007$  & $4$ & $0.007$  & $4$ & $0.006$  & $4$ & $0.005$ \\% & ~~ & ~~ \\
$q = 2$   & $4$ & $0.004$  & $4$ & $0.009$  & $5$ & $0.016$  & $4$ & $0.007$  & $4$ & $0.006$ \\% & ~~ & ~~ \\
$q = 3$   & $4$ & $0.005$  & $5$ & $0.015$  & $5$ & $0.015$  & $4$ & $0.009$  & $4$ & $0.006$ \\% & ~~ & ~~ \\
$q = 4$   & $4$ & $0.005$  & $5$ & $0.016$  & $5$ & $0.016$  & $4$ & $0.009$  & $4$ & $0.008$ \\% & ~~ & ~~ \\
$q = 5$   & $4$ & $0.005$  & $5$ & $0.018$  & $5$ & $0.015$  & $4$ & $0.009$  & $4$ & $0.008$ \\% & ~~ & ~~ \\
$q = 6$   & $4$ & $0.005$  & $5$ & $0.019$  & $5$ & $0.015$  & $4$ & $0.008$  & $4$ & $0.007$ \\% & ~~ & ~~ \\
\end{tabular} \vspace{2ex}
\caption{Example~\ref{ex:1}: case [b], one smoothing step ($m=1$)}\label{table:b_bilinear_alg1_W_m1}
 \end{center}
\end{table}

\begin{example} \label{ex:2} 
The second set of experiments is related to the test case~[c] where, similarly to 
Example~\ref{ex:1}, we examine the performance of the preconditioner for the bilinear
form~\eqref{WH-div-prod}.  Here we compare the ASMG preconditioners defined by
Algorithm~\ref{algorithm1} and Algorithm~\ref{algorithm2}. In this example the
finest mesh is always composed of $256\times 256$ elements meaning that changing
the number of levels $\ell$ refers to a different size of the coarse-grid problem.
Tables~\ref{table:c44_bilinear_V}--\ref{table:c74_bilinear_W} report the number
of outer iterations $n_{ASMG}$ and the maximum number of inner iterations $n_i$ needed
to reduce the residual of the linear systems with the matrix $R\widetilde{D}R^T$ by a
factor of~$10^6$.
\end{example}

Tables~\ref{table:c44_bilinear_V} and~\ref{table:c44_bilinear_W} contain the results
for SPE10 slice 44. We see that while the number of inner iterations is about the
same, the number $n_{ASMG}$ of outer iterations in case of the V-cycle is on average
$2.5$ times higher than those for the W-cycle. However, since the complexity of the V-cycle
is lower, the overall performance of these two methods is comparable. Comparing the
two algorithms, we see that they have approximately the same number of inner and outer
iterations. Due to its lower memory requirements we could therefore recommend
Algorithm~\ref{algorithm1} for these kinds of problems. 
On Tables \ref{table:c74_bilinear_V}--\ref{table:c74_bilinear_W}  we present the results
for SPE10 slice 74, which has slightly different permeability distribution but has
the same contrast, $\kappa=10^7$. Computational results are pretty much the same for
this case as well.

\begin{table}[ht!]
 \begin{center}
 \begin{tabular}{c| c  c  c  | c c c | c c c | c c c }
 \multicolumn {13}{c}{ASMG V-cycle: bilinear form~\eqref{WH-div-prod}} \\
\multicolumn{1}{c}{~} & \multicolumn{6}{c|}{Algorithm~\ref{algorithm1}} & \multicolumn{6}{c}{Algorithm~\ref{algorithm2}}
\\
\cline{2-13}
 & \multicolumn{3}{ c |}{$m=0$} & \multicolumn{3}{c|}{$m=1$} 
& \multicolumn{3}{c|}{$m=0$} & \multicolumn{3}{c}{$m=1$}\\
\cline{2-13}
& $n_{ASMG}$ & $\rho$ & $n_{i}$ & $n_{ASMG}$ & $\rho$ & $n_{i}$   &   $n_{ASMG}$ & $\rho$ & $n_{i}$ & $n_{ASMG}$ & $\rho$ & $n_{i}$  \\
\hline 
$\ell = 3$   & $8$  & $0.080$ & $4$ & $7$  & $0.064$ & $5$    & $8$  & $0.090$ & $5$ & $8$  & $0.080$ & $5$   \\ 
$\ell = 4$   & $10$ & $0.157$ & $6$ & $9$  & $0.122$ & $6$    & $11$ & $0.184$ & $6$ & $10$ & $0.143$ & $6$   \\ 
$\ell = 5$   & $12$ & $0.209$ & $6$ & $10$ & $0.154$ & $6$    & $13$ & $0.231$ & $6$ & $11$ & $0.185$ & $6$   \\
$\ell = 6$   & $13$ & $0.239$ & $6$ & $11$ & $0.179$ & $6$    & $14$ & $0.264$ & $6$ & $12$ & $0.207$ & $6$   \\
$\ell = 7$   & $13$ & $0.239$ & $6$ & $11$ & $0.179$ & $6$    & $14$ & $0.264$ & $6$ & $12$ & $0.207$ & $6$   \\
\end{tabular} \vspace{2ex}
\caption{Example~\ref{ex:2}: case [c] - slice 44 of SPE10 benchmark}\label{table:c44_bilinear_V}
 \end{center}
\end{table}
\begin{table}[ht!]
 \begin{center}
 \begin{tabular}{c| c  c  c  | c c c | c c c | c c c }
 \multicolumn {13}{c}{ASMG W-cycle: bilinear form~\eqref{WH-div-prod} } \\
\multicolumn{1}{c}{~} & \multicolumn{6}{c|}{Algorithm~\ref{algorithm1}} & \multicolumn{6}{c}{Algorithm~\ref{algorithm2}}
\\
\cline{2-13}
 & \multicolumn{3}{ c |}{$m=0$} & \multicolumn{3}{c|}{$m=1$} 
& \multicolumn{3}{c|}{$m=0$} & \multicolumn{3}{c}{$m=1$}\\
\cline{2-13}
& $n_{ASMG}$ & $\rho$ & $n_{i}$ & $n_{ASMG}$ & $\rho$ & $n_{i}$   &   $n_{ASMG}$ & $\rho$ & $n_{i}$ & $n_{ASMG}$ & $\rho$ & $n_{i}$  \\
\hline 
$\ell = 3$   & $5$ & $0.019$ & $6$ & $5$ & $0.014$ & $5$    & $5$ & $0.025$ & $5$ & $5$ & $0.017$ & $6$    \\ 
$\ell = 4$   & $5$ & $0.019$ & $6$ & $5$ & $0.014$ & $5$    & $5$ & $0.025$ & $5$ & $5$ & $0.017$ & $6$    \\
$\ell = 5$   & $5$ & $0.019$ & $6$ & $5$ & $0.014$ & $5$    & $6$ & $0.026$ & $6$ & $5$ & $0.017$ & $6$    \\
$\ell = 6$   & $5$ & $0.019$ & $6$ & $5$ & $0.014$ & $5$    & $6$ & $0.026$ & $6$ & $5$ & $0.017$ & $6$    \\
$\ell = 7$   & $5$ & $0.019$ & $6$ & $5$ & $0.014$ & $5$    & $6$ & $0.026$ & $6$ & $5$ & $0.017$ & $6$    \\
\end{tabular} \vspace{2ex}
\caption{Example~\ref{ex:2}: case [c] - slice 44 of SPE10 benchmark}\label{table:c44_bilinear_W}
 \end{center}
\end{table}
\begin{table}[ht!]
 \begin{center}
 \begin{tabular}{c| c  c  c  | c c c | c c c | c c c }
 \multicolumn {13}{c}{ASMG V-cycle: bilinear form~\eqref{WH-div-prod}} \\
\multicolumn{1}{c}{~} & \multicolumn{6}{c|}{Algorithm~\ref{algorithm1}} & \multicolumn{6}{c}{Algorithm~\ref{algorithm2}}
\\
\cline{2-13}
 & \multicolumn{3}{ c |}{$m=0$} & \multicolumn{3}{c|}{$m=1$} 
& \multicolumn{3}{c|}{$m=0$} & \multicolumn{3}{c}{$m=1$}\\
\cline{2-13}
& $n_{ASMG}$ & $\rho$ & $n_{i}$ & $n_{ASMG}$ & $\rho$ & $n_{i}$   &   $n_{ASMG}$ & $\rho$ & $n_{i}$ & $n_{ASMG}$ & $\rho$ & $n_{i}$  \\
\hline 
$\ell = 3$   & $7$  & $0.070$ & $4$ & $7$  & $0.059$ & $4$    &  $8$  & $0.079$ & $4$ & $7$  & $0.069$ & $4$    \\ 
$\ell = 4$   & $10$ & $0.156$ & $5$ & $9$  & $0.122$ & $6$    &  $11$ & $0.173$ & $5$ & $10$ & $0.143$ & $6$    \\
$\ell = 5$   & $13$ & $0.236$ & $5$ & $11$ & $0.173$ & $6$    &  $14$ & $0.256$ & $5$ & $12$ & $0.196$ & $6$    \\
$\ell = 6$   & $14$ & $0.253$ & $5$ & $11$ & $0.183$ & $6$    &  $15$ & $0.283$ & $5$ & $12$ & $0.210$ & $6$    \\
$\ell = 7$   & $14$ & $0.253$ & $6$ & $11$ & $0.183$ & $6$    &  $15$ & $0.283$ & $5$ & $12$ & $0.210$ & $6$    \\
\end{tabular} \vspace{2ex}
\caption{Example~\ref{ex:2}: case [c] - slice 54 of SPE10 benchmark}\label{table:c54_bilinear_V}
 \end{center}
\end{table}
\begin{table}[ht!]
 \begin{center}
 \begin{tabular}{c| c  c  c  | c c c | c c c | c c c }
 \multicolumn {13}{c}{ASMG W-cycle: bilinear form~\eqref{WH-div-prod} } \\
\multicolumn{1}{c}{~} & \multicolumn{6}{c|}{Algorithm~\ref{algorithm1}} & \multicolumn{6}{c}{Algorithm~\ref{algorithm2}}
\\
\cline{2-13}
 & \multicolumn{3}{ c |}{$m=0$} & \multicolumn{3}{c|}{$m=1$} 
& \multicolumn{3}{c|}{$m=0$} & \multicolumn{3}{c}{$m=1$}\\
\cline{2-13}
& $n_{ASMG}$ & $\rho$ & $n_{i}$ & $n_{ASMG}$ & $\rho$ & $n_{i}$   &   $n_{ASMG}$ & $\rho$ & $n_{i}$ & $n_{ASMG}$ & $\rho$ & $n_{i}$  \\
\hline 
$\ell = 3$   & $5$ & $0.016$ & $4$ & $5$ & $0.013$ & $4$    & $5$ & $0.021$ & $4$ & $5$ & $0.016$ & $4$    \\ 
$\ell = 4$   & $5$ & $0.017$ & $6$ & $5$ & $0.013$ & $5$    & $5$ & $0.023$ & $6$ & $5$ & $0.017$ & $6$    \\
$\ell = 5$   & $5$ & $0.018$ & $6$ & $5$ & $0.013$ & $6$    & $5$ & $0.023$ & $6$ & $5$ & $0.017$ & $6$    \\
$\ell = 6$   & $5$ & $0.018$ & $6$ & $5$ & $0.013$ & $6$    & $5$ & $0.023$ & $6$ & $5$ & $0.017$ & $6$    \\
$\ell = 7$   & $5$ & $0.018$ & $6$ & $5$ & $0.013$ & $6$    & $5$ & $0.023$ & $6$ & $5$ & $0.017$ & $6$    \\
\end{tabular} \vspace{2ex}
\caption{Example~\ref{ex:2}: case [c] - slice 54 of SPE10 benchmark}\label{table:c54_bilinear_W}
 \end{center}
\end{table}
\begin{table}[ht!]
 \begin{center}
 \begin{tabular}{c| c  c  c  | c c c | c c c | c c c }
 \multicolumn {13}{c}{ASMG V-cycle: bilinear form~\eqref{WH-div-prod}} \\
\multicolumn{1}{c}{~} & \multicolumn{6}{c|}{Algorithm~\ref{algorithm1}} & \multicolumn{6}{c}{Algorithm~\ref{algorithm2}}
\\
\cline{2-13}
 & \multicolumn{3}{ c |}{$m=0$} & \multicolumn{3}{c|}{$m=1$} 
& \multicolumn{3}{c|}{$m=0$} & \multicolumn{3}{c}{$m=1$}\\
\cline{2-13}
& $n_{ASMG}$ & $\rho$ & $n_{i}$ & $n_{ASMG}$ & $\rho$ & $n_{i}$  &  $n_{ASMG}$ & $\rho$ & $n_{i}$
& $n_{ASMG}$ & $\rho$ & $n_{i}$  \\
\hline 
$\ell = 3$   & $8$  & $0.090$ & $4$ & $7$  & $0.070$ & $4$    & $8$  & $0.097$ & $4$ & $8$  & $0.087$ & $4$    \\ 
$\ell = 4$   & $11$ & $0.178$ & $5$ & $10$ & $0.145$ & $5$    & $12$ & $0.199$ & $5$ & $11$ & $0.162$ & $6$    \\
$\ell = 5$   & $13$ & $0.229$ & $5$ & $11$ & $0.166$ & $6$    & $14$ & $0.257$ & $5$ & $11$ & $0.183$ & $6$    \\
$\ell = 6$   & $13$ & $0.242$ & $6$ & $11$ & $0.180$ & $6$    & $15$ & $0.276$ & $6$ & $13$ & $0.219$ & $6$    \\
$\ell = 7$   & $13$ & $0.242$ & $6$ & $11$ & $0.180$ & $6$    & $15$ & $0.276$ & $6$ & $13$ & $0.219$ & $6$    \\
\end{tabular} \vspace{2ex}
\caption{Example~\ref{ex:2}: case [c] - slice 74 of SPE10 benchmark}\label{table:c74_bilinear_V}
 \end{center}
\end{table}
\begin{table}[ht!]
 \begin{center}
 \begin{tabular}{c| c  c  c  | c c c | c c c | c c c }
 \multicolumn {13}{c}{ASMG W-cycle: bilinear form~\eqref{WH-div-prod} } \\
\multicolumn{1}{c}{~} & \multicolumn{6}{c|}{Algorithm~\ref{algorithm1}} & \multicolumn{6}{c}{Algorithm~\ref{algorithm2}}
\\
\cline{2-13}
 & \multicolumn{3}{ c |}{$m=0$} & \multicolumn{3}{c|}{$m=1$} 
& \multicolumn{3}{c|}{$m=0$} & \multicolumn{3}{c}{$m=1$}\\
\cline{2-13}
& $n_{ASMG}$ & $\rho$ & $n_{i}$ & $n_{ASMG}$ & $\rho$ & $n_{i}$   &   $n_{ASMG}$ & $\rho$ & $n_{i}$ & $n_{ASMG}$ & $\rho$ & $n_{i}$  \\
\hline 
$\ell = 3$   & $5$ & $0.019$ & $4$ & $5$ & $0.014$ & $4$    & $5$ & $0.025$ & $4$ & $5$ & $0.017$ & $5$    \\ 
$\ell = 4$   & $5$ & $0.020$ & $5$ & $5$ & $0.015$ & $5$    & $6$ & $0.030$ & $5$ & $5$ & $0.018$ & $6$    \\
$\ell = 5$   & $5$ & $0.020$ & $6$ & $5$ & $0.015$ & $6$    & $6$ & $0.030$ & $6$ & $5$ & $0.018$ & $6$    \\
$\ell = 6$   & $5$ & $0.020$ & $6$ & $5$ & $0.015$ & $6$    & $6$ & $0.030$ & $6$ & $5$ & $0.018$ & $6$    \\
$\ell = 7$   & $5$ & $0.020$ & $6$ & $5$ & $0.015$ & $6$    & $6$ & $0.030$ & $6$ & $5$ & $0.018$ & $6$    \\
\end{tabular} \vspace{2ex}
\caption{Example~\ref{ex:2}: case [c] - slice 74 of SPE10 benchmark}\label{table:c74_bilinear_W}
 \end{center}
\end{table}

\subsection{Testing of block-diagonal preconditioner for system~\eqref{eq:saddle_point_sys}
within MinRes iteration}\label{sec:saddle_point_sys}
Now we present a number of numerical experiments for solving the mixed finite element
system~\eqref{eq:saddle_point_sys} by using
a preconditioned MinRes method.
We consider two different examples, first, Example~\ref{ex:3} in which the performance
of the block-diagonal preconditioner and its dependence on the accuracy of the inner
solves with W-cycle ASMG preconditioner is evaluated, and second, 
Example~\ref{ex:4}
that tests the scalability of the MinRes iteration, again using a W-cycle ASMG preconditioner
with one smoothing step for the inner iterations.
\begin{example}\label{ex:3} 
  Here we apply the MinRes iteration to
  solve~\eqref{eq:saddle_point_sys} for test case [c]. The hierarchy 
  of meshes is the same as in Example~\ref{ex:2}. 
An ASMG W-cycle
  based on Algorithm~\ref{algorithm1} with one smoothing step has been
  used as a preconditioner on the ${{\mathcal{RT}}_{{0}}}$
  space. Table~\ref{table:c44_saddle_W_m1} shows the number of MinRes
  iterations denoted by $n_{MinRes}$, the
  maximum number of ASMG iterations $n_{ASMG}$ needed to achieve an
  ASMG residual reduction by $\varpi$. 
\end{example}
\begin{table}[ht!]
 \begin{center}
 \begin{tabular}{c |  c  c  | c c | c c }
 \multicolumn {7}{c}{MinRes iteration: saddle point system~\eqref{eq:saddle_point_sys}} \\
\multicolumn{1}{c}{~} & \multicolumn{2}{c|}{$\varpi=10^6$} & \multicolumn{2}{c|}{$\varpi=10^8$} 
& \multicolumn{2}{c}{$\varpi=10^{10}$}
 \\
\cline{2-7}
&  $n_{MinRes}$ & $n_{ASMG}$ 
& $n_{MinRes}$ & $n_{ASMG}$  
& $n_{MinRes}$ & $n_{ASMG}$  
\\
\hline 
$\ell = 3$ & $24$ & $4$ 
& $17$ & $6$ 
& $15$ & $8$ 
\\ 
$\ell = 4$ & $15$ & $5$ 
& $13$ & $6$ 
& $13$ & $8$ 
\\
$\ell = 5$ & $21$ & $5$ 
& $17$ & $6$ 
& $15$ & $8$ 
\\
$\ell = 6$ & $22$ & $5$ 
& $17$ & $6$ 
& $15$ & $8$ 
\\
$\ell = 7$ & $22$ & $5$ 
& $17$ & $6$ 
& $15$ & $8$ 
\\
\end{tabular} \vspace{2ex}
\caption{Example~\ref{ex:3}: case [c] - slice 44 of SPE10 benchmark. The hierarchy 
  of meshes is the same as in Example~\ref{ex:2}.}\label{table:c44_saddle_W_m1}
 \end{center}
\end{table}

\begin{example}\label{ex:4}
In the last set of experiments the MinRes iteration has been used to 
solve~\eqref{eq:saddle_point_sys} for test case~[c] on the same hierarchy of 
meshes as in Example~\ref{ex:1}. An ASMG W-cycle based on Algorithm~\ref{algorithm1} with
one smoothing step has been used as a preconditioner on the ${{\mathcal{RT}}_{{0}}}$ space for a residual
reduction by $10^8$. Table~\ref{table:c44_saddle_W_m1_m} shows the number of MinRes
iterations $n_{MinRes}$,
the maximum number of (inner) ASMG iterations $n_{ASMG}$ per (outer) MinRes iteration,
and the number of DOF. Note that as long as the product $n_{MinRes}n_{ASMG}$ is constant, 
the total number of arithmetic operations required to achieve any prescribed 
accuracy is proportional to the number of DOF. 
\end{example}
\begin{table}[ht!]
 \begin{center}
 \begin{tabular}{c|ccr}
 \multicolumn {4}{c}{MinRes iteration: saddle point system~\eqref{eq:saddle_point_sys}} \\
 & $n_{MinRes}$ & $n_{ASMG}$ & \multicolumn{1}{c}{DOF}
\\
\hline 
$\ell = 4$ & $13$ & $5$ & $3,136$   
\\
$\ell = 5$ & $13$ & $6$ & $12,416$  
\\
$\ell = 6$ & $15$ & $6$ & $49,408$  
\\
$\ell = 7$ & $17$ & $6$ & $197,120$ 
\\
$\ell = 8$ & $17$ & $6$ & $787,456$ 
\\
\end{tabular} \vspace{2ex}
\caption{Example~\ref{ex:3}: case [c] - slice 44 of SPE10
  benchmark. The hierarchy of meshes is the same as in
  Example~\ref{ex:1}.}\label{table:c44_saddle_W_m1_m}
 \end{center}
\end{table}

\subsection{Some conclusions and general comments regarding the numerical experiments}

The presented numerical results clearly demonstrate the efficiency of the proposed algebraic
multilevel iteration (AMLI)-cycle auxiliary space multigrid (ASMG) preconditioner
for problems with highly varying
coefficients
as they typically arise in the mathematical modelling of physical processes in high-contrast
and high-frequency media. 

The first group of tests examines
the convergence behavior of the nonlinear ASMG method
for the weighted bilinear 
form~\eqref{WH-div-prod}. 
This is a key 
point in the presented study.
The cases~[a] and~[b] are designed to represent a typical multiscale geometry with
islands and channels. Although case~[b] (a background with a random coefficient)
appears to be more complicated, the impact of the multiscale heterogeneity seems to
be stronger in the binary case~[a] where the number of iterations is slightly larger. 
However, in both cases we
observe a uniformly converging
ASMG V-cycle 
with $m=2$ and W-cycle ($\nu=2$) with $m=1$. Some small 
fluctuations of the results
in the right-lower corner of the tables could be due to some round off effects. 
Case~[c] (SPE10) is
a popular benchmark problem in the petroleum engineering
community. Here we observe robust and uniform convergence with respect
to the number of levels $\ell$, or, equivalently, mesh-size $h$. Note
that such uniform convergence is present for the ASMG V-cycle even
without smoothing iterations (i.e.  $m=0$) and for both,
Algorithm~\ref{algorithm1} and Algorithm~\ref{algorithm2}. In
addition, Algorithm~\ref{algorithm2} is computationally more favorable
when compared to Algorithm~\ref{algorithm1} because
the matrices used in Algorithm~\ref{algorithm2}
have fewer non-zeroes (they are sparser).

The last two tables,
Table~\ref{table:c44_saddle_W_m1}--\ref{table:c44_saddle_W_m1_m},
confirm the expected optimal convergence rate
of the block-diagonally preconditioned MinRes iteration applied
to the coupled saddle point system~\eqref{eq:saddle_point_sys}.
The results in Table~\ref{table:c44_saddle_W_m1} demonstrate 
how the efficiency (in terms of the product $n_{MinRes} n_{ASMG}$) is achieved 
for a relative accuracy of $10^{-8}$ of the inner ASMG solver. 
Table \ref{table:c44_saddle_W_m1_m} illustrates
the scalability of the solver
indicated by a (almost) constant number of  MinRes and ASMG
iterations since the total computational work is
in terms of fine grid matrix vector multiplications is proportional to
the product $n_{MinRes}n_{ASMG}$.

Although not in the scope of this study, we note that the proposed
auxiliary space multigrid method would be suitable for implementation
on distributed memory computer architectures.

\appendix
\section{Discrete inf-sup condition}\label{append}
Here we provide a proof of the discrete inf-sup condition
\eqref{inf_sup_fin_set} for the bilinear form arising in the mixed
finite element method.  We begin by introducing some details regarding
the finite element spaces involved in the approximation of
problem~\eqref{eq:dual_mixed} or \eqref{eq:LS}.

\subsection{Raviart-Thomas-{N\'ed\'elec\ } space} 

We consider the standard lowest order
Raviart-Thomas-{N\'ed\'elec\ } space ${{\boldsymbol V}}_h$. 
Recall that every element ${{\mathbf v}}\in {{\boldsymbol V}}_h$ can be written as
\begin{equation}\label{eq:RT}
{{\mathbf v}} = \sum_{e\in \mathcal{E}_h} \sigma_e({{\mathbf v}} ){{\boldsymbol \psi}}_e({{\mathbf x}}),
\quad \sigma_e({{\mathbf v}})  = \int_e {{\mathbf v}}\cdot{{\mathbf n}}_e.
\end{equation}
Here the vector ${{\mathbf n}}_e$ has a fixed direction (normal to $e$), and
this direction is set once and for all for every
face $e\in \mathcal{E}_h$.  For an element $T\in \mathcal{T}_h$
let ${{\mathbf n}}_{e,T}$ define the unit normal vector to $e\in \partial
T\cap \mathcal{E}_h$ which points outward with respect to $T$. 
Now, if $e$ is the intersection of two elements from
$\mathcal{T}_h$, $e=T_e^+\cap T_e^-$, then $T^+_e$ is the element for
which  ${{\mathbf n}}_e\cdot{{\mathbf n}}_{e,T}=1$ and 
$T^{-}_e$ is the element for which 
${{\mathbf n}}_e\cdot{{\mathbf n}}_{e,T}=-1$.  
If $e$ is on  the boundary of the domain then the corresponding
element is $T^{+}_e$ and $T_e^{-}$ is missing. Finally, for a
piece-wise constant function $q\in W_h$ we denote
\[
q_{\pm,e} = q\big|_{T^\pm_e}, \quad e\in \mathcal{E}_h.
\]

The basis function ${{\boldsymbol \psi}}_e\in {{\boldsymbol V}_{\hspace{-0.2mm}h}}$, for $e\in \mathcal{E}_h$
corresponds to an edge/face $e\in \partial T$, for some $T\in {\mathcal T}_h$. If
$e$ is the face opposite to the vertex $P_e$ of the triangle/tetrahedron $T$,
then
\begin{equation}\label{eq:RT-basis}
{{\boldsymbol \psi}}_e\big|_T = \frac{c_d}{|T|}({{\mathbf x}}-{{\mathbf x}}_{P_e}).
\end{equation}
where $c_d$ is a constant depending only on the spatial dimension $d$
and is chosen so that $\int_e {{\boldsymbol \psi}}_e\cdot{{\mathbf n}}_e = 1$. 

  We note that explicit formulas similar to~\eqref{eq:RT-basis} 
are available also for the case of lowest order 
Raviart-Thomas-{N\'ed\'elec\ } elements on $d$-dimensional cuboids
(parallelograms or rectangular parallelepipeds). Indeed, if a
  cuboidal element $T \in \mathcal{T}_h$ is with faces
  parallel to the coordinate planes, this can be seen as
  follows. We denote by ${{\boldsymbol \psi}}_{k}^{\pm}$, the basis function which
  corresponds to a face $F_k^{\pm}$ with outward normal vector
  $\pm{{\mathbf e}}_k$, $k=1,\ldots,d$. Here, ${{\mathbf e}}_k$ is the $k$-th coordinate
  vector in $\mathbb{R}^d$. Let ${{\mathbf x}}_{M,k}^{\pm} \in \mathbb{R}^d$ be the mass center of
  the face $F_{k}^{\pm}$, $k=1,\ldots,d$. We then have
\begin{equation}\label{eq:RT-basis-cube}
{{\boldsymbol \psi}}_k^{\pm}({{\mathbf x}}) = \frac{({{\mathbf x}}-{{\mathbf x}}_{M,k}^{\mp})^T {{\mathbf e}}_k  }{|T|} {{\mathbf e}}_k .
\end{equation}
Here for simplicity we have used the notation ${{\mathbf x}}^T {{\mathbf e}}_k$ for the standard Euclidean inner 
product of vectors ${{\mathbf x}}, {{\mathbf e}}_k \in \mathbb{R}^d$. From this formula we see that over 
the finite element $T$ the basis functions
$ {{\boldsymbol \psi}}_k^{\pm}({{\mathbf x}})$ are linear of the variable $x_k$ and constant in the
remaining variables in ${\mathbb{R}}^d$. 

These are exactly $2d$ basis functions satisfying
\[
\int_{F_j^{\pm}}{{\boldsymbol \psi}}_k^+\cdot {{\mathbf n}}_{F^\pm_j} = \int_{F_k^{-}}{{\boldsymbol \psi}}_k^+\cdot {{\mathbf n}}_{F^-_k} = 0,
\;\mbox{for}\;  j\neq k,\; 
\;\mbox{and}\; 
\int_{F_k^{+}}{{\boldsymbol \psi}}_k^+\cdot {{\mathbf n}}_{F^+_k} = 1.
\]
with a similar relation for ${{\boldsymbol \psi}}_k^-$, $k=1,\ldots,d$. We note that all numerical experiments in the 
subsequent sections were done using these finite elements.

These
simple formulas also show that on a shape regular mesh we have:
\begin{equation}\label{eq:scaling}
\|{{\boldsymbol \psi}}_e\|_{0,T}^2 \sim h_e^{2-d},\quad h_e=\operatorname{diam}(e),\quad
\quad T^+\cap T^-=e\in \mathcal{E}_h. 
\end{equation}
The constants in the norm equivalence in \eqref{eq:scaling}
depend on the shape of the finite elements, but not on the
contrast $\kappa$.

We treat Dirichlet, Neumann boundary conditions (or mixture of these)
in a unified fashion. In order to do this we set $\mathcal{E}_h$ to
denote the set of faces (edges) of $\mathcal{T}_h$ minus the set of
faces (edges) on which we have Neumann conditions
and $|\mathcal{E}_h|$ is the number of these edges/faces. Thus, in the case of pure
Neumann conditions on $\partial \Omega$ the set of faces (edges) $\mathcal{E}_h$
is the set of interior faces (edges). Further, $W_h$ is the space of all
piece-wise constant scalar valued functions with zero mean value over $\Omega$, namely,
$\int_\Omega q=0$. In the case of pure Dirichlet conditions, $W_h$ is
the space of piece-wise constants (without restriction) and
$\mathcal{E}_h$ is the set of all faces (edges) in $\mathcal{T}_h$.
 
\subsection{Scalar products and discrete gradients}
We now define a weighted scalar product on ${{\boldsymbol V}_{\hspace{-0.2mm}h}}$: For a vector of
weights $\omega\in \mathbb{R}^{|\mathcal{E}_h|}$ we set
\[
({{\mathbf u}},{{\mathbf v}})_{*,\omega} = \sum_{e\in \mathcal{E}_h} 
\omega_{e}\sigma_{e}({{\mathbf u}}) \sigma_{e}({{\mathbf v}})\| {{\boldsymbol \psi}}_e\|^2, \quad
({{\mathbf u}},{{\mathbf v}})_{*,1} = \sum_{e\in \mathcal{E}_h} 
\sigma_{e}({{\mathbf u}}) \sigma_{e}({{\mathbf v}})\|{{\boldsymbol \psi}}_e\|^2.
\]
Note that $({{\mathbf u}},{{\mathbf v}})_{*,1}$ is an inner product for which
$\omega_e=1$ for all $e\in \mathcal{E}_h$. The corresponding norms are
denoted by $\|\cdot\|_{*,\omega}$ and $\|\cdot\|_{*,1}$. 

Next, with the vector $\omega\in \mathbb{R}^{|\mathcal{E}_h|}$, we
associate an operator $D_{\omega}:{{\boldsymbol V}_{\hspace{-0.2mm}h}}\mapsto {{\boldsymbol V}_{\hspace{-0.2mm}h}}$. The
action of the operator is determined by defining
$\sigma_e(D_{\omega}{{\mathbf v}})$ for $e\in \mathcal{E}_h$, since these
degrees of freedom uniquely determine any element from ${{\boldsymbol V}_{\hspace{-0.2mm}h}}$. We
set
\begin{equation}\label{eq:operator}
\sigma_e(D_{\omega} {{\mathbf v}}) = \omega_e\sigma_e({{\mathbf v}}). 
\end{equation}

The discrete gradient is defined as adjoint of the ``${\operatorname{div}}$'' in the 
$(\cdot,\cdot)_{*,1}$ inner product and denoted by ``$\nabla_h$''. For
$q\in W_h$, $\nabla_h q$ is the unique element in ${{\boldsymbol V}_{\hspace{-0.2mm}h}}$ satisfying
\begin{equation}\label{discrete-grad}
({{\mathbf v}},\nabla_h q)_{*,1}  = -({\operatorname{div}} {{\mathbf v}},q),
\quad\mbox{for all}\quad {{\mathbf v}}\in{{\boldsymbol V}_{\hspace{-0.2mm}h}}.
\end{equation}
A straightforward computation (by taking ${{\mathbf v}}={{\boldsymbol \psi}}_e$)  shows that 
$$
\sigma_e(\nabla_h q) = 
\frac{{\lbrack\!\lbrack {q} \rbrack\!\rbrack}_e}{\|{{\boldsymbol \psi}}_e\|^2}, \quad {\lbrack\!\lbrack {q} \rbrack\!\rbrack}_e = q_{-,e}-q_{+,e}.
$$
For Dirichlet boundary faces (edges) ${\lbrack\!\lbrack {q} \rbrack\!\rbrack}_e=q_{+,e}$. 
We have the following proposition:
\begin{proposition}\label{proposition:simple} 
Let ${{\mathbf v}}\in {{\boldsymbol V}_{\hspace{-0.2mm}h}}$ and $\omega_e>0$ for all
  $e\in \mathcal{E}_h$. We then have 
\begin{itemize}
\item[(i)] $({{\mathbf u}},{{\mathbf v}})_{*,\omega}= (D_\omega {{\mathbf u}}, {{\mathbf v}})_{*,1}$;
\item[(ii)]  The bilinear form on $W_h\times W_h$ defined by
\[
(D_\omega \nabla_h \varphi,\nabla_h \chi)_{*,1} = 
\sum_{e\in\mathcal{E}_h}
\frac{\omega_e}{\| {{\boldsymbol \psi}}_e\|^2}{\lbrack\!\lbrack {\varphi} \rbrack\!\rbrack}_e{\lbrack\!\lbrack {\chi} \rbrack\!\rbrack}_e
\]
is symmetric and positive definite on $W_h$. 
\end{itemize}
\end{proposition}
\begin{proof} (i) follows directly from the definition of
  $D_\omega$ and $(\cdot,\cdot)_{*,\omega}$. 

Regarding (ii), we take $\chi=\varphi$ and we obtain
\[
(D_\omega \nabla_h \varphi,\nabla_h \varphi)_{*,1} =
(\nabla_h \varphi,\nabla_h \varphi)_{*,\omega} = 
\sum_{e\in\mathcal{E}_h}
\frac{\omega_e}{\|{{\boldsymbol \psi}}_e\|^2}({\lbrack\!\lbrack {\varphi} \rbrack\!\rbrack}_e)^2\ge 0.
\]
To show that this inequality is strict (which implies positive
definiteness of the bilinear form) we observe that if we have Dirichlet
conditions on part of, or, the whole boundary then the matrix
corresponding to this bilinear form is a symmetric, weakly diagonally
dominant $M$-matrix and hence it is positive definite. If we have
Neumann conditions on $\partial\Omega$,  then setting 
$(\nabla_h \varphi,\nabla_h \varphi)_{*,\omega}=0$ implies that
$q_{+,e}=q_{-,e}$ for all $e\in \mathcal{E}_h$ and we conclude that $q$
is a constant on $\Omega$. Since the average of $q$ is $0$ (in case of
Neumann conditions) we conclude that $q=0$. Thus, the bilinear form
has a trivial kernel and is positive definite on $W_h$. The proof of (ii) is
complete. 
\end{proof}

\subsection{A discrete \emph{a priori} estimate}
Recall that $K(x)$ and $\alpha$ are a piece-wise constant
tensor-functions whose discontinuities are aligned with the
triangulation $\mathcal{T}_h$. For any face $e\in \mathcal{E}_h$, 
$e=\overline T_e^+\cap \overline T_e^{-}$ we
now define
\begin{equation}\label{eq:alphae}
\begin{aligned}
\widehat{\alpha}\in \mathbb{R}^{|\mathcal{E}_h|}, \quad 
\widehat{\alpha}_e =  
\frac{\| {{\boldsymbol \psi}}_e\|^2_{0,\alpha,T_e^+}}{\| {{\boldsymbol \psi}}_e\|^2}+
\frac{\|{{\boldsymbol \psi}}_e\|^2_{0,\alpha,T_e^-}}{\| {{\boldsymbol \psi}}_e\|^2},
\quad \widehat{\kappa}\in \mathbb{R}^{|\mathcal{E}_h|}, 
\quad \widehat{\kappa}_e = \widehat{\alpha}_e^{-1}.
\end{aligned}
\end{equation}
We have the following lemma which is a crucial ingredient in the proof
of the discrete inf-sup condition. 
\begin{lemma}\label{lemma:mass-equivalence} 
Let $\widehat{\alpha}\in\mathbb{R}^{|\mathcal{E}_h|}$ and
$\widehat{\kappa}\in\mathbb{R}^{|\mathcal{E}_h|}$ be defined as
in~\eqref{eq:alphae}. Then
\begin{itemize}
\item[(i)]  
$\|{{\mathbf u}}\|^2_{0,\alpha} \le
  (d+1)\|{{\mathbf u}}\|^2_{*,\widehat{\alpha}}$ and 
$\|{{\mathbf u}}\|^2 \eqsim \|{{\mathbf u}}\|^2_{*,1}$.
\item[(ii)] $D_{\widehat{\alpha}}D_{\widehat{\kappa}} = I$.
\item[(iii)] $\widehat{\alpha}_e\le 2$ and  
$\widehat{\kappa}_e\ge \frac12$.  
\end{itemize}
\end{lemma}
\begin{proof}
To prove the first inequality in (i), let ${{\mathbf u}}\in
    {{\boldsymbol V}}_h$. Then, for all $T\in T_h$ we have ${{\mathbf u}}\big|_T =
    \sum_{e\subset\partial T}\sigma_e({{\mathbf u}}) {{\boldsymbol \psi}}_e$,  and,  hence, 
\begin{eqnarray*}
\|{{\mathbf u}}\|^2_{0,\alpha} & = & 
\sum_T \int_T \alpha_T{{\mathbf u}}\cdot {{\mathbf u}}  
=  
\sum_T 
\sum_{e\in\partial T}\sum_{e'\in \partial T}
\sigma_e({{\mathbf u}})\sigma_{e'}({{\mathbf u}})\int_T \alpha_T{{\boldsymbol \psi}}_e\cdot {{\boldsymbol \psi}}_{e'}
\\[1.7ex]
&\le &
\sum_T 
\sum_{e\in\partial T}\sum_{e'\in \partial T}
|\sigma_e({{\mathbf u}})||\sigma_{e'}({{\mathbf u}})|\|{{\boldsymbol \psi}}_e\|_{0,\alpha,T}\|{{\boldsymbol \psi}}_{e'}\|_{0,\alpha,T}
\\[1.7ex]
&\le &
\frac12\sum_T 
\sum_{e\in\partial T}\sum_{e'\in \partial T}
\left([\sigma_e({{\mathbf u}})]^2\|{{\boldsymbol \psi}}_e\|^2_{0,\alpha,T}+
[\sigma_{e'}({{\mathbf u}})]^2\|{{\boldsymbol \psi}}_{e'}\|^2_{0,\alpha,T}\right)
\\[1.7ex]
& = & 
(d+1)\sum_T 
\sum_{e\in\partial T}[\sigma_e({{\mathbf u}})]^2\|{{\boldsymbol \psi}}_e\|^2_{0,\alpha,T}. 
\end{eqnarray*}
In the last step we have used $\sum_{e\in \partial T} 1 = (d+1)$. 
Switching the order of summation in the right side of the inequality above then gives 
\begin{equation}\label{eq:norm-equivalence}
\begin{array}{rcl}
\|{{\mathbf u}}\|^2_{0,\alpha} & \le & 
(d+1)\sum_{e\in\mathcal{E}_h}\left(\| {{\boldsymbol \psi}}_e\|^2_{0,\alpha,T^+_e}
  +\|{{\boldsymbol \psi}}_e\|^2_{0,\alpha,T_e^-} 
\right)
[\sigma_e({{\mathbf u}})]^2 \\[1.7ex]
& = & 
(d+1)\sum_{e\in\mathcal{E}_h}\widehat{\alpha}_{e}[\sigma_e({{\mathbf u}})]^2\|  {{\boldsymbol \psi}}_e\|^2
 = (d+1)\|{{\mathbf u}}\|^2_{*,\widehat{\alpha}}.
\end{array}
\end{equation}
To prove the equivalence in (i) for $\widehat{\alpha}_e=1$,
$e\in\mathcal{E}_h$, we observe that for $T\in \mathcal T_h$ 
we have the following equivalence  relation
with constants only depending on the geometry of $T$ 
(bounded for shape regular triangulation):
\begin{equation}\label{eq:mass-matrix}
\sum_{e\in\partial T}\sum_{e'\in \partial T}
 \sigma_e({{\mathbf u}} )\sigma_{e'}({{\mathbf u}})\int_T {{\boldsymbol \psi}}_e {{\boldsymbol \psi}}_{e'}
 \eqsim \sum_{e\in\partial T} [\sigma_e({{\mathbf u}})]^2\| {{\boldsymbol \psi}}_e\|^2.
\end{equation}
The equivalence \eqref{eq:mass-matrix}  is just spectral equivalence between the
  element mass matrix 
$\{({{\boldsymbol \psi}}_e,{{\boldsymbol \psi}}_{e'})\}_{e,e'\in\partial T}$
  and its diagonal $\{\|{{\boldsymbol \psi}}\|^2\}_{e\in\partial T}$ which is easy 
  to establish, for example, by using the explicit form of the basis 
  functions given in~\eqref{eq:RT-basis}
  and~\eqref{eq:RT-basis-cube}. 
This completes the proof of (i). 

Items (ii) and (iii) directly follow from the definitions and
the inequality~\eqref{bound_below_K} which implies that
$\|\alpha\|_{\ell^2}\le 1$.  
\end{proof}

An immediate corollary from Proposition~\ref{proposition:simple}
and Lemma~\ref{lemma:mass-equivalence} is the following  {Poincar\'e } inequality.
\begin{corollary}\label{lemma:Poincare} 
If $\omega_e = 1$ for all $e\in \mathcal{E}_h$ we have 
\begin{equation}\label{eq:Poincare}
\|\chi\|^2 \le C_P\|\nabla_h \chi\|^2_{*,1}, 
\end{equation}
with a constant $C_P$ independent of $h$. 
\end{corollary}
\begin{proof}
  To prove this {Poincar\'e }{} inequality we recall a classical result on the solvability of the
  mixed discretizations of the Laplace equation. It is well known
  (see, for example, \cite[(7.1.28) and Proposition
  5.4.3]{2013BoffiD_BrezziF_FortinM-aa}; also \cite{2002ArnoldD-aa}), that the
  following estimate holds with $\widetilde{\gamma}>0$ independent of
  $h$ for all $q\in W_h$:
\begin{equation}\label{eq:inf-sup-1}
\sup_{{{\mathbf v}}\in {{\boldsymbol V}_{\hspace{-0.2mm}h}}}\frac{(q,{\operatorname{div}} {{\mathbf v}})}{\|{{\mathbf v}}\|_{\Lambda_1}}\ge \widetilde{\gamma}\|q\|. 
\end{equation}
Since by Lemma~\ref{lemma:mass-equivalence}(i) the norms $\|\cdot\|$
and $\|\cdot\|_{*,1}$ are equivalent norms on ${{\boldsymbol V}}_h$, for any
$\chi\in W_h$:
\begin{eqnarray*}
\|\nabla_h \chi\|_{*,1} 
&= &\sup_{{{\mathbf v}}\in {{\boldsymbol V}}_h}\frac{(\nabla_h \chi,{{\mathbf v}})_{*,1}}{\|{{\mathbf v}}\|_{*,1}}
= \sup_{{{\mathbf v}}\in {{\boldsymbol V}}_h}\frac{(\chi,{\operatorname{div}} {{\mathbf v}})}{\|{{\mathbf v}}\|_{*,1}}\\
&\gtrsim& \sup_{{{\mathbf v}}\in {{\boldsymbol V}}_h}\frac{(\chi,{\operatorname{div}} {{\mathbf v}})}{\|{{\mathbf v}}\|}
\ge \sup_{{{\mathbf v}}\in {{\boldsymbol V}}_h}\frac{(\chi,{\operatorname{div}} {{\mathbf v}})}{\|{{\mathbf v}}\|_{\Lambda_1}}
\ge \widetilde{\gamma}\|\chi\|.
\end{eqnarray*}
Therefore, \eqref{eq:Poincare} holds with $C_P=\widetilde{\gamma}^{-2}$. 
\end{proof}

\begin{remark}
Clearly, for a scalar coefficient $K(x)$ the edge weights $\widehat{\kappa}_e$ are weighted 
harmonic averages of the values of the coefficient on the
two elements sharing the face $e$. On a uniform mesh, where 
$
\| {{\boldsymbol \psi}}_e\|^2_{0,\alpha,T_e^\pm}/\| {{\boldsymbol \psi}}_e\|^2= 
|\alpha_{T_e^{\pm}}|/2$ we have that 
\[
\widehat{\kappa}_e 
=\frac{2K_{+,e}K_{-,e}}{K_{+,e}+K_{-,e}}
\]
On a non-uniform mesh or for tensor coefficient $K(x)$ 
the average is more complicated and involves weights depending
on geometrical quantities describing pairs of neighboring elements. 
\end{remark}

Consider now $q\in W_h$ 
and let $\varphi\in W_h$ be the solution of the variational problem
\begin{equation}\label{eq:variational}
(D_{\widehat{\kappa}}\nabla_h \varphi,\nabla_h \chi)_{*,1} = (q,\chi),\quad \mbox{for all}\quad \chi\in 
W_h. 
\end{equation}
In accordance with Proposition~\ref{proposition:simple}(ii) the
solution to the variational problem~\eqref{eq:variational} 
$\varphi$ exists and is unique. 
We further set 
\begin{equation}\label{eq:def-w}
{{\mathbf w}} = D_{\widehat{\kappa}}\nabla_h \varphi
\end{equation}
and by the definition of $\nabla_h$ we have 
$-({\operatorname{div}} D_{\widehat{\kappa}}\nabla_h 
\varphi,\chi)_{*,1} = (q,\chi)$ for all $\chi\in W_h$, which implies that
\begin{equation}\label{eq:div-w}
{\operatorname{div}} D_{\widehat{\kappa}}\nabla_h\varphi=-q.
\end{equation}

Next, we prove an a priori estimate needed for the proof of a contrast
independent inf-sup condition.
\begin{lemma}\label{lemma:a-priori}
  Let $\varphi\in W_h$ be the solution of~\eqref{eq:variational} and
  let ${{\mathbf w}} \in {{\boldsymbol V}_{\hspace{-0.2mm}h}}$ be defined as in~\eqref{eq:def-w}, namely, 
  ${{\mathbf w}} =D_{\widehat{\kappa}}\nabla_h\varphi$. Then the following 
\emph{a  priori} estimate holds
\begin{equation}\label{eq:a-priori}
\|{{\mathbf w}}\|^2_{*,\widehat{\alpha}} \le 2C_P\|q\|^2.
\end{equation}
\end{lemma}
\begin{proof}
The proof follows from Proposition~\ref{proposition:simple}, 
Lemma~\ref{lemma:mass-equivalence} and some simple equalities. We have
\begin{equation*}
\begin{aligned}
\|{{\mathbf w}}\|^2_{*,\widehat{\alpha}}& =  
(D_{\widehat{\alpha}}{{\mathbf w}},{{\mathbf w}})_{*,1} 
&\qquad\mbox{[by Proposition~\ref{proposition:simple}(i)]}\\
& = 
(D_{\widehat{\alpha}}\;D_{\widehat{\kappa}}\nabla_h\varphi,{{\mathbf w}})_{*,1}
& \qquad\mbox{[by the definition of ${{\mathbf w}}$]}
\\
& =   (\nabla_h\varphi,{{\mathbf w}})_{*,1}
&\qquad \mbox{by [Lemma~\ref{lemma:mass-equivalence}(ii)]}\\
&=  -(\varphi,{\operatorname{div}} {{\mathbf w}})=(\varphi, q) \le  \|\varphi\|\|q\|. 
&\qquad\mbox{[by~\eqref{eq:div-w}]}
\end{aligned}
\end{equation*}

We now use the {Poincar\'e } inequality to estimate $\|\varphi\|$:
\begin{equation*}
\begin{aligned}
\|\varphi\|^2 &\le C_P(\nabla_h\varphi,\nabla_h\varphi)_{*,1} 
&\qquad\mbox{[by \eqref{eq:Poincare}]}\\
&\le 2C_P(D_{\widehat{\kappa}}\nabla_h\varphi,\nabla_h\varphi)_{*,1}
&\qquad\mbox{[by Lemma~\ref{lemma:mass-equivalence}(iii)]}\\
&=
2C_P(D_{\widehat{\kappa}}\nabla_h\varphi,D_{\widehat{\alpha}}D_{\widehat{\kappa}}
  \nabla_h\varphi)_{*,1},
&\qquad\mbox{[by Lemma~\ref{lemma:mass-equivalence}(ii)]}\\
& =  2C_P\|{{\mathbf w}}\|^2_{*,\widehat{\alpha}} 
& \qquad\mbox{[by the definition of ${{\mathbf w}}$]}
\end{aligned}
\end{equation*}
Note that  we have used the inequality  
$\widehat{\kappa}_e\ge \frac12$ for all $e\in \mathcal{E}_h$. 
\end{proof}

After all the preparatory work, we are ready to prove the discrete
inf-sup condition. 
\begin{theorem}\label{theorem:inf-sup} The following inequality holds
  with constant $\gamma>0$ independent of the contrast $\kappa$:
\begin{equation}
\inf_{q\in W_h}\sup_{{{\mathbf v}}\in {{\boldsymbol V}_{\hspace{-0.2mm}h}}}\frac{(q,{\operatorname{div}}
  {{\mathbf v}})}{\|{{\mathbf v}}\|_{\Lambda_\alpha}\;\|q\|}\ge \gamma.
\end{equation}
\end{theorem}
\begin{proof}
For all $q\in W_h$  and with ${{\mathbf w}}$ defined as in~\eqref{eq:def-w} we have  
\begin{equation*}
\begin{aligned}
  \sup_{{{\mathbf v}}\in {{\boldsymbol V}_{\hspace{-0.2mm}h}}} \frac{(q,{\operatorname{div}} {{\mathbf v}})}{\|{{\mathbf v}}\|_{\Lambda_\alpha}} &
  \ge \|q\|^2 /\left (\|{{\mathbf w}}\|_{0,\alpha}^2+\|{\operatorname{div}} {{\mathbf w}}\|^2
    \right )^\frac12  &\qquad \mbox{[by the definition of $\sup$]}
  \\
  &\ge  \|q\|^2 / \left (  (d+1)\|{{\mathbf w}}\|^2_{*,\widehat{\alpha}}+\|{\operatorname{div}} {{\mathbf w}}\|^2
      \right )^\frac12 &\qquad 
\mbox{[by      Lemma~\ref{lemma:mass-equivalence}(i)]} 
\\
&= \|q\|^2 / \left ( (d+1)\|{{\mathbf w}}\|^2_{*,\widehat{\alpha}}+\|q\|^2 \right )^\frac12
&\qquad \mbox{[by \eqref{eq:div-w}]}\\
&\ge \|q\|/ \sqrt{2(d+1)C_P+1},
& \qquad \mbox{[by Lemma~\ref{lemma:a-priori}]}
\end{aligned}
\end{equation*}
which completes the proof.
\end{proof}
The standard theory of mixed finite element methods then immediately
implies the following approximation result
\begin{theorem}\label{theorem:cea} 
Let $({{\mathbf u}},p)\in {{\boldsymbol H}}_N({\operatorname{div}})\times  L_2$ be the solution of the mixed 
problem~\eqref{eq:dual_mixed} and
  $({{\mathbf u}}_h,p_h)\in {{\boldsymbol V}_{\hspace{-0.2mm}h}}\times W_h$ be the solution of the
  discrete problem~\eqref{eq:dual_mixed_FEM}. Then the following estimate
  holds
\begin{equation}\label{eq:cea}
  \|{{\mathbf u}}-{{\mathbf u}}_h\|_{\Lambda_\alpha} +\|p-p_h\|
\le 
C \left( \inf_{{{\mathbf v}}\in {{\boldsymbol V}_{\hspace{-0.2mm}h}}}\|{{\mathbf u}}-{{\mathbf v}}\|_{\Lambda_\alpha}+
\inf_{q\in W_h}\|p-q\|\right),
\end{equation}
where the constant $C$ 
depend on the shape
regularity of the mesh, the Poincar\'e constant $C_P$, and the spatial dimension $d$, but is
independent of the variations in $K(x)$. 
\end{theorem}

\section*{Acknowledgments}
This work has been partially supported by the Bulgarian NSF Grant DCVP
02/1, FP7 Grant AComIn, and the Austrian NSF Grant P22989. R. Lazarov
has been supported in parts by US NSF Grant DMS-1016525, by Award
No. KUS-C1-016-04, made by KAUST. L. Zikatanov has been supported in
part by US NSF Grants DMS-1418843, DMS-1217142 and DMS-1016525.

\bibliographystyle{abbrv}
\bibliography{for_arxiv}
\end{document}


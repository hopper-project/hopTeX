

\documentclass{amsart}

\usepackage[mathscr]{eucal}
\usepackage{amssymb}
\usepackage[usenames,dvipsnames]{color}
\usepackage{amsthm}
\usepackage{bbold}
\usepackage{enumerate}
\usepackage{array}
\usepackage{amsmath}

\usepackage[colorlinks=true]{hyperref}

\numberwithin{equation}{section}
\setcounter{tocdepth}{1}

\usepackage[all]{xy}
\SelectTips{cm}{}

\newdir{ >}{{}*!/-10pt/\dir{>}}

\swapnumbers 

\newtheorem*{Thm*}{Theorem}
\newtheorem{Cor}[equation]{Corollary}
\newtheorem{Lem}[equation]{Lemma}
\newtheorem{Prop}[equation]{Proposition}
\newtheorem{Thm}[equation]{Theorem}

\theoremstyle{remark}
\newtheorem{Def}[equation]{Definition}
\newtheorem{Exa}[equation]{Example}
\newtheorem{Hyp}[equation]{Hypotheses}
\newtheorem{Not}[equation]{Notation}
\newtheorem{Rem}[equation]{Remark}
\newtheorem{Que}[equation]{Question}

{coker}{\colim}{colim}
{coker}{\cone}{cone}
{coker}{\End}{End}
{coker}{\Hom}{Hom}
{coker}{\Id}{Id}
{coker}{\Ind}{Ind}
{coker}{\Ker}{Ker}
{coker}{\opname}{op}
{coker}{\supp}{supp}
{coker}{\Spc}{Spc}
{coker}{\Spec}{Spec}

{coker}{\DM}{DM}
{coker}{\im}{im}
{coker}{\SH}{SH}

\begin{document}

\title[On surjectivity on tensor-triangular spectra]{On the surjectivity of the map of spectra\\associated to a tensor-triangulated functor}
\author{Paul Balmer}
\date{\today}

\address{Paul Balmer, Mathematics Department, UCLA, Los Angeles, CA 90095-1555, USA}
\email{balmer@math.ucla.edu}
\urladdr{http://www.math.ucla.edu/$\sim$balmer}

\begin{abstract}
We prove a few results about the map $\Spc(F)$ induced on tensor-triangular spectra by a tensor-triangulated functor~$F$. First, $F$ is conservative if and only if $\Spc(F)$ is surjective on closed points. Second, if $F$ detects tensor-nilpotence of morphisms then $\Spc(F)$ is surjective on the whole spectrum. In fact, surjectivity of~$\Spc(F)$ is equivalent to $F$ detecting the nilpotence of some class of morphisms, namely those morphisms which are nilpotent on their cone.
\end{abstract}

\subjclass{18E30; 14F42, 19K35, 55U35}
\keywords{Tensor-triangular, spectra, nilpotence, conservative, surjectivity}

\thanks{Research supported by NSF grant~DMS-1600032.}

\maketitle

\section{Introduction}

\begin{Hyp}
\label{hyp}Throughout the paper, $F\colon{\mathscr{{K}}}\to {\mathscr{{L}}}$ is a tensor-triangulated functor between essentially small tensor-triangulated categories~${\mathscr{{K}}}$ and~${\mathscr{{L}}}$. Consider
\[
\varphi=\Spc(F)\colon {\Spc({\mathscr{L}})}\to {\Spc({\mathscr{K}})}
\]
the induced map on spectra, in the sense of tensor-triangular geometry~\cite{Balmer05a,BalmerICM,Stevenson16pp}.
Assume that ${\mathscr{{K}}}$ is \emph{rigid}, {{\sl i.e.}\ } every object has a dual (Remark~\ref{rem:rigid}).
\end{Hyp}

Our first result is a characterization of conservativity of~$F$.

\begin{Thm}
\label{thm:surj-closed}Under Hypotheses~\ref{hyp}, the following properties are equivalent:
\begin{enumerate}[\rm(a)]
\item
The functor $F\colon{\mathscr{{K}}}\to {\mathscr{{L}}}$ is conservative, {{\sl i.e.}\ } it detects isomorphisms.
\smallbreak
\item
The induced map~$\varphi\colon{\Spc({\mathscr{L}})}\to {\Spc({\mathscr{K}})}$ is surjective on closed points, {{\sl i.e.}\ } for every closed point ${{\mathscr{{P}}}}$ in~${\Spc({\mathscr{K}})}$, there exists ${{\mathscr{{Q}}}}$ in~${\Spc({\mathscr{L}})}$ such that $\varphi({{\mathscr{{Q}}}})={{\mathscr{{P}}}}$.
\end{enumerate}
\end{Thm}

We can remove the assumption that ${\mathscr{{K}}}$ is rigid, at the cost of replacing~(a) by:
\begin{enumerate}[\rm(a')]
\item
\label{it:a'}$F$ detects {$\otimes$-nilpotence}\ of objects, {{\sl i.e.}\ } $F(x)=0 {\Rightarrow} x{^{\otimes {n}}}=0$ for some $n\ge 1$.
\end{enumerate}

\medbreak

Our main results are dedicated to surjectivity of~$\varphi$ on the whole of~${\Spc({\mathscr{K}})}$.

\begin{Thm}
\label{thm:surj-nil}Under Hypotheses~\ref{hyp}, suppose that the functor $F\colon{\mathscr{{K}}}\to {\mathscr{{L}}}$ detects {$\otimes$-nilpotence}\ of morphisms, {{\sl i.e.}\ } every $f\colon x\to y$ in~${\mathscr{{K}}}$ such that $F(f)=0$ satisfies $f{^{\otimes {n}}}=0$ for some~$n\ge1$. Then the induced map~$\varphi\colon{\Spc({\mathscr{L}})}\to {\Spc({\mathscr{K}})}$ is surjective.
\end{Thm}

This result is clearly a corollary of (b)${\Rightarrow}$(a) in the following more technical result:
\begin{Thm}
\label{thm:main}Under Hypotheses~\ref{hyp}, the following properties are equivalent:
\begin{enumerate}[\rm(a)]
\item
The morphism $\varphi\colon{\Spc({\mathscr{L}})}\to {\Spc({\mathscr{K}})}$ is surjective.
\smallbreak
\item
\label{it:nil-cone}The functor $F\colon {\mathscr{{K}}}\to {\mathscr{{L}}}$ detects {$\otimes$-nilpotence}\ of morphisms which are already {$\otimes$-nilpotent}\ on their cone, {{\sl i.e.}\ } every $f\colon x\to y$ in~${\mathscr{{K}}}$ such that $F(f)=0$ and such that $f{^{\otimes {m}}}\otimes\cone(f)=0$ for some $m\ge 1$ satisfies $f{^{\otimes {n}}}=0$ for some~$n\ge1$.
\end{enumerate}
\end{Thm}

At this point, the Devinatz-Hopkins-Smith~\cite{DevinatzHopkinsSmith88} Nilpotence Theorem might come to some readers' mind. This celebrated result asserts that a morphism between finite objects in the topological stable homotopy category~$\SH$ must be $\otimes$-nilpotent if it vanishes on complex cobordism. Hopkins and Smith used the Nilpotence Theorem in the subsequent work~\cite{HopkinsSmith98} to prove the Chromatic Tower Theorem. A reformulation of the latter, in terms of~$\Spc(\SH^c)$, can be found in~\cite[\S\,9]{Balmer10b}. From the Nilpotence Theorem it follows that every prime of~$\SH^c$ is the kernel of some Morava $K$-theory. This implication is in essence the surjectivity of Theorem~\ref{thm:surj-nil} at work in the special case of~$\SH$.

Let us stress however that the scope of Theorems~\ref{thm:surj-closed} and~\ref{thm:surj-nil} is broader than the topological example. In fact, $\SH$ plays among general tensor-triangulated categories the same role that ${\mathbb{Z}}$ plays among general commutative rings. Commutative algebra is not only the study of ${\mathbb{Z}}$, and tt-geometry is not only the study of~$\SH$. For the reader who never heard of tensor-triangulated categories and yet had the fortitude to read thus far, let us recall that tt-categories also appear in algebraic geometry ({{\sl e.g.}}\ derived categories of schemes), in representation theory ({{\sl e.g.}}\ derived and stable categories of finite groups), in noncommutative topology ({{\sl e.g.}}\ $KK$-categories of $C^*$-algebras), in motivic theory ({{\sl e.g.}}\ stable ${\mathbb{A}}^1$-homotopy and derived categories of motives), and in equivariant analogues ({{\sl e.g.}}\ equivariant stable homotopy theory). A good introduction can be found in~\cite[\S\,1.2]{HoveyPalmieriStrickland97}. Tensor-triangular geometry is an umbrella theory for all those examples. In particular, computing ${\Spc({\mathscr{K}})}$ is \emph{the} fundamental problem for every tt-category~${\mathscr{{K}}}$ out there; see~\cite[Thm.\,4.10]{Balmer05a}.

\medbreak

After this motivational digression, let us return to the development of our results. It is interesting to know whether the converse of Theorem~\ref{thm:surj-nil} holds true in glorious generality: Does surjectivity of~$\Spc(F)$ alone guarantee that $F$ detects {$\otimes$-nilpotence}\ of morphisms? By Theorem~\ref{thm:main}, this problem can be reduced as follows.
\begin{Que}
\label{que}Under Hypotheses~\ref{hyp}, if $\varphi\colon{\Spc({\mathscr{L}})}\to {\Spc({\mathscr{K}})}$ is surjective and if $f\colon x\to y$ satisfies $F(f)=0$, is $f$ necessarily {$\otimes$-nilpotent}\ on its cone?
\end{Que}

We do not know any counter-example. In fact, we can give a positive answer under the assumption that $F\colon {\mathscr{{K}}}\to {\mathscr{{L}}}$ admits a right adjoint. Since ${\mathscr{{K}}}$ and ${\mathscr{{L}}}$ are essentially small (typically the `compact' objects of some big ambient category), existence of such a right adjoint is rather restrictive. In the context  of~\cite{BalmerDellAmbrogioSanders16}, it would be equivalent to having `Grothendieck-Neeman' duality. To give an example, this right adjoint exists in the case of a finite separable extension, see~\cite{Balmer16b}. The following are generalizations of some of the results in~\cite{Balmer16a}.

\begin{Thm}
\label{thm:FU}Under Hypotheses~\ref{hyp}, suppose that $F\colon {\mathscr{{K}}}\to {\mathscr{{L}}}$ admits a right adjoint~$U\colon{\mathscr{{L}}}\to {\mathscr{{K}}}$. Then the map $\varphi\colon{\Spc({\mathscr{L}})}\to {\Spc({\mathscr{K}})}$ is surjective if and only if the functor $F\colon{\mathscr{{K}}}\to {\mathscr{{L}}}$ detects {$\otimes$-nilpotence}\ of morphisms.
\end{Thm}

Again, this is a special case of a sharper, slightly more technical result.

\begin{Thm}
\label{thm:FU+}Under Hypotheses~\ref{hyp}, suppose that $F\colon {\mathscr{{K}}}\to {\mathscr{{L}}}$ admits a right adjoint~$U\colon{\mathscr{{L}}}\to {\mathscr{{K}}}$ and consider the image~$U({\mathbb{1}})\in{\mathscr{{K}}}$ of the $\otimes$-unit. Then the image of the map $\varphi\colon{\Spc({\mathscr{L}})}\to {\Spc({\mathscr{K}})}$ is exactly the support of the object~$U({\mathbb{1}})$:
\[
\im(\Spc(F))=\supp(U({\mathbb{1}}))\,.
\]
\end{Thm}

An example of the latter, not covered by the separable extensions of~\cite{Balmer16a}, can be obtained by `modding out' coefficients in motivic categories, see~\cite[Chap.\,5]{VoevodskySuslinFriedlander00}. For instance, if ${\mathscr{{K}}}={\DM_{\text{gm}}}(X;{\mathbb{Z}}){\overset{{F}}{\,{\mathop{\longrightarrow}\limits}\,}}{\DM_{\text{gm}}}(X;{\mathbb{Z}}/p)={\mathscr{{L}}}$ then $\im(\Spc(F))=\supp({\mathbb{Z}}/p)$. From these techniques, one can easily reduce the computation of the (yet unknown) spectrum of the integral derived category of geometric motives~${\DM_{\text{gm}}}(X,{\mathbb{Z}})$ to the case of field coefficients:
\[
\Spc({\DM_{\text{gm}}}(X;{\mathbb{Z}}))=\im(\Spc({\DM_{\text{gm}}}(X;{\mathbb{Q}})))\sqcup \ \bigsqcup_{p}\ \im(\Spc({\DM_{\text{gm}}}(X;{\mathbb{Z}}/p)))\,.
\]
These considerations will be pursued elsewhere.

\medbreak

Let us now state a direct consequence of Theorem~\ref{thm:surj-nil}, that was apparently never noticed despite its importance and simplicity. It is the case where $F$ is faithful.
\begin{Cor}
\label{cor:surj-ff}Suppose that ${\mathscr{{K}}}\subset{\mathscr{{L}}}$ is a rigid tensor-triangulated subcategory. Then every prime ${{\mathscr{{P}}}}\in{\Spc({\mathscr{K}})}$ is the intersection of a prime~${{\mathscr{{Q}}}}\in{\Spc({\mathscr{L}})}$ with~${\mathscr{{K}}}$.
\end{Cor}

A special sub-case of interest is that of `cellular' subcategories, {{\sl i.e.}\ } those ${\mathscr{{K}}}\subseteq{\mathscr{{L}}}$ generated by a collection of `nice' objects of~${\mathscr{{L}}}$, typically $\otimes$-invertible ones (spheres). Such cellular subcategories~${\mathscr{{K}}}$ are commonly studied when the ambient~${\mathscr{{L}}}$ appears out-of-reach of known methods. For instance, Dell'Ambrogio~\cite{DellAmbrogio10} used this approach for equivariant $KK$-theory, and later with Tabuada~\cite{DellAmbrogioTabuada12} for non-commutative motives. Peter~\cite{Peter13} is an example in the case of mixed Tate motives. Similarly, Heller-Ormsby~\cite{HellerOrmsby16pp} consider cellular subcategories in their recent study of tt-geometry in stable motivic homotopy theory. In all cases, Corollary~\ref{cor:surj-ff} says that whatever can be detected via these cellular subcategories~${\mathscr{{K}}}$ is actually relevant information about the bigger and more mysterious ambient category~${\mathscr{{L}}}$. In particular, surjectivity of the comparison homomorphisms introduced in~\cite{Balmer10b} can be tested on the cellular subcategory:
\begin{Cor}
\label{cor:surj-rho}Let $u\in{\mathscr{{L}}}$ be a $\otimes$-invertible object and ${\mathscr{{K}}}$ the full thick triangulated subcategory of~${\mathscr{{L}}}$ generated by~${\big\{\,{u{^{\otimes {n}}}}\,\big|\,{n\in{\mathbb{Z}}}\,\big\}}$, which is supposed rigid\,\textrm{\rm(\footnote{\,This is automatic if ${\mathscr{{L}}}$ lives in a `big' ambient category with internal hom, where rigid objects are closed under triangles. See~\cite[Thm.\,A.2.5\,(a)]{HoveyPalmieriStrickland97}.})}. Note that the graded rings $R^{{\scriptscriptstyle\bullet}}_{{\mathscr{{K}}},u}$ and $R^{{\scriptscriptstyle\bullet}}_{{\mathscr{{L}}},u}$ associated to~$u$ are the same in~${\mathscr{{K}}}$ and in~${\mathscr{{L}}}$:
\[
R^{{\scriptscriptstyle\bullet}}_{{\mathscr{{K}}},u}\ \overset{\textrm{def}}=\ {\Hom_{{\mathscr{{K}}}}}({\mathbb{1}}, u{^{\otimes {{\scriptscriptstyle\bullet}}}})={\Hom_{{\mathscr{{L}}}}}({\mathbb{1}}, u{^{\otimes {{\scriptscriptstyle\bullet}}}})\ \overset{\textrm{def}}=\ R^{{\scriptscriptstyle\bullet}}_{{\mathscr{{L}}},u}\,.
\]
If the comparison map $\rho^{{\scriptscriptstyle\bullet}}_{{\mathscr{{K}}},u}$ for~${\mathscr{{K}}}$ (recalled below) is surjective for the `cellular' subcategory~${\mathscr{{K}}}$ then the comparison map $\rho^{{\scriptscriptstyle\bullet}}_{{\mathscr{{L}}},u}$ for the ambient~${\mathscr{{L}}}$ is also surjective:
\[
\xymatrix@R=2em{
{\Spc({\mathscr{L}})} \ \ar@{->>}[r]^-{\textrm{Cor.\,\ref{cor:surj-ff}}} \ar[d]_-{\rho^{{\scriptscriptstyle\bullet}}_{{\mathscr{{L}}},u}} \ar@{}[rd]|-{\circlearrowright}
& \ {\Spc({\mathscr{K}})} \ar[d]^-{\rho^{{\scriptscriptstyle\bullet}}_{{\mathscr{{K}}},u}}
& {{\mathscr{{P}}}} \ar@{|->}[d]_-{\rho^{{\scriptscriptstyle\bullet}}_{{\mathscr{{K}}},u}}  \ar@{}[l]|(.7){\ni}
\\
\Spec^{{\scriptscriptstyle\bullet}}(R^{{\scriptscriptstyle\bullet}}_{{\mathscr{{L}}},u}) \ \ar@{=}[r]
& \ \Spec^{{\scriptscriptstyle\bullet}}(R^{{\scriptscriptstyle\bullet}}_{{\mathscr{{K}}},u})
& \rho^{{\scriptscriptstyle\bullet}}_{{\mathscr{{K}}},u}({{\mathscr{{P}}}})\ \overset{\textrm{def}}=\ {\big\{\,{f\in R^{{\scriptscriptstyle\bullet}}_{{\mathscr{{K}}},u}}\,\big|\,{\cone(f)\notin{{\mathscr{{P}}}}}\,\big\}} \ar@{}[l]|(.7){\ni} \,.\!\!
}
\]
\end{Cor}

For an introduction to these comparison maps and their importance, the reader is invited to consult the above references~\cite{Balmer10b,DellAmbrogio10,DellAmbrogioTabuada12,HellerOrmsby16pp} or~\cite{Sanders13}.

\medbreak
\noindent
\textbf{Acknowledgments}: I am thankful to Beren Sanders for observing in a previous version of this article that my proof of surjectivity of~$\varphi$ reduced to $F$ detecting nilpotence of morphisms of the form $\eta_x\otimes y\colon y\to x^\vee\otimes x\otimes y$. Beren's idea led me to the `morphisms which are nilpotent on their cone' and to Theorem~\ref{thm:main}. I also thank Martin Gallauer, Jeremiah Heller and Kyle Ormsby for their comments.

\goodbreak
\section{The proofs}
\label{se:proofs}\medbreak

The tensor $\otimes\colon{\mathscr{{K}}}\times{\mathscr{{K}}}{\mathop{\longrightarrow}\limits} {\mathscr{{K}}}$ is exact in each variable and~${\mathbb{1}}$ stands for the $\otimes$-unit in~${\mathscr{{K}}}$. Recall that a \emph{tt-ideal ${\mathscr{{J}}}\subseteq {\mathscr{{K}}}$} is a triangulated, thick, $\otimes$-ideal subcategory, {{\sl i.e.}\ } it is non-empty, is closed under taking cones, direct summands and under tensoring by any object of~${\mathscr{{K}}}$. For $\mathcal{E}\subseteq {\mathscr{{K}}}$, we denote by~${\langle {\mathcal{E}}\rangle}\subseteq{\mathscr{{K}}}$ the tt-ideal it generates.

A proper tt-ideal ${{\mathscr{{P}}}}\subsetneq{\mathscr{{K}}}$ is \emph{prime} if $x\otimes y\in{{\mathscr{{P}}}}$ implies $x\in{{\mathscr{{P}}}}$ or $y\in{{\mathscr{{P}}}}$. The \emph{spectrum} ${\Spc({\mathscr{K}})}={\big\{\,{{{\mathscr{{P}}}}\subset{\mathscr{{K}}}}\,\big|\,{{{\mathscr{{P}}}}\textrm{ is prime}}\,\big\}}$ has a topology whose basis of open is given by the subsets $U(x)={\big\{\,{{{\mathscr{{P}}}}\in{\Spc({\mathscr{K}})}}\,\big|\,{x\in {{\mathscr{{P}}}}}\,\big\}}$, for every~$x\in{\mathscr{{K}}}$. The closed complement $\supp(x)={\big\{\,{{{\mathscr{{P}}}}\in{\Spc({\mathscr{K}})}}\,\big|\,{x\notin{{\mathscr{{P}}}}}\,\big\}}$ is called the \emph{support} of the object~$x$. A tensor-triangulated functor $F\colon{\mathscr{{K}}}\to{\mathscr{{L}}}$ induces a continuous map $\varphi=\Spc(F)\colon{\Spc({\mathscr{L}})}\to {\Spc({\mathscr{K}})}$ given explicitly by~$\varphi({{\mathscr{{Q}}}})=F{^{-1}}({{\mathscr{{Q}}}})$, for every prime~${{\mathscr{{Q}}}}\subset{\mathscr{{L}}}$.

\begin{Rem}
\label{rem:rigid}Our assumption that the tensor category~${\mathscr{{K}}}$ is \emph{rigid}, means that there exists an exact functor called the \emph{dual}
\[
(-)^\vee\colon {\mathscr{{K}}}{^{\opname}} {\mathop{\longrightarrow}\limits} {\mathscr{{K}}}
\]
that provides an adjoint to tensoring with any object $x\in {\mathscr{{K}}}$ as follows:
\begin{equation}
\label{eq:adj-dual}
\vcenter{\xymatrix@R=2em{
{\mathscr{{K}}} \ar@<-.5em>[d]_-{x\otimes-} \ar@{}[d]|-{\dashv}
\\
{\mathscr{{K}}} \ar@<-.5em>[u]_-{x^\vee\otimes-}
}}
\end{equation}
Some authors call such objects~$x$ \emph{strongly dualizable}, {{\sl e.g.}}~\cite{HoveyPalmieriStrickland97}. The adjunction~\eqref{eq:adj-dual} comes with units (coevaluation) and counits (evaluation)
\begin{equation}
\label{eq:(co)units}\eta_x\colon{\mathbb{1}}\to x^\vee\otimes x
{\qquad\textrm{{and}}\qquad}
{\epsilon}_x\colon x\otimes x^\vee\to {\mathbb{1}}
\end{equation}
which satisfy the relation
\begin{equation}
\label{eq:(co)unit-relation}({\epsilon}_x\otimes x)\circ (x\otimes \eta_x)=1_x\,.
\end{equation}
It follows from~\eqref{eq:(co)unit-relation} that $x$ is a direct summand of $x\otimes x^\vee \otimes x\cong x{^{\otimes {2}}}\otimes x^\vee$.

It is a general fact that any tensor functor $F\colon{\mathscr{{K}}}\to {\mathscr{{L}}}$ preserves rigidity, since we can use $F(x^\vee)$ as $F(x)^\vee$ with $F(\eta_x)$ and $F({\epsilon}_x)$ as units and counits. See for instance~\cite[Prop.\,3.1]{FauskHuMay03}. In particular, although we do not assume ${\mathscr{{L}}}$ rigid, every object we use below will be rigid as long as it comes from~${\mathscr{{K}}}$.
\end{Rem}

\begin{Rem}
\label{rem:non-rigid}In a not-necessarily rigid tt-category, an object~$x$ with empty support, $\supp(x)=\varnothing$, is {$\otimes$-nilpotent}, {{\sl i.e.}\ } $x{^{\otimes {n}}}=0$ for some $n\ge 1$. See~\cite[Cor.\,2.4]{Balmer05a}. When $x$ is rigid, $x{^{\otimes {n}}}=0$ forces $x=0$ since $x$ is a summand of $x{^{\otimes {n}}}\otimes (x^\vee){^{\otimes {(n-1)}}}$.
\end{Rem}

\smallbreak

We begin with Theorem~\ref{thm:surj-closed}, which is relatively straightforward. We only need a few standard facts from basic tt-geometry, which do not use rigidity, namely:
\begin{enumerate}[\rm(A)]
\item
\label{it:A}Given a $\otimes$-multiplicative class~$S$ of objects in~${\mathscr{{K}}}$ ({{\sl i.e.}\ } ${\mathbb{1}}\in S$ and $x,y\in S {\Rightarrow} x\otimes y\in S$) and a tt-ideal ${\mathscr{{J}}}\subset{\mathscr{{K}}}$ such that ${\mathscr{{J}}}\cap S=\varnothing$, then there exists a prime ${{\mathscr{{P}}}}\in{\Spc({\mathscr{K}})}$ such that ${\mathscr{{J}}}\subseteq{{\mathscr{{P}}}}$ and ${{\mathscr{{P}}}}\cap S=\varnothing$. This fact uses that ${\mathscr{{K}}}$ is essentially small and is proven in~\cite[Lemma~2.2]{Balmer05a}.
\smallbreak
\item
\label{it:B}A point ${{\mathscr{{P}}}}\in{\Spc({\mathscr{K}})}$ is closed if and only ${{\mathscr{{P}}}}$ is a \emph{minimal} prime for inclusion in~${\mathscr{{K}}}$ ({{\sl i.e.}\ } ${{\mathscr{{P}}}}'\subseteq{{\mathscr{{P}}}}{\Rightarrow} {{\mathscr{{P}}}}'={{\mathscr{{P}}}}$). See~\cite[Prop.\,2.9]{Balmer05a}.
\smallbreak
\item
\label{it:C}Any non-empty closed subset, for instance ${\overline{\{{{\mathscr{{P}}}}\}}}$ for a point~${{\mathscr{{P}}}}$, or $\supp(x)$ for a non-trivial object~$x$, contains a closed point. See~\cite[Cor.\,2.12]{Balmer05a}.
\smallbreak
\item
\label{it:D}For $F\colon{\mathscr{{K}}}\to {\mathscr{{L}}}$ and $\varphi=\Spc(F)\colon{\Spc({\mathscr{L}})}\to {\Spc({\mathscr{K}})}$, and every object~$x\in{\mathscr{{K}}}$, we have $\supp(F(x))=\varphi{^{-1}}(\supp(x))$ in~${\Spc({\mathscr{L}})}$. See~\cite[Prop.\,3.6]{Balmer05a}.
\end{enumerate}

\begin{proof}[Proof of Theorem~\ref{thm:surj-closed}]
Suppose that $F\colon{\mathscr{{K}}}\to {\mathscr{{L}}}$ is conservative and let ${{\mathscr{{P}}}}\in{\Spc({\mathscr{K}})}$ be a closed point, {{\sl i.e.}\ } a minimal prime. Consider its complement $S={\mathscr{{K}}}{\!\smallsetminus\!}{{\mathscr{{P}}}}$. Since ${{\mathscr{{P}}}}$ is prime, $S$ is $\otimes$-multiplicative in~${\mathscr{{K}}}$ and does not contain zero. Since $F$ is a conservative tensor functor, the same holds for the class $F(S)$ in~${\mathscr{{L}}}$. (Recall that for a triangulated functor~$F$, conservativity is equivalent to $F(x)=0{\Rightarrow} x=0$, since a morphism is an isomorphism if and only if its cone is zero.) By the general fact~\eqref{it:A} recalled above, for the $\otimes$-multiplicative class $F(S)$ and for the tt-ideal ${\mathscr{{J}}}=0$ in~${\mathscr{{L}}}$, there exists a prime~${{\mathscr{{Q}}}}\in{\Spc({\mathscr{L}})}$ such that ${{\mathscr{{Q}}}}\cap F(S)=\varnothing$. This relation implies that $F{^{-1}}({{\mathscr{{Q}}}})\subseteq {{\mathscr{{P}}}}$. By minimality of the closed point~${{\mathscr{{P}}}}$, see~\eqref{it:B}, this inclusion $F{^{-1}}({{\mathscr{{Q}}}})\subseteq {{\mathscr{{P}}}}$ forces ${{\mathscr{{P}}}}=F{^{-1}}({{\mathscr{{Q}}}})=\varphi({{\mathscr{{Q}}}})$.

Conversely, suppose that $\varphi\colon{\Spc({\mathscr{L}})}\to {\Spc({\mathscr{K}})}$ is surjective on closed points and let $x\in{\mathscr{{K}}}$ be such that $F(x)=0$. We want to show that $x=0$. Suppose {{\sl ab absurdo}}\ that $x\neq 0$. Then we have $\supp(x)\neq\varnothing$. By~\eqref{it:C}, we know that there exists a closed point ${{\mathscr{{P}}}}\in\supp(x)$, which by assumption belongs to the image of~$\varphi$, say ${{\mathscr{{P}}}}=\varphi({{\mathscr{{Q}}}})$. But then ${{\mathscr{{Q}}}}\in\varphi{^{-1}}(\supp(x))=\supp(F(x))$ by~\eqref{it:D}. This last statement contradicts $\supp(F(x))=\supp(0)=\varnothing$. So $x=0$ as claimed.
\end{proof}

\begin{Rem}
The proof also gives a statement for ${\mathscr{{K}}}$ not rigid. In that case, the property $\supp(x)=\varnothing$ does not necessarily imply that $x=0$ but that $x$ is {$\otimes$-nilpotent}, as an object. See Remark~\ref{rem:non-rigid}. Surjectivity of $\varphi$ onto closed points is therefore equivalent to $F$ detecting {$\otimes$-nilpotence}\ of objects. See Theorem~\ref{thm:surj-closed}\,(\ref{it:a'}').
\end{Rem}

\begin{Rem}
In complete generality, if a closed point ${{\mathscr{{P}}}}\in{\Spc({\mathscr{K}})}$ belongs to the image of $\varphi\colon{\Spc({\mathscr{L}})}\to {\Spc({\mathscr{K}})}$, say ${{\mathscr{{P}}}}=\varphi({{\mathscr{{Q}}}})$, then ${{\mathscr{{P}}}}$ is also the image of a \emph{closed} point~${{\mathscr{{Q}}}}'$, which can be chosen in the closure of~${{\mathscr{{Q}}}}$. Indeed, there exists a closed point ${{\mathscr{{Q}}}}'\in{\overline{\{{{\mathscr{{Q}}}}\}}}$ by~\eqref{it:C} and continuity of~$\varphi$ implies $\varphi({{\mathscr{{Q}}}}')\in{\overline{\{{{\mathscr{{P}}}}\}}}=\{{{\mathscr{{P}}}}\}$.
\end{Rem}

\begin{center}*\ *\ *\end{center}

We now turn to Theorem~\ref{thm:main}, which is slightly more tricky. For the sake of clarity, we spell out the now probably obvious terminology:
\begin{Def}
\label{def:tens-nil}A morphism $f\colon x\to y$ is called \emph{$\otimes$-nilpotent}\  if $f{^{\otimes {n}}}\colon x{^{\otimes {n}}}\to y{^{\otimes {n}}}$ is zero for some~$n\ge 1$.
We say that $f\colon x\to y$ is \emph{$\otimes$-nilpotent\ on an object~$z$} in~${\mathscr{{K}}}$ if there exists $n\ge 1$ such that $f{^{\otimes {n}}}\otimes z$ is the zero morphism $x{^{\otimes {n}}}\otimes z\to y{^{\otimes {n}}}\otimes z$. In particular, $f$ is \emph{{$\otimes$-nilpotent}\ on its cone} if there exists $n\ge 1$ such that $f{^{\otimes {n}}}\otimes \cone(f)=0$.
\end{Def}

The following useful fact was already observed in~\cite[Prop.\,2.12]{Balmer10b}:
\begin{Prop}
\label{prop:nil-tt-ideal}Let $f\colon x\to y$ be a morphism in~${\mathscr{{K}}}$. Then
\[
{\big\{\,{z\in{\mathscr{{K}}}}\,\big|\,{f\textrm{ is {$\otimes$-nilpotent}\ on }z}\,\big\}}
\]
forms a tt-ideal, even if ${\mathscr{{K}}}$ is not rigid.
\end{Prop}

Closure under direct summands and $\otimes$ is clear from the definition. The trick for closure under cones, is that if $f{^{\otimes {n_i}}}\otimes z_i=0$ for $i=1,2$ and if $z_1\to z_2\to z_3\to \Sigma z_1$ is an exact triangle, then $f{^{\otimes {(n_1+n_2)}}}\otimes z_3$ will vanish. This is the place where the same statement would fail with `$f$ vanishes on~$z$' (instead of `$f$ {$\otimes$-nilpotent}\ on~$z$').

\begin{Prop}
\label{prop:spectra^3}Let $\xi\colon w\to {\mathbb{1}}$ be a morphism in~${\mathscr{{K}}}$ (not necessarily rigid) such that $\xi\otimes\cone(\xi)=0$. Then the cone of~$\xi{^{\otimes {n}}}$ generates the same tt-ideal, for all~$n$:
\[
{\langle {\cone(\xi)}\rangle}={\big\{\,{z\in{\mathscr{{K}}}}\,\big|\,{\xi\textrm{ is {$\otimes$-nilpotent}\ on }z}\,\big\}}={\langle {\cone(\xi{^{\otimes {n}}})}\rangle}\,.
\]
\end{Prop}

\begin{proof}
The assumption $\xi\otimes \cone(\xi)=0$ implies that the object $\cone(\xi)$ belongs to~${\big\{\,{z\in{\mathscr{{K}}}}\,\big|\,{\xi\textrm{ is {$\otimes$-nilpotent}\ on }z}\,\big\}}$, which is a tt-ideal by Proposition~\ref{prop:nil-tt-ideal}. On the other hand, if the morphism $\xi{^{\otimes {n}}}\otimes z$ is zero then the exact triangle
\[
\xymatrix@C=2em{
w{^{\otimes {n}}}\otimes z \ar[rr]^-{\xi{^{\otimes {n}}}\otimes z=0}
&& z \ar[r]
& \cone(\xi{^{\otimes {n}}})\otimes z \ar[r]
& \Sigma w{^{\otimes {n}}}\otimes z
}
\]
implies that $z$ is a summand of~$\cone(\xi{^{\otimes {n}}})\otimes z$. Hence $z$ belongs to~${\langle {\cone(\xi{^{\otimes {n}}})}\rangle}$. Finally, in the Verdier quotient ${\mathscr{{K}}}/{\langle {\cone(\xi)}\rangle}$, the morphism $\xi$ is an isomorphism, hence so is~$\xi{^{\otimes {n}}}$. Therefore $\cone(\xi{^{\otimes {n}}})\in{\langle {\cone(\xi)}\rangle}$. In short, we have obtained
\[
{\langle {\cone(\xi)}\rangle} \subseteq {\big\{\,{z\in{\mathscr{{K}}}}\,\big|\,{\xi{^{\otimes {n}}}\otimes z=0\textrm{ for some }n\ge1}\,\big\}}\subseteq \cup_{n\ge1}{\langle {\cone(\xi{^{\otimes {n}}})}\rangle}
\subseteq{\langle {\cone(\xi)}\rangle}\,.
\]
This proves the claim. Compare~\cite[\S\,2]{Balmer10b}.
\end{proof}

We can now establish the key observation of the paper:
\begin{Cor}
\label{cor:key}Let $x\in{\mathscr{{K}}}$ be a rigid object in a (not necessarily rigid) tt-category~${\mathscr{{K}}}$. Choose $\xi_x$ a `homotopy fiber' of the coevaluation morphism~$\eta_x$ of~\eqref{eq:(co)units}, {{\sl i.e.}\ } choose an exact triangle in~${\mathscr{{K}}}$
\begin{equation}
\label{eq:triangle}
\xymatrix{
w_x \ar[r]^-{\displaystyle\xi_x}
& {\mathbb{1}} \ar[r]^-{\displaystyle\eta_x}
& x^\vee\otimes x \ar[r]^-{}
& \Sigma w_x
}
\end{equation}
for a morphism~$\xi_x$ (which will play an important role below). Then the tt-ideal ${\langle {x}\rangle}$ generated by our object is exactly the subcategory on which $\xi_x$ is $\otimes$-nilpotent:
\begin{equation}
\label{eq:key}{\langle {x}\rangle} = {\big\{\,{z\in{\mathscr{{K}}}}\,\big|\,{\xi_x{^{\otimes {n}}}\otimes z=0\textrm{ for some }n\ge1}\,\big\}}\,.
\end{equation}
Moreover, for every $n\ge1$ the morphism $\xi_x{^{\otimes {n}}}$ is {$\otimes$-nilpotent}\ on its cone.
\end{Cor}

\begin{proof}
Consider the exact triangle obtained by tensoring~\eqref{eq:triangle} with~$x$:
\[
\xymatrix@C=4em{
x\otimes w_x \ar[r]^-{x\otimes \xi_x}
& x \ar[r]^-{x\otimes \eta_x}
& x\otimes x^\vee\otimes x \ar[r]^-{x\otimes \zeta_x}
& \Sigma x\otimes w_x
}
\]
By the unit-counit relation~\eqref{eq:(co)unit-relation}, the morphism $x\otimes \eta_x$ is a monomorphism. This forces $x\otimes \xi_x=0$. Hence $\xi_x\otimes \cone(\xi_x)\simeq \xi_x\otimes x^\vee\otimes x=0$ and we can apply Proposition~\ref{prop:spectra^3} to $\xi=\xi_x$. It gives us~\eqref{eq:key} since ${\langle {\cone(\xi_x)}\rangle}={\langle {x^\vee\otimes x}\rangle}={\langle {x}\rangle}$ by rigidity of~$x$. The `moreover part' also follows from Proposition~\ref{prop:spectra^3} where we proved that $\xi$ is {$\otimes$-nilpotent}\ on~$\cone(\xi{^{\otimes {n}}})$.
\end{proof}

The above result allows us to translate questions about tt-ideals into a {$\otimes$-nilpotence}\ problem. We isolate a surjectivity argument that we shall use in two different instances.
\begin{Lem}
\label{lem:key}Under Hypotheses~\ref{hyp}, choose for every $x\in{\mathscr{{K}}}$ an exact triangle as in~\eqref{eq:triangle}. Let ${{\mathscr{{P}}}}\in{\Spc({\mathscr{K}})}$ be a prime. Suppose that ${{\mathscr{{P}}}}$ satisfies the following technical condition:
\begin{equation}
\label{eq:N}\textrm{For all }x\in {{\mathscr{{P}}}},\textrm{ all }s\in{\mathscr{{K}}}{\!\smallsetminus\!}{{\mathscr{{P}}}}\textrm{ and all }n\ge1,\textrm{ we have }F(\xi_x{^{\otimes {n}}}\otimes s)\neq0.
\end{equation}
Then ${{\mathscr{{P}}}}$ belongs to the image of~$\varphi\colon{\Spc({\mathscr{L}})}\to {\Spc({\mathscr{K}})}$.
\end{Lem}

\begin{proof}
Consider the complement $S={\mathscr{{K}}}{\!\smallsetminus\!} {{\mathscr{{P}}}}$. Let ${\mathscr{{J}}}\subseteq{\mathscr{{L}}}$ be the tt-ideal generated by~$F({{\mathscr{{P}}}})$, just viewed as a class of objects in~${\mathscr{{L}}}$. We claim that ${\mathscr{{J}}}={\langle {F({{\mathscr{{P}}}})}\rangle}$ equals
\[
{\mathscr{{J}}}':={\big\{\,{y\in{\mathscr{{L}}}}\,\big|\,{\textrm{there exists }x\in{{\mathscr{{P}}}}\textrm{ such that }y\in{\langle {F(x)}\rangle}}\,\big\}}.
\]
Indeed, since we have $F({{\mathscr{{P}}}})\subseteq{\mathscr{{J}}}'\subseteq{\mathscr{{J}}}$ directly form the definitions, it suffices to show that ${\mathscr{{J}}}'$ is a tt-ideal. It is clearly thick and $\otimes$-ideal. For closure under cones, if $y_1\to y_2\to y_3\to \Sigma y_1$ is exact in~${\mathscr{{L}}}$ and $y_i\in{\langle {F(x_i)}\rangle}$ for $x_i\in{{\mathscr{{P}}}}$ and $i=1,2$, then $y_3\in{\langle {y_1,y_2}\rangle}\subseteq{\langle {F(x_1),F(x_2)}\rangle}={\langle {F(x_1\oplus x_2)}\rangle}$ and $x_1\oplus x_2$ still belongs to~${{\mathscr{{P}}}}$.

Now, for every object~$x\in{\mathscr{{K}}}$, the tt-functor $F\colon{\mathscr{{K}}}\to {\mathscr{{L}}}$ sends an exact triangle over the unit~$\eta_x$ as in~\eqref{eq:triangle} to an exact triangle in~${\mathscr{{L}}}$:
\[
\xymatrix@C=4em{
F(w_x) \ar[r]^-{F(\xi_x)}
& {\mathbb{1}} \ar[r]^-{\eta_{F(x)}}
& F(x)^\vee\otimes F(x) \ar[r]^-{F(\zeta_x)}
& \Sigma F(w_x)\,.
}
\]
Here we use that $F(\eta_x)=\eta_{F(x)}$ which is another way of saying that $F$ preserves duals. See Remark~\ref{rem:rigid}. Using this last exact triangle in Corollary~\ref{cor:key} for the rigid object~$F(x)$ in the tt-category~${\mathscr{{L}}}$, we see that
\[
{\langle {F(x)}\rangle}={\big\{\,{y\in{\mathscr{{L}}}}\,\big|\,{F(\xi_x){^{\otimes {n}}}\otimes y=0\textrm{ for some }n\ge 1}\,\big\}}\,.
\]
Combining this with the description of~${\mathscr{{J}}}={\langle {F({{\mathscr{{P}}}})}\rangle}$ as ${\mathscr{{J}}}'$ above, we obtain
\[
{\langle {F({{\mathscr{{P}}}})}\rangle}={\big\{\,{y\in{\mathscr{{L}}}}\,\big|\,{F(\xi_x){^{\otimes {n}}}\otimes y=0\textrm{ for some }n\ge 1\textrm{ and some }x\in{{\mathscr{{P}}}}}\,\big\}}\,.
\]
It follows that if $s\in S={\mathscr{{K}}}{\!\smallsetminus\!}{{\mathscr{{P}}}}$ then $F(s)$ cannot belong to~${\mathscr{{J}}}={\langle {F({{\mathscr{{P}}}})}\rangle}$. Indeed, if $F(s)\in{\langle {F({{\mathscr{{P}}}})}\rangle}$ then by the above there exists $x\in{{\mathscr{{P}}}}$ and $n\ge1$ such that $0=F(\xi_x){^{\otimes {n}}}\otimes F(s)\cong F(\xi_x{^{\otimes {n}}}\otimes s)$ since $F$ is a $\otimes$-functor. This contradicts~\eqref{eq:N}.

In short, we have shown that the $\otimes$-multiplicative class $F(S)=F({\mathscr{{K}}}{\!\smallsetminus\!} {{\mathscr{{P}}}})$ does not meet the tt-ideal ${\mathscr{{J}}}={\langle {F({{\mathscr{{P}}}})}\rangle}$, in the tt-category~${\mathscr{{L}}}$. By the existence trick~\eqref{it:A} again, there exists a prime~${{\mathscr{{Q}}}}$ satisfying the following two relations: ${\mathscr{{J}}}\subseteq {{\mathscr{{Q}}}}$ and $F(S)\cap {{\mathscr{{Q}}}}=\varnothing$. Unpacking the definition of $S={\mathscr{{K}}}{\!\smallsetminus\!}{{\mathscr{{P}}}}$ and ${\mathscr{{J}}}={\langle {F({{\mathscr{{P}}}})}\rangle}$, these two relations mean respectively ${{\mathscr{{P}}}}\subseteq F{^{-1}}({{\mathscr{{Q}}}})$ and $F{^{-1}}({{\mathscr{{Q}}}})\subseteq {{\mathscr{{P}}}}$. Hence ${{\mathscr{{P}}}}=F{^{-1}}({{\mathscr{{Q}}}})=\varphi({{\mathscr{{Q}}}})$ as wanted.
\end{proof}

We are now ready to prove our main result.

\begin{proof}[Proof of Theorem~\ref{thm:main}]
\

(a)${\Rightarrow}$(b): Suppose that $\varphi\colon{\Spc({\mathscr{L}})}\to {\Spc({\mathscr{K}})}$ is surjective and let $f\colon x\to y$ be a morphism such that $F(f)=0$ and which is {$\otimes$-nilpotent}\ on its cone, say $f{^{\otimes {m}}}\otimes\cone(f)=0$. It follows from the exact triangle $x{\overset{{f}}\to} y\to \cone(f)\to \Sigma x$ in~${\mathscr{{K}}}$ and from $F(f)=0$ that $F(\cone(f))\simeq F(y)\oplus \Sigma F(x)$ in~${\mathscr{{L}}}$. Taking supports, we have $\supp(F(\cone(f)))=\supp(F(x))\cup \supp(F(y))$. By~\eqref{it:D}, this translates into
\[
\varphi{^{-1}}(\supp(\cone(f)))=\varphi{^{-1}}(\supp(x))\cup \varphi{^{-1}}(\supp(y))=\varphi{^{-1}}(\supp(x)\cup \supp(y))\,.
\]
Since $\varphi$ is surjective, this implies $\supp(\cone(f))=\supp(x)\cup \supp(y)$. Therefore $x,y\in{\langle {\cone(f)}\rangle}$. But we assumed that $f$ is {$\otimes$-nilpotent}\ on~$\cone(f)$ and it follows from Proposition~\ref{prop:nil-tt-ideal} that $f$ is also {$\otimes$-nilpotent}\ on~$x$ and on~$y$. This means that there exists $n\ge 1$ such that $f{^{\otimes {n}}}\otimes x=0\colon x{^{\otimes {(n+1)}}}\to y{^{\otimes {n}}}\otimes x$. But then $f{^{\otimes {(n+1)}}}$ decomposes as
\[
\xymatrix@C=5em{
x{^{\otimes {n+1}}} \ar[r]_-{f{^{\otimes {n}}}\otimes x=0} \ar@/^1em/[rr]^-{f{^{\otimes {(n+1)}}}}
& y{^{\otimes {n}}}\otimes x \ar[r]_-{y{^{\otimes {n}}}\otimes f}
& y{^{\otimes {(n+1)}}}
}
\]
and is therefore also zero, that is, $f{^{\otimes {(n+1)}}}=0$ as wanted.

(b)${\Rightarrow}$(a):
Suppose that $F\colon{\mathscr{{K}}}\to {\mathscr{{L}}}$ detects {$\otimes$-nilpotence}\ of those morphisms which are already zero on their cone. Let ${{\mathscr{{P}}}}\in{\Spc({\mathscr{K}})}$ be a prime and let us show that property~\eqref{eq:N} in Lemma~\ref{lem:key} is satisfied. Let $g=\xi_x{^{\otimes {n}}}\otimes s$ be the morphism in~\eqref{eq:N} for some objects $x\in{{\mathscr{{P}}}}$ and $s\in{\mathscr{{K}}}{\!\smallsetminus\!}{{\mathscr{{P}}}}$ and for $n\ge1$. Suppose {{\sl ab absurdo}}\ that $F(g)=0$. The cone of~$g=\xi_x{^{\otimes {n}}}\otimes s$ is simply $\cone(\xi_x{^{\otimes {n}}})\otimes s$. By Corollary~\ref{cor:key}, $\xi_x{^{\otimes {n}}}$ is {$\otimes$-nilpotent}\ on its cone. Hence $g$ is {$\otimes$-nilpotent}\ on its cone as well. We can therefore apply our assumption~\eqref{it:nil-cone} to~$g$ and deduce from the (absurd) assumption $F(g)=0$ that $g=\xi_x{^{\otimes {n}}}\otimes s$ is {$\otimes$-nilpotent}. In other words, $\xi_x$ is $\otimes$-nilpotent on~$s{^{\otimes {m}}}$ for some $m\ge1$. By Corollary~\ref{cor:key} again, this implies that $s{^{\otimes {m}}}$ belongs to~${\langle {x}\rangle} \subseteq{{\mathscr{{P}}}}$, and therefore $s\in{{\mathscr{{P}}}}$ since ${{\mathscr{{P}}}}$ is prime, a contradiction with the choice of~$s$ in~$S={\mathscr{{K}}}{\!\smallsetminus\!}{{\mathscr{{P}}}}$. In short, we have verified property~\eqref{eq:N} of Lemma~\ref{lem:key} for the prime~${{\mathscr{{P}}}}$, which tells us that ${{\mathscr{{P}}}}$ belongs to the image of~$\varphi$ as claimed.
\end{proof}

\begin{center}*\ *\ *\end{center}

Let us now prove Theorems~\ref{thm:FU} and~\ref{thm:FU+}. We therefore assume the existence of an adjoint~$U\colon {\mathscr{{L}}}\to {\mathscr{{K}}}$ to our tensor-triangulated functor~$F$:
\begin{equation}
\label{eq:FU}\vcenter{\xymatrix@R=2em{
{\mathscr{{K}}} \ar@<-.5em>[d]_-{F} \ar@{}[d]|-{\dashv}
\\
{\mathscr{{L}}} \ar@<-.5em>[u]_-{U}
}}
\end{equation}
By general theory, $U$ must satisfy a projection formula
\begin{equation}
\label{eq:proj-form}U(F(x)\otimes z)\cong x\otimes U(z)
\end{equation}
for all~$x\in{\mathscr{{K}}}$ and $z\in{\mathscr{{L}}}$. The latter is an easy consequence of rigidity of~$x$ and the adjunctions~\eqref{eq:adj-dual} and~\eqref{eq:FU}. See for instance~\cite[Prop.\,3.2]{FauskHuMay03}.

\begin{proof}[Proof of Theorem~\ref{thm:FU+}]
Let ${{\mathscr{{P}}}}\in{\Spc({\mathscr{K}})}$. We need to show that ${{\mathscr{{P}}}}\in\im(\varphi)$ if and only if~${{\mathscr{{P}}}}\in\supp(U({\mathbb{1}}))$. The latter means $U({\mathbb{1}})\notin{{\mathscr{{P}}}}$.

Suppose first that ${{\mathscr{{P}}}}=\varphi({{\mathscr{{Q}}}})$ for some ${{\mathscr{{Q}}}}\in{\Spc({\mathscr{L}})}$. Then ${{\mathscr{{P}}}}=F{^{-1}}({{\mathscr{{Q}}}})$. To show $U({\mathbb{1}})\notin{{\mathscr{{P}}}}$ it therefore suffices to show that $FU({\mathbb{1}})\notin{{\mathscr{{Q}}}}$. This is easy since, by the unit-counit relation for~\eqref{eq:FU}, the object $FU({\mathbb{1}}_{{\mathscr{{L}}}})\cong FUF({\mathbb{1}}_{{\mathscr{{K}}}})$ admits $F({\mathbb{1}}_{{\mathscr{{K}}}})\cong{\mathbb{1}}_{{\mathscr{{L}}}}$ as a direct summand and ${\mathbb{1}}$ cannot belong to any prime.

The reverse inclusion is the interesting one. So, let ${{\mathscr{{P}}}}\in\supp(U({\mathbb{1}}))$, meaning $U({\mathbb{1}})\notin{{\mathscr{{P}}}}$. Let us show that ${{\mathscr{{P}}}}$ satisfies condition~\eqref{eq:N} of Lemma~\ref{lem:key}. Take two objects $x\in{{\mathscr{{P}}}}$ and $s\in{\mathscr{{K}}}{\!\smallsetminus\!}{{\mathscr{{P}}}}$ and $n\ge 1$ suppose {{\sl ab absurdo}}\ that $F(g)=0$ where $g=\xi_x{^{\otimes {n}}}\otimes s$ as before. By the projection formula~\eqref{eq:proj-form} for~$z={\mathbb{1}}$, the property $UF(g)=U(0)=0$ implies $g\otimes U({\mathbb{1}})=0$. Consequently we have an exact triangle
\[
\xymatrix@C=2em{
w_x{^{\otimes {n}}}\otimes s\otimes U({\mathbb{1}}) \ar[rr]^-{g\otimes U({\mathbb{1}})=0}
&& s\otimes U({\mathbb{1}}) \ar[r]
& \cone(g)\otimes U({\mathbb{1}}) \ar[r]
& \Sigma w_x{^{\otimes {n}}}\otimes s\otimes U({\mathbb{1}})
}
\]
in~${\mathscr{{K}}}$. This proves that $s\otimes U({\mathbb{1}})$ is a direct summand of~$\cone(g)\otimes U({\mathbb{1}})\in {\langle {\cone(g)}\rangle} \subseteq{\langle {\cone(\xi_x{^{\otimes {n}}})}\rangle}$. By Proposition~\ref{prop:spectra^3}, the latter is contained in~${\langle {x}\rangle} \subseteq{{\mathscr{{P}}}}$. In short, we have $s\otimes U({\mathbb{1}})\in{{\mathscr{{P}}}}$. Since ${{\mathscr{{P}}}}$ is prime this forces $s\in{{\mathscr{{P}}}}$ or $U({\mathbb{1}})\in{{\mathscr{{P}}}}$, which are both absurd. So we have proven~\eqref{eq:N} for~${{\mathscr{{P}}}}$ and we conclude by Lemma~\ref{lem:key} again.
\end{proof}

\begin{proof}[Proof of Theorem~\ref{thm:FU}]
In view of Theorem~\ref{thm:FU+} it suffices to prove that $F\colon{\mathscr{{K}}}\to {\mathscr{{L}}}$ detects {$\otimes$-nilpotence}\ if and only if $\supp(U({\mathbb{1}}))={\Spc({\mathscr{K}})}$, which means ${\langle {U({\mathbb{1}})}\rangle}={\mathscr{{K}}}$. This is a standard argument, as in~\cite[Prop.\,3.15]{Balmer16a} for instance. Let us outline it for completeness. The point is that $A:=U({\mathbb{1}})$ is a ring-object (for $U$ is lax-monoidal). Let $J {\overset{{\xi}}\to} {\mathbb{1}} {\overset{{u}}\to} A \to \Sigma J$ be an exact triangle over the unit $u\colon{\mathbb{1}} \to A$ (the unit of the $F{\dashv} U$ adjunction at~${\mathbb{1}}$). We have $A\otimes \xi=0$ (since $A\otimes u$ is a split monomorphism, retracted by multiplication~$A\otimes A\to A$). A morphism $f\colon x\to y$ satisfies $F(f)=0$ if and only if the composite $x{\overset{{f}}\to}y{\overset{{u\otimes y}}\to}A\otimes y$ is zero (by adjunction and the projection formula: $A\otimes-\simeq UF(-)$); this is in turn equivalent to the morphism $f\colon x\to y$ factoring via~$\xi\otimes y\colon J\otimes y\to y$ (by the exact triangle $J\otimes y{\overset{{\xi\otimes y}}{\,{\mathop{\longrightarrow}\limits}\,}}y{\overset{{u\otimes y}}{\,{\mathop{\longrightarrow}\limits}\,}}A\otimes y{\overset{{}}{\,{\mathop{\longrightarrow}\limits}\,}}\Sigma J\otimes y$). So we are down to proving that $\xi\colon J\to {\mathbb{1}}$ is {$\otimes$-nilpotent}\ if and only if~${\langle {A}\rangle}={\mathscr{{K}}}$. This is now immediate from Proposition~\ref{prop:spectra^3}, which says that ${\langle {A}\rangle}={\big\{\,{z\in{\mathscr{{K}}}}\,\big|\,{\xi\textrm{ is {$\otimes$-nilpotent}\ on }z}\,\big\}}$. Indeed, ${\mathbb{1}}\in{\langle {A}\rangle}$ if and only if~$\xi$ is {$\otimes$-nilpotent}\ on~${\mathbb{1}}$.
\end{proof}

\begin{thebibliography}{FHM03}

\bibitem[Bal05]{Balmer05a}
Paul Balmer.
\newblock The spectrum of prime ideals in tensor triangulated categories.
\newblock {\em J. Reine Angew. Math.}, 588:149--168, 2005.

\bibitem[Bal10a]{Balmer10b}
Paul Balmer.
\newblock Spectra, spectra, spectra -- tensor triangular spectra versus
  {Z}ariski spectra of endomorphism rings.
\newblock {\em Algebr. Geom. Topol.}, 10(3):1521--1563, 2010.

\bibitem[Bal10b]{BalmerICM}
Paul Balmer.
\newblock Tensor triangular geometry.
\newblock In {\em International {C}ongress of {M}athematicians, Hyderabad
  (2010), {V}ol. {II}}, pages 85--112. Hindustan Book Agency, 2010.

\bibitem[Bal16a]{Balmer16a}
Paul Balmer.
\newblock The derived category of an \'etale extension and the separable
  {N}eeman-{T}homason theorem.
\newblock {\em J. Inst. Math. Jussieu}, 15(3):613--623, 2016.

\bibitem[Bal16b]{Balmer16b}
Paul Balmer.
\newblock Separable extensions in tensor-triangular geometry and generalized
  {Q}uillen stratification.
\newblock {\em Ann. Sci. \'Ec. Norm. Sup\'er. (4)}, 49(4):907--925, 2016.

\bibitem[BDS16]{BalmerDellAmbrogioSanders16}
Paul Balmer, Ivo Dell'Ambrogio, and Beren Sanders.
\newblock Grothendieck-{N}eeman duality and the {W}irthm\"uller isomorphism.
\newblock {\em Compos. Math.}, 152(8):1740--1776, 2016.

\bibitem[Del10]{DellAmbrogio10}
Ivo Dell'Ambrogio.
\newblock Tensor triangular geometry and {$KK$}-theory.
\newblock {\em J. Homotopy Relat. Struct.}, 5(1):319--358, 2010.

\bibitem[DHS88]{DevinatzHopkinsSmith88}
Ethan~S. Devinatz, Michael~J. Hopkins, and Jeffrey~H. Smith.
\newblock Nilpotence and stable homotopy theory. {I}.
\newblock {\em Ann. of Math. (2)}, 128(2):207--241, 1988.

\bibitem[DT12]{DellAmbrogioTabuada12}
Ivo Dell'Ambrogio and Gon{\c{c}}alo Tabuada.
\newblock Tensor triangular geometry of non-commutative motives.
\newblock {\em Adv. Math.}, 229(2):1329--1357, 2012.

\bibitem[FHM03]{FauskHuMay03}
H.~Fausk, P.~Hu, and J.~P. May.
\newblock Isomorphisms between left and right adjoints.
\newblock {\em Theory Appl. Categ.}, 11:No. 4, 107--131, 2003.

\bibitem[HO16]{HellerOrmsby16pp}
Jeremiah Heller and Kyle Ormsby.
\newblock Primes and fields in stable motivic homotopy theory.
\newblock Preprint \texttt{arXiv:1608.02876}, 2016.

\bibitem[HPS97]{HoveyPalmieriStrickland97}
Mark Hovey, John~H. Palmieri, and Neil~P. Strickland.
\newblock Axiomatic stable homotopy theory.
\newblock {\em Mem. Amer. Math. Soc.}, 128(610), 1997.

\bibitem[HS98]{HopkinsSmith98}
Michael~J. Hopkins and Jeffrey~H. Smith.
\newblock Nilpotence and stable homotopy theory. {II}.
\newblock {\em Ann. of Math. (2)}, 148(1):1--49, 1998.

\bibitem[Pet13]{Peter13}
Tobias~J. Peter.
\newblock Prime ideals of mixed {A}rtin-{T}ate motives.
\newblock {\em Journal of K-theory}, 11(2):331--349, 2013.

\bibitem[San13]{Sanders13}
Beren Sanders.
\newblock Higher comparison maps for the spectrum of a tensor triangulated
  category.
\newblock {\em Adv. Math.}, 247:71--102, 2013.

\bibitem[Ste16]{Stevenson16pp}
Greg Stevenson.
\newblock A tour of support theory for triangulated categories through tensor
  triangular geometry.
\newblock Preprint available at \url{http://arxiv.org/abs/1601.03595}, 2016.

\bibitem[VSF00]{VoevodskySuslinFriedlander00}
Vladimir Voevodsky, Andrei Suslin, and Eric~M. Friedlander.
\newblock {\em Cycles, transfers, and motivic homology theories}, volume 143 of
  {\em Annals of Mathematics Studies}.
\newblock Princeton University Press, Princeton, NJ, 2000.

\end{thebibliography}

\end{document}


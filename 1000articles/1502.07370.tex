\documentclass[a4paper, 12pt]{amsart}
\usepackage[top=80pt,bottom=65pt,left=70pt,right=70pt]{geometry}
\usepackage{graphicx, eepic, bm, times}
\usepackage{mathpazo}
\usepackage{amsthm, amsmath,amsfonts,amssymb}
\usepackage{hyperref}
\usepackage{multirow, url}
\usepackage{color}
\usepackage[all]{xy}

\newtheorem{thm}{Theorem}[section]
\newtheorem{cor}[thm]{Corollary}
\newtheorem{lem}[thm]{Lemma}
\newtheorem{prop}[thm]{Proposition}
\newtheorem{prob}[thm]{Problem}
\newtheorem{assu}[thm]{Assumption}
\newtheorem{conj}[thm]{Conjecture}
\newtheorem{fact}[thm]{Fact}
\newtheorem{obse}[thm]{Observation}
\newtheorem{goal}[thm]{Goal}
\newtheorem{summ}[thm]{Summary}
\newtheorem{idea}[thm]{Idea}
\theoremstyle{definition}
\newtheorem{defn}[thm]{Definition}
\theoremstyle{remark}
\newtheorem{rem}[thm]{Remark}
\newtheorem{recall}[thm]{Recall}
\newtheorem{claim}[thm]{Claim}
\newtheorem{note}[thm]{Note}
\newtheorem{ques}[thm]{Question}
\newtheorem{answ}[thm]{Answer}
\newtheorem{exam}[thm]{Example}
\newtheorem{conv}[thm]{Convention}
\newtheorem{notation}[thm]{Notation}

\numberwithin{equation}{section} \numberwithin{table}{section}

 
\mathchardef\hyp="2D

 

  

  

\begin{document}                                                                          

\title{Rational torsion points on Jacobians of Shimura curves}
\author{Hwajong Yoo}
\email{hwajong@gmail.com}
\address{Universit\'e du Luxembourg, Facult\'e des Sciences, de la Technologie et de la Communication, 6, rue Richard Coudenhove-Kalergi, L-1359 Luxembourg, Luxembourg}

\date{\today}
\subjclass[2010]{11G10, 11G18, 14G05}
\keywords{Rational points, Shimura curves, Eisenstein ideals}

\begin{abstract}
Let $p$ and $q$ be distinct primes. Consider the Shimura curve ${{\mathcal{X}}}$ associated to the indefinite quaternion algebra of discriminant $pq$ over ${{\mathbb{Q}}}$. Let $J$ be the Jacobian variety of ${{\mathcal{X}}}$, which is an abelian variety over ${{\mathbb{Q}}}$. For an odd prime $\ell$, we provide sufficient conditions for the non-existence of rational points of order $\ell$ on $J$. As an application, we find some non-trivial subgroups of the kernel of an isogeny from the new quotient of $J_0(pq)$ to $J$.
\end{abstract} 
\maketitle
\setcounter{tocdepth}{1}
\tableofcontents

\section{Introduction}
Let $p$ and $q$ be distinct primes. Consider the modular curves $X_0(p)$ and $X_0(pq)$ over ${{\mathbb{Q}}}$; and their Jacobians $J_0(p)$ and $J_0(pq)$ over ${{\mathbb{Q}}}$. By the Mordell-Weil theorem, the rational points on $J_0(p)$ and $J_0(pq)$ are finitely generated abelian groups, and hence:
$$
J_0(p)({{\mathbb{Q}}})={{\mathbb{Z}}}^a \oplus J_0(p)({{\mathbb{Q}}})_{{\mathrm{tor}}}\quad\text{and}\quad J_0(pq)({{\mathbb{Q}}})={{\mathbb{Z}}}^b \oplus J_0(pq)({{\mathbb{Q}}})_{{\mathrm{tor}}},
$$
where $J_0(p)({{\mathbb{Q}}})_{{\mathrm{tor}}}$ and $J_0(pq)({{\mathbb{Q}}})_{{\mathrm{tor}}}$ are finite abelian groups. 

In early 1970s Ogg \cite{Og75} conjectured that the group $J_0(p)({{\mathbb{Q}}})_{{\mathrm{tor}}}$ is generated by the cuspidal divisor $[0-\infty]$. In his landmark paper \cite{M77}, Mazur proved this conjecture by the careful study of submodules of $J_0(p)$ annihilated by the Eisenstein ideal of the Hecke ring of level $p$. As an application, he was able to prove a classification theorem of the rational torsion subgroups of elliptic curves over ${{\mathbb{Q}}}$. 
A natural generalization is as follows.

\begin{conj}[Generalized Ogg's conjecture]
All rational torsion points on $J_0(pq)$ are cuspidal, i.e.,
$$
J_0(pq)({{\mathbb{Q}}})_{{\mathrm{tor}}} = {{\mathcal{C}}}(pq),
$$
where ${{\mathcal{C}}}(pq)$ is the cuspidal group of $J_0(pq)$.
\end{conj}
Our present knowledge is insufficient to prove the above conjecture completely. As Mazur pointed out \cite[p. 34]{M77}, the control of the $2$-torsion part of $J_0(pq)({{\mathbb{Q}}})_{{\mathrm{tor}}}$ is very difficult. Moreover, for a prime $\ell\geq 5$, if $p \equiv q \equiv 1 {{ \!\pmod {\ell}}}$, the control of the $\ell$-torsion part of $J_0(pq)({{\mathbb{Q}}})_{{\mathrm{tor}}}$ is also very difficult because the intersection of the $p$-old and $q$-old subvarieties has non-trivial $\ell$-torsion points (cf. \cite[Theorem 1]{R89}).

In this paper, instead of studying the above conjecture, we consider the Jacobian variety $J$ of the Shimura curve ${{\mathcal{X}}}$ associated to the quaternion algebra of discriminant $pq$ over ${{\mathbb{Q}}}$, which is isogenous to the new quotient of $J_0(pq)$. (cf. \cite[Th\`eor\`eme 2]{R80}.) 
Note that $J$ is an abelian variety over ${{\mathbb{Q}}}$ of dimension $g({{\mathcal{X}}})$, the genus of ${{\mathcal{X}}}$. 
From now on, we always assume that $g({{\mathcal{X}}}) \neq 0$. 

\begin{thm}\label{thm:maintheorem}
For a prime $\ell\geq 5$, the Jacobian $J$ does not have rational points of order $\ell$ if the following hold:
\begin{itemize}
\item If $p\equiv 1 {{ \!\pmod {\ell}}}$, then $q\not\equiv 1 {{ \!\pmod {\ell}}}$ and $q^{\frac{p-1}{\ell}} \not\equiv 1 \pmod p$;
\item If $q\equiv 1 {{ \!\pmod {\ell}}}$, then $p\not\equiv 1 {{ \!\pmod {\ell}}}$ and $p^{\frac{q-1}{\ell}} \not\equiv 1 \pmod q$.
\end{itemize}
Furthermore, the Jacobian $J$ does not have rational points of order $3$ if $(p-1)(q-1)$ is not divisible by $3$.
\end{thm}

Let us remark that in the first part of the theorem, the assumption $\ell \geq 5$ is crucial. For instance, the case where $p=7$ and $q=2$ satisfies the two conditions above for $\ell=3$. However, the Jacobian $J$ has rational torsion points of order $3$. Therefore we need a stronger condition such as the non-divisibility of $(p-1)(q-1)$ by $3$ to prove the non-existence of rational torsion points of order $3$ on $J$.

As an application of this theorem, we get information about the kernel of an isogeny between the new quotient of $J_0(pq)$ and $J$. More precisely, let $J_0(pq)^{{\mathrm{new}}}$ denote the new quotient of $J_0(pq)$,
$\Psi$ denote such an isogeny over ${{\mathbb{Q}}}$, and $K(pq)$ denote the kernel of $\Psi$:
$$
\xymatrix{
0 \ar[r] & J_0(pq)_{{\mathrm{old}}} \ar[r] & J_0(pq) \ar[r]^-{\pi} & J_0(pq)^{{\mathrm{new}}} \ar[r] & 0;
}
$$
$$
\xymatrix{
0\ar[r] & K(pq) \ar[r] & J_0(pq)^{{\mathrm{new}}} \ar[r]^-{\Psi} & J \ar[r] & 0.
}
$$
By his careful study of bad reduction of Shimura curves, Ogg \cite{Og85} conjectured that the image of some cuspidal divisors in $J_0(pq)$ is contained in $K(pq)$. In the case of low genus Shimura curves, this conjecture was proved by Gonz\'alez and Molina \cite{GM11}.
More precisely, they found an equation of ${{\mathcal{X}}}$ in the case where $g({{\mathcal{X}}}) \leq 3$ and computed $K(pq)$ by the consideration of bad reduction of ${{\mathcal{X}}}$.
Instead of finding an explicit equation of ${{\mathcal{X}}}$ to compute $K(pq)$, we prove that $K(pq)$ contains $\pi({{\mathcal{C}}}_{\ell}(pq))$ if $\ell$ satisfies certain conditions, where ${{\mathcal{C}}}_{\ell}(pq)$ is the $\ell$-primary subgroup of ${{\mathcal{C}}}(pq)$.

\begin{thm}\label{thm:kernelofisogeny}
Let $m$ and $n$ be the exact powers of $\ell$ dividing $p+1$ and $q+1$, respectively, i.e., $m=v_{\ell}(p+1)$ and $n=v_{\ell}(q+1)$, where $v_{\ell}$ is the normalized $\ell$-adic valuation such that $v_{\ell}(\ell)=1$.
If $\ell\geq 5$, then $K(pq)$ contains $\pi({{\mathcal{C}}}_{\ell}(pq))$, which is isomorphic to ${{{{\mathbb{Z}}}/{{\ell^m}}{{\mathbb{Z}}}}} \oplus {{{{\mathbb{Z}}}/{{\ell^n}}{{\mathbb{Z}}}}}$ if the following hold:
\begin{itemize}
\item If $p\equiv 1 {{ \!\pmod {\ell}}}$, then $q\not\equiv 1 {{ \!\pmod {\ell}}}$ and $q^{\frac{p-1}{\ell}} \not\equiv 1 \pmod p$;
\item If $q\equiv 1 {{ \!\pmod {\ell}}}$, then $p\not\equiv 1 {{ \!\pmod {\ell}}}$ and $p^{\frac{q-1}{\ell}} \not\equiv 1 \pmod q$.
\end{itemize}
If $\ell=3$, then $K(pq)$ contains $\pi({{\mathcal{C}}}_3(pq))$, which is isomorphic to ${{{{\mathbb{Z}}}/{{3^{\alpha}}}{{\mathbb{Z}}}}} \oplus {{{{\mathbb{Z}}}/{{3^{\beta}}}{{\mathbb{Z}}}}}$ if $\ell$ does not divide $(p-1)(q-1)$, where $\alpha=\max \{0,~m-1\}$ and $\beta=\max \{0, ~n-1\}$.
\end{thm}

The organization of this article is as follows. In \textsection \ref{sec:Eisenstein}, we discuss all possible new Eisenstein maximal ideals of level $pq$. In \textsection \ref{sec:criteriaformaximality}, we give certain criteria on primes $p$ and $q$ for an Eisenstein ideal discussed in the previous section to be maximal. In \textsection \ref{sec:structure}, we study the structure of the kernels of Eisenstein maximal ideals on Jacobians. In \textsection \ref{sec:proof}, we deduce Theorem \ref{thm:maintheorem} from the above results. Finally, we prove Theorem \ref{thm:kernelofisogeny} in \textsection \ref{sec:kernelofisogeny}. 

\subsubsection*{Acknowledgements} We are grateful to Sara Arias-de-Reyna, Kenneth Ribet and Sug Woo Shin for valuable comments and discussions.

\subsection{Notation}
Let $B$ be a quaternion algebra over ${{\mathbb{Q}}}$ of discriminant $D$ such that $\phi : B\otimes_{{\mathbb{Q}}} {{\mathbb{R}}} \simeq M_2({{\mathbb{R}}})$. Let ${{\mathcal{O}}}$ be an Eichler order of level $N$ of $B$ and ${{\mathcal{O}}}^{\times, 1}$ be the set of (reduced) norm 1 elements in ${{\mathcal{O}}}$. We define $\Gamma_0^D(N):=\phi({{\mathcal{O}}}^{\times, 1})$. Let $X_0^D(N)$ be the Shimura curve over ${{\mathbb{Q}}}$ associated to $B$ with $\Gamma_0^D(N)$ level structure and $J_0^D(N):={{\mathrm{Pic}}}^0(X_0^D(N))$ be its Jacobian variety. If $D=1$, then $X_0(N)=X_0^1(N)$ denotes the modular curve for $\Gamma_0(N)$ and $J_0(N)=J_0^1(N)$ denotes its Jacobian variety. If $D\neq 1$, then $X_0^D(N)({{\mathbb{C}}}) \simeq \Gamma_0^D(N)\backslash {{\mathbb{H}}}$, where ${{\mathbb{H}}}$ is the complex upper half plane. (Thus, ${{\mathcal{X}}}=X_0^{pq}(1)$ and $J=J_0^{pq}(1)$.)

For an integer $n\geq 1$, we can consider a Hecke operator $T_n$ acting on $J_0^D(N)$. We denote by ${{\mathbb{T}}}^D(N)$ the ${{\mathbb{Z}}}$-subalgebra of the endomorphism ring of $J_0^D(N)$ generated by
all $T_n$. In the case where $D=1$ (resp. $N=1$), we simply denote by ${{\mathbb{T}}}(N)$ (resp. ${{\mathbb{T}}}^D$) the Hecke ring ${{\mathbb{T}}}^1(N)$ (resp. ${{\mathbb{T}}}^D(1)$). 
If $p$ divides $DN$, we denote by $U_p$ the Hecke operator $T_p$ on $J_0^D(N)$. 
For a prime $p$ dividing $N$, we denote by $w_p$ the Atkin-Lehner involution on $J_0^D(N)$. 
For a maximal ideal ${{\mathfrak{m}}}$ of a Hecke ring ${{\mathbb{T}}}$, we denote by ${{\mathbb{T}}}_{{\mathfrak{m}}}$ the completion of ${{\mathbb{T}}}$ at ${{\mathfrak{m}}}$, i.e.: 
$$
{{\mathbb{T}}}_{{\mathfrak{m}}}:= {\lim_{\substack{\longleftarrow\\{n}}}\;} {{\mathbb{T}}}/{{{\mathfrak{m}}}^n}.
$$

\section{Eisenstein ideals in ${{\mathbb{T}}}^{pq}$}\label{sec:Eisenstein}
From now on, we fix distinct primes $p$ and $q$; and $\ell$ denotes an odd prime. Let ${{\mathbb{T}}}:={{\mathbb{T}}}^{pq}$ and $I_0:=(T_r-r-1~:~\text{for primes } r \nmid pq)$ be an Eisenstein ideal of ${{\mathbb{T}}}$. (An ideal is called \textit{Eisenstein} if it contains $I_0$.)

\begin{lem}\label{lem:Up^2=1}
We have $U_p^2=U_q^2=1 \in {{\mathbb{T}}}$.
\end{lem}
\begin{proof}
Let $w_p$ and $w_q$ be the Atkin-Lehner involutions on $J_0(pq)$. Then $U_p+w_p=0$ on the space of newforms (cf. \cite[Proposition 3.7]{R90}). Because $\Psi$ is Hecke-equivariant (cf. \cite[\textsection 4]{R90}), $U_p$ and $U_q$ are also involutions on $J$.
\end{proof}

\begin{defn}
We define some Eisenstein ideals as follows: 
$$
I_1 := (U_p-1,~U_q-1~, I_0), \quad\quad\quad\quad\quad I_2 := (U_p+1,~U_q+1,~I_0),
$$
$$
I_3 := (U_p-1,~U_q+1~, I_0) \quad\quad\text{and}\quad\quad I_4 := (U_p+1,~U_q-1,~I_0).
$$
Moreover, we set ${{\mathfrak{m}}}_i :=(\ell, ~I_i)$. They are all possible Eisenstein maximal ideals in ${{\mathbb{T}}}$ by the above lemma.
\end{defn}

Let ${{\mathbb{T}}}_{\ell}:={{\mathbb{T}}} \otimes_{{\mathbb{Z}}} {{\mathbb{Z}}}_{\ell}$. Then it is a semi-local ring and we have:
$$
{{\mathbb{T}}}_{\ell} = \prod_{\ell \in {{\mathfrak{m}}}~ \text{maximal}} {{\mathbb{T}}}_{{\mathfrak{m}}}.
$$

Using the above description of Eisenstein maximal ideals, we prove the following.

\begin{prop}\label{prop:decomposition}
The quotient ${{\mathbb{T}}}_{\ell}/{I_0}$ decomposes as follows:
$$
{{\mathbb{T}}}_{\ell}/{I_0} = \prod_{i=1}^4 {{\mathbb{T}}}_{{{\mathfrak{m}}}_i}/I_0 \simeq \prod_{i=1}^4 {{\mathbb{T}}}_{\ell}/{I_i}.
$$
\end{prop}

\begin{proof}
It suffices to prove that ${{\mathbb{T}}}_{\ell}/{I} \simeq {{\mathbb{T}}}_{{\mathfrak{m}}}/I_0$, where $I:=I_i$ and ${{\mathfrak{m}}}:=(\ell, ~I)$. 
We discuss the case where $i=1$; the other cases are basically the same. 

Note that ${{\mathfrak{m}}}$ is either maximal or ${{\mathfrak{m}}} = {{\mathbb{T}}}_{\ell}$.
Therefore if ${{\mathfrak{m}}}$ is not maximal, then ${{\mathbb{T}}}_{{\mathfrak{m}}}=0={{\mathbb{T}}}_{\ell}/I$. Thus, we may assume that ${{\mathfrak{m}}}$ is maximal. 
Since $\ell$ is odd and $U_p-1 \in {{\mathfrak{m}}}$, we have $U_p+1 \not\in {{\mathfrak{m}}}$. In other words, $U_p+1$ is a unit in ${{\mathbb{T}}}_{{\mathfrak{m}}}$. By Lemma \ref{lem:Up^2=1}, we have $U_p-1=0$ in ${{\mathbb{T}}}_{{\mathfrak{m}}}$ and hence $U_p-1 \in I_0 {{\mathbb{T}}}_{{\mathfrak{m}}}$. Similarly we have $U_q-1 \in I_0{{\mathbb{T}}}_{{\mathfrak{m}}}$. Therefore ${{\mathbb{T}}}_{{\mathfrak{m}}}/{I_0} = {{\mathbb{T}}}_{{\mathfrak{m}}}/I$. Since the index of $I$ in ${{\mathbb{T}}}$ is finite (cf. \cite[Lemma 3.1]{Yoo14}), for sufficiently large $n$ we have ${{\mathfrak{m}}}^n \subseteq I$. Thus, we have $({{\mathbb{T}}}_{\ell}/{{{\mathfrak{m}}}^n})/I = {{\mathbb{T}}}_{\ell}/I$ and hence ${{\mathbb{T}}}_{{\mathfrak{m}}}/I \simeq {{\mathbb{T}}}_{\ell}/I$.
\end{proof}

\section{Criteria for ${{\mathfrak{m}}}$ to be maximal}\label{sec:criteriaformaximality}
In this section, we discuss certain conditions on the primes $p$ and $q$ for which ${{\mathfrak{m}}}_i$ is maximal. By the Jacquet-Langlands correspondence, it suffices to show that ${{\mathfrak{m}}}_i$ is new maximal in ${{\mathbb{T}}}(pq)$ under the given assumption. 

\subsection{Maximality of ${{\mathfrak{m}}}_1$} \label{sec:s=2}
\begin{thm}\label{thm:m1}
The ideal ${{\mathfrak{m}}}_1$ is maximal in ${{\mathbb{T}}}$ if and only if one of the following holds:
\begin{itemize}
\item $p \equiv q \equiv 1 {{ \!\pmod {\ell}}}$;
\item $\ell$ divides the numerator of $\frac{p-1}{3}$ and $q^{\frac{p-1}{\ell}}\equiv 1 \pmod p$;
\item $\ell$ divides the numerator of $\frac{q-1}{3}$ and $p^{\frac{q-1}{\ell}}\equiv 1 \pmod q$.
\end{itemize} 
\end{thm}
\begin{proof}
Let ${{\mathfrak{m}}}:={{\mathfrak{m}}}_1$ and $I:=I_1$.

If $\ell\geq 5$ and $\ell$ does not divide $pq$, this is proved by Ribet \cite[Theorem 2.4]{Yoo14a}.

Since the index of $I$ in ${{\mathbb{T}}}(pq)$ is equal to the numerator of $\frac{(p-1)(q-1)}{3}$ up to powers of 2 \cite[Theorem 5.1]{Yoo15a}, we assume that $\ell$ divides the numerator of $\frac{(p-1)(q-1)}{3}$, and hence ${{\mathfrak{m}}}$ is maximal in ${{\mathbb{T}}}(pq)$.

Let $\ell=3$ and $\lambda$ be the corresponding ideal of ${{\mathfrak{m}}}$ in the Hecke ring ${{\mathbb{T}}}(p)$ of level $p$. By Mazur \cite{M77}, $\lambda$ is maximal if and only if $p \equiv 1 \pmod 9$. 
First, we assume that $p \equiv 1 \pmod 9$. Let $R:={{\mathbb{T}}}(p)_{\lambda}$ be the completion of ${{\mathbb{T}}}(p)$ at $\lambda$ and $I$ be the Eisenstein ideal of ${{\mathbb{T}}}(p)$. 
Then, $IR\neq (T_q-q-1)R$ if and only if either $q\equiv 1 \pmod 3$ or $q^{\frac{p-1}{3}}\equiv 1 \pmod p$. By the same argument as in the proof of \cite[Theorem 2.4]{Yoo14a}, ${{\mathfrak{m}}}$ is new maximal if and only if $IR \neq (T_q-q-1)R$. Therefore by symmetry, the result follows unless $p-1$ and $q-1$ are exactly divisible by $3$. Next, we assume that $p-1$ and $q-1$ are exactly divisible by $3$. Then ${{\mathfrak{m}}}$ is new because it is neither $p$-old nor $q$-old. 

If $\ell\geq 5$, the same method as above works and the result follows directly.
\end{proof}

\begin{rem}
In the above proof, we don't need to assume that $\ell$ does not divide $pq$. Note that one direction in the proof of \cite[Theorem 2.4]{Yoo14a} relies on the saturation property of ${{\mathbb{T}}}(pq)$ in ${{\mathrm{End}}}(J_0(pq))$ locally at ${{\mathfrak{m}}}$. If either $\ell=p$ or $\ell=q$, this property follows from the second case of \cite[Theorem 3.3]{Yoo14a} because $T_{\ell} \equiv 1 \pmod {{\mathfrak{m}}}$ (cf. \cite[Lemma 1.1]{R08} or \cite[Remark 3.5]{Yoo14a}). The other direction follows by the same argument as in the proof of \cite[Theorem 2.4]{Yoo14a} without further difficulties.
\end{rem}

\subsection{Maximality of ${{\mathfrak{m}}}_2$}
\begin{thm}\label{thm:m2}
The ideal ${{\mathfrak{m}}}_2$ cannot be maximal.
\end{thm}
\begin{proof}
For an Eisenstein maximal ideal ${{\mathfrak{m}}}$, we have $T_{\ell}\equiv 1 \pmod {{\mathfrak{m}}}$. Therefore ${{\mathfrak{m}}}_2$ is not maximal if either $\ell=p$ or $\ell=q$ because $\ell$ is odd. 
Thus, we assume that $\ell$ does not divide $pq$ and ${{\mathfrak{m}}}_2$ is maximal. By \cite[Theorem 1.2.(3)]{Yoo14a}, we have $p\equiv q \equiv -1 {{ \!\pmod {\ell}}}$ and ${{\mathfrak{m}}}_2=(\ell, ~U_p-p,~U_q-q,~I_0)$. By \cite[Proposition 5.5]{Yoo15a}, ${{\mathfrak{m}}}_2$ cannot be maximal, which is a contradiction. Therefore the result follows.
\end{proof}

\subsection{Maximality of ${{\mathfrak{m}}}_3$ and ${{\mathfrak{m}}}_4$}\label{sec:m3}
\begin{thm}\label{thm:m3}
The ideal ${{\mathfrak{m}}}_3$ is maximal if and only if $\ell$ divides the numerator of $\frac{q+1}{(3, ~p(p+1))}$.
By symmetry, the ideal ${{\mathfrak{m}}}_4$ is maximal if and only if $\ell$ divides the numerator of $\frac{p+1}{(3, ~q(q+1))}$.
\end{thm}
\begin{proof}
Let ${{\mathfrak{m}}}:={{\mathfrak{m}}}_3$ and $I:=I_3$.

If $\ell\geq 5$ and $\ell$ does not divide $pq$, this is proved by Ribet \cite[Theorem 1.4(2)]{Yoo14a}.

Let $n$ be the numerator of $\frac{q+1}{\gcd(3, ~p(p+1))}$. Since ${{\mathbb{T}}}$ is a quotient of ${{\mathbb{T}}}(pq)$ and the index of $I$ in ${{\mathbb{T}}}(pq)$ is equal to $n$ up to powers of 2 \cite[Theorem 3.4]{Yoo14}, ${{\mathfrak{m}}}$ is not maximal if $\ell$ does not divide $n$. Conversely, if $\ell$ divides $n$ then $S_q[{{\mathfrak{m}}}] \neq 0$ by Proposition \ref{prop:skoro}, where $S_q$ is the Skorobogatov subgroup of $J$ from the level structure at $q$. Therefore ${{\mathfrak{m}}}$ is maximal.
\end{proof}

\section{The structure of $J[{{\mathfrak{m}}}]$}\label{sec:structure}
In this section, we discuss the structure of $J[{{\mathfrak{m}}}]$, where ${{\mathfrak{m}}}={{\mathfrak{m}}}_i$ for $1\leq i \leq 4$. 
If ${{\mathfrak{m}}}_i$ is not maximal, then $J[{{\mathfrak{m}}}]=0$. Therefore it suffices to study $J[{{\mathfrak{m}}}_1]$ and $J[{{\mathfrak{m}}}_3]$. (The structure of $J[{{\mathfrak{m}}}_4]$ follows by symmetry.)

\subsection{Multiplicity one for Jacobians of Shimura curves}
In this subsection, we prove the following multiplicity one result. 
\begin{thm}\label{thm:multiplicityone}
Assume that ${{\mathfrak{m}}}={{\mathfrak{m}}}_3$ is maximal. 
If $\ell=3$, we further assume that $3$ does not divide $(p-1)(q-1)$.
Then, $J[{{\mathfrak{m}}}]$ is a non-trivial extension of ${{{{\mathbb{Z}}}/{\ell{{\mathbb{Z}}}}}}$ by ${{\mu_{\ell}}}$. Moreover, $J[{{\mathfrak{m}}}]$ is ramified at $p$ but is unramified at $q$. Therefore we have ${{{{\mathbb{Z}}}/{\ell{{\mathbb{Z}}}}}} \nsubseteq J[{{\mathfrak{m}}}]$.
\end{thm}

When ${{\mathfrak{m}}}={{\mathfrak{m}}}_1$ is maximal, the structure of $J[{{\mathfrak{m}}}]$ is more complicated than the one of $J[{{\mathfrak{m}}}_3]$. However, if one of $p-1$ and $q-1$ is not divisible by $\ell$, then we have the similar result as above. Note that the theorem below is not used in the proof of our main theorem.
\begin{thm}\label{thm:multionem=1}
Assume that ${{\mathfrak{m}}}={{\mathfrak{m}}}_1$ is maximal. Assume further $\ell\geq 5$ and $q \not\equiv 1 {{ \!\pmod {\ell}}}$. Then, $J[{{\mathfrak{m}}}]$ is of dimension $2$ and is ramified at $p$.
\end{thm}

\begin{proof}[Proof of Theorem \ref{thm:multiplicityone}]
Let ${{\mathfrak{m}}}={{\mathfrak{m}}}_3$. By Theorem \ref{thm:m3}, ${{\mathfrak{m}}}$ is maximal if and only if $\ell$ divides the numerator of $\frac{q+1}{\gcd(3, ~p(p+1))}$. In particular, we assume that $q\not\equiv 1 {{ \!\pmod {\ell}}}$. 

For a square-free integer $DN$, we denote by $J_0^D(N)_{/{{{\mathbb{F}}}_p}}$ the special fiber of the N\'eron model of $J_0^D(N)$ over ${{\mathbb{F}}}_p$. If $p$ is a divisor of $N$ (resp. $D$), then it is given by a Deligne-Rapoport (resp. Cerednik-Drinfeld) model \cite{Bu97, Ce76, DR73, Dr76} and the theory of Raynaud \cite{Ra70}. We denote by $\Phi_p(J_0^D(N))$ (resp. $X_p(J_0^D(N))$) the component (resp. character) group of $J_0^D(N)_{/{{{\mathbb{F}}}_p}}$. 

We shall carry out a proof in several steps. 
\begin{enumerate}
\item Step 1 : We show that $\Phi_p(J)[{{\mathfrak{m}}}]=0$. 

By Ribet \cite[Theorem 4.3]{R90}, there is a Hecke-equivariant exact sequence:
$$
\xymatrix{
0 \ar[r] & K \ar[r] & (X\oplus X)/{\delta_p (X\oplus X)} \ar[r] & \Phi_p(J) \ar[r] & C \ar[r] & 0,
}
$$
where $X:=X_q(J_0(q))$ and $\delta_p = {
 \left(  \begin{smallmatrix} {p+1} & {T_p} \\ {T_p} & {p+1} \end{smallmatrix} \right)}$; and
$K$ (resp. $C$) is the kernel (resp. cokernel) of the map:
$$
\gamma_p : \Phi_q(J_0(q)) \times \Phi_q(J_0(q)) \rightarrow \Phi_q(J_0(pq))
$$
induced by the degeneracy map $\gamma_p : J_0(q) \times J_0(q) \rightarrow J_0(pq)$.
Since $q \not\equiv 1 {{ \!\pmod {\ell}}}$, there is no Eisenstein ideal of level $q$ containing $\ell$.
Therefore the first and second terms of the above exact sequence have no support at ${{\mathfrak{m}}}$.

\begin{itemize}
\item 
If $\ell\geq 5$, then $C[{{\mathfrak{m}}}]=0$ by \cite[Proposition A.5]{Yoo14a} and \cite[Corollary A.6]{Yoo14a}.
Therefore $\Phi_p(J)[{{\mathfrak{m}}}]=0$.
\item
If $\ell=3$ and $q>3$, then the $3$-primary part of $\Phi_q(J_0(pq))$ is cyclic by \cite[\textsection 4.4.1]{Ed91} because $p\not\equiv 1 \pmod {3}$.  Since $U_q$ acts as $1$ on it (cf. \cite[Proposition A.2]{Yoo14a}), we have $C[{{\mathfrak{m}}}]=0$ and hence $\Phi_p(J)[{{\mathfrak{m}}}]=0$.
\item
If $\ell=3$ and $q=2$, we have $\Phi_p(J)[{{\mathfrak{m}}}]=0$ by the table in \cite[p. 210]{Og85} because $p\equiv -1\pmod 3$.  
\end{itemize}

\item Step 2 : We show that $Y_{{\mathfrak{m}}}$ is self-dual and free of rank 1 over ${{\mathbb{T}}}_{{\mathfrak{m}}}$, where $Y:=X_p(J)$, the character group of $J_{/{{\mathbb{F}}}_p}$.

By Ribet \cite{R90}, there is a Hecke-equivariant exact sequence:
$$
\xymatrix{
0 \ar[r] & Y \ar[r] & L \ar[r] & X \oplus X \ar[r] & 0,
}
$$
where $L:=X_q(J_0(pq))$.
By localizing at ${{\mathfrak{m}}}$, we have $Y_{{\mathfrak{m}}} \simeq L_{{\mathfrak{m}}}$.
Since the dimension of $L/{{{\mathfrak{m}}} L}$ is $1$ (cf. \cite[Theorem 4.5.(4)]{Yoo14}), the dimension of $Y/{{{\mathfrak{m}}} Y}$ is $1$ as well, and hence $Y_{{\mathfrak{m}}}$ is free of rank 1 over ${{\mathbb{T}}}_{{\mathfrak{m}}}$.
Moreover by the monodromy exact sequence, we have a Hecke-equivariant exact sequence:
$$
\xymatrix{
0 \ar[r] & Y \ar[r] & {{\mathrm{Hom}}}(Y, \,{{\mathbb{Z}}}) \ar[r] & \Phi_p(J) \ar[r] & 0.
}
$$
Since $\Phi_p(J)[{{\mathfrak{m}}}]=0$, we have $Y_{{\mathfrak{m}}} \simeq {{\mathrm{Hom}}}(Y_{{\mathfrak{m}}}, \,{{\mathbb{Z}}}_{\ell})$. In other words, $Y_{{\mathfrak{m}}}$ is self-dual and free of rank 1 over ${{\mathbb{T}}}_{{\mathfrak{m}}}$, and hence ${{\mathbb{T}}}_{{\mathfrak{m}}}$ is Gorenstein.

\item Step 3 : We show that $J[{{\mathfrak{m}}}]$ is of dimension 2.

By Grothendieck \cite{Gro72}, there is an exact sequence:
$$
\xymatrix{
0 \ar[r] & {{\mathrm{Hom}}}(Y/{\ell^n Y}, \,\mu_{\ell^n}) \ar[r] & J[\ell^n] \ar[r] & Y/{\ell^n Y} \ar[r] & 0.
}
$$
(For details, see \cite[\textsection 3.3]{R76}.)
By taking projective limits, we have:
$$
\xymatrix{
0 \ar[r] & {{\mathrm{Hom}}}(Y_{\ell}, \,{{\mathbb{Z}}}_{\ell}(1)) \ar[r] & {{\mathrm{Ta}}}_{\ell} J \ar[r] & Y_{\ell} \ar[r] & 0,
}
$$
where ${{\mathrm{Ta}}}_{\ell}J$ is the $\ell$-adic Tate module of $J$ and ${{\mathbb{Z}}}_{\ell}(1)$ is the Tate twist of ${{\mathbb{Z}}}_{\ell}$. Since ${{\mathbb{T}}}_{{\mathfrak{m}}}$ is a direct factor of ${{\mathbb{T}}}_{\ell}$, we have: 
$$
\xymatrix{
0 \ar[r] & {{\mathrm{Hom}}}(Y_{{\mathfrak{m}}}, \,{{\mathbb{Z}}}_{\ell}(1)) \ar[r] & {{\mathrm{Ta}}}_{{\mathfrak{m}}} J \ar[r] & Y_{{\mathfrak{m}}} \ar[r] & 0.
}
$$
Since $Y_{{\mathfrak{m}}}$ is self-dual and free of rank 1 over ${{\mathbb{T}}}_{{\mathfrak{m}}}$, ${{\mathrm{Ta}}}_{{\mathfrak{m}}} J$ is a free ${{\mathbb{T}}}_{{\mathfrak{m}}}$-module of rank 2. Therefore $J[{{\mathfrak{m}}}]$ is of dimension 2.

\item Step 4 : We show that $J[{{\mathfrak{m}}}]$ is ramified at $p$.

Let $I_p$ be an inertia subgroup of ${{\mathrm{Gal}(\overline{\mathbb{Q}}/{\mathbb{Q}})}}$ at $p$. Then by Serre-Tate \cite{ST68}, we have $J[{{\mathfrak{m}}}]^{I_p} \simeq J_{/{{{\mathbb{F}}}_p}}[{{\mathfrak{m}}}]$. Since $\Phi_p(J)[{{\mathfrak{m}}}]=0$ and $J^0[{{\mathfrak{m}}}] = {{\mathrm{Hom}}}(Y/{{{\mathfrak{m}}} Y},\, {{\mu_{\ell}}})$ is of dimension 1, $J[{{\mathfrak{m}}}]^{I_p} \simeq J^0[{{\mathfrak{m}}}]$ is of dimension 1 as well, where $J^0$ is the identity component of $J_{/{{{\mathbb{F}}}_p}}$. Therefore $J[{{\mathfrak{m}}}]$ is ramified at $p$.

\item Step 5 :  We show that $J[{{\mathfrak{m}}}]$ contains ${{\mu_{\ell}}}$.

By Proposition \ref{prop:skoro}, our assumption on ${{\mathfrak{m}}}$ implies ${{\mu_{\ell}}} \simeq S_q[{{\mathfrak{m}}}] \subseteq J[{{\mathfrak{m}}}]$, where $S_q$ is the Skorobogatov subgroup of $J$ from the level structure at $q$.

\item Step 6 : We show that $J[{{\mathfrak{m}}}]$ is a non-trivial extension of ${{{{\mathbb{Z}}}/{\ell{{\mathbb{Z}}}}}}$ by ${{\mu_{\ell}}}$.

Since all Jordan-H\"older factors of $J[{{\mathfrak{m}}}]$ are either ${{\mu_{\ell}}}$ or ${{{{\mathbb{Z}}}/{\ell{{\mathbb{Z}}}}}}$ 
(cf. \cite[Proposition 14.1]{M77}), the quotient
$J[{{\mathfrak{m}}}]/{{\mu_{\ell}}}$ is isomorphic to either ${{{{\mathbb{Z}}}/{\ell{{\mathbb{Z}}}}}}$ or ${{\mu_{\ell}}}$. If $J[{{\mathfrak{m}}}]/{{\mu_{\ell}}} \simeq {{\mu_{\ell}}}$, then $J[{{\mathfrak{m}}}^{\infty}]$ is a multiplicative ${{\mathfrak{m}}}$-divisible module, which is a contradiction. Therefore $J[{{\mathfrak{m}}}]$ is an extension of ${{{{\mathbb{Z}}}/{\ell{{\mathbb{Z}}}}}}$ by ${{\mu_{\ell}}}$. Since $J[{{\mathfrak{m}}}]$ is ramified at $p$, we have ${{{{\mathbb{Z}}}/{\ell{{\mathbb{Z}}}}}} \nsubseteq J[{{\mathfrak{m}}}]$.

\item Step 7 : We finish the proof by showing that $J[{{\mathfrak{m}}}]$ is unramified at $q$.

Let ${{\mathrm{Frob}}}_q$ be the Frobenius endomorphism in characteristic $q$. Then, ${{\mathrm{Frob}}}_q$ acts by $qU_q$ on the torus $T$ of $J_{/{{{\mathbb{F}}}_q}}$ (cf. \cite[Theorem 3.1]{JL86}, \cite{R89b}). Since $U_q \equiv -1 \pmod {{\mathfrak{m}}}$, $T[{{\mathfrak{m}}}]$ cannot contain ${{\mu_{\ell}}}$, which is included in $J[{{\mathfrak{m}}}]^{I_q}$.
Since ${{\mathbb{T}}}$ acts faithfully on $X_q(J)$ and ${{\mathfrak{m}}}$ is maximal, $T[{{\mathfrak{m}}}]$ is at least of dimension 1. Therefore $J[{{\mathfrak{m}}}]^{I_q} \simeq J_{/{{{\mathbb{F}}}_q}}[{{\mathfrak{m}}}]$ is at least of dimension 2. Thus, $J[{{\mathfrak{m}}}]$ is unramified at $q$.
\end{enumerate}
\end{proof}

\begin{rem}
Most of the above proof was given by Ribet in \cite[Appendix B]{Yoo14a} under the assumption that $p \not\equiv 1{{ \!\pmod {\ell}}}$ and $\ell\geq 5$. However we duplicate the proof here to point out where our assumption plays a role. For instance, if $p=7$ and $q=2$, then $\Phi_p(J)[{{\mathfrak{m}}}] \simeq {{{{\mathbb{Z}}}/{{3}}{{\mathbb{Z}}}}}$, and hence the above proof does not hold.
\end{rem}

\begin{proof}[Proof of Theorem \ref{thm:multionem=1}] Since $q\not\equiv 1{{ \!\pmod {\ell}}}$, ${{\mathfrak{m}}}$ is not $p$-old by Mazur \cite{M77}. Moreover we have $C[{{\mathfrak{m}}}]=0$ as in Step 1 of the above proof because $U_q$ acts by $q$ on $C[\ell]$. Therefore the argument in Step 1 works in this case as well. With our assumption on $q$, the arguments in Step 2 to 4 are also valid as above, and hence the result follows.
\end{proof}

\subsection{The Skorobogatov subgroups of $J$}\label{sec:Skorobogatov}
In this subsection, we discuss a subgroup of $J[{{\mathfrak{m}}}_3]$, which is the $\ell$-torsion subgroup of the Skorobogatov subgroup $S_q$ from the level structure at $q$. 
In \cite[Appendix C]{Yoo14a}, we studied the actions of the Hecke operators on $S_q$ and computed its order {{up to products of powers of 2 and 3}}. Since we include the discussion with $\ell=3$, we compute the $\ell$-torsion subgroup on $S_q$ for any odd prime $\ell$. 
\begin{prop}\label{prop:skoro}
We have $S_q[\ell] \neq 0$ if and only if $\ell$ divides the numerator of $\frac{q+1}{\gcd(3, ~p(p+1))}$. If $S_q[\ell]\neq 0$, then we have $S_q[\ell]=S_q[{{\mathfrak{m}}}_3]\simeq {{\mu_{\ell}}}$.
\end{prop}
\begin{proof}
Since the order of $S_q$ is equal to $\frac{q+1}{\epsilon(q)}$ (up to powers of $2$), the first statement follows by the definition of $\epsilon(q)$ in \cite[p.~781]{Sk05}. 
Since $S_q$ is the Cartier dual of the constant cyclic group scheme (cf. \textit{loc. cit.}), $S_q[\ell]$ is isomorphic to ${{\mu_{\ell}}}$ if it is not zero. Therefore we have $S_q[\ell]=S_q[{{\mathfrak{m}}}_3]\simeq {{\mu_{\ell}}}$ by \cite[Proposition C.2]{Yoo14a} if $S_q[\ell]\neq 0$. 
\end{proof}

\section{Proof of Theorem \ref{thm:maintheorem}}\label{sec:proof}
In this section, we prove our main theorem.
\begin{thm}
For a prime $\ell\geq 5$, the Jacobian $J$ does not have rational points of order $\ell$ if following hold:
\begin{itemize}
\item If $p\equiv 1 {{ \!\pmod {\ell}}}$, then $q\not\equiv 1 {{ \!\pmod {\ell}}}$ and $q^{\frac{p-1}{\ell}} \not\equiv 1 \pmod p$;
\item If $q\equiv 1 {{ \!\pmod {\ell}}}$, then $p\not\equiv 1 {{ \!\pmod {\ell}}}$ and $p^{\frac{q-1}{\ell}} \not\equiv 1 \pmod q$.
\end{itemize}
Furthermore, the Jacobian $J$ does not have rational points of order $3$ if $(p-1)(q-1)$ is not divisible by $3$.
\end{thm}
\begin{proof}
Let $A:=J({{\mathbb{Q}}})_{{\mathrm{tor}}}$ and $A_{\ell}:=A \otimes_{{\mathbb{Z}}} {{\mathbb{Z}}}_{\ell}$. Then $A_{\ell}$ is a ${{\mathbb{T}}}_{\ell}$-module. For a prime $r$ not dividing $pq$, by the Eichler-Shimura relation and the isogeny $\Psi$ we have:
$$
T_r \equiv {{\mathrm{Frob}}}_r+{{\mathrm{Ver}}}_r  ~\text{ on }~ J_{/{{{\mathbb{F}}}_r}},
$$
where ${{\mathrm{Frob}}}_r$ is the Frobenius morphism in characteristic $r$ and ${{\mathrm{Ver}}}_r$ is its transpose. Therefore $T_r-1-r$ kills $A$ and hence $A_{\ell}$ is annihilated by $I_0 {{\mathbb{T}}}_{\ell}$, i.e., $A_{\ell}$ is a ${{\mathbb{T}}}_{\ell}/{I_0}$-module. By Proposition \ref{prop:decomposition}, it decomposes into $A_{\ell}^i$, where each $A_{\ell}^i$ is a ${{\mathbb{T}}}_{\ell}/{I_i}$-module. More precisely, we have 
$A_{\ell}^i = A_{\ell} \cap J[I_i] = A_{\ell}[I_i]$. 
Thus, it suffices to prove that $A_{\ell}^i=0$ for all $1\leq i \leq 4$.

Assume that the two conditions above hold. Then, by Theorem \ref{thm:m1}, ${{\mathfrak{m}}}_1$ is not maximal and hence $A_{\ell}^1=0$. By Theorem \ref{thm:m2}, we have $A_{\ell}^2=0$ as well. Now we assume that $A_{\ell}^3 \neq 0$. If $\ell=3$, then we further assume that $\ell$ does not divide $(p-1)(q-1)$. 
Then $A_{\ell}^3[\ell] \simeq ({{{{\mathbb{Z}}}/{\ell{{\mathbb{Z}}}}}})^a$ for some $a \geq 1$. Since $A_{\ell}^3[\ell]=A_{\ell}[\ell, ~I_3]=A_{\ell}[{{\mathfrak{m}}}_3]$, we have ${{{{\mathbb{Z}}}/{\ell{{\mathbb{Z}}}}}} \subseteq J[{{\mathfrak{m}}}_3]$. This contradicts Theorem \ref{thm:multiplicityone}. Thus, we have $A_{\ell}^3=0$ and hence $A_{\ell}^4=0$ by symmetry. 
\end{proof}

\section{The kernel of an isogeny due to Ribet}\label{sec:kernelofisogeny}
In this section, we provide an application of our main theorem. As before, let $J_0(pq)^{{\mathrm{new}}}$ denote the new quotient of $J_0(pq)$,
$\Psi$ denote an isogeny from $J_0(pq)^{{\mathrm{new}}}$ to $J$, and $K(pq)$ denote the kernel of $\Psi$:
$$
\xymatrix{
0 \ar[r] & J_0(pq)_{{\mathrm{old}}} \ar[r] & J_0(pq) \ar[r]^-{\pi} & J_0(pq)^{{\mathrm{new}}} \ar[r] & 0;
}
$$
$$
\xymatrix{
0\ar[r] & K(pq) \ar[r] & J_0(pq)^{{\mathrm{new}}} \ar[r]^-{\Psi} & J \ar[r] & 0.
}
$$
Ogg \cite{Og85} conjectured that the image of some cuspidal divisors in $J_0(pq)$ is contained in $K(pq)$. This conjecture was proved by Gonz\'alez and Molina \cite{GM11} if $g({{\mathcal{X}}})$, the genus of ${{\mathcal{X}}}$, is at most $3$ . We prove some part of the conjecture by Ogg as follows:

\begin{thm} 
Let $m$ and $n$ be the exact powers of $\ell$ dividing $p+1$ and $q+1$, respectively, i.e., $m=v_{\ell}(p+1)$ and $n=v_{\ell}(q+1)$, where $v_{\ell}$ is the normalized $\ell$-adic valuation such that $v_{\ell}(\ell)=1$.
Let ${{\mathcal{C}}}_{\ell}(pq)$ be the $\ell$-primary subgroup of the cuspidal group ${{\mathcal{C}}}(pq)$ of $J_0(pq)$.
If $\ell\geq 5$, then $K(pq)$ contains $\pi({{\mathcal{C}}}_{\ell}(pq))$, which is isomorphic to ${{{{\mathbb{Z}}}/{{\ell^m}}{{\mathbb{Z}}}}} \oplus {{{{\mathbb{Z}}}/{{\ell^n}}{{\mathbb{Z}}}}}$  
if the following hold:
\begin{itemize}
\item If $p\equiv 1 {{ \!\pmod {\ell}}}$, then $q\not\equiv 1 {{ \!\pmod {\ell}}}$ and $q^{\frac{p-1}{\ell}} \not\equiv 1 \pmod p$;
\item If $q\equiv 1 {{ \!\pmod {\ell}}}$, then $p\not\equiv 1 {{ \!\pmod {\ell}}}$ and $p^{\frac{q-1}{\ell}} \not\equiv 1 \pmod q$.
\end{itemize}
If $\ell=3$, then $K(pq)$ contains $\pi({{\mathcal{C}}}_3(pq))$, which is isomorphic to ${{{{\mathbb{Z}}}/{{3^{\alpha}}}{{\mathbb{Z}}}}} \oplus {{{{\mathbb{Z}}}/{{3^{\beta}}}{{\mathbb{Z}}}}}$ if $\ell$ does not divide $(p-1)(q-1)$, where $\alpha=\max \{0,~m-1\}$ and $\beta=\max \{0, ~n-1\}$.
\end{thm}

\begin{proof} 
Let $C_p:=[P_1-P_p]$ and $C_q:=[P_1-P_q]$ be elements in ${{\mathcal{C}}}(pq)$, where $P_t$ is the cusp of $X_0(pq)$ corresponding to $1/t \in {{\mathbb{P}}}^1({{\mathbb{Q}}})$. 

Assume that $\ell\geq 5$. 
Let $(p-1)(q^2-1)=\ell^a \times x$ and $(q-1)(p^2-1)=\ell^b \times y$, where $\ell$ does not divide $xy$. Let $D_p:=xC_p$ and $D_q:=yC_q$. Assume that the two conditions above hold. Then by Chua-Ling \cite[Lemma 1.(i)]{CL97}, we have ${{\mathcal{C}}}_{\ell}(pq) \simeq {\langle {D_p} \rangle} \oplus {\langle {D_q} \rangle}$ and it is contained in $J_0(pq)({{\mathbb{Q}}})_{{\mathrm{tor}}}$. By symmetry, we may assume that $q\not\equiv 1 {{ \!\pmod {\ell}}}$. Then, the intersection of ${{\mathcal{C}}}_{\ell}(pq)$ and $J_0(pq)_{{\mathrm{old}}}$ is isomorphic to ${\langle {\ell^n D_p} \rangle} \oplus {\langle {\ell^m D_q} \rangle}$ (cf. \cite[Theorem 2]{CL97}). Thus, $\pi(C_{\ell}(pq)) \simeq {{{{\mathbb{Z}}}/{{\ell^n}}{{\mathbb{Z}}}}} \oplus {{{{\mathbb{Z}}}/{{\ell^m}}{{\mathbb{Z}}}}}$. Since $J({{\mathbb{Q}}})_{{{\mathrm{tor}}}, ~\ell}=0$ by Theorem \ref{thm:maintheorem}, $K(pq)$ contains $\pi(C_{\ell}(pq))$.

Assume that $\ell=3$ and $3$ does not divide $(p-1)(q-1)$.
Note that the order of $C_p$ (resp. $C_q$) is the numerator of $\frac{(p-1)(q^2-1)}{3}$ (resp. $\frac{(q-1)(p^2-1)}{3}$) up to powers of $2$. Thus, ${{\mathcal{C}}}_3(pq)$ is isomorphic to ${{{{\mathbb{Z}}}/{{3^{\alpha}}}{{\mathbb{Z}}}}}\oplus {{{{\mathbb{Z}}}/{{3^{\beta}}}{{\mathbb{Z}}}}}$. Since the $3$-primary subgroups of the rational torsion subgroups of $J_0(pq)_{{\mathrm{old}}}$ and $J$ are zero, $K(pq)$ contains $\pi(C_3(pq))$, which is isomorphic to ${{{{\mathbb{Z}}}/{{3^{\alpha}}}{{\mathbb{Z}}}}}\oplus {{{{\mathbb{Z}}}/{{3^{\beta}}}{{\mathbb{Z}}}}}$.
\end{proof}

\begin{rem}
Let $p$ and $q$ be distinct primes with $p<q$ 
and let $S$ be the set of pairs $(p,~q)$ such that $g({{\mathcal{X}}}) \leq 3$.
Let $S_{\ell}$ be the subset of $S$ consisting of the pairs $(p,~q)$ satisfying the two conditions above for $\ell \geq 5$. Similarly, let $S_3$ be the subset of $S$ consisting of the pairs $(p,~q)$ such that $3$ does not divide $(p-1)(q-1)$. The following table describes the orders of $K(pq)$ and 
$$
{{\mathcal{D}}}(pq) := \bigoplus_{\substack{\ell~\text{odd primes } < q\\\text{such that }(p, ~q) \in S_{\ell}}} 
\pi({{\mathcal{C}}}_{\ell}(pq)).
$$
From its definition, we have ${{\mathcal{D}}}(pq) \subseteq \pi({{\mathcal{C}}}(pq)) \cap K(pq)$. We can see that $K(pq)/{{\mathcal{D}}}(pq)$ is a $2$-group for any $(p,~q) \in S$ from the table below.
\end{rem}

\begin{center}
Table 1.

\vspace{2mm}
\begin{tabular}{| c | c | c | c | c | c | c |}
\hline
$S$ & $g({{\mathcal{X}}})$ & $\in S_3$? &$\in S_5$?&$\in S_7$?&$\# {{\mathcal{D}}}(pq)$& $\# K(pq)$ \\ \hline
$(2,~7)$ &1 &No &Yes &Yes & 1&2 \\ \hline
$(2,~17)$ &1 &Yes &Yes &Yes &3& 3\\ \hline
$(3,~5)$ &1 &Yes &Yes &Yes &1 &1 \\ \hline
$(3,~7)$ &1 &No &Yes &Yes & 1&2 \\ \hline
$(3,~11)$ &1 &Yes &Yes &Yes &1 &1 \\ \hline
$(2,~13)$ &2 &No &Yes &Yes & 7&7 \\ \hline
$(2,~19)$ &2 &No &Yes &Yes & 5&5 \\ \hline
$(2,~29)$ &2 &Yes &Yes &Yes & 5& 5\\ \hline
$(2,~31)$ &3 &No &Yes &Yes &1 &8 \\ \hline
$(2,~41)$ &3 &Yes &Yes &Yes & 7&7 \\ \hline
$(2,~47)$ &3 &Yes &Yes &Yes & 1&4 \\ \hline
$(3,~13)$ &3 &No &Yes &Yes & 7&7 \\ \hline
$(3,~17)$ &3 &Yes &Yes &Yes & 3&3 \\ \hline
$(3,~19)$ &3 &No &Yes &Yes & 5&20 \\ \hline
$(3,~23)$ &3 &Yes &Yes &Yes & 1&8 \\ \hline
$(5,~7)$ &3 &No &Yes &Yes & 1&2 \\ \hline
$(5,~11)$ &3 &Yes &Yes &Yes & 1&1 \\ \hline
\end{tabular}
\end{center}

\bibliographystyle{annotation}

\begin{thebibliography}{99}

\bibitem{Bu97} Kevin Buzzard, \emph{Integral models of certain Shimura curves}, Duke Math. Journal, Vol \textbf{87}, no 3. (1997), 591--612.
    

\bibitem{Ce76} I.V. Cerednik, \emph{Uniformization of algebraic curves by discrete arithmetic subgroups of $\mathrm{PGL}_2(k_w)$ with compact quotients (in Russian)}, Math. Sb. \textbf{100}, 59--88 (1976). Translation in Math. USSR Sb. \textbf{29} (1976), 55--78.

\bibitem{CL97} Seng-Kiat Chua and San Ling, \emph{On the rational cuspidal subgroup and the rational torsion points of $J_0(pq)$}, Proceedings of AMS, Vol \textbf{125}, Number \textbf{8} (1997), 2255--2263.

\bibitem{DR73} Pierre Deligne and Michael Rapoport, \emph{Les sch\'emas de modules de courbes elliptiques}, Modular functions of one variable II, Lecture notes in Math., Vol. \textbf{349} (1973), 143--316.

\bibitem{Dr76} Vladimir Drinfeld, \emph{Coverings of $p$-adic symmetric regions (in Russian)}, Funkts. Anal. Prilozn \textbf{10}, 29--40 (1976). Translation in Funct. Anal. Appli. \textbf{10}, 107--115 (1976).

\bibitem{Ed91} Bas Edixhoven, \emph{L'action de l'alg\`ebre de Hecke sur les groupes de composantes des jacobiennes des courbes modulaires est ``Eisenstein"}, Courbes modulaires et courbes de Shimura (Orsay, 1987/1988), Ast\'erisque No. \textbf{196-197} (1991), 159--170.

\bibitem{GM11} Josep Goz\'alez and Santiago Molina, \emph{The kernel of Ribet’s isogeny for genus three Shimura curves}, submitted, available at \url{https://www.math.uni-bielefeld.de/sfb701/files/preprints/sfb12007.pdf} (2011). 

\bibitem{Gro72} Alexander Grothendieck, \emph{SGA 7 I. Expose IX}, Lecture Notes in Math., Vol \textbf{288} (1972), 313--523.

\bibitem{JL86} Bruce W. Jordan and Ron A. Livn\'e, \emph{On the N\'eron models of Jacobians of Shimura curves}, Compositio Math., tome \textbf{60}, no 2. (1986), 227--236.

\bibitem{M77} Barry Mazur, \emph{Modular curves and Eisenstein Ideals}, Publications Math. de l'I.H.\'E.S., tome \textbf{47} (1977), 33--186.

\bibitem{Og75} Andrew Ogg, \emph{Diophantine equations and modular forms}, Bull. A.M.S., Vol. \textbf{81} (1975), 14--27.

\bibitem{Og85} Andrew Ogg, \emph{Mauvaise r\'eduction des courbes de Shimura}, S\'eminaire de th\'eorie des nombres, Paris 1983-84 Progress in Math. \textbf{59} (1985) 199--217.

\bibitem{Ra70} Michel Raynaud, \emph{Sp\'ecialization du foncteur de Picard}, Publications Math. de l'I.H.\'E.S., tome \textbf{38} (1970), 27--76.

\bibitem{R76} Kenneth Ribet, \emph{Galois action on division points of Abelian varieties with real multiplications}, American Journal of Math., Vol. \textbf{98} (1976), 751--804.

\bibitem{R80} Kenneth Ribet, \emph{Sur les vari\'et\'es ab\'eliennes \`a multiplications r\'eelles}, C. R. Acad. Sci. Paris. t. \textbf{291}, S\'erie A--B (1980), no. 2, A121--A123.

\bibitem{R89} Kenneth Ribet, \emph{The old subvariety of $J_0(pq)$}, Arithmetic algebraic geometry (Texel, 1989), Vol. \textbf{89}, 293--307.

\bibitem{R89b} Kenneth Ribet, \emph{Bimodules and abelian surfaces}, Algebraic number theory. Adv. Stud. Pure Math., \textbf{17} (1989), 359--407.

\bibitem{R90} Kenneth Ribet, \emph{On modular representations of ${{\mathrm{Gal}}}(\overline {{\mathbb{Q}}}/{{\mathbb{Q}}})$ arising from modular forms}, Invent. Math. \textbf{100}, no. 2 (1990), 431--476.

\bibitem{R08} Kenneth Ribet, \emph{Eisenstein primes for $J_0(pq)$}, 2008 June, unpublished.

\bibitem{ST68} Jean-Pierre Serre and John Tate, \emph{Good reduction of abelian varieties}, Ann. of Math., Vol \textbf{88} (1968), 492--517.

\bibitem{Sk05} Alexei Skorobogatov, \emph{Shimura coverings of Shimura curves and the Manin obstruction}, Mathematical Research Letter, Vol \textbf{12} (2005), 779--788. 

\bibitem{Yoo14} Hwajong Yoo, \emph{The index of an Eisenstein ideal of multiplicity one}, submitted, available at \url{http://arxiv.org/pdf/1311.5275.pdf} (2014).

\bibitem{Yoo14a} Hwajong Yoo, \emph{Non-optimal levels of a reducible mod $\ell$ modular representation}, submitted, available at \url{http://arxiv.org/pdf/1409.8342.pdf} (2014).

\bibitem{Yoo15a} Hwajong Yoo, \emph{On Eisenstein ideals and the cuspidal group of $J_0(N)$}, submitted, available at \url{http://arxiv.org/pdf/1502.01571.pdf} (2015).
\end{thebibliography}

\end{document}

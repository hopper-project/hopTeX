\documentclass[leqno,11pt,twoside]{amsart}
\usepackage{geometry}

\usepackage{times}

\usepackage[all]{xy} 
\usepackage{amsmath, amssymb, amsfonts, latexsym, mdwlist, amsthm}
\usepackage{caption, subcaption}

\usepackage{url}

\usepackage[bookmarks, colorlinks, breaklinks, pdftitle={A short proof},
pdfauthor={Arend Bayer}]{hyperref}
\hypersetup{linkcolor=blue,citecolor=blue,filecolor=black,urlcolor=blue}

\usepackage{paralist}
\setdefaultenum{(a)}{(i)}{}{}
\usepackage{enumitem} 

\makeatletter
\newtheorem*{rep@theorem}{{Conjecture} \ref##}

\makeatother

\newtheorem{Thm}{Theorem}[section]
{\newenvironment{rep{Thm}}[1]{ } \begin{rep@theorem}} \end{rep@theorem}}
\newtheorem{Prop}[Thm]{Proposition}
\newtheorem{PropDef}[Thm]{Proposition and Definition}
\newtheorem{Lem}[Thm]{Lemma}
\newtheorem{PosLem}[Thm]{Positivity Lemma}
\newtheorem{Cor}[Thm]{Corollary}
{\newenvironment{rep{Cor}}[1]{ } \begin{rep@theorem}} \end{rep@theorem}}
\newtheorem{Con}[Thm]{Conjecture}
{\newenvironment{rep{Con}}[1]{ } \begin{rep@theorem}} \end{rep@theorem}}
\newtheorem{Ques}[Thm]{Question}
\newtheorem{Obs}[Thm]{Observation}
\newtheorem{Ass}[Thm]{Assumption}
\newtheorem{Note}[Thm]{Note}

\newtheorem{thm-int}{Theorem}

\theoremstyle{definition}
\newtheorem{Def-s}[Thm]{Definition}
\newtheorem{Def}[Thm]{Definition}
\newtheorem{Rem}[Thm]{Remark}
\newtheorem{DefRem}[Thm]{Definition and Remark}
\newtheorem{Prob}[Thm]{Problem}
\newtheorem{Ex}[Thm]{Example}

\usepackage{todonotes}

\begin{document}

\title[A short proof]{A short proof of the deformation property of Bridgeland stability
conditions}

\author{Arend Bayer}
\address{School of Mathematics and Maxwell Institute,
University of Edinburgh,
James Clerk Maxwell Building,
Peter Guthrie Tait Road, Edinburgh, EH9 3FD,
United Kingdom}
\email{arend.bayer@ed.ac.uk}
\urladdr{http://www.maths.ed.ac.uk/~abayer/}

\keywords{Bridgeland stability conditions, Derived category, Wall-crossing}

\begin{abstract}
The key result in the theory of Bridgeland stability conditions is the property that they form a
complex manifold. This comes from the fact that given any small deformation of the central charge, there is a
unique way to correspondingly deform the stability condition.

We give a short direct proof of a strong version of this deformation property.
\end{abstract}

\maketitle
\setcounter{tocdepth}{1}
\tableofcontents

\section{Introduction}

Stability conditions on triangulated categories,  introduced in \cite{Bridgeland:Stab}, have
been hugely influential, due to their connections to physics \cite{Tom-Ivan:quadratic,GMN:WKB}, to
mirror symmetry \cite{Bridgeland:spaces} and to representation theory
\cite{Anno-Bezrukavnikov-Mirkovic:stability}, and due to their applications within algebraic
geometry, for example
to Donaldson-Thomas invariants \cite{Yukinobu:DTsurvey},
to the derived category itself \cite{Daniel:intro-stability, K3Pic1}, or to the birational geometry of moduli spaces
\cite{ABCH:MMP, BM:walls, wallcrossing-BrillNoether, Emolo-Benjamin:lecture-notes}.

Their distinguishing property, crucial for  all applications, is a strong deformation
property: by the main result of \cite{Bridgeland:Stab}, there is a complex manifold of stability
conditions, with map to a vector space that is a locally an isomorphism. We
give a short proof of an effective version of this result.

\subsection*{Result}
We refer to Section \ref{sect:review} for the complete definitions; here we briefly
review notation and the support property. Let ${\ensuremath{\mathcal D}}$ be a triangulated category, and let
$v \colon K({\ensuremath{\mathcal D}}) \to \Lambda$ be a homomorphism from its K-group to a finitely generated free
abelian group $\Lambda$. A pre-stability condition on ${\ensuremath{\mathcal D}}$ with respect to $v$ is a pair
$\sigma = (Z, {\ensuremath{\mathcal P}})$, where
${\ensuremath{\mathcal P}}$ is a \emph{slicing} (see Definition \ref{def:slicing}) and 
$Z \colon \Lambda \to {\ensuremath{\mathbb{C}}}$ is a compatible (see Definition \ref{def:prestability}) group homomorphism.

\begin{Def}[{\cite{Bridgeland:Stab}, \cite{Kontsevich-Soibelman:stability}}]
\label{def:supportproperty}
Let $Q \colon \Lambda_{\ensuremath{\mathbb{R}}} \to {\ensuremath{\mathbb{R}}}$ be a quadratic form. We say that a pre-stability condition $(Z,
{\ensuremath{\mathcal P}})$ satisfies the support property with respect to $Q$ if 
\begin{enumerate*}
\item \label{item:QKerneg} 
the kernel ${\mathop{\mathrm{Ker}}\nolimits} Z \subset \Lambda_{\ensuremath{\mathbb{R}}}$ of the central charge is negative definite with respect to $Q$, and
\item \label{item:QEnonneg}
for any semistable object $E$, i.e. $E \in {\ensuremath{\mathcal P}}(\phi)$ for some $\phi \in {\ensuremath{\mathbb{R}}}$, we have
$Q(v(E)) \ge 0$.
\end{enumerate*}
\end{Def}
If such $Q$ exists, we call $\sigma$ a stability condition. Let ${\mathop{\mathrm{Stab}}\nolimits}_\Lambda({\ensuremath{\mathcal D}})$ denote the
topological space (see Section \ref{sect:review}) of stability conditions. It comes with a canonical
map ${\ensuremath{\mathcal Z}} \colon {\mathop{\mathrm{Stab}}\nolimits}_\Lambda({\ensuremath{\mathcal D}}) \to {\mathop{\mathrm{Hom}}\nolimits}(\Lambda, {\ensuremath{\mathbb{C}}})$ given by
${\ensuremath{\mathcal Z}}(Z, {\ensuremath{\mathcal P}}) = Z$. 
We will prove:

\begin{Thm} \label{thm:mainthm}
Let $Q$ be a quadratic form on $\Lambda \otimes {\ensuremath{\mathbb{R}}}$, and assume that the stability condition
$\sigma = (Z, {\ensuremath{\mathcal P}})$ satisfies the support property with respect to $Q$. Then:
\begin{enumerate*}
\item \label{item:deformeffective}
There is an open neighboorhood $\sigma \in U_\sigma \subset {\mathop{\mathrm{Stab}}\nolimits}_\Lambda({\ensuremath{\mathcal D}})$ such that the
restriction ${\ensuremath{\mathcal Z}} \colon U_\sigma \to {\mathop{\mathrm{Hom}}\nolimits}(\Lambda, {\ensuremath{\mathbb{C}}})$ is a covering of the set of 
$Z'$ such that $Q$ is negative definite on ${\mathop{\mathrm{Ker}}\nolimits} Z'$.
\item \label{item:Qremains}
All $\sigma \in U_\sigma$ satisfy the support property with respect to ${\ensuremath{\mathcal Q}}$.
\end{enumerate*}
\end{Thm}
In other words, ${\mathop{\mathrm{Stab}}\nolimits}_\Lambda({\ensuremath{\mathcal D}})$ is a manifold, and any path
$Z_t \in {\mathop{\mathrm{Hom}}\nolimits}(\Lambda, {\ensuremath{\mathbb{C}}})$ for $t \in [0,1]$ with $Z_0 = Z$ and
${\mathop{\mathrm{Ker}}\nolimits} Z_t$ negative definite for all $t \in [0,1]$ lifts uniquely to a continuous path
$\sigma_t = (Z_t, {\ensuremath{\mathcal P}}_t)$ in the space of stability conditions starting at $\sigma_0 = \sigma$.

Part \eqref{item:deformeffective} is an effective variant of \cite[Theorem 1.2]{Bridgeland:Stab}
(which says that there is \emph{some} neighbourhood of $Z_0$ in which paths can be lifted uniquely). The
entire result first appeared as \cite[Proposition A.5]{BMS:stabCY3s} with an indirect proof based on reduction to Bridgeland's previous result.

\subsection*{Remarks} 
The support property can be a deep and interesting property in itself:
a quadratic Bogomolov-Gieseker type inequality for Chern classes of semi-stable objects which, by
Theorem \ref{thm:mainthm}, is preserved under wall-crossing.

Theorem \ref{thm:mainthm} was crucial in \cite{BMS:stabCY3s} in order to describe an entire
component of the space of stability conditions on abelian threefolds, and on some Calabi-Yau
threefolds. It also greatly simplifies the
construction of stability conditions on surfaces (or of \emph{tilt-stability} on higher-dimensional
varieties \cite{BMT:3folds-BG}). In this case, the quadratic form
$Q$ is the classical Bogomolov-Gieseker inequality, and 
Theorem \ref{thm:mainthm} gives an open subset of stability conditions that
otherwise has to be glued together from many small pieces (see e.g.~\cite[Section 4]{localP2}).

Theorem 1.2 of \cite{Bridgeland:Stab} also allows for components of the space of stability
conditions modelled on a linear subspace $L \subset {\mathop{\mathrm{Hom}}\nolimits}(\Lambda, {\ensuremath{\mathbb{C}}})$. When $L$ is defined over ${\ensuremath{\mathbb{Q}}}$, we can recover
that statement by replacing $\Lambda$ with $\Lambda/{\mathop{\mathrm{Ker}}\nolimits} L$.
(See \cite{Sven-Holger:quotcategories} for examples where this is not satisfied; however, 
to achieve well-behaved wall-crossing one always has to assume that $L$ is defined over ${\ensuremath{\mathbb{Q}}}$.)

\subsubsection*{Proof idea}
Our proof is based on two ideas. First, we reduce to the case where the imaginary part of $Z$
is constant; then we only have to deform stability in a fixed abelian
category. Secondly, we use the elementary convex geometry of the \emph{Harder-Narasimhan
polygon}, see Section \ref{sect:HNpolygon}.

This avoids the need for \emph{quasi-abelian categories}, of $\epsilon$ or of $\frac 18$. It also avoids some of the 
more technical arguments of \cite[Section 7]{Bridgeland:Stab}. We still need a few arguments
similar to ones in \cite{Bridgeland:Stab}; we have reproduced most of them,
except for the proofs of Proposition \ref{prop:stabviaheart} and Lemma
\ref{lem:locallyinjective}. 

\subsection*{Application}
Assume that ${\ensuremath{\mathcal D}}$ is a 2-Calabi-Yau category, i.e. for all $E, F \in {\ensuremath{\mathcal D}}$ we have a bi-functorial
isomorphism
${\mathop{\mathrm{Hom}}\nolimits}(E, F) = {\mathop{\mathrm{Hom}}\nolimits}(F, E[2])^\vee$. Let $\Lambda$ be the \emph{numerical
$K$-group} of ${\ensuremath{\mathcal D}}$, and assume that $\Lambda$ is finitely generated. Then
there is a surjection $v \colon K({\ensuremath{\mathcal D}}) \to \Lambda$, and $\Lambda$ admits a non-degenerate bilinear form $({\underline{\hphantom{A}}}, {\underline{\hphantom{A}}})$, called Mukai-pairing, with
\[ \chi(E, F) = -\bigl(v(E), v(F) \bigr). \]

Let ${\ensuremath{\mathcal P}}_0({\ensuremath{\mathcal D}}) \subset {\mathop{\mathrm{Hom}}\nolimits}(\Lambda, {\ensuremath{\mathbb{C}}})$ be the set of central charges $Z$ such that
${\mathop{\mathrm{Ker}}\nolimits} Z$ is negative definite with respect to the Mukai pairing, and such that ${\mathop{\mathrm{Ker}}\nolimits} Z$ contains
no root $\delta \in \Lambda, (\delta, \delta) = -2$. 

\begin{Cor}
\label{cor:P0covering}
The restriction
$ {\ensuremath{\mathcal Z}}^{-1}\left({\ensuremath{\mathcal P}}_0({\ensuremath{\mathcal D}})\right) \xrightarrow{\ensuremath{\mathcal Z}} {\ensuremath{\mathcal P}}_0({\ensuremath{\mathcal D}}) $
is a covering map.
\end{Cor}
The proof, given in Section \ref{sect:application}, is fairly similar to the case of
K3 surfaces \cite[Proposition 8.3]{Bridgeland:K3}.
The point of including it here is
to show that in terms of the support property via quadratic forms, and equipped with Theorem
\ref{thm:mainthm}, the proof becomes natural and short.
This result was also proved previously for preprojective algebras of
quivers in \cite{Thomas:stability, Bridgeland:ADE, Ikeda:stability-preprojective}. In each of these
cases, there is in fact a connected component of ${\mathop{\mathrm{Stab}}\nolimits}({\ensuremath{\mathcal D}})$ that is a covering of a connected of
${\ensuremath{\mathcal P}}_0(X)$; such deeper statements rely crucially on \emph{non-emptiness} of moduli spaces
of stable objects.

\subsection*{Acknowledgements}
I would like to thank Emanuele Macr{\`{\i}} and Paolo Stellari; as indicated above, Theorem
\ref{thm:mainthm} first appeared with a different proof in our joint work \cite{BMS:stabCY3s}. 
I presented a clumsier version of the arguments in this article at my lectures
at the Hausdorff school  on derived categories in Bonn, April 2016;
I am grateful to the organisers for the opportunity, and the participants for their
feedback.
My work was supported by the ERC starting grant WallXBirGeom 337039.

\section{Review: definitions and basic properties}
\label{sect:review}

Throughout, ${\ensuremath{\mathcal D}}$ will be a triangulated category, equipped with a group homomorphism
\[ v \colon K({\ensuremath{\mathcal D}}) \to \Lambda \]
from its $K$-group to an abelian group $\Lambda \cong {\ensuremath{\mathbb{Z}}}^m$.

\subsection*{Definitions}
We first recall the main definitions from \cite{Bridgeland:Stab}.

\begin{Def} \label{def:slicing}
A \emph{slicing} ${\ensuremath{\mathcal P}}$ on ${\ensuremath{\mathcal D}}$ is a collection of full subcategories ${\ensuremath{\mathcal P}}(\phi)$ for all $\phi \in
{\ensuremath{\mathbb{R}}}$ with
\begin{enumerate}
\item ${\ensuremath{\mathcal P}}(\phi+1) = {\ensuremath{\mathcal P}}(\phi)[1]$ for all $\phi \in {\ensuremath{\mathbb{R}}}$;
\item for $\phi_1 > \phi_2$ and $E_i \in {\ensuremath{\mathcal P}}(\phi_i),  i = 1,2$, we have ${\mathop{\mathrm{Hom}}\nolimits}(E_1, E_2) = 0$; and
\item \label{item:HN} 
for any $E \in {\ensuremath{\mathcal D}}$ there is a sequence of maps
$ 0 = E_0 \xrightarrow{i_1} E_1 \to \dots \xrightarrow{i_m} E_m$
and of real numbers $\phi_1 > \phi_2 > \dots > \phi_m$
such that the cone of $i_j$ is in ${\ensuremath{\mathcal P}}(\phi_j)$
for $j = 1, \dots, m$.
\end{enumerate}
\end{Def}
The objects of ${\ensuremath{\mathcal P}}(\phi)$ are called \emph{semistable of phase $\phi$}; its 
simple objects are called \emph{stable}. The sequence of maps in \eqref{item:HN} is called the HN
filtration of $E$.

\begin{Def} \label{def:prestability}
A pre-stability condition on ${\ensuremath{\mathcal D}}$ is a pair $\sigma = (Z, {\ensuremath{\mathcal P}})$ where ${\ensuremath{\mathcal P}}$ is a slicing, and
$Z \colon \Lambda \to {\ensuremath{\mathbb{C}}}$ is a group homomorphism, that satisfy the following condition:
for all $0 \neq E \in {\ensuremath{\mathcal P}}(\phi)$, we have $Z(v(E)) \in {\ensuremath{\mathbb{R}}}_{>0}\cdot e^{i \pi \phi}$.
\end{Def}
We will abuse notation and write $Z(E)$ instead of $Z(v(E))$.

\subsection*{Basic properties}
Let ${\mathop{\mathrm{GL}}\nolimits}_2^+({\ensuremath{\mathbb{R}}})$ denote the group of real $2 \times 2$-matrices with positive determinant, and
let ${\widetilde{{\mathop{\mathrm{GL}}\nolimits}_2^+({\ensuremath{\mathbb{R}}})}}$ be its universal cover. Since ${\mathop{\mathrm{GL}}\nolimits}_2^+({\ensuremath{\mathbb{R}}})$ acts on $S^1$, its universal
cover  acts on the universal cover ${\ensuremath{\mathbb{R}}} \to S^1$ given explicitly by $\phi \mapsto e^{i\pi\phi}$.
For $\tilde g \in {\widetilde{{\mathop{\mathrm{GL}}\nolimits}_2^+({\ensuremath{\mathbb{R}}})}}$ we will write $g$  for the corresponding element of ${\mathop{\mathrm{GL}}\nolimits}_2^+({\ensuremath{\mathbb{R}}})$, and
$\tilde g.\phi$ for the given action on ${\ensuremath{\mathbb{R}}}$.

\begin{Prop} There is a natural action of ${\widetilde{{\mathop{\mathrm{GL}}\nolimits}_2^+({\ensuremath{\mathbb{R}}})}}$ on the set of pre-stability conditions given by
$\tilde g.(Z, {\ensuremath{\mathcal P}}) = (Z', {\ensuremath{\mathcal P}}')$ with
\[ 
Z' = g \circ Z \quad \text{and} \quad
{\ensuremath{\mathcal P}}'(\tilde g.\phi) = {\ensuremath{\mathcal P}}(\phi). \]
\end{Prop}

The \emph{heart of a bounded t-structure} is a full subcategory ${\ensuremath{\mathcal A}} \subset {\ensuremath{\mathcal D}}$ such that
\[ 
{\ensuremath{\mathcal S}}(\phi) := \begin{cases} {\ensuremath{\mathcal A}}[\phi] & \text{if $\phi \in {\ensuremath{\mathbb{Z}}}$} \\
			\emptyset & \text{if $\phi \notin {\ensuremath{\mathbb{Z}}}$} \end{cases}
\]
is a slicing (see \cite[Lemma 3.2]{Bridgeland:Stab}). It is automatically an abelian subcategory; 
and stability conditions on ${\ensuremath{\mathcal D}}$ can be constructed from slope-stability in ${\ensuremath{\mathcal A}}$.

\begin{Def}
A stability function $Z$ on an abelian category ${\ensuremath{\mathcal A}}$ is a morphism
$Z \colon K({\ensuremath{\mathcal A}}) \to {\ensuremath{\mathbb{C}}}$ of abelian groups such that for all $0 \neq E \in {\ensuremath{\mathcal A}}$, the complex number 
$Z(E)$ is in the semi-closed upper half plane
\[  {\ensuremath{\mathbb{H}}}:= {\left\{{z \in {\ensuremath{\mathbb{C}}}}\,\colon\,{\Im Z > 0,  \quad \text{or} \ \Im Z = 0 \ \text{and} \ \Re Z < 0}\right\}}.
\]
\end{Def}

For $ 0 \neq E \in {\ensuremath{\mathcal A}}$ we define its phase by $\phi(E) := \frac 1\pi \arg Z(E) \in (0,1]$.
An object $E \in {\ensuremath{\mathcal A}}$ is called $Z$-semistable if for all
subobjects $A {\ensuremath{\hookrightarrow}} E$, we have $\phi(A) \le \phi(E)$. 

\begin{Def} We say that a stability function $Z$ on an abelian category ${\ensuremath{\mathcal A}}$ satisfies the
\emph{HN property} if every object $E \in {\ensuremath{\mathcal A}}$ admits a Harder-Narasimhan (HN) filtration: a
sequence $0 = E_0 {\ensuremath{\hookrightarrow}} E_1 {\ensuremath{\hookrightarrow}} E_2 {\ensuremath{\hookrightarrow}} \dots {\ensuremath{\hookrightarrow}} E_m = E$ such that
$E_i/E_{i-1}$ is $Z$-semistable for $i = 1, \dots, m$, with
\[ \phi\left(E_1/E_0\right) > \phi\left(E_2/E_1\right) > \dots > \phi\left(E_m/E_{m-1}\right). \]
\end{Def}

\begin{Prop}[{\cite[Proposition 5.3]{Bridgeland:Stab}}] \label{prop:stabviaheart}
To give a pre-stability condition on ${\ensuremath{\mathcal D}}$ is equivalent to giving a heart ${\ensuremath{\mathcal A}}$ of a bounded
t-structure, and a stability function $Z$ on ${\ensuremath{\mathcal A}}$ with the HN property.
\end{Prop}
Here we tacitly assume that the stability function $Z$ on ${\ensuremath{\mathcal A}}$ also factors via
$K({\ensuremath{\mathcal A}}) = K({\ensuremath{\mathcal D}}) \xrightarrow{v} \Lambda$. Given $(Z, {\ensuremath{\mathcal A}})$, the slicing is determined by
setting ${\ensuremath{\mathcal P}}(\phi)$ to be the $Z$-semistable objects $E \in {\ensuremath{\mathcal A}}$ of phase $\phi$ for $\phi \in (0,
1]$. Conversely,
given  $(Z, {\ensuremath{\mathcal P}})$, the heart ${\ensuremath{\mathcal A}}$ is the smallest extension-closed subcategory of ${\ensuremath{\mathcal D}}$ containing
${\ensuremath{\mathcal P}}(\phi)$ for $\phi \in (0, 1]$.

\begin{Def} A stability condition $\sigma$ is a pre-stability condition that satisfies the support
property in the sense of Definition \ref{def:supportproperty} with respect to some quadratic
form $Q$ on $\Lambda \otimes {\ensuremath{\mathbb{R}}}$.
\end{Def}

\subsection*{Topology and local injectivity} 
There is a generalised metric, and thus a topology, on the set of slicings ${\mathop{\mathrm{Slice}}\nolimits}({\ensuremath{\mathcal D}})$ given as follows. Given two slicings ${\ensuremath{\mathcal P}}, {\ensuremath{\mathcal Q}}$, we write
$\phi^{\pm}(E)$ and $\psi^{\pm}(E)$ for the largest and smallest phase in the associated HN
filtration of an object $E$ for ${\ensuremath{\mathcal P}}$ and ${\ensuremath{\mathcal Q}}$, respectively. Then we define the distance of ${\ensuremath{\mathcal P}}$ and ${\ensuremath{\mathcal Q}}$ by
\[
d({\ensuremath{\mathcal P}}, {\ensuremath{\mathcal Q}}) := \sup {\left\{{ {\left\lvert{\phi^+(E) - \psi^+(E)}\right\rvert}, {\left\lvert{\phi^-(E) - \psi^-(E)}\right\rvert}}\,\colon\,{E \in {\ensuremath{\mathcal D}}}\right\}} \in [0, +\infty].
\]

We recall that this distance can be computed by considering ${\ensuremath{\mathcal P}}$-semistable objects alone:
\begin{Lem}[{\cite[Lemma 6.1]{Bridgeland:Stab}}] \label{lem:distviasemistables}
We have $d({\ensuremath{\mathcal P}}, {\ensuremath{\mathcal Q}}) = d'({\ensuremath{\mathcal P}}, {\ensuremath{\mathcal Q}})$, where the latter is defined by
\[ d'({\ensuremath{\mathcal P}}, {\ensuremath{\mathcal Q}}) := \sup {\left\{{\psi^+(E) - \phi, \phi - \psi^-(E)}\,\colon\,{\phi \in {\ensuremath{\mathbb{R}}}, 0 \neq E \in {\ensuremath{\mathcal P}}(\phi)}\right\}}.\]
\end{Lem}
\begin{proof}
The inequality $d({\ensuremath{\mathcal P}}, {\ensuremath{\mathcal Q}}) \ge d'({\ensuremath{\mathcal P}}, {\ensuremath{\mathcal Q}})$ is immediate. For the converse,
consider $E \in {\ensuremath{\mathcal D}}$, and let $A_i$ be one of its HN factors with respect to ${\ensuremath{\mathcal P}}$.
Then $\psi^+(A_i) \le \phi(A_i) + d'({\ensuremath{\mathcal P}}, {\ensuremath{\mathcal Q}}) \le \phi^+(E) + d'({\ensuremath{\mathcal P}}, {\ensuremath{\mathcal Q}})$. 
Hence $E$ admits no maps from ${\ensuremath{\mathcal Q}}$-stable objects
of phase bigger than $\phi^+(E) + d'({\ensuremath{\mathcal P}}, {\ensuremath{\mathcal Q}})$, and so $\psi^+(E) \le \phi^+(E) + d'({\ensuremath{\mathcal P}}, {\ensuremath{\mathcal Q}})$. 
The analogous inequality for $\psi^-(E)$ follows similarly; combined, they imply the claim.
\end{proof}

The topology on ${\mathop{\mathrm{Stab}}\nolimits}_{\Lambda}({\ensuremath{\mathcal D}})$ (and similarly on the set of pre-stability conditions)
is the finest topology such that both maps
\begin{align*}
{\mathop{\mathrm{Stab}}\nolimits}_{\Lambda}({\ensuremath{\mathcal D}}) & \to {\mathop{\mathrm{Slice}}\nolimits}({\ensuremath{\mathcal D}}),  \quad (Z, {\ensuremath{\mathcal P}}) \mapsto {\ensuremath{\mathcal P}} \\
{\mathop{\mathrm{Stab}}\nolimits}_{\Lambda}({\ensuremath{\mathcal D}}) & \to {\mathop{\mathrm{Hom}}\nolimits}(\Lambda, {\ensuremath{\mathbb{C}}}),  \quad (Z, {\ensuremath{\mathcal P}}) \mapsto Z 
\end{align*}
are continuous.

\begin{Lem}[{\cite[Lemma 6.4]{Bridgeland:Stab}}] \label{lem:locallyinjective}
If $\sigma = (Z, {\ensuremath{\mathcal P}})$ and $\tau = (Z, {\ensuremath{\mathcal Q}})$ are two pre-stability conditions with the same central charge
$Z$ and $d({\ensuremath{\mathcal P}}, {\ensuremath{\mathcal Q}}) < 1$,  then $\sigma = \tau$.
\end{Lem}
\begin{Cor} \label{cor:locallyinjective}
The  map ${\mathop{\mathrm{Stab}}\nolimits}_{\Lambda}({\ensuremath{\mathcal D}}) \to {\mathop{\mathrm{Hom}}\nolimits}(\Lambda, {\ensuremath{\mathbb{C}}}), (Z, {\ensuremath{\mathcal P}}) \mapsto Z$ is locally injective.
\end{Cor}

\section{Harder-Narasimhan filtrations via the Harder-Narasimhan polygon}\label{sect:HNpolygon}

Throughout this section, let ${\ensuremath{\mathcal A}}$ be an abelian category with a stability function $Z$.

\begin{Def}
The \emph{Harder-Narasimhan polygon} ${\mathop{\mathrm{HN}}\nolimits}^Z(E)$ of an object $E \in {\ensuremath{\mathcal A}}$ is the convex
hull of the central charges $Z(A)$ of all subobjects $A \subset E$ of $E$.
\end{Def}
(The trivial subobjects $A = 0$ or $A = E$ are included in the definition.) The idea to consider
this convex set in the context of slope-stability goes back at least 40 years
\cite{Shatz:Degeneration}.

\begin{Def} We say that the Harder-Narasimhan polygon ${\mathop{\mathrm{HN}}\nolimits}^Z(E)$ of an object $E \in {\ensuremath{\mathcal A}}$
is \emph{polyhedral on the left} if the set has finitely many extremal points
$0 = z_0, z_1, \dots, z_m = Z(E)$ such that ${\mathop{\mathrm{HN}}\nolimits}^Z(E)$ lies to the right of the 
path $z_0z_1z_2\dots z_m$;
see fig.~\ref{fig:HNpoly}.
\end{Def}

\begin{figure}[htb]
  \begin{center}
        \includegraphics{HN-poly}
    \caption{Polyhedral on the left}
    \label{fig:HNpoly}
  \end{center}
\end{figure}

In other words, the intersection of ${\mathop{\mathrm{HN}}\nolimits}^Z(E)$ with the closed half-plane to the
left of the line through $0$ and $Z(E)$ is the polygon with vertices $z_0, z_1, \dots, z_m$.
Our proof of Theorem \ref{thm:mainthm} is based on the following well-known statement; we provide a
proof for completeness:

\begin{Prop} \label{prop:HNviapoly}
The object $E$ has a Harder-Narasimhan filtration with respect to $Z$ if and only if its
Harder-Narasimhan polygon ${\mathop{\mathrm{HN}}\nolimits}^Z(E)$ is polyhedral on the left.
\end{Prop}

Assume that ${\mathop{\mathrm{HN}}\nolimits}^Z(E)$ is polyhedral on the left. For each $i = 1, \dots, m$, choose a subobject $E_i \subset E$ such that $Z(E_i) = z_i$. (This exists as $z_i$ is extremal.)
\begin{Lem}  \label{lem:filtration}
This is a filtration, i.e. $E_{i-1} \subset E_{i}$ for $i = 1, \dots, m$. 
\end{Lem}
\begin{proof}
Let $A := E_{i-1} \cap E_i \subset E$ be the intersection of two subsequent objects, and
$B := E_{i-1} + E_i \subset E$ be their span inside $E$. Then there is a short exact sequence
\[ A {\ensuremath{\hookrightarrow}} E_{i-1} \oplus E_i {\ensuremath{\twoheadrightarrow}} B. \]
Hence the midpoint of $Z(A)$ and $Z(B)$ is also the midpoint of $z_{i-1}$ and $z_i$,
see also figure \ref{fig:HNproofdetail}.

\begin{figure}[!htb]
\centering
\begin{minipage}{0.45\textwidth}
	\centering
        \includegraphics{HN-detail}
    \captionof{figure}{Lemma \ref{lem:filtration}}
    \label{fig:HNproofdetail}
\end{minipage}
\begin{minipage}{0.45\textwidth}
	\centering
        \includegraphics{HN-detail2}
    \captionof{figure}{Lemma \ref{lem:quotsemistable}}
    \label{fig:HNproofdetail2}
\end{minipage}
\end{figure}

On the other hand, $Z(A), Z(B)$ lie in ${\mathop{\mathrm{HN}}\nolimits}^Z(E)$; by convexity and the choice of $z_{i-1}, z_i$,
they both have to lie either in the open half-plane to the right of the line
$\left(z_{i-1}z_i\right)$, or on the line segment $\overline{z_{i-1}z_i}$. The former would be a
contradiction to the previous paragraph, and so $Z(A) \in \overline{z_{i-1}z_i}$. 

Since $A \subset E_{i-1}$, this implies $Z(A) = z_{i-1}$ and $A \cong E_{i-1}$; therefore, 
$E_{i-1} \subset E_i$.
\end{proof}

\begin{Lem} \label{lem:quotsemistable}
The filtration quotient $E_i/E_{i-1}$ is semistable. 
\end{Lem}
\begin{proof}
Otherwise, there is an object $A$ with $E_{i-1} \subset A \subset E_i$ such that
$A/E_{i-1}$ has bigger phase than $E_i/E_{i-1}$, see fig.~\ref{fig:HNproofdetail2}. It follows that $Z(A)$ lies to the left of the line
segment $\overline{z_{i-1}z_i}$. Since $A \subset E$ and hence $Z(A) \in {\mathop{\mathrm{HN}}\nolimits}^Z(E)$, this is a
contradiction.
\end{proof}

\begin{proof}[Proof of Propostion \ref{prop:HNviapoly}]
The phase of $E_i/E_{i-1}$ is determined by the argument of $z_i - z_{i-1}$;
by convexity this shows $\phi(E_1/E_0) > \dots > \phi(E_m/E_{m-1})$, and so the
$E_i$ form a HN filtration.

Conversely, assume that we are given a HN filtration $0 = E_0 {\ensuremath{\hookrightarrow}} E_1 {\ensuremath{\hookrightarrow}} \dots {\ensuremath{\hookrightarrow}} E_m$
and a subobject $A {\ensuremath{\hookrightarrow}} E$.
We have to show that $Z(A)$ lies to the right of the 
path $z_0z_1 \dots z_m$ with vertices $z_i := Z(E_i)$. By induction on $m$, we may 
assume that $Z(A \cap E_{m-1})$ lies to the right of the path $z_0z_1 \dots z_{m-1}$. On the other hand,
$A/\left(A \cap E_{m-1}\right)$ is a subobject of $E_m/E_{m-1}$, which is semistable; thus the central charge
of $Z(A/\left(A \cap E_{m-1}\right)$ lies to the right of the line segment from $0$ to $z_m - z_{m-1}$. 
Therefore, $Z(A) = Z(A \cap E_{m-1}) + Z(A/\left(A \cap E_{m-1}\right)$ lies to the right of the path
$z_0z_1\dots z_m$ as claimed.
\end{proof}

\begin{Cor} \label{cor:HNfromfinite}
Given $E \in {\ensuremath{\mathcal A}}$, assume that there are only finitely many classes
$v(A)$ of subobjects $A \subset E$ with $\Re Z(A) < \max \left\{0, \Re Z(E) \right\}$. Then $E$ admits a HN filtration.
\end{Cor}

\section{Proof of the deformation property}
\label{sect:proof}

Throughout Section \ref{sect:proof} and \ref{sect:Qpreserved}, we will make the following assumption:
\begin{Ass} \label{ass:nondeg}
The quadratic form $Q$ has signature $(2, {\mathop{\mathrm{rk}}} \Lambda - 2)$. 
\end{Ass}

\begin{Lem} \label{lem:coords}
Up to the action of ${\widetilde{{\mathop{\mathrm{GL}}\nolimits}_2^+({\ensuremath{\mathbb{R}}})}}$ on ${\mathop{\mathrm{Stab}}\nolimits}_\Lambda({\ensuremath{\mathcal D}})$, we may assume that we are in
the following situation.
There is a norm $\left\|\cdot\right\|$ on ${\mathop{\mathrm{Ker}}\nolimits} Z$ such that if $p \colon \Lambda_{\ensuremath{\mathbb{R}}} \to {\mathop{\mathrm{Ker}}\nolimits} Z$ denotes the orthogonal
projection with respect to $Q$, then
\[ Q(v) = {\left\lvert{Z(v)}\right\rvert}^2 - {\left\|{p(v)}\right\|}^2.\]
\end{Lem}
\begin{proof}
Let $K$ the kernel of $Z$, and let $K^\perp$ denote its orthogonal complement. Then $Q$ is negative
definite on $K$; let ${\left\|{\cdot}\right\|}$ be the norm associated to $-Q$. 
As $Z|_{K^\perp}$ is injective, Assumption \ref{ass:nondeg} can only hold
if $Q$ is positive definite on $K^\perp$, and  if we have an isomorphism of real vector spaces
\[ Z|_{K^\perp} \colon K^\perp \to {\ensuremath{\mathbb{C}}}. \]
Using the ${\mathop{\mathrm{GL}}\nolimits}_2^+({\ensuremath{\mathbb{R}}})$-action, we may assume this to be an
isometry. Then the the claim follows.
\end{proof}

\begin{Rem} In different context, namely for the Mukai quadratic form instead of $Q$, the analogous normalisation is used extensively in \cite{Bridgeland:K3}.
\end{Rem}

Consider the subset in ${\mathop{\mathrm{Hom}}\nolimits}(\Lambda, {\ensuremath{\mathbb{C}}})$ of central charges whose kernel is
negative definite with respect to $Q$; let ${\ensuremath{\mathcal P}}_Z(Q)$ be its connected component containing $Z$.
\begin{Lem} \label{lem:deformincoords}
Assume we are in the situation of Lemma \ref{lem:coords}.
Up to the action of ${\mathop{\mathrm{GL}}\nolimits}_2^+({\ensuremath{\mathbb{R}}})$, each $Z' \in {\ensuremath{\mathcal P}}_Z(Q)$ is of the form 
\[ Z' = Z + u \circ p \]
where $u \colon {\mathop{\mathrm{Ker}}\nolimits} Z \to {\ensuremath{\mathbb{C}}}$ is a linear map with operator norm satisfying ${\left\|{u}\right\|} < 1$.
\end{Lem}
\begin{proof}
As in the previous Lemma, let $K^\perp$ be the orthogonal complement of ${\mathop{\mathrm{Ker}}\nolimits} Z$. 
The restriction of $Z'$ to $K^\perp$ is an isomorphism for any $Z' \in {\ensuremath{\mathcal P}}_Z(Q)$. Hence for any path $Z(t)$
in ${\ensuremath{\mathcal P}}_Z(Q)$ starting at $Z$ there is a corresponding path $\gamma(t) \in {\mathop{\mathrm{GL}}\nolimits}_2^+({\ensuremath{\mathbb{R}}})$ 
such that $\gamma(t) \circ Z(t)$ is constant. So we may assume that
$Z'$ and $Z$ agree when restricted to $K^\perp$. Let $u$ be the restriction of $Z'$ to
${\mathop{\mathrm{Ker}}\nolimits} Z$, and the claim follows.
\end{proof}

\begin{Lem} \label{lem:onlyreal}
In order to prove Theorem \ref{thm:mainthm}, it is enough to show
the following: given any stability conditions $\sigma_0 = (Z_0, {\ensuremath{\mathcal P}}_0)$, and any path
of central charges of the form  $t \mapsto Z_t = Z + t \cdot u \circ p$ for $t \in [0,1]$, where $u \colon
{\mathop{\mathrm{Ker}}\nolimits} Z \to {\ensuremath{\mathbb{R}}}$ is a linear map to the \emph{real numbers} with ${\left\|{u}\right\|} < 1$, there exists a
continuous lift $t \mapsto \sigma_t$ to the space of stability conditions; moreover, all $\sigma_t$ satisfy the support
property with respect to the same quadratic form $Q$.
\end{Lem}
\begin{proof}
Due to Corollary \ref{cor:locallyinjective}, it is enough to prove the \emph{existence} of a lift for any given path, and moreover we can freely replace any path in ${\ensuremath{\mathcal P}}_Z(Q)$ by a homotopic one. 
Observe that due the $\widetilde{{\mathop{\mathrm{GL}}\nolimits}_2^+}({\ensuremath{\mathbb{R}}})$-action such a result would equally hold when $u$ is purely
imaginary.  Now write $Z_1 = Z + u \circ p$ and $u = \Re u + i \Im u$. Since ${\left\|{\Re u}\right\|} \le {\left\|{u}\right\|}$, we first obtain
a path from $\sigma_0 = \sigma$ to a stability condition $\sigma_1 = (Z_1, {\ensuremath{\mathcal P}}_1)$ with
$Z_1 = Z + \Re u \circ p$. By part \eqref{item:Qremains} of Theorem \ref{thm:mainthm}, we can apply the result again starting at $\sigma_1$ to construct the desired
stability condition with central charge $Z + u\circ p = Z_1 + i \Im u \circ p$.
\end{proof}

Our next key observation is that when 
$u$ is real, we may (and in fact, have to) leave the heart ${\ensuremath{\mathcal A}} := {\ensuremath{\mathcal P}}(0, 1]$ unchanged. Hence we
will apply Proposition \ref{prop:stabviaheart} and prove
that $({\ensuremath{\mathcal A}}, Z_t)$ produces a stability condition for all $t \in [0, 1]$.
Clearly we just need to prove the case $t = 1$. 
\begin{Lem} Let $Z, u$ be as in Lemma \ref{lem:onlyreal}. Then $Z_1 = Z + u \circ p$ is a stability
function on ${\ensuremath{\mathcal A}}$.
\end{Lem}
\begin{proof} Consider $E \in {\ensuremath{\mathcal A}}$; if $\Im Z(E) = \Im Z_1(E) > 0$, there is nothing to prove.
Otherwise, $E$ is semistable with $Z(E) \in {\ensuremath{\mathbb{R}}}_{<0}$ and thus ${\left\|{p(E)}\right\|} \le - Z(E)$.  From
 ${\left\|{u}\right\|} < 1$ we conclude
\[
Z_1(E) = Z(E) + u\circ p(E) \le Z(E) + {\left\|{u}\right\|} {\left\|{p(E)}\right\|}  < Z(E) - Z(E) = 0. 
\]
\end{proof}

Next, we want to prove that $({\ensuremath{\mathcal A}}, Z_1)$ satisfies the HN property. We will use
Proposition \ref{prop:HNviapoly} and Corollary \ref{cor:HNfromfinite}. 

Let us define the \emph{mass} $m^Z(E)$ of $E$ with respect to $Z$ as the length of the boundary
of ${\mathop{\mathrm{HN}}\nolimits}^Z(E)$ on the left between $0$ and $Z(E)$.

\begin{Lem} \label{lem:boundZ}
For all $E \in {\ensuremath{\mathcal A}}$ we have
${\left\|{p(E)}\right\|} \le m^Z(E)$.
\end{Lem}
\begin{proof}
If $E$ is semistable, then
$0 \le Q(E) = {\left\lvert{Z(E)}\right\rvert}^2 - {\left\|{p(E)}\right\|}^2
= \left(m^Z(E)\right)^2 - {\left\|{p(E)}\right\|}^2$, which is exactly the claim. 
Otherwise, consider the HN filtration $E_0 {\ensuremath{\hookrightarrow}} E_1
{\ensuremath{\hookrightarrow}} \dots {\ensuremath{\hookrightarrow}} E_m = E$. Combined with the triangle inequality, this gives
\[
{\left\|{p(E)}\right\|} \le \sum_i {\left\|{p(E_i/E_{i-1})}\right\|} \le \sum_i {\left\lvert{Z(E_i/E_{i-1})}\right\rvert}
= \sum_i {\left\lvert{Z(E_i) - Z(E_{i-1})}\right\rvert} = m^Z(E).
\]
\end{proof}

The following Lemma needs no proof:
\begin{Lem} \label{lem:subHNpoly}
If $A \subset E$, then ${\mathop{\mathrm{HN}}\nolimits}^Z(A) \subset {\mathop{\mathrm{HN}}\nolimits}^Z(E)$.
\end{Lem}

\begin{Lem} \label{lem:boundlength}
Given any subobject $A \subset E$, we have
\[ m^Z(A) - \Re Z(A) \le m^Z(E) - \Re Z(E). \]
\end{Lem}
\begin{proof}
This follows from the previous Lemma, convexity and a picture, see
fig.~\ref{fig:bound-subobject}. Indeed, choose $x > \Re Z(A), \Re Z(E)$; let $a = x + i \Im Z(A)$ and
$e = x + i \Im Z(E)$. Let $\gamma_A$ be the path that follows by boundary of ${\mathop{\mathrm{HN}}\nolimits}^Z(A)$ from $0$ to
$Z(A)$, and then continues horizontally to $a$; similarly $\gamma_E$ follows the boundary of
${\mathop{\mathrm{HN}}\nolimits}^Z(E)$ and then continues to $e$. Their lengths are given as
\[ {\left\lvert{\gamma_A}\right\rvert} = m^Z(A) + x - \Re Z(A), \quad {\left\lvert{\gamma_E}\right\rvert} = m^Z(E) + x - \Re Z(E).
\]
On the other hand, convexity and Lemma \ref{lem:subHNpoly} imply ${\left\lvert{\gamma_A}\right\rvert} \le
{\left\lvert{\gamma_E}\right\rvert}$; for example, if $\gamma_I$ denotes the intermediate path that follows the boundary
of ${\mathop{\mathrm{HN}}\nolimits}^Z(E)$ up to height $\Im Z(A)$ and then goes horizontally to $a$, we clearly have
${\left\lvert{\gamma_A}\right\rvert} \le  {\left\lvert{\gamma_I}\right\rvert} \le  {\left\lvert{\gamma_E}\right\rvert}$.
\end{proof}

\begin{figure}
\centering
\begin{minipage}{0.45\textwidth}
\centering
        \includegraphics{bound-subobject}
    \captionof{figure}{Proof of Lemma \ref{lem:boundlength}}
    \label{fig:bound-subobject}
\end{minipage}
\begin{minipage}{0.45\textwidth}
	\centering
        \includegraphics{continuous}
    \captionof{figure}{Proof of Lemma \ref{lem:continuous}}
    \label{fig:continuous}
\end{minipage}
\end{figure}

\begin{Lem} Given $C \in {\ensuremath{\mathbb{R}}}$, there are only finitely classes $v(A)$ of subobjects $A \subset E$ with
$\Re \left(Z + u \circ p\right)(A) < C$.
\end{Lem}
\begin{proof}
Given any such $A$, we 
use Lemmas \ref{lem:boundZ} and \ref{lem:boundlength} to obtain
\begin{align*}
C  > \left(\Re Z + u \circ p\right)(A) \ge \Re Z(A) - {\left\|{u}\right\|} {\left\|{p(A)}\right\|}
& > (1 - {\left\|{u}\right\|})\Re Z(A) - {\left\|{u}\right\|} \left(m^Z(A) - \Re Z(A)\right) \\
& \ge (1 - {\left\|{u}\right\|})\Re Z(A) - {\left\|{u}\right\|} \left(m^Z(E) - \Re Z(E)\right).
\end{align*}
Since ${\left\|{u}\right\|} < 1$, this bounds $\Re Z(A)$ from above.
On the other hand, $Z(A) \in {\mathop{\mathrm{HN}}\nolimits}^Z(E)$, and thus $Z(A)$ is constrained to lie in a compact region of ${\ensuremath{\mathbb{C}}}$.
Using Lemmas \ref{lem:boundlength} and \ref{lem:boundZ} again, this gives an upper bound first for $m^Z(A)$
and consequently for ${\left\|{p(A)}\right\|}$. Hence $v(A)$ is contained in a compact
region of $\Lambda \otimes {\ensuremath{\mathbb{R}}}$ depending only on $E$ and $C$, and the claim follows.
\end{proof}

Therefore, Corollary \ref{cor:HNfromfinite} implies the existence of HN filtrations for
$Z_1$ on ${\ensuremath{\mathcal A}}$. 

\subsection*{Continuity} So far, we have constructed a pre-stability condition
$\sigma_t = ({\ensuremath{\mathcal A}}, Z_t)$ for each $t \in [0, 1]$.
\begin{Lem} \label{lem:continuous}
The map $t \mapsto \sigma_t = ({\ensuremath{\mathcal A}}, Z_t)$ defines a continuous path in the space of
pre-stability conditions.
\end{Lem}
\begin{proof}
Let ${\ensuremath{\mathcal P}}_t$ denote the associated slicing. It is enough to prove that for $t$ sufficiently small,
the distance $d({\ensuremath{\mathcal P}}_0, {\ensuremath{\mathcal P}}_t)$ becomes arbitrarily small. We will apply Lemma
\ref{lem:distviasemistables}. Thus consider a ${\ensuremath{\mathcal P}}_0$-semistable object $E \in
{\ensuremath{\mathcal D}}$; up to shift, we may assume $E \in {\ensuremath{\mathcal A}}$. Let $A {\ensuremath{\hookrightarrow}} E$ be the leading HN filtration factor
of $E$ with respect to $Z_t$. Write $Z_0(A) = a + x$ where $a \in {\ensuremath{\mathbb{C}}}$ has the same phase as
$Z_0(E)$ and $x \ge 0$, see fig.~\ref{fig:continuous}.
By convexity, $m^{Z_0}(A) \le {\left\lvert{a}\right\rvert} + x$. Therefore
\[
\Re Z_t(A)
\ge \Re {Z_0}(A) - t {\left\|{u}\right\|} {\left\|{p(A)}\right\|}
\ge \Re {Z_0}(A) - t m^{Z_0}(A)
\ge \Re {Z_0}(A) - t \left({\left\lvert{a}\right\rvert} + x\right)
\ge \Re a - t {\left\lvert{a}\right\rvert}.
\]
Note that $\pi\cdot \left(\psi^+(E) - \phi(E)\right)$ is the argument of $\frac 1a Z_t(A)$; hence
\[ \psi^+(E) - \phi(E) \le \frac 1{\pi} \sin^{-1} \frac{{\left\lvert{Z_t(A) - a}\right\rvert}}{{\left\lvert{a}\right\rvert}} < \frac
1{\pi}\sin^{-1}(t).\]
Combined with an analogous argument for $\psi^-(E)$ we obtain $d'({\ensuremath{\mathcal P}}_0, {\ensuremath{\mathcal P}}_t) \le \frac 1{\pi}
t$ as claimed.
\end{proof}

\section{Preservation of the quadratic inequality}
\label{sect:Qpreserved}

It remains to show that the pre-stability condition $(Z_1, {\ensuremath{\mathcal A}})$ satisfies the support property
with respect to $Q$, i.e. that 
that $Q(v(E)) \ge 0$ for all $E \in {\ensuremath{\mathcal A}}$ that are $Z_1$-stable. The basic reason is that the quadratic inequality is preserved by wall-crossing:

\begin{Lem} \label{lem:Qpreserved}
Let $\sigma = (Z, {\ensuremath{\mathcal P}})$ be pre-stability condition.
Assume that $Q$ is a non-degenerate quadratic form on $\Lambda_{\ensuremath{\mathbb{R}}}$ of signature $(2, {\mathop{\mathrm{rk}}} \Lambda-2)$ such
that $Q$ is negative definite on ${\mathop{\mathrm{Ker}}\nolimits} Z$. If $E$ is strictly $\sigma$-semistable and admits a Jordan-H\"older filtration with factors $E_1, \dots, E_m$, and if $Q(v(E_i)) \ge 0$ for $i = 1, \dots, m$, then
$Q(v(E)) = 0$.
\end{Lem}
\begin{proof}
We apply Lemma \ref{lem:coords}; then $Q(v) \ge 0$ is equivalent to ${\left\lvert{Z(v)}\right\rvert} \ge {\left\|{p(v)}\right\|}$. We
obtain
\[ {\left\lvert{Z(E)}\right\rvert} = \sum_i {\left\lvert{Z(E_i)}\right\rvert} \ge \sum_i {\left\|{p(v(E_i))}\right\|} \ge {\left\|{\sum_i p(v(E_i))}\right\|} = {\left\|{p(v(E))}\right\|} \]
where the first equality holds since the central charges of all $E_i$ are aligned, the first inequality holds by assumption, and the second inequality is the triangle inequality.
\end{proof}

The proof strategy is thus clear: if $E \in {\ensuremath{\mathcal A}}$ is $Z_1$-stable with $Q(v(E)) < 0$, then it must be
$Z_0$-unstable; wall-crossing gives a $t \in [0, 1)$ such that $E$ is strictly $Z_t$-semistable; by the Lemma, one of its Jordan-H\"older factors will also violate the inequality, and we proceed by induction. To make this argument work, we have to show that we can find such a wall, and that this process terminates.

\begin{Lem} \label{lem:duh}
Given two objects $A, E \in {\ensuremath{\mathcal A}}$, denote their phases with respect to $Z_t$ by
$\phi^t(A), \phi^t(E)$, respectively. If the set of $t \in [0,1]$ with $\phi^t(A) \ge \phi^t(E)$ is
non-empty, then it is a closed subinterval of $[0,1]$ containing one of its endpoints.
\end{Lem}
\begin{proof}
The condition is equivalent to $\frac{-\Re Z_t(A)}{\Im Z_t(A)} \ge \frac{-\Re Z_t(E)}{\Im Z_t(E)}$,
which is a linear inequality in $t$.
\end{proof}

\begin{figure}[htb]
  \begin{center}
        \includegraphics{HN-poly-truncated}
    \caption{The truncated HN polygon}
    \label{fig:HNpolytruncated}
  \end{center}
\end{figure}
Consider the polygon whose vertices are the extremal points of ${\mathop{\mathrm{HN}}\nolimits}^{Z_0}(E)$ on the left; we will call this the
\emph{truncated HN polygon of $E$}, see fig.~\ref{fig:HNpolytruncated}. Note that if $A \subset E$ is a
subobject with $\phi_0(A) \ge \phi_0(E)$, then $Z_0(A)$ is contained in the truncated HN polygon of
$E$; by Lemmas \ref{lem:boundlength} and \ref{lem:boundZ} there are only finitely
many classes $v(A)$ of such subobjects.

\begin{Lem}
Every $Z_1$-semistable object $E \in {\ensuremath{\mathcal A}}$ satisfies $Q(E) \ge 0$.
\end{Lem}
\begin{proof}
Otherwise, $E$ must be $Z_0$-unstable. By Lemma \ref{lem:duh} and the following observation, there are only finitely many classes
$v(A)$ of subobjects $A {\ensuremath{\hookrightarrow}} E$ that destabilise $E$ with respect $Z_t$ for any $t\in [0,1]$. Hence
there is a wall $t_1 \in (0, 1]$ such
that $E$ is strictly semistable with respect to $Z_{t_1}$, and moreover $E$ admits a Jordan-H\"older filtration with respect to $Z_{t_1}$. By Lemma 
\ref{lem:Qpreserved}, there are subobjects $G_1 {\ensuremath{\hookrightarrow}} F_1 {\ensuremath{\hookrightarrow}} E$ of the same phase, such that
$Q(v(F_1/G_1)) < 0$.

Applying the same logic to $F_1/G_1$, we obtain $t_2 \in (0, t_1)$ and subobjects $G_1 \subset G_2 \subset F_2 \subset F_1 \subset E$
such that $F_2/G_1, G_2/G_1$ and $F_1/G_1$ all have the same phase with respect to $t_2$, and such that
$Q(v(F_2/G_2)) < 0$. Continuing by induction, we obtain a sequence $t_1 > t_2 > t_3 > \dots$ of real
number and  chains $G_1 \subset G_2 \subset G_3 \subset \dots \subset E$ and $E \supset F_1 \supset
F_2 \supset F_3 \supset \dots$ of subobjects of $E$. 

Lemma \ref{lem:duh} gives $\phi^{t_2}(F_1) \ge \phi^{t_2}(E)$ and $\phi^{t_2}(G_1) \ge
\phi^{t_2}(E)$. Since the central charge of $Z_{t_2}(F_2)$ lies on the line segment
connecting $Z_{t_2}(F_1)$ and $Z_{t_2}(G_1)$, we also have $\phi^{t_2}(F_2) \ge \phi^{t_2}(E)$ (and
therefore $\phi^{t}(F_2) \ge \phi^t(E)$ for all $t \in [0, t_2]$;
similarly for $G_2$. Continuing
this argument by induction, we see that $Z_0(F_i)$ and
$Z_0(G_i)$ are all contained in the truncated HN polygon of $E$. Thus this process terminates.
\end{proof}

By Lemma \ref{lem:onlyreal}, this concludes the proof of Theorem \ref{thm:mainthm} whenever
Assumption \ref{ass:nondeg} holds.

\section{Reductions}

Finally, we will show that we can always reduce the general situation to the case where Assumption \ref{ass:nondeg} holds.
By abuse of language, we call a quadratic form degenerate or non-degenerate if the associated symmetric bilinear form
is degenerate or non-degenerate, respectively.

\begin{Lem}
Assume that the quadratic form $Q$ on $\Lambda_{\ensuremath{\mathbb{R}}}$ is degenerate. Then there exists an injective map $\Lambda_{\ensuremath{\mathbb{R}}} {\ensuremath{\hookrightarrow}} \overline{\Lambda}$ of real vector spaces and a non-degenerate quadratic form $\overline{Q}$ on $\overline{\Lambda}$, extending $Q$, such that any central charge
$Z \colon \Lambda_{\ensuremath{\mathbb{R}}} \to {\ensuremath{\mathbb{C}}}$ whose kernel is negative definite with respect to $Q$ extends to a central charge
$\overline{Z} \colon \overline{\Lambda} \to {\ensuremath{\mathbb{C}}}$ whose kernel is negative definite with respect to $\overline{Q}$.
\end{Lem}
\begin{proof}
Let $N {\ensuremath{\hookrightarrow}} \Lambda_{\ensuremath{\mathbb{R}}}$ be the null space of $Q$; we will only treat the case ${\mathop{\mathrm{dim}}\nolimits}_{\ensuremath{\mathbb{R}}} N = 1$
(otherwise, we can iterate the construction that follows). Choose a splitting $\Lambda_{\ensuremath{\mathbb{R}}} \cong N
\oplus C$; then for $n \in N, c \in C$, we have $Q(n \oplus c) = Q(c)$. Let
$\overline{\Lambda_{\ensuremath{\mathbb{R}}}} := N \oplus N^\vee \oplus C$, let $q$ be the canonical quadratic form on
the hyperbolic plane $N \oplus N^\vee$, and set $\overline{Q} := q \oplus Q|_C$. 

Given $Z$ as above, the restriction $Z|_N$ is  injective, and we may
assume that $Z$ maps $N$ to the real line. Let $n \in N$ be such that $Z(n) = 1$, and let
$n^\vee \in N^\vee$ be the dual vector with $(n, n^\vee) = 1$. We claim that for $\alpha \gg 0$,
the extension of $Z$ defined by $Z'(n^\vee) = \alpha$ has the desired property.

Let $K := {\mathop{\mathrm{Ker}}\nolimits} Z$; then the kernel of $Z'$ is contained in $N \oplus N^\vee \oplus K$, and
given by vectors of the form $a \cdot n - \frac{a}{\alpha}\cdot n^\vee + k$ for $k \in K, a \in {\ensuremath{\mathbb{R}}}$.
For such vectors, we have
\[
 Q\left(a\cdot n  - \frac{a}{\alpha}\cdot n^\vee + k \right)
= - \frac{2a^2}{\alpha} - \frac{2a}{\alpha}(n^\vee, k) + Q(k).
\]
This is a quadratic function in $a$ with negative constant term; its discriminant is negative if
\[ \alpha > \max {\left\{{\frac{(n^\vee, k)^2}{-Q(k)}}\,\colon\,{k \in K, k \neq 0}\right\}} \]
(which is finite since $-Q(\cdot)$ is a positive definite form on $K$).
\end{proof}

Replacing $\Lambda$ by $\Lambda \oplus {\ensuremath{\mathbb{Z}}}$ and $v$ by 
\[ K({\ensuremath{\mathcal D}}) \xrightarrow{v} \Lambda {\ensuremath{\hookrightarrow}} \Lambda \oplus {\ensuremath{\mathbb{Z}}} \]
we can therefore restrict to the case where $Q$ is non-degenerate: given a path $Z_t$ of central
charges in ${\mathop{\mathrm{Hom}}\nolimits}(\Lambda_{\ensuremath{\mathbb{R}}}, {\ensuremath{\mathbb{C}}})$ that are negative definite with respect to $Q$, we can choose
extensions $\overline{Z}_t$ as in the Lemma that form a continuous path in ${\mathop{\mathrm{Hom}}\nolimits}(\overline{\Lambda}, {\ensuremath{\mathbb{C}}})$.
If we can lift the latter path to a path of stability conditions
$\overline{\sigma}_t = (\overline{Z}_t, {\ensuremath{\mathcal P}}_t)$ that satisfy the support property 
with respect to $\overline{Q}$, then $\sigma_t := (Z_t, {\ensuremath{\mathcal P}}_t)$ is a
path of stability conditions satisfying the support property with respect to $Q$.
The reduction to the case where $Q$ has signature
$(2, {\mathop{\mathrm{rk}}} \Lambda -2)$ works similarly:

\begin{Lem} Assume that $Q$ is non-degenerate and of signature $(p, {\mathop{\mathrm{rk}}} \Lambda -p)$ for $p \in \{
0, 1\}$. 
Let $\overline{\Lambda} := \Lambda_{\ensuremath{\mathbb{R}}} \oplus {\ensuremath{\mathbb{R}}}$, and let $\overline{Q}$ be the extension given by
$\overline{Q}(v, \alpha) = Q(v) + \alpha^2$ for $v \in \Lambda_{\ensuremath{\mathbb{R}}}$ and $\alpha \in {\ensuremath{\mathbb{R}}}$. 
Then any central charge $Z$ on $\Lambda_{\ensuremath{\mathbb{R}}}$ whose kernel is negative definite with respect to $Q$ extends to
a central charge $\overline{Z}$ on $\overline{\Lambda}$ whose kernel is negative definite with respect to
$\overline{Q}$. 
\end{Lem}
\begin{proof}
We claim that there exists $z \in {\ensuremath{\mathbb{C}}}$ such that for all $v \in \Lambda_{\ensuremath{\mathbb{R}}}$ with $Z(v) = z$, we have
$Q(v) < -1$. Indeed, 
let $K \subset \Lambda_{\ensuremath{\mathbb{R}}}$ be the kernel of $Z$, and let $K^\perp$ be its orthogonal complement. Then clearly
we may assume $v \in K^\perp$. Since the restriction of $Z$ to $K^\perp$ is injective, and since
$K^\perp$ either has rank one, or has signature $(1, -1)$ with respect to $Q$, the claim is evident.

Using the claim, we can set $\overline Z(v, \alpha) := Z(v) + \alpha z$. 
\end{proof}
This concludes the proof of Theorem \ref{thm:mainthm}. 

\section{Application} \label{sect:application}

\begin{proof}[Proof of Corollary \ref{cor:P0covering}]
Using the same arguments as in the previous section, we may assume that the Mukai pairing
on $\Lambda$ has signature $(2, {\mathop{\mathrm{rk}}} \Lambda -2)$. 

By Serre duality, any $\sigma$-stable object $E \in {\ensuremath{\mathcal D}}$ satisfies ${\mathop{\mathrm{Hom}}\nolimits}(E, E[i]) = 0$ for
$i < 0$ or $i > 3$ and ${\mathop{\mathrm{Hom}}\nolimits}(E, E) = {\ensuremath{\mathbb{C}}} = {\mathop{\mathrm{Hom}}\nolimits}(E, E[2])$; therefore, $(v(E), v(E)) \ge -2$.
Moreover, Serre duality induces a non-degenerate symplectic form on ${\mathop{\mathrm{Ext}}\nolimits}^1(E, E)$, and it
has even dimension; thus $(v(E), v(E)) = -2$ or $(v(E), v(E)) \ge 0$.

Let $\sigma = (Z, {\ensuremath{\mathcal P}})$ be a stability condition with $Z \in {\ensuremath{\mathcal P}}_0({\ensuremath{\mathcal D}})$. By the same argument as in 
Lemma \ref{lem:coords} we may assume
\[ (v, v) = {\left\lvert{Z(v)}\right\rvert}^2 - {\left\|{p(v)}\right\|}^2, \]
where $p \colon \Lambda_{\ensuremath{\mathbb{R}}} \to {\mathop{\mathrm{Ker}}\nolimits} Z$ is the orthogonal projection onto the kernel of $Z$, and
where ${\left\|{\cdot}\right\|}$ denotes the norm on ${\mathop{\mathrm{Ker}}\nolimits} Z$ induced by the negative of the Mukai pairing.
We claim that  
\begin{equation} \label{eq:defZ}
C:= {\mathop{\mathrm{inf}}\nolimits} {\left\{{{\left\lvert{Z(\delta)}\right\rvert}}\,\colon\,{\delta \in \Lambda, (\delta, \delta) = -2}\right\}} > 0.
\end{equation}
Indeed, if ${\left\lvert{Z(\delta)}\right\rvert} \le 1$, then ${\left\|{p(\delta)}\right\|} \le \sqrt{3}$;
as ${\left\lvert{Z(\cdot)}\right\rvert} + {\left\|{p(\cdot)}\right\|}$ is a norm on $\Lambda_{\ensuremath{\mathbb{R}}}$,
there are only finitely many integral classes satisfying both inequalities.
Since $Z(\delta) \neq 0$ by assumption, the claim follows.

Now set 
\[ Q(v) := (v, v) + \frac 2{C^2} {\left\lvert{Z(v)}\right\rvert}^2. \]
Clearly $Q$ is negative definite on ${\mathop{\mathrm{Ker}}\nolimits} Z$. Moreover, if $E$ is $\sigma$-stable, then
either $(v(E), v(E)) \ge 0$ or $v(E) = \delta$ is a root in $\Lambda$; in both cases,  $Q(v(E)) \ge 0$ is evident
from the construction.

Therefore, $\sigma$ satisfies the support property with respect to $Q$. Theorem \ref{thm:mainthm} gives
an open neighbourhood ${\ensuremath{\mathcal P}}_Z(Q) \subset {\mathop{\mathrm{Hom}}\nolimits}(\Lambda, {\ensuremath{\mathbb{C}}})$ of $Z$ and an open neighbourhood $U_\sigma$ of $\sigma$ such that $U_\sigma \xrightarrow{\ensuremath{\mathcal Z}} {\ensuremath{\mathcal P}}_Z(Q)$ is a covering.

By compactness, any path $Z_t$ in ${\ensuremath{\mathcal P}}_0({\ensuremath{\mathcal D}})$ is contained in a finite number of such sets
${\ensuremath{\mathcal P}}_{Z_{t_i}}(Q_i)$, where $Q_i$ the quadratic form associated to $Z_{t_i}$.
Thus $Z_t$ lifts uniquely to a path in ${\ensuremath{\mathcal Z}}^{-1}({\ensuremath{\mathcal P}}_0({\ensuremath{\mathcal D}}))$.
\end{proof}

\bibliography{all}                      
\bibliographystyle{halpha}     

\end{document}



\documentclass{amsart}
\usepackage{amssymb,amscd}

\newtheorem{thm}{Theorem}[section]

\newtheorem*{prob*}{Problem}

\newtheorem{prop}[thm]{Proposition}
\newtheorem{lem}[thm]{Lemma}
\newtheorem{cor}[thm]{Corollary}

\theoremstyle{definition}
\newtheorem{defn}[thm]{Definition}
\newtheorem{exa}[thm]{Example}

\theoremstyle{remark}
\newtheorem*{rem*}{Remark}
\newtheorem{rem}[thm]{Remark}

\frenchspacing

\begin{document}

\title{Covariants, Invariant Subsets, and First Integrals}

\author{Frank Grosshans and Hanspeter Kraft}
\date{February 22, 2017}

\address{Department of Mathematics
\newline
\indent West Chester University, West Chester, PA 19383, USA}
\email{fgrosshans@wcupa.edu}
\address{Departement Mathematik und Informatik, 
Universit\"at Basel,\newline
\indent Spiegelgasse 1, CH-4051 Basel, Switzerland}
\email{Hanspeter.Kraft@unibas.ch}
\thanks{The first author thanks Sebastian Walcher for his advice on an earlier version of this paper.}

\keywords{Covariants, endomorphisms, invariant subsets, vector fields, first integrals, ind-varieties}
\subjclass[2010]{Primary 14L30, 22E47, secondary 13A50, 34C14}

\begin{abstract}
Let ${\Bbbk}$ be an algebraically closed field of characteristic $0$, and
let $V$ be a finite-dimensional vector space. Let $\operatorname{End}(V)$ be the
semigroup of all polynomial endomorphisms of $V$. Let ${\mathcal E}{\subseteq}
\operatorname{End}(V)$ be a linear subspace which is also a semi-subgroup. Both $\operatorname{End}(V)$ and
${\mathcal E}$ are ind-varieties which act on $V$ in the obvious way. 

In this paper, we study important aspects
of such actions. We assign to ${\mathcal E}$ a linear subspace
${\mathcal D}_{\mathcal E}$ of the vector fields on $V$. A subvariety $X$ of $V$ is said to
be ${\mathcal D}_{\mathcal E}$-invariant if $\xi(x)\in T_{x}X$ for all $\xi
\in{\mathcal D}_{\mathcal E}$. We show that $X$ is ${\mathcal D}_{\mathcal E}$ -invariant if and
only if it is the union of ${\mathcal E}$-orbits.
For such $X$, we define first integrals and construct a quotient space for the ${\mathcal E}$-action. 

An important case occurs when $G$ is an algebraic subgroup of $\operatorname{GL}(V)$ and
${\mathcal E}$ consists of the $G$-equivariant polynomial endomorphisms.
In this case, the associated ${\mathcal D}_{\mathcal E}$ is the space the $G$-invariant vector fields.
A significant question here is whether there are non-constant $G$-invariant first integrals on $X$. 
As examples, we study the adjoint
representation, orbit closures of highest weight vectors, and representations of the
additive group. We also look at finite-dimensional
irreducible representations of $\operatorname{SL}_{2}$ and
its nullcone.
\end{abstract}

\maketitle

\section{Introduction and main results}
The study of algebraic group actions, which began in the nineteenth
century with the action of $\operatorname{SL}_{2}$ on binary forms, has proved to be
incredibly fertile. In this
paper, we consider some important aspects of that subject in a new setting,
ind-varieties (see Appendix). Let ${\Bbbk}$ be an algebraically closed field
of characteristic $0$, and let $X$ be an irreducible affine variety over ${\Bbbk}$. Let
${\mathcal D}$ be a set of vector fields on $X$. A closed subvariety
$Y\subseteq X$ is called ${\mathcal D}$-invariant if $\xi(y)\in T_{y}Y$ for all
$y\in Y$ and $\xi\in{\mathcal D}$. We establish some basic properties of
invariant subsets including the following: for any $x\in X$, there is a smallest
${\mathcal D}$-invariant closed subvariety, $M(x)$, which contains $x$ (Lemma~\ref{D-invariant.lem}). 
When ${\mathcal D}$ is a linear subspace, we define a first integral of
${\mathcal D}$ to be a function $f\in {\Bbbk}(X)$ such that $\xi f=0$ for
all $\xi\in{\mathcal D}$. We show that first integrals are precisely those
functions which are constant on the spaces $M(x)$ (Lemma~\ref{first-integral.lem}).

We next consider the semigroup $\operatorname{End}(X)$ consisting of all endomorphisms of $X
$. An important fact is that $\operatorname{End}(X)$ is an ind-variety. This allows us to define the (Zariski) tangent space
$T_\operatorname{id}\operatorname{End}(X)$ and to  associate to any $A\in T_\operatorname{id}\operatorname{End}(X)$ a vector field $\xi_{A}$ on $X$ (section~\ref{endo.subsec}). 
For ${\mathcal E}{\subseteq} \operatorname{End}(X)$, a closed semi-subgroup, we denote by ${\mathcal D}_{\mathcal E}$
the associated vector fields. The ${\mathcal E}$-orbit of an element $x\in X$ is defined as ${\mathcal E}(x):=\{{\varphi}(x)\mid {\varphi}\in {\mathcal E}\}$. We first show that if a closed subvariety $Y{\subseteq} X$ is the union of ${\mathcal E}$-orbits, then $Y$ is ${\mathcal D}_{\mathcal E}$-invariant (Proposition~\ref{E&D-invariance.prop}).

Our main application of these ideas is to the case where $X$ is a finite
dimensional vector space and $Y$ is a closed subvariety. We change notation
and denote the vector space by $V$ and the subvariety by $X$. We assume from
now on that ${\mathcal E}{\subseteq} \operatorname{End}(V)$ is a linear subspace which is also a
semi-subgroup. In this context, we show that for $v\in V$, ${\mathcal E}(v)=M(v)$ and
that a subvariety $X{\subseteq} V$ is ${\mathcal D}_{\mathcal E}$ -invariant if and only if
it is the union of ${\mathcal E}$-orbits (Theorem~\ref{main-theorem-1}). This means that ${\mathcal D}_{\mathcal E}$-invariant 
subvarieties are precisely those which are stable under the action of ${\mathcal E}$. Furthermore,
first integrals separate ${\mathcal E}$-orbits on an open set and can be used to
construct a quotient space for the action of ${\mathcal E}$ on $X$ (Proposition~\ref{quotient-mod-E.prop}).
This construction includes an algebraic (and global) version of a classical
theorem of {\textsc{{Frobenius}\/}}  \cite[Theorem 1.60]{Wa1971Foundations-of-dif}.

The most important instances of the above occur when $G$ is an algebraic
group acting linearly on $V$, and ${\mathcal E}$ is the semigroup of covariants, i.e., 
$$
{\mathcal E}=\operatorname{End}_{G}(V) :=
\{{\varphi}\in \operatorname{End}(V)\mid {\varphi}(g\cdot v) = g \cdot {\varphi}(v)\text{ for all } g\in G \text{ and } v\in V\}.
$$ 
An important question here is whether or not there are non-constant $G$-invariant
first integrals. Examples show that such can occur. However, in those cases
where they do not, the field of first integrals is the field of rational
functions on a homogeneous space (Lemma~\ref{first-integrals.lem} and Theorem~\ref{first-integrals.thm}).

When $G$ is reductive and the orbit $Gv$ is closed, ${\mathcal E}(v)$ can be described in
terms of the stabilizer of $v$ (Proposition~\ref{Panyushev.prop}). Furthermore, when the
generic $G$-orbit in $X$ is closed and ${\mathcal E}=\operatorname{End}_{G}(V)$, we show that there are
no $G$-invariant first integrals (Theorem~\ref{first-integrals.thm}). Finally, in section~\ref{SLtwo.sec}, we study the
case where $G={\operatorname{SL}_{2}}$ and either $X=V_{d}$,  the binary forms of degree
$d$, or $X$ is the nullcone of $V_{d}$.

The background for some of the ideas in this paper comes from ordinary
differential equations, see \cite{GrScWa2012Invariant-sets-for}. In that context, it can
be shown that a Zariski-closed set $X$ is ${\mathcal D}_{\mathcal E}$-invariant if and
only if it is the union of trajectories of solutions to $\frac{dx}{dt}=F(x)$, 
$F\in {\mathcal E}$. The quotient construction given in this paper leads to a
description of all such sets. The study of the ${\mathcal E}(v)$ began with the paper
\cite{LeSp1999A-note-concerning-} by {\textsc{{Lehrer}\/}} and {\textsc{{Springer}\/}}
which was subsequently extended by {\textsc{{Panyushev}\/}} in \cite{Pa2002On-covariants-of-r}. The
difficult problem of constructing the module of covariants on $X$ was first
considered in the nineteenth century \cite{El1913An-Introduction-to} and continues to be of interest
\cite{Do2008Covariants-and-the}. In this paper, we consider that problem in the context of finding
the maximum dimension of the orbits ${\mathcal E}(v)$.

{\par\medskip}
\section{Basic material}
\subsection{Vector fields and ${\mathcal D}$-invariant subsets}
Our base field ${\Bbbk}$ is algebraically closed of characteristic zero.
We start with a lemma which translates the concept of invariant subsets with respect to an ordinary differential equation into the algebraic setting. For an affine variety $X$ an {\it algebraic vector field} $\xi=(\xi(x))_{x\in X}$ is a collection of tangent vectors $\xi(x)\in T_{x}X$ such that, for every regular function $f \in \operatorname{\mathcal O}(X)$, the function $\xi f\colon x \mapsto \xi(x)f$ is again regular. It is easy to see that this is the same as a derivation of the coordinate ring $\operatorname{\mathcal O}(X)$. Note that $\xi f$ is also defined for rational functions $f$.

In addition, one can define the  the {\it tangent bundle\/} $TX$ of $X$ which is a variety together with a projection $p\colon TX \to X$ such that the fibers $p^{-1}(x)$ are the Zariski tangent spaces $T_{x}X$. Then the sections are the algebraic vector fields (see e.g. \cite[Appendix A.4.5]{Kr2014Algebraic-Transfor}). It is clear that the algebraic vector fields form a $\operatorname{\mathcal O}(X)$-module which will be denoted by $\operatorname{Vec}(X)$ and which can be identified with the $\operatorname{\mathcal O}(X)$-module $\operatorname{Der}(\operatorname{\mathcal O}(X))$ of derivations of $\operatorname{\mathcal O}(X)$.

\begin{lem}
Let $X$ be a smooth complex variety, and let $\xi \in \operatorname{Vec}(X)$ be an algebraic vector field. Then a Zariski-closed subvariety $Y {\subseteq} X$ is invariant with respect to the flow defined by the differential equation $\dot x = \xi(x)$ if and only if $\xi(y) \in T_{y}Y$ for all $y \in Y$.
\end{lem}
\begin{proof}
Let $\Phi\colon X \times {\mathbb R} \to X$ be the local flow of $\xi$, defined in an open neighborhood of $X \times \{0\}$. By definition, 
$$
\frac{\partial }{\partial t} \Phi(x,t)|_{t=0} = \xi(x) \text{ for all } x \in X.
$$
This implies that if $Y$ is invariant under $\Phi$, then $\xi(y) \in T_{y}Y$ for all $y \in Y$. On the other hand, assume that $\xi(y) \in T_{y}Y$ for all $y \in Y$, and denote by $Y' {\subseteq} Y$ the open dense set of smooth points of $Y$. Then $\xi|_{Y'}$ defines a local flow $\Phi_{Y'}\colon Y' \times {\mathbb R} \to Y'$ such that $\frac{\partial }{\partial t} \Phi_{Y'}(y',t)|_{t=0} = \xi(y')$ for  all  $y \in Y'$. By the uniqueness of the local flow, we have $\Phi_{Y'} = \Phi|_{Y' \times {\mathbb R}}$, and so $Y'$ is invariant under $\Phi$. Since $Y = \overline{Y'}$ we see that $Y$ is also invariant under $\Phi$.
\end{proof}
This lemma allows to define the invariance of subvarieties with respect to a set of vector fields for an arbitrary ${\Bbbk}$-variety $X$.
\begin{defn}
Let ${\mathcal D} {\subseteq} \operatorname{Vec}(X)$ be a set of vector fields. 
\begin{enumerate}
\item
A closed subvariety $Y {\subseteq} X$ is called {\it ${\mathcal D}$-invariant\/} if $\xi(y)\in T_{y}Y$ for all $y\in Y$ and all $\xi \in {\mathcal D}$. We also say that the vector fields $\xi \in {\mathcal D}$ are {\it parallel to $Y$}.
\item
A subspace $W {\subseteq} \operatorname{\mathcal O}(X)$ is called {\it ${\mathcal D}$-invariant} if $\xi(W) {\subseteq} W$ for all $\xi \in {\mathcal D}$.
\end{enumerate}
\end{defn}
\begin{rem} 
We will constantly use the following easy fact. If $\xi$ is a vector field parallel to $Y$ and $f$  a rational function on $X$ defined in a neighborhood $U$ of $y \in Y$, then $\xi(y) f = \xi(y) (f|_{U\cap Y})$. In particular, if $f$ is regular on $X$, then $(\xi f)|_{Y}=\xi|_{Y} (f|_{Y})$.
\end{rem} 

{\par\smallskip}
\subsection{${\mathcal D}$-invariant ideals}
Let ${\mathcal D} {\subseteq} \operatorname{Vec}(X)$ be a set of vector fields.
\begin{lem}
If $I(Y) {\subseteq} \operatorname{\mathcal O}(X)$ denotes the vanishing ideal of $Y$, then $Y$ is ${\mathcal D}$-invariant if and only if $I(Y)$ is ${\mathcal D}$-invariant.
\end{lem}
\begin{proof}
If $f \in I(Y)$, then, for $y \in Y$,  $(\xi f)(y) = \xi(y) f = \xi(y) f|_{Y}= 0$, hence $\xi f \in I(Y)$. Conversely, if $\xi(I(Y)) {\subseteq} I(Y)$, then $\xi$ induces a derivation of $\operatorname{\mathcal O}(X)/I(Y) = \operatorname{\mathcal O}(Y)$, and the claim follows.
\end{proof}

For a closed subvariety $Y {\subseteq} X$ we can define the  Lie subalgebra of the vector fields on $X$ parallel to $Y$:
$$
\operatorname{Vec}_{Y}(X):=\{\xi \in \operatorname{Vec}(X) \mid \xi(y) \in T_{y}Y \text{ for all }y \in Y\} {\subseteq} \operatorname{Vec}(X).
$$ 
We have a surjective homomorphism of Lie algebras 
$$
\rho\colon \operatorname{Vec}_{Y}(X) {\twoheadrightarrow} \operatorname{Vec}(Y), \quad\xi\mapsto \xi|_{Y},
$$  
whose kernel consists of the vector fields on $X$ vanishing on $Y$. In order to see that this map is surjective it suffices to consider the case where $X$ is a vector space, and then the statement is clear. With this notation we see that  $Y$ is ${\mathcal D}$-invariant if and only if  ${\mathcal D} {\subseteq} \operatorname{Vec}_{Y}(X)$.

\begin{lem}{\label{D-invariant.lem}}
\begin{enumerate}
\item Sums and intersections of ${\mathcal D}$-invariant ideals are ${\mathcal D}$-invariant.
\item If $I {\subseteq} \operatorname{\mathcal O}(X)$ is a ${\mathcal D}$-invariant ideal, then so is $\sqrt{I}$.
\item If $Y_{i}{\subseteq} X$, $i\in I$, are ${\mathcal D}$-invariant closed subvarieties, then so is $\bigcap_{i\in I}Y_{i}$.
\item For any $x \in X$ there is a uniquely defined minimal ${\mathcal D}$-invariant closed subvariety $M(x)\subseteq X$ containing $x$.
\item If the closed subvariety $Y {\subseteq} X$ is ${\mathcal D}$-invariant, then every irreducible component of $Y$ is ${\mathcal D}$-invariant.
\end{enumerate}
\end{lem}
\begin{proof}
(1) is clear, (3) follows from (1) and (2), and (4) follows from (3).
{\par\smallskip}
(2) It suffices to show that if $f^{n}=0$, then $(\xi f)^{m}=0$ for some $m>0$. Let $e_{0}\geq 0$ be the minimal $e$ such that there exists a $q\geq 0$ with $f^{e}\cdot(\xi f)^{q}=0$. If $e_{0}=0$, we are done. So assume that $e_{0}>0$. Then 
$$
0=\xi(f^{e_{0}}\cdot(\xi f)^{q})\cdot\xi f = e_{0}f^{e_{0}-1}\cdot(\xi f)^{q+1} + q f^{e_{0}}\cdot(\xi f)^{q}\cdot \xi^{2}f 
= e_{0}f^{e_{0}-1}\cdot(\xi f)^{q+1},
$$
contradicting the minimality of $e_{0}$.
{\par\smallskip}
(5) It suffices to consider the case where $Y=X$, hence $(0) = {\mathfrak p}_{1}\cap\ldots\cap {\mathfrak p}_{k}$ where the ${\mathfrak p}_{i}$ are the minimal primes of $\operatorname{\mathcal O}(X)$. For every $i$ choose an element $p_{i}\in\bigcap_{j\neq i}{\mathfrak p}_{j}\setminus{\mathfrak p}_{i}$. Then ${\mathfrak p}_{i}=\{p\in\operatorname{\mathcal O}(X)\mid p_{i} p = 0\}$, and the same holds for every power of $p_{i}$. For every $p \in {\mathfrak p}_{i}$ we find
$$
0 = p_{i} \xi(p_{i} p) = p_{i}(p_{i} \xi p + p\xi p_{i}) = p_{i}^{2} \xi p,
$$
hence $\xi p \in {\mathfrak p}_{i}$. 
\end{proof}

\begin{defn}
The closed subvarieties $M(x) {\subseteq} X$ from Lemma~\ref{D-invariant.lem}(4) are called {\it minimal ${\mathcal D}$-invariant subvarieties}. 
By Lemma~\ref{D-invariant.lem}(5) they are irreducible.
\end{defn}

{\par\smallskip}
\subsection{Linear spaces of vector fields}
In the following, we will mainly deal with the case where ${\mathcal D} {\subseteq} \operatorname{Vec}(X)$ is a linear subspace. In this case, we set
$$
{\mathcal D}(x):= {\varepsilon}_{x}({\mathcal D}):=\{\xi(x) \mid \xi \in {\mathcal D}\} {\subseteq} T_{x}X
$$
where  ${\varepsilon}_{x}\colon \operatorname{Vec}(X) \to T_{x}X$ is the (linear) evaluation map $\xi\mapsto \xi(x)$. 
Note that a closed subvariety $Y {\subseteq} X$ is ${\mathcal D}$-invariant if and only if ${\mathcal D}(y) {\subseteq} T_{y}Y$ for all $y \in Y$.

The following lemma is clear.
\begin{lem}\label{lower-semi-cont.lem}
For a linear subspace ${\mathcal D} {\subseteq} \operatorname{Vec}(X)$ the function $x \mapsto \dim{\mathcal D}(x)$ is lower semicontinuous, i.e., for every $x\in X$ the set $U_{x}:=\{u \in X\mid \dim{\mathcal D}(u)\geq \dim{\mathcal D}(x)\}$ is a (Zariski-) open neighborhood of $x$.
\end{lem}
Setting $d_{\mathcal D}(X):=\max_{x\in X}\dim {\mathcal D}(x)$ the lemma implies that 
$$
X':=\{x\in X \mid \dim{\mathcal D}(x)=d_{\mathcal D}(X)\}
$$ 
is open (and non-empty) in $X$.

{\par\medskip}
\section{Endomorphisms}
\subsection{The semigroup of endomorphisms}\label{endo.subsec}
We now study the semigroup $\operatorname{End}(X)$ of endomorphisms  of $X$. An important fact is that $\operatorname{End}(X)$ is an {\it ind-variety} (see Appendix) which allows to define the (Zariski) tangent space $T_\operatorname{id}\operatorname{End}(X)$. We have a canonical inclusion
$$
\Xi\colon T_\operatorname{id}\operatorname{End}(X) {\hookrightarrow} \operatorname{Vec}(X), \  A \mapsto \xi_{A},
$$ 
where the vector field $\xi_{A}$ is defined in the following way (see Appendix, Proposition~\ref{vector-fields.prop}). For any $x \in X$ consider the ``orbit map'' $\mu_{x}\colon \operatorname{End}(X) \to X$, ${\varphi}\mapsto {\varphi}(x)$, and its differential 
$$
(d\mu_{x})_\operatorname{id}\colon T_\operatorname{id}\operatorname{End}(X) \to T_{x}X.
$$ 
Then define $\xi_{A}(x):=(d\mu_{x})_\operatorname{id}(A)$. 

Note that $\operatorname{End}(X)$ acts on $X$,  
$$
\Phi\colon \operatorname{End}(X) \times X \to X, \quad({\varphi},x)\mapsto {\varphi}(x),
$$ 
and this action is a morphism of ind-varieties. For the differential we find
\[\tag{$*$}
d\Phi_{(\operatorname{id},x_{0})}\colon T_\operatorname{id}\operatorname{End}(X) \oplus T_{x_{0}}X \to T_{x_{0}}X, \quad (A,\delta)\mapsto \xi_{A}(x_{0})+\delta.
\]
\begin{defn}
If ${\mathcal E} {\subseteq} \operatorname{End}(X)$ is a closed semi-subgroup we say that a subset $Y {\subseteq} X$ is {\it stable under ${\mathcal E}$} (shortly ${\mathcal E}$-stable), if 
${\varphi}(y) \in Y$ for all $y\in Y$ and all ${\varphi}\in{\mathcal E}$. Equivalently,
$Y$ contains with every point $y$ the {\it orbit  ${\mathcal E}(y)$ of $y$} defined as
$$
{\mathcal E}(y):=\{{\varphi}(y) \mid {\varphi}\in {\mathcal E}\}.
$$
\end{defn}
The closed semi-subgroup ${\mathcal E} {\subseteq} \operatorname{End}(X)$ defines a linear subspace ${\mathcal D}_{\mathcal E}{\subseteq}\operatorname{Vec}(X)$ as the image of the tangent space $T_\operatorname{id}{\mathcal E}$ under $\Xi$:
$$
{\mathcal D}_{\mathcal E}:= \{\xi_{A}\mid A \in T_\operatorname{id}{\mathcal E}\} {\subseteq} \operatorname{Vec}(X).
$$
The main point of this section is to relate the invariance under ${\mathcal D}_{\mathcal E}$ with the stability under the semigroup ${\mathcal E}$. A first and easy result is the following.

\begin{prop}\label{E&D-invariance.prop}
Let ${\mathcal E} \subseteq \operatorname{End}(X)$ be a closed semi-subgroup. If $Y {\subseteq} X$ is a closed ${\mathcal E}$-stable subvariety, 
then $Y$ is ${\mathcal D}_{\mathcal E}$-invariant.
\end{prop}
\begin{proof}
Since $Y$ is ${\mathcal E}$-invariant we have a morphism $\Phi\colon {\mathcal E} \times Y \to Y$ whose differential
$$
d\Phi_{(\operatorname{id},y)}\colon T_\operatorname{id} {\mathcal E} \oplus T_{y}Y \to T_{y}Y
$$
sends $(A,0)$ to $\xi_{A}(y)$, by  formula $(*)$ above. Thus $\xi(y) \in T_{y}Y$ for all $\xi \in {\mathcal D}_{\mathcal E}$ which means that $Y$ is ${\mathcal D}_{\mathcal E}$-invariant.
\end{proof}
We will see below that under stronger assumptions on ${\mathcal E}$ the reverse implication also holds, i.e., a closed subset $Y {\subseteq} X$ is ${\mathcal E}$-stable if and only if it is ${\mathcal D}_{\mathcal E}$-invariant.
\begin{rem}
We do not know what the structure of the subsets ${\mathcal E}(x) {\subseteq} X$ is. If ${\mathcal E}$ is curve-connected (i.e. any two points of ${\mathcal E}$ can be connected by an irreducible curve, see Definition~\ref{affine-algebraic.def}(5)), then one can show that ${\mathcal E}(x)$ contains a set $U$ which is open and dense in $\overline{{\mathcal E}(x)}$. But it is not clear whether ${\mathcal E}(x)$ is constructible.
\end{rem}

{\par\smallskip}
\subsection{The case of a vector space}
In case of a vector space $X = V$ the situation is much simpler, because we can identify every tangent space $T_{v}V$ with $V$. In particular,  vector fields $\xi\in\operatorname{Vec}(V)$ correspond to morphisms $\xi\colon V\to V$. Choosing a basis of $V$ we have
$$
\xi = \sum_{i=1}^{n}p_{i}{\frac{\partial}{\partial x_{i}}} \text{ where }p_{i} := \xi x_{i}.
$$
In this situation, the semigroup $\operatorname{End}(V) = \operatorname{\mathcal O}(V) \otimes V$ is a vector space, hence $T_\operatorname{id}\operatorname{End}(V) = \operatorname{End}(V)$ in a canonical way, and 
$$
\Xi\colon \operatorname{End}(V) = T_\operatorname{id}\operatorname{End}(V) {\xrightarrow{\sim}} \operatorname{Vec}(V)
$$ 
is the obvious isomorphism given as follows. In terms of coordinates an endomorphism ${\varphi}$ has the form ${\varphi}=(p_{1},\ldots,p_{n})\colon {\Bbbk}^{n} \to {\Bbbk}^{n}$ where $p_{i}={\varphi}^{*}(x_{i})$, and the corresponding vector field $\xi:=\Xi({\varphi})$ is given by  $\xi= \sum_{i=1}^{n}p_{i}{\frac{\partial}{\partial x_{i}}}$. 

The same formula holds for a semigroup  ${\mathcal E} {\subseteq} \operatorname{End}(V)$ which is a  {\it linear} subspace.
However, for a general closed semigroup ${\mathcal E}{\subseteq}\operatorname{End}(V)$, we cannot identify ${\mathcal E}$ with $T_\operatorname{id}{\mathcal E}$, and so the formula above does not make sense. For example, if ${\varphi}\in\operatorname{End}(V)$ is any endomorphism, then the semigroup ${\mathcal E}:=\{\operatorname{id},{\varphi},{\varphi}^{2},\ldots\}$ is discrete, hence $T_\operatorname{id}{\mathcal E}$ is trivial, and so ${\mathcal D}_{\mathcal E}$ is also trivial.
\par\smallskip
The following result is crucial. We will identify $T_{v}V$ with $V$ and thus consider the subspace ${\mathcal D}(v) \in T_{v}V$ as a subspace of $V$. 

\begin{lem}\label{main-lemma}
Let ${\mathcal E} {\subseteq} \operatorname{End}(V)$ be a linear subspace which is a semigroup. Then 
$$
{\mathcal E}(v) = M(v) = {\mathcal D}_{\mathcal E}(v) \ \text{ for all } v \in V.
$$
In particular, a subset $Y {\subseteq} V$ is ${\mathcal D}_{\mathcal E}$-invariant if and only if it is ${\mathcal E}$-stable.
Moreover, we have ${\mathcal E}(w) = {\mathcal E}(v)$ for all $w$ in an open neighborhood of $v$ in ${\mathcal E}(v)$.
\end{lem}

\begin{proof}
(a) 
We have seen in Proposition~\ref{E&D-invariance.prop} that ${\mathcal E}(v)$ is ${\mathcal D}_{\mathcal E}$-invariant, because it is stable under ${\mathcal E}$. Hence, ${\mathcal D}_{\mathcal E}(w) {\subseteq} T_{w}{\mathcal E}(v)$ for all $w \in {\mathcal E}(v)$.
\par\smallskip
(b) The evaluation map $\mu_{v}\colon {\mathcal E} \to V$ is linear with image ${\mathcal E}(v)$, hence ${\mathcal E}(v) {\subseteq} V$ is a linear subspace and ${\mathcal D}_{\mathcal E}(v) = T_{v}{\mathcal E}(v) = {\mathcal E}(v)$.
\par\smallskip
(c) By Lemma~\ref{lower-semi-cont.lem} there is an open neighborhood  $U_{v}$ of $v$ in ${\mathcal E}(v)$ such that $\dim{\mathcal D}_{\mathcal E}(w) \geq \dim{\mathcal D}_{\mathcal E}(v)$ for all $w \in U_{v}$. Hence, ${\mathcal E}(v)={\mathcal D}_{\mathcal E}(v) = {\mathcal D}_{\mathcal E}(w)={\mathcal E}(w)$ for $w \in U_{v}$, by (a).
\par\smallskip
(d) It remains to prove the minimality, i.e. that ${\mathcal E}(v) = M(v)$. Let $Y {\subseteq} {\mathcal E}(v)$ be closed and ${\mathcal D}_{\mathcal E}$-invariant with $v \in Y$. Then, for every $w \in U_{v}\cap Y$, we have ${\mathcal E}(v) = {\mathcal D}_{\mathcal E}(w) {\subseteq} T_{w}Y {\subseteq} {\mathcal E}(v)$. Hence, $\dim Y \geq \dim {\mathcal E}(v)$, and so $Y = {\mathcal E}(v)$.
\end{proof}

\begin{rem}\label{closed-cone.rem}
If ${\mathcal E} {\subseteq} \operatorname{End}(V)$ is as in the lemma above, then it contains the scalar multiplication ${\Bbbk} \cdot\operatorname{id}$, and so ${\mathcal E}(v) \supset {\Bbbk} v$ for all $v \in V$. Therefore, every ${\mathcal D}_{\mathcal E}$-invariant closed subvariety $X$ is a closed cone, i.e., contains with every point $x\neq 0$ the line ${\Bbbk}\cdot x$, and  every ${\mathcal D}_{\mathcal E}$-invariant ideal is homogeneous.
\end{rem}

{\par\smallskip}
\subsection{Linear semigroups}
One would like to extend the lemma above to a statement of the form that a subvariety $Y {\subseteq} X$ is stable under a closed semigroup ${\mathcal E} {\subseteq} \operatorname{End}(X)$ if and only if it is ${\mathcal D}_{\mathcal E}$-invariant where ${\mathcal D}_{\mathcal E}$ is the image of $T_\operatorname{id}{\mathcal E}$ in $\operatorname{Vec}(X)$. We do not know if such a result holds in general, but we can prove it for so-called {\it linear semigroups\/} ${\mathcal E} {\subseteq} \operatorname{End}(X)$ which is sufficient for the applications we have in mind.

If $X {\subseteq} V$ is a closed subvariety, then $\operatorname{End}(X) {\subseteq} \operatorname{Mor}(X,V)$. Thus we can form linear combinations of endomorphisms of $X$, but in general the resulting morphism does not have its image in $X$.

\begin{defn} A semi-subgroup ${\mathcal E} {\subseteq} \operatorname{End}(X)$ is called {\it linear} if there is a closed embedding $X {\hookrightarrow} V$ into a vector space $V$ such that the image of ${\mathcal E}$ in $\operatorname{Mor}(X,V)$ is a linear subspace.
\end{defn}

\begin{thm}\label{main-theorem-1}
Let $X$ be an affine variety, and let ${\mathcal E} {\subseteq} \operatorname{End}(X)$ be a linear semigroup. 
\begin{enumerate}
\item
For any $x \in X$ we have ${\mathcal E}(x) = M(x)$. 
\item
The subsets ${\mathcal E}(x){\subseteq} X$ are closed and isomorphic to vector spaces.
\item
$T_{x}M(x) = T_{x}{\mathcal E}(x) = {\mathcal D}_{\mathcal E}(x)$ for all $x \in X$.
\end{enumerate}
In particular, a closed subvariety $Y {\subseteq} X$ is ${\mathcal D}_{\mathcal E}$-invariant if and only if it is ${\mathcal E}$-stable, i.e. it is a union of ${\mathcal E}$-orbits.
\end{thm}
\begin{proof}
Choose a closed embedding $X {\subseteq} V$  such that ${\mathcal E} {\subseteq} \operatorname{Mor}(X,V)$ is a linear subspace. Since the map $\operatorname{End}(V) \to \operatorname{Mor}(X,V)$ is linear and surjective  there is a linear subspace $\tilde{\mathcal E} {\subseteq}\operatorname{End}(V)$ whose image in $\operatorname{Mor}(X,V)$ is ${\mathcal E}$. In particular, $X$ is stable under $\tilde{\mathcal E}$ and so  ${\mathcal D}_{\tilde{\mathcal E}} {\subseteq} \operatorname{Vec}_{X}(V)$. The linearity of  $\operatorname{End}(V) \to \operatorname{Mor}(X,V)$ implies that the image of  ${\mathcal D}_{\tilde{\mathcal E}}$ under $\operatorname{Vec}_{X}(V) \to \operatorname{Vec}(X)$ is ${\mathcal D}_{\mathcal E}$, i.e. ${\mathcal D}_{\tilde{\mathcal E}}(x) = {\mathcal D}_{\mathcal E}(x)$ for all $x \in X$.

Now we apply  Lemma~\ref{main-lemma} to $\tilde {\mathcal E}$ and find that ${\mathcal E}(x) = \tilde{\mathcal E}(x) = M(x)$, hence (1) and (2). Moreover, $T_{x}{\mathcal E}(x) = T_{x}\tilde{\mathcal E}(x) = \tilde{\mathcal E}(x) = {\mathcal D}_{\tilde{\mathcal E}}(x) = {\mathcal D}_{\mathcal E}(x)$, hence (3). 

Finally, for a closed subvariety $Y {\subseteq} X {\subseteq} V$ the ${\mathcal D}_{\mathcal E}$-invariance is the same as the ${\mathcal D}_{\tilde{\mathcal E}}$-invariance, and $Y$ is stable under $\tilde E$ if and only if it is stable under ${\mathcal E}$. Hence the last claim follows also from the lemma.
\end{proof}

If ${\mathcal E} {\subseteq} \operatorname{End}(X)$ is a linear semigroup we define
$d_{\mathcal E}(X) := \max_{x\in X}\dim{\mathcal E}(x)$. By our theorem above we have $d_{\mathcal E}(X) = d_{{\mathcal D}_{\mathcal E}}(X)$. We shall denote this common value simply by $d(X)$.

\begin{cor}
Let $X$ be an irreducible variety and ${\mathcal E} {\subseteq} \operatorname{End}(X)$ a linear semigroup. Then  $X':=\{x\in X\mid \dim {\mathcal E}(x) = d(X)\}$ is open and dense in $X$, and the subsets ${\mathcal E}(x)\cap X'$ for $x \in X'$ form a partition of $X'$. 
\end{cor}
\begin{proof}
The first part is Lemma~\ref{lower-semi-cont.lem}. If $y \in {\mathcal E}(x)$, then ${\mathcal E}(y){\subseteq} {\mathcal E}(x)$. Since, by the theorem above,  the ${\mathcal E}(x)$ are vector spaces and $\dim {\mathcal E}(x) = \dim {\mathcal D}_{\mathcal E}(x)$, we have ${\mathcal E}(y) = {\mathcal E}(x)$ in case $y \in X'$. This proves the second claim.
\end{proof}

\begin{rem}
If an algebraic group $G$ acts on a variety $X$, then every element $A \in \operatorname{Lie} G$ defines a vector field $\xi_{A}$. It is known that for a connected group $G$ a closed subvariety $Y {\subseteq} X$ is $G$-stable if and only if $Y$ is $\xi_{A}$-invariant for all $A \in \operatorname{Lie} G$. A proof can be found in \cite[III.4.4, Corollary~4.4.7]{Kr2014Algebraic-Transfor}, and generalization to actions of connected ind-groups on affine varieties is given in \cite{FuKr2015On-the-geometry-of}. Our main theorem above shows that a similar statement holds for linear semigroups.
\end{rem}

{\par\medskip}
\section{$G$-symmetry}
\subsection{$G$-equivariant endomorphisms}
Now consider an action of an algebraic group $G$ on the affine variety $X$. Then the induced actions of $G$ on the coordinate ring $\operatorname{\mathcal O}(X)$ and on the vector fields $\operatorname{Vec}(X)$ are locally finite and rational, and the $G$-invariant vector fields ${\operatorname{Vec}_{G}}(X)$ form an $\operatorname{\mathcal O}(V)^{G}$-module. Note that the (linear) action of $G$ on $\operatorname{Vec}(X)$ is given by $g\xi:= dg \circ \xi \circ g^{-1}$ if we consider $\xi$ as a section of the tangent bundle. If we regard $\xi$ as a derivation $\delta$ of $\operatorname{\mathcal O}(X)$, then  $g\delta := (g^{*})^{-1}\circ \delta \circ g^{*}$ where $g^{*}\colon \operatorname{\mathcal O}(X) \to \operatorname{\mathcal O}(X)$ is the comorphism of $g \colon X \to X$.

The action of $G$ on $\operatorname{End}(X)$ by conjugation induces a linear action on the tangent space $T_\operatorname{id}\operatorname{End}(X)$ which we denote by $g\mapsto \operatorname{Ad} g$. It follows that the canonical map $\Xi\colon T_\operatorname{id}\operatorname{End}(X) {\hookrightarrow} \operatorname{Vec}(X)$ is $G$-equivariant. In fact, one has the formula
$$
\xi_{\operatorname{Ad} g (A)}(gx) = dg \, \xi_{A}(x) \text{ for } A \in T_\operatorname{id}\operatorname{End}(X), \ g \in  G \text{ and }  x \in X.
$$
This proves the first part of the following lemma.

\begin{lem}\label{G-invariantVF.lem}
We have $\xi(T_\operatorname{id}\operatorname{End}_{G}(X)) \subseteq \operatorname{Vec}_{G}(X)$ with equality if $X$ is a vector space $V$ with a linear action of $G$.
\end{lem}
\begin{proof}
It remains to see that for a linear action of $G$ on the vector space $V$ we have $T_\operatorname{id}\operatorname{End}_{G}(V) = (T_\operatorname{id}\operatorname{End}(V))^{G}$. But this is clear, because  $\operatorname{End}(V)$ is a vector space, $\operatorname{End}_{G}(V) = \operatorname{End}(V)^{G}$  is a linear subspace, and $\Xi\colon T_\operatorname{id}\operatorname{End}(V) {\xrightarrow{\sim}} \operatorname{Vec}(V)$ is a $G$-equivariant linear isomorphism.
\end{proof}

{\par\smallskip}
\subsection{$G$-symmetric subvarieties}
We now come to the main notion of this paper, the {\it $G$-symmetry of subvarieties}. This was already discussed in the introduction.
\begin{defn}
Let $X$ be an affine variety with an action of an algebraic group $G$. A closed subvariety $Y \subseteq X$ is called {\it $G$-symmetric\/} if $Y$ is 
$\operatorname{Vec}_{G}(X)$-invariant, i.e., $Y$ is parallel to all $G$-invariant vector fields $\xi$.
\end{defn}

If $V$ is a vector space with a linear action of the algebraic group $G$, then $\operatorname{End}_{G}(V) {\subseteq} \operatorname{End}(V)$ is a linear subspace and, by Lemma~\ref{G-invariantVF.lem} above,  the image of $T_\operatorname{id}\operatorname{End}_{G}(V)$ in $\operatorname{Vec}(V)$ is the subspace $\operatorname{Vec}_{G}(V)$ of $G$-invariant vector fields. Hence Theorem~\ref{main-theorem-1} implies the following result.

\begin{thm}\label{main-theorem-2}
Let $V$ be a vector space with a linear action of an algebraic group $G$. Then a closed subvariety $X {\subseteq} V$ is $G$-symmetric if and only if it is stable under $\operatorname{End}_{G}(V)$.
\end{thm}

\begin{exa}
Let $V$ be a $G$-module, and assume that $V^{G}=\{0\}$. Define the {\it null cone}
$$
{\mathcal N}_{0}:= \{v \in V \mid f(v) = 0 \text{ for all }f \in \operatorname{\mathcal O}(V)^{G}\text{ such that }f(0)=0\}.
$$
Then ${\mathcal N}_{0}{\subseteq} V$ is a closed $G$-symmetric subvariety.
\begin{proof}
We have $\operatorname{\mathcal O}(V)={\Bbbk}\oplus {\mathfrak m}_{0}$ where ${\mathfrak m}_{0}$ is the maximal ideal of $0\in V$, and ${\mathcal N}_{0}$ is the zero set of ${\mathfrak m}_{0}^{G}$. Since $V^{G}$ is fixed under every $G$-equivariant endomorphism ${\varphi}$ of $V$, we get ${\varphi}^{*}({\mathfrak m}_{0}^{G}) {\subseteq} {\mathfrak m}_{0}^{G}$, and so ${\mathcal N}_{0}$ is stable under $\operatorname{End}_{G}(V)$. Now the claim follows from the theorem above.
\end{proof}
\end{exa}

{\par\smallskip}
\subsection{Stabilizers}
The next result deals with the relation between $G$-symmetric subvarieties and the $G$-action on $X$. We denote by $G_{x} \subseteq G$ the stabilizer of $x\in X$, and by $M(x)$ the minimal $\operatorname{Vec}_{G}(X)$-symmetric subvariety containing $x$ (Lemma~\ref{D-invariant.lem}(4)).

\begin{lem}
Let $X$ be an affine $G$-variety.
\begin{enumerate}
\item
If $Y {\subseteq} X$ is a $G$-symmetric closed subvariety, then $gY {\subseteq} X$ is $G$-symmetric for all $g \in G$. 
\item
For $x \in X$ we have $\operatorname{Vec}_{G}(X)(x) \subseteq (T_{x}X)^{G_{x}}$.
\item
For $x \in X$ and $g \in G$ we have  $gM(x)=M(gx)$, and so $gM(x)=M(x)$ for $g \in G_{x}$.
\end{enumerate}
\end{lem}
\begin{proof}
(1) If $\xi$ is a $G$-invariant vector field, then $dg \,\xi(x) = \xi(gx)$ for $x \in X$, $g\in G$. This shows that $\xi(y) \in T_{y}Y$ if and only if $\xi(gy) \in T_{gy}gY$, and the claim follows.
\par\smallskip
(2) The formula in (1) shows that for a $G$-invariant vector field $\xi$ we get $dg \,\xi(x) = \xi(x)$ for $g \in G_{x}$. Hence $\xi(x) \in (T_{x}X)^{G_{x}}$.
\par\smallskip
(3) This follows from the  minimality of $M(x)$.
\end{proof}

In case of a linear action of $G$ on a vector space $V$ we get the following result.
\begin{prop}
Let $V$ be a  $G$-module.
\begin{enumerate}
\item For every closed subgroup $H {\subseteq} G$ the fixed point set $V^{H}$ is $G$-symmetric.
\item For all $v \in V$ we have $M(v)=\operatorname{End}_{G}(V)(v) {\subseteq} V^{G_{v}}$.
\end{enumerate}
\end{prop}
\begin{proof}
(1) It is clear that $V^{H}$ is stable under all $G$-equivariant endomorphisms, and so the claim follows from Theorem~\ref{main-theorem-2}.
\par\smallskip
(2) By (1), $V^{G_{v}}$ is $G$-symmetric and contains $v$, hence  $M(v) {\subseteq} V^{G_{v}}$ by the minimality of $M(v)$.
\end{proof}
\begin{exa}\label{diagonal.exa}
Let $G\to\operatorname{GL}(V)$ be a diagonalizable representation of an algebraic group $G$. Then, for a generic $v\in V$, we have $\operatorname{End}_{G}(V)(v) = V$. In particular, $d_{\operatorname{End}_{G}(V)}(V)=\dim V$.

In fact, let $V = \bigoplus_{\chi\in \Omega} V_{\chi}$ be the decomposition into weight spaces where $\Omega {\subseteq} X(G)$ are those characters $\chi$ of $G$ such that $V_{\chi}:=\{v\in V \mid gv=\chi(g)\cdot v\}$ is nontrivial. Then $\operatorname{End}_{G}(V)$ contains ${\mathcal L}:=\bigoplus_{\chi\in\Omega}{\mathcal L}(V_{\chi})$ where ${\mathcal L}(W)$ denotes the linear endomorphisms of the vector space $W$.  It follows that for any $v = (v_{\chi})_{\chi\in\Omega}$ such that $v_{\chi}\neq 0$ for all $\chi\in\Omega$ we have ${\mathcal L}(V) = V$, thus the claim.
\end{exa}

{\par\smallskip}
\subsection{Reductive groups}\label{RedGr.subsec}
If $X$ is an affine $G$-variety and $Y {\subseteq} X$ a closed and $G$-stable subvariety, then  $\operatorname{Vec}_{Y}(X) {\subseteq} \operatorname{Vec}(X)$ is a $G$-submodule and the linear map $\rho\colon\operatorname{Vec}_Y(X) {\twoheadrightarrow} \operatorname{Vec}(Y)$ is $G$-equivariant and surjective. If $Y$ is also $G$-symmetric, then $\rho(\operatorname{Vec}_{G}(X)) {\subseteq} \operatorname{Vec}_{G}(Y)$. But this might be a strict inclusion, i.e., not every $G$-invariant vector field on $Y$ is obtained by restricting a $G$-invariant vector field from $X$ (see Example~\ref{non-liftable-VF.exa} in section 6). However, if $G$ is reductive, then we get $\rho(\operatorname{Vec}_{G}(X)) = \operatorname{Vec}_{G}(Y)$, because $\rho$ is surjective and thus maps $G$-invariants onto $G$-invariants. This gives the following result.

\begin{lem}\label{surjectivity-of-VecG.lem}
Let $V$ be a $G$-module and $X {\subseteq} V$ a closed $G$-stable and $G$-symmetric subvariety. If a closed subvariety $Y {\subseteq} X$ is $G$-symmetric with respect to the action of $G$ on $X$, then it is also $G$-symmetric with respect to the action on $V$. If $G$ is reductive, then the converse also holds.
\end{lem}

\begin{exa}
Consider the adjoint representation of ${\operatorname{GL}_{n}}=\operatorname{GL}_{n}({\Bbbk})$ on the matrices $\operatorname{M}_{n}=\operatorname{M}_{n}({\Bbbk})$. It follows from classical invariant theory that $\operatorname{End}_{\operatorname{GL}_{n}}(\operatorname{M}_{n})$ is a free module over the invariants $\operatorname{\mathcal O}(\operatorname{M}_{n})^{\operatorname{GL}_{n}}$, with basis $(p_{i}\colon A \mapsto A^{i}\mid i=0,\ldots,n-1)$. Note that $p_{0}$ is the constant map $A \mapsto E$. It follows that the minimal symmetric subspaces $M(A)$ are given by
$$
M(A) = \sum_{i=0}^{n-1}{\Bbbk} A^{i}.
$$
In particular, a closed subset $Y {\subseteq} V$ is $\operatorname{GL}_{n}$-symmetric if and only if, for any $A \in Y$, the vector space spanned by  all  powers $A^{0}=E, A, A^{2}, \ldots$ is contained in $Y$. Note that the minimal subsets $M(A) {\subseteq} \operatorname{M}_{n}$ are exactly the {\it commutative unitary subalgebras of $\operatorname{M}_{n}({\Bbbk})$} generated by one element.

Recall that a matrix $A$ is {\it regular\/} if its centralizer $({\operatorname{GL}_{n}})_{A}$ has dimension $n$ which is the minimal dimension of a centralizer.  Equivalently, the minimal polynomial of $A$ coincides with the characteristic polynomial of $A$. The following is known. 
\begin{enumerate}
\item
$A$ is regular if and only if $\dim M(A) = n$.
\item
For a regular matrix  $A$ one has $M(A) = (\operatorname{M}_{n})^{(\operatorname{GL}_{n})_{A}}$. 
\end{enumerate}
An example of a closed $G$-symmetric subvariety is the nilpotent cone ${\mathcal N}{\subseteq} M_{n}$ consisting of all nilpotent matrices. It is also known that for a nilpotent matrix $N$ all powers $N^{k}$ are contained in the closure of the conjugacy class $C(N)$ of $N$, as well as their linear combinations. (In fact, $N':=\sum_{k\geq0}a_{k}N^{k}$ is conjugate to $N$ if $a_{0}\neq 0$, because $\ker {N'}^{j}=\ker N^{j}$ for all $j$.)  Hence these closures $\overline{C(N)}$ are $G$-symmetric as well.
\end{exa}

In the example above we have $M(A) = (\operatorname{M}_{n})^{({\operatorname{GL}_{n}})_{A}}$ for a regular matrix $A$. This is an instance of the following general result which is due to {\textsc{{Panyushev}\/}} \cite[Theorem~1]{Pa2002On-covariants-of-r}. For the convenience of the reader we give a short proof.

\begin{prop}\label{Panyushev.prop}
Let $V$ be a $G$-module where $G$ is reductive. If the closure $\overline{Gv}$ of the orbit of $v$ is normal and if 
$\operatorname{codim}_{\overline{Gv}}(\overline{Gv}\setminus Gv) \geq 2$, then $M(v) = V^{G_{v}}$.
\end{prop}

\begin{proof}
The assumptions on the orbit closure imply that $\operatorname{\mathcal O}(\overline{Gv}) = \operatorname{\mathcal O}(Gv)$. Let $w \in V^{G_{v}}$. 
We will show that there is a $G$-equivariant morphism ${\varphi}\colon V \to V$ such that ${\varphi}(v)=w$. Since $G_{w}\supseteq G_{v}$ there is a $G$-equivariant morphism $\mu\colon Gv \to V$ such that $\mu(v)=w$. The comorphism has the form $\mu^{*}\colon \operatorname{\mathcal O}(V) \to \operatorname{\mathcal O}(Gv) = \operatorname{\mathcal O}(\overline{Gv})$, hence $\mu$ extends to a morphism $\tilde\mu\colon \overline{Gv} \to V$ which is again $G$-equivariant. Since $G$ is reductive and $\overline{Gv}{\subseteq} V$ closed and $G$-stable, the morphism $\tilde\mu$ extends to a $G$-equivariant morphism ${\varphi}\colon V \to V$ with ${\varphi}(v)=w$.
\end{proof}

{\par\smallskip}
\subsection{Dense orbits}
Let $X$ be an irreducible affine variety, and let ${\mathcal E} {\subseteq} \operatorname{End}(X)$ be a semi-subgroup. An interesting question is whether ${\mathcal E}$ has a dense orbit, i.e. whether there exists an $x \in X$ such that $\overline{{\mathcal E}(x)}=X$. 

\begin{lem}\label{DenseOrbit.lem}
Let ${\mathcal E} {\subseteq} \operatorname{End}(X)$ be a linear semigroup.
Then the following are equivalent.
\begin{enumerate}
\item[(i)] ${\mathcal E}$ has a dense orbit in $X$.
\item[(ii)] $d(X) = \dim X$.
\item[(iii)] There exists an $x \in X$ such that ${\mathcal E}(x)=X$.
\item[(iv)] One has ${\mathcal E}(x) = X$ for all $x$ in an open dense subset of $X$.
\end{enumerate}
If this holds, then $X$ is a vector space.
\end{lem}
\begin{proof}
If ${\mathcal E}$ is a linear semigroup, then ${\mathcal E}(v){\subseteq} V$ is a linear subspace and therefore closed in $X$. It is now clear that the first three statements are equivalent, and (iv) follows from (iii) and the last statement of Lemma~\ref{main-lemma}.
\end{proof}
\begin{prop}
Let $G$ be a reductive group, and let $V$ be a faithful $G$-module. 
\begin{enumerate}
\item
If the generic $G$-orbits in $V$ are closed with trivial stabilizer, then $\operatorname{End}_{G}(V)$ has a dense orbit in $V$, i.e.  
$d(V)=\dim V$.
\item
If $G$ is semisimple and $d(V)=\dim V$, then the generic $G$-orbits in $V$ are closed with trivial stabilizer.
\end{enumerate}
\end{prop}
\begin{proof} Set ${\mathcal E}:=\operatorname{End}_{G}(V)$.
\par\smallskip
(1) If the orbit $Gv$ is closed and $G_{v}$ trivial, then ${\mathcal E}(v) = V$ by Proposition~\ref{Panyushev.prop}.
\par\smallskip
(2) If $G$ is semisimple, then the generic $G$-orbits in $V$ are closed. If $d(V)=\dim V$, then, by the lemma above, we have ${\mathcal E}(v) = V$ for all $v$ from a dense open subset $U {\subseteq} V$. Since ${\mathcal E}(v) {\subseteq} V^{G_{v}}$ and since the action is faithful, we see that $G_{v}$ is trivial for all $v \in U$, i.e. the generic stabilizer is trivial.
\end{proof}
\begin{rem}
Example~\ref{diagonal.exa} shows that the assumption in (2) that $G$ is semisimple is necessary.
\end{rem}

{\par\medskip}
\section{First integrals}
\subsection{The field of first integrals}
Let $X$ be an irreducible affine variety, and let ${\mathcal D} {\subseteq} \operatorname{Vec}(X)$ be a linear subspace.
\begin{defn}
A {\it first integral of ${\mathcal D}$\/}  is a rational function $f \in {\Bbbk}(X)$ with the property that $\xi f = 0$ for all $\xi \in {\mathcal D}$. If $X$ is a $G$-variety and ${\mathcal D} := \operatorname{Vec}_{G}(X)$, then a first integral of ${\mathcal D}$ will be called a {\it first integral for the $G$-action on $X$}.
\end{defn}
It is easy to see that the first integrals of ${\mathcal D}$ form a subfield  of ${\Bbbk}(X)$ which we denote by ${\mathcal F}_{\mathcal D}(X)$. If ${\mathcal D} = \operatorname{Vec}_{G}(X)$, then we write ${\mathcal F}_{G}(X)$ instead of ${\mathcal F}_{\operatorname{Vec}_{G}(X)}(X)$.

{\par\smallskip}
From now on assume that $X$ is an irreducible affine variety, and that ${\mathcal D} {\subseteq} \operatorname{Vec}(X)$ is a linear subspace. 
We want to show that the first integrals are the rational functions on a certain ``quotient'' of the variety $X$.

\begin{lem}\label{first-integral.lem}
Let $f \in {\Bbbk}(X)$ be a rational function.
\begin{enumerate}
\item
Assume that there is an open dense $U {\subseteq} X$ where $f$ is defined and has the property that $f$ is constant on $M(x)\cap U$ for all $x \in U$. Then $f$ is a first integral of ${\mathcal D}$.
\item
Assume that $f$ is a first integral of ${\mathcal D}$. If $f$ is defined in $x \in X$ and if $T_{x}M(x) = {\mathcal D}(x)$, then $f$ is constant on $M(x)$.
\end{enumerate}
\end{lem}
\begin{proof}
(1) Since $M(x)$ is ${\mathcal D}$-invariant we have $\xi(x) \in T_{x}\,M(x)$ for all $x \in U$ and all $\xi \in {\mathcal D}$. Hence $(\xi f)(x) = \xi(x) f = \xi(x) f|_{M(x)\cap U} = 0$, because $f|_{M(x)\cap U}$ is constant, and so $\xi f = 0$ for all $\xi \in {\mathcal D}$.
\par\smallskip
(2) There is $d\geq 0$ such that $\dim {\mathcal D}(y)\leq d$ for all $y \in M(x)$, with equality on a dense open set $M' {\subseteq} M(x)$ (Lemma~\ref{lower-semi-cont.lem}). In particular, $\dim M(x) \leq \dim T_{x}M(x) = \dim {\mathcal D}_{x} \leq d$. On the other hand, ${\mathcal D}_{y}{\subseteq} T_{y}M(x)$ for all $y \in M(x)$. We can assume that $M'$ consists of smooth point of $M(x)$. Then, for every $y \in M'$, we get $d=\dim {\mathcal D}_{y}\leq \dim T_{y}M(x) = \dim M(x)$. Hence $d = \dim M(x)$, and so $T_{y}M(x)={\mathcal D}(y)$ for all $y\in M'$. 
Since $f$ is defined in $x$, it is defined in a dense open set $M''{\subseteq} M'$. But then $f|_{M''}$ is constant, because $\delta f=0$ for all $u \in M''$ and all $\delta \in T_{u}M(x)$.
\end{proof}
\begin{rem}\label{first-integral.rem}
If ${\mathcal E} {\subseteq} \operatorname{End}(X)$ is a linear semigroup and ${\mathcal D}:={\mathcal D}_{\mathcal E}$, then a rational function $f\in{\Bbbk}(X)$ defined on an open set $U {\subseteq} X$ is a first integral for ${\mathcal D}$ if and only if $f$ is constant on ${\mathcal E}(x)\cap U$ for all $x \in U$. This follows from the lemma above, because in this case we have ${\mathcal E}(x) = M(x)$ and  $T_{x}M(x) = {\mathcal D}(x)$ for all $x \in X$, by Theorem~\ref{main-theorem-1}.
\end{rem}

Now  choose a closed embedding $X{\subseteq} V$ into a vector space $V$. We know from Lemma~\ref{lower-semi-cont.lem} that $X':=\{x\in X \mid \dim {\mathcal D}(x)=d(X)\}$ is open and dense in $X$. Consider the map 
$$
\pi\colon X' \to \operatorname{Gr}_{d_{\mathcal D}(X)}(V) \text{ given by }\pi(x) := {\mathcal D}(x){\subseteq} T_{x}X {\subseteq} V.
$$
\begin{lem}
The map $\pi\colon X' \to \operatorname{Gr}_{d(X)}(V)$ is a morphism of varieties.
\end{lem}
\begin{proof}
We will use the Pl\"ucker-embedding $\operatorname{Gr}_{d}(V) {\hookrightarrow} {\mathbb P}(\bigwedge^{d}V)$, $d := d(X)$. For  $x \in X'$ choose $\xi_{1},\ldots,\xi_{d} \in {\mathcal D}$ such that $\xi_{1}(x),\ldots,\xi_{d}(x)$ is a basis of ${\mathcal D}(x)$. Then ${\mathcal D}(x)= \xi_{1}(x)\wedge \xi_{2}(x)\wedge\dots\wedge\xi_{d}(x)\in\bigwedge^{d}V$. It follows that  there is an open neighborhood $U_{x}{\subseteq} X'$ of $x$ such that $\xi_{1}(u),\ldots,\xi_{d}(u)$ is a basis of ${\mathcal D}(u)$ for all $u \in U_{x}$. Since $\pi(u) = [\xi_{1}(u)\wedge\cdots\wedge\xi_{d}(u)] \in {\mathbb P}(\bigwedge^{d}V)$ we see that $\pi|_{U_{x}}$ is a morphism, and  the claim follows.
\end{proof}

{\par\smallskip}
\subsection{The quotient mod ${\mathcal E}$}
Let ${\mathcal E} {\subseteq} \operatorname{End}(V)$ be a linear semigroup, and let ${\mathcal D}_{\mathcal E} {\subseteq} \operatorname{Vec}(V)$ denote the image of $T_\operatorname{id}{\mathcal E} = {\mathcal E}$. 
Let $X {\subseteq} V$ be a closed irreducible ${\mathcal E}$-stable subvariety.
Under these assumptions we have ${\mathcal E}(x)=M(x)={\mathcal D}_{\mathcal E}(x) {\subseteq} V$  for all $x \in X$ (Lemma~\ref{main-lemma}).
As above, define 
$$
X':=\{x\in X\mid \dim {\mathcal E}(x) = d(X)\},
$$
and consider the morphism  $\pi\colon X' \to \operatorname{Gr}_{d(X)}(V)$, $x\mapsto {\mathcal E}(x){\subseteq} T_{x}X {\subseteq} V$.

\begin{prop}\label{quotient-mod-E.prop}\begin{enumerate}
\item
For all $x \in X'$ we have $\pi^{-1}(\pi(x)) = {\mathcal E}(x)\cap X'$.
\item 
$\pi$ induces an isomorphism $\pi^{*}\colon {\Bbbk}(\overline{\pi(X')}) {\xrightarrow{\sim}} {\mathcal F}_{{\mathcal D}_{\mathcal E}}(X)$.
\item
We have $\operatorname{tdeg}_{\Bbbk}{\mathcal F}_{{\mathcal D}_{\mathcal E}}(X) = \dim X - d(X)= \dim \overline{\pi(X')}$. 
\item
${\mathcal F}_{{\mathcal D}_{\mathcal E}}(X) = {\Bbbk}$ if and only if $d(X) = \dim X$, and then $X {\subseteq} V$ is a linear subspace.
\end{enumerate}
\end{prop}
We will use the notion $X{/\!\!/} {\mathcal E}$ for the closure $\overline{\pi(X')}$ reflecting the fact that the map $\pi$ can be regarded as the ``quotient'' under the action of the semigroup ${\mathcal E}$ of endomorphisms.
\begin{proof}
(1) For $y\in {\mathcal E}(x)\cap X'$ we have ${\mathcal E}(y)={\mathcal E}(x)$, hence $\pi(y) = \pi(x)$. If $y \in X' \setminus {\mathcal E}(x)$, then ${\mathcal E}(y) \neq {\mathcal E}(x)$ and so $\pi(y)\neq\pi(x)$.
\par\smallskip
(2) By Remark~\ref{first-integral.rem} a rational function $f \in {\Bbbk}(X)$ defined on an open set $U{\subseteq} X'$ is a first integral if and only if it is constant on the subsets ${\mathcal E}(x)\cap U$ for all $x \in U$. We can assume that $\pi(U) {\subseteq} \operatorname{Gr}_{d(X)}(V)$ is locally closed and smooth and that $\pi\colon U \to \pi(U)$ is smooth.  Then it is a well-known fact that $\pi^{*}(\operatorname{\mathcal O}(\pi(U))) {\subseteq} \operatorname{\mathcal O}(U)$ are the regular functions on $U$ which are constant on the fibers.
\par\smallskip
(3) This is clear. 
{\par\smallskip}
(4) If $d(X) = \dim X$, then $X = {\mathcal E}(x)$ for a generic $x\in X$ (Lemma~\ref{DenseOrbit.lem}), and so $X$ is a linear subspace of $V$.
\end{proof}
\begin{cor}
If $X {\subseteq} V$ is not a linear subspace, then there exist non-constant first integrals.
\end{cor}
Note that if $X$ is smooth, then it is a linear subspace, because $X$ is a closed cone, see Remark~\ref{closed-cone.rem}.

\begin{exa}\label{kId.exa}
Let $X {\subseteq} V$ be a closed cone, and let ${\mathcal E}:={\Bbbk}\cdot\operatorname{id}_{V} {\subseteq} \operatorname{End}(V)$. Then ${\mathcal E}(x) = {\Bbbk} x$ for all $x \in X$, hence $X{/\!\!/} {\mathcal E} = {\mathbb P}(X)$ and ${\mathcal F}_{{\mathcal D}_{\mathcal E}}(X) = {\Bbbk}({\mathbb P}(X))$.
\end{exa}

{\par\smallskip}
\subsection{The symmetric case}
Assume that $V$ is a representation of an algebraic group and that ${\mathcal E} := \operatorname{End}_{G}(V)$, hence ${\mathcal D}_{\mathcal E} = \operatorname{Vec}_{G}(V)$. Then, for every $G$-stable and $G$-symmetric closed irreducible subvariety $X {\subseteq} V$, the open subset $X' {\subseteq} X$ is $G$-stable and the morphism $\pi\colon X' \to \operatorname{Gr}_{d(X)}(V)$ is $G$-equivariant. In particular, $\pi^{*}\colon {\Bbbk}(\overline{\pi(X')}) {\xrightarrow{\sim}} {\mathcal F}_{G}(X)$ is a $G$-equivariant isomorphism. It follows that for any $x\in X'$ we have
$$
G_{\pi(x)} = \operatorname{Norm}_{G}({\mathcal E}(x))
$$
where $\operatorname{Norm}_{G}(W)$ denotes the normalizer in $G$ of the subspace $W {\subseteq} V$. 

\begin{lem}\label{first-integrals.lem}
\begin{enumerate}
\item
For $x \in X'$ we have
$$
\operatorname{tdeg}{\mathcal F}_{G}(X) \geq \operatorname{tdeg}{\mathcal F}_{G}(X)^{G} + \dim G - \dim \operatorname{Norm}_{G}({\mathcal E}(x))
$$
with equality on a dense open set  $U {\subseteq} X'$.
\item
If ${\mathcal F}_{G}(X)^{G}={\Bbbk}$, then ${\mathcal F}_{G}(X)$ is $G$-isomorphic to ${\Bbbk}(G/\operatorname{Norm}_{G}({\mathcal E}(x))$ for any $x$ in a dense open set of $X'$.
\end{enumerate}
\end{lem}
\begin{proof}
(1) By {\textsc{{Rosenlicht}\/}}'s theorem (see \cite[Satz~2.2]{Sp1989Aktionen-reduktive}) there is an open dense $G$-stable subset $O {\subseteq} \pi(X')$ which admits a geometric quotient $q \colon O \to O/G$. In particular, the fibers of $q$  are $G$-orbits and have all the same dimension. Hence $\operatorname{tdeg}{\mathcal F}_{G}(X) = \dim O = \dim O/G + \dim Gu$ for $u \in O$, and ${\Bbbk}(O/G) = {\Bbbk}(O)^{G}={\mathcal F}_{G}(X)^{G}$. If $u = \pi(x)$, then $G_{u}= \operatorname{Norm}_{G}({\mathcal E}(x))$ and so
$$
\operatorname{tdeg}{\mathcal F}_{G}(X) = \operatorname{tdeg}{\mathcal F}_{G}(X)^{G} + \dim G - \dim \operatorname{Norm}_{G}({\mathcal E}(x))
$$
for all $x \in U:=\pi^{-1}(O)$. Since $\dim Gu$ is maximal for $u \in O$ the claim follows.
\par\smallskip
(2) If ${\mathcal F}_{G}(X)^{G}={\Bbbk}$, then, as a consequence of {\textsc{{Rosenlicht}\/}}'s theorem, $G$ has a dense orbit $Gu$ in $\overline{\pi(X')}$ and so ${\mathcal F}_{G}(X) = {\Bbbk}(Gu)$. If $u = \pi(x)$, then $Gu \simeq G/\operatorname{Norm}_{G}({\mathcal E}(x))$, and the claim follows.
\end{proof}

\begin{rem}
Note that ${\mathcal F}_{G}(X)^{G}={\Bbbk}$ if and only if $G {\mathcal E}(x)$ is dense in $X$ for a generic $x \in X$, or, equivalently, $\dim X =  d(X) + \dim G - \dim \operatorname{Norm}_{G}({\mathcal E}(x))$ for a generic $x \in X$.
\end{rem}

\begin{exa}\label{exa2}
Consider the adjoint representation of $\operatorname{GL}_{2}$ on $\operatorname{M}_{2}$. Then $\operatorname{M}_{2}' = \operatorname{M}_{2}\setminus {\Bbbk} I_{2}$ where $I_{2}=\left[\begin{smallmatrix}1&0\\0&1\end{smallmatrix}\right]$, and the morphism $\pi$ is equal to the composition
$$
\pi\colon \operatorname{M}_{2}' {\twoheadrightarrow} (\operatorname{M}_{2}/{\Bbbk} I_{2}) \setminus \{0\} {\twoheadrightarrow} {\mathbb P}(\operatorname{M}_{2}/{\Bbbk} I_{2}).
$$
Choosing the basis 
$\overline{\left[\begin{smallmatrix}0&1\\0&0\end{smallmatrix}\right]}$, 
$\overline{\left[\begin{smallmatrix}0&0\\1&0\end{smallmatrix}\right]}$, 
$\overline{\left[\begin{smallmatrix}1&\phantom{-}0\\0&-1\end{smallmatrix}\right]}$  of $\operatorname{M}_{2}/{\Bbbk} I_{2}$, the pullbacks of the coordinate functions are $b,c,\frac{a-d}{2}$, and so 
${\mathcal F}_{\operatorname{GL}_{2}}(\operatorname{M}_{2}) = {\Bbbk}(\frac{a-d}{b},\frac{a-d}{c})$.
\end{exa}
\begin{exa}
For the adjoint representation of $\operatorname{GL}_{n}$ on $\operatorname{M}_{n}$ we claim that $\operatorname{GL}_{n}$ has a dense orbit in $\pi(\operatorname{M}_{n}')$. In fact, let $S \in \operatorname{M}_{n}$ be a generic diagonal matrix. Then the span ${\mathcal E}(S)=\sum_{i=0}^{n-1}{\Bbbk} S^{i}$ has dimension $n$, hence it is the subspace of diagonal matrices, and so $\operatorname{GL}_{n}{\mathcal E}(S) {\subseteq} \operatorname{M}_{n}$ is the dense subset of all diagonalizable matrices. Moreover,  the normalizer of ${\mathcal E}(S)$ is equal to $N$, the normalizer of the diagonal torus $T {\subseteq} \operatorname{GL}_{n}$, and so ${\mathcal F}_{\operatorname{GL}_{n}}(\operatorname{M}_{n})\simeq {\Bbbk}(\operatorname{GL}_{n}/N)$.
\end{exa}

\begin{exa}
The previous example carries over to the adjoint representation of an arbitrary semisimple group $G$ on its Lie algebra ${\mathfrak g}:=\operatorname{Lie} G$. If $s \in {\mathfrak g}$ is a regular semisimple element, then the orbit $Gs$ is closed and the stabilizer of $s$ is a maximal torus $T$. This implies by Proposition~\ref{Panyushev.prop} that ${\mathcal E}(s) = {\mathfrak g}^{T}=\operatorname{Lie} T$ which is a toral subalgebra of ${\mathfrak g}$. Again, $G {\mathcal E}(s){\subseteq} {\mathfrak g}$ is the dense set of semisimple elements of ${\mathfrak g}$, and the normalizer of ${\mathcal E}(s)$ is equal to $N$, the normalizer of $T$ in $G$. Hence ${\mathcal F}_{G}(\operatorname{Lie} G) \simeq {\Bbbk}(G/N)$.
\end{exa}

In the examples above there are no $G$-invariant first integrals: ${\mathcal F}_{G}(X)^{G}={\Bbbk}$. This is not always the case as the next two examples show. However, it holds for a representation of a reductive group $G$ in case the generic fiber of the quotient map contains a dense orbit (Proposition~\ref{FGinvariantsAreTrivial.prop}).

\begin{exa} Let $U=\{{\left[\begin{smallmatrix}1&a&b\\&1&c\\&&1\end{smallmatrix}\right]}\mid a,b,c\in{\Bbbk}\}{\subseteq}\operatorname{GL}_{3}({\Bbbk})$ be the unipotent group of upper triangular matrices, and consider the adjoint representation of $U$ on its Lie algebra ${\mathfrak u}:=\operatorname{Lie} U =  \{{\left[\begin{smallmatrix}0&x&y\\&0&z\\&&0\end{smallmatrix}\right]}\mid x,y,z\in{\Bbbk}\}$. For $u={\left[\begin{smallmatrix}1&a&b\\&1&c\\&&1\end{smallmatrix}\right]} \in U$ and $v={\left[\begin{smallmatrix}0&x&y\\&0&z\\&&0\end{smallmatrix}\right]} \in {\mathfrak u}$ we find
\[\tag{$**$}
\operatorname{Ad}(u) v = u v u^{-1} = \begin{bmatrix} 0 & x & -c x + y + a z\\ &0&z\\&&0\end{bmatrix}
\]
which shows that the fixed points are ${\mathfrak u}^{U}={\Bbbk}{\left[\begin{smallmatrix}0&0&1\\&0&0\\&&0\end{smallmatrix}\right]}$ and the other orbits are the parallel lines $\operatorname{Ad}(U){\left[\begin{smallmatrix}0&x&y\\&0&z\\&&0\end{smallmatrix}\right]} ={\left[\begin{smallmatrix}0&x&0\\&0&z\\&&0\end{smallmatrix}\right]}+{\mathfrak u}^{U}$. It follows that the invariant ring is given by $\operatorname{\mathcal O}({\mathfrak u})^{U}={\Bbbk}[x,z]$. We have an exact sequence of $U$-modules
$$
0\to {\mathfrak u}^{U} {\hookrightarrow} {\mathfrak u} \overset{p}{\to} {\Bbbk}^{2}\to 0 \ \text{ where }\ p({\left[\begin{smallmatrix}0&x&y\\&0&z\\&&0\end{smallmatrix}\right]}):=(x,z).
$$
We claim that  the covariants ${\mathcal E}:=\operatorname{Cov}({\mathfrak u},{\mathfrak u})$ are generated as a $\operatorname{\mathcal O}({\mathfrak u})^{U}$-module by $\operatorname{id}_{\mathfrak u}$ and ${\varphi}_{0}\colon  {\left[\begin{smallmatrix}0&x&y\\&0&z\\&&0\end{smallmatrix}\right]} \mapsto {\left[\begin{smallmatrix}0&0&1\\&0&0\\&&0\end{smallmatrix}\right]}$.
This implies that ${\mathcal E}(v) = {\Bbbk} v +{\mathfrak u}^{U}$ for $v\in{\mathfrak u}\setminus {\mathfrak u}^{U}$, hence $d({\mathfrak u}) = 2$ and ${\mathfrak u}'={\mathfrak u}\smallsetminus {\mathfrak u}^{U}$. It follows that  
$$
{\mathfrak u}{/\!\!/} {\mathcal E} = {\mathbb P}({\mathfrak u}/{\mathfrak u}^{U}){\xrightarrow{\sim}} {\mathbb P}^{1}.
$$ 
In particular, the action of $U$ on the quotient is trivial, and so 
$$
{\mathcal F}_{U}({\mathfrak u}) = {\mathcal F}_{U}({\mathfrak u})^{U}={\Bbbk}(\frac{x}{z}).
$$
In order to prove the claim, let ${\varphi}\colon {\mathfrak u} \to {\mathfrak u}$ be a covariant, 
$$
{\varphi}({\left[\begin{smallmatrix}0&x&y\\&0&z\\&&0\end{smallmatrix}\right]})={\left[\begin{smallmatrix}0&{p(x,y,z)}&{q(x,y,z)}\\&0&{r(x,y,z)}\\&&0\end{smallmatrix}\right]}\ \text{ where }\ p,q,r \in \operatorname{\mathcal O}({\mathfrak u})={\Bbbk}[x,y,z].
$$
Then, by $(**)$, we get for $a,b,c\in {\Bbbk}$
\[\tag{1}
q(x,-c x + y + a z,z) = -c\cdot p(x,y,z)+q(x,y,z)+a\cdot r(x,y,z).
\]
This shows that $q$ is linear in $y$, i.e. $q(x,y,z) = q_{0}(x,z) + q_{1}(x,z)y$, and so
\[\tag{2}
q(x,-c x + y + a z,z) = q_{0}(x,y) + q_{1}(x,z)(-c x + y + a z) = q_{0} -c\cdot q_{1}x+ q_{1}y + a\cdot q_{1}z .
\]
Comparing (2) with (1) we get
\[\tag{3}
p= q_{1} x, \quad q = q_{0} + q_{1}y, \quad r = q_{1}z,
\]
hence ${\varphi} = q_{1}\operatorname{id}_{\mathfrak u}+ q_{0}{\varphi}_{0}$, as claimed.
\end{exa}

\begin{exa}
Let $G$ be a reductive group and $V$ an irreducible $G$-module. If the connected component of the center $Z(G)^{0}$ acts nontrivially, then $\operatorname{End}_{G}(V) ={\Bbbk} \operatorname{id}_{V}$. Hence, by Example~\ref{kId.exa}, we get $V{/\!\!/}\operatorname{End}_{G}(V) \simeq {\mathbb P}(V)$, ${\mathcal F}_{G}(V) = {\Bbbk}({\mathbb P}(V))$, and ${\mathcal F}_{G}(V)^{G} = {\Bbbk}({\mathbb P}(V{/\!\!/} (G,G)))$.
\newline
(In order to see that $\operatorname{End}_{G}(V) ={\Bbbk} \operatorname{id}_{V}$ we just remark that the $G$-module $V^{*}$ occurs only once in $\operatorname{\mathcal O}(V)$, namely in degree 1. In fact, $Z(G)^{0}$ acts on $V$ via a character $\chi$, and thus via $\chi^{-d}$ on the homogeneous functions $\operatorname{\mathcal O}(V)_{d}$ of degree $d$.)
\newline
\end{exa}
This example generalizes to the situation where $V$ is a reducible $G$-module such that the characters of $Z(G)^{0}$ on the irreducible components of $V$ are linearly independent. 

\begin{exa}\label{Omin.exa}
Let $V$ be an irreducible representation of a reductive group $G$. For the orbit ${O_{\text{\it min}}} {\subseteq} V$ of highest weight vectors we have
${\overline{O_{\text{\it min}}}} = {O_{\text{\it min}}} \cup \{0\}$, and ${\overline{O_{\text{\it min}}}}$ is normal with rational singularities (see \cite{He1979The-normality-of-c}). Clearly, ${\overline{O_{\text{\it min}}}}$ is $G$-symmetric, i.e. stable under all $G$-equivariant endomorphisms of $V$. We claim that ${\mathcal E}:=\operatorname{End}_{G}({\overline{O_{\text{\it min}}}}) = {\Bbbk}\cdot\operatorname{id}$. In fact, if $v \in V$ is a highest weight vector, then the $G$-orbit of $[v]\in{\mathbb P}(V)$ is closed, and thus the normalizer $P$ of $[v]$ is a parabolic subgroup. Hence $P$ is the normalizer of $G_{v}$ in $G$,  and so $P/G_{v}={\Bbbk^{*}}$. Since, $\operatorname{Aut}_{G}({\overline{O_{\text{\it min}}}})=\operatorname{Aut}_{G}({O_{\text{\it min}}})\simeq P/G_{v}={\Bbbk^{*}}$ the claim follows. 

As a consequence we get ${\overline{O_{\text{\it min}}}}'={O_{\text{\it min}}}$, ${\overline{O_{\text{\it min}}}}{/\!\!/} {\mathcal E} = {O_{\text{\it min}}}/{\Bbbk^{*}} = {\mathbb P}({\overline{O_{\text{\it min}}}}) {\subseteq} {\mathbb P}(V)$, and so ${\mathbb P}({\overline{O_{\text{\it min}}}})$ is the closed orbit of highest weight vectors in ${\mathbb P}(V)$. In particular, ${\mathcal F}_{G}({\overline{O_{\text{\it min}}}}) = {\Bbbk}({\mathbb P}({O_{\text{\it min}}}))$, and ${\mathcal F}_{G}({\overline{O_{\text{\it min}}}})^{G}={\Bbbk}$.
\end{exa}

{\par\smallskip}
\subsection{First integrals for reductive groups}
Let $G$ be a reductive group, and let $X$ be an irreducible $G$-variety. Denote by $q\colon X \to X{/\!\!/} G$ the quotient. Then {\textsc{{Luna}\/}}'s slice theorem (see \cite[pp.~97--98]{Lu1973Slices-etales}) implies the existence of a {\it principal isotropy group\/} $H {\subseteq} G$. This means the following:
\begin{enumerate}
\item If $Gx {\subseteq} X$ is a closed orbit, then $G_{x}$ contains a conjugate of $H$.
\item The set $(X{/\!\!/} G)_{\text{\it pr}}$ of points $\xi\in X{/\!\!/} G$ such that the closed orbit in the fiber $q^{-1}(\xi)$ is $G$-isomorphic to $G/H$ is open and dense in $X{/\!\!/} G$. 
\end{enumerate}
It follows that every closed orbit contains a fixed point of $H$, hence $\pi(X^{H}) = X {/\!\!/} G$. 
\par\smallskip
The open dense subset $(X{/\!\!/} G)_{\text{\rm pr}}$ of $X{/\!\!/} G$ is called the {\it principal stratum}, and the closed orbits over the principal stratum are the {\it principal orbits}. If the action on $X$ is {\it stable}, i.e. if the generic orbits of $X$ are closed, then the principal orbits are generic.

\begin{thm}\label{first-integrals.thm}
Let $G$ be reductive, $V$ a $G$-module, and let $X {\subseteq} V$ be a $G$-stable and $G$-symmetric irreducible closed subvariety. Assume that the generic orbit of $X$ is closed, with principal isotropy group $H {\subseteq} G$. Then ${\mathcal F}_{G}(X) = {\Bbbk}(G/N)$ where $N := \operatorname{Norm}_{G}(H)$. In particular, ${\mathcal F}_{G}(X)^{G}={\Bbbk}$.
\end{thm}
\begin{proof}
By assumption, the orbit $Gx$ is principal for a generic $x \in X^{H}$.  The minimal invariant subset $M(x) {\subseteq} X$ is also minimal invariant as a subset of $V$ (Lemma~\ref{surjectivity-of-VecG.lem}).  Hence,  $M(x) = V^{H}$ by Proposition~\ref{Panyushev.prop}. Since $M(x) {\subseteq} X^{H}{\subseteq} V^{H}$, we finally get $M(x) = X^{H}=V^{H}$. As we have seen above, $G X^{H}$ contains all closed orbits, and in particular all $M(y)$ for $y$ in the dense open set of principal orbits. This implies that $G$ has a dense orbit in $X{/\!\!/}\operatorname{End}_{G}(X)$. Since the stabilizer of the image $\pi(V^{H})$ is the normalizer $\operatorname{Norm}_{G}(V^{H})$, it remains to see $\operatorname{Norm}_{G}(V^{H})=\operatorname{Norm}_{G}(H)$. Since $g(V^{H}) = V^{gHg^{-1}}$  we get $V^{H}=V^{H\cap gHg^{-1}}$ for any $g\in\operatorname{Norm}(V^{H})$, hence $H = gHg^{-1}$, because the stabilizer of a generic elements from $V^{H}$ is $H$.  
\end{proof}
Since the generic orbits in a representation of a semisimple group are closed, we get the following consequence.
\begin{cor}
Let $V$ be a representation of a semisimple group $G$, and let  $H {\subseteq} G$ be the principal isotropy group. Then ${\mathcal F}_{G}(V) = {\Bbbk}(G/N)$ where $N := \operatorname{Norm}_{G}(H)$, and ${\mathcal F}_{G}(V)^{G}={\Bbbk}$. In particular, ${\mathcal F}_{G}(V) = {\Bbbk}$ if the principal isotropy group is trivial.
\end{cor}
Note that for a ``generic'' representation of a semisimple group $G$ the principal isotropy group is trivial, hence there are no nonconstant first integrals. The irreducible representations of simple groups with a nontrivial principal isotropy group have been classified (\cite{AnViEl1967Orbits-of-highest-}, cf. \cite[\S7]{ViPo1994Invariant-theory}).

The fact that there are no $G$-invariant first integrals is a consequence of the following slightly more general result.

\begin{prop}\label{FGinvariantsAreTrivial.prop}
Let $V$ be representation of a reductive group $G$. Assume that the generic fiber of the quotient map $q\colon V \to V {/\!\!/} G$ contains a dense orbit $O\simeq G/K$, or, equivalently, ${\Bbbk}(V)^{G}$ is the field of fractions of $\operatorname{\mathcal O}(V)^{G}$.  Then ${\mathcal F}_{G}(V) \simeq {\Bbbk}(G/\operatorname{Norm}_{G}(K))$ and ${\mathcal F}_{G}(V)^{G}={\Bbbk}$.
\end{prop}
\begin{proof}
Let $F$ be a fiber of the quotient map $q$ over the principal stratum, and let $O {\subseteq} F$ be the dense orbit. Consider the morphism $\pi\colon V' \to V{/\!\!/}\operatorname{End}_{G}(V) {\subseteq} \operatorname{Gr}_{d}(V)$, $d:=d_{\operatorname{End}_{G}(V)}(V)$. We claim that $O{\subseteq} V'$, that $\overline{\pi(O)} = \overline{\pi(V')}=V{/\!\!/}\operatorname{End}_{G}(V)$, and that the image of $O$ under $\pi$ is $G/\operatorname{Norm}_{G}(K)$. This proves the proposition.

{\textsc{{Luna}\/}}'s slice theorem tells us that the fibers of the quotient map over the principal stratum are $G$-isomorphic. This implies that $\operatorname{End}_{G}(V)$ acts transitively on the set of these fibers (see the argument in the proof of Proposition~\ref{Panyushev.prop}), hence $\overline{\pi(V')} = \overline{\pi(F\cap V')}$. Since  $F\cap V'$ is open and $G$-stable, we have $O {\subseteq} V'$. If ${\varphi} \in \operatorname{End}_{G}(V)$ and ${\varphi}(v) \in O$ for some $v \in O$, then ${\varphi}(O) = O$, and ${\varphi}|_{O}$ is a $G$-equivariant automorphism. Hence, $\pi(O)\simeq O/\operatorname{Aut}_{G}(O) \simeq G/\operatorname{Norm}_{G}(K)$, and the claims follow.
\end{proof}

{\par\medskip}
\section{Actions of ${\operatorname{SL}_{2}}$}\label{SLtwo.sec}
\subsection{Representations}
The standard representation of ${\operatorname{SL}_{2}}$ on $V := {\Bbbk}^{2}$ defines a linear action on the coordinate ring $\operatorname{\mathcal O}(V) = {\Bbbk}[x,y]$ given by $gf(v):=f(g^{-1} v)$. It is well-known that the homogeneous components $V_{d}:={\Bbbk}[x,y]_{d}$, $d=0,1,2,\ldots$, represent all irreducible representations of ${\operatorname{SL}_{2}}$, i.e. all simple ${\operatorname{SL}_{2}}$-modules. For these representations we have the following result.

\begin{prop} Define ${\mathcal E}_{d}:=\operatorname{End}_{\operatorname{SL}_{2}}(V_{d})$.
\begin{enumerate}
\item ${\mathcal E}_{1} = {\Bbbk}\operatorname{id}_{V_{1}}$, hence $d(V_{1}) = 1$. Moreover, we have $V'_{1}=V_{1}\setminus\{0\}$, $V_{1}{/\!\!/} {\mathcal E}_{1} = {\mathbb P}(V_{1})$ and ${\mathcal F}_{\operatorname{SL}_{2}}(V_{1})\simeq {\Bbbk}(\operatorname{SL}_{2}/B)$.
\item ${\mathcal E}_{2} = J\operatorname{id}_{V_{2}}$ where $J := \operatorname{\mathcal O}(V_{2})^{\operatorname{SL}_{2}}={\Bbbk}[D]$ is the ring of invariants where $D$ is the discriminant. Hence $d(V_{2}) = 1$,  $V_{2}' = V_{2}\setminus \{0\}$, $V_{2}{/\!\!/} {\mathcal E}_{2} = {\mathbb P}(V_{2})$ and ${\mathcal F}_{\operatorname{SL}_{2}}(V_{2}) {\xrightarrow{\sim}} {\Bbbk}({\operatorname{SL}_{2}}/N)$.
\item For $d\geq 3$, we have ${\mathcal E}_{d}(f) = V_{d}$ for a generic $f \in V_{d}$, hence $d(V_{d}) = \dim V_{d}$ and 
${\mathcal F}_{\operatorname{SL}_{2}}(V_{d})={\Bbbk}$.
\end{enumerate}
\end{prop}
\begin{proof}
(1) The coordinate ring $\operatorname{\mathcal O}(V_{1})$ contains every irreducible representation exactly once. Hence every covariant  of $V_{1}$ is a multiple of the identity, and so ${\mathcal E}(v) = {\Bbbk} v$ for all $v \in V_{1}$. Now the claims follow.
\par\smallskip
(2) This follows immediately from Example~\ref{exa2}.
\par\smallskip
(3) For $d\geq 3$ the stabilizer of a generic element $f\in V_{d}$ is trivial for $d$ odd and $\pm I_{2}$ for $d$ even. Hence, ${\mathcal E}(f) = V_{d}$ by Proposition~\ref{Panyushev.prop}, and the claims follow.
\end{proof}
\begin{rem}\label{covariant.rem}
An ${\operatorname{SL}_{2}}$-equivariant morphism ${\varphi}\colon V \to W$ between two ${\operatorname{SL}_{2}}$-modules is called a {\it covariant}. Every covariant is a sum of homogeneous covariants: 
$$
\operatorname{Cov}(V,W)=\bigoplus_{j\in{\mathbb N}}\operatorname{Cov}(V,W)_{j}.
$$ 
Moreover, $\operatorname{End}_{\operatorname{SL}_{2}}(V) = \operatorname{Cov}(V,V)$ is a module under $\operatorname{\mathcal O}(V)^{\operatorname{SL}_{2}}$ where the module structure is given by  $f{\varphi}(v) := f(v)\cdot{\varphi}(v)$.
\end{rem}

{\par\smallskip}
\subsection{The nullcone ${\mathcal N}(V)$}
A very interesting object in this setting is the {\it nullcone} ${\mathcal N}(V) {\subseteq} V$ of a representation $V$ of ${\operatorname{SL}_{2}}$ which is defined in the following way. Denote by $q \colon V \to V{/\!\!/}{\operatorname{SL}_{2}}$ the quotient morphism, i.e., $V{/\!\!/}{\operatorname{SL}_{2}} = \operatorname{Spec}\operatorname{\mathcal O}(V)^{\operatorname{SL}_{2}}$ and $q$ is induced by the inclusion $\operatorname{\mathcal O}(V)^{\operatorname{SL}_{2}} {\subseteq} \operatorname{\mathcal O}(V)$. Then ${\mathcal N}(V):=q^{-1}(q(0))$, or equivalently,  ${\mathcal N}(V)$ is the zero set of all homogeneous invariants of positive degree. In case $V = V_{d}$ the elements from ${\mathcal N}(V_{d})$ are classically called {\it nullforms}. One has the following description. Denote by $T {\subseteq} {\operatorname{SL}_{2}}$ the diagonal torus, and define the {\it weight spaces}
$$
V[i]:=\{f \in V \mid \left[\begin{smallmatrix} t & 0 \\ 0 & t^{-1}\end{smallmatrix}\right] f = t^{i}f \text{  for all }t \in {\Bbbk}^{*}\} \text{ \ for } i \in {\mathbb N}.
$$
Since the representation of $T$ is completely reducible we have $V = \bigoplus_{j} V[j]$. For $V = V_{d}$ we get $V_{d} = \bigoplus_{i=0}^{d} V_{d}[d-2i]$, and the weight spaces are one-dimensional.
Note that $\left[\begin{smallmatrix} t & 0 \\ 0 & t^{-1}\end{smallmatrix}\right] x = t^{-1}x$, and $\left[\begin{smallmatrix} t & 0 \\ 0 & t^{-1}\end{smallmatrix}\right] y = t y$, and so
$$
V_{d}[d-2i]={\Bbbk} x^{i}y^{d-i}.
$$
\begin{lem}
The following statements for a form $f \in V_{d}$ are equivalent.
\begin{enumerate}
\item[(i)] $f$ is a nullform, i.e. $f \in {\mathcal N}(V_{d})$.
\item[(ii)] There is a one-parameter subgroup $\lambda\colon {\Bbbk}^{*} \to {\operatorname{SL}_{2}}$ such that $\lim_{t\to 0}\lambda(t) f = 0$.
\item[(iii)] $f$ is in the ${\operatorname{SL}_{2}}$-orbit of an element from $V_{d}^{+}:=\bigoplus_{i>0} V_{d}[i] {\subseteq} V_{d}$.
\item[(iv)] $f$ contains a linear factor with multiplicity $> \frac{d}{2}$.
\end{enumerate}
\end{lem}
\begin{proof}
(a) The equivalence of (i) and (ii) is a consequence of the famous {\textsc{{Hilbert-Mumford}\/}}-Criterion and holds for any representation of a reductive group.
\par\smallskip
(b) (ii) and (iii) are equivalent, because every one-parameter subgroup of ${\operatorname{SL}_{2}}$ is conjugate to a one-parameter subgroup of $T$. This holds for any representation of ${\operatorname{SL}_{2}}$.
\par\smallskip
(c) The equivalence of (iii) and (iv) is clear, because $V_{d}^{+}$ are the forms which contain $y$ with multiplicity $>\frac{d}{2}$.
\end{proof}

Let $V$ be a representation of ${\operatorname{SL}_{2}}$.  If ${\varphi}\in \operatorname{End}_{\operatorname{SL}_{2}}(V)$ is homogeneous of degree $k$, then ${\varphi}(V[j]) {\subseteq} V[kj]$.  
It follows that 
${\varphi}(\bigoplus_{j\geq j_{0}}V[j]) {\subseteq} \bigoplus_{j\geq k j_{0}}V[j]$.
In particular, the  subspaces $\bigoplus_{j\geq j_{0}}V[j]$ are $G$-symmetric for any $j_{0}\geq 0$, because any endomorphism is a sum of homogeneous endomorphisms (Remark~\ref{covariant.rem}). Since every element $f \in {\mathcal N}(V)$  is ${\operatorname{SL}_{2}}$-equivalent to an element from $V^{+}:=\bigoplus_{j>0}V[j]$ it suffices to study the 
${\operatorname{SL}_{2}}$-symmetric subspaces of $V^{+}$. Note that such a subspace is $T$-stable. 

{\par\smallskip}
\subsection{Covariants}
We will now construct some special covariants for the binary forms $V_{d}$.
Let ${\varphi}\colon V_{d} \to \operatorname{End}(V_{d})$ and $\psi\colon V_{d} \to V_{d}$ be homogeneous covariants. Then we define covariants $\Phi_{s}=\Phi_{s}({\varphi},\psi)\in \operatorname{End}_{\operatorname{SL}_{2}}(V_{d})$ by
$$
\Phi_{s}({\varphi},\psi)f := {\varphi}(f)^{s}\psi(f)=({\varphi}(f)\circ{\varphi}(f)\circ\cdots\circ{\varphi}(f))(\psi(f)).
$$
This is a homogeneous covariant of degree $\deg\Phi_{s}= s\deg{\varphi} + \deg \psi$. 

Let $\operatorname{{\mathfrak sl}_{2}} := \operatorname{Lie} {\operatorname{SL}_{2}}$ be the Lie algebra of ${\operatorname{SL}_{2}}$ which acts on a representation  $V$ of ${\operatorname{SL}_{2}}$ by the adjoint representation $\operatorname{ad}\colon \operatorname{{\mathfrak sl}_{2}} \to \operatorname{End}(V)$. As an ${\operatorname{SL}_{2}}$-module we have $\operatorname{{\mathfrak sl}_{2}} {\xrightarrow{\sim}} V_{2}$, and $\operatorname{{\mathfrak sl}_{2}}[2] = {\Bbbk}{\left[\begin{smallmatrix} 0 & 1 \\ 0 & 0\end{smallmatrix}\right]}$.

\begin{lem}\label{covariants.lem}
Let $V_{d}$ denote the binary forms of degree $d$, considered as a representation of ${\operatorname{SL}_{2}}$.
\begin{enumerate}
\item If $d$ is odd, then there is a quadratic covariant ${\varphi}_{0}\colon V_{d}\to\operatorname{{\mathfrak sl}_{2}}$ such that
${\varphi}_{0}(V_{d}[1]) = \operatorname{{\mathfrak sl}_{2}}[2]={\Bbbk}{\left[\begin{smallmatrix} 0 & 1 \\ 0 & 0\end{smallmatrix}\right]}$.
\item If $d$ is even, then there is a quadratic covariant ${\varphi}_{0}\colon V_{d} \to \operatorname{{\mathfrak sl}_{2}} \otimes \operatorname{{\mathfrak sl}_{2}}$ such that ${\varphi}_{0}(V_{d}[2]) = \operatorname{{\mathfrak sl}_{2}}[2]\otimes\operatorname{{\mathfrak sl}_{2}}[2]$.
\item  If $d\equiv 0\mod 4$, then there is a quadratic covariant $\psi\colon V_{d} \to V_{d}$ such that $\psi(V_{d}[2]) =V_{d}[4]$.
\item If  $d\equiv 2\mod 4$ and $d\geq 10$, then there is a homogeneous covariant $\psi\colon V_{d} \to V_{d}$ of degree 4 such that $\psi(V_{d}[2]) = V_{d}[8]$, and there is no quadratic covariant.
\end{enumerate}
\end{lem}

For the proof let us recall the {\textsc{{Clebsch-Gordan}\/}}-decomposition of the tensor product $V_{d}\otimes V_{e}$ as an ${\operatorname{SL}_{2}}$-module where we assume that $d \geq e$:
$$
V_{d}\otimes V_{e} \simeq \bigoplus_{r=0}^{e} V_{d+e-2r}.
$$
The projection $\tau_{r}\colon V_{d}\otimes V_{e} \to V_{d+e-2r}$ is classically called the {\it $r$th transvection}, and it is given by the following formula:
\[\tag{T1}\label{transvec1.form}
f\otimes h \mapsto (f,h)_{r} := \sum_{i=0}^{r}(-1)^{i}\binom{r}{i} \frac{\partial^{r}f}{\partial x^{r-i}
\partial y^{i}} \frac{\partial^{r}h}{\partial x^{i}\partial y^{r-i}}.
\]
The second symmetric power $S^{2}(V_{d})$ has the decomposition 
$$
S^{2}(V_{d}) {\xrightarrow{\sim}} V_{2d}\oplus V_{2d-4}\oplus V_{2d-8}\oplus \cdots,
$$
and thus the quadratic covariants $f\mapsto (f,f)_{r}$ are non-zero only for even $r$, and are given by
\[\tag{T2}\label{transvec2.form}
(f,f)_{r} := \sum_{i=0}^{r}(-1)^{i}\binom{r}{i} \frac{\partial^{r}f}{\partial x^{r-i}
\partial y^{i}} \frac{\partial^{r}f}{\partial x^{i}\partial y^{r-i}} \in V_{2d-2r}.
\]
\begin{lem}\label{nonzero-transvection.lem}
For $d =2m$ and an even $r < d$, the transvection
$$
(x^{m-1}y^{m+1},x^{m-1}y^{m+1})_{r} \in V_{2d-2r}
$$
is a nonzero multiple of $x^{d-r-2}y^{d-r+2}$.
\end{lem}

\begin{proof}
For $r < d=2m$ we have $(x^{m-1}y^{m+1},x^{m-1}y^{m+1})_{r} = c_{m,r} x^{d-r-2}y^{d-r+2}$, 
\[\tag{$*$}
c_{m,r}=\sum_{i=r-m+1}^{m-1} (-1)^{i}\binom{r}{i} (m-r+i)_{r-i}(m-i+2)_{i}(m-i)_{i}(m-r+i+2)_{r-i}
\]
where $(x)_{n}:= x(x+1)(x+2)\cdots (x+n-1)$. Using Mathematica \cite{Re2016Mathematica}, we find
{\small
$$
c_{m,r} = (-1)^{m+r-1}4^{m-1}\frac{\Gamma(m)\Gamma(m+2)\Gamma(2m-\frac{r}{2}+1)\Gamma(\frac{r+1}{2})}{\Gamma(2m-r-1)\Gamma(2m-r+1)\Gamma(m-\frac{r}{2}+2)\Gamma(-m+\frac{r+3}{2})}.
$$}
Using some standard functional equations for $\Gamma$ we get for $m,s\in{\mathbb N}$, $m\geq 1$ and $0\leq r=2s<d=2m$:
\[\tag{$**$}
\begin{split}
c_{m,2s} &=(-1)^{s} \frac{(2s)!(m-1)!(m+1)!(2m-s)!}{s!(m-s-1)!(m-s+1)!(2m-2s)!} = \\
&=(-1)^{s}(2s)!(s!)^{2}\binom{m-1}{s}\binom{m+1}{s}\binom{2m-s}{s} \neq 0
\end{split}
\]
\end{proof}

\begin{rem} With a similar computation one shows that for $d=2m+1$ and an even $r<d$ the transvection $(x^{m}y^{m+1},x^{m}y^{m+1})_{r}$ is a nonzero multiple of $x^{2m-r}y^{2m-r+2}$. But this will not be used in the following.
\end{rem}

\begin{proof}[Proof of Lemma~\ref{covariants.lem}]
(a) If $d = 2m+1$, then the $2m$-th transvection is a covariant $\tau_{2m}\colon V_{d} \to V_{2}\simeq \operatorname{{\mathfrak sl}_{2}}$, and  $\tau_{2m}(x^{m}y^{m+1})$ is a non-zero multiple of $y^{2}$. In fact, for $r=2m$, the sum (\ref{transvec2.form}) has a single term, namely for $i=m$. This proves (1).
\par\smallskip
(b) Now assume that $d$ is even, $d = 2m$. Then the transvection $\tau_{2m-2}\colon V_{d} \to V_{4}$ has the property that $\tau_{2m-2}(x^{m-1}y^{m+1})$ is a non-zero multiple of $y^{4}$. In fact, the sum (\ref{transvec2.form}) has a single term, namely for $i=m-1$. Since  $\operatorname{{\mathfrak sl}_{2}} \otimes \operatorname{{\mathfrak sl}_{2}} \simeq V_{0}\oplus V_{2}\oplus V_{4}$ and $(\operatorname{{\mathfrak sl}_{2}}\otimes\operatorname{{\mathfrak sl}_{2}})[4] = \operatorname{{\mathfrak sl}_{2}}[2]\otimes \operatorname{{\mathfrak sl}_{2}}[2]$, we thus get a covariant ${\varphi}_{0}\colon V_{d}\to \operatorname{{\mathfrak sl}_{2}}\otimes\operatorname{{\mathfrak sl}_{2}}$ as claimed in (2).
\par\smallskip
(c) If $m$ is even, then, again by Lemma~\ref{nonzero-transvection.lem}, $\psi:=\tau_{m}\colon V_{d} \to V_{d}$ is a quadratic covariant such that $\psi(V_{d}[2])=V_{d}[4]$, proving (3). Here we only use that $c_{m,m}\neq 0$.
\par\smallskip
(d) Finally, if $m$ is odd, $m=2k+1$, there is no quadratic covariant, because $V_{d}$ does not appear in the decomposition of $S^{2}(V_{d})$. But, for $m\geq 5$, there is a homogeneous covariant $\psi$ of degree 4 with the required property. For even $k$ we take 
$$
\psi\colon V_{d} \to V_{d}, \ f\mapsto ((f,f)_{3k},(f,f)_{3k+2})_{1},
$$
and for odd $k$
$$
\psi\colon V_{d} \to V_{d}, \ f\mapsto ((f,f)_{3k-1},(f,f)_{3k+3})_{1}.
$$
By Lemma~\ref{nonzero-transvection.lem}, $(x^{m-1}y^{m+1},x^{m-1}y^{m+1})_{r}$ is a nonzero multiple of $x^{2m-r-1}y^{2m-r+1}$ for even $r$ such that  $0\leq r\leq m-1$. It remains to see that the  transvections $(x^{k}y^{k+4},x^{k-2}y^{k+2})_{1}$ for even $k$ and $(x^{k+1}y^{k+5},x^{k-3}y^{k+1})_{1}$  for odd $k$ are nonzero.  This follows from the transvection formula~(\ref{transvec1.form}) above which gives
\begin{gather*}
(x^{k}y^{k+4},x^{k-2}y^{k+2})_{1}= 8\cdot x^{2k-3}y^{2k+5} = 8 \cdot x^{m-4}y^{m+4},\\
(x^{k+1}y^{k+5},x^{k-3}y^{k+1})_{1}= 16\cdot x^{2k-3}y^{2k+5} = 16 \cdot x^{m-4}y^{m+4}.
\end{gather*}
This proves (4).
\end{proof}

\begin{rem}\label{Konvalinka.rem}
Each summand in the expression $(*)$ for $c_{m,r}$ can be rewritten as a certain multiple of a product of three binomial coefficients, and one obtains
\[
c_{m,r}= \frac{(m-1)!\,(m+1)!\,r!}{(2m-r)!}\sum_{i=r-m+1}^{m-1}(-1)^{i} \binom{m-1}{r-i}\binom{m + 1}{ i}\binom{2 m - r}{ m - i - 1}.
\]
In case of $r=m$ we find
\[\tag{$*\!*\!*$}
c_{m,m}= \frac{(m+1)!^{2}}{m^{2}}\;\sum_{i=1}^{m-1}(-1)^{i} \binom{m}{i-1}\binom{m}{ i}\binom{m}{i+1}.
\]
{\textsc{{Konvalinka}\/}} has shown (see \cite[formula~1.1]{Ko2008An-inverse-matrix-}) that this alternating sum is nonzero  for even $m$. This implies that we have a rigorous proof for the first three statements of Lemma~\ref{covariants.lem}, because only in part (d) of the proof we used the non-vanishing of a general $c_{m,r}$.
\end{rem}
{\par\smallskip}
\subsection{Symmetric subspaces of the nullforms}
We will now determine the minimal ${\operatorname{SL}_{2}}$-symmetric subspaces of the nullforms ${\mathcal N}(V_{d})$ and calculate the first integrals.

\begin{thm}\label{thm3}
Let $d =2m+1$ be odd, $d\geq 3$.
\begin{enumerate}
\item $d({\mathcal N}(V_{d})) = m$.
\item $V_{d}^{+}$ is a minimal ${\operatorname{SL}_{2}}$-symmetric subspace of ${\mathcal N}(V_{d})$ of dimension $m$.
\item If $M {\subseteq} {\mathcal N}(V_{d})$ is a minimal ${\operatorname{SL}_{2}}$-symmetric subspace of dimension $m$, then $M = gV_{d}^{+}$ for some $g \in{\operatorname{SL}_{2}}$.
\item ${\mathcal N}(V_{d}){/\!\!/} \operatorname{End}_{\operatorname{SL}_{2}}(V_{d}) \simeq {\operatorname{SL}_{2}}/B \simeq {\mathbb P}^{1}$.
\item ${\mathcal F}_{\operatorname{SL}_{2}}({\mathcal N}(V_{d})) \simeq {\Bbbk}({\operatorname{SL}_{2}}/B)$, in particular ${\mathcal F}_{\operatorname{SL}_{2}}({\mathcal N}(V_{d}))^{\operatorname{SL}_{2}} ={\Bbbk}$.
\end{enumerate}
\end{thm}
\begin{proof}
(a) Consider the covariants $\Phi_{s}({\varphi},\operatorname{id})\colon V_{d}\to V_{d}$ defined above where ${\varphi}$ is the composition
$$
\begin{CD}
{\varphi} \colon V_{d} @>{{\varphi}_{0}}>> \operatorname{{\mathfrak sl}_{2}} @>\operatorname{ad}>> \operatorname{End}(V_{d})
\end{CD}
$$
and ${\varphi}_{0}\colon V_{d}\to \operatorname{{\mathfrak sl}_{2}}$ is from Lemma~\ref{covariants.lem}(1). By construction, we get
$$
\Phi_{s}(V_{d}[1]) = \operatorname{ad}{\left[\begin{smallmatrix} 0 & 1 \\ 0 & 0\end{smallmatrix}\right]}^{s}V_{d}[1] = V_{d}[2s+1].
$$
This shows that $\operatorname{End}_{\operatorname{SL}_{2}}(V_{d}) (V_{d}[1] )= V_{d}^{+}$, hence (1) and (2).
\par\smallskip
(b) Let  $M = M(f)$ be of dimension $m$. There is a $g\in{\operatorname{SL}_{2}}$ such that $gf \in V_{d}^{+}$, hence $gM(f) = M(gf) \subseteq V_{d}^{+}$. Since $\dim M(f) = m$ we get $gM(f)  = V_{d}^{+}$. This gives (3) and shows that ${\operatorname{SL}_{2}}$ acts transitively on the subspaces $M(f){\subseteq} {\mathcal N}(V_{d})$ of dimension $m$ and thus on the image of $\pi\colon {\mathcal N}(V_{d}) \to \operatorname{Gr}_{m}(V_{d})$. Since the normalizer of $V_{d}^{+}$ is $B$, we finally get (4) and (5).
\end{proof}

\begin{thm}\label{d=4m.thm}
Let $d = 2m$ and $m$ even.
\begin{enumerate}
\item $d({\mathcal N}(V_{d})) = m$.
\item $V_{d}^{+}$ is a minimal ${\operatorname{SL}_{2}}$-symmetric subspace of ${\mathcal N}(V_{d})$ of dimension $m$.
\item If $M {\subseteq} {\mathcal N}(V_{d})$ is a minimal ${\operatorname{SL}_{2}}$-symmetric subspace of dimension $m$, then $M = gV_{d}^{+}$ for some $g \in{\operatorname{SL}_{2}}$.
\item ${\mathcal N}(V_{d}){/\!\!/} \operatorname{End}_{\operatorname{SL}_{2}}(V_{d}) \simeq {\operatorname{SL}_{2}}/B \simeq {\mathbb P}^{1}$.
\item ${\mathcal F}_{\operatorname{SL}_{2}}({\mathcal N}(V_{d})) \simeq {\Bbbk}({\operatorname{SL}_{2}}/B)$, in particular ${\mathcal F}_{\operatorname{SL}_{2}}({\mathcal N}(V_{d}))^{\operatorname{SL}_{2}} ={\Bbbk}$.
\end{enumerate}
\end{thm}
\begin{proof}
Define the following covariant
$$
\begin{CD}
{\varphi} \colon V_{d} @>{{\varphi}_{0}}>> \operatorname{{\mathfrak sl}_{2}}\otimes\operatorname{{\mathfrak sl}_{2}} @>{\alpha}>> \operatorname{End}(V_{d})
\end{CD}
$$
where ${\varphi}_{0}$ is from Lemma~\ref{covariants.lem}(2), and $\alpha$ is the linear ${\operatorname{SL}_{2}}$-equivariant map $A\otimes B \mapsto \operatorname{ad} A \circ \operatorname{ad} B$. Then the covariants $\Phi_{s}({\varphi},\operatorname{id})\colon V_{d}\to V_{d}$ satisfy
$\Phi_{s}(V_{d}[2]) = (\operatorname{ad}{\left[\begin{smallmatrix} 0 & 1 \\ 0 & 0\end{smallmatrix}\right]})^{2s} V_{d}[2]=V_{d}[4s + 2]$, and for the covariants $\Phi_{s}({\varphi},\psi)$ where $\psi$ is from Lemma~\ref{covariants.lem}(3) we get $\Phi_{s}(V_{d}[2]) = \operatorname{ad}{\left[\begin{smallmatrix} 0 & 1 \\ 0 & 0\end{smallmatrix}\right]}^{2s}V_{d}[4] = V_{d}[4s+4]$. As a consequence, we get  $\operatorname{End}_{\operatorname{SL}_{2}}(V_{d})(V_{d}[2]) = V_{d}^{+}$, hence (1) and (2). The remaining claims follow as in the proof of Theorem~\ref{thm3}.
\end{proof}

If $d = 2m$ and $m$ odd we define $V_{d}^{++}:=V_{d}[2] \oplus V_{d}[6] \oplus V_{d}[8] \oplus \cdots$.
\begin{thm} Let $d =2m$ and $m$ odd, $m\geq 3$.
\begin{enumerate}
\item $d({\mathcal N}(V_{d})) = m-1$.
\item $V_{d}^{++}$ is a minimal ${\operatorname{SL}_{2}}$-symmetric subspace of ${\mathcal N}(V_{d})$ of dimension $m-1$.
\item If $M {\subseteq} {\mathcal N}(V_{d})$ is a minimal ${\operatorname{SL}_{2}}$-symmetric subspace of dimension $m-1$, then $M = gV_{d}^{++}$ for some $g \in{\operatorname{SL}_{2}}$.
\item ${\mathcal N}(V_{d}){/\!\!/} \operatorname{End}_{\operatorname{SL}_{2}}(V_{d}) \simeq {\operatorname{SL}_{2}}/T$.
\item ${\mathcal F}_{\operatorname{SL}_{2}}({\mathcal N}(V_{d})) \simeq {\Bbbk}({\operatorname{SL}_{2}}/T)$, in particular ${\mathcal F}_{\operatorname{SL}_{2}}({\mathcal N}(V_{d}))^{\operatorname{SL}_{2}} ={\Bbbk}$.
\end{enumerate}
\end{thm}
\begin{proof}
(a) We first remark that there is no quadratic covariant ${\varphi}\colon V_{d}\to V_{d}$, and so $V_{d}^{++}$ is stable under ${\mathcal E}:=\operatorname{End}_{\operatorname{SL}_{2}}(V_{d})$. Now we use the covariants $\Phi_{s}({\varphi},\operatorname{id})$, as in the proof of the previous theorem, to show that ${\mathcal E} (V_{d}[2]) \supset V_{d}[4s+2]$. Moreover, the covariants $\Phi_{s}({\varphi},\psi)$ with $\psi$ from Lemma~\ref{covariants.lem}(4) imply that ${\mathcal E}(V_{d}[2]) \supset V_{d}[4s+8]$. It follows that ${\mathcal E} (V_{d}[2]) = V_{d}^{++}$, hence (1) and (2). 
\par\smallskip
(b) Using again that there are no quadratic covariants, we see that ${\mathcal E} (V_{d}[4]) \subseteq V_{d}[4] \oplus V_{d}[8] \oplus V_{d}[10]\oplus\cdots$, hence 
$\dim {\mathcal E} (V_{d}[4]) \leq m-2$. Therefore, $V_{d}^{++}$ is the only minimal ${\operatorname{SL}_{2}}$-symmetric subspace of $V_{d}^{+}$ of dimension $m-1$. Now the remaining claims follow as before, using that the normalizer of $V_{d}^{++}$ is $T$.
\end{proof}

\begin{rem} 
Part (4) of Lemma~\ref{covariants.lem} was only used in the proof of the last theorem.
Hence we have a rigorous proof of Theorem~\ref{thm3} and Theorem~\ref{d=4m.thm}, independent of the symbolic calculations done with Mathematica (see Remark~\ref{Konvalinka.rem}).
\end{rem} 
\begin{exa}
The minimal orbit $O_{0} {\subseteq} V_{d}$ is the orbit of $y^{d}$. Denote by $O_{1}$ the orbit of $xy^{d-1}$. Then $X:=\overline{O_{1}} = O_{1}\cup O_{0}\cup \{0\}$. We claim that $X$ is ${\operatorname{SL}_{2}}$-symmetric and that $\operatorname{End}_{\operatorname{SL}_{2}}(X) = {\Bbbk}\cdot\operatorname{id}$ in case $d \geq 5$. In fact, the image of $xy^{d-1}\in V_{d}^{+}$ under ${\varphi}\in\operatorname{End}_{\operatorname{SL}_{2}}(X)$ is again a weight vector of positive weight, hence a multiple of some $x^{\ell}y^{d-\ell}$ where  $\ell < d- \ell$. Since the stabilizer of $x^{\ell}y^{d-\ell}$ in ${\operatorname{SL}_{2}}$ is cyclic of order $d-2\ell$ for 
$\ell < d-\ell$, we see that ${\varphi}(xy^{d-1})$ is a multiple of $xy^{d-1}$ if $d>4$. (For $d=4$, $X$ is the nullcone ${\mathcal N}(V_{4})$, and the quadratic covariant ${\varphi}$ sends $O_{1}$ onto $O_{0}$, see Theorem~\ref{d=4m.thm}.) This implies that ${\varphi}|_{O_{1}} = \lambda\cdot \operatorname{id}$ for some $\lambda \in{\Bbbk}$, hence ${\varphi}|_{X} = \lambda\cdot \operatorname{id}$. As a consequence, $X'=X \setminus\{0\}$, and $X{/\!\!/} {\mathcal E} = {\mathbb P}(X) {\subseteq} {\mathbb P}(V_{d})$.
\end{exa}

\begin{exa}\label{non-liftable-VF.exa}
Let $d =2m$ be even and consider $V_{d}^{+}$ as a $B$-module. It is not difficult to see that there is always a $B$-covariant ${\varphi}$ of degree 2. E.g. for $d=6$ it is given by 
$$
{\varphi}(a_{1}\cdot x^{2}y^{4}+a_{2}\cdot xy^{5}+a_{3}\cdot y^{6}) =  2a_{1}^{2}\cdot xy^{5} + a_{1}a_{2}\cdot y^{6}.
$$
On the other hand, for $d=2m\geq 6$ and $m$ odd there is no ${\operatorname{SL}_{2}}$-covariant of $V_{d}$ of degree 2 (Lemma~\ref{covariants.lem}(4)).
Since $\operatorname{End}_{\operatorname{SL}_{2}}(V)=\operatorname{End}_{B}(V)$ for every ${\operatorname{SL}_{2}}$-module $V$, we see that the restriction map $\operatorname{End}_{B}(V_{d}) \to \operatorname{End}_{B}(V_{d}^{+})$ for $d\equiv 2\mod 4$ and $d\geq 6$ is not surjective.
\end{exa}
\addtocounter{section}{1}
\setcounter{subsection}{0}

{\par\medskip}
\section*{Appendix: Ind-varieties and ind-semigroups}
An introduction to ind-varieties and ind-groups can be found in {\textsc{{Kumar}\/}}'s book \cite[Chapter IV]{Ku2002Kac-Moody-groups-t}.
\subsection{Basic definitions} 
The following is borrowed from \cite{FuKr2015On-the-geometry-of}.

\begin{defn}\label{indvar.def}
An {\it ind-variety} ${\mathcal V}$ is a set together with an ascending filtration ${\mathcal V}_{0}{\subseteq} {\mathcal V}_{1}{\subseteq} {\mathcal V}_{2}{\subseteq} \cdots{\subseteq} {\mathcal V}$ such that the following holds:
\begin{enumerate}
\item ${\mathcal V} = \bigcup_{k \in {\mathbb N}}{\mathcal V}_{k}$;
\item Each ${\mathcal V}_{k}$ has the structure of an algebraic variety;
\item For all $k \in {\mathbb N}$ the inclusion  ${\mathcal V}_{k}{\hookrightarrow} {\mathcal V}_{k+1}$ is closed immersion of algebraic varieties.
\end{enumerate}
\end{defn}

A {\it morphism} between ind-varieties ${\mathcal V}$ and ${\mathcal W}$  is a map ${\varphi}\colon {\mathcal V} \to {\mathcal W}$  such that for any $k$ there is an $m$ such that ${\varphi}({\mathcal V}_{k}) {\subseteq} {\mathcal W}_{m}$ and that the induced map ${\mathcal V}_{k}\to {\mathcal W}_{m}$ is a morphism of varieties. {\it Isomorphisms} of ind-varieties are defined in the obvious way.

Two filtrations ${\mathcal V} = \bigcup_{k \in {\mathbb N}} {\mathcal V}_{k}$ and ${\mathcal V} = \bigcup_{k \in {\mathbb N}} {\mathcal V}_{k}'$ are called {\it equivalent\/} if for any $k$ there is an $m$ such that ${\mathcal V}_{k}{\subseteq} {\mathcal V}_{m}'$ is a closed subvariety as well as ${\mathcal V}_{k}'{\subseteq} {\mathcal V}_{m}$. Equivalently,  the identity map $\operatorname{id} \colon {\mathcal V} = \bigcup_{k \in {\mathbb N}} {\mathcal V}_{k} \to {\mathcal V} = \bigcup_{k \in {\mathbb N}} {\mathcal V}_{k}'$ is an isomorphism of ind-varieties.

\begin{defn} 
The {\it Zariski topology} of an ind-variety ${\mathcal V}=\bigcup_{k}{\mathcal V}_{k}$ is defined by declaring a subset $U {\subseteq} {\mathcal V}$ to be open if the intersections $U \cap{\mathcal V}_{k}$ are Zariski-open in ${\mathcal V}_{k}$ for all $k$. It is obvious that $A {\subseteq} {\mathcal V}$ is closed if and only if  $A \cap{\mathcal V}_{k}$ is Zariski-closed in ${\mathcal V}_{k}$ for all $k$. It follows that a locally closed subset ${\mathcal W} {\subseteq} {\mathcal V}$ has a natural structure of an ind-variety, given by the filtration ${\mathcal W}_{k}:={\mathcal W} \cap {\mathcal V}_{k}$ which are locally closed subvarieties of ${\mathcal V}_{k}$. These subsets are called {\it ind-subvarieties}.

A morphism ${\varphi}\colon {\mathcal V} \to {\mathcal W}$ is called an {\it immersion} if the image ${\varphi}({\mathcal V}) {\subseteq} {\mathcal W}$ is locally closed and ${\varphi}$ induces an isomorphism ${\mathcal V} {\xrightarrow{\sim}} {\varphi}({\mathcal V})$ of ind-varieties. An immersion ${\varphi}$ is called a {\it closed (resp. open) immersion} if ${\varphi}({\mathcal V}) {\subseteq} {\mathcal W}$ is closed (resp. open).
\end{defn}

\begin{defn}\label{affine-algebraic.def}\strut
\begin{enumerate}
\item An ind-variety ${\mathcal V}$ is called {\it affine} if it admits a filtration such that all ${\mathcal V}_{k}$ are affine. It follows that any filtration of ${\mathcal V}$ has this property. 
\item The {\it algebra of regular functions\/} on ${\mathcal V}=\bigcup {\mathcal V}_{k}$ is defined as 
$$
\operatorname{\mathcal O}({\mathcal V}):= \operatorname{Mor}({\mathcal V},{\AA^{1}}) = \varprojlim \operatorname{\mathcal O}({\mathcal V}_{k})
$$
It will always be regarded as a topological algebra with the obvious topology as an inverse  limit of finitely generated algebras. For any morphism ${\varphi}\colon {\mathcal V} \to {\mathcal W}$ the induced homomorphism ${\varphi}^{*}\colon\operatorname{\mathcal O}({\mathcal W}) \to \operatorname{\mathcal O}({\mathcal V})$ is continuous. Moreover, an affine ind-variety ${\mathcal V}$ is uniquely determined by the topological algebra $\operatorname{\mathcal O}({\mathcal V})$.
\item
The {\it Zariski tangent space\/} of an ind-variety ${\mathcal V}=\bigcup_{k}{\mathcal V}_{k}$ is defined in the obvious way:
$$
T_{v}{\mathcal V}:=\varinjlim T_{v}{\mathcal V}_{k}.
$$
If ${\mathcal V}$ is affine, then a tangent vector $A \in T_{v}{\mathcal V}$ is the same as a continuous derivation $A\colon\operatorname{\mathcal O}({\mathcal V})\to {\Bbbk}$ in $v$. It is clear that a morphism ${\varphi}\colon {\mathcal V} \to {\mathcal W}$ between two ind-varieties induces a linear map $d{\varphi}_{v}\colon T_{v}{\mathcal V} \to T_{{\varphi}(v)}{\mathcal W}$, {\it the differential of ${\varphi}$ in $v$}.
\item
The {\it product of two ind-varieties ${\mathcal V}=\bigcup_{k}{\mathcal V}_{k}$ and ${\mathcal W}=\bigcup_{j} {\mathcal W}_{j}$\/} is the ind-variety defined as ${\mathcal V}\times{\mathcal W}:=\bigcup_{k}{\mathcal V}_{k}\times {\mathcal W}_{k}$. It has the usual universal properties.
\item An ind-variety ${\mathcal V}$ is {\it curve-connected} if for every pair $v,w \in {\mathcal V}$ there is an irreducible algebraic curve $C$ and a morphism $\gamma\colon C \to {\mathcal V}$ such that $v,w \in\gamma(C)$. One can show that this is equivalent to the existence of a filtration ${\mathcal V}=\bigcup_{k}{\mathcal V}_{k}$ such that all ${\mathcal V}_{k}$ are irreducible (see \cite{FuKr2015On-the-geometry-of}).
\end{enumerate}
\end{defn}

Since products exist in the category of ind-varieties we can define ind-groups and ind-semigroups.

\begin{defn}
An {\it ind-group} ${\mathcal G}$ is an ind-variety with a group structure such that multiplication ${\mathcal G}\times{\mathcal G} \to {\mathcal G}$ and inverse ${\mathcal G} \to {\mathcal G}$ are morphisms. An {\it ind-semigroup ${\mathcal S}$} is defined in a similar way.
\end{defn}

An {\it action of an ind-group ${\mathcal G}$ on a variety $X$} is a homomorphism ${\mathcal G} \to \operatorname{Aut}(X)$ such that the induced map ${\mathcal G}\times X \to X$ is a morphism of ind-varieties. 
If $X$ is an affine variety, it is shown in \cite{FuKr2015On-the-geometry-of} that $\operatorname{End}(X)$ is an affine ind-semigroup and $\operatorname{Aut}(X)$ is an affine ind-group which is locally closed in $\operatorname{End}(X)$. It follows that an action of an ind-group ${\mathcal G}$ on $X$ is the same as a homomorphism of ind-groups ${\mathcal G} \to \operatorname{Aut}(X)$.

All this carries over to actions of ind-semigroups ${\mathcal S}$.

{\par\smallskip}
\subsection{Vector fields and Lie algebras}
A {\it vector field $\delta$} on an affine variety $X$ is a collection $\delta=(\delta(x))_{x\in X}$ of tangent vectors $\delta(x)\in T_{x}X$ such that, for all $f \in \operatorname{\mathcal O}(X)$, we have $\delta f \in \operatorname{\mathcal O}(X)$ where $(\delta f)(x):=\delta(x)f$. It follows that the vector fields $\operatorname{Vec}(X)$ can be identified with the derivations of $\operatorname{\mathcal O}(X)$ which we denote by $\operatorname{Der}(\operatorname{\mathcal O}(X))$.

The same definition can be used for an affine ind-variety ${\mathcal V}$, and one gets an identification of $\operatorname{Vec}({\mathcal V})$ with the {\it continuous\/} derivations ${\operatorname{Der}^{c}}(\operatorname{\mathcal O}({\mathcal V}))$. For an affine ind-group ${\mathcal G}$ one shows that the tangent space $T_{e}{\mathcal G}$ has a natural structure of a Lie algebra. It will be denoted by $\operatorname{Lie}{\mathcal G}$.

If ${\mathcal G}$ acts on the variety $X$ and $x \in X$ we denote by $\mu_{x}\colon {\mathcal G} \to X$ the orbit map $g\mapsto gx$.

\begin{prop}\label{vector-fields.prop}
Assume that an affine ind-group ${\mathcal G}$ acts on an affine variety $X$. For  $A \in \operatorname{Lie} {\mathcal G}$ and $x\in X$ define the tangent vector $\xi_{A}(x) \in T_{x}X$ to be the image of $A$ under $d\mu_{x}\colon \operatorname{Lie}{\mathcal G} \to T_{x}X$. Then $\xi_{A}$ is a vector field on $X$. The resulting linear map $\Xi\colon \operatorname{Lie}{\mathcal G} \to \operatorname{Vec}(X)$, $A \mapsto \xi_{A}$, is a anti-homomorphism of Lie algebras.
\end{prop}
\begin{proof}[Outline of Proof]
The action ${\varphi}\colon{\mathcal G}\times X \to X$ defines a homomorphism ${\varphi}^{*}\colon \operatorname{\mathcal O}(X) \to \operatorname{\mathcal O}({\mathcal G})\otimes\operatorname{\mathcal O}(X)$. Now consider the following derivation of $\operatorname{\mathcal O}(X)$:
$$
\begin{CD}
\delta\colon \operatorname{\mathcal O}(X) @>{{\varphi}^{*}}>> \operatorname{\mathcal O}({\mathcal G})\otimes\operatorname{\mathcal O}(X) @>{A\otimes\operatorname{id}}>> \operatorname{\mathcal O}(X).
\end{CD}
$$
An easy calculation shows that $(\delta f)(x) = A \mu_{x}^{*}(f)  = d\mu_{x}(A) f$, hence $\delta = \xi_{A}$.
\end{proof}

It is easy to see that this generalizes to the action of an affine ind-semigroup ${\mathcal E}$ on an affine variety $X$, $\mu\colon{\mathcal E} \to \operatorname{End}(X)$, and defines a linear map $\Xi\colon T_\operatorname{id}{\mathcal E} \to \operatorname{Vec}(X)$ whose image will be denoted by ${\mathcal D}_{\mathcal E}$ and called the {\it corresponding vector fields}. 

\bigskip

\begin{thebibliography}{GSW12}

\bibitem[AVE67]{AnViEl1967Orbits-of-highest-}
E.~M. Andreev, E.~B. Vinberg, and A.~G. Elashvili, \emph{Orbits of highest
  dimension of semisimple linear {L}ie groups}, Funkcional. Anal. i Prilozen.
  \textbf{1} (1967), no.~4, 3--7.
  
\bibitem[Dom08]{Do2008Covariants-and-the}
M{{\'a}}ty{{\'a}}s Domokos, \emph{Covariants and the no-name lemma}, J. Lie
  Theory \textbf{18} (2008), no.~4, 839--849.

\bibitem[Ell13]{El1913An-Introduction-to}
E.~B. Elliott, \emph{An {I}ntroduction to the {A}lgebra of {B}inary
  {Q}uantics}, Oxford University Press, 1913 (reprinted by Chelsea Publishing
  Company, 1964).

\bibitem[FK16]{FuKr2015On-the-geometry-of}
Jean-Philippe Furter and Hanspeter Kraft, \emph{On the geometry of automorphism
groups of affine varieties}, in preparation, 2015/16.

\bibitem[GSW12]{GrScWa2012Invariant-sets-for}
Frank~D. Grosshans, J{{\"u}}rgen Scheurle, and Sebastian Walcher,
  \emph{Invariant sets forced by symmetry}, J. Geom. Mech. \textbf{4} (2012),
  no.~3, 271--296.

\bibitem[Hes79]{He1979The-normality-of-c}
Wim Hesselink, \emph{The normality of closures of orbits in a {L}ie algebra},
  Comment. Math. Helv. \textbf{54} (1979), no.~1, 105--110.
  
  \bibitem[Kon08]{Ko2008An-inverse-matrix-}
Matja{\v{z}} Konvalinka, \emph{An inverse matrix formula in the right-quantum
  algebra}, Electron. J. Combin. \textbf{15} (2008), no.~1, Research Paper 23,
  19.
  
\bibitem[Kra14]{Kr2014Algebraic-Transfor}
Hanspeter Kraft, \emph{Algebraic {T}ransformation {G}roups: {A}n
{I}ntroduction}. Preliminary version from July 2014. ({\tt http://www.math.unibas.ch/kraft})

\bibitem[Kum02]{Ku2002Kac-Moody-groups-t}
Shrawan Kumar, \emph{Kac-{M}oody groups, their flag varieties and
  representation theory}, Progress in Mathematics, vol. 204, Birkh{\"a}user
  Boston Inc., Boston, MA, 2002.

\bibitem[LS99]{LeSp1999A-note-concerning-}
G.~I. Lehrer and T.~A. Springer, \emph{A note concerning fixed points of
  parabolic subgroups of unitary reflection groups}, Indag. Math. (N.S.)
  \textbf{10} (1999), no.~4, 549--553.

\bibitem[Lun73]{Lu1973Slices-etales}
Domingo Luna, \emph{Slices {\'e}tales}, Sur les groupes alg{\'e}briques, Soc.
  Math. France, Paris, 1973, pp.~81--105. Bull. Soc. Math. France, Paris,
  M{\'e}moire 33.

\bibitem[Pan02]{Pa2002On-covariants-of-r}
Dmitri~I. Panyushev, \emph{On covariants of reductive algebraic groups}, Indag.
Math. (N.S.) \textbf{13} (2002), no.~1, 125--129.

\bibitem[Res16]{Re2016Mathematica}
Wolfram Research, \emph{Mathematica}, version 10.4 ed., Wolfram Research, Inc.,
  Champaign, Illinois, 2016.

\bibitem[Spr89]{Sp1989Aktionen-reduktive}
Tonny~A. Springer, \emph{Aktionen reduktiver {G}ruppen auf {V}ariet{\"a}ten},
Algebraische {T}ransformationsgruppen und {I}nvariantentheorie, DMV Sem.,
vol.~13, Birkh{\"a}user, Basel, 1989, pp.~3--39.

\bibitem[VP94]{ViPo1994Invariant-theory}
E.~B. Vinberg and V.~L. Popov, \emph{Invariant theory}, Algebraic geometry IV
  (A.~N. Parshin and I.~R. Shafarevich, eds.), Encyclopaedia of Mathematical
  Sciences, vol.~55, Springer-Verlag, 1994, pp.~123--284.

\bibitem[War71]{Wa1971Foundations-of-dif}
Frank~W. Warner, \emph{Foundations of differentiable manifolds and {L}ie
  groups}, Scott, Foresman and Co., Glenview, Ill.-London, 1971.

\end{thebibliography}
\end{document}


\documentclass{amsart}
\usepackage{latexsym,amsmath}
\usepackage{amsthm,amssymb}
\usepackage{lipsum}

\newtheorem{theorem}{Theorem}[section]
\newtheorem{lemma}[theorem]{Lemma}
\newtheorem{corollary}[theorem]{Corollary}
\newtheorem{proposition}[theorem]{Proposition}

\theoremstyle{definition}
\newtheorem{definition}{Definition}
\newtheorem{example}[theorem]{Example}
\newtheorem{remark}[theorem]{Remark}

\numberwithin{equation}{section}

 
 
 
 
 
 
 
 
 
 
 
 
 
 
 
 
 
 
 
  

\begin{document}

\title[Fiberwise Ricci-flat metrics]{Semi-positivity of fiberwise Ricci-flat metrics on Calabi-Yau fibrations}

\author{Young-Jun Choi}

\address{School of Mathematics, Korea Institute for Advanced Study(KIAS), 85 Hoegiro Dongdaemun-gu, Seoul 130-722 790-784, Republic of Korea}

\email{choiyj@kias.re.kr}

\subjclass[2010]{32Q25, 32Q20, 32G05, 32W20}

\keywords{Calabi-Yau manifold, Ricci-flat metric, K\"ahler-Einstein metric, a family of Calabi-Yau manifolds, variation}

\maketitle

\begin{abstract}
Let $X$ be a K\"ahler manifold which is fibered over a complex manifold $Y$ such that every fiber is a Calabi-Yau manifold. Let $\omega$ be a fixed K\"ahler form on $X$. By the theorem due to Yau, there exists a unique Ricci-flat K\"ahler form $\rho\vert_{X_y}$ for each fiber, which is cohomologous to $\omega\vert_{X_y}$. This family of Ricci-flat K\"ahler forms $\rho\vert_{X_y}$ induce a smooth $(1,1)$-form $\rho$ on $X$. In this paper, we prove that $\rho$ is semi-positive on the total space $X$. We also discuss several byproducts, among them the local triviality of families of Calabi-Yau manifolds.
\end{abstract}

\section{Introduction}

Let $p:X\rightarrow Y$ be a proper surjective holomorphic mapping between complex manifolds $X$ and $Y$ whose differential has maximal rank everywhere such that every fiber $X_y:=p^{-1}(y)$ is a compact K\"ahler manifolds. This is called a \emph{smooth family of compact K\"ahler manifolds} or a \emph{compact K\"ahler fibration}. If every fiber $X_y$ is a Calabi-Yau manifold, i.e., a compact K\"ahler manifold whose canonical line bundle $K_{X_y}$ is trivial, then the family is called a \emph{smooth family of Calabi-Yau manifolds} or a \emph{Calabi-Yau fibration}. By the celebrated theorem due to Yau, every Calabi-Yau manifold equpped with a fixed K\"ahler class has a unique Ricci-flat K\"ahler metric whose associated K\"ahler form belongs to the fixed K\"ahler class (\cite{Yau}).
\medskip

The main theorem of this paper is the following:

\begin{theorem} \label{T:main_theorem}
Let $p:X\rightarrow Y$ be a smooth family of Calabi-Yau manifolds. 
If $(X,\omega)$ is a K\"ahler manifold, then there exists a unique function ${\varphi}\in C^\infty(X)$ which satisfies the following properties:
\begin{itemize}
\item[(\romannumeral1)] $\int_{X_y}{\varphi}(\omega_y)^n=0$ for every $y\in Y$,
\item[(\romannumeral2)] $\omega+dd^c{\varphi}\vert_{X_y}$ is a Ricci flat K\"ahler form on $X_y$ for every $y\in Y$ and
\item[(\romannumeral3)] $\omega+dd^c{\varphi}$ is a semi-positive $(1,1)$-form on $X$. 
\end{itemize}
\end{theorem}

Here $d^c$ means the real operator defined by
\begin{equation*}
d^c=\frac{\sqrt{-1}}{2}{\left({\partial-\bar\partial}\right)}.
\end{equation*}
Then we have $dd^c={\sqrt{-1}}{\partial\bar\partial}$. 
We call the $(1,1)$-form $\rho:=\omega+dd^c{\varphi}$ which satisfies the property $(\romannumeral2)$ the \emph{fiberwise Ricci-flat metric} or the \emph{fiberwise Ricci-flat K\"ahler form} on a Calabi-Yau fibration $p:X\rightarrow Y$. 
Note that a real $(1,1)$-form on $X$ satisfying $(\romannumeral2)$ is not uniquely determined. 
With the normalization condition $(\romannumeral1)$, the fiberwise Ricci-flat metric is uniquely determined. From now on, the fiberwise Ricci-flat metric on a Calabi-Yau fibration means the real $(1,1)$-form which satisfies $(\romannumeral1)$ and $(\romannumeral2)$.
\medskip

For a family of canonically polarized compact K\"ahler manifolds, we have a fiberwise K\"ahler-Einstein metric by the same way. The positivity of the fiberwise K\"ahler-Einstein metric on a family of compact K\"ahler manifolds was first studied by Schumacher. In his paper \cite{Schumacher}, he have proved that the fiberwise K\"ahler-Einstein metric on a family of canonically polarized compact K\"ahler manifolds is semi-positive. Moreover he have also proved that it is strictly positive if the family is not locally trivial. This is equivalent to the semi-positivity or positivity of the relative canonical line bundle of the fibration, respectively. P\v aun have shown that if the relative adjoint line bundle is positive on each fiber, then it is semi-positive on the total space by generalizing the method of Schumacher (\cite{Paun2}). In case of a family of complete K\"ahler manifolds, Choi have proved that the fiberwise K\"ahler-Einstein metric on a family of bounded pseudoconvex domains is semi-positive or positive if the total space is pseudoconvex or strongly pseudoconvex, resepctively (\cite{Choi1, Choi2}).
\medskip

The proof of Theorem \ref{T:main_theorem} starts with the following identity from \cite{Semmes}: For a real $(1,1)$-form $\tau$ on $X$,
\begin{equation*}\label{E:Semmes}
\tau^{n+1}
=
c(\tau)\tau^n{\sqrt{-1}} ds\wedge d\bar s
\end{equation*}
where $\tau^n$ is the $n$-fold exterior power divided by $n!$. Here $c(\tau)$ is called a \emph{geodesic curvature} of $\tau$. (For the detail, see Section \ref{SS:horizontal_lift}.) Now suppose that $\tau$ is positive-definite on each fiber $X_y$. Then \eqref{E:Semmes} says that $\tau$ is semi-positive or positive if and only if $c(\tau)\ge0$ or $c(\tau)>0$, respectively.

In \cite{Schumacher}, Schumacher have proved that the geodesic curvature of fiberwise K\"ahler-Einstein metric on a family of canonically polarized compact K\"ahler manifolds satisfies a certain second order elliptic partial differential equation. This PDE gives a lower bound of the geodesic curvature by the maximum principle or an estimate on the heat kernel. This is how Schumacher have shown that the fiberwise K\"ahler-Einstein metric is positive. 

However, in case of a Calabi-Yau fibration, the PDE which the geodesic curvature of fiberwise Ricci-flat metric satisfies is different from the previous one. (See Section \ref{S:fiberwiseRFf}.) In particular, it does not give a lower bound directly. This difference arises from the difference of complex Monge-Amp\`ere equations which give the K\"ahler-Einstein metrics. More precisely, the complex Monge-Amp\`ere equation of type:
\begin{equation}\label{E:1}
{\left({\omega+dd^c{\varphi}}\right)}^n
=
e^{\lambda{\varphi}+f}\omega^n,
\end{equation}
for some constant $\lambda>0$ and some suitable smooth function $f$, gives the K\"aher-Einstein metric on a canonically polarized compact K\"aher manifold. On the other hand, the complex Monge-Amp\`ere equation of type:
\begin{equation}\label{E:0}
{\left({\omega+dd^c{\varphi}}\right)}^n
=
e^{\tilde{f}}\omega^n
\end{equation}
for some suitable smooth function $\tilde f$, gives the K\"ahler-Einstein (in this case Ricci-flat) metric on a Calabi-Yau manifold. It is remarkable to note that if $f$ and $\tilde f$ coincide,  then \eqref{E:1} converges to \eqref{E:0} as $\lambda\rightarrow0$.
Then by the a priori estimate for complex Monge-Amp\`ere equation, it is well known that the solutions ${\varphi}_\lambda$ of \eqref{E:1} converges to the solution  of \eqref{E:0} by passing through a subsequence (see Section \ref{S:approximation}). 
This is the key idea of the proof of Theorem \ref{T:main_theorem}.
\medskip

In the meantime, the second order elliptic PDE for the geodesic curvature $c(\rho)$ of the fiberwise Ricci-flat metric of a Calabi-Yau fibration gives several informations about Calabi-Yau fibrations. 
Among them, there is a result about the local triviality of the Calabi-Yau fibrations.

\begin{theorem}
Let $p:X\rightarrow Y$ be a smooth family of Calabi-Yau manifolds. Let $E:=p_*(K_{X/Y})$ be the direct image bundle of the relative canonical line bundle $K_{X/Y}$. We denote by $\Theta(E)$ the curvature of the natural $L^2$ metric of $E$. If $\Theta(E)$ vanishes along a  complex curve, then the family is trivial along the complex curve.
\end{theorem}

\noindent{\bf Acknowlegement.}
The author happily acknowledges his thanks to Mihai P\v aun who suggested this problem, shared his ideas and Dano Kim for very helpful discussions. He is also indebted to Hoang Lu Chinh for teaching him the approximation process of complex Monge-Amp\`ere equations and Bo Berndtsson for enlightening discussions about many topics, including the direct image bundles and Griffiths theorem. 

\section{Preliminaries}

Let $p:X^{n+d}\rightarrow Y^d$ be a smooth family of K\"ahler manifolds. Taking a local coordinate $(s^1,\dots,s^d)$ of $Y$ and a local coordinate $(z^1,\dots,z^n)$ of a fiber of $p$, $(z^1,\dots,z^n,s^1,\dots,s^d)$ forms a local coordinate of $X$ such that under this coordinate, the holomorphic mapping $p$ is locally given by
\begin{equation*}
p(z^1,\dots,z^n,s^1,\dots,s^d)
=
(s^1,\dots,s^d).
\end{equation*}
We call this an \emph{admissible coordinate of $p$}.

Throughout this paper we use  small Greek letters, $\alpha,\beta,\dots=1,\dots,n$ for indices on $z=(z^1,\dots,z^n)$ and small roman letters, $i,j,\dots=1,\dots,m$ for indices on $s=(s^1,\dots,s^d)$ unless otherwise specified. For a properly differentiable function $f$ on $X$, we denote by
\begin{equation} \label{E:convention}
f_\alpha
    ={\frac{\partial{f}}{\partial{z^\alpha}}},\;\;
f_{\bar\beta}
    ={\frac{\partial{f}}{\partial{z^{\bar\beta}}}},\;\;
   \;\;\text{and}\;\;
f_{i}
    ={\frac{\partial{f}}{\partial{s^i}}},\;\;
f_{\bar{j}}
    ={\frac{\partial{f}}{\partial{s^{\bar{j}}}}},
\end{equation}
where $z^{\bar\beta}$ and $s^{\bar{j}}$ mean $\bar{z}^\beta$ and $\bar{s}^j$, respectively. In case $d=1$, we denote by
\begin{equation*}
f_{s}
	={\frac{\partial{f}}{\partial{s}}}\;\;
\text{and}\;\;
f_{\bar{s}}
	={\frac{\partial{f}}{\partial{\bar{s}}}}.
\end{equation*}
If there is no confusion, we always use the Einstein convention. For simplicity we denote by $v_i:=\partial/\partial{s^i}$. If $d=1$, then we denote by $v:=\partial/\partial{s}$.

\subsection{Horizontal lifts and geodesic curvatures}\label{SS:horizontal_lift}

For a complex manifold $M$, we denote by $T'M$ the complex tangent bundle of type $(1,0)$.
\begin{definition}
Let $V\in T'_yY$ and $\tau$ be a real $(1,1)$-form on $X$. Suppose that $\tau$ is positive definite on a fiber $X_y$.
\begin{itemize}
\item[1.] A vector field $V_\tau$ of type $(1,0)$ is  called a \emph{horizontal lift} along $X_y$ of $V$ if $V_\tau$ satisfies the following:
\begin{itemize}
\item [(\romannumeral1)]${\left\langle{{V_\tau,W}}\right\rangle}_\tau=0$ for all $W\in{T'X_y}$,
\item [(\romannumeral2)]$d\pi(V_\tau)=V$.
\end{itemize}
\item[2.] The \emph{geodesic curvature} $c(\tau)(V)$ of $\tau$ along $V$ is defined by the norm of $V_\tau$ with respect to the sesquilinear form ${\left\langle{{\cdot,\cdot}}\right\rangle}_\tau$ induced by $\tau$, namely,
\begin{equation*}
c(\tau)(V)={\left\langle{{V_\tau,V_\tau}}\right\rangle}_\tau.
\end{equation*}
\end{itemize}
\end{definition}

\begin{remark} \label{R:horizontal_lift}
Let $(z^1,\dots,z^n,s^1,\dots,s^d)$ be an admissible coordinate of $p$. Then we can write $\tau$ as follows:
\begin{equation*}
\tau
=
{\sqrt{-1}}{\left({\tau_{i\bar{j}}ds^i\wedge{ds}^{\bar{j}} 
+\tau_{i\bar\beta}ds^i\wedge{dz}^{\bar\beta} 
+\tau_{\alpha\bar{j}}dz^\alpha\wedge{ds}^{\bar{j}} 
+\tau_{\alpha\bar\beta}dz^\alpha\wedge{dz}^{\bar\beta}
}\right)}.
\end{equation*}
Since $\tau$ is positive-definite on each fiber $X_y$, the matrix $(\tau_{\alpha\bar\beta})$ is invertible. We denote by $(\tau^{\bar\beta\alpha})$ the inverse matrix. Then it is easy to see that the horizontal lift of $\partial/\partial{s^i}$ is given as follows.
\begin{equation*}
{\left({{\frac{\partial{}}{\partial{s^i}}}}\right)}_\tau
=
{\frac{\partial{}}{\partial{s^i}}}-\tau_{i\bar\beta}\tau^{\bar\beta\alpha}{\frac{\partial{}}{\partial{z^\alpha}}},
\end{equation*}
in particular, any horizontal lift with respect to $\tau$ is uniquely determined. 

On the other hand, the geodesic curvature $c(\tau)(v_i)$ is computed as follows:
\begin{equation} \label{E:function}
\begin{aligned}
c(\tau)(v_i)
&=
{\left\langle{{(v_i)_\tau,(v_i)_\tau}}\right\rangle}_\tau \\
&=
{\left\langle{{{\frac{\partial{}}{\partial{s^i}}}-\tau_{i\bar\beta}\tau^{\bar\beta\alpha}{\frac{\partial{}}{\partial{z^\alpha}}},
{\frac{\partial{}}{\partial{s^i}}}-\tau_{i\bar\delta}\tau^{\bar\delta\gamma}{\frac{\partial{}}{\partial{z^\gamma}}}
}}\right\rangle}_\tau \\
&=
\tau_{i\bar i}
-\overline{\tau_{i\bar\delta}\tau^{\bar\delta\gamma}}\tau_{i\bar\gamma}
-\tau_{i\bar\beta}\tau^{\bar\beta\alpha}\tau_{\alpha\bar i}
+\tau_{i\bar\beta}\tau^{\bar\beta\alpha}
\overline{\tau_{i\bar\delta}\tau^{\bar\delta\gamma}}\tau_{\alpha\bar\gamma} \\
&=
\tau_{i\bar i}
-\tau_{i\bar\beta}\tau^{\bar\beta\alpha}\tau_{\alpha\bar i},
\end{aligned}
\end{equation}
because $\tau$ is a real $(1,1)$-form.
\end{remark}
We denote by $\tau^n$ the $n$-fold exterior power divided by $n!$. Suppose that $Y$ is $1$-dimensional. We already know that $\tau$ is positive definite when restricted to $X_y$, hence it has at least $n$ positive eigenvalues. In order to show that the $(n+1)$-th eigenvalue (in the “base direction”) is equally positive, we consider the form $\tau^{n+1}$ on $X$.
It is well known (cf, see \cite{Semmes}) that 
\begin{equation}\label{E:Semmes}
\tau^{n+1}=c(\tau)\cdot \tau^n\wedge{\sqrt{-1}}{d}s\wedge{d}\bar{s}.
\end{equation}
It is remarkable to note that if $c(\tau)>0 \; (\ge0)$, then $\tau$ is a positive (semi-positive) real $(1,1)$-form by \eqref{E:Semmes}.

\subsection{Kodaira-Spencer classes and Direct image bundles}

Let $p:X\rightarrow Y$ be a smooth family of compact K\"ahler manifolds. We denote the Kodaira-Spencer map for the family $p:X\rightarrow Y$ at a given point $y\in Y$ by
\begin{equation*}
K_y:T'_y Y\rightarrow
H^1(X_y,T'X_y).
\end{equation*}
The Kodaira-Spencer map is induced by the edge homomorphism for the short exact sequence
\begin{equation*}
0\rightarrow
T'_{X/Y}
\rightarrow
T'X
\rightarrow
p^*T'Y
\rightarrow0.
\end{equation*}
If $V\in T'_y Y$ is a tangent vector, and if 
\begin{equation*}
V+b^\alpha{\frac{\partial{}}{\partial{z^\alpha}}}
\end{equation*}
is any smooth lifting of $V$ along $X_y$, then 
\begin{equation*}
\bar\partial
{\left({
	V+b^\alpha{\frac{\partial{}}{\partial{z^\alpha}}}
}\right)}
=
{\frac{\partial{b^\alpha}}{\partial{z^{\bar\beta}}}}
{\frac{\partial{}}{\partial{z^\alpha}}}
\otimes
dz^{\bar\beta}
\end{equation*}
is a $\bar\partial$-closed form on $X$, which represents $K_y(V)$, i.e.,
\begin{equation*}
K_y(V)
=
\left[
{\frac{\partial{b^\alpha}}{\partial{z^{\bar\beta}}}}
{\frac{\partial{}}{\partial{z^\alpha}}}
\otimes
dz^{\bar\beta}
\right]
\in H^{0,1}(X_y,T'X_y).
\end{equation*}
This cohomology class $K_y(V)$ is called the \emph{Kodaira-Spencer class} of $V$. The celebrated theorem of Kodaira and Spencer says that if the Kodaira-Spencer class vanishes locally, then the family is locally trivial (\cite{Kodaira:Spencer}, see also \cite{Kodaira}).
\medskip

The direct image bundle $E:=p_*(K_{X/Y})$ of $K_{X/Y}$ is defined by the vector bundle over $Y$ whose fiber $E_y$ is given by
\begin{equation*}
E_y=H^0(X_y,K_{X_y}).
\end{equation*}
$E$ is a hermitian vector bundle with $L^2$ metric which is a hermitian metric given by following: For $u_y, v_y$, define ${\left\langle{{u_y,v_y}}\right\rangle}$ by
\begin{equation*}
{\left\langle{{u_y,v_y}}\right\rangle}_y^2
=
\int_{X_y}c_n u_y\wedge \overline{v_y}
\end{equation*}
where $c_n=({\sqrt{-1}})^{n^2}$ chosen to make the form positive. 

The Kodaira-Spencer class acts on $u_y\in E_y$ as follows: Let $k_y(V)$ is any representative of $K_y(V)$, i.e., $T'X_y$-valued $(0,1)$-form in $K_y(V)$, which locally decomposes as 
\begin{equation*}
k_y=\zeta\otimes w
\end{equation*}
where $\zeta$ is a $(0,1)$-form and $w$ is a vector field of type $(1,0)$. Then $k_y(V)$ acts on $u_y$ by
\begin{equation*}
k_y(V)\cdot u_y=\zeta\wedge(u_y\rfloor w),
\end{equation*}
where $\rfloor$ is a contraction. This gives a globally defined $\bar\partial$-closed form of type $(n-1,1)$ and
\begin{equation*}
K_y(V)\cdot u_y
:=
{\left[{k_y(V)\cdot u_y
}\right]}
\in H^{(n-1,1)}(X_y).
\end{equation*}
The following theorem due to Griffiths says the curvature of $E$ is computed in terms of Kodaira-Spencer classes (\cite{Griffiths}, see also \cite{Berndtsson2}).

\begin{theorem}
Let $\Theta(E)$ be the curvature of $E$ with $L^2$-metric. Then for $V\in T_y'Y$,
\begin{equation}\label{E:Griffiths}
{\left\langle{{\Theta_{V\bar V}(E)u,u}}\right\rangle}={\left\|{K_y(V)\cdot u}\right\|}^2,
\end{equation}
where ${\left\|{K_y(V)\cdot u}\right\|}$ is the norm of its unique harmonic representative. It does not depend on the choice of K\"ahler metric.
\end{theorem}

\subsection{Estimates for resolvent and heat kernel}

This subsection essentially same with Section 3 in \cite{Schumacher}. But we give most of details for readers' convenience.

Let $(X,\omega)$ be a compact K\"ahler manifold. We denote by $\Delta_\omega$ the Laplace-Beltrami operator of $\omega$. Note that $-\Delta_\omega$ is self-adjoint with nonnegative eigenvalues.

\begin{proposition} \label{P:heat_kernel}
Suppose that the Ricci curvature of $\omega$ is bounded from below by negative constant $-1$. Then there exists a strictly positive function $P_n(d(X))$, depending on the dimension $n$ of $X$ and the diameter $d(X)$ with the following property:

Let $0<{\varepsilon}\le1$. If $g$ is a continuous function and $f$ is a solution of
\begin{equation}\label{E:pde_kernel}
(-\Delta_\omega+{\varepsilon})f=g,
\end{equation}
then
\begin{equation*}
f(z)\ge
P_n(d(X))\cdot\int_Xg dV_\omega
\end{equation*}
for $z\in X$.
\end{proposition}

\begin{proof}
Let $\{e_\nu\}$ be an orthonormal basis of the space of square integrable real valued functions consisting of eigenfunctions of $-\Delta_\omega$ with eigenvalues $\lambda_\nu$. Then the resolvent kernel for \eqref{E:pde_kernel} is given by
\begin{equation*}
P_{\varepsilon}(z,w)
=
\sum_\nu\frac{1}{{\varepsilon}+\lambda_\nu}e_\nu(z)e_\nu(w).
\end{equation*}
It is well known that the solution $f$ is equals to
\begin{equation*}
f(z):=(-\Delta_\omega+{\varepsilon})^{-1}(g)(z)
=
\int_X P(z,w)g(w) dV_\omega,
\end{equation*}
where $dV_\omega=\omega^n$ is the volume form of $\omega$.

On the other hand, we denote by $P(t,z,w)$ the integral kernel for the heat operator ${\frac{\partial{}}{\partial{t}}}-\Delta_\omega$. Then the solution of the heat equation 
\begin{equation*}
{\left({{\frac{\partial{}}{\partial{t}}}-\Delta_\omega}\right)}f=0
\end{equation*}
with initial function $g(z)$ for $t=0$ is given by the following:
\begin{equation*}
\int_X P(t,z,w)g(w) dV_\omega.
\end{equation*}
Note that $P(t,z,w)$ is given by
\begin{equation*}
P(t,z,w)
=
\sum_\nu e^{-t\lambda_\nu}e_\nu(z)e_\nu(w).
\end{equation*}
Then direct computation implies that
\begin{equation*}
P_{\varepsilon}(z,w)
=
\int_0^\infty e^{-{\varepsilon} t}P(t,z,w) dt.
\end{equation*}

Since the Ricci curvature is bounded from below by $-1$, we use the estimates in the form of \cite{Sturm} (cf, see \cite{Cheeger:Yau}).
\begin{equation*}
P(t,z,w)\ge
Q_n(t,r(z,w)):=
\frac{1}{(2\pi t)^n}
e^{-\frac{r^2(z,w)}{t}}
e^{-\frac{2n-1}{4}t},
\end{equation*}
where $r=r(z,w)$ denotes the geodesic distance. Let
\begin{equation*}
P_{n,{\varepsilon}}(r)=\int_0^\infty e^{-{\varepsilon} t}Q_n(t,r)dt
\end{equation*}
Since $e^{-{\varepsilon} t}$ is decreasing with respect to ${\varepsilon}$, we get
\begin{equation*}
P_{\varepsilon}(z,w)\ge
P_{n,{\varepsilon}}(r(z,w))\ge
P_{n,{\varepsilon}}(d(X))\ge
P_{n,1}(d(X)),
\end{equation*}
where $d(X)$ is the diameter of $X$. This completes the proof.
\end{proof}

\section{Approximations of complex Monge-Amp\`ere equations}\label{S:approximation}
\label{S:approximation}

In this section, we discuss approximations of a solution of complex Monge-Amp\`ere equation \eqref{E:0} in terms of the solutions of \eqref{E:1}. First we consider the approximation on a single compact K\"ahler manifold. After that, we apply the approximation procedure to a family of complex Monge-Amp\`ere equations. First, we recall the existence and uniqueness theorem of complex Monge-Amp\`ere equations due to Aubin and Yau.

Let $(X,\omega)$ be a compact K\"ahler manifold. Let $f$ be a smooth function on $X$. The complex Monge-Amp\`ere equation is given by the following:
\begin{equation} \label{E:CMAE}
\begin{aligned}
{\left({\omega+dd^c{\varphi}}\right)}^n
&=
e^{\lambda{\varphi}+f}\omega^n,
\\
\omega+dd^c{\varphi}
&>0.
\end{aligned}
\end{equation}
This fully nonlinear complex partial differential equation was first raised by E. Calabi. The easiest case, $\lambda>0$, was solved by Aubin and Yau, independently (\cite{Aubin, Yau}).  The next case, $\lambda=0$, was solved by Yau (\cite{Yau}). The last case, $\lambda<0$, is not solved in general. This is why a compact K\"ahler manifold with positive first Chern class does not have the K\"ahler-Einstein metric in general. (Cf, see \cite{Tian}.)

\begin{theorem} \label{T:AY}
The following holds:
\begin{itemize}
\item [1.] (Aubin/Yau) If $\lambda>0$, then there exists a unique smooth function ${\varphi}$ satisfying \eqref{E:CMAE} for every smooth function $f\in C^\infty(X)$. 
\item [2.] (Yau) If $\lambda=0$, then there exists a smooth function ${\varphi}$ satisfying \eqref{E:CMAE} for $f\in C^\infty(X)$ such that $\int_X e^f\omega^n=\int_X \omega^n$. Moreover, the solution is unique up to the addition of constants.
\end{itemize}
\end{theorem}

\subsection{Approximation on a compact K\"ahler manifold}

Let $(X,\omega)$ be a compact K\"ahler manifold and $f$ be a smooth function on $X$ satisfying
\begin{equation*}
\int_X e^f\omega^n=\int_X\omega^n.
\end{equation*}
Consider the following complex Monge-Amp\`ere equation:
\begin{equation} \label{E:CMAE0}
\begin{aligned}
{\left({\omega+dd^c\varphi}\right)}^n &= e^f\omega^n, \\
\omega+dd^c&\varphi>0.
\end{aligned}
\end{equation}
By Theorem \ref{T:AY}, we already know that there exists a solution which is unique up to addition of constants. 

Let $\{f_{\varepsilon}\}$ be a sequence of smooth functions in $X$ which converges to $f$ as ${\varepsilon}$ goes to $0$ in $C^{k,\alpha}$-topology for any $k\in{\mathbb{N}}$ and $\alpha\in(0,1)$. We want to approximate a solution of \eqref{E:CMAE0} by the solutions ${\varphi}_{\varepsilon}$ of the following complex Monge Amp\`ere equations:
\begin{equation} \label{E:CMAE1}
\begin{aligned}
{\left({\omega+dd^c{\varphi}_{\varepsilon}}\right)}^n &= 
e^{{\varepsilon}{\varphi}_{\varepsilon}+f_{\varepsilon}}\omega^n \\
\omega+dd^c&{\varphi}_{\varepsilon}>0,
\end{aligned}
\end{equation}
as ${\varepsilon}$ goes to $0$.  Note that if ${\varepsilon}$ goes to $0$, then Equation \eqref{E:CMAE1} converges to \eqref{E:CMAE0}. 

The convention all over this paper is that we will use the same letter ``$C$'' to denote a generic constant, which may change from one line to another, but it is independent of the pertinent parameters involved (especially ${\varepsilon}$).

\begin{proposition} \label{P:approximation1}
For each ${\varepsilon}$ with $0<{\varepsilon}\le1$, let ${\varphi}_{\varepsilon}$ be the solution of \eqref{E:CMAE1}. Then for any $k\in{\mathbb{N}}$ and $\alpha\in(0,1)$, there exists a constant $C>0$ which depend only on $k$, $\alpha$, the geometry of $(X,\omega)$ and the function $f$ such that 
\begin{equation*}
{\left\|{{\varphi}_{\varepsilon}}\right\|}_{C^{k,\alpha}(X)}<C.
\end{equation*}
In particular, $\{{\varphi}_{\varepsilon}\}$ is a relatively compact subset of $C^{k,\alpha}(X)$ for any positive integer $k$ and $\alpha\in(0,1)$.
\end{proposition}

\begin{proof}
We may assume that
\begin{equation*}
\mathrm{Vol}(X)
=
\int_X\omega^n
=1.
\end{equation*}
The first step is obtaining a uniform upper bound for $\varphi_\varepsilon$. For each $\varepsilon>0$, the solution $\varphi_\varepsilon$ of \eqref{E:CMAE1} satisfies that 
\begin{equation*}
1=\int_X{\left({\omega+dd^c{\varphi}_{\varepsilon}}\right)}^n
=
\int_X e^{{\varepsilon}{\varphi}_{\varepsilon}} e^{f_{\varepsilon}}\omega^n
\end{equation*}
Then Jensen inequality implies that
\begin{equation*}
1\ge
\exp{\left({\int_X{\varepsilon}{\varphi}_{\varepsilon} e^{f_{\varepsilon}}\omega^n}\right)},
\end{equation*}
it is equivalent to
\begin{equation*}
\int_X{\varphi}_{\varepsilon} e^{f_{\varepsilon}}\omega^n
\le0.
\end{equation*}
Note that $f_{\varepsilon}$ converges to $f$ as ${\varepsilon}\rightarrow0$. The Hartogs lemma for quasi-plurisubharmonic functions implies that
\begin{equation}\label{E:upper}
\sup_X{\varphi}_{\varepsilon}
<
C,
\end{equation}
where $C$ is a constant which depends only on the geometry of $(X,\omega)$ and $f$ (\cite{Guedj_Zerihai}). Here we recall the theorem about the uniform estimates (\cite{Yau, Kolodziej}).

\begin{theorem}\label{T:uniform}
Let $(M,\omega)$ be a compact K\"ahler manifold. Assume that $\varphi$ satisfies the following complex Monge-Amp\`ere equation:
\begin{align*}
{\left({\omega+dd^c\varphi}\right)}^n 
&=
F\omega^n, \\
\omega+dd^c\varphi
&>
0.
\end{align*}
Then
\begin{equation*}
\mathrm{osc}\;\varphi
:=
\sup_M\varphi-\inf_M\varphi
\le
C
\end{equation*}
where $C>0$ depends only on $(M,\omega)$ and on an upper bound for ${\left\|{F}\right\|}_p$ for some $1<p\le\infty$.
\end{theorem}

If we set $F=e^{{\varepsilon}{\varphi}+f_{\varepsilon}}$, then ${\left\vert{F}\right\vert}<C$ for some $C>0$ by \eqref{E:upper}. Then it follows from Theorem \ref{T:uniform} that 
\begin{equation}\label{E:uniform}
{\left\|{{\varphi}_{\varepsilon}}\right\|}_{C^0(X)}<C
\end{equation}
for some $C>0$ which depends only on $M$ and the function $f$.
\medskip

The second step is obtaining the Laplacian estimates. We recall the following theorem in \cite{Di Nezza_Lu}, which is essentially due to M. P\v{a}un  (\cite{Paun1}, cf. see \cite{Siu}).
\begin{theorem}\label{T:Laplacian}
Let $\psi^+$ and $\psi^-$ be smooth quasi-plurisubharmonic functions on $X$. Let $\varphi\in~C^\infty(X)$ be such that $sup_X\varphi=0$ and
\begin{equation*}
(\omega+dd^c\varphi)^n
=
e^{\psi^+-\psi^-}\omega^n.
\end{equation*}
Assume given a constant $C>0$ such that
\begin{equation*}
dd^c\psi^\pm\ge-C\omega,
\;\;\;
\sup_X\psi^+\le C.
\end{equation*}
Assume also that the holomorphic bisectional curvature of $\omega$ is bounded from below by $-C$. Then there exists $A>0$ depending on $C$ and $\int_Xe^{-2(4C+1)\varphi}\omega^n$ such that
\begin{equation*}
0\le
n+\Delta_\omega\varphi
\le
Ae^{-2\psi^-}.
\end{equation*}
\end{theorem}

We take $\psi^+={\varepsilon}{\varphi}_{\varepsilon}+f_{\varepsilon}$ and $\psi^-=0$. Since $f_{\varepsilon}$ converges to $f$ as ${\varepsilon}\rightarrow0$ and every ${\varphi}_{\varepsilon}$ satisfies that
\begin{equation*}
dd^c{\varphi}_{\varepsilon} > -\omega,
\end{equation*}
it follows from \eqref{E:uniform} that $\psi^+$ satisfies the hypothesis of Theorem~\ref{T:Laplacian}. Note that $\{{\varphi}_{\varepsilon}\}_{0<{\varepsilon}\le1}$ is a relatively compact subset of $L^1(X,\omega)$. This implies the Laplacian estimates for ${\varphi}_{\varepsilon}$:
\begin{equation*}
{\left\vert{\Delta_\omega{\varphi}_{\varepsilon}}\right\vert}<C
\end{equation*}
for some constant $C>0$ which depends only on the geometry of $(M,\omega)$ and the function $f$ by the Uniform Skoda Integrability Theorem due to Zeriahi (\cite{Zeriahi}).
\medskip

The final step is $C^{2,\alpha}(X)$-estimate. For $k\ge2$ and $\alpha\in(0,1)$, the standard Evans-Krylov method (\cite{Evans, Krylov}) and Schauder estimates (cf, see \cite{Gilbarg_Trudinger}) implies
\begin{equation*}
{\left\|{{\varphi}_{\varepsilon}}\right\|}_{C^{k,\alpha}(X)}
\le
C,
\end{equation*}
where $C$ is some constant which depends only on $k,\alpha$, the geometry of $(X,\omega)$ and the function $f$. This completes the proof.
\end{proof}

Proposition \ref{P:approximation1} implies that there exists a $\hat{\varphi}\in C^\infty(X)$ such that ${\varphi}_{\varepsilon}\rightarrow\hat{\varphi}$ as ${\varepsilon}\rightarrow0$ by passing through a subsequence. It is obvious that $\hat{\varphi}$ satisfies Equation \eqref{E:CMAE0}. Hence we have the following corollary.

\begin{corollary}
There exists a subsequence of $\{{\varphi}_{\varepsilon}\}$ converging to $\hat{\varphi}\in C^\infty(X)$ as ${\varepsilon}\rightarrow0$ such that $\hat{\varphi}$ is a solution of \eqref{E:CMAE0}.
\end{corollary}

\subsection{Approximation on a family of complex Monge-Amp\'ere equations}
\label{SS:AFCMAE}

Let $p:X^{n+d}\rightarrow Y^d$ be a smooth family of compact K\"ahler manifolds. For a form $\theta$ on $X$, the push-forward $p_*(\theta)$ is defined by the form on $Y$ which satisfies
\begin{equation*}
\int_Y p_*(\theta)\wedge\beta
=
\int_X \theta\wedge p^*(\beta),
\end{equation*}
for any form $\beta$ on $Y$. Note that the push-forward commutes with the exterior derivative $d$ (cf, see \cite{Berndtsson1}). Let $\omega$ be a positive real $(1,1)$-form on $X$. We denote by
\begin{equation*}
\omega_y:=\omega\vert_{X_y}=(\iota_y)^*\omega,
\end{equation*}
where $\iota:X_y\hookrightarrow X$ is the standard inclusion. It is easy to see that
\begin{equation*}
\mathrm{Vol}_{\omega_y}(X_y)
=
p_*{\left({\omega_y}\right)}^n
=
\int_{X_y}{\left({\omega_y}\right)}^n.
\end{equation*}
If $\omega$ is K\"ahler, then it follows that
\begin{equation*}
d\mathrm{Vol}_{\omega_y}(X_y)
=
dp_*{\left({\omega_y}\right)}^n
=
p_*{\left({d\omega_y}\right)}^n
=
0.
\end{equation*}
Therefore we have $\mathrm{Vol}(X_y)=C$ for some constant $C>0$. From now on, we always assume that
\begin{equation*}
\mathrm{Vol}_{\omega_y}(X_y)=\int_{X_y}{\left({\omega_y}\right)}^n=1
\end{equation*}
for every $y\in Y$.
\medskip

From now on, we consider a smooth family $p:X\rightarrow{\mathbf{D}}$ of compact K\"ahler manifolds over the unit disc in ${\mathbb{C}}$. Suppose that $\omega$ is a K\"ahler form on $X$. Under an admissible coordinate $(z^1,\dots,z^n,s)$ in $X$, $\omega$ is written as follows:
\begin{equation*}
\omega
=
{\sqrt{-1}}{\left({g_{s\bar s}ds\wedge d\bar s
+g_{s\bar\beta}ds\wedge{dz}^{\bar\beta} 
+g_{\alpha\bar s}dz^\alpha\wedge d\bar s
+g_{\alpha\bar\beta}dz^\alpha\wedge{dz}^{\bar\beta}
}\right)}.
\end{equation*}
Let $f$ be a smooth function on $X$. For each $0<{\varepsilon}\le1$, let $\{f_{\varepsilon}\}$ be a sequence of a smooth function on $X$ converging to $f$ in $C^{k,\alpha}(X)$-topology. We consider the following fiberwise complex Monge-Amp\`ere equations:
\begin{equation} \label{E:ACMAE}
\begin{aligned}
{\left({\omega_y+dd^c{\varphi}_y}\right)}^n
&=
e^{{\varepsilon}{\varphi}_y+f_{\varepsilon}\vert_{X_y}}\omega_y^n,
\\
\omega_y+dd^c{\varphi}_y
&>0
\end{aligned}
\end{equation}
on $X_y$ for $y\in{\mathbf{D}}$. Theorem \ref{T:AY} implies that for each $y$, there exists a unique solution of \eqref{E:ACMAE}, call it ${\varphi}_{y,{\varepsilon}}\in C^\infty(X_y)$. It is remarkable to note that the function ${\varphi}_{\varepsilon}$ defined by 
\begin{equation*}
{\varphi}_{\varepsilon}(x)={\varphi}_{y,{\varepsilon}}(x).
\end{equation*}
is a smooth function on $X$. This is followed from the openness part of the a priori estimate for complex Monge-Amp\`ere equations and the implicit function theorem (\cite{Yau}).

\begin{proposition}\label{P:approximation2}
For a fixed fiber $X_y$, there exits a constant $C$ which depends only on the geometry of $(X_y,\omega_y)$ and the function $f$ such that
\begin{equation*}
{\left\|{({\varphi}_{\varepsilon})_s}\right\|}_{C^{k,\alpha}(X_y)}<C
\;\;\;\text{and}\;\;\;
{\left\|{({\varphi}_{\varepsilon})_{s\bar s}}\right\|}_{C^{k,\alpha}(X_y)}<C,
\end{equation*}
for $0<{\varepsilon}\le1$. In particular, $\{({\varphi}_{\varepsilon})_s\}_{0<{\varepsilon}\le1}$ and $\{({\varphi}_{\varepsilon})_{s\bar s}\}_{0<{\varepsilon}\le1}$ are relatively compact subsets in $C^{k,\alpha}(X_y)$ for any $k\in{\mathbb{N}}$ and $\alpha\in(0,1)$.
\end{proposition}

\begin{proof}
We denote by $\theta_{\varepsilon}=\omega+dd^c{\varphi}_{\varepsilon}$. Note that Proposition \ref{P:approximation1} implies that there exists a uniform constant $C>0$ such that
\begin{equation}\label{E:equivalence}
\frac{1}{C}\omega_y
<
\theta_{\varepsilon}\vert_{X_y}
<
C\omega_y,
\end{equation}
for $0<{\varepsilon}\le1$. Under an admissible coordinate $(z^1,\dots,z^n,s)$, the first equation of \eqref{E:ACMAE} is written as follows:
\begin{equation} \label{E:CMAE_coord}
\det(g_{\alpha\bar\beta}+(\varphi_{\varepsilon})_{\alpha\bar\beta})
=
e^{\varepsilon c_\varepsilon}e^{\varepsilon\varphi_{\varepsilon}+\eta}
\det(g_{\alpha\bar\beta})
\end{equation}
on each $X_y$. Taking logarithm of \eqref{E:CMAE_coord} and differentiating it with respect to $s$, we have
\begin{equation*}
\Delta_{\theta_{\varepsilon}}{\left({g_s+({\varphi}_{\varepsilon})_s}\right)}
=
{\varepsilon}({\varphi}_{\varepsilon})_s
+
(f_{\varepsilon})_s+{\varepsilon}(c_{\varepsilon})_s
+
\Delta_\omega g_s,
\end{equation*}
that is,
\begin{equation}\label{E:pde_vpve}
-\Delta_{\theta_{\varepsilon}}({\varphi}_{\varepsilon})_s
+
{\varepsilon}({\varphi}_{\varepsilon})_s
=
-(f_{\varepsilon})_s
+
(\Delta_{\theta_{\varepsilon}}-\Delta_\omega){g}_s.
\end{equation}
It follows that the right hand side is globally defined function on $X_y$. It is remarkable to note that the last term of \eqref{E:pde_vpve} is written as follows:
\begin{equation*}
\Delta_{\theta_{\varepsilon}}{g}_s
=
\mathrm{tr}_{\theta_{\varepsilon}}
{\left({
	i_{\partial_s}\omega
	}\right)}
\;\;\;\text{and}\;\;\;
\Delta_\omega g_s
=
\mathrm{tr}_{\omega}
	{\left({i_{\partial_s}\omega
	}\right)}.
\end{equation*}

Let $\tilde R_{\varepsilon}$ be the right hand side of \eqref{E:pde_vpve}. Then we have
\begin{equation*}
-\Delta_{\theta_{\varepsilon}}({\varphi}_{\varepsilon})_s
+
{\varepsilon}({\varphi}_{\varepsilon})_s
=
\tilde R_{\varepsilon}.
\end{equation*}
This is a second order elliptic partial differential equation. By \eqref{E:equivalence}, the Schauder estimate implies that there exists a constant $C>0$ which does not depend on ${\varepsilon}$ such that
\begin{equation*}
{\left\|{({\varphi}_{\varepsilon})_s}\right\|}_{C^{k,\alpha}(X_y)}
<
C.
\end{equation*}
Differentiating \eqref{E:pde_vpve} with respect to $\bar s$, we have
\begin{equation*}
-\Delta_{\theta_{\varepsilon}}({\varphi}_{\varepsilon})_{s\bar s}
+
{\varepsilon}({\varphi}_{\varepsilon})_{s\bar s}
=
{\frac{\partial{}}{\partial{\bar s}}}(\theta_{\varepsilon})^{\bar\beta\alpha}\cdot(({\varphi}_{\varepsilon})_s)_{\alpha\bar\beta}
+
(R_{\varepsilon})_{\bar s}.
\end{equation*}
Since ${\left\|{{\varphi}_{\varepsilon}}\right\|}_{C^{k,\alpha}(X_y)}$ and ${\left\|{({\varphi}_{\varepsilon})_s}\right\|}_{C^{k,\alpha}(X_y)}$ are bounded, the $C^{k,\alpha}$-norm of the right hand side is also bounded.
Therefore, the Schauder estimate completes the proof.
\end{proof}

\section{Fiberwise Ricci-flat K\"ahler forms on Calabi-Yau fibrations}\label{S:fiberwiseRFf}

In this section, we discuss the properties of the fiberwise Ricci-flat metric $\rho$. More precisely, we first discuss Schumacher's partial differential equation, which the geodesic curvature $c(\rho)$ satisfies and several applications of the PDE. Before going to the fiberwise Ricci-flat metric, we introduce hermitian metrics for the relative canonical line bundle induced by the real $(1,1)$-form on $X$ which is postive-definite when restricted on each fiber.

Let $p:X\rightarrow Y$ be a holomorphic family of compact K\"ahler manifolds. We consider a smooth $(1,1)$-form $\tau$ on $X$, whose restriction to any fiber of $p$ is positive definite. Then $\tau$ induces a metric on the bundle $K_{X/Y}$ as follows:

Let $x\in{X}$ be a point and $y=p(x)$. Let $(z^1,\dots,z^n,s^1,\dots,s^d)$ be an admissible coordinate centered at $x$.
In this coordinate, $\tau$ is written as Remark \ref{R:horizontal_lift}. Since $\tau$ is positive-definite on each fiber, $(\tau_{\alpha\bar\beta})$ is positive-definite. Hence 
$$
\sum\tau_{\alpha\bar\beta}(z,s)dz^\alpha\wedge dz^{\bar\beta}
$$ 
gives a K\"ahler metric on each fiber $X_s$. It follows that 
\begin{equation}\label{E:metric_on_RCLB}
\det(\tau_{\alpha\bar\beta}(z,s))^{-1}
\end{equation}
gives a hermitian metric on the relative line bundle $K_{X/Y}$. We denote this metric by $h^\tau_{X/Y}$. The curvature form $\Theta_{h^\tau_{X/Y}}(K_{X/Y})$ of this metric is real $(1,1)$-form on $X$ which is given by
\begin{equation*}
\Theta_{h^\tau_{X/Y}}(K_{X/Y})
=
dd^c\det(\tau_{\alpha\bar\beta}(z,s)).
\end{equation*}
\medskip

Now we consider a smooth family of Calabi-Yau manfiolds $p:X\rightarrow Y$ and suppose that $\omega$ is a fixed K\"ahler form on $X$. Under an admissible coordinate $(z^1,\dots,z^n,s^1,\dots,s^d)$ in $X$, $\omega$ is written as follows:
\begin{equation*}
\omega
=
{\sqrt{-1}}{\left({g_{i\bar j}ds^i\wedge ds^{\bar j}
+g_{i\bar\beta}ds^i\wedge{dz}^{\bar\beta} 
+g_{\alpha\bar j}dz^\alpha\wedge ds^{\bar j}
+g_{\alpha\bar\beta}dz^\alpha\wedge{dz}^{\bar\beta}
}\right)}.
\end{equation*}
Since every fiber $X_y$ is a Calabi-Yau manifold, the first Chern class $c_1(X_y)$ vanishes. Since $c_1(X_y)$ is represented by the Ricci form of $\omega_y$, we know that 
\begin{equation*}
\left[
	-dd^c\log\det(g_{\alpha\bar\beta}(\cdot,y))
\right]
=0.
\end{equation*}
By the $dd^c$-lemma, there exists a unique function $\eta_y\in C^\infty(X_y)$ such that
\begin{itemize}
\item $dd^c\eta_y=dd^c\log\det(g_{\alpha\bar\beta})$ and
\item $\int_{X_y} e^{\eta_y}(\omega_y)^n=\int_{X_y}(\omega_y)^n$.
\end{itemize}
For each $y\in Y$, there exists a unique solution ${\varphi}_y\in C^\infty(X_y)$ of the following complex Monge-Amp\`ere equation on each fiber $X_y$:
\begin{equation}
\begin{aligned}
{\left({\omega_y+dd^c\varphi_y}\right)}^n
&=
e^{\eta_y}(\omega_y)^n, \\
\omega_y+dd^c\varphi_y
&>
0,
\end{aligned}
\end{equation}
which is normalized by
\begin{equation*}
\int_{X_y}{\varphi}_y\,(\omega_y)^n=0.
\end{equation*}
Then it is easy to see that $\omega_y+dd^c\varphi_y$ is the Ricci-flat K\"ahler metric on $X_y$. As we already mentioned, we can consider ${\varphi}$ as a smooth function on $X$ by letting ${\varphi}(x)={\varphi}_y(x)$ where $y=p(x)$. Define a real $(1,1)$-form $\rho$ on $X$ by
\begin{equation*}
\rho=\omega+dd^c{\varphi}.
\end{equation*}
Then this is the fiberwise Ricci-flat metric in Theorem \ref{T:main_theorem}. 
\medskip

As we mentioned, $\rho$ induces a hermitian metric $h^\rho_{X/Y}$ on $K_{X/Y}$. To compute the curvature of $h^\rho_{X/Y}$, we write the function $\eta$ more precisely:
Fix a point $x\in X$ and fix the fiber $X_y$ where $y=p(x)$. Take an admissible coordinate system $(z^1,\dots,z^n,s^1,\dots,s^d)$ centered at $x$ in $X$. In this coordinate, $y=(0,\dots,0)$. So by abusing notations, we write $X_0=X_y$.
Since $X_0$ is a Calabi-Yau manifold, there exists a non-vanishing holomorphic $n$-form $u_0$ on $X_0$. Then Ohsawa-Takegoshi extension theorem (\cite{Berndtsson2}) implies that there exists a holomorphic $(n,0)$-form $u$ on $p^{-1}(U)$, where $U$ is a neighborhood of $y$, such that
\begin{equation*}
u\vert_{X_0}=u_0.
\end{equation*}
The following proposition gives a precise formula of $\eta$.

\begin{proposition} \label{P:key_p}
On $p^{-1}(U)$, $\eta$ is written as follows:
\begin{equation}\label{E:eta}
\eta(z,s)
=
-\log\frac{\omega^n\wedge dV_s}
	{c_n u\wedge \overline{u}\wedge dV_s}
-\log{\left\|{u}\right\|}^2_s.
\end{equation}
where $dV_s=c_d ds\wedge d\bar s$.
\end{proposition}

\begin{proof}
Denote the right hand side of \eqref{E:eta} by $\tilde\eta$.
It is enough to show the following:
\begin{itemize}
\item [1.]$\int_{X_s}e^{\tilde\eta}(\omega_s)^n=1$.
\item [2.]$dd^c\tilde\eta\vert_{X_s}=-dd^c\log\det(g_{\alpha\bar\beta})\vert_{X_s}.$
\end{itemize}
First we compute
\begin{align*}
\int_{X_s}e^{\tilde\eta}(\omega_s)^n
&=
\int_{X_s}\left[\exp{\left({-\log\frac{\omega^n\wedge dV_s}
	{c_n u\wedge \overline{u}\wedge dV_s}
-\log{\left\|{u}\right\|}_s^2}\right)}\right](\omega_s)^n \\
\end{align*}
If we write $dz=dz^1\wedge\dots\wedge dz^n$, then 
\begin{equation*}
(\omega_s)^n=\det(g_{\alpha\bar\beta})c_n dz\wedge d\bar z
\;\;\;\text{and}\;\;\;
u\vert_{X_s}=\hat u(z,s)dz
\end{equation*}
for some local holomorphic function $\hat u(z,s)$. It follows that
\begin{align*}
\int_{X_s}e^{\tilde\eta}(\omega_s)^n
&=
\int_{X_s}\exp
{\left({-\log\frac{\det(g_{\alpha\bar\beta})}{c_n{\left\vert{\hat u(z,s)}\right\vert}^2}-\log{\left\|{u}\right\|}_s^2
}\right)}
(\omega_s)^n \\
&=
\frac{1}{{\left\|{u}\right\|}_s^2}\int_{X_s}
\frac{c_n{\left\vert{\hat u(z,t)}\right\vert}^2}{\det(g_{\alpha\bar\beta})}
{\det(g_{\alpha\bar\beta})}c_n dz\wedge d\bar z \\
&=
\frac{1}{{\left\|{u}\right\|}_s^2}\int_{X_s}
\frac{{\left\vert{\hat u(z,s)}\right\vert}^2}{\det(g_{\alpha\bar\beta})}
{\det(g_{\alpha\bar\beta})}c_n dz\wedge d\bar z \\
&=
\frac{1}{{\left\|{u}\right\|}_s^2}\int_{X_s}
c_n{\left\vert{\hat u(z,s)}\right\vert}^2 dz\wedge d\bar z 
=1.
\end{align*}
This yields the first assertion. For the second assertion, 
\begin{align*}
dd^c\tilde\eta\vert_{X_s}
&=
-dd^c
{\left({\log\frac{\omega^n\wedge dV_s}
	{c_n u\wedge \overline{u}\wedge dV_s}-\log{\left\|{u_s}\right\|}^2
}\right)}\Big\vert_{X_s}
\\
&=
-dd^c
{\left({\log\det(g_{\alpha\bar\beta})
	-\log{\left\vert{\hat u(z,s)}\right\vert}^2
}\right)}\Big\vert_{X_s} 
\\
&=
-dd^c\log\det(g_{\alpha\bar\beta})\vert_{X_s}
\end{align*}
This completes the proof.
\end{proof}
Now we can compute the curvature of $h_{X/Y}^\rho$ on $K_{X/Y}$. Proposition \ref{P:key_p} implies that
\begin{align*}
\Theta_{h^\rho_{X/Y}}(K_{X/Y})
& =
dd^c\log{\left({(\omega+{\sqrt{-1}}{\partial\bar\partial}\varphi)^n\wedge{\sqrt{-1}} ds\wedge d\bar s}\right)} 
\\
& =
dd^c\log{\left({e^{\eta}\omega^n\wedge{\sqrt{-1}} ds\wedge d\bar s}\right)} 
\\
& = 
dd^c\eta+\Theta_{h^\omega_{X/Y}}(K_{X/Y}) 
\\
& =
dd^c{\left({\log{\left\vert{\hat u(z,s)}\right\vert}^2-\log\det(g_{\alpha\bar\beta})-\log{\left\|{u}\right\|}_s^2)}\right)}
\\
&\;\;\;\;+\Theta_{h^\omega_{X/Y}}(K_{X/Y}) 
\\
& =
-dd^c\log{\left\|{u}\right\|}_s^2.
\end{align*}
Let $(E,{\left\|\cdot\right\|})$ be the direct image bundle $p_*(K_{X/Y})$ of relative canonical line bundle equipped with the $L^2$ metric ${\left\|\cdot\right\|}$. Then the fiber $E_y$ is $H^0(X_y,K_{X_y})$. Since $X_y$ is Calabi-Yau, $H^0(X_y,K_{X_y})$ is a $1$-dimensional vector space. This implies that $E$ is a line bundle. Note that $u$ is a local holomorphic section of $E$. Denote by $\Theta(E)$ the curvature of $(E,{\left\|\cdot\right\|})$. Then it follows that
\begin{equation} \label{E:curvature_of_RCLB}
\Theta_{h^\rho_{X/Y}}(K_{X/Y})
=
-dd^c\log{\left\|{u}\right\|}_s^2
=
\Theta(E)
\end{equation}
Here $\Theta(E)$ means $p^*\Theta(E)$. This formula enables us to compute the Laplacian of $c(\rho)$ on each fiber $X_y$:

\begin{theorem}\label{T:PDE0}
Let $V\in T_yY$. Then the following PDE holds on $X_y$:
\begin{equation}\label{E:PDE0}
-\Delta_\rho c(\rho)(V)
=
{\left\vert{\bar\partial V_\rho}\right\vert}_\rho^2-\Theta_{V\bar V}(E).
\end{equation}
\end{theorem}
The computation is quite straight forward. Later, we will prove this for more general situation (See Theorem \ref{T:PDE}). 

\begin{remark}\label{R:sufficient_condition}
To show that $\rho$ is semi-positive on $X$, it is enough to consider a Calabi-Yau fibration over the unit disc by the following: 
\begin{itemize}
\item [1.] Let $\sigma_1$ and $\sigma_2$ be real $(1,1)$-forms on $X$. We assume that for each  disc $\gamma:{\mathbf{D}}\rightarrow Y$, we have
\begin{equation*}
\sigma_1\vert_{X_\gamma}\ge\sigma_2\vert_{X_\gamma},
\end{equation*}
where $X_\gamma:=p^{-1}(\gamma({\mathbf{D}}))$. Then we have that $\sigma_1\ge\sigma_2$ on $X$.
\item [2.] Every computation concerning the semi-positivity of $\rho$ is local in $s$-variable, which is a local coordinate in $Y$.
\end{itemize}
Therefore we only consider a famliy of Calabi-Yau manifolds over the unit disc as long as we are only interested in semi-positivity properties of $\rho$. In this case, \eqref{E:PDE0} turns out to be
\begin{equation}\label{E:PDE0'}
-\Delta_\rho c(\rho)
=
{\left\vert{\bar\partial v_\rho}\right\vert}_\rho^2-\Theta_{s\bar s}(E),
\end{equation}
where $v=\partial/\partial s$ and $\Theta_{s\bar s}(E)=\Theta(E)(v,\bar v)$. As we mentioned in Section \ref{SS:horizontal_lift}, the semi-positivity of $\rho$ is equivalent to $c(\rho)\ge0$. Thus it is enough to show that $c(\rho)\vert_{X_y}\ge0$ on a fixed fiber $X_y$.

In case of a family of canonically polarized compact complex manifolds $p:X\rightarrow{\mathbf{D}}$, Schumacher have proved that the geodesic curvature $c(\tilde\rho)$ of the form $\tilde\rho$, which is induced by the fiberwise K\"ahler-Einstein metrics of Ricci curvature $-1$, satisfies the following PDE:
\begin{equation*}
-\Delta_\rho c(\tilde\rho)+c(\tilde\rho)
=
{\left\vert{\bar\partial v_{\tilde\rho}}\right\vert}_{\tilde\rho}^2
\end{equation*}
for each fiber $X_y$ (\cite{Schumacher}). This PDE gives a lower bound of $c(\tilde\rho)$ directly by maximum principle. (More precise lower bound is also obtained using heat kernel estimates by Schumacher.) Hence the fiberwise K\"ahler-Einstein form $\tilde\rho$ is a semi-positive form on $X$. However \eqref{E:PDE0'} does not gives a lower bound. This makes it complicated to show $c(\rho)\ge0$. In the next section, we shall show that $c(\rho)\ge0$.
\end{remark}

Before going further, we discuss some applications of Theorem \ref{T:PDE0}.

\begin{proposition} \label{P:norm_dbarv}
For $V\in TY$, the following holds:
\begin{equation*}
{\left\|{\bar\partial V_\rho}\right\|}_\rho^2
=
\Theta_{V\bar V}(E).
\end{equation*}
\end{proposition}
\begin{proof}
Integrating \eqref{E:PDE0} on $X_y$ gives the conclusion. 
\end{proof}

\begin{proposition}
$\bar\partial V_\rho\rfloor u_y$ is the harmonic representative of the cohomology class $K_y(V)\cdot u_y$ with respect to $\rho\vert_{X_y}$.
\end{proposition}

\begin{proof}
Since $E$ is a line bundle, Griffiths' theorem implies that
\begin{equation*}
\Theta_{V\bar V}(E)
=
\frac{{\left\|{K_y(V)\cdot u_y}\right\|}^2}{{\left\|{u_y}\right\|}^2}.
\end{equation*}
Note that
\begin{equation*}
\bar\partial V_\rho
\in
K_y(V).
\end{equation*}
It follows that
\begin{equation*}
\frac{{\left\|{K_y(V)\cdot u_y}\right\|}^2}{{\left\|{u_y}\right\|}^2}
\le
\frac{{\left\|{\bar\partial V_\rho\rfloor u_y}\right\|}^2}{{\left\|{u_y}\right\|}^2}.
\end{equation*}
The following lemma is well known (cf, see \cite{Popovici}).
\begin{lemma}
Let $(X,\omega)$ be a Calabi-Yau manifold. Let $u$ be a non-vanishing holomorphic $n$-form on $X$ such that 
\begin{equation*}
{\left\|{u}\right\|}^2_\omega
:=\int_X {\left\vert{u}\right\vert}^2_\omega\;dV_\omega
=\int_X dV_\omega
=1.
\end{equation*}
Denote by $A^{(p,q)}(E)$ the space of smooth $(p,q)$-forms with values in $E$. Define a map 
$$
T_u:A^{(0,1)}(T'X)\rightarrow A^{(n-1,1)}(X)
$$
by $T_u(V)=V\rfloor u$.
Then $T_u$ is an isometry with respect to the pointwise scalar product induced by $\omega$.
\end{lemma}
Hence Propositioin 4.4 implies that
\begin{equation*}
{\left\|{\bar\partial V_\rho}\right\|}_\rho^2
=
\Theta_{V\bar V}(E)
=
\frac{{\left\|{K_y(V)\cdot u_y}\right\|}^2}{{\left\|{u_y}\right\|}^2}
\le
\frac{{\left\|{\bar\partial V_\rho\rfloor u_y}\right\|}^2}{{\left\|{u_y}\right\|}^2}
=
{\left\|{\bar\partial V_\rho}\right\|}_\rho^2.
\end{equation*}
It follows that $\bar\partial V_\rho\rfloor u_y$ is the harmonic representative with respect to $\rho\vert_{X_y}$ of $K_y(V)\cdot u_y$. This completes the proof.
\end{proof}

\begin{proposition}
Let $p:X\rightarrow Y$ be a Calabi-Yau fibration. If the curvature of the direct image bundle $p_*(K_{X/Y})$ vanishes along a complex curve, then the fibration is trivial along the complex curve.
\end{proposition}

\begin{proof}
Denote by $\gamma$ the complex curve in $Y$. Then $p\vert_\gamma:X_\gamma\rightarrow\gamma$ is a Calabi-Yau fibration over a $1$-dimensional base. If we take $s$ be a holomorphic coordinate of $\gamma$, then we have Equation \eqref{E:PDE0'} on each fiber $X_y$ for $y\in\gamma$. By the Hypothesis, $\Theta_{s\bar s}(E)$ vanishes on $\gamma$. Proposition \ref{P:norm_dbarv} implies that $v_\rho$ is a holomorphic vector field on $X_\gamma$. The flow of $v_\rho$ makes $X_\gamma$ a trivial fibration.
\end{proof}

\section{Proof of Theorem \ref{T:main_theorem}}

In this section we shall prove the main theorem. As we mentioned in Remark \ref{R:sufficient_condition}, it is enough to show that $c(\rho)\ge0$ for a family of Calabi-Yau manifolds over a unit disc. 

Let $p:X\rightarrow{\mathbf{D}}$ be a family of Calabi-Yau manifolds. For each ${\varepsilon}$ with $0<{\varepsilon}\le1$, we consider the following fiberwise complex Monge-Amp\`ere equation on each fiber $X_y$:
\begin{equation}\label{E:PDE1'}
\begin{aligned}
{\left({\omega_y+dd^c\psi_y}\right)}^n
&=
e^{\varepsilon\psi_y}e^{\eta_y}(\omega_y)^n
\;\;\text{and}
\\
\omega_y+dd^c\psi_y
&>0,
\end{aligned}
\end{equation}
where $\eta$ is defined in Section \ref{SS:AFCMAE}.
Theorem \ref{T:AY} implies that there exists a unique solution $\psi_{y,{\varepsilon}}\in C^\infty(X_y)$. As we mentioned, we can consider $\psi_{\varepsilon}$ as a smooth function on $X$ by letting
\begin{equation*}
\psi_{\varepsilon}(x):=\psi_{y,{\varepsilon}}(x),
\end{equation*}
where $y=p(x)$. 
We define the constant $c_\varepsilon$ by
\begin{equation*}
c_\varepsilon(y)=\int_{X_y}\psi_{\varepsilon}(\omega_y)^n.
\end{equation*}
Since $\psi_\varepsilon$ is smooth on $X$, $c_\varepsilon$ is also smooth on $Y$. Now we define $\varphi_\varepsilon$ by
\begin{equation*}
\varphi_\varepsilon
=
\psi_\varepsilon-c_\varepsilon.
\end{equation*}
Then ${\varphi}_{\varepsilon}$ is smooth on $X$ and satisfies that
\begin{equation*}
\int_{X_y}\varphi_{\varepsilon}(\omega_y)^n=0.
\end{equation*}
Moreover, it is obvious that ${\varphi}_{\varepsilon}$ is the unique solution of the following equation on each fiber $X_y$:
\begin{equation}\label{E:PDE1}
\begin{aligned}
{\left({\omega_y+dd^c{\varphi}_{\varepsilon}\vert_{X_y}}\right)}^n
&=
e^{{\varepsilon} c_{\varepsilon}}e^{{\varepsilon}{\varphi}_{\varepsilon}\vert_{X_y}}e^{\eta_y}(\omega_y)^n, 
\\
\omega_y+dd^c{\varphi}_{\varepsilon}\vert_{X_y}
&>0.
\end{aligned}
\end{equation}
We consider next the $(1,1)$-form
\begin{equation} \label{E:rho}
\rho_\varepsilon
	:=\omega+dd^c\varphi_\varepsilon
\end{equation}
on the manifold $X$. Note that $\rho_\varepsilon$ is positive definite when restricted to $X_y$. Thus, we can define a metric $h^{\rho_\varepsilon}_{X/Y}$ on the bundle $K_{X/Y}\vert_{X_0}$. By Proposition \ref{P:key_p}, its curvature is rapidly computed as follows:
\begin{align*}
\Theta_{h^{\rho_\varepsilon}_{X/Y}}(K_{X/Y})
& =
dd^c\log
{\left({
	(\omega+{\sqrt{-1}}{\partial\bar\partial}\varphi_\varepsilon)^n\wedge{\sqrt{-1}} ds\wedge d\bar s
}\right)} 
\\
& =
dd^c\log
{\left({
	e^{\varepsilon c_\varepsilon}
	e^{\varepsilon\varphi_\epsilon+\eta}
	\omega^n\wedge{\sqrt{-1}} ds\wedge d\bar s}\right)} \\
& = 
dd^c(\varepsilon c_\varepsilon+\varepsilon\varphi_\varepsilon+\eta)
+
\Theta_{h^\omega_{X/Y}}(K_{X/Y}) \\
& =
dd^c{\left({\log{\left\vert{\hat u(z,s)}\right\vert}^2-\log\det(g_{\alpha\bar\beta})-\log{\left\|{u_s}\right\|}^2)}\right)} \\
&
\;\;\;\;+\Theta_{h^\omega_{X/Y}}(K_{X/Y})
+
\varepsilon{\sqrt{-1}}{\partial\bar\partial}\varphi_\varepsilon
+ 
{\varepsilon} dd^c c_\varepsilon.
\end{align*}
From \eqref{E:rho}, we have $dd^c\varphi_\varepsilon=\rho_\varepsilon-\omega$. 
Since $\hat u(z,s)$ is holomorphic, it follows that
\begin{equation}\label{E:Ricci}
\Theta_{h^{\rho_\varepsilon}_{X/Y}}(K_{X/Y})
=
\varepsilon{\rho_\varepsilon}
-
\varepsilon\omega
+ 
{\varepsilon} dd^c c_\varepsilon
-
dd^c\log{\left\|{u_s}\right\|}^2,
\end{equation}
in the other expression,
\begin{equation*}
{\varepsilon}{\rho_\varepsilon}
=
{\varepsilon}{\left({
	\omega-dd^c c_\varepsilon
}\right)}
+
\Theta_{h^{\rho_\varepsilon}_{X/Y}}(K_{X/Y})
-
\Theta(E).
\end{equation*}
Our next claim is the geodesic curvature $c(\rho_{\varepsilon})$ satisfies a certain elliptic partial differential equation of second order on each fiber $X_y$. 

Under an admissible coordinate $(z^1,\dots,z^n,s)\in X$, $\rho_{\varepsilon}$ is written as follows:
\begin{equation*}
\rho_{\varepsilon}
=
{\sqrt{-1}}{\left({(h_{\varepsilon})_{s\bar s}ds\wedge d\bar s
+(h_{\varepsilon})_{s\bar\beta}ds\wedge{dz}^{\bar\beta} 
+(h_{\varepsilon})_{\alpha\bar s}dz^\alpha\wedge d\bar s
+(h_{\varepsilon})_{\alpha\bar\beta}dz^\alpha\wedge{dz}^{\bar\beta}
}\right)}.
\end{equation*}
For each $y\in{\mathbf{D}}$, $(h_{\varepsilon})_{\alpha\bar\beta}(\cdot,y)$ gives a K\"{a}hler metric on $X_y$. (If there is no confusion, we simply write $(h_{\varepsilon})_{\alpha\bar\beta}$.) Thus we can define contraction and covariant derivative on each $X_y$ with respect to $(h_{\varepsilon})_{\alpha\bar\beta}$. We use raising and lowering of indices as well as the semi-colon for the contractions and the covariant derivative with respect to the K\"{a}hler metric $(h_{\varepsilon})_{\alpha\bar\beta}$, respectively, on the fiber $X_y$.  We denote by $\Delta_{\rho_{\varepsilon}}=\Delta_{\rho_{\varepsilon}\vert_{X_y}}$ the Laplace operator with negative eigenvalues on the fiber $X_y$ with respect to $\rho_{\varepsilon}\vert_{X_y}$. 

By raising of indices, we can write the horizontal lift $v_{\rho_\varepsilon}$ of $v=\partial/\partial s$ with respect to $\rho_\varepsilon$ by
\begin{equation*}
v_{\rho_\varepsilon}
=
{\frac{\partial{}}{\partial{s}}}
-(h_{\varepsilon})_{s\bar\beta}(h_{\varepsilon})^{\bar\beta\alpha}{\frac{\partial{}}{\partial{z^\alpha}}}
=
{\frac{\partial{}}{\partial{s}}}
-(h_{\varepsilon}){_{{s}\phantom{{\alpha}}{}}^{\phantom{{s}}{\alpha}}}{\frac{\partial{}}{\partial{z^\alpha}}}.
\end{equation*}
Then $\bar\partial{v_{\rho_\varepsilon}}$ is a $T'X_y$-valued $(0,1)$-form which is defined by
\begin{align*}
\bar\partial{v_{\rho_\varepsilon}}
&=
\bar\partial{\left({{\frac{\partial{}}{\partial{s}}}
-(h_{\varepsilon})_{s}^{\phantom{i}\alpha}{\frac{\partial{}}{\partial{z^\alpha}}}}\right)} \\
&=
{\left({-\bar\partial{(h_{\varepsilon})}{_{{s}\phantom{{\alpha}}{}}^{\phantom{{s}}{\alpha}}}}\right)}\otimes{\frac{\partial{}}{\partial{z^\alpha}}} \\
&=
-{\frac{\partial{(h_{\varepsilon}){_{{s}\phantom{{\alpha}}{}}^{\phantom{{s}}{\alpha}}}}}{\partial{z^{\bar\beta}}}}dz^{\bar\beta}\otimes{\frac{\partial{}}{\partial{z^\alpha}}}.
\end{align*}
Since $(h_{\varepsilon})_{\alpha\bar\beta}$ is a K\"{a}hler metric and we use holomorphic coordinates, $\bar\partial v_{\rho_\varepsilon}$ is written by
\begin{equation*}
\bar\partial v_{\rho_\varepsilon}
=
-(h_{\varepsilon}){_{{s}\phantom{{\alpha}}{;\bar\beta}}^{\phantom{{s}}{\alpha}}}dz^{\bar\beta}\otimes{\frac{\partial{}}{\partial{z^\alpha}}}.
\end{equation*}
Then the geodesic curvature $c(\rho_{\varepsilon}):X\rightarrow{\mathbb{R}}$ is given by
\begin{equation*}
c(\rho_{\varepsilon})(z,s)={\left\langle{{v_{\rho_\varepsilon},v_{\rho_\varepsilon}}}\right\rangle}_{\rho_\varepsilon},
\end{equation*}
By the same computation as in \eqref{E:function}, we have
\begin{equation*}
\begin{aligned}
c(\rho_{\varepsilon})
=
(h_{\varepsilon})_{s\bar{s}}
-
(h_{\varepsilon})_{s\bar\beta}(h_{\varepsilon})^{\bar\beta\alpha}(h_{\varepsilon})_{\alpha\bar{s}}.
\end{aligned}
\end{equation*}

The following theorem is inspired by Schumacher's method in \cite{Schumacher}. Paun generalized the computation to the twisted K\"ahler-Einstein metric case (\cite{Paun2}). (See also \cite{Choi1}.)

\begin{theorem}\label{T:PDE}
The following partial differential equation holds on each fiber $X_y$:
\begin{equation*}
-\Delta_{\rho_{\varepsilon}} c(\rho_\varepsilon)
+\varepsilon c(\rho_\varepsilon)
= 
\varepsilon\omega(v_{\rho_{\varepsilon}},\overline{v_{\rho_{\varepsilon}}})
-
\varepsilon(c_\varepsilon)_{s\bar s}
+{\left\vert{\bar\partial v_{\rho_{\varepsilon}}}\right\vert}_{\rho_{\varepsilon}}^2
-\Theta_{s\bar s}(E),
\end{equation*}
where ${\left\vert{\bar\partial v_{\rho_{\varepsilon}}}\right\vert}_{\rho_{\varepsilon}}$ is the pointwise norm of $\bar\partial v_{\rho_{\varepsilon}}$ with respect to the K\"ahler metric $\rho_{\varepsilon}\vert_{X_y}$.
\end{theorem}

\begin{proof}
We fix a fiber $X_y$ and ${\varepsilon}>0$. During this proof, if there is no confusion, then we omit the subscript ${\varepsilon}$ in the components in $\rho_{\varepsilon}$ for simplicity, namely, we write as follows:
\begin{equation*}
h_{s\bar s}=(h_{\varepsilon})_{s\bar s},
\quad
h_{s\bar\beta}=(h_{\varepsilon})_{s\bar\beta}
\quad\text{and}\quad
h_{\alpha\bar\beta}=(h_{\varepsilon})_{\alpha\bar\beta}.
\end{equation*}

We have to compute the following:
\begin{equation*}
\Delta_{\rho_{\varepsilon}} c(\rho_{\varepsilon})
	=h^{\bar\delta\gamma}(c(\rho_{\varepsilon}))_{;\gamma\bar\delta}
	=h^{\bar\delta\gamma}{\left({
	h_{s\bar{s}}-h_{s\bar\beta}h^{\bar\beta\alpha}h_{\alpha\bar{s}}
	}\right)}_{;\gamma\bar\delta}.
\end{equation*}
First we consider the term $h^{\bar\delta\gamma}h_{s\bar{s};\gamma\bar\delta}$. 
Since $\omega$ is a K\"ahler form on $X$, $\rho_{\varepsilon}$ is locally ${\partial\bar\partial}$-exact. So we have that 
\begin{align*}
h_{s\bar{s};\gamma\bar\delta}
	& ={\frac{\partial{^2h_{s\bar{s}}}}{\partial{z^\gamma\partial{z}^{\bar\delta}}}}
	= {\frac{\partial{^2}}{\partial{s\partial\bar{s}}}}h_{\gamma\bar\delta}.
\end{align*}
Then it follows that
\begin{align*}
h^{\bar\delta\gamma}h_{s\bar{s};\gamma\bar\delta}
	& = h^{\bar\delta\gamma}{\frac{\partial{^2}}{\partial{s\partial\bar{s}}}}h_{\gamma\bar\delta} \\
	& = {\frac{\partial{}}{\partial{s}}}{\left({h^{\bar\delta\gamma}{\frac{\partial{}}{\partial{\bar{s}}}}h_{\gamma\bar\delta}}\right)}
	- {\frac{\partial{h^{\bar\delta\gamma}}}{\partial{s}}}{\frac{\partial{h_{\gamma\bar\delta}}}{\partial{\bar{s}}}}
	\\
	& ={\frac{\partial{^2}}{\partial{s\partial\bar{s}}}}\log{\det(h_{\alpha\bar\beta})}
	+h^{\bar\delta\alpha}{\frac{\partial{h_{\alpha\bar\beta}}}{\partial{s}}}
	h^{\bar\beta\gamma}{\frac{\partial{h_{\gamma\bar\delta}}}{\partial{\bar{s}}}}
\end{align*}
By \eqref{E:Ricci}, we have
\begin{equation*}
{\frac{\partial{^2}}{\partial{s\partial\bar{s}}}}\log{\det(h_{\alpha\bar\beta})}
=
{\varepsilon}\rho_{\varepsilon}{\left({{\frac{\partial{}}{\partial{s}}},{\frac{\partial{}}{\partial{\bar s}}}}\right)}
-
{\varepsilon}\omega{\left({{\frac{\partial{}}{\partial{s}}},{\frac{\partial{}}{\partial{\bar s}}}}\right)}
+
{\left({c_{\varepsilon}}\right)}_{s\bar s}
-
{\frac{\partial{^2}}{\partial{s\partial\bar{s}}}}\log{\left\|{u}\right\|}_s^2.
\end{equation*}
Hence it follows that
\begin{equation}\label{E:first_term}
h^{\bar\delta\gamma}h_{s\bar{s};\gamma\bar\delta}
=
{\varepsilon}{\left({h_{s\bar s}-g_{s\bar s}+(c_{\varepsilon})_{s\bar s}}\right)}
-
{\frac{\partial{^2}}{\partial{s\partial\bar{s}}}}\log{\left\|{u}\right\|}_s^2
+
h_{s\bar\beta;\alpha}
h_{\bar{s}\gamma;\bar\delta}
h^{\bar\beta\gamma}h^{\bar\delta\alpha}.
\end{equation}

Next we consider the term 
$h^{\bar\delta\gamma}{\left({h_{s\bar\beta}h^{\bar\beta\alpha}h_{\alpha\bar{s}}}\right)}_{;\gamma\bar\delta}$, which can be written by 
$$
h^{\bar\delta\gamma}{\left({h{_{{s}\phantom{{\alpha}}{}}^{\phantom{{s}}{\alpha}}} h_{\alpha\bar{s}}}\right)}_{;\gamma\bar\delta}.
$$
Define a tensor $\{A{_{{s}\phantom{{\alpha}}{\bar\beta}}^{\phantom{{s}}{\alpha}}}\}$ by 
$$
A{_{{s}\phantom{{\alpha}}{\bar\beta}}^{\phantom{{s}}{\alpha}}}=-h{_{{s}\phantom{{\alpha}}{;\bar\beta}}^{\phantom{{s}}{\alpha}}}.
$$ 
Then it follows that
\begin{equation*}
\bar\partial{v_\rho}=A{_{{s}\phantom{{\alpha}}{\bar\beta}}^{\phantom{{s}}{\alpha}}}{\frac{\partial{}}{\partial{z^\alpha}}}\otimes{dz}^{\bar\beta}.
\end{equation*}
The next term is computed by
\begin{align*}
h^{\bar\delta\gamma}{\left({h{_{{s}\phantom{{\sigma}}{}}^{\phantom{{s}}{\sigma}}}h_{\bar{s}\delta}}\right)}_{;\gamma\bar\delta}
	& = h^{\bar\delta\gamma}{\left({
	h{_{{s}\phantom{{\sigma}}{;\gamma\bar\delta}}^{\phantom{{s}}{\sigma}}}h_{\bar{s}\sigma}
	+A{_{{s}\phantom{{\sigma}}{\bar\delta}}^{\phantom{{s}}{\sigma}}}A_{\bar{s}\sigma\gamma}
	+h{_{{s}\phantom{{\sigma}}{;\bar\delta}}^{\phantom{{s}}{\sigma}}}h_{\bar{s}\sigma;\bar\delta}
	+h{_{{s}\phantom{{\sigma}}{}}^{\phantom{{s}}{\sigma}}}A_{\bar{s}\sigma\gamma;\bar\delta}
	}\right)}
	\\
	& := I_1+I_2+I_3+I_4.
\end{align*}
Before further computation, we recall the commutation formula for covariant derivatives: For every $1$-tensor $\{T^\alpha\}$,
\begin{equation} \label{E:commutation}
T{_{{}\phantom{{\alpha}}{;\bar\beta\gamma}}^{\phantom{{}}{\alpha}}}
	-T{_{{}\phantom{{\alpha}}{;\gamma\bar\beta}}^{\phantom{{}}{\alpha}}}
	=R{_{{}\phantom{{\alpha}}{\delta\bar\beta\gamma}}^{\phantom{{}}{\alpha}}}T^\delta,
\end{equation}
where $R{_{{}\phantom{{\delta}}{\alpha\bar\beta\gamma}}^{\phantom{{}}{\delta}}}$ is the Riemann curvature tensor.
\medskip

First of all, it is obvious that 
\begin{align*}
I_2
=
A{_{{s}\phantom{{\sigma}}{\bar\delta}}^{\phantom{{s}}{\sigma}}}A_{\bar{s}\sigma\gamma}h^{\bar\delta\gamma}
=
{\left\vert{\bar\partial{v_{\rho_{\varepsilon}}}}\right\vert}_{\rho_{\varepsilon}}^2.
\end{align*}
And the term $I_3$ is equal to $h_{s\bar\beta;\alpha}h_{\bar{s}\gamma;\bar\delta}h^{\bar\beta\gamma}h^{\bar\delta\alpha}$, which is appeared in \eqref{E:first_term}. So these terms are cancelled in the last computation.
Before computing $I_1$ and $I_4$, we introduce the following lemma.

\begin{lemma} \label{L:harmonic}
Let $\bar\partial^*_{\rho_{\varepsilon}}$ be the adjoint of $\bar\partial$ with respect to the $L^2$-inner product with  $\rho_\varepsilon\vert_{X_y}$, which is defined by 
\begin{equation*}
\bar\partial^*{\left({A{_{{s}\phantom{{\alpha}}{\bar\beta}}^{\phantom{{s}}{\alpha}}}
{\frac{\partial{}}{\partial{z^\alpha}}}\otimes{dz}^{\bar\beta}}\right)}
	:=h^{\bar\beta\gamma}A{_{{s}\phantom{{\alpha}}{\bar\beta;\gamma}}^{\phantom{{s}}{\alpha}}}{\frac{\partial{}}{\partial{z^\alpha}}}
\end{equation*}
Then we have the following:
\begin{equation}\label{E:dbarstar}
\bar\partial^*
{\left({
\bar\partial v_{\rho^\varepsilon}
}\right)}
=
\varepsilon
{\left({
	g_{s\bar\delta}h^{\bar\delta\alpha}
	-h_{s\bar\delta}g^{\bar\delta\alpha}
}\right)}
{\frac{\partial{}}{\partial{z^\alpha}}}.
\end{equation}
In particular, we have
\begin{equation*}
h^{\bar\beta\gamma}A{_{{s}\phantom{{\alpha}}{\bar\beta;\gamma}}^{\phantom{{s}}{\alpha}}}
=
\varepsilon
{\left({
	g_{s\bar\delta}h^{\bar\delta\alpha}
	-h_{s\bar\delta}g^{\bar\delta\alpha}
}\right)}.
\end{equation*}
\end{lemma}

\begin{proof}
Since  the Riemannian connection induced by a K\"{a}hler metric is torsion-free, we have
\begin{align*}
h^{\bar\beta\gamma}A{_{{s}\phantom{{\alpha}}{\bar\beta;\gamma}}^{\phantom{{s}}{\alpha}}}
	= -h^{\bar\beta\gamma}h^{\bar\delta\alpha}h_{s\bar\delta;\bar\beta\gamma} 
	= -h^{\bar\beta\gamma}h^{\bar\delta\alpha}h_{s\bar\beta;\bar\delta\gamma}
\end{align*}
By \eqref{E:Ricci} and \eqref{E:commutation}, it follows that
\begin{align*}
h^{\bar\beta\gamma}A{_{{s}\phantom{{\alpha}}{\bar\beta;\gamma}}^{\phantom{{s}}{\alpha}}}
&=
	 -h^{\bar\beta\gamma}h^{\bar\delta\alpha}
	{\left[{h_{s\bar\beta;\gamma\bar\delta}
	-h_{s\bar\tau}R{_{{}\phantom{{\bar\tau}}{\bar\beta\bar\delta\gamma}}^{\phantom{{}}{\bar\tau}}}
	}\right]} 
	\\
&=
	-h^{\bar\delta\alpha}
	{\left[{
	{\left({h^{\bar\beta\gamma}{\frac{\partial{h_{\bar\beta\gamma}}}{\partial{s}}}}\right)}_{;\bar\delta}
	-h_{s\bar\tau}h^{\bar\beta\gamma}R{_{{}\phantom{{\bar\tau}}{\bar\beta\bar\delta\gamma}}^{\phantom{{}}{\bar\tau}}}
	}\right]}
	\\
&=
	-h^{\bar\delta\alpha}
	{\left[{
	{\left({{\frac{\partial{}}{\partial{s}}}\log\det(h_{\alpha\bar\beta})}\right)}_{;\bar\delta}
	+h_{s\bar\tau}R{_{{}\phantom{{\bar\tau}}{\bar\delta}}^{\phantom{{}}{\bar\tau}}}
	}\right]} 
	\\
&=
	-h^{\bar\delta\alpha}
	{\left[{
	(\Theta_{h^{\rho_{\varepsilon}}_{X/Y}})_{s\bar\delta}
	+h_{s\bar\tau}h^{\bar\tau\gamma}R_{\gamma\bar\delta}
	}\right]} 
	\\
&=
	-h^{\bar\delta\alpha}
	{\left[{
	(\Theta_{h^{\rho_{\varepsilon}}_{X/Y}})_{s\bar\delta}
	-h_{s\bar\tau}h^{\bar\tau\gamma}(\Theta_{h^{\rho_{\varepsilon}}_{X/Y}})_{\gamma\bar\delta}
	}\right]} 
	\\
&=
	-\varepsilon
	h^{\bar\delta\alpha}
	{\left[{
	h_{s\bar\delta}-g_{s\bar\delta}
	-h_{s\bar\tau}h^{\bar\tau\gamma}
	{\left({h_{\gamma\bar\delta}-g_{\gamma\bar\delta}}\right)}
	}\right]}
	\\
	&=
	\varepsilon
	{\left({
		g_{s\bar\delta}h^{\bar\delta\alpha}
		-h_{s\bar\delta}g^{\bar\delta\alpha}
	}\right)}
\end{align*}
This completes the proof.
\end{proof}

Next we compute the term $I_1$.
\begin{align*}
I_1 
&=
	h_{\bar{s}\sigma}h{_{{s}\phantom{{\sigma}}{;\gamma\bar\delta}}^{\phantom{{s}}{\sigma}}}h^{\bar\delta\gamma} 
	\\
	&=
	h_{\bar{s}\sigma}
	{\left({
		-A{_{{s}\phantom{{\sigma}}{\bar\delta;\gamma}}^{\phantom{{s}}{\sigma}}}h^{\bar\delta\gamma} 
		+h{_{{s}\phantom{{\lambda}}{}}^{\phantom{{s}}{\lambda}}}R{_{{}\phantom{{\sigma}}{\lambda\gamma\bar\delta}}^{\phantom{{}}{\sigma}}}
		h^{\bar\delta\gamma}
	}\right)} 
	\\
&=
	h_{\bar{s}\sigma}
	{\left[{
		-\varepsilon
		{\left({
			g_{s\bar\delta}h^{\bar\delta\sigma}
			-h_{s\bar\delta}g^{\bar\delta\sigma}
		}\right)}
		-h{_{{s}\phantom{{\lambda}}{}}^{\phantom{{s}}{\lambda}}}R{_{{}\phantom{{\sigma}}{\lambda}}^{\phantom{{}}{\sigma}}}
	}\right]}
	\\
&=
	h_{\bar{s}\sigma}
	{\left[{
		-\varepsilon
		{\left({
			g_{s\bar\delta}h^{\bar\delta\sigma}
			-h_{s\bar\delta}g^{\bar\delta\sigma}
		}\right)}
		-h_{s\bar\lambda}R^{\sigma\bar\lambda}
	}\right]}
	\\
&=
	h_{\bar{s}\sigma}
	{\left[{
		-\varepsilon
		{\left({
			g_{s\bar\delta}h^{\bar\delta\sigma}
			-h_{s\bar\delta}g^{\bar\delta\sigma}
		}\right)}
		+h_{s\bar\lambda}\varepsilon
		{\left({h^{\sigma\bar\lambda}-g^{\sigma\bar\lambda}
		}\right)}
	}\right]}
	\\
&=
	\varepsilon
	h_{\bar{s}\sigma}
	{\left[{
		-g_{s\bar\delta}h^{\bar\delta\sigma}
		+h_{s\bar\delta}g^{\bar\delta\sigma}
		+
		h_{s\bar\lambda}{\left({
			h^{\sigma\bar\lambda}-g^{\sigma\bar\lambda}
		}\right)}
	}\right]}
	\\
&=
	\varepsilon{\left({
	h_{s\bar\beta}h^{\bar\beta\alpha}h_{\alpha\bar s}
	-
	g_{s\bar\beta}h^{\bar\beta\alpha}h_{\alpha\bar s}
	}\right)}.
\end{align*}
Finally we compute the term $I_4$.
\begin{align*}
I_4
&=
	h^{\gamma\bar\delta}h{_{{s}\phantom{{\sigma}}{}}^{\phantom{{s}}{\sigma}}}A_{\bar{s}\sigma\gamma;\bar\delta}
	\\
&=
	h_{s\bar\sigma}
	h^{\gamma\bar\delta}A{_{{\bar{s}}\phantom{{\bar\sigma}}{\gamma;\bar\delta}}^{\phantom{{\bar{s}}}{\bar\sigma}}}
	\\
&=
	h_{s\bar\sigma}
	\varepsilon{\left({
		g_{\bar s\delta}h^{\delta\bar\sigma}
		-h_{\bar s\delta}g^{\delta\bar\sigma}
	}\right)}
	\\
&=
	\varepsilon{\left({
	h_{s\bar\beta}h^{\bar\beta\alpha}g_{\alpha\bar s}
	-
	h_{s\bar\beta}g^{\bar\beta\alpha}h_{\alpha\bar s}
	}\right)}.
\end{align*}
Together with all computation, it follows that
\begin{align*}
\Delta_{\rho_{\varepsilon}} c(H)
&=
{\varepsilon}(h_{s\bar s}-g_{s\bar s}+(c_{\varepsilon})_{s\bar s})
+
{\frac{\partial{^2}}{\partial{s\partial\bar{s}}}}\log{\left\|{u}\right\|}_s^2
-
{\left\vert{\bar\partial{v_{\rho_{\varepsilon}}}}\right\vert}_{\rho_{\varepsilon}}^2
\\
&\;\;\;-
	\varepsilon{\left({
	h_{s\bar\beta}h^{\bar\beta\alpha}h_{\alpha\bar s}
	-
	g_{s\bar\beta}h^{\bar\beta\alpha}h_{\alpha\bar s}
	}\right)}
	\\
&\;\;\;-
	\varepsilon{\left({
	h_{s\bar\beta}h^{\bar\beta\alpha}g_{\alpha\bar s}
	-
	h_{s\bar\beta}g^{\bar\beta\alpha}h_{\alpha\bar s}
	}\right)}
	\\
&=
	{\frac{\partial{^2}}{\partial{s\partial\bar{s}}}}\log{\left\|{u}\right\|}_s^2
	-
	{\left\vert{\bar\partial{v_{\rho_{\varepsilon}}}}\right\vert}_{\rho_{\varepsilon}}^2
	+
	{\varepsilon}	
	{\left({h_{s\bar s}-h_{s\bar\beta}h^{\bar\beta\alpha}h_{\alpha\bar s}
	}\right)}
	\\
&\;\;\;+	
	\varepsilon{\left({g_{s\bar s}-g_{s\bar\beta}h^{\bar\beta\alpha}h_{\alpha\bar s}
		-h_{s\bar\beta}h^{\bar\beta\alpha}g_{\alpha\bar s}
		+h_{s\bar\beta}g^{\bar\beta\alpha}h_{\alpha\bar s}
	}\right)}
	+{\varepsilon}{\left({c_{\varepsilon}}\right)}_{s\bar s}.
\end{align*}
Note that 
\begin{equation*}
\omega(v_{\rho_{\varepsilon}},\overline{v_{\rho_{\varepsilon}}})
=
g_{s\bar s}-g_{s\bar\beta}h^{\bar\beta\alpha}h_{\alpha\bar s}
	-h_{s\bar\beta}h^{\bar\beta\alpha}g_{\alpha\bar s}
	+h_{s\bar\beta}g^{\bar\beta\alpha}h_{\alpha\bar s}.
\end{equation*}
Therefore, we have
\begin{equation*}
-\Delta_{\rho_{\varepsilon}} c(\rho_\varepsilon)
+\varepsilon c(\rho_\varepsilon)
=
{\varepsilon}{\left({\omega(v_{\rho_{\varepsilon}},\overline{v_{\rho_{\varepsilon}}})
	-{\left({c_{\varepsilon}}\right)}_{s\bar s}
}\right)}
+
{\left\vert{\bar\partial{v_{\rho_{\varepsilon}}}}\right\vert}_{\rho_{\varepsilon}}^2
+
{\frac{\partial{^2}}{\partial{s\partial\bar{s}}}}\log{\left\|{u}\right\|}_s^2.
\end{equation*}
Therefore, we have the conclusion.
\end{proof}

\begin{corollary}
Let $\rho$ be a fiberwise Ricci-flat metric in Theorem \ref{T:main_theorem}. Then the following holds:
\begin{equation*}
-\Delta_\rho c(\rho)={\left\vert{\bar\partial v_\rho}\right\vert}_\rho^2-\Theta_{s\bar s}(E).
\end{equation*}
\end{corollary}

\begin{proof}
Recall that the fiberwise Ricci-flat form $\rho$ satisfies the following:
\begin{equation*} 
\Theta_{h^\rho_{X/Y}}(K_{X/Y})
=
-dd^c\log{\left\|{u}\right\|}_s^2
=
\Theta(E)
\end{equation*}
If we apply the same computation with the proof of Theorem \ref{T:PDE} to $\rho$ using the above equation, then we have the conclusion.

On the other hand, it is also a easy consequence of the convergence of the K\"ahler metric $\rho_{\varepsilon}\vert_{X_y}$ to $\rho\vert_{X_y}$ as ${\varepsilon}\rightarrow0$ by passing through a subsequence for each $y\in Y$. This will be proved in the next section.
\end{proof}

Proposition \ref{P:heat_kernel} and Theorem \ref{T:PDE} gives a lower bound of $c(\rho_{\varepsilon})$:
\begin{equation*}
\inf c(\rho_\varepsilon)
\ge
C\int_{X_y}
{\left[{
	{\varepsilon}{\left({\omega(v_{\rho_{\varepsilon}},\overline{v_{\rho_{\varepsilon}}})
	-(c_{\varepsilon})_{s\bar s}
	}\right)}
	+{\left\vert{\bar\partial v_{\rho_{\varepsilon}}}\right\vert}_{\rho_{\varepsilon}}^2
	-\Theta^E_{s\bar s}
}\right]}
dV_{\rho_\varepsilon},
\end{equation*}
where $C$ only depends on the dimension $n$, the diameter of the fiber $X_y$  and $\varepsilon$. Since the volume of $X_y$ with respect to $dV_{\rho_{\varepsilon}}$ is 1, it follows that
\begin{equation}
\inf c(\rho_\varepsilon)
\ge
C{\varepsilon}{\left[{\omega(v_{\rho_{\varepsilon}},\overline{v_{\rho_{\varepsilon}}})
		-(c_{\varepsilon})_{s\bar s}
	}\right]}
	+
	C\int_{X_y}{\left({
		{\left\vert{\bar\partial v_{\rho_{\varepsilon}}}\right\vert}_{\rho_{\varepsilon}}^2
		-\Theta^E_{s\bar s}
	}\right)}
dV_{\rho_\varepsilon}.
\end{equation}
\begin{proposition} \label{P:conv_geo_curv}
On each fiber $X_y$, there exists a sequence $\{{\varepsilon}_j\}_{j\in{\mathbb{N}}}$ converging to $0$ as $j\rightarrow\infty$ such that
\begin{equation*}
c(\rho_{{\varepsilon}_j})\rightarrow c(\rho)
\;\;\;\text{and}\;\;\;
\bar\partial v_{\rho_{{\varepsilon}_j}}
\rightarrow
\bar\partial v_{\rho}
\;\;\;\text{as}\;\;\;
j\rightarrow\infty.
\end{equation*}
\end{proposition}
\begin{proof}
See Section \ref{S:app_geo_curv}.
\end{proof}

From Section 3, we already know that $\rho_{\varepsilon}\vert_{X_y}$ converges to $\rho\vert_{X_y}$ as ${\varepsilon}\rightarrow0$ by passing through a subsequence. 
Since ${\left\|{(\psi_{\varepsilon})_{s\bar s}}\right\|}$ is uniformly bounded by Proposition \ref{P:approximation2}, so is $(c_{\varepsilon})_{s\bar s}$. 
Hence the first term of lower bound in \eqref{E:lower_bound} converges to $0$ when ${\varepsilon}$ goes to $0$. 
Hence Proposition \ref{P:conv_geo_curv} gives the lower bound of $c(\rho)$:
\begin{equation}\label{E:lower_bound}
\inf c(\rho)
\ge
C\int_{X_y}
{\left({
		{\left\vert{\bar\partial v_\rho}\right\vert}_{\rho}^2
		-\Theta^E_{s\bar s}
}\right)}
dV_\rho.
\end{equation}
Proposition \ref{P:norm_dbarv} implies that
\begin{equation*}
\Theta^E_{s\bar s}
=
{\left\|{\bar\partial v_\rho}\right\|}_{\rho}^2.
\end{equation*}
Therefore $\inf c(\rho)\ge0$. 
This completes the proof.

\section{Approximation of the geodesic curvature}\label{S:app_geo_curv}

In this section, we shall prove Proposition \ref{P:conv_geo_curv}.

Let $p:X\rightarrow{\mathbf{D}}$ be a Calabi-Yau fibration and let $\omega$ be a fixed K\"ahler form on $X$. For each fiber $X_y$, we have the solution ${\varphi}_y$ of the following complex Monge-Amp\`ere equation:
\begin{equation}\label{E:CMAE0'}
\begin{aligned}
{\left({\omega_y+dd^c\varphi_y}\right)}^n &= e^{\eta\vert_{X_y}}(\omega_y)^n, \\
\omega_y+dd^c&\varphi_y>0,
\end{aligned}
\end{equation}
which is normalized by 
\begin{equation}\label{E:normalization}
\int_{X_y}\varphi_y(\omega_y)^n=1.
\end{equation}
For each fiber $X_y$ and each $0<{\varepsilon}\le1$, we also have the solution ${\varphi}_{y,{\varepsilon}}$ of the following complex Monge-Amp\`ere equation:
\begin{equation}\label{E:CMAE1'}
\begin{aligned}
{\left({\omega_y+dd^c{\varphi}_{\varepsilon}}\right)}^n &= 
e^{{\varepsilon} c_{\varepsilon}+\eta\vert_{X_y}}e^{{\varepsilon}{\varphi}_{\varepsilon}}(\omega_y)^n \\
\omega+dd^c&{\varphi}_{\varepsilon}>0.
\end{aligned}
\end{equation}

\begin{theorem}
For any fixed fiber $X_y$, the following holds:
\begin{equation*}
{\varphi}_{\varepsilon}\rightarrow {\varphi},
\;\;\;
({\varphi}_{\varepsilon})_s\rightarrow {\varphi}_s
\;\;\;
\text{and}
\;\;\;
({\varphi}_{\varepsilon})_{s\bar s}\rightarrow {\varphi}_{s\bar s}
\end{equation*}
as ${\varepsilon}\rightarrow0$ in $C^{k,\alpha}(X_y)$-topology for any $k\in{\mathbb{N}}$ and $\alpha\in(0,1)$ by passing through a subsequence.
\end{theorem}
It is obvious that this proposition implies Proposition \ref{P:conv_geo_curv}. 
\medskip

In the proof, we fix a fiber $X_y$ and omit the subscript $y$, if there is no confusion. Every convergence means the convergence by passing through a subsequence in the topology of $C^{k,\alpha}(X_y)$ for any $k\in{\mathbb{N}}$ and $\alpha\in(0,1)$. 

Proposition \ref{P:approximation1} says that for any $k\in{\mathbb{N}}$ and $\alpha\in(0,1)$, 
\begin{equation*}
{\left\|{{\varphi}_{\varepsilon}}\right\|}_{C^{k,\alpha}(X_y)}<C.
\end{equation*}
So $\{c_{\varepsilon}\}$ is bounded, it follows that Equation \eqref{E:CMAE1'} converges to Equation \eqref{E:CMAE0'} as ${\varepsilon}\rightarrow0$. Moreover, it also implies that there exists a smooth function $\hat{\varphi}\in C^\infty(X_y)$ such that ${\varphi}_{\varepsilon}$ converges to $\hat{\varphi}$ as ${\varepsilon}\rightarrow0$. Note that every ${\varphi}_{\varepsilon}$ satisfies that
\begin{equation*}
\int_{X_y}{\varphi}_{\varepsilon}(\omega_y)^n
=
0.
\end{equation*}
Hence $\hat{\varphi}$ is the unique solution of \eqref{E:CMAE0'} with the normalization:
\begin{equation*}
\int_{X_y}\hat{\varphi}(\omega_y)^n
=
0.
\end{equation*}
By the uniqueness of the solution of \eqref{E:CMAE0'}, ${\varphi}=\hat{\varphi}$. This conclude the first assertion.
\medskip

In an admissible coordinate $(z^1,\dots,z^n,s)$, the first equation of \eqref{E:CMAE1'} is written as follows:
\begin{equation} \label{E:FCMAE}
\det(g_{\alpha\bar\beta}+(\varphi_{\varepsilon})_{\alpha\bar\beta})
=
e^{\varepsilon c_\varepsilon}e^{\varepsilon\varphi_{\varepsilon}+\eta}
\det(g_{\alpha\bar\beta}).
\end{equation}
Taking logarithm of \eqref{E:FCMAE} and differentiating it with respect to $s$, we have
\begin{equation*}
\Delta_{\rho_{\varepsilon}}{\left({g_s+({\varphi}_{\varepsilon})_s}\right)}
=
\varepsilon({\varphi}_{\varepsilon})_s
+
\eta_s+{\varepsilon}(c_{\varepsilon})_s
+
\Delta_\omega g_s,
\end{equation*}
where $\Delta_{\rho_{\varepsilon}}$ and $\Delta_\omega$ are Laplace-Beltrami operators of $\rho_{\varepsilon}$ and $\omega$, repsectively. It follows that
\begin{equation} \label{E:pde1}
-\Delta_{\rho_{\varepsilon}}({\varphi}_{\varepsilon})_s
+
\varepsilon({\varphi}_{\varepsilon})_s
=
-\eta_s-{\varepsilon}(c_{\varepsilon})_s
+
(\Delta_{\rho_{\varepsilon}}-\Delta_\omega){g}_s.
\end{equation}
We denote the right hand side by $R_{\varepsilon}$. Hence ${\varphi}_{\varepsilon}$ satisfies the following equation fiberwise:
\begin{equation} \label{E:pde1'}
-\Delta_{\varepsilon}({\varphi}_{\varepsilon})_s
+
\varepsilon({\varphi}_{\varepsilon})_s
=
R_{\varepsilon}.
\end{equation}
Proposition \ref{P:approximation2} implies that there exists a uniform constant $C>0$ such that 
\begin{equation*}
{\left\|{({\varphi}_{\varepsilon})_s}\right\|}_{C^{k,\alpha}(X_y)}<C.
\end{equation*}
Hence $\{({\varphi}_{\varepsilon})_s\}$ converges to some smooth function on $X_y$ in $C^{k,\alpha}(X_y)$-topology. We call the limit $\Phi$. Then it is obvious that $\psi$ satisfies the following:
\begin{equation*}
-\Delta_\rho \Phi
=
R_0,
\end{equation*}
where $R_0=\lim_{{\varepsilon}\rightarrow0}R_{\varepsilon}$. On the other hand, the same computation implies that ${\varphi}_s$ satisfies that
\begin{equation*}
-\Delta_\rho{\varphi}_s
=
R,
\end{equation*}
where 
$$
R=-\eta_s+(\Delta_\rho-\Delta_\omega)g_s.
$$
Since $R_0=R$, the following lemma says that $\Phi={\varphi}_s$. This yields that $({\varphi}_{\varepsilon})_s$ converges to ${\varphi}_s$ as ${\varepsilon}$ goes to $0$.

\begin{lemma} \label{L:bdry1}
The following holds:
\begin{equation*}
\lim_{{\varepsilon}\rightarrow\infty}
\int_{X_y}({\varphi}_{\varepsilon})_s(\omega_y)^n
=
\int_{X_y}{\varphi}_s(\omega_y)^n.
\end{equation*}
\end{lemma}

\begin{proof}
Note that ${\varphi}$ and ${\varphi}_{\varepsilon}$ satisfy that
\begin{equation*}
f_*({\varphi}_{\varepsilon}\omega^n)
=
\int_{X_y}{\varphi}_{\varepsilon}\,\omega^n
=0
\quad\text{and}\quad
f_*({\varphi}\omega^n)
=
\int_{X_y}{\varphi}\,\omega^n
=0.
\end{equation*}
Since the push forward $f_*$ commutes the exterior differentiation, it follows that
\begin{equation*}
\partial f_*({\varphi}_{\varepsilon}\omega^n)
=
f_*{\left[{(\partial{\varphi}_{\varepsilon})\wedge\omega^n}\right]}
=
f_*
{\left[{
	{\left({{\frac{\partial{{\varphi}_{\varepsilon}}}{\partial{s}}}ds+{\frac{\partial{{\varphi}_{\varepsilon}}}{\partial{z^\gamma}}}dz^\gamma
	}\right)}\wedge\omega^n
}\right]},
\end{equation*}
or
\begin{equation*}
f_*
{\left[{
	{\left({({\varphi}_{\varepsilon})_sds+({\varphi}_{\varepsilon})_\gamma dz^\gamma
	}\right)}\wedge\omega^n
}\right]}
=0.
\end{equation*}
By the same way, we have
\begin{equation*}
\partial f_*({\varphi}\omega^n)
=
{\left[{
	{\left({{\varphi}_sds+{\varphi}_\gamma dz^\gamma
	}\right)}\wedge\omega^n
}\right]}=0.
\end{equation*}
Since ${\varphi}_{\varepsilon}$ converges to ${\varphi}$ as ${\varepsilon}\rightarrow0$ in $C^{k,\alpha}$-topology, it follows that
\begin{equation*}
f_*
{\left[{({\varphi}_{\varepsilon})_\gamma dz^\gamma\wedge\omega^n
}\right]}
\rightarrow
f_*
{\left[{{\varphi}_\gamma dz^\gamma\wedge\omega^n
}\right]}
\;\;\;
\text{as}
\;\;\;
{\varepsilon}\rightarrow0.
\end{equation*}
It follows that
\begin{equation*}
f_*
{\left[{({\varphi}_{\varepsilon})_s ds\wedge\omega^n
}\right]}
\rightarrow
f_*
{\left[{{\varphi}_s ds\wedge\omega^n
}\right]}
\;\;\;
\text{as}
\;\;\;
{\varepsilon}\rightarrow0.
\end{equation*}
Note that
\begin{equation*}
f_*
{\left[{({\varphi}_{\varepsilon})_s ds\wedge\omega^n
}\right]}
=
{\left[{\int_{X_y}({\varphi}_{\varepsilon})_s(\omega_y)^n
}\right]}
ds
\;\;\;
\text{and}
\;\;\;
f_*
{\left[{{\varphi}_s ds\wedge\omega^n
}\right]}
=
{\left[{\int_{X_y}{\varphi}_s(\omega_y)^n
}\right]}
ds.
\end{equation*}
This completes the proof.
\end{proof}

It remains to show the last assertion. Differentiating \eqref{E:pde1'} with respect to $\bar s$, we have
\begin{equation*}
-\Delta_{\rho_{\varepsilon}}({\varphi}_{\varepsilon})_{s\bar s}
+
\varepsilon({\varphi}_{\varepsilon})_{s\bar s}
=
{\frac{\partial{}}{\partial{s}}}(h^{\varepsilon})^{\bar\beta\alpha}\cdot(({\varphi}_{\varepsilon})_s)_{\alpha\bar\beta}
+
(R_{\varepsilon})_{\bar s}.
\end{equation*}
By the same way, ${\varphi}_{s\bar s}$ satisfies that
\begin{equation}\label{E:PDEss}
-\Delta_\rho{\varphi}_{s\bar s}
=
{\frac{\partial{}}{\partial{s}}}h^{\bar\beta\alpha}\cdot({\varphi}_s)_{\alpha\bar\beta}
+
(R)_{\bar s}.
\end{equation}
Proposition \ref{P:approximation2} implies that there exists a uniform constant $C>0$ such that 
\begin{equation*}
{\left\|{({\varphi}_{\varepsilon})_{s\bar s}}\right\|}_{C^{k,\alpha}(X_y)}<C.
\end{equation*}
Hence we have a subsequential limit $\hat\Phi$ of $\{({\varphi}_{\varepsilon})_{s\bar s}\}$ as ${\varepsilon}\rightarrow0$. By the same arguments, $\hat\Phi$ satisfies \eqref{E:PDEss}. The following lemma says that $\hat\Phi={\varphi}_{s\bar s}$. This completes the proof.
\begin{lemma} \label{L:boundary2}
The following holds:
\begin{equation*}
\lim_{{\varepsilon}\rightarrow\infty}
\int_{X_y}({\varphi}_{\varepsilon})_{s\bar s}\,(\omega_y)^n
=
\int_{X_y}{\varphi}_{s\bar s}\,(\omega_y)^n.
\end{equation*}
\end{lemma}

\begin{proof}
The proof is essentially same with the one of Lemma \ref{L:bdry1}. We write $dd^c{\varphi}_{\varepsilon}$ as follows:
\begin{align*}
dd^c{\varphi}_{\varepsilon}
=
{\sqrt{-1}}({\varphi}_{\varepsilon})_{s\bar s}ds\wedge d\bar s
+
\Sigma_{\varepsilon}
\end{align*}
where
\begin{equation*}
\Sigma_{\varepsilon}
=
{\sqrt{-1}}{\left[{({\varphi}_{\varepsilon})_{s\bar\beta}ds\wedge{dz}^{\bar\beta} 
+({\varphi}_{\varepsilon})_{\alpha\bar s}dz^\alpha\wedge d\bar s
+({\varphi}_{\varepsilon})_{\alpha\bar\beta}dz^\alpha\wedge dz^{\bar\beta}
}\right]}.
\end{equation*}
Likewise we write $dd^c{\varphi}$ as follows:
\begin{align*}
dd^c{\varphi}
=
{\sqrt{-1}}{\varphi}_{s\bar s}ds\wedge d\bar s
+
\Sigma
\end{align*}
where
\begin{equation*}
\Sigma
=
{\sqrt{-1}}{\left[{{\varphi}_{s\bar\beta}ds\wedge{dz}^{\bar\beta} 
+{\varphi}_{\alpha\bar s}dz^\alpha\wedge d\bar s
+{\varphi}_{\alpha\bar\beta}dz^\alpha\wedge dz^{\bar\beta}
}\right]}.
\end{equation*}
Since ${\varphi}_{\varepsilon}$ converges to ${\varphi}$ and $({\varphi}_{\varepsilon})_s$ converges to ${\varphi}_s$ as ${\varepsilon}\rightarrow0$ in $C^{k,\alpha}$-topology on $X_y$, it is obvious that every coefficient function in $\Sigma_{\varepsilon}$ converges to the corresponding coefficient function in $\Sigma$.

Recall that ${\varphi}$ and ${\varphi}_{\varepsilon}$ satisfy that
\begin{equation*}
f_*({\varphi}_{\varepsilon}\omega^n)
=
\int_{X_y}{\varphi}_{\varepsilon}\,\omega^n
=0
\quad\text{and}\quad
f_*({\varphi}\omega^n)
=
\int_{X_y}{\varphi}\,\omega^n
=0.
\end{equation*}
Taking $dd^c$, it follows that
\begin{equation*}
p_*(dd^c{\varphi}_{\varepsilon}\wedge\omega^n)
=
p_*(dd^c{\varphi}\wedge\omega^n),
\end{equation*}
or
\begin{align*}
p_*{\left({({\varphi}_{\varepsilon})_{s\bar s}\omega^n\wedge
{\sqrt{-1}} ds\wedge d\bar s}\right)}
&+
p_*(\Sigma_{\varepsilon}\wedge\omega^n) 
\\
&=
p_*{\left({{\varphi}_{s\bar s}\omega^n\wedge
{\sqrt{-1}} ds\wedge d\bar s}\right)}
+
p_*(\Sigma\wedge\omega^n).
\end{align*}
Note that 
\begin{equation*}
p_*{\left({{\varphi}_{s\bar s}\omega^n\wedge
{\sqrt{-1}} ds\wedge d\bar s}\right)}
=
{\left[{
	\int_{X_y}{\varphi}_{s\bar s}(\omega_y)^n
}\right]}
\cdot{\sqrt{-1}} ds\wedge d\bar s.
\end{equation*}
This completes the proof.
\end{proof}

\begin{thebibliography} {2010}

\bibitem{Aubin} Aubin, T., \emph{Equations du type Monge-Amp\`ere sur les vari\'et\'es k\"ahleriennes compactes}, C. R. Acad. Sci. Paris Sér. A-B 283 (1976), no. 3, Aiii, A119--A121.

\bibitem{Berndtsson1} B. Berndtsson, \emph{Curvature of vector bundles associated to holomorphic fibrations}, Ann. of Math. (2) 169 (2009), no. 2, 531--560.

\bibitem{Berndtsson2} B. Berndtssom, \emph{Strict and nonstrict positivity of direct image bundles}, Math. Z. (2011), no. 3--4, 1201--1218.

\bibitem{Berndtsson_Paun}  B. Berndtsson, M. P\v{a}un, \emph{Bergman kernels and the pseudoeffectivity of relative canonical bundles}, Duke Math. J. 145 (2008), no. 2, 341--378

\bibitem{Cheeger:Yau} Cheeger, J, Yau, S.-T., \emph{A lower bound on heat kernel}, Comm. Pure Appl. Math. 34 (1981), no. 4, 465--480. 

\bibitem{Choi1} Choi, Y.-J., \emph{Variations of K\"ahler-Einstein metrics on strongly pseudoconvex domains}, Math. Ann. 362 (2015) no. 1--2, 121--146.

\bibitem{Choi2} Choi, Y.-J., \emph{A study of variations of pesudoconex domains via K\"ahler-Einstein metrics}, to appear in Math. Z.

\bibitem{DeTurck_Kazdan} DeTurck, D. M., Kazdan, J. L., \emph{Some regularity theorems in Riemannian geometry}, Ann. Sci. Ecole Norm. Sup. (4) 14 (1981), no. 3, 249--260.

\bibitem{Di Nezza_Lu} Di Nezza, E., Lu, H. C., \emph{Complex Monge-Amp\`ere equations on quasi-projective varieties}, arXiv:1401.6398.

\bibitem{Evans} Evans, L. C., 
\emph{Classical solutions of fully nonlinear, convex, second order elliptic equations},
Comm. Pure Appl. Math 25 (1982), 333–363.

\bibitem{Guedj_Zerihai} Guedj, V., Zeriahi, A., \emph{Intrinsic capacities on compact K\"ahler manifolds}, J. Geom. Anal. 15 (2005), no. 4, 607--639. 

\bibitem{Gilbarg_Trudinger} Gilbarg, D., Trudinger, N, \emph{Elliptic partial differential equations of second order}, Grundlehren der Mathematischen Wissenschaften, Vol. 224. Springer-Verlag, Berlin-New York, 1977. x+401 pp.

\bibitem{Griffiths}  Griffiths, P.A., \emph{Curvature properties of the Hodge bundles (Notes written by Loring Tu) Topics in Transcendental Algebraic Geometry}, Annals of Mathematics Studies. Princeton University Press, Princeton (1984)

\bibitem{Kodaira}  Kodaira, K. \emph{Complex manifolds and deformation of complex structures}, Grundlehren der Mathematischen Wissenschaften, 283. Springer-Verlag, New York, 1986. x+465 pp.

\bibitem{Kodaira:Spencer} Kodaira, K., Spencer, D. C., 
\emph{On deformations of complex analytic structures. I, II},
Ann. of Math. (2) 67 1958 328--466. 

\bibitem{Kolodziej} Kolodziej, S., \emph{The complex Monge-Amp\`ere equation}, Acta Math., 180 (1998), 69--117.

\bibitem{Krylov} Krylov, N.V., 
\emph{Boundedly nonhomogeneous elliptic and parabolic equations}, 
Izvestia Akad. Nauk. SSSR 46 (1982), 487–523; English translation in Math. USSR Izv. 20
(1983), no. 3, 459--492.

\bibitem{Paun1} P\v{a}un, M., 
\emph{Regularity properties of the degenerate Monge-Amp\`ere equations on compact K\"ahler manifolds}, 
Chin. Ann. Math. 29B(6), 2008, 623--630.

\bibitem{Paun2} P\v{a}un, M., 
\emph{Relative adjoint transcendental classes and Albanese maps of compact K\"ahler manifolds with nef Ricci curvature}, 
arxiv:1209.2195[math.CV]

\bibitem{Popovici} Popovich, D., \emph{Holomorphic deformations of balanced Calabi-Yau $\partial\bar\partial$-manifolds}, arXiv:1304.0331v1 [math.AG]. 

\bibitem{Schumacher} Schumacher, G.,\emph{Positivity of relative canonical bundles and applications}, Invent. Math. 190 (2012), no. 1, 1--56.

\bibitem{Semmes} Semmes, S., \emph{Interpolation of Banach spaces, differential geometry and differential equations}, Rev. Mat. Iberoamericana 4(1), 155--176 (1988).

\bibitem{Siu} Siu, Y.-T.,
\emph{Lectures on Hermitian-Einstein metrics for stable bundles and K\"ahler-Einstein metrics}, 
DMV Seminar, 8. Birkhäuser Verlag, Basel, 1987. 171 pp.

\bibitem{Sturm} Sturm, K. T., 
\emph{Heat kernel bounds on manifolds}, 
Math. Ann. 292, 149--162 (1992)

\bibitem{Tian} Tian, G., 
\emph{Canonical metrics in K\"ahler geometry}, 
Notes taken by Meike Akveld. Lectures in Mathematics ETH Zürich. Birkhäuser Verlag, Basel, 2000. vi+101 pp.

\bibitem{Yau} Yau, S.-T.,
\emph{On the Ricci curvature of a compact K\"ahler manifold and the complex Monge-Amp\`ere equation. I}, 
Comm. Pure Appl. Math., no. 31(3), 339--411, 1978.

\bibitem{Zeriahi} Zeriahi, A., \emph{Volume and capacity of sub level sets of a Lelong class of plurisubharmonic functions}, Indiana Univ. Math. J. 50 (2001), 671--703.

\end{thebibliography}

\end{document}


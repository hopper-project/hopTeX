\documentclass[12pt]{amsart}

\usepackage{graphicx}
\usepackage{mathptmx}

\usepackage{amscd}
\usepackage{amsthm}
\usepackage{amsxtra}
\usepackage{a4wide}
\usepackage{latexsym}
\usepackage{amssymb}
\usepackage{amsfonts}
\usepackage{amsmath}
\usepackage{amsrefs}
\usepackage{mathrsfs}

\usepackage{upref}
\usepackage{txfonts}

\usepackage[bookmarksnumbered, colorlinks, plainpages]{hyperref}
\hypersetup{colorlinks=true,linkcolor=red, anchorcolor=green, citecolor=cyan, urlcolor=red, filecolor=magenta, pdftoolbar=true}

\allowdisplaybreaks

\theoremstyle{plain}
\newtheorem{thm}{Theorem}[section]
\newtheorem{lem}[thm]{Lemma}
\newtheorem{cor}[thm]{Corollary}
\newtheorem{prop}[thm]{Proposition}
\theoremstyle{definition}
\newtheorem{rem}[thm]{Remark}
\newtheorem{rems}[thm]{Remarks}
\newtheorem{defi}[thm]{Definition}
\newtheorem{ex}[thm]{Example}
\newtheorem{exs}[thm]{Examples}
\newtheorem{conv}[thm]{Convention}

\numberwithin{thm}{section}
\numberwithin{equation}{section}

\begin{document}

\title{A note on weighted iterated Hardy-type inequalities}

\author[R.Ch. Mustafayev]{Rza Mustafayev}
\address{Department of Mathematics \\ Faculty of Science and Arts \\ Kirikkale
	University \\ 71450 Yahsihan, Kirikkale, Turkey}
\email{rzamustafayev@gmail.com}

\subjclass[2010]{26D10, 26D15}

\keywords{quasilinear operators, iterated Hardy inequalities, weights}

\begin{abstract}
In this paper the inequality
$$
\bigg( \int_0^{\infty} \bigg( \int_x^{\infty} \bigg( \int_t^{\infty} h \bigg)^q w(t)\,dt
\bigg)^{r / q} u(x)\,ds \bigg)^{1/r}\leq C \,\int_0^{\infty} h v, \quad h \in {\mathfrak M}^+(0,\infty)
$$
is characterized. Here $0 < q ,\, r < \infty$ and $u,\,v,\,w$ are weight functions on $(0,\infty)$.
\end{abstract}

\maketitle

\section{Introduction}\label{in}

Throughout this paper by ${\mathfrak M}^+ (0,\infty)$ we denote the set of all non-negative measurable functions on $(0,\infty)$.
A weight is a function $v \in {\mathfrak M}^+ (0,\infty)$ such that
$$
0 < \int_0^x v(t)\,dt < \infty \quad \mbox{for all} \quad x > 0.
$$
The family of all weight functions (also called just weights) on $(0,\infty)$ is given by ${\mathcal W}{(0,\infty)}$. In the following, assume that $u,\,v,\,w \in {\mathcal W}{(0,\infty)}$.

Let $0 < q ,\, r < \infty$. The inequality
\begin{equation}\label{IHI.0.1}
\bigg( \int_0^{\infty} \bigg( \int_x^{\infty} \bigg( \int_t^{\infty} h \bigg)^q w(t)\,dt
\bigg)^{r / q} u(x)\,ds \bigg)^{1/r}\leq C \,\, \bigg( \int_0^{\infty} h^p v \bigg)^{1 / p}, \quad h \in {\mathfrak M}^+(0,\infty)
\end{equation}
was characterized in \cite[Theorem 6.5]{GogMusIHI} with $1 < p < \infty$. The same paper contains solutions of inequalities
\begin{equation}\label{IHI.0.2}
\bigg( \int_0^{\infty} \bigg( \int_x^{\infty} \bigg( \int_t^{\infty} h \bigg)^q w(t)\,dt
\bigg)^{r / q} u(x)\,ds \bigg)^{1/r}\leq C \, \int_0^{\infty} h(x)  \, \bigg( \int_x^{\infty} v\bigg)^{-1}\,dx, \quad h \in {\mathfrak M}^+(0,\infty),
\end{equation}
(see \cite[Theorem 6.6]{GogMusIHI}), and
\begin{equation}\label{IHI.0.3}
\bigg( \int_0^{\infty} \bigg( \int_x^{\infty} \bigg( \int_t^{\infty} h \bigg)^q w(t)\,dt
\bigg)^{r / q} u(x)\,ds \bigg)^{1/r}\leq C \, \int_0^{\infty} h(x)  \, \bigg( \int_0^x v\bigg)\,dx, \quad h \in {\mathfrak M}^+(0,\infty),
\end{equation}
(see, for instance, \cite[Corollary 3.24 and Theorem 5.2]{GogMusIHI}, as well. In the present paper the inequality
\begin{equation}\label{main}
\bigg( \int_0^{\infty} \bigg( \int_x^{\infty} \bigg( \int_t^{\infty} h \bigg)^q w(t)\,dt
\bigg)^{r / q} u(x)\,ds \bigg)^{1/r}\leq C \,\int_0^{\infty} h v, \quad h \in {\mathfrak M}^+(0,\infty)
\end{equation}
is characterized.

In the case when $q=1$, using the Fubini Theorem, inequality \eqref{main} can be reduced to the weighted $L^1 - L^r$ boundedness problem of the  Volterra operator
$$
(Kh)(x) : = \int_x^{\infty} k(x,y) h(y)\,dy, \quad x > 0,
$$
with the kernel
$$
k(x,y) : = \int_x^y w(t)\,dt, \quad 0 < x \le y < \infty,
$$
and consequently, can be easily solved. Indeed: By  the Fubini Theorem, we see that
$$
\int_x^{\infty} \left( \int_t^{\infty}
h(y)\,dy\right)u(t)\,dt = \int_x^{\infty} k(x,y) h(y)\,dy, \qquad h\in {\mathfrak M}^+(0,\infty).
$$
Recall that the weighted $L^1 - L^q$ boundedness of Volterra operators $K$, that is, inequality
\begin{equation}\label{volter}
\bigg( \int_0^{\infty} \bigg( \int_x^{\infty} k(x,y) h(y)\,dy \bigg)^q w(x)\,dx\bigg)^{1/q} \le C \int_0^{\infty} hv, \qquad h \in {{\mathfrak M}}^+ (0,\infty)
\end{equation}
is completely characterized for $0 <  q < \infty$ (see Theorems \ref{Oinar} and \ref{Krep}).

Investigation of the weighted iterated Hardy-type inequalities started in \cite{GogMusPers1} and \cite{GogMusPers2}. In \cite{ProkhStep1} a unified method was created for solution of these inequalities for all possible values of parameters. In particular, inequality \eqref{IHI.0.1} with $0<q<\infty$, $0 < r < \infty$, $1 \le p < \infty$ was characterized in integral forms in \cite{ProkhStep1}. But this characterization involve auxiliary functions, which make conditions more complicated. 

Our approach consists of the following steps: We prove that
\begin{align*}
	{\operatorname{LHS}}{\eqref{main}} \thickapprox &  \bigg\| \bigg\{ 2^{k / r} \bigg(
	\int_{x_k}^{x_{k+1}} \bigg( \int_s^{x_{k+1}} h \bigg)^q
	w(s)\,ds\bigg)^{ 1 / q} \bigg\} \bigg\|_{\ell^r({\mathcal Z})}  \\
	&  +  \bigg( \int_0^{\infty}  u(t) \sup_{t < s} \bigg(
	\int_t^s w \bigg)^{r / q} \bigg( \int_s^{\infty} h\bigg)^r \,dt\bigg)^{1/r} 
\end{align*}
with constants independent of $h \in {\mathfrak M}^+(0,\infty)$, where  $\{x_k\}_{k=-\infty}^{M + 1}$ is a covering sequence mentioned in Remark \ref{rem.disc.} (see Lemma \ref{lem.1.4}). 
Then we give the proof of fact that the following assertions are equivalent:
\begin{align}
	\bigg( \int_0^{\infty}  u(t) \sup_{t < s} \bigg(
	\int_t^s w \bigg)^{r / q} \bigg( \int_s^{\infty} h\bigg)^r \,dt\bigg)^{1/r} \leq C \, \int_0^{\infty} h v, \quad h \in {\mathfrak M}^+(0,\infty), \label{1.1} \\
	\bigg( \int_0^{\infty}  u(t) \bigg( \int_t^{\infty} \bigg(
	\int_t^s w \bigg)^{1 / q}  h(s) \,ds\bigg)^r \,dt\bigg)^{1/r} \leq C \,\int_0^{\infty} h v, \quad h \in {\mathfrak M}^+(0,\infty) \label{1.2}
\end{align}
(see Lemma \ref{lem.1.5}). Recall that the solution of inequality \eqref{1.2} is known (see Theorems \ref{Oinar} and \ref{Krep}). 
Then noting that the best constant in \eqref{main} is unchanged when weight function $v$ is replaced with the greatest non-decreasing minorant of $v$ (see Theorem \ref{Sinnamon.thm.2}), the characterization of the discrete inequality
\begin{equation*}
	\bigg\| \bigg\{ 2^{k / r} \bigg( \int_{x_k}^{x_{k+1}} \bigg( \int_s^{x_{k+1}} h \bigg)^q
	w(s)\,ds\bigg)^{ 1 / q} \bigg\} \bigg\|_{\ell^r({\mathcal Z})} \le C \bigg\| \bigg\{ \int_{x_n}^{x_{n+1}} h v \bigg\} \bigg\|_{\ell^1 ({\mathcal Z})}, \quad h \in {\mathfrak M}^+(0,\infty) 
\end{equation*}
is presented with non-decreasing weight function $v$ (see Lemma \ref{lem.1.7}).

Our main result reads as follows:
\begin{thm}\label{main1}
	Let $0 < q,\, r < \infty$ and  $u,\,v,\,w \in {\mathcal W}{(0,\infty)}$.
	
	{\rm (i)} Let $r < 1$. Then inequality \eqref{main} holds if and only if
	\begin{align*}
		F_1 : & = \bigg( \int_0^{\infty}  \bigg( \int_0^t u \bigg)^{r'}  u(t) \bigg( {\operatornamewithlimits{ess\,sup}}_{t \le s < \infty} \bigg(
		\int_t^s w \bigg)^{r' / q} \bigg({\operatornamewithlimits{ess\,sup}}_{s \le \tau < \infty }v(\tau)^{-1}\bigg)^{r'} \bigg) \,dt\bigg)^{1/r'} < \infty, \\
		F_2 : & = \bigg( \int_0^{\infty}  \bigg( \int_0^t u(x) \bigg( \int_x^t w \bigg)^{r / q} dx \bigg)^{r'} u(t) \bigg( {\operatornamewithlimits{ess\,sup}}_{t \le s < \infty} \bigg( \int_t^s w \bigg)^{r/q} \bigg({\operatornamewithlimits{ess\,sup}}_{s \le \tau < \infty }v(\tau)^{-1}\bigg)^{r'} \bigg) \,dt\bigg)^{1/r'}  < \infty, \\
		F_3 : & = \bigg( \int_0^{\infty} \bigg( \int_0^t u\bigg)^{r'} \bigg( \int_t^{\infty} \bigg({\operatornamewithlimits{ess\,sup}}_{s \le \tau < \infty} v(\tau)^{-1}\bigg)^{q} w(s)\,ds \bigg)^{ r' / q}u(t)\,dt \bigg)^{1/r'} < \infty.
	\end{align*}	
	
	Moreover, if $C$ is the best constant in \eqref{main}, then $C \approx F_1 + F_2 + F_3$.
	
	{\rm (ii)} Let $r \ge 1$. Then inequality \eqref{main} holds if and only if
	\begin{align*}
		G_1 : & = \sup_{t \in (0,\infty)} \bigg( \int_0^t   u(x) \bigg( \int_x^t w \bigg)^{r / q}  \,dx \bigg)^{1 / r}  \bigg({\operatornamewithlimits{ess\,sup}}_{t \le \tau < \infty }v(\tau)^{-1}\bigg) < \infty, \\
		G_2 : & = \sup_{t \in (0,\infty)}  \bigg( \int_0^t u(x) \,dx \bigg)^{1/r}  \bigg( {\operatornamewithlimits{ess\,sup}}_{t \le s < \infty} \bigg( \int_t^s w \bigg)^{1/q} \bigg({\operatornamewithlimits{ess\,sup}}_{s \le \tau < \infty }v(\tau)^{-1}\bigg) \bigg) < \infty, \\
		G_3 : & = \sup_{t > 0} \bigg( \int_0^t u \bigg)^{1 / r} \bigg( \int_t^{\infty} \bigg({\operatornamewithlimits{ess\,sup}}_{s \le \tau < \infty }v(\tau)^{-1}\bigg)^{q} w(s)\,ds \bigg)^{ 1 / q} < \infty.	
	\end{align*}
	
	Moreover, if $C$ is the best constant in \eqref{main}, then $C \approx G_1 + G_2 + G_3$.
\end{thm}

It is worth to mention that the characterizations of "dual"
inequality
\begin{equation}\label{main.dual}
\bigg( \int_0^{\infty} \bigg( \int_0^x \bigg( \int_0^t h \bigg)^q w(t)\,dt
\bigg)^{r / q} u(x)\,ds \bigg)^{1/r}\leq C \,\int_0^{\infty} h v, \quad h \in {\mathfrak M}^+(0,\infty)
\end{equation}
can be easily obtained  from the solution of inequality
\eqref{main}, by change of variables.

We pronounce that the characterizations of inequalities
\eqref{main} and \eqref{main.dual} are important because many inequalities
for classical operators  can be reduced to them. These inequalities play an important role in the theory of Morrey-type
spaces and other topics (see \cite{BGGM1}, \cite{BGGM2} and \cite{BO}). Note that using characterizations of weighted Hardy inequalities it is easy to see that the characterization of the boundedness of bilinear Hardy-type inequality
\begin{equation}
\bigg( \int_0^{\infty} \bigg( \int_0^x f \bigg)^q \, \bigg( \int_x^{\infty} g \bigg)^q w(x)\,dx \bigg)^{1/q} \le C \,\, \bigg( \int_0^{\infty} f^p v_1 \bigg)^{1/p}  \int_0^{\infty} g v_2,
\end{equation}
for all $f \in L^{p}(v_1,(0,\infty))$ and $g \in L^{1}(v_2,(0,\infty))$ with constant $C$ independent of $f$ and $g$,
can be reduced to inequality \eqref{main} when $q < p$.

The paper is organized as follows. Section \ref{pre} contains some
preliminaries along with the standard ingredients used in the
proofs. In Section \ref{mon} we prove that the norm of the composition of a monotone operator $T: {{\mathfrak M}}^+ (0,\infty){\rightarrow} {{\mathfrak M}}^+ (0,\infty)$ with the Hardy operator $Hf$ (the Copson operator $H^* f$) from $L_{\mu}^1(v)$ to a quasi-normed space of measurable functions $X$ defined on ${{\mathfrak M}}^+ (0,\infty)$ with the lattice property is unchanged when weight function  $v$ is replaced with the greatest non-increasing minorant of $v$ (the greatest non-decreasing minorant of $v$) (see Theorems \ref{Sinnamon.thm.1} and \ref{Sinnamon.thm.2}), where $\mu$ is a ${\sigma}$-finite measure on $(0,\infty)$. In Section \ref{eq.red.thms} the equivalence and reduction theorems are proved which allow to obtain our main result. 

\section{Notations and Preliminaries}\label{pre}

Throughout the paper, we always denote by $C$ a positive
constant, which is independent of main parameters but it may vary
from line to line. However a constant with subscript or superscript such as $c_1$
does not change in different occurrences. By $a\lesssim b$,
($b\gtrsim a$) we mean that $a\leq \lambda b$, where $\lambda >0$ depends on
inessential parameters. If $a\lesssim b$ and $b\lesssim a$, we write
$a\approx b$ and say that $a$ and $b$ are  equivalent. Unless a
special remark is made, the differential element $dx$ is omitted
when the integrals under consideration are the Lebesgue integrals.
Everywhere in the paper, $u$, $v$ and $w$ are weights. We use the
abbreviation ${\operatorname{LHS}} (*)$ for the left hand side
of the relation $(*)$.

We adopt the following usual conventions.
\begin{conv}\label{Notat.and.prelim.conv.1.1}
	{\rm (i)} We put $0 \cdot \infty = 0$, $\infty / \infty =
	0$ and $0/0 = 0$.
	
	{\rm (ii)} We denote by $r' = r / (1- r )$, if $0 < r < 1$.
\end{conv}

For $p\in (0,\infty]$ and $w\in {\mathfrak M}^+(I)$ with $I : = (a,b)\subseteq (0,\infty)$ we define the functional
$\|\cdot\|_{p,w,I}$ on ${\mathfrak M} (I)$ by
\begin{equation*}
\|f\|_{p,w,I} : = \left\{\begin{array}{cl}
\left(\int_I |f(x)|^p w(x)\,dx \right)^{1/p} & \qquad\mbox{if}\qquad p<\infty \\
{\operatornamewithlimits{ess\,sup}}_{I} |f(x)|w(x) & \qquad\mbox{if}\qquad p=\infty.
\end{array}
\right.
\end{equation*}

If, in addition, $w\in {\mathcal W}{(0,\infty)}$, then the weighted Lebesgue space
$L^p(w,I)$ is given by
\begin{equation*}
L^p(w,I) = \{f\in {\mathfrak M} (I):\,\, \|f\|_{p,w,I} < \infty\},
\end{equation*}
and it is equipped with the quasi-norm $\|\cdot\|_{p,w,I}$.

When $w\equiv 1$ on $I$, we write simply $L^p(I)$ and
$\|\cdot\|_{p,I}$ instead of $L^p(w,I)$ and $\|\cdot\|_{p,w,I}$,
respectively.

Let $\emptyset \neq {\mathcal Z} \subseteq \overline{\mathbb Z} : = {\mathbb Z} \cup \{
-\infty ,+\infty\}$, $0 < q \le +\infty$ and $\{w_k\} = \{w_k\}_{k \in
	{\mathbb Z}}$ be a sequence of positive numbers. We denote by
$\ell^q(\{w_k\},{\mathcal Z})$ the following discrete analogue of a weighted
Lebesgue space: if $0 < q < + \infty$, then
\begin{align*}
&\ell^q(\{w_k\},{\mathcal Z}) = \bigg\{ \{a_k\}_{k \in {\mathcal Z}}:
\|\{a_k\}\|_{\ell^q(\{w_k\},{\mathcal Z})}
: = \bigg(\sum_{k\in{\mathcal Z}}|a_k w_k|^q\bigg)^{1 / q}<+\infty \bigg\},\\
\intertext{and} &
\ell^\infty(\{w_k\},{\mathcal Z}) =\left\{ \{a_k\}_{k\in{\mathcal Z}}:
\|\{a_k\}\|_{\ell^\infty(\{w_k\},{\mathcal Z})} : =
\sup_{k\in{\mathcal Z}}|a_kw_k|<+\infty \right\}.
\end{align*}
If $w_k=1$ for all $k \in {\mathcal Z}$, we write simply $\ell^q({\mathcal Z})$ instead
of $\ell^q(\{w_k\},{\mathcal Z})$. 

\begin{defi} \label{D:2.1}
	Let $N,M\in \overline{\mathbb Z}$, $N<M$. A positive sequence $\{\tau _k\} _{k=N}^M$ is called geometrically increasing if
	there is $\alpha \in (1, +\infty )$ such that
	$$
	\tau _k\geq \alpha \tau _{k-1} \quad \text{for all} \quad k\in
	\{ N+1, \dots ,M\}.
	$$
\end{defi}

We quote some known results. Proofs  can be found in \cite{Le1} and
\cite{Le2}.
\begin{lem}\label{L:1.2}
	Let $q\in (0,+\infty ]$, $N, M \in \overline{\mathbb Z}$, $ N\le M$, ${\mathcal Z} =
	\{N,N+1,\ldots,M-1,M\}$ and let $\{ \tau_k\} _{k=N}^M$ be a	geometrically increasing sequence. Then
	
	\begin{equation}\label{E1.1.1}
	\left\| \left\{\tau _k \sum_{m=k}^{M}a_m\right\}\right\|
	_{\ell^q({\mathcal Z})} \approx \| \{\tau _ka_k\}\| _{\ell^q({\mathcal Z})}
	\end{equation}
	and
	
	\begin{equation}\label{E1.2.1}
	\bigg\| \bigg\{\tau _k \sup _{k\leq m\leq M}a_m \bigg\}\bigg\|
	_{\ell^q({\mathcal Z})} \approx \|\{\tau _ka_k\}\| _{\ell^q({\mathcal Z})}
	\end{equation}
	for all non-negative sequences $\{ a_k\} _{k=N}^M$.
\end{lem}

Given two (quasi-) Banach spaces $X$ and $Y$, we write $X
\hookrightarrow Y$ if $X \subset Y$ and if the natural embedding of
$X$ in $Y$ is continuous.

The following two lemmas are discrete version of the classical
Landau resonance theorems. Proofs can be found, for example, in
\cite{gp1}.
\begin{prop}\label{prop.2.1}{\rm(\cite[Proposition 4.1]{gp1})}
	Let $0 < r < +\infty$, $\emptyset \neq {\mathcal Z} \subseteq
	\overline{\mathbb Z}$ and let $\{v_k\}_{k\in{\mathcal Z}}$ and $\{w_k\}_{k\in{\mathcal Z}}$ be
	two sequences of positive numbers. Assume that
	\begin{equation}\label{eq31-4651}
	\ell^1 (\{v_k\},{\mathcal Z}) \hookrightarrow \ell^r (\{w_k\},{\mathcal Z}).
	\end{equation}
	Then
	\begin{equation*}\label{eq31-46519009}
	\big\|\big\{w_k v_k^{-1}\big\}\big\|_{\ell^\rho({\mathcal Z})} \le C,
	\end{equation*}
	where $1 / \rho : = ( 1 / r - 1)_+$ \footnote{For any $a\in{\mathbb R}$ denote
		by $a_+ = a$ when $a>0$ and $a_+ = 0$ when $a \le 0$.} and $C$ stands for the norm of
	embedding \eqref{eq31-4651}.
\end{prop}

The kernel $k : [0,\infty)^2 \longrightarrow [0,\infty)$ is measurable function which has the following properties:

{\rm (i)} $k(x,y)$ is non-increasing in $x$ and non-decreasing in $y$;

{\rm (ii)} there exists a constant $d > 0$ such that for all $0 \le x < y < z < \infty$ it holds
$$
k(x,z) \le d(k(x,y) + k (y,z));
$$

{\rm (iii)} $k(0,y) > 0$ for all $y > 0$.

We recall the following statement, which may be recovered from \cite[Theorem 1.1]{Oinar} by correctly interpreting the expressions involving the symbol $p'$ in there.
\begin{thm}\cite[Theorem 1.1]{Oinar}\label{Oinar}
	Let $1 \le q < \infty$ and $v,\, w \in {\mathcal W}{(0,\infty)}$. Assume that $k$ is a kernel, satifying (i) - (iii) with some $0 < d < \infty$. Then the inequality
	\begin{equation}\label{HIwithKer.1}
	\bigg( \int_0^{\infty} \bigg( \int_x^{\infty} k(x,y) h(y)\,dy \bigg)^q w(x)\,dx\bigg)^{1/q} \le C \int_0^{\infty} hv, \qquad h \in {{\mathfrak M}}^+ (0,\infty)
	\end{equation}
	holds if and only if
	\begin{align*}
	O_1 : = \sup_{t \in (0,\infty)} \bigg( \int_0^t w(x) k^q (x,t) \,dx\bigg)^{1/q} {\operatornamewithlimits{ess\,sup}}_{t \le  s < \infty} v(s)^{-1} < \infty \\
	\intertext{and}
	O_2 : = \sup_{t \in (0,\infty)} \bigg( \int_0^t w(x)\,dx\bigg)^{1/q} {\operatornamewithlimits{ess\,sup}}_{t \le  s < \infty} k (t,s) v(s)^{-1} < \infty.
	\end{align*} 
	Moreover, the best constant in \eqref{HIwithKer.1} satisfies $C \approx O_1 + O_2$.
\end{thm}	

\begin{thm}\cite[Theorem 8]{Krep}\label{Krep}
	Let $0 < q < 1$ and $v,\, w \in {\mathcal W}{(0,\infty)}$. Assume that $k$ is a kernel, satifying (i) - (iii) with some $0 < d < \infty$. Then  inequality \eqref{HIwithKer.1}	holds if and only if
	\begin{align*}
	K_1 : = \bigg( \int_0^{\infty} \bigg( \int_0^t w(x)\,dx\bigg)^{q'} w(t) \bigg( {\operatornamewithlimits{ess\,sup}}_{t \le s < \infty} k^{q'}(t,s)v(s)^{-q'}\bigg) \,dt \bigg)^{1 / q'} < \infty \\
	\intertext{and}
	K_2 : = \bigg( \int_0^{\infty} \bigg( \int_0^t w(x) k^q (x,t )\,dx\bigg)^{q'} w(t) \bigg( {\operatornamewithlimits{ess\,sup}}_{t \le s < \infty} k^{q}(t,s)v(s)^{-q'}\bigg) \,dt \bigg)^{1 / q'} < \infty.
	\end{align*} 
	Moreover, the best constant in \eqref{HIwithKer.1} satisfies $C \approx K_1 + K_2$.
\end{thm}	

\section{Monotone envelopes}\label{mon}

Let $\mu$ be a ${\sigma}$-finite measure on $(0,\infty)$. In order for
monotone functions to be $\mu$-measurable we assume that all Borel
sets are $\mu$-measurable. Let ${{\mathfrak M}}_{\mu}^+$ be the collection of
all non-negative, $\mu$-measurable functions on $(0,\infty)$.

The two operators of integration we will need are $H$ and $H^*$
defined for any $x > 0$ by
$$
H f(x) = \int_{(0,x]} f \,d\mu \qquad \mbox{and} \qquad H^* f(x) =
\int_{[x,\infty)} f \,d \mu.
$$
Note that for all $u,\,v \in {{\mathfrak M}}_{\mu}^+$ we have
$$
\int_{(0,\infty)} (H u)v\,d\mu = \int_{(0,\infty)} u (H^* v)\,d\mu.
$$

For $v \in {{\mathfrak M}}_{\mu}^+$ we define the monotone envelopes of $v$ as
follows: the greatest non-increasing minorant of $v$ is
$$
v_{\downarrow}(x) : = {\operatornamewithlimits{ess\,inf}}_{0 < t \le x}v(t)
$$
and the greatest non-decreasing minorant of $v$ is
$$
v_{\uparrow}(x) : = {\operatornamewithlimits{ess\,inf}}_{x \le t < \infty}v(t).
$$

The following theorem is true.
\begin{thm}[\cite{Sinn}, Theorem 2.1]\label{Sinnamon.thm.1}
	Suppose $f,\, v \in{{\mathfrak M}}_{\mu}^+$. Then
	\begin{equation}\label{Sinnamon.thm.1.eq.1}
	\inf_{Hg \ge Hf} \int_{(0,\infty)} gv\,d\mu = \int_{(0,\infty)} f v_{\downarrow}\,d\mu,
	\end{equation}
	and
	\begin{equation}\label{Sinnamon.thm.1.eq.2}
	\inf_{H^* g \ge H^* f} \int_{(0,\infty)} gv\,d\mu = \int_{(0,\infty)} f
	v_{\uparrow}\,d\mu.
	\end{equation}
\end{thm}

An operator $T: {{\mathfrak M}}^+ {\rightarrow} {{\mathfrak M}}^+$ such that $Tf(x) \le c Tg(x)$ for almost all $x \in {\mathbb R}_+$ if $f(x) \le g(x)$ for almost all $x \in {\mathbb R}_+$, with constant $c > 0$ independent of $f$ and $g$, is called a monotone operator.
\begin{thm}\label{Sinnamon.thm.2}
	Let $X$ be a quasi-normed space  of functions from ${{\mathfrak M}}^+$ with the
	lattice property, that is,
	$$
	0 \le g \le f \quad \Rightarrow \quad \|g\|_X \lesssim \|f\|_X.
	$$
	Suppose $v \in{{\mathfrak M}}_{\mu}^+$, and let $T: {{\mathfrak M}}^+ {\rightarrow} {{\mathfrak M}}^+$ be a monotone operator. Then the least constant $C$, finite
	or infinite, for which
	$$
	\left\|T \bigg(\int_{(0,x]} f \,d\mu\bigg)\right\|_X \le C \int_{(0,\infty)}
	fv\,d\mu,~ f \in {{\mathfrak M}}_{\mu}^+,
	$$
	holds is unchanged when $v$ is replaced by $v_{\downarrow}$. That is,
	\begin{equation}\label{Sinnamon.thm.2.eq.1}
	\sup_{f \ge 0} \frac{\left\|T \bigg(\int_{(0,x]} f
		\,d\mu\bigg)\right\|_X}{\int_\Ifv\,d\mu} = \sup_{f \ge 0}
	\frac{\left\|T \bigg(\int_{(0,x]} f
		\,d\mu\bigg)\right\|_X}{\int_\Ifv_{\downarrow}\,d\mu}.
	\end{equation}
\end{thm}
\begin{proof}
	Since $v_{\downarrow} \le v$ $\mu$-almost everywhere the inequality "$\le$"
	in \eqref{Sinnamon.thm.2.eq.1} is immediate.
	
	To establish the reverse inequality we apply
	\eqref{Sinnamon.thm.1.eq.1} of Theorem \ref{Sinnamon.thm.1} to get
	\begin{align*}
	\sup_{f \ge 0} \frac{\left\|T \left(\int_{(0,x]} f
		\,d\mu\right)\right\|_X}{\int_\Ifv_{\downarrow}\,d\mu} & = \sup_{f \ge 0}
	\frac{\left\|T \left(\int_{(0,x]} f
		\,d\mu\right)\right\|_X}{\inf_{Hg
			\ge Hf} \int_{(0,\infty)} gv\,d\mu} \\
	& = \sup_{f \ge 0} \sup_{Hg \ge Hf} \frac{\left\|T
		\left(\int_{(0,x]} f \,d\mu\right)\right\|_X}{\int_\Igv\,d\mu}.
	\end{align*}
	Now if $Hf \le Hg$ then the monotonicity of $T$ shows that
	$$
	T \left(\int_{(0,x]} f \,d\mu\right) \le T \left(\int_{(0,x]} g
	\,d\mu\right),
	$$
	and since $X$ has the lattice property we have
	$$
	\left\|T \left(\int_{(0,x]} f \,d\mu\right)\right\|_X \le \left\|T
	\left(\int_{(0,x]} g \,d\mu\right)\right\|_X.
	$$
	Thus
	\begin{align*}
	\sup_{f \ge 0} \frac{\left\|T \left(\int_{(0,x]} f
		\,d\mu\right)\right\|_X}{\int_\Ifv_{\downarrow}\,d\mu} & \le \sup_{f \ge
		0} \sup_{Hg \ge Hf} \frac{\left\|T \left(\int_{(0,x]} g
		\,d\mu\right)\right\|_X}{\int_\Igv\,d\mu} \le \sup_{g \ge 0}
	\frac{\left\|T \left(\int_{(0,x]} g
		\,d\mu\right)\right\|_X}{\int_\Igv\,d\mu}.
	\end{align*}
	This completes the proof.
\end{proof}

The proof of the following theorem is analogous to that of Theorem
\ref{Sinnamon.thm.2} and we omit it.
\begin{thm}\label{Sinnamon.thm.3}
	Let $X$ be a quasi-normed space  of functions from ${{\mathfrak M}}^+$ with the
	lattice property. Suppose $v \in{{\mathfrak M}}_{\mu}^+$, and let $T: {{\mathfrak M}}^+ {\rightarrow} {{\mathfrak M}} ^+$ be a monotone operator. Then the least constant $C$, finite
	or infinite, for which
	$$
	\left\|T \bigg(\int_{[x,\infty)} f \,d\mu\bigg)\right\|_X \le C
	\int_{(0,\infty)} fv\,d\mu,~ f \in {{\mathfrak M}}_{\mu}^+,
	$$
	holds is unchanged when $v$ is replaced by $v_{\uparrow}$. That is,
	\begin{equation}\label{Sinnamon.thm.3.eq.1}
	\sup_{f \ge 0} \frac{\left\|T \bigg(\int_{[x,\infty)} f
		\,d\mu\bigg)\right\|_X}{\int_\Ifv\,d\mu} = \sup_{f \ge 0}
	\frac{\left\|T \bigg(\int_{[x,\infty)} f
		\,d\mu\bigg)\right\|_X}{\int_\Ifv_{\uparrow}\,d\mu}.
	\end{equation}
\end{thm}

\begin{rem}
	In \cite{ss} the notion of transfering monotonicity from the kernel of an operator to the weight was introduced to study a special case of the weighted Hardy inequality. This property was placed in a more general setting in \cite{Sinn}:
	
	Let $\lambda$ be a $\sigma$-finte measure on $(0,\infty)$ for which non-increasing functions are $\lambda$-measurable. Assume that  $\mu$ is any measure on any set and $X$ is a Banach Function Space of $\mu$-measurable functions.
	It was proved in \cite[Theorem 3.1 and Corollary 3.2]{Sinn} that the best constant, finite or infinite, for which
	$$
	\bigg\| \int_{(0,\infty)} k(\cdot,t) f(t)\,d\lambda (t) \bigg\|_X \le C \int_{(0,\infty)} fv \, d \lambda, \quad f\in {\mathfrak M}_{\lambda}^+
	$$
	holds, is unchanged when $v$ is replaced by $v_{\downarrow}$ (by $x_{\uparrow}$), where $k(x,t)$ is non-negative $\mu \times \lambda$-measurable function, non-increasing in $t$ for each $x$ (non-decreasing in $t$ for each $x$).
\end{rem}

\section{Equivalence and reduction theorems}\label{eq.red.thms}

In this section we prove the equivalence and reduction theorems.
\begin{rem}\label{rem.disc.}
Let $u$ be a weight functyon on $(0,+\infty)$. It is easy to see that if $\int_0^{\infty} u(t)\,dt = +\infty$, then there exists strictly increasing sequence $\{x_k\}_{-\infty}^{+\infty} \subset (0,\infty)$ such that $\int_0^{x_k} u(t)\,dt = 2^k$, $k \in {\mathbb Z}$, and $\bigcup_{k \in {\mathbb Z}} (x_k,x_{k+1}] = (0,\infty)$. Consequently, $\{x_k\}_{k \in {\mathbb Z}}$ is covering sequence, i.e. partition of $(0,\infty)$ (cf. \cite{Lai1999}). In the case when $\int_0^{\infty} u(t)\,dt < +\infty$ define a sequence $\{x_k\}_{k = -\infty}^M$ such that $\int_0^{x_k} u(t)\,dt = 2^k$ if $-\infty < k \le M$, where $M$ is defined by the inequality $2^M \le \int_0^{\infty} u(t)\,dt < 2^{M+1}$. Denote by $x_{M+1} : = \infty$. Thus $\{x_k\}_{k = -\infty}^{M+1}$ is covering sequence of $(0,\infty)$ as well. We will consider in the proofs of all statements below the case when $\int_0^{\infty} u(t)\,dt < \infty$, for it is much simpler to deal with the case when $\int_0^{\infty} u(t)\,dt = \infty$. We will denote ${\mathcal Z} : = \{k \in {\mathbb Z}:\, k \le M\}$, when $\int_0^{\infty} u(t)\,dt < \infty$, and note that ${\mathcal Z} = {\mathbb Z}$, when $\int_0^{\infty} u(t)\,dt = \infty$.
\end{rem}

\begin{lem}\label{lem.1.1}
Let $0 < q,\, r < \infty$ and  $u,\,v,\,w \in {\mathcal W}{(0,\infty)}$.
Assume that $\{x_k\}_{k=-\infty}^{M + 1}$ is a covering sequence mentioned in Remark \ref{rem.disc.}. Then
\begin{align}
{\operatorname{LHS}}{\eqref{main}} \thickapprox &  \bigg\| \bigg\{  2^{k / r} \bigg(
\int_{x_k}^{x_{k+1}} \bigg( \int_s^{x_{k+1}} h \bigg)^q
w(s)\,ds\bigg)^{ 1 / q}\bigg\} \bigg\|_{\ell^r({\mathcal Z})} \notag\\
&  +  \bigg\| \bigg\{  2^{k / r} \bigg(
\int_{x_k}^{x_{k+1}} w \bigg)^{1 / q} \bigg( \int_{x_{k+1}}^{\infty} h
\bigg)\bigg\} \bigg\|_{\ell^r({\mathcal Z})}
\end{align}
with constants independent of $h \in {\mathfrak M}^+(0,\infty)$. 
\end{lem}

\begin{proof}
	It is clear that
	\begin{align*}
	{\operatorname{LHS}}{\eqref{main}} & = \, \bigg\| \bigg\{ \bigg(\int_{x_k}^{x_{k+1}} \bigg( \int_x^{\infty} \bigg( \int_t^{\infty} h \bigg)^q w(t)\,dt \bigg)^{r / q} u(x)\,dx \bigg)^{1/r} \bigg\} \bigg\|_{\ell^r({\mathcal Z})} \\
	& \, \approx \bigg\| \bigg\{ 2^{k / r} \bigg( \int_{x_k}^{\infty}	\bigg( \int_s^{\infty} h \bigg)^q w(s)\,ds\bigg)^{ 1 / q} \bigg\} \bigg\|_{\ell^r({\mathcal Z})}.
	\end{align*}

By Lemma \ref{L:1.2}, we get that
	\begin{align*}
	{\operatorname{LHS}}{\eqref{main}}  \approx & \, \bigg\| \bigg\{ 2^{k / r} \bigg( \int_{x_k}^{x_{k+1}}	\bigg( \int_s^{\infty} h \bigg)^q w(s)\,ds\bigg)^{ 1 / q} \bigg\} \bigg\|_{\ell^r({\mathcal Z})} \\
	\approx & \, \bigg\| \bigg\{ 2^{k / r} \bigg( \int_{x_k}^{x_{k+1}}	\bigg( \int_s^{x_{k + 1}} h \bigg)^q w(s)\,ds\bigg)^{ 1 / q} \bigg\} \bigg\|_{\ell^r({\mathcal Z})} \\
	& + \bigg\| \bigg\{ 2^{k / r} \bigg( \int_{x_k}^{x_{k+1}} w \bigg)^{1 / q} \bigg( \int_{x_{k+1}}^{\infty} h \bigg) \bigg\} \bigg\|_{\ell^r({\mathcal Z})}.
	\end{align*}
	
The proof is completed. 
\end{proof}

\begin{lem}\label{lem.1.2}
	Let $0 < q,\, r < \infty$ and  $u,\,v,\,w \in {\mathcal W}{(0,\infty)}$.
	Assume that $\{x_k\}_{k=-\infty}^{M + 1}$ is a covering sequence mentioned in Remark \ref{rem.disc.}. Then
	\begin{align}
	{\operatorname{LHS}}{\eqref{main}} \thickapprox &  \, \bigg\| \bigg\{ 2^{k / r} \bigg(
	\int_{x_k}^{x_{k+1}} \bigg( \int_s^{x_{k+1}} h \bigg)^q
	w(s)\,ds\bigg)^{ 1 / q} \bigg\} \bigg\|_{\ell^r({\mathcal Z})} \notag \\
	&  +  \bigg\| \bigg\{ \bigg( \int_{x_{k-1}}^{x_k} u(t) \sup_{t < s \le x_{k+1}} \bigg(
	\int_{x_k}^s w \bigg)^{r / q} \bigg( \int_s^{\infty} h
	\bigg)^r \,dt\bigg)^{1/r} \bigg\} \bigg\|_{\ell^r({\mathcal Z})}
	\end{align}
	with constants independent of $h \in {\mathfrak M}^+(0,\infty)$.
\end{lem}

\begin{proof}
{\bf $\lesssim:$} Since
	\begin{align*}
	\bigg\| \bigg\{ 2^{k / r} \bigg( \int_{x_k}^{x_{k+1}} w \bigg)^{1 / q} \bigg( \int_{x_{k+1}}^{\infty} h \bigg) \bigg\} \bigg\|_{\ell^r({\mathcal Z})} & \\
	& \hspace{-3cm} \lesssim \bigg\| \bigg\{\bigg( \int_{x_{k-1}}^{x_k} u(t)\,dt \sup_{x_k < s  \le x_{k+1}} \bigg( \int_{x_k}^s w \bigg)^{r / q} \bigg( \int_s^{\infty} h \bigg)^r \bigg)^{1/r} \bigg\} \bigg\|_{\ell^r({\mathcal Z})} \\
	& \hspace{-3cm} \le \bigg\| \bigg\{ \bigg( \int_{x_{k-1}}^{x_k} u(t) \sup_{t < s \le x_{k+1}} \bigg(
	\int_{x_k}^s w \bigg)^{r / q} \bigg( \int_s^{\infty} h	\bigg)^r \,dt\bigg)^{1/r} \bigg\} \bigg\|_{\ell^r({\mathcal Z})},
	\end{align*}
	by Lemma \ref{lem.1.1}, we obtain that
	\begin{align*}
	{\operatorname{LHS}}{\eqref{main}} \lesssim & \, \bigg\| \bigg\{ 2^{k / r} \bigg(
	\int_{x_k}^{x_{k+1}} \bigg( \int_s^{x_{k+1}} h \bigg)^q
	w(s)\,ds\bigg)^{ 1 / q} \bigg\} \bigg\|_{\ell^r({\mathcal Z})} \notag \\
	&  +  \bigg\| \bigg\{ \bigg( \int_{x_{k-1}}^{x_k} u(t) \sup_{t < s \le x_{k+1}} \bigg(
	\int_{x_k}^s w \bigg)^{r / q} \bigg( \int_s^{\infty} h
	\bigg)^r \,dt\bigg)^{1/r} \bigg\} \bigg\|_{\ell^r({\mathcal Z})}. 
	\end{align*}
	{\bf $\gtrsim:$} It is easy to see that
	\begin{align*}
	\bigg\| \bigg\{ \bigg( \int_{x_{k-1}}^{x_k} u(t) \sup_{t < s \le x_{k+1}} \bigg(
	\int_{x_k}^s w \bigg)^{r / q} \bigg( \int_s^{\infty} h
	\bigg)^r \,dt\bigg)^{1/r} \bigg\} \bigg\|_{\ell^r({\mathcal Z})} & \\
    & \hspace{-5cm}\le  \bigg\| \bigg\{ 2^{k / r} \sup_{x_{k-1} < s \le x_{k+1}} \bigg(
	\int_{x_{k-1}}^s w \bigg)^{1 / q} \bigg( \int_s^{\infty} h
	\bigg) \bigg\} \bigg\|_{\ell^r({\mathcal Z})}.
	\end{align*}

	Let $y_k \in (x_{k-1},x_{k+1}]$, $k \in {\mathcal Z}$, be such that
	$$
	\sup_{x_{k-1} < s \le x_{k+1}} \bigg( \int_{x_{k-1}}^s w \bigg)^{r / q} \bigg( \int_s^{\infty} h
	\bigg)^r \lesssim \bigg( \int_{x_{k-1}}^{y_k} w \bigg)^{r / q} \bigg( \int_{y_k}^{\infty} h	\bigg)^r.
	$$

	Thus
	\begin{align*}
	\bigg\| \bigg\{ \bigg( \int_{x_{k-1}}^{x_k} u(t) \sup_{t < s \le x_{k+1}} \bigg(
	\int_{x_k}^s w \bigg)^{r / q} \bigg( \int_s^{\infty} h
	\bigg)^r \,dt\bigg)^{1/r} \bigg\} \bigg\|_{\ell^r({\mathcal Z})} & \\
	&\hspace{-5cm} \lesssim  \bigg\| \bigg\{ 2^{k / r} \bigg( \int_{x_{k-1}}^{y_k} w \bigg)^{1 / q} \bigg( \int_{y_k}^{\infty} h	\bigg) \bigg\} \bigg\|_{\ell^r({\mathcal Z})} \\
	& \hspace{-5cm} \le \bigg\| \bigg\{ 2^{k / r} \bigg( \int_{x_{k-1}}^{y_k} w (s) \bigg( \int_s^{\infty} h	\bigg)^q \,ds \bigg)^{1 / q} \bigg\} \bigg\|_{\ell^r({\mathcal Z})} \\
	&\hspace{-5cm} \lesssim \bigg\| \bigg\{ \bigg( \int_{x_{k-2}}^{x_{k-1}} u(t)\,dt \bigg)^{1 / r} \,\bigg( \int_{x_{k-1}}^{\infty} w (s) \bigg( \int_s^{\infty} h	\bigg)^q \,ds \bigg)^{1 / q}  \bigg\} \bigg\|_{\ell^r({\mathcal Z})}\\
	& \hspace{-5cm}\le \bigg\| \bigg\{ \bigg( \int_{x_{k-2}}^{x_{k-1}} u(t) \bigg( \int_t^{\infty} w (s) \bigg( \int_s^{\infty} h	\bigg)^q \,ds \bigg)^{r / q} \,dt \bigg)^{1/r} \bigg\} \bigg\|_{\ell^r({\mathcal Z})} \\
	& \hspace{-5cm} \le \bigg(\int_0^{\infty} u(t) \bigg( \int_t^{\infty} w (s) \bigg( \int_s^{\infty} h	\bigg)^q \,ds \bigg)^{r / q} \,dt \bigg)^{1/r} = {\operatorname{LHS}} \eqref{main}.
	\end{align*}

	Consequently, by Lemma \ref{lem.1.1}, we have that
	\begin{align*}
	\bigg\| \bigg\{ 2^{k / r} \bigg(
	\int_{x_k}^{x_{k+1}} \bigg( \int_s^{x_{k+1}} h \bigg)^q
	w(s)\,ds\bigg)^{ 1 / q} \bigg\} \bigg\|_{\ell^r({\mathcal Z})}  & \\
	& \hspace{-5cm}  +  \bigg\| \bigg\{ \bigg( \int_{x_{k-1}}^{x_k} u(t) \sup_{t < s \le x_{k+1}} \bigg(
	\int_{x_k}^s w \bigg)^{r / q} \bigg( \int_s^{\infty} h
	\bigg)^r \,dt\bigg)^{1/r} \bigg\} \bigg\|_{\ell^r({\mathcal Z})} \lesssim {\operatorname{LHS}}{\eqref{main}}.
	\end{align*}	
\end{proof}

\begin{lem}\label{lem.1.4}
	Let $0 < q,\, r < \infty$ and  $u,\,v,\,w \in {\mathcal W}{(0,\infty)}$.	Assume that $\{x_k\}_{k=-\infty}^{M + 1}$ is a covering sequence mentioned in Remark \ref{rem.disc.}.
	Then
	\begin{align}
	{\operatorname{LHS}}{\eqref{main}} \thickapprox &  \bigg\| \bigg\{ 2^{k / r} \bigg(
	\int_{x_k}^{x_{k+1}} \bigg( \int_s^{x_{k+1}} h \bigg)^q
	w(s)\,ds\bigg)^{ 1 / q} \bigg\} \bigg\|_{\ell^r({\mathcal Z})} \notag \\
	&  +  \bigg( \int_0^{\infty}  u(t) \sup_{t < s} \bigg(
	\int_t^s w \bigg)^{r / q} \bigg( \int_s^{\infty} h\bigg)^r \,dt\bigg)^{1/r} 
	\end{align}
	with constants independent of $h \in {\mathfrak M}^+(0,\infty)$.
\end{lem}

\begin{proof}
	{\bf $\lesssim:$} Since 
	\begin{align*}
	\bigg\| \bigg\{ \bigg( \int_{x_{k-1}}^{x_k} u(t) \sup_{t < s \le x_{k+1}} \bigg(
	\int_{x_k}^s w \bigg)^{r / q} \bigg( \int_s^{\infty} h
	\bigg)^r \,dt\bigg)^{1/r} \bigg\} \bigg\|_{\ell^r({\mathcal Z})} & \\
	& \hspace{-5cm} \le \bigg\| \bigg\{ \bigg( \int_{x_{k-1}}^{x_k} u(t) \sup_{t < s} \bigg(
	\int_t^s w \bigg)^{r / q} \bigg( \int_s^{\infty} h\bigg)^r \,dt\bigg)^{1/r} \bigg\} \bigg\|_{\ell^r({\mathcal Z})} \\
	& \hspace{-5cm} \le \bigg( \int_0^{\infty}  u(t) \sup_{t < s} \bigg(
	\int_t^s w \bigg)^{r / q} \bigg( \int_s^{\infty} h\bigg)^r \,dt\bigg)^{1/r},
	\end{align*}
	then $\lesssim$ part of the statement follows by Lemma \ref{lem.1.2}.
	
	{\bf $\gtrsim:$} Since, by Lemma \ref{lem.1.2},
	$$
	\bigg\| \bigg\{ 2^{k / r} \bigg(
	\int_{x_k}^{x_{k+1}} \bigg( \int_s^{x_{k+1}} h \bigg)^q
	w(s)\,ds\bigg)^{ 1 / q} \bigg\} \bigg\|_{\ell^r({\mathcal Z})} \lesssim {\operatorname{LHS}}{\eqref{main}},
	$$
	it is enough to show that 
	$$
	\bigg( \int_0^{\infty}  u(t) \sup_{t < s} \bigg(
	\int_t^s w \bigg)^{r / q} \bigg( \int_s^{\infty} h\bigg)^r \,dt\bigg)^{1/r} \lesssim {\operatorname{LHS}}{\eqref{main}}.
	$$

	On using Lemma \ref{L:1.2}, it is easy to see that
	\begin{align*}
    \bigg( \int_0^{\infty}  u(t) \sup_{t < s} \bigg(
	\int_t^s w \bigg)^{r / q} \bigg( \int_s^{\infty} h\bigg)^r \,dt\bigg)^{1/r} & \\
	& \hspace{-3cm}  \approx \bigg\| \bigg\{  2^{n / r} \sup_{x_n < s} \bigg(
	\int_{x_n}^s w \bigg)^{1 / q} \bigg( \int_s^{\infty} h\bigg) \bigg\} \bigg\|_{\ell^r({\mathcal Z})} \\
	& \hspace{-3cm}  =  \bigg\| \bigg\{  2^{n / r} \sup_{n \le i \le M} \sup_{x_i < s \le x_{i+1}} \bigg(
	\int_{x_n}^s w \bigg)^{1 / q} \bigg( \int_s^{\infty} h\bigg) \bigg\} \bigg\|_{\ell^r({\mathcal Z})} \\
	& \hspace{-3cm}  \approx  \bigg\| \bigg\{  2^{n / r} \sup_{n \le i \le M} \sup_{x_i < s \le x_{i+1}} \bigg(
	\int_{x_i}^s w \bigg)^{1 / q} \bigg( \int_s^{\infty} h\bigg) \bigg\} \bigg\|_{\ell^r({\mathcal Z})} \\
	& \hspace{-2.5cm}  +  \bigg\| \bigg\{  2^{n / r} \sup_{n + 1\le i \le M}  \bigg(
	\int_{x_n}^{x_i} w \bigg)^{1 / q} \bigg( \int_{x_i}^{\infty} h\bigg)\bigg\} \bigg\|_{\ell^r({\mathcal Z})} \\
	& \hspace{-3cm}  \approx  \bigg\| \bigg\{  2^{n / r} \sup_{x_n < s \le x_{n+1}} \bigg(
	\int_{x_n}^s w \bigg)^{1 / q} \bigg( \int_s^{\infty} h\bigg) \bigg\} \bigg\|_{\ell^r({\mathcal Z})} \\
	& \hspace{-2.5cm}  +  \bigg\| \bigg\{  2^{n / r} \sup_{n + 1 \le i \le M}  \bigg(
	\int_{x_n}^{x_i} w \bigg)^{1 / q} \bigg( \int_{x_i}^{\infty} h\bigg)\bigg\} \bigg\|_{\ell^r({\mathcal Z})} : = B_1 + B_2.
	\end{align*}

	Note that
	\begin{align*}
    B_1 & \approx \bigg\| \bigg\{ \bigg( \int_{x_{n-1}}^{x_n} u(t)\,dt \bigg)^{1/r} \, \sup_{x_n < s \le x_{n+1}} \bigg(
	\int_{x_n}^s w \bigg)^{1 / q} \bigg( \int_s^{\infty} h\bigg)  \bigg\} \bigg\|_{\ell^r({\mathcal Z})} \\
	& \le \bigg\| \bigg\{ \bigg( \int_{x_{n-1}}^{x_n} u(t) \sup_{t < s \le x_{n+1}} \bigg(
	\int_t^s w \bigg)^{r / q} \bigg( \int_s^{\infty} h\bigg)^r \,dt \bigg)^{1/r} \bigg\} \bigg\|_{\ell^r({\mathcal Z})},
	\end{align*}
	and
	\begin{align*}
	B_2	& \le \bigg\| \bigg\{ 2^{n / r} \sup_{n + 1 \le i \le M}  \bigg(
	\int_{x_n}^{x_i} w(s) \,\bigg( \int_s^{\infty} h\bigg)^q \,ds\bigg)^{1 / q} \bigg\} \bigg\|_{\ell^r({\mathcal Z})} \\
	& \le \bigg\| \bigg\{ 2^{n / r}  \bigg(
	\int_{x_n}^{\infty} w(s) \bigg( \int_s^{\infty} h\bigg)^q \,ds \bigg)^{1 / q} \bigg\} \bigg\|_{\ell^r({\mathcal Z})} \\
	& \approx \bigg\| \bigg\{ \bigg( \int_{x_{n-1}}^{x_n} u(t)\,dt \bigg)^{1/r}  \, \bigg(
	\int_{x_n}^{\infty} w(s) \bigg( \int_s^{\infty} h\bigg)^q \,ds \bigg)^{1 / q} \bigg\} \bigg\|_{\ell^r({\mathcal Z})} \\
	& \le \bigg\| \bigg\{ \bigg( \int_{x_{n-1}}^{x_n} u(t) \bigg(
	\int_t^{\infty} w(s) \bigg( \int_s^{\infty} h\bigg)^q \,ds \bigg)^{r / q}  \,dt \bigg)^{1/r} \bigg\} \bigg\|_{\ell^r({\mathcal Z})} \\
	& \le \bigg(\int_0^{\infty} u(t) \bigg( \int_t^{\infty} w (s) \bigg( \int_s^{\infty} h	\bigg)^q \,ds \bigg)^{r / q} \,dt \bigg)^{1/r} = {\operatorname{LHS}} \eqref{main}.
	\end{align*}

	Consequently, on using lemma \ref{lem.1.2}, we have that
	\begin{align*}
	\bigg( \int_0^{\infty}  u(t) \sup_{t < s} \bigg(
	\int_t^s w \bigg)^{r / q} \bigg( \int_s^{\infty} h\bigg)^r \,dt\bigg)^{1/r} & \\
	& \hspace{-5cm}  \lesssim  \bigg\| \bigg\{ \bigg( \int_{x_{n-1}}^{x_n} u(t) \sup_{t < s \le x_{n+1}} \bigg(
	\int_t^s w \bigg)^{r / q} \bigg( \int_s^{\infty} h\bigg)^r \,dt \bigg)^{1/r} \bigg\} \bigg\|_{\ell^r({\mathcal Z})} +  {\operatorname{LHS}} \eqref{main} \lesssim {\operatorname{LHS}} \eqref{main}.
	\end{align*}	

	The proof is completed.	
\end{proof}	

The proof of the following statement is similar to the proof of \cite[Theorem 4.1, $(4.2) \Leftrightarrow (4.3)$]{GogStep1}.
For the convenience of the reader we give here the complete proof.
\begin{lem}\label{lem.1.5} 
	Let $0 < q,\,r < \infty$ and  $u,\,v,\,w \in {\mathcal W}{(0,\infty)}$.
    Then the following assertions are equivalent:
	\begin{align}
	\bigg( \int_0^{\infty}  u(t) \sup_{t < s} \bigg(
	\int_t^s w \bigg)^{r / q} \bigg( \int_s^{\infty} h\bigg)^r \,dt\bigg)^{1/r} \leq C \, \int_0^{\infty} h v, \quad h \in {\mathfrak M}^+(0,\infty), \label{eq.1.3} \\
	\bigg( \int_0^{\infty}  u(t) \bigg( \int_t^{\infty} \bigg(
	\int_t^s w \bigg)^{1 / q}  h(s) \,ds\bigg)^r \,dt\bigg)^{1/r} \leq C \,\int_0^{\infty} h v, \quad h \in {\mathfrak M}^+(0,\infty). \label{eq.1.4}
	\end{align}
\end{lem}

\begin{proof}
	At first note that
	\begin{align*}
	\int_t^{\infty} \bigg( \int_t^s w \bigg)^{1 / q}  h(s) \,ds = \sup_{t < \tau} \int_{\tau}^{\infty} \bigg( \int_t^s w \bigg)^{1 / q}  h(s) \,ds \ge \sup_{t < \tau} \bigg( \int_t^{\tau} w \bigg)^{1 / q} \int_{\tau}^{\infty}  h(s) \,ds.
	\end{align*}	
	Thus ${\operatorname{LHS}}{\eqref{eq.1.3}} \le {\operatorname{LHS}}{\eqref{eq.1.4}}$, and, consequently, $\eqref{eq.1.4} \Rightarrow \eqref{eq.1.3}$.
	
	Assume that \eqref{eq.1.3} holds. Let $\{x_k\}_{k=-\infty}^{M + 1}$ be a covering sequence mentioned in Remark \ref{rem.disc.}. We know that
	\begin{align*}
	{\operatorname{LHS}}{\eqref{eq.1.3}}  \approx B_1 + B_2.
	\end{align*}

	On using Lemma \ref{L:1.2}, it is easy to see that
	\begin{align*}
	{\operatorname{LHS}}{\eqref{eq.1.4}}  \approx &  \,\, \bigg\| \bigg\{ 2^{n / r}  \int_{x_n}^{\infty} \bigg(
	\int_{x_n}^s w \bigg)^{1 / q} h(s) \,ds \bigg\} \bigg\|_{\ell^r ({\mathcal Z})} \\
	= & \,\, \bigg\| \bigg\{ 2^{n / r}  \sum_{i = n}^M \int_{x_i}^{x_{i+1}} \bigg(
	\int_{x_n}^s w \bigg)^{1 / q} h(s) \,ds \bigg\} \bigg\|_{\ell^r ({\mathcal Z})} \\
	\approx & \,\,\bigg\| \bigg\{ 2^{n / r}  \sum_{i = n}^M \int_{x_i}^{x_{i+1}} \bigg(
	\int_{x_i}^s w \bigg)^{1 / q} h(s) \,ds \bigg\} \bigg\|_{\ell^r ({\mathcal Z})} \\
	& + \bigg\| \bigg\{ 2^{n / r} \sum_{i = n + 1}^M \int_{x_i}^{x_{i+1}} \bigg(
	\int_{x_n}^{x_i} w \bigg)^{1 / q} h(s) \,ds \bigg\} \bigg\|_{\ell^r ({\mathcal Z})} \\
	\approx & \,\,\bigg\| \bigg\{ 2^{n / r}  \int_{x_n}^{x_{n+1}} \bigg(
	\int_{x_n}^s w \bigg)^{1 / q} h(s) \,ds \bigg\} \bigg\|_{\ell^r ({\mathcal Z})} \\
	& + \bigg\| \bigg\{ 2^{n / r}  \sum_{i = n + 1}^M \bigg(
	\int_{x_n}^{x_i} w \bigg)^{1 / q} \bigg( \int_{x_i}^{x_{i+1}} h \bigg) \bigg\} \bigg\|_{\ell^r ({\mathcal Z})} : = A_1 + A_2.	
	\end{align*}

	Let $q \ge 1$. Using Jensen's inequality, by Lemma \ref{L:1.2}, we get that	
	\begin{align*}
	A_2 = & \,\, \bigg\| \bigg\{ 2^{n / r}  \sum_{i = n + 1}^M \bigg( \sum_{j = n+1}^i \int_{x_{j-1}}^{x_j} w \bigg)^{1 / q} \bigg( \int_{x_i}^{x_{i+1}} h \bigg) \bigg\} \bigg\|_{\ell^r ({\mathcal Z})} \\	
	\le & \,\,\bigg\| \bigg\{ 2^{n / r} \sum_{i = n + 1}^M \sum_{j = n+1}^i \bigg(
	\int_{x_{j-1}}^{x_j} w \bigg)^{1 / q} \bigg( \int_{x_i}^{x_{i+1}} h \bigg)\bigg\} \bigg\|_{\ell^r ({\mathcal Z})} \\	
	= & \,\, \bigg\| \bigg\{ 2^{n / r}  \sum_{j = n + 1}^M \sum_{i = j}^M \bigg(
	\int_{x_{j-1}}^{x_j} w \bigg)^{1 / q} \bigg( \int_{x_i}^{x_{i+1}} h \bigg) \bigg\} \bigg\|_{\ell^r ({\mathcal Z})} \\	
	= & \,\,\bigg\| \bigg\{ 2^{n / r}  \sum_{j = n + 1}^M \bigg(
	\int_{x_{j-1}}^{x_j} w \bigg)^{1 / q} \bigg( \int_{x_i}^{\infty} h \bigg) \bigg\} \bigg\|_{\ell^r ({\mathcal Z})} \\	
	\approx & \,\,\bigg\| \bigg\{ 2^{n / r} \bigg(
	\int_{x_n}^{x_{n+1}} w \bigg)^{1 / q} \bigg( \int_{x_{n+1}}^{\infty} h \bigg) \bigg\} \bigg\|_{\ell^r ({\mathcal Z})}\\
	\le & \,\,\bigg\| \bigg\{  2^{n / r} \sup_{n + 1 \le i \le M}  \bigg(
	\int_{x_n}^{x_i} w \bigg)^{1 / q} \bigg( \int_{x_i}^{\infty} h\bigg)\bigg\} \bigg\|_{\ell^r({\mathcal Z})} = B_2.
	\end{align*}
	
	Let $q < 1$. Then, by Minkowski's inequality, we have that
	\begin{align*}
	\sum_{i = n + 1}^M \bigg(
	\int_{x_n}^{x_i} w \bigg)^{1 / q} \bigg( \int_{x_i}^{x_{i+1}} h \bigg)  & = \sum_{i = n+ 1}^M \bigg(
	\sum_{j=n+1}^i \int_{x_{j-1}}^{x_j} w \bigg)^{1 / q} \bigg( \int_{x_i}^{x_{i+1}} h \bigg) \\
	& = \left\{ \left[ \sum_{i=n+ 1}^M \bigg( \sum_{j=n+1}^i \bigg( \int_{x_{j-1}}^{x_j} w\bigg) \bigg( \int_{x_i}^{x_{i+1}} h \bigg)^q \bigg)^{1/q}\right]^q \right\}^{1/q} \\
	& \le \left\{ \sum_{j=n+1}^M \left[ \sum_{i=j}^M \bigg(\int_{x_{j-1}}^{x_j} w\bigg)^{1/q} \bigg( \int_{x_i}^{x_{i+1}} h \bigg) \right]^q \right\}^{1/q} \\ 
	& = \left\{ \sum_{j=n+1}^M \bigg(\int_{x_{j-1}}^{x_j} w\bigg) \left[ \sum_{i=j}^M \bigg( \int_{x_i}^{x_{i+1}} h \bigg) \right]^q \right\}^{1/q} \\ 
	& = \left\{ \sum_{j=n+1}^M \bigg(\int_{x_{j-1}}^{x_j} w\bigg) \bigg( \int_{x_j}^{\infty} h \bigg)^q \right\}^{1/q}.
	\end{align*}

	Hence, by Lemma \ref{L:1.2}, we get that
	\begin{align*}
    A_2 & \le \bigg\| \bigg\{ 2^{n / r}  \bigg\{ \sum_{j=n+1}^M \bigg(\int_{x_{j-1}}^{x_j} w\bigg) \bigg( \int_{x_j}^{\infty} h \bigg)^q \bigg\}^{1/q}\bigg\} \bigg\|_{\ell^r ({\mathcal Z})} \\
    & = \bigg\| \bigg\{ 2^{n q / r}  \sum_{j=n+1}^M \bigg(\int_{x_{j-1}}^{x_j} w\bigg) \bigg( \int_{x_j}^{\infty} h \bigg)^q \bigg\} \bigg\|_{\ell^{r / q} ({\mathcal Z})}^{1 / q} \\
    & \approx \bigg\| \bigg\{ 2^{n q / r}  \bigg(\int_{x_n}^{x_{n+1}} w\bigg) \bigg( \int_{x_{n+1}}^{\infty} h \bigg)^q \bigg\} \bigg\|_{\ell^{r / q} ({\mathcal Z})}^{1 / q} \\
    & = \bigg\| \bigg\{ 2^{n / r} \bigg(
    \int_{x_n}^{x_{n+1}} w \bigg)^{1 / q} \bigg( \int_{x_{n+1}}^{\infty} h \bigg) \bigg\} \bigg\|_{\ell^r ({\mathcal Z})} \le B_2.
	\end{align*}
	
	Thus in both cases
	$$
	A_2 \lesssim B_2 \le C \, \int_0^{\infty} h v, \quad h \in {\mathfrak M}^+(0,\infty).
	$$	
	
	On the other hand, by H\"{o}lder's inequality, we have that
	\begin{align*}
	A_1 \le \bigg\| \bigg\{ 2^{n / r} \bigg( \sup_{{x_n} < s \le x_{n+1}} \bigg(
	\int_{x_n}^s w \bigg)^{1 / q} v(s)^{-1} \bigg) \bigg( \int_{x_n}^{x_{n+1}} h v\bigg)\bigg\} \bigg\|_{\ell^r ({\mathcal Z})}. 
	\end{align*}

	Using inequality
	\begin{equation}\label{discrete.Hold.}
	\|\{ a_k b_k \}\|_{\ell^r ({\mathcal Z})} \le \|\{ a_k \}\|_{\ell^{\rho} ({\mathcal Z})} \|\{
	b_k \}\|_{\ell^1 ({\mathcal Z})},
	\end{equation}
	where $1 / \rho = (1 / r - 1)_+$, we get that
	\begin{align*}
	A_1 \le & \, \bigg\| \bigg\{ 2^{n/r} \bigg( \sup_{{x_n} < s \le x_{n+1}} \bigg(
	\int_{x_n}^s w \bigg)^{1 / q} v(s)^{-1} \bigg) \bigg\} \bigg\|_{\ell^{\rho}({\mathcal Z})} \,  \int_0^{\infty} h v =: \, D  \, \int_0^{\infty} h v.
	\end{align*}

	Note that
	\begin{align*}
	\bigg\| \bigg\{ 2^{n/r} \sup_{x_n < s \le x_{n+1}} \bigg( \int_{x_n}^s w \bigg)^{1 / q} \bigg( \int_s^{x_{n+1}} h\bigg) \bigg\} \bigg\|_{\ell^r ({\mathcal Z})} \le B_1 & \lesssim C\, \int_0^{\infty} h v \\
	& = C\, \bigg\| \bigg\{ \int_{x_n}^{x_{n+1}} h v \bigg\} \bigg\|_{\ell^1 ({\mathcal Z})}, \quad h \in {\mathfrak M}^+(0,\infty).
	\end{align*}

	Since 
	$$
	\sup \bigg\{ \sup_{x_n < s \le x_{n+1}} \bigg(
	\int_{x_n}^s w \bigg)^{1 / q} \bigg( \int_s^{x_{n+1}} h\bigg) :\,  \int_{x_n}^{x_{n+1}} hv \le 1 \bigg\} = \sup_{{x_n} < s \le x_{n+1}} \bigg(	\int_{x_n}^s w \bigg)^{1 / q} v(s)^{-1}
	$$
	for any $n \in {\mathcal Z}$, there exist $h_n$ such that ${\operatorname{supp}} h_n \subset [x_n,x_{n+1}]$, $\int_{x_n}^{x_{n+1}} h_n v = 1$ and
	$$
	\sup_{x_n < s \le x_{n+1}}\bigg(\int_{x_n}^s w \bigg)^{1 / q} \bigg( \int_s^{x_{n+1}} h_n\bigg) \ge \frac{1}{2}
	\sup_{{x_n} < s \le x_{n+1}} \bigg(	\int_{x_n}^s w \bigg)^{1 / q} v(s)^{-1}.
	$$

	Consequently, by Proposition \ref{prop.2.1}, we obtain that
	\begin{align*}
	C & \gtrsim \sup_{h \ge 0} \frac{\bigg\| \bigg\{ 2^{n/r} \sup_{x_n < s \le x_{n+1}} \bigg( \int_{x_n}^s w \bigg)^{1 / q} \bigg( \int_s^{x_{n+1}} h\bigg) \bigg\} \bigg\|_{\ell^r ({\mathcal Z})}}{\bigg\| \bigg\{ \int_{x_n}^{x_{n+1}} h v \bigg\} \bigg\|_{\ell^1 ({\mathcal Z})}} \\
	& \ge \sup_{h = \sum_{n = - \infty}^M a_n h_n} \frac{\bigg\| \bigg\{ a_n 2^{n/r} \sup_{x_n < s \le x_{n+1}} \bigg( \int_{x_n}^s w \bigg)^{1 / q} \bigg( \int_s^{x_{n+1}} h_n \bigg) \bigg\} \bigg\|_{\ell^r ({\mathcal Z})}}{\bigg\| \bigg\{a_n \int_{x_n}^{x_{n+1}} h_n v \bigg\} \bigg\|_{\ell^1 ({\mathcal Z})}} \\
	& \ge \sup_{h = \sum_{n = - \infty}^M a_n h_n} \frac{\bigg\| \bigg\{ a_n 2^{n/r} \sup_{{x_n} < s \le x_{n+1}} \bigg(	\int_{x_n}^s w \bigg)^{1 / q} v(s)^{-1} \bigg\} \bigg\|_{\ell^r ({\mathcal Z})}}{\| \{a_n  \} \|_{\ell^1 ({\mathcal Z})}} \\	
	& = \sup_{\{a_n\}_{n \in {\mathcal Z}}: a_n \ge 0} \frac{\bigg\| \bigg\{ a_n 2^{n/r} \sup_{{x_n} < s \le x_{n+1}} \bigg(	\int_{x_n}^s w \bigg)^{1 / q} v(s)^{-1} \bigg\} \bigg\|_{\ell^r ({\mathcal Z})}}{\| \{a_n  \} \|_{\ell^1 ({\mathcal Z})}} \\
	& \ge \bigg\| \bigg\{ 2^{n/r} \bigg( \sup_{{x_n} < s \le x_{n+1}} \bigg(
	\int_{x_n}^s w \bigg)^{1 / q} v(s)^{-1} \bigg) \bigg\} \bigg\|_{\ell^{\rho}({\mathcal Z})} = D.
	\end{align*}

	Hence
	$$
	A_1 \lesssim \, D  \, \int_0^{\infty} h v \lesssim \, C  \, \int_0^{\infty} h v.
	$$

	Combining, we get that
	$$
	{\operatorname{LHS}}{\eqref{eq.1.4}} \approx A_1 + A_2 \lesssim \, C  \, \int_0^{\infty} h v.
	$$

	We have proved that $\eqref{eq.1.3} \Rightarrow \eqref{eq.1.4}$.
	
	The proof is completed.
\end{proof}

Combining Theorem \ref{lem.1.5} with Theorems \ref{Oinar} and \ref{Krep}, we get the following statement.
\begin{cor}\label{cor.1.1}
	Let $0 < r < \infty$, $0 < q < \infty$, and  $u,\,v,\,w \in {\mathcal W}{(0,\infty)}$.
	
	{\rm (a)} Let $r \ge 1$. Then inequality \eqref{eq.1.3} holds if and only if
	\begin{align*}
	D_1 : & = \sup_{t \in (0,\infty)} \bigg( \int_0^t   u(x) \bigg( \int_x^t w \bigg)^{r / q}  \,dx \bigg)^{1 / r} {\operatornamewithlimits{ess\,sup}}_{t \le  s < \infty } v(s)^{-1} < \infty, \\
	D_2 : & = \sup_{t \in (0,\infty)}  \bigg( \int_0^t u(x) \,dx \bigg)^{1/r}  \bigg( {\operatornamewithlimits{ess\,sup}}_{t \le s < \infty} \bigg( \int_t^s w \bigg)^{1/q} v(s)^{-1} \bigg) < \infty.
	\end{align*}
	Moreover, if $C$ is the best constant in \eqref{eq.1.3}, then 
	$$
	C \approx D_1 + D_2. 
	$$
	
	{\rm (b)} Let $r < 1$. Then inequality \eqref{eq.1.3} holds if and only if
	\begin{align*}
	E_1 : & = \bigg( \int_0^{\infty}  \bigg( \int_0^t u \bigg)^{r'}  u(t) \bigg( {\operatornamewithlimits{ess\,sup}}_{t \le s < \infty} \bigg(
	\int_t^s w \bigg)^{r' / q} v(s)^{-r'}\bigg) \,dt\bigg)^{1/r'} < \infty, \\
	E_2 : & = \bigg( \int_0^{\infty}  \bigg( \int_0^t u(x) \bigg( \int_x^t w \bigg)^{r / q} dx \bigg)^{r'} u(t) \bigg( {\operatornamewithlimits{ess\,sup}}_{t \le s < \infty} \bigg( \int_t^s w \bigg)^{r/q} v(s)^{-r'} \bigg) \,dt\bigg)^{1/r'}  < \infty.
	\end{align*}
	Moreover, if $C$ is the best constant in \eqref{eq.1.3}, then 
	$$
	C \approx E_1 + E_2. 
	$$	
\end{cor}

Now we give characterization of the following discrete inequality.
\begin{lem}\label{lem.1.6}
	Let $0 < q,\, r < \infty$ and  $u,\,w \in {\mathcal W}(0,\infty)$ and $v \in {\mathcal W}(0,\infty) \cap {{\mathfrak M}}^{\uparrow}(0,\infty)$. Assume that $\{x_k\}_{k=-\infty}^{M + 1}$ is a covering sequence mentioned in Remark \ref{rem.disc.}.
	Then the inequality
	\begin{equation}\label{disc.ineq.1}
	\bigg\| \bigg\{ 2^{k / r} \bigg( \int_{x_k}^{x_{k+1}} \bigg( \int_s^{x_{k+1}} h \bigg)^q
	w(s)\,ds\bigg)^{ 1 / q} \bigg\} \bigg\|_{\ell^r({\mathcal Z})} \le C \bigg\| \bigg\{ \int_{x_n}^{x_{n+1}} h v \bigg\} \bigg\|_{\ell^1 ({\mathcal Z})}, \quad h \in {\mathfrak M}^+(0,\infty) 
	\end{equation}
	holds with constant independent of $h \in {\mathfrak M}^+(0,\infty)$ if and only if $A < \infty$, where
	$$
	A = \bigg\| \bigg\{ 2^{k / r} \bigg( \int_{x_k}^{x_{k+1}} v(s)^{-q} w(s)\,ds\bigg)^{ 1 / q} \bigg\} \bigg\|_{\ell^{\rho}({\mathcal Z})}.
	$$
	Moreover, if $C$ is the best constant in \eqref{disc.ineq.1}, then $C \approx A$.	
\end{lem}

\begin{proof} {\bf Necessity.}	Assume that $A < \infty$.	By monotonicity of $v$, we have that
	\begin{align*}
	{\operatorname{LHS}} \eqref{disc.ineq.1} & \le \bigg\| \bigg\{ 2^{k / r} \bigg( \int_{x_k}^{x_{k+1}}\bigg( \int_s^{x_{k+1}} h v\bigg)^q v(s)^{-q} w(s)\,ds\bigg)^{ 1 / q} \bigg\} \bigg\|_{\ell^r({\mathcal Z})} \\
    & \le \bigg\| \bigg\{ 2^{k / r} \bigg( \int_{x_k}^{x_{k+1}} v(s)^{-q} 
	w(s)\,ds\bigg)^{ 1 / q} \bigg( \int_{x_k}^{x_{k+1}} h v\bigg)\bigg\} \bigg\|_{\ell^r({\mathcal Z})}.	
	\end{align*} 

	By \eqref{discrete.Hold.}, we get that
	$$
	{\operatorname{LHS}} \eqref{disc.ineq.1} \le 
	\bigg\| \bigg\{ 2^{k / r} \bigg( \int_{x_k}^{x_{k+1}} v(s)^{-q} w(s)\,ds\bigg)^{ 1 / q} \bigg\} \bigg\|_{\ell^{\rho}({\mathcal Z})}  \bigg\| \bigg\{ \int_{x_n}^{x_{n+1}} h v \bigg\} \bigg\|_{\ell^1 ({\mathcal Z})}.
	$$ 

	Thus	
    $$
    C \le \bigg\| \bigg\{ 2^{k / r} \bigg( \int_{x_k}^{x_{k+1}} v(s)^{-q} w(s)\,ds\bigg)^{ 1 / q} \bigg\} \bigg\|_{\ell^{\rho}({\mathcal Z})} = A.
    $$	
    
    {\bf Sufficiency.}	Assume that \eqref{disc.ineq.1} holds. Since
    $$
    \sup_{h \ge 0} \frac{\bigg(	\int_{x_k}^{x_{k+1}} \bigg( \int_s^{x_{k+1}} h \bigg)^q
   	w(s)\,ds\bigg)^{ 1 / q}}{\int_{x_n}^{x_{n+1}} h v} = \bigg( \int_{x_k}^{x_{k+1}} v(s)^{-q} w(s)\,ds\bigg)^{ 1 / q},
    $$
    then for any $k \in {\mathcal Z}$ there exists $h_k \in {{\mathfrak M}}^+(0,\infty)$ such that ${\operatorname{supp}} h_k \in (x_k,x_{k+1})$, $\int_{x_k}^{x_{k+1}} h_k v = 1$ and
    $$
    \bigg(	\int_{x_k}^{x_{k+1}} \bigg( \int_s^{x_{k+1}} h_k \bigg)^q
    w(s)\,ds\bigg)^{ 1 / q} \ge \frac{1}{2} \bigg( \int_{x_k}^{x_{k+1}} v(s)^{-q} w(s)\,ds\bigg)^{ 1 / q}.
    $$

    Define
    \begin{equation}\label{g}
    	h = \sum_{m \in {\mathcal Z}} a_m h_m,
    \end{equation}
    where $\{a_m\}_{m \in {\mathcal Z}}$ is any sequence of positive numbers. Then the inequality
    \begin{equation}\label{disc.ineq.1.1}
    \bigg\| \bigg\{ a_k 2^{k / r} \bigg( \int_{x_k}^{x_{k+1}} v(s)^{-q} w(s)\,ds\bigg)^{ 1 / q} \bigg\} \bigg\|_{\ell^r ({\mathcal Z})} \le C \, \|\{a_k\}\|_{\ell^1({\mathcal Z})}
    \end{equation}
    holds. 
    
    By Proposition \ref{prop.2.1}, we obtain that
    $$
    A = \bigg\| \bigg\{2^{k / r} \bigg( \int_{x_k}^{x_{k+1}} v(s)^{-q} w(s)\,ds\bigg)^{ 1 / q}\bigg\}\bigg\|_{\ell^{\rho}({\mathcal Z})} \lesssim C.
    $$
    
    The proof is completed.
\end{proof}

\begin{lem}\label{lem.1.7}
	Let $0 < q,\, r < \infty$ and  $u,\,w \in {\mathcal W}(0,\infty)$ and $v \in {\mathcal W}(0,\infty) \cap {{\mathfrak M}}^{\uparrow}(0,\infty)$. Assume that $\{x_k\}_{k=-\infty}^{M + 1}$ is a covering sequence mentioned in Remark \ref{rem.disc.}.
	Then inequality \eqref{disc.ineq.1}	holds with constant independent of $h \in {\mathfrak M}^+(0,\infty)$ if and only if $B < \infty$, where
	$$
	B = 
	\begin{cases}
	\bigg( \int_0^{\infty} \bigg( \int_0^t u(s)\,ds\bigg)^{r'} \bigg( \int_t^{\infty} v(s)^{-q} w(s)\,ds \bigg)^{ r' / q}u(t)\,dt \bigg)^{1/r'}  & \quad \mbox{if} \quad r < 1, \\
	\sup_{t > 0} \bigg( \int_0^t u(s)\,ds \bigg)^{1 / r} \bigg( \int_t^{\infty} v(s)^{-q} w(s)\,ds \bigg)^{ 1 / q}  & \quad \mbox{if} \quad r \ge 1.
	\end{cases}
	$$
	Moreover, if $C$ is the best constant in \eqref{disc.ineq.1}, then $C \approx B$.	
\end{lem}

\begin{proof}
	{\bf Necessity.} Let inequality \eqref{disc.ineq.1} holds. Then, by Lemma \ref{lem.1.6}, $A < \infty$ and the best constant in \eqref{disc.ineq.1} satisfies $C \approx A$.
	
	{\rm (a)} Let $r < 1$. Then, on using Lemma \ref{L:1.2}, we get that
	\begin{align*}
	B = \bigg( \int_0^{\infty} \bigg( \int_0^t u(s)\,ds\bigg)^{r'} \bigg( \int_t^{\infty} v(s)^{-q} w(s)\,ds \bigg)^{ r' / q}u(t)\,dt \bigg)^{1/r'} & \\
	& \hspace{-7cm} = \bigg( \sum_{k = -\infty}^M \int_{x_k}^{x_{k+1}} \bigg( \int_0^t u(s)\,ds\bigg)^{r'} \bigg( \int_t^{\infty} v(s)^{-q} w(s)\,ds \bigg)^{ r' / q}u(t)\,dt \bigg)^{1/r'} \\
	& \hspace{-7cm} \le \bigg( \sum_{k = -\infty}^M \int_{x_k}^{x_{k+1}} \bigg( \int_0^t u(s)\,ds\bigg)^{r'} u(t)\,dt \,\bigg( \int_{x_k}^{\infty} v(s)^{-q} w(s)\,ds \bigg)^{ r' / q} \bigg)^{1/r'} \\
	& \hspace{-7cm} \approx \bigg( \sum_{k = -\infty}^M \bigg( \bigg( \int_0^{x_{k+1}} u(s)\,ds \bigg)^{r' / r} - \bigg( \int_0^{x_{k}} u(s)\,ds\bigg)^{r' / r} \bigg)\,\bigg( \int_{x_k}^{\infty} v(s)^{-q} w(s)\,ds \bigg)^{ r' / q} \bigg)^{1/r'} \\
	& \hspace{-7cm} \approx \bigg( \sum_{k = -\infty}^M 2^{kr' / r} \,\bigg( \int_{x_k}^{\infty} v(s)^{-q} w(s)\,ds \bigg)^{ r' / q} \bigg)^{1/r'} \\
	& \hspace{-7cm} \approx \bigg( \sum_{k = -\infty}^M 2^{kr' / r} \,\bigg( \int_{x_k}^{x_{k+1}} v(s)^{-q} w(s)\,ds \bigg)^{ r' / q} \bigg)^{1/r'} = A.
	\end{align*}
	
	{\rm (b)} Let $r \ge 1$. Then, on using Lemma \ref{L:1.2}, we get that
	\begin{align*}
	B = \sup_{t > 0} \bigg( \int_0^t u(s)\,ds \bigg)^{1 / r} \bigg( \int_t^{\infty} v(s)^{-q} w(s)\,ds \bigg)^{ 1 / q} & \\
	& \hspace{-7cm} =  \sup_{-\infty < k \le M} \sup_{{x_k} < t \le {x_{k+1}}} \bigg( \int_0^t u(s)\,ds \bigg)^{1 / r} \bigg( \int_t^{\infty} v(s)^{-q} w(s)\,ds \bigg)^{ 1 / q} \\
	& \hspace{-7cm} \le \sup_{-\infty < k \le M} \bigg( \int_0^{x_{k+1}} u(s)\,ds \bigg)^{1 / r} \bigg( \int_{x_k}^{\infty} v(s)^{-q} w(s)\,ds \bigg)^{ 1 / q} \\	
	& \hspace{-7cm} \approx \sup_{-\infty < k \le M} 2^{k / r} \bigg( \int_{x_k}^{\infty} v(s)^{-q} w(s)\,ds \bigg)^{ 1 / q} \\
	& \hspace{-7cm} \approx \sup_{-\infty < k \le M} 2^{k / r} \bigg( \int_{x_k}^{x_{k+1}} v(s)^{-q} w(s)\,ds \bigg)^{ 1 / q} = A.
	\end{align*}	
	
	Consequently, in both cases, by Lemma \ref{lem.1.6}, we have that $B \lesssim A \approx C < \infty$.
	
    {\bf Sufficiency.} Assume that $B < \infty$.
    
   	{\rm (a)} Let $r < 1$. Since $\int_{x_{k-1}}^{x_k} \bigg( \int_{x_{k-1}}^t u(s)\,ds \bigg)^{r'} u(t)\,dt \approx 2^{k r' / r}$, ${-\infty < k \le M}$, we have that
   	\begin{align*}
   	A = & \, \bigg( \sum_{k = -\infty}^M 2^{k r' / r} \bigg( \int_{x_k}^{x_{k+1}} v(s)^{-q} w(s)\,ds\bigg)^{ r' / q} \bigg)^{1/r'} \\
   	\approx & \, \bigg( \sum_{k = -\infty}^M \int_{x_{k-1}}^{x_k} \bigg( \int_{x_{k-1}}^t u(s)\,ds \bigg)^{r'} u(t)\,dt \bigg( \int_{x_k}^{x_{k+1}} v(s)^{-q} w(s)\,ds\bigg)^{ r' / q} \bigg)^{1/r'} \\
   	\le & \, \bigg( \sum_{k = -\infty}^M \int_{x_{k-1}}^{x_k} \bigg( \int_{x_{k-1}}^t u(s)\,ds \bigg)^{r'} \bigg( \int_t^{x_{k+1}} v(s)^{-q} w(s)\,ds\bigg)^{ r' / q} \,u(t)\,dt  \bigg)^{1/r'} \\
   	\le & \, \bigg( \sum_{k = -\infty}^M \int_{x_{k-1}}^{x_k} \bigg( \int_0^t u(s)\,ds \bigg)^{r'} \bigg( \int_t^{\infty} v(s)^{-q} w(s)\,ds\bigg)^{ r' / q} \,u(t)\,dt \bigg)^{1/r'} \\
   	\le & \, \bigg( \int_0^{\infty} \bigg( \int_0^t u(s)\,ds\bigg)^{r'} \bigg( \int_t^{\infty} v(s)^{-q} w(s)\,ds \bigg)^{ r' / q}u(t)\,dt \bigg)^{1/r'} = B.
   	\end{align*}
   	
	{\rm (b)} Let $r \ge 1$. In this case, it is easy to see that
	\begin{align*}
	A = & \, \sup_{-\infty < k \le M} 2^{k / r} \bigg( \int_{x_k}^{x_{k+1}} v(s)^{-q} w(s)\,ds\bigg)^{ 1 / q} \\
	\le & \, \sup_{-\infty < k \le M} \bigg( \int_0^{x_k} u(s)\,ds \bigg)^{1/ r}\bigg( \int_{x_k}^{\infty} v(s)^{-q} w(s)\,ds\bigg)^{ 1 / q} \\
	\le & \, \sup_{t > 0} \bigg( \int_0^t u(s)\,ds \bigg)^{1/ r} \bigg( \int_t^{\infty} v(s)^{-q} w(s)\,ds \bigg)^{ 1 / q} = B.
	\end{align*}
	By Lemma \ref{lem.1.6}, in both cases we obtain that $C \approx A \lesssim B < \infty$.
\end{proof}

We are now in position to prove our main result.

\noindent{\bf Proof of Theorem \ref{main1}.} By Theorem \ref{Sinnamon.thm.3} (applied with $d\mu = dx$ on ${(0,\infty)}$, $X = L^r(u,{(0,\infty)})$ and $Tg (x) = \|g\|_{q,w,(x,\infty)}$, $x > 0$), inequality \eqref{main} holds if and only if the inequality
\begin{equation}\label{main5}
\bigg( \int_0^{\infty} \bigg( \int_x^{\infty} \bigg( \int_t^{\infty} h \bigg)^q w(t)\,dt
\bigg)^{r / q} u(x)\,ds \bigg)^{1/r}\leq c \,\int_0^{\infty} h v_{\uparrow}, \quad h \in {\mathfrak M}^+(0,\infty)
\end{equation}
holds. Moreover, the best constants in both inequalities are the same. Now the statement follows by Lemma \ref{lem.1.4}, combined with Lemma \ref{lem.1.7} and Corollary \ref{cor.1.1}.

\hspace{16.8cm}$\square$

\begin{bibdiv}
    \begin{biblist}

        \bib{BGGM1}{article}{
            author={Burenkov, V. I.},
            author={Gogatishvili, A.},
            author={Guliyev, V. S.},
            author={Mustafayev, R. Ch.},
            title={Boundedness of the fractional maximal operator in local
                Morrey-type spaces},
            journal={Complex Var. Elliptic Equ.},
            volume={55},
            date={2010},
            number={8-10},
            pages={739--758},
            issn={1747-6933},
            review={\MR{2674862 (2011f:42015)}},
        }

        \bib{BGGM2}{article}{
            author={Burenkov, V.I.},
            author={Gogatishvili, A.},
            author={Guliyev, V.S.},
            author={Mustafayev, R.Ch.},
            title={Boundedness of the Riesz potential in local Morrey-type spaces},
            journal={Potential Anal.},
            volume={35},
            date={2011},
            number={1},
            pages={67--87},
            issn={0926-2601},
            review={\MR{2804553 (2012d:42027)}},
        }

        \bib{BO}{article}{
            author={Burenkov, V.I.},
            author={Oinarov, R.},
            title={Necessary and sufficient conditions for boundedness of the
                Hardy-type operator from a weighted Lebesgue space to a Morrey-type
                space},
            journal={Math. Inequal. Appl.},
            volume={16},
            date={2013},
            number={1},
            pages={1--19},
            issn={1331-4343},
            review={\MR{3060376}},
        }

        \bib{GogMusIHI}{article}{
        	author={Gogatishvili, A.},
        	author={Mustafayev, R. Ch.},
        	title={Weighted iterated Hardy-type inequalities},
        	journal={Math. Inequal. Appl. (accepted)},
        	date={2016},
        	pages={},
        	issn={},
        	doi={},
        }
        
        \bib{GogMusPers1}{article}{
            author={Gogatishvili, A.},
            author={Mustafayev, R. Ch.},
            author={Persson, L.-E.},
            title={Some new iterated Hardy-type inequalities},
            journal={J. Funct. Spaces Appl.},
            date={2012},
            pages={Art. ID 734194, 30},
            issn={0972-6802},
            review={\MR{3000818}},
        }

        \bib{GogMusPers2}{article}{
            author={Gogatishvili, A.},
            author={Mustafayev, R. Ch.},
            author={Persson, L.-E.},
            title={Some new iterated Hardy-type inequalities: the case $\theta = 1$},
            journal={J. Inequal. Appl.},
            date={2013},
            pages={29 pp.},
            issn={},
            doi={10.1186/1029-242X-2013-515},
        }

        \bib{GogStep1}{article}{
            author={Gogatishvili, A.},
            author={Stepanov, V.D.},
            title={Reduction theorems for operators on the cones of monotone
                functions},
            journal={J. Math. Anal. Appl.},
            volume={405},
            date={2013},
            number={1},
            pages={156--172},
            issn={0022-247X},
            review={\MR{3053495}},
            doi={10.1016/j.jmaa.2013.03.046},
        }

        \bib{GogStep}{article}{
            author={Gogatishvili, A.},
            author={Stepanov, V. D.},
            title={Reduction theorems for weighted integral inequalities on the cone
                of monotone functions},
            language={Russian, with Russian summary},
            journal={Uspekhi Mat. Nauk},
            volume={68},
            date={2013},
            number={4(412)},
            pages={3--68},
            issn={0042-1316},
            translation={
                journal={Russian Math. Surveys},
                volume={68},
                date={2013},
                number={4},
                pages={597--664},
                issn={0036-0279},
            },
            review={\MR{3154814}},
        }
    
        \bib{gp1}{article}{
        	author={Gogatishvili, A.},
        	author={Pick, L.},
        	title={Discretization and anti-discretization of rearrangement-invariant
        		norms},
        	journal={Publ. Mat.},
        	volume={47},
        	date={2003},
        	number={2},
        	pages={311--358},
        	issn={0214-1493},
        	review={\MR{2006487 (2005f:46053)}},
        }
        
		\bib{Krep}{article}{
			author={K\v{r}epela, M.},			
			title={Boundedness of Hardy-type operators with a kernel integral weighted conditions for the case $0 < q < 1 \le p < \infty$},
			journal={Preprint},
			volume={},
			date={2016},
			number={},
			pages={},
			issn={},
			review={},
		}
      
        \bib{Lai1999}{article}{
        	author={Lai, Q.},
        	title={Weighted modular inequalities for Hardy type operators},
        	journal={Proc. London Math. Soc. (3)},
        	volume={79},
        	date={1999},
        	number={3},
        	pages={649--672},
        	issn={0024-6115},
        	review={\MR{1710168}},
        	doi={10.1112/S0024611599012010},
        }
        
        \bib{Le1}{article}{
        	author={Leindler, L.},
        	title={Inequalities of Hardy-Littlewood type},
        	language={English, with Russian summary},
        	journal={Anal. Math.},
        	volume={2},
        	date={1976},
        	number={2},
        	pages={117--123},
        	issn={0133-3852},
        	review={\MR{0427566 (55 \#597)}},
        }
        
        \bib{Le2}{article}{
        	author={Leindler, L. },
        	title={On the converses of inequalities of Hardy and Littlewood},
        	journal={Acta Sci. Math. (Szeged)},
        	volume={58},
        	date={1993},
        	number={1-4},
        	pages={191--196},
        	issn={0001-6969},
        	review={\MR{1264231 (95j:40001)}},
        }

        \bib{Oinar}{article}{
        	author={O{\u\i}narov, R.},
        	title={Two-sided estimates for the norm of some classes of integral
        		operators},
        	language={Russian},
        	journal={Trudy Mat. Inst. Steklov.},
        	volume={204},
        	date={1993},
        	number={Issled. po Teor. Differ. Funktsii Mnogikh Peremen. i ee
        		Prilozh. 16},
        	pages={240--250},
        	issn={0371-9685},
        	translation={
        		journal={Proc. Steklov Inst. Math.},
        		date={1994},
        		number={3 (204)},
        		pages={205--214},
        		issn={0081-5438},
        	},
        	review={\MR{1320028}},
        }

        \bib{ProkhStep1}{article}{
            author={Prokhorov, D. V.},
            author={Stepanov, V. D.},
            title={On weighted Hardy inequalities in mixed norms},
            journal={Proc. Steklov Inst. Math.},
            volume={283},
            date={2013},
            pages={149–-164},
        }

        \bib{Sinn}{article}{
            author={Sinnamon, G.},
            title={Transferring monotonicity in weighted norm inequalities},
            journal={Collect. Math.},
            volume={54},
            date={2003},
            number={2},
            pages={181--216},
            issn={0010-0757},
            review={\MR{1995140 (2004m:26031)}},
        }

        \bib{ss}{article}{
            author={Sinnamon, G.},
            author={Stepanov, V.D.},
            title={The weighted Hardy inequality: new proofs and the case $p=1$},
            journal={J. London Math. Soc. (2)},
            volume={54},
            date={1996},
            number={1},
            pages={89--101},
            issn={0024-6107},
            review={\MR{1395069 (97e:26021)}},
            doi={10.1112/jlms/54.1.89},
        }

\end{biblist}
\end{bibdiv}

\end{document}


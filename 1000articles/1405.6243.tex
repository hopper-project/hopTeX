\documentclass[12pt,twoside]{amsart}
\usepackage{amsmath}
\usepackage{amssymb}
\usepackage{amscd}
\usepackage{xypic}
\xyoption{all}
\setlength{\textwidth}{15.1cm}
\setlength{\evensidemargin}{0mm} \setlength{\oddsidemargin}{0mm}

\title[Higher residue pairing over $P$-rings and Crystalline cohomology]{Higher residue pairing over $P$-rings and Crystalline cohomology}

\author{Mohammad Reza Rahmati}
\thanks{}
\address{ Centro de Investigacion en Matematicas, CIMAT
\hfill\break 
\hfill\break \\
\hfill\break }
\email{mrahmati@cimat.mx}

 

\newtheorem{theorem}{Theorem}[section]
\newtheorem{proposition}[theorem]{Proposition}
\newtheorem{corollary}[theorem]{Corollary}
\newtheorem{lemma}[theorem]{Lemma}
\newtheorem{definition}[theorem]{Definition}
\newtheorem{remark}[theorem]{Remark}
\newtheorem{example}[theorem]{Example}
\newtheorem{claim}[theorem]{Claim}
\newtheorem{conjecture}[theorem]{Conjecture}

\keywords{Ring of Witt vectors, Complete local rings, Higher residue pairing, Crystalline cohomology}

\subjclass{}

\begin{document}

\begin{abstract}
We have generalized the formal definition of K. Saito higher residue pairing to a degenerate family of algebraic varieties defined over $Spec(W(k))$, the ring of Witt vectors over a perfect field $k$ of char $>0$. We also restate the compatibility of this generalization by the period isomorphism with etale cohomology with coefficient 
in $\mathbb{C}_p$.
\end{abstract}

\maketitle

\section*{Introduction}

\vspace{0.5cm}

The construction of the higher residue pairing originally belonged to K. Saito, \cite{SA1}, may be re-phrased in terms of an identification of twisted de Rham complex and the formal complex of poly-vector fields, \cite{LLS}. This allows to formulate higher residue by the trace map, via symplectic pairing as;

\[ K^f( \ , \ ):\mathcal{H}_{(0)}^f \times \mathcal{H}_{(0)}^f \to \mathcal{O}_{S,0}[[t]] \]

\vspace{0.5cm}

\noindent
The construction is compatible with base change and is compatible with the lifting along an inverse system defined by a filtration on the structure sheaf of the parameter manifold, which can be chosen to have dimension 1. This suggests if one can formulate a version of this duality over p-adic rings or more generally a P-ring. These rings are characterized as universal rings of Witt vectors over arbitrary rings. Briefly a Witt ring over a ring $A$ or the ring of Witt vectors of $A$, is a copy of the infinite product $A^{\infty}$, but with specific sum and products given in each component by polynomials is $char =p$. Such a ring has characteristic $0$. The definition of higher residue as a Serre duality on Brieskorn lattice can be repeated when the base space is defined over a complete local ring of un-equal characteristic, where the residue field is perfect of char $>0$, as:

\[ WK^f( \ , \ ):W\widehat{\mathcal{H}}_{(0)}^f \times W\widehat{\mathcal{H}}_{(0)}^f \to W\widehat{\mathcal{O}}_{S,0}[[t]] \]

\vspace{0.5cm}

\noindent
such that the induced pairing on 

\[ \dfrac{W\mathcal{H}_{(0)}^f}{t.\ W\mathcal{H}_{(0)}^f} \otimes \dfrac{W\mathcal{H}_{(0)}^f}{t.\ W\mathcal{H}_{(0)}^f} \to \overline{Frac(W(k))} \]

\vspace{0.5cm}

\noindent
is the classical Grothendieck residue defined similar to the case over $\mathbb{C}$. Here we may identify 

\[ \dfrac{W\mathcal{H}_{(0)}^f}{t.\ W\mathcal{H}_{(0)}^f} \cong A_f \]

\vspace{0.5cm}

\noindent
the Jacobi ring of $f$ over the Witt ring, also defined analogously. In this way the theorem of K. Saito on defining higher residues can be stated for crystals obtained by family of crystalline cohomologies, of the degenerate fibered space in the crystalline topos. One may note that in this case the characteristic is $0$. This helps that every thing goes right with the integral structures. We will also apply the result to the period map construction between crystalline and etale cohomology with coefficient in $\mathbb{C}_p$. 

\[ R\Gamma_{dR}^{alg}(X) \otimes_{\overline{K}}\mathbb{C}_p \rightarrow R\Gamma_{et}(X,\mathbb{Q}_p) \otimes_{\mathbb{Q}_p} \mathbb{C}_p \]

\vspace{0.5cm}

\noindent 
where $K$ is the field of fractions of $W(k)$, and $\bar{K}$ is a fixed algebraic closure. 
This small note is based on a formal construction of Higher residue and a completion process. Despite of technicalities concerning the crystalline topos and cohomologies, the text can be understood by a reader not quite familiar with this concept, that is familier with K. Saito pairing. However it does need a clear understanding of the ring of Witt vectors associated to a commutative ring $A$
and its universal properties. 

\vspace{0.5cm}

\section{Higher residues}

\vspace{0.5cm}

In this section we express the compatibility of residue pairing with the lifting of local systems along a filtration of the base ring. Such a situation is happenning in a crystalline topos, in which one defines crystalline varieties and cohomology. In a formal set up if $(\mathcal{O}_{S,0},\mathfrak{m})$ is a commutative local ring we may form an inverse system rings $\{\mathcal{O}/\mathfrak{m}^l\}$ and simply define the $\mathfrak{m}$-adic completion of $\mathcal{O}_{S,0}$;

\[ \hat{\mathcal{O}}_{S,0}:=\displaystyle{\lim_{\leftarrow}\mathcal{O}_{S,0}}/\mathfrak{m}^l \]

\vspace{0.5cm}

\noindent
This procedure may also be sheafified analogously. An $\hat{\mathcal{O}}_{S,0}$-module $\hat{\mathcal{E}}$ may be defined via the inverse system. If we begin with $(\mathcal{E}, \nabla)$, an integrable connection 
then it can be extended to

\[ \hat{\nabla}:\hat{\mathcal{E}} \to \hat{\mathcal{E}} \times \Omega_S^1 \]

\vspace{0.5cm}

\noindent
Most of the algebraic structures already defined over $\mathcal{O}_{S,0}$ may be extended over the completion $\hat{\mathcal{O}}_{S,0}$ via the natural map 

\[ \mathcal{O}_{S,0} \to \hat{\mathcal{O}}_{S,0} \]

\vspace{0.5cm}

\noindent
Suppose for the moment $\dim(S)=1$ and it is defined over a field $k$ of $\text{char}k=0$. Suppose $f:X \to S$ is a family which is generically proper and flat. We are going to apply this to the construction of higher residues and primitive forms originally belonged to K. Saito, \cite{SA1}, \cite{LLS}. This can be done by a quasi-identification of the space of covariant $PV(X)$ of smooth polyvector-fields, with the space of cotravariant $A(X)=\sum A^{i,j}(X)$ of smooth complex differential forms on $X$, \cite{LLS}. This gives a quasi-isomorphism 

\[ (PV(X)((t)), Q_f=\bar{\partial}_f+t\partial) \leftrightarrows (A(X)((t)), d+t^{-1}df \wedge \bullet) \]

\vspace{0.5cm}

\noindent
where $Q_f$ is corresponding coboundary to $d+t^{-1}df \wedge \bullet$ via a specific  isomorphism $PV(X)((t)) \cong (A(X)((t))$. The natural embedding 

\[ \imath:(PV_c(X)[[t]], Q_f) \hookrightarrow (PV(X)[[t]], Q_f) \]

\vspace{0.5cm}

\noindent
defines a quasi-isomorphism, and if we set

\[ \mathcal{H}_{(0)}^f:=H^*(PV(X)[[t]], Q_f) \]

\vspace{0.5cm}

\noindent
then the trace map

\[ Tr:PV_c(X) \to k \]

\vspace{0.5cm}

\noindent
provides a $k[[t]]$-homomorphism $\widehat{Res}^f$ as follows.

\vspace{0.5cm}

\begin{center}

$\mathcal{H}_{(0)}^f \longrightarrow \mathcal{O}_{S,0}[[t]],  \qquad \widehat{Res}^f=\sum_k \widehat{Res}_k^f(\bullet)t^k $

\end{center}

\vspace{0.5cm}

\noindent
with $\widehat{Res}_k^f$ the higher residues. Similarly, we obtain the higher residue pairing 

\[ K^f( \ , \ ):\mathcal{H}_{(0)}^f \times \mathcal{H}_{(0)}^f \to \mathcal{O}_{S,0}[[t]], \qquad  K^f( \ , 1 ):=\widehat{Res}^f \]

\vspace{0.5cm}

\noindent
$\mathcal{H}_{(0)}^f$ will also inherits a connection as

\[ \nabla:\mathcal{H}_{(0)}^f \to t^{-1}.\mathcal{H}_{(0)}^f \otimes \Omega_{S,0}^1 \].

\vspace{0.5cm}

\noindent
All of these constructions can be extended over $\hat{\mathcal{O}}_{S,0}$ by the flat base change $\mathcal{O}_{S,0} \to \hat{\mathcal{O}}_{S,0}$, \cite{SA1}, \cite{LLS}. In other words if we have an inverse system of vector bundles with connections

\[ E_N =E \otimes R_N,\qquad  \nabla: \Gamma(S,E_N) \to \Omega_S^1 \otimes \Gamma(S,E_N)  \] 

\vspace{0.5cm}

\noindent
and we can define 

\[ K_N^f : (\mathcal{H}_{(0)}^f \otimes R_N) \otimes (\mathcal{H}_{(0)}^f \otimes R_N) \to {{\mathbb R}}_N[[t]] \]

\vspace{0.5cm}

\noindent
and then obtain 

\[ \widehat{K}^f(s_1,s_2):=\lim_{\leftarrow}K_N^f (s_{1,N},s_{2,N})\]

\vspace{0.5cm}

\noindent
One may replace the variety $S$ by any variety defined over a field of $char=0$. As we work over complete local rings the major examples are $\mathbb{C}$ and $\mathbb{C}_p$ the algebraic closure of the field Witt ring $\mathbb{Q}_p$, which is the concept of discussion in the next sections.

\vspace{0.5cm}

\section{Ring of Witt vectors}

\vspace{0.5cm}

\noindent
We provide basic definitions and properties of Witt rings following J. P. Serre, in \cite{SE} and S. Bloch cf. \cite{BL}. Let $A$ be a complete discrete valuation ring, with residue field $k$. Suppose that $A$ and $k$ have the same characteristic, and $k$ is perfect. Then $A$ is isomorphic to $k[[T]]$. This fact can be proved by showing that $A$ contains a system of representatives of the residue field, which is a field. If $S$ is such a system of representatives, then any $a \in A$ can be written as a convergent series 

\[ a=\displaystyle{\sum_{n=0}^{\infty} s_n. \pi^n , \qquad s_n \in S} \]

\vspace{0.5cm}

\noindent
Now suppose that $A$ and $k$ have different characteristic. This is possible only when $\text{char}(A)=0$ and $\text{char}(k)=p>0$. Then $v(p)=e \geq 0$ is called the absolute ramification index of $A$. The injection $\mathbb{Z} \hookrightarrow A$ extends by continuity to an injection of the ring $\mathbb{Z}_p$ of p-adic integers into $A$. When the residue field is a finite field with $q=p^f$ element, then $A$ is a free $\mathbb{Z}_p$ module of rank $n=ef$. \textit{For any perfect field $k$ of characteristic $p$, there exists a complete discrete valuation ring and only one up to a unique isomorphism which is absolutely un-ramified and has $k$ as its residue field. It is denoted $W(k)$}. In the ramified case, one has: 

\vspace{0.5cm}

\begin{theorem}\cite{SE}
Let $A$ be a complete discrete valuation ring of characteristic unequal to that of its residue field $k$. Let $e$ be its absolute ramification index. Then there exists a unique homomorphism of $W(k)$ into $A$ which makes commutative the diagram:

\vspace{0.5cm}

\begin{center}
$\begin{array}[c]{ccc}
W(k)&\rightarrow&A\\
&\searrow&\downarrow\\
&&k
\end{array}$
\end{center}

\vspace{0.5cm}

\noindent
This homomorphism is injective and $A$ is a free $W(k)$-module of rank equal to $e$. 
\end{theorem}

\vspace{0.5cm}

\begin{example}\cite{SE} As an example let $X_{\alpha}$ be a family of indeterminate, and let $S$ be the ring of $p^{-\infty}$-polynomials in the $X_{\alpha}$'s, with integer coefficients. That is

\[ S=\displaystyle{\bigcup_{\alpha, n} \mathbb{Z}[X_{\alpha}^{p^{-n}}]} \]

\vspace{0.5cm}

\noindent
Then provide $S$ with the $p$-adic filtration $\{ p^nS\}_{n\geq0}$ and complete it. One obtains 

\[ \hat{S}=\hat{\mathbb{Z}}[X_{\alpha}^{p^{-\infty}}] \]

\vspace{0.5cm}

\noindent
The residue ring $\hat{S}/p.\hat{S}$ is the ring $F_p[X_{\alpha}^{p^{-\infty}}]$. It is perfect of $\text{char}=p$.
 
\end{example}

\vspace{0.5cm}

It follows almost evident from the universal property 2.1 stated above that: \textit{For every perfect ring $k$ of characteristic $p$, there exists a unique $p$-ring $W(k)$ with residue ring $k$}. The uniqueness follows easily from the privious notes. If $k$ has the form $F_p[X_{\alpha}^{p^{-\infty}}]$ one takes $W(k)=\hat{\mathbb{Z}}[X_{\alpha}^{p^{-\infty}}]$. The general case follows from the fact that every perfect ring is a quotient of the rings in the former case. Thus $W(k)$ is a functor of $k$, and $Hom(k,k^{\prime})=Hom(W(k),W(k^{\prime}))$.

\vspace{0.5cm}

\noindent
Then for an arbitrary commutative ring $A$, The elements of $A^{\mathbb{N}}$ is equipped with the following addition and multiplication laws given by polynomials $S_n$ and $P_n$,

\[ (a_n)+(b_n)= (S_n((a_n),(b_n))) , \qquad (a_n) \times (b_n)= (P_n((a_n),(b_n))) \]

\vspace{0.5cm}

\noindent
and these operations make $A^{\mathbb{N}}$ into a commutative unitary ring, called the ring of Witt vectors with coefficients in $A$. There is a canonical map 

\[ W_*:W(A) \to A^{\mathbb{N}}, \qquad (a_n) \mapsto (W_n((a_n)) \]

\vspace{0.5cm}

\noindent
There is also a natural shift map namely

\vspace{0.5cm}

\begin{center}
$V(a_0,...)=(0,a_0,...)$, 
\end{center} 

\vspace{0.5cm}

\noindent
which is transformed to 

\vspace{0.5cm}

\begin{center}
$(w_0,...) \mapsto (0,pw_0,...)$ 
\end{center}

\vspace{0.5cm}

\noindent
by the homomorphism $W_*$. Another natural map 

\vspace{0.5cm}

\begin{center}
$r:A \to W(A), \qquad r(x)=(x,0,...)$ \\ 
\end{center}

\vspace{0.5cm}

\noindent
which satisfies $r(xy)=r(x)r(y)$, and is transformed to 

\vspace{0.5cm}

\begin{center}
$x \mapsto (x,x^p,..., x^{p^n},...)$ 
\end{center}

\vspace{0.5cm}

\noindent
under $W_*$. Another structural map is the action of Frobenius 

\[ F:W(k) \to W(k) , \qquad F((a_n)):=((a_n^p)) \]

\vspace{0.5cm}

\noindent
which is a ring homomorphism satisfying $VF=p=FV$. In Grothendieck language of schemes $S_n=\text{Spec}(W_n(A))$ is affine and of finite type over $\text{Spec}(Z)$. 

\vspace{0.5cm}

\noindent
There is an alternative definition for the ring of Witt vectors of a commutative ring $R$  and a natural filtration on it, \cite{BL}, as:

\[ W(R)=(1+tR[[t]])^* , \qquad Filt^nW(R)=(1+t^{n+1}R[[t]])^* , \qquad W_n(R)=W(R)/Filt^n \]

\vspace{0.5cm}

\noindent
Any element $P(t) \in 1+tR[[t]]$ can be written as 

\[ P(t)=\prod_{n \geq 1}(1-a_nt^n)^{-1} ,\qquad P(t) \leftrightarrows (a_1,...,a_n,...)=\omega(P), \qquad W_n \leftrightarrows (a_1,...,a_n) \]

\vspace{0.5cm}

\noindent
Then the product structure is given by 

\[ \omega(1-at^n)^{-1} \omega(1-bt^n)^{-1} =\omega(1-a^{n/r}b^{m/r}t^{mn/r})^{-r}, \qquad r=g.c.d(m,n) \]

\vspace{0.5cm}

\noindent
We will obtain a set of maps 

\[ V_n\omega(P)=\omega(P(t^n)), \qquad V_n\omega(1-at^m)=\omega(1-aT^{nm})^{-1},\]

\[ F_n\omega(P)=\sum_{\zeta^n=1}\omega(P(\zeta t^{1/n})), \qquad F_n\omega(1-at^m)=\omega^(1-a^{n/r}T^{m/r})^{-r}
, \ r=g.c.d(m,n) \]

\[ V_nFilt^m \subset Filt^{mn+n-1}, \qquad F_n Filt^{mn} \subset Filt^m \]

\vspace{0.5cm}

\noindent
If $\mu :\mathbb{N} \to \{0,1,-1\}$ be the Mubius function then $\pi=\sum_{n \in I(p)}\dfrac{\mu(n)}{n} V_nF_n$, where $I(p)$ is the set of positive integers not divisible by $p$, is a ring homomorphism. It commutes with the ghost map 

\[ W(R) \to \prod_{\infty} R, \qquad W(R) \cong (1+tR[[t]])^* \stackrel{t\frac{d}{dt}\log}{\longrightarrow} tR[[t]]^+ \cong \prod_{\infty} R \]

\vspace{0.5cm}

\noindent
which is a ring homomorphism and if $R$ is torsion free is injective. Moreover if 

\[ \rho:\prod R \to \prod R, \qquad (a_1,a_2,...) \mapsto (a_1,0,...,0,a_p,0,...) \]

\vspace{0.5cm}

\noindent
then

\begin{equation}
\begin{CD}
W(R)  @>gh>>  \prod R\\
@V{\pi}VV        @VV{\rho}V\\
W(R)  @>>gh> \prod R
\end{CD} , \qquad  
\begin{CD}
W(R)  @>gh>>  \prod R\\
@V{V_n}VV        @VV{\mathcal{V}_n}V\\
W(R)  @>>gh> \prod R
\end{CD} , \qquad  
\begin{CD}
W(R)  @>gh>>  \prod R\\
@V{F_n}VV        @VV{\mathcal{F}_n}V\\
W(R)  @>>gh> \prod R
\end{CD}
\end{equation}

\vspace{0.5cm}

\noindent
where $\mathcal{V}_n(a_1,a_2,...)=(0,...,0,na_1,0,...,0,na_2,...), \ \mathcal{F}_n(a_1,a_2,...)=(a_n,a_{2n},...)$.

\vspace{0.5cm}

\begin{proposition}\cite{LZ}
Let $R$ be a complete local ring whose residue class field is a perfect field of characteristic $p$. Denote by $\mathfrak{m}$ the maximal ideal of $R$. Then $W_n(R)$ is for each number $n$, is a noetherian complete local ring, whose maximal ideal $\mathfrak{n}$ is the kernel of the homomorphism $W_n(R) \to R \to R/\mathfrak{m}$. The $\mathfrak{n}$-adic topology of $W_n(R)$ coincides with the topology defined by the filtration by the ideals $W_n(\mathfrak{m}^s)$.

\end{proposition}

\vspace{0.5cm}

\noindent
The proposition states a compatibility for passing to the filtrations. The above two method of presenting ring of Witt vectors are equivalent in the way that the corresponding maps $F,V$ and $r$ carry over respectively. This can be checked out by the relations involved, \cite{BL}. 

\vspace{0.5cm}
 
\section{De Rham-Witt complex and Crystalline cohomology}

\vspace{0.5cm}

Let $X$ be a smooth and proper scheme over a perfect field $k$ of $\text{char}>0$. Assume $X$ lifts to a scheme $\tilde{X}$ over $W(K)$.  It was discovered by Grothendieck, that the hyper-cohomology of the de Rham complex $\Omega_{\tilde{X}/W(k)}$ does not depend on the lifting, but only on $X$. The crystalline cohomology defines this hyper-cohomology in terms of $X$. It will also make sense without existence of any liftings $\tilde{X}$. Berthelot proved that this cohomology enjoys all good properties, i.e it is a Weil cohomology on the category of proper smooth schemes over $X$, \cite{LZ}, \cite{BEO}, \cite{BL}. For $n \geq 1$ define $S_n=\text{Spec}(W_n(k))$. In addition let $W_n(\mathcal{O}_X)$ be the Zariski sheaf of rings obtained by taking the $p$-Witt vectors of length $n$ on $\mathcal{O}_X$. Then $X_n=(X,W_n(\mathcal{O}_X))$ is a scheme of finite type over $S_n$. The Crystalline topos $(X/S_n)_{\text{cris}}$ is the category of sheaves over the site $\text{Cris}(X/S_n)$. 

\vspace{0.5cm}

\noindent
Let $A$ be a commutative ring and $I$ an ideal. By divided powers on $I$ we mean a collection of maps $\gamma^{(i)}:I \to A, \ i \geq 0$. 

\vspace{0.5cm}

\begin{itemize}
\item for all $x \in I, \ \gamma^{(0)}(x)=1,  \ \gamma^{(1)}(x)=x, \ \gamma^{(i)}(x) \in I $

\vspace{0.4cm}

\item $\ \gamma^{(k)}(x+y)=\sum_{i+j=k}\ \gamma^{(i)}(x) \gamma^{(j)}(x) $

\vspace{0.3cm}

\item For $\lambda \in A, \ \gamma^{(k)}(\lambda x)= \lambda^k \gamma^{(k)}(x) $

\vspace{0.3cm}

\item $ \gamma^{(i)}(x)=\dfrac{(i+j)!}{i!j!} \gamma^{(j)}(x) \gamma^{(i+j)}(x)$

\vspace{0.2cm}

\item $ \gamma^{(p)}(\gamma^{(q)}(x))= \dfrac{(pq)!}{p!q!^p}\gamma^{(pq)}(x)$
\end{itemize}

\vspace{0.5cm}

\noindent
Then the divided power $I^{[n]}$ is the ideal generated by 
$ \gamma^{(i_1)}(x) \gamma^{(i_2)}(x)... \gamma^{(i_k)}(x), \sum i_j \geq n$. It is convenient to denote $\gamma^{(n)}(x)$ by $x^{[n]}$. Let $B$ a commutative unitary $A$-algebra, and $\mathfrak{b} \subset B$ an ideal which is equipped with divided powers $\gamma_n :\mathfrak{b} \to \mathfrak{b}, \ n \geq 1$. We set $\gamma_0(b)=1, \ b \in \mathfrak{b}$. A basic example is to take $A$ a ring of $char =p>0$ and $I$ an ideal s.t $I^p=0$. Then $\gamma^{(n)}(x)=\frac{1}{n!}.x^n, \ n <p$ and $\gamma^{(n)}(x)=0, \ n>p$ defines a devided power structure on $I$. The idea of divided powers is a fundamental tool in the theory of PD differential operators and crystalline cohomology, where it is used to overcome the difficulties in char. $p >0$. A divided power structure is a way to to make expressions as $x^n/n!$ meaningful even when it is not actually devide by $n!$. 

\vspace{0.5cm}

\noindent
Let $M$ be a $B$-module. A pd-derivation $\nu:B \to M$ over $A$, is an $A$-linear derivation $\nu$, which satisfies 

\[ \nu (\gamma_n (b))=\gamma_{n-1}(b)\nu(b), \qquad n \geq 1, \ b \in \mathfrak{b} \]

\vspace{0.5cm}

\noindent
This is a formal way to define derivatives or differentials with devided powers. There exists a map 

\vspace{0.5cm}

\begin{center}
$j_n^*:H_{cris}^*(X/S_n) \to H^*(X,\Omega_{X_n/S_n, \gamma}^{\bullet})$ 
\end{center}

\vspace{0.5cm}

\noindent
where the right hand side means the de Rham complex of $X_n/S_n$, with a certain compatibility with divided powers imposed, \cite{BL}.

\vspace{0.5cm} 

\begin{theorem} \cite{BL}
There is a canonical map :

\[ j_n^*:H_{cris}(X/W_n) \to H^*(X,\Omega_{X_n/S_n,\gamma}) \]

\vspace{0.5cm}

\noindent
where $H^*(X,\Omega_{X_n/S_n,\gamma})$ is the de Rham complex with compatibility relations like 

\[ d(\gamma^{(m)}(x)dy)=\gamma^{(m-1)}dxdy \]

\vspace{0.5cm}

\noindent
imposed. $j_0: H_{cris}(X/k) \to H^*(X,\Omega_X^{\bullet}) $ is the standard identification, and the following diagram commutes for $m \geq n$.

\begin{equation}
\begin{CD}
H_{cris}(X/W_m) @>{j_m^*}>> H^*(X,\Omega_{X_m/S_m,\gamma}) \\
@VVV        @VVV\\
H_{cris}(X/W_n) @>>{j_n^*}> H^*(X,\Omega_{X_n/S_n,\gamma}) 
\end{CD} , \qquad  
\end{equation}

\end{theorem}

\vspace{0.5cm}

\noindent
We have an inverse system of pd-differential graded $W_m(R)$-algebras

\[ W_n\Omega_{R} \to W_{n-1}\Omega_R \to ... \to \Omega_R , \qquad W\Omega_{R}= \lim_{\leftarrow}W_n\Omega_{R} \]

\vspace{0.5cm}

\noindent
called the de Rham-Witt complex; introduced by Ellusie. It is a complex of sheaves of $W(k)$-modules on $X$, whose hyper-cohomology is crystalline cohomology, \cite{LZ}. The de Rham-Witt complex of a scheme $X$ over $R$, is a projective system indexed by $\mathbb{N}$ of complexes $W_n\Omega_{X/R}$ of $W_n(R)$-algebras on $X$. If $p$ is nilpotent in $R$, and $X$ is smooth over $Spec(R)$, the hyper-cohomology of $W_n\Omega_{X/R}$ is isomorphic to the crystalline cohomology

\vspace{0.5cm}
 
\begin{center}
$H_{cris}^*(X/W_n(R),\mathcal{O}_{X/W_n(R)}^{cris})$ 
\end{center}

\vspace{0.5cm}

\noindent
of the crystalline structure sheaf. 

\vspace{0.5cm}

\begin{theorem} (Comparison Theorem) \cite{LZ}
There is a canonical isomorphism 

\begin{equation} 
H^i(X/W_n(R))_{\text{crys}}, \mathcal{O}_{X/W_n(R)}) \cong H^i(X,W_n\Omega_{X/R}) 
\end{equation}

\end{theorem}
 
\vspace{0.5cm} 

\noindent
The notion of lifting differential forms in the above pattern appears similarly when considering vector bundles.
More systematically it is known as ''Stratification'' on a vector bundle $E$. Roughly speaking a stratification is given by a set modules $\mathcal{P}_X^n$ with a set of compatibility homomorphisms $\epsilon_n:\mathcal{P}_X^n \to \mathcal{P}_X^m$ for $m \leq n$ where $\epsilon_0=id$. The set of these data must satisfy some specific co-cycle conditions, cf. \cite{BEO}. Then, one knows how to pass from the category of vector bundles to the category of vector bundles with stratification in a functorial way.
When such vector bundles are equipped with integrable connections are called crystals, cf. \cite{BEO} Chap. 5.   

\vspace{0.5cm}

\noindent
Let $E$ be a crystal on $Crys(X/W_n(R))$. By this we mean $E$ is a vector bundle with a flat connection, or equivalently a $D$-module. We consider an affine open set $U=Spec(S) \subset X$ and a pd-thickening
$A \to S$ relative to $W_n(R)$, then we have the pd-differential de Rham complex with coefficients in $E$,

\[ (E_A \otimes_A \Omega_{A/W_n(R)},\nabla) \]

\vspace{0.5cm}

\noindent
Setting $E_n=E_{W_n(\mathcal{O}_X)}$, we define de Rham-Witt complex with coefficients in $E$:

\[ (E_n \otimes_{W_n(\mathcal{O}_X)}W_n\Omega_{X/R},\nabla) \]

\vspace{0.5cm}

\noindent
Again the hyper-cohomology of this complex is the crystalline cohomology of $E$, if $E$ is flat and $X$ smooth over $R$. This also holds at the level of Cech complexes, \cite{LZ}. Then we have 

\[ H^q(X, E \otimes W\Omega_X) \cong \lim_{\leftarrow}H^q(X, E_n \otimes W_n\Omega_X) \]

\vspace{0.5cm}

The reader should consider the afore-mentioned isomorphism as different approaches to define a type of de Rham cohomology in positive characteristic, and the compatibility between these certain differently defined cohomologies. We refer to Chap. 1 of \cite{BEO} for more detailed historical remarks. 

\vspace{0.5cm}

\noindent
Local systems of crystalline cohomologies of varieties can be considered similar to that over $\mathbb{C}$. Thus we will consider similar correspondence between local systems 
in characteristic $0$ in this case and the flat connections as crystals. To consider local systems satisfying Hodge structure one uses the period isomorphism 

\[ R\Gamma_{dR}^{alg}(X) \otimes_{\overline{K}}\mathbb{C}_p \rightarrow R\Gamma_{et}(X,\mathbb{Q}_p) \otimes_{\mathbb{Q}_p} \mathbb{C}_p \]

\vspace{0.5cm}

\noindent 
where $K$ is the field of fractions of $W(k)$, and $\bar{K}$ is a fixed algebraic closure. The period isomorphism mainly asserts that the crystalline and etale cohomology are equivalent after a suitable base change. Therefore one defines Hodge structures of pure or mixed weights on etale cohomology. Hodge structures both in the pure and mixed case can be defined over the etale cite by the Weil conjectures. There the weights are defined via the eigen-values of the Frobenius. 

\vspace{0.5cm}

 
\section{Higher residue Pairing over P-rings}

\vspace{0.5cm}

We apply the procedure of the section to family of schemes over the Witt rings. In this way the ring $\mathcal{O}_S=\bigcup_n \mathbb{Z}[X_{\alpha}^{p^{-n}}]$ filtered by $\{p^n\mathcal{O}_S\}_{n \geq 0}$. If we want to repeat the construction in the first section to schemes over $W_n(k)$, then the de Rham complex would be replaced by the de Rham-Witt complexes, and the corresponding formal poly-vector field complex as its co-variant mirror. Because the characteristic is $0$, the isomorphism proceeds word by word to in this case and we still get a mirror type identification between these two formal complexes. Then analogous isomorphisms 

\vspace{0.5cm}

\begin{center}
$(W_NPV_{S}(X)((t)), Q_f=\bar{\partial}_f+t\partial) \leftrightarrows (W_NA_{S}(X)((t)), d+t^{-1}df \wedge \bullet)$\\[0.6cm]
$\imath:(W_NPV_{S,c}(X)[[t]], Q_f) \hookrightarrow (W_NPV_{S}(X)[[t]], Q_f)$ 
\end{center}

\vspace{0.5cm}

\noindent
still hold for $W_n$s and also in the limit for $W(k)$, and we can define,

\[ W_N\mathcal{H}_{(0),N}^f:=H^*(W_NPV(X)[[t]], Q_f) \]

\vspace{0.5cm}

\noindent
By the same method as before we obtain: 

\[ W_N\widehat{Res}_N^f=\sum_k W_N\widehat{Res}_{k,N}^f(\bullet)t^k \]

\vspace{0.5cm}

\noindent
with $\widehat{Res}_{k,N}^f$ the higher residues. Similarly, we obtain the higher residue pairing 

\[ W_NK_N^f( \ , \ ):W_N\mathcal{H}_{(0),N}^f \times W_N\mathcal{H}_{(0),N}^f \to \mathcal{O}_{S,0}[[t]], \qquad  W_NK_N^f( \ , 1 ):=W\widehat{Res}_N^f \]

\vspace{0.5cm}

\noindent
Now by applying the completion process explained in section (1) we obtain 

\[ WK^f( \ , \ ):W\widehat{\mathcal{H}}_{(0)}^f \times W\widehat{\mathcal{H}}_{(0)}^f \to W\widehat{\mathcal{O}}_{S,0}[[t]] \]

\vspace{0.5cm}

\noindent
It means that if we consider the inverse system of crystals $E_{n,N}=(E \otimes W_n \Omega_{X/R}, \nabla ) \otimes \mathcal{O}_S/p^N\mathcal{O}_S$, and repeat the process in section (1) word by word to obtain the following generalization of K. Saito theorem on crystalline site, \cite{LLS}, \cite{SA1}. 

\vspace{0.5cm}

\begin{theorem}(Higher residue pairing on crystalline site) There exists a $K=\text{Frac}(W(k))$-sesquilinear form 

\[ WK^f( \ , \ ):W\widehat{\mathcal{H}}_{(0)}^f \times W\widehat{\mathcal{H}}_{(0)}^f \to W\widehat{\mathcal{O}}_{S,0}[[t]] \]

\vspace{0.5cm}

\noindent
Let $s_1,s_2$ be local sections of $W\mathcal{H}_{(0)}^f$, then;

\vspace{0.5cm}

\begin{itemize}
\item $WK^f(s_1,s_2)=\overline{WK^f(s_2,s_1)}$.

\vspace{0.2cm}

\item $WK^f(v(t)s_1,s_2)=WK^f(s_1,v(-t)s_2)=v(t)WK^f(s_1,s_2)$, $v(t) \in \mathcal{O}_S[[t]]$.

\vspace{0.2cm}

\item $\partial_V.WK^f(s_1,s_2)=WK^f(\partial_Vs_1,s_2)+WK^f(s_1,\partial_Vs_2)$, for any local section of $T_S$.

\vspace{0.2cm}

\item $(t\partial_t+n)WK^f(s_1,s_2)=WK^f(t\partial_t.s_2,s_1)+WK^f(s_1,t \partial_t.s_2)$

\vspace{0.2cm}

\item The induced pairing on 
\[ W\mathcal{H}_{(0)}^f/t.W\mathcal{H}_{(0)}^f \otimes W\mathcal{H}_{(0)}^f/t.W\mathcal{H}_{(0)}^f \to \bar{K} \]

\vspace{0.3cm}

\noindent
is the classical Grothendieck residue.

\end{itemize}
\end{theorem}

\vspace{0.5cm}

\noindent
The conjugation is formally done by $\overline{g(t) \otimes \eta}=g(-t). \eta$, for $g \in W(\mathcal{O}_S),\ \eta \in WA_S(X)$. 

\vspace{0.5cm}

\noindent
There are essentially two type of proof for higher residue pairing over $\mathbb{C}$. The one cited in \cite{SA1} is mainly a comparison of two construction. One an application of local (Serre) duality theorem to Brieskorn lattices and their duals. This amounts to define the Brieskorn modules $(\mathcal{H}_f^{(-k)}, \nabla:\mathcal{H}^{(-k-1)} \to \mathcal{H}_f^{(-k)})$ together with their duals $(\check{\mathcal{H}}^{(k)}, \check{\nabla}: \check{\mathcal{H}}^{(k)} \to  \check{\mathcal{H}}^{(k+1)})$ which satisfy a local duality as 

\vspace{0.5cm}

\begin{center}
$\mathcal{H}^{(k)} \times \check{\mathcal{H}}^{(k)} \to \mathcal{O}_S$
\end{center}

\vspace{0.5cm}

\noindent
Then this duality is related to the twisted de Rham complex by 

\vspace{0.5cm}

\begin{center}
$\hat{\alpha}_k: \widehat{\mathcal{H}^{(-k)}} \cong R^{n+1}f_*(F^{-k}\Omega, \hat{d}), \qquad k \geq 1$.
\end{center}

\vspace{0.5cm}

\noindent
where $F$ is the Hodge filtration, \cite{S1}. Specifically

\begin{equation}
\hat{\alpha}: \widehat{\mathcal{H}^{(0)}} \cong R^{n+1}f_*(F^{0}\Omega, \hat{d}), \qquad k \geq 1.
\end{equation}

\vspace{0.5cm}

\noindent
The second method is a duality isomorphism between the twisted de Rham complex and the twisted differential complex of poly-vector fields as in section 1. Both of these constructions are algebraic and can be stated similarly over any field of characteristic $0$ and can be applied over Witt ring construction. Thus a proof of theorem 4.1 follows from the formality (algebraicity) of the construction in \cite{SA1} in characteristic $0$ and equivalent with the one mentioned section 1. The reader should convince himself that the method explained in Section 1 can be applied to prove the Higher residue pairing using a basic algebra.

\vspace{0.5cm}

\begin{corollary}
The form $K^f$ of higher residue can be defined for family of schemes on $Spec(\mathbb{Q}_p[[t]])$ as a pairing

\[ K_p^f( \ , \ ):\widehat{\mathcal{H}}_{(0),p}^f \times \widehat{\mathcal{H}}_{(0),p}^f \to \overline{\mathbb{Q}_{p}}[[t]] \]

\vspace{0.5cm}

\noindent
with the same properties as in 4.1
\end{corollary}

\vspace{0.5cm}

The corollary is just the special case $W(\mathbb{F}_p)=\mathbb{Z}_p$.

\vspace{0.5cm}

\noindent
The notion of opposite filtration and formal primitive form may also be generalized to this case easily. That is 
if 

\[ W_N\mathcal{H}_{N}^f:=H^*(W_NPV(X)((t)), Q_f) \]

\vspace{0.5cm}

\noindent
then we may find a subspace $W_N\mathcal{L}$ such that 

\[ W_N\mathcal{H}_{N}^f :=  W_N\mathcal{H}_{(0),N}^f \oplus W_N\mathcal{L} , \qquad t^{-1}W_N\mathcal{L} \subset \mathcal{L} , \qquad Jac(f) \cong  W_N\mathcal{H}_{N}^f \oplus t.W_N\mathcal{L} \]

\vspace{0.5cm}

\noindent
Then taking the inverse limits would obtain the desired opposite filtration. Similar to case of the field $\mathbb{C}$, a Hodge filtration

\vspace{0.5cm}
 
\begin{center}
$t^k. W\mathcal{H}_{(0)}^f:=W\mathcal{H}_{(-k)}^f \supset  W\mathcal{H}_{(-k-1)}^f$ 
\end{center}

\vspace{0.5cm}

\noindent
can also been defined. The natural map

\[ \dfrac{f_*W\Omega^{n+1}}{df \wedge dW\Omega^{n-1}} \to  W\mathcal{H}_{(0)}^f, \qquad \xi \to \dfrac{\xi}{dx_0 \wedge ...\wedge dx_n} \]

\vspace{0.5cm}

\noindent
makes the classical filtration on Brieskorn lattice correspond to the filtration we mentioned above. One may also define the contravariant Higher residue for the left hand side, as historically defined by K. Saito.

\vspace{0.5cm}

\noindent
A good section is an equivalent notion to opposite filtration under the identification, 

\[ [\nu :Jac(f) \to  W\mathcal{H}_{(0)}^f] \mapsto \mathcal{L}:=t^{-1}\nu(Jac(f))[t^{-1}] \]

\vspace{0.5cm}

\noindent
Also, as in complex case a primitive form $W\zeta_0$ can be defined as an element of $ W\mathcal{H}_{(0)}^f$ which its reduction to $ W\mathcal{H}_{(0)}^f/t. W\mathcal{H}_{(0)}^f$ generates $Jac(f)$ and is homogeneous, i.e. $t\partial_t \zeta_0-r \zeta_0 \in \mathcal{L}$, for some $r \in \mathbb{C}$. If $F$ is a universal unfolding of $f$ with critical set $C(F)$. The Kodaira-Spencer map is

\[ KS: T_S \to p_*\mathcal{O}_{C(F)} , \qquad KS(\xi):=\tilde{\xi} .F|_{C(F)} \]

\vspace{0.5cm}

\noindent
where $\tilde{\xi}$ is a lifting of $\xi$, under $p:\mathbb{C}^{n+1+\mu} \to S$. Then, the Euler vector field is by definition
 
\[
E:=KS^{-1}(F) \in \Gamma(S,T_S) \]

\vspace{0.5cm}

\section{Relation with etale cohomology}

\vspace{0.5cm}

Classically over the field $\mathbb{C}$, the period map provides a commutative triangle in the following form between the de Rham and singular complexes

\vspace{0.5cm}

\begin{center}
$\begin{array}[c]{ccc}
R\Gamma_{dR}^{alg}(X)&\rightarrow& R\Gamma(X_{top},\mathbb{C})\\
&\searrow&\downarrow\\
&&R\Gamma_{dR}^{an}
\end{array}$
\end{center}

\vspace{0.5cm}

\noindent
where the vertical arrow is a quasi-isomorphism via Poincare lemma. The horizonlal arrow is a filtered quasi isomorphism namely Period isomorphism which more or less is given by the integration of complex differential forms along integral homology classes. This commutative triangle is fundamental in theory of motives and their periods. Classically periods come out from any comparison of this type, and a motive is the collection of the whole data of the cohomologies. Such a diagram can be also defined in the $p$-adic setting introducing similar concepts in that case. Then, the period isomorphism is the natural filtered quasi-isomorphism,

\[ R\Gamma_{dR}^{alg}(X) \otimes_{\overline{K}}\mathcal{B}_{dr} \rightarrow R\Gamma_{et}(X,\mathbb{Q}_p) \otimes_{\mathbb{Q}_p} \mathcal{B}_{dR} \] 

\vspace{0.5cm}

\noindent
where $K$ is the field of fractions of $W(k)$ and $\mathcal{B}_{dR}$ is a discrete valuation field whose valuation ring is called Fontaine ring and its residue field is $\mathbb{C}_p$. It descends to ,

\[ R\Gamma_{dR}^{alg}(X) \otimes_{\overline{K}}\mathbb{C}_p \rightarrow R\Gamma_{et}(X,\mathbb{Q}_p) \otimes_{\mathbb{Q}_p} \mathbb{C}_p \]

\vspace{0.5cm}

\noindent
The comparison theorem indicates that for any DVR namely $V$, there exists a ring $B(V)$, such that for $X$ smooth and proper $V$-scheme, the etale cohomology of the generic fiber $X/W(k)$ is related to the crystalline cohomology of $X/W(k)$ by

\[ H_{et}^*(X \otimes_{W(k)} \bar{K}) \otimes_{\mathbb{Q}_p} B(V) =H_{crys}^*(X/W(k)) \otimes_{W(k)} B(V) \]

\vspace{0.5cm}

\noindent
with $K= \text{quot}\ W(k)$ a totally ramified extension of degree $e$, \cite{FA}. In fact, for $n , \ i\in \mathbb{N}$, the specialization map induces isomorphisms compatible with the action of Galois group $G_K$:

\[ H^i((X \times_{\mathcal{O}_K} \bar{k})_{et},\mathbb{Z}/l^n. \mathbb{Z}) \cong H^i((X \times_{\mathcal{O}_K} \bar{K})_{et},\mathbb{Z}/l^n. \mathbb{Z}) \]

\vspace{0.5cm}

\noindent
The period isomorphism say that crystalline and etale cohomologies in some way determine one another. Using the period isomorphism we can state the Higher residue pairing on the etale site if the ground filed would be $\mathbb{C}_p$. Simply in theorem 4.2 if we tensor every thing with $\mathbb{C}_p$ we obtain the same result on the etale site over $\mathbb{C}_p$.

\vspace{0.5cm}

\begin{theorem} (Higher residue pairing on etale site)
There exists a sesqui-linear form 

\[ K_p^f( \ , \ ):\widehat{\mathcal{H}}_{(0),p}^f \times \widehat{\mathcal{H}}_{(0),p}^f \to \widehat{\mathcal{O}}_{S,0}[[t]] \]

\vspace{0.5cm}

\noindent
Let $s_1,s_2$ be local sections of $\mathcal{H}_{(0,p),p}^{f,\mathbb{C}_p}$.

\vspace{0.5cm}

\begin{itemize}
\item $K_{et}^f(s_1,s_2)=\overline{K_{et}^f(s_2,s_1)}$.

\vspace{0.2cm}

\item $K_{et}^f(v(t)s_1,s_2)=K_{et}^f(s_1,v(-t)s_2)=v(t)K_{et}^f(s_1,s_2)$, $v(t) \in \mathcal{O}_S[[t]]$.

\vspace{0.2cm}

\item $\partial_V.K_{et}^f(s_1,s_2)=K_{et}^f(\partial_Vs_1,s_2)+K_{et}^f(s_1,\partial_Vs_2)$, for any local section of $T_S$.

\vspace{0.2cm}

\item $(t\partial_t+n)K_{et}^f(s_1,s_2)=K_{et}^f(t\partial_t.s_2,s_1)+K_{et}^f(s_1,t \partial_t.s_2)$

\vspace{0.2cm}

\item The induced pairing on 
\[ \mathcal{H}_{(0),p}^f/t.\mathcal{H}_{(0),p}^f \otimes \mathcal{H}_{(0),p}^f/t.\mathcal{H}_{(0),p}^f \to \mathbb{C}_p \]

\vspace{0.5cm}

\noindent
is the classical Grothendieck residue.

\end{itemize}
\end{theorem}

\vspace{0.5cm}

\begin{example}
$\mathbb{P}^1 \setminus \{0,\infty \}$ with $f=z_1+...+z_n+q/z_1...z_n$, the element $1$, is a  primitive forms, and we have similar identities

\[ K^f(1/z,1/z)=0, \qquad K^f(1,q/z)=-1 \]

\vspace{0.5cm}

\noindent
with respect to the choice of volume form $\dfrac{dz_1}{z_1} \wedge ... \wedge \dfrac{dz_n}{z_n}$, cf. \cite{LLS}.
\end{example}

\vspace{0.5cm}

\begin{remark}
Because the characteristic is $0$ all the formal computations carried over in \cite{LLS}, can be done similarly in this new set up. For instance if $f=x^3+y^7$
then 

\[ \zeta_+=1+\dfrac{4}{3.7^2}u_{11}u_{12}^2-.... \]

\vspace{0.5cm}

\noindent
would be a primitive form. Here $u_{ij}$ are the coordinates of unfolding space, cf. \cite{LLS}.
\end{remark}

\vspace{0.5cm}

\section{Action of Frobenius and variation of slope filtration}

\vspace{0.5cm}

Given a functor $F: R$-alg $\to Ab$, then for a commutative ring $R$, one defines the curves of length $n$ on $F$, $C_nF$, by: 

\[ C_nF=\ker(F(R[T]/T^n) \to F(R)) \]

\vspace{0.5cm}

\noindent
In our case, take 

\[ F_j^i=H^i(X_k \times Spec(A), K_j) \]

\vspace{0.5cm}

\noindent
where $K_j$ is the sheaf of $K$-groups of $X$. A crucial observation due to Katz is 

\[ TC_nF_j^i=H^i(X,TC_nK_j), \qquad TC_n=C_{p^n} \] 

\vspace{0.5cm}

\noindent
We will obtain an inverse system of sheaves 

\[ TC_nK_1 \stackrel{\delta}{\rightarrow} TC_nK_2 \to ...\to TC_nK_{d+1} , \qquad C_n^q:=TC_nK_{q+1}  \]

\[ C^q=\{TC_nK_{q+1}\}_n , \qquad \widehat{TC_nF}=\lim_{\leftarrow}TC_nF \] 

\vspace{0.5cm}

\noindent
Evident is $C_n^0=W_n$ the sheaf of Witt vectors, and $C_n^q=0, \ q >\dim X$. $C_n^q$ is a $W_n$-module and $\delta^q:C_n^q \to C_n^{q+1}$ is $W_n$-linear. According to \cite{BL}, 

\[ \ C_n^{\bullet}/p.C_n^{\bullet}=\Omega^{\bullet} \]

\vspace{0.5cm}

\noindent
the usual de Rham complex. The afore-mentioned inverse system has endomorphisms $F_q,V_q$ such that $F_qV_q=V_qF_q=p$ and induce $F,V$ on $C^{\bullet}$ given by $p^qF_q$ and $p^{\dim X-q}V_q$ on $C^q$ respectively, \cite{BL}. We will have the isomorphism 

\[ H^*_{cris}(X/W) \cong \lim_{\leftarrow} H^*(X,C^q) \]

\vspace{0.5cm}

\noindent
and under this isomorphism the action of Frobenius is carried over the map $F$ explained hereabove. Let $Slope^{\bullet}H_{cris}^{\bullet}$ be the filtration induced by the slope spectral sequence 

\[ E_1^{s,l}:H^l(X,C^s) \Longrightarrow H_{cris}^{s+l}(X/W) \]

\vspace{0.5cm}

\noindent
then the Frobenius preserve this filtration and $Slope^qH_{cris}^{\bullet}$ is the greatest $Frob$-stable subspace $H_{cris}^{\bullet}$ which the slopes are $>q$ ,\cite{BL}.

\vspace{0.5cm}

\noindent
We are interested to the deformation of crystalline cohomology groups on the punctured plane $\mathbb{P}^1 \setminus 0, \infty$, i.e is to study the complex $\Omega_{X/W}^{\bullet}[u,u^{-1}]$ in a manner similar to the usual de Rham complex, twisted by the variable $u$. As we explained all of the formal definitions of the twisted de Rham complex may be stated for the de Rham-Witt complex with connection. The reader may easily check with the proof as in \cite{SAB1}, For $(M,F)$ a filtered coherent $D$-module , the Hodge filtration can be defined by 

\[ F_k(M \otimes_{\bar{K}} \bar{K}[u,u^{-1}])=\oplus_jF_{j+k}Mu^{-j}, \qquad F_k=u^kF_0 \] , 

\[ gr_0^F(M \otimes_{\bar{K}} \bar{K}[u,u^{-1}])=F_0/u^{-1}.F_0 =gr^F M \] 

\vspace{0.5cm}

\noindent
and the crystalline cohomologies

\[ H_{cris}^i(X,W\Omega_X \otimes_{\mathcal{O}_X} M[u,u^{-1}], u^{-1} \nabla -df \wedge)) \leftrightarrows H_{cris}^i(X,W\Omega_X \otimes_{\mathcal{O}_X} M[u,u^{-1}],  \nabla - udf \wedge )) \]

\vspace{0.5cm}

\noindent
are finite dimensional and explain mutually the solution local system.

\vspace{0.5cm}

\noindent
Now considering the inverse system of curves over sheaves of $K$-groups  $H^*(X,C^q)$, we may repeat the procedure of defining Saito form for the cohomology cycles in these cohomologies in order to obtain

\[ K_{C^q}^f( \ , \ ):\widehat{\mathcal{H}}_{(0),C^q}^f \times \widehat{\mathcal{H}}_{(0),C^q}^f \to \widehat{\mathcal{O}}_{S,0}[[t]] \]

\vspace{0.5cm}

\noindent
The action of Frobenius on $H_{cris}^*(X/W)$ would carry over $p^qF$ on $H^*(X,C^q)$. 

\vspace{0.5cm}

The form of K. Saito plays a crucial role in the inter-relation between Hodge theory and Mirror symmetry. It provides an interesting background to discuss about different positivity questions in complex algebraic geometry. Normally, the ring $A$ working in algebraic geometry is regular, and according to the classification theorems for complete regular local rings, it would be enough to consider the two cases of power series rings and the ring of Witt vectors of the corresponding residue field of $A$. The above theorem may concern some motivations toward a positivity in algebraic geometry in the latter case.

\vspace{0.5cm}

\vspace{0.5cm}

\vspace{0.5cm}

\begin{thebibliography}{99}

\bibitem[BJ]{BJ} Bhatt B. , De Jong A. , Crystalline cohomology and de Rham cohomology, arxiv preprint.

\bibitem[BEI]{BEI} beilinson A. , P-adic periods and derived de Rham cohomology, Journal of AMS, Vol 25, 715-738, 2012

\bibitem[BEO]{BEO} P. Berthelot, Ogus A. , Notes on crystalline cohomology, Princeton University Press, 1978

\bibitem[BL]{BL}  Bloch S. , Algebraic K-theory and crystalline cohomology, publications mathematique de IHES Volume 47, Issue 1, pp 188-268, 1977

\bibitem[BM]{BM} Breuil C. , Messing W. , Torsion etale and Crystalline cohomologies, Astérisque, Soc. math. Astérisque, Soc. math. 281-327 (1997)

\bibitem[FA]{FA}  Faltings G. , Integral crystalline cohomology over very ramified rings, J. Amer. Math. Soc. 12 (1999), 117-144

\bibitem[F1]{F1}  Fredrich W. , Cycle classes for algebraic de Rham cohomology and Crystalline cohomology, PhD dessertation, Universitat Bonn, 2002

\bibitem[LLS]{LLS} Li C., Li S. , Saito K., Primitive forms via polyvector fields, arxiv:1311.1659v3, 2014. 

\bibitem[LZ]{LZ}  Langer A. , T. Zink, De Rham-Witt cohomology for a proper and smooth morphism, Journal of the Institute of Mathematics of Jussieu, 231 - 314, 2003

\bibitem[KA]{KA}  Katz N. , Crystalline cohomology, Diodonne modules, and jacobi sums, Automorphic Forms, Representation Theory and Arithmetic 
Tata Institute of Fundamental Research Studies in Mathematics 1981,   pp 165-246 

\bibitem[SAB1]{SAB1}  Sabbah C. , on a twisted de Rham complex, Tohoku Math. J. (2) 
Volume 51, Number 1 (1999), 125-140.

\bibitem[SE]{SE}  Serre J. P. , Local fields, Series: Graduate Texts in Mathematics, Vol. 67, Springer Verlag 1959

\bibitem[SA1]{SA1}  Saito K. , Period mapping associated to a primitive form, Publications of Research Inst. Math. Sci. , Kyoto Univ., Vol 19, No 3, 1983

\bibitem[ST]{ST}  Stack Project, Crystalline cohomology, Cotangent complex

\end{thebibliography}

\end{document}


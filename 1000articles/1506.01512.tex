

\nonstopmode
\documentclass[12pt]{amsart}
\usepackage{graphicx}
\usepackage{latexsym}
\usepackage{fancyhdr}
\usepackage{amsmath, amssymb}
\usepackage[all]{xy}
\usepackage{pdflscape}
\usepackage{longtable}
\usepackage{rotating}
\usepackage{verbatim}
\usepackage{hyperref}
\usepackage{subfigure}
\usepackage{mathrsfs}
\usepackage{tensor}
\usepackage{mdwlist}
\usepackage{etoolbox}
\usepackage{mathtools}
\usepackage{todonotes}

\setlength{\oddsidemargin}{0in} \setlength{\evensidemargin}{0in}
\setlength{\textwidth}{6.5in} \setlength{\topmargin}{0in}
\setlength{\textheight}{8.5in} \setlength{\parindent}{1pc}
\setlength{\parskip}{0in} \setlength{\baselineskip}{.21in}

\theoremstyle{plain}
\newtheorem*{lemma*}{Lemma}
\newtheorem{lemma}{Lemma}
\newtheorem*{theorem*}{Theorem}
\newtheorem{theorem}{Theorem}
\newtheorem*{proposition*}{Proposition}
\newtheorem{proposition}{Proposition}
\newtheorem*{corollary*}{Corollary}
\newtheorem{corollary}{Corollary}
\newtheorem*{claim*}{Claim}
\theoremstyle{definition}
\newtheorem*{assumption*}{Assumption}
\newtheorem{assumption}{Assumption}
\newtheorem*{definition*}{Definition}
\newtheorem{definition}{Definition}
\newtheorem*{convention*}{Convention}
\newtheorem*{example*}{Example}
\newtheorem{example}{Example}
\newtheorem{examples}{Examples}
\newtheorem{algorithm}{Algorithm}
\newtheorem*{algorithm*}{Algorithm}
\newtheorem*{remark*}{Remark}
\newtheorem{remark}{Remark}
\newtheorem{remarks}{Remarks}
\newtheorem*{remarks*}{Remarks}

\numberwithin{equation}{section}

\newenvironment{demo}[1]{\par\smallskip\noindent{\bf #1.}}{\par\smallskip}
\newenvironment{private}{\par\smallskip{\makebox[0pt][r]{\bf PRIVATE} $\ulcorner$}}{$\lrcorner$\par\smallskip}

\sloppy

\let\on=\operatorname

\title[Optimal regularity of roots of polynomials]
{Optimal regularity of roots of polynomials}

\author[Adam Parusi\'nski and  Armin Rainer]
{Adam Parusi\'nski and Armin Rainer}

\address {Adam Parusi\'nski: Univ. Nice Sophia Antipolis, CNRS,  LJAD, UMR 7351, 06100 Nice, France}

\email{adam.parusinski@unice.fr}

\address{Armin Rainer: Fakult\"at f\"ur Mathematik, Universit\"at Wien, 
Oskar-Morgenstern-Platz~1, A-1090 Wien, Austria}

\email{armin.rainer@univie.ac.at}

\begin{document}

\begin{abstract}
  We study the regularity of the roots of complex univariate polynomials 
  whose coefficients depend smoothly on parameters. We show that any 
  continuous choice of the roots of a $C^{n-1,1}$-curve of monic polynomials 
  of degree $n$ is locally absolutely continuous with locally $p$-integrable 
  derivatives for every $1 \le p < n/(n-1)$, uniformly with respect to the coefficients. 
  This result is optimal: in general, the derivatives of the roots of a smooth 
  curve of monic polynomials of degree $n$ are not locally $n/(n-1)$-integrable, 
  and the roots may have locally unbounded variation if the coefficients are only 
  of class $C^{n-1,\alpha}$ for $\alpha <1$. 
  We also give a generalization of Ghisi and Gobbino's higher order Glaeser inequalities.
\end{abstract}

\thanks{Supported by the Austrian Science Fund (FWF), Grant P~26735-N25, and by ANR project STAAVF (ANR-2011 BS01 009).}
\keywords{Perturbation of complex polynomials, absolute continuity of roots, optimal regularity of the roots among Sobolev spaces $W^{1,p}$, 
higher order Glaeser inequalities}
\subjclass[2010]{
26C10, 
26A46, 
26D10, 
30C15} 
 
\maketitle

\section{Introduction}

This paper is dedicated to the problem of determining the optimal regularity of the roots of 
univariate polynomials whose coefficients depend smoothly on parameters. There is a vast literature on this 
problem, but most contributions treat special cases: 
\begin{itemize}
    \item the polynomial is assumed to have only real roots 
    (\cite{Bronshtein79}, \cite{Mandai85}, \cite{Wakabayashi86}, \cite{AKLM98}, \cite{KLM04}, 
    \cite{BBCP06}, \cite{BonyColombiniPernazza06}, \cite{Tarama06}, \cite{BonyColombiniPernazza10}, \cite{ColombiniOrruPernazza12}, 
    \cite{ParusinskiRainerHyp}),
    \item only radicals of functions are considered (\cite{Glaeser63R}, \cite{CJS83}, \cite{Tarama00}, \cite{CL03}, \cite{GhisiGobbino13}),    
    \item it is assumed that the roots meet only of finite order, e.g., if the coefficients are real analytic or in some other quasianalytic class,
    (\cite{CC04}, \cite{RainerAC}, \cite{RainerQA}, \cite{RainerOmin}, \cite{RainerFin}),
    \item quadratic and cubic polynomials (\cite{Spagnolo99}), etc. 
\end{itemize}  
In this paper we consider the general case: let $({\alpha},{\beta}) \subseteq {\mathbb{R}}$ be a bounded open interval and let 
  \begin{equation} \label{curveofpolynomials}
    P_a(t)(Z)= P_{a(t)}(Z) = Z^n + \sum_{j=1}^n a_j(t) Z^{n-j}, \quad t \in ({\alpha},{\beta}), 
  \end{equation}
be a monic polynomial whose coefficients are complex valued smooth functions $a_j : ({\alpha},{\beta}) \to {\mathbb{C}}$, $j = 1,\ldots,n$. 
It is not hard to see that $P_a$ always admits a continuous system of roots (e.g.\ \cite[Ch.~II Theorem~5.2]{Kato76}), 
but in general the roots cannot satisfy a local Lipschitz condition. 
For a long time it was unclear whether the roots of $P_a$ admit locally absolutely continuous parameterizations. 
This question was 
affirmatively solved in our recent paper \cite{ParusinskiRainerAC}: we proved that there is a positive integer $k=k(n)$ and a 
rational number $p=p(n)>1$ such that, if the coefficients are of class $C^k$, then each continuous root ${\lambda}$ is locally absolutely continuous 
with derivative ${\lambda}'$ being locally $q$-integrable for each $1 \le q < p$, uniformly with respect to the coefficients. 
See the introduction of \cite{ParusinskiRainerAC} for the history of the problem and for applications. 
The main tool of \cite{ParusinskiRainerAC} was the resolution of singularities. With this technique we could not determine the 
optimal parameters $k$ and $p$.

In the present paper we prove the optimal result by elementary methods.
Our main result is the following theorem.   

\begin{theorem} \label{main}
  Let $({\alpha},{\beta}) \subseteq {\mathbb{R}}$ be a bounded open interval and let $P_a$   
  be a monic polynomial \eqref{curveofpolynomials} with coefficients $a_j \in C^{n-1,1}([{\alpha},{\beta}])$, $j = 1,\ldots,n$. 
  Let ${\lambda} \in C^0(({\alpha},{\beta}))$ be a continuous root of $P_a$ on $({\alpha},{\beta})$.
  Then ${\lambda}$ is absolutely continuous on $({\alpha},{\beta})$ and belongs to the Sobolev space 
  $W^{1,p}(({\alpha},{\beta}))$ for every $1 \le p < n/(n-1)$. The derivative ${\lambda}'$ satisfies  
  \begin{align} \label{bound} 
   \| {\lambda}' \|_{L^p(({\alpha},{\beta}))}  
   &\le C(n,p) \max\{1, ({\beta}-{\alpha})^{1/p}, ({\beta}-{\alpha})^{-1+1/p}\} 
   \max_{1 \le j \le n} \|a_j\|^{1/j}_{C^{n-1,1}([{\alpha},{\beta}])},
  \end{align}
  where the constant $C(n,p)$ depends only on $n$ and $p$.
\end{theorem}

\begin{remark*}
  The factor $({\beta}-{\alpha})^{-1+1/p}$, which makes the bound \eqref{bound} blow up if the length of the interval $({\alpha},{\beta})$ tends to $0$ 
  unless $p=1$, appears only in very special situations. For details and more precise bounds see Section \ref{end}.
\end{remark*}

This result is best possible in the following sense:
\begin{itemize}
  \item In general the roots of a polynomial of degree $n$ cannot lie locally in $W^{1,n/(n-1)}$, even when the coefficients are real analytic. 
  For instance, $Z^n = t$, $t \in {\mathbb{R}}$. 
  \item If the coefficients are just in $C^{n-1,{\delta}}([{\alpha},{\beta}])$ for every ${\delta}<1$, then the roots need not have bounded variation in 
  $({\alpha},{\beta})$. See \cite[Example 4.4]{GhisiGobbino13}. 
\end{itemize}

A \emph{curve} of complex monic polynomials \eqref{curveofpolynomials} admits a continuous choice of its roots. This is 
no longer true if the dimension of the parameter space is at least two. In that case monodromy may 
prevent the existence of continuous roots. We get however the following multiparameter result, where we 
impose the existence of a continuous root. 

\begin{theorem}  
  Let $U \subseteq {\mathbb{R}}^m$ be open and let  
  \[
    P_a(x)(Z)= P_{a(x)}(Z) = Z^n + \sum_{j=1}^n a_j(x) Z^{n-j}, \quad x \in U,   
  \]
  be a monic polynomial with coefficients $a_j \in C^{n-1,1}(U)$, $j = 1,\ldots,n$.   
  Let ${\lambda} \in C^0(V)$ be a root of $P_a$ on a relatively compact open subset $V \Subset U$. 
  Then ${\lambda}$ belongs to the Sobolev space $W^{1,p}(V)$ for every $1 \le p < n/(n-1)$. 
  The distributional gradient $\nabla {\lambda}$ satisfies  
  \begin{equation} \label{multbound} 
   \|\nabla {\lambda} \|_{L^p(V)}  \le  C(n,p,V,W) \max_{1 \le j \le n} \|a_j\|^{1/j}_{C^{n-1,1}(\overline W)},
  \end{equation}
  where $W$ is a neighborhood of $\overline V$ in $U$ and 
  the constant $C(n,p,V,W)$ depends only on $n$, $p$, $V$, and $W$.   
\end{theorem}

\begin{proof}
  Follow the proof of Theorem 4.1 in \cite{ParusinskiRainerAC} and use Theorem \ref{main}.
\end{proof}

The proof of Theorem \ref{main} makes essential use of the recent result of Ghisi and Gobbino \cite{GhisiGobbino13} 
who found the optimal regularity of radicals of functions (we will need a version for complex valued functions; see Section \ref{radicals}):

\begin{theorem}[Ghisi and Gobbino \cite{GhisiGobbino13}] \label{GhisiGobbino}
  Let $k$ be a positive integer, let ${\alpha} \in (0,1]$, let $I \subseteq {\mathbb{R}}$ be an open bounded interval, and let $f : I \to {\mathbb{R}}$ 
  be a function.
  Assume that $f$ is continuous and that there exists $g \in C^{k,{\alpha}}({\overline I},{\mathbb{R}})$ such that 
  \[
    |f|^{k +{\alpha}} = |g|.
  \]
  Let $p$ be defined by 
  \[
    \frac 1 p + \frac 1 {k+{\alpha}} =1.
  \]
  Then we have $f' \in L^p_w(I)$ and 
  \begin{equation} \label{GG}
    \|f'\|_{p,w,I} \le 
    C(k) \max\Big\{\big({\on{H\ddot{o}ld}}_{{\alpha},I}(g^{(k)})\big)^{1/(k+{\alpha})}|I|^{1/p}, 
    \|g'\|_{L^\infty(I)}^{1/(k+{\alpha})}\Big\}, 
  \end{equation}
  where $C(k)$ is a constant that depends only on $k$.
\end{theorem}

Here $L^p_w(I)$ denotes the weak Lebesgue space equipped with the quasinorm $\|\cdot\|_{p,w,I}$ (see Section \ref{Lebesgue}), 
and ${\on{H\ddot{o}ld}}_{{\alpha},I}(g^{(k)})$ is the ${\alpha}$-H\"older constant of $g^{(k)}$ on $I$. 

Let us briefly describe the strategy of our proof of Theorem \ref{main}. 
It is by induction on the degree of the polynomial and its heart is Proposition~\ref{induction}.
First we reduce the polynomial $P_a$ to Tschirnhausen form $P_{\tilde a}$ (indicated by adding \emph{tilde}), where $\tilde a_1 \equiv 0$ 
(see Section \ref{Tschirnhausen}), and split it near points $t_0$, where not all coefficients vanish, 
\[
  P_{\tilde a}(t) = P_b(t) P_{b^*}(t), \quad t \in I, \quad (t_0 \in I).
\]
The splitting is universal and gives formulas for the coefficients $b_i$ (and $b^*_i$) in terms of $\tilde a_j$ (see Section \ref{ssec:split}); 
hereby the differentiability class is preserved. 
After Tschirnhausen transformation, $P_b \leadsto P_{\tilde b}$, we split $P_{\tilde b}$ near points $t_1 \in I$, where not all $\tilde b_i$ vanish, 
\[
  P_{\tilde b}(t) = P_c(t) P_{c^*}(t), \quad t \in J, \quad (t_1 \in J).
\]
Again the splitting is universal, we get formulas for $c_h$ (and $c^*_h$) in terms of $\tilde b_j$, and the differentiability class is preserved.
Apply the Tschirnhausen transformation, $P_c \leadsto P_{\tilde c}$.
Let $k \in \{2,\ldots,n\}$ be such that 
\[
   |\tilde a_k(t_0)|^{1/k} = \max_{2 \le j \le n} |\tilde a_j(t_0)|^{1/j}  
\]
and $\ell \in \{2,\ldots, \deg P_b\}$ such that 
\[
   |\tilde b_\ell(t_1)|^{1/\ell} = \max_{2 \le i \le \deg P_b} |\tilde b_i(t_1)|^{1/i}.  
\]
The central idea consists in showing that, for $1 \le p < n/(n-1)$, we have an estimate of the form 
\begin{align} \label{central}
  \||J|^{-1} |\tilde b_{\ell}(t)|^{1/\ell} \|_{L^p (J)}
  + \sum_{h=2}^{\deg P_{c}} \|(\tilde c_{h}^{1/h})'\|_{L^p (J)}
  &\le C \Big( \| |I|^{-1}  {|\tilde a_k(t_0)|^{1/k}} \|_{L^p (J)} 
  + \sum_{i=2}^{\deg P_b} \|(\tilde b_i^{1/i})'\|_{L^p (J)}\Big), 
\end{align}
for a universal constant $C=C(n,p)$.
In the derivation of this estimate we make essential use of \eqref{GG} in order to bound $\|(\tilde c_{h}^{1/h})'\|_{L^p (J)}$.
We get the estimate \eqref{central} on neighborhoods $J$ of all points $t_1 \in I$, where not all $\tilde b_i$ vanish. 
In order to glue them we prove in Proposition \ref{cover} that there is a countable subcollection of intervals $J$ such that 
every point in their union is covered at most by two intervals. 
In this gluing process we use the ${\sigma}$-additivity of $\|\cdot\|^p_{L^p}$. 
Since the $L^p_w$-quasinorm lacks this property, we are forced to switch from $L^{n/(n-1)}_w$- to $L^p$-bounds for $p<n/(n-1)$.

The paper is structured as follows. We fix notation and recall facts on function spaces in Section \ref{functionspaces}. 
Ghisi and Gobbino's result on radicals (Theorem \ref{GhisiGobbino}) is extended to complex valued functions in Section \ref{radicals}.
We collect preliminaries on polynomials and define a universal splitting of such in Section \ref{polynomials}.
We derive bounds for the coefficients of a polynomial and generalize Ghisi and Gobbino's higher order Glaeser inequalities 
\cite[Prop.~3.4]{GhisiGobbino13}
in Section~\ref{Glaeser}, by applying these bounds to the Taylor polynomial.  
In Sections \ref{aestimates} and \ref{bestimates} we deduce estimates for the iterated derivatives of the coefficients before 
and after the splitting. 
Section \ref{specialcover} is dedicated to the proof of Proposition \ref{cover}. The proof of Theorem \ref{main} is finally carried out 
in Section \ref{proof}.

\section{Function spaces} \label{functionspaces}

In this section we fix notation for function spaces and recall well-known facts. 

\subsection{H\"older spaces}

Let ${\Omega} \subseteq {\mathbb{R}}^n$ be open. We denote by $C^0({\Omega})$ the space of continuos complex valued functions on ${\Omega}$.
For $k \in {\mathbb{N}} \cup \{\infty\}$ we set 
\begin{align*}
  C^k({\Omega}) &= \{f \in {\mathbb{C}}^{\Omega} : {\partial}^{\alpha} f \in C^0({\Omega}), 0 \le |{\alpha}| \le k\},\\
  C^k(\overline {\Omega}) &= \{f \in C^k({\Omega}) : {\partial}^{\alpha} f \text{ has a continuous extension to } \overline {\Omega}, 
  0 \le |{\alpha}| \le k\}.
\end{align*}
For ${\alpha} \in (0,1]$ a function $f : {\Omega} \to {\mathbb{C}}$ belongs to $C^{0,{\alpha}}(\overline {\Omega})$ if it is ${\alpha}$-H\"older continuous 
in ${\Omega}$, i.e.,
\[
{\on{H\ddot{o}ld}}_{{\alpha},{\Omega}}(f) := \sup_{x,y \in {\Omega}, x \ne y} \frac{|f(x)-f(y)|}{|x-y|^{\alpha}} < \infty.
\]
If $f$ is Lipschitz, i.e., $f \in C^{0,1}({\overline} {\Omega})$, we use 
\[
{\on{Lip}}_{\Omega} (f) ={\on{H\ddot{o}ld}}_{1,{\Omega}}(f).
\]
We define 
\[
 C^{k,{\alpha}}(\overline {\Omega}) = \{f \in C^k(\overline {\Omega}) : {\partial}^{\beta} f \in C^{0,{\alpha}}(\overline {\Omega}), |{\beta}|=k\}.
\]
Note that $C^{k,{\alpha}}(\overline {\Omega})$ is a Banach space when provided with the norm
\[
\|f\|_{C^{k,{\alpha}}(\overline {\Omega})} 
:= \sup_{\substack{|{\beta}| \le k\\ x \in {\Omega}}} |{\partial}^{\beta} f(x)| + \sup_{|{\beta}|=k} {\on{H\ddot{o}ld}}_{{\alpha},{\Omega}}({\partial}^{\beta} f).
\]

 
\subsection{Lebesgue spaces and weak Lebesgue spaces}  \label{Lebesgue}

Let ${\Omega} \subseteq {\mathbb{R}}^n$ be open and bounded, and let $1 \le p \le \infty$.
We denote by $L^p({\Omega})$ the Lebesgue space with respect to the $n$-dimensional Lebesgue measure ${\mathcal{L}}^n$. 
For Lebesgue measurable sets $E \subseteq {\mathbb{R}}^n$ we will denote by 
\[
  |E| = {\mathcal{L}}^n(E)
\]
its $n$-dimensional Lebesgue measure. We denote by $p'$ the conjugate exponent of $p$ defined by 
\[
  \frac 1p + \frac 1 {p'} =1
\]
with the convention $1' = \infty$ and $\infty' =1$.

Let $1 \le p < \infty$. 
A measurable function $f : {\Omega} \to {\mathbb{C}}$ belongs to the weak $L^p$-space $L_w^p({\Omega})$ if 
\[
\|f\|_{p,w,{\Omega}} := \sup_{r\ge 0}  r\, |\{x \in {\Omega} : |f(x)| > r\}|^{1/p} < \infty.
\]  
For $1 \le q < p < \infty$ we have (cf.\ \cite[Ex.\ 1.1.11]{Grafakos08})
\begin{equation} \label{eq:qp}
  \|f\|_{q,w,{\Omega}} \le \|f\|_{L^q({\Omega})} \le \Big(\frac{p}{p-q}\Big)^{1/q} 
  |{\Omega}|^{1/q-1/p} \|f\|_{p,w,{\Omega}}
\end{equation}
and hence
$L^p({\Omega}) \subseteq L_w^p({\Omega}) \subseteq L^q({\Omega}) \subseteq L_w^q({\Omega})$
with strict inclusions. 
It will be convenient to \emph{normalize} the $L^p$-norm and the $L^p_w$-quasinorm, i.e., we will consider 
\begin{align*}
  \|f\|^*_{L^p({\Omega})} &:= |{\Omega}|^{-1/p} \|f\|_{L^p({\Omega})},\\
  \|f\|^*_{p,w,{\Omega}} &:= |{\Omega}|^{-1/p} \|f\|_{p,w,{\Omega}}.
\end{align*}
Note that $\|1\|^*_{L^p({\Omega})} = \|1\|^*_{p,w,{\Omega}} =1$.
Then, for $1 \le q < p < \infty$,
\begin{gather}
  \|f\|^*_{L^q({\Omega})} \le \|f\|^*_{L^p({\Omega})}, \label{inclusions0}\\
  \|f\|^*_{q,w,{\Omega}} \le \|f\|^*_{L^q({\Omega})} \le \Big(\frac{p}{p-q}\Big)^{1/q} 
   \|f\|^*_{p,w,{\Omega}}. \label{inclusions}
\end{gather}
We remark that $\|\cdot\|_{p,w,{\Omega}}$ is only a quasinorm; the triangle inequality fails, but for  
$f_j \in L_w^p({\Omega})$ 
we still have
\[
\Big\|\sum_{j=1}^m f_j \Big\|_{p,w,{\Omega}} \le m \sum_{j=1}^m \|f_j\|_{p,w,{\Omega}}.
\]
There exists a norm equivalent to $\|\cdot\|_{p,w,{\Omega}}$ which makes $L_w^p({\Omega})$ into a Banach space if $p>1$. 

The $L^p_w$-quasinorm is ${\sigma}$-subadditive: if $\{{\Omega}_j\}$ is a countable family of open sets with 
${\Omega} = \bigcup {\Omega}_j$ then 
\begin{equation}
  \|f\|^p_{p,w,{\Omega}}  \le \sum_j \|f\|^p_{p,w,{\Omega}_j} \quad \text{ for every } f \in L^p_w({\Omega}).
\end{equation}
But it is not ${\sigma}$-additive:
for instance, for $h : (0,\infty) \to {\mathbb{R}}$, $h(t):= t^{-1/p}$, we have  
$\|h\|_{p,w,(0,{\epsilon})}^p = 1$ for every ${\epsilon}>0$,  
but $\|h\|_{p,w,(1,2)}^p = 1/2$.

\subsection{Sobolev spaces}

For $k \in {\mathbb{N}}$ and $1 \le p \le \infty$ we consider the Sobolev space
\[
  W^{k,p}({\Omega}) = \{f \in L^p({\Omega}) : {\partial}^{\alpha} f \in L^p({\Omega}), 0 \le |{\alpha}| \le k\},
\]
where ${\partial}^{\alpha} f$ denote distributional derivatives. On bounded intervals $I \subseteq {\mathbb{R}}$ the Sobolev space $W^{1,1}(I)$ 
coincides with the space $AC(I)$ of absolutely continuous functions on $I$
if we identify each $W^{1,1}$-functions with its unique continuous representative. 
Recall that a function $f : {\Omega} \to {\mathbb{R}}$ on an open subset ${\Omega} \subseteq {\mathbb{R}}$ 
is absolutely continuous if for every ${\epsilon}>0$ there exists ${\delta}>0$ so that  
$\sum_{i=1}^n |a_i -b_i| < {\delta}$ implies $\sum_{i=1}^n |f(a_i) -f(b_i)| < {\epsilon}$ whenever $[a_i,b_i]$, $i =1,\ldots,n$, 
are non-overlapping intervals contained in ${\Omega}$.

We shall also use $W^{k,p}_{\on{loc}}$, $AC_{\on{loc}}$, etc.\ with the obvious meaning.

\subsection{Extension lemma}

We will use the following extension lemma. The analogue for $L^p_w$-quasinorms may be found in \cite[Lemma 2.1]{ParusinskiRainerAC}
which is a slight generalization of \cite[Lemma 3.2]{GhisiGobbino13}. Here we need a version for $L^p$-norms.  

\begin{lemma} \label{lem:extend}
  Let ${\Omega} \subseteq {\mathbb{R}}$ be open and bounded, let $f : {\Omega} \to {\mathbb{C}}$ be continuous, and set ${\Omega}_0 := \{t \in {\Omega} : f(t) \ne 0\}$.  
  Assume that $f|_{{\Omega}_0} \in AC_{\on{loc}}({\Omega}_0)$ and that $f|_{{\Omega}_0}' \in L^p({\Omega}_0)$ for some $p>1$ 
  (note that $f$ is differentiable a.e.\ in ${\Omega}_0$).
  Then the distributional derivative of $f$ in ${\Omega}$ is a measurable function $f' \in L^p({\Omega})$ and 
  \begin{equation} \label{eq:extend}
  \|f'\|_{L^p({\Omega})} = \|f|_{{\Omega}_0}'\|_{L^p({\Omega}_0)}.
  \end{equation}
\end{lemma}

\begin{proof} 
The function ${\psi} : {\Omega} \to {\mathbb{C}}$ defined by  
\[
{\psi}(t) := 
\begin{cases}
  f'(t) & \text{ if } t \in {\Omega}_0\\
  0     & \text{ if } t \in {\Omega} \setminus {\Omega}_0  
\end{cases}
\]
clearly belongs to $L^p({\Omega})$. We show that ${\psi}$ is the distributional derivative of $f$ in ${\Omega}$.
Let ${\phi} \in C^\infty_c({\Omega})$ be a test function with compact support in ${\Omega}$ and 
let ${\mathcal{C}}$ denote the (at most countable) set of connected components of ${\Omega}_0$. 
Then, using integration by parts for the Lebesgue integral (see e.g.\ \cite{Leoni09} Corollary 3.37) 
\begin{align*}
  \int_{\Omega} f {\phi}' \, dt = \int_{{\Omega}_0} f {\phi}'  \, dt = \sum_{J \in {\mathcal{C}}} \int_{J} f {\phi}'  \, dt
  = -\sum_{J \in {\mathcal{C}}} \int_{J} f' {\phi}  \, dt = -\int_{{\Omega}_0} f' {\phi}  \, dt = -\int_{\Omega} {\psi} {\phi}  \, dt.   
\end{align*}  
(If $J=(a,b)$ then $\int_a^b f {\phi}'  \, dt = \lim_{{\epsilon} \to 0^+} \int_{a+{\epsilon}}^{b-{\epsilon}} f {\phi}'  \, dt 
= - \lim_{{\epsilon} \to 0^+} \int_{a+{\epsilon}}^{b-{\epsilon}} f' {\phi}  \, dt
= - \int_a^b f' {\phi}  \, dt$, by the dominated convergence theorem, continuity of $f$, and \eqref{inclusions0}.)
Moreover, we have 
$\|f'\|_{L^p({\Omega})} = \|{\psi}\|_{L^p({\Omega})} = \|{\psi}\|_{L^p({\Omega}_0)} = \|f|_{{\Omega}_0}'\|_{L^p({\Omega}_0)}$, 
that is \eqref{eq:extend}.
\end{proof}

\section{Radicals of differentiable functions} \label{radicals}

We derive an analogue of Theorem \ref{GhisiGobbino} for complex valued functions.

\begin{proposition} \label{prop:radicals}
  Let $I \subseteq {\mathbb{R}}$ be a bounded interval, let $k \in {\mathbb{N}}_{>0}$, and ${\alpha} \in (0,1]$.
  For each $g \in C^{k,{\alpha}}(\overline I)$ we have 
  \begin{equation} \label{est1}
    |g'(t)| \le {\Lambda}_{k+{\alpha}}(t) |g(t)|^{1-1/(k+{\alpha})}, \quad \text{ a.e.\ in } I,
  \end{equation}  
  for some ${\Lambda}_{k+{\alpha}}={\Lambda}_{k+{\alpha},g} \in L_w^p(I,{\mathbb{R}}_{\ge0})$, 
  where $1/p + 1/(k+{\alpha}) = 1$, and such that
  \begin{equation} \label{est2}
    \|{\Lambda}_{k+{\alpha}}\|_{p,w,I} \le C(k) \max\Big\{\big({\on{H\ddot{o}ld}}_{{\alpha},I}(g^{(k)})\big)^{1/(k+{\alpha})}|I|^{1/p}, 
    \|g'\|_{L^\infty(I)}^{1/(k+{\alpha})}\Big\}.
  \end{equation}
\end{proposition}

\begin{proof}
  Analogous to the proof of \cite[Proposition~3.1]{ParusinskiRainerAC}.
\end{proof}

\begin{corollary} \label{cor:radicals}
  Let $n$ be a positive integer and 
  let $I \subseteq {\mathbb{R}}$ be an open bounded interval. Assume that $f : I \to {\mathbb{C}}$ is a continuous function such that 
  $f^n = g \in C^{n-1,1}({\overline I})$.
  Then we have $f' \in L^{n'}_w(I)$ and 
  \begin{equation} \label{est}
    \|f'\|_{n',w,I} \le 
    C(n) \max\Big\{\big({\on{Lip}}_{I}(g^{(n-1)})\big)^{1/n}|I|^{1/n'}, 
    \|g'\|_{L^\infty(I)}^{1/n}\Big\}, 
  \end{equation}
  where $C(n)$ is a constant that depends only on $n$ and $1/n +1/n'=1$.  
\end{corollary}

\begin{proof}
  On the set ${\Omega}_0 = \{t \in I : f(t) \ne 0\}$, $f$ is differentiable and satisfies
  \[
    |f'(t)|   
    = \frac 1 n \frac{|g'(t)|}{|g(t)|^{1-1/n}}.
  \]  
  So the assertion follows from Proposition \ref{prop:radicals} and the $L^p_w$-analogue of Lemma \ref{lem:extend}; see 
  \cite[Lemma~2.1]{ParusinskiRainerAC}.
\end{proof}

\begin{remark}
  Proposition~\ref{prop:radicals} and hence also Corollary \ref{cor:radicals} are optimal in the following sense:
  \begin{itemize}
    \item ${\Lambda}_{k+{\alpha}}$ can in general not be chosen in $L^p$. Indeed, for $g : (-1,1) \to {\mathbb{R}}$, $g(t)=t$, we have 
    \[
    \Big(\frac{|g'|}{|g|^{1-1/(k+{\alpha})}}\Big)^p  = |t|^{-1},
    \] 
    which is not integrable near $0$. See \cite[Example 4.3]{GhisiGobbino13}.
    \item If $f$ is just in $C^{k,{\beta}}(\overline I)$ for every ${\beta}<{\alpha}$, 
    then \eqref{est1} does in general not hold with ${\Lambda}_{k+{\alpha}} \in L^1(I)$.
    Indeed, in \cite[Example 4.4]{GhisiGobbino13} there is constructed a non-negative function $f : I \to {\mathbb{R}}$ 
    which belongs to $C^{k,{\beta}}(\overline I) \cap C^{\infty}(I)$ for every ${\beta} < {\alpha}$, but not for ${\beta} = {\alpha}$, 
    and whose non-negative $(k+{\alpha})$-root has unbounded variation in $I$.    
  \end{itemize}
\end{remark}

\section{Preliminaries on polynomials} \label{polynomials}

\subsection{Tschirnhausen transformation}

A monic polynomial 
\[
  P_a(Z) = Z^n + \sum_{j=1}^n a_j Z^{n-j}, \quad a=(a_1,\ldots,a_n) \in {\mathbb{C}}^n
\] 
is said to be in \emph{Tschirnhausen form} if $a_1=0$.
Every polynomial $P_a$ can be transformed to a polynomial $P_{\tilde a}$ in Tschirnhausen form 
by the substitution $Z \mapsto Z-a_1/n$, which we refer to as the 
\emph{Tschirnhausen transformation}, 
\[
  P_{\tilde a}(Z) = P_a(Z-a_1/n) = Z^n + \sum_{j=2}^n \tilde a_j Z^{n-j}, \quad \tilde a=(\tilde a_2,\ldots,\tilde a_n) \in {\mathbb{C}}^{n-1}.
\]
We have the formulas
\begin{equation}\label{Tschirnhausen}
  \tilde a_j = \sum_{\ell=0}^j C_\ell\, a_\ell\, {a_1}^{j-\ell}, \quad j= 2,\ldots,n, 
\end{equation} 
where $C_\ell$ are universal constants.
The effect of the Tschirnhausen transformation will always be indicated by adding \emph{tilde} to the coefficients, $P_a \leadsto P_{\tilde a}$.  

We will identify the set of monic complex polynomials $P_a$ of degree $n$ with the set ${\mathbb{C}}^n$ (via $P_a \mapsto a$) and 
the set of monic complex polynomials $P_{\tilde a}$ of degree $n$ in Tschirnhausen form 
with the set ${\mathbb{C}}^{n-1}$ (via $P_{\tilde a} \mapsto \tilde a$).

\subsection{Splitting} \label{ssec:split}

The following well-known lemma (see e.g.\ \cite{AKLM98} or \cite{BM90}) is a consequence of the 
inverse function theorem. 

\begin{lemma} \label{split}
Let $P_a = P_b P_c$, where $P_b$ and $P_c$ are monic complex polynomials without common root.
Then for $P$ near $P_a$ we have $P = P_{b(P)} P_{c(P)}$
for analytic mappings of monic polynomials $P \mapsto b(P)$ and $P \mapsto c(P)$,
defined for $P$ near $P_a$, with the given initial values.
\end{lemma}

\begin{proof}
The splitting $P_a = P_b P_c$ defines on the coefficients a polynomial mapping ${\varphi}$ such that $a = {\varphi}(b,c)$, 
where $a=(a_i)$, $b=(b_i)$, and $c=(c_i)$. The Jacobian determinant  
$\det d{\varphi}(b,c)$ equals the resultant of $P_b$ and $P_c$ which is nonzero by assumption. 
Thus ${\varphi}$ can be inverted locally. 
\end{proof}

If $P_{\tilde a}$ is in Tschirnhausen form and if $\tilde a \ne 0$, then $P_{\tilde a}$ \emph{splits}, i.e., 
$P_{\tilde a} = P_b P_c$ for monic polynomials $P_b$ and $P_c$ with positive degree and without common zero. 
For, if ${\lambda}_1,\ldots,{\lambda}_n$ denote the roots of $P_{\tilde a}$ and they all coincide, then since 
\[
    {\lambda}_1+\cdots+{\lambda}_n = \tilde a_1 = 0
\] 
they all must vanish, contradicting $\tilde a \ne 0$.

Let us identify the set of monic complex polynomials $P_{\tilde a}$ of degree $n$ in Tschirnhausen form with the set ${\mathbb{C}}^{n-1}$, and 
let $\tilde a_2,\ldots,\tilde a_n$ denote the coordinates in ${\mathbb{C}}^{n-1}$.
Fix $k \in \{2,\ldots,n\}$ and let $p \in {\mathbb{C}}^{n-1} \cap \{\tilde a_k \ne 0\}$; $p$ corresponds to the polynomial $P_{\tilde a}$. 
We associate the polynomial 
\begin{gather*}
  Q_{\underline a}(Z) := \tilde a_k^{- n/k} P_{\tilde a} (\tilde a_k^{1/k} Z) 
  = Z^n + \sum_{j=2}^n \tilde a_k^{- j/k} \tilde a_j Z^{n-j},\\
  \underline a_j := \tilde a_k^{- j/k} \tilde a_j, \quad j = 2,\ldots, n, 
\end{gather*}
where the radicals are interpreted as multi-valued functions. Then $Q_{\underline a}$ is in Tschirnhausen form and $\underline a_k =1$. 
By Lemma~\ref{split} we have a splitting $Q_{\underline a} = Q_{\underline b} Q_{\underline c}$ on some open ball $B_{\rho}(p)$ 
centered at $p$ with radius ${\rho}>0$. 
In particular, there exist analytic functions ${\psi}_i$ so that, on $B_{\rho}(p)$,
\begin{equation*} 
  \underline b_i = {\psi}_i \big(\tilde a_k^{-2/k} \tilde a_2, \tilde a_k^{-3/k} \tilde a_3, \ldots, \tilde a_k^{-n/k} \tilde a_n\big), 
    \quad i = 1,\ldots,\deg P_{\underline b}.
\end{equation*}
The splitting $Q_{\underline a} = Q_{\underline b} Q_{\underline c}$ induces a splitting 
$P_{\tilde a} = P_b P_c$, where 
\begin{equation} \label{eq:bj}
    b_i = \tilde a_k^{i/k} {\psi}_i \big(\tilde a_k^{-2/k} \tilde a_2, \tilde a_k^{-3/k} \tilde a_3, \ldots, \tilde a_k^{-n/k} \tilde a_n\big), 
    \quad i = 1,\ldots,\deg P_b;
\end{equation}
likewise for $c_j$. 
Shrinking ${\rho}$ slightly, we may assume that all partial derivatives of ${\psi}_i$ are separately bounded on $B_{\rho}(p)$. 
If $\tilde b_j$ denote the coefficients of the polynomial $P_{\tilde b}$ resulting from $P_b$ by the  
Tschirnhausen transformation, then, by \eqref{Tschirnhausen}, 
\begin{equation} \label{eq:tildebj}
    \tilde b_i = \tilde a_k^{i/k} 
    \tilde {\psi}_i \big(\tilde a_k^{-2/k} \tilde a_2, \tilde a_k^{-3/k} \tilde a_3, \ldots, \tilde a_k^{-n/k} \tilde a_n\big), 
    \quad i = 2,\ldots,\deg P_b;
\end{equation}
for analytic functions $\tilde {\psi}_i$ all partial derivatives of which are separately bounded on $B_{\rho}(p)$.

\subsection{Coefficient estimates}

We shall need the following estimates. (Here it is convenient to number the coefficients in reversed order.)

\begin{lemma} \label{lem:interpol}
  Let $m\ge 1$ be an integer and ${\alpha} \in (0,1]$.
  Let $P(x) = a_1 x + \cdots + a_{m} x^{m} \in {\mathbb{C}}[x]$ satisfy 
  \begin{equation} \label{interpol1}
    |P(x)| \le A (1+M x^{m+{\alpha}}), \quad \text{ for }~ x \in [0,B] \subseteq {\mathbb{R}},
  \end{equation}
  and constants $A, M \ge 0$ and $B>0$. Then 
  \begin{equation} \label{interpol2}
    |a_j| \le C A (1+M^{j/(m+{\alpha})} B^j) B^{-j}, \quad j=1,\ldots,m,
  \end{equation}
  for a constant $C$ depending only on $m$ and ${\alpha}$.
\end{lemma}

\begin{proof}
The statement is well-known if $M=0$; see \cite[Lemma 3.4]{ParusinskiRainerHyp}. Assume that $M>0$.

It suffices to consider the special case $A=B=1$.  
The general case follows by applying the special case to $Q(x)= A^{-1} P(Bx) =  b_1 x + \cdots + b_{m} x^{m}$, where 
$b_i= A^{-1} B^i  {a_i}$.  

Fix $k \in \{1,\ldots,m\}$ and write the inequality \eqref{interpol1} in the form 
\begin{equation} \label{interpol3}
  |x^{-k} P(x)| \le x^{-k} + M x^{m+{\alpha}-k}.
\end{equation}
The function on the right-hand side of \eqref{interpol3} attains is minimum on $\{x>0\}$ at the point 
\begin{equation} \label{interpol5}
  x_k = \big(\tfrac k{m+{\alpha}-k}\big)^{1/(m+{\alpha})} M^{-1/(m+{\alpha})}  
\end{equation}
and this minimum is of the form $C_k M^{k/(m+{\alpha})}$ for some $C_k$ depending only on $k$, $m$, and ${\alpha}$.
Thus, 
\begin{equation} \label{interpol4}
  |P(x_k)| \le \tilde C_k, 
\end{equation}
for some $\tilde C_k$ depending only on $k$, $m$, and ${\alpha}$.

Suppose first that $x_k \le 1$ for all $k= 1,\ldots,m$ and consider 
\begin{equation*}
    a_1 x_k + \cdots + a_{m} x_k^{m} = P(x_k), \quad k= 1,\ldots,m, 
\end{equation*}
as a system of linear equations with the unknowns $a_k M^{-k/(m+{\alpha})}$ and the (Vandermonde-like) matrix 
\begin{equation*}
  L = \big(\big(\tfrac{k}{m+{\alpha}-k}\big)^{j/(m+{\alpha})}\big)_{k,j=1}^{m}.   
\end{equation*} 
Then the vector of unknowns is given by 
\[
  (a_1 M^{-1/(m+{\alpha})}, \ldots , a_{m} M^{-m/(m+{\alpha})})^T = L^{-1}  (P(x_1), P(x_2), \ldots , P(x_{m}))^T.
\] 
By \eqref{interpol4}, we may conclude that 
\[
  |a_j| \le C M^{j/(m+{\alpha})}, \quad j=1,\ldots,m,
\]
for a constant $C$ depending only on $m$ and ${\alpha}$.

If $x_k > 1$ then $M < k/(m+{\alpha}-k)$, by \eqref{interpol5}, and hence for $x \in [0,1]$,
\[
  |P(x)| \le 1+ M x^{m+{\alpha}} \le 1+ \tfrac k {m+{\alpha}-k} \le \tfrac{m+{\alpha}}{\alpha}. 
\]
In this case we may apply the lemma with $M=0$, $A = (m+{\alpha})/{\alpha}$, and $B=1$, and obtain 
\[
  |a_j| \le C , \quad j=1,\ldots,m,
\]
for a constant $C$ depending only on $m$ and ${\alpha}$.

Summing up, we obtained \eqref{interpol2} in the case $A=B=1$.
\end{proof} 

As a consequence we get estimates for the intermediate derivatives of a finitely differentiable function in terms of the 
function and its highest derivative. 
For an interval $I \subseteq {\mathbb{R}}$ and a function $f : I \to {\mathbb{C}}$ we define
\[
  V_I(f) := \sup_{t,s \in I} |f(t)-f(s)|.
\]

\begin{lemma} \label{taylor} 
  Let $I \subseteq {\mathbb{R}}$ be a bounded open interval, $m \in {\mathbb{N}}$, and ${\alpha} \in (0,1]$.
  If $f\in C^{m,{\alpha}}(\overline I)$, then there is a universal constant $C$, depending only on $m$ and ${\alpha}$, 
  such that for all $t\in I$ 
  and  $s = 1,\ldots,m$,  
    \begin{align}\label{eq:1}  
    |f^{(s)}(t) | \le C |I|^{-s} \bigl(V_I(f) + V_I(f)^{(m+{\alpha}-s)/(m+{\alpha})} ({\on{H\ddot{o}ld}}_{{\alpha},I}(f^{(m)}))^{s/(m+{\alpha})}  |I|^s
    \bigr)  .  
  \end{align}
\end{lemma}

\begin{proof}
We may suppose that $I=(-\delta, \delta)$. If $t \in I$ then at least one of the two intervals
 $[t,t\pm \delta )$, say $[t,t+ \delta )$, is included in $I$.  
 By Taylor's formula, for $t_1\in [t,t +\delta )$, 
\begin{align*}
      \sum_{s=1}^{m}  \frac{{f}^{(s)}(t)}{s!} (t_1-t)^s 
       & = f(t_1)-f(t) -  \int_0^1 \frac {(1-{\tau})^{m-1}}{(m-1)!} \big(f^{(m)}(t+{\tau}(t_1-t))-f^{(m)}(t)\big)\, d{\tau}\, (t_1-t)^m      
\end{align*}
and hence
    \begin{align*}
      \Big|\sum_{s=1}^{m}  \frac{{f}^{(s)}(t)}{s!} (t_1-t)^s\Big| 
      & \le V_I(f)+  {\on{H\ddot{o}ld}}_{{\alpha},I}(f^{(m)})  (t_1-t)^{m+{\alpha}}
      \\
      & = V_I(f)\big(1 + V_I(f)^{-1} {\on{H\ddot{o}ld}}_{{\alpha},I}(f^{(m)})  (t_1-t)^{m+{\alpha}} \big).        
    \end{align*}
The assertion follows from Lemma~\ref{lem:interpol}.   
\end{proof}

\subsection{Higher order Glaeser inequalities} \label{Glaeser}

As a corollary of Lemma \ref{taylor} we obtain a generalization of 
Ghisi and Gobbino's higher order Glaeser inequalities \cite[Prop.~3.4]{GhisiGobbino13}.

\begin{corollary}
  Let $m \in {\mathbb{N}}$ and ${\alpha} \in (0,1]$.
  Let $I = (t_0- {\delta},t_0+{\delta})$ with $t_0 \in {\mathbb{R}}$ and ${\delta}>0$.
  If $f\in C^{m,{\alpha}}(\overline I)$ is such that $f$ and $f'$ do not change their sign on $I$,
  then there is a universal constant $C$, depending only on $m$ and ${\alpha}$, such that for all $s = 1,\ldots,m$,
  \begin{align}\label{eq:2}  
    |f^{(s)}(t_0) | \le C |I|^{-s} \bigl(|f(t_0)| + |f(t_0)|^{(m+{\alpha}-s)/(m+{\alpha})} ({\on{H\ddot{o}ld}}_{{\alpha},I}(f^{(m)}))^{s/(m+{\alpha})}  |I|^s
    \bigr) .  
  \end{align}
\end{corollary}

\begin{proof}
  For simplicity assume $t_0=0$.
  Changing $f$ to $-f$ and $t$ to $-t$ if necessary, we may assume that $f(t) \ge 0$ and $f'(t)\le 0$ for all $t \ge 0$.
  Then $V_{[0,{\delta})}(f) \le f(0)$ and so \eqref{eq:2} follows from \eqref{eq:1}.
\end{proof}

For $s=1$ we recover \cite[Prop.~3.4]{GhisiGobbino13}. Indeed, for $s =1$ we may write \eqref{eq:2} in the form 
\begin{align} \label{eq:3}   
    |f'(t_0) | \le C  |f(t_0)|^{(m+{\alpha}-1)/(m+{\alpha})} \max\bigl\{|f(t_0)|^{1/(m+{\alpha})} |I|^{-1}, ({\on{H\ddot{o}ld}}_{{\alpha},I}(f^{(m)}))^{1/(m+{\alpha})}  \bigr\},  
\end{align}
and the inequality in \cite[Prop.~3.4]{GhisiGobbino13} can be written as 
\begin{align} \label{eq:4}  
    |f'(t_0) | \le C  |f(t_0)|^{(m+{\alpha}-1)/(m+{\alpha})} 
    \max\bigl\{|f'(t_0)|^{1/(m+{\alpha})} |I|^{-1+1/(m+{\alpha})}, ({\on{H\ddot{o}ld}}_{{\alpha},I}(f^{(m)}))^{1/(m+{\alpha})}  \bigr\}.  
\end{align}
These two inequalities are equivalent in the following sense.  
 If \eqref{eq:3} holds with the constant $C>0$ then \eqref{eq:4} holds with the constant 
 $\max \{C,C^{(m+{\alpha}-1)/(m+{\alpha})}\}$ and 
 symmetrically,  if \eqref{eq:4} holds with the constant $C>0$ then \eqref{eq:3} holds with the constant 
 $\max \{C,C^{(m+{\alpha})/(m+{\alpha}-1)}\}$.  For instance, suppose that \eqref{eq:3} holds.  
 If the second term in the maximum is dominant then \eqref{eq:4} holds with the same constant.  If the first term is dominant in the 
 maximum, that is $|f'(t_0)| \le C |f(t_0)| |I|^{-1}$, then $|f'(t_0)| ^{(m+{\alpha}-1)/(m+{\alpha})} \le (C
  |f(t_0)| |I|^{-1})^{(m+{\alpha}-1)/(m+{\alpha})}$ and \eqref{eq:4} holds 
  with the constant  $C^{(m+{\alpha}-1)/(m+{\alpha})}$.    

\section{Estimates for the iterated derivatives of the coefficients} \label{aestimates}

\subsection{Preparations for the splitting}

Let $I\subseteq {\mathbb{R}}$ be a bounded open interval and let 
\begin{equation} \label{eq:polynomial}
  P_{\tilde a(t)}(Z) = Z^n + \sum_{j=2}^n \tilde a_j(t) Z^{n-j},\quad t \in I,    
\end{equation} 
be a monic complex polynomial in Tschirnhausen form with coefficients $\tilde a_j \in C^{n-1,1}( {\overline I} )$, $j = 2,\ldots,n$.
We make the following assumptions.
Suppose that $t_0\in I$ and $k \in \{2,\dots,n\}$ are such that 
\begin{equation} \label{eq:k}
  |\tilde a_k(t_0)|^{1/k} =  \max_{2 \le j\le n} |\tilde a_j(t_0)|^{1/j}\ne 0  
 \end{equation} 
and that, for some positive constant $B<  1/3$,  
\begin{align}\label{assumpt}
\sum_{j=2}^n \|(\tilde a_j^{1/j})'\|_{L^1 (I)} \le  B |\tilde a_k(t_0)|^\frac{1}{k} .
\end{align}

By Corollary~\ref{cor:radicals}, 
every continuous selection $f$ of 
the multi-valued function $\tilde a_j^{1/j}$ is absolutely continuous on $I$ and $\|f'\|_{L^1(I)}$ is independent of the 
choice of the selection, by \eqref{inclusions} and \eqref{est}. 
(By a selection of a set-valued function $F : X \leadsto Y$ we mean a single-valued function $f : X \to Y$ such that $f(x) \in F(x)$ 
for all $x \in X$.)
So henceforth we shall fix one continuous selection of $\tilde a_j^{1/j}$, and, abusing notation, denote it by $\tilde a_j^{1/j}$ as well.

\begin{lemma} \label{lem1}
  Assume that the polynomial \eqref{eq:polynomial} satisfies \eqref{eq:k}--\eqref{assumpt}. Then for all $t \in I$ and $j=2,\ldots,n$,
  \begin{align} \label{eq:ass10} 
    |\tilde a_j^{1/j} (t) - \tilde a_j^{1/j} (t_0)| \le  B |\tilde a_k (t_0)|^{1/k},  
  \end{align}
  \begin{align} \label{eq:ass11}  
    \frac 2 3 <    1-B \le \Big | \frac{\tilde a_k(t)}{\tilde a_k(t_0)} \Big |^{1/k}   \le 1+B < \frac4 3,  
  \end{align}
  \begin{equation} \label{eq:ass12}
    |\tilde a_j(t)|^{1/j}\le \frac 4 3 |\tilde a_k(t_0)|^{1/k} \le  2  |\tilde a_k(t)|^{1/k}.
  \end{equation}
\end{lemma}

\begin{proof}
  In fact, by \eqref{assumpt},
  \begin{align*}
    |\tilde a_j^{1/j} (t) - \tilde a_j^{1/j} (t_0)| = |\int_{t_0}^t (\tilde a_j^{1/j})' \,ds| 
    \le \| (\tilde a_j^{1/j})'\|_{L^1(I)} \le B |\tilde a_k(t_0)|^{1/k}, 
  \end{align*}
  that is \eqref{eq:ass10}. For $j=k$ it implies
  \begin{align*}
      \Big|\Big | \frac{\tilde a_k(t)}{\tilde a_k(t_0)}\Big |^{1/k} -1\Big| \le  B, 
  \end{align*}
  and thus \eqref{eq:ass11}.
  By \eqref{eq:k}, \eqref{eq:ass10}, and \eqref{eq:ass11}, 
  \[
    |\tilde a_j(t)|^{1/j} \le (1+B) |\tilde a_k(t_0)|^{1/k} \le 2 |\tilde a_k(t)|^{1/k},
  \]
  that is \eqref{eq:ass12}.
\end{proof}

By \eqref{eq:ass11}, $\tilde a_k$ does not vanish on the interval $I$ and so the curve
\begin{align} \label{curve}
    \underline a : I &\to \{(\underline a_2,\dots,\underline a_n) \in {\mathbb{C}}^{n-1} : \underline a_k=1\} \\
    t &\mapsto {\underline} a(t) := (\tilde a_k^{-2/k} \tilde a_2, \ldots, \tilde a_k^{-n/k} \tilde a_n)(t) \notag
\end{align}
is well-defined.  

\begin{lemma} \label{lem2}
  Assume that the polynomial \eqref{eq:polynomial} satisfies \eqref{eq:k}--\eqref{assumpt}. 
  Then the length of the curve \eqref{curve} 
  is bounded by $3 n^2\, 2^{n} B$.
\end{lemma}

\begin{proof} 
  The estimates \eqref{eq:ass10}, \eqref{eq:ass11}, and \eqref{eq:ass12} 
  imply 
  \begin{align*}
    |\tilde a_k^{- j/k} \tilde a_j'| &\le  2^n |\tilde a_j^{- 1+1/j } \tilde a_j' \tilde a_k^{-1/k}|  
    \le  3n\, 2^{n-1} |(\tilde a_j^{1/j})'|   |\tilde a_k (t_0)|^{-1/k}    
    \\
   | (\tilde a_k^{-j/k})' \tilde a_j| &\le n 2^n |\tilde a_k^{-1/k} (\tilde a_k^{1/k})'| 
   \le  3n\, 2^{n-1} |(\tilde a_k^{1/k})'|  
   |\tilde a_k (t_0)|^{-1/k},
  \end{align*}
  and thus 
  \[
    |(\tilde a_k^{-j/k} \tilde a_j)'| \le 3n\, 2^{n-1} |\tilde a_k (t_0)|^{-1/k} \Big(|(\tilde a_j^{1/j})'| + |(\tilde a_k^{1/k})'|\Big).
  \]
  Consequently, using \eqref{assumpt},
  \begin{align*}
    \int_I |{\underline} a'| \,ds \le 3 n^2\, 2^{n} B, 
  \end{align*}
  as required.
\end{proof}

\subsection{Estimates for the derivatives of the coefficients}
Let us replace \eqref{assumpt} by the stronger assumption
\begin{align}\label{assumption}
  {M} |I| + \sum_{j=2}^n \|(\tilde a_j^{1/j})'\|_{L^1 (I)} \le  B |\tilde a_k(t_0)|^{1/k},
\end{align} 
where 
\begin{equation} \label{M}
  {M} = \max_{2 \le j\le n}  ({\on{Lip}}_I(\tilde a_j^{(n-1)}))^{1/n} |\tilde a_k (t_0)|^{(n-j)/(kn)}.
\end{equation}

\begin{lemma} \label{bounds2}
Assume that the polynomial \eqref{eq:polynomial} satisfies \eqref{eq:k} and \eqref{assumption}. 
Then 
for all $t\in I$, 
$ j = 2,\ldots,n$, and  $ s = 1,\ldots,n$,  
    \begin{align}\label{est:a}  
    |\tilde a_j^{(s)}(t) | \le C(n)  |I|^{-s}  |\tilde a_k (t_0)|^{j/k}   .  
  \end{align}
\end{lemma}

\begin{proof}
By Lemma \ref{taylor},    
  \begin{align*}  
    |\tilde a_j^{(s)}(t) | \le C |I|^{-s} \bigl( V_I(\tilde a_j) +  V_I(\tilde a_j)^{(n-s)/n} {\on{Lip}}_I(\tilde a_j^{(n-1)})^{s/n}  |I|^s\bigr).   
  \end{align*}
By \eqref{eq:ass12}, 
\[
  V_I(\tilde a_j) \le 2 \|\tilde a_j\|_{L^{\infty}(I)} \le 2\, (4/3)^n |\tilde a_k (t_0)|^{j/k} 
\]
and, by \eqref{assumption},
\[
   \max_{2 \le j\le n}  ({\on{Lip}}_I(\tilde a_j^{(n-1)}))^{s/n} |\tilde a_k (t_0)|^{-js/(kn)} |I|^s 
   = |\tilde a_k (t_0)|^{-s/k} {M}^s |I|^s \le  1.
\]
Thus 
\begin{align*}
  \MoveEqLeft
  V_I(\tilde a_j) +  V_I(\tilde a_j)^{(n-s)/n} {\on{Lip}}_I(\tilde a_j^{(n-1)})^{s/n}  |I|^s
  \\
  &\le
   |\tilde a_k (t_0)|^{j/k} \big(C_1 + C_2  {\on{Lip}}_I(\tilde a_j^{(n-1)})^{s/n} |\tilde a_k (t_0)|^{-js/(kn)} |I|^s \big) 
  \\
  &\le C_3 |\tilde a_k (t_0)|^{j/k}, 
\end{align*}
for constants $C_i$ that depend only on $n$.
So \eqref{est:a} is proved.  
\end{proof}

\section{The estimates after splitting} \label{bestimates}

\subsection{Estimates after splitting on \texorpdfstring{$I$}{I}}

Assume that the polynomial \eqref{eq:polynomial} satisfies \eqref{eq:k}-\eqref{assumpt} and the estimates \eqref{est:a}.

We suppose additionally  
that the curve $\underline a$ as defined in \eqref{curve} lies entirely in one of the balls $B_{\rho}(p)$ 
from Section \ref{ssec:split} on which we have a splitting. 
Then 
$P_{\tilde a}$ splits on $I$, 
\begin{align}\label{splitting}
  P_{\tilde a}(t) = P_b(t) P_{b^*}(t), \quad t \in I. 
\end{align}
By \eqref{eq:bj} and \eqref{eq:tildebj}, the coefficients $b_i$ are of the form 
\begin{equation} \label{eq:b_i0}
  b_i = \tilde a_k^{i/k} {\psi}_i \big(\tilde a_k^{-2/k} \tilde a_2, \ldots, \tilde a_k^{-n/k} \tilde a_n\big), 
  \quad i = 1,\ldots, \deg P_b,
\end{equation}
and after Tschirnhausen transformation $P_b \leadsto P_{\tilde b}$, we get 
  \begin{equation} \label{eq:b_i}
  \tilde b_i = \tilde a_k^{i/k} \tilde {\psi}_i \big(\tilde a_k^{-2/k} \tilde a_2, \ldots, \tilde a_k^{-n/k} \tilde a_n\big),
  \quad i = 2,\ldots, \deg P_b,
  \end{equation}  
where ${\psi}_i$ and $\tilde {\psi}_i$ are the analytic functions specified in Section \ref{ssec:split}. 

\begin{lemma} \label{lem:B}
  Assume that the polynomial \eqref{eq:polynomial} satisfies \eqref{eq:k}--\eqref{assumpt}, \eqref{est:a}, and \eqref{splitting}--\eqref{eq:b_i}. 
  Then there is a universal constant $C$, depending only on $n$ and on the functions $\tilde {\psi}_i$, 
  such that for all $t \in I$, $i=2,\ldots,\deg P_b$, and $s = 1,\ldots,n$,
  \begin{equation} \label{eq:b_ider}
      |\tilde b_i^{(s)}(t) | \le C  |I|^{-s}  |\tilde a_k (t_0)|^{i/k}.   
  \end{equation}  
  Moreover, for all $t \in I$,
  \begin{equation} \label{b1prime}
      |b_1'(t) | \le C  |I|^{-1}  |\tilde a_k (t_0)|^{1/k}.   
  \end{equation}
\end{lemma}

\begin{proof}
  We claim that the functions $\tilde {\psi}_i {\circ} {\underline} a$ satisfy
  \begin{equation} \label{eq:B2}
    |{\partial}_t^s (\tilde {\psi}_i {\circ} {\underline} a)| \le C  |I|^{-s},
  \end{equation}
  for a constant $C$ as required in the lemma.
  Using induction, \eqref{est:a}, and differentiating the following equation $(s-1)$ times,
  \begin{align*}
    {\partial}_t (\tilde {\psi}_i {\circ} {\underline} a) = \sum_{j=2}^n (({\partial}_{j-1} \tilde {\psi}_i){\circ} {\underline} a)\, {\partial}_t \big(\tilde a_k^{- j/k} \tilde a_j\big), 
  \end{align*}
  \eqref{eq:B2} follows easily.
  (Here we used the fact that all partial derivatives of the functions $\tilde {\psi}_i$ are separately bounded and that these bounds 
  are universal.)
  Now \eqref{eq:b_ider} is a consequence of \eqref{eq:b_i} and \eqref{eq:B2}. 

  The proof of \eqref{b1prime} is analogous.
\end{proof}

\subsection{Intervals of first and second kind} \label{intervals}

Assume that the polynomial \eqref{eq:polynomial} satisfies \eqref{eq:k}--\eqref{assumpt}, \eqref{est:a}, and \eqref{splitting}--\eqref{eq:b_i}.
Suppose that $t_1 \in I$ and $\ell \in \{2,\ldots,\deg P_b\}$ are such that  
\begin{equation} \label{eq:ell}
  |\tilde b_\ell(t_1)|^{1/\ell} = \max_{2 \le i \le \deg P_b} |\tilde b_i(t_1)|^{1 /i} \ne 0.
\end{equation}
By \eqref{eq:ass12} and \eqref{eq:b_i}, for all $t \in I$ and $i = 2,\ldots,\deg P_b$,
      \begin{equation} \label{eq:b_ibound}
|\tilde b_i(t) | \le C_1  |\tilde a_k (t_0)|^{i/k},   
  \end{equation}
where the constant $C_1$ depends only on the functions $\tilde {\psi}_i$.
As an immediate consequence of \eqref{eq:b_ibound} we may conclude that we can choose  
a universal constant $D< 1/3$ and that there is an open interval $J$, $t_1 \subseteq J \subseteq I$, such that 
\begin{align}\label{assumption2a}
  | J|  |I|^{-1}  {|\tilde a_k(t_0)|^{1/k}}  + \sum_{i=2}^{\deg P_b} \|(\tilde b _i^{1/i})'\|_{L^1 (J)} =  D |\tilde b_\ell(t_1)|^{1/\ell} .
\end{align}
It suffices to choose $D < C_1^{-1}$ where $C_1$ is the constant in \eqref{eq:b_ibound}; $\tilde b _i^{1/i}$ is absolutely continuous 
by Corollary \ref{cor:radicals}. 
Let us set 
\begin{align*}
  {\varphi}_{t_1,+}(s) &:= (s-t_1) |I|^{-1} {|\tilde a_k(t_0)|^{1/k}}  + \sum_{i=2}^{\deg P_b} \|(\tilde b _i^{1/i})'\|_{L^1 ([t_1,s))}, \quad s\ge t_1, \\
  {\varphi}_{t_1,-}(s) &:= (t_1-s) |I|^{-1} {|\tilde a_k(t_0)|^{1/k}}  + \sum_{i=2}^{\deg P_b} \|(\tilde b _i^{1/i})'\|_{L^1 ((s,t_1])}, \quad s\le t_1. 
\end{align*}
Then ${\varphi}_{t_1,\pm} \ge 0$ are continuous with ${\varphi}_{t_1,\pm}(t_1) = 0$. We let ${\varphi}_{t_1,\pm}$ grow until 
${\varphi}_{t_1,-}(s_-) + {\varphi}_{t_1,+}(s_+) = D |\tilde b_\ell(t_1)|^{1/\ell}$, that is \eqref{assumption2a} with $J=(s_-,s_+)$. And we do this 
\emph{symmetrically} whenever possible:  
\begin{enumerate}
   \item[(i)] We say that the interval $J=(s_-,s_+)$ is of \emph{first kind} if 
   \begin{equation} \label{1kind}
     {\varphi}_{t_1,-}(s_-) = {\varphi}_{t_1,+}(s_+) = \frac D 2|\tilde b_\ell(t_1)|^{1/\ell}.
   \end{equation}
   \item[(ii)] If \eqref{1kind} is not possible, i.e., we reach the boundary of the interval $I$ before either ${\varphi}_{t_1,-}$ or ${\varphi}_{t_1,+}$ has 
   grown to the value $(D/2)|\tilde b_\ell(t_1)|^{1/\ell}$, then we say that $J=(s_-,s_+)$ is of \emph{second kind}.  
\end{enumerate} 

\subsection{Estimates after splitting on \texorpdfstring{$J$}{J}}

\begin{lemma} \label{lemB}
  Assume that the polynomial \eqref{eq:polynomial} satisfies \eqref{eq:k}--\eqref{assumpt}, \eqref{est:a}, 
  \eqref{splitting}--\eqref{eq:b_i}, and \eqref{eq:ell}. 
  Let $D$ and $J$ be as in \eqref{assumption2a}. 
  Then the functions $\tilde b_i$ on $J$ satisfy the conclusions of Lemmas \ref{lem1}, \ref{lem2}, and \ref{bounds2}. More precisely, 
  for all $t \in J$ and $i = 2,\ldots,\deg P_b$,
  \begin{gather}  
    |\tilde b_i^{1/i} (t) - \tilde b_i^{1/i} (t_1)| \le  D |\tilde b_\ell (t_1)|^{1/\ell}, \label{b1} \\ 
    \frac 2 3 <     1-D \le \Big | \frac{\tilde b_\ell(t)}{\tilde b_\ell(t_1)} \Big |^{1/\ell}   \le 1+D < \frac 4 3,  \label{b2}\\
    |\tilde b_i(t)|^{1/i}\le \frac 4 3 |\tilde b_\ell(t_1)|^{1/\ell} \le 2 |\tilde b_\ell(t)|^{1/\ell}. \label{b3}
  \end{gather}  
  The length of the curve 
  \begin{equation}
    J \ni t \mapsto {\underline} b(t) := (\tilde b_\ell^{-2/\ell} \tilde b_2, \ldots, \tilde b_\ell^{-\deg P_b/\ell} \tilde b_{\deg P_b})(t)
  \end{equation}
  is bounded by $3 (\deg P_b)^2\, 2^{\deg P_b} D$.  
  There is a universal constant $C$, depending only on $n$ and $\tilde {\psi}_i$, such that for all $t\in J$, 
  $ i = 2,\ldots,\deg P_b$, and  $ s = 1,\ldots,n$,  
    \begin{align}\label{b4}  
    |\tilde b_i^{(s)}(t) | \le C |J|^{-s}  |\tilde b_\ell (t_1)|^{i/\ell}.  
  \end{align}
\end{lemma}

\begin{proof}
  The proof of \eqref{b1}--\eqref{b3} is analogous to the proof of Lemma \ref{lem1}; use \eqref{eq:ell} and \eqref{assumption2a} 
  instead of \eqref{eq:k} and \eqref{assumpt}. The bound for the length of the curve $J \ni t \mapsto {\underline} b(t)$ (which is 
  well-defined by \eqref{b2}) 
  follows from \eqref{assumption2a} and \eqref{b1}--\eqref{b3}; see the proof of Lemma \ref{lem2}. 

  Let us prove \eqref{b4}.
  By \eqref{eq:b_ider}, for $t\in I$ and $i = 2,\ldots,\deg P_b$,
  \begin{equation} \label{eq:b_iderj} 
    |\tilde b_i^{(i)} (t) | \le C  |I|^{-i} |\tilde a_k(t_0)|^{i/k}, 
  \end{equation}
  where $C = C(n,\tilde {\psi}_i)$.
  Thus, for $t\in J$ and $s=1, ..., i$,
  \begin{align*}
  |\tilde b_i^{(s)}(t) | 
    &\le C | J |^{-s} \bigl (  V_J(\tilde b_i)  + V_J(\tilde b_i)^{(i-s)/i} \|\tilde b_i^{(i)}\|^{s/i}_{L^\infty(J)} |J|^s \bigr) 
    \hspace{21mm} \text{by Lemma \ref{taylor}}\\
    &\le C_1 | J |^{-s} \Bigl (  |\tilde b_\ell(t_1)|^{i/\ell}  + |\tilde b_\ell(t_1)|^{(i-s)/\ell} |J|^s |I|^{-s} |\tilde a_k(t_0)|^{s/k} \Bigr) 
    \hspace{5mm} \text{by \eqref{b3} and \eqref{eq:b_iderj}} \\  
  &\le C_2  | J |^{-s}  |\tilde b_\ell(t_1)|^{i/\ell}  \hspace{69mm} \text{by \eqref{assumption2a}}, 
  \end{align*}
  for constants $C = C(i)$ and $C_i = C_i(n,\tilde {\psi}_i)$.
  For $s>i$, \eqref{eq:b_ider} implies 
   \begin{align*}
    |\tilde b_i^{(s)}(t) |  & \le   C  | I |^{-s}  |\tilde a_k(t_0)|^{i/k}    
    = C   | J |^{-s}  \big(|J| |I|^{-1} \big)^s  |\tilde a_k(t_0)|^{i/k}  \\
    &    \le C   | J |^{-s} \big(|J| |I|^{-1} |\tilde a_k(t_0)|^{1/k} \big)^i  
    \le C    | J |^{-s}  |\tilde b_\ell(t_1)|^{i/ \ell},  
  \end{align*}
  where the last inequality follows from \eqref{assumption2a}.
  Thus \eqref{b4} is proved.
\end{proof} 

\section{A special cover by intervals} \label{specialcover}

Assume that the polynomial \eqref{eq:polynomial} satisfies \eqref{eq:k}--\eqref{assumpt}, \eqref{est:a}, and \eqref{splitting}--\eqref{eq:b_i}. 
The arguments in Section \ref{intervals} show that for each point $t_1$ in  
\[
  I' := I \setminus \{t \in I : \tilde b_2 (t) = \cdots = \tilde b_{\deg P_b}(t) = 0\}
\]
there exists $\ell \in \{2, \ldots, \deg P_b\}$ such that \eqref{eq:ell} 
and there is an open interval $J = J(t_1)$, $t_1 \in J \subseteq I'$, such that \eqref{assumption2a}; that $J \subseteq I'$ 
follows from \eqref{b2}.

The goal of this section is to prove the following proposition.

\begin{proposition} \label{cover}
  The collection $\{J(t_1)\}_{t_1\in I'}$ has a countable subcollection ${\mathcal{J}}$ that still 
  covers $I'$ and such that every point in $I'$ belongs to at most two intervals in ${\mathcal{J}}$. 
  In particular, 
  \[
    \sum_{J \in {\mathcal{J}}} |J| \le 2 |I'|.
  \]  
\end{proposition}

\begin{remark}
   It is essential for us that ${\mathcal{J}}$ is a subcollection and not a refinement; by shrinking the intervals we would lose equality in  
   \eqref{assumption2a}. 
\end{remark} 

We can treat the connected components of $I'$ separately. 
So let $({\alpha},{\beta})$ be any connected component of $I'$ and let ${\mathcal{I}} := \{J(t_1)\}_{t_1 \in ({\alpha},{\beta})}$. 
The coefficients $\tilde b_i$ may or may not all vanish at the endpoints.
We distinguish 
three cases:
\begin{enumerate}
  \item[(i)] $\tilde b$ vanishes at both endpoints,
  \begin{equation}
     \label{1case}
    \sum_{i=2}^{\deg P_b} |\tilde b_i({\alpha})|^{1/i}  = \sum_{i=2}^{\deg P_b} |\tilde b_i({\beta})|^{1/i} = 0. 
  \end{equation}
  \item[(ii)] $\tilde b$ vanishes at one endpoint, say ${\alpha}$, but not at the other, 
  \begin{equation}
     \label{2case}
    \sum_{i=2}^{\deg P_b} |\tilde b_i({\alpha})|^{1/i}   = 0, ~\sum_{i=2}^{\deg P_b} |\tilde b_i({\beta})|^{1/i} \ne 0.
   \end{equation} 
   \item[(iii)] $\tilde b$ does not vanishes at either endpoint,
   \begin{equation}
    \label{3case}
    \sum_{i=2}^{\deg P_b} |\tilde b_i({\alpha})|^{1/i}   \ne 0, ~\sum_{i=2}^{\deg P_b} |\tilde b_i({\beta})|^{1/i} \ne 0. 
   \end{equation}
\end{enumerate}

\begin{lemma} \label{endpoints} We have:
\begin{enumerate}
  \item If $\tilde b({\alpha}) = 0$, then no interval $J \in {\mathcal{I}}$ has left endpoint ${\alpha}$ and $|J(t_1)| \to 0$ as $t_1 \to {\alpha}$. 
  If $\tilde b({\beta}) = 0$, then no interval $J \in {\mathcal{I}}$ has right endpoint ${\beta}$ and $|J(t_1)| \to 0$ as $t_1 \to {\beta}$. 
  \item If $\tilde b({\alpha}) \ne 0$, then there exists an interval $J \in {\mathcal{I}}$ of second kind (with endpoint ${\alpha}$). 
  If $\tilde b({\beta}) \ne 0$, then there exists an interval $J \in {\mathcal{I}}$ of second kind (with endpoint ${\beta}$).
\end{enumerate}
\end{lemma}

\begin{proof}
  (1) By \eqref{b2}, $\tilde b$ cannot vanish at the endpoints of $J$. 
  That $|J(t_1)| \to 0$ as $t_1$ tends to an endpoint, where $\tilde b$ vanishes, is immediate from \eqref{assumption2a}.

  (2) Suppose that $\tilde b({\beta}) \ne 0$. If all intervals $J(t_1)$ in ${\mathcal{I}}$ were of first kind then, by \eqref{assumption2a} and \eqref{1kind}, 
  \begin{equation} \label{endpoints1}
    {\varphi}_{t_1,+}({\beta}) \ge \frac{D}{2} |\tilde b_\ell(t_1)|^{1/\ell} 
    = \frac{D}{2} \max_{2 \le i \le \deg P_b} |\tilde b_i(t_1)|^{1/i}, \quad t_1 \in ({\alpha},{\beta}).
  \end{equation}
  But ${\varphi}_{t_1,+}({\beta}) \to 0$ as $t_1 \to {\beta}$, while the right-hand side of \eqref{endpoints1} tends to a positive constant, a contradiction. 
\end{proof}

\begin{lemma} \label{lem:interlace}
  Suppose that $J \in {\mathcal{I}}$ and $t_1 \not\in J$ such that $J(t_1)$ is of first kind. 
  Then $J \not \subseteq J(t_1)$.
\end{lemma}

\begin{proof}
  Let $J=J(s_1) = ({\alpha}_{s_1},{\beta}_{s_1})$ and assume without loss of generality that ${\beta}_{s_1} \le t_1$. 
  Suppose that $J(s_1) \subseteq J(t_1)$.
  Since $J(t_1)=({\alpha}_{t_1},{\beta}_{t_1})$ is of first kind (cf.\ \eqref{1kind}), we have 
  \begin{align*}
    (t_1 -{\alpha}_{t_1})  |I|^{-1}  {|\tilde a_k(t_0)|^{1/k}}  + \sum_{i=2}^{\deg P_b} \|(\tilde b_i^{1/i})'\|_{L^1 (({\alpha}_{t_1},t_1])} 
    =  \frac{D}{2} |\tilde b_{\ell_{t_1}}(t_1)|^{1/\ell_{t_1}} 
    < D |\tilde b_{\ell_{s_1}}(s_1)|^{1/\ell_{s_1}},
  \end{align*}
  because by \eqref{b2} and \eqref{b3},
  \[
    |\tilde b_{\ell_{t_1}}(t_1)|^{1/\ell_{t_1}} < \frac 3 2 |\tilde b_{\ell_{t_1}}(s_1)|^{1/\ell_{t_1}} 
    \le 2 |\tilde b_{\ell_{s_1}}(s_1)|^{1/\ell_{s_1}}. 
  \]
  But this leads to a contradiction in view of \eqref{assumption2a}.
\end{proof}

\subsection*{Case (i)}
  By \eqref{1case} and Lemma \ref{endpoints}, 
  each $J \in {\mathcal{I}}$ is an interval of first kind. 

  Choose any interval $J(t_1)$, $t_1 \in ({\alpha},{\beta})$, and denote it by $J_0 = ({\alpha}_0,{\beta}_0)$. 
  Define recursively (for ${\gamma} \in {\mathbb{Z}}$)
  \begin{align*}
    J_{\gamma} = ({\alpha}_{\gamma},{\beta}_{\gamma}) := 
    \begin{cases}
      J({\beta}_{{\gamma}-1}) & \text{if } {\gamma} \ge 1, \\
      J({\alpha}_{{\gamma}+1}) & \text{if } {\gamma} \le -1.  
    \end{cases} 
  \end{align*} 
  By Lemma \ref{lem:interlace}, we have ${\alpha} < {\alpha}_{\gamma} < {\alpha}_{{\gamma}+1}$ and ${\beta}_{\gamma} < {\beta}_{{\gamma}+1} < {\beta}$ for all ${\gamma}$.
  Let us show that 
  the collection ${\mathcal{J}} = \{J_{\gamma}\}_{{\gamma} \in {\mathbb{Z}}}$ covers $({\alpha},{\beta})$.   
  Suppose that, say,
  ${\tau} := \sup_{\gamma} {\beta}_{\gamma} <{\beta}$.  
  By \eqref{assumption2a} and since all intervals are of first kind (cf.\ \eqref{1kind}),
  \begin{align*}
    ({\tau} -{\beta}_{\gamma})  |I|^{-1}  {|\tilde a_k(t_0)|^{1/k}}  + \sum_{i=2}^{\deg P_b} \|(\tilde b_i^{1/i})'\|_{L^1 (({\beta}_{\gamma},{\tau}])} 
    \ge   
    \frac{D}{2} \max_{2 \le i \le \deg P_b} |\tilde b_i({\beta}_{\gamma})|^{1/i}. 
  \end{align*}
  But the left-hand side tends to $0$ as ${\gamma} \to +\infty$, whereas the right-hand side converges to 
  $(D/2) \max_{2 \le i \le \deg P_b} |\tilde b_i({\tau})|^{1/i}>0$, a contradiction. 

   
  

  Now Proposition \ref{cover} follows from Lemma \ref{endpoints} and the following lemma.

  \begin{lemma} \label{glue}
  Let ${\mathcal{J}} = \{J_{\gamma}\}_{{\gamma} \in {\mathbb{Z}}}$ be a countable collection of bounded open intervals $J_{\gamma} = ({\alpha}_{\gamma},{\beta}_{\gamma}) \subseteq {\mathbb{R}}$ such that 
  \begin{enumerate}
    \item $\bigcup {\mathcal{J}} = ({\alpha},{\beta})$ is a bounded open interval,
    \item ${\alpha} < {\alpha}_{\gamma} < {\alpha}_{{\gamma}+1}$ and ${\beta}_{\gamma} < {\beta}_{{\gamma}+1}< {\beta}$ for all ${\gamma} \in {\mathbb{Z}}$, 
    \item $|J_{\gamma}| \to 0$ as ${\gamma} \to \pm \infty$.
  \end{enumerate}
  Then there is a subcollection ${\mathcal{J}}_0 \subseteq {\mathcal{J}}$ with 
  $\bigcup {\mathcal{J}}_0 = ({\alpha},{\beta})$ and such that every point in $({\alpha},{\beta})$ 
  belongs to at most two intervals in ${\mathcal{J}}_0$. 
\end{lemma}

\begin{proof}
  The assumptions imply that the sequence of left endpoints $({\alpha}_{\gamma})$ converges to ${\beta}$ as ${\gamma} \to \infty$,
  and the sequence of right endpoints $({\beta}_{\gamma})$ converges to ${\alpha}$ as ${\gamma} \to -\infty$. 
  Thus, there exists ${\gamma}_1 > 0$ such that ${\alpha}_{{\gamma}_1} < {\beta}_0 \le {\alpha}_{{\gamma}_1 + 1}$, 
  there exists ${\gamma}_2 > {\gamma}_1$ such that ${\alpha}_{{\gamma}_2} < {\beta}_{{\gamma}_1} \le {\alpha}_{{\gamma}_2 + 1}$, and iteratively, 
  there exists ${\gamma}_j > {\gamma}_{j-1}$ such that ${\alpha}_{{\gamma}_j} < {\beta}_{{\gamma}_{j-1}} \le {\alpha}_{{\gamma}_j + 1}$. 
  Symmetrically,  
  there exist integers ${\gamma}_{j-1} < {\gamma}_{j} < 0$ ($j \in {\mathbb{Z}}_{<0}$) such that ${\beta}_{{\gamma}_{j-1}-1} \le {\alpha}_{{\gamma}_j} < {\beta}_{{\gamma}_{j-1}}$.
  Set ${\gamma}_0 := 0$ and define 
  \[
    {\mathcal{J}}_0 := \{J_{{\gamma}_j}\}_{j \in {\mathbb{Z}}}.
  \]
  By construction ${\mathcal{J}}_0$ still covers $({\alpha},{\beta})$ and the left and right endpoints of the intervals $J_{{\gamma}_j}$ are interlacing,
  \[
    \cdots < {\beta}_{{\gamma}_{j-2}}< {\alpha}_{{\gamma}_j} < {\beta}_{{\gamma}_{j-1}} < {\alpha}_{{\gamma}_{j+1}} < {\beta}_{{\gamma}_{j}} < {\alpha}_{{\gamma}_{j+2}} < \cdots 
  \]
  Thus ${\mathcal{J}}_0$ has the required properties.
\end{proof}

\subsection*{Case (ii)} By \eqref{2case} and Lemma \ref{endpoints}, the collection ${\mathcal{I}}$ contains an interval of second kind. 
  Since $\tilde b({\alpha}) = 0$, all intervals of second kind in ${\mathcal{I}}$ must have endpoint ${\beta}$. 
  Thus, and because $|J(t_1)| \to 0$ as $t\to {\alpha}$ by Lemma \ref{endpoints},  
  \[
    {\tau} := \inf\{t_1 : J(t_1) \in {\mathcal{I}} \text{ is of second kind}\} > {\alpha}.
  \] 
  The interval $J({\tau})$ is of first kind (being of second kind is an open condition).  
  There is an interval $J_0 = ({\alpha}_0,{\beta}_0={\beta})$ of second kind in ${\mathcal{I}}$ with $J({\tau}) \cap J_0 \ne \emptyset$. 
  Let us denote $J({\tau})$ by $J_{-1} = ({\alpha}_{-1},{\beta}_{-1})$ and define recursively  
  \begin{align*}
    J_{\gamma} = ({\alpha}_{\gamma},{\beta}_{\gamma}) := 
      J({\alpha}_{{\gamma}+1}), \quad {\gamma} \le -1. 
  \end{align*} 
  The arguments in Case (i) imply that the collection ${\mathcal{J}}:= \{J_{\gamma}\}_{{\gamma} \le 0}$ is a countable cover of $({\alpha},{\beta})$ 
  satisfying ${\alpha} < {\alpha}_{\gamma} < {\alpha}_{{\gamma}+1}$ and $|J_{\gamma}| \to 0$.

  Proposition \ref{cover} follows from (an obvious modification of) Lemma \ref{glue}.

  \subsection*{Case (iii)}
  In this case ${\mathcal{I}}$ has a finite subcollection ${\mathcal{J}}$ that still covers $({\alpha},{\beta})$. Indeed,
  by \eqref{3case} and Lemma \ref{endpoints}, the collection ${\mathcal{I}}$ contains intervals of second kind with endpoints ${\alpha}$ and ${\beta}$, 
  say, $({\alpha},{\delta})$ and $({\epsilon},{\beta})$. If their intersection is non-empty we are done. Otherwise there are finitely many   
  intervals in ${\mathcal{I}}$ that cover the compact interval $[{\delta},{\epsilon}]$.

  Proposition \ref{cover} follows from the following lemma.  

  \begin{lemma}
    Every finite collection ${\mathcal{J}}$ of open intervals with $\bigcup {\mathcal{J}} = ({\alpha},{\beta})$ 
    has a subcollection that still covers $({\alpha},{\beta})$ and every point in $({\alpha},{\beta})$ belongs to at most two 
    intervals in the subcollection.  
  \end{lemma}

  \begin{proof}
    The collection ${\mathcal{J}}$ contains an interval with endpoint ${\alpha}$; let $J_0 = ({\alpha} = {\alpha}_0,{\beta}_0)$ be the biggest among them. 
    If ${\beta}_0 < {\beta}$,
    let $J_1 = ({\alpha}_1,{\beta}_1)$ denote the interval among all intervals in ${\mathcal{J}}$ containing ${\beta}_0$ whose right endpoint is maximal.
    If ${\beta}_1 < {\beta}$,
    let $J_2 = ({\alpha}_2,{\beta}_2)$ denote the interval among all intervals in ${\mathcal{J}}$ containing ${\beta}_0$ whose right endpoint is maximal, etc. 
    This yields a finite cover of $({\alpha},{\beta})$ by intervals $J_i = ({\alpha}_i,{\beta}_i)$, $i = 0,1,\ldots,N$, such that ${\alpha}_0 < {\alpha}_1 < \cdots < {\alpha}_N$.
    Define 
    \[
      i_1 := \max_{{\alpha}_i <{\beta}_0} i, \qquad i_j := \max_{{\alpha}_i <{\beta}_{i_{j-1}}} i,  \quad j \ge 2.  
    \]
    Then $\{J_0,J_{i_1},J_{i_2},\ldots,J_N\}$ has the required properties.
  \end{proof}

\section{Proof of Theorem \ref{main}} \label{proof}

Let $({\alpha},{\beta}) \subseteq {\mathbb{R}}$ be a bounded open interval and let
\begin{equation} \label{polynomial}
  P_a(t)(Z)= P_{a(t)}(Z) = Z^n + \sum_{j=1}^n a_j(t) Z^{n-j}, \quad t \in ({\alpha},{\beta}),
\end{equation}
be a monic polynomial with coefficients $a_j \in C^{n-1,1}([{\alpha},{\beta}])$, $j = 1,\ldots,n$. 

The proof of Theorem \ref{main} is by induction on the degree of the polynomial. We will reduce the degree by splitting $P_a$ locally. 
Since the first splitting is atypical we shall consider subsequent splittings before we apply the inductive hypothesis. 

\subsection{Reduction to Tschirnhausen form}

Without loss of generality we may assume that $n \ge 2$ and that $P_a = P_{\tilde a}$ is in Tschirnhausen form, i.e., $\tilde a_1=0$.
We shall see in Section \ref{secbound} how to get the bound \eqref{bound} from a corresponding bound involving the $\tilde a_j$. 

Let $\{{\lambda}_j(t)\}_{j=1}^n$, $t \in ({\alpha},{\beta})$, be any system of the roots of $P_{\tilde a}$ (not necessarily continuous). 
Since $P_{\tilde a}$ is in Tschirnhausen form, for fixed $t \in ({\alpha},{\beta})$,
\begin{equation} \label{zero}
  \forall_{i,j} ~{\lambda}_i(t) = {\lambda}_j(t) \quad \Longleftrightarrow \quad 
  \forall_i ~{\lambda}_i(t) =0 \quad \Longleftrightarrow \quad 
  \forall_i ~\tilde a_i(t) =0.
\end{equation}

\subsection{Universal splitting of \texorpdfstring{$P_{\tilde a}$}{Pa}} \label{universal}

The space of monic polynomials of degree $n$ in Tschirnhausen form can be identified with ${\mathbb{C}}^{n-1}$; let the coordinates in 
${\mathbb{C}}^{n-1}$ be 
denoted by ${\underline} a_2,{\underline} a_3, \ldots,{\underline} a_n$. The set 
\[
  K:= \bigcup_{k=2}^n\{({\underline} a_2, \ldots,{\underline} a_n) \in {\mathbb{C}}^{n-1} : {\underline} a_k=1,~ |{\underline} a_j| \le 1 \text{ for }j \ne k\}
\]
is compact. For each point $p \in K$
there exists ${\rho}_p>0$ such that 
$P_{\tilde a}$ splits on the open ball $B_{{\rho}_p}(p)$; cf.\ Section \ref{ssec:split}.
Choose a finite subcover of $K$ by open balls $B_{{\rho}_{\delta}}(p_{\delta})$, ${\delta} \in {\Delta}$. 
Then there exists ${\rho}>0$ so that for every $p \in K$ there is a ${\delta} \in {\Delta}$ such that $B_{\rho}(p) \subseteq B_{{\rho}_{\delta}}(p_{\delta})$. 
Fix a universal positive constant $B$ satisfying 
\begin{gather} \label{eq:constB}
  B < \min\Big\{\frac1 3, \frac{\rho}{3 n^2 2^{n}}\Big\}. 
\end{gather}

\subsection{First splitting} \label{first}

Fix $t_0 \in ({\alpha},{\beta})$ and $k \in \{2,\ldots,n\}$ such that \eqref{eq:k} holds, i.e., 
\begin{equation} \label{k}
  |\tilde a_k(t_0)|^{1/k} =  \max_{2 \le j \le n} |\tilde a_j(t_0)|^{1/j}\ne 0
\end{equation}
This is possible unless $\tilde a \equiv 0$ in which case nothing is to prove. 
Choose a maximal open interval $I \subseteq ({\alpha},{\beta})$ containing $t_0$ such that we have \eqref{assumption}, i.e., 
\begin{align}\label{assumption<=}
{M} |I| + \sum_{j=2}^n \|(\tilde a_j^{1/j})'\|_{L^1 (I)} \le  B |\tilde a_k(t_0)|^{1/k},
\end{align}
with ${M}$ given by \eqref{M}.

In particular, all conclusions of Section \ref{aestimates} hold true.

Consider the point $p = {\underline} a(t_0)$, where ${\underline} a$ is the curve defined in \eqref{curve}. By \eqref{k}, $p \in K$ and 
thus there exists ${\delta} \in {\Delta}$ such that $B_{\rho}(p) \subseteq B_{{\rho}_{\delta}}(p_{\delta})$. By Lemma \ref{lem2} and by \eqref{eq:constB}, 
the length of the curve ${\underline} a|_I$ is bounded by ${\rho}$. It follows that we have a splitting on $I$, 
\[
  P_{\tilde a}(t) = P_b(t) P_{b^*}(t), \quad t \in I.
\] 
By \eqref{eq:bj}, the coefficients $b_i$ of $P_b$ are of the form 
\[
  b_i = \tilde a_k^{i/k} {\psi}_i \big(\tilde a_k^{-2/k} \tilde a_2, \ldots, \tilde a_k^{-n/k} \tilde a_n\big), \quad i = 1,\ldots, \deg P_b,
\]  
and after Tschirnhausen transformation $P_b \leadsto P_{\tilde b}$, see \eqref{eq:tildebj}, 
\[
  \tilde b_i = \tilde a_k^{i/k} \tilde {\psi}_i \big(\tilde a_k^{-2/k} \tilde a_2, \ldots, \tilde a_k^{-n/k} \tilde a_n\big), 
  \quad i = 2,\ldots, \deg P_b,
\] 
where ${\psi}_i$, respectively, $\tilde {\psi}_i$, are analytic functions all whose partial derivatives are separately bounded on $B_{\rho}(p)$.
(Similar formulas hold for $b_i^*$ and $\tilde b_i^*$.)

In summary, the restriction of the curve of polynomials $P_{\tilde a}$ to the interval $I$ satisfies all assumptions and thus all conclusions of  
Sections \ref{aestimates} and \ref{bestimates}.

It follows that the assumptions of the following proposition are satisfied.

\begin{proposition} \label{induction}
  Let $I \subseteq {\mathbb{R}}$ be a bounded open interval and 
  let $P_{\tilde a}$ be a monic polynomial in Tschirnhausen form with coefficients of class $C^{\deg P_{\tilde a}-1,1}({\overline I})$.
  Let $t_0 \in I$ and $k \in \{2,\ldots, \deg P_{\tilde a}\}$ be such that 
  \begin{enumerate}
    \item $|\tilde a_k(t_0)|^{1/k} =  \max_{2 \le j\le \deg P_{\tilde a}} |\tilde a_j(t_0)|^{1/j}\ne 0$,
    \item $\sum_{j=2}^{\deg P_{\tilde a}} \|(\tilde a_j^{1/j})'\|_{L^1(I)} \le B |\tilde a_k(t_0)|^{1/k}$ for some constant $B< 1/3$, 
    \item $|\tilde a_j^{(s)}(t) | \le C |I|^{-s}  |\tilde a_k (t_0)|^{j/k}$ for all $t \in I$, $j = 2, \ldots,\deg P_{\tilde a}$, and 
    $s = 1,\ldots,\deg P_{\tilde a}$, and some constant $C= C(\deg P_{\tilde a})$.
    \item Assume that $P_{\tilde a}$ splits on $I$, i.e., $P_{\tilde a}(t) = P_b(t) P_{b^*}(t)$ for $t \in I$.
  \end{enumerate}
  Then every continuous root $\mu \in C^0(I)$ of $P_{\tilde b}$ is absolutely continuous and satisfies
  \begin{equation} \label{muab}
    \|\mu'\|_{L^p(I)} \le C  \Big( \| |I|^{-1}  {|\tilde a_k(t_0)|^{1/k}} \|_{L^p (I)} 
    + \sum_{i=2}^{\deg P_b} \|(\tilde b_i^{1/i})'\|_{L^p (I)}\Big),
  \end{equation}   
  for all $1 \le p < (\deg P_{\tilde a})'$ and a constant $C$ depending only on $\deg P_{\tilde a}$ and $p$.  
\end{proposition}

In this proposition and from now on we apply the following convention:
\begin{quote}
  \it Any dependence of constants on parameters of the universal splitting, like ${\rho}$, $\tilde {\psi}_i$, etc., will no longer 
      be explicitly stated. For simplicity it will henceforth be subsumed by saying that the constants depend on the degree of 
      the polynomials. Universal constants will be denoted by $C$ and may vary from line to line.   
\end{quote}

We shall prove this proposition by induction on the degree.  
The assumptions of the proposition amount exactly to the assumptions \eqref{eq:polynomial}--\eqref{assumpt}, \eqref{est:a}, and 
\eqref{splitting}--\eqref{eq:b_i}. Thus we may rely on all conclusions of Sections \ref{aestimates} and \ref{bestimates}. 

\subsection{Second splitting} \label{second}

By \eqref{eq:ass11}, $\tilde a_k$ does not vanish on $I$, and thus $b_i$ and $\tilde b_i$ belong to $C^{n-1,1}({\overline I})$.
Let us set 
\[
  I' := I \setminus \{t \in I : \tilde b_2 (t) = \cdots = \tilde b_{\deg P_b}(t) = 0\}.
\]
For each $t_1 \in I'$ there is $\ell \in \{2, \ldots, \deg P_b\}$ such that \eqref{eq:ell} holds, i.e.,
\[
  |\tilde b_\ell(t_1)|^{1/\ell} = \max_{2 \le i \le \deg P_b} |\tilde b_1(t_1)|^{1/i} \ne 0,
\]
and, by Section \ref{intervals}, there is an open interval $J = J(t_1)$, $t_1 \in J \subseteq I'$, such that \eqref{assumption2a}, i.e., 
\begin{align*}
    | J|  |I|^{-1}  {|a_k(t_0)|^{1/k}}  + \sum_{i=2}^{\deg P_b} \|(\tilde b _i^{1/i})'\|_{L^1 (J)} =  D |\tilde b_\ell(t_1)|^{1/\ell} .
\end{align*}
The universal constant $D$ can be chosen sufficiently small such that on $J$ we have a splitting 
\begin{equation*}
  P_{\tilde b}(t) = P_c(t) P_{c^*}(t), \quad t \in J;
\end{equation*}
in fact, it suffices to choose 
\begin{equation} \label{D}
  D < \min\Big\{\frac 1 3, \frac{\sigma}{3 (\deg P_b)^2 2^{\deg P_b}}, C_1^{-1}\Big\}, 
\end{equation}
where $C_1$ is the constant in \eqref{eq:b_ibound} and where ${\sigma}$ is the analogue of ${\rho}$ in Section \ref{universal}
for a cover of 
\[
  \bigcup_{\ell=2}^{\deg P_b}\{({\underline} b_2, \ldots,{\underline} b_{\deg P_b}) \in {\mathbb{C}}^{\deg P_b-1} 
    : {\underline} b_\ell=1,~ |{\underline} b_i| \le 1 \text{ for }i \ne \ell\}, \quad {\underline} b_i := \tilde b_\ell^{-i/\ell} \tilde b_i. 
\]
This follows from Lemma \ref{lemB} and 
the arguments in Sections \ref{universal} and \ref{first} applied to $P_{\tilde b}$. 

By Proposition \ref{cover}, 
we may conclude that there is a countable family $\{J_{\gamma}\}$ of open intervals $J_{\gamma} \subseteq I'$, of points $t_{\gamma} \in J_{\gamma}$, 
and of integers $\ell_{\gamma} \in \{2, \ldots, \deg P_b\}$ 
satisfying
\begin{gather}
    |\tilde b_{\ell_{\gamma}}(t_{\gamma})|^{1/{\ell_{\gamma}}} = \max_{2 \le i \le \deg P_b} |\tilde b_i(t_{\gamma})|^{1/i} \ne 0, \\
    |J_{\gamma}|  |I|^{-1}  {|\tilde a_k(t_0)|^{1/k}}  + \sum_{i=2}^{\deg P_b} \|(\tilde b_i^{1/i})'\|_{L^1 (J_{\gamma})} =  
    D |\tilde b_{\ell_{\gamma}}(t_{\gamma})|^{1/\ell_{\gamma}}, \label{Jba}\\ 
    P_{\tilde b}(t) = P_{c_{\gamma}}(t) P_{c_{\gamma}^*}(t), \quad t \in J_{\gamma}, \label{Pb}\\
    \bigcup_{\gamma} J_{\gamma} = I', \quad \sum_{\gamma} |J_{\gamma}| \le 2 |I'|. \label{Jga}
\end{gather}

In particular, for every ${\gamma}$, the polynomial $P_{\tilde b}(t) = P_{c_{\gamma}}(t) P_{c_{\gamma}^*}(t)$, $t \in J_{\gamma}$, satisfies the 
assumptions of Proposition \ref{induction}; indeed, (3) corresponds to \eqref{b4}. 

\subsection{Inductive step}

Let $\mu \in C^0(I)$ be a continuous root of $P_{\tilde b}$. 
We may assume without loss of generality that in $J_{\gamma}$, 
\begin{equation} \label{tildemu}
  \tilde \mu(t) := \mu(t) + \frac{c_{{\gamma} 1}(t)}{\deg P_{c_{\gamma}}}, \quad t \in J_{\gamma},
\end{equation}
is a root of $P_{\tilde c_{\gamma}}$. 
Since $\deg P_{\tilde c_{\gamma}} < \deg P_{\tilde b} < \deg P_{\tilde a}$, the induction hypothesis implies that
$\tilde \mu$ is absolutely continuous and satisfies 
\begin{equation} \label{mu}
    \|\tilde \mu'\|_{L^p(J_{\gamma})} \le C \Big( \| |J_{\gamma}|^{-1}  {|\tilde b_{\ell_{\gamma}}(t_{\gamma})|^{1/\ell_{\gamma}}} \|_{L^p (J_{\gamma})} 
  + \sum_{h=2}^{\deg P_{c_{\gamma}}} \|(\tilde c_{{\gamma} h}^{1/h})'\|_{L^p (J_{\gamma})}\Big),
\end{equation}
for all $1 \le p < (\deg P_b)'$, for a constant $C$ depending only on $\deg P_b$ and $p$.

\subsection{\texorpdfstring{$L^p$}{Lp}-estimates on \texorpdfstring{$I$}{I}}

By Section \ref{ssec:split}, the coefficients $c_{{\gamma} h}$ of $P_{c_{\gamma}}$ are of the form 
\[
  c_{{\gamma} h} = \tilde b_{\ell_{\gamma}}^{h/\ell_{\gamma}} 
  {\theta}_h \big(\tilde b_{\ell_{\gamma}}^{-2/\ell_{\gamma}} \tilde b_2, \ldots, \tilde b_{\ell_{\gamma}}^{-\deg P_b/\ell_{\gamma}} \tilde b_{\deg P_b}\big), 
  \quad h = 1,\ldots, \deg P_{c_{\gamma}},
\]  
and after Tschirnhausen transformation $P_{c_{\gamma}} \leadsto P_{\tilde c_{\gamma}}$, see \eqref{eq:tildebj}, 
\[
  \tilde c_{{\gamma} h} = \tilde b_{\ell_{\gamma}}^{h/\ell_{\gamma}} 
  \tilde {\theta}_h \big(\tilde b_{\ell_{\gamma}}^{-2/\ell_{\gamma}} \tilde b_2, \ldots, \tilde b_{\ell_{\gamma}}^{-\deg P_b/\ell_{\gamma}} \tilde b_{\deg P_b}\big), 
  \quad h = 2,\ldots, \deg P_{c_{\gamma}},
\] 
where ${\theta}_h$, respectively, $\tilde {\theta}_h$, are analytic functions all whose partial derivatives are separately bounded.
(Similar formulas hold for $c_{{\gamma} h}^*$ and $\tilde c_{{\gamma} h}^*$.) 
By \eqref{b2}, $\tilde b_{\ell_{\gamma}}$ does not vanish on $J_{\gamma}$ and thus $c_{{\gamma} h}$ and $\tilde c_{{\gamma} h}$ belong to 
$C^{\deg P_{\tilde a}-1,1}(\overline J_{\gamma})$.
Analogously to \eqref{eq:b_ider} we find that, for $t \in J_{\gamma}$, $h = 2,\ldots, \deg P_{c_{\gamma}}$, and $s=1,\dots,\deg P_{\tilde a}$,
\[
  |\tilde c_{{\gamma} h}^{(s)}(t) | \le C  |J_{\gamma}|^{-s}  |\tilde b_{\ell_{\gamma}} (t_{\gamma})|^{h/\ell_{\gamma}},
\]
where $C=C(\deg P_{\tilde a})$.
Together with \eqref{est}, it implies 
\begin{align*} 
\|(\tilde c_{{\gamma} h}^{1/h})'\|_{h',w,J_{\gamma}} &\le C(h) \max\Big\{\big(\on{Lip}_{J_{\gamma}}(\tilde c_{{\gamma} h}^{(h-1)})\big)^{1/h}|J_{\gamma}|^{1/h'}, 
    \|\tilde c_{{\gamma} h}'\|_{L^\infty(J_{\gamma})}^{1/h}\Big\} \\
    &\le C  |J_{\gamma}|^{-1 +1/h'} |\tilde b_{\ell_{\gamma}} (t_{\gamma})|^{1/\ell_{\gamma}},
\end{align*} 
where the constant $C$ depends only on $\deg P_{\tilde a}$.
Thus,
\begin{equation*}
  \|(\tilde c_{{\gamma} h}^{1/h})'\|^*_{h',w,J_{\gamma}} \le C  |J_{\gamma}|^{-1} |\tilde b_{\ell_{\gamma}} (t_{\gamma})|^{1/\ell_{\gamma}},
\end{equation*}
and so, in view of \eqref{inclusions}, for all $p$, $1 \le p < (\deg P_{c_{\gamma}})'$, 
\begin{equation*}
  \sum_{h=2}^{\deg P_{c_{\gamma}}} \|(\tilde c_{{\gamma} h}^{1/h})'\|^*_{L^p(J_{\gamma})} \le C  |J_{\gamma}|^{-1} |\tilde b_{\ell_{\gamma}} (t_{\gamma})|^{1/\ell_{\gamma}}, 
\end{equation*}
for a constant $C$ that depends only on $\deg P_{\tilde a}$ and $p$.
Consequently, by \eqref{Jba} and \eqref{inclusions0},
\begin{align*}
\MoveEqLeft
  \||J_{\gamma}|^{-1} |\tilde b_{\ell_{\gamma}}(t_{\gamma})|^{1/\ell_{\gamma}} \|^*_{L^p (J_{\gamma})}
  + \sum_{h=2}^{\deg P_{c_{\gamma}}} \|(\tilde c_{{\gamma} h}^{1/h})'\|^*_{L^p (J_{\gamma})}
  \\
  &\le (1+C) |J_{\gamma}|^{-1} |\tilde b_{\ell_{\gamma}}(t_{\gamma})|^{1/\ell_{\gamma}}
  \\
  &= (1+C) D^{-1} \Big( \| |I|^{-1}  {|\tilde a_k(t_0)|^{1/k}} \|^*_{L^1 (J_{\gamma})} 
  + \sum_{i=2}^{\deg P_b} \|(\tilde b_i^{1/i})'\|^*_{L^1 (J_{\gamma})}\Big)
  \\
  &\le (1+C)   D^{-1}   \Big( \| |I|^{-1}  {|\tilde a_k(t_0)|^{1/k}} \|^*_{L^p (J_{\gamma})} 
  + \sum_{i=2}^{\deg P_b} \|(\tilde b_i^{1/i})'\|^*_{L^p (J_{\gamma})}\Big) 
\end{align*}
and therefore 
\begin{align}
\MoveEqLeft
  \||J_{\gamma}|^{-1} |\tilde b_{\ell_{\gamma}}(t_{\gamma})|^{1/\ell_{\gamma}} \|^p_{L^p (J_{\gamma})}
  + \sum_{h=2}^{\deg P_{c_{\gamma}}} \|(\tilde c_{{\gamma} h}^{1/h})'\|^p_{L^p (J_{\gamma})}
  \notag \\
  &\le C D^{-p}  \Big( \| |I|^{-1}  {|\tilde a_k(t_0)|^{1/k}} \|^p_{L^p (J_{\gamma})} 
  + \sum_{i=2}^{\deg P_b} \|(\tilde b_i^{1/i})'\|^p_{L^p (J_{\gamma})}\Big), \label{cba}
\end{align}
for a constant $C$ that depends only on $\deg P_{\tilde a}$ and $p$.

Furthermore, the analogue of \eqref{b1prime} gives 
\begin{align*}  
    \|c_{{\gamma} 1}'\|_{L^\infty(J_{\gamma})}   \le C  |J_{\gamma}|^{-1}  |\tilde b_{\ell_{\gamma}} (t_{\gamma})|^{1/\ell_{\gamma}},
\end{align*} 
where $C=C(\deg P_{\tilde a})$.
Thus, by \eqref{Jba} and \eqref{inclusions0},
\begin{align*}  
    \|c_{{\gamma} 1}'\|^*_{L^p(J_{\gamma})} \le C D^{-1}  \Big( \| |I|^{-1}  {|\tilde a_k(t_0)|^{1/k}} \|^*_{L^p (J_{\gamma})} 
  + \sum_{i=2}^{\deg P_b} \|(\tilde b_i^{1/i})'\|^*_{L^p (J_{\gamma})}\Big)
\end{align*}
and hence
\begin{align}  \label{c1prime}
    \|c_{{\gamma} 1}'\|^p_{L^p(J_{\gamma})} \le C D^{-p}  \Big( \| |I|^{-1}  {|\tilde a_k(t_0)|^{1/k}} \|^p_{L^p (J_{\gamma})} 
  + \sum_{i=2}^{\deg P_b} \|(\tilde b_i^{1/i})'\|^p_{L^p (J_{\gamma})}\Big),
\end{align}
for a constant $C$ that depends only on $\deg P_{\tilde a}$ and $p$.

Hence, by \eqref{Jga}, \eqref{mu}, and  \eqref{cba},
\begin{align} \label{ind11}
  \sum_{\gamma} \|\tilde \mu'\|^p_{L^p(J_{\gamma})} 
  &\le C D^{-p}  \sum_{\gamma} \Big( \| |I|^{-1}  {|\tilde a_k(t_0)|^{1/k}} \|^p_{L^p (J_{\gamma})} 
  + \sum_{i=2}^{\deg P_b} \|(\tilde b_i^{1/i})'\|^p_{L^p (J_{\gamma})}\Big)
  \notag \\
  &\le 2 C D^{-p}  \Big( \| |I|^{-1}  {|\tilde a_k(t_0)|^{1/k}} \|^p_{L^p (I)} 
  + \sum_{i=2}^{\deg P_b} \|(\tilde b_i^{1/i})'\|^p_{L^p (I)}\Big). 
\end{align}
Similarly, with \eqref{c1prime} we get
\begin{align} \label{ind22}
  \sum_{\gamma} \|c_{{\gamma} 1}'\|^p_{L^p(J_{\gamma})} 
  \le C D^{-p}  \Big( \| |I|^{-1}  {|\tilde a_k(t_0)|^{1/k}} \|^p_{L^p (I)} 
  + \sum_{i=2}^{\deg P_b} \|(\tilde b_i^{1/i})'\|^p_{L^p (I)}\Big). 
\end{align}
By \eqref{Jga}, \eqref{tildemu}, \eqref{ind11}, and \eqref{ind22}, we may conclude that $\mu$ is absolutely continuous on $I'$ and 
\begin{align*}
  \|\mu'\|^p_{L^p(I')} \le \sum_{\gamma} \|\mu'\|^p_{L^p(J_{\gamma})} 
  \le C D^{-p}  \Big( \| |I|^{-1}  {|\tilde a_k(t_0)|^{1/k}} \|^p_{L^p (I)} 
  + \sum_{i=2}^{\deg P_b} \|(\tilde b_i^{1/i})'\|^p_{L^p (I)}\Big), 
\end{align*}
and hence
\begin{align*}
  \|\mu'\|_{L^p(I')} 
  \le C D^{-1}  \Big( \| |I|^{-1}  {|\tilde a_k(t_0)|^{1/k}} \|_{L^p (I)} 
  + \sum_{i=2}^{\deg P_b} \|(\tilde b_i^{1/i})'\|_{L^p (I)}\Big),
\end{align*}
for a constant $C$ that depends only on $\deg P_{\tilde a}$ and $p$.
Since $\mu$ vanishes on $I \setminus I'$, Lemma \ref{lem:extend} implies that $\mu$ is absolutely continuous on $I$ and  
\begin{equation*} 
  \|\mu'\|_{L^p(I)} \le C D^{-1}  \Big( \| |I|^{-1}  {|\tilde a_k(t_0)|^{1/k}} \|_{L^p (I)} 
  + \sum_{i=2}^{\deg P_b} \|(\tilde b_i^{1/i})'\|_{L^p (I)}\Big).
\end{equation*}
This completes the proof of Proposition \ref{induction}, since $C = C(\deg P_{\tilde a},p)$ 
and $D = D(\deg P_{\tilde a})$ by \eqref{D}.

\subsection{End of proof of Theorem \ref{main}} \label{end}

We have seen in Section \ref{first} that for a polynomial $P_{\tilde a}$ in Tschirnhausen form satisfying \eqref{k} and \eqref{assumption<=}
the assumptions of Proposition \ref{induction} hold with the constant $B$ 
fulfilling \eqref{eq:constB}. 

Let ${\lambda} \in C^0(({\alpha},{\beta}))$ be a continuous root of $P_{\tilde a}$. We may assume without loss of generality that in $I$,  
it is a root of $P_b$.
Then it has the form 
\begin{align} \label{lambda}
  {\lambda}(t) &= - \frac{b_1(t)}{\deg P_b} + \mu(t), \quad t \in I, 
\end{align}
where $\mu$ is a continuous root of $P_{\tilde b}$. 
By Proposition \ref{induction}, $\mu$ is absolutely continuous on $I$ and satisfies \eqref{muab}.
Let us estimate the right-hand side of \eqref{muab}.

By Lemma \ref{lem:B}, we have \eqref{eq:b_ider}, and thus together with \eqref{est},
\begin{align*} 
\|(\tilde b_i^{1/i})'\|_{i',w,I} &\le C(i) \max\Big\{\big(\on{Lip}_{I}(\tilde b_i^{(i-1)})\big)^{1/i}|I|^{1/i'}, 
    \|\tilde b_i'\|_{L^\infty(I)}^{1/i}\Big\} \\
    &\le C(n)  |I|^{-1 +1/i'} |\tilde a_k (t_0)|^{1/k}.
\end{align*} 
Hence
\begin{align*} 
\|(\tilde b_i^{1/i})'\|^*_{i',w,I} &\le C(n)  |I|^{-1} |\tilde a_k (t_0)|^{1/k}.
\end{align*}
Since $n' < \min_{2 \le i \le \deg P_b} i'$ and by \eqref{inclusions}, we get for all $p$, $1 \le p < n'$, 
\begin{equation*}
  \sum_{i=2}^{\deg P_b} \|(\tilde b_i^{1/i})'\|^*_{L^p(I)} \le C  |I|^{-1} |\tilde a_k (t_0)|^{1/k}, 
\end{equation*}
where the constant $C$ depends only on $n$ and $p$.
It follows that  
\begin{align} \label{beforecases}
  \||I|^{-1} |\tilde a_k (t_0)|^{1/k} \|^*_{L^p (I)} + \sum_{i=2}^{\deg P_b} \|(\tilde b_i^{1/i})'\|^*_{L^p(I)} 
  \le (1+C) |I|^{-1} |\tilde a_k (t_0)|^{1/k}.
\end{align}
At this stage we distinguish the following two cases:
\begin{enumerate}
	\item[(i)] Either we have equality in \eqref{assumption<=}, i.e., 
	\begin{align}\label{assumption=}
		{M} |I| + \sum_{j=2}^n \|(\tilde a_j^{1/j})'\|_{L^1 (I)} =  B |\tilde a_k(t_0)|^{1/k} .
	\end{align}
	\item[(ii)] Or $I = ({\alpha},{\beta})$ and 
	\begin{align}\label{assumption<}
		{M} |I| + \sum_{j=2}^n \|(\tilde a_j^{1/j})'\|_{L^1 (I)} <  B |\tilde a_k(t_0)|^{1/k} .
	\end{align}
\end{enumerate}

\subsection*{Case (i)} In this cases we can estimate \eqref{beforecases} by \eqref{assumption=} and obtain
\begin{align*}
  \MoveEqLeft
  \||I|^{-1} |\tilde a_k (t_0)|^{1/k} \|^*_{L^p (I)} + \sum_{i=2}^{\deg P_b} \|(\tilde b_i^{1/i})'\|^*_{L^p(I)} 
  \\
  &\le C B^{-1}  \Big({M} \|1\|^*_{L^1(I)} + \sum_{j=2}^n \|(\tilde a_j^{1/j})'\|^*_{L^1 (I)}\Big) \qquad \text{ by \eqref{assumption=}}
  \\
  &\le C  B^{-1}  \Big({M} \|1\|^*_{L^p(I)} + \sum_{j=2}^n \|(\tilde a_j^{1/j})'\|^*_{L^p (I)}\Big)  \qquad \text{ by \eqref{inclusions0}}
\end{align*}
and therefore 
\begin{align}
  \MoveEqLeft
  \||I|^{-1} |a_k (t_0)|^{1/k} \|_{L^p (I)} + \sum_{i=2}^{\deg P_b} \|(\tilde b_i^{1/i})'\|_{L^p(I)} \notag
  \\
  &\le C B^{-1}   \Big({M}\|1\|_{L^p(I)} + \sum_{j=2}^n \|(\tilde a_j^{1/j})'\|_{L^p (I)}\Big), \label{Lpestimate} 
\end{align}
for a constant $C$ that depends only on $n$ and $p$.

Thus, by \eqref{muab} and \eqref{Lpestimate}, 
\begin{equation} \label{mu1a}
  \|\mu'\|_{L^p(I)} \le C B^{-1} \Big({M} \|1\|_{L^p(I)} + \sum_{j=2}^n \|(\tilde a_j^{1/j})'\|_{L^p (I)}\Big).
\end{equation}
Similarly, by \eqref{inclusions0}, \eqref{b1prime}, \eqref{assumption=}, and \eqref{Lpestimate}, 
\begin{align}  \label{b11a}
    \|b_1'\|_{L^p(I)} \le C B^{-1}  \Big({M} \|1\|_{L^p(I)} + \sum_{j=2}^n \|(\tilde a_j^{1/j})'\|_{L^p (I)}\Big).
\end{align}

In view of \eqref{lambda}, \eqref{mu1a}, and \eqref{b11a} we obtain that ${\lambda}$ is absolutely continuous on $I$ and 
\begin{align*} 
  \|{\lambda}'\|_{L^p(I)} \le C B^{-1}  \Big({M} \|1\|_{L^p(I)} + \sum_{j=2}^n \|(\tilde a_j^{1/j})'\|_{L^p (I)}\Big).
\end{align*}
The constant ${M}$ given \eqref{M} depends on $t_0$; thus we 
set 
\begin{equation} \label{tildeA}
  \tilde A := \max_{2 \le j \le n} \|\tilde a_j\|^{1/j}_{C^{n-1,1}([{\alpha},{\beta}])}
\end{equation}
and 
estimate ${M}$ by
\begin{align*}
  {M} &= \max_{2 \le j\le n}  ({\on{Lip}}_I(\tilde a_j^{(n-1)}))^{1/n} |\tilde a_k (t_0)|^{(n-j)/(kn)}
  \\
  &\le \max_{2 \le j\le n} \tilde A^{j/n} \tilde A^{(n-j)/n} = \tilde A.  
\end{align*}
Thus,
\begin{align} \label{laIA}
  \|{\lambda}'\|_{L^p(I)} \le C B^{-1}  \Big(\tilde A \|1\|_{L^p(I)} + \sum_{j=2}^n \|(\tilde a_j^{1/j})'\|_{L^p (I)}\Big).
\end{align}

\subsection*{Case (ii)}  In this case we have a splitting  $P_{\tilde a} = P_b P_{b^*}$ on the whole interval $I=({\alpha},{\beta})$; cf.\ 
Section \ref{first}. Thus, \eqref{beforecases} becomes 
\begin{align*}
  \MoveEqLeft
  \||({\beta}-{\alpha})^{-1} |\tilde a_k (t_0)|^{1/k} \|_{L^p (({\alpha},{\beta}))} + \sum_{i=2}^{\deg P_b} \|(\tilde b_i^{1/i})'\|_{L^p(({\alpha},{\beta}))} 
  \\
  &\le C  ({\beta}-{\alpha})^{-1+1/p} |\tilde a_k (t_0)|^{1/k}
  \\
  &\le C  ({\beta}-{\alpha})^{-1+1/p} \max_{2 \le j \le n} \|\tilde a_j\|^{1/j}_{L^\infty(({\alpha},{\beta}))}
\end{align*}
Similarly, \eqref{b1prime} implies 
\begin{equation*}
  \|b_1'\|_{L^p(({\alpha},{\beta}))} \le C ({\beta}-{\alpha})^{-1+1/p} \max_{2 \le j \le n} \|\tilde a_j\|^{1/j}_{L^\infty(({\alpha},{\beta}))}.
\end{equation*}
In view of \eqref{lambda} and \eqref{muab} we obtain that ${\lambda}$ is absolutely continuous on $({\alpha},{\beta})$ and 
\begin{align} \label{lacase2}
  \|{\lambda}'\|_{L^p(({\alpha},{\beta}))} \le C ({\beta}-{\alpha})^{-1+1/p} \max_{2 \le j \le n} \|\tilde a_j\|^{1/j}_{L^\infty(({\alpha},{\beta}))},
\end{align}
where $C = C(n,p)$.

\subsection*{Gluing the estimates}
In Case (ii) the bound \eqref{lacase2} holds on the whole interval; no gluing is required. 
Hence, if there is at least one point $t_0$ in $({\alpha},{\beta})$ at which Case (ii) occurs, we are done. 

Let us assume that at all points in $({\alpha},{\beta})$, that do not satisfy \eqref{zero}, Case (i) occurs. 
In analogy to Section \ref{second}, we can cover the complement in $({\alpha},{\beta})$ of the points $t$ satisfying \eqref{zero} 
by a countable family ${\mathcal{I}}$ of open intervals $I$ on which \eqref{laIA} holds and such that 
$\sum_{I \in {\mathcal{I}}} |I| \le 2 ({\beta}-{\alpha})$.
Since ${\lambda}$ vanishes on the points $t$ satisfying \eqref{zero}, we can apply Lemma~\ref{lem:extend} and obtain
that ${\lambda}$ is absolutely continuous on $({\alpha},{\beta})$ and satisfies
\begin{equation*}
  \|{\lambda}'\|_{L^p(({\alpha},{\beta}))} \le  C B^{-1}  \Big(\tilde A \|1\|_{L^p(({\alpha},{\beta}))} + \sum_{j=2}^n \|(\tilde a_j^{1/j})'\|_{L^p (({\alpha},{\beta}))}\Big).
\end{equation*}
By \eqref{est}, we may conclude that ${\lambda}$ is absolutely continuous on $({\alpha},{\beta})$ and satisfies  
\begin{equation} \label{lacase1}
  \|{\lambda}'\|_{L^p(({\alpha},{\beta}))} \le  C B^{-1}  \Big(\tilde A ({\beta}-{\alpha})^{1/p} 
  + \sum_{j=2}^n \max \Big\{ ({\on{Lip}}_{({\alpha},{\beta})}(\tilde a_j^{(j-1)}))^{1/j}  ({\beta}-{\alpha})^{1-1/j}, 
  \|\tilde a_j'\|^{1/j}_{L^\infty(({\alpha},{\beta}))}
   \Big\} \Big),
\end{equation}
where $C = C(n,p)$ and $B= B(n)$ by \eqref{eq:constB}.

\begin{remarks*}
  (1) We can avoid the distinction of cases in Section \ref{end} if we require that the constant $B$ also satisfies 
  \begin{equation}
    B  \max_{2 \le j \le n} \|\tilde a_j\|_{L^\infty(({\alpha},{\beta}))}^{1/j} \le {M} ({\beta} -{\alpha}). \label{eq:constB2}
  \end{equation} 
  which enforces Case (i). Then, however, the factor $B^{-1}$ that appears in \eqref{lacase1} blows up as 
  ${\beta} - {\alpha} \to 0$.

  (2) Also the bound in Case (ii) for $\|{\lambda}'\|_{L^p(({\alpha},{\beta}))}$ in \eqref{lacase2} tends to infinity if ${\beta} - {\alpha} \to 0$ 
  unless $p=1$.

  (3) A sufficient condition for the elimination of this blow-up phenomenon is the following. 
  \emph{Assume that for all $j = 2,\ldots,n$ there is a point $s = s(j) \in ({\alpha},{\beta})$ such that $\tilde a_j(s) =0$.}
  In that case we have, for $t \in ({\alpha},{\beta})$, 
  \[
    |\tilde a_j^{1/j}(t)| = |\int_s^t (\tilde a_j^{1/j})'\, d {\tau}| \le \|(\tilde a_j^{1/j})'\|_{L^1(({\alpha},{\beta}))}
  \]
  and hence 
  \begin{equation} \label{sufficient}
      \max_{2 \le j \le n} \|\tilde a_j\|_{L^\infty(({\alpha},{\beta}))}^{1/j} \le \sum_{j=2}^n \|(\tilde a_j^{1/j})'\|_{L^1(({\alpha},{\beta}))}. 
  \end{equation}
  Since $B<1$ (by \eqref{eq:constB}), \eqref{sufficient} enforces equality in \eqref{assumption<=} and thus only Case (i) occurs. 
  Since the constant $B$ is only restricted by \eqref{eq:constB} it is universal.
\end{remarks*}

\subsection{The uniform bound \texorpdfstring{\eqref{bound}}{(1.2)}} \label{secbound}

The bounds \eqref{lacase1} and \eqref{lacase2} imply
\begin{equation} \label{boundT}
  \|{\lambda}' \|_{L^p(({\alpha},{\beta}))}  \le C(n,p) \max \{1, ({\beta}-{\alpha})^{1/p}, ({\beta}-{\alpha})^{-1+1/p}\}   \tilde A,
\end{equation}
where $\tilde A$ is given by \eqref{tildeA}.

In order to get the bound in terms of the $a_j$ (i.e., \emph{before} Tschirnhausen transformation) 
let $\hat {\lambda} := {\lambda} - a_1/n$ and set
\[
  A := \max_{1 \le j \le n} \|a_j\|^{1/j}_{C^{n-1,1}([{\alpha},{\beta}])}. 
\] 
Then
\begin{align*}
  \|\hat {\lambda}'\|_{L^p(({\alpha},{\beta}))} \le \|{\lambda}'\|_{L^p(({\alpha},{\beta}))} + (1/n)\|a_1'\|_{L^p(({\alpha},{\beta}))} 
\end{align*}
and 
\begin{align*}
  \|a_1'\|_{L^p(({\alpha},{\beta}))} \le ({\beta}-{\alpha})^{1/p} \|a_1'\|_{L^\infty(({\alpha},{\beta}))}.
\end{align*}
Observe that
\[
  \tilde A \le C(n) A, 
\] 
by the weighted homogeneity of the formulas \eqref{Tschirnhausen}.
Hence, by \eqref{boundT}, 
\begin{align*} 
  \|\hat {\lambda}' \|_{L^p(({\alpha},{\beta}))}  &\le C(n,p) \max\{1, ({\beta}-{\alpha})^{1/p}, ({\beta}-{\alpha})^{-1+1/p}\}   A,
\end{align*}
that is \eqref{bound}.
The proof of Theorem \ref{main} is complete.

\providecommand{\bysame}{\leavevmode\hbox to3em{\hrulefill}\thinspace}
\providecommand{\MR}{\relax\ifhmode\unskip\space\fi MR }
\providecommand{\MRhref}[2]{  \href{http://www.ams.org/mathscinet-getitem?mr=#1}{#2}
}
\providecommand{\href}[2]{#2}
\begin{thebibliography}{10}

\bibitem{AKLM98}
D.~Alekseevsky, A.~Kriegl, M.~Losik, and P.~W. Michor, \emph{Choosing roots of
  polynomials smoothly}, Israel J. Math. \textbf{105} (1998), 203--233.

\bibitem{BM90}
E.~Bierstone and P.~D. Milman, \emph{Arc-analytic functions}, Invent. Math.
  \textbf{101} (1990), no.~2, 411--424.

\bibitem{BBCP06}
J.-M. Bony, F.~Broglia, F.~Colombini, and L.~Pernazza, \emph{Nonnegative
  functions as squares or sums of squares}, J. Funct. Anal. \textbf{232}
  (2006), no.~1, 137--147.

\bibitem{BonyColombiniPernazza06}
J.-M. Bony, F.~Colombini, and L.~Pernazza, \emph{On the differentiability class
  of the admissible square roots of regular nonnegative functions}, Phase space
  analysis of partial differential equations, Progr. Nonlinear Differential
  Equations Appl., vol.~69, Birkh\"auser Boston, Boston, MA, 2006, pp.~45--53.

\bibitem{BonyColombiniPernazza10}
\bysame, \emph{On square roots of class {$C^m$} of nonnegative functions of one
  variable}, Ann. Sc. Norm. Super. Pisa Cl. Sci. (5) \textbf{9} (2010), no.~3,
  635--644. 

\bibitem{Bronshtein79}
M.~D. Bronshtein, \emph{Smoothness of roots of polynomials depending on
  parameters}, Sibirsk. Mat. Zh. \textbf{20} (1979), no.~3, 493--501, 690,
  English transl. in Siberian Math. J. \textbf{20} (1980), 347--352.

\bibitem{CC04}
J.~Chaumat and A.-M. Chollet, \emph{Division par un polyn\^ome hyperbolique},
  Canad. J. Math. \textbf{56} (2004), no.~6, 1121--1144.

\bibitem{CJS83}
F.~Colombini, E.~Jannelli, and S.~Spagnolo, \emph{Well-posedness in the
  {G}evrey classes of the {C}auchy problem for a nonstrictly hyperbolic
  equation with coefficients depending on time}, Ann. Scuola Norm. Sup. Pisa
  Cl. Sci. (4) \textbf{10} (1983), no.~2, 291--312.

\bibitem{CL03}
F.~Colombini and N.~Lerner, \emph{Une procedure de {C}alder\'on-{Z}ygmund pour
  le probl\`eme de la racine {$k$}-i\`eme}, Ann. Mat. Pura Appl. (4)
  \textbf{182} (2003), no.~2, 231--246.

\bibitem{ColombiniOrruPernazza12}
F.~Colombini, N.~Orr{{\`u}}, and L.~Pernazza, \emph{On the regularity of the
  roots of hyperbolic polynomials}, Israel J. Math. \textbf{191} (2012),
  923--944. 

\bibitem{GhisiGobbino13}
M.~Ghisi and M.~Gobbino, \emph{Higher order {G}laeser inequalities and optimal
  regularity of roots of real functions}, Ann. Sc. Norm. Super. Pisa Cl. Sci.
  (5) \textbf{12} (2013), no.~4, 1001--1021. 

\bibitem{Glaeser63R}
G.~Glaeser, \emph{Racine carr\'ee d'une fonction diff\'erentiable}, Ann. Inst.
  Fourier (Grenoble) \textbf{13} (1963), no.~2, 203--210.

\bibitem{Grafakos08}
L.~Grafakos, \emph{Classical {F}ourier analysis}, second ed., Graduate Texts in
  Mathematics, vol. 249, Springer, New York, 2008. 

\bibitem{Kato76}
T.~Kato, \emph{Perturbation theory for linear operators}, second ed.,
  Grundlehren der Mathematischen Wissenschaften, vol. 132, Springer-Verlag,
  Berlin, 1976.

\bibitem{KLM04}
A.~Kriegl, M.~Losik, and P.~W. Michor, \emph{Choosing roots of polynomials
  smoothly. {II}}, Israel J. Math. \textbf{139} (2004), 183--188.

\bibitem{Leoni09}
G.~Leoni, \emph{A first course in {S}obolev spaces}, Graduate Studies in
  Mathematics, vol. 105, American Mathematical Society, Providence, RI, 2009.
  

\bibitem{Mandai85}
T.~Mandai, \emph{Smoothness of roots of hyperbolic polynomials with respect to
  one-dimensional parameter}, Bull. Fac. Gen. Ed. Gifu Univ. (1985), no.~21,
  115--118.

\bibitem{ParusinskiRainerHyp}
A.~Parusi{{\'n}}ski and A.~Rainer, \emph{A new proof of {B}ronshtein's
  theorem}, J. Hyperbolic Differ. Equ., to appear, ar{X}iv:1309.2150.

\bibitem{ParusinskiRainerAC}
\bysame, \emph{Regularity of roots of polynomials}, Ann.\ Sc.\ Norm.\ Super.\
  Pisa Cl.\ Sci.\ (5), to appear, ar{X}iv:1309.2151.

\bibitem{RainerAC}
A.~Rainer, \emph{Perturbation of complex polynomials and normal operators},
  Math. Nachr. \textbf{282} (2009), no.~12, 1623--1636. 
  

\bibitem{RainerQA}
\bysame, \emph{Quasianalytic multiparameter perturbation of polynomials and
  normal matrices}, Trans. Amer. Math. Soc. \textbf{363} (2011), no.~9,
  4945--4977.

\bibitem{RainerOmin}
\bysame, \emph{Smooth roots of hyperbolic polynomials with definable
  coefficients}, Israel J. Math. \textbf{184} (2011), 157--182. 
  

\bibitem{RainerFin}
\bysame, \emph{Differentiable roots, eigenvalues, and eigenvectors}, Israel J.
  Math. \textbf{201} (2014), no.~1, 99--122. 

\bibitem{Spagnolo99}
S.~Spagnolo, \emph{On the absolute continuity of the roots of some algebraic
  equations}, Ann. Univ. Ferrara Sez. VII (N.S.) \textbf{45} (1999),
  no.~suppl., 327--337 (2000), Workshop on Partial Differential Equations
  (Ferrara, 1999).

\bibitem{Tarama00}
S.~Tarama, \emph{{On the lemma of Colombini, Jannelli and Spagnolo}}, Memoirs
  of the Faculty of Engineering, Osaka City University \textbf{41} (2000),
  111--115.

\bibitem{Tarama06}
\bysame, \emph{Note on the {B}ronshtein theorem concerning hyperbolic
  polynomials}, Sci. Math. Jpn. \textbf{63} (2006), no.~2, 247--285.

\bibitem{Wakabayashi86}
S.~Wakabayashi, \emph{Remarks on hyperbolic polynomials}, Tsukuba J. Math.
  \textbf{10} (1986), no.~1, 17--28.

\end{thebibliography}

\end{document}


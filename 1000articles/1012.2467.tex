\documentclass{amsart}
\usepackage[all]{xy}
\usepackage{latexsym}
\usepackage{amssymb}
\usepackage{amsmath}
\usepackage{amsthm}
\usepackage{amscd}
\usepackage{fancyhdr}
\usepackage[dvips]{graphicx}
\usepackage{fullpage}
\usepackage{mathrsfs}

\xyoption{arc}

\setlength{\parindent}{0pt} \addtolength{\headsep}{0.5cm}

 \newfam\cyrfam

  \font\tencyr=wncyr10

  \font\sevencyr=wncyr7

  \font\fivecyr=wncyr5

  

  \textfont\cyrfam=\tencyr \scriptfont\cyrfam=\sevencyr

    \scriptscriptfont\cyrfam=\fivecyr

  \newfam\cyifam

  \font\tencyi=wncyi10

  \font\sevencyi=wncyi7

  \font\fivecyi=wncyi5

  
  

  \textfont\cyifam=\tencyi \scriptfont\cyifam=\sevencyi

    \scriptscriptfont\cyifam=\fivecyi

  

 

 

 

 

  

  

   
 
 
 
 
 
 
 
 

 
 

 
 
 
 
 
 

   
 

  
  

 
 
 
 
 
 
 
 
 
 
 
 
 
 
 
 
 
 
 

 

 
 

 
 
 
 
 
 
 
 
 
 
 
 
 
 
 
 
 
 
 
 
 
 
 
 
 
 
 
 

 

 
 
 
 
 
 
 
 
 
 
 
 
 
 
 
 
 

 

 
 
 
 
 
 
 

 
 
 
 
 

 
 
 
 
 
 
 
 
 
 
 
 
 
  
 
 
 
 
 
 

\theoremstyle{plain}
\swapnumbers
\newtheorem{theorem}{Theorem}[subsection]
\newtheorem{corollary}[theorem]{Corollary}
\newtheorem{observation}[theorem]{Observation}
\newtheorem{lemma}[theorem]{Lemma}
\newtheorem{proposition}[theorem]{Proposition}
\newtheorem{problem}[theorem]{Problem}
\newtheorem{conjecture}[theorem]{Conjecture}
\newtheorem{prop-def}[theorem]{Proposition-definition}

\newtheorem{f-theorem}{Formality Theorem}[section]
\newtheorem{main-theorem}{Main~Theorem}[section]
\newtheorem{section-theorem}{Theorem}[section]
\newtheorem{section-corollary}{Corollary}[section]

\theoremstyle{definition}
\newtheorem{example}[theorem]{Example}
\newtheorem{remark}[theorem]{Remark}
\newtheorem{definition}[theorem]{Definition}

\newtheorem{fact}{Fact}[subsection]

\newtheorem{fact-me}{Fact \cite{Me1}}[subsection]

 
  \begin{document}

 \sloppy

 \newenvironment{proo}{\begin{trivlist} \item{\sc {Proof.}}}
  {\hfill $\square$ \end{trivlist}}

\long

 \title{Grothendieck-Teichm\"uller  and Batalin-Vilkovisky}
  

 \author{ Sergei\ Merkulov and Thomas Willwacher}
\address{Sergei~Merkulov: Department of Mathematics, Stockholm University, 10691 Stockholm, Sweden, and Mathematics Research Unit, University of Luxembourg, Grand Duchy of Luxembourg (Current address)}
\email{sergei.merkulov@uni.lu}
\address{Institute of Mathematics\\ University of Zurich\\ Winterthurerstrasse 190 \\ 8057 Zurich, Switzerland}
\email{thomas.willwacher@math.uzh.ch}

\subjclass[2010]{81R99, 13D10 }
\keywords{Grothendieck-Teichm\"uller Group, Batalin-Vilkovisky Algebras, Master Equation}

 \begin{abstract} It is proven that, for  any affine supermanifold $M$  equipped with a constant odd symplectic structure, there is a universal
  action (up to homotopy) of the  Grothendieck-Teichm\"uller Lie algebra ${{\mathfrak{{grt}}}_1}$ on the set of quantum  BV structures (i. e.\ solutions of the quantum master equation) on $M$.

\end{abstract}
 \maketitle
\markboth{S.\ Merkulov and T.\ Willwacher}{GRT and BV}

{\large
\section{\bf Introduction}
}

Let $M$ be a finite dimensional affine ${{\mathbb Z}}$-graded manifold $M$  over a field ${{\mathbb K}}$ equipped with a constant degree 1 symplectic structure
${\omega}$. In particular, the ring of functions ${{\mathcal O}}_M$ is a Batalin-Vilkovisky algebra, with Batalin-Vilkovisky operator $\Delta$ and bracket $\{\ ,\ \}$. A degree 2 function $S\in {{\mathcal O}}_M[[{u}]]$ is a solution the quantum master equation on $M$ if\footnote{See \cite{Sc} for an introduction into the geometry of the BV formalism.}
$$
{u} \Delta S + \frac{1}{2}\{S,S\}=0,
$$
where $u$ is a formal variable of degree 2. In other words $S$ is a Maurer-Cartan element in the differential graded (dg) Lie algebra $\left({{\mathcal O}}_M[[{u}]][1], {u}\Delta, \{\ ,\ \}\right)$.

{\smallskip}

The Grothendieck-Teichm\"uller group ${GRT_1}$ is a pro-unipotent group  introduced
by Drinfeld in [Dr]; we denote its Lie algebra by ${{\mathfrak{{grt}}}_1}$.
In this paper we show the following result.

 {\smallskip}

\subsection{ Main Theorem} {\em There is an $L_\infty$ action of the Lie algebra ${{\mathfrak{{grt}}}_1}$ on the differential graded Lie algebra $\left({{\mathcal O}}_M[[{u}]][1], {u}\Delta, \{\ ,\ \}\right)$ by $L_\infty$ derivations. In particular, it follows that there is an action of ${GRT_1}$ on the set of gauge equivalence classes of formal solutions of the quantum master equation, i.~e., on gauge equivalence classes of Maurer-Cartan elements in the differential graded Lie algebra $\left(\hbar{{\mathcal O}}_M[[{u}]][[\hbar]][1], {u}\Delta, \{\ ,\ \}\right)$, where $\hbar$ is a formal deformation parameter of degree 0. }

 {\smallskip}

 Our main technical tool is a  version of the Kontsevich graph complex, $({{\mathsf G}}{{\mathsf C}}_2[[{u}]], d_{u})$ which
 controls universal deformations of $\left({{\mathcal O}}_M[[{u}]][1], {u}\Delta, \{\ ,\ \}\right)$  in the category of $L_\infty$ algebras.
 Using the main result of \cite{Wi} we show in Sect.\ 2 that
 $$
 H^0({{\mathsf G}}{{\mathsf C}}_2[[{u}]], d_{u})\simeq{{\mathfrak{{grt}}}_1}
 $$
and then use this isomorphism in Sect.\ 3 to prove the Main Theorem.

\subsection{ Some notation} In this paper $\mathbb K$ denotes a field of characteristic $0$. If $V=\oplus_{i\in {{\mathbb Z}}} V^i$ is a graded vector space over ${{\mathbb K}}$, then
$V[k]$ stands for the graded vector space with $V[k]^i:=V^{i+k}$. For $v\in V^i$ we set $|v|:=i$.
The phrase \emph{differential graded} is abbreviated by dg.
The $n$-fold symmetric product of a (dg) vector space $V$ is denoted  by $\odot^n V$, the full symmetric product space by  $\odot^\bullet V$.
For a finite group $G$ acting on a vector space $V$, we
denote via $V^G$ the space of invariants with respect to the action of $G$, and by $V_G$
the space of coinvariants $V_G = V/\{gv- v| v\in V, g\in G\}$. As we always work over a field ${{\mathbb K}}$ of characteristic zero, we have a canonical isomorphism $V_G\cong V^G$.

We  use freely the language of operads. For a background on operads we refer to the textbook \cite{LV}.
For an operad ${{\mathcal P}}$ we denote by ${{\mathcal P}}\{k\}$ the unique operad which has the following property:
for any graded vector space $V$ there is a one-to-one correspondence between representations of
${{\mathcal P}}\{k\}$ in $V$ and representations of
${{\mathcal P}}$ in $V[-k]$; in particular, ${{\mathcal E}} nd_V\{k\}={{\mathcal E}} nd_{V[k]}$.

{\bigskip}

{\large
\section{\bf A variant of the Kontsevich graph complex}
}

{\bigskip}
\subsection{From operads to Lie algebras}
Let ${{\mathcal P}}=\{{{\mathcal P}}(n)\}_{n\geq 1}$ be an operad in the category of dg vector spaces with the partial compositions $\circ_i: {{\mathcal P}}(n){\otimes} {{\mathcal P}}(m) {\rightarrow} {{\mathcal P}}(m+n-1)$, $1\leq i\leq n$.
Then the map
$$
\begin{array}{rccc}
[\ ,\ ]:&  {{\mathsf P}} {\otimes} {{\mathsf P}} & {\longrightarrow} & {{\mathsf P}}\\
& (a\in {{\mathcal P}}(n), b\in {{\mathcal P}}(m)) & {\longrightarrow} &
[a, b]:= \sum_{i=1}^n a\circ_i b - (-1)^{|a||b|}\sum_{i=1}^m b\circ_i a
\end{array}
$$
makes the vector space
$
{{\mathsf P}}:= \prod_{n\geq 1}{{\mathcal P}}(n)$  
into a dg Lie algebra \cite{GV,KM}. Moreover, the Lie algebra structure descends to the subspace
of coinvariants ${{\mathsf P}}_{{\mathbb S}}:=  \prod_{n\geq 1}{{\mathcal P}}(n)_{{{\mathbb S}}_n}$. Via the identification of invariants and coinvariants ${{\mathsf P}}_{{\mathbb S}} \cong {{\mathsf P}}^{{\mathbb S}}$, we furthermore obtain a Lie algebra structure on the space of invariants
${{\mathsf P}}^{{\mathbb S}}:=  \prod_{n\geq 1}{{\mathcal P}}(n)^{{{\mathbb S}}_n}$ as well.

\subsection{An operad of graphs and the Kontsevich graph complex}
For any integers $n\geq 1$ and $l\geq 0$ we denote by ${{\mathsf G}}_{n,l}$ a set of graphs\footnote{ A {\em graph}\, ${\Gamma}$ is, by definition,  a 1-dimensional $CW$-complex whose $0$-cells are called {\em vertices}\, and $1$-dimensional cells are called {\em edges}. The set of vertices of ${\Gamma}$ is denoted by $V({\Gamma})$ and the set of edges by $E({\Gamma})$.}, $\{{\Gamma}\}$, with $n$ vertices and $l$ edges
such that (i) the vertices of ${\Gamma}$ are labelled by elements of $[n]:=\{1,\ldots, n\}$,
(ii) the set of edges, $E({\Gamma})$, is totally ordered up to an even permutation.
For example, $\xy
(0,2)*{_{1}},
(7,2)*{_{2}},
 (0,0)*{\bullet}="a",
(7,0)*{\bullet}="b",
\ar @{-} "a";"b" <0pt>
\endxy\in {{\mathsf G}}_{2,1}$.
The group ${{\mathbb Z}}_2$ acts freely  on ${{\mathsf G}}_{n,l}$ for $l\geq 2$ by changes of the total ordering;  its orbit
is denoted by $\{{\Gamma}, {\Gamma}_{opp}\}$. Let ${{\mathbb K}}\langle {{\mathsf G}}_{n,l}\rangle$  be the vector space over  a field ${{\mathbb K}}$ spanned by isomorphism classes, $[{\Gamma}]$, of elements of ${{\mathsf G}}_{n,l}$ modulo the relation\footnote{Abusing notations we identify from now an equivalence class $[{\Gamma}]$ with any
of its representative ${\Gamma}$.}  ${\Gamma}_{opp}=-{\Gamma}$, and consider
a ${{\mathbb Z}}$-graded ${{\mathbb S}}_n$-module,
$$
{{\mathsf G}} {{\mathsf r}}{{\mathsf a}} (n):=\bigoplus_{l=0}^\infty {{\mathbb K}}\langle {{\mathsf G}}_{n,l}\rangle[l].
$$
Note that graphs with two or more edges between any fixed pair of vertices do not contribute to
${{\mathsf G}} {{\mathsf r}}{{\mathsf a}} (n)$ so that we could have assumed right from the beginning that the sets ${{\mathsf G}}_{n,l}$ do not contain graphs with multiple edges. The ${{\mathbb S}}$-module, ${{\mathsf G}} {{\mathsf r}}{{\mathsf a}} :=\{{{\mathsf G}} {{\mathsf r}}{{\mathsf a}} (n)\}_{n\geq 1}$, is naturally an operad with the  operadic compositions given by
$$
\begin{array}{rccc}
\circ_i: & {{\mathsf G}} {{\mathsf r}}{{\mathsf a}} (n){\otimes} {{\mathsf G}} {{\mathsf r}}{{\mathsf a}} (m) & {\longrightarrow} &  {{\mathsf G}} {{\mathsf r}}{{\mathsf a}} (m+n-1)\\
&  {\Gamma}_1 {\otimes} {\Gamma}_2   &{\longrightarrow} & \sum_{{\Gamma}\in {{\mathsf G}}_{{\Gamma}_1, {\Gamma}_2}^i} (-1)^{\sigma_{\Gamma}} {\Gamma}
\end{array}
$$
where $ {{\mathsf G}}_{{\Gamma}_1, {\Gamma}_2}^i$ is the subset of ${{\mathsf G}}_{n+m-1, \# E({\Gamma}_1) + \#E({\Gamma}_2)}$ consisting
of graphs, ${\Gamma}$, satisfying the condition: the full subgraph of ${\Gamma}$ spanned by the vertices labeled by
the set $\{i,i+1, \ldots, i+m-1\}$ is isomorphic to ${\Gamma}_2$ and the quotient graph, ${\Gamma}/{\Gamma}_2$, obtained by contracting that subgraph to a single vertex, is isomorphic to ${\Gamma}_1$. The sign $(-1)^{\sigma_\Gamma}$ is determined by the equality
$$
\bigwedge_{e\in E({\Gamma})}e= (-1)^{\sigma_\Gamma}\bigwedge_{e'\in E({\Gamma}_1)}e' \wedge \bigwedge_{e''\in E({\Gamma}_2)}e''.
$$
The unique element in ${{\mathsf G}}_{1,0}$ serves as the unit element in the operad  ${{\mathsf G}} {{\mathsf r}}{{\mathsf a}}$. The associated  Lie algebra of  ${{\mathbb S}}$-invariants, $(({{\mathsf G}} {{\mathsf r}}{{\mathsf a}}\{-2\})^{{\mathbb S}},[\ ,\ ])$ is denoted, following notations
of \cite{Wi}, by ${{\mathsf f}}{{\mathsf G}}{{\mathsf C}}_2$. Its elements can be understood as
graphs from ${{\mathsf G}}_{n,l}$ but with labeling of vertices forgotten, e.g.
$$
\xy
 (0,0)*{\bullet}="a",
(6,0)*{\bullet}="b",
\ar @{-} "a";"b" <0pt>
\endxy = \frac{1}{2}\left(\xy
(0,2)*{_{1}},
(6,2)*{_{2}},
 (0,0)*{\bullet}="a",
(6,0)*{\bullet}="b",
\ar @{-} "a";"b" <0pt>
\endxy + \xy
(0,2)*{_{2}},
(6,2)*{_{1}},
 (0,0)*{\bullet}="a",
(6,0)*{\bullet}="b",
\ar @{-} "a";"b" <0pt>
\endxy\right)\in {{\mathsf f}}{{\mathsf G}}{{\mathsf C}}_2.
$$
The cohomological degree of a graph with $n$ vertices and $l$ edges is $2(n-1)-l$.
It is easy to check that $\xy
 (0,0)*{\bullet}="a",
(6,0)*{\bullet}="b",
\ar @{-} "a";"b" <0pt>
\endxy$ is a Maurer-Cartan element in the Lie algebra ${{\mathsf f}}{{\mathsf G}}{{\mathsf C}}_2$. Hence we obtain a dg Lie algebra
$$
\left({{\mathsf f}}{{\mathsf G}}{{\mathsf C}}_2, [\ ,\ ], d:=[\xy
 (0,0)*{\bullet}="a",
(6,0)*{\bullet}="b",
\ar @{-} "a";"b" <0pt>
\endxy,\ ] \right).
$$
One may define a dg Lie subalgebra, ${{\mathsf G}}{{\mathsf C}}_2$, spanned by connected graphs with at least trivalent vertices and no edges beginning and ending at the same vertex. It is called
the {\em Kontsevich graph complex} \cite{Ko}.
We leave it to the reader to verify that the subspace ${{\mathsf G}}{{\mathsf C}}_2$ is indeed closed under both the differential and the Lie bracket.
We refer to \cite{Wi} for a detailed explanation of why studying
the dg Lie subalgebra  ${{\mathsf G}}{{\mathsf C}}_2$ rather than full Lie algebra ${{\mathsf f}}{{\mathsf G}}{{\mathsf C}}_2$ should be enough for most purposes. The cohomologies of ${{\mathsf G}}{{\mathsf C}}_2$ and ${{\mathsf f}}{{\mathsf G}}{{\mathsf C}}_2$ were partially computed in \cite{Wi}.

\subsubsection{\bf Theorem \cite{Wi}} \label{thm:negzero} (i) $
H^0({{\mathsf G}}{{\mathsf C}}_2, d)\simeq {{\mathfrak{{grt}}}_1}.$ (ii)  {\em For any negative integer}\, $i$, $H^i({{\mathsf G}}{{\mathsf C}}_2, d)=0$.
 

{\smallskip}

We shall introduce  next a new graph complex which is responsible for the action of ${GRT_1}$ on the set of quantum master functions on an odd symplectic supermanifold.

\subsection{A variant of the Kontsevich graph complex} The graph
$
\xy
(0,-2)*{\bullet}="A";
(0,-2)*{\bullet}="B";
"A"; "B" **\crv{(6,6) & (-6,6)};
\endxy
 \in {{\mathsf f}}{{\mathsf G}}{{\mathsf C}}_2
$
has degree $-1$ and satisfies
$$
[\xy
(0,-2)*{\bullet}="A";
(0,-2)*{\bullet}="B";
"A"; "B" **\crv{(6,6) & (-6,6)};
\endxy  ,   \xy
(0,-2)*{\bullet}="A";
(0,-2)*{\bullet}="B";
"A"; "B" **\crv{(6,6) & (-6,6)};
\endxy  ]=[\xy
(0,-2)*{\bullet}="A";
(0,-2)*{\bullet}="B";
"A"; "B" **\crv{(6,6) & (-6,6)};
\endxy , \xy
 (0,0)*{\bullet}="a",
(6,0)*{\bullet}="b",
\ar @{-} "a";"b" <0pt>
\endxy]=0.
$$
Let $u$  be a formal variable of degree $2$ and consider the graph complex ${{\mathsf f}}{{\mathsf G}}{{\mathsf C}}_2[[u]]$ with the differential
$$
d_u:= d+ u \Delta,\ \ \ \ \ \mbox{where}\ \ \ \Delta:=[\xy
(0,-2)*{\bullet}="A";
(0,-2)*{\bullet}="B";
"A"; "B" **\crv{(6,6) & (-6,6)};
\endxy  , \ \ ].
$$
The subspace ${{\mathsf G}}{{\mathsf C}}_2[[u]]\subset {{\mathsf f}}{{\mathsf G}}{{\mathsf C}}_2[[u]]$ is a subcomplex of $({{\mathsf f}}{{\mathsf G}}{{\mathsf C}}_2[[u]], d_u)$.

\begin{proposition}\label{2: prop on grt}
$H^0\left({{\mathsf G}}{{\mathsf C}}_2[[u]], d_u\right)\simeq {{\mathfrak{{grt}}}_1}$ and $H^{\leq -1}\left({{\mathsf G}}{{\mathsf C}}_2[[u]]\right)=0$.
\end{proposition}
\begin{proof}
Consider a decreasing filtration of ${{\mathsf G}}{{\mathsf C}}_2[[{u}]]$ by the powers in ${u}$. The first term of the associated spectral sequence is
$$
{{\mathcal E}}_1= \bigoplus_{i\in {{\mathbb Z}}} {{\mathcal E}}_1^i,\ \ \ \ {{\mathcal E}}_1^i=\prod_{p\geq 0} H^{i-2p}({{\mathsf G}}{{\mathsf C}}_2, d) u^p
$$
with the differential equal to $u\Delta$.
As $H^0({{\mathsf G}}{{\mathsf C}}_2, d)\simeq {{\mathfrak{{grt}}}_1}$
and $H^{\leq -1}({{\mathsf G}}{{\mathsf C}}_2, d)=0$, one gets the desired result.  
$H^0\left({{\mathsf f}}{{\mathsf G}}{{\mathsf C}}_2[[u]], d_u\right)\simeq {{\mathfrak{{grt}}}_1}$.

The projections $({{\mathsf G}}{{\mathsf C}}_2[[u]],d_u)\to ({{\mathsf G}}{{\mathsf C}}_2,d)$ and $({{\mathsf f}}{{\mathsf G}}{{\mathsf C}}_2[[u]],d_u)\to ({{\mathsf f}}{{\mathsf G}}{{\mathsf C}}_2,d)$ sending $u$ to 0 are maps of Lie algebras and induce isomorphisms in degree 0 cohomology. Since the isomorphisms of Theorem {\ref{thm:negzero}} (i) are maps of Lie algebras as shown in \cite{Wi}, so are the maps in the above Proposition.
\end{proof}

\subsection{Remark}\label{2:Remark on induction}  Let $\sigma$ be an element of ${{\mathfrak{{grt}}}_1}$ and let ${\Gamma}_\sigma^{(0)}$ be any cycle representing the cohomology class  $\sigma$  in the graph complex
$({{\mathsf G}}{{\mathsf C}}_2, d)$.
Then one can construct a cocycle,
\begin{equation}\label{cyclic_repr}
\Gamma^{u}_\sigma= {\Gamma}_\sigma^{(0)}  + {\Gamma}_\sigma^{(1)}{u} +  {\Gamma}_\sigma^{(2)}{u}^2 +  {\Gamma}_\sigma^{(3)}{u}^3+ \ldots,
\end{equation}
 representing the cohomology class $\sigma\in{{\mathfrak{{grt}}}_1}$ in the complex $\left({{\mathsf G}}{{\mathsf C}}_2[[u]], d_u\right)$ by the following induction:
{\smallskip}

{\em 1st step}: As $d{\Gamma}_\sigma^{(0)}=0$, we have $d (\Delta {\Gamma}_\sigma^{(0)})=0$. As $H^{-1}( {{\mathsf G}}{{\mathsf C}}_2, d)=0$,
there exists ${\Gamma}_\sigma^{(1)}$ of degree $-2$ such that $\Delta {\Gamma}_\sigma^{(0)}= - d{\Gamma}_\sigma^{(1)}$ and hence
$$
(d +u \Delta) \left({\Gamma}_\sigma^{(0)}  + {\Gamma}_\sigma^{(1)}{u}\right)=0 \bmod O({u}^2).
$$

{\em n-th step}: Assume we have constructed  a polynomial $\sum_{i=1}^n  {\Gamma}_\sigma^{(i)}{u}^i$ such that
 $$
(d +{u} \Delta) \sum_{i=1}^n  {\Gamma}_\sigma^{(i)}{u}^i=0 \bmod O({u}^{n+1}).
$$
Then $d (\Delta {\Gamma}_\sigma^{(n)})=0$, and, as  $H^{-2n-1}( {{\mathsf G}}{{\mathsf C}}_2, d)=0$, there exists a  graph
${\Gamma}_\sigma^{(n+1)}$ in ${{\mathsf G}}{{\mathsf C}}_2$ of degree $-2n-2$ such that $\Delta {\Gamma}_\sigma^{(n)})=- d {\Gamma}_\sigma^{(n+1)}$. Hence $(d +{u} \Delta) \sum_{i=1}^{n+1}  {\Gamma}_\sigma^{(i)}{u}^i=0 \bmod O({u}^{n+2})$.

{\bigskip}

{\bigskip}

{\large
\section{\bf Quantum BV structures on odd symplectic manifolds}
}

{\bigskip}

\subsection{Maurer-Cartan elements and gauge tranformations}
Let $({{\mathfrak g}}=\oplus_{i\in {{\mathbb Z}}} {{\mathfrak g}}^i, [\ ,\ ], d)$ be a dg Lie algebra and consider the dg Lie algebra ${{\mathfrak g}}_\hbar := \hbar {{\mathfrak g}}[[\hbar]]=:\oplus_{i\in {{\mathbb Z}}} {{\mathfrak g}}^i_\hbar$, where $\hbar$ is a formal deformation parameter.
The group $G:=\exp({{\mathfrak g}}_{\hbar}^0)$ (which is, as a set, ${{\mathfrak g}}_{\hbar}^0$ equipped with the standard Baker-Campbell-Hausdorff  multiplication) acts on ${{\mathfrak g}}^1_\hbar$,
$$
{\gamma} {\rightarrow} \exp(h)\cdot {\gamma}:= e^{\mathrm{ad}_h}{\gamma} -\frac{e^{\mathrm{ad}_h}-1}{\mathrm{ad}_h}dh,
$$
preserving its subset of Maurer-Cartan elements
$$
{{\mathcal M}}{{\mathcal C}}({{\mathfrak g}}_\hbar) = \{{\gamma}\in {{\mathfrak g}}^1_\hbar | d{\gamma} + \frac{1}{2}[{\gamma},{\gamma}]=0\}.
$$
We call the $G$-orbits in ${{\mathcal M}}{{\mathcal C}}({{\mathfrak g}}_\hbar)$ the gauge equivalence classes of Maurer-Cartan elements.

 {\smallskip}

The group of $L_\infty$ automorphism of ${{\mathfrak g}}$ acts on ${{\mathcal M}}{{\mathcal C}}({{\mathfrak g}}_\hbar)$ by the formula
$$
F\cdot {\gamma} := \sum_{n\geq 1}\frac{1}{n!} F_n({\gamma}, \ldots, {\gamma})
$$
where $F_n$ is the $n$-th component of the $L_\infty$ morphism.
In particular, let $f$ be an $L_\infty$ derivation of ${{\mathfrak g}}$ without linear term. It exponentiates to an $L_\infty$ automorphism $\exp(f)$ of ${{\mathfrak g}}$, which acts on ${{\mathcal M}}{{\mathcal C}}({{\mathfrak g}}_\hbar)$, and in particular on the set of gauge equivalence classes.
By a small calculation one may check that if we change $f$ by homotopy, i.~e., by adding $dh$ for some degree 0 element $h$ of the Chevalley-Eilenberg complex of ${{\mathfrak g}}$, then the induced actions of $\exp(f)$ and $\exp(f+dh)$ on the set of gauge equivalence classes agree.

\subsection{Quantum BV manifolds} Let $M$ be a ${{\mathbb Z}}$-graded manifold equipped with an odd symplectic structure ${\omega}$ (of degree $1$). There always exist so called Darboux coordinates,
$(x^a, \psi_a)_{1\leq a\leq n}$, on $M$ such that $|\psi_a|=-|x^a| + 1$ and ${\omega}=\sum_a dx^a\wedge d\psi_a$. The odd symplectic structure makes, in the obvious way, the structure sheaf  
into a Lie algebra with brackets, $\{\ ,\ \}$, of degree $-1$. A less obvious fact is that ${\omega}$ induces
a  degree $-1$ differential operator, $\Delta_{\omega}$, on the invertible sheaf of semidensities, ${{\mathit B}{\mathit e} {\mathit r}}(M)^{\frac{1}{2}}$ \cite{Kh}. Any choice of a Darboux coordinate system on $M$ defines an associated trivialization of the sheaf ${{\mathit B}{\mathit e} {\mathit r}}(M)^{\frac{1}{2}}$; if one denotes the associated basis section
of  ${{\mathit B}{\mathit e} {\mathit r}}(M)^{\frac{1}{2}}$ by $D_{x,\psi}$, then  any semidensity $D$ is of the form $f(x,\psi) D_{x,\psi}$ for some smooth function $f(x,\psi)$, and the operator $\Delta_{\omega}$ is given by
 $$
 \Delta_\omega\left(f(x,\psi) D_{x,\psi}\right)=\sum_{a=1}^n 
  \frac{{{\partial}}^2f}{{{\partial}} x^a {{\partial}} \psi_a} D_{x,\psi}.
 $$
Let ${u}$ be a formal parameter of degree $2$. A {\em quantum
master function}\ on $M$ is a ${u}$-dependent semidensity $D$ which satisfies the equation
$$
\Delta_\omega D=0
$$
and which admits, in some Darboux coordinate system, a form
$$
D=e^{\frac{S}{u}}D_{x,\psi},
$$
for some  $S\in {{\mathcal O}}_M[[{u}]]$ of total degree $2$, where ${{\mathcal O}}_M$ is the algebra of functions on $M$. In the literature it is this formal power series
in ${u}$ which is often called a quantum master function. Let us denote the set of all quantum master functions on $M$ by ${{\mathcal Q}}{{\mathcal M}}(M)$. It is easy to check that the equation $\Delta_{\omega} D=0$ is equivalent to the following one,
\begin{equation}\label{3:QME}
{u}\Delta S + \frac{1}{2}\{S,S\}=0,
\end{equation}
where $\Delta:=\sum_{a=1}^n  \frac{{{\partial}}^2}{{{\partial}} x^a {{\partial}} \psi_a}$. This equation is often called the {\em quantum master equation}, while a triple $(M, {\omega}, S\in {{\mathcal Q}}{{\mathcal M}}(M))$ a {\em quantum BV manifold}.

{\smallskip}

Let us assume from now on that $M$ is affine or formal (i.~e., we work with $\infty$-jets of functions at some point) and that a particular Darboux coordinate system is fixed on $M$ up to affine transformations\footnote{This is not a serious loss of generality as any quantum master equation can be represented in the form (\ref{3:QME}). Our action of ${GRT_1}$ on ${{\mathcal Q}}{{\mathcal M}}_\hbar(M)$
depends on the choice of an affine structure on $M$ in exactly the same way as
the classical Kontsevich's  formula for a universal formality  map \cite{Ko2} depends on such a choice. A choice of an appropriate affine connection on $M$ and methods of the paper \cite{D} can make our formulae for the ${GRT_1}$ action invariant under the group of symplectomorphsims of $(M,{\omega})$; we do not address this {\em globalization}\, issue in the present note.} so that the algebra of function on $M$ is ${{\mathcal O}}_M\cong {{\mathbb K}}[x^a,\psi_a]$ or ${{\mathcal O}}_M\cong {{\mathbb K}}[[x^a,\psi_a]]$.

{\smallskip}

For later reference we will also consider solutions of \eqref{3:QME} that depend on a formal deformation parameter $\hbar$ of degree 0, $S\in \hbar {{\mathcal O}}_M[[u]][[\hbar]]$. We will call the set of such $S$ the \emph{set of formal solutions of the quantum master equation}\, and denote it by ${{\mathcal Q}}{{\mathcal M}}_\hbar(M)$.

\subsection{An action of ${GRT_1}$ on quantum master functions}\label{3: subsection on GRT action on qmfunctions}
The constant odd symplectic structure on $M$ makes ${{\mathcal O}}_M$ into a representation
\begin{equation}
\begin{array}{rccc}
\rho: & {{\mathsf G}} {{\mathsf r}}{{\mathsf a}}(n) & {\longrightarrow} & {{\mathsf E}} \mathsf n \mathsf d_V(n)={{\mathrm H\mathrm o\mathrm m}}_{cont}({{\mathcal O}}_M^{{\otimes} n},{{\mathcal O}}_M)\\
      & {\Gamma} &{\longrightarrow} & \Phi_{\Gamma}
\end{array}
\end{equation}
of the operad ${{\mathsf G}} {{\mathsf r}}{{\mathsf a}}$ as follows:
$$
\Phi_{\Gamma}(S_1,\ldots, S_n) :=\pi\left(\prod_{e\in E({\Gamma})} \Delta_e \left(S_1(x_{(1)}, \psi_{(1)},{u}){\otimes} S_2(x_{(2)}, \psi_{(2)},{u}){\otimes} \ldots{\otimes} S_n(x_{(n)}, \psi_{(n)},{u}) \right)\right)
$$
where, for an edge $e$ connecting vertices labeled by integers $i$ and $j$,
$$
\Delta_e= \sum_{a=1}^n \frac{{\partial}}{{{\partial}} x_{(i)}^a}  \frac{{\partial}}{{{\partial}} \psi_{a(j)}}
+
\frac{{\partial}}{{{\partial}} \psi_{a(i)}}  \frac{{\partial}}{{{\partial}} x_{(j)}^a}
$$
with the subscript $(i)$ or $(j)$ indicating that the derivative operator is to be applied to the $i$-th of $j$-th factor in the tensor product.
The symbol $\pi$ in \eqref{3: repr rho of Gra} denotes the multiplication map,
$$
\begin{array}{rccc}
\pi:&   V^{{\otimes} n} & {\longrightarrow} & V\\
   & S_1{\otimes} S_2{\otimes} \ldots {\otimes} S_n &{\longrightarrow} & S_1S_2\cdots S_n.
\end{array}
$$

Let $V:={{\mathcal O}}_M[[{u}]]$. Then by ${u}$-linear extension we obtain a continuous representation (in the category of topological ${{\mathbb K}}[[{u}]]$-modules)
\begin{equation} \label{3: repr rho of Gra}
 {{\mathsf G}} {{\mathsf r}}{{\mathsf a}}[[u]]  {\longrightarrow}  {{\mathsf E}} \mathsf n \mathsf d_V={{\mathrm H\mathrm o\mathrm m}}_{cont}(V^{{\otimes} \cdot},V).
\end{equation}

The space $V[1]$ is a topological dg Lie algebra with differential ${u} \Delta$ and
Lie bracket $\{\ ,\ \}$. These data define a Maurer-Cartan element, ${\gamma}_{{{\mathcal Q}}{{\mathcal M}}}:={u}\Delta \oplus
 \{\ ,\ \}$ in the Lie algebra $({{\mathsf E}} \mathsf n \mathsf d_V\{-2\})^{{\mathbb S}}\subset CE^{\bullet}(V,V)$, where $CE^{\bullet}(V,V)$ is the Lie algebra of  coderivations
 $$
 CE^{\bullet}(V,V)=\left(\mbox{Coder}(\odot^{{\bullet}\geq 1}(V[2])), [\ ,\ ] \right)\ \ \mathrm{with}\ \ \ CE^{\bullet}(V,V)_{(m)}:={{\mathrm H\mathrm o\mathrm m}}(\odot^{{\bullet}\geq m+1}(V[2]), V[2]),
 $$
of the standard graded co-commutative coalgebra, $\odot^{{\bullet}\geq 1}(V[2])$, co-generated by a vector space $V$. The set ${{\mathcal M}}{{\mathcal C}}(CE^{\bullet}(V,V))$ can be identified with the set of
 $L_\infty$ structures 
 on the space $V[1]$.
 

{\smallskip}

The map sending an operad ${{\mathcal P}}$ to the Lie algebra of invariants $\prod_n {{\mathcal P}}\{-2\}(n)^{{{\mathbb S}}_n}$ is functorial. Hence, from the representation \eqref{3: repr rho of Gra} we obtain a map of graded Lie algebras
$$
{{\mathsf f}}{{\mathsf G}}{{\mathsf C}}_2[[u]]\cong ({{\mathsf G}} {{\mathsf r}}{{\mathsf a}}\{-2\}[[u]])^{{\mathbb S}} \to ({{\mathsf E}} \mathsf n \mathsf d_V\{-2\})^{{\mathbb S}} \subset CE^{\bullet}(V,V)
$$
One checks that the Maurer-Cartan element
$$ \xy
 (0,0)*{\bullet}="a",
(6,0)*{\bullet}="b",
\ar @{-} "a";"b" <0pt>
\endxy
+ {u} \, \xy
(0,-2)*{\bullet}="A";
(0,-2)*{\bullet}="B";
"A"; "B" **\crv{(6,6) & (-6,6)};
\endxy \in {{\mathsf f}}{{\mathsf G}}{{\mathsf C}}_2[[{u}]]$$
is sent to the Maurer-Cartan element ${\gamma}_{{{\mathcal Q}}{{\mathcal M}}}\in CE^{\bullet}(V,V)$. Hence we obtain a morphism of dg Lie algebras
$$
 \left({{\mathsf f}}{{\mathsf G}}{{\mathsf C}}_2[[{u}]], [\ ,\ ], d_h\right) {\longrightarrow} \left(CE^{\bullet}(V,V), [\ ,\ ], \delta:=[{\gamma}_{{{\mathcal Q}}{{\mathcal M}}},\ ]\right),
$$
and by restriction a morphism
$$
 \Phi\colon \left({{\mathsf G}}{{\mathsf C}}_2[[{u}]], [\ ,\ ], d_h\right) {\longrightarrow} \left(CE^{\bullet}(V,V), [\ ,\ ], \delta:=[{\gamma}_{{{\mathcal Q}}{{\mathcal M}}},\ ]\right),
$$
Hence we also obtain a morphism of their cohomology groups,
$$
{{\mathfrak{{grt}}}_1}\simeq H^0\left({{\mathsf G}}{{\mathsf C}}_2[[{u}]], d_{u}\right) {\longrightarrow} H^0\left(CE^{\bullet}(V,V), \delta\right).
$$
 Let $\sigma$ be an arbitrary element in ${{\mathfrak{{grt}}}_1}$ and let
 ${\Gamma}^{u}_\sigma$ be a cocycle representing $\sigma$ in the graph complex  $({{\mathsf G}}{{\mathsf C}}_2[[{u}]], d_{u})$.
 We may assume that ${\Gamma}^{u}_\sigma$ consists of graphs with at least 4 vertices, see \cite{Wi}.
 Then the element $\Phi({\Gamma}_\sigma^{u})$ describes an $L_\infty$ derivation of the Lie algebra $V[1]$ without linear term.
 By exponentiation we obtain an $L_\infty$ automorphism,
$$
F^\sigma=\left\{F^\sigma_n: \odot^n V {\longrightarrow} V[2-2n]\right\}_{n\geq 1},
$$
of the dg Lie algebra $(V[1], {u}\Delta, \{\ ,\ \})$  with $F^\sigma_1={{\mathrm I\mathrm d}}$.
Hence,  for any  formal quantum master function $S\in {{\mathcal Q}}{{\mathcal M}}_\hbar(M)$ the series
$$
S^\sigma:= S + \sum_{n\geq 2}\frac{1}{n!} F^\sigma_n(S, \ldots, S)
$$
gives again a formal quantum master function.\footnote{The series trivially converges since we work in the formal setting, i.~e., $S=\hbar(\cdots)$. Ideally, of course, one hopes to have a non-zero convergence radius in $\hbar$, but we cannot guarantee this.}
The induced action on gauge equivalence classes of such functions is well defined, i.~e., it does not depend on the representative ${\Gamma}^{u}_\sigma$ chosen. 
This is the acclaimed homotopy action of ${GRT_1}$ on  ${{\mathcal Q}}{{\mathcal M}}_\hbar(M)$
for any affine odd symplectic manifold $M$.

\begin{remark}
 As pointed out by one of the referees, there is also a stronger notion of ``homotopy action'' that holds in our setting. We will only consider the infinitesimal version.
 Then, we do not only have a Lie algebra morphism ${{\mathfrak{{grt}}}_1}\to H^0\left(CE^{\bullet}(V,V)\right)$, but an $L_\infty$ morphism ${{\mathfrak{{grt}}}_1}\to CE^{\bullet}(V,V)$ as follows. First, consider the truncated version $\left({{\mathsf G}}{{\mathsf C}}_2[[{u}]]\right)^{tr}$ of the dg Lie algebra ${{\mathsf G}}{{\mathsf C}}_2[[{u}]]$, which is by definition the same as ${{\mathsf G}}{{\mathsf C}}_2[[{u}]]$ in negative degrees, zero in positive degrees, and consists of the degree zero cocycles in degree zero. By Proposition {\ref{2: prop on grt}} the canonical projection $\left({{\mathsf G}}{{\mathsf C}}_2[[{u}]]\right)^{tr}\to {{\mathfrak{{grt}}}_1}$ is a quasi-isomorphism. Hence we can obtain the desired $L_\infty$ morphism ${{\mathfrak{{grt}}}_1}\to CE^{\bullet}(V,V)$ by lifting the zig-zag
 \[
  {{\mathfrak{{grt}}}_1} \stackrel{\sim}{\longrightarrow} \left({{\mathsf G}}{{\mathsf C}}_2[[{u}]]\right)^{tr} {\longrightarrow} CE^{\bullet}(V,V).
 \]
 This proves the first claim of the main Theorem.
\end{remark}

\subsection{Remark}
It is a well known result due to D. Tamarkin \cite{T} that the Grothendieck Teichm\"uller group ${GRT_1}$ acts on the operad of chains of the little disks operad.
In fact, one can show that this ${GRT_1}$ action extends to an action on the operad of chains of the framed little disks operad, which is quasi-isomorphic to the Batalin-Vilkovisky operad. Hence one obtains in particular an action of ${GRT_1}$ on the set of Batalin-Vilkovisky algebra structures on any vector space, and on their deformations, up to homotopy.
In our setting the algebra ${{\mathcal O}}_M$ is an algebra over the framed little disks operad. Any solution $S=S_0+u S_1+u^2 S_2+\cdots$ of the master equation \eqref{3:QME} yields a deformation of the Batalin-Vilkovisky structure on ${{\mathcal O}}_M$, up to homotopy. Concretely, to $S$ one may associate a $BV_\infty^{com}$-structure (see \cite{Kr} or \cite[section 5.3]{CMW}), whose $n$-th order ``BV'' operator is defined as $\Delta_n := [S_n,\cdot]$ (notation as in \cite[section 5.3]{CMW}).
The ${GRT_1}$ action on solutions of the master equation described above can hence be seen as a shadow of this more general action of ${GRT_1}$ on the framed little disks operad.
However, we leave the details to elsewhere.

\subsection*{Acknowledgements} We are grateful to K. Costello and to the anonymous referees for useful critical comments.

\begin{thebibliography}{10}

\bibitem[CMW]{CMW} R. Campos, S. Merkulov and T. Willwacher, {\em The Frobenius properad is Koszul}, preprint arXiv:1402.4048.

\bibitem[Do]{D} V.\ Dolgushev, {\em Covariant and Equivariant Formality Theorems},
Adv. Math., Vol. {\bf 191}, 1 (2005) 147--177.

\bibitem[Dr]{Dr}
V. Drinfeld, {\em On quasitriangular quasi-Hopf algebras and a group closely connected
with $Gal(\bar{Q}/Q)$}, Leningrad Math. J. {\bf 2}, No.\ 4 (1991),  829--860.

\bibitem[GV]{GV} M.\ Gerstenhaber and A.~A.\ Voronov, {\em Homotopy $G$-algebras and moduli space operad}. IMRN {\bf 3} (1995) 141--153.

\bibitem[KM]{KM} M.\ Kapranov and Yu.~I.\ Manin, {\em Modules and Morita theorem for operads}. Amer. J. Math. {\bf 123} (2001), no. 5, 811--838.

\bibitem[Kh]{Kh} H.\ Khudaverdian, {\em Semidensities on odd symplectic supermanifolds},
 Commun.\ Math.\ Phys. {\bf 247} (2004), 353--390.

 \bibitem[Ko1]{Ko} M. Kontsevich, {\em Formality Conjecture}, In: D. Sternheimer et al. (eds.),
Deformation Theory and Symplectic
Geometry, Kluwer 1997, 139--156.

\bibitem[Ko2]{Ko2} M.\ Kontsevich, {\em Deformation quantization
 of Poisson manifolds}, Lett.\ Math.\ Phys. {\bf 66} (2003), 157--216.

  \bibitem[Kr]{Kr}
 O.\ Kravchenko, {\em Deformations of Batalin-Vilkovisky algebras}, In: Poisson Geometry
 (Warsaw, 1998), Banach Center Publ.,
vol. 51, Polish Acad. Sci., Warsaw, 2000, pp. 131--139.

 \bibitem[LV]{LV}
J.-L. Loday and B.~Vallette.
\newblock {\em {Algebraic Operads}}.
\newblock Number 346 in {Grundlehren der mathematischen Wissenschaften}.
  {Springer, Berlin}, {2012}.

\bibitem[Sc]{Sc} A.\ Schwarz, {\em Geometry of Batalin-Vilkovisky quantization},
Commun.\ Math.\ Phys.\
{\bf 155} (1993), 249--260.

\bibitem[T]{T}
D. Tamarkin, {\em Action of the Grothendieck-Teichm\"uller group on the operad of
  Gerstenhaber algebras}, preprint arXiv:math/0202039.

\bibitem[Wi]{Wi} T.\ Willwacher, {\em M.\ Kontsevich's graph complex and the Grothendieck-Teichm\"uller Lie algebra},
preprint arXiv:1009.1654.

 \end{thebibliography}

\end{document} 

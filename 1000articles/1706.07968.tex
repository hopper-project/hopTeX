\documentclass[12pt,reqno]{amsart}
\usepackage{amsmath,amssymb,hyperref,mathrsfs,graphicx}
\usepackage[utf8x]{inputenc}
\usepackage{amsmath}
\usepackage{amsfonts}
\usepackage{latexsym}
\usepackage{amsthm}
\usepackage{amssymb,amscd}
\usepackage{xargs}
\usepackage{graphicx}
\usepackage{color}
\usepackage{enumitem}

\usepackage{lmodern}

\usepackage[text={5.7in, 8in}, centering]{geometry}
 
\usepackage{microtype}

\usepackage{hyperref}

\newtheorem{thm}{Theorem}[section]
\newtheorem{conj}{Conjecture}[section]
\newtheorem{lem}[thm]{Lemma}
\newtheorem{defn}[thm]{Definition}
\newtheorem{prop}[thm]{Proposition}
\newtheorem{cor}[thm]{Corollary}

\newtheorem{ex}[thm]{Example}

\theoremstyle{remark}
\newtheorem*{rmk}{Remark}

\subjclass[2010]{Primary 37J40, Secondary 37J50, 70H08}
\keywords{billiard map, rational caustic, KAM theorem, Nekhoroshev theorem}

\title{Density of convex billiards with rational caustics}

\author{Vadim Kaloshin}
\author{Ke Zhang}

\begin{document}
\maketitle
\begin{abstract}
	We show that in the space of all convex billiard boundaries, the set of boundaries with rational caustics is dense. More precisely, the set of billiard boundaries with caustics of rotation number $1/q$ is polynomially sense in the smooth case, and exponentially dense in the analytic case. 
\end{abstract}

\section{Introduction}

Let $\Omega\subset {\mathbb{R}}^2$ be a strictly convex billiard table, assume that its boundary is given by 
\[
	\partial \Omega = 
	\left\{ r(s) \in {\mathbb{R}}^2 \right\}, \quad r(s) = r(s + 1),\,  s \in {\mathbb{R}}.
\]
By applying a translation, we can assume that $0 \in \Omega$. Let $\kappa(s)$ denote the curvature $\partial\Omega$ at $r(s)$, we assume for $D > 1$:
\begin{equation}
  \label{eq:real-assumption}
  	|\dot{r}(s)| \ge D^{-1}, \quad \kappa(s) \ge D^{-1}. \tag{A1} 
\end{equation}

We are interested in the existence of rational caustics of rotation number $1/q$. Baryshnikov and Zharnitsky (\cite{BZ2006}) showed that boundaries admitting caustics of a fixed rational rotation number is a finite co-dimension sub-manifold among all boundaries. It is natural to investigate how dense such sub-manifolds are, within the space of all boundaries. Analog to Hamiltonian averaging suggests the density of such billiards should be polynomial in $q$ in the smooth case, and exponential in $q$ in the analytic case. Indeed, Mart\'{i}n, Ram\'{i}rez-Roz and Tamarit-Sariol showed (\cite{MRT2016}), in the analytic case, Mather's $\Delta W_{p/q}$ function is exponentially small. This means that the billiard is ``exponentially close'' to having a caustic. This, however, does not mean one can perturb the boundary by an exponentially small amount to create a caustic. The reason is the billiard dynamics depends rather implicitly on the boundary, and it is not obvious how to perturb the boundary to obtain the desired caustics. 

Our main result is that this indeed can be done. 
\begin{thm}\label{thm:main}
\begin{enumerate}
 \item  Suppose $r$ is a real analytic boundary, extensible to a complex neighborhood of size $\sigma_0$.  Then there is $C>1$ depending only on $\sigma_0, D$ and the analytic norm of $r$, such that for $q$ sufficiently large, there a real analytic boundary $r_{\mathrm{cau}}$ admitting a caustic of rotation number $1/q$, satisfying 
 \[
 	\|r - r_{\mathrm{cau}}\|_{\sigma_0/8} < C e^{- C^{-1} q},
 \] 
 there $\|\cdot\|_{\sigma}$ is the supremum norm of the complex extension. 
 \item Suppose $r$ is $C^m$, then for each $l < m$,  there is $C > 1$ depending only on $D$, $\|r\|_{C^m}$, $l$, $m$, and a $C^\infty$ boundary $r_{\mathrm{cau}}$ admitting a caustic of rotation number $1/q$, such that 
 \[
 	\|r - r_{\mathrm{cau}}\|_{C^l} < C q^{\frac{m-l}{4}}. 
 \]
\end{enumerate}
\end{thm}
\begin{rmk}
The density obtained in item (2) is not optimal. We expect, one can get to $(1-\delta)(m-l)$ for arbitrarily small $\delta$, as long as $m$ is large enough. Doing this requires an improvement of Proposition~\ref{prop:gen-laz} to arbitrary order, and more technical estimates. To limit the degree of technicality, we prefer to just prove the weaker result. 
\end{rmk}

The billiard problem has a caustic of rotation number $1/q$ if there exists a homeomorphism $u: {\mathbb{R}} \to {\mathbb{R}}$ satisfying $u(\theta +1) = u(\theta) + 1$, such that 
\begin{equation}
	\label{eq:inv-cur}
	E(r, u) := \partial_{s'} L(r, u(\theta - \alpha), u(\theta)) + \partial_s L(r, u(\theta), u(\theta + \alpha)), 
\end{equation}
where 
\[
	L(r; s, s') = |r(s') - r(s)|, \quad  \alpha = 1/q.
\]
This idea is due to Moser and Levi (\cite{LM2001}) where they prove a KAM theorem using the Lagrangian setting. 

To obtain a solution to $E(r, u) = 0$, we first perform coordinate changes to obtain approximate solutions. This is done in two steps: First we perform averaging of Lazutkin-type (\cite{Laz1973}), to convert the billiard to a map close to rigid rotation. We then apply a Nekhoroshev-type (see \cite{LN1992} for more background) averaging, which in the analytic case, proves the existence of an approximate caustic with only exponential error. Note that this step is in fact done in \cite{MRT2016}, however, their result does not provide the quantitative estimates needed for the smooth version. We provide an alternative approach which only use the Lagrangian setting. 

Suppose $(r, u)$ is an approximate solution to \eqref{eq:E}, we seek  a real solution $E({\tilde{r}}, \tilde{u}) = 0$ close to $(r, u)$. Given any function $g: {\mathbb{T}} \to {\mathbb{R}}$ with $g(\theta) = \sum_{k \in {\mathbb{Z}}} g_k e^{2 \pi i k \theta}$, we define 
\[
	[g]_q = \sum_{k \in q{\mathbb{Z}}} g_k e^{2\pi k \theta} = \frac{1}{q}\sum_{i =1}^q g(\theta + k \alpha), \quad \{g\}_q = g - [g]_q.  
\]
We call a function $g$ resonant if $[g]_q = g$, and non-resonant if $[g]_q = 0$. 

The main idea is to use the $r$ deformation to kill the resonant part, and adjust $u$ to kill the non-resonant part. For the first part, we show that there is an explicit deformation which projects $E(r, u)$ to the space of non-resonant functions. Indeed, in Corollary~\ref{cor:proj} we show that for each $(r, u)$ there is $a: {\mathbb{T}} \to {\mathbb{R}}$ such that 
\[
	[u_\theta E(e^a r, u)]_q = 0. 
\]

Therefore, we may assume that $[u_\theta E(r, u)]_q = 0$. In this case, the method of Moser and Levi (\cite{LM2001}) provides a solution to 
\[
	u_\theta \partial_u E(r, u) \cdot v = - u_\theta E(r, u) - v \frac{d}{d\theta} E(r, u),
\]
which allows a KAM-type iteration to find a solution. Moreover, due to the identity 
\[
	E(r \circ u, {\mathrm{id}}) = u_\theta E(r, u), 
\]
we can perform the iteration at $u = {\mathrm{id}}$. 

The outline of this paper is as follows. In Section~\ref{sec:comp}, we construct the projection to non-resonant space, and recall the Moser-Levi algorithm. In Section~\ref{sec:analytic-est}, we perform basic estimates in the analytic norm. The Nekhoroshev averaging is performed in Section~\ref{sec:nek}, and the Lazutkin normal form is in Section~\ref{sec:laz} with some details deferred to the appendix. The KAM iteration is done in Section~\ref{sec:kam}, where we also prove the main theorem. 

\section{Basic computations}
\label{sec:comp}

Let us use the following notations:
\begin{enumerate}
	\item Denote $u(\theta)$ by $u$ when there is no confusion, we write $u^- = u(\theta  -\alpha)$, $u^+ = u(\theta + \alpha)$.
	\item For any function $f$ of $s$, $\Delta f$ denotes $f(u^+) - f(u)$. Under the same convention, $\Delta f^-$ denotes $f(u) - f(u^-)$. Note that the composition with $u$ is implied whenever $\Delta$ notation is used. 
\end{enumerate}
Then
\begin{equation}
	\label{eq:E}
	E(r, u)  = \partial_{s'} L(r, u^-, u) + \partial_{s} L(r, u, u^+) 
	= \left\langle   
	\frac{\Delta r^-}{|\Delta r^-|} - \frac{\Delta r}{|\Delta r|} , \dot{r}(u) 
	\right\rangle .
\end{equation}

\begin{lem}\label{lem:prjection}
Let $a: {\mathbb{T}} \to {\mathbb{R}}$ be such that $a(u^+) = a(u)$. Then 
\[
	E(e^{a}r, u) = e^{a(u)} \left( \dot{a}(u) F(r, u) +  E(r, u) \right), 
\]
where 
\[
	F(r, u) = \left\langle
	\frac{\Delta r^-}{|\Delta r^-|} - \frac{\Delta r}{|\Delta r|} , r(u)
	\right\rangle. 
\]
\end{lem}
\begin{proof}
	Write ${\tilde{r}} = e^a r$, then
	\[
		\Delta {\tilde{r}} = e^{a(u^+)} r(u^+)  - e^{a(u)} r(u) = 
		e^{a(u)}\Delta r, 
	\]
	as a result, $\Delta {\tilde{r}} / |\Delta {\tilde{r}}| = \Delta r / |\Delta r|$. Since 
	\[
		\frac{d}{ds} {\tilde{r}} = e^a\left(  \dot{a} r +  \dot{r} \right), 
	\]
	from \eqref{eq:E} we get 
	\[
		E({\tilde{r}}, u) = \dot{a}(u) e^{a(u)} \left\langle  \frac{\Delta r^-}{|\Delta r^-|} - \frac{\Delta r}{|\Delta r|} ,  r \right\rangle + a e^{a(u)} \left\langle  \frac{\Delta r^-}{|\Delta r^-|} - \frac{\Delta r}{|\Delta r|} , \dot{r} \right\rangle. 
	\]
\end{proof}

\begin{lem}\label{lem:F-const}
The function $F(r, u)$ has the following special property:
\[
	[F(r, u)]_q = [|\Delta r|]_q, 
	\quad 
	\frac{d}{d\theta}[F(r, u)]_q =  [u_\theta E(r, u)]_q. 
\]
\end{lem}
\begin{proof}
	We have 
	\[
		\begin{aligned}
			{[F(r, u)]_q}  &= \frac{1}{q} \sum_{k = 1}^q \left \langle \frac{\Delta r^-}{|\Delta r^-|} , r \right \rangle \circ (\theta + k \alpha) -
			\frac{1}{q} \sum_{k = 1}^q \left \langle \frac{\Delta r}{|\Delta r|} , r \right \rangle \circ (\theta + k \alpha) \\
			& = \frac{1}{q} \sum_{k = 1}^q \left \langle \frac{\Delta r}{|\Delta r|} , r^+ \right \rangle \circ (\theta + k \alpha) -
			\frac{1}{q} \sum_{k = 1}^q \left \langle \frac{\Delta r}{|\Delta r|} , r \right \rangle \circ (\theta + k \alpha) \\
			& = \frac{1}{q} \sum_{k = 1}^q \left \langle \frac{\Delta r}{|\Delta r|} , \Delta r \right \rangle \circ (\theta + k \alpha) = \frac{1}{q} \sum_{k = 1}^q |\Delta r| \circ  (\theta + k \alpha). 
		\end{aligned}
	\]
	The second formula follows a standard computation. 
\end{proof}

\begin{cor}\label{cor:proj}
Given $(r, u)$ define $a$ by the equation 
\begin{equation}
	\label{eq:a}
	\frac{d}{d\theta} a(u) = \dot{a}(u) u_\theta = - \frac{[u_\theta E(r, u)]_q}{[F(r, u)]_q}. 
\end{equation}
Then for ${\tilde{r}} = e^a r$, we have 
\[
	[u_\theta E({\tilde{r}}, u)]_q = 0. 
\]
\end{cor}
\begin{proof}
	We first need to show that \eqref{eq:a} defines a function satisfying $a(u^+) = a(u)$. By Lemma~\ref{lem:F-const} we have 
	\[
		a\circ u(\theta + \alpha) - a \circ u(\theta) = \int_{\theta}^{\theta + \alpha} \frac{d}{d \tau} \log [F(r, u)](\tau) d\tau = \log[F(r, u)]\bigr|_\theta^{\theta + \alpha} = 0. 
	\]
	Using Lemma~\ref{lem:prjection}, and noting that $a(u)$ and $\dot{a}(u) \cdot u_\theta$ are $\alpha$ periodic, we have 
	\[
		[u_\theta E({\tilde{r}}, u)]_q = e^{a(u)} \left( \dot{a}(u) u_\theta [F(r, u)]_q + [u_\theta E(r, u)]_q \right) = 0. 
	\]
\end{proof}

We now compute the linearized operator in $u$. Following \cite{LM2001}, for a function $g(\theta)$, define 
\[
	\nabla g(\theta) = g(\theta + \alpha) - g(\theta), \quad 
	\nabla^* g(\theta) = g(\theta) - g(\theta - \alpha), 
\]
and write 
\[
	L_{12}(r, u) = \partial^2_{ss'}L(r, u, u^+). 
\]

\begin{lem}\cite{LM2001}
We have 
\begin{equation}
	\label{eq:moser-levi}
	u_\theta \partial_u E(r, u)\cdot v + v \frac{d}{d\theta} E(r, u) = 
	\nabla^*\left( L_{12} u_\theta u_\theta^+ \nabla w \right),
\end{equation}
where $w = v/u_\theta$. 
\end{lem}

\section{Estimates of the analytic norm}
\label{sec:analytic-est}

In this section we introduce the function space and provide basic estimates in the analytic norm. 

For $\sigma >0$, we define ${\mathcal{A}}_\sigma$ to be the set of bounded complex analytic functions on  the set ${\mathbb{T}}_\sigma = \{|{\mathrm{Im}\,} \theta| < \sigma\} \in {\mathbb{C}}/{\mathbb{Z}}$ which takes real values for real $\theta$ (namely $\overline{f(\theta)} = f(\overline{\theta})$). It is a Banach space with the norm given by 
\[
	\|f\|_\sigma = \sup_{\theta \in {\mathbb{T}}_\sigma} |f(\theta)|.
\]
For $l \in {\mathbb{N}}$, define 
\[
	{\mathcal{A}}_{\sigma, l} = \left\{ f: {\mathbb{T}}_\sigma \to {\mathbb{C}}; \quad \overline{f(\theta)} = f(\overline{\theta}), \quad  \|f\|_{\sigma, l} := \sup_{0 \le k \le l} \|f^{(k)}(\theta)\|_\sigma < \infty \right\}. 
\]
We will use the same notations when $f$ is a vector valued function.

When $r$ is analytic, we will assume
there is $\sigma_0 >0$, such that
\begin{equation}
	\label{eq:std-assumption}
	\|r\|_{\sigma_0, 2} \le D. \tag{A2}
\end{equation}

We extend the notation $|\cdot|$ as norm of vectors in ${\mathbb{R}}^2$ into a function on ${\mathbb{C}}^2$, namely
\[
	|(z_1, z_2)| = \sqrt{z_1^2 + z_2^2}, 
\]
where we take the principle branch of the function $\sqrt{\cdot}$ in ${\mathbb{C}}$. Note that $|\cdot|$ can take a complex value and is no longer a norm in ${\mathbb{C}}^2$. We use $\|r\| = \|(z_1, z_2)\|$ to denote the standard norm in ${\mathbb{C}}^2$, and note that $\bigl\|\,  |r| \, \bigr\| \le \|r\|$. 

\begin{lem}\label{lem:func-def}
Let $r$ satisfy \eqref{eq:std-assumption}, and suppose 
\begin{equation}
	\label{eq:init-bound}
	\|u_\theta - 1\|_\sigma < \frac12, \quad
	\sigma < \min\left\{   \frac{\alpha}{4D^2}, \sigma_0/2  \right\}. 
\end{equation}
Then $E(r, u)$ (as a function of $\theta$) is analytic on ${\mathbb{T}}_\sigma$.
\end{lem}
\begin{proof}
	The assumption	$\|u_\theta - 1\|_\sigma < \frac12$ implies $u: {\mathbb{T}}_\sigma \to {\mathbb{T}}_\sigma$ is well defined on the space ${\mathbb{T}}_\sigma \to {\mathbb{T}}_{2\sigma}$. Moreover, we have $\frac12 \le \|u\|_\sigma \le 2$. 
	We have 
	\[
		\|r \circ u (\theta) - r \circ u ({\mathrm{Re}\,} \theta)\|_\sigma \le \|r'\|_{2\sigma} \|u_\theta\|_\sigma \|{\mathrm{Im}\,} \theta\| \le 2D \sigma, 
	\]
	and using \eqref{eq:real-assumption}, 
	\[
		|r\circ u ({\mathrm{Re}\,} \theta + \alpha) - r\circ u({\mathrm{Re}\,} \theta)| \ge D^{-1}\alpha. 
	\]
	Writing $\theta^+ = \theta + \alpha$,  \eqref{eq:init-bound} implies that for all $\sigma \in{\mathbb{T}}_\sigma$, 
	\[
		\begin{aligned}
		   & {\mathrm{Re}\,}\|r\circ u(\theta^+) - r\circ u(\theta)\|  \\ 
		   &\ge	 |r \circ u ({\mathrm{Re}\,} \theta^+) - r \circ u({\mathrm{Re}\,} \theta)| \\
		 & \quad\quad		   - \|r \circ u (\theta) - r \circ u ({\mathrm{Re}\,} \theta)\| - 
		 \|r \circ u (\theta^+) - r \circ u ({\mathrm{Re}\,} \theta^+)\| \\
		 & \ge D^{-1} \alpha - 4D \sigma \ge (2D)^{-1} \alpha. 
		\end{aligned}
	\]
	Observe that the function $|p_1 - p_2|$ as a function on ${\mathbb{C}}^2$ is analytic on the space ${\mathrm{Re}\,} |p_1 - p_2| > 0$, so $|\Delta r| = |r(u^+) - r(u)|$ is a well defined analytic function on ${\mathbb{T}}_\sigma$. Using \eqref{eq:E} we obtain the analyticity of $E(r, u)$. 

	Moreover, we get the estimate for all $\theta \in {\mathbb{T}}_\sigma$:
	\begin{equation}
	  \label{eq:delta-r}
	  (2D)^{-1} \alpha \le \| \Delta r\|   \le 2D \alpha. 
	\end{equation}
\end{proof}

We now discuss the inverse of the operators $\Delta$ and $\Delta^*$. 
\begin{lem}
	Given any function $g \in {\mathcal{A}}_\sigma$ satisfying $[g]_\sigma = 0$, there is a unique function $\phi$ with $[\phi]_q =0$,  such that 
	\[
		\nabla \phi = g, \quad \|\phi\|_\sigma \le q \|g\|_\sigma. 
	\]
	The same holds with $\nabla$ replaced with $\nabla^*$. 
\end{lem}
\begin{proof}
	It's easy to see that the function we seek is given by the Fourier series $\phi = \sum \phi_k e^{2\pi i k \theta}$, where
	\[
		\begin{cases}
			\phi_k = g_k/(e^{2 \pi i k/q} - 1), & k \notin q {\mathbb{Z}}, \\
			\phi_k = 0, & k \in q {\mathbb{Z}}. 
		\end{cases}
	\]

	To get the norm estimate, we compute  $\varphi$ in a different way. 
	Since $\nabla \phi = \phi(\theta + \alpha) - \phi(\theta) = g(\theta)$, we have, for $j = 1, \cdots, q$ , 
	\[
		\phi(\theta + j \alpha) - \phi(\theta) = j g(\theta + (j-1) \alpha). 
	\]
	Sum over all $j$, we get 
	\[
		\sum_{j = 1}^q \phi(\theta + j \alpha) - q \phi(\theta) = \sum_{j = 1}^q j g(\theta + (j-1)\alpha), 
	\]
	since $\sum_{j = 1}^q \phi(\theta + j \alpha) = q[\phi]_q = 0$, then 
	\[
		\phi(\theta) = - \frac{1}{q} \sum_{j = 1}^q j g(\theta + (j-1)\alpha). 
	\]
	The norm estimate follows easily. 
\end{proof}

Before solving the equation \eqref{eq:moser-levi}, we provide some estimates on $\partial_{12}L$. 
\begin{lem}\label{lem:est-analytic}
There is $C_1, D_1 > 1$ depending only on $D$, such that if $r, u$ satisfies the assumptions of  Lemma~\ref{lem:func-def}, and in addition
\[
	\sigma < C_1^{-1} \alpha, 
\]
then the following estimates hold for all $\theta \in {\mathbb{T}}_\sigma$: 
\begin{enumerate}
 \item $\|E(r, u)\| < D_1$;
 \item $D_1^{-1} \alpha^{-1} \le  \|\partial_{12}L(u, u^+)\| \le D_1 \alpha^{-1}$;
 \item $\left\| \frac{d}{d\theta} \partial_{12}L(u, u^+) \right\| \le D_1 \alpha^{-2}$. 
\end{enumerate}
\end{lem}
\begin{proof}
In this proof, let $C$ denote an unspecified constant depending only on $D$ 
\footnote{We will keep the same convention throughout the paper, but only within proofs. } .
Using \eqref{eq:E}, we have 
\[
	\|E(r, u)\|_\sigma \le 2 \frac{\max \|\Delta r\|}{\min \|\Delta r\|} \|\dot{r}\|_\sigma \le C  
\]
where $\max, \min$ are taking on ${\mathbb{T}}_\sigma$. 

For the upper bound in the  second item, since
\[
		\partial_{12}L(u, u^+) = - \frac{\langle \dot{r}(s), \dot{r}(s')\rangle}{|r(s')- r(s)|}  
		- \frac{\langle r(s') - r(s), \dot{r}(s)\rangle^2 }{|r(s') - r(s)|^3}, 
\]
we get 
\[
	\|\partial_{12}L(u, u^+)\|_\sigma \le C \frac{\|\dot{r}\|_\sigma^2}{ \min \|\Delta r\|} + C \frac{\|\Delta r\|_\sigma^2 \|\dot{r}\|_\sigma^2}{ \min \|\Delta r\|^3 } \le C \alpha^{-1}.  
\]
Furthermore, using similar computations, we have 
\[
	\left\|  \frac{d}{d\theta}  \partial_{12}L(u, u^+) \right\|_\sigma \le C \alpha^{-2}, 
\]
obtaining the third item. 

For the lower bound, we first consider real values, i.e. $\theta \in {\mathbb{T}}$, then 
\[
	\langle \dot{r}(u^+), \dot{r}(u)\rangle \ge |\dot{r}(u^+)|^2 - |\langle \dot{r}(u^+) - \dot{r}(u), \dot{r}(u)\rangle| 
	\ge \frac{1}{D^2} - 2\alpha D^2 \ge \frac{1}{2D^2\alpha}. 
\]
The for real values of $\theta$, 
\[
	- \partial_{12}L(u, u^+) \ge \frac{1}{2D^2\alpha}. 
\]
Using the bound on the derivative of $\partial_{12}L$, we get 
\[
	\|\partial_{12}L(u, u^+)\| \ge \frac{1}{2D^2\alpha} - \left\|  \frac{d}{d\theta}  \partial_{12}L(u, u^+) \right\|_\sigma \cdot \sigma \ge \frac{1}{2D^2 \alpha} - C \alpha^{-2} \cdot \sigma \ge \frac{1}{4D \alpha}
\]
if $\sigma < C^{-1} \alpha$ for sufficiently large $C$. 
\end{proof}

\begin{cor}\label{cor:inverse-bound}
Suppose $r, u$ satisfies the conditions of Lemma~\ref{lem:est-analytic}. Then for each $[g]_q = 0$, then there exists unique $[w]_q = 0$ such that 
\[
	\nabla^*\left( L_{12}(r, {\mathrm{id}}) u_\theta u_\theta^+ \nabla w \right) = g. 
\]
Moreover,  there is $C_2, D_2 >1 $ depending only on $D$ such that if $q^{-1} < C_2^{-1}$, we have 
\[
	\|w\|_\sigma \le D_2 q \|g\|_\sigma. 
\]
\end{cor}
\begin{proof}
	Let $p^{-1} = L_{12} u_\theta u_\theta^+$, we attempt to solve 
	\[
		\begin{cases}
			\nabla^* h = g \\
			p^{-1}\nabla w = h + h_1. 
		\end{cases}
	\]
	The first equation can be solved directly. For the second equation, we need to solve 
	\[
		\nabla w = ph + p h_1. 
	\]
	The equation has a unique solution if  $[ph]_q + [ph_1]_q = 0$. Suppose $h_1(\theta + \alpha) = h_1(\theta)$, then $[ph_1]_q = h_1 [p]_q$, 	therefore we can take
	\[
		h_1 = -  [ph]_q / [p]_q . 
	\]

	We now estimate the norm. 	Lemma~\ref{lem:est-analytic}, item (2) implies
	\[
		\|p\|_\sigma \le C \alpha^{-1}, \quad \|p\| \le C \alpha. 
	\]
	We now have the estimate: 
	\[
		\|h_1\|_{\sigma} \le \frac{\|p\|_{\sigma} \|h\|_{\sigma}}{ \min_{{\mathbb{T}}_{\sigma}} |p|} \le C \alpha \|p^{-1}\|_{\sigma}  \|h\|_{\sigma}  = C \alpha \alpha^{-1} \|h\|_\sigma = C \|h\|_\sigma .
	\]
	 As a result, 
	\[
		\|w\|_\sigma \le C \alpha^{-1} \|p\|_\sigma \|h\|_\sigma \le C \alpha \alpha^{-1} \|h\|_\sigma, \quad 
		\|h\|_\sigma \le \alpha^{-1} \|g\|_\sigma,
	\]
	since $\alpha = q^{-1}$, we have
	\[
		\|w\|_\sigma \le C q \|g\|_\sigma. 
	\]
\end{proof}

\section{Nekhoroshev-type iteration}
\label{sec:nek}

Let us fix some $r_0$ satisfying \eqref{eq:real-assumption} and \eqref{eq:std-assumption}. There is a constant $C_3> 1$ depending only on $D$, such that if 
\begin{equation}
  \label{eq:B-sigma}
  	r \in B_\sigma := \left\{  \|r - r_0\|_{\sigma, 2} < C_3^{-1}  \right\}, 
\end{equation}
$r$ satisfies the conditions \eqref{eq:real-assumption} and \eqref{eq:std-assumption} with $D$ replaced by $2D$. 

The goal of this section is to prove the following Nekhoroshev-type result, which allows us to reduce the non-resonant part of the function $E(r, {\mathrm{id}})$ to exponentially small. 
\begin{prop}\label{prop:nek}
Suppose $\|r - r_0\|_{\sigma_0, 2} < (2C_3)^{-1}$, then there exists $C_4, D_4 > 1$ depending on $D$, such that if 
\[
	q^{-1} < C_4^{-1} \sigma_0, \quad E(r, {\mathrm{id}}) =: \epsilon  < C_4^{-1} q^{-4}, 
\]
there exists analytic diffeomorphism  $u_{{\mathrm{nek}}}: {\mathbb{T}} \to {\mathbb{T}}$ satisfying $\|u_{{\mathrm{nek}}} - {\mathrm{id}}\|_{\sigma_0/2}< D_4 q\epsilon$, and for  $r_{{\mathrm{nek}}} = r \circ u_{{\mathrm{nek}}}$, we have  $\|r_{{\mathrm{nek}}} - r\|_{\sigma_0/2} < 2 \epsilon$ and 
\[
	\left\| E(r_{{\mathrm{nek}}}, {\mathrm{id}})\right\|_{\sigma_0/4} < D_4  \sigma_0^{-1}  e^{- \sigma_0 q/4}. 
\]
\end{prop}

We now describe the iteration process. 
\begin{lem}
	[Iteration lemma for Moser-Levi procedure] \label{lem:nek-iter} Suppose $\|r - r_0\|_{\sigma, 2} < (2C_3)^{-1}$, 
	and let $v$ solve
	\begin{equation}
		\label{eq:v1}
		\nabla^*(L_{12}(r, {\mathrm{id}}) \nabla v) = - \{E(r, {\mathrm{id}})\}_q, 
	\end{equation}
	then there exists constants $C_4, D_4 > 1$ depending only on $D$ such that: if for $\eta, \epsilon > 0$ and $0 < \sigma' < \sigma$,  
	\[
	  \|E(r, {\mathrm{id}})\|_\sigma < \eta, \quad  \|\{E(r, {\mathrm{id}})\}_q\|_\sigma < \epsilon, \quad C_4 q \epsilon < \sigma - \sigma',
	\]
	we have $r_+ = r \circ ({\mathrm{id}} + v) \in {\mathcal{A}}_{\sigma', 2}$, and 
	\[
		\|r_+ - r\|_{\sigma', 2} < \frac{D_4 q\epsilon}{(\sigma - \sigma')^2} , \quad
		\|v\|_{\sigma'} < D_4  q \epsilon, 
	\]
	and 
	\[
		\left\| E(r_+, {\mathrm{id}}) - [E(r, {\mathrm{id}})]_q  \right\|_{\sigma'}, \, \,  \left\| \{E(r_+, {\mathrm{id}})\}_q  \right\|_{\sigma'} \le  D_4 \left( \frac{  q \eta }{(\sigma - \sigma')} +  q^4 \epsilon \right) \epsilon. 
	\]
\end{lem}
\begin{proof}
By Corollary~\ref{cor:inverse-bound}, we have $\|v\|_\sigma \le C q \epsilon$. 
	Let $\delta = (\sigma - \sigma')/2$, our assumption ensures $C q \epsilon < \delta$. Then
	\[
		\|r_+ - r\|_{\sigma - \delta} = \|r\circ ({\mathrm{id}} + v) - r\|_{\sigma - \delta} \le \|r\|_{\sigma} \|v\|_{\sigma - \delta} \le C q \epsilon.
	\]
	Then 
	\[
		\|r_+ - r\|_{\sigma- 2\delta, 2} = \delta^{-2} \|r_+ - r\|_{\sigma - \delta} \le \frac{Cq\epsilon}{(\sigma - \sigma')^2} . 
	\]

	Using an estimate similar to Lemma~\ref{lem:est-analytic} we have 
	\[
		\left\| \frac{d^2}{dt^2} E(r, {\mathrm{id}} + tv) \right\|_{\sigma} \le C \alpha^{-2} \|v\|_\sigma^2 = C q^2 \|v\|_\sigma^2, 
	\]
	and 
	\[
		\begin{aligned}
			& \left\| E(r_+, {\mathrm{id}}) - [E(r, {\mathrm{id}})]_q  \right\|_{\sigma'} \le  \|E(r, {\mathrm{id}}) - [E(r, {\mathrm{id}})]_q + \partial_u E(r, {\mathrm{id}}) \cdot v\|_{\sigma'} + C q \|v\|_\sigma^2 \\
			& \le \| v \frac{d}{d\theta} E(r, {\mathrm{id}})\|_{\sigma'} + C q^2 \|v\|_\sigma^2
			\le C \frac{q  \eta \epsilon}{\sigma - \sigma'} + C q^4 \epsilon^2. 
		\end{aligned}
	\]
	We get our final estimate by applying the operator $\{\cdot\}_q$ to the above estimate. 
\end{proof}

\begin{proof}[Proof of Proposition~\ref{prop:nek}]
We now perform the Nekhoroshev iteration. As before $C>1$ denote a generic constant, but in this proof $C$ may depend on both $D$ and $\sigma_0$. 

\emph{Step 1}. We take initially $\delta = \sigma_0/4$ and write $\sigma_1 = \sigma_0 - \delta$. This trick of using a large initial step to improve estimate is due to Neishtadt (see for example \cite{LN1992}). Since $Cq \epsilon < \sigma_0$, we apply Lemma~\ref{lem:nek-iter} and $r_1 = r \circ ({\mathrm{id}} + v_1)$, , with the estimates
\[
	\|r_1 - r\|_{\sigma_1, 2} < C \sigma_0^{-1} q \epsilon, \quad\|u_0 - {\mathrm{id}}\|_{\sigma_1} < C q \epsilon, 
\]
\[
\left\| E(r_1, {\mathrm{id}}) - [E(r, {\mathrm{id}})]_q \right\|_{\sigma_1} , \, \,   \|\{E(r_1, {\mathrm{id}})\}_q\|_{\sigma_1, 2}  <   \epsilon_1 := D_4 \left( q^2 /\sigma_0 + q^4 \right) \epsilon^2 < C q^4 \epsilon^2,
\]
since $\sigma_0 > q^{-1}$. In particular, we obtain $\eta_1 := \|E(r_1, {\mathrm{id}})\|_{\sigma_1} < \epsilon + C q^4 \epsilon^2 < 2\epsilon$ for $\epsilon < C q^{-4}$. 

\emph{Step 2}.
We now take $\delta_* = C' q^{\frac52}\epsilon$ for a sufficiently large $C'> 1$. Define $\sigma_n = \sigma_1 - (n-1)\delta_*$, let 
$v_n$ solve \eqref{eq:v1} for $r = r_{n-1}$,  define  $r_n = r_{n-1} \circ ({\mathrm{id}} + v_n)$. Then we have 
\[
\begin{aligned}
 	\epsilon_2 & = \left\| E(r_2, {\mathrm{id}}) - [E(r_1, {\mathrm{id}})]_q \right\|_{\sigma_2} < D_4 \left( \frac{q \eta_1}{\delta_*}+ q^4 \epsilon_1 \right) \epsilon_1  \\
  &= D_4 \left( q^{-\frac32} (C')^{-1} + C q^8 \epsilon^2  \right) \epsilon_1 < \frac{\epsilon_1}2  
\end{aligned}
\]
as long as $\epsilon < C^{-1} q^{-4}$ for large enough $C$. Furthermore,
\[
	\|r_2 - r_1\|_{\sigma_2, 2} < \frac{D_4 q \epsilon_1}{\delta_*^2} = \frac{D_4 q^5 \epsilon^2}{(C')^2 q^5 \epsilon^2} < (4C_3)^{-1}
\]
if $C'$ is large enough. This ensures $r_2 \in B_{\sigma_2}$. 

 We now check that as long as $\sigma_n = \sigma_1 - n \delta_* > 0$, the following holds inductively:
\begin{enumerate}
\item $\epsilon_n = \left\| E(r_n, {\mathrm{id}}) - [E(r_{n-1}, {\mathrm{id}})]_q \right\|_{\sigma_n} < 2^{n-1}\epsilon_1$. 
\item For $n \ge 2$, $\|r_n - r_{n-1}\|_{\sigma_n} < 2^{-(n-1)} (4C_3)^{-1}$, and $\|r_n - r_1\|_{\sigma_n} < (2C_3)^{-1}$, in particular, $r_n \in B_{\sigma_n}$.  
\item $\eta_n = \|E(r_n, {\mathrm{id}})\|_{\sigma_n} \le \epsilon_n + \eta_{n-1} < 2 \epsilon$. 
\item $u_n = ({\mathrm{id}} + v_n) \circ \cdots \circ ({\mathrm{id}} + v_1)$ satisfies $\|u_n - {\mathrm{id}}\|_{\sigma_n} < C q\epsilon$. 
\end{enumerate}
Pick $N = \sigma_0/(4\delta_*) = (4C')^{-1} \sigma_0 q^{-\frac52}\epsilon^{-1}$, we get $\sigma_N = \sigma_1 - \sigma_0/4 =\sigma_0/2$,  the above estimates implies $\|E(r_N, {\mathrm{id}})\|_{\sigma_0/2} < 2\epsilon$, and
\[
	  \|\{E(r_N, {\mathrm{id}})\}_q\|_{\sigma_0/2} = \epsilon_N < C  q^4 \epsilon^2 \cdot 2^{- C^{-1} q^{-\frac52} \epsilon^{-1}} < C \epsilon \exp\left\{ - C^{-1} \sigma_0 q^{-\frac52} \epsilon^{-1}  \right\} . 
\]
Keep in mind $\epsilon^{-1} > q^4$, the upper bound is then estimated by $C \epsilon e^{- C^{-1} \sigma_0 q^{4 - 5/2}} = C \epsilon e^{- C^{-1} \sigma_0 q^{-\frac32}}$. Moreover, by applying a standard estimate (Lemma~\ref{lem:res-exp}), we have 
\[
	\left\| [E(r_N, {\mathrm{id}})]_q \right\|_{\sigma_0/2} \le C \sigma_0^{-1} e^{-q\sigma_0/4} \epsilon.  
\]
The proposition follows by taking $u_{{\mathrm{nek}}} = u_N$, $r_{{\mathrm{nek}}} = r_N$ and using the larger of the two upper bounds. 
\end{proof}

\begin{lem}\label{lem:res-exp}
Let $f\in {\mathcal{A}}_{\sigma}$ satisfy  $[f]_q = f$ and $\int_0^1 f(\theta) d\theta = 0$ , then there is an explicit $C' > 0$ such that 
\[
	\|f\|_{\sigma/2} < C' \sigma^{-1} e^{-q\sigma/4} \|f\|_\sigma. 
\]
\end{lem}
\begin{proof}
Our conditions implies $f(\theta) = \sum_{k \in {\mathbb{Z}} \setminus\{0\}} f_{kq} e^{2\pi i k q \theta}$. Denote $\epsilon = \|f\|_\sigma$, standard estimates of the analytic function implies $|f_k| \le \epsilon e^{- k q \sigma}$, then 
\[
	\|f\|_{\sigma/2} = \sup_{|{\mathrm{Im}\,} \theta| < \sigma/2} \epsilon \left|  \sum_{k \in {\mathbb{Z}} \setminus \{0\}} e^{- k q \sigma} \right| \le \epsilon e^{-q\sigma/4} \sum_{k \in {\mathbb{Z}} \setminus \{0\}}  e^{-kq\sigma/4} \le C \frac{\epsilon}{\sigma} e^{-q\sigma/4}. 
\]
 \end{proof}

\section{Lazutkin-type normal form and smooth approximation}
\label{sec:laz}

In this section we perform an initial step of normal form due to Lazutkin (\cite{Laz1973}). 
Let $s$ be the arc-length parameter on $\Omega$,  $\varphi$ be the angle of reflection, and $\theta = \frac{\pi}2 - \varphi$. Near the boundary (see for example \cite{Laz1973}), the billiard map is can be written approximately as:
\[
	T(s, \theta) = {\begin{bmatrix} { s + 2 \rho(s) \theta + O(\theta^2) \\ \theta - 2 \rho'(s) \theta^2/3 + O(\theta^3)} \end{bmatrix}},
\]
where $\rho(s)$ is the radius of curvature at $r(s)$. 

Let $B_\sigma \subset {\mathbb{C}}$ denote the ball of radius $\delta$, let us consider the space ${\mathcal{D}}_{a, b}$ of real analytic functions that can be extended to the complex neighborhood ${\mathbb{T}}_a \times B_b$. 

\begin{lem}\label{lem:billiard-analytic}
Given $D>0$, there exists a constant $C_4>1$ depending only on $D$ such that: if $r$ satisfies conditions \eqref{eq:real-assumption} and \eqref{eq:std-assumption} with parameters $\sigma_0 < C^{-1}$ and $D$, the map $T$ is real analytic, and can be extended to a complex analytic map on  ${\mathbb{T}}_{\sigma_0} \times B_{\sigma_0}$. Moreover, for each $\sigma < \sigma_0$
\[
	\|T\|_{\sigma} \le C \|r\|_{\sigma, 1} . 
\]
\end{lem}
\begin{proof}
Suppose $T(s, \varphi) = (s^+, \varphi^+)$, then
\[
	\cos \theta = \partial_1 L(s, s^+), \quad  - \cos \theta^+ = \partial_2 L(s, s^+). 
\]
As a result, $(s^+, \varphi^+)$ is defined implicitly by an equation which involves the first derivative of $r$. The implicit function applies due to the fact that $\partial_{12}L$ is uniformly bounded away from $0$, depending only on $D$.  As a result, the $C^r$ norm of $T$ is bounded by  the $C^{r+1}$ norm of $r$, up to a constant depending on $D$. The map can be extended to a complex domain of width depending only on $D$, on which the implicit function theorem still applies. The norm estimate is straight forward, from implicit function theorem. 
\end{proof}

\begin{prop}\label{prop:gen-laz}
Under the same assumption as Lemma~\ref{lem:billiard-analytic}, there exists a real analytic coordinate change 
\[
	(x, y) = \Phi(s, \theta) , 
\]
such that $T_{\mathrm{laz}} = \Phi \circ T \circ \Phi^{-1}$ takes the form:
\[
	T_{\mathrm{laz}}(x, y) = (x + y + y^6 f(x, y), y + y^6 g(x, y)). 
\]
Moreover, there is a constant $C_6>1$ depending only on $D$ such that, for every $\sigma < \sigma_0/2$, we have 
\[
	\|f\|_{\sigma}, \|g\|_{\sigma} \le C_6 \sigma_0^{-6} . 
\]
\end{prop}
The proof is presented in the Appendix. 

\begin{cor}\label{cor:appox-general}
Under the same assumptions as Proposition~\ref{prop:gen-laz}, there is $C_7 > 1$ such that, for $\sigma_1 \le  \sigma_0/2$,  there is a real analytic diffeomorphism  $u_q: {\mathbb{T}} \to {\mathbb{T}}$ extensible to ${\mathbb{T}}_{\sigma_1}$, such that 
\[
	\|E(r, u_q)\|_{\sigma_1} < C_7 \sigma_0^{-6} q^{-6}. 
\]
\end{cor}
\begin{proof}
Denote $\alpha = 1/q$ as before, 
Let $\gamma(x) = (x, \frac{1}{q}) \in {\mathbb{T}} \times (-b_*', b_*')$, then $\|T_{\mathrm{laz}}(\gamma(x)) - \gamma(x + \alpha)\|_{\sigma_1} \le C \sigma_0^{-6} q^{-6}$. Let 
\[
	\eta(x) = (u(x), w(x)) = \Phi(\gamma(x)),  
\]
then $\|T(\eta(x)) - \eta(x+\alpha)\|_{\sigma_1} \le C \sigma_0^{-6} q^{-6}$. 

Let ${\overline{u}}(x) = \pi_1 T(u(x), w(x))$, and ${\underline{u}}(x) = \pi_1 T^{-1}(u(x), w(x))$, then we have 
\[
	\partial_s' L(r; {\underline{u}}(x), u(x)) + \partial_s L(r, u(x), {\underline{u}}(x)) = \cos w(x) - \cos w(x) = 0 , 
\]
and 
\[
	\|{\overline{u}}(x) - u(x + \alpha)\|_{\sigma_1}, \|{\underline{u}}(x) - u(x -\alpha)\|_{\sigma_1} \le C \sigma_0^{-6}q^{-6}, 
\]
we conclude that  
\[
	\|E(r, u)\|_{\sigma_1} \le C \sigma_0^{-6} q^{-6}. 
\]
\end{proof}

Apply Corollary~\ref{cor:appox-general} and then Proposition~\ref{prop:nek}, we obtain:
\begin{cor}
[Analytic case] \label{cor:app-analytic}
Suppose $r$ satisfies the conditions \eqref{eq:real-assumption} and \eqref{eq:std-assumption}, then there is $C_8 > 1$ depending only on $\sigma_0$ and $D$ such that for $q > C$, there exists $u_{\mathrm{app}}: {\mathbb{T}} \to {\mathbb{T}}$ extensible to ${\mathbb{T}}_{\sigma/4}$ such that
\[
	\|E(r \circ u_{\mathrm{app}}, {\mathrm{id}})\|_{\sigma_0/4} <  C_8 q^{-6} e^{- \sigma_0 q/ 4}. 
\] 
\end{cor}

In the case $r$ is smooth, we use a standard analytic approximation. 
\begin{prop}
[See \cite{Zeh1975}, Lemma 2.1, 2.3]
Suppose $f: {\mathbb{R}}^n \to {\mathbb{R}}$ is $C^m$, and let $l < m$. Then for each $t> 0$ there exists an analytic function $S_t f$ satisfying the following estimates:
\[
	\|S_t f\|_{t^{-1}} \le C_{n, m} \|f\|_{C^m}, \quad
	\|S_t f - f\|_{C^l} \le C_{n, m, l} t^{-(m - l)} \|f\|_{C^m}. 
\]
\end{prop}

For each $r \in C^m$, we consider the analytic approximation $S_t \rho$ of the function $\rho = |\ddot{r}|$, such that 
\[
	\|S_t \rho\|_{t^{-1}} \le C_m \|r\|_{C^m}, \quad \|S_t \rho - \rho\|_{C^{l-2}} \le C_{m, l} t^{-(m - l)} \|r\|_{C^m}. 
\]
We will pick $t = q^{\frac14}$, $\sigma_q = q^{-\frac14}$, and denote $r_\omega$ the associated boundary with $\ddot{r}_\omega = S_t \rho$. We get 
\[
	\|r_\omega\|_{\sigma_q, 2} \le C_m \|r\|_{C^m}, \quad \|r_\omega - r\|_{C^{l}} \le C_{m, l} q^{-\frac{m-l}{4}} \|r\|_{C^m}. 
\]
Apply Corollary~\ref{cor:appox-general} to $r_\omega$, we get an approximate solution $u_q$ such that $\|E(r_\omega, u_q)\|_{\sigma_q/2} < C \sigma_q^{-6} q^{-6} < C q^{-\frac{9}2}$ which is smaller than $q^{-4}$ as required by Proposition~\ref{prop:nek}. Therefore, we obtain:
\begin{cor} 
[Smooth case] \label{cor:app-smooth}
Suppose $r \in C^m$ satisfies \eqref{eq:real-assumption} and $\|r\|_{C^m} < D$. Then there is $C_9> 1$ depending only on $D$ and $m, l$ such that, for 
\[
	q^{-1} < C^{-1}, \quad \sigma_q = q^{-\frac14}, 
\]
there is $r_\omega \in {\mathcal{A}}_{\sigma_q, 2}$ and $u_{\mathrm{app}} \in {\mathcal{A}}_q$ satisfying $\|r_\omega\|_{\sigma_q, 2} \le CD$, and 
\[
	\|r_\omega - r\|_{C^l} \le C_9 q^{-\frac{m-l}{4}}, \quad
	\|E(r_\omega \circ u_{\mathrm{app}}, {\mathrm{id}})\|_{\sigma_q/4} \le C_9 q^{\frac14} e^{-  C^{-1} q^{\frac34}}. 
\]
\end{cor}

\section{KAM algorithm and proof of the main theorem}
\label{sec:kam}

We prove the following KAM-type theorem. 
\begin{thm}\label{thm:analytic-KAM}
Suppose $0 < \sigma < \sigma_0/2$, and $r_0$ satisfies our standing assumptions \eqref{eq:real-assumption} and  \eqref{eq:std-assumption}. There exists constants $C_{10}, D_{10} > 1$ depending only on $D$ such that if 
\[
	q^{-1} < C_{10}^{-1}, \quad \sigma < C_{10}^{-1} q^{-1}, \quad \|E(r, {\mathrm{id}})\|_\sigma :=  \epsilon < C_{10}^{-1} q^{-6}, \quad  \epsilon <  C_{10}^{-1} q^{-3} \sigma^3, 
\] 
 there is $r_\infty \in {\mathcal{A}}_{\sigma/2}$ such that 
\[
	E(r_\infty, {\mathrm{id}}) = 0, \quad \|r_\infty - r\|_{\sigma/2, 2} < D_{10} q\epsilon/\sigma^2. 
\]
\end{thm}

We now describe the iteration process. 
\begin{prop}
	[Iteration lemma] \label{prop:iter} Suppose $\|r - r_0\|_{\sigma, 2} < (2C_3)^{-1}$ ($C_3$ is from \eqref{eq:B-sigma}), define 
	\begin{equation}
		\label{eq:a-iter} 
		a(\theta) = - \int_0^\theta \frac{[E(r, {\mathrm{id}})]_q}{[F(r, {\mathrm{id}})]_q} (\tau) d\tau, \quad r_* = e^a r,
	\end{equation}
	and let $v$ solve
	\begin{equation}
		\label{eq:v}
		\nabla^*(L_{12}(r_*, {\mathrm{id}}) \nabla v) = - E(r_*, {\mathrm{id}}), 
	\end{equation}
	then there exists constants $C_4, D_4 > 1$ depending only on $D$ such that: if for $\epsilon > 0$ and $0 < \sigma' < \sigma$,  
	\[
	 \|E(r, {\mathrm{id}})\|_\sigma < \epsilon, \quad C_4 q \epsilon < \sigma - \sigma',
	\]
	we have $r_+ = r_* \circ ({\mathrm{id}} + v) \in {\mathcal{A}}_{\sigma', 2}$, and 
	\[
		\|r_+ - r\|_{\sigma', 2} < \frac{D_4 q\epsilon}{(\sigma - \sigma')^2} , \quad
		\|v\|_{\sigma'} < D_4  q \epsilon, \quad
		\|E(r_+, {\mathrm{id}})\|_{\sigma'} \le  \frac{D_4  q^2\epsilon^2}{(\sigma - \sigma')} + D_4 q^4 \epsilon^2. 
	\]
\end{prop}

\begin{proof}
	Since $\|E(r, {\mathrm{id}})\|_\sigma < \epsilon$, and by \eqref{eq:delta-r}, 
	\[
		C^{-1} \le \|[F(r, {\mathrm{id}})]_q\|_\sigma = \|[|\Delta r|]_q\|_\sigma \le C, 
	\]
	noting $a$ is $\alpha$-periodic, we get 
	\[
		\|\dot{a}\|_{\sigma} \le \frac{\min \|F\|}{\|E\|_\sigma} \le C q \epsilon, \quad
		\|a\|_\sigma \le  \max_{0 \le t \le \alpha, \,  \theta \in {\mathbb{T}}_\sigma} \left|\int_0^t \dot{a}(\theta + \tau) d\tau \right| \le C \alpha q \epsilon = C \epsilon. 
	\]
	As a result
	\[
		\|r_* - r\|_{\sigma', 2} \le \frac{C}{(\sigma - \sigma')} \|a\|_{\sigma,2} \le \frac{Cq\epsilon}{(\sigma - \sigma')}. 
	\]
	Moreover, 
	\[
		\|E(r_*, {\mathrm{id}})\|_{\sigma} \le \|e^a\|_\sigma (\|\dot{a}\|_\sigma \|F(r, u)\|_\sigma + \|E(r, u)\|_\sigma) \le  C q\epsilon \cdot  \alpha + C\epsilon \le C \epsilon. 
	\]
	We now apply Lemma~\ref{lem:nek-iter} to $r_*$, to obtain all the estimates for $r_+$. 
\end{proof}

To apply KAM we have the following standard induction lemma: 
\begin{prop}
Suppose $0 < \sigma < \sigma_0/2$, and $\|r - r_0\| < (4D)^{-1}$. Denote, 
\[
	\sigma(1) = \sigma, \quad \sigma(n+1) = \sigma(n) - \delta(n), \quad \delta(n) = 2^{-n-1}\sigma, 
\]
\[
	 \epsilon(1) = \epsilon, \quad \epsilon(n+1) = (\epsilon(n))^{\frac43}, \quad n \ge 1. 
\]
There exist constants $C_{10}, D_{10} > 1$ depending only on $D$, such that if:
\[
 C_{10} \epsilon < q^{-6}, \quad C_{10} \epsilon < q^{-3} \sigma^3, 
\]
the following hold for all $n \ge 1$:
\begin{enumerate}
 \item $C_{10} \epsilon(n) < q^{-6}$, $C_{10} \epsilon(n) < q^{-3} (\delta(n))^3$. 
 \item For $r_1 = r$ and $r_{n+1} = (r_n)_+$ using Proposition~\ref{prop:iter}, we have 
 	\[
		\|r_n- r_{n-1}\|_{\sigma(n), 2} < D_{10}\frac{q \epsilon(n)}{\delta(n)^2}, \quad \|E(r_n, {\mathrm{id}})\|_{\sigma(n)} < \epsilon(n),
	\] 
	\[
		C_3 q \epsilon(n) < \delta(n).
	\]
\item $r_n \to r_\infty$ in ${\mathcal{A}}_{\sigma/2}$, with
	\[
		E(r_\infty, {\mathrm{id}}) = 0, \quad \|r_\infty - r_0\|_{\sigma/2, 2} < D_{10} q \epsilon /\sigma^2. 
	\]  
\end{enumerate}
\end{prop}
\begin{proof}
Since the sequence $\epsilon(n)/(\delta(n))^3$ is decreasing if $\epsilon$ is small enough, item (1) is obvious. The only non-trivial estimate in (2) is the estimate of $\|E(r_n, {\mathrm{id}})\|_{\sigma(n)}$. Suppose item (2) hold up to index $n-1$.  We use Proposition~\ref{prop:iter} to get 
\[
	\|E(r_n, {\mathrm{id}})\|_{\sigma(n)} \le ( \epsilon(n-1))^{\frac43} \cdot C_4 (\epsilon(n-1))^{\frac23} \left( 2q^2 (\delta(n-1))^{-2} + q^4  \right) 
\]
where the second group in the product is smaller than $2 C_5^{-1}$ by item (1). Choosing $C_5$ large enough yields the desired estimate. Item (3) follows from item (2) by observing that $\epsilon(n)/\sigma(n)^2$  is summable, and $\sigma(n) \to \sigma/2$.  
\end{proof}

Theorem~\ref{thm:analytic-KAM} follows directly from what we just proved. 

\begin{proof}[Proof of Theorem~\ref{thm:main}]
We now prove our main theorem.

\emph{Case 1, the analytic case}: Suppose $r$ satisfies \eqref{eq:real-assumption} and \eqref{eq:std-assumption}. For $q$ sufficiently large depending only on $D$ and $\sigma_0$, Corollary~\ref{cor:app-analytic} applies, and 
\[
	\|E(r \circ u_{\mathrm{app}}, {\mathrm{id}})\|_{\sigma_0/4} < \epsilon: = C q^{-6} e^{-\sigma_0 q/4}. 
\]
Clearly the assumptions of Theorem~\ref{thm:analytic-KAM} applies, we obtain $r_\infty \in {\mathcal{A}}_{\sigma_0/8, 2}$ such that $E(r_\infty, {\mathrm{id}}) = 0$, and $\|r_\infty - r \circ u_{\mathrm{app}}\|_{{\sigma_0}/8} < C q^{-5} e^{-\sigma_0 q/4}$. Then the boundary $r_\infty \circ (u_{\mathrm{app}})^{-1}$ satisfies our conclusion. 

\emph{Case 2, the smooth case}. Assume $\|r\|_{C^m} \le C$.  Apply Corollary~\ref{cor:app-smooth}, we obtain $\|r_\omega - r\|_{C^l} \le C q^{-\frac{m-l}4}$, such that $\|E(r_\omega \circ u_{\mathrm{app}}, {\mathrm{id}})\|_{\sigma_q/4} \le C q^{\frac14} e^{-C^{-1} q^{\frac34}}$. For large $q$ depending only on $D$, Theorem~\ref{thm:analytic-KAM} applies, and we obtain a perturbed boundary for which $E(\rho_\omega, {\mathrm{id}}) = 0$. Note that for $q$ large enough, $\|r_\infty \circ (u_{\mathrm{app}})^{-1} - r_\omega\|$ is bounded by $q^{-\frac{m-l}4}$, since the former is exponentially small. Then $r_\infty \circ (u_{\mathrm{app}})^{-1}$ is the boundary we seek. 
\end{proof}

\appendix

\section{Higher order Lazutkin normal form}

Lazutkin (\cite{Laz1973})  showed existence of smooth coordinate changes taking the map to higher order normal forms
\[
	(x, y) \mapsto (x + y + O(y^m), \quad y + O(y^{m+1})). 
\] 
\cite{MRT2016} contains an analytic version of this normal form and \cite. We need a version with explicit estimates on the width of analyticity, but only to the order $O(y^6)$. 
These calculations are done by J. De Simoi and A. Sorrentino (unpublished notes \cite{DeSimoi}, \cite{Sorrentino}), which we reproduce here. 

Let $\vartheta = \frac{\pi}2 - \varphi$, and write $T(s, \vartheta) = (s^+, \vartheta^+)$  and $T^{-1}(s, \vartheta) =(s^-, \vartheta^-)$. Denote 
\[
	\begin{aligned}
	& s^\pm  = s  \pm b_1 \vartheta + b_2  \vartheta^2 \pm b_3 \vartheta^3 + b_4 \vartheta^4 + O(\vartheta^5)\\
	& \vartheta^\pm = \vartheta \pm d_2 \vartheta^2 + d_3 \vartheta^3 \pm d_4 \vartheta^4 + O(\vartheta^5) 
	\end{aligned}
\]
Here $b_i = b_i(s)$ and $d_i = d_i(s)$ are functions of $s$. The relation between derivatives of $s^\pm$ and $\vartheta^\pm$ is due to the time reversal symmetry of the map: if we denote $I(s, \vartheta) = (s, -\vartheta)$, then $I \circ T \circ I = T$. 
Let's call the following expression a differential monomial in $\rho$:
\[
	P(\rho)(s)  = \prod_{k = 0}^m \left( \rho^{(k)}(s) \right)^{a_k}, \quad a_k \in {\mathbb{N}}, \, k \ge 1,  
\]
where $\rho = \rho(s)$ is a function, and $\rho^{(k)}$ denote it's $k$th derivatives. The sum $\sum_{k =1}^m k a_k$ is called the degree of $P$. A direct computation shows that $b_i$, $d_i$ are homogeneous differential polynomials of degree $i-1$, in the function $\rho = |\ddot{r}|^{-1}$, which is the radius of curvature.
\footnote{In all likelihoods, this holds for all $i \in {\mathbb{N}}$, however we do not prove this. We only check that it holds up to $i = 4$ from explicit computations. }
For example, 
\[
	b_1 = 2\rho, \quad b_2 = \frac43 \rho \dot{\rho}, \quad b_3 = \frac49 \rho \dot{\rho}^2 + \frac23 \rho^2 \ddot{\rho}, \quad d_2 = -\frac23 \dot{\rho}, \quad
	\text{etc.}
\]
To obtain a normal form, we consider a change of coordinate 
\begin{equation}
  \label{eq:6th-lautkin}
  	x = X(s, \vartheta) = F(s) + G(s) \vartheta^2, \quad y = Y(s, \vartheta) =  X(s, \vartheta) - X(s^-(s, \vartheta), \vartheta^-(s, \vartheta)). 
\end{equation}
Let us note that 
\begin{equation}
  \label{eq:Phi-y}
 	y = F(s) - F(s^-) + \left( G(s) \vartheta^2 - G(s^-) (\vartheta^-)^2 \right) = F'(s) \vartheta + O(\vartheta^2). 
\end{equation} 

We attempt to  solve the equation $y^+ - y = O(\vartheta^6)$, which is 
\[
	X(s^+, \vartheta^+) - 2X(s, \vartheta) + X(s^-, \vartheta^-) = O(\vartheta^6). 
\]
Observe that $F(s^+) - 2 F(s) + F(s^-)$ must be an even series of $\vartheta$, an explicit computation yields
\[
\begin{aligned}
 	F(s^+) - 2F(s) + F(s^-) 
& = F'(s) (s^+ + s^- - 2s) + \frac12 F''(s) \left( (s^+-s)^2 + (s^- - s)^2 \right) \\
& \quad  + \frac16 F'''(s)\left( (s^+ - s)^3  + (s^- - s)^3\right) + O(\vartheta^6) \\
& = F'(s) \left( 2 b_2 \vartheta^2 + {\textcolor{blue}}{2}b_4 \vartheta^4 \right)  + 
\frac12 F''(s) \left(  2 b_1^2 \vartheta^2 + (2b_1 b_3 + b_2^2) \vartheta^4 \right)\\
& \quad + 
\frac16 F'''(s) (2b_1^2 b_2) \vartheta^4 + O(\vartheta^6). 
\end{aligned}
\]
Attempting to eliminate the $\vartheta^2$ term leads to the equation 
\[
	2b_2F'(s) + b_1^2 F''(s) = 0, \quad \frac43\left(  2 \rho \dot{\rho} F' + 3\rho^2 F'' \right) =0, 
\]
whose solution is
\begin{equation}
  \label{eq:F-laz}
  F(s) = C \int_0^s \rho^{-\frac23}(z)dz, 
\end{equation}
which is identical to Lazutkin's choice. Moreover, by plugging in this choice of $F(s)$, we have 
\[
	F(s^+) - 2F(s) + F(s^-) = P_4(s) \vartheta^4 + O(\vartheta^6), 
\]
where $P_4(s)$ is a differential polynomial of degree $4$ (The power of $\rho$ will be fractional, however, they do not count in degrees). A similar computation yields
\[
	 	G(s^+) - 2 G(s) + G(s^-) =  2b_2  G'(s) \vartheta^2  + b_1^2 G''(s) \vartheta^2    + O(\vartheta^4), 
\]
leading to the equation 
\[
	2 \rho \dot{\rho} G' + 3\rho^2 G'' = - \frac34 P_4(s).  
\]
By using the substitution $G' = \rho^{{\textcolor{blue}}{- \frac{2}{3}}} g$, we obtain the equation 
\begin{equation}
  \label{eq:g-explicit}
  	{\textcolor{blue}}{3}\rho^{{\textcolor{blue}}{\frac{4}{3}}} g' = - \frac34 P_4, \quad g(s) = - \int_0^s \frac14 \rho^{{\textcolor{blue}}{-\frac43}} P_4(z) dz + C_1, 
\end{equation}
note that $C_1$ is uniquely defined by the condition $\int_0^1 \rho^{\frac23}g(z) dz = 0$. Then 
\begin{equation}
  \label{eq:G-laz}
  	G(s) = \int_0^s  C_1 \rho^{-\frac23}(z) dz + \int_0^s \rho^{-\frac23} g(z) dz. 
\end{equation}

\begin{proof}
[Proof of Proposition~\ref{prop:gen-laz}]
Define the coordinate $(x, y) = \Phi(s, \vartheta)$ change using \eqref{eq:6th-lautkin}, \eqref{eq:F-laz} and \eqref{eq:G-laz}.
Denote $A(x, y) = (x + y, y)$, then 
\[
\begin{aligned}
  & 	(\Phi \circ T - A \circ \Phi) (s, \vartheta) = (Y(s^+, \vartheta^+) - Y(s, \vartheta), Y(s^+, \vartheta^+) - Y(s, \vartheta))  \\
  & \quad = (R_1(s, \vartheta) \vartheta^6, R_2(s, \vartheta) \vartheta^6)
\end{aligned}
\]
where $R_1, R_2$ are analytic functions in $(s, \vartheta)$. The norm of $R_1, R_2$ can be estimated by the $6$th derivatives of $\Phi \circ T - A \circ \Phi$. 
Using similar computations as before, the $6$th derivative of $F(s^+)- 2F(s) + F(s^-)$ in $\vartheta$ depends on up to $5$ combined derivatives in $F'$ and $\partial_\vartheta(s^\pm)$, and since $F' = C \rho^{-\frac23}$, while $\|\partial_\vartheta(s^\pm)\|_\sigma \le C\|\rho\|_\sigma$  due to Lemma~\ref{lem:billiard-analytic}, $\|\partial^6_\vartheta\left( F(s^+)- 2F(s) + F(s^-) \right)\|_\sigma$ is estimated by $\|\rho\|_{\sigma, 5}$. Similarly, since $G''$ is a  $4$th degree differential polynomial of $\rho$, 
\[
\partial^6_\vartheta\left( \left( G(s^+) - 2G(s) + G(s^-) \right) \vartheta^2 \right) = Q_8 \vartheta^2 + Q_7 \vartheta + Q_6,
\]
where $\|Q_k\|_\sigma \le \|\rho\|_{\sigma, k}$. Combining all estimates, we obtain 
\[
	\left\|  \Phi \circ T - A \circ \Phi  \right\|_{\sigma} \le 
	C \left( \|\rho\|_{\sigma, 6} + \|\rho\|_{\sigma, 7} \sigma + \|\rho\|_{\sigma, 8} \sigma^2  \right) \le C \sigma_0^{-6} \|\rho\|_{\sigma_0}
\]
if $\sigma < \sigma_0/2$. By \eqref{eq:Phi-y}, the same estimate, with possibly a different constant, holds for $\Phi \circ T \circ \Phi^{-1} - A$. 
\end{proof}

\bibliographystyle{plain}
\bibliography{billiard}

\end{document}


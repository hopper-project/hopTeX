\documentclass{amsart}
\usepackage[T1]{fontenc}
\usepackage[english]{babel}
\usepackage{amsmath,amsthm,amssymb, latexsym,geometry,stackrel,enumerate}
\usepackage[all]{xy}

\usepackage[dvips]{graphicx}
\DeclareGraphicsExtensions{.bmp,.png,.pdf,.jpgg}
\usepackage[all]{xy}
\usepackage{verbatim}
\usepackage[mathcal]{euscript}
\usepackage{epsfig}
\usepackage{multicol}
\usepackage{color}
\usepackage{verbatim}
\usepackage{graphicx,float}

\swapnumbers
\theoremstyle{plain}
\newtheorem{teo}{Theorem}[section]
\newtheorem{lema}[teo]{Lemma}
\newtheorem{prop}[teo]{Proposition}
\newtheorem{coro}[teo]{Corollary}
\newtheorem*{Teo}{Theorem}
\newtheorem{convencion}[teo]{\bf{Convention}}

\theoremstyle{definition}
\newtheorem{defi}[teo]{Definition}
\newtheorem{obs}[teo]{Remark}
\newtheorem{obss}[teo]{Remarks}
\newtheorem{ejem}[teo]{Example}
\newtheorem{algo}[teo]{Algorithm}
\newtheorem{nota}[teo]{Notation}

\newenvironment{note}{\noindent NOTE: \rm}

\title[ Hochschild cohomology of  $m$-branched algebras]{Hochschild cohomology of $m$-Cluster tilted algebras of type $\widetilde{\mathbb{A}}$}

\author[V. Gubitosi]{Viviana Gubitosi}
\address{Instituto de Matem\'{a}tica y Estad\'{\i}stica Rafael Laguardia, Facultad de Ingenier\'{\i}a - UdelaR, Montevideo, Uruguay, 11200 }

\email{gubitosi@fing.edu.uy}

\keywords{$m$-cluster tilted algebras; gentle algebras; derived equivalence; Hochschild cohomology; branched algebras}

\begin{document}
\maketitle

\begin{abstract}
 In this paper, we compute the dimension of the Hochschild cohomology groups of any $m$-cluster tilted algebra of type $\tilde{\mathbb{A}}$. Moreover, we give conditions on the bounded quiver of an $m$-cluster tilted algebra $\Lambda$ of type $\tilde{\mathbb{A}}$ such that the Gerstenhaber algebra $\operatorname{HH}^*(\Lambda)$ has non-trivial multiplicative structures. We also show that the derived class of gentle $m$-cluster tilted algebras is not always completely determined by the Hochschild cohomology.
\end{abstract}

\section*{Introduction}

The Hochschild cohomology groups $\operatorname{HH}^{i}(A)$ of an algebra $A$, where $i\geq 0$, were introduced by Hochschild in \cite{Ho46}. The
low-dimensional groups, namely for $i=0,1,2$, have a concrete interpretation of classical algebraic structures, but in general it is quite hard to compute
them. However, an explicit formula for the dimension of the Hochschild cohomology $\operatorname{HH}^{i}(A)$ of some subclasses of
special biserial algebras had been computed in terms of combinatorial data, for example in \cite{B06,ST10,Lad12b}. Recently Redondo and
Rom\'an obtained a formula for quadratic string algebras and therefore for gentle algebras (see \cite{RR15}).

Cluster categories were introduced in \cite{Buan2006} as a representation theoretic framework for the  cluster algebras of Fomin and Zelevinski \cite{FZ02}. The clusters correspond to the tilting objects in the cluster category.
Given an hereditary finite dimensional algebra $H$  over  an algebraically closed field ${{\normalfont{\textsf{k}}}}$ the $m$-cluster category is defined to be $\mathcal{C}_m(H):=\mathcal{D}^b(H)/ \tau^{-1} [m]$, where $[m]$  denotes the $m$-th power of  the shift functor $[1]$ and $\tau$ is the Auslander - Reiten translation in $\mathcal{D}^b(H)$. By a result of Keller \cite{K05}, the $m$-cluster category is triangulated.  For the  $m$-cluster category,  $m$-cluster tilting objects have been defined by Thomas, in \cite{Thomas2007}, who in addition showed that they are in bijective correspondence with the $m$-clusters associated by  Fomin and Reading to a finite root system in \cite{FR05}. The endomorphism algebras of the $m$-cluster tilting objects are called $m$-cluster tilted algebras or, in case $m=1$, cluster tilted algebras.

In \cite{ABCP09} it has been shown that cluster tilted algebras are gentle if and only if they are of type  $\mathbb{A}$ or $\tilde{\mathbb{A}}$. On the other hand, using arguments similar to those of \cite{ABCP09}, Murphy showed in \cite{Murphy2010} that $m$-cluster tilted algebras of type $\mathbb{A}$ are gentle and he described the connected components of $m$-cluster tilted algebras up to derived equivalence, a result analogous to that of \cite{Buan2008}. Later, a similar work has been done in \cite{Gubitosi} for  $m$-cluster tilted algebras of type $\mathbb{\widetilde{A}}$, where it is shown that $m$-cluster tilted algebras of type $\mathbb{\widetilde{A}}$ are gentle and their possible bound quivers are described. Moreover, in \cite{BG14} and \cite{Gub1} the algebras that are derived equivalent to $m$-cluster tilted algebras of type  $\mathbb{A}$ and $\mathbb{\widetilde{A}}$ have been classified. They are called $\mathbb{A}$-branched algebras   \cite[Definition 4.3]{BG14} and  $\mathbb{\widetilde{A}}$-branched algebras \cite[Definition 3.2 ]{Gub1} respectively.

The aim of this work is to compute the dimension of $\operatorname{HH}^{n}(\Lambda)$ for any $m$-cluster tilted algebra $\Lambda$ of type $\tilde{\mathbb{A}}$. Since the  Hochschild cohomology is a derived invariant \cite{Ric91} we can extend the result to  the class  of $\tilde{\mathbb{A}}$-branched algebras.  Moreover, using the results of Redondo and Rom\'an we obtain that the non-trivial multiplicative structure of the Gerstenhaber algebra $\rm HH^*(\Lambda)$ for  $\Lambda$  an $m$-cluster tilted algebra  of type $\tilde{\mathbb{A}}$ (or more generally  an  $\tilde{\mathbb{A}}$-branched algebra) depends on the existence of  $m$-saturated cycles.

We now state the main results of this paper (for the definitions of the terms used, we refer the reader to section 1.4  below).

\subsection*{Theorem A}\textit{ Let $\Lambda$ be an   $\tilde{\mathbb{A}}$-branched algebra with parameters $s_1,s_2,k_1,k_2,r$ and  $m\geq 1$.   Then:}

\begin{itemize}
\item[(a)] ${\operatorname{dim}_k\operatorname{HH}^{{0}}(\Lambda)}= \begin{cases} 2 & \mbox{ if \ }  r=1,  \  k_1=s_1=0 \text{ or  }  r=-1,  \  k_2=s_2=0;
\\
 1  & \mbox{otherwise}. \end{cases}$

\item[(b)] ${\operatorname{dim}_k\operatorname{HH}^{{1}}(\Lambda)}= \begin{cases} 3 & \mbox{ if \ }  k_1=k_2=0, \  s_1=s_2=1 ;
\\
 k_1+k_2+2  & \mbox{if \ }  r=0,  \  s_1=1,  \  k_1=0 \text{ or  }  r=0,  \  s_2=1,  \  k_2=0 ; \\
k_1+k_2 + 1 & \mbox{otherwise}. \end{cases}$

\item[(c)] If $\operatorname{char}{{\normalfont{\textsf{k}}}}\neq 2$ and $n\geq 2$, then $${\operatorname{dim}_k\operatorname{HH}^{{n}}(\Lambda)}= \begin{cases} 1-\delta_{1,m}+ k_1+k_2 & \mbox{ if \ } n\equiv 0,1 \pmod{\rm lcm(m+2,2)}; \\ 1-\delta_{1,m} &
\mbox{otherwise.} \end{cases}$$

\item[(d)] If $\operatorname{char}{{\normalfont{\textsf{k}}}}= 2$ and $n\geq 2$, then $${\operatorname{dim}_k\operatorname{HH}^{{n}}(\Lambda)}= \begin{cases} 1-\delta_{1,m}+k_1+k_2 & \mbox{ if \ } n\equiv 0,1 \pmod{m+2};\\ 1-\delta_{1,m} & \mbox{otherwise.} \end{cases}.$$
\end{itemize}

where $\delta_{1,m}=1$ if $m=1$ and $0$ otherwise.

Moreover, we have:

\subsection*{Theorem B}\textit{ Let $\Lambda$ be an  $m$-cluster tilted algebra of type $\tilde{\mathbb{A}}$.
If its bound quiver contains at least one $m$-saturated cycle,
then the cup product defined in $\operatorname{HH}^*(\Lambda)$ is non trivial; and if $\operatorname{char}{{\normalfont{\textsf{k}}}}=0$, then the Lie bracket is also
non trivial.}
\medskip

In particular, specializing to the case $m=1$, we recover known results of \cite{Va}.\\

The paper is organized as follows: In section 1 we recall facts about gentle algebras, Hochschild cohomology, $m$-cluster tilted algebras and $\widetilde{\mathbb{A}}$-branched algebras. Also  we establish the facts about $m$-cluster tilted algebras of type $\mathbb{\widetilde{A}}$ and $\widetilde{\mathbb{A}}$-branched algebras that will be used in the sequel. Section 2  and 3 are devoted to the proof of the theorem  A and B respectively.
In section 4 we show that the Hochschild cohomology  is not a complete invariant for $\tilde{\mathbb{A}}$-branched algebras.\\

\section{Preliminaries}

\subsection{Gentle algebras}
While we briefly recall some  concepts concerning bound quivers and algebras, we refer the reader to \cite{ASS06} or \cite{ARS95}, for instance, for unexplained notions.

Let {{\normalfont{\textsf{k}}}} \  be a commutative field. A quiver $Q$ is the data of two sets, $Q_0$ (the \textit{vertices}) and $Q_1$ (the \textit{arrows}) and two maps {${s,t}\colon {Q_1} \to {Q_0}$} that assign to each arrow ${\alpha}$ its \textit{source} $s({\alpha})$ and its \textit{target} $t({\alpha})$. We write {${\alpha}\colon {s({\alpha})} \to {t({\alpha})}$}. If ${\beta}\in Q_1$ is such that $t({\alpha})=s({\beta})$ then the composition of ${\alpha}$ and ${\beta}$ is the path ${\alpha}{\beta}$. This extends naturally to paths of arbitrary positive length. The \emph{path algebra} ${{\normalfont{\textsf{k}}}} Q$ is the ${{\normalfont{\textsf{k}}}}$-algebra whose basis is the set of all paths in $Q$, including one stationary path $e_x$ at each vertex $x\in Q_0$, endowed with the  multiplication induced from the composition of paths. In case $|Q_0|$ is finite, the sum of the stationary paths  - one for each vertex - is the identity.

If the quiver $Q$ has no oriented cycles, it is called \emph{acyclic}. A \emph{relation} in $Q$ is a ${{\normalfont{\textsf{k}}}}$-linear combination of paths of length at least $2$ sharing source and target.  A relation which is a path is called \emph{monomial}, and the relation is \emph{quadratic} if the paths appearing in it have all length $2$. Let $\mathcal{R}$ be a set of relations.
 Given $\mathcal{R}$ one can consider the two-sided ideal of ${{\normalfont{\textsf{k}}}} Q$ it generates $I=\langle \mathcal{R}\rangle \subseteq  \langle Q_1 \rangle^2$. It is called \emph{admissible} if there exists a natural number $r\geqslant 2$ such that $\left\langle Q_1 \right\rangle^r \subseteq I$. The pair $(Q,I)$ is a \emph{bound quiver}, and associated to it is the algebra $A={{\normalfont{\textsf{k}}}} Q/I$.
It is known that any finite dimensional basic algebra over an algebraically closed field is obtained in this way, see \cite{ASS06}, for instance.

The class of gentle algebras  defined by Assem and Skowro\'nski in \cite{AH81}  has been extensively studied, see \cite{AS87, AG08, BB10, Buan2008, Murphy2010, SZ03}, for instance, and is particularly well understood, at least from the representation theoretic point of view. This class includes, among others, iterated tilted,  cluster tilted and $m$-cluster tilted algebras of types $\mathbb{A}$ and $\tilde{\mathbb{A}}$, and, as shown in \cite{SZ03}, is closed under derived equivalence.

Recall  that an algebra  $A= {{\normalfont{\textsf{k}}}} Q/I$ is said to be \emph{gentle} if  $I$ is generated by a set of monomial quadratic relations such that:
 

\begin{enumerate}
 \item[G1.] For every vertex $x\in Q_0$  at most two arrows enter or leave $x$;
 \item[G2.] For every arrow $\alpha\in Q_1$ there exists at most one arrow $\beta$ and one arrow $\gamma$ in $ Q_1$ such that $\alpha\beta\not\in I$, $\gamma\alpha\not\in I $;
 \item[G3.] For every arrow $\alpha\in Q_1$ there exists at most one arrow $\beta$ and one arrow $\gamma$ in $Q_1$ such that $\alpha\beta\in I$, $\gamma\alpha\in I$.
\end{enumerate}

\subsection{Hochschild cohomology}

Given an algebra $A$, the $n$-th Hochschild cohomology group of $A$ with coefficients in the bimodule $_A A_{A}$ is the extension group $\operatorname{HH}^{n}(A)={{\normalfont{\textsf{{Ext}}}}_{{A-A}}^{{^n}}({A},\,{A})}$. The sum $\operatorname{HH}^{\ast}(A)= \bigoplus_{n\geqslant 0} \operatorname{HH}^{n}(A)$ has the additional structure of a Gerstenhaber algebra, see \cite{G63}. From \cite{R89, K04} this structure is known to be a derived invariant, that is, invariant under derived equivalence.

Let $A={{\normalfont{\textsf{k}}}} Q /I$ be a monomial quadratic algebra. Define ${\Gamma}_0={\Gamma}_0(Q)=Q_0,\ {\Gamma}_1={\Gamma}_1(Q)=Q_1$, and for $n\geqslant 2$, ${\Gamma}_n={\Gamma}_n(Q,I)=\{{\alpha}_1{\alpha}_2\cdots {\alpha}_n|\ {\alpha}_i{\alpha}_{i+1}\in I\}$. Moreover, let $E={\normalfont{\textsf{{k}}}}Q_0$ be the semi-simple algebra isomorphic to $A/{{\normalfont{\textsf{{rad}}}} {A}}$, and ${\normalfont{\textsf{{k}}}}{\Gamma}_n$ the ${\normalfont{\textsf{{k}}}}$-vector space with basis ${\Gamma}_n$. The latter are also $E-E$-bimodules in an obvious way. In what follows, tensor products are taken over $E$.

As mentioned before, the sum $\operatorname{HH}^{\ast}(A) = \bigoplus_{n\geqslant 0} \operatorname{HH}^{n}(A))$ has additional structure given by two products, which we now describe, see \cite{G63}. The two products are defined using the standard resolution of $A$, but using appropriate explicit maps between the resolutions, see \cite[Section 2]{SF08} and \cite[Section 1]{B06}, we can carry them to our context.

Given $f\in {{\normalfont{\textsf{{Hom}}}}_{{E-E}}({{{\normalfont{\textsf{k}}}}{\Gamma}_n},\,{A})}$, $g\in{{\normalfont{\textsf{{Hom}}}}_{{E-E}}({{{\normalfont{\textsf{k}}}}{\Gamma}_m},\,{A})}$, and $i\in\{1,2,\ldots n\}$, define the element $f\circ_i g$ as $f \left({l}^{i-1} \otimes g \otimes {l}^{n-i}\right)$. In addition, define the \textit{composition product} as
$$f \circ g = \sum_{i=1}^n (-1)^{(i-1)(m-1)} f\circ_i g$$
and the \emph{bracket} to be
$$[f,g]= f\circ g - (-1)^{(n-1)(m-1)}g \circ f$$

On the other hand, denote by {${\sigma}\colon {A\otimes A} \to {A}$} the multiplication of $A$. The \emph{cup-product} $f\cup g$ of $f$ and $g$ is  the  element of ${{\normalfont{\textsf{{Hom}}}}_{{E-E}}({{{\normalfont{\textsf{k}}}} {\Gamma}_{n+m}},\,{A})}$ defined by $f\cup g = \sigma(f \otimes g)$.

Recall that a Gerstenhaber algebra is a graded ${{\normalfont{\textsf{k}}}}$-vector space $A$ endowed with a product which makes $A$ into a graded commutative algebra, and a bracket $[ - ,  - ]$ of degree $\^{a}ˆ’1$ that makes $A$ into a graded Lie algebra, and such that $[x, yz] = [x, y]z + (-1)^{(|x|-1)|y|} y[x, z]$, that is, a graded analogous of a Poisson algebra. The cup product $\cup$ and the bracket $[ - , - ]$ defined above define products in $\operatorname{HH}^{\ast}(A)$, which becomes then a Gerstenhaber algebra (see \cite{G63}).

\subsection{ Hochschild cohomology groups of gentle algebras}

The Hochschild cohomology groups of gentle algebras have already been computed in \cite{Lad12b} by Ladkani and in \cite{RR15} by Redondo and Rom\'an. In the
first case,  these results have been expressed in terms of the derived invariant introduced by Avella-Alaminos and Geiss (AG-invariant for short) in \cite{AG08}. In the second one, Redondo and Rom\'an used Bardzell's resolution (see \cite{Ba97}).

The AG-invariant is a function $\phi_{A}:\mathbb N^2\rightarrow \mathbb N$ depending on the ordered pairs generated by a certain algorithm. The
number $\phi_{A}(n,m)$ counts how often each pair $(n,m)$ appears in the algorithm.  See \cite{AG08} for a complete definition of AG-invariant.

Since $m$-cluster tilted algebras of type $\tilde{\mathbb{A}}$  are gentle, we use the computation of Ladkani to prove our main result. The statement of Ladkani is the following:

\begin{teo}\cite[Corollary 1]{Lad12b}\label{TeoLad} Let $A$ be a gentle algebra. Define $\psi_A(n)=\sum_{d\mid n} \phi_A(0,d)$ for $n\geq 1$. Then
\begin{itemize}
\item[(a)] ${\operatorname{dim}_k\operatorname{HH}^{{0}}(\Lambda)}=1+ \phi_A(1,0)$.
\item[(b)] ${\operatorname{dim}_k\operatorname{HH}^{{1}}(\Lambda)}=1+{\lvert {Q_1}\rvert}-{\lvert {Q_0}\rvert}+\phi_A(1,1)+ \begin{cases} \phi_A(0,1) & \mbox{if }
\operatorname{char}(k)=2\\ 0& \mbox{otherwise} \end{cases}$.
\item[(c)] ${\operatorname{dim}_k\operatorname{HH}^{{n}}(\Lambda)}=\phi_A(1,n)+a_n\psi_A(n)+b_n\psi_A(n-1)$ for $n\geq 2$, where $$(a_n,b_n)=
\begin{cases} (1,0)& \mbox{if } \operatorname{char}k\neq 2 \mbox{ and $n$ is even}\\ (0,1)&\mbox{if } \operatorname{char}k\neq 2 \mbox{ and $n$ is odd}\\
(1,1)&\mbox{if } \operatorname{char}k=2 \end{cases} $$
\end{itemize}

\end{teo}

\subsection{$m$-Cluster tilted algebras of type $\widetilde{\mathbb{A}}$}

Given a bound quiver $(Q,I)$ and an integer $m$, a cycle is called \emph{ $m$-saturated} if it is an oriented cycle consisting of $m+2$ arrows such that the composition of any two consecutive arrows on this cycle belongs to $I$. Recall that two relations  $r$ and $r'$ in the bound quiver $(Q,I)$ are said to be   \textit{consecutive} if there is a walk  $v=wr=r'w'$ in  $(Q,I)$ such that $r$ and $r'$ point in the same direction and share an arrow. Also, recall that given a connected quiver $Q$, its \emph{Euler characteristic} is  $\chi(Q)=|Q_1|-|Q_0|+1$.\\

Recall from \cite[Definition 7.2]{Gubitosi} that an algebra  $A\cong {{\normalfont{\textsf{k}}}} Q/I$ is an  \textit{algebra with root } if its bound quiver is gentle, connected, having $\chi(Q)-1$  $m$-saturated cycles and no loops.

  Since  $\chi(Q)$ is the number of $m$-saturated cycles plus 1, we know that  $(Q,I)$  has at least a non $m$-saturated cycle $\widetilde{\mathcal{C}}$. We will refer to the  cycle $\widetilde{\mathcal{C}}$ as the  \textit{root cycle}. Moreover, since $A$ is a finite dimensional algebra, if $\widetilde{\mathcal{C}}$ is an oriented cycle, then it must have at least one relation.

Once we fix a root cycle $\widetilde{\mathcal{C}}$, let $\mathfrak{C}$ be the set of $m$-saturated cycles sharing at least two vertices with $\widetilde{\mathcal{C}}$. Let $(Q',I')$ be the bound quiver such that $v\in Q'_0$ if there is $\alpha \not \in (\widetilde{\mathcal{C}}\cup\mathfrak{C})_1$ such that   $s(\alpha)=v$ or $t(\alpha)=v$ and $Q'_1=Q_1\setminus (\widetilde{\mathcal{C}}\cup\mathfrak{C})_1$. If $\mathcal{F}$ is a minimal set of relations generating $I$, let $\mathcal{F'}$ be obtained by deleting from $\mathcal{F}$ the relations involving arrows in $(\widetilde{\mathcal{C}}\cup\mathfrak{C})_1$. Then $I'=\langle \mathcal{F'} \rangle $.  Every connected component of $Q'$ is said to be a \textit{ray}.

Therefore every ray is  a gentle  quiver $\mathcal{R}$ having $\chi(\mathcal{R})$ $m$-saturated cycles; i.e. is a quiver derived equivalent to a quiver of an $m$-cluster tilted algebra of type $\mathbb{A}$. \\
If the ray $\mathcal{R}$ shares just one vertex with the cycle $\widetilde{\mathcal{C}}$, this vertex is called the \emph{union vertex} of the ray. For each union vertex there is at least one relation $\rho$ involving at least one arrow of $\widetilde{\mathcal{C}}$. If both arrows of $\rho$ belong to the root cycle, $\rho$ is called \textit{internal union relation} of the ray. If instead  just one arrow of $\rho$ belongs to the root cycle, $\rho$ is called \textit{external union relation} of the ray.
If the ray $\mathcal{R}$  and the cycle $\widetilde{\mathcal{C}}$ are connected through an  $m$-saturated cycle, we say that $\mathcal{R}$ is a ray without union relations.

\subsection*{Theorem }\cite[Theorem 7.16]{Gubitosi}\textit{ A connected algebra $A={{\normalfont{\textsf{k}}}} Q/I$ is  a connected component of an $m$-cluster tilted algebra of type $\widetilde{\mathbb{A}}$ if and only if  $(Q,I)$ is a gentle bound quiver satisfying the following conditions: }
\textit{\begin{itemize}
  \item [(a)] It can contain a non-saturated  cycle  $\widetilde{\mathcal{C}}$ in such a way that $A$ is an algebra with root $\widetilde{\mathcal{C}}$.
  \item [(b)] If it does not contain a non-saturated  cycle as in $(a)$ , then the only  possible cycles  are  $m$-saturated.
  \item [(c)] Outside of an $m$-saturated cycle it can have at most  $m-1$ consecutive relations.
 
  \item [(d)] If there are internal relations in the root cycle, then   the number  of clockwise oriented  relations   is equal modulo $m$ to the number of counterclockwise oriented.
\end{itemize}}

Let $(Q,I)$ be the quiver of an $m$-cluster tilted algebra of type $\widetilde{\mathbb{A}}$.  As in \cite[Section 7]{Gub1} we define five parameters $s_1,s_2,k_1,k_2$ and $r$ for $(Q,I)$ as follows:

\begin{defi}\cite[Definition 7.2]{Gub1}
Let $s_1$ be the number of arrows which are not part of any $m$-saturated cycle and which fulfill one of the following conditions:

\begin{itemize}
  \item  [a)] These arrows are part of the root cycle and they are oriented in the counterclockwise direction.
  \item  [b)] These arrows belong to a ray attached to the root cycle by a counterclockwise internal union relation and this relation does not involve the arrows.
  \item  [c)] These arrows belong to a ray attached to the root cycle by a clockwise internal union relation and this relation  involves the arrows.
  \item  [d)] These arrows belong to a ray attached to the root cycle by a counterclockwise external union relation or a ray without union relations.
\end{itemize}

Let $k_1$ be the number of  $m$-saturated cycles  which fulfill one of the following conditions:

\begin{itemize}
  \item  [a)] These cycles share one arrow $\alpha$ with the root cycle and $\alpha$ is oriented in the counterclockwise direction.
  \item  [b)] These cycles belong to a ray attached to the root cycle by a clockwise internal union relation.
  \item  [c)] These cycles belong to a ray attached to the root cycle by a counterclockwise external union relation or a ray without union relations..
\end{itemize}

Similary we define the parameters $s_2$ and $k_2$   permuting the words  'counterclockwise' and 'clockwise'.\\
Let $r$ be the number of clockwise internal relations minus the number of counterclockwise internal relations.
\end{defi}

\subsection{$\widetilde{\mathbb{A}}$-branched algebras}

Let $A$ be an algebra with root and let $\mathcal{S}$ be the set of all arrows in the quiver of $A$ not belonging to any   $m$-saturated cycle.\\

The   \textit{number of free clockwise arrows} in $\mathcal{S}$ is equal to the number of clockwise oriented arrows on the root cycle that are not  involved in any internal union relation   plus the number of clockwise  internal union relations  plus the number of arrows on the rays associated to clockwise union relations  (internal or external).

Dually, we define the number  of free counterclockwise arrows.

The algebras that satisfy the following definition are the algebras derived equivalents to $m$-cluster tilted algebras of type $\widetilde{\mathbb{A}}$ with a root cycle \cite[Theorem 6.4]{Gub1}.

\begin{defi}
We say that a connected algebra  $B={{\normalfont{\textsf{k}}}} Q/I$  is \textit{$\tilde{\mathbb{A}}$}\textit{-branched}  if $B$ satisfies the following conditions:

\begin{itemize}
\item [(a)] There is a  cycle $\widetilde{\mathcal{C}}$ in $Q$ in such a way that $B$ is an algebra with root $\widetilde{\mathcal{C}}$.
\item [(b)] In the root cycle  the number  $r_h$  of clockwise oriented relations is the same modulo  $m$  that the number  $r_a$ of counterclockwise oriented relations.
\item [(c)] If $|r_h-r_a|=r=\alpha (m-1) + \beta$ (with $\beta< m-1$), then there must exist $r+1+\varepsilon$ free arrows  not belonging to any   $m$-saturated cycle on the clockwise sense if  $r_h>r_a$ or in the counterclockwise sense otherwise. Here,

    $$\varepsilon =\left\{
              \begin{array}{ll}
                \alpha - 1, & \hbox{\text{if }  $\beta=0$ ;} \\
                \alpha, & \hbox{\text{if}  $\beta\neq 0$ .}
              \end{array}
            \right.$$
\end{itemize}

\end{defi}

\section{Main result}

According to \cite[Theorem 6.4]{Gub1} any $m$-cluster tilted algebra of type $\tilde{\mathbb{A}}$ ( or $\tilde{\mathbb{A}}$-branched algebra) is derived equivalent to  an  $\tilde{\mathbb{A}}$-branched algebra with \textit{normal form}. See \cite[Definitions 4.1 and 4.3]{Gub1}.

\begin{figure}[H]

$$\SelectTips{eu}{10}\xymatrix@C=.1pc@R=.2pc{  &&& &&& &&& &&& &&& &&& & .\ar[dddl] \ar@{.}@/^/[r]& . &\\
 &&& &&& &&& &&& &&& &&& &&&\\
 &&& &&& &&& &&& &&& &&& &&&\\
&&& .\ar[rrr]^{{\alpha}_1} & \ar@{.}@/^/[lld] & &    \ar@{.}[rrr] &&\ar@{.}@/_/[rr]& . \ar[rrr]^{{\alpha}_r} &&& . \ar@{.}[rrr] &&& . \ar[rrr]^{{\alpha}_{r+s_2}} &&&  . \ar[rrr]_{{\alpha}_{s_2+1}} &&& .\ar[uuul] \ar@{.}[rrrdd] &&&  && .\ar[ddll] \ar@{.}@/^/[dr] &&& &&&\\
&&&  &&& &&& &&& &&& &&& &&& &&& &&& . &&& &&& \\
 &&& &&& &&& &&& &&& &&& &&& &&&  . \ar[rrdd]_{{\alpha}_{s_2+k_2}}    \\
  &&&  &&& &&& &&& &&& &&& &&& &&& &&& \\
 \scriptstyle{0} \ar[rrruuuu]^{{\alpha}_0} \ar[rrrdddd]_{{\beta}_0}  &&& &&& &&& &&& &&& &&& &&&  &&& && .\ar[uuur] \ar[dddr]&&& \\
 &&&  &&& &&& &&& &&& &&& &&& &&& &&& \\
  &&&  &&& &&& &&& &&& &&& &&& &&&   \ar[rruu]^{{\beta}_{s_1+k_1}}  . &&& \\
  &&&  &&& &&& &&& &&& &&& &&&  &&& &&& .  \\
 &&& .\ar[rrr]_{{\beta}_1}  &&& \ar@{.}[rrrrrrrrr] &&&   &&&   &&& . \ar[rrr]_{{\beta}_{s_1}} &&&. \ar[rrr]^{{\beta}_{s_1+1}} &&&  . \ar[dddl] \ar@{.}[rrruu] &&& && . \ar[uull] \ar@{.}@/_/[ur]  &&& &&&\\
&&&  &&& &&& &&& &&& &&& &&&\\
&&&  &&& &&& &&& &&& &&& &&&\\
 &&&  &&& &&& &&& &&& &&& & . \ar[uuul]\ar@{.}@/_/[r]& . &\\
 &&&&&&&&&&&&&&&&&&}$$

\caption{The bound quiver of a normal form.}
\end{figure}

Observe that we allow  $s_1=k_1=0$.\\

Then writing \cite[Propositions 4.3 and 4.5]{Gub1} in terms of the previous parameters we obtain:

\begin{prop}\label{phibranched}
Let $\Lambda$ be an   $\tilde{\mathbb{A}}$-branched algebra with parameters $s_1,s_2,k_1,k_2$ and $r$. Then
$$\phi_{\Lambda}=(mk_1+s_1+r,s_1)^*+ (mk_2+s_2-r,s_2)^*+ (k_1+k_2).(0,m+2)^*$$
\end{prop}

where given a pair $(a,b)\in \mathbb{N} \times \mathbb{N}$,  $(a,b)^\ast$ denotes the characteristic function of the set  $\{(a,b)\} \subseteq \mathbb{N} \times \mathbb{N}$.\qed

\medskip

We are now able to prove our main result.

\begin{proof}[Proof of Theorem A] Using  Theorem \ref{TeoLad} and Proposition \ref{phibranched}, we compute the dimension of Hochschild cohomology groups for an $m$-cluster tilted algebra of type $\widetilde{\mathbb{A}}$:

\begin{itemize}

\item[(i)] Assume $(mk_1+s_1+r,s_1)=(1,0)$ or $(mk_2+s_2-r,s_2)=(1,0)$. A simple computation gives $s_1=k_1=0$ and $r=1$ or  $s_2=k_2=0$ and $r=-1$ and in both cases $\phi_{\Lambda}(1,0)=1$.

\item[(ii)] The equality $(mk_1+s_1+r,s_1)=(1,1)$ holds if $r=0$,  $ s_1=1$ and   $k_1=0$. Analogously  $(mk_2+s_2-r,s_2)=(1,1)$ if $r=0$,  $s_2=1$ and  $k_2=0$.  Then $\phi_{\Lambda}(1,1)=2$  if both equalities hold in which case $ s_1=s_2=1$ and   $k_1=k_2=0$ or $\phi_{\Lambda}(1,1)=1$ if just one equality holds.

\item[(iii)] Since $m\geq1$ there are no pairs $(0,1)$ and therefore $\phi_{\Lambda}(0,1)=0$.

\item[(iv)]  Assume  $(mk_1+s_1+r,s_1)=(1,n)$ or  $(mk_2+s_2-r,s_2)=(1,n)$ with $n\geq 2$. The first equality never holds and the second one holds if and only if $k_2=0$ and $r+1=n=s_2$. An algebra with those parameters is  $m$-cluster tilted of type $\tilde{\mathbb{A}}$ if and only if $m\geq n$.  Then $\phi_{\Lambda}(1,n)=1$ unless $m=1$.

\item[(v)] Since $\phi_{\Lambda}(0,d)= k_1+k_2$ if $d=m+2$ and $0$ otherwise, the function $\psi_{\Lambda}(n)$, defined in Theorem \ref{TeoLad}, depends on $n\equiv 0\pmod{m+2}$, then $$\psi_{\Lambda}(n)=\begin{cases} k_1+k_2 &
    \mbox{if } n\equiv 0\pmod{m+2}\\ 0 & \mbox{otherwise.} \end{cases}$$

Therefore the final expression of the ${\operatorname{dim}_k\operatorname{HH}^{{n}}(\Lambda)}$ depends on the value of $n$ and $n-1$ modulo $m+2$, the characteristic of ${{\normalfont{\textsf{k}}}}$ and the parity of $n$. We start assuming that $\operatorname{char} {{\normalfont{\textsf{k}}}} \neq 2$. If $n\equiv 0 \pmod{m+2}$, we have $\psi_{\Lambda}(n)= k_1+k_2$ and $\psi_{\Lambda}(n-1)=0$, then

\begin{equation*}  {\operatorname{dim}_k\operatorname{HH}^{{n}}(\Lambda)}= \phi_{\Lambda}(1,n)+ \begin{cases} k_1+k_2 &  \text{if } n \mbox{ is  even }\\ 0 & \text{if } n \mbox{ is   odd} \end{cases} \end{equation*}

If $n\equiv 1 \pmod{m+2}$, we have $\psi_{\Lambda}(n-1)= k_1+k_2$ and $\psi_{\Lambda}(n)=0$, then

\begin{equation*}  {\operatorname{dim}_k\operatorname{HH}^{{n}}(\Lambda)}= \phi_{\Lambda}(1,n)+ \begin{cases} k_1+k_2 &  \text{if } n \mbox{ is  odd }\\ 0 & \text{if } n \mbox{ is   even} \end{cases} \end{equation*}

Since the parity of $n$ can be described in terms of the value of $n$ module $2$, ${\operatorname{dim}_k\operatorname{HH}^{{n}}(\Lambda)}$ depends on the value of $n$ modulo $\rm lcm(m+2,2)$. Therefore,  $${\operatorname{dim}_k\operatorname{HH}^{{n}}(\Lambda)}= \phi_{\Lambda}(1,n)+ \begin{cases} k_1+k_2 & \mbox{ if \ } n\equiv 0,1 \pmod{\rm lcm(m+2,2)}; \\ 0 &
\mbox{otherwise.} \end{cases}$$

Finally if $\operatorname{char} {{\normalfont{\textsf{k}}}}=2$, the parity of $n$ is not important and  ${\operatorname{dim}_k\operatorname{HH}^{{n}}(\Lambda)}$ depends only   on  the value of $n$ module
$m+2$. Then:

$${\operatorname{dim}_k\operatorname{HH}^{{n}}(\Lambda)}=\phi_{\Lambda}(1,n) + \begin{cases} k_1+k_2 & \mbox{ if $n\equiv 0,1\pmod {m+2} $}\\ 0 & \mbox{ otherwise.} \end{cases}$$

\end{itemize}
\end{proof}

\section{Non-trivial structure of the Gerstenhaber algebra $\operatorname{HH}^*(\Lambda)$}

Finally we show that the non-trivial structure of the Gerstenhaber algebra $\operatorname{HH}^*(\Lambda)$ depends only on the existence of $m$-saturated cycles. From now on, we follow the notation of \cite{RR15}. \\

For $n\geq 2$, write ${\Gamma}_n=\{\alpha_1\alpha_2\cdots\alpha_n\mid  \alpha_i\alpha_{i+1}\in I\}$.
  Let $\mathcal{C}_n$ be the set of  pairs  $(\alpha_1\alpha_2\cdots\alpha_n, e_r) \in {\Gamma}_n\times Q_0$ such that $s(\alpha_1)=t(\alpha_n)=e_r$ and $\alpha_n\alpha_1\in I$. The pairs belonging to  any set $\mathcal{C}_n$ are called \textit{complete pairs}. Let $\mathcal{C}_n(0)$ be the subset of complete pairs $(\alpha_1\alpha_2\cdots\alpha_n, e_r) \in {\Gamma}_n\times Q_0$ such that  it does not exist $\gamma\in Q_1\setminus\{\alpha_n\}$  nor $\beta\in Q_1\setminus\{\alpha_1\}$  with $\alpha_n\beta, \gamma\alpha_1\in I$.
The cyclic group $\mathbb{Z}_n=<t>$ of order $n$ acts on the set  $\mathcal{C}_n$ with the action given by $t(\alpha_1\alpha_2\cdots\alpha_n, e_{s(\alpha_1)})=(\alpha_n\alpha_1\alpha_2\cdots\alpha_{n-1}, e_{s(\alpha_n)})$. A complete pair $(\alpha_1\alpha_2\cdots\alpha_n, e_r)$ is called  \textit{gentle} if $t^m(\alpha_1\alpha_2\cdots\alpha_n, e_r)\in \mathcal{C}_n(0)$ for any $m\in \mathbb{Z}$.  We will denote by  $\mathcal G_{n}$  the set of gentle pairs in ${\Gamma}_n\times Q_0$.\\

\begin{teo}\cite[Theorems 4.5 and 4.9]{RR15}\label{gerstenhaberstructure}
Let $A={{\normalfont{\textsf{k}}}} Q/I$ be a gentle algebra such that $\mathcal G_{n}$ is not empty for some $n>0$. Then the cup product defined in $\operatorname{HH}^*(A)$   is non-trivial. If in addition $char{{\normalfont{\textsf{k}}}}=0$, then the Lie bracket is also non-trivial.
\end{teo}

\begin{proof}[Proof of Theorem B] By the previous theorem 
it is enough to observe that the set $\mathcal G_{n}$ is not empty if and only if $n\equiv 0\pmod{m+2}$ and $\Lambda$ contains at least one $m$-saturated cycle.

\end{proof}

\section{Further consequences}

We conclude this work showing that the Hochschild cohomology groups, together with the number of vertices of the ordinary quiver, do not yield a complete system of invariants  for $\tilde{\mathbb{A}}$-branched algebras. In \cite[Theorem 1.2]{BG14} the authors  showed that any two  $m$-cluster tilted algebras $A$ and $B$ of type $\mathbb{A}$  are derived equivalent if and only if  $\operatorname{HH}^*(A)\cong \operatorname{HH}^*(B)$ and ${\lvert {Q_0(A)}\rvert}={\lvert {Q_0(B)}\rvert}$. However, the following example show that the Hochschild cohomology loses information, and then it is not always  a good derived invariant.

\begin{ejem}
 Let $(Q,I)$ and $(Q',I')$ be the bound quivers
\medskip
\begin{center}
\begin{tabular}{ccccc}
$\SelectTips{eu}{10}\xymatrix@R=1pc@C=.6pc{
&&.\ar[drr]	&	&	& &\\
.\ar[urr] \ar[dr]	&&&&   .\ar[dl]	&& \\
&.\ar[rr]^{{\alpha}_0}	&& . \ar[d]^{{\alpha}_1}	&	&  & \\
& . \ar[u]^{{\alpha}_3} & & .\ar[ll]^{{\alpha}_2}&  &   &  }$    & && &     $\SelectTips{eu}{10}\xymatrix@R=1pc@C=.6pc{
&&.\ar[drr]	&	&	& &\\
.\ar[urr] \ar[dr]	&&&&   .\ar[dl]_{{\alpha}_0}	&& \\
&.\ar[rr]	&& . \ar[dr]_{{\alpha}_1}	&	&  . \ar[ul]_{{\alpha}_3}& \\
&  & && . \ar[ur]_{{\alpha}_2} & &  }$ \\
&&&&\\

$I=\langle {\alpha}_i{\alpha}_{i+1}| 0\leqslant i \leqslant 3\rangle$  &&&&  $I'=\langle {\alpha}_i{\alpha}_{i+1}| 0\leqslant i \leqslant 3 \rangle$  \\
and indices are read modulo 3.&&&& and indices are read modulo 3.
   \end{tabular}
\end{center}
\medskip

\end{ejem}

The algebras $A={{\normalfont{\textsf{k}}}} Q/I$ an $B={{\normalfont{\textsf{k}}}} Q'/I'$ are both $2$-cluster tilted algebras of type $\tilde{\mathbb{A}}$ not derived equivalents because $\phi_A=(3,1)^*+(3,3)^*+(0,4)^*$ and $\phi_B=(2,2)^*+(4,2)^*+(0,4)^*$. However ${\lvert {Q_0(A)}\rvert}={\lvert {Q_0(B)}\rvert}$ and $\operatorname{HH}^i(A)\cong \operatorname{HH}^i(B)$ for all $i$.

\section*{Acknowledgements}
The author gratefully  acknowledges financial support from  the  \emph{Agencia Nacional de Investigaci\'{o}n e Innovaci\'{o}n (ANII)} of Uruguay.

\begin{thebibliography}{10} \expandafter\ifx\csname url\endcsname\relax
  \fi
\expandafter\ifx\csname urlprefix\endcsname\relax\fi \expandafter\ifx\csname href\endcsname\relax
   \fi

\bibitem{ABCP09} I.Assem, T. Brüstle, G.Charbonneau-Jodoin,  P.G. Plamondon,  \textit{Gentle algebras arising from
surface triangulations}, Algebra Number Theory 4, 2 (2010), 201-229.

\bibitem{AH81} I. Assem,  D. Happel, \textit{Generalized tilted algebras of type An}, Commun. Algebra 9(20), 2101–
2125 (1981)

\bibitem{AS87} I. Assem, A.~Skowro{\'n}ski, \textit{Iterated tilted algebras of type ~An}, Math. Z. 195, 2 (1987), 269-290.

\bibitem{ASS06} I. Assem, D. Simson, A. Skowro{\'n}ski, \textit{Elements of the representation theory of associative
algebras 1: techniques of representation theory}, In: London Mathematical Society Student
Texts, vol. 65. Cambridge University Press, Cambridge (2006). Techniques of Representation
Theory

\bibitem{ARS95} M. Auslander, I. Reiten, O.S. Sverre : \textit{Representation theory of Artin algebras}, Cambridge
Studies in Advanced Mathematics, vol. 36. Cambridge University Press, Cambridge (1995)

\bibitem{AG08} D.~Avella-Alaminos, C.~Geiss, \textit{Combinatorial derived invariants for gentle algebras}, J. Pure Appl. Algebra 212~(1) (2008) 228--243.

\bibitem{Ba97} M.~J. Bardzell, \textit{The alternating syzygy behavior of monomial algebras}, J. Algebra 188~(1) (1997) 69--89.

\bibitem{BB10} G. Bobi«ski, A. B. Buan,  \textit{The algebras derived equivalent to gentle cluster tilted algebras}, J. Algebra
Appl. 11, 1 (2012), 1250012, 26.

\bibitem{Buan2006} A. B. Buan, R. Marsh, M. Reineke, I. Reiten and  G. Todorov, \textit{Tilting theory and cluster combinatorics},
Advances in Mathematics 204, 2 (Aug. 2006), 572-618.

\bibitem{Buan2008} A. B. Buan, D. F. Vatne,  \textit{Derived equivalence classification for cluster-tilted algebras of type An},
Journal of Algebra 319, 7 (Apr. 2008), 2723-2738.

\bibitem{B06} J.~C. Bustamante, \textit{The cohomology structure of string algebras}, J. Pure Appl. Algebra 204~(3) (2006) 616--626.

\bibitem{BG14} J.~C. Bustamante, V.~Gubitosi, \textit{Hochschild cohomology and the derived class of {$m$}-cluster tilted algebras of type {$\Bbb{A}$}},
  Algebr. Represent. Theory 17~(2) (2014) 445--467.

\bibitem{FR05} S. Fomin and N. Reading,  \textit{Generalized cluster complexes and Coxeter combinatorics}, Int. Math. Res. Not.
44 (2005), 2709-2757.

\bibitem{FZ02} S. Fomin and  A. Zelevinsky, \textit{Cluster algebras. I. Foundations}, J. Amer. Math. Soc 15, 2 (2002), 497-529.

\bibitem{G63}  M. Gerstenhaber,\textit{The cohomology structure of an associative ring},  Ann. Math. 78(2), 267–288 (1963)

\bibitem{Gubitosi} V. Gubitosi, \textit{m-cluster tilted algebras of type {$\tilde{\Bbb{A}}$}}, arXiv:1506.08874

\bibitem{Gub1} V. Gubitosi, \textit{Derived class of m-cluster tilted algebras of type {$\tilde{\Bbb{A}}$}}, arXiv:1507.07484

\bibitem{Ha89} D.~Happel, \textit{Hochschild cohomology of finite-dimensional algebras}, in: S\'eminaire d'{A}lg\`ebre {P}aul {D}ubreil
  et {M}arie-{P}aul {M}alliavin, 39\`eme {A}nn\'ee ({P}aris, 1987/1988), Vol. 1404 of Lecture Notes in Math., Springer, Berlin, 1989, pp. 108--126.

\bibitem{Ho46} G.~Hochschild, \textit{On the cohomology theory for associative algebras}, Ann. of Math. (2) 47 (1946) 568--579.

\bibitem{K04} B. Keller,\textit{Hochschild cohomology and derived Picard groups}, J. Pure Appl. Algebra 190(1–3),
177–196 (2004)

\bibitem{K05} B. Keller,  \textit{On triangulated orbit categories}, Doc. Math. 10 (2005), 551-581.

\bibitem{Lad12b} S.~Ladkani, \textit{Hochschild cohomology of gentle algebras}, arXiv:1208.2230.

\bibitem{Murphy2010} G. J. Murphy,  \textit{Derived equivalence classification of m-cluster tilted algebras of type An}, Journal of Algebra 323, 4 (Feb. 2010), 920-965.

\bibitem{R89} J. Rickard, \textit{Morita theory for derived categories}, J. Lond. Math. Soc. 39(2–3), 436–456 (1989)

\bibitem{Ric91} J.~Rickard, \textit{Derived equivalences as derived functors}, J. London Math. Soc. (2) 43~(1) (1991) 37--48.

\bibitem{RR15} L.~Roman, M.~J. Redondo, \textit{Gerstenhaber algebra of quadratic string algebras}, arXiv:1504.02495.

\bibitem{SF08} S. Sánchez-Flores, \textit{ The Lie module structure on theHochschild cohomology groups of monomial
algebras with radical square zero}, J. Algebra 320(12), 4249–4269 (2008)

\bibitem{SZ03} J. Schröer, A. Zimmermann,  \textit{Stable endomorphism algebras of modules over special biserial algebras}, Math. Z. 244, 3 (2003), 515-530.

\bibitem{SW83} A.~Skowro{\'n}ski, J.~Waschb{\"u}sch,  \textit{Representation-finite biserial algebras},
  J. Reine Angew. Math. 345 (1983) 172--181.

\bibitem{ST10} N.~Snashall, R.~Taillefer, \textit{The {H}ochschild cohomology ring of a class of special biserial algebras}, J. Algebra Appl. 9~(1) (2010) 73--122.

\bibitem{Thomas2007} H. Thomas,  \textit{Defining an m-cluster category}, Journal of Algebra 318, 1 (Dec. 2007), 37-46.

\bibitem{Va} Y. Valdivieso Diaz,  \textit{Hochschild cohomology of Jacobian algebras from unpunctured surfaces: A geometric computation.}, arXiv:1512.00738.

\end{thebibliography}

\end{document} 

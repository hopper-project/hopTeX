\documentclass[reqno,a4paper]{amsart}
\usepackage[english,activeacute]{babel}
\usepackage{amssymb,amsmath,amsthm,amsfonts,mathrsfs,scalefnt,graphicx,graphics,color,hyperref,}
\usepackage{lmodern}
\usepackage{tikz}
\usetikzlibrary{trees}
\usetikzlibrary{shapes}
\usepackage[on]{auto-pst-pdf}
\usepackage{pst-tree}
\usepackage{pst-all}
\usepackage{tabularx}
\usepackage[T1]{fontenc}
\usepackage{url}
\usepackage[latin1]{inputenc}
\theoremstyle{plain}
\newtheorem{theorem}{Theorem}
\newtheorem{corollary}{Corollary}
\theoremstyle{definition}
\newtheorem{definition}{Definition}
\theoremstyle{plain}
\newtheorem{lemma}{Lemma}
\newtheorem{proposition}{Proposition}
\theoremstyle{plain}
\newtheorem{remark}{Remark}
\theoremstyle{plain}
\newtheorem{hyp}{Assumption}

\title[Mixed markets]{Asymptotic arbitrage in fractional mixed markets}
\author{Fernando Cordero}
\address{Faculty of Technology, University of Bielefeld, Universit\"{a}tsstr. 25, 33615 Bielefeld, Germany}
\email{fcordero@techfak.uni-bielefeld.de}

\author{Irene Klein}
\address{Department of Statistics and Operations Research, University of Vienna, Oskar-Morgenstern-Platz 1, 1090 Vienna, Austria}
\email{irene.klein@univie.ac.at}

\author{Lavinia Perez-Ostafe}
\address{Department of Mathematics, University of Vienna, Oskar-Morgenstern-Platz 1, 1090 Vienna, Austria}
\email{lavinia.ostafe@univie.ac.at}
\thanks{The third author gratefully acknowledges financial support from the Austrian Science Fund (FWF): J3453-N25}
\date{\today}
\begin{document}
\subjclass[2010]{}
\keywords{Mixed fractional Brownian motion, relative entropy, large financial market, entire asymptotic separation, strong asymptotic arbitrage}

\begin{abstract}
 We consider a family of mixed processes given as the sum of a fractional Brownian motion with Hurst parameter $H\in(3/4,1)$ and a multiple of an independent standard Brownian motion, the family being indexed by the scaling factor in front of the Brownian motion. We analyze the family of underlying markets from the perspective of large financial markets. More precisely, we show the existence of a strong asymptotic arbitrage (SAA) when the scaling factor converges to zero. The proofs are based on a dichotomy result for sequences of Gaussian measures and the concept of relative entropy. Additionally, the relation between SAA and the notion of entire asymptotic separation for sequences of probability measures is exploited.
\end{abstract}
\maketitle

\section{Introduction}
Empirical studies of financial time series led to the conclusion that the log-return increments exhibit long-range dependence. This fact supports the idea of modelling the randomness of a risky asset using a fractional Brownian motion with Hurst parameter $H>1/2$. However, markets driven by a fractional Brownian motion have been extensively disputed, as the latter fails to be a semimartingale and, hence, they allow for a free lunch with vanishing risk (see \cite{G:R:S:2008}). 

Many attempts were proposed to overcome the drawbacks of the fractional Brownian motion. In this work, we deal with the regularization method proposed by Cheridito in \cite{Ch,Ch1} when $H>3/4$. The latter consists in adding to the fractional Brownian motion a multiple of an independent Brownian motion, the resulting process, called \textit{mixed fractional Brownian motion}, being Gaussian with the long-range dependence property. Moreover, as shown in \cite{Ch,Ch1}, when $H>3/4$ the mixed fractional Brownian motion is equivalent to a multiple of a Brownian motion. Therefore, a Black-Scholes type model in which the randomness of the risky asset is driven by a mixed fractional Brownian motion is arbitrage free and complete. We call such a model a \textit{mixed fractional Black-Scholes model}. 

In this paper, we study from the perspective of large financial markets the asymptotic arbitrage opportunities in the sequence of mixed fractional Black-Scholes models when the scaling factor in front of the Brownian motion converges to zero. We focus on the notion of \textit{strong asymptotic arbitrage} introduced by Kabanov and Kramkov in \cite{Kakra} as the possibility of getting arbitrarily rich with probability arbitrarily close to one by taking a vanishing risk. Our model fits the standard framework of large financial markets, as each mixed fractional Black-Scholes model is arbitrage free (and even complete). In this setting, a connection between the existence of strong asymptotic arbitrage and the entire asymptotic separation of the sequence of objective probability measures and the sequence of equivalent martingale measures was established in \cite{Kakra}.

We aim to show the existence of strong asymptotic arbitrage in the previously described large financial market. Based on the above discussion, we show that the sequence of objective probability measures is entirely asymptotically separable from the sequence of equivalent martingale measures. The proof uses the notion of relative entropy and a dichotomy result for sequences of Gaussian measures. Indeed, inspired by the work of Cheridito \cite{Ch,Ch1}, we first show, for each fixed market, that the entropy of the objective probability measure relative to the equivalent martingale measure, both restricted to a discrete partition, converges to infinity. Our proof then follows using tightness arguments for the sequence of Radon-Nikodym derivatives of the objective probability measures with respect to equivalent martingale measures and the fact that two sequences of Gaussian measures are either mutually contiguous or entirely separable. The latter is known in the literature as the equivalence/singularity dichotomy for sequences of Gaussian processes, see \cite{Eag81}.

The paper is structured as follows. In Section~\ref{S1}, we set the mixed fractional Black-Scholes model and recall the framework of the large financial market. At the end of this part, we state the main result (Theorem~\ref{t1}). Section \ref{S2} is dedicated to the proof of Theorem~\ref{t1}, whereas Section \ref{S3} provides a discussion about the existence of strong asymptotic arbitrage using self-financing strategies constrained to jump only in a finite set of times. We end our work with Appendix~\ref{A1} in which we recall the definition of relative entropy and an equivalent characterization in terms of the Radon-Nikodym derivative.
\section{Preliminaries and main results}\label{S1}

\subsection{Setting the model}\label{S1.1}
Let $(\Omega, {{\mathcal F}}, P)$ be a probability space.

\begin{definition}
A fractional Brownian motion $Z^H=(Z^H_t)_{t\geq0}$ with Hurst parameter $H\in(0,1)$ is a continuous centred Gaussian process with covariance function
$$\mathrm{Cov}(Z^H_t,Z^H_s)=E(Z^H_tZ^H_s)=\frac12\left(t^{2H}+s^{2H}-|t-s|^{2H}\right),\ s,t\geq0.$$
In particular, $Z^{\frac12}$ is a standard Brownian motion.
\end{definition}
A linear combination of different fractional Brownian motions is refered in the literature as a \textit{mixed fractional Brownian motion}. In order to avoid localization arguments we only consider finite time horizon processes. In addition we focus on linear combinations of a standard Brownian motion $(B_t)_{t\in[0,1]}$ and an independent fractional Brownian motion $(Z_t^H)_{t\in[0,1]}$ with Hurst parameter $H\in(3/4,1)$, both defined on $(\Omega, {{\mathcal F}}, P)$. Cheridito shows in \cite{Ch,Ch1} that, for each $\alpha\in{{\mathbb R}}$ the mixed process $M^{H,\alpha}:=(M_t^{H,\alpha})_{t\in[0,1]}$ defined by $$M_t^{H,\alpha}:=\alpha\, Z_t^H+ B_t,\quad t\in[0,1],$$
is equivalent to a Brownian motion. By this we mean that the measure $Q^{H,\alpha}$ induced on ${{\mathcal C}}[0,1]$ by $M^{H,\alpha}$ and the Wiener measure $Q_W$ (induced by the Brownian motion on ${{\mathcal C}}[0,1]$) are equivalent. As a consequence, the process $M^{H,\alpha}$ is a $({{\mathcal F}}_t^{H,\alpha})_{t\in[0,1]}$-semimartingale, where, for each $t\geq 0$, ${{\mathcal F}}_t^{H,\alpha}:=\overline{\sigma((M_s^{H,\alpha})_{s\in[0,t]})}$ is the right-continuous natural filtration augmented by the nullsets.

Now, for each $\alpha>0$, we refer as the \textit{$\alpha$-mixed fractional Black-Scholes model} to the financial market consisting of a risk free asset normalized to one and a risky asset $(S_t^{H,\alpha})_{t\in[0,1]}$ given by
\begin{equation}\label{smm}
 S_t^{H,\alpha}:=S_0^{H,\alpha} \exp\left(\left(\mu-\frac{\sigma^2}{2\alpha^2}\right)\, t+\sigma \left(Z_t^H+\frac{1}{\alpha} B_t\right)\right),\quad t\in [0,1],
\end{equation}
where $\mu\in{{\mathbb R}}$ and $\sigma>0$ represent the drift and the volatility of the asset. We denote by $X:=(X_{t})_{t\in[0,1]}$ the coordinate process in ${{\mathcal C}}[0,1]$ and we define the process $S^\alpha:=(S^\alpha_t)_{t\in[0,1]}$ as
\begin{equation}\label{sm}
 S^\alpha_t:=S^\alpha_0 \exp\left(\left(\mu-\frac{\sigma^2}{2\alpha^2}\right)\, t+\frac{\sigma}{\alpha} X_t\right),\quad t\in [0,1].
\end{equation}
From the above discussion, we conclude that $S^{H,\alpha}$ under $P$ is equivalent to $S^\alpha$ under $Q_W$, which is a martingale when $\mu=0$. For a general drift, we denote by $Q_{\frac{\mu\alpha}{\sigma}}$ the measure induced on ${{\mathcal C}}[0,1]$ by the Brownian motion with drift $-\frac{\mu\alpha}{\sigma}$ (in particular $Q_0=Q_W$). Thanks to the Girsanov theorem, the process $S^{H,\alpha}$ under $P$ is also equivalent to $S^\alpha$ under $Q_{\frac{\mu\alpha}{\sigma}}$, which is a martingale. Therefore, the $\alpha$-mixed fractional Black-Scholes model with the filtration $({{\mathcal F}}_t^{H,\alpha})_{t\in[0,1]}$ has a unique equivalent martingale measure, and therefore is arbitrage-free and complete.
\subsection{Asymptotic arbitrage}
In this work, we treat the collection of $\alpha$-mixed fractional Black-Scholes models from the perspective of large financial markets. This idea is formalized in the following definition.
\begin{definition}[The large mixed fractional market]
We call \textit{large mixed fractional market} to the family of $\alpha$-mixed fractional Black-Scholes models, $\alpha>0$, i.e.,  the large financial market $${{\mathbb L}}^H:=\left(\Omega,{{\mathcal F}},({{\mathcal F}}_t^{H,\alpha})_{t\in[0,1]},P,S^{H,\alpha}\right)_{\alpha>0}.$$
 \end{definition}
We aim to study the presence of asymptotic arbitrage in this large financial market when $\alpha$ tends to infinity, i.e. when the Brownian component asymptotically disappears. More precisely, we intend to investigate, using methods of \cite{Kakra}, the presence of a so-called \textit{strong asymptotic arbitrage}. The latter is an analogue concept of arbitrage but for sequences of markets rather than for a single market model. Intuitively, this kind of arbitrage for sequences of markets gives the possibility of getting arbitrarily rich with probability arbitrarily close to one while taking a vanishing risk. The definition below, adapted to our situation, reflects this idea.
\begin{definition}\label{SAA}
There exists a strong asymptotic arbitrage (SAA) in the large mixed fractional market as $\alpha$ tends to infinity if there exists a sequence $(\alpha_n)_{n\geq 1}$ converging to infinity and self--financing admissible trading strategies $({\varphi}^n)_{n\geq 1}$, with zero endowment for $S^{H,\alpha_n}$ such that
\begin{enumerate}
\item $V_t^n({\varphi}^n)\geq-c_n$, for all $0\leq t\leq 1$,
\item $\lim_{n\to\infty}P(V_1^n({\varphi}^n)\geq C_n)=1$,
\end{enumerate}
where $V_t^n({\varphi}^n)$ denotes the value process at time $t\in[0,1]$ in the $\alpha_n$-mixed fractional Black-Scholes model and $c_n$ and $C_n$ are sequences of positive real numbers with $c_n\to0$ and $C_n\to\infty$ as $n\to\infty$.
\end{definition}
We do not give the definitions of self-financing strategies, admissibility and of value processes here, since our approach will be not constructive. Instead,
we use an equivalent characterization of strong asymptotic arbitrage based on the notion of \textit{entire asymptotic separability of sequences of measures}, which is defined as follows.
\begin{definition}\label{asymsep}
A sequence of probability measures ${(P^n)}$ is entirely asymptotically separable from the sequence of probability measures ${(Q^n)}$, if
there exists a subsequence $n_k$ and a sequence of sets $A^k\in\mathcal{F}^{n_k}$ such that $\lim_{k\to\infty}P^{n_k}(A^{k})=1$
and $\lim_{k\to\infty}Q^{n_k}(A^k)=0$. In this case we write $(P^n)\vartriangle(Q^n)$. This notion is clearly symmetric, i.e. $(P^n)\vartriangle(Q^n)$ implies that  $(Q^n)\vartriangle(P^n)$ (this follows by taking the complements of the sets $A^k$).
\end{definition}
The precise relation between this notion and the existence of SAA is given in \cite[Proposition 4]{Kakra}. In the case of complete markets, this result takes the following simple form.
\begin{proposition}\label{pkk}
Consider the large financial market $(\Omega^n,{{\mathcal F}}^n,({{\mathcal F}}_t^n)_{t\in[0,T]}, P^n)_{n\geq 0}$ and assume that each small market is complete. For each $n\geq 0$, let $Q^n\sim P^n$ be the unique equivalent martingale measure. Then the following conditions are equivalent
\begin{enumerate}
\item There is a SAA.
 \item $(P^n)\vartriangle(Q^n)$.
  \end{enumerate}
\end{proposition}
Therefore the study of SAA in ${{\mathbb L}}^H$ reduces to determining whether $(Q^{H,\alpha})_{\alpha>0}$ is entirely asymptotically separable from $(Q_{\frac{\mu\alpha}{\sigma}})_{\alpha>0}$ or not\footnote{In the case when $\mu=0$, the study of SAA reduces to show that $(Q^{H,\alpha})_{\alpha>0}\vartriangle Q_W$.}.

\subsection{Main result}\label{S1.3}
We state now our main result.
\begin{theorem}\label{t1}
There exists a strong asymptotic arbitrage in the large mixed fractional market ${{\mathbb L}}^H$ for $\alpha\to\infty$.
\end{theorem}
As mentioned we will show that $(Q^{H,\alpha})_{\alpha>0}\vartriangle (Q_{\frac{\mu\alpha}{\sigma}})_{\alpha>0}$.
\section{Proof of Theorem~\ref{t1}}\label{S2}
In order to prove Theorem~\ref{t1}, we provide a series of lemmas from which the desired result is obtained as a direct consequence. Before proceeding, we introduce though some notations.

Following the lines of \cite{Ch,Ch1}, we define, for all $n\in{{\mathbb N}}$, $Y_n:C[0,1]\rightarrow{{\mathbb R}}^n$ by:
$$Y_n(\omega)=\left(\omega\left(\frac1{n}\right)-\omega(0),\omega\left(\frac2{n}\right)-\omega\left(\frac1{n}\right),\ldots,\omega(1)-\omega\left(\frac{n-1}n\right)\right)^T$$
and denote $Q^{H,\alpha,n}$ and $Q_{\frac{\mu\alpha}{\sigma}}^n$ the restrictions of $Q^{H,\alpha}$ and $Q_{\frac{\mu\alpha}{\sigma}}$ to the $\sigma$-algebra ${{\mathcal F}}_n:=\sigma(Y_n)$.
We fix the Hurst parameter $H\in(3/4,1)$ and we avoid to mention the dependence on it by setting $Q^{H,\alpha}\equiv Q^{\alpha}$ and $Q^{H,\alpha,n}\equiv Q^{\alpha,n}$.

We denote by $C_n$ the covariance matrix of the increments of the fractional Brownian motion $Z^H$:
$$C_n^{i,j}:=\textrm{Cov}\left(Z_{\frac{i}{n}}^H-Z_{\frac{i-1}{n}}^H,Z_{\frac{j}{n}}^H-Z_{\frac{j-1}{n}}^H\right),\quad 1\leq i,j\leq n,$$
and by $\lambda_1^n,\dots,\lambda_n^n$ its eigenvalues. Since the matrix $C_n$ is symmetric and positive semi-definite, all the $\lambda_i^n$, $1\leq i\leq n$, are real and nonnegative.

We moreover set
$$\Sigma_0:=\frac{1}{n}I_n+\alpha^2 C_n\quad\textrm{and}\quad \Sigma_1:=\frac{1}{n}I_n+\frac1{n^2} \frac{\mu^2\alpha^2}{\sigma^2}\,1_{n\times n},$$
where $I_n$ is the identity matrix and $1_{n\times n}$ is the $n\times n$ matrix with all coefficients equal to $1$.

The proof of Theorem \ref{t1} strongly relies on the concept of relative entropy (also called sometimes Kullback-Leibler divergence) of the probability measure $Q^{\alpha,n}$ (respectively, $Q^{\alpha}$) relative to $Q_{\frac{\mu\alpha}{\sigma}}^n$ (respectively, $Q_{\frac{\mu\alpha}{\sigma}}$), denoted by $H(Q^{\alpha,n}|Q_{\frac{\mu\alpha}{\sigma}}^n)$ (respectively, $H(Q^{\alpha}|Q_{\frac{\mu\alpha}{\sigma}})$), see \cite[Section 6]{Hihi}. We recall the definition of relative entropy and some relevant results in Appendix~\ref{A1}.

\begin{lemma}\label{relent}
For each $n\geq 1$, we have
\begin{equation}\label{eg}
  H\left(Q^{\alpha,n}|Q_{\frac{\mu\alpha}{\sigma}}^n\right)=\frac{1}{2}\left[\textrm{tr}(\Sigma_1^{-1}\Sigma_0)-n+\frac{\mu^2\alpha^2}{\sigma^2 n^2} 1_n^T\Sigma_1^{-1}1_n+\ln\left(\frac{\textrm{det}\left(\Sigma_1\right)}{\textrm{det}\left(\Sigma_0\right)}\right)\right],
\end{equation}
where $1_n\in{{\mathbb R}}^n$  is the vector with all coordinates equal to $1$, and, for each square matrix $A$, $\textrm{tr}(A)$ and $\textrm{det}(A)$ denote the trace and the determinant of $A$, respectively.    
\end{lemma}
\begin{proof}
Note first that
 $$E_{Q^{\alpha,n}}\left[Y_nY_n^T\right]=\Sigma_0 \quad\textrm{and}\quad E_{Q_{\frac{\mu\alpha}{\sigma}}^n}\left[Y_nY_n^T\right]=\Sigma_1.$$
Note also that
$$E_{Q^{\alpha,n}}\left[Y_n\right]=0_n\quad\textrm{and}\quad E_{Q_{\frac{\mu\alpha}{\sigma}}^n}\left[Y_n\right]=-\frac{\mu\alpha}{\sigma n} 1_n,$$
where $0_n\in{{\mathbb R}}^n$ is the vector with all coordinates equal to $0$. Since $Y_n$ is a Gaussian vector under the two measures, the result follows using Lemma \ref{a1.2} and performing a straightforward calculation.
\end{proof}
Using standard properties of the trace and the determinant, it is not difficult to see that
\begin{equation}\label{trdet}
\textrm{tr}\left(\Sigma_0\right)=\sum\limits_{i=1}^n\left(\frac{1}{n}+\alpha^2\lambda_i^n\right)\quad\textrm{and}\quad\ln\left(\textrm{det}\left(\Sigma_0\right)\right)=\sum\limits_{i=1}^n\ln \left(\frac{1}{n}+\alpha^2\lambda_i^n\right).
\end{equation}
We set $a_n=\frac{1}{n}\frac{\mu^2\alpha^2}{\sigma^2}$ and note that $\Sigma_1=\frac{1}{n}(I_n+a_n1_{n\times n})$.
The next lemma summarizes the properties of the matrix $\Sigma_1$.
\begin{lemma}\label{linalg}
For each $n>1$, the eigenvalues of $\Sigma_1$ are $1/n$ with multiplicity $n-1$ and $\frac1{n}+a_n$ with multiplicity $1$. In particular, we have
$$\textrm{det}\left(\Sigma_1\right)=\frac{n a_n+1}{n^n}.$$
The inverse of $\Sigma_1$ is given by
$$\Sigma_1^{-1}=n\left(I_n-\frac{a_n}{n a_n+1}1_{n\times n}\right).$$
\end{lemma}
\begin{proof}
Denote by $d_n^\lambda:=\text{det}(\Sigma_1-\lambda I_n)=\text{det}((\frac1{n}-\lambda)I_n+\frac{a_n}{n}1_{n\times n})$. Subtracting the row $i$ from the row $i+1$, for each $1\leq i<n$, in the matrix $\Sigma_1-\lambda I_n$, we see that $d_n^\lambda$ is equal to the determinant of the matrix
\begin{equation}
 \begin{pmatrix}
       \frac1{n}-\lambda+\frac{a_n}{n}&\frac{a_n}{n}&\frac{a_n}{n}&\cdots&\frac{a_n}{n}\\
      \lambda-\frac1{n}&\frac1{n}-\lambda&0&\ldots&0\\
       0&\ddots&\ddots&\ddots&0\\
       \vdots&\ddots&\lambda-\frac1{n}&\frac1{n}-\lambda&0\\
       0&\cdots&0&\lambda-\frac1{n}&\frac1{n}-\lambda
\end{pmatrix}.
\end{equation}
Developing the determinant with respect to the last column we get
$$d_n^\lambda=\left(\frac1{n}-\lambda\right)d_{n-1}^{\lambda}+\frac{a_n}{n}\left(\frac1{n}-\lambda\right)^{n-1},\quad n>2.$$
Iterating this identity, we obtain
$$d_n^\lambda=\left(\frac1{n}-\lambda\right)^{n-2}d_2^{\lambda}+(n-2)\frac{a_n}{n}\left(\frac1{n}-\lambda\right)^{n-1}=\left(\frac1{n}-\lambda\right)^{n-1}\left(\frac1{n}-\lambda+a_n\right).$$
The first two statements follow. For the last statement, one can easily check that
$$\Sigma_1\times n\left(I_n-\frac{a_n}{n a_n+1}1_{n\times n}\right)=I_n.$$
This shows the desired result.
\end{proof}

\begin{lemma}\label{l1}
For all $n> 1$ and $H\in(\frac34,1)$, we have
$$\lim\limits_{\alpha\rightarrow\infty}H\left(Q^{\alpha,n}|Q_{\frac{\mu\alpha}{\sigma}}^n\right)=\infty.$$
\end{lemma}
\begin{proof}
Our starting point is Lemma~\ref{relent}. Evaluating each term entering \eqref{eg}, we first obtain
\begin{align}\label{term1.1}
 \text{tr}(\Sigma_1^{-1}\Sigma_0)&=n\,\text{tr}(\Sigma_0)-\frac{n a_n}{n a_n+1}\,\text{tr}(1_{n\times n}\Sigma_0)\nonumber\\
 &=\sum_{i=1}^n(1+n\alpha^2\lambda_i^n)-\frac{n a_n}{n a_n+1}(1+\alpha^2\text{tr}(1_{n\times n}C_n)).
\end{align}
Since $\text{tr}(1_{n\times n}C_n)=\sum_{1\leq i,j\leq n}C_n^{i,j}$, it follows that $\text{tr}(1_{n\times n}C_n)=1$ and replacing $a_n=\frac{1}{n}\frac{\mu^2\alpha^2}{\sigma^2}$, equation \eqref{term1.1} becomes
\begin{align}\label{term1.2}
 \text{tr}(\Sigma_1^{-1}\Sigma_0)&=n+\sum_{i=1}^nn\alpha^2\lambda_i^n-\frac{\mu^2\alpha^2}{\mu^2\alpha^2+\sigma^2}(1+\alpha^2).
\end{align}
For the third term in \eqref{eg}, using that $1_n^T1_{n\times n}1_n=n^2$, one can easily derive that
\begin{align}\label{term2}
 1_n^T\Sigma_1^{-1}1_n=\frac{n^2}{n a_n+1}=\frac{n^2\sigma^2}{\mu^2\alpha^2+\sigma^2}.
\end{align}
For the last term in \eqref{eg}, we use \eqref{trdet} and Lemma~\ref{linalg} to obtain
\begin{align}\label{term3}
 \ln\left(\frac{\textrm{det}\left(\Sigma_1\right)}{\textrm{det}\left(\Sigma_0\right)}\right)&=\ln(n a_n+1)-\sum_{i=1}^n\ln(1+n\alpha^2\lambda_i^n)\nonumber\\
 &=\ln\left(\frac{\mu^2\alpha^2+\sigma^2}{\sigma^2}\right)-\sum_{i=1}^n\ln(1+n\alpha^2\lambda_i^n).
\end{align}
Inserting \eqref{term1.2}, \eqref{term2} and \eqref{term3} in \eqref{eg} it follows
\begin{align}\label{2entrop}
 H\left(Q^{\alpha,n}|Q_{\frac{\mu\alpha}{\sigma}}^n\right)=\frac12\left[\sum\limits_{i=1}^n (n\alpha^2\lambda_i^n-\ln(1+n\alpha^2\lambda_i^n))-\frac{\mu^2\alpha^4}{\mu^2\alpha^2+\sigma^2}+\ln\left(\frac{\mu^2\alpha^2+\sigma^2}{\sigma^2}\right)\right].
\end{align}
Since the trace is similarity-invariant, we deduce that $$\sum_{i=1}^n\lambda_i^n=\textrm{tr}(C_n)=\sum_{i=1}^nC_n^{i,i}=\frac1{n^{2H-1}}.$$ In addition, we have  $\ln\left(\frac{\mu^2\alpha^2+\sigma^2}{\sigma^2}\right)\geq0$. Therefore, \eqref{2entrop} leads to
\begin{align}\label{lowb1}
 H\left(Q^{\alpha,n}|Q_{\frac{\mu\alpha}{\sigma}}^n\right)&\geq\frac{\alpha^2}2\left(n^{2-2H}-\frac{\mu^2\alpha^2}{\mu^2\alpha^2+\sigma^2}\right)-\frac{n}2 \ln(1+n\alpha^2\lambda_{\text{max}}^n)\nonumber\\
 &\geq\frac{\alpha^2}2(n^{2-2H}-1)-\frac{n}2 \ln(1+n\alpha^2\lambda_{\text{max}}^n)\nonumber\\
 &=\frac12\ln\left(\frac{e^{\theta_n\alpha^2}}{(1+n\alpha^2\lambda_{\text{max}}^n)^n}\right),
\end{align}
where $\theta_n:=n^{2-2H}-1>0$ and $\lambda_{\text{max}}^n=\max_{i=1\ldots n}\lambda_i^n$. The result follows taking the limit when $\alpha$ tends to infinity in the previous expression.
\end{proof}
\begin{remark}\label{rmk0}
 If $\mu=0$, using Lemma \ref{relent}, the previous result extends directly to the case $n=1$.
\end{remark}

\begin{remark}\label{rmk1}
The above proof also gives us the relation between the relative entropy of $Q^{\alpha,n}$ relative to $Q_{\frac{\mu\alpha}{\sigma}}^n$, i.e.~$H\left(Q^{\alpha,n}|Q_{\frac{\mu\alpha}{\sigma}}^n\right)$, and the relative entropy of $Q^{\alpha,n}$ relative to $Q_W^n$, i.e.~$H\left(Q^{\alpha,n}|Q_W^n\right)$. Indeed, using \cite[Lemma 5.3]{Ch1} one can deduce from \eqref{2entrop} that
\begin{equation}\label{rmkeq}
H\left(Q^{\alpha,n}|Q_{\frac{\mu\alpha}{\sigma}}^n\right)=H\left(Q^{\alpha,n}|Q_W^n\right)-\frac12\frac{\mu^2\alpha^4}{\mu^2\alpha^2+\sigma^2}+\frac12\ln\left(\frac{\mu^2\alpha^2+\sigma^2}{\sigma^2}\right).
\end{equation}
\end{remark}

\begin{remark}
We point out that we also have $$\lim\limits_{\alpha\rightarrow\infty}H\left(Q^{\alpha}|Q_{\frac{\mu\alpha}{\sigma}}\right)=\infty.$$
Indeed, we know from \cite[Lemma 5.3]{Ch1} that $\sup_n H\left(Q^{\alpha,n}|Q_W^n\right)<\infty$, which directly implies that also $\sup_n H\left(Q^{\alpha,n}|Q_{\frac{\mu\alpha}{\sigma}}^n\right)<\infty$. Therefore, applying \cite[Lemma 6.3]{Hihi} we obtain
$$H\left(Q^{\alpha}|Q_W\right)=\sup_nH\left(Q^{\alpha,n}|Q_W^n\right)\ \text{and}\ H\left(Q^{\alpha}|Q_{\frac{\mu\alpha}{\sigma}}\right)=\sup_nH\left(Q^{\alpha,n}|Q_{\frac{\mu\alpha}{\sigma}}^n\right).$$
The statement then follows from the result for the restrictions.
\end{remark}

For each $n\geq 1$, we denote the Radon-Nikodym derivative of $Q^{\alpha,n}$ relative to $Q_{\frac{\mu\alpha}{\sigma}}^n$  by $$L_\alpha^n:=\frac{dQ^{\alpha,n}}{dQ_{\frac{\mu\alpha}{\sigma}}^n}.$$
Using \cite[Lemma 6.1]{Hihi} (see Lemma~\ref{a1.2} in Appendix~\ref{A1}), we see that
$$H\left(Q^{\alpha,n}|Q_{\frac{\mu\alpha}{\sigma}}^n\right)=E_{Q^{\alpha,n}}[\ln(L_\alpha^n)]=E_{Q_{\frac{\mu\alpha}{\sigma}}^n}[L_\alpha^n\ln(L_\alpha^n)].$$

Moreover, let us recall the notion of $(Q^{\alpha,n})_{\alpha>0}$-tightness: $(L_\alpha^n)_{\alpha>0}$ is $(Q^{\alpha,n})_{\alpha>0}$-tight if the following holds:
$$\lim_{N\uparrow\infty}\limsup_{\alpha\to\infty}Q^{\alpha,n}(L_\alpha^n>N)=0.$$

\begin{lemma}\label{l2}
For each $n>1$, the family $(L_\alpha^n)_{\alpha>0}$ is not $(Q^{\alpha,n})_{\alpha>0}$-tight.
\end{lemma}
\begin{proof}
We know, by Lemma~\ref{l1}, that $E_{Q^{\alpha,n}}[\ln(L_\alpha^n)]=H\left(Q^{\alpha,n}|Q_{\frac{\mu\alpha}{\sigma}}^n\right)$ tends to infinity when $\alpha$ tends to $\infty$. Since the measures $Q^{\alpha,n}$ and $Q_{\frac{\mu\alpha}{\sigma}}^n$ are Gaussian, the result follows as a direct application of the remark on \cite[p. 457]{Eag81} which says that tightness is equivalent to the boundedness of the following two families: $E_{Q^{\alpha,n}}[\ln(L_\alpha^n)]$, $\alpha>0$, and $\text{Var}_{Q^{\alpha,n}}[\ln(L_\alpha^n)]$, $\alpha>0$.
\end{proof}

Before we can state and prove the last lemma of this section, we recall now the definition of contiguity of sequences of probability measures.
\begin{definition}\label{con}
A sequence of probability measures ${(P^n)}$ is contiguous with respect to the sequence of probability measures ${(Q^n)}$,
${(P^n)}{\triangleleft}{(Q^n)}$, if the following holds:
for any sequence $A^n\in\mathcal{F}^n$ with $Q^n(A^n)\to0$, for $n\to\infty$, we have that $P^n(A^n)\to0$, for $n\to\infty$. We say that ${(P^n)}$ and ${(Q^n)}$ are mutually contiguous if ${(P^n)}{\triangleleft}{(Q^n)}$ and ${(Q^n)}{\triangleleft}{(P^n)}$, in which case we write ${(P^n)}\triangleleft\triangleright {(Q^n)}$.
\end{definition}

\begin{lemma}\label{finiten}
For each $n>1$, we have $$(Q^{\alpha,n})_{\alpha>0}\vartriangle \left(Q_{\frac{\mu\alpha}{\sigma}}^n\right)_{\alpha>0}.$$
\end{lemma}
\begin{proof}
 Since, by Lemma~\ref{l2},  $(L_\alpha^n)_{\alpha>0}$ is not tight with respect to $Q^{\alpha,n}$ we apply \cite[Lemma V.1.6]{Jashi} and deduce that, for each $n> 1$, $(Q^{\alpha,n})_{\alpha}\ntriangleleft Q_{\frac{\mu\alpha}{\sigma}}^n$. The dichotomy for sequences of  Gaussian measures  of \cite[Corollary 4]{Eag81}
 says that two sequences of Gaussian measures on $\mathbb{R}^n$ are either mutually contiguous or entirely separable.
 So we conclude that, for each $n> 1$, $(Q^{\alpha,n})_{\alpha>0}\vartriangle (Q_{\frac{\mu\alpha}{\sigma}}^n)_{\alpha>0}$.
\end{proof}
\begin{remark}\label{final}
From Remark \ref{rmk0}, when $\mu=0$, the same argumentation leads to the conclusion that Lemma \eqref{finiten} holds true for $n=1$.
\end{remark}
\begin{proof}[Proof of Theorem~\ref{t1}]
From  Proposition \ref{pkk} (see also \cite[Proposition 4]{Kakra}), we know that there is a SAA if and only if $(Q^{\alpha})_{\alpha>0}\vartriangle (Q_{\frac{\mu\alpha}{\sigma}})_{\alpha>0}$.

Fix $n>1$. By Lemma \ref{l2}, for each $\alpha>0$, there exists $A_\alpha\in{{\mathcal F}}_n$ such that $$\lim_{\alpha\to\infty} Q^{\alpha}(A_\alpha)=\lim_{\alpha\to\infty}Q^{\alpha,n}(A_\alpha)=0$$ and $$\lim_{\alpha\to\infty} Q_{\frac{\mu\alpha}{\sigma}}(A_\alpha)=\lim_{\alpha\to\infty}Q_{\frac{\mu\alpha}{\sigma}}^n(A_\alpha)=1.$$ The result follows.
\end{proof}

\section{Interpretation of the results in the restricted markets}\label{S3}
 Lemma~\ref{finiten} might suggest that, for each $n>1$ (or following Remark \ref{final}, for each $n\geq1$ if $\mu=0$), there exists also some kind of asymptotic arbitrage in the large financial market consisting of the restrictions of the $\alpha$-mixed fractional Black-Scholes models, $\alpha>0$, to the grid $E_n:=\{0,\frac{1}{n},...,\frac{n-1}{n},1\}$. However, we will show that this is impossible. 
 
For simplicity, we only consider the case $n=1$ and $\mu=0$.
We also assume that $S^{H,\alpha}_0=1$. Thus, for each $\alpha>0$, the corresponding market is

\begin{equation} 
S^{H,\alpha}_t=\exp{\left(\sigma\left(Z^H_t+\frac1{\alpha} B_t\right)-\frac{\sigma^2}{2\alpha^2}t\right)},\quad t=0,1.\label{2step}
\end{equation}
In this case all possible strategies are constants and hence the value process $V^{\alpha}_1$ takes the following form
$$V^{\alpha}_1=c_{\alpha}(S^{H,\alpha}_1-1),$$
where $c_{\alpha}\in\mathbb{R}$. Obviously, we cannot hope for admissibility (boundedness from below), see the discussion in the introduction of \cite{Mi}. But even if we do not require any admissibility here, there is no way to choose a sequence of $\alpha_n\to\infty$ and corresponding value processes $V^{\alpha_n}$ such that the following holds: there exists $\beta>0$ and ${\varepsilon}_n\to0$ with
\begin{align}\label{nafl}
&(i)\quad P(V^{\alpha_n}_1>\beta)>\beta, \text{ for all $n$},\nonumber\\
&(ii) \lim_{n\to\infty}P(V^{\alpha_n}_1\geq -{\varepsilon}_n)=1.
\end{align}
This is not possible as  $Z^{H}_1$ as well as $B_1$ are independent $N(0,1)$ and hence are strictly positive as well as strictly negative with positive $P$-probability (and here neither letting $\alpha\to\infty$ nor multiplying $S^{H,\alpha}_1-1$ by some constants either positive or negative will be of any help: whenever there will be a strictly positive part in the limit there will also be a strictly negative part in the limit with a non-disappearing probability). Hence there is no such thing as (\ref{nafl}) which, in our discrete time $t=0,1$ situation, is the appropriate version
of an asymptotic arbitrage.
 
The reason behind this apparent contradiction is that in contrast to the continuous time large financial market its discrete counterpart is not complete.
Under the original measure $P$ (which induces $Q^{\alpha}$ on $\mathcal{C}[0,1]$) we have that $Z^H_1\sim N(0,1)$ and $B_1\sim N(0,1)$ and the two random variables are independent. We know that the Wiener measure $Q_W$ is a martingale measure for $S^{H,\alpha}$ (understood on $\mathcal{C}[0,1]$) for all $\alpha$, hence $Q_W|_{\mathcal{F}_1}$ is an equivalent martingale measure for (\ref{2step}). We will now construct a different martingale measure for the process (\ref{2step}) which is equivalent to $P$ on $(\Omega,\mathcal{F})$.

Indeed, define a measure $\tilde{P}$ on $(\Omega,\mathcal{F})$ as follows: $\frac{d\tilde{P}}{dP}=g(X)$ where we have $X:=\exp(\sigma Z^H_1)$ and $g(x)=e^{-x}\frac1{h(x)}$ where $h(x)=\frac1{\sqrt{2\pi}\sigma x} \exp(-\frac12(\frac{\ln(x)}{\sigma})^2)$ is the density of a lognormal distribution, i.e., the density  of the law of $X$  under $P$. Obviously this measure change has the purpose to make the distribution of $X$ exponential with parameter 1. Recall that the measure $Q^{\alpha,1}$ is considered as a measure on ${{\mathbb R}}$ (the measure induced by $M_1^{H,\alpha}:=\alpha Z_1^H+B_1$).

\begin{lemma}\label{newemm}
The measure $\tilde{P}$ satisfies:
\begin{enumerate}
\item $\tilde{P}\sim P$
\item $E_{\tilde{P}}[S^{H,\alpha}_1]=1=S^{H,\alpha}_0$, which means that $\tilde{P}$ is a martingale measure for (\ref{2step}), for each $\alpha$.
\item Let $\tilde{Q}^{\alpha,1}$ be the measure that is induced by $(M_1^{H,\alpha},\tilde{P})$ on ${{\mathbb R}}$, for each $\alpha>0$. Then
$(\tilde{Q}^{\alpha,1})_{\alpha>0}\triangleleft\triangleright (Q^{\alpha,1})_{\alpha>0}$.
\end{enumerate}
\end{lemma}

\begin{proof}
To prove (1) observe that $g(x)>0$ for all $0<x<\infty$ and $0<X<\infty$ $P$-a.s. and hence $\frac{d\tilde{P}^{\alpha}}{dP}=g(X)>0$ $P$-a.s. Moreover
$$E_P[g(X)]=\int_0^{\infty}g(x)h(x)dx=\int_0^{\infty}e^{-x}dx=1,$$
and so $\tilde{P}$ is a probability equivalent to $P$.
\newline\newline
For (2) we see that
\begin{align}
E_{\tilde{P}}[S^{H,\alpha}_1] &=E_P\left[g\left(e^{\sigma Z^{H}_1}\right)\exp{\left(\sigma\left(Z^H_1+\frac1{\alpha} B_1\right)-\frac{\sigma^2}{2\alpha^2}\right)}\right]\nonumber\\
&=E_P\left[g\left(e^{\sigma Z^{H}_1}\right)\exp{\left(\sigma Z^H_1\right)}\right]E_P\left[\exp{\left(\frac{\sigma}{\alpha} B_1-\frac{\sigma^2}{2\alpha^2}\right)}\right]\nonumber\\
&=E_P[g(X)X],\nonumber
\end{align}
where we used the independence of $Z^{H}_1$ and $B_1$ under $P$. Finally we have that
$$E_P[g(X)X]=\int_0^{\infty}g(x)xh(x)dx=\int_0^{\infty}xe^{-x}dx=1,$$
proving (2).\newline\newline
For (3) note that, for each $A\in\mathcal{B}({{\mathbb R}})$, we have $Q^{\alpha,1}(A)=P(M_1^{H,\alpha}\in A)$ and $\tilde{Q}^{\alpha,1}(A)=\tilde{P}(M^{H,\alpha}_1\in A)$. Thus, using that $\tilde{P}\sim P$, we infer, for a family of sets $A^{\alpha}$, that $Q^{\alpha,1}(A^{\alpha})=P(D^{\alpha})\to0$, for $\alpha\to\infty$, if and only if $\tilde{Q}^{\alpha,1}(A^{\alpha})=\tilde{P}(D^{\alpha})\to0$, where $D^{\alpha}=\{M_1^{H,\alpha}\in A^{\alpha}\}\in\mathcal{F}_1$. The result follows.
\end{proof}

In conclusion, Lemma~\ref{newemm} shows that there exists a family of equivalent martingale measures for the model (\ref{2step}) with good properties, in this case mutual contiguity. And this fact is reflected by the impossibility to find asymptotic arbitrage opportunities for the family of models (\ref{2step}), $\alpha>0$.

\appendix
\section{Relative entropy}\label{A1}
In this section, we recall the concept of relative entropy and some equivalent characterization. A more detailed presentation of the topic can be found in \cite{Hihi}.
\begin{definition}\label{a1.1}
 Let $Q_1$ and $Q_2$ be probability measures on a measurable space $(\Omega,{{\mathcal F}})$ and let $P=\{F_i: i=1,\ldots,n\}$ be a finite partition of $\Omega$, i.e. $\Omega=\cup_{i=1}^nF_i$ and $F_i$ are pairwise disjoint. The entropy of the measure $Q_1$ relative to $Q_2$ is the quantity
 $$H(Q_1|Q_2)=\sup_{{\mathcal P}}\sum_{j=1}^nQ_1(F_j)\ln\left(\frac{Q_1(F_j)}{Q_2(F_j)}\right),$$
 where ${{\mathcal P}}$ is the class of all possible finite partitions $P$ of $\Omega$. In the above formula, we assume  that $0\ln0=0$ and $\ln0=-\infty$.
\end{definition}

\begin{lemma}(\cite[Lemma 6.1]{Hihi})\label{a1.2}
 If a probability measure $Q_1$ is absolutely continuous w.r.t. another probability measure $Q_2$, then the relative entropy $H(Q_1|Q_2)$ is related to the Radon-Nikodym derivative ${\varphi}=\frac{dQ_1}{dQ_2}$ as follows:
 $$H(Q_1|Q_2)=E_{Q_1}[\ln({\varphi})]=E_{Q_2}[{\varphi}\ln({\varphi})].$$
 \end{lemma}

\bibliographystyle{acm}
\bibliography{reference2}
\end{document}


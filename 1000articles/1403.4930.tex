
\documentclass{amsart}
\usepackage{graphicx}
\usepackage{amsmath}
\usepackage{amssymb}\setcounter{MaxMatrixCols}{30}\usepackage{amsfonts}\usepackage{graphicx}
\usepackage[all]{xy}
\usepackage[mathscr]{euscript}

\newtheorem{theorem}{Theorem}[section]
\newtheorem{lemma}[theorem]{Lemma}

\theoremstyle{definition}
\newtheorem{definition}[theorem]{Definition}
\newtheorem{example}[theorem]{Example}
\newtheorem{proposition}[theorem]{Proposition}
\newtheorem{corollary}[theorem]{Corollary}
\newtheorem{xca}[theorem]{Exercise}
\newtheorem{problem}[theorem]{Problem}
\usepackage[all]{xy}
\usepackage{graphicx}
\theoremstyle{remark}
\newtheorem{remark}[theorem]{Remark}

\numberwithin{equation}{section}
\usepackage{graphicx}

\begin{document}
\title{Length Minimising Bounded Curvature Paths in Homotopy Classes}
\author{Jos\'{e} Ayala}
\address{FIA, Universidad Arturo Prat, Iquique, Chile}
\email{jayalhoff@gmail.com}
\subjclass[2000]{Primary 49Q10; Secondary 90C47, 51E99, 68R99}
\keywords{Regular homotopy, bounded curvature, Dubins paths}
\maketitle

\begin{abstract} Choose two points in the tangent bundle of the Euclidean plane $(x,X),(y,Y)\in T{ R}^2$. In this work we characterise the immersed length minimising paths with a prescribed bound on the curvature starting at $x$, tangent to $X$; finishing at $y$, tangent to $Y$, in each connected component of the space of paths with a prescribed bound on the curvature from $(x,X)$ to $(y,Y)$.
\end{abstract}

\section{Introduction}
Length minimising paths with a bound on the curvature and with fixed initial and final points and directions, widely known as Dubins paths, have proven to be extraordinarily useful in robotics since a bound on the curvature is a turning circle constraint for the trajectory of a robot along a path. Paths with a bound on the curvature admitting self intersections arise naturally in applications, for example, in the design of a system of interconnected tunnels navigated by vehicles. These paths are important as projections of embedded 3-dimensional paths, compare for example \cite{brazil 1}. Counterintuitively, paths with a bound on the curvature admitting self intersections may be length minimisers for certain choices of initial and final points in $T{R}^2$. This work correspond to the culmination of the study of minimal length elements in spaces of bounded curvature paths and evokes some of the machinery developed in \cite{papera}, \cite{paperd} and \cite{paperc}. In \cite{papera} we developed a method for finding length minimising paths for any given initial and final points and vectors. We start with an arbitrary bounded curvature path and divided up into pieces of suitable length called fragments. Then, for each fragment, we construct a piecewise constant curvature path (possibly arbitrarily close to the fragment) of length at most the length of the fragment. The process of interchanging\footnote{We avoid using homotopy since the interchanging does not involve a continuous one parameter family of paths.} a fragment by a piecewise constant curvature path is called {\it replacement} and after all the fragments of the given path are replaced what is left is a concatenation of piecewise constant curvature paths (possibly arbitrarily close to the original path) called {\it normalisation}. Subsequently, we developed a series of results involving larger pieces of the normalisation, and again, we interchange such pieces (or components) by piecewise constant curvature paths of less complexity observing that a path that can be perturbed while decreasing its length cannot be candidate to be length minimiser; we refer to this as {\it reduction process}. After a finite number of steps we obtain the result known as Dubins theorem, compare \cite{dubins 1}. A crucial observation is that the act of replacing a fragment by a piecewise constant curvature path does not involve a continuity argument, and therefore, without and explicit homotopy (preserving the bound on the curvature) between these paths nothing can be say in terms of homotopy classes. This rather technical issue is addresses in \cite{paperd} were we classified the homotopy classes of bounded curvature paths for any given initial and final points in $TR^2$.  In \cite{paperc} we proved that for certain initial and final points there exists a homotopy class whose elements are embedded and with the property that they cannot be deformed to paths with self intersections. 

Throughout this note we develop a machinery to justify following procedure. Given a bounded curvature path (possibly with loops) in a prescribed homotopy class, we continuously deform the path into a piecewise constant curvature path by applying the continuity argument developed in \cite{paperd}, so we make sure both paths lie in the same connected component. Then, we homotope the obtained $cs$ path to a shorter one while reducing the number of arcs of circle and/or line segments (complexity of the path) taking special care into components with loops. After applying this process a finite number of times, we achieve the desired characterisation for the minimum length elements in spaces of bounded curvature paths with prescribed {turning number}. Our main result, Theorem \ref{singudub}, gives in particular, the global length minimisers, generalising with this the Dubins theorem. The methodology used to prove our main result can be seen in Figure \ref{figsingmovccc}. Observe that a global length minimiser may not be unique and, in such a case, these paths are elements in different homotopy classes, see Figure \ref{figwinjust}. 

Many of the results exposed in this note have been heuristically verified with {\sc dubins explorer}, a mathematical software developed by the author and J. Diaz which computes the length minimisers and the number of connected components of the space of paths with a prescribed bound on the curvature from $(x,X)$ to $(y,Y)$. In this note, we also give an elementary proof for the following fact: The length of a closed loop whose absolute curvature is bounded by 1 is at least $2\pi$. We suggest the reader to read this work in conjunction with \cite{papera}, \cite{paperd} and \cite{paperc}. 

\section{preliminaries}
Denote by $T{ R}^2$ the tangent bundle of ${ R}^2$. Recall the elements in $T{ R}^2$ correspond to pairs $(x,X)$ sometimes denoted just by {\sc x}. Here the first coordinate corresponds to a point in ${ R}^2$ and the second to a tangent vector to ${ R}^2$ at $x$.

\begin{definition} \label{adm_pat} Given $(x,X),(y,Y) \in T{ R}^2$,  a path $\gamma: [0,s]\rightarrow { R}^2$ connecting these points is a {\it bounded curvature path} if:
\end{definition}
 \begin{itemize}
\item $\gamma$ is $C^1$ and piecewise $C^2$.
\item $\gamma$ is parametrised by arc length (i.e $||\gamma'(t)||=1$ for all $t\in [0,s]$).
\item $\gamma(0)=x$,  $\gamma'(0)=X$;  $\gamma(s)=y$,  $\gamma'(s)=Y.$
\item $||\gamma''(t)||\leq \kappa$, for all $t\in [0,s]$ when defined, $\kappa>0$ a constant.
\end{itemize}
Of course, $s$ is the arc-length of $\gamma$.

The first condition means that a bounded curvature path has continuous first derivative and piecewise continuous second derivative. For the third condition makes sense, without loss of generality, we extend the domain of $\gamma$ to $(-\epsilon,s+\epsilon)$ for $\epsilon>0$. The third item is called endpoint condition. The fourth item means that  bounded curvature paths have absolute curvature bounded above by a positive constant which can be choose to be $\kappa=1$. In addition, note that the first item is a completeness condition since the length minimising paths satisfying simultaneously the last three items in Definition \ref{adm_pat} are is $C^1$ and piecewise $C^2$, compare \cite{papera} or \cite{dubins 1}. Generically, the interval $[0,s]$ is denoted by $I$. Recall that path a $\gamma: I \rightarrow {R}^2$ has a self intersection if there exists $t_1,t_2 \in I$, with $t_1\neq t_2$ such that $\gamma(t_1)=\gamma(t_2)$.

\begin{definition} A path $cs$ path corresponds to a finite number of concatenations of line segments and arcs of circle. The number of line segments plus the number of circular arcs is called the {\it complexity} of the path.
\end{definition}

 \begin{definition} \label{admsp} Given $\mbox{\sc x,y}\in T{ R}^2$. The space of bounded curvature paths satisfying the given endpoint condition is denoted by $\Gamma(\mbox{\sc x,y})$. \end{definition}

  When a path is continuously deformed under parameter $p$, we reparametrise each of the deformed paths by its arc-length. Thus $\gamma: [0,s_p]\rightarrow { R}^2$ describes a deformed path at parameter $p$, with $s_p$ corresponding to its arc-length. This idea will be applied in the next definition.

\begin{definition}  \label{hom_adm} Given $\gamma,\eta \in \Gamma(\mbox{\sc x,y})$. A {\it bounded curvature homotopy}  between $\gamma: [0,s_0] \rightarrow  R^2$ and $\eta: [0,s_1] \rightarrow  R^2$ corresponds to a continuous one-parameter family of immersed paths $ {\mathcal H}_t: [0,1] \rightarrow \Gamma(\mbox{\sc x,y})$ such that:
\begin{itemize}
\item ${\mathcal H}_t(p): [0,s_p] \rightarrow  R^2$ for $t\in [0,s_p]$ is an element of $\Gamma(\mbox{\sc x,y})$ for all $p\in [0,1]$.
\item $ {\mathcal H}_t(0)=\gamma(t)$ for $t\in [0,s_0]$ and ${\mathcal H}_t(1)=\eta(t)$ for $t\in [0,s_1]$.
\end{itemize}
We say that $\gamma$ and $\eta$ are bounded-homtopic paths.
\end{definition}

The following settings were introduced in \cite{paperc}. The idea is to correlate the endpoint condition with the number of connected components the space $\Gamma(\mbox{\sc x,y})$ has. 

Let $\mbox{\sc C}_ l(\mbox{\sc x})$ be the unit circle tangent to $x$ and to the left of $X$. Analogous interpretations apply for $\mbox{\sc C}_ r(\mbox{\sc x})$, $\mbox{\sc C}_ l(\mbox{\sc y})$ and $\mbox{\sc C}_ r(\mbox{\sc y})$.  Denote their centres with lower-case letters, so the centre of $\mbox{\sc C}_ l(\mbox{\sc x})$ is denoted by $c_l(\mbox{\sc x})$. The following inequalities give qualitative information about the topology and geometry of $\Gamma(\mbox{\sc x,y})$.
\begin{equation} d(c_l(\mbox{\sc x}),c_l(\mbox{\sc y}))\geq 4 \quad \mbox{and}\quad d(c_r(\mbox{\sc x}),c_r(\mbox{\sc y}))\geq4 \label{con_a}\tag{i}\end{equation}
 \begin{equation} d(c_l(\mbox{\sc x}),c_l(\mbox{\sc y}))< 4 \quad \mbox{and}\quad d(c_r(\mbox{\sc x}),c_r(\mbox{\sc y}))\geq 4 \label{con_b}\tag{ii} \end{equation}
  \begin{equation} d(c_l(\mbox{\sc x}),c_l(\mbox{\sc y}))\geq4 \quad \mbox{and}\quad d(c_r(\mbox{\sc x}),c_r(\mbox{\sc y}))< 4  \label{con_b'}\tag{iii} \end{equation}
   \begin{equation} d(c_l(\mbox{\sc x}),c_l(\mbox{\sc y}))< 4 \quad \mbox{and}\quad d(c_r(\mbox{\sc x}),c_r(\mbox{\sc y}))< 4 \label{con_c}\tag{iv} \end{equation}

If the endpoint condition satisfies (i) we say that $\Gamma({\mbox{\sc x,y}})$ satisfies proximity condition {\sc A}. If the endpoint condition satisfies (ii) or (iii) we say that $\Gamma({\mbox{\sc x,y}})$ satisfies proximity condition {\sc B}. If the endpoint condition satisfies (iv) and the elements in $\Gamma({\mbox{\sc x,y}})$ are bounded-homotopic to paths of arbitrary length we say that $\Gamma({\mbox{\sc x,y}})$ satisfies proximity condition {\sc C}. A crucial result in \cite{paperc} states that certain endpoint conditions have associated a homotopy classes of embedded bounded curvature paths, such paths are trapped in a planar compact region (see Figure \ref{figfunlem}). Whenever the endpoint condition satisfies (iv) we say that $\Gamma({\mbox{\sc x,y}})$ satisfies proximity condition {\sc D} if $\Gamma({\mbox{\sc x,y}})$ has a homotopy class of embedded paths. 

\section{Normalisation of paths and the homotopy argument}
 
 Next we describe the overall strategy we follow to prove Theorem \ref{singudub}, our main result. The concepts in this section can be found in detail in \cite{papera} and \cite{paperd}. We now introduce a notion called {\it normalisation}. The idea is to divide up any given bounded curvature path into sufficiently small pieces called fragments. Let ${\mathcal L}(\gamma,a,b)$ be the length of $\gamma:I \rightarrow {R}^2$ restricted to $[a,b]\subset I$. We write ${\mathcal L}(\gamma,0,s)={\mathcal L}(\gamma)$. A {\it fragmentation} of $\gamma$ corresponds to a finite sequence $0=t_0<t_1\ldots <t_m=s$ such that, ${\mathcal L}(\gamma,t_{i-1},t_i)<  1$ with $\sum_{i=1}^m {\mathcal L}(\gamma,t_{i-1},t_i) =s$. A {\it fragment}, is the restriction of $\gamma$ to the interval determined by two consecutive elements in the fragmentation. We start with an arbitrary bounded curvature path and consider a fragmentation. The critical argument here is that {\it a fragment is bounded homotopic to a {\sc csc} path} called the replacement path (see Proposition 2.6. in  \cite{paperd}).  A {\sc csc} path corresponds to a concatenation of an arc of a unit radius circle, followed by a line segment, followed by an arc of a unit radius circle (see dashed trace in Figure \ref{figfunlem}). The other crucial observation is that {\it the length of the fragment is at most the length of the replacement path} (see Lemma 2.12. in \cite{papera}).  By applying these two results we have constructed a {\it bounded curvature homotopy} between a fragment and a {\sc csc} path with the additional property that the length of the {\sc csc} path is at most the length of the fragment.

  
  
 \begin{center}
\includegraphics[width=.8\textwidth,angle=0]{figfunlemb}
\end{center}
\caption{Embedded bounded curvature paths cannot be deformed outside the grey regions. The dashed path corresponds to the {\sc csc} path bounded-homotopic to the fragment $\gamma$. Note that $\gamma$ in between {\sc x} and $L_1$
 is longer than the dashed path in between {\sc x} and $L_1$ (see Lemma 2.12 in \cite{papera}).}
 \label{figfunlem}
\end{figure}}
  \begin{theorem} (normalisation)\label{homotarg} A bounded curvature path $\gamma$ is bounded-homotopic to a cs path of length at most the length of $\gamma$.
\end{theorem}

\begin{proof} Consider a fragmentation for $\gamma$. By applying Theorem 2.7. in \cite{paperd} and Lemma 2.12. in \cite{papera} to each of the fragments the result follows.
\end{proof}
 
The next step is to define three types of generic components obtained by concatenating {\sc csc} paths. We denote by a component of type ${\mathscr C}_1$ a path of type {\sc cscsc} shown in Figure \ref{figrep1} left and center. A component of type ${\mathscr C}_2$ is a path of type {\sc csccsc} shown in Figure \ref{figrep1} right. A component of type ${\mathscr C}_3$ contains a loop and will be defined later in this work. A component is called admissible if it satisfies proximity condition {\sc A}, {\sc B} or {\sc D}. A non-admissible component satisfy proximity condition {\sc C}. A component is said to be degenerate if one of their sub arcs or line segment has length zero. The {\it reduction process} consist on substituing the components of type ${\mathscr C}_1$, ${\mathscr C}_2$ or ${\mathscr C}_3$ by $cs$ paths with less complexity without increasing the length at any stage of the deformation.
  

{ \begin{figure} [[htbp]
 \begin{center}
\includegraphics[width=1\textwidth,angle=0]{figrep01}
\end{center}
\caption{Examples of components of type ${\mathscr C}_1$ and ${\mathscr C}_2$. The dashed trace at the left and right figures represent {\sc csc} paths. The middle illustration correspond to a non-admissible component of type ${\mathscr C}_1$.}
 \label{figrep1}
\end{figure}}

Theorem \ref{singudub}, in particular, generalises the following well known result. 

\begin{theorem} \label{embdub} (Dubins) (Theorem 3.9. in  \cite{papera}) Choose $\mbox{\sc x,y} \in T{ R}^2$. The minimal length bounded curvature path in $\Gamma(\mbox{\sc x,y})$ is either a {\sc ccc} path having its middle component of length greater than $\pi $ or a {\sc csc} path where some of the circular arcs or line segments can have zero length.
\end{theorem}

 \section{Reducing paths with loops} \label{windingnumberofpaths}

It is not hard to see that the two bounded curvature paths in Figure \ref{figwinjust} cannot be made bounded-homotopic one to another. In order to capture such a situation, we develop the following machinery. Consider the exponential map $\exp: {R} \rightarrow {S}^1$.

{ \begin{figure} [[htbp]
 \begin{center}
\includegraphics[width=0.55\textwidth,angle=0]{figwinjust}
\end{center}
\caption{The paths $\gamma_1$ and $\gamma_2$ are both length minimisers with different turning numbers.}
\label{figwinjust}
\end{figure}}

\begin{definition} The {\it turning map} $\tau$ is defined in the following diagram,
\[ \xymatrix{ I  \ar[d]_{\tau}   \ar@{>}[dr]^{w} &  \\
                     { R} \ar[r]_{\exp}  & { S}^1} \]
The map $w:I \rightarrow { S}^1$ is called the {\it direction map} and gives the derivative $\gamma'(t)$ of the path $\gamma$ at $t\in I$. The turning map $\tau:I\rightarrow {R}$ gives the turning angle $\gamma'(t)$ makes with respect to the $x$-axis.
\end{definition}

\begin{remark} The turning maps of two elements in $\Gamma(\mbox{\sc x,y})$ must differ by an integer multiple called the {\it turning number} denoted by $\tau(\gamma)$. In addition, note there is a one to one correspondence between homotopy classes and these multiples.
\end{remark}
  \begin{definition} Given $\mbox{\sc x,y}\in T{ R}^2$. The space of bounded curvature paths satisfying the given endpoint condition and having turning number $n$ is denoted by,
$${\Gamma}(n)=\{\gamma \in {\Gamma}(\mbox{\sc x,y}) \,|\,\,\, \tau(\gamma)=n,\,n\in {Z}\}.$$
\end{definition}

From the Graustein-Whitney theorem in \cite{whitney} we have the following consequence.

\begin{corollary} \label{wcpm} For fixed $\mbox{\sc x,y}\in T{R}^2$ we have ${\Gamma}(m)\cap {\Gamma}(n)=\emptyset$ for $m\neq n$.
\end{corollary}

A theorem of Whitney, see \cite{whitney}, establishes that any curve can be perturbed to have a finite number of transversal self intersections. The number of transversal self intersections of $\gamma$ can be chosen to be minimal and may be denoted by $\chi$. 
 A natural question arises: Can this perturbation be done with bounded curvature homotopies? Recall in \cite{paperd} we developed a continuity argument preserving the bounded curvature property for homotopies between general bounded curvature paths. 
 
 
 In order to describe a meaningful reduction process for bounded curvature paths in homotopy classes we need to develop a method for lowering the complexity of {\it components with loops} while the length of such components is never increasing through the process. 

\begin{proposition}\label{noclor} There are no closed bounded curvature paths lying in the interior of a unit radius disk.
\end{proposition}
\begin{proof} Let  $\gamma$ be a closed bounded curvature path lying in the interior of a unit radius disk. By considering $\gamma(0)$ and $\gamma(s)$ as in Corollary 2.4 in \cite{papere}  we conclude that $\gamma$ has a pair of points which are distant at least $2$. So, there exists $t_1,t_2\in I$ such that $d(\gamma(t_1),\gamma(t_2))\geq 2$, implying that the path cannot be completely contained in the interior of the disk.
\end{proof}

\begin{theorem} \label{loopbound} The shortest closed bounded curvature path is the boundary of a unit disk.
\end{theorem}

\begin{proof}  Let $\gamma$ be a closed bounded curvature path. By Proposition \ref{noclor} there are no closed bounded curvature paths lying in the interior of a unit disk. By the Pestov-Ionin Lemma see \cite{pestov}, {\it a closed bounded curvature path contains a unit disk in its interior component}. Denote by $C$ the boundary of such a disk and consider a coordinate system with origin the center $o$ of $C$. By applying Lemma 2.5 in \cite{papera} to $\gamma$ with respect to $o$ we conclude that the length of any closed bounded curvature path must be at least $2 \pi$ concluding the proof.\end{proof}

\begin{remark}
 Let $\gamma$ be a bounded curvature path with self intersections. Observe that there are four possible ways for $\gamma$ to intersect itself (transversally) for the first time (see Figure \ref{figfirstkink}).
 \end{remark}

{ \begin{figure} [[htbp]
 \begin{center}
\includegraphics[width=0.8\textwidth,angle=0]{figfirstkink}
\end{center}
\caption{The four possible types of first self intersection.}
 \label{figfirstkink}
\end{figure}}

\begin{definition} Suppose that $\gamma$ intersects itself for first time at $\gamma(t_1)=\gamma(t_2)$ for $t_1<t_2$. The restriction of $\gamma$ to the interval $[t_1,t_2]$ is a loop. Given $\delta>0$, the restriction of $\gamma$ to the interval $[t_1-\delta,t_2+\delta]$ is a kink.
\end{definition}

\begin{corollary}\label{looplength} The length of a loop is at least $2 \pi $.
\end{corollary}
 \begin{proof}  Let $\gamma$ be a loop. By a slightly more general version of Pestov-Ionin Lemma see \cite {hee}, {\it a loop contains a unit disk in its interior component}. Denote by $C$ the boundary of such a disk and consider a coordinate system with origin the center $o$ of $C$. By applying Lemma 2.5 in \cite{papera} to $\gamma$ with respect to $o$ we conclude that the length of the loop must be at least $2 \pi$.
\end{proof}

\begin{definition} \label{c3comp}A component of type ${\mathscr C}_3$ corresponds to a {\sc cscsc} path with a loop (see Figure \ref{figrep3scsnomin}).  A component of type ${\mathscr C}_3$ is called admissible if it satisfies proximity condition {\rm A, B} or {\rm D}. A component of type ${\mathscr C}_3$ is called non-admissible if it satisfies proximity condition {\rm C}. A component as the one at the centre in Figure \ref{figrep3scsnomin} is called degenerate.
\end{definition}

{\begin{figure} [[htbp]
\begin{center}
\includegraphics[width=1\textwidth,angle=0]{figrep3scsnomin}
\end{center}
\caption{Left: A path with a loop. Centre: An admissible degenerate component of type ${\mathscr C}_3$. Right: A  non-admissible component of type ${\mathscr C}_3$.}
\label{figrep3scsnomin}
\end{figure}}

\begin{proposition} \label{kink} A kink satisfying proximity condition {\sc A}, {\sc B} or {\sc D} can be homotoped to a component of type ${\mathscr C}_3$ of length at most the length of the kink.
\end{proposition}

\begin{proof} Let $\gamma$ be a normalisation of a bounded curvature path with self intersections. Let $\gamma(t_1)=\gamma(t_2)$ for $t_1<t_2$ be the first self intersection of $\gamma$ and suppose that for some $\delta>0$ the associated kink satisfies proximity condition {\sc A}, {\sc B} or {\sc D}. By applying a similar argument as in Proposition 3.3 in \cite{papera} to the interval $[t_1-\delta,t_2+\delta]$, we have that,
$${\mathcal L(\gamma, t_1-\delta, t_1)}+{\mathcal L(\gamma, t_2, t_2+\delta)}>{\mathcal L(\beta)}$$
where $\beta$ is the {\sc csc} replacement path constructed between $\gamma(t_1-\delta)$ and $\gamma(t_2+\delta)$ as in Proposition 2.10 in \cite{papera}. On the other hand, by Corollary \ref{looplength} we have that,
$${\mathcal L}(\gamma, t_1, t_2)\geq 2 \pi $$
 Therefore we have that,
 $${\mathcal L}(\gamma, t_1-\delta, t_2+\delta)\geq{\mathcal L(\beta)}+2 \pi $$

Here ${\mathcal L(\beta)}+2 \pi $ corresponds to the length of a ${\mathscr C}_3$ component. Since the normalisation of $\gamma$ and the ${\mathscr C}_3$ component are paths of piecewise constant curvature, it is easy to see that these paths are bounded-homotopic since both paths satisfy proximity condition {\sc A}, {\sc B} or {\sc D}. 
 \end{proof}

\section {Length Minimising Bounded Curvature Paths in $\Gamma(n)$}

\begin{proposition} \label{c3nomin}A non-degenerate component of type ${\mathscr C}_3$ is not a path of minimal length.
\end{proposition}

\begin{proof} Consider a non-degenerate component of type ${\mathscr C}_3$. Let $G$ be the self intersection in the component and, consider the points $P$ and $Q$ as shown at the right in Figure \ref{figrep3scsnomin}. By applying Theorem 3.4 in \cite{papera} to the non-degenerate component of type ${\mathscr C}_3$ with respect to the points $P$ and $Q$ (see Figure 10 in \cite{papera}) we constructed a shorter length non-degenerate component of type ${\mathscr C}_3$, concluding the proof.
\end{proof}

\begin{definition} We denote by $\mbox{\sc c}^\chi $ a circle traversed $2 \chi  \pi $ times.
\end{definition}

 Some $cs$ paths can be presented in several ways. For example, a $\mbox{\sc r}^{\chi } \mbox{\sc s}\mbox{\sc r}$ path can also be presented as $\mbox{\sc r} \mbox{\sc s}\mbox{\sc r}^{\chi }\mbox{\sc s}\mbox{\sc r}$ having both paths the same length. In addition, a $cs$ path is called {\it symmetric} if their arcs of circle have the same orientation, otherwise the path is called {\it skew}. In the next result we present the paths having minimal complexity. In addition, the circle traversed $2 \chi  \pi $ times is placed at the beginning (if possible). 

\begin{theorem} \label{singudub} Given $\mbox{\sc x,y} \in T{R}^2$ and $n\in { Z}$. Then the minimal length bounded curvature path in $\Gamma(n)$ for $n\in {Z}$ must be of the form:

\begin{itemize}
\item {\sc csc} or {\sc ccc}.
\item Symmetric $\mbox{\sc c}^\chi \mbox{\sc s}\mbox{\sc c}$ or $\mbox{\sc c}^{\chi}\mbox{\sc c} \mbox{\sc s}\mbox{\sc c}$.
\item Skew $\mbox{\sc c}^{\chi}\mbox{\sc s}\mbox{\sc c}$ or  $\mbox{\sc c}\mbox{\sc s}\mbox{\sc c}^{\chi}$.
\item $\mbox{\sc c}^\chi \mbox{\sc c}\mbox{\sc c}$ or $\mbox{\sc c} \mbox{\sc c}^{\chi}\mbox{\sc c}$.
 \end{itemize}
Here $\chi$ is the minimal number of crossings for paths in $\Gamma(n)$. In addition, some of the circular arcs or line segments may have zero length. In particular, we have Theorem \ref{embdub} in the homotopy class containing the minimum length element in $\Gamma(\mbox{\sc x},\mbox{\sc y})$ (See Figure \ref{figsingmov}).
\end{theorem}

\begin{proof} Consider a path $\gamma \in \Gamma(n)$ and a fragmentation for $\gamma$. By applying Theorem \ref{homotarg} we obtain a normalisation of $\gamma$ of length at most the length of $\gamma$. Consider the first self intersection of the normalisation of $\gamma$. By recursively applying Proposition \ref{kink} we replace the kinks by degenerates components of type ${\mathscr C}_3$ (see Figure \ref{figrep3scsnomin} centre). Since translations are isometries, we slide the middle circle in the component of type ${\mathscr C}_3$ along the $cs$ path and place it to be tangent to $X$ (or $Y$). By continuing with this procedure until all the loops are tangent to $X$, and after cancelling out oppositely oriented loops, we end up with a component of type $\mbox{\sc c}^{\chi}$ concatenated with a $cs$ path without loops. Then by applying Theorem \ref{embdub} to the $cs$ path, we obtain the minimal length path as one of the six paths {\sc csc} or {\sc ccc} (concatenated with the $\mbox{\sc c}^{\chi}$ component). Figure \ref{figsingmov} illustrates a {\sc csc} path concatenated with a degenerated $\mbox{\sc c}^{\chi}$ component. Figure \ref{figsingmovccc} illustrates a {\sc ccc} path concatenated with a non-degenerate $\mbox{\sc c}^{\chi}$ component. There is one case left. After reducing the fragmentation we may end up with a non-admissible component of type ${\mathscr C}_3$ which by Proposition \ref{c3nomin} it is not a path of minimal length. Depending on the boundary condition we may homotope such a component to a {\sc csc}, {\sc ccc} path (see Figure \ref{figrepclosesingtwo}) or to a path of higher complexity. It is not hard to see that the higher complexity path is not a path of minimal length (compare Theorem 3.4 and Corollary 3.5 in \cite{papera}). Then we homotope the path to a {\sc csc} concatenated with a $\mbox{\sc c}^{\chi}$ component (possibly degenerated) or to a $\mbox{\sc c}^\chi \mbox{\sc c}\mbox{\sc c}$ or $\mbox{\sc c} \mbox{\sc c}^{\chi}\mbox{\sc c}$ (possibly degenerated) see Figure \ref{figrepclosesingtwo}.
 \end{proof}

{ \begin{figure} [[htbp]
 \begin{center}
\includegraphics[width=1.0\textwidth,angle=0]{figsingmov}
\end{center}
\caption{Via Theorem \ref{singudub} we obtain the minimal length element in $\Gamma(1)$, corresponding in this case, to the global length minimiser in $\Gamma(\mbox{\sc x,y})$. Other length minimisers can be seen in Figures \ref{figsingdubpath1}, \ref{figsingdubpath2} and \ref{figsingdubpath3}.}
\label{figsingmov}
\end{figure}}

{ \begin{figure} [[htbp]
 \begin{center}
\includegraphics[width=1.5\textwidth,angle=90]{figsingmovccc}
\end{center}
\caption{An illustration of Theorem \ref{singudub} applied to a path in $\Gamma(-1)$ to obtain the length minimiser in its homotopy class.}
\label{figsingmovccc}
\end{figure}}

{ \begin{figure} [[htbp]
 \begin{center}
\includegraphics[width=1\textwidth,angle=0]{figrepclosesingtwo}
\end{center}
\caption{Top: The minimal length element in the homotopy class of $\gamma$ is a $\mbox{\sc c}\mbox{\sc s}\mbox{\sc c}$. Bottom: The minimal length element in the homotopy class of $\gamma$ is a $\mbox{\sc c} \mbox{\sc c}\mbox{\sc c}$ with ${\chi}=1$.}
\label{figrepclosesingtwo}
\end{figure}}

{ \begin{figure} [[htbp]
 \begin{center}
\includegraphics[width=1\textwidth,angle=0]{figsingdubpath1}
\end{center}
\caption{Examples of minimal length paths bounded-homotopic to a $\mbox{\sc r}^\chi \mbox{\sc s}\mbox{\sc r}$ path.}
\label{figsingdubpath1}
\end{figure}}

{ \begin{figure} [[htbp]
 \begin{center}
\includegraphics[width=1\textwidth,angle=0]{figsingdubpath2}
\end{center}
\caption{Examples of minimal length paths bounded-homotopic to a $\mbox{\sc l}^{\chi}\mbox{\sc r}\mbox{\sc s}\mbox{\sc r}$ path.}
\label{figsingdubpath2}
\end{figure}}

{ \begin{figure} [[htbp]
 \begin{center}
\includegraphics[width=1\textwidth,angle=0]{figsingdubpath3}
\end{center}
\caption{Examples of minimal length paths bounded-homotopic to a $\mbox{\sc l}^\chi \mbox{\sc r}\mbox{\sc l}$ path.}
\label{figsingdubpath3}
\end{figure}}

\bibliographystyle{amsplain}
   \begin{thebibliography}{11} \rm

   \bibitem{hee} H. Ahn, O. Cheon, J. Matousek, A. Vigneron, Reachability by Paths of Bounded Curvature in a Convex Polygon, {Computational Geometry: Theory and Applications, v.45, no.1-2, 2012 Jan-Feb, p.21(12).}
 
    \bibitem{papera} J. Ayala and J.H. Rubinstein, A Geometric Approach to Shortest Bounded Curvature Paths (2014) arXiv:1403.4899v1 [math.MG].

   \bibitem{thesisayala}J. Ayala, Classification of the Homotopy Classes and Minimal Length Elements in Spaces of Bounded Curvature Paths, PhD Thesis, University of Melbourne 2014.
   
       \bibitem{paperd} J. Ayala and J.H. Rubinstein, The Classification of Homotopy Classes of Bounded Curvature Paths (2014) arXiv:1403.5314v1 [math.MG]. {To appear in the Israel Journal of Mathematics.}
   
       \bibitem{paperc}J. Ayala and J.H. Rubinstein, Non-uniqueness of the Homotopy Class of Bounded Curvature Paths (2014) arXiv:1403.4911 [math.MG].
   
 
   
     \bibitem{papere} J. Ayala, On the topology of the spaces of plane curves (2014) arXiv:1404.4378v1 [math.GT]
 

  
\bibitem{dubinsexplorer}J. Ayala, J. Diaz, Dubins Explorer: Software for bounded curvature paths.  

  

 \bibitem {brazil 1}  M. Brazil, P.A. Grossman, D.A. Thomas, J.H. Rubinstein, D. Lee, N.C. Wormald, Constrained path optimisation for underground mine layout, {The 2007 International Conference of Applied and Engineering Mathematics (ICAEMÕ07), London, (2007),856-861}
\bibitem {dubins 1} L.E. Dubins,  On curves of minimal length with constraint on average curvature, and with prescribed initial and terminal positions and tangents, { American Journal of Mathematics 79 (1957), 139-155.}
\bibitem{dubins 2}  L.E. Dubins, On plane curve with curvature, {Pacific J. Math. Volume 11, Number 2 (1961), 471-481.}

 \bibitem{pestov} G. Pestov, V. Ionin, On the largest possible circle imbedded in a given closed curve, {Dok. Akad. Nauk SSSR, 127:1170-1172, 1959}. In Russian.

\bibitem{smale} S. Smale, Regular curves on Riemannian manifolds, Transactions of the American Mathematical Society, Vol. 87, No. 2. (Mar., 1958), pp. 492-512.

\bibitem{whitney}H. Whitney,  On regular closed curves in the plane, {Compositio Math. 4 (1937), 276-284.}
\end{thebibliography}
\end{document}


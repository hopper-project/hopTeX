\documentclass[10pt]{article}


\usepackage{amsmath,amsbsy,amssymb,latexsym,amsthm,dsfont}
\usepackage[top=2.3cm, bottom=2cm, left=2cm, right=2cm]{geometry}
\usepackage{graphicx}
\usepackage{cite}


\newtheorem{theorem}{Theorem}
\newtheorem{corollary}[theorem]{Corollary}
\newtheorem{lemma}[theorem]{Lemma}
\newtheorem{definition}[theorem]{Definition}
\newtheorem{remark}[theorem]{Remark}
\newtheorem{example}[theorem]{Example}


\newenvironment{keywords}{\begin{center}
\begin{minipage}[c]{13.4cm} {\bf Keywords:}} {\end{minipage}
\end{center}}

\newenvironment{msc}{\begin{center}
\begin{minipage}[c]{13.4cm} {\bf MSC 2010:}} {\end{minipage}
\end{center}}


\begin{document}

\title{Fractional variational problems\\
depending on indefinite integrals and with delay}

\author{Ricardo Almeida\footnote{ricardo.almeida@ua.pt, CIDMA -- Center for Research and Development in Mathematics and Applications,
Department of Mathematics, University of Aveiro, Portugal.}}

\date{}

\maketitle


\begin{abstract}
The aim of this paper is to exhibit a necessary and sufficient condition of optimality for functionals depending on fractional  integrals and derivatives, on indefinite integrals
and on presence of time delay. We exemplify with one example, where we find analytically the minimizer.
\end{abstract}

\begin{msc}
49K05, 49S05, 26A33, 34A08.
\end{msc}

\begin{keywords}
calculus of variations, fractional calculus, Caputo derivatives, time delay.
\end{keywords}


\section{Introduction}\label{sec:intro}
In this paper we proceed the work started in \cite{Almeida5}, where the authors studied fractional variational problems with the Lagrangian containing not only
 fractional integrals and fractional derivatives, but an indefinite integral as well. With this approach, we tried not only to obtain new results but also
 generalize some already known. The novelty of this paper is that we consider dependence on time delay in the cost functional. Since fractional derivatives are
 characterized by retaining memory, it is natural to state the system at an earlier time and many phenomena have time delays inherent in them. This is a field under
 strong research, namely for optimal control problems, differential equations, biology, etc (see e.g. \cite{Chen,Dehghan,Liu0,Liu,Mo,Udaltsov,Xu,Zhu}).
 For some literature on what this paper concerns, we suggest the reader to \cite{AGRA1,Almeida,Baleanu1,Bhrawy,Chen2,Gastao0,Loghmani,Malinowska,Mozyrska,Yueqiang} for fractional variational problems dealing with Caputo derivative, in \cite{Almeida1} for Lagrangians depending on fractional integrals, and in \cite{Gregory,Nat} when presence of indefinite integrals.
 For a standard variational approach to systems in presence of time delay or more general topics, we suggest the interested reader to the papers \cite{AGRA0,Rosenblueth1,Rosenblueth2,Wang}, and for the fractional
 approach to \cite{Baleanu,Jarad}.

 The paper is organized in the following way. For the reader's convenience, in section \ref{sec:frac} we recall some definitions and results on
 fractional calculus; namely the definitions of fractional integral and fractional derivative, and some fractional integration by parts formulas.
 Section \ref{sec:ELequation} is the main core of the paper: we exhibit a necessary and sufficient condition of optimality for the functional that we purpose to study in  this paper.

\section{Review on fractional calculus}\label{sec:frac}

Let  us  now  explain  the  notation  used. For more, see e.g. \cite{Kilbas,Miller,samko}.

Given a function $f:[a,b]\to\mathbb{R}$, $\alpha\in(0,1)$ and $\beta>0$, the left and right fractional integrals of order $\beta$ of $f$ are respectively
$${_aI_x^\beta}f(x)=\frac{1}{\Gamma(\beta)}\int_a^x (x-t)^{\beta-1}f(t)dt,$$
and
$${_xI_b^\beta}f(x)=\frac{1}{\Gamma(\beta)}\int_x^b(t-x)^{\beta-1} f(t)dt.$$
The left and right Riemann--Liouville fractional derivatives of order $\alpha$ of $f$ are respectively
$${_aD_x^\alpha}f(x)=\frac{1}{\Gamma(1-\alpha)}\frac{d}{dx}\int_a^x(x-t)^{-\alpha}f(t)dt$$
and
$${_xD_b^\alpha}f(x)=\frac{-1}{\Gamma(1-\alpha)}\frac{d}{dx}\int_x^b (t-x)^{-\alpha} f(t)dt.$$
The left and right Caputo fractional derivatives of order $\alpha$ of $f$ are respectively
$${_a^CD_x^\alpha}f(x)=\frac{1}{\Gamma(1-\alpha)}\int_a^x (x-t)^{-\alpha}\frac{d}{dt}f(t)dt$$
and
$${_x^CD_b^\alpha}f(x)=\frac{-1}{\Gamma(1-\alpha)}\int_x^b(t-x)^{-\alpha}\frac{d}{dt} f(t)dt.$$
It is obvious that these operators are linear, and in some sense fractional differentiation and fractional integration are inverse operations.

Caputo fractional derivative seems to be more natural than the Riemann-Liouville fractional derivative. There are two main reasons for that.
The first one is that the Caputo derivative of a constant is zero, while the  Riemann-Liouville  derivative of $f(x)=C$ is $C(x-a)^{-\alpha}/\Gamma(1-\alpha)$. The second one is that the Laplace transform of the Caputo derivative depends on the derivative of integer order of the function
$$(\mathcal{L}\,{_0^CD^\alpha_s}f)(s)=s^\alpha (\mathcal{L}\,f)(s)-\sum_{k=0}^{n-1}s^{\alpha-k-1}\frac{d^kf}{ds^k}(0),$$
in opposite to the Riemann-Liouville  derivative  that uses fractional integrals evaluated at the initial value.

A basic result needed to apply variational methods is the integration by parts formula, that in case for fractional integrals is
\begin{equation}\label{Int2}\displaystyle\int_{a}^{b}  g(x) \cdot {_aI_x^\beta}f(x)dx=\int_a^b f(x) \cdot {_x I_b^\beta} g(x)dx \, ,\end{equation}
and for Caputo fractional derivatives, we have
\begin{equation}\label{Int}\int_{a}^{b}g(x)\cdot {_a^C D_x^\alpha}f(x)dx=\int_a^b f(x)\cdot {_x D_b^\alpha} g(x)dx+\left[f(x)\cdot{_xI_b^{1-\alpha}}g(x)\right]_a^b.
\end{equation}


Formula \eqref{Int} can be generalized in a way to include the case where the lower bound of the integral is distinct of the lower bound of the Caputo derivative.

\begin{lemma}\label{LemmaInt} Let $f$ and $g$ be two functions of class $C^1$ on $[a,b]$, and let $r\in(a,b)$. Then
\begin{multline}\label{GenInt}\int_{r}^{b}g(x)\cdot {_a^C D_x^\alpha}f(x)dx=\int_r^b f(x)\cdot {_x D_b^\alpha} g(x)dx\\
-\int_a^r\frac{f(x)}{\Gamma(1-\alpha)}\, \frac{d}{dx}\left(\int_r^b (t-x)^{-\alpha}g(t)\,dt\right)dx-
\frac{f(a)}{\Gamma(1-\alpha)}\int_r^b(t-a)^{-\alpha}g(t)dt.\end{multline}
\end{lemma}
\begin{proof} It follows due the next relations:
$$\begin{array}{ll}
\displaystyle \int_{r}^{b}g(x)\cdot {_a^C D_x^\alpha}f(x)dx & =\displaystyle \int_{a}^{b}g(x)\cdot {_a^C D_x^\alpha}f(x)dx-\int_{a}^{r}g(x)\cdot {_a^C D_x^\alpha}f(x)dx\\
                    & = \displaystyle\int_{a}^{b}f(x)\cdot {_x D_b^\alpha}g(x)dx+ \left[f(x) \cdot{_xI_b^{1-\alpha}}g(x) \right]_a^b\\
                    &\quad - \displaystyle \int_{a}^{r}f(x)\cdot {_x D_r^\alpha}g(x)dx-\left[f(x)\cdot {_xI_r^{1-\alpha}}g(x) \right]_a^r\\
                     &=\displaystyle\int_{r}^{b}f(x)\cdot {_x D_b^\alpha}g(x)dx+\int_{a}^{r}f(x)\cdot \left({_x D_b^\alpha}g(x)-{_x D_r^\alpha}g(x)\right)dx\\
                     & \displaystyle\quad +\left[f(x)\cdot {_xI_b^{1-\alpha}}g(x) \right]_a^b-\left[f(x)\cdot {_xI_r^{1-\alpha}}g(x) \right]_a^r\\
                     &=\displaystyle\int_r^b f(x)\cdot {_x D_b^\alpha} g(x)dx\\
&\displaystyle-\int_a^r\frac{f(x)}{\Gamma(1-\alpha)}\, \frac{d}{dx}\left(\int_r^b (t-x)^{-\alpha}g(t)\,dt\right)dx-
\frac{f(a)}{\Gamma(1-\alpha)}\int_r^b(t-a)^{-\alpha}g(t)dt.
\end{array}$$
\end{proof}



\section{The Euler-Lagrange equation}\label{sec:ELequation}
\label{sec:ELequation}

The cost functional that we will study is given by the expression
\begin{equation}
\label{funct}
J(y)=\int_a^b L(x,y(x),{^C_aD_x^\alpha}y(x),{_aI_x^\beta}y(x),z(x), y(x-\tau), y'(x-\tau))dx,
\end{equation}
defined on $C^1[a-\tau,b]$, where
$$\left\{
\begin{array}{l}
\tau>0, \mbox{ and } \tau<b-a,\\
\alpha\in(0,1) \mbox{ and }\beta>0,\\
z(x)=\int_a^x l(t,y(t),{^C_aD_t^\alpha}y(t),{_aI_t^\beta}y(t))dt,\\
L=L(x,y,v,w,z,y_\tau,v_\tau) \mbox{ and } l=l(x,y,v,w) \mbox{ are of class } C^1
\end{array}\right.$$
and the admissible functions are such that
$$\left\{
\begin{array}{l}
{^C_aD_x^\alpha}y(x) \mbox{ and } {_aI_x^\beta}y(x) \mbox{ exist and are continuous on } [a,b],\\
y(b)=y_b\in \mathbb R,\\
y(x)=\phi(x), \mbox{ for all } x\in [a-\tau,a], \, \phi \mbox{ a fixed function.}
\end{array}\right.$$
The set of variation functions of $y$ that we will consider are those of type $y+\epsilon h$, such that $|\epsilon| \ll1$ and $h\in C^1[a-\tau,b]$ with
$$\left\{
\begin{array}{l}
h(b)=0,\\
h(x)=0, \mbox{ for all } x\in [a-\tau,a].
\end{array}\right.$$

An important result in variational calculus is the so called du Bois-Reymond Theorem:

\begin{theorem}\label{dubois} (see e.g. \cite{Brunt}) Let $f:[a,b]\to\mathbb R$ be a continuous functions, and suppose that the relation
$$\int_a^b f(x)h(x)dx=0$$
holds for every $h\in C^k[a,b]$, with $k\geq 0$. Then $f(x)=0$ on $[a,b]$.
\end{theorem}

Theorem \ref{dubois} still holds if we impose the auxiliary conditions $h(a)=h(b)=0$.

From now on, to simplify writing, by $[y](x)$ and $\{y\}(x)$ we denote the vectors
$$[y](x)=(x,y(x),{^C_aD_x^\alpha}y(x),{_aI_x^\beta}y(x),z(x), y(x-\tau), y'(x-\tau))\quad \mbox{and}\quad \{y\}(x)=(x,y(x),{^C_aD_x^\alpha}y(x),{_aI_x^\beta}y(x)).$$
Let $y$ be a minimizer or maximizer of $J$ as in \eqref{funct}. As it is known, at the extremizers of the functional we have
$$\frac{d}{d\epsilon}J(y+\epsilon h)=0,$$
where $y+\epsilon h$ is any variation of $y$. Proceeding with the necessary calculations, we deduce that
\begin{multline*}
\int_a^b \left[ \frac{\partial L}{\partial y}[y](x)h(x)
+ \frac{\partial L}{\partial v}[y](x){^C_aD^\alpha_x}h(x)
+ \frac{\partial L}{\partial w}[y](x){_aI^\beta_x}h(x)\right.\\
\left.+\frac{\partial L}{\partial z}[y](x)\int_a^x\left(
\frac{\partial l}{\partial y}\{y\}(t)h(t)
+\frac{\partial l}{\partial v}\{y\}(t){^C_aD^\alpha_t}h(t)
+\frac{\partial l}{\partial w}\{y\}(t){_aI^\beta_t}h(t)\right)dt\right.\\
\left. +\frac{\partial L}{\partial y_\tau}[y](x)h(x-\tau)
+ \frac{\partial L}{\partial v_\tau}[y](x)h'(x-\tau)\right]dx=0.
\end{multline*}
Next, we use the following relations

\textit{R1:}
$$\int_a^b  \frac{\partial L}{\partial y}[y](x)h(x)dx=
\int_a^{b-\tau} \frac{\partial L}{\partial y}[y](x)h(x)dx+\int_{b-\tau}^b \frac{\partial L}{\partial y}[y](x)h(x)dx;$$

\textit{R2:} Since $h(a)=0$, and using formulas \eqref{Int} and \eqref{GenInt}
\begin{align*}
 \int_a^b \frac{\partial L}{\partial v}[y](x){^C_aD^\alpha_x}h(x) dx&=\int_a^{b-\tau}
                   \frac{\partial L}{\partial v}[y](x){^C_aD^\alpha_x}h(x) dx
       +\int_{b-\tau}^b \frac{\partial L}{\partial v}[y](x){^C_aD^\alpha_x}h(x) dx\\
     &=\int_a^{b-\tau} {_x D_{b-\tau}^\alpha} \left(\frac{\partial L}{\partial v}[y](x) \right)h(x)dx
     +\left[{_x I_{b-\tau}^{1-\alpha}} \left(\frac{\partial L}{\partial v}[y](x) \right)h(x)\right]_a^{b-\tau}\\
     &\quad +\int_{b-\tau}^b {_x D_{b}^\alpha} \left(\frac{\partial L}{\partial v}[y](x) \right)h(x)dx
     -\frac{h(a)}{\Gamma(1-\alpha)}\int_{b-\tau}^b(t-a)^{-\alpha}\frac{\partial L}{\partial v}[y](t)dt  \\
     &\quad -\int_a^{b-\tau}\frac{h(x)}{\Gamma(1-\alpha)}\, \frac{d}{dx}
     \left( \int_{b-\tau}^b(t-x)^{-\alpha} \frac{\partial L}{\partial v}[y](t)dt\right) dx\\
     &=\int_a^{b-\tau} \left[{_x D_{b-\tau}^\alpha} \left(\frac{\partial L}{\partial v}[y](x)\right)-
     \frac{1}{\Gamma(1-\alpha)}\, \frac{d}{dx}\left( \int_{b-\tau}^b(t-x)^{-\alpha} \frac{\partial L}{\partial v}[y](t)dt\right)\right]h(x)dx\\
     &\quad + \int_{b-\tau}^b {_x D_{b}^\alpha} \left(\frac{\partial L}{\partial v}[y](x) \right)h(x)dx;\\
\end{align*}

\textit{R3:} Using equation \eqref{Int2} and Lemma 2(b) of \cite{Baleanu},
\begin{align*}\int_a^b \frac{\partial L}{\partial w}[y](x){_aI^\beta_x}h(x) dx&=
  \int_a^{b-\tau} \frac{\partial L}{\partial w}[y](x){_aI^\beta_x}h(x) dx+\int_{b-\tau}^b \frac{\partial L}{\partial w}[y](x){_aI^\beta_x}h(x) dx\\
&= \int_a^{b-\tau} {_xI^\beta_{b-\tau}}\left(\frac{\partial L}{\partial w}[y](x)\right)h(x) dx\\
  &\quad +\int_{b-\tau}^b {_xI^\beta_b}\left(\frac{\partial L}{\partial w}[y](x)\right)h(x) dx
    +\frac{1}{\Gamma(\beta)}\int_a^{b-\tau}h(x)\left( \int_{b-\tau}^b (t-x)^{\beta-1} \frac{\partial L}{\partial w}[y](t)dt \right)dx\\
  &= \int_a^{b-\tau} \left[ {_xI^\beta_{b-\tau}}\left(\frac{\partial L}{\partial w}[y](x)\right) +
     \frac{1}{\Gamma(\beta)}  \left( \int_{b-\tau}^b (t-x)^{\beta-1} \frac{\partial L}{\partial w}[y](t)dt \right) \right]h(x) dx\\
     &\quad +\int_{b-\tau}^b {_xI^\beta_b}\left(\frac{\partial L}{\partial w}[y](x)\right)h(x) dx;
\end{align*}


\textit{R4:} Using standard integration by parts,
\begin{align*}
\int_a^b & \frac{\partial L}{\partial z}[y](x)\left(\int_a^x\frac{\partial l}{\partial y}\{y\}(t)h(t) dt \right) dx
=\int_a^b \left( -\frac{d}{dx}\int_x^b\frac{\partial L}{\partial z}[y](t)dt \right)
\left( \int_a^x \frac{\partial l}{\partial y}\{y\}(t)h(t) dt \right) dx\\
&= \int_a^b \left(\int_x^b\frac{\partial L}{\partial z}[y](t)dt \right)
\frac{\partial l}{\partial y}\{y\}(x)h(x) \, dx\\
&= \int_a^{b-\tau} \left(\int_x^b\frac{\partial L}{\partial z}[y](t)dt \right)\frac{\partial l}{\partial y}\{y\}(x)h(x) dx
+ \int_{b-\tau}^b \left(\int_x^b\frac{\partial L}{\partial z}[y](t)dt \right)\frac{\partial l}{\partial y}\{y\}(x)h(x) dx;
\end{align*}

\textit{R5:} Using standard integration by parts, formulas \eqref{Int} and \eqref{GenInt} and since $h(a)=0$,
\begin{align*}
&\int_a^b  \frac{\partial L}{\partial z}[y](x)\left(\int_a^x\frac{\partial l}{\partial v}\{y\}(t){^C_aD^\alpha_t}h(t) dt \right) dx
= \int_a^b \left( -\frac{d}{dx}\int_x^b\frac{\partial L}{\partial z}[y](t)dt \right)\left(\int_a^x \frac{\partial l}{\partial v}\{y\}(t){^C_aD^\alpha_t}h(t)dt\right)dx\\
&= \int_a^b \left(\int_x^b\frac{\partial L}{\partial z}[y](t)dt \right)\frac{\partial l}{\partial v}\{y\}(x){^C_aD^\alpha_x}h(x) dx\\
&= \int_a^{b-\tau} \left(\int_x^b\frac{\partial L}{\partial z}[y](t)dt \right)\frac{\partial l}{\partial v}\{y\}(x){^C_aD^\alpha_x}h(x)dx
  +\int_{b-\tau}^b \left(\int_x^b\frac{\partial L}{\partial z}[y](t)dt \right)\frac{\partial l}{\partial v}\{y\}(x){^C_aD^\alpha_x}h(x)dx\\
&= \int_a^{b-\tau} {_xD^\alpha_{b-\tau}}\left(\int_x^b\frac{\partial L}{\partial z}[y](t)dt\frac{\partial l}{\partial v}\{y\}(x)\right)h(x)dx
+\left[ {_xI^{1-\alpha}_{b-\tau}}\left(\int_x^b\frac{\partial L}{\partial z}[y](t)dt\frac{\partial l}{\partial v}\{y\}(x) \right) h(x) \right]_a^{b-\tau}\\
  & \quad +\int_{b-\tau}^b {_xD^\alpha_b}\left(\int_x^b\frac{\partial L}{\partial z}[y](t)dt\frac{\partial l}{\partial v}\{y\}(x)\right)h(x)dx
  -\frac{h(a)}{\Gamma(1-\alpha)}\int_{b-\tau}^b(t-a)^{-\alpha}\int_t^b \frac{\partial L}{\partial z}[y](k)dk \frac{\partial l}{\partial v}\{y\}(t)dt  \\
     &\quad -\int_a^{b-\tau}\frac{h(x)}{\Gamma(1-\alpha)}\, \frac{d}{dx}
     \left( \int_{b-\tau}^b(t-x)^{-\alpha}\int_t^b \frac{\partial L}{\partial z}[y](k)dk \frac{\partial l}{\partial v}\{y\}(t)dt \right)dx\\
     &=\int_a^{b-\tau} \left[{_xD^\alpha_{b-\tau}}\left(\int_x^b\frac{\partial L}{\partial z}[y](t)dt\frac{\partial l}{\partial v}\{y\}(x)\right)
     -\frac{1}{\Gamma(1-\alpha)} \frac{d}{dx}
     \left( \int_{b-\tau}^b(t-x)^{-\alpha}\int_t^b \frac{\partial L}{\partial z}[y](k)dk \frac{\partial l}{\partial v}\{y\}(t)dt \right)\right]h(x)dx\\
     &\quad + \int_{b-\tau}^b {_xD^\alpha_b}\left(\int_x^b\frac{\partial L}{\partial z}[y](t)dt\frac{\partial l}{\partial v}\{y\}(x)\right)h(x)dx
\end{align*}

\textit{R6:} Using standard integration by parts, equation \eqref{Int2} and Lemma 2(b) of \cite{Baleanu},
\begin{align*}
&\int_a^b \frac{\partial L}{\partial z}[y](x)\left(\int_a^x\frac{\partial l}{\partial w}\{y\}(t){_aI^\beta_t}h(t) dt \right) dx=
\int_a^b \left( -\frac{d}{dx}\int_x^b\frac{\partial L}{\partial z}[y](t)dt\right)\left(\int_a^x\frac{\partial l}{\partial w}\{y\}(t){_aI^\beta_t}h(t) dt \right) dx\\
&=\int_a^b \left( \int_x^b\frac{\partial L}{\partial z}[y](t)dt\right) \frac{\partial l}{\partial w}\{y\}(x){_aI^\beta_x}h(x) dx\\
&=\int_a^{b-\tau}  \left(\int_x^b\frac{\partial L}{\partial z}[y](t)dt\right) \frac{\partial l}{\partial w}\{y\}(x){_aI^\beta_x}h(x) dx
+\int_{b-\tau}^b  \left(\int_x^b\frac{\partial L}{\partial z}[y](t)dt\right) \frac{\partial l}{\partial w}\{y\}(x){_aI^\beta_x}h(x) dx\\
&=\int_a^{b-\tau}\left[ {_xI^\beta_{b-\tau}}\left(\int_x^b\frac{\partial L}{\partial z}[y](t)dt\frac{\partial l}{\partial w}\{y\}(x)\right)
+\frac{1}{\Gamma(\beta)}\left(  \int_{b-\tau}^b (t-x)^{\beta-1}\int_t^b  \frac{\partial L}{\partial z}[y](k)dk \, \frac{\partial l}{\partial w}\{y\}(t)dt\right)
\right]h(x) dx \\
&+ \int_{b-\tau}^b {_xI^\beta_b}\left(\int_x^b\frac{\partial L}{\partial z}[y](t)dt\frac{\partial l}{\partial w}\{y\}(x)\right)h(x)dx.
\end{align*}

\textit{R7:} Since $h(x)=0$ for all $x\in[a-\tau,a]$,
$$\int_a^b \frac{\partial L}{\partial y_\tau}[y](x)h(x-\tau)dx=\int_{a-\tau}^{b-\tau} \frac{\partial L}{\partial y_\tau}[y](x+\tau)h(x)dx
=\int_a^{b-\tau} \frac{\partial L}{\partial y_\tau}[y](x+\tau)h(x)dx$$

\textit{R8:} Since $h(x)=0$ for all $x\in[a-\tau,a]$, using standard integration by parts, we have
$$\int_a^b \frac{\partial L}{\partial v_\tau}[y](x)h'(x-\tau)dx=\int_a^{b-\tau} \frac{\partial L}{\partial v_\tau}[y](x+\tau)h'(x)dx=
\frac{\partial L}{\partial v_\tau}[y](b)h(b-\tau)-\int_a^{b-\tau} \frac{d}{dx}\left(\frac{\partial L}{\partial v_\tau}[y](x+\tau)\right)h(x)dx$$

We are now in position to obtain a necessary condition of optimality when in presence of the time delay $\tau>0$.


\begin{theorem}\label{Teo1} If $y$ is a minimizer or maximizer of $J$ as in \eqref{funct}, then $y$ is a solution of the system of equations
\begin{enumerate}
\item $\displaystyle \frac{\partial L}{\partial v_\tau}[y](b)=0$;
\item for every $x\in[a,b-\tau]$,
\begin{align*}
&\frac{\partial L}{\partial y}[y](x)+{_xD^\alpha_{b-\tau}}\left( \frac{\partial L}{\partial v}[y](x) \right)
- \frac{1}{\Gamma(1-\alpha)}\, \frac{d}{dx}\left( \int_{b-\tau}^b(t-x)^{-\alpha} \frac{\partial L}{\partial v}[y](t)dt\right)\\
&\quad +{_xI_{b-\tau}^\beta}\left(\frac{\partial L}{\partial w}[y](x)\right)
+\frac{1}{\Gamma(\beta)}  \left( \int_{b-\tau}^b (t-x)^{\beta-1} \frac{\partial L}{\partial w}[y](t)dt \right)
 +\int_x^b \frac{\partial L}{\partial z}[y](t)dt  \frac{\partial l}{\partial y}\{y\}(x)\\
&\quad+{_xD^\alpha_{b-\tau}}\left( \int_x^b \frac{\partial L}{\partial z}[y](t)dt \frac{\partial l}{\partial v}\{y\}(x)\right)
-\frac{1}{\Gamma(1-\alpha)} \frac{d}{dx}
\left( \int_{b-\tau}^b(t-x)^{-\alpha}\int_t^b \frac{\partial L}{\partial z}[y](k)dk \frac{\partial l}{\partial v}\{y\}(t)dt \right)\\
&\quad +{_xI^\beta_{b-\tau}}\left( \int_x^b \frac{\partial L}{\partial z}[y](t)dt \frac{\partial l}{\partial w}\{y\}(x)  \right)
+\frac{1}{\Gamma(\beta)}\left(  \int_{b-\tau}^b (t-x)^{\beta-1}\int_t^b  \frac{\partial L}{\partial z}[y](k)dk \, \frac{\partial l}{\partial w}\{y\}(t)dt\right)\\
& \quad +\frac{\partial L}{\partial y_\tau}[y](x+\tau) -\frac{d}{dx}\frac{\partial L}{\partial v_\tau}[y](x+\tau)=0;
\end{align*}
\item for every $x\in[b-\tau,b]$,
\begin{align*}
&\frac{\partial L}{\partial y}[y](x)+{_xD^\alpha_b}\left( \frac{\partial L}{\partial v}[y](x) \right)+{_xI_b^\beta}\left(\frac{\partial L}{\partial w}[y](x)\right)\\
&\quad +\int_x^b \frac{\partial L}{\partial z}[y](t)dt \frac{\partial l}{\partial y}\{y\}(x)
+{_xD^\alpha_b}\left( \int_x^b \frac{\partial L}{\partial z}[y](t)dt \frac{\partial l}{\partial v}\{y\}(x)\right)
+{_xI^\beta_b}\left( \int_x^b \frac{\partial L}{\partial z}[y](t)dt \frac{\partial l}{\partial w}\{y\}(x)  \right)=0.
\end{align*}
\end{enumerate}
\end{theorem}

\begin{proof} If follows combining relations \textit{R1}-\textit{R8}, the arbitrariness of $h$ and from Theorem \ref{dubois}.
\end{proof}

\begin{example}\label{example2}
Consider the function
$$y_\alpha(x)=\left\{
\begin{array}{lll}
0&\mbox{ if }& x\in[-1,0]\\
x^{\alpha+1}&\mbox{ if }& x\in[0,2].\\
\end{array}\right.$$
Then
$${^C_0D_x^\alpha}y_\alpha(x)=\Gamma(\alpha+2)x.$$
For the cost functional, let
\begin{equation}
\label{example}
J(y)=\int_0^2 ({^C_0D_x^\alpha}y(x)-\Gamma(\alpha+2)x)^2+z(x)+(y'(x-1)-y'_\alpha(x-1))^2dx,
\end{equation}
where
$$z(x)=\int_0^x (y(t)-t^{\alpha+1})^2 \, dt,$$
defined on the set $C^1[-1,2]$, under the constraints
$$\left\{
\begin{array}{l}
y(2)=2^{\alpha+1},\\
y(x)=0, \mbox{ for all } x\in [-1,0].
\end{array}\right.$$
Since $J(y)\geq0$ for all admissible functions $y$, and $J(y_\alpha)=0$, we have that $y_\alpha$ is a minimizer of $J$ and zero is its minimum value.
Equations \textit{1-3} of Theorem \ref{Teo1} applied to $J$ read as
\begin{enumerate}
\item $\displaystyle \left[y'(x-1)-y'_\alpha(x-1)\right]_{x=2}=0$;
\item for every $x\in[0,1]$,
\begin{align*}
&{_xD_1^\alpha}({^C_0D_x^\alpha}y(x)-\Gamma(\alpha+2)x)
- \frac{1}{\Gamma(1-\alpha)}\,\frac{d}{dx}\left( \int_{1}^2(t-x)^{-\alpha}({^C_0D_t^\alpha}y(t)-\Gamma(\alpha+2)t)dt\right)\\
&\quad +\int_x^21dt \, (y(x)-x^{\alpha+1})-\frac{d}{dx}\left(y'(x)-y'_\alpha(x)\right)=0\\
\end{align*}
\item for every $x\in[1,2]$,
\begin{align*}
&{_xD_2^\alpha}({^C_0D_x^\alpha}y(x)-\Gamma(\alpha+2)x)+\int_x^21dt \, (y(x)-x^{\alpha+1})=0.\\
\end{align*}
\end{enumerate}

Obviously, $y_\alpha$ is a solution for the three previous conditions 1--3.
\end{example}

\begin{remark} In \cite{Jarad} fractional variational problems in presence of Caputo derivatives and delays are considered. Since the variational functions $h$ are
chosen in such a way that take the value zero at the extrema, the Caputo and the Riemann-Liouville derivative of these functions  are equal.
Using a general integration by parts formula of \cite{Baleanu} similar to our Lemma \ref{LemmaInt}, but for Riemann-Liouville derivative, the problem of
\cite{Jarad} is solved for Caputo derivative. Here we choose to obtain the equivalent formula of  \cite{Baleanu} for the Caputo derivative.
\end{remark}

\begin{remark} Consider the case when $\alpha$ goes to 1 and $\beta$ goes to zero. If so, we obtain the standard functional derived from \eqref{funct}:
\begin{equation}
\label{funct2}
J(y)=\int_a^b L(x,y(x),y'(x),z(x), y(x-\tau), y'(x-\tau))dx,
\end{equation}
where
$$z(x)=\int_a^x l(t,y(t),y'(t))dt,$$
defined for $y\in C^1[a-\tau,b]$ satisfying the boundary conditions
$$\left\{
\begin{array}{l}
y(b)=y_b\in \mathbb R,\\
y(x)=\phi(x), \mbox{ for all } x\in [a-\tau,a].
\end{array}\right.$$
If $y$ is a minimizer or maximizer of $J$ as in \eqref{funct2}, then $y$ is a solution of the system of equations
\begin{enumerate}
\item $\displaystyle \frac{\partial L}{\partial v_\tau}[y](b)=0$;
\item for every $x\in[a,b-\tau]$,
\begin{align*}
&\frac{\partial L}{\partial y}[y](x)-\frac{d}{dx}\left( \frac{\partial L}{\partial v}[y](x) \right)
+\int_x^b \frac{\partial L}{\partial z}[y](t)dt  \frac{\partial l}{\partial y}\{y\}(x)\\
&\quad -\frac{d}{dx}\left(\int_x^b \frac{\partial L}{\partial z}[y](t)dt \frac{\partial l}{\partial v}\{y\}(x)\right)
+\frac{\partial L}{\partial y_\tau}[y](x+\tau) -\frac{d}{dx}\frac{\partial L}{\partial v_\tau}[y](x+\tau)=0;
\end{align*}
\item for every $x\in[b-\tau,b]$,
$$\frac{\partial L}{\partial y}[y](x)-\frac{d}{dx}\left(\frac{\partial L}{\partial v}[y](x) \right)
+\int_x^b \frac{\partial L}{\partial z}[y](t)dt \frac{\partial l}{\partial y}\{y\}(x)
-\frac{d}{dx}\left( \int_x^b \frac{\partial L}{\partial z}[y](t)dt \frac{\partial l}{\partial v}\{y\}(x)\right)=0.$$
\end{enumerate}
This result is apparently new also.
\end{remark}

\begin{remark} Theorem \ref{Teo1} can be generalized for functionals with several dependent variables. Let us consider
\begin{equation}
\label{funct3}
J(y_1,\ldots,y_n)=\int_a^b L(x,y_1(x),\ldots,y_n(x),{^C_aD_x^{\alpha_1}}y_1(x),\ldots,{^C_aD_x^{\alpha_n}}y_n(x),\end{equation}
$${_aI_x^{\beta_1}}y_1(x),\ldots,{_aI_x^{\beta_n}}y_n(x),z(x), y_1(x-\tau_1),\ldots,y_n(x-\tau_n), y_1'(x-\tau_1),\ldots, y_n'(x-\tau_n))dx,$$
defined on $C^1 \prod^{n}_{i=1} [a-\tau_i,b]$, where for all $i\in\{1,\ldots,n\}$,
$$\left\{
\begin{array}{l}
\tau_i>0, \mbox{ and } \tau_i<b-a,\\
\alpha_i\in(0,1) \mbox{ and }\beta_i>0,\\
z(x)=\int_a^x l(t,y_1(t),\ldots,y_n(t),{^C_aD_t^{\alpha_1}}y_1(t),\ldots,{^C_aD_t^{\alpha_n}}y_n(t),{_aI_t^{\beta_1}}y_1(t),\ldots,{_aI_t^{\beta_n}}y_n(t))dt,\\
L=L(x,y_1,\ldots,y_n,v_1,\ldots,v_n,w_1,\ldots,w_n,z,y_{\tau_1},\ldots,y_{\tau_n},v_{\tau_1},\ldots,v_{\tau_n}) \mbox{ and } \\
l=l(x,y_1,\ldots,y_n,v_1,\ldots,v_n,w_1,\ldots,w_n) \mbox{ are of class } C^1
\end{array}\right.$$
and the admissible functions are such that
$$\left\{
\begin{array}{l}
{^C_aD_x^{\alpha_i}}y_i(x) \mbox{ and } {_aI_x^{\beta_i}}y_i(x) \mbox{ exist and are continuous on } [a,b],\\
y_i(b)=y_{b_i}\in \mathbb R,\\
y_i(x)=\phi_i(x), \mbox{ for all } x\in [a-\tau_i,a], \, \phi_i \mbox{ a fixed function.}
\end{array}\right.$$
If the \textit{n}-uple function $(y_1,\ldots,y_n)$ is a minimizer or maximizer of $J$ as in \eqref{funct3},
then $(y_1,\ldots,y_n)$ is a solution of the system of equations similar to the ones of \textit{1-3} of Theorem \ref{Teo1}, replacing the variables
$$y\to y_i,\quad v\to v_i,\quad w\to w_i,\quad y_\tau\to y_{\tau_i},\quad v_\tau\to v_{\tau_i},\quad \alpha\to\alpha_i,\quad \beta\to\beta_i,\quad\tau\to\tau_i,$$
for all $i\in\{1,\dots,n\}$.
\end{remark}



\section{Sufficient condition}\label{sec:SufConditions}


Assuming some convexity conditions on the Lagrangian $L$ and on the supplementary function $l$, we may present a necessary condition that guarantees the existence
of minimizers for the problem. For convenience, recall the definition of convex and concave function. Given $k\in\{1,\ldots,n\}$ and $f:D\subseteq\mathbb{R}^n\to \mathbb{R}$ a
 function differentiable with respect to $x_k,\ldots,x_n$, we say that $f$ is convex (resp. concave) in  $(x_k,\ldots,x_n)$ if
$$f(x_1+c_1,\ldots,x_n+c_n)-f(x_1,\ldots,x_n)\geq \, (resp. \leq) \, \sum_{i=k}^n\frac{\partial f}{\partial x_i}(x_1,\ldots,x_n)c_i,$$
for all $(x_1,\ldots,x_n),(x_1+c_1,\ldots,x_n+c_n)\in D$.


\begin{theorem} Let $y$ be a function satisfying conditions 1--3 of Theorem \ref{Teo1}. If $L$ is convex in $(y,v,w,z,y_\tau,v_\tau)$ and one of the two following
conditions are met
\begin{enumerate}
\item $l$ is convex in $(y,v,w)$ and $\frac{\partial L}{\partial z}[y](x) \geq 0$ for all $x \in [a,b]$,
\item $l$ is concave in $(y,v,w)$ and $\frac{\partial L}{\partial z}[y](x) \leq 0$ for all $x \in [a,b]$,
\end{enumerate}
then $y$ is a minimizer of the functional $J$ as in \eqref{funct}.
\end{theorem}

\begin{proof}
Let $y+h$ be a variation of $y$. Using relations  \textit{R1}-\textit{R8} of Section \ref{sec:ELequation}, we have that
$$\begin{array}{ll}
J(y+h) - J(y) & \displaystyle\geq \int_a^{b-\tau}\left[  \frac{\partial L}{\partial y}[y](x)+{_xD^\alpha_{b-\tau}}\left( \frac{\partial L}{\partial v}[y](x) \right)
- \frac{1}{\Gamma(1-\alpha)}\, \frac{d}{dx}\left( \int_{b-\tau}^b(t-x)^{-\alpha} \frac{\partial L}{\partial v}[y](t)dt\right)\right.\\
&\displaystyle\quad+{_xI_{b-\tau}^\beta}\left(\frac{\partial L}{\partial w}[y](x)\right)+\frac{1}{\Gamma(\beta)}  \left( \int_{b-\tau}^b (t-x)^{\beta-1} \frac{\partial L}{\partial w}[y](t)dt \right)\\
&\displaystyle\quad+\int_x^b \frac{\partial L}{\partial z}[y](t)dt  \frac{\partial l}{\partial y}\{y\}(x)
+{_xD^\alpha_{b-\tau}}\left( \int_x^b \frac{\partial L}{\partial z}[y](t)dt \frac{\partial l}{\partial v}\{y\}(x)\right)\\
&\displaystyle\quad-\frac{1}{\Gamma(1-\alpha)} \frac{d}{dx}
\left( \int_{b-\tau}^b(t-x)^{-\alpha}\int_t^b \frac{\partial L}{\partial z}[y](k)dk \frac{\partial l}{\partial v}\{y\}(t)dt \right)\\
 &\displaystyle\quad+{_xI^\beta_{b-\tau}}\left( \int_x^b \frac{\partial L}{\partial z}[y](t)dt \frac{\partial l}{\partial w}\{y\}(x)  \right)
+\frac{1}{\Gamma(\beta)}\left(  \int_{b-\tau}^b (t-x)^{\beta-1}\int_t^b  \frac{\partial L}{\partial z}[y](k)dk \, \frac{\partial l}{\partial w}\{y\}(t)dt\right)\\
&\displaystyle\left.\quad +\frac{\partial L}{\partial y_\tau}[y](x+\tau) -\frac{d}{dx}\frac{\partial L}{\partial v_\tau}[y](x+\tau)\right] h(x)dx\\
&\displaystyle\quad+ \int_{b-\tau}^b\left[  \frac{\partial L}{\partial y}[y](x)+{_xD^\alpha_b}\left( \frac{\partial L}{\partial v}[y](x) \right)+{_xI_b^\beta}\left(\frac{\partial L}{\partial w}[y](x)\right)+\int_x^b \frac{\partial L}{\partial z}[y](t)dt \frac{\partial l}{\partial y}\{y\}(x)\right.\\
&\displaystyle\quad\left. +{_xD^\alpha_b}\left( \int_x^b \frac{\partial L}{\partial z}[y](t)dt \frac{\partial l}{\partial v}\{y\}(x)\right)
+{_xI^\beta_b}\left( \int_x^b \frac{\partial L}{\partial z}[y](t)dt \frac{\partial l}{\partial w}\{y\}(x)  \right)\right] h(x)dx\\
&\displaystyle\quad+ \frac{\partial L}{\partial v_\tau}[y](b)h(b-\tau)=0.\\
\end{array}$$
\end{proof}

For example, report to the example \ref{example2}. For this case,
$$L(x,y,v,w,z,y_\tau,v_\tau) =(v-\Gamma(\alpha+2)x)^2+z+(y_\tau-(\alpha+1)(x-1)^\alpha)^2\mbox{ and }l(x,y,v,w)=(y-x^{\alpha+1})^2 $$
are both convex, and $\frac{\partial L}{\partial z}[y](x) =1$. Observe that $y_\alpha$ is a solution of equations  \textit{1--3} of Theorem \ref{Teo1}, and in fact is a
minimizer of $J$.

\section{Conclusion}

The aim of the paper is to generalize the main result of \cite{Almeida5}, by considering delays in our system. Necessary conditions are proven in case the Lagrange function depends on fractional derivatives and on indefinite integral as well. For future work, we will study numerical tools to solve directly these kind of problems, avoiding to solve analytically fractional differential equations. 
 

\section*{Acknowledgements}

This work was supported by Portuguese funds through the CIDMA - Center for Research and Development in Mathematics and Applications, and the Portuguese Foundation for Science and Technology (“FCT–-Funda\c{c}\~{a}o para a Ci\^{e}ncia e a Tecnologia”), within project PEst-OE/MAT/UI4106/2014.


\begin{thebibliography}{99}

\bibitem{AGRA0}
O. P. Agrawal,
A Bliss-type multiplier rule for constrained variational problems with time delay.
J. Math. Anal. Appl. {\bf 210} (1997), no. 2, 702-–711.

\bibitem{AGRA1}
O. P. Agrawal,
Generalized Euler-Lagrange equations and transversality
conditions for FVPs in terms of the Caputo derivative,
J. Vib. Control {\bf 13} (2007), no.~9-10, 1217--1237.

\bibitem{Almeida1}
R. Almeida\ and\ D. F. M. Torres,
Calculus of variations with fractional
derivatives and fractional integrals,
Appl. Math. Lett. {\bf 22} (2009), no.~12, 1816--1820.

\bibitem{Almeida}
R. Almeida\ and\ D. F. M. Torres,
Necessary and sufficient conditions for the fractional calculus
of variations with Caputo derivatives,
Commun. Nonlinear Sci. Numer. Simulat. {\bf 16} (2011), no.~3, 1490--1500.

\bibitem{Almeida5}
R. Almeida, S. Pooseh e D.F.M. Torres,
Fractional variational problems depending on indefinite integrals,
Nonlinear Anal. {\bf 75} (2012), no. 3, 1009-–1025.

\bibitem{Baleanu1}
D. Baleanu, amd O. P. Agrawal,
Fractional Hamilton formalism within Caputo's derivative.
Czechoslovak J. Phys.  {\bf 56} (2006), no. 10-11, 1087-–1092.

\bibitem{Baleanu}
D. Baleanu, T. Maaraba and F. Jarad,
Fractional variational principles with delay.
J. Phys. A {\bf 41} (2008), no. 31, 315403, 8 pp.

\bibitem{Bhrawy}
A.H. Bhrawy, M.A. Alghamdi, M.M. Tharwat. A new operational matrix of fractional integration for shifted Jacobi polynomials, 
Bull. Malays. Math. Sci. Soc. (2), accepted.

\bibitem{Chen2}
J. Chen and X. H. Tang, 
Infinitely many solutions for a class of fractional boundary value problem,
Bull. Malays. Math. Sci. Soc. (2)  {\bf 36}  (2013),  no. 4, 1083--1097.

\bibitem{Chen}
Y Chen, H. Wang, A. Xue and R. Lu,
Renquan Passivity analysis of stochastic time-delay neural networks.
Nonlinear Dynam. {\bf 61} (2010), no. 1-2, 71-–82.


\bibitem{Dehghan}
M. Dehghan and R. Salehi,
Solution of a nonlinear time-delay model in biology via semi-analytical approaches.
Comput. Phys. Comm. {\bf 181} (2010), no. 7, 1255–-1265.

\bibitem{Gastao0}
G. S. F. Frederico\ and\ D. F. M. Torres,
Fractional optimal control in the sense of Caputo
and the fractional Noether's theorem,
Int. Math. Forum {\bf 3} (2008), no.~9-12, 479--493.

\bibitem{Gregory}
J. Gregory,
Generalizing variational theory to include
the indefinite integral, higher derivatives,
and a variety of means as cost variables,
Methods Appl. Anal. {\bf 15} (2008), no.~4, 427--435.

\bibitem{Jarad}
F. Jarad, T. Abdeljawad and D. Baleanu,
Fractional variational principles with delay within Caputo derivatives.
Rep. Math. Phys. {\bf 65} (2010), no. 1, 17–--28.

\bibitem{Kilbas}
A. A. Kilbas, H. M. Srivastava\ and\ J. J. Trujillo,
{\it Theory and applications of fractional differential equations},
North-Holland Mathematics Studies, 204, Elsevier, Amsterdam, 2006.

\bibitem{Liu0}
Y. Liu and J. Zhao,
Nonfragile control for a class of uncertain switching fuzzy time-delay systems.
J. Control Theory Appl. {\bf 8} (2010), no. 2, 229–-232.

\bibitem{Liu}
B. Liu, Q. Zhang and Y. Gao,
The dynamics of pest control pollution model with age structure and time delay.
Appl. Math. Comput. {\bf 216} (2010), no. 10, 2814–-2823.

\bibitem{Loghmani}
G. B. Loghmani and S. Javanmardi,
Numerical methods for sequential fractional differential equations for Caputo operator.
Bull. Malays. Math. Sci. Soc.   (2) {\bf 35}  (2012),  no. 2, 315--323.

\bibitem{Malinowska}
A. B. Malinowska\ and\ D. F. M. Torres,
Generalized natural boundary conditions for fractional
variational problems in terms of the Caputo derivative,
Comput. Math. Appl. {\bf 59} (2010), no.~9, 3110--3116.

\bibitem{Mo}
J. Mo and Z. Wen,
Singularly perturbed reaction diffusion equations with time delay.
Appl. Math. Mech. (English Ed.) {\bf 31} (2010), no. 6, 769-–774.

\bibitem{Nat}
N. Martins\ and\ D. F. M. Torres,
Generalizing the variational theory on time scales
to include the delta indefinite integral,
Comput. Math. Appl., {\bf 61} (2011), 2424--2435.

\bibitem{Miller}
K. S. Miller\ and\ B. Ross,
{\it An introduction to the fractional calculus
and fractional differential equations},
A Wiley-Interscience Publication, Wiley, New York, 1993.

\bibitem{Mozyrska}
D. Mozyrska\ and\ D. F. M. Torres,
Minimal modified energy control for fractional
linear control systems with the Caputo derivative,
Carpathian J. Math. {\bf 26} (2010), no.~2, 210--221.

\bibitem{Rosenblueth1}
J. F. Rosenblueth,
Systems with time delay in the calculus of variations: the method of steps.
IMA J. Math. Control Inform. {\bf 5} (1988), no. 4, 285–-299.

\bibitem{Rosenblueth2}
J. F. Rosenblueth,
Systems with time delay in the calculus of variations: a variational approach.
IMA J. Math. Control Inform. {\bf 5} (1988), no. 2, 125–-145.

\bibitem{samko}
S. G. Samko, A. A. Kilbas\ and\ O. I. Marichev,
{\it Fractional integrals and derivatives},
Translated from the 1987 Russian original,
Gordon and Breach, Yverdon, 1993.

\bibitem{Udaltsov}
V. S. Udaltsov, J. P. Goedgebuer, L. Larger, J. B. Cuenot, P. Levy and W. T. Rhodes,
Cracking chaos-based encryption systems ruled by nonlinear time delay differential equations.
Phys. Lett. A {\bf 308} (2003), no. 1, 55–-60.

\bibitem{Brunt}
B. van Brunt,
{\it The calculus of variations},
Universitext, Springer, New York, 2004.

\bibitem{Wang}
H. Wang and Y., 
Necessary and sufficient optimality conditions for fvps with generalized boundary conditions, 
J. Nonlinear Anal. Optim. {\bf 3} (2012), 33--43

\bibitem{Xu}
S. Xu and Z. Feng,
Analysis of a mathematical model for tumor growth under indirect effect of inhibitors with time delay in proliferation.
J. Math. Anal. Appl. {\bf 374} (2011), no. 1, 178–-186.

\bibitem{Yueqiang}
S. Yueqiang. 
Existence of positive solutions for a three-point boundary value problem with fractional q-differences, 
Bull. Malays. Math. Sci. Soc. (2), accepted.

\bibitem{Zhenhai}
L. Zhenhai and L. Jitai. Multiple solutions of nonlinear boundary value problems for fractional differential equations,
Bull. Malays. Math. Sci. Soc. (2), accepted.

\bibitem{Zhu}
Q. Zhu and J. Cao,
Adaptive synchronization of chaotic Cohen-Crossberg neural networks with mixed time delays.
Nonlinear Dynam. {\bf 61} (2010), no. 3, 517–-534.

\end{thebibliography}


\end{document}



\documentclass[11pt,english]{smfart}
\usepackage[french]{babel}

\usepackage[utf8]{inputenc}
\usepackage[T1]{fontenc}

\usepackage[]{hyperref}
\hypersetup{
    colorlinks=true,       
    linkcolor=red,          
    citecolor=blue,        
    filecolor=magenta,      
    urlcolor=cyan           
}

\usepackage[leqno]{amsmath}
\usepackage{amssymb}
\usepackage{mathrsfs}
\usepackage{amsthm}
\usepackage{amsxtra}
\usepackage{bm}
\setcounter{tocdepth}{1}
\usepackage[titletoc]{appendix}

\parskip=5pt
\parindent=0pt

\usepackage{graphicx}

\theoremstyle{plain}
\newtheorem{theo}{Theorem}[section]
\newtheorem{prop}[theo]{Proposition}
\newtheorem{lemm}[theo]{Lemma}
\newtheorem{coro}[theo]{Corollary}
\newtheorem{assu}[theo]{Assumption}
\newtheorem{defi}[theo]{Definition}
\theoremstyle{definition}
\newtheorem{rema}[theo]{Remark}
\newtheorem{nota}[theo]{Notation}

\numberwithin{equation}{section}

\title[A paradifferential reduction for the gravity-capillary waves]{A paradifferential reduction for the gravity-capillary waves system at low regularity and applications}

\author{Thibault de Poyferr\'{e}}
\address{UMR 8553 du CNRS, Laboratoire de Mathématiques et Applications de l'Ecole Normale Supérieure, 75005 Paris, France}
\email{tdepoyfe@dma.ens.fr}
\author{Quang-Huy Nguyen}
\address{UMR 8628 du CNRS, Laboratoire de Math\'ematiques d'Orsay, Universit\'e Paris-Sud, 91405 Orsay Cedex, France}
\email{quang-huy.nguyen@math.u-psud.fr }
\date{}
\keywords{gravity-capillary waves, paradifferential approach, blow-up criterion, a priori estimate, contraction estimate}

\DeclareSymbolFont{pletters}{OT1}{cmr}{m}{sl}
\DeclareMathSymbol{s}{\mathalpha}{pletters}{`s}

\begin{document}

\begin{abstract}
We consider in this article the system of gravity-capillary waves (in any dimension) under the Zakharov/Craig-Sulem formulation. Using a paradifferential approach introduced by Alazard-Burq-Zuily we symmetrize this system into a quasilinear equation whose principal term is of order $3/2$. The main novelty compare to earlier studies is that this reduction is performed at the  Sobolev regularity of quasilinear pdes: $H^s({\mathbf{R}}^d)$ with $s>3/2+d/2$ ($d$ is the horizontal dimension). From this reduction, we deduce a blow-up criterion and then an a priori estimate for the solution and the Lipschitz continuity of the flow map in terms of the Sobolev norm and the Strichartz norm.
\end{abstract}
\maketitle

\tableofcontents
\section{Introduction}
{\hspace*{.15in}} We consider the system of gravity-capillary waves describing the motion of a fluid interface under the effect of gravity and surface tension. From the well-posedness result in Sobolev spaces  Yosihara \cite{Yosihara} (see also Wu \cite{WuInvent, WuJAMS} for pure gravity waves) it is known that the system is quasilinear in nature. In a more recent work \cite{ABZ1} Alazard-Burq-Zuily showed explicitly this quasilinearity by using a paradifferential approach (see Appendix \ref{Appendix}) to symmetrize the system into the following paradifferential equation
\begin{equation}\label{intro:reduce}
\big(\partial_t +T_{V(t, x)}\cdot\nabla +iT_{\gamma(t, x, \xi)} \big) u(t,x)=f(t,x)
\end{equation}
where $V$ is the horizontal component of the trace of the velocity field on the free surface, $\gamma$ is a paradifferential symbol of order $3/2$, depending on the solution. This reduction has many consequences, among them are the local well-posedness and smoothing effect in \cite{ABZ1}, Strichartz estimates in \cite{ABZ2}  for $u\in L^\infty_tH_x^s({\mathbf{R}}^d)$ with $s>2+d/2$. As remarked in \cite{ABZ1} $s>2+d/2$ is the minimal Sobolev index (in term of Sobolev's embedding) to ensure that the velocity filed is Lipschitz up to the boundary, without taking into account the dispersive property.  From the works of Alazard-Burq-Zuily \cite{ABZ3, ABZ4},  Hunter-Ifrim-Tataru \cite{HuIfTa} for pure gravity waves it seems natural to assume that the gradient of the velocity is Lipschitz so that the particles flow is well-defined. On the other hand, from the standard theory of quasilinear pdes, it is natural to ask if the reduction \eqref{intro:reduce} holds at the Sobolev threshold $s>3/2+d/2$ and then, if a local-wellposedness theory holds at the same level of regularity?  The two observations above motivate us to study the gravity-capillary system at the following regularity level: 
\begin{equation}\label{intro:reg}
u\in \mathcal{X}:=L^\infty_tH^s_x\cap L^p_tW^{r, \infty}_x \quad\text{with}~
s>\frac{3}{2}+\frac{d}{2},~r>2,
\end{equation}
which exhibits a gap of $1/2$ derivatives that may be filled up by Strichartz estimates. One of our main results will be a blow-up criterion at this scaling with $p=1$ (i.e. merely integrable in time), which states that the solution can be prolonged as long as the $\mathcal{X}$-norm of $u$  remained bounded (at least in the case of infinite depth). To derive our criterion, the main difficulty compare to the reduction in \cite{ABZ1} is that we have to keep all the estimates in the analysis to be {\it tame}, i.e., linear with respect to the highest norm-the H\"older norm $W^{r,\infty}$.\\
{\hspace*{.15in}} First of all, let us recall the Zakharov/Craig-Sulem formulation of water waves.
\subsection{The Zakharov/Craig-Sulem formulation}
 We consider an incompressible inviscid fluid with unit density moving in a time-dependent domain  
$$
\Omega = \{(t,x,y) \in[0,T] \times {\mathbf{R}}^d \times {\mathbf{R}}:(x, y)\in \Omega_t\} 
$$
where each $\Omega_t$ is a domain located underneath a free surface 
$$
\Sigma_t = \{(x,y)  \times {\mathbf{R}}^d \times {\mathbf{R}}: y=\eta(t, x)\} 
$$
and above a fixed bottom $\Gamma=\partial\Omega_t\setminus \Sigma_t$. We make the following separation assumption $(H_t)$ on the  domain at time~$t$:\\
{\it
$\Omega_t$ is the intersection of the half space 
\[
\Omega_{1,t}= \{(x,y)  \times {\mathbf{R}}^d \times {\mathbf{R}}: y=\eta(t, x)\} 
\]
and an open connected set $\mathcal{O}$ containing a fixed strip around $\Sigma_t$, i.e., there exists $h>0$ such that 
\begin{equation}\label{sepbot}
 \{(x,y)  \times {\mathbf{R}}^d \times {\mathbf{R}}: \eta(x)-h{\leq} y{\leq}\eta(t, x)\} \subset \mathcal{O}.
\end{equation}
}
{\hspace*{.15in}} Assume that the velocity field $v$ admits a potential $\phi:\Omega \to {\mathbf{R}}$, i.e, $v=\nabla \phi$. Using the  idea of Zakharov, we introduce the trace of $\phi$ on the free surface
$$\psi(t,x)= \phi(t,x,\eta(t,x)).$$ 
 Then $\phi(t, x, y)$ is the unique variational solution of 
\begin{equation}
\Delta\phi =0\text{~in}~\Omega_t,\quad \phi(t, x, \eta(t, x))=\psi(t, x).
\end{equation}
The Dirichlet-Neumann operator is then defined by
\begin{align*}
G(\eta) \psi &= \sqrt{1 + \vert \nabla_x \eta \vert ^2}
\Big( \frac{\partial \phi}{\partial n} \Big \arrowvert_{\Sigma}\Big)\\
&= (\partial_y \phi)(t,x,\eta(t,x)) - \nabla_x \eta(t,x) \cdot(\nabla_x \phi)(t,x,\eta(t,x)).
\end{align*}
The gravity-capillary water waves problem with surface tension consists in solving the following so-called Zakharov-Craig-Sulem system on $\eta,\psi$:
\begin{equation}\label{ww}
\left\{
\begin{aligned}
&\partial_t \eta = G(\eta) \psi,\\
&\partial_t \psi + g\eta-H(\eta)+{\frac{1}{2}} \vert \nabla_x \psi \vert^2 - {\frac{1}{2}} \frac{(\nabla_x \eta \cdot \nabla_x \psi + G(\eta)\psi)^2}{1+ \vert \nabla_x \eta \vert^2}=0.
\end{aligned}
\right.
\end{equation}
Here, $H(\eta)$ is the mean curvature of the free surface:
\[
H(\eta)=\operatorname{div}\left( \frac{\nabla\eta}{\sqrt{1+|\nabla\eta|^2}}\right).
\]
It is important to introduce the vertical and horizontal components of the velocity on $\Sigma$, which can be expressed in terms of $\eta$ and $\psi$:
\begin{equation}\label{BV}
B = (v_y)\arrowvert_\Sigma = \frac{ \nabla_x \eta \cdot \nabla_x \psi + G(\eta)\psi} {1+ \vert \nabla_x \eta \vert^2},\quad V= (v_x)\arrowvert_\Sigma  =\nabla_x \psi - B \nabla_x \eta.
 \end{equation}
\subsection{Main results}
The Cauchy problem has been extensively studied, for example in Nalimov \cite{Nalimov}, Yosihara \cite{Yosihara},  Coutand- Shkoller \cite{CS}, Craig \cite{Craig1985}, Shatah-Zeng \cite{SZ1, SZ2, SZ3},  Ming-Zhang \cite{MiZh},  Lannes \cite{LannesLivre}: for sufficiently smooth solutions and  Alazard-Burq-Zuily \cite{ABZ1} for solutions at the energy threshold. See also Craig \cite{Craig1985}, Wu \cite{WuInvent, WuJAMS}, Lannes \cite{LannesJAMS} for the studies on gravity waves. Observe that the linearized system of \eqref{ww} about the rest state $(\eta=0, \psi=0)$ when $g=0$ reads
\[
\begin{cases}
\partial_t\eta-|D_x|\psi=0,\\
\partial_t\psi-\Delta \eta=0
\end{cases}
\]
which becomes 
\begin{equation}\label{eq:lin}
\partial_t\Phi +i|D_x|^{\frac{3}{2}} \Phi=0,\quad\text{with}~\Phi=|D_x|^{\frac{1}{2}}\eta+i\psi.
\end{equation}
Therefore, it is natural to study \eqref{ww} at the following algebraic scaling
\[
(\eta, \psi)\in H^{s+{\frac{1}{2}}}({\mathbf{R}}^d)\times H^s({\mathbf{R}}^d).
\]
From the formula \eqref{BV} for the trace of velocity on the surface, we have that the Lipschitz threshold in \cite{ABZ1} corresponds to $s> 2+d/2.$  On the other hand, the threshold $s>3/2+d/2$ suggested by the quasilinear nature \eqref{intro:reduce} is also the minimal Sobolev index to ensure that the mean curvature  $H(\eta)$ is bounded. The question we are concerned with is:\\
{\hspace*{.15in}} ({\bf Q}) If the Cauchy problem for \eqref{ww} is solvable for initial data 
\begin{equation}\label{intro:data}
(\eta_0, \psi_0)\in H^{s+{\frac{1}{2}}}\times H^s,\quad s>{\frac{3}{2}}+\frac d2.
\end{equation}
Assume now that 
\begin{equation}
(\eta,\psi)\in L^{\infty}{\left(}[0, T]; H^{s+{\frac{1}{2}}}\times H^s{\right)}\cap L^p{\left(}[0, T]; W^{r+{\frac{1}{2}}, \infty}\times W^{r, \infty}{\right)}
\end{equation}
with 
\begin{equation}\label{intro:reg}
 s>{\frac{3}{2}}+\frac{d}{2},\quad2<r<s+\frac12-\frac d2,~p\ge 1.
\end{equation}
 is a solution whose data is \eqref{intro:data}. We shall prove in Proposition \ref{singleeq:Phi} that the quasilinear reduction \eqref{intro:reduce} of system \eqref{ww} still holds with the right-hand-side term $f(t, x)$ satisfying a tame estimate. To be concise in the following statements let us define the quantities that control the system:
\begin{align*}
&\text{Sobolev norms}:~M_{\sigma,T}=\Vert (\eta, \psi)\Vert_{L^{\infty}([0, T]; H^{\sigma+{\frac{1}{2}}}\times H^\sigma)},~ M_{\sigma,0}=\Vert (\eta_0, \psi_0)\Vert_{H^{\sigma+{\frac{1}{2}}}\times H^\sigma},\\
&\text{Blow-up norm}: N_{\sigma,T}=\Vert (\eta, \psi)\Vert_{L^1([0, T]; W^{\sigma+{\frac{1}{2}}, \infty}\times W^{\sigma, \infty})},\\
& \text{Strichartz norm}:Z_{\sigma,T}=\Vert (\eta, \psi)\Vert_{L^p([0, T]; W^{\sigma+{\frac{1}{2}}, \infty}\times W^{\sigma, \infty})}.
\end{align*}
Our first result concerns  an a priori estimate for the Sobolev norm $M_{s, T}$ in terms of itself and the Strichartz norm $Z_{r,T}$.
\begin{theo}\label{intro:theo:apriori}
Let~$d\geq 1$, $h>0,~p> 1$. Then there exists a non-negative, non-decreasing function~${ \mathcal{F}}$  such that: for all $T\in (0, 1]$ and all $(\eta, \psi)$ solution to \eqref{ww} on $[0, T]$ with regularity \eqref{intro:reg} and initial data \eqref{intro:data} and satisfies $\inf_{t\in [0,T]}\operatorname{dist}(\eta(t), \Gamma)>h$, there holds
\[
 M_{s, T}\leq{ \mathcal{F}}{\left(} M_{s,0}+T{ \mathcal{F}}{\left(} M_{s, T}+Z_{r,T}{\right)}{\right)}.
\]
\end{theo}
As a  consequence, when $s>2+d/2$ one retrieves by Sobolev embeddings the a priori estimate in \cite{ABZ1}.\\
{\hspace*{.15in}} 
  Our second result provides a blow-up criterion for solutions of \eqref{ww}.
\begin{theo}\label{intro:theo:blowup}
	Let~$d\geq 1,~h>0$ and indices
	$$\frac32+\frac d2<s_0<s-{\frac{1}{2}},\quad2<r<s_0+\frac12-\frac d2.$$
Let $T^*=T^*(\eta_0, \psi_0, h)$ be the maximal time of existence defined by \eqref{def:T*} and
\[
(\eta,\psi)\in L^{\infty}{\left(}[0, T^*); H^{s+{\frac{1}{2}}}\times H^s{\right)}
\]
	be the maximal solution of ~\eqref{ww}  with prescribed data $(\eta_0, \psi_0)$ satisfying $\operatorname{dist}(\eta_0, \Gamma)>h$.
	Then if ~$T^*$ is finite, we have 
\[
\limsup_{T\rightarrow T^*}\Big( M_{s_0, T}+N_{r, T}+\frac{1}{h(T)}\Big)=+\infty,
\]
where $h(T)$ is the distance from the surface $\eta$ to the bottom $\Gamma$ over the time  interval $[0, T]$.
\end{theo}
 Remark that the Sobolev regularity in the above criterion  is exactly the one given in question ({\bf Q}). In contrast, for pure gravity waves in which the surface tension is neglected, it was shown in \cite{Thibault} and \cite{HuIfTa} that boundedness of the curvature is irrelevant: that  $\int_0^T{\left\Vert} \nabla\eta(t){\right\Vert}_{W^{1/2}}dt<+\infty$  is enough.\\
{\hspace*{.15in}} In the survey paper \cite{CrWa} Craig-Wayne posed (see Problem $3$) the following questions on {\it How do solutions break down?}: \\
{\bf (Q1)}~{\it For which $\alpha$ is it true that, if one knows a priori that $\sup_{[-T, T]}\| (\eta, \psi)\|_{C^\alpha}<+\infty$  then $C^\infty$ data $(\eta_0, \psi_0)$ implies that the solution is $C^\infty$ over the time interval $[-T, T]$?}\\
{\bf (Q2)}~{\it 
It would be more satisfying to say that the solution fails to exist because the "curvature of the surface has diverged at some point", or a related geometrical and/or physical statement}.\\
 Theorem \ref{intro:theo:blowup} gives a partial answer to {\bf (Q2)}: our criteria on $M_{s_0, T}$ and $N_{r, T}$ are directly imposed on the regularity of the solution yet they correspond to the following geometrical and physical statements:
\begin{itemize}
\item $\| \eta\|_{L^\infty([0, T^*); H^{s_0+{\frac{1}{2}}})}=+\infty$ is the minimal Sobolev-based criterion to say that the curvature explodes,
\item $N_{r,T^*}=+\infty$ corresponds to the condition that the  (slightly above) Lipschitz norm of the trace of velocity on the surface fail to be intergable in time,
\item $h(T^*)=0$ means that the bottom rises to the surface.
\end{itemize}
Concerning question ({\bf Q1}) we have by virtue of Theorem \ref{intro:theo:blowup} the following result on persistence of Sobolev regularity.
\begin{coro}\label{intro:prop:regularity}
Let $T\in (0, +\infty)$ and $(\eta, \psi)$ be a distributional solution to system \eqref{ww} on the time interval $[0, T]$  such that  $\inf_{[0, T]}\operatorname{dist}(\eta(t), \Gamma)>0.$ Then the following property holds: if one knows a priori that 
\begin{equation}\label{answer:Graig}
\sup_{[0, T]}\| (\eta(t), \psi(t))\|_{H^{2+\frac{d}{2}+}\times H^{{\frac{3}{2}}+\frac{d}{2}+}}+\int_{0}^T\| (\eta(t), \psi(t))\|_{C^{\frac{5}{2}+}\times C^{2+}}{\,\mathrm{d}} t<+\infty
\end{equation}
 then $(\eta(0), \psi(0))\in  H^\infty({\mathbf{R}}^d)^2$ implies that $(\eta, \psi)\in  L^\infty([0, T]; H^\infty({\mathbf{R}}^d))^2$.
\end{coro}
As a consequence, by Sobolev's embedding one can replace \eqref{answer:Graig} by a stronger condition involving only Sobolev regularity
\[
\sup_{[0, T]}\| (\eta(t), \psi(t))\|_{H^{\frac{5}{2}+\frac{d}{2}+}\times H^{2+\frac{d}{2}+}}<+\infty
\]
and thus obtain an answer for the Sobolev version of {\bf (Q1)}.\\ 
{\hspace*{.15in}}  Finally, we observe that the relation \eqref{intro:reg} exhibits a gap of $1/2$ derivative from $H^s$ to $W^{r,\infty}$ in term of Sobolev embedding. To fill up this gap we need to take into account the dispersive property of water waves to prove a Strichartz estimate with a gain of $1/2$ derivative. As remarked in \cite{NgPo} this gain can be achieved for the 3D linearized system (i.e. $d=2$) and corresponds to the {\it semiclassical Strichartz estimate}.  By virtue of Theorems \ref{intro:theo:apriori},  \ref{intro:theo:blowup} and Theorem \ref{theo:contraction} on the Lipschitz continuity of the solution map one would end up with an affirmative answer for ({\bf Q}). Therefore, the problem boils down to studying Strichartz estimates for \eqref{ww}. As a first effort in this direction, we prove in the companion paper \cite{NgPo} Strichartz estimates with an intermediate gain $0<\mu<1/2$ which will yield a Cauchy theory (see also \cite{NgPo}) in which the initial velocity may fail to be Lipschitz (up to the boundary) but becomes Lipschitz at almost all later time; this is an analogue of the result in \cite{ABZ4} for pure gravity waves.\\
{\hspace*{.15in}} The article is organized as follows. Section \ref{section:DN} is devoted for the elliptic estimates needed to study the Dirichlet-Neumann operator: bound estimates and paralinearizations. Next, in Section \ref{section:para} we adapt the method in \cite{ABZ1} to paralinrarize and then symmetrize system \eqref{ww} at our level of regularity \eqref{intro:reg}. Having this reduction, we use the energy method to derive a blow-up criterion and then an a priori estimate  in Section \ref{section:apriori}. Section \ref{section:contraction} is  devoted for contraction estimates, more precisely we establish the Lipschitz continuity of the flow map in spaces of 1-derivative less. Finally, we gather some basic features of the paradifferential calculus theory and technical results in Appendix \ref{Appendix}, most of which comes from \cite{ABZ3, ABZ4}.
\section*{Acknowledgment}
{\hspace*{.15in}} The authors would like to sincerely thank T.Alazard, N.Burq and C.Zuily for many fruitful discussions, suggestions when this work was preparing, as well as their helpful comments at the final stage of the work. Quang Huy Nguyen was partially supported by the labex LMH through the grant no ANR-11-LABX-0056-LMH in the "Programme des Investissements d'Avenir".
\section{Elliptic estimates and the Dirichlet-Neumann operator}\label{section:DN}
\begin{nota}
Throughout this article, for spatial regularity we shall denote for simplicity the Zygmund spaces $C^\sigma_*({\mathbf{R}}^d)~(\sigma\in {\mathbf{R}})$ by $C^\sigma$; while for temporal variable,  $C^k~(k\in {\mathbf{N}})$ are the usual spaces of functions having continuous derivatives up to order $k$.
\end{nota}
\subsection{The elliptic problem}
Let ~$\eta \in W^{1, \infty}({\mathbf{R}}^d)$  and~$ f\in H^{\frac{1}{2}}({\mathbf{R}}^d)$.  It was proved in \cite{ABZ3} that there exists a unique variational solution $\phi$ to the boundary value problem
\begin{equation}
\Delta_{x,y}\phi=0\text{ in }\Omega,\quad
\phi\arrowvert_{\Sigma}=f,\quad \partial_n \phi\arrowvert_{\Gamma}=0.
\end{equation}
Define 
\begin{equation}
{\left\{}\begin{aligned}
   	\Omega_1:=&{\left\{}(x,y):x\in{\bm{\mathrm{R}}}^d,\eta(x)-h<y<\eta(x){\right\}},\\
   	\Omega_2:=&{\left\{}(x,y)\in\mathcal{O}:y\leq\eta(x)-h{\right\}},\\
   	\Omega:=&\Omega_1\cup\Omega_2,
   \end{aligned}
   \right.
\end{equation}
and
\begin{equation}
{\left\{}\begin{aligned}
   	\widetilde{\Omega}_1:=&{\left\{}(x,z):x\in{\bm{\mathrm{R}}}^d,z\in I{\right\}},\quad I=(-1,0),\\
   	\widetilde{\Omega}_2:=&{\left\{}(x,z)\in{\bm{\mathrm{R}}}^d\times(-\infty,-1]:(x,z+1+\eta(x)-h)\in\Omega_2{\right\}},\\
   	\widetilde{\Omega}:=&\widetilde{\Omega}_1\cup\widetilde{\Omega}_2.
   \end{aligned}
   \right.
\end{equation}
To study the regularity of $\phi$, we follow \cite{LannesJAMS}, \cite{ABZ3} straighten out the fluid domain using the map~$(x,z)\mapsto\rho(x,z)$ from~$\widetilde{\Omega}$ to~$\Omega$, defined as
\begin{equation}
{\left\{}\begin{aligned}
	  \rho(x,z):=&(1+z)e^{\delta z{\langle {D_x} \rangle}}\eta(x)-z{\left\{} e^{-(1+z)\delta{\langle {D_x} \rangle}}\eta(x)-h{\right\}}&\text{ if }(x,z)\in\widetilde{\Omega}_1,\\
	  \rho(x,z):=&z+1+\eta(x)-h&\text{ if }(x,z)\in\widetilde{\Omega}_2,
   \end{aligned}
   \right.
\end{equation}
with~$\delta>0$.
It has been proven in~\cite{ABZ3} that if~$\eta\in W^{1,\infty}$, for~$\delta=\delta({\left\Vert}\eta{\right\Vert}_{W^{1,\infty}({\bm{\mathrm{R}}}^d)})$ small enough, the map~$(x,z)\mapsto(x,\rho(x,z))$ is a Lipschitz diffeomorphism from~$\widetilde{\Omega}_1$ to~$\Omega_1$.\\
Introduce for~$\mu\in{\bm{\mathrm{R}}}$ and~$J\subset{\bm{\mathrm{R}}}$ the interpolation spaces
\begin{equation}
\begin{aligned}
	X^\mu(I)&=C^0_z(I;H^\mu({\bm{\mathrm{R}}}^d))\cap L^2_z(I;H^{\mu+{\frac{1}{2}}}({\bm{\mathrm{R}}}^d)),\\
	Y^\mu(I)&=L^1_z(I;H^\mu({\bm{\mathrm{R}}}^d))+L^2_z(I;H^{\mu-{\frac{1}{2}}}({\bm{\mathrm{R}}}^d)).
\end{aligned}
\end{equation}
Remark that $\|\cdot\|_{Y^\mu}{\leq} \|\cdot\|_{X^{\mu-1}}$ for any $\mu\in {\mathbf{R}}$. In these spaces, we have from~\cite{ABZ3} and some easy computations 
\begin{lemm} \label{lem:elldif}
	If~$s>{\frac{1}{2}}+\frac d2$, there exists a positive function~${ \mathcal{F}}$ such that for every~$\eta\in H^{s+{\frac{1}{2}}}({\bm{\mathrm{R}}}^d)$ there holds
	\begin{equation}
		{\left\{}
		\begin{aligned}
			{\left\Vert}\partial_z\rho-h{\right\Vert}_{X^{s-{\frac{1}{2}}}(I)}&\leq{ \mathcal{F}}{\left(}{\left\Vert}\eta{\right\Vert}_{W^{1,\infty}({\bm{\mathrm{R}}}^d)}{\right)}{\left\Vert}\eta{\right\Vert}_{H^{s+{\frac{1}{2}}}({\bm{\mathrm{R}}}^d)},\\
			{\left\Vert}\partial^2_z\rho{\right\Vert}_{X^{s-{\frac{3}{2}}}(I)}&\leq { \mathcal{F}}{\left(}{\left\Vert}\eta{\right\Vert}_{W^{1,\infty}({\bm{\mathrm{R}}}^d)}{\right)}{\left\Vert}\eta{\right\Vert}_{H^{s+{\frac{1}{2}}}({\bm{\mathrm{R}}}^d)},\\
			{\left\Vert}\partial^3_z\rho{\right\Vert}_{X^{s-\frac52}(I)}&\leq { \mathcal{F}}{\left(}{\left\Vert}\eta{\right\Vert}_{W^{1,\infty}({\bm{\mathrm{R}}}^d)}{\right)}{\left\Vert}\eta{\right\Vert}_{H^{s+{\frac{1}{2}}}({\bm{\mathrm{R}}}^d)},\\
			{\left\Vert}\nabla_x\rho{\right\Vert}_{X^{s-{\frac{1}{2}}}(I)}&\leq{ \mathcal{F}}{\left(}{\left\Vert}\eta{\right\Vert}_{W^{1,\infty}({\bm{\mathrm{R}}}^d)}{\right)}{\left\Vert}\eta{\right\Vert}_{H^{s+{\frac{1}{2}}}({\bm{\mathrm{R}}}^d)}.
		\end{aligned}
		\right.
	\end{equation}
\end{lemm}

Then if we take
\begin{equation}\label{defi:v}
v(x,z)=\phi(x,\rho(x,z)),\forall(x,z)\in\widetilde{\Omega},
\end{equation}
the pullback of~$\phi$ by this diffeomorphism, it solves 
\begin{equation}\label{eq:v}	(\partial_z^2+\alpha\Delta_x+\beta\cdot\nabla_x\partial_z-\gamma\partial_z)v=0,
\end{equation}
where
\begin{equation}
	\begin{gathered}
		\alpha:=\frac{(\partial_z\rho)^2}{1+{\left\vert}\nabla_x\rho{\right\vert}^2},
		  \quad\beta:=-2\frac{\partial_z\rho\nabla_x\rho}{1+{\left\vert}\nabla_x\rho{\right\vert}^2},\quad
		\gamma:=\frac{1}{\partial_z\rho}(\partial_z^2\rho+\alpha\Delta_x\rho+\beta\cdot\nabla_x\partial_z\rho).
	\end{gathered}
\end{equation}
We have the following control on those coefficients :
\begin{lemm} \label{lem:estcoefellsob}
	Assume~$s\geq s_0>3/2+d/2$. Then for~$\eta\in H^{s+{\frac{1}{2}}}$, there holds with $X^\mu=X^\mu(I)$:
	\begin{align} \label{eq:estcoefellsob}
		&{\left\Vert}\alpha-h^2{\right\Vert}_{X^{s-{\frac{1}{2}}}}+{\left\Vert}\beta{\right\Vert}_{X^{s-{\frac{1}{2}}}}+{\left\Vert}\gamma{\right\Vert}_{X^{s-{\frac{3}{2}}}}\leq{ \mathcal{F}}{\left(}{\left\Vert}\eta{\right\Vert}_{H^{s_0+{\frac{1}{2}}}}{\right)}
		{\left\Vert}\eta{\right\Vert}_{H^{s+{\frac{1}{2}}}},\\
             &{\left\Vert} \alpha{\right\Vert}_{C^0(I; C^{r-{\frac{1}{2}}})}+{\left\Vert} \beta{\right\Vert}_{C^0(I; C^{r-{\frac{1}{2}}})}+{\left\Vert} \gamma{\right\Vert}_{C^0(I; C^{r-{\frac{3}{2}}})}\leq { \mathcal{F}}{\left(}{\left\Vert}\eta{\right\Vert}_{H^{s_0+{\frac{1}{2}}}}{\right)}
		{\left\Vert}\eta{\right\Vert}_{C^{r+{\frac{1}{2}}}},\\
            &{\left\Vert}\partial_z\alpha{\right\Vert}_{X^{s-{\frac{3}{2}}}}+{\left\Vert}\partial_z\beta{\right\Vert}_{X^{s-{\frac{3}{2}}}}+{\left\Vert}\partial_z\gamma{\right\Vert}_{X^{s-\frac{5}{2}}}\leq{ \mathcal{F}}{\left(}{\left\Vert}\eta{\right\Vert}_{H^{s_0+{\frac{1}{2}}}}{\right)}
		{\left\Vert}\eta{\right\Vert}_{H^{s+{\frac{1}{2}}}}\label{eq:estdercoefellsob}.
	\end{align}
\end{lemm}
	Those are all consequences of the product rules and nonlinear estimates of Proposition~\ref{tame}.
Now a consequence of Proposition~3.16  and the estimate $(3.5)$ of~\cite{ABZ3} is that our solution~$v$ satisfies
\begin{prop} \label{prop:ellregbase}
	Let ~$d\ge 1$,
\[
s_0>1/2+d/2,~-1/2{\leq} \sigma{\leq} s_0-1/2
\]
and $\eta\in H^{s_0+1/2}$. If~$f\in H^{\sigma+1}$, then for any~$z_0\in(-1,0)$, $\nabla_{x,z}v\in X^{\sigma}([z_0,0])$, and
	\begin{equation}
		{\left\Vert}\nabla_{x,z}v{\right\Vert}_{X^{\sigma}([z_0,0])}\leq{ \mathcal{F}}{\left(}{\left\Vert}\eta{\right\Vert}_{H^{s_0+{\frac{1}{2}}}}{\right)} {\left\Vert} f{\right\Vert}_{H^{\sigma+1}},
	\end{equation}
	for some non-decreasing positive function~${ \mathcal{F}}$ depending only on~$s_0$ and $\sigma$. 
\end{prop}
It was deduced from the preceding Proposition the following Sobolev estimate for the Dirichlet-Neumann operator (see Theorem $3.12$, \cite{ABZ3})
\begin{theo}\label{DN:Sobolev}
Let~$d\ge 1$,~$s_0>{\frac{1}{2}}+\frac{d}{2}$ and~$\frac 1 2 \leq \sigma \leq s_0+ \frac 1 2$. 
Then there exists a non-decreasing function~$\mathcal{F}\colon{\mathbf{R}}_+\rightarrow{\mathbf{R}}_+$ 
such that, for all 
$\eta\in H^{s_0+{\frac{1}{2}}}({\mathbf{R}}^d)$ and all~$f\in H^{\sigma}({\mathbf{R}}^d)$, we have 
\begin{equation}
{\left\Vert} G(\eta)f {\right\Vert}_{H^{\sigma-1}({\mathbf{R}}^d)}{\leq} 
\mathcal{F}\bigl(\| \eta \|_{H^{s_0+{\frac{1}{2}}}({\mathbf{R}}^d)}\bigr){\left\Vert} f{\right\Vert}_{H^{\sigma}({\mathbf{R}}^d)}.
\end{equation}
\end{theo}
Since we authorize the control on our quantities to depend non-linearly on the~$H^{s_0}$ norms and only want linearity in the higher order~$H^s$ norm, this means we can use Proposition \ref{prop:ellregbase} as a base case for a bootstrap to control the~$H^s$ and~$C^r$ norms.
We want to prove the following proposition :
\begin{prop} \label{prop:regellsob}
	Let
	$$s\geq s_0>{\frac{3}{2}}+\frac d2,$$
	~$f\in H^{s}$ and~$\eta\in H^{s+{\frac{1}{2}}}$. Then for any~$z_0\in(-1,0)$, $\nabla_{x,z}v\in X^{s-1}([z_0,0])$ and 
	$${\left\Vert}\nabla_{x,z}v{\right\Vert}_{X^{s-1}([z_0,0])}\leq{ \mathcal{F}}{\left(}{\left\Vert}\eta{\right\Vert}_{H^{s_0+{\frac{1}{2}}}},{\left\Vert} f{\right\Vert}_{H^{s_0}}{\right)}{\left[} {\left\Vert} f{\right\Vert}_{H^{s_0}}+{\left\Vert} f{\right\Vert}_{H^s}+{\left\Vert}\eta{\right\Vert}_{H^{s+{\frac{1}{2}}}}{\right]}.$$
	for some non-decreasing positive function~${ \mathcal{F}}$ depending only on~$s_0$ and~$s$.
\end{prop}
The proof will be a simple bootstrap procedure on~$s$. Calling~$\mathcal{H}_s$ the proposition for~$s$, Proposition~\ref{prop:ellregbase} applied with $\sigma=s_0-1$ tells us that~$\mathcal{H}_{s_0}$ is true. We will show that if~$\mathcal{H}_s$ is true, then so is~$\mathcal{H}_{s+{\varepsilon}}$ with
$$0<{\varepsilon}\leq{\frac{1}{2}},\quad{\varepsilon}<s_0-{\frac{3}{2}}-\frac d2.$$
First we paralinearize  equation \eqref{eq:v} of $v$ :
\begin{lemm}\label{para:eq:v}
	There is a function~${ \mathcal{F}}$ such that for all~$I\subset[-1,0]$, $v$ satisfies
\begin{gather*} \partial_z^2v+T_\alpha\Delta_x v+T_\beta\cdot\nabla_x\partial_zv=F:=\gamma\partial_zv+(T_\alpha-\alpha)\Delta_xv+(T_\beta-\beta)\cdot\nabla\partial_zv,\\
	{\left\Vert} F{\right\Vert}_{Y^{s-1+{\varepsilon}}(I)}\leq{ \mathcal{F}}{\left(}{\left\Vert}\eta{\right\Vert}_{H^{s_0+{\frac{1}{2}}}},{\left\Vert} f{\right\Vert}_{H^{s_0}}{\right)}{\left[} {\left\Vert} f{\right\Vert}_{H^{s_0}}+{\left\Vert}\nabla_{x,z}v{\right\Vert}_{X^{s-1}(I)}+{\left\Vert}\eta{\right\Vert}_{H^{s+{\frac{1}{2}}}}{\right]}.
\end{gather*}
\end{lemm}
\begin{proof}
The above expression of $F$ follows directly from equation \eqref{eq:v} satisfied by $v$. Now, using~(\ref{tame:S}) and Hölder inequality in~$z$ we have	$${\left\Vert}\gamma\partial_zv{\right\Vert}_{L^2(I;H^{s-{\frac{3}{2}}+{\varepsilon}})}\lesssim{\left\Vert}\gamma{\right\Vert}_{L^2(I;H^{s_0-1})}{\left\Vert}\partial_zv{\right\Vert}_{L^\infty(I;H^{s-1})}+
	{\left\Vert}\partial_zv{\right\Vert}_{L^\infty(I;H^{s_0-1})}{\left\Vert}\gamma{\right\Vert}_{L^2(I;H^{s-1})},$$
	so that using~(\ref{eq:estcoefellsob}) to control~$\gamma$ and Proposition~\ref{prop:ellregbase} to control~${\left\Vert}\partial_zv{\right\Vert}_{L^\infty(I;H^{s_0-1})}$ gives 
	$${\left\Vert}\gamma\partial_zv{\right\Vert}_{L^2(I;H^{s-{\frac{3}{2}}+{\varepsilon}})}\leq{ \mathcal{F}}{\left(}{\left\Vert}\eta{\right\Vert}_{H^{s_0+{\frac{1}{2}}}},{\left\Vert} f{\right\Vert}_{H^{s_0}}{\right)}{\left[}
	{\left\Vert}\nabla_{x,z}v{\right\Vert}_{X^{s-1}(I)}+{\left\Vert}\eta{\right\Vert}_{H^{s+{\frac{1}{2}}}}{\right]}.$$
	Next by \eqref{boundpara2} we have	$${\left\Vert}(T_\alpha-\alpha)\Delta_xv{\right\Vert}_{L^1(I;H^{s-1+{\varepsilon}})}\lesssim{\left\Vert}\Delta_xv{\right\Vert}_{L^2(I;H^{s_0-{\frac{3}{2}}})}{\left[}1+{\left\Vert}\alpha-h^2{\right\Vert}_{L^2(I;H^s)}{\right]},$$
	so that again we can conclude using~(\ref{eq:estcoefellsob}) and Proposition~\ref{prop:ellregbase}.
	The last remainder term can be controlled identically.
\end{proof}
We then decouple the equation into a forward and a backward parabolic equation :
\begin{lemm} \label{lem:decouplesob}
	There exist two symbols~$a^{(1)},A^{(1)}\in\Gamma^1_{\varepsilon}([-1,0])$ satisfying
	$$\mathcal{M}^1_{1+{\varepsilon}}(a^{(1)})+\mathcal{M}^1_{1+{\varepsilon}}(A^{(1)})\leq{ \mathcal{F}}{\left(}{\left\Vert}\eta{\right\Vert}_{H^{s_0+{\frac{1}{2}}}}{\right)},\quad  \Re(-a^{(1)})+\Re(A^{(1)})\geq c{\left\vert}\xi{\right\vert}$$
	for some constant~$c=c({\left\Vert}\eta{\right\Vert}_{H^{s_0+{\frac{1}{2}}}})>0$, such that
	$$\partial_z^2+T_\alpha\Delta_x+T_\beta\cdot\nabla_x\partial_z=(\partial_z-T_{a^{(1)}})(\partial_z-T_{A^{(1)}})+R,$$
where $R$ is of order $1$ (see Definition \ref{defi:order}) having its norm bounded by ${ \mathcal{F}}({\left\Vert}\eta{\right\Vert}_{H^{s_0+{\frac{1}{2}}}})$. In particular, 
	\begin{equation} \label{est:Rv} {\left\Vert} Rv{\right\Vert}_{Y^{s-1+{\varepsilon}}(I)}\leq{ \mathcal{F}}{\left(}{\left\Vert}\eta{\right\Vert}_{H^{s_0+{\frac{1}{2}}}}{\right)}{\left\Vert}\nabla_{x,z}v{\right\Vert}_{X^{s-1}(I)}.\end{equation}
\end{lemm}
\begin{proof}
Take
\begin{equation}
	\begin{aligned}
		a^{(1)}&={\frac{1}{2}}{\left(}-i\beta\cdot\xi-\sqrt{4\alpha{\left\vert}\xi{\right\vert}^2-(\beta\cdot\xi)^2}{\right)},\\
		A^{(1)}&={\frac{1}{2}}{\left(}-i\beta\cdot\xi+\sqrt{4\alpha{\left\vert}\xi{\right\vert}^2-(\beta\cdot\xi)^2}{\right)}
	\end{aligned}
\end{equation}
so that $a^{(1)}+A^{(1)}=-i\beta \cdot \xi, \quad a^{(1)}A^{(1)}=-\alpha|\xi|^2.$\\
Then the control of the semi-norm of $a^{(1)}$ and $A^{(1)}$ is a consequence of the boundedness of the coefficients  $\alpha, \beta$ from~(\ref{eq:estcoefellsob}).
From the expressions of~$\alpha$ and~$\beta$, and the fact that ~$|\partial_z\rho| \ge c_0>0$, we get
$$\exists c>0, \sqrt{4\alpha{\left\vert}\xi{\right\vert}^2-(\beta\cdot\xi)^2}\geq c{\left\vert}\xi{\right\vert},$$
which gives the ellipticity.
At last, $R=(T_{a^{(1)}}T_{A^{(1)}}-T_\alpha\Delta_x)-T_{\partial_z{A^{(1)}}}.$
The first difference is of order $2-(1+{\varepsilon})=1-{\varepsilon}$ by Theorem \ref{theo:sc} $(ii)$, and  the second term~$\partial_zA^{(1)}\in\Gamma^1_{\varepsilon}$ by~(\ref{eq:estdercoefellsob}). Consequently, the remainder $R$ has order $1$ and \eqref{est:Rv} follows. Here, we can replace~${\left\Vert} v{\right\Vert}_{H^{s+{\frac{1}{2}}}}$ by~${\left\Vert}\nabla v{\right\Vert}_{H^{s-{\frac{1}{2}}}}$ since the paradifferential operator $T_p$ can be repalced by~$T_p(1-\Psi(D_x))$, for a low frequency cutoff~$\Psi$, at no cost.
\end{proof}
To conclude the proof of Proposition~\ref{prop:regellsob}, we want to apply Theorem~\ref{regularitysob} two times.
Take~$0>z_1>z_0>-1$. Since~$\mathcal{H}_s$ is true, there holds
$${\left\Vert}\nabla_{x,z}v{\right\Vert}_{X^{s-1}([z_0,0])}\leq{ \mathcal{F}}{\left(}{\left\Vert}\eta{\right\Vert}_{H^{s_0+{\frac{1}{2}}}},{\left\Vert} f{\right\Vert}_{H^{s_0}}{\right)}{\left[} {\left\Vert} f{\right\Vert}_{H^{s_0}}+{\left\Vert} f{\right\Vert}_{H^s}+{\left\Vert}\eta{\right\Vert}_{H^{s+{\frac{1}{2}}}}{\right]}.$$
We will prove
$${\left\Vert}\nabla_{x,z}v{\right\Vert}_{X^{s-1+{\varepsilon}}([z_1,0])}\leq{ \mathcal{F}}{\left(}{\left\Vert}\eta{\right\Vert}_{H^{s_0+{\frac{1}{2}}}},{\left\Vert} f{\right\Vert}_{H^{s_0}}{\right)}{\left[} {\left\Vert} f{\right\Vert}_{H^{s_0}}+{\left\Vert} f{\right\Vert}_{H^{s+{\varepsilon}}}+{\left\Vert}\eta{\right\Vert}_{H^{s+{\frac{1}{2}}+{\varepsilon}}}{\right]}.$$
Since~$z_0$ and~$z_1$ are arbitrary, this will complete the proof.

We now introduce a cutoff~$\chi$ satisfying~$\chi\rvert_{z<z_0}=0$,  $\chi\rvert_{z>z_1}=1$, and set~$w=\chi(z)(\partial_z-T_{A^{(1)}})v$. From Lemma~\ref{lem:decouplesob} we have~$\partial_zw-T_{a^{(1)}}w=F'$, with
\begin{equation} F'=\chi(z)(F+Rv)+\chi'(z)(\partial_z-T_{A^{(1)}})v.\end{equation}
We have the trivial estimate
$${\left\Vert}\chi'(z)(\partial_z-T_A)v{\right\Vert}_{Y^{s-1+{\varepsilon}}([z_0,0])}\leq{ \mathcal{F}}{\left(}{\left\Vert}\eta{\right\Vert}_{H^{s_0+{\frac{1}{2}}}}{\right)}{\left\Vert}\nabla_{x,z}v{\right\Vert}_{X^{s-1}([z_0,0])}.$$
Together with the preceding lemmas, we obtain that
$${\left\Vert} F'{\right\Vert}_{Y^{s-1+{\varepsilon}}([z_0,0])}\leq{ \mathcal{F}}{\left(}{\left\Vert}\eta{\right\Vert}_{H^{s_0+{\frac{1}{2}}}},{\left\Vert} f{\right\Vert}_{H^{s_0}}{\right)}{\left[}{\left\Vert} f{\right\Vert}_{H^{s_0}}+{\left\Vert} f{\right\Vert}_{H^s}+{\left\Vert}\eta{\right\Vert}_{H^{s+{\frac{1}{2}}}}{\right]}.$$
Since~$w(z_0)=0$ and~$\Re(-a)\geq c{\left\vert}\xi{\right\vert}$, Theorem~\ref{regularitysob} implies
\begin{equation}\label{est:w:indu}{\left\Vert} w{\right\Vert}_{X^{s-1+{\varepsilon}}([z_0,0])}\leq{ \mathcal{F}}{\left(}{\left\Vert}\eta{\right\Vert}_{H^{s_0+{\frac{1}{2}}}},{\left\Vert} f{\right\Vert}_{H^{s_0}}{\right)}{\left[}{\left\Vert} f{\right\Vert}_{H^{s_0}}+{\left\Vert} f{\right\Vert}_{H^s}+{\left\Vert}\eta{\right\Vert}_{H^{s+{\frac{1}{2}}}}{\right]}.\end{equation}
Consequently,
$${\left\Vert} w{\right\Vert}_{Y^{s+{\varepsilon}}([z_0,0])}\leq{ \mathcal{F}}{\left(}{\left\Vert}\eta{\right\Vert}_{H^{s_0+{\frac{1}{2}}}},{\left\Vert} f{\right\Vert}_{H^{s_0}}{\right)}{\left[}{\left\Vert} f{\right\Vert}_{H^{s_0}}+{\left\Vert} f{\right\Vert}_{H^s}+{\left\Vert}\eta{\right\Vert}_{H^{s+{\frac{1}{2}}}}{\right]}.$$
Then because~$\chi=1$ on~$[z_1,0]$, $\partial_zv-T_{A^{(1)}}v=w \quad\text{for }z\in [z_1, 0].$
At last, applying again Theorem~\ref{regularitysob} with $v(0)=f$, after inversing~$z$ into~$-z$, we obtain
$${\left\Vert} v{\right\Vert}_{X^{s+{\varepsilon}}([z_1,0])}\leq{ \mathcal{F}}{\left(}{\left\Vert}\eta{\right\Vert}_{H^{s_0+{\frac{1}{2}}}},{\left\Vert} f{\right\Vert}_{H^{s_0}}{\right)}{\left[}{\left\Vert} f{\right\Vert}_{H^{s_0}}+{\left\Vert} f{\right\Vert}_{H^{s+{\varepsilon}}}+{\left\Vert}\eta{\right\Vert}_{H^{s+{\frac{1}{2}}+{\varepsilon}}}{\right]}.$$
Using the relation~$\partial_zv=T_{A^{(1)}}v+w$ and take into account the estimate \eqref{est:w:indu} we can finally estimate~$\nabla_{x,z}v$ as claimed.

Next, we prove a H\"older estimate for $\nabla_{x,z}v$.
\begin{prop} \label{prop:regellhol}
	Let
	$$s_0>{\frac{3}{2}}+\frac d2,\quad r<s_0-\frac d2+{\frac{1}{2}},\quad 1{\leq} \mu{\leq} \frac 52,$$
	and~$f\in H^{s_0}\cap C^{r}$,~$\eta\in H^{s_0+{\frac{1}{2}}}\cap C^{r+{\frac{1}{2}}}$. Then for any~$z_0\in(-1,0)$, we have
	$${\left\Vert}\nabla_{x,z}v{\right\Vert}_{C^0([z_0,0];C^{r-\mu})}\leq{ \mathcal{F}}{\left(}{\left\Vert}\eta{\right\Vert}_{H^{s_0+{\frac{1}{2}}}},{\left\Vert} f{\right\Vert}_{H^{s_0}}{\right)}{\left(} {\left\Vert} f{\right\Vert}_{H^{s_0-\mu+1}}+{\left\Vert} f{\right\Vert}_{C^{r-\mu+1}}{\right)}.$$
	for some non-decreasing positive function~${ \mathcal{F}}$ depending only on~$s_0$ and~$r$.
\end{prop}
\begin{proof}
Similar to the proof above, we take $-1<z_0<z_1<0$, introduce a cutoff~$\chi$ satisfying~$\chi\rvert_{z<z_0}=0$,  $\chi\rvert_{z>z_1}=1$ and set~$w=\chi(z)(\partial_z-T_{A^{(1)}})v$. We use the estimate $(3.56)$  in \cite{ABZ3}: for
\[
0{\leq} {\varepsilon}{\leq} {\frac{1}{2}},~{\varepsilon}<s_0-{\frac{1}{2}}-\frac d2,~-{\frac{1}{2}}{\leq} \sigma{\leq} s_0-{\frac{1}{2}}-{\varepsilon}
\]
it holds that
\[
{\left\Vert}  w{\right\Vert}_{X^{\sigma+{\varepsilon}}([z_0, 0])}{\leq} \mathcal{F}(\| \eta \|_{H^{s_0+{\frac{1}{2}}}}){\left\Vert}  \nabla_{x,z} v{\right\Vert}_{X^\sigma([z_0, 0])}.
\]  
Then, applying this inequality with ${\varepsilon}=1/2,~\sigma=s_0-\mu$ gives
\[
{\left\Vert}  w{\right\Vert}_{X^{s_0-\mu+{\frac{1}{2}}}([z_0, 0])}
{\leq} \mathcal{F}(\| \eta \|_{H^{s_0+{\frac{1}{2}}}}){\left\Vert}  \nabla_{x,z} v{\right\Vert}_{X^{s_0-\mu}([z_0, 0])}.
\]
On the other hand, it follows from Proposition \ref{prop:ellregbase} that
\begin{equation}\label{apriori:v}
{\left\Vert}  \nabla_{x,z} v{\right\Vert}_{X^{s_0-\mu}([z_0, 0])}
{\leq} \mathcal{F}(\| \eta \|_{H^{s_0+{\frac{1}{2}}}}){\left\Vert} f{\right\Vert}_{H^{s_0-\mu+1}};
\end{equation}
consequently,
\begin{equation}\label{apriori:w}
{\left\Vert}  w{\right\Vert}_{X^{s_0-\mu+{\frac{1}{2}}}([z_0, 0])}{\leq} \mathcal{F}(\| \eta \|_{H^{s_0+{\frac{1}{2}}}}){\left\Vert} f{\right\Vert}_{H^{s_0-\mu+1}}.
\end{equation}
Now, on $[z_1, 0]$~$(\partial_z - T_{A^{(1)}})v =w$ so after inversing $z$ to $-z$ one can apply Theorem \ref{regularityhold}with $r_1=r-\mu+1,~r_0<r_1,~q=\infty$ to get with $J=[z_1, 0]$
\[
{\left\Vert} v{\right\Vert}_{C(J; C^{r-\mu+1})}{\leq} \mathcal{F}(\| \eta \|_{H^{s_0+{\frac{1}{2}}}}){\left(} {\left\Vert} f{\right\Vert}_{C^{r-\mu+1}}+{\left\Vert} w{\right\Vert}_{L^\infty(J; C^{r-\mu+\delta})} +{\left\Vert} v{\right\Vert}_{L^\infty(J; C^{r_0})} {\right)}.
\]
Using \eqref{apriori:w}, Sobolev's embedding and the relation between $r$ and $s_0$, one deduces
\[
{\left\Vert} w{\right\Vert}_{L^\infty(J; C^{r-\mu+\delta})}{\lesssim} {\left\Vert} w{\right\Vert}_{L^\infty(J; H^{s_0-\mu+{\frac{1}{2}}})} {\leq} \mathcal{F}(\| \eta \|_{H^{s_0+{\frac{1}{2}}}}){\left\Vert} f{\right\Vert}_{H^{s_0-\mu+1}},
\]
where we have taken $0<\delta<s_0+1/2-d/2-r$. Finally, for the last term on the right-hand side, one chooses $r_0$ small enough so that the desired estimate can be deduced from \eqref{apriori:v} via Sobolev embeddings.
\end{proof}
\subsection{Dirichlet-Neumann operator}
We now apply the elliptic estimates in the previous paragraph to derive estimates for the Dirichlet-Neumann operator.
\begin{prop}	\label{prop:DNreg}
	Let~$d\geq1$, and 
	$$s\geq s_0>{\frac{3}{2}}+\frac d2,\quad 1{\leq} \mu{\leq} \frac 52,\quad \frac{\mu+1}{2}<r<s_0-\frac d2+{\frac{1}{2}}.$$
	Then there exists a positive nondecreasing function~${ \mathcal{F}}$ 
	such that 
	\begin{align}
		
		{\left\Vert} G(\eta)f{\right\Vert}_{H^{s-1}}&\leq{ \mathcal{F}}{\left(}{\left\Vert}\eta{\right\Vert}_{H^{s_0+{\frac{1}{2}}}},{\left\Vert} f{\right\Vert}_{H^{s_0}}{\right)}{\left[} {\left\Vert} f{\right\Vert}_{H^{s_0}}+{\left\Vert} f{\right\Vert}_{H^s}+{\left\Vert}\eta{\right\Vert}_{H^{s+{\frac{1}{2}}}}{\right]},\\
		{\left\Vert} G(\eta)f{\right\Vert}_{C^{r-\mu}}&\leq{ \mathcal{F}}{\left(}{\left\Vert}\eta{\right\Vert}_{H^{s_0+{\frac{1}{2}}}}{\right)}{\left[} {\left\Vert} f{\right\Vert}_{H^{s_0-\mu+1}}+{\left\Vert} f{\right\Vert}_{C^{r-\mu+1}}{\right]}\label{est:DN:Holder}.
		
	\end{align}
\end{prop}
\begin{proof}
	By definition the Dirichlet-Neumann operator is given by
	$$G(\eta)f=\frac{1+{\left\vert}\nabla_x\rho{\right\vert}^2}{\partial_z\rho}\partial_zv-\nabla_x\rho\cdot\nabla_xv\rvert_{z=0}.$$
	Thus the result is a consequence of Propositions~\ref{prop:regellsob}, \ref{prop:regellhol}, of the estimations on~$\rho$ of Lemma~\ref{lem:elldif}, and of the product and nonlinear estimates of Proposition~\ref{tame}. Here, we need to take some care for the second estimate.\\
 1.  If $r-\mu\ge 0$, the rule \eqref{tame:H} implies at $z=0$
\[
{\left\Vert} \nabla_x\rho\cdot\nabla_xv{\right\Vert}_{C^{r-\mu}}{\leq} {\left\Vert} \nabla_x\rho{\right\Vert}_{C^{r-\mu}}{\left\Vert} \nabla_xv{\right\Vert}_{C^{r-\mu}}{\leq} {\left\Vert} \nabla_x\rho{\right\Vert}_{H^{s_0-\mu+{\frac{1}{2}}}}{\left\Vert} \nabla_xv{\right\Vert}_{C^{r-\mu}}
\]
Then since $s_0-\mu+1/2{\leq} s_0-1/2$ and 
\[
{\left\Vert} \nabla_x\rho{\right\Vert}_{H^{s_0-{\frac{1}{2}}}}{\leq} { \mathcal{F}}{\left(}{\left\Vert} \eta{\right\Vert}_{H^{s_0+{\frac{1}{2}}}}{\right)}
\]
the right-hand side is bounded as claimed by virtue of Proposition  \ref{prop:regellhol}.\\
2. If $r-\mu<0$ one applies \eqref{tame:H<0} with $\alpha:=\mu-r<\beta:=r-1$ to get (at $z=0$) 
\[
{\left\Vert} \nabla_x\rho\cdot\nabla_xv{\right\Vert}_{C^{r-\mu}}{\leq} {\left\Vert} \nabla_x\rho{\right\Vert}_{C^{r-1}}{\left\Vert} \nabla_xv{\right\Vert}_{C^{r-\mu}}{\leq} {\left\Vert} \nabla_x\rho{\right\Vert}_{H^{s_0-{\frac{1}{2}}}}{\left\Vert} \nabla_xv{\right\Vert}_{C^{r-\mu}}
\]
from which \eqref{est:DN:Holder} follows.\\
The first term in the expression of $G(\eta)f$ is treated in the same way by writing
\[
\frac{1+{\left\vert}\nabla_x\rho{\right\vert}^2}{\partial_z\rho}\partial_zv=\Big(\frac{1+{\left\vert}\nabla_x\rho{\right\vert}^2}{\partial_z\rho}-\frac 1h \Big)\partial_zv+\frac 1h\partial_zv.
\]
\end{proof}
Recall the expression of the trace of the  velocity at the free surface
\[
B = \frac{ \nabla \eta \cdot \nabla \psi + G(\eta)\psi} {1+ \vert \nabla \eta \vert^2},\quad V= \nabla \psi - B \nabla\eta.
\]
As a consequence, we have the following estimates on~$V$ and~$B$.
\begin{coro}	\label{cor:BV}
	Let~$d\geq1$, and
	$$s\geq s_0>{\frac{3}{2}}+\frac d2,\quad2<r<s_0-\frac d2+{\frac{1}{2}}.$$
	Then there exists a positive nondecreasing function~${ \mathcal{F}}$ 
	such that 
		\begin{align}
		{\left\Vert} (B,V){\right\Vert}_{H^{s_0-1}\times H^{s_0-1}}&\leq{ \mathcal{F}}{\left(}{\left\Vert}\eta{\right\Vert}_{H^{s_0+{\frac{1}{2}}}},{\left\Vert}\psi{\right\Vert}_{H^{s_0}}{\right)},\\
		{\left\Vert} (B,V){\right\Vert}_{H^{s-1}\times H^{s-1}}&\leq{ \mathcal{F}}{\left(}{\left\Vert}\eta{\right\Vert}_{H^{s_0+{\frac{1}{2}}}},{\left\Vert}\psi{\right\Vert}_{H^{s_0}}{\right)}{\left[}1+{\left\Vert}\psi{\right\Vert}_{H^s}+{\left\Vert}\eta{\right\Vert}_{H^{s+{\frac{1}{2}}}}{\right]},\\
		{\left\Vert} (B,V){\right\Vert}_{C^{r-1}\times C^{r-1}}&\leq{ \mathcal{F}}{\left(}{\left\Vert}\eta{\right\Vert}_{H^{s_0+{\frac{1}{2}}}},{\left\Vert}\psi{\right\Vert}_{H^{s_0}}{\right)}{\left[}1+{\left\Vert}\psi{\right\Vert}_{C^r}+{\left\Vert}\eta{\right\Vert}_{C^{r+{\frac{1}{2}}}}{\right]}.
	\end{align}
\end{coro}
\begin{proof}
We only need to prove estimates for $B$, then those for $V$ will follow immediately. This is done by decomposing $B$ as
\begin{equation} \label{decompose:B} B=\frac{\nabla\eta}{1+{\left\vert}\nabla\eta{\right\vert}^2}\cdot\nabla\psi+\frac1{1+{\left\vert}\nabla\eta{\right\vert}^2}G(\eta)\psi=:K(\nabla\eta)\cdot\nabla\psi+L(\nabla\eta)G(\eta)\psi,
\end{equation}
	with~$K$ and~$L$ smooth.
	The first estimate is a consequence of Theorem \ref{DN:Sobolev} and the fact that~$H^{s_0-1}$ is an algebra since~$s_0>{\frac{3}{2}}+\frac d2$.
	
	For the second and the third, we use estimates~(\ref{F(u):C}), (\ref{F(u):H}),(\ref{tame:H}), (\ref{tame:H}),
	and Proposition \ref{prop:DNreg}.
\end{proof}
We also prove, following~\cite{AM}, that the Dirichlet-Neumann operator can be paralinearized. We show that it is possible to obtain tame estimates on the remainder.
Define
\begin{equation} \label{eq:lambda}
\lambda:=\lambda^{(1)}+\lambda^{(0)}
\end{equation}
a symbol with an order one part 
\begin{equation} 
\lambda^{(1)}:=\sqrt{(1+{\left\vert}\nabla\eta{\right\vert}^2){\left\vert}\xi{\right\vert}^2-(\nabla\eta\cdot\xi)^2},
\end{equation}
and an order zero part
\begin{equation}		
\lambda^{(0)}:=\frac{1+{\left\vert}\nabla\eta{\right\vert}^2}{2\lambda^{(1)}}{\left[}\operatorname{div}(\alpha^{(1)}\nabla\eta)+i\partial_\xi\lambda^{(1)}\cdot\nabla\alpha^{(1)}{\right]},~~\alpha^{(1)}:=\frac{1}{1+{\left\vert}\nabla\eta{\right\vert}^2}(\lambda^{(1)}+i\nabla\eta\cdot\xi).
\end{equation}
\begin{prop}	\label{prop:paradir}
	Let~$d\geq1$, and
	$$s\geq s_0>{\frac{3}{2}}+\frac d2,~r>2.$$
	Then there exists a positive nondecreasing function~${ \mathcal{F}}$ 
	such that 
	for
\[
(\eta,\psi)\in {\left(} H^{s+{\frac{1}{2}}}\times H^s {\right)}\cap {\left(} C^{r+{\frac{1}{2}}}\times C^r{\right)},
\] there holds
	$$G(\eta)\psi=T_\lambda(\psi-T_B\eta)-T_V\cdot\nabla\eta+f(\eta,\psi),$$
	with
	\begin{equation}\label{para:DN:main}{\left\Vert} f(\eta,\psi){\right\Vert}_{H^{s+{\frac{1}{2}}}}\leq{ \mathcal{F}}{\left(}{\left\Vert}\eta{\right\Vert}_{H^{s_0+{\frac{1}{2}}}},{\left\Vert}\psi{\right\Vert}_{H^{s_0}}{\right)}{\left[}1+{\left\Vert}\psi{\right\Vert}_{C^r}+{\left\Vert}\eta{\right\Vert}_{C^{r+{\frac{1}{2}}}}{\right]}{\left[}1+{\left\Vert}\psi{\right\Vert}_{H^s}+{\left\Vert}\eta{\right\Vert}_{H^{s+{\frac{1}{2}}}}{\right]}.\end{equation}
\end{prop}
The rest of this section is devoted to the proof of this Proposition. Recall that in the preceding section we have straightened the domain using the diffeomorphism~$\rho$ to obtain from $\phi$ (the potential velocity) a new unknown~$v$ satisfying
$$(\partial_z^2+\alpha\Delta_x+\beta\cdot\nabla_x\partial_z-\gamma\partial_z)v=0.$$
We then established in Proposition \ref{prop:regellsob} that
$${\left\Vert}\nabla_{x,z}v{\right\Vert}_{X^{s-1}([z_0,0])}\leq{ \mathcal{F}}{\left(}{\left\Vert}\eta{\right\Vert}_{H^{s_0+{\frac{1}{2}}}},{\left\Vert}\psi{\right\Vert}_{H^{s_0}}{\right)}{\left[}1+{\left\Vert}\psi{\right\Vert}_{H^s}+{\left\Vert}\eta{\right\Vert}_{H^{s+{\frac{1}{2}}}}{\right]}.$$
Now using the above equation on $v$, the estimates on its coefficients and their $z$-derivatives from Lemma~\ref{lem:estcoefellsob} one gets
$${\left\Vert}\partial^2_zv{\right\Vert}_{X^{s-2}([z_0,0])}+{\left\Vert}\partial_z^3 v{\right\Vert}_{X^{s-3}([z_0,0])}\leq{ \mathcal{F}}{\left(}{\left\Vert}\eta{\right\Vert}_{H^{s_0+{\frac{1}{2}}}},{\left\Vert}\psi{\right\Vert}_{H^{s_0}}{\right)}{\left[}1+{\left\Vert}\psi{\right\Vert}_{H^s}+{\left\Vert}\eta{\right\Vert}_{H^{s+{\frac{1}{2}}}}{\right]}.$$
On the other hand, by Proposition~\ref{prop:regellhol} we have
$${\left\Vert}\nabla_{x,z}v{\right\Vert}_{C^0([z_0,0];C^{r-1})}\leq{ \mathcal{F}}{\left(}{\left\Vert}\eta{\right\Vert}_{H^{s_0+{\frac{1}{2}}}},{\left\Vert}\psi{\right\Vert}_{H^{s_0}}{\right)}{\left[}1+{\left\Vert}\psi{\right\Vert}_{C^r}+{\left\Vert}\eta{\right\Vert}_{C^{r+{\frac{1}{2}}}}{\right]},$$
and again with product rules,
$${\left\Vert}\partial^2_zv{\right\Vert}_{C^0([z_0,0];C^{r-2})}+{\left\Vert}\partial_{z}^3v{\right\Vert}_{C^0([z_0,0];C^{r-3})}\leq{ \mathcal{F}}{\left(}{\left\Vert}\eta{\right\Vert}_{H^{s_0+{\frac{1}{2}}}},{\left\Vert}\psi{\right\Vert}_{H^{s_0}}{\right)}{\left[}1+{\left\Vert}\psi{\right\Vert}_{C^r}+{\left\Vert}\eta{\right\Vert}_{C^{r+{\frac{1}{2}}}}{\right]}.
$$
The result we want to prove is linked to the so-called good unknown of Alinhac (cf~\cite{Alipara,AliXEDP}): we introduce
\begin{equation}\label{def:b,u}
b:=\frac{\partial_zv}{\partial_z\rho},\quad\text{and }u:=v-T_b\rho,
\end{equation}
so that 
$$b\vert_{z=0}=B,\quad u\rvert_{z=0}=\psi-T_B\eta.$$
The interest of the good unknown is that we expect it to satisfy a better paradifferential equation than~$v$ itself. Indeed, we have the following lemma.
\begin{lemm}\label{para:eq:u}
	The good unknown~$u=v-T_b\rho$ satisfies the equation
	\begin{gather} \partial^2_zu+T_\alpha\Delta_xu+T_\beta\cdot\nabla_x\partial_zu-T_\gamma\partial_zu=f,\nonumber\\
{\left\Vert} f{\right\Vert}_{Y^{s+{\frac{1}{2}}}([-1,0])}\leq{ \mathcal{F}}{\left(}{\left\Vert}\eta{\right\Vert}_{H^{s_0+{\frac{1}{2}}}},{\left\Vert}\psi{\right\Vert}_{H^{s_0}}{\right)}{\left[}1+{\left\Vert}\psi{\right\Vert}_{C^r}+{\left\Vert}\eta{\right\Vert}_{C^{r+{\frac{1}{2}}}}{\right]}{\left[}1+{\left\Vert}\psi{\right\Vert}_{H^s}+{\left\Vert}\eta{\right\Vert}_{H^{s+{\frac{1}{2}}}}{\right]}.
\label{eq:remali}
	\end{gather}
\end{lemm}
\begin{proof}
	To simplify the proof, we will write ~$f_1\sim f_2$ iff ~${\left\Vert} f_1-f_2{\right\Vert}_{Y^{s+{\frac{1}{2}}}([-1,0])}$ is bounded by the right-hand side of~(\ref{eq:remali}). In particular, $f_1\sim f_2$ if ${\left\Vert} f_1-f_2{\right\Vert}_{X^{s-{\frac{1}{2}}}([-1,0])}$ is bounded by the right-hand side of~(\ref{eq:remali}).
	Introduce
	\begin{gather*}
		E:=\partial^2_z+\alpha\Delta +\beta\cdot\nabla \partial_z-\gamma\partial_z,\quad
		P:=\partial^2_z+T_\alpha\Delta +T_\beta\cdot\nabla \partial_z-T_\gamma\partial_z.
	\end{gather*}
	We have $Ev=0$. By decomposing each term in $Ev$ with the Bony decomposition, using the estimates \eqref{esti:quant1},~\eqref{Bony2}, the previous estimates on~$\partial_zv,~\partial^2_zv$, and the estimates on the coefficients~(\ref{eq:estcoefellsob}) we obtain
	$$0=Ev\sim Pv-T_{\partial_zv}\gamma,$$ 
	which gives since~$v=u+T_b\rho$ that
	$$Pu+PT_b\rho-T_{\partial_zv}\gamma\sim0.$$ 
	The proof then  boils down to showing that
	\begin{equation}\label{DN:key} PT_b\rho-T_{\partial_zv}\gamma\sim0.\end{equation} 
	Now 
	$$PT_b\rho=\partial^2_zT_b\rho+T_\alpha\Delta T_b\rho+T_\beta\cdot\nabla \partial_zT_b\rho-T_\gamma\partial_zT_b\rho.$$
	Using Leibniz rule for the $z$-derivatives and neglecting the terms $\sim 0$ it holds that
	\[
	PT_b\rho\sim T_b\partial^2_z\rho+T_\alpha T_b\Delta \rho+T_\beta\cdot T_b\nabla \partial_z\rho.
	\]
	To suppress the terms in~$\gamma$, we use  $T_bE\rho=T_b0=0,$ which implies
	$$
		T_b\partial^2_z\rho+T_bT_\alpha\Delta \rho+T_bT_\beta\cdot\nabla \partial_z\rho-T_bT_{\partial_z\rho}\gamma\sim0.
	$$
	Now again by the symbolic calculus, we get
	$$[T_b,T_\alpha]\Delta \rho+[T_b,T_\beta]\cdot\nabla \partial_z\rho\sim0$$
	and~$T_{\partial_zv}\gamma=T_{b\partial_z\rho}\gamma\sim T_bT_{\partial_z\rho}\gamma$, hence we obtain \eqref{DN:key}.
\end{proof}
The next step of the proof is again to decouple between a forward and a backward parabolic equations, using a refinement of Lemma~\ref{lem:decouplesob}.
\begin{lemm}\label{dep:parabolics}
	For~$0{\leq} {\varepsilon}<\min{\left(} r-2,{\frac{1}{2}}{\right)}$,
	there exist two symbols~$a$ and~$A$ satisfying
	$$\Re(-a)+\Re(A)\geq c{\left\vert}\xi{\right\vert}$$
	for a constant~$c({\left\Vert}\eta{\right\Vert}_{H^{s_0+{\frac{1}{2}}}})>0$, such that
	$$\partial_z^2+T_\alpha\Delta_x+T_\beta\cdot\nabla_x\partial_z-T_\gamma\partial_z=(\partial_z-T_a)(\partial_z-T_A)+R,$$
	where $R$ is of order $1/2-{\varepsilon}$. In particular, for any $z_0\in (-1, 0)$ we have
	$${\left\Vert} Ru{\right\Vert}_{Y^{s+{\frac{1}{2}}}([z_0, 0])}\leq{ \mathcal{F}}{\left(}{\left\Vert}\eta{\right\Vert}_{H^{s_0+{\frac{1}{2}}}},{\left\Vert}\psi{\right\Vert}_{H^{s_0}}{\right)}{\left[}1+{\left\Vert}\eta{\right\Vert}_{C^{r+{\frac{1}{2}}}}{\right]}
	{\left[}1+{\left\Vert}\psi{\right\Vert}_{H^s}+{\left\Vert}\eta{\right\Vert}_{H^{s+{\frac{1}{2}}}}{\right]}.$$
\end{lemm}
\begin{proof}
	We  look for symbols of the following form:
	\begin{gather*}		a=a^{(1)}+a^{(0)}\in\dot{\Gamma}^1_{{\frac{3}{2}}+{\varepsilon}}+\dot{\Gamma}^0_{{\frac{1}{2}}+{\varepsilon}},\quad A=A^{(1)}+A^{(0)}\in\dot{\Gamma}^1_{{\frac{3}{2}}+{\varepsilon}}+\dot{\Gamma}^0_{{\frac{1}{2}}+{\varepsilon}}.
	\end{gather*}
	We already found
	\begin{equation*}
	\begin{aligned}
		a^{(1)}&={\frac{1}{2}}\Big(-i\beta\cdot\xi-\sqrt{4\alpha{\left\vert}\xi{\right\vert}^2-(\beta\cdot\xi)^2}\Big),\\
		A^{(1)}&={\frac{1}{2}}\Big(-i\beta\cdot\xi+\sqrt{4\alpha{\left\vert}\xi{\right\vert}^2-(\beta\cdot\xi)^2}\Big),
	\end{aligned}
	\end{equation*}
which satisfy
\begin{align*}
&M^1_{{\frac{3}{2}}+{\varepsilon}}{\left(} A^{(1)}(z){\right)}+M^1_{{\frac{3}{2}}+{\varepsilon}}{\left(} A^{(1)}(z){\right)} {\leq} { \mathcal{F}}{\left(}{\left\Vert} \eta{\right\Vert}_{H^{s_0+{\frac{1}{2}}}}{\right)}{\left[} 1+{\left\Vert} \eta{\right\Vert}_{C^{r+{\frac{1}{2}}}}{\right]},\\
 &M^1_0{\left(} A^{(1)}(z){\right)}+M^1_0{\left(} A^{(1)}(z){\right)} {\leq} { \mathcal{F}}{\left(}{\left\Vert} \eta{\right\Vert}_{H^{s_0+{\frac{1}{2}}}}{\right)}.
\end{align*}
Then we take
	\begin{equation*}
	\begin{aligned}
		a^{(0)}&=\frac1{A^{(1)}-a^{(1)}}{\left(} i\partial_\xi a^{(1)}\partial_xA^{(1)}-\gamma a^{(1)}{\right)},\\
		A^{(0)}&=\frac1{a^{(1)}-A^{(1)}}{\left(} i\partial_\xi a^{(1)}\partial_xA^{(1)}-\gamma A^{(1)}{\right)}
	\end{aligned}
	\end{equation*}
so that
\begin{gather*}
		a^{(1)}A^{(1)}+\frac1i\partial_\xi a^{(1)}\cdot\partial_xA^{(1)}+a^{(1)}A^{(0)}+a^{(0)}A^{(1)}=-\alpha{\left\vert}\xi{\right\vert}^2,
		\quad a+A=-i\beta\cdot\xi+\gamma.
	\end{gather*}
	We can easily verify that 
\begin{align*}
&M^0_{{\frac{1}{2}}+{\varepsilon}}{\left(} A^{(0)}(z){\right)}+M^1_{{\frac{1}{2}}+{\varepsilon}}{\left(} A^{(0)}(z){\right)} {\leq} { \mathcal{F}}{\left(}{\left\Vert} \eta{\right\Vert}_{H^{s_0+{\frac{1}{2}}}}{\right)}{\left[} 1+{\left\Vert} \eta{\right\Vert}_{C^{r+{\frac{1}{2}}}}{\right]},\\
 &M^0_0{\left(} A^{(0)}(z){\right)}+M^0_0{\left(} A^{(0)}(z){\right)} {\leq} { \mathcal{F}}{\left(}{\left\Vert} \eta{\right\Vert}_{H^{s_0+{\frac{1}{2}}}}{\right)}.
\end{align*}
 The remainder will be 
	$$R={\left(} T_aT_A-T_\alpha\Delta{\right)}+{\left(}(T_a+T_A)+(T_\beta\cdot\nabla-T_\gamma){\right)}\partial_z=T_aT_A-T_\alpha\Delta.$$
Using the symbolic calculus we obtain that $R$ is of order ${\frac{1}{2}}-{\varepsilon}$, hence by virtue of Proposition \ref{prop:regellsob} we conclude 
\begin{multline*}
{\left\Vert} Ru{\right\Vert}_{Y^{s+{\frac{1}{2}}}}{\lesssim} {\left\Vert} Ru{\right\Vert}_{L^2H^s}{\leq} { \mathcal{F}}{\left(}{\left\Vert} \eta{\right\Vert}_{H^{s_0+{\frac{1}{2}}}}{\right)}{\left[} 1+{\left\Vert} \eta{\right\Vert}_{C^{r+{\frac{1}{2}}}}{\right]} {\left\Vert} \nabla u{\right\Vert}_{L^2H^{s-{\frac{1}{2}}}}\\
{\leq} { \mathcal{F}}{\left(}{\left\Vert}\eta{\right\Vert}_{H^{s_0+{\frac{1}{2}}}},{\left\Vert} f{\right\Vert}_{H^{s_0}}{\right)}{\left[}1+{\left\Vert}\eta{\right\Vert}_{C^{r+{\frac{1}{2}}}}{\right]}{\left[}1+{\left\Vert} f{\right\Vert}_{H^s}+{\left\Vert}\eta{\right\Vert}_{H^{s+{\frac{1}{2}}}}{\right]}.
\end{multline*}
\end{proof}
{\it Proof of Proposition~\ref{prop:paradir}}.\\
For the sake of conciseness we denote in this proof by~$\Xi$ the right-hand  side of \eqref{para:DN:main}. Again we introduce~$w:=\chi(z)(\partial_z-T_A)u$ with~$\chi$ satisfying~$\chi\rvert_{z<z_0}=0$ and $\chi\rvert_{z>z_1}=1$, for~$-1<z_0<z_1<0$.
Then
$$\partial_zw-T_aw=\chi(z)Ru+\chi'(z)(\partial_z-T_A)u,$$
with ${\left\Vert} Ru{\right\Vert}_{Y^{s+{\frac{1}{2}}}([z_0, 0])}\leq \Xi$. We turn to estimate ~$\omega:=\chi'(z)(\partial_z-T_A)u$ in $Y^{s+{\frac{1}{2}}}$, it is non-zero only on~$(z_0,z_1)$ and satisfies
$$\partial_z\omega-T_a\omega=\chi'(z)Ru+\chi''(z)\omega:=f_0.$$
We have trivially
$${\left\Vert}(\partial_z-T_A)u{\right\Vert}_{Y^{s}([z_0,0])}\leq{ \mathcal{F}}{\left(}{\left\Vert}\eta{\right\Vert}_{H^{s_0+{\frac{1}{2}}}}{\right)}{\left\Vert}\nabla_{x,z}u{\right\Vert}_{X^{s-1}([-1,0])}.$$
From the study of~$v$ and the expression~$u=v-T_b\rho$, it holds that
${\left\Vert}\nabla_{x,z}u{\right\Vert}_{X^{s-1}([z_0,0])}{\leq} \Xi$. Consequently, ${\left\Vert} \omega{\right\Vert}_{Y^s}\leq \Xi$ and ${\left\Vert} f_0{\right\Vert}_{Y^s}{\leq} \Xi$. Applying Theorem~\ref{regularitysob} with the boundary condition ~$\omega(z_1)=0$ gives
${\left\Vert}\omega{\right\Vert}_{X^{s}([z_1, 0]}{\leq} \Xi$. Since~$X^s\subset Y^{s+{\frac{1}{2}}}$, we have proved that
$\partial_zw-T_aw=f$ with ${\left\Vert} f{\right\Vert}_{Y^{s+{\frac{1}{2}}}([z_1,0])}\leq \Xi$. Then using Theorem~\ref{regularitysob} once again gives
\begin{equation} \label{lem:tangpara}
{\left\Vert}\partial_zu-T_Au{\right\Vert}_{X^{s+{\frac{1}{2}}}([z_1,0])}\leq \Xi.
\end{equation}
To finish the proof of Proposition~\ref{prop:paradir}, we recall that by definition 
$$G(\eta)f=\frac{1+{\left\vert}\nabla\rho{\right\vert}^2}{\partial_z\rho}\partial_zv\arrowvert_{z=0}-\nabla\rho\cdot\nabla v\arrowvert_{z=0}.$$
We will say ~$f_1\sim f_2$ if ${\left\Vert} f_1-f_2{\right\Vert}_{X^{s+{\frac{1}{2}}}([z_1,0])}\leq \Xi$. By paralinearizing we have
\begin{multline*}
	\frac{1+{\left\vert}\nabla\rho{\right\vert}^2}{\partial_z\rho}\partial_zv-\nabla\rho\cdot\nabla v\sim T_{\frac{1+{\left\vert}\nabla\rho{\right\vert}^2}{\partial_z\rho}}\partial_zv+2T_{b\nabla\rho}\cdot\nabla\rho-T_{b\frac{1+{\left\vert}\nabla\rho{\right\vert}^2}{\partial_z\rho}}\partial_z\rho-T_{\nabla\rho}\cdot\nabla v-T_{\nabla v}\cdot\nabla\rho.
\end{multline*}
Then replacing~$v$ with~$u+T_b\rho$ we have after some computations
$$\frac{1+{\left\vert}\nabla\rho{\right\vert}^2}{\partial_z\rho}\partial_zv-\nabla\rho\cdot\nabla v\sim T_{\frac{1+{\left\vert}\nabla\rho{\right\vert}^2}{\partial_z\rho}}\partial_zu-T_{\nabla\rho}\cdot\nabla u+T_{b\nabla\rho-\nabla v}\cdot\nabla\rho.$$
Lemma~\ref{lem:tangpara} gives
$$T_{\frac{1+{\left\vert}\nabla\rho{\right\vert}^2}{\partial_z\rho}}\partial_zu-T_{\nabla\rho}\cdot\nabla u\sim T_\Lambda u,$$
with~$\Lambda\rvert_{z=0}=\lambda$ as announced. Now at~$z=0$,
$\nabla\rho\rvert_{z=0}=\nabla\eta$, $u\rvert_{z=0}=\psi-T_B\eta$, $\nabla v-b\nabla\rho\rvert_{z=0}=V$, so
$$G(\eta)\psi\sim T_\lambda(\psi-T_B\eta)-T_V\nabla\eta$$
as claimed. The proof of Proposition \ref{prop:paradir} is complete.\\
Now, we want a paralinearization result for $G(\eta)f$ in term of the principle symbol $\lambda^{(1)}$ only, with a remainder of order $0$.
\begin{prop}\label{DN2}
Let $d\ge 1$ and $s>3/2+d/2$. Let $1/2{\leq}\mu{\leq} s-1$ then there exists a non-decreasing function $F$ such that for any $f\in H^\mu$ there hold
\[
G(\eta)f=T_{\lambda^{(1)}}f+F(\eta, f),\quad
{\left\Vert} F(\eta, f){\right\Vert}_{H^{\mu}}{\leq} F({\left\Vert} \eta{\right\Vert}_{H^{s+{\frac{1}{2}}}}){\left\Vert} f{\right\Vert}_{H^{\mu}}.
\]
\end{prop}
\begin{rema}
The same result was proved in \cite{ABZ1} when $s>2+d/2$.
\end{rema}
\begin{proof}
{\it Step 1}. Again, with $v$ a solution to \eqref{eq:v}, Proposition \ref{prop:regellsob} gives with $I=[-1,0]$
\begin{equation}\label{DN2:dv}
{\left\Vert} \nabla_{x,z}v{\right\Vert}_{X^{\mu-1}(I)}{\leq} { \mathcal{F}}{\left(}{\left\Vert} \eta{\right\Vert}_{H^{s+{\frac{1}{2}}}}{\right)}{\left\Vert} f{\right\Vert}_{H^\mu}.
\end{equation}
According to Lemma \ref{para:eq:v} we have the paralinearization of \eqref{eq:v}
\begin{equation}
\partial_z^2v+T_\alpha\Delta_x v+T_\beta\cdot\nabla_x\partial_zv=F_0:=\gamma\partial_zv+(T_\alpha-\alpha)\Delta_xv+(T_\beta-\beta)\cdot\nabla\partial_zv.
\end{equation}
We claim that 
\begin{equation}\label{DN2:F0}
{\left\Vert} F_0{\right\Vert}_{Y^\mu} {\leq} F({\left\Vert} \eta{\right\Vert}_{H^{s+{\frac{1}{2}}}}){\left\Vert} \nabla_{x,z}v{\right\Vert}_{H^{\mu-1}}.
\end{equation}
Indeed, for the first term one estimates using the product rule \eqref{boundpara} with $s_0=\mu-1/2,~s_1=s-3/2,~s_2=\mu-1/2$ to get
\[
{\left\Vert} \gamma\partial_z v{\right\Vert}_{L^2H^{\mu-{\frac{1}{2}}}}{\lesssim} {\left\Vert} \gamma{\right\Vert}_{L^\infty H^{s-{\frac{3}{2}}}}{\left\Vert} \partial_zv{\right\Vert}_{L^2H^{\mu-{\frac{1}{2}}}} {\leq}  { \mathcal{F}}{\left(}{\left\Vert} \eta{\right\Vert}_{H^{s+{\frac{1}{2}}}}{\right)}{\left\Vert} \nabla_{x,z}v{\right\Vert}_{X^{\mu-1}}.
\]
For the second term, the rule \eqref{boundpara2} yields
\[
{\left\Vert} (T_\alpha-\alpha)\Delta_xv{\right\Vert}_{H^{\mu-{\frac{1}{2}}}}{\lesssim} {\left\Vert} \alpha{\right\Vert}_{L^\infty H^{s-{\frac{1}{2}}}}{\left\Vert} \Delta v{\right\Vert}_{L^2H^{\mu-{\frac{3}{2}}}} {\leq}  { \mathcal{F}}{\left(}{\left\Vert} \eta{\right\Vert}_{H^{s+{\frac{1}{2}}}}{\right)}{\left\Vert} \nabla_{x,z}v{\right\Vert}_{X^{\mu-1}}.
\]
The last term is estimated identically, we thus obtain \eqref{DN2:F0}.\\
{\it Step 2.} Next, according to Lemma \ref{lem:decouplesob}
$$\partial_z^2+T_\alpha\Delta_x+T_\beta\cdot\nabla_x\partial_z=(\partial_z-T_{a^{(1)}})(\partial_z-T_{A^{(1)}})+R,$$
with $R$ is of order $1$ and thus
$
{\left\Vert} Rv{\right\Vert}_{Y^\mu}{\leq} { \mathcal{F}}({\left\Vert} \eta{\right\Vert}_{H^{s+{\frac{1}{2}}}}){\left\Vert} \nabla_{x,z}v{\right\Vert}_{X^{\mu-1}}.
$
In view of \eqref{DN2:dv} there holds
\[
(\partial_z-T_{a^{(1)}}){\left[} (\partial_z-T_{A^{(1)}})v{\right]}=F_1,\quad 
{\left\Vert} F_1{\right\Vert}_{Y^\mu}{\leq} { \mathcal{F}}{\left(}{\left\Vert} \eta{\right\Vert}_{H^{s+{\frac{1}{2}}}}{\right)}{\left\Vert} f{\right\Vert}_{H^\mu}.
\]
By virtue of  Theorem \ref{regularitysob} we can obtain as before 
\begin{equation}\label{DN2:key}
{\left\Vert} (\partial_z-T_{A^{(1)}})v{\right\Vert}_{X^\mu([z_1, 0])}{\leq} { \mathcal{F}}{\left(}{\left\Vert} \eta{\right\Vert}_{H^{s+{\frac{1}{2}}}}{\right)}{\left\Vert} f{\right\Vert}_{H^\mu}.
\end{equation}
{\it Step 4.} Writing $f_1\sim f_2$ iff the $X^{\mu}([z_1, 0])$-norm of $f_1-f_2$ is bounded by the right-hand side of \eqref{DN2:key}, we have  (notice that $A^{(1)}\in \Gamma^1_1$ with semi-norms bounded by ${ \mathcal{F}}({\left\Vert} \eta{\right\Vert}_{H^{s+{\frac{1}{2}}}})$)
\begin{align*}
\frac{1+{\left\vert}\nabla\rho{\right\vert}^2}{\partial_z\rho}\partial_zv-\nabla\rho\cdot\nabla v&\sim T_{\frac{1+{\left\vert}\nabla\rho{\right\vert}^2}{\partial_z\rho}}\partial_zv-T_{\nabla \rho}\nabla v\\
&\sim T_{\frac{1+{\left\vert}\nabla\rho{\right\vert}^2}{\partial_z\rho}}T_{A^{(1)}}v-T_{\nabla\rho}\nabla v\sim T_{\frac{1+{\left\vert}\nabla\rho{\right\vert}^2}{\partial_z\rho}A^{(1)}}v-T_{\nabla \rho}\nabla v
\end{align*}
which concludes the proof since at $z=0$,~
$
\frac{1+{\left\vert}\nabla\rho{\right\vert}^2}{\partial_z\rho}A^{(1)}-i\nabla\rho\cdot\xi =\lambda^{(1)}.
$
\end{proof}
To conclude this section, let us recall the following result on the {\it shape derivative} of the Dirichlet-Neumann operator.
\begin{theo}[\protect{\cite[Theorem~3.21]{LannesLivre}}]\label{shape}
Let $\psi\in H^{\frac{3}{2}}$ and $s>1/2+d/2,~d\ge 1$. Then the map 
\[
G(\cdot)\psi: H^{s+{\frac{1}{2}}}\to H^{\frac{1}{2}}
\]
is differentiable and 
\[
d_\eta G(\eta)\psi\cdot f=\lim_{{\varepsilon}\to 0}\frac{1}{\varepsilon}\{G(\eta+{\varepsilon} f)\psi-G(\eta)f\}=-G(\eta)(Bf)-\operatorname{div}(Vf)
\]
where $B$ and $V$ are functions of $(\eta, \psi)$ as in \eqref{BV}.
\end{theo}
\section{Paralinearization and symmetrization of the system}\label{section:para}
\subsection{Paralinearization of the system}
We want to replace all the nonlinear terms in the Zakharov-Craig-Sulem system~(\ref{ww}) with paradifferential expressions. We have already paralinearized the Dirichlet-Neumann map, so we need to transform the nonlinear terms appearing in the second equation.\\
{\hspace*{.15in}} Throughout this paragraph, we fix $d\ge 1,~p\in [1, +\infty],~I=[0, T]$ and $(\eta, \psi)$ be a solution to system \eqref{ww} such that
\begin{equation}\label{assum:sym}
\left\{
\begin{aligned}
&s\geq s_0>{\frac{3}{2}}+\frac d2,\quad 2<r<s_0-\frac d2+{\frac{1}{2}},\\
&\psi\in L^\infty(I; H^s)\cap L^p(I; W^{r,\infty}),\\
& \eta\in L^\infty(I; H^{s+{\frac{1}{2}}})\cap L^p(I; W^{r+{\frac{1}{2}}, \infty}),\quad \inf_{t\in I}\operatorname{dist}(\eta(t), \Gamma)\ge h>0.
\end{aligned}
\right.
\end{equation}
\begin{lemm}	\label{lem:paracurv}
	There exists a nondecreasing function~${ \mathcal{F}}$ such that
	$$H(\eta)=-T_l\eta+f,$$
	where~$l=l^{(2)}+l^{(1)}$ with
	\begin{equation}	\label{eq:l}
	\begin{aligned}
		l^{(2)}&={\left(}1+{\left\vert}\nabla\eta{\right\vert}^2{\right)}^{-{\frac{1}{2}}}{\left(}{\left\vert}\xi{\right\vert}^2-\frac{(\nabla\eta\cdot\xi)^2}{1+{\left\vert}\nabla\eta{\right\vert}^2}{\right)},\quad
		l^{(1)}&=-\frac i2(\partial_x\cdot\partial_\xi)l^{(2)},
	\end{aligned}
	\end{equation}
	and~$f\in H^{s+r-2}$ satisfying
	$${\left\Vert} f{\right\Vert}_{H^{s+r-2}}\leq{ \mathcal{F}}{\left(}{\left\Vert}\eta{\right\Vert}_{H^{s_0+{\frac{1}{2}}}}{\right)}{\left\Vert}\eta{\right\Vert}_{C^{r+{\frac{1}{2}}}}{\left\Vert}\eta{\right\Vert}_{H^{s+{\frac{1}{2}}}}.$$
\end{lemm}
\begin{proof}
Applying Theorem \ref{paralin} with  $u=\nabla \eta$, $\mu=s-{\frac{1}{2}}$ and $\rho= r-{\frac{1}{2}}$ we obtain 
\[
\frac{\nabla\eta}{\sqrt{1+|\nabla\eta|^2}}=T_{p}\nabla\eta+f_1, \quad p=\frac{1}{(1+|\nabla\eta|^2)^{\frac{1}{2}}}I-\frac{\nabla \eta\otimes\nabla\eta}{(1+|\nabla\eta|^2)^{\frac{3}{2}}}
\]
and $f_1$ satisfies
\[
{\left\Vert} f_1{\right\Vert}_{H^{s+r-1}}{\leq} { \mathcal{F}}{\left(}{\left\Vert} \nabla\eta{\right\Vert}_{L^{\infty}}{\right)}{\left\Vert} \nabla\eta{\right\Vert}_{C^{r-{\frac{1}{2}}}}{\left\Vert} \nabla\eta{\right\Vert}_{H^{s-{\frac{1}{2}}}}.
\]
Since $s_0>\frac{3}{2}+\frac{d}{2}$, this yields
\begin{equation}
{\left\Vert} f_1{\right\Vert}_{H^{s+r-1}}{\leq} { \mathcal{F}}{\left(}{\left\Vert} \eta{\right\Vert}_{H^{s_0+{\frac{1}{2}}}}{\right)}{\left\Vert}  \eta{\right\Vert}_{C^{r+{\frac{1}{2}}}}{\left\Vert} \eta{\right\Vert}_{H^{s+{\frac{1}{2}}}}.
\end{equation}
Hence,
\[
H(\eta)=\operatorname{div}(T_{p}\nabla\eta+f_1)=T_{-p\xi\cdot\xi+i\operatorname{div} p\xi}\eta+\operatorname{div} f_1.
\]
This gives the conclusion with $l^{(2)}=p\xi\cdot\xi,~l^{(1)}=-i\operatorname{div} p\xi,~ f=\operatorname{div} f_1.$
\end{proof}
We next tackle the other nonlinear terms.
Recall the notations
$$B=\frac{\nabla\eta\cdot\nabla\psi+G(\eta)\psi}{1+{\left\vert}\nabla\eta{\right\vert}^2},\quad V=\nabla\psi-B\nabla\eta.$$
\begin{lemm}	\label{lem:paraquad}
	There exists a nondecreasing function~${ \mathcal{F}}$ such that
	$${\frac{1}{2}}{\left\vert}\nabla\psi{\right\vert}^2-{\frac{1}{2}}\frac{(\nabla\eta\cdot\nabla\psi-G(\eta)\psi)^2}{1+{\left\vert}\nabla\eta{\right\vert}^2}=
	T_V\cdot\nabla\psi-T_VT_B\cdot\nabla\eta-T_BG(\eta)\psi+f,$$
	with~$f\in H^{s+r-2}$ and
	$${\left\Vert} f{\right\Vert}_{H^{s+r-2}}\leq{ \mathcal{F}}{\left(}{\left\Vert}\eta{\right\Vert}_{H^{s_0+{\frac{1}{2}}}},{\left\Vert}\psi{\right\Vert}_{H^{s_0}}{\right)}{\left[}1+{\left\Vert}\psi{\right\Vert}_{C^r}+{\left\Vert}\eta{\right\Vert}_{C^{r+{\frac{1}{2}}}}{\right]}{\left[}1+{\left\Vert}\psi{\right\Vert}_{H^s}+{\left\Vert}\eta{\right\Vert}_{H^{s+{\frac{1}{2}}}}{\right]}.$$
\end{lemm}
\begin{proof}
	Consider
$$
F(a, b, c)={\frac{1}{2}}\frac{(ab+c)^2}{1+{\left\vert} a{\right\vert}^2}, \quad (a, b, c)\in {\bm{\mathrm{R}}}^d\times{\bm{\mathrm{R}}}^d\times{\bm{\mathrm{R}}}.
$$
We compute 
$$
\partial_aF=\frac{(ab+c)}{1+{\left\vert} a{\right\vert}^2}{\left(} b-\frac{(ab+c)}{1+{\left\vert} a{\right\vert}^2}a{\right)},\quad \partial_bF=\frac{(ab+c)}{1+{\left\vert} a{\right\vert}^2}a,\quad \partial_cF=\frac{(ab+c)}{1+{\left\vert} a{\right\vert}^2}.
$$
Now we take~$a=\nabla\eta$, $b=\nabla\psi$, and~$c=G(\eta)\psi$.
Using Proposition~\ref{prop:DNreg} and the hypothesis~$s_0>{\frac{3}{2}}+\frac d2$, we have
\begin{align*}
	{\left\Vert}(a,b,c){\right\Vert}_{L^\infty}&\leq{ \mathcal{F}}{\left(}{\left\Vert}\eta{\right\Vert}_{H^{s_0+{\frac{1}{2}}}},{\left\Vert}\psi{\right\Vert}_{H^{s_0}}{\right)},\\
	{\left\Vert}(a,b,c){\right\Vert}_{H^{s-1}}&\leq{ \mathcal{F}}{\left(}{\left\Vert}\eta{\right\Vert}_{H^{s_0+{\frac{1}{2}}}},{\left\Vert}\psi{\right\Vert}_{H^{s_0}}{\right)}{\left[}1+{\left\Vert}\psi{\right\Vert}_{H^s}+{\left\Vert}\eta{\right\Vert}_{H^{s+{\frac{1}{2}}}}{\right]},\\
	{\left\Vert}(a,b,c){\right\Vert}_{C^{r-1}}&\leq{ \mathcal{F}}{\left(}{\left\Vert}\eta{\right\Vert}_{H^{s_0+{\frac{1}{2}}}},{\left\Vert}\psi{\right\Vert}_{H^{s_0}}{\right)}{\left[}1+{\left\Vert}\psi{\right\Vert}_{C^r}+{\left\Vert}\eta{\right\Vert}_{C^{r+{\frac{1}{2}}}}{\right]}.
\end{align*}
Then combining this with Theorem~\ref{paralin} gives
$${\frac{1}{2}}\frac{(\nabla\eta\cdot\nabla\psi-G(\eta)\psi)^2}{1+{\left\vert}\nabla\eta{\right\vert}^2}=T_{VB}\cdot\nabla\eta+T_{B\nabla\eta}\cdot\nabla\psi+T_BG(\eta)\psi+f_1,$$
with
$${\left\Vert} f_1{\right\Vert}_{H^{s+r-2}}\leq{ \mathcal{F}}{\left(}{\left\Vert}\eta{\right\Vert}_{H^{s_0+{\frac{1}{2}}}},{\left\Vert}\psi{\right\Vert}_{H^{s_0}}{\right)}{\left[}1+{\left\Vert}\psi{\right\Vert}_{C^r}+{\left\Vert}\eta{\right\Vert}_{C^{r+{\frac{1}{2}}}}{\right]}{\left[}1+{\left\Vert}\psi{\right\Vert}_{H^s}+{\left\Vert}\eta{\right\Vert}_{H^{s+{\frac{1}{2}}}}{\right]}.$$
By the same theorem, there holds
\[ {\frac{1}{2}}{\left\vert}\nabla\psi{\right\vert}^2=T_{\nabla\psi}\cdot\nabla\psi+f_2,\quad {\left\Vert} f_2{\right\Vert}_{H^{s+r-2}}\leq{ \mathcal{F}}{\left(}{\left\Vert}\psi{\right\Vert}_{H^{s_0}}{\right)}{\left\Vert}\psi{\right\Vert}_{C^r}{\left\Vert}\psi{\right\Vert}_{H^s}.
\]
At last, we deduce  from~(\ref{esti:quant2}) and the estimates on~$(B,V)$ from Corollary~\ref{cor:BV} that
$${\left\Vert}(T_{BV}-T_VT_B)\cdot\nabla\eta{\right\Vert}_{H^{s+r-2}}\leq{ \mathcal{F}}{\left(}{\left\Vert}\eta{\right\Vert}_{H^{s_0+{\frac{1}{2}}}},{\left\Vert}\psi{\right\Vert}_{H^{s_0}}{\right)}{\left\Vert}\nabla\eta{\right\Vert}_{H^{s-{\frac{1}{2}}}},$$
from which we can conclude the proposition.
\end{proof}

To replace the original unknown with the new good unknown, we will need an estimate on~$T_{\partial_tB}\eta$.
This is contained in the following lemma.
\begin{lemm} \label{lem:derB}
We have
	$${\left\Vert} T_{\partial_tB}\eta{\right\Vert}_{H^{s}}\leq{ \mathcal{F}}{\left(}{\left\Vert}\eta{\right\Vert}_{H^{s_0+{\frac{1}{2}}}},{\left\Vert}\psi{\right\Vert}_{H^{s_0}}{\right)}{\left[}1+{\left\Vert}\psi{\right\Vert}_{C^r}+{\left\Vert}\eta{\right\Vert}_{C^{r+{\frac{1}{2}}}}{\right]}{\left\Vert}\eta{\right\Vert}_{H^{s+{\frac{1}{2}}}}.$$
\end{lemm}
\begin{proof}
	First, using the equations~(\ref{ww}), the product and nonlinear estimates, and the estimates on~$G(\eta)\psi$ of Proposition~\ref{prop:DNreg}, we have
\begin{gather*}{\left\Vert}\partial_t\eta{\right\Vert}_{H^{s_0-1}}+{\left\Vert}\partial_t\psi{\right\Vert}_{H^{s_0-\frac32}}\leq{ \mathcal{F}}{\left(}{\left\Vert}\eta{\right\Vert}_{H^{s_0+{\frac{1}{2}}}},{\left\Vert}\psi{\right\Vert}_{H^{s_0}}{\right)},\\	{\left\Vert}\partial_t\eta{\right\Vert}_{C^{r-1}}+{\left\Vert}\partial_t\psi{\right\Vert}_{C^{r-\frac32}}\leq{ \mathcal{F}}{\left(}{\left\Vert}\eta{\right\Vert}_{H^{s_0+{\frac{1}{2}}}},{\left\Vert}\psi{\right\Vert}_{H^{s_0}}{\right)}{\left[}1+{\left\Vert}\psi{\right\Vert}_{C^r}+{\left\Vert}\eta{\right\Vert}_{C^{r+{\frac{1}{2}}}}{\right]}.
\end{gather*}
	Then using Theorem \ref{shape} for the shape derivative of the Dirichlet-Neumann, we have 
	$$\partial_t{\left[} G(\eta)\psi{\right]}=G(\eta)(\partial_t\psi-B\partial_t\eta)-\operatorname{div}(V\partial_t\eta).$$
	Then using the preceding estimates and the estimate on~$B$ from Corollary~\ref{cor:BV},
	\begin{gather*}{\left\Vert}\partial_t\psi-B\partial_t\eta{\right\Vert}_{H^{s_0-\frac32}}\leq{ \mathcal{F}}{\left(}{\left\Vert}\eta{\right\Vert}_{H^{s_0+{\frac{1}{2}}}},{\left\Vert}\psi{\right\Vert}_{H^{s_0}}{\right)},\\
{\left\Vert}\partial_t\psi-B\partial_t\eta{\right\Vert}_{C^{r-\frac32}}\leq{ \mathcal{F}}{\left(}{\left\Vert}\eta{\right\Vert}_{H^{s_0+{\frac{1}{2}}}},{\left\Vert}\psi{\right\Vert}_{H^{s_0}}{\right)}{\left[}1+{\left\Vert}\psi{\right\Vert}_{C^r}+{\left\Vert}\eta{\right\Vert}_{C^{r+{\frac{1}{2}}}}{\right]}.
\end{gather*}
	Thus the last estimates of Proposition~\ref{prop:DNreg} give
	$$
	{\left\Vert} G(\eta)(\partial_t\psi-B\partial_t\eta){\right\Vert}_{C^{r-\frac52}}\leq{ \mathcal{F}}{\left(}{\left\Vert}\eta{\right\Vert}_{H^{s_0+{\frac{1}{2}}}},{\left\Vert}\psi{\right\Vert}_{H^{s_0}}{\right)}{\left[}1+{\left\Vert}\psi{\right\Vert}_{C^r}+{\left\Vert}\eta{\right\Vert}_{C^{r+{\frac{1}{2}}}}{\right]}.
	$$
	There also holds
	$${\left\Vert}\operatorname{div}(V\partial_t\eta){\right\Vert}_{C^{r-\frac52}}{\leq} {\left\Vert} V\partial_t\eta{\right\Vert}_{C^{r-\frac32}}\leq{ \mathcal{F}}{\left(}{\left\Vert}\eta{\right\Vert}_{H^{s_0+{\frac{1}{2}}}},{\left\Vert}\psi{\right\Vert}_{H^{s_0}}{\right)}{\left[}1+{\left\Vert}\psi{\right\Vert}_{C^r}+{\left\Vert}\eta{\right\Vert}_{C^{r+{\frac{1}{2}}}}{\right]},$$
	so that
	$${\left\Vert}\partial_tG(\eta)\psi{\right\Vert}_{C^{r-\frac52}}\leq{ \mathcal{F}}{\left(}{\left\Vert}\eta{\right\Vert}_{H^{s_0+{\frac{1}{2}}}},{\left\Vert}\psi{\right\Vert}_{H^{s_0}}{\right)}{\left[}1+{\left\Vert}\psi{\right\Vert}_{C^r}+{\left\Vert}\eta{\right\Vert}_{C^{r+{\frac{1}{2}}}}{\right]}.$$
	
	At last, as in \eqref{decompose:B},
\[
B=K(\nabla\eta)\cdot\nabla\psi+L(\nabla\eta)G(\eta)\psi.
\]
 Differentiating this expression and using the preceding estimates on the time derivatives, we have
	$${\left\Vert}\partial_tB{\right\Vert}_{C^{r-\frac52}}\leq{ \mathcal{F}}{\left(}{\left\Vert}\eta{\right\Vert}_{H^{s_0+{\frac{1}{2}}}},{\left\Vert}\psi{\right\Vert}_{H^{s_0}}{\right)}{\left[}1+{\left\Vert}\psi{\right\Vert}_{C^r}+{\left\Vert}\eta{\right\Vert}_{C^{r+{\frac{1}{2}}}}{\right]},$$
	from which the lemma follows immediately by \eqref{pest1} and the fact that ~$r-\frac52>-{\frac{1}{2}}$.
\end{proof}
We now have all the ingredients needed to paralinearize the equations.
Recall that~$\lambda$ has been defined in~\eqref{eq:lambda}, and~$l$ in~\eqref{eq:l}.
\begin{prop}
	There exists a nondecreasing function~${ \mathcal{F}}$ such that with ~$U:=\psi-T_B\eta$ there holds
	\begin{equation}	\label{eq:PL}
	{\left\{}\begin{aligned}
		   \partial_t\eta+T_V\cdot\nabla\eta-T_\lambda U=&f_1,\\
		   \partial_tU+T_V\cdot\nabla U-T_l\eta=&f_2,
	    \end{aligned}
	\right.
	\end{equation}
	with~$(f_1,f_2)$ satisfying
	\begin{multline*}{\left\Vert}(f_1,f_2){\right\Vert}_{H^{s+{\frac{1}{2}}}\times H^s}\leq{ \mathcal{F}}{\left(}{\left\Vert}\eta{\right\Vert}_{H^{s_0+{\frac{1}{2}}}},{\left\Vert}\psi{\right\Vert}_{H^{s_0}}{\right)}{\left[}1+{\left\Vert}\psi{\right\Vert}_{C^r}+{\left\Vert}\eta{\right\Vert}_{C^{r+{\frac{1}{2}}}}{\right]}\times\\
\times{\left[}1+{\left\Vert}\psi{\right\Vert}_{H^s}+{\left\Vert}\eta{\right\Vert}_{H^{s+{\frac{1}{2}}}}{\right]}.\end{multline*}
\end{prop}
\begin{proof}
	The first equation is just Proposition~\ref{prop:paradir}.
	For the second one, we use the equation satisfied by~$\psi$ and Lemmas~\ref{lem:paracurv}--\ref{lem:paraquad}
	to see that
	$$\partial_t\psi+T_l\eta+T_V\cdot\nabla\psi-T_VT_B\cdot\nabla\eta-T_BG(\eta)\psi=R$$
	with
	$${\left\Vert} R{\right\Vert}_{H^s}\leq{ \mathcal{F}}{\left(}{\left\Vert}\eta{\right\Vert}_{H^{s_0+{\frac{1}{2}}}},{\left\Vert}\psi{\right\Vert}_{H^{s_0}}{\right)}{\left[}1+{\left\Vert}\psi{\right\Vert}_{C^r}+{\left\Vert}\eta{\right\Vert}_{C^{r+{\frac{1}{2}}}}{\right]}{\left[}1+{\left\Vert}\psi{\right\Vert}_{H^s}+{\left\Vert}\eta{\right\Vert}_{H^{s+{\frac{1}{2}}}}{\right]},$$
	and since
	$$\partial_tU=\partial_t\psi-T_B\partial_t\eta-T_{\partial_tB}\eta,$$
	we can use Lemma~\ref{lem:derB}, the fact that~$\partial_t\eta=G(\eta)\psi,$
	and 
	$$T_V\cdot\nabla\psi-T_VT_B\cdot\nabla\eta=T_V\cdot\nabla U+R'$$
	with
	$${\left\Vert} R'{\right\Vert}_{H^s}\leq{ \mathcal{F}}{\left(}{\left\Vert}\eta{\right\Vert}_{H^{s_0+{\frac{1}{2}}}},{\left\Vert}\psi{\right\Vert}_{H^{s_0}}{\right)}{\left[}1+{\left\Vert}\psi{\right\Vert}_{H^s}+{\left\Vert}\eta{\right\Vert}_{H^{s+{\frac{1}{2}}}}{\right]}$$
	to conclude.
\end{proof}

\subsection{Symmetrization of the system}
As in \cite{ABZ1} we shall deal with a class of symbols having special structure that we recall here for the reader's convenience.
\begin{defi} Given $m\in {\mathbf{R}}$, $\Sigma^m$ denotes the class of symbols $a$ of the form 
\[
a=a^{(m)}+a^{(m-1)}
\]
with 
\[
a^{(m)}( x, \xi)=F(\nabla\eta(x), \xi),\quad a^{(m-1)}( x, \xi)=\sum_{|\alpha|=2}F_{\alpha}(\nabla\eta(x), \xi)\partial_x^{\alpha}\eta( x)
\]
such that 
\begin{enumerate}
\item  $T_a$ maps real-valued functions to real-valued functions;
\item $F$ is a $C^{\infty}$ real-valued function of $(\zeta, \xi)\in {\mathbf{R}}^d\times{\mathbf{R}}^d\setminus\{0\}$, homogeneous of order $m$ in $\xi$, and there exists a function $K=K(\zeta)>0$ such that 
\[
F(\zeta, \xi)\ge K(\zeta)|\xi|^m, \quad\forall (\zeta, \xi)\in  {\mathbf{R}}^d\times{\mathbf{R}}^d\setminus\{0\};
\]
\item the $F_{\alpha}$s are complex-valued functions of  $(\zeta, \xi)\in{\mathbf{R}}^d\times{\mathbf{R}}^d\setminus\{0\}$, homogeneous of order $m-1$ in $\xi$.
\end{enumerate}
\end{defi}
{\hspace*{.15in}} In the sequel, we often need an estimate for $u$ from $T_au$. For this purpose, we prove
\begin{prop}\label{inverse}
Let $m,~\mu \in {\mathbf{R}}$, and~$s_0>\frac32+\frac d2$. Then there exists a function ${ \mathcal{F}}$ such that for all~$\eta\in H^{s_0-{\frac{1}{2}}}$, for all~$a\in\Sigma^m$, we have
\begin{gather}
\Vert u\Vert_{H^{\mu+m}}{\leq} { \mathcal{F}}{\left(}\Vert \eta\Vert_{H^{s_0-{\frac{1}{2}}}}{\right)}\left(\Vert T_{a}u\Vert_{H^{\mu}}+\Vert u\Vert_{L^2}\right)\label{est:inverse},\\
\Vert u\Vert_{C_*^{\mu+m}}{\leq} { \mathcal{F}}{\left(}\Vert \eta\Vert_{H^{s_0-{\frac{1}{2}}}}{\right)}\left(\Vert T_{a}u\Vert_{C_*^{\mu}}+\Vert u\Vert_{C^0_*}\right)\label{est:inverse:H}.
\end{gather}
\end{prop}
\begin{rema}
1. The same result was proved in Proposition $4.6$ of~\cite{ABZ1} where the constant in the right hand side is ${ \mathcal{F}}(\Vert \eta(t)\Vert_{H^{s-1}})$. Here, for less regular $\eta$ we prove a worse estimate. However, it turns out that \eqref{est:inverse} is sufficient to obtain  a priori bounds.\\
2. In \eqref{est:inverse} (resp. \eqref{est:inverse:H}) one can freely replace $\|u\|_{L^2}$ (reps. $\Vert u\Vert_{C^0_*}$) by any lower order Sobolev (resp. H\"older) norm.
\end{rema}
\begin{proof}
We give the proof for \eqref{est:inverse}, the one of \eqref{est:inverse:H} follows identically. We write $a=a^{(m)}+a^{(m-1)}$. Introduce $b=\frac{1}{a^{(m)}}$ and 
$$0<{\varepsilon}<\min{\left\{}1, s_0-{\frac{3}{2}}-\frac{d}{2}{\right\}}.$$
Applying Theorem \ref{theo:sc} $(ii)$ with $\rho={\varepsilon}$ gives $T_bT_{a^{(m)}}=I+r$ where~$r$ is of order~$-{\varepsilon}$ and 
\begin{equation}\label{inverse:1}
\Vert ru\Vert_{H^{\mu+{\varepsilon}}}{\leq} { \mathcal{F}}{\left(}\Vert \nabla\eta\Vert_{C^{\varepsilon}}{\right)}\Vert u\Vert_{H^{\mu}}{\leq} { \mathcal{F}}{\left(}\Vert \eta\Vert_{C^{1+{\varepsilon}}}{\right)}\Vert u\Vert_{H^{\mu}}{\leq} { \mathcal{F}}{\left(}\Vert \eta\Vert_{H^{s_0-{\frac{1}{2}}}}{\right)}\Vert u\Vert_{H^{\mu}}.
\end{equation}
Then, setting $R=-r-T_bT_{a^{(m-1)}}$ we have
\[
(I-R)u=T_bT_au.
\]
Let us consider the symbol $a^{(m-1)}$. For any $\alpha\in {\mathbf{N}}^d$ with $|\alpha|=2$ and fixed $\xi$, since $s_0>{\frac{3}{2}}+\frac{d}{3}$, Sobolev embedding  and  estimates \eqref{pr}, \eqref{F(u):H} give
\begin{multline*}
\Vert F_{\alpha}(\nabla\eta, \xi)\partial_x^{\alpha}\eta\Vert_{C^{-1+{\varepsilon}}_*}{\leq} \Vert F_{\alpha}(\nabla\eta, \xi)\partial_x^{\alpha}\eta\Vert_{H^{-1+{\varepsilon}+\frac{d}{2}}}
{\leq} \Vert F_{\alpha}(\nabla\eta, \xi)\Vert_{H^{s_0-{\frac{3}{2}}}}\Vert\partial_x^{\alpha}\eta\Vert_{H^{s_0-\frac{5}{2}}}\\
{\leq} { \mathcal{F}}{\left(}\Vert\nabla\eta\Vert_{L^{\infty}}{\right)}\Vert \eta\Vert_{H^{s_0-{\frac{1}{2}}}}^2{\leq} { \mathcal{F}}{\left(}\Vert\eta\Vert_{H^{s_0-{\frac{1}{2}}}}{\right)}.
\end{multline*}
Consequently, one deduces $a^{(m-1)}\in \dot{\Gamma}^{m-1}_{-1+{\varepsilon}}$ and thus by Proposition \ref{regu<0},
\[
\Vert T_{a^{(m-1)}}u\Vert_{H^{\mu-m+{\varepsilon}}}{\leq} { \mathcal{F}}{\left(}\Vert\eta\Vert_{H^{s_0-{\frac{1}{2}}}}{\right)}\Vert u\Vert_{H^{\mu}}.
\]
Because $b\in \Gamma^{-m}_0$ with semi-norm bounded by ${ \mathcal{F}}(\Vert\eta\Vert_{H^{s_0-{\frac{1}{2}}}})$ we get
\begin{equation}
\label{inverse:2}
\Vert T_bT_{a^{(m-1)}}u\Vert_{H^{\mu+{\varepsilon}}}{\leq} { \mathcal{F}}{\left(}\Vert\eta\Vert_{H^{s_0-{\frac{1}{2}}}}{\right)}\Vert u\Vert_{H^{\mu}}.
\end{equation}
Combining \eqref{inverse:1} with \eqref{inverse:2} yields 
\[
\Vert Ru\Vert_{H^{\mu+{\varepsilon}}}{\leq} { \mathcal{F}}{\left(}\Vert\eta(t)\Vert_{H^{s_0-{\frac{1}{2}}}}{\right)}\Vert u\Vert_{H^{\mu}}.
\]
The rest of the proof is identical to that of Proposition $4.6$ in~\cite{ABZ1}.
\end{proof}
For the sake of conciseness, we give the following definition.
\begin{defi}\label{equi:operators}
Let $m\in {\mathbf{R}}$ and consider two families of operators of order~$m$,
\[
\{ A(t): t\in [0, T]\},\quad \{ B(t): t\in [0, T]\}.
\]
Let~$s_0>\frac32+\frac d2$ and~$2<r\leq s_0+{\frac{1}{2}}-\frac d2$.

We write $A\sim B$ if $A-B$ is of order $m-{\frac{3}{2}}$ and the following condition is fulfilled: for all $\mu\in {\mathbf{R}}$, there exists a nondecreasing function ${ \mathcal{F}}$ such that for a.e. $t\in [0, T]$, 
\[
{\left\Vert} A(t)-B(t){\right\Vert}_{H^{\mu}\to H^{\mu-m+{\frac{3}{2}}}}{\leq} { \mathcal{F}}{\left(}{\left\Vert} \eta(t){\right\Vert}_{H^{s_0+{\frac{1}{2}}}}{\right)}{\left(}1+{\left\Vert} \eta(t){\right\Vert}_{C^{r+{\frac{1}{2}}}}{\right)}.
\]
\end{defi}
\begin{rema}\label{rema:class}
Let $a=a^{(m)}+a^{(m-1)}\in \Sigma^m$. We make the following remarks.\\
$(i)$ Because the principal symbol $a^{(m)}(t)$ contains only the first derivative $\nabla\eta\in C^{r-{\frac{1}{2}}}({\mathbf{R}}^d)\cap H^{s_0-{\frac{1}{2}}}({\mathbf{R}}^d)$ with $r-{\frac{1}{2}}>{\frac{3}{2}},~s_0-{\frac{1}{2}}>1+\frac{d}{2}$, applying the nonlinear estimate \eqref{F(u):H} we obtain
\[
M^m_{\frac{3}{2}}(a^{(m)}(t)){\leq} { \mathcal{F}}{\left(}{\left\Vert} \eta(t){\right\Vert}_{H^{s_0-{\frac{1}{2}}}}{\right)}{\left\Vert} \eta(t){\right\Vert}_{C^{r+{\frac{1}{2}}}}.
\]
On the other hand, 
\[
M^m_0(a^{(m)}(t)){\leq} { \mathcal{F}}{\left(}{\left\Vert} \eta{\right\Vert}_{H^{s_0-{\frac{1}{2}}}}{\right)}.
\]
$(ii)$ The subprincipal symbol $a^{(m-1)}(t)$ depends on $\partial^{\alpha}\eta~, |\alpha|=2$ which belongs to $C^{r-{\frac{3}{2}}}({\mathbf{R}}^d)\hookrightarrow C^{\frac{1}{2}}({\mathbf{R}}^d)$. Hence, $a^{(m-1)}\in \Gamma^{m-1}_{1/2}$ and by \eqref{tame:H} and \eqref{F(u):H} we have uniformly for $|\xi|=1$,
\begin{align*}
&{\left\Vert} F_{\alpha}(\nabla\eta(t,x), \xi)\partial_x^{\alpha}\eta(t, x) {\right\Vert}_{C^{\frac{1}{2}}}\\
&{\leq} {\left\Vert} [F_{\alpha}(\nabla\eta(t,\cdot), \xi)-F_{\alpha}(0, \xi)]\partial_x^{\alpha}\eta(t,\cdot) {\right\Vert}_{C^{\frac{1}{2}}}+{\left\vert} F_{\alpha}(0, \xi){\right\vert} {\left\Vert} \partial_x^{\alpha}\eta(t, \cdot) {\right\Vert}_{C^{\frac{1}{2}}}\\
&{\leq} { \mathcal{F}}{\left(}{\left\Vert} \nabla\eta(t){\right\Vert}_{L^{\infty}}{\right)}{\left\Vert} \nabla\eta(t){\right\Vert}_{C^{\frac{1}{2}}}{\left\Vert} \eta{\right\Vert}_{C^{r+{\frac{1}{2}}}}+{\left\vert} F_{\alpha}(0, \xi){\right\vert} {\left\Vert} \eta(t) {\right\Vert}_{C^{r+{\frac{1}{2}}}}\\
&{\leq} { \mathcal{F}}{\left(}{\left\Vert} \nabla\eta(t){\right\Vert}_{L^{\infty}}{\right)}{\left\Vert} \eta(t){\right\Vert}_{H^{s_0}}{\left\Vert} \eta{\right\Vert}_{C^{r+{\frac{1}{2}}}}+{\left\vert} F_{\alpha}(0,\xi){\right\vert} {\left\Vert} \eta(t) {\right\Vert}_{C^{r+{\frac{1}{2}}}}.
\end{align*}
The same estimates hold when one takes derivatives in $\xi$, consequently
\[
M^{m-1}_{\frac{1}{2}}(a^{(m-1)}(t)){\leq}  { \mathcal{F}}{\left(}{\left\Vert} \eta{\right\Vert}_{H^{s_0}}{\right)}{\left\Vert} \eta(t) {\right\Vert}_{C^{r+{\frac{1}{2}}}}.
\]
On the other hand, due to the fact that $s_0>{\frac{3}{2}}+\frac{d}{2}$ we have
\[
M^{m-1}_0(a^{(m-1)}(t)){\leq}  { \mathcal{F}}{\left(}{\left\Vert} \eta{\right\Vert}_{H^{s_0-{\frac{1}{2}}}}{\right)}{\left\Vert}\eta{\right\Vert}_{C^2}{\leq} { \mathcal{F}}{\left(}{\left\Vert} \eta{\right\Vert}_{H^{s_0+{\frac{1}{2}}}}{\right)}.
\]
From $(i)$ and $(ii)$ we observe that when one applies the symbolic calculus Theorem \ref{theo:sc}, the operator-norm estimates are always linear in the highest norm of $\eta$, namely ${\left\Vert} \eta{\right\Vert}_{C^{r+{\frac{1}{2}}}}$.
\end{rema}
Using this remark, one can verify easily that Proposition $4.3$ in \cite{ABZ1} is still valid and hence so is Proposition $4.8$, \cite{ABZ1}:
\begin{prop}\label{pq}
Let $q\in \Sigma^0,~p\in \Sigma^{\frac{1}{2}},~\gamma\in \Sigma^{\frac{3}{2}}$ defined by 
\[
\begin{aligned}
q&=(1+|\partial_x\eta|^2)^{-{\frac{1}{2}}},\\
p&=(1+|\partial_x\eta|^2)^{-\frac{5}{4}}\sqrt{\lambda^{(1)}}+p^{(-1/2)},\\
\gamma&=\sqrt{\ell^{(2)}\lambda^{(1)}}+\sqrt{\frac{\ell^{(2)}}{\lambda^{(1)}}}\frac{\Re \lambda^{(0)}}{2}-\frac{i}{2}(\partial_{\xi}\cdot\partial_x)\sqrt{\ell^{(2)}\lambda^{(1)}},
\end{aligned}
\]
where 
\[
p^{(-1/2)}=\frac{1}{\gamma^{(3/2)}}\left\{q^{(0)}\ell^{(1)}-\gamma^{(1/2)}p^{(1/2)}+i\partial_{\xi}\gamma^{(3/2)}\partial_xp^{(1/2)} \right\}.
\]
Then, it holds that
\[
T_pT_{\lambda}\sim T_{\gamma}T_q,\quad T_qT_{\ell}\sim T_{\gamma}T_p,\quad T_{\gamma}\sim (T_{\gamma})^*.
\]
\end{prop}
{\hspace*{.15in}} Using this Proposition, we now perform the symmetrization of the system~\eqref{eq:PL}. Remark that in \cite{ABZ2}, for $s>2+{\frac{1}{2}}$, this is achieved by using a technical result in Lemma $4.4$, \cite{ABZ2}: for any $m~,\mu\in {\mathbf{R}}$ there exists a function $C$ such that for all $a\in \Sigma^m$ and $t\in [0, T]$,
\[
\Vert T_{a(t)}u\Vert_{H^{\mu-m}}{\leq} C(\Vert \eta(t)\Vert_{H^{s-1}})\Vert u\Vert_{H^{\mu}}
\]
which says that the operator norm of $T_{a(t)}$ depends only on $\Vert \eta(t)\Vert_{H^{s-1}}$ instead of $\Vert \eta(t)\Vert_{H^{s}}$ when one applies Theorem \ref{theo:sc} $(i)$. In our situation, we shall  use Proposition \ref{regu<0} to handle symbols with negative regularity.
\begin{prop}\label{sym:prop}
Introduce two new unknowns
\[
\Phi_1=T_p\eta,\quad \Phi_2=T_qU.
\]
Then $\Phi_1,~\Phi_2\in C^0([0, T], H^s({\mathbf{R}}))$ and satisfy 
\begin{equation}
\left\{
\begin{aligned}
&\partial_t\Phi_1+T_V\cdot\nabla\Phi_1- T_{\gamma}\Phi_2=F_1,\\
&\partial_t\Phi_2+T_V\cdot\nabla\Phi_2+T_{\gamma}\Phi_2=F_2,
\end{aligned}
\right.
\end{equation}
and there exists a nondecreasing function ${ \mathcal{F}}$ independent of $\eta, \psi$ such that for each $t\in [0, T]$, there holds
\begin{equation}\label{SS:Fi}
{\left\Vert} (F_1, F_2){\right\Vert}_{ H^s\times H^s}{\leq} { \mathcal{F}}{\left(}{\left\Vert} \eta{\right\Vert}_{H^{s_0+{\frac{1}{2}}}}, {\left\Vert} \psi{\right\Vert}_{H^{s_0}}{\right)}\left(1+{\left\Vert}  \eta{\right\Vert}_{C^{r+{\frac{1}{2}}}}+{\left\Vert}  \psi{\right\Vert}_{C^{r}}\right){\left(}1+{\left\Vert}\eta{\right\Vert}_{H^{s+{\frac{1}{2}}}}+{\left\Vert}  \psi{\right\Vert}_{H^{s}}\right).
\end{equation}
\end{prop}
\begin{proof}
It follows directly from the parlinearizedsystem \eqref{eq:PL} and Proposition \ref{pq} that $\Phi_1,~\Phi_2$ satisfy
\begin{equation}
\left\{
\begin{aligned}
&\partial_t\Phi_1+T_V\cdot\nabla\Phi_1- T_{\gamma}\Phi_2=T_pf_1+T_{\partial_tp}\eta+[T_V\cdot\nabla,T_p]\eta,\\
&\partial_t\Phi_2+T_V\cdot\nabla\Phi_2+T_{\gamma}\Phi_2=T_qf_2+T_{\partial_tq}U+[T_V\cdot\nabla,T_q]U.
\end{aligned}
\right.
\end{equation}
For simplicity in notation, we denote the right-hand side of \eqref{SS:Fi} by RHS.
First, Remark \ref{rema:class} and the symbolic calculus from Theorem \ref{theo:sc} $(ii)$ applied with  $\rho=1$ gives
\[
{\left\Vert} [T_V\cdot\nabla,T_p]\eta{\right\Vert}_{H^s}+{\left\Vert} [T_V\cdot\nabla,T_q]U{\right\Vert}_{H^s}{\leq} ~\text{RHS}.
\]
It remains to estimate 
\[
{\left\Vert} T_{\partial_tp}{\right\Vert}_{H^{s+{\frac{1}{2}}}\to H^s}, \quad {\left\Vert} T_{\partial_tq}{\right\Vert}_{H^s\to H^s}.
\]
Recall that we have from the estimates on the Dirichlet-Neumann in Proposition~\ref{prop:DNreg} 
$${\left\Vert}\partial_t\nabla\eta{\right\Vert}_{H^{s_0-2}}\leq { \mathcal{F}}{\left(}{\left\Vert} \eta{\right\Vert}_{H^{s_0+{\frac{1}{2}}}}, {\left\Vert} \psi{\right\Vert}_{H^{s_0}}{\right)},$$
and
\begin{equation}\label{dtdx:eta}
{\left\Vert}\partial_t\nabla\eta{\right\Vert}_{C^{r-2}}\leq { \mathcal{F}}{\left(}{\left\Vert} \eta{\right\Vert}_{H^{s_0+{\frac{1}{2}}}}, {\left\Vert} \psi{\right\Vert}_{H^{s_0}}{\right)}{\left[}1+{\left\Vert}  \eta{\right\Vert}_{C^{r+{\frac{1}{2}}}}+{\left\Vert}  \psi{\right\Vert}_{C^{r}}{\right]}.
\end{equation}
We thus get by Theorem \ref{theo:sc} $(i)$ that
\[
{\left\Vert} T_{\partial_tp^{(1/2)}}{\right\Vert}_{H^{s+{\frac{1}{2}}}\to H^s}{\leq} { \mathcal{F}}{\left(}{\left\Vert} \eta{\right\Vert}_{H^{s_0+{\frac{1}{2}}}}, {\left\Vert} \psi{\right\Vert}_{H^{s_0}}{\right)}{\left[}1+{\left\Vert}  \eta{\right\Vert}_{C^{r+{\frac{1}{2}}}}+{\left\Vert}  \psi{\right\Vert}_{C^{r}}{\right]}.
\]
Thus, we are left with the estimate of ${\left\Vert} T_{\partial_tp^{(-1/2)}}{\right\Vert}_{H^{s+{\frac{1}{2}}}\to H^s}$. Recall that $p^{(-1/2)}$ is of the form 
\[
p^{(-1/2)}=\sum_{|\alpha|=2}F_{\alpha}(\nabla \eta, \xi)\partial_x^{\alpha}\eta,
\]
where the~$F_{\alpha}$s are smooth functions of $\xi$ and homogeneous of order $-{\frac{1}{2}}$. Hence,
\[
\partial_tp^{(-1/2)}=\sum_{|\alpha|=2}[\partial_tF_{\alpha}(\nabla \eta, \xi)]\partial_x^{\alpha}\eta+\sum_{|\alpha|=2}F_{\alpha}(\nabla \eta, \xi)\partial_t\partial_x^{\alpha}\eta.
\]
$(i)$ Since $s_0>{\frac{3}{2}}+\frac{d}{2}$, we have for all $|\alpha|=2$
\[
{\left\Vert} \partial_x^{\alpha}\eta{\right\Vert}_{L^{\infty}}{\leq} {\left\Vert} \partial_x^{\alpha}\eta{\right\Vert}_{H^{s_0-{\frac{3}{2}}}} {\leq} {\left\Vert} \eta{\right\Vert}_{H^{s_0+{\frac{1}{2}}}}.
\]
This estimate together with \eqref{dtdx:eta} implies that $(\partial_tF_{\alpha}(\nabla \eta, \xi))\partial_x^{\alpha}\eta\in \Gamma^{-{\frac{1}{2}}}_0$ and 
\[
M^{-{\frac{1}{2}}}_0\left([\partial_tF_{\alpha}(\nabla \eta, \xi)]\partial_x^{\alpha}\eta\right){\leq} { \mathcal{F}}{\left(}{\left\Vert} \eta{\right\Vert}_{H^{s_0+{\frac{1}{2}}}}, {\left\Vert} \psi{\right\Vert}_{H^{s_0}}{\right)}{\left[}1+{\left\Vert}  \eta{\right\Vert}_{C^{r+{\frac{1}{2}}}}+{\left\Vert}  \psi{\right\Vert}_{C^{r}}{\right]}.
\]
Theorem \ref{theo:sc} $(i)$ then yields
\[
{\left\Vert} T_{[\partial_tF_{\alpha}((\nabla \eta), \xi)]\partial_x^{\alpha}\eta}\eta{\right\Vert}_{H^s}{\leq} 
M^{-{\frac{1}{2}}}_0\left([\partial_tF_{\alpha}(\nabla \eta, \xi)]\partial_x^{\alpha}\eta\right){\left\Vert} \eta{\right\Vert}_{H^{s-{\frac{1}{2}}}}{\leq} ~\text{RHS}.
\]
$(ii)$ Let $G$ be an arbitrary smooth function of $\nabla\eta$. For any $|\alpha|=2$, we apply~\eqref{tame:H<0} with $1<s_0-{\frac{1}{2}}-\frac{d}{2}$ to get
\begin{align*}
{\left\Vert} G(\nabla\eta)\partial_t\partial_x^{\alpha}\eta{\right\Vert}_{C^{-1}}&{\leq} {\left\Vert} G(\nabla\eta ){\right\Vert}_{C^{s_0-{\frac{1}{2}}-\frac{d}{2}}}{\left\Vert} \partial_t\partial_x^2\eta{\right\Vert}_{C^{-1}}\\
&{\leq} \left({\left\Vert} G(\nabla \eta)-G(0){\right\Vert}_{H^{s_0-{\frac{1}{2}}}}+|G(0)|\right){\left\Vert} \partial_t\partial_x^{\alpha}\eta{\right\Vert}_{C^{-1}}.
\end{align*}
Clearly, 
\[
{\left\Vert} G(\nabla \eta)-G(0){\right\Vert}_{H^{s_0-{\frac{1}{2}}}}{\leq} { \mathcal{F}}{\left(}{\left\Vert} \eta{\right\Vert}_{H^{s_0+{\frac{1}{2}}}}{\right)}.
\]
On the other hand, by virtue of Proposition \ref{prop:regellhol}, 
\begin{align*}
{\left\Vert} \partial_t\partial_x^{\alpha}\eta{\right\Vert}_{C^{-1}}&{\leq}{\left\Vert} G(\eta)\psi{\right\Vert}_{C^1}{\leq} {\left\Vert} G(\eta)\psi{\right\Vert}_{C^{r-1}}\\
&{\leq} { \mathcal{F}}{\left(}{\left\Vert} \eta{\right\Vert}_{H^{s_0+{\frac{1}{2}}}}, {\left\Vert} \psi{\right\Vert}_{H^{s_0}}{\right)}{\left[}1+{\left\Vert}  \eta{\right\Vert}_{C^{r+{\frac{1}{2}}}}+{\left\Vert}  \psi{\right\Vert}_{C^{r}}{\right]}.
\end{align*}
Consequently, 
\[
{\left\Vert} G(\nabla \eta)\partial_t\partial_x^{\alpha}\eta{\right\Vert}_{C^{-1}}{\leq} { \mathcal{F}}{\left(}{\left\Vert} \eta{\right\Vert}_{H^{s_0+{\frac{1}{2}}}}, {\left\Vert} \psi{\right\Vert}_{H^{s_0}}{\right)}{\left[}1+{\left\Vert}  \eta{\right\Vert}_{C^{r+{\frac{1}{2}}}}+{\left\Vert}  \psi{\right\Vert}_{C^{r}}{\right]}.
\]
Therefore, according to Definition \ref{defi:symbol<0}, $F_{\alpha}(\nabla \eta, \xi)\partial_t\partial_x^{\alpha}\eta\in \Gamma^{-{\frac{1}{2}}}_{-1}$ with semi-norm 
\[
M^{-{\frac{1}{2}}}_{-1}(F_{\alpha}(\nabla \eta, \xi)\partial_t\partial_x^{\alpha}\eta){\leq} { \mathcal{F}}{\left(}{\left\Vert} \eta{\right\Vert}_{H^{s_0+{\frac{1}{2}}}}, {\left\Vert} \psi{\right\Vert}_{H^{s_0}}{\right)}{\left[}1+{\left\Vert}  \eta{\right\Vert}_{C^{r+{\frac{1}{2}}}}+{\left\Vert}  \psi{\right\Vert}_{C^{r}}{\right]}.
\]
We then obtain by virtue of Proposition \ref{regu<0}
\[
{\left\Vert} T_{F_{\alpha}(\nabla \eta, \xi)\partial_t\partial_x^{\alpha}\eta}\eta{\right\Vert}_{H^s}{\leq} ~\text{RHS}.
\]

For~$\partial_tq$, the proof is the same as for the principal part of~$\partial_tp$, and we only need to remark that
\begin{equation}	\label{eq:estU}
	{\left\Vert} U{\right\Vert}_{H^s}\leq{ \mathcal{F}}{\left(}{\left\Vert} \eta{\right\Vert}_{H^{s_0+{\frac{1}{2}}}}, {\left\Vert} \psi{\right\Vert}_{H^{s_0}}{\right)}{\left[} {\left\Vert}  \eta{\right\Vert}_{H^{s+{\frac{1}{2}}}}+{\left\Vert}  \psi{\right\Vert}_{H^{s}}{\right]}.
\end{equation}

This concludes the proof of the Proposition.
\end{proof}
\section{Blow-up criterion and a priori estimate}\label{section:apriori}
First of all, it follows straightforwardly from Proposition \ref{sym:prop}, that one can reduce the water waves system to a single equation of a complex-valued unknown:
\begin{prop}\label{singleeq:Phi}
Assume that \eqref{assum:sym} holds and let $\Phi_1, \Phi_2$ be as in Proposition \ref{sym:prop} then 
\[
\Phi{\mathrel{:=}} \Phi_1+i\Phi_2=T_p\eta+iT_qU
\]
 satisfies
\begin{gather}
\left( \partial_t+T_V\cdot\nabla+iT_{\gamma}\right)\Phi=F,\label{Phi}\\
{\left\Vert} F(t){\right\Vert}_{ H^s}{\leq} { \mathcal{F}}{\left(}{\left\Vert} \eta{\right\Vert}_{H^{s_0+{\frac{1}{2}}}}, {\left\Vert} \psi{\right\Vert}_{H^{s_0}}{\right)}{\left[}1+{\left\Vert}  \eta{\right\Vert}_{C^{r+{\frac{1}{2}}}}+{\left\Vert}  \psi{\right\Vert}_{C^{r}}{\right]}{\left[}1+{\left\Vert}\eta{\right\Vert}_{H^{s+{\frac{1}{2}}}}+{\left\Vert}  \psi{\right\Vert}_{H^{s}}{\right]} \label{F-tame}.
\end{gather}
\end{prop}
{\hspace*{.15in}} To obtain estimates in Sobolev spaces, we shall commute equation \eqref{Phi} with an elliptic operator $\wp$ of order $s$ and then perform an $L^2$-energy estimate. Since $\gamma^{(3/2)}$ is of order $3/2>1$ we need to choose $\wp$ function of $\gamma^{(3/2)}$ as in \cite{ABZ1}:
\begin{equation}\label{def:wp}
\wp{\mathrel{:=}} (\gamma^{(3/2)})^{2s/3},
\end{equation}
and take ${\varphi}=T_{\wp}\Phi$. Since we want to obtain energy estimates in terms of the original variables~$\eta$ and~$\psi$, we have to link them with this new variable~${\varphi}$. 
\begin{lemm}	\label{lem:estphi}
	For~$s\geq s_0>\frac32+\frac d2$, there is a function~${ \mathcal{F}}$ such that there holds
\begin{gather}
{\left\Vert}{\varphi}{\right\Vert}_{L^2}\leq{ \mathcal{F}}{\left(}{\left\Vert}\eta{\right\Vert}_{H^{s_0+\frac12}}{\right)}{\left[}{\left\Vert}\eta{\right\Vert}_{H^{s+{\frac{1}{2}}}}+{\left\Vert}\psi{\right\Vert}_{H^s}{\right]}\\
{\left\Vert}\eta{\right\Vert}_{H^{s+{\frac{1}{2}}}}+{\left\Vert}\psi{\right\Vert}_{H^s}\leq{ \mathcal{F}}{\left(}{\left\Vert}\eta{\right\Vert}_{H^{s_0+\frac12}},{\left\Vert}\psi{\right\Vert}_{H^{s_0}}{\right)}{\left[} 1+{\left\Vert}{\varphi}{\right\Vert}_{L^2}{\right]} \label{original:varphi} .
\end{gather}
\end{lemm}
\begin{proof}
	Recall that~$p\in\Sigma^s$, $q\in\Sigma^0$, and~$\wp\in\Sigma^s$ since~$\gamma^{(\frac32)}\in\Sigma^{\frac32}$.
	Thus we have
	\begin{multline*}	{\left\Vert}{\varphi}{\right\Vert}_{L^2}\leq{ \mathcal{F}}{\left(}{\left\Vert}\eta{\right\Vert}_{H^{s_0+\frac12}}{\right)}{\left\Vert}\Phi{\right\Vert}_{H^s}
			       \leq{ \mathcal{F}}{\left(}{\left\Vert}\eta{\right\Vert}_{H^{s_0+\frac12}}{\right)}{\left[}{\left\Vert} U{\right\Vert}_{H^s}+{\left\Vert}\eta{\right\Vert}_{H^{s+\frac12}}{\right]}\\
 \leq{ \mathcal{F}}{\left(}{\left\Vert}\eta{\right\Vert}_{H^{s_0+\frac12}}{\right)}{\left[}{\left\Vert}\eta{\right\Vert}_{H^{s+{\frac{1}{2}}}}+{\left\Vert}\psi{\right\Vert}_{H^s}{\right]},
	\end{multline*}
	where we have used~\eqref{eq:estU} to estimate~$U$.
	To prove \eqref{original:varphi} we apply Proposition~\ref{inverse} two times to get
	\begin{align}	\label{eq:esteta}
	{\left\Vert}\eta{\right\Vert}_{H^{s+{\frac{1}{2}}}}\leq{ \mathcal{F}}{\left(}{\left\Vert}\eta{\right\Vert}_{H^{s_0-\frac12}}{\right)}{\left[}{\left\Vert} T_\wp T_p\eta{\right\Vert}_{L^2}+{\left\Vert}\eta{\right\Vert}_{H^{\frac{1}{2}}}{\right]},\\
	\label{eq:estpsi1}
	{\left\Vert}\psi{\right\Vert}_{H^s}\leq{ \mathcal{F}}{\left(}{\left\Vert}\eta{\right\Vert}_{H^{s_0-\frac12}}{\right)}{\left[}{\left\Vert} T_\wp T_q\psi{\right\Vert}_{L^2}+{\left\Vert}\psi{\right\Vert}_{L^2}{\right]}.
	\end{align}
	Clearly, ${\left\Vert} T_\wp T_p\eta{\right\Vert}_{L^2}{\leq} {\left\Vert} \varphi{\right\Vert}_{L^2}$. On the other hand,
	\begin{equation}
			{\left\Vert} T_\wp T_q\psi{\right\Vert}_{L^2}\leq{\left\Vert} T_\wp T_qU{\right\Vert}_{L^2}+{\left\Vert} T_\wp T_qT_B\eta{\right\Vert}_{L^2}
	\end{equation}
	and
	\begin{equation}	
	\begin{aligned}
	{\left\Vert} T_\wp T_qT_B\eta{\right\Vert}_{L^2}&\leq{ \mathcal{F}}{\left(}{\left\Vert}\eta{\right\Vert}_{H^{s_0+\frac12}},{\left\Vert}\psi{\right\Vert}_{H^{s_0}}{\right)}{\left\Vert}\eta{\right\Vert}_{H^{s+{\frac{1}{2}}}}\\
			      &\leq{ \mathcal{F}}{\left(}{\left\Vert}\eta{\right\Vert}_{H^{s_0+\frac12}},{\left\Vert}\psi{\right\Vert}_{H^{s_0}}{\right)}{\left[}{\left\Vert} T_\wp T_p\eta{\right\Vert}_{L^2}+1{\right]},
	\end{aligned}
	\end{equation}
	using~\eqref{eq:esteta}.
	Putting together these estimates proves the proposition.
\end{proof}
For the blow-up criterion and energy estimate below, we recall the following quantities controlling the system \eqref{ww}.
\begin{nota}
The Sobolev norm, blow-up norm and Strichartz norm for $(\eta, \psi)$ are denoted by $M_{s,T},~N_{r, T},~Z_{r,T}$ respectively:
\begin{align}\label{MN}
&M_{s,T}=\Vert (\eta, \psi)\Vert_{L^{\infty}([0, T]; H^{s+{\frac{1}{2}}}\times H^s)}, \quad M_{s,0}=\Vert (\eta, \psi)\arrowvert_{t=0}\Vert_{H^{s+{\frac{1}{2}}}\times H^s},\\
&N_{r,T}=\Vert (\eta, \psi)\Vert_{L^1([0, T]; W^{r+{\frac{1}{2}}, \infty}\times W^{r, \infty})},\\
& Z_{r,T}=\Vert (\eta, \psi)\Vert_{L^p([0, T]; W^{r+{\frac{1}{2}}, \infty}\times W^{r, \infty})}.
\end{align}
\end{nota}
\begin{prop}\label{L2varphi}
	Let~$d\geq 1$, $h>0$, and indices
	$$\frac32+\frac d2<s_0\leq s,\quad2<r<s_0+\frac12-\frac d2.$$
	Then there exists a non-negative, non-decreasing function~${ \mathcal{F}}_h$ such that for all $T\ge 0$ and
	$$(\eta,\psi)\in L^{\infty}{\left(}[0, T]; H^{s+{\frac{1}{2}}}\times H^s{\right)}\cap L^1{\left(}[0, T]; W^{r+{\frac{1}{2}}, \infty}\times W^{r, \infty}{\right)}$$
	solution of the Zacharov-Craig-Sulem system~\eqref{ww} satisfying condition~\eqref{sepbot}, there holds
	$${\left\Vert} \varphi{\right\Vert}_{L^\infty([0, T]; L^2)}\leq { \mathcal{F}}_h{\left(} M_{s, 0}+{ \mathcal{F}}{\left(} M_{s_0, T}{\right)}{\left[} T+N_{r, T}{\right]}{\right)}.$$
\end{prop}
\begin{rema}
In general, ${ \mathcal{F}}_h$ depends also on $d,~s,~r,~s_0$.
\end{rema}
\begin{proof}
	Using Grönwall lemma and the fact that ${\left\Vert} {\varphi}(0){\right\Vert}_{L^2}{\leq} { \mathcal{F}}(M_{s,0})$, we see that the Proposition will be a consequence of the following estimate for~${\varphi}$ :
	\begin{equation}	\label{eq:enerL^2}
		\frac{\,\mathrm{d}}{{\,\mathrm{d}} t}{\left\Vert}{\varphi}{\right\Vert}_{L^2}^2\leq { \mathcal{F}}{\left(}{\left\Vert} \eta{\right\Vert}_{H^{s_0+{\frac{1}{2}}}}, {\left\Vert} \psi{\right\Vert}_{H^{s_0}}{\right)}{\left[}1+{\left\Vert}  \eta{\right\Vert}_{C^{r+{\frac{1}{2}}}}+{\left\Vert}  \psi{\right\Vert}_{C^{r}}{\right]}{\left[}1+{\left\Vert}{\varphi}{\right\Vert}_{L^2}{\right]}{\left\Vert}{\varphi}{\right\Vert}_{L^2}.
	\end{equation}
	
	To prove this estimate, we see from~\eqref{Phi} that~${\varphi}$ solves the equation
	\begin{equation}\label{varphi}
	\left( \partial_t+T_V\cdot\nabla+iT_{\gamma}\right)\varphi=T_{\wp}F+G
	\end{equation}
	where 
	\[
	G=T_{\partial_t\wp}\Phi+[T_V\cdot\nabla, T_{\wp}]\Phi+i[T_{\gamma}, T_{\wp}]\Phi.
	\]
	 First, remark that since $\partial_{\xi}\wp\cdot\partial_x\gamma^{(3/2)}=\partial_{\xi}\gamma^{(3/2)}\cdot\partial_x\wp$ we can apply Theorem~\ref{theo:sc} $(ii)$ with $m=s,~ m'={\frac{3}{2}},~\rho={\frac{3}{2}}$ to find (keep in mind Remark \ref{rema:class})
\[
\Vert [T_{\wp}, T_{\gamma}]\Vert_{H^s\to L^2}{\leq} { \mathcal{F}}{\left(}{\left\Vert}\eta{\right\Vert}_{H^{s_0+\frac12}},{\left\Vert}\psi{\right\Vert}_{H^{s_0}}{\right)}{\left[}1+{\left\Vert}\psi{\right\Vert}_{C^r}+{\left\Vert}\eta{\right\Vert}_{C^{r+{\frac{1}{2}}}}{\right]}.
\]
The same theorem (applied with $m=1,~m'=s,~\rho=1$) also gives
\[
\Vert [T_V\cdot\nabla, T_{\wp}]\Vert_{H^s\to L^2}{\leq} { \mathcal{F}}{\left(}{\left\Vert}\eta{\right\Vert}_{H^{s_0+\frac12}},{\left\Vert}\psi{\right\Vert}_{H^{s_0}}{\right)}{\left[}1+{\left\Vert}\psi{\right\Vert}_{C^r}+{\left\Vert}\eta{\right\Vert}_{C^{r+{\frac{1}{2}}}}{\right]}.
\]
Finally, one can write $\partial_t\wp=L(\nabla\eta, \partial_t\nabla \eta, \xi)$ for some smooth function $L$ homogeneous of order $s$ in $\xi$, so that
\[
\Vert T_{\partial_t\wp}\Vert_{H^s\to L^2}{\leq} { \mathcal{F}}{\left(}{\left\Vert}\eta{\right\Vert}_{H^{s_0+\frac12}},{\left\Vert}\psi{\right\Vert}_{H^{s_0}}{\right)}{\left[}1+{\left\Vert}\psi{\right\Vert}_{C^r}+{\left\Vert}\eta{\right\Vert}_{C^{r+{\frac{1}{2}}}}{\right]}.
\]
The estimates above imply 
$${\left\Vert} G{\right\Vert}_{L^2}\leq{ \mathcal{F}}{\left(}{\left\Vert}\eta{\right\Vert}_{H^{s_0+\frac12}},{\left\Vert}\psi{\right\Vert}_{H^{s_0}}{\right)}{\left[}1+{\left\Vert}\psi{\right\Vert}_{C^r}+{\left\Vert}\eta{\right\Vert}_{C^{r+{\frac{1}{2}}}}{\right]}{\left\Vert}\Phi{\right\Vert}_{H^s}.$$
On the other hand,  Proposition~\ref{inverse} applied to $u=\Phi,~a=\wp\in \Sigma^s$ yields
$${\left\Vert}\Phi{\right\Vert}_{H^s}\leq{ \mathcal{F}}{\left(}{\left\Vert}\eta{\right\Vert}_{H^{s_0+\frac12}}, {\left\Vert}\psi{\right\Vert}_{H^{s_0}}{\right)}{\left[}{\left\Vert}{\varphi}{\right\Vert}_{L^2}+1{\right]}.$$
Therefore,
$${\left\Vert} G{\right\Vert}_{L^2}\leq{ \mathcal{F}}{\left(}{\left\Vert}\eta{\right\Vert}_{H^{s_0+\frac12}},{\left\Vert}\psi{\right\Vert}_{H^{s_0}}{\right)}{\left[}1+{\left\Vert}\psi{\right\Vert}_{C^r}+{\left\Vert}\eta{\right\Vert}_{C^{r+{\frac{1}{2}}}}{\right]}{\left[}1+{\left\Vert}{\varphi}{\right\Vert}_{L^2}{\right]}.$$
On the other hand, we see from~\eqref{F-tame} that
\begin{multline*}{\left\Vert} T_\wp F{\right\Vert}_{L^2}\leq{ \mathcal{F}}{\left(}{\left\Vert}\eta{\right\Vert}_{H^{s_0+\frac12}},{\left\Vert}\psi{\right\Vert}_{H^{s_0}}{\right)}{\left[}1+{\left\Vert}\psi{\right\Vert}_{C^r}+{\left\Vert}\eta{\right\Vert}_{C^{r+{\frac{1}{2}}}}{\right]}\times\\
\times {\left[}1+{\left\Vert}\psi{\right\Vert}_{H^s}+{\left\Vert}\eta{\right\Vert}_{H^{s+{\frac{1}{2}}}}{\right]},\end{multline*}
so that thanks to Lemma~\ref{lem:estphi} we have
$${\left\Vert} T_\wp F{\right\Vert}_{L^2}\leq{ \mathcal{F}}{\left(}{\left\Vert}\eta{\right\Vert}_{H^{s_0+\frac12}},{\left\Vert}\psi{\right\Vert}_{H^{s_0}}{\right)}{\left[}1+{\left\Vert}\psi{\right\Vert}_{C^r}+{\left\Vert}\eta{\right\Vert}_{C^{r+{\frac{1}{2}}}}{\right]}{\left[}1+{\left\Vert}{\varphi}{\right\Vert}_{L^2}{\right]}.$$
Now, using Theorem \ref{theo:sc} $(iii)$ we see easily that 
\begin{equation}\label{energy:gamma}
\Vert (T_V\cdot\nabla)+(T_V\cdot\nabla)^*\Vert_{L^2\to L^2}\leq{ \mathcal{F}}{\left(}{\left\Vert}\eta{\right\Vert}_{H^{s_0+\frac12}},{\left\Vert}\psi{\right\Vert}_{H^{s_0}}{\right)}{\left[}1+{\left\Vert}\psi{\right\Vert}_{C^r}+{\left\Vert}\eta{\right\Vert}_{C^{r+{\frac{1}{2}}}}{\right]}
\end{equation}
and
\begin{equation}\label{energy:V}
\Vert (T_{\gamma})+(T_{\gamma})^*\Vert_{L^2\to L^2}\leq{ \mathcal{F}}{\left(}{\left\Vert}\eta{\right\Vert}_{H^{s_0+\frac12}},{\left\Vert}\psi{\right\Vert}_{H^{s_0}}{\right)}{\left[}1+{\left\Vert}\psi{\right\Vert}_{C^r}+{\left\Vert}\eta{\right\Vert}_{C^{r+{\frac{1}{2}}}}{\right]}.
\end{equation}
Then using equation \eqref{varphi} we conclude the proof of~\eqref{eq:enerL^2} and thus of the Proposition.
\end{proof}
Now, taking~$s>2+\frac d2$ and 
\[
(\eta_0, \psi_0)\in  H^{s+{\frac{1}{2}}}\times H^s
\]
such that $\operatorname{dist}(\eta_0, \Gamma)>h>0$, we know from Theorem $1.1$, \cite{ABZ1} that there exists a time $ T\in (0, \infty)$ such that the Cauchy problem for system \eqref{ww} with initial condition $(\eta_0, \psi_0)$ has a unique solution 
\[
(\eta, \psi)\in C{\left(}[0, T];  H^{s+{\frac{1}{2}}}\times H^s{\right)}
\]
 The maximal time of existence $T^*>0$ then can be defined as
\begin{multline}\label{def:T*}
T^*=T^*(\eta_0, \psi_0, h):=\sup\left\{ T'>0:  ~\text {the Cauchy problem for \eqref{ww} with data}\right.\\
\left.  ~(\eta_0, \psi_0)~\text{has a solution}~(\eta, \psi)\in C([0, T'];  H^{s+{\frac{1}{2}}}\times H^s) \right.\\
\left. ~\text{satisfying}~ \inf_{[0, T']}\operatorname{dist}(\eta(t), \Gamma)>0\right\}.
\end{multline}
  By the uniqueness statement of Proposition $6.4$, \cite{ABZ1} (it is because of this Proposition that we require the separation condition in the definition \eqref{def:T*}) the solution $(\eta, \psi)$ is defined for all $t<T^*$ and 
\[
(\eta, \psi)\in C{\left(}[0, T^*);  H^{s+{\frac{1}{2}}}\times H^s{\right)},
\]
which will be called the maximal solution.
\begin{theo}\label{theo:blowup}
	Let~$d\geq 1,~h>0$ and indices
	$$\frac32+\frac d2<s_0<s-{\frac{1}{2}},\quad2<r<s_0+\frac12-\frac d2.$$
 Let $T^*=T^*(\eta_0, \psi_0, h)$ be the maximal time of existence defined by \eqref{def:T*} and
 \begin{equation}
(\eta,\psi)\in L^{\infty}{\left(}[0, T^*); H^{s+{\frac{1}{2}}}\times H^s{\right)}
\end{equation}
	be the maximal solution of ~\eqref{ww} with prescribed data $(\eta_0, \psi_0)$ satisfying $\operatorname{dist}(\eta_0, \Gamma)>h$.
	Then if ~$T^*$ is finite, it holds that
	$$\label{criterion:blowup}\limsup_{T\rightarrow T^*}\Big( M_{s_0, T}+N_{r, T}+\frac{1}{h(T)}\Big)=+\infty,$$
where $h(T)$ is the distance from the surface $\eta$ to the bottom $\Gamma$ over the time  interval $[0, T]$.
\end{theo}
\begin{proof}
Suppose that $T^*<+\infty$ and
\[
K:=\limsup_{T\rightarrow T^*}\Big( M_{s_0, T}+N_{r, T}+\frac{1}{h(T)}\Big)<+\infty.
\]
Let $T\in [0, T^*)$ arbitrary then $h(T)>1/K>0$. It follows from Proposition \ref{L2varphi} and the estimate \eqref{original:varphi} that
\begin{equation}\label{estforblowup}
 M_{s,T}{\leq} { \mathcal{F}}_{1/K}{\left(} T, M_{s, 0}, M_{s_0, T}, N_{r, T}{\right)}{\leq} L
\end{equation}
for some function ${ \mathcal{F}}_{1/K}$ increasing in each argument and $L=L(K, M_{s,0}, T^*)$. On the other hand, from the a priori estimate in Proposition $5.2$, \cite{ABZ1} we deduce that the existence time for local solutions can be chosen uniformly for data lying in a bounded subset of $H^{s+{\frac{1}{2}}}\times H^s$ and satisfy uniformly the separation condition $(H_0)$. In particular, call $T_1$ be the time of existence for data in the ball $B(0, L)$ of  $H^{s+{\frac{1}{2}}}\times H^s$ whose surface is away from the bottom a distance (at least) $1/K$. Choosing $\eta(T^*-\frac{T_1}{2})$ as such a datum we can prolong the solution up to the time $T^*+\frac{T_1}{2}$ which contradicts the maximality of $T^*$, the theorem is proved.
\end{proof}
The preceding result means that one can continue a solution satisfying the separation condition $(H_t)$ as long as the Sobolev norm $M_{s_0}$ for any index~$s_0>{\frac{3}{2}}+\frac d2$ stays bounded, and the time integral of the H\"older norms at regularity~$r=2+{\varepsilon}$, $N_r$, is finite. \\
{\hspace*{.15in}} Now we give the proof of Corollary \ref{intro:prop:regularity} which is stated again for reader's convenience.
\begin{coro}\label{theo:regularity}
Let $T\in (0, +\infty)$ and $(\eta, \psi)$ be a distributional solution to system \eqref{ww} on the time interval $[0, T]$  such that  $\inf_{[0, T]}\operatorname{dist}(\eta(t), \Gamma)>0.$ Then the following property holds: if one knows a priori that 
\begin{equation}\label{condition:reg}
\sup_{[0, T]}\| (\eta(t), \psi(t))\|_{H^{2+\frac{d}{2}+}\times H^{{\frac{3}{2}}+\frac{d}{2}+}}+\int_{0}^T\| (\eta(t), \psi(t))\|_{C^{\frac{5}{2}+}\times C^{2+}}{\,\mathrm{d}} t<+\infty
\end{equation}
 then $(\eta(0), \psi(0))\in  H^\infty({\mathbf{R}}^d)^2$ implies that $(\eta, \psi)\in  L^\infty([0, T]; H^\infty({\mathbf{R}}^d))^2$.
\end{coro}
\begin{proof}
From condition \eqref{condition:reg} one can find $s_0,~r$ verifying
	$$\frac32+\frac d2<s_0,\quad2<r<s_0+\frac12-\frac d2$$
such that 
\begin{equation}\label{reg:1}
M_{s_0, T}+N_{r,T}<\infty.
\end{equation}
Take $s>s_0+{\frac{1}{2}}$ arbitrary, it suffices to prove that if $(\eta(0), \psi(0))\in  H^{s+{\frac{1}{2}}}\times H^{s}$ then $(\eta, \psi)\in  L^\infty([0, T]; H^{s+{\frac{1}{2}}}\times H^{s})$. Since $s>2+\frac{d}{2}$, according to the Cauchy theory in \cite{ABZ1} one has a maximal solution 
\[
(\eta, \psi)\in  L^\infty([0, T_s); H^{s+{\frac{1}{2}}}\times H^{s}).
\]
By uniqueness, we only need to show that $T_s\ge T$. Suppose that $T_s<T<+\infty$ we get by applying Theorem \ref{theo:blowup}
	$$\limsup_{T'\rightarrow T_s}\Big( M_{s_0, T'}+N_{r, T'}+\frac{1}{h(T')}\Big)=+\infty.$$
By our assumption, $h(T')\ge h(T_s)>0$, hence
$$\limsup_{T'\rightarrow T_s}\Big( M_{s_0, T'}+N_{r, T'}\Big)=+\infty$$
which contradicts \eqref{reg:1}.
\end{proof}
\begin{rema}
In fact, the proof of Corollary \ref{theo:regularity} above shows  a stronger regularity result: under condition \eqref{condition:reg}, for any $s>2+\frac{d}{2}+$ we have $(\eta(0), \psi(0))\in  H^{s+{\frac{1}{2}}}\times H^{s}$ implies $(\eta, \psi)\in  L^\infty([0, T]; H^{s+{\frac{1}{2}}}\times H^{s})$. The value of this result over the Cauchy theory in \cite{ABZ1} is that here the existence time $T$ is given and is independent of the Sobolev index $s$.
\end{rema}
{\hspace*{.15in}} We next derive from Proposition \ref{L2varphi} an a priori estimate for the Sobolev norm $M_{s,T}$ by means of itself and the Strichartz norm $Z_{r, T}$.
\begin{theo}\label{theo:aprioriestimate}
	Let~$d\geq 1$, $h>0,~p> 1$, and indices
	$$s>\frac32+\frac d2,\quad2<r<s+\frac12-\frac d2.$$
	Then there exists a non-negative, non-decreasing function~${ \mathcal{F}}_h$ such that for all  $T\in [0, 1)$ and 
	$$(\eta,\psi)\in L^{\infty}{\left(}[0, T]; H^{s+{\frac{1}{2}}}\times H^s{\right)}\cap L^p{\left(}[0, T]; W^{r+{\frac{1}{2}}, \infty}\times W^{r, \infty}{\right)}$$
	 solution of the Zacharov-Craig-Sulem system~\eqref{ww} satisfying~$\inf_{t\in [0,T]}\operatorname{dist}(\eta(t), \Gamma)>h$ we have
	\begin{equation}\label{aprioriestimate} M_{s, T}\leq{ \mathcal{F}}_h{\left(} M_{s,0}+T^\delta{ \mathcal{F}}{\left(} M_{s, T}+Z_{r,T}{\right)}{\right)},\end{equation}
where $\delta=\min\{1-\frac{1}{p},~{\frac{1}{2}}\}$.
\end{theo}
\begin{proof}
First, by H\"older inequality we have $N_{r, T}{\leq} T^{1-\frac{1}{p}}Z_{r,T}$ and thus Proposition \ref{L2varphi} applied with $s_0=s$ implies that
\begin{equation}\label{energy:1}
{\left\Vert} T_\wp T_p\eta{\right\Vert}_{L^\infty([0, T]; L^2)}+{\left\Vert} T_\wp T_qU{\right\Vert}_{L^\infty([0, T]; L^2)}\leq { \mathcal{F}}{\left(} M_{s, 0}+T^{1-\frac{1}{p}}{ \mathcal{F}}{\left(} M_{s, T}+Z_{r, T}{\right)}{\right)}.
\end{equation}
We denote by $\Xi$ the right-hand side of the preceding inequality, where ${ \mathcal{F}}$ may change from line to line. Using the estimate for the Dirichlet-Neumann operator in Proposition \ref{prop:DNreg} we get
\begin{equation}\label{eta:s-1}
\Vert\eta(t)-\eta(0)\Vert_{H^{s-1}}{\leq}\int_0^t\Vert\partial_t\eta(m)\Vert_{H^{s-1}}dm= \int_0^t\Vert G(\eta)\psi(m)\Vert_{H^{s-1}}dm{\leq} T{ \mathcal{F}}{\left(} M_{s,T}{\right)}.
\end{equation}
Consequently,
\begin{multline}\label{eta:s-mez}
\Vert\eta(t)\Vert_{H^{s-{\frac{1}{2}}}}{\leq} \Vert\eta(0)\Vert_{H^{s-{\frac{1}{2}}}}+\Vert\eta(t)-\eta(0)\Vert_{H^{s-{\frac{1}{2}}}}
\\   {\leq} \Vert\eta(0)\Vert_{H^{s-{\frac{1}{2}}}}+\Vert\eta(t)-\eta(0)\Vert_{H^{s-1}}^{\frac{1}{2}}\Vert\eta(t)-\eta(0)\Vert_{H^s}^{\frac{1}{2}}
{\leq} M_{s, 0}+T^{\frac{1}{2}}{ \mathcal{F}}{\left(} M_s(T){\right)}.
\end{multline}
The estimates \eqref{eq:esteta}, \eqref{energy:1} and \eqref{eta:s-mez} then give 
\begin{equation}\label{energy:eta}
{\left\Vert} \eta{\right\Vert}_{L^\infty H^{s+{\frac{1}{2}}}}{\leq} \Xi.
\end{equation}
 We turn to estimate ${\left\Vert} \psi{\right\Vert}_{L^\infty H^s}$, for which we use the second equation in \eqref{ww} to get
\[
\Vert\psi(t)-\psi(0)\Vert_{ H^{s-{\frac{3}{2}}}}{\leq} T{ \mathcal{F}}{\left(} M_{s,T}{\right)}.
\]
By interpolation as in \eqref{eta:s-1}, there holds
\begin{equation}\label{psi:s-1}
{\left\Vert} \psi(t){\right\Vert}_{H^{s-1}}{\leq} {\left\Vert} \psi(0){\right\Vert}_{H^{s-1}} +\sqrt{T} { \mathcal{F}}{\left(} M_{s,T}{\right)}.
\end{equation}
Then, in views of \eqref{eq:estpsi1} and \eqref{eta:s-mez}  it remains to estimate $\Vert T_{\wp}T_q\psi\Vert_{L^{\infty}([0, T], L^2)}$. To do this, one writes by definition of $U$
\[
\Vert T_{\wp}T_q\psi\Vert_{L^{\infty}([0, T], L^2)}{\leq} \Vert T_{\wp}T_qU\Vert_{L^{\infty}([0, T], L^2)}+\Vert T_{\wp}T_qT_B\eta\Vert_{L^{\infty}([0, T], L^2)}.
\]
The second term on the right-hand side is bounded by \eqref{energy:1}. For the second term, one uses \eqref{energy:eta} to have
\[
\Vert T_{\wp}T_qT_B\eta\Vert_{L^{\infty}([0, T], L^2)} {\leq} \Xi {\left\Vert} T_B\eta{\right\Vert}_{L^\infty H^s}{\leq} \Xi{\left\Vert} B{\right\Vert}_{L^\infty C^{-{\frac{1}{2}}}}{\left\Vert} \eta {\right\Vert}_{L^\infty H^{s+{\frac{1}{2}}}}.
\]
Thus, to complete the proof we are left with ${\left\Vert} B{\right\Vert}_{L^\infty C^{-{\frac{1}{2}}}}$, for which we use again the decomposition \eqref{decompose:B} for $B$:
\[
B=K(\nabla\eta)\cdot\nabla\psi+L(\nabla\eta)G(\eta)\psi.
\]
Then by \eqref{tame:H<0} and the estimate for $G(\eta)\psi$ in Theorem  \ref{DN:Sobolev}, there hold
\begin{multline*}
{\left\Vert} B{\right\Vert}_{C^{-{\frac{1}{2}}}}{\leq}{\left\Vert} K(\nabla\eta){\right\Vert}_{C^1}{\left\Vert} \nabla\psi{\right\Vert}_{C^{-{\frac{1}{2}}}}+{\left\Vert} L(\nabla\eta){\right\Vert}_{C^1}{\left\Vert} G(\eta)\psi{\right\Vert}_{C^{-{\frac{1}{2}}}}\\
 {\leq} { \mathcal{F}}{\left(}{\left\Vert} \eta{\right\Vert}_{H^{s+{\frac{1}{2}}}}{\right)}{\left(} {\left\Vert} \nabla \psi{\right\Vert}_{H^{s-2}} +{\left\Vert} G(\eta)\psi{\right\Vert}_{H^{s-2}} {\right)}
 {\leq} { \mathcal{F}}{\left(}{\left\Vert} \eta{\right\Vert}_{H^{s+{\frac{1}{2}}}}{\right)}{\left\Vert}\psi{\right\Vert}_{H^{s-1}}.
\end{multline*}
The estimates \eqref{energy:eta} and \eqref{psi:s-1} then conclude the proof.
\end{proof}
\section{Contraction estimates}\label{section:contraction}
Our goal in this section is to prove a contraction estimate for two solutions to \eqref{ww} in  weaker norms. This will be used in the proof of the convergence of the scheme and in establishing uniqueness for the Cauchy theory in our companion paper \cite{NgPo}. To get started, we have by straightforward computations the following assertion: $(\eta, \psi)$ is a solution to system \eqref{ww} if and only if 
\[
(\partial_t+T_V\cdot\nabla+\mathcal{L})\begin{pmatrix}\eta \\ \psi\end{pmatrix} =f(\eta, \psi)
\]
with
\begin{equation}\label{defi:L}
\mathcal{L}:=
\begin{pmatrix}
I &  0\\
T_B& I
\end{pmatrix}
\begin{pmatrix}
0 &  -T_\lambda\\
T_l& 0
\end{pmatrix}
\begin{pmatrix}
I &  0\\
-T_B& I
\end{pmatrix},\quad
f(\eta, \psi):=\begin{pmatrix}
I &  0\\
T_B& I
\end{pmatrix}
\begin{pmatrix}
f^1\\
f^2
\end{pmatrix}
.
\end{equation}
where
\begin{equation}\label{f1f2}
\begin{aligned}
f^1(\eta, \psi)&=G(\eta)\psi-{\left(} T_{\lambda}(\psi-T_B\eta)-T_V\cdot\nabla\eta {\right)},\\
f^2(\eta, \psi)&=-{\frac{1}{2}}|\nabla\psi|^2+{\frac{1}{2}}\frac{(\nabla \eta \cdot \nabla\psi + G(\eta)\psi)^2}{1+ \vert \nabla \eta \vert^2}\\
\quad & +T_V\cdot\nabla \psi-T_BT_V\cdot\nabla\eta-T_BG(\eta)\psi+ H(\eta)+T_l\eta-g\eta.
\end{aligned}
\end{equation}
We consider $(\eta, \psi)$ at the following regularity level
\[
(\eta, \psi)\in L^\infty{\left(}[0, T]; H^{s+{\frac{1}{2}}}\times H^s{\right)}\cap L^p{\left(}[0 ,T]; W^{r+{\frac{1}{2}},\infty}\times W^{r,\infty}{\right)},
\]
with
\[
s>{\frac{3}{2}}+\frac d2,~\quad 2<r<s-\frac d2+{\frac{1}{2}}.
\]Now, let $(\eta_1, \psi_1)$ and $(\eta_2, \psi_2)$ be two solutions of system \eqref{ww} on $[0, T]$. Set 
\[
\delta \eta=\eta_1-\eta_2,\quad \delta\psi=\psi_1-\psi_2, \quad \delta B=B_1-B_2,\quad \delta V=V_1-V_2.
\]
Define the following quantities
\begin{equation}\label{quantities:contraction}
\begin{aligned}
&P_S(t)={\left\Vert} \delta \eta(t){\right\Vert}_{H^{s-1}}+{\left\Vert} \delta \psi(t){\right\Vert}_{H^{s-{\frac{3}{2}}}},\quad P_H(t)={\left\Vert} \delta\eta(t){\right\Vert}_{C^{r-1}}+{\left\Vert} \delta\psi(t){\right\Vert}_{C^{r-{\frac{3}{2}}}},\\
&P_{S, T}={\left\Vert} P_S{\right\Vert}_{L^\infty([0, T])}, \quad P_{H,T}={\left\Vert} P_S{\right\Vert}_{L^p([0, T])},\\ &P(t)=P_S(t)+P_H(t), \quad P_T=P_{S,T}+P_{H,T}.
\end{aligned}
\end{equation}
\begin{nota}
Throughout this section, we write $A{\lesssim} B$ if there exists a non-decreasing function ${ \mathcal{F}}:{\mathbf{R}}^+\to {\mathbf{R}}^+$ such that $A{\leq} { \mathcal{F}}(M^1_{s, T}, M^2_{s,T})B$,  where $M^j_{s,T}$ is defined by \eqref{MN}: $M^j_{s,T}=\Vert (\eta_j, \psi_j)\Vert_{L^{\infty}([0, T]; H^{s+{\frac{1}{2}}}\times H^s)}.$
\end{nota}
\subsection{Contraction estimate for $f^2$}
Recall that we consider $B,~V$ as functions of $(\eta, \psi)$ defined by \eqref{BV}.
\begin{lemm}\label{dB,dV}
We have for a.e. $t\in [0, T]$
\[
{\left\Vert} \delta B(t){\right\Vert}_{C^{-{\frac{1}{2}}}}+{\left\Vert} \delta V(t){\right\Vert}_{C^{-{\frac{1}{2}}}}{\lesssim} P(t).
\]
\end{lemm}
\begin{proof}
Assume that the estimate for $\delta B$ is proved, we have
\[
\delta V=\nabla \delta \psi-\delta B\nabla \eta_1-B_2\nabla\delta\eta.
\]
Obviously, 
\[
\|\nabla \delta \psi(t)\|_{C^{-{\frac{1}{2}}}}{\leq} \| \delta \psi(t)\|_{C^{\frac{1}{2}}}{\leq} \| \delta \psi(t)\|_{C^{r-{\frac{1}{2}}}}{\leq} P_H(t).
\]
On the other hand, 
\[
{\left\Vert} B_2\nabla\delta\eta(t){\right\Vert}_{C^{-{\frac{1}{2}}}}{\leq} {\left\Vert} B_2\nabla\delta\eta(t){\right\Vert}_{L^\infty}{\lesssim} {\left\Vert} \delta\eta(t){\right\Vert}_{W^{1,\infty}}{\leq}  P_H(t)
\]
and from the product rule \eqref{tame:H<0} for negative H\"older indices there holds
\[
{\left\Vert} \delta B\nabla \eta_1(t){\right\Vert}_{C^{-{\frac{1}{2}}}}{\lesssim} {\left\Vert} \delta B(t){\right\Vert}_{C^{-{\frac{1}{2}}}}{\left\Vert} \nabla\eta_1(t){\right\Vert}_{C^{{\frac{1}{2}}+{\varepsilon}}}{\lesssim} P(t)
\]
since for ${\varepsilon}>0$ small enough ${\left\Vert} \nabla\eta_1(t){\right\Vert}_{C^{{\frac{1}{2}}+{\varepsilon}}}{\lesssim} {\left\Vert} \nabla\eta_1(t){\right\Vert}_{H^{s-{\frac{1}{2}}}}.$
Therefore, we are left with the estimate for $\delta B$, for which we use again the formula \eqref{decompose:B}
\[
B=K(\nabla\eta)\cdot\nabla\psi+L(\nabla\eta)G(\eta)\psi,
\]
	with~$K$ and~$L$ smooth. Observe that $G(\eta)$ has order $1$, hence these two terms are at the same regularity structure. We give the proof for the second one since it involves the  Dirichlet-Neumann operator:
\begin{multline*}
L(\nabla\eta_1)G(\eta_1)\psi_1-L(\nabla\eta_2)G(\eta_2)\psi_2=[L(\nabla\eta_1)-L(\nabla\eta_2)]G(\eta_1)\psi_1\\
+L(\nabla\eta_2)[G(\eta_1)\psi_1-G(\eta_2)\psi_2].
\end{multline*}
From this expression and the product rule \eqref{tame:H<0} we only need to estimate the $C^{-1/2}$ norm of 
\[
G(\eta_1)\psi_1-G(\eta_2)\psi_2=G(\eta_1)\delta\psi-[G(\eta_1)-G(\eta_2)]\psi_2,
\]
where the H\"older estimate \eqref{est:DN:Holder} applied with $\mu=5/2$ gives ${\left\Vert} G(\eta_1)\delta\psi{\right\Vert}_{C^{-{\frac{1}{2}}}}{\lesssim} P(t).$ For the second term on the right-hand side, we apply for example, Theorem $5.3$, \cite{ABZ3} on the  contraction estimate for the Dirichlet-Neumann operator to get (since $s-1/2>1+d/2$)
\[
{\left\Vert} [G(\eta_1)-G(\eta_2)]\psi_2{\right\Vert}_{H^{s-2}}{\lesssim} {\left\Vert} \delta \eta{\right\Vert}_{H^{s-1}}.
\]
Then the embedding $H^{s-2}\hookrightarrow C^{-1/2}$ concludes the proof.
\end{proof}
We introduce the following notation.
\begin{nota}
Let $f:{\mathbf{R}}^d\to {\mathbf{C}}^d$ be a  function of $u$, we set 
\[
{\,\mathrm{d}}_uf(u)\dot u=\lim_{{\varepsilon}\to 0}\{f(u+{\varepsilon} \dot u)-f(u)\}.
\]
\end{nota}
\begin{prop}\label{contraction:f2}
With $f^2$ defined in \eqref{f1f2}, it holds for a.e. $t\in [0, T]$ that
\[
{\left\Vert} f^2(\eta_1,\psi_1)(t)-f^2(\eta_2, \psi_2)(t){\right\Vert}_{H^{s-{\frac{3}{2}}}}{\lesssim} P(t).
\]
\end{prop}
\begin{proof}
It suffices to prove that 
\begin{equation}\label{df2}
{\left\Vert} d_\eta f^2(\eta, \psi)\dot\eta+ {\,\mathrm{d}}_\psi f^2(\eta, \psi)\dot\psi {\right\Vert}_{H^{s-{\frac{3}{2}}}}{\lesssim} \|\dot\eta\|_{H^{s-1}}+\| \dot\eta\|_{C^{r-1}}+\| \dot\psi\|_{H^{s-{\frac{3}{2}}}}+\| \dot\psi\|_{C^{r-{\frac{3}{2}}}}.
\end{equation}
We have $f^2(\eta, \psi)=I_1+I_2+I_3$ with
\begin{align*}
I_1&:=H(\eta)+T_l\eta,\\
I_2&:=-{\frac{1}{2}}|\nabla\psi|^2+{\frac{1}{2}}\frac{(\nabla \eta \cdot \nabla\psi + G(\eta)\psi)^2}{1+ \vert \nabla \eta \vert^2}+T_V\cdot\nabla \psi-T_BT_V\cdot\nabla\eta-T_BG(\eta)\psi,\\
I_3&:=-g\eta.
\end{align*}
Observe that ${\,\mathrm{d}}_\psi I_1={\,\mathrm{d}}_\psi I_3=0$. The estimate for $d_\eta I_3\dot\eta=-g\dot\eta$ is obvious. Observe that $I_1$ and $I_2$ are the remainder of the paralinearization in Lemmas \ref{lem:paracurv} and  \ref{lem:paraquad}, respectively. Putting $f(x)=-x(1+|x|^2)^{-1/2},~x\in {\mathbf{R}}^d$, we have $H(\eta)=\operatorname{div} f(\nabla \eta)$. Since
\[
d_\eta f(\nabla \eta)\dot\eta=f'(\nabla\eta)\nabla\dot\eta,
\]
it follows that 
\[
d_\eta H(\eta)\dot \eta=\operatorname{div}(f'(\nabla\eta)\nabla)\dot\eta+f'(\nabla\eta)\nabla\cdot\nabla\dot\eta.
\]
Then using the Bony decomposition we get 
\[
d_\eta H(\eta)\dot \eta=T_{i\operatorname{div}(f'(\nabla\eta)\xi)}\dot\eta+T_{-f'(\nabla\eta)\xi\cdot\xi}\dot\eta+R=T_{-l}\dot\eta+R
\]
with $\| R\|_{H^{s-3/2}}{\lesssim}  {\left\Vert} \dot\eta{\right\Vert}_{H^{s-1}}+{\left\Vert} \dot\eta{\right\Vert}_{C^{r-1}}$.  Then by Leibnitz rule
\[
d_\eta I_1(\eta)\dot\eta=T_{\dot l}\eta+R
\]
where $\dot l:=d_\eta l\dot\eta$, so we only need to show that $\| T_{\dot l}\eta\|_{H^{s-3/2}}{\lesssim} {\left\Vert} \dot\eta{\right\Vert}_{H^{s-1}}+{\left\Vert} \dot\eta{\right\Vert}_{C^{r-1}}$. Indeed, observe that $\dot l$ is of the form 
\[
\dot l=F_1(\nabla\eta, \xi)\nabla\dot\eta+F_2(\nabla\eta, \xi)\nabla^2\dot\eta+F_3(\nabla\eta, \xi)\nabla\dot\eta\nabla^2\eta=:\sum_{j=1}^3G_j(x, \xi),
\]
where $F_j,~j=1,2,3$ are smooth in ${\mathbf{R}}^d\times {\mathbf{R}}^d\setminus\{0\}$; $F_1$ is homogeneous of order $2$ in $\xi$, $F_2,~F_3$ are homogeneous of order $1$ in $\xi$. By virtue of Theorem \ref{theo:sc} $(i)$ and Proposition \ref{regu<0} we see that to obtain the desired bound for $\| T_{\dot l}\eta\|_{H^{s-3/2}}$ it suffices to prove
\[
\sup_{|\xi|=1}\|\partial_\xi^\alpha G_1(\cdot, \xi)\|_{L^\infty}+\sup_{|\xi|=1}\|\partial_\xi^\alpha G_j(\cdot, \xi)\|_{C^{-1}}{\lesssim} C_\alpha{\left\Vert} \dot\eta{\right\Vert}_{C^{r-1}},\quad\forall \alpha\in {\mathbf{N}}^d,~j=2,~3.
\]
This is true because (assuming without loss of generality that $F_j(0, \xi)=0$, for all $\xi$) uniformly in $|\xi|=1$,
\begin{align*}
{\left\Vert} F_1(\nabla\eta)\nabla\dot\eta{\right\Vert}_{L^\infty}&{\lesssim} {\left\Vert} \dot\eta{\right\Vert}_{W^{1,\infty}}{\lesssim} {\left\Vert} \dot\eta{\right\Vert}_{C^{r-1}},\\
{\left\Vert} F_2(\nabla\eta)\nabla^2\dot\eta{\right\Vert}_{C^{-1}}&{\lesssim} {\left\Vert} F_2(\nabla\eta){\right\Vert}_{C^{1+{\varepsilon}}}{\left\Vert} \nabla^2\dot\eta{\right\Vert}_{C^{-1}}{\lesssim} {\left\Vert} \dot\eta{\right\Vert}_{C^{r-1}},\\
{\left\Vert} F_3(\nabla\eta)\nabla\dot\eta\nabla^2\eta{\right\Vert}_{C^{-1}}&{\lesssim} {\left\Vert} F_3(\nabla\eta)\nabla\dot\eta\nabla^2\eta{\right\Vert}_{L^\infty}{\lesssim} {\left\Vert} \dot\eta{\right\Vert}_{W^{1,\infty}}{\lesssim} {\left\Vert} \dot\eta{\right\Vert}_{C^{r-1}}.
\end{align*}
(here, we chose $0<{\varepsilon}<s-3/2-d/2$).\\
We have shown the desired estimate for $I_1$. By inspecting the proof of Lemma  \ref{lem:paraquad}, the estimate for $H^{s-3/2}$ norm of  ${\,\mathrm{d}}_\eta I_2\dot\eta+{\,\mathrm{d}}_\psi I_2\dot\psi$ can be obtained in the same way.
\end{proof}
\subsection{Contraction estimate for $f^1$}
Our goal in this paragraph is to derive the following estimate.
\begin{prop}\label{contraction:f1}
With $f^1$ defined in \eqref{f1f2}, it holds for a.e. $t\in [0, T]$ that
\[
{\left\Vert} f^1(\eta_1,\psi_1)(t)-f^1(\eta_2, \psi_2)(t){\right\Vert}_{H^{s-1}}{\lesssim} P_H(t)+P_S(t)Q(t)
\]
with 
\begin{equation}\label{Q(t)}
Q(t):=1+\sum_{j=1}^2{\left\Vert} \eta_j(t){\right\Vert}_{C^{r+{\frac{1}{2}}}}+\sum_{j=1}^2{\left\Vert} \psi_j(t){\right\Vert}_{C^r}.
\end{equation}
\end{prop}
 Proposition \ref{contraction:f1} will be a consequence of the following estimates:
\begin{align}
&\|{\,\mathrm{d}}_\eta f^1(\eta, \psi)\dot\eta\|_{H^{s-1}} {\lesssim} \| \dot\eta\|_{H^{s-1}}{\left(}1+{\left\Vert} \eta{\right\Vert}_{C^{r+{\frac{1}{2}}}}+{\left\Vert} \psi{\right\Vert}_{C^r} {\right)}+{\left\Vert} \dot\eta{\right\Vert}_{C^{r-1}},~\forall \dot \eta\in H^{s+{\frac{1}{2}}}\cap C^{r+{\frac{1}{2}}},\label{est:f:deta}\\
&\| {\,\mathrm{d}}_\psi f^1(\eta, \psi)\dot\psi\|_{H^{s-1}}{\lesssim} \| \dot\psi\|_{H^{s-{\frac{3}{2}}}}+\| \dot\psi\|_{C^{r-{\frac{3}{2}}}},~\forall \dot\psi\in H^s\cap C^r.\label{est:f:dpsi}
\end{align}
\begin{lemm}
The estimate \eqref{est:f:deta} holds.
\end{lemm}
\begin{proof}
From the  definition of $f^1$ we have
\begin{align*}
{\,\mathrm{d}}_\eta f^1(\eta, \psi)\dot\eta&=-G(B\dot\eta)-\operatorname{div}(V\dot\eta)\\
&\quad -\left\{ T_{\dot\lambda}(\psi-T_B\eta)-T_{\lambda}T_{\dot B}\eta-T_{\lambda}T_B\dot\eta-T_{\dot V}\nabla\eta-T_V\nabla\dot\eta \right\}\\
&=\sum_{j=1}^5 I_j,
\end{align*}
where $\dot B:=d_\eta B(\eta, \psi\dot)\eta$ and similarly for $\dot V,~\dot\lambda$; and 
\begin{align*}
&I_1:=T_{\dot V}\eta,\quad I_2:=-V\nabla\dot\eta+T_V\nabla\dot\eta,\quad I_3:=-T_{\dot\lambda}(\psi-T_B\eta),\\
& I_4:=T_{\lambda}T_{\dot B}\eta,\quad I_5:=-G(B\dot\eta)-(\operatorname{div} V)\dot\eta+T_{\lambda}T_B\dot\eta.
\end{align*}
1. For $I_2$ we write $I_2=-T_{\nabla\dot\eta}V-R(\nabla\dot\eta,  V)$ and use  \eqref{Bony1}, \eqref{pest3} to estimate 
\[
{\left\Vert} I_2{\right\Vert}_{H^{s-1}}{\lesssim} {\left\Vert} V{\right\Vert}_{H^{s-1}}\|\nabla\dot\eta\|_{L^\infty}{\lesssim} {\left\Vert} \dot\eta{\right\Vert}_{C^{r-1}}.
\]
2. To estimate the other terms, we need to study $\dot B$ and $\dot V$. For the former, the only nontrivial point is ${\,\mathrm{d}}_\eta [G(\eta)\psi]\dot \eta$:
\begin{equation}\label{dshape}
{\,\mathrm{d}}_\eta [G(\eta)\psi]\dot \eta=-G(\eta)(B\dot\eta)-\operatorname{div}(V\dot\eta).
\end{equation}
Consequently,
\[
{\left\Vert} {\,\mathrm{d}}_\eta [G(\eta)\psi]\dot \eta{\right\Vert}_{H^{s-2}}{\lesssim} {\left\Vert} \dot\eta{\right\Vert}_{H^{s-1}}+{\left\Vert} V\dot\eta{\right\Vert}_{H^{s-1}}{\lesssim} {\left\Vert} \dot\eta{\right\Vert}_{H^{s-1}}.
\]
Therefore, $\| \dot B\|_{H^{s-2}}{\lesssim} \|\dot\eta\|_{H^{s-1}}$. This together with the relation  $V=\nabla\psi-B\nabla\eta$ imply that
\[
\| \dot B\|_{H^{s-2}}+\| \dot V\|_{H^{s-2}}{\lesssim} \| \dot \eta\|_{H^{s-1}}.
\]
A a consequence, the paraproduct rule \eqref{boundpara} gives with $s-3/2>d/2$
\[
{\left\Vert} I_1{\right\Vert}_{H^{s-1}}{\lesssim} \| \dot V\|_{H^{s-2}}\| \nabla \eta\|_{H^{s-{\frac{1}{2}}}}{\lesssim} \| \dot\eta\|_{H^{s-1}}.
\]
Similarly, 
\[
{\left\Vert} I_4{\right\Vert}_{H^{s-1}}{\lesssim} \|T_{\dot B}\eta\|_{H^s}{\lesssim} \| \dot B\|_{H^{s-2}}\|\eta\|_{H^{s+{\frac{1}{2}}}}{\lesssim} \|\dot \eta\|_{H^{s-1}}.
\]
3. For $I_3$ one estimates $\dot \lambda$ exactly as for $\dot l$ in the proof of Proposition \ref{contraction:f2}.\\
4. For $I_5$ we follow \cite{ABZ1} using a key cancellation in Lemma $2.12$, \cite{ABZ1} whose proof applies also at our regularity level:
\[
G(\eta)B=-\operatorname{div} V+R,~\|R\|_{H^{s-1}}{\lesssim} 1.
\]
On the other hand, it follows from Proposition \ref{DN2} that
\[
G(\eta)(B\dot\eta)=T_{\lambda^{(1)}}B\dot\eta+F(\eta, B\dot\eta),\quad G(\eta)(B)=T_{\lambda^{(1)}}B+F(\eta, B)
\]
with 
\[
{\left\Vert} F(\eta, B\dot\eta){\right\Vert}_{H^{s-1}}{\lesssim} {\left\Vert} \dot\eta{\right\Vert}_{H^{s-1}},\quad {\left\Vert} F(\eta, B){\right\Vert}_{H^{s-1}}{\lesssim} 1.
\]
Then plugging these paralinearizations into the expression of $I_5$ gives  $I_5=J_1+J_2$ with
\begin{gather*}
J_1=-T_{\lambda^{(1)}}\left(B\dot\eta-T_B\dot\eta-T_{\dot\eta}B\right),\\
J_2=T_{\lambda^{(0)}}T_B\dot\eta+[T_{\dot\eta}, T_{\lambda^{(1)}}]B+T_{\dot\eta}F(\eta, B)+(\dot\eta-T_{\dot\eta})\operatorname{div} V-F(\eta, B\dot\eta)-T_{\dot\eta}R.
\end{gather*}
For $J_1$ one applies \eqref{Bony2} to have
\[
{\left\Vert} J_1{\right\Vert}_{H^{s-1}}{\lesssim} {\left\Vert} R(B, \dot\eta){\right\Vert}_{H^s}{\lesssim} {\left\Vert} \dot\eta{\right\Vert}_{H^{s-1}}{\left\Vert} B{\right\Vert}_{C^{1}}{\lesssim}  \| \dot\eta\|_{H^{s-1}}\Big(1+{\left\Vert} \eta{\right\Vert}_{C^{r+{\frac{1}{2}}}}+{\left\Vert} \psi{\right\Vert}_{C^r} \Big). 
\]
For $J_2$ we only need to take care of the commutator $[T_{\dot\eta}, T_{\lambda^{(1)}}]B$. Since $\| B\|_{H^{s-1}}{\lesssim} 1$ it suffices to prove that $[T_{\dot\eta}, T_{\lambda^{(1)}}]$ has order $0$ with norm from $H^{s-1}\to H^{s-1}$ bounded by the right hand side of \eqref{est:f:deta}. This is in turn a consequence of Theorem \ref{theo:sc} $(ii)$ and the fact that $r-1>1$. This concludes the proof.
\end{proof}
Finally, we prove
\begin{lemm}\label{lemm:est:f:dpsi}
The estimate \eqref{est:f:dpsi} holds.
\end{lemm}
Writing $B=B(\eta, \psi),~V=V(\eta, \psi)$ we have since $f^1$ is linear with respect to $\psi$ that
\[
{\,\mathrm{d}}_\psi f^1(\eta, \psi)\dot\psi=G(\eta)\dot\psi -T_\lambda(\dot\psi-T_{B(\eta, \dot\psi)}\eta)-T_{V(\eta, \dot\psi)}\cdot\nabla\eta=:R(\eta, \dot\psi).
\]
Estimate \eqref{est:f:dpsi} means that $R$ is of order $-1/2$ in $\dot\psi$ and acts from $H^{s-3/2}$ to $H^{s-1}$. In fact, we have proved in Proposition \ref{prop:paradir} that $R$ is of order $-1/2$ and acts from $H^s$ to $H^{s+1/2}$. Here, we shall follow the proof of   Proposition \ref{prop:paradir} except that we do not need to use the good unknown $u$ in \eqref{def:b,u} and  we do not need to track the lower Sobolev index $s_0$. Lemma \ref{lemm:est:f:dpsi} is a consequence of the following.
\begin{lemm}
Let $d\ge 1,~h>0$ and 
\[
s>{\frac{3}{2}}+\frac d2,~2<r<s+\frac d2+{\frac{1}{2}}.
\]
Then there exist a non-decreasing function ${ \mathcal{F}}:{\mathbf{R}}^+\times {\mathbf{R}}^+\to {\mathbf{R}}^+$ such that for any $\eta \in H^{s+{\frac{1}{2}}}$ satisfying $\operatorname{dist}(\eta, \Gamma)\ge h>0$ and $\psi\in H^s\cap C^r$, we have
\begin{multline}\label{DN:low}
{\left\Vert} G(\eta)\psi-T_{\lambda}(\psi-T_B\eta)-T_V\cdot\nabla \eta{\right\Vert}_{H^{s-1}}{\leq} { \mathcal{F}}{\left(}{\left\Vert} \eta{\right\Vert}_{H^{s+{\frac{1}{2}}}}, {\left\Vert} \psi{\right\Vert}_{H^s}{\right)}\times \\
\times \left(\| \psi\|_{H^{s-{\frac{3}{2}}}}+\| \psi\|_{C^{r-{\frac{3}{2}}}}\right).
\end{multline}
\end{lemm}
\begin{proof}
We first remark that 
\[
{\left\Vert} T_{\lambda} T_B\eta{\right\Vert}_{H^{s-1}}+{\left\Vert} T_V\cdot\nabla\eta{\right\Vert}_{H^{s-1}}{\leq} { \mathcal{F}}{\left(}{\left\Vert} \eta{\right\Vert}_{H^{s+{\frac{1}{2}}}}, {\left\Vert} \psi{\right\Vert}_{H^s}{\right)}{\left(} {\left\Vert} B{\right\Vert}_{C^{-{\frac{1}{2}}}}+{\left\Vert} V{\right\Vert}_{C^{-{\frac{1}{2}}}}{\right)}.
\]
On the other hand, \eqref{est:DN:Holder} implies 
\[
 {\left\Vert} B{\right\Vert}_{C^{-{\frac{1}{2}}}}+{\left\Vert} V{\right\Vert}_{C^{-{\frac{1}{2}}}}{\leq} { \mathcal{F}}{\left(}{\left\Vert} \eta{\right\Vert}_{H^{s+{\frac{1}{2}}}}{\right)} \left(\| \psi\|_{H^{s-{\frac{3}{2}}}}+\| \psi\|_{C^{r-{\frac{3}{2}}}}\right).
\]
Therefore, the proof of\eqref{DN:low} reduces to showing that
\begin{equation}\label{DN:low1}
{\left\Vert} G(\eta)\psi-T_{\lambda}\psi{\right\Vert}_{H^{s-1}}{\leq} { \mathcal{F}}{\left(}{\left\Vert} \eta{\right\Vert}_{H^{s+{\frac{1}{2}}}}, {\left\Vert} \psi{\right\Vert}_{H^s}{\right)} \left(\| \psi\|_{H^{s-{\frac{3}{2}}}}+\| \psi\|_{C^{r-{\frac{3}{2}}}}\right).
\end{equation}
{\it Step 1} (Estimates for $v$.) First, let $v$ be as in \eqref{defi:v}, which satisfies equation~\eqref{eq:v}. Let $z_0\in (-1, 0)$ and set $J=[z_0, 0]$. Proposition \ref{prop:ellregbase} applied with~$\sigma=s-5/2\ge -1/2$ gives the Sobolev estimates
\begin{equation}\label{v:low:Sobolev}
{\left\Vert} \nabla_{x,z}v{\right\Vert}_{X^{s-\frac 52}(J)}{\leq} { \mathcal{F}}{\left(}{\left\Vert} \eta{\right\Vert}_{H^{s+{\frac{1}{2}}}}{\right)}{\left\Vert} \psi{\right\Vert}_{H^{s-\frac 32}}.
\end{equation}
Then from  equation \eqref{eq:v} itself and the product rule \eqref{boundpara2} we obtain
\begin{equation}\label{d2v:low:Sobolev}
{\left\Vert} \partial_z^2v{\right\Vert}_{X^{s-\frac 72}(J)}{\leq} {\left\Vert} \nabla_{x,z}v{\right\Vert}_{X^{s-\frac 52}(I)}{\leq} { \mathcal{F}}{\left(}{\left\Vert} \eta{\right\Vert}_{H^{s+{\frac{1}{2}}}}{\right)}{\left\Vert} \psi{\right\Vert}_{H^{s-\frac 32}}.
\end{equation}
Proposition \ref{prop:regellhol} applied with $\mu=5/2$ on the other hand, implies the following H\"older estimate 
\begin{equation}\label{v:low:Holder}
{\left\Vert}\nabla_{x,z}v{\right\Vert}_{C(J;C^{r-\frac 52})}\leq{ \mathcal{F}}{\left(}{\left\Vert}\eta{\right\Vert}_{H^{s+{\frac{1}{2}}}},{\left\Vert} \psi{\right\Vert}_{H^{s}}{\right)}{\left(} {\left\Vert} \psi{\right\Vert}_{H^{r-\frac 32}}+{\left\Vert} \psi{\right\Vert}_{C^{r-\frac 32}}{\right)}.
\end{equation}
Again, we shall use \eqref{eq:v} to derive a bound for $\partial_z^2v$ in $C(J; C^{-3/2}_*)$, for which we use the Bony decomposition
\[
\alpha\Delta_x v=T_\alpha \Delta_x v+T_{\Delta_x v}\alpha+R(\alpha, \Delta_x v)
\]
where the paraproduct terms are estimated using \eqref{pest3}, \eqref{pest2} together with~\eqref{v:low:Holder} for $\Delta_xv$; \eqref{eq:estcoefellsob} and Sobolev embedding for $\alpha$. For the remainder term one uses \eqref{Bony3} as follows:
\[
{\left\Vert} R(\alpha, \Delta_x v){\right\Vert}_{L^\infty C^{-{\frac{3}{2}}}}{\lesssim} {\left\Vert} R(\alpha, \Delta_x v){\right\Vert}_{L^\infty H^{-{\frac{3}{2}}+d-\frac d2}}{\lesssim} {\left\Vert} \alpha{\right\Vert}_{L^\infty H^{s-{\frac{1}{2}}}}{\left\Vert} \Delta_x v{\right\Vert}_{L^\infty H^{s-\frac 72}},
\]
noticing that $s>3/2+d/2$, hence $s-7/2+s-1/2>\max\{0, -3/2+d\}$. The term $\beta\nabla_x\partial_zv$ is treated identically, so we are left with $\gamma\partial_zv$:
\[
{\left\Vert} \gamma\partial_zv{\right\Vert}_{L^\infty C^{-{\frac{3}{2}}}}{\lesssim} {\left\Vert} \gamma\partial_zv{\right\Vert}_{L^\infty H^{-{\frac{3}{2}}+\frac d2}}{\lesssim} {\left\Vert} \gamma{\right\Vert}_{L^\infty H^{s-{\frac{3}{2}}}} {\left\Vert} \partial_z v{\right\Vert}_{L^\infty H^{s-\frac 52}},
\]
where we have applied \eqref{pr}. Therefore, we obtain
\begin{equation}\label{d2v:low:Holder}
{\left\Vert}\partial_z^2v{\right\Vert}_{C(J; C^{-\frac 32})}\leq{ \mathcal{F}}{\left(}{\left\Vert}\eta{\right\Vert}_{H^{s+{\frac{1}{2}}}},{\left\Vert} \psi{\right\Vert}_{H^{s}}{\right)}{\left(} {\left\Vert} \psi{\right\Vert}_{H^{s-\frac 32}}+{\left\Vert} \psi{\right\Vert}_{C^{r-\frac 32}}{\right)}.
\end{equation}
{\it Step 2.}  To simplify notations, we shall write $g_1\sim_E g_2$ iff the $E$-norm of $g_1-g_2$ is bounded by the right-hand side of \eqref{DN:low1}, which shall be denoted by r.h.s. As in Proposition \ref{para:eq:u}, set
\[
P:= \partial_z^2+T_\alpha\Delta_x +T_\beta\cdot\nabla_x\partial_z-T_{\gamma}\partial_z.
\]
 From equation \eqref{eq:v} there holds
\[
0=(\partial_z^2+\alpha\Delta_x+\beta\cdot\nabla_x\partial_z-\gamma\partial_z)v=Pv+Qv
\]
with 
\[
Qv:=[T_{\Delta v}\alpha+R(\Delta v, \alpha)]+[T_{\nabla\partial_z v}\beta+R(\nabla\partial_zv, \beta)]-[T_{\partial_z v}\gamma+R(\partial_z v, \gamma)].
\]
For the first bracket, we have  according to \eqref{pest1}, \eqref{Bony2} and \eqref{v:low:Sobolev}, \eqref{v:low:Holder}
\[
{\left\Vert} T_{\Delta v}\alpha{\right\Vert}_{L^2H^{s-{\frac{3}{2}}}}+{\left\Vert} R(\Delta v, \alpha){\right\Vert}_{L^2H^{s-{\frac{3}{2}}}}{\lesssim} {\left\Vert} \Delta v{\right\Vert}_{L^\infty C^{-\frac 32}}{\left(} {\left\Vert} \alpha-h^2{\right\Vert}_{L^2H^s}+1{\right)}{\lesssim} r.h.s.\\
\]
Estimates for other terms follow along the same lines. We conclude that~$Pu\sim 0$. Next, by virtue of Lemma \ref{dep:parabolics}, 
\[
(\partial_z-T_a)(\partial_z-T_A)v\sim_{Y^{s-1}} 0.
\]
Then, following exactly the proof of Proposition \ref{prop:paradir}, we obtain as in \eqref{lem:tangpara} that
\[
{\left\Vert} \partial_zv-T_Av{\right\Vert}_{X^{s-1}}{\lesssim} r.h.s.
\]
Consequently, we deduce by using again the Bony decomposition
\begin{align*}
\frac{1+{\left\vert}\nabla\rho{\right\vert}^2}{\partial_z\rho}\partial_zv-\nabla\rho\cdot\nabla v&\sim_{X^{s-1}} T_{\frac{1+{\left\vert}\nabla\rho{\right\vert}^2}{\partial_z\rho}}\partial_zv-T_{\nabla \rho}\nabla v\\
&\sim_{X^{s-1}} T_{\frac{1+{\left\vert}\nabla\rho{\right\vert}^2}{\partial_z\rho}}T_{A}v-T_{\nabla\rho}\nabla v\sim T_{\frac{1+{\left\vert}\nabla\rho{\right\vert}^2}{\partial_z\rho}A}v-T_{\nabla \rho}\nabla v\\
&\sim_{X^{s-1}} T_{\Lambda}v
\end{align*}
with $\Lambda=\frac{1+{\left\vert}\nabla\rho{\right\vert}^2}{\partial_z\rho}A-i\nabla\eta\cdot\xi$ satisfying  $\Lambda\arrowvert_{z=0}=\lambda$.  The proof of \eqref{DN:low1} is complete.
\end{proof}
\subsection{Contraction estimate for solutions}
In views of notations \eqref{f1f2}, \eqref{quantities:contraction} and \eqref{Q(t)} , we have proved in subsections~$5.1$, $5.2$ the following result for a.e.~$~t\in [0, T]$.
\[
{\left\Vert} f(\eta_1, \psi_1)(t)-f(\eta_2, \psi_2)(t){\right\Vert}_{H^{s-1}\times H^{s-{\frac{3}{2}}}}{\leq}  { \mathcal{F}}{\left(} M^1_{s, T}, M^2_{s,T}{\right)}{\left(} P_H(t)+P_S(t)Q(t){\right)}.
\]
Consequently, this together with Lemma \ref{dB,dV} implies that the difference of solutions satisfies
\begin{equation}\label{eq:difference}
(\partial_t+T_{V_1}\cdot\nabla+\mathcal{L}_1)\begin{pmatrix}\delta\eta \\ \delta\psi\end{pmatrix} =\begin{pmatrix}g_1 \\ g_2\end{pmatrix} 
\end{equation}
where, again
\begin{equation}\label{contraction:g1g2}
{\left\Vert} (g_1(t), g_2(t)){\right\Vert}_{H^{s-1}\times H^{s-{\frac{3}{2}}}}{\leq}  { \mathcal{F}}{\left(} M^1_{s, T}, M^2_{s,T}{\right)}{\left(} P_H(t)+P_S(t)Q(t){\right)},\quad a.e. ~t\in [0, T].
\end{equation}
\subsubsection{Symmetrization}
Now, we symmetrize \eqref{eq:difference} using the symmetrizer 
\[
S=
\begin{pmatrix}
T_{p_1} &0\\
0&T_{q_1}
\end{pmatrix}
\begin{pmatrix}
I &0\\
-T_{B_1}&I
\end{pmatrix}.
\]
{\it The dispersive part $\mathcal{L}$}. Recall Definition \ref{equi:operators} on equivalence of two families of operators $A(t)$ and $B(t)$,~$t\in [0, T]$:
\[
{\left\Vert} A(t)-B(t){\right\Vert}_{H^{\mu}\to H^{\mu-m+{\frac{3}{2}}}}{\leq} { \mathcal{F}}{\left(}{\left\Vert} \eta(t){\right\Vert}_{H^{s_0+{\frac{1}{2}}}}{\right)}{\left(}1+{\left\Vert} \eta(t){\right\Vert}_{C^{r+{\frac{1}{2}}}}{\right)}.
\]
By virtue of Proposition \ref{pq} we obtain (we skip the subscript 1 in the following computations)
\begin{align*}
&\begin{pmatrix} 
0&T_q
\end{pmatrix}
\begin{pmatrix}
I &0\\
-T_B&I
\end{pmatrix}
\begin{pmatrix}
I &  0\\
T_B& I
\end{pmatrix}
\begin{pmatrix}
0 &  -T_\lambda\\
T_l& 0
\end{pmatrix}\\
&=\begin{pmatrix} 
0&T_q
\end{pmatrix}
\begin{pmatrix}
0 &  -T_\lambda\\
T_l& 0
\end{pmatrix}
= \begin{pmatrix} 
T_qT_l& 0
\end{pmatrix}\\ 
&\sim \begin{pmatrix} 
T_\gamma T_p& 0
\end{pmatrix}
=\begin{pmatrix} 
T_\gamma & 0
\end{pmatrix}
\begin{pmatrix}
T_p & 0\\
0 & T_q
\end{pmatrix}.
\end{align*}
Consequently,
\[
S\mathcal{L}_1\sim 
\begin{pmatrix}
0 & -T_{\gamma_1} \\
T_{\gamma_1} & 0
\end{pmatrix}
\begin{pmatrix}
T_{p_1} & 0\\
0 & T_{q_1}
\end{pmatrix}
\begin{pmatrix}
I & 0\\
-T_{B_1} & I
\end{pmatrix}.
\]
Therefore, if we set 
\[
\Phi_1:=T_{p_1}\delta \eta,\quad \Phi_2:=T_{q_1}(\delta \psi-T_{B_1}\delta \eta),
\]
then $\Phi_1,~\Phi_2$ satisfy 
\[
S\mathcal{L}_1
\begin{pmatrix}
\delta\eta\\
\delta\psi
\end{pmatrix}
\sim 
\begin{pmatrix}
-T_{\gamma_1}\Phi_2\\
T_{\gamma_1}\Phi_1
\end{pmatrix}
,
\]
which means that 
\[
{\left\Vert}  S\mathcal{L}_1
\begin{pmatrix}
\delta\eta\\
\delta\psi
\end{pmatrix}
-
\begin{pmatrix}
-T_{\gamma_1}\Phi_2\\
T_{\gamma_1}\Phi_1
\end{pmatrix}
{\right\Vert}_{H^{s-{\frac{3}{2}}}}(t)
{\leq} { \mathcal{F}}{\left(} M_{s,T}^1, M_{s,T}^2{\right)}\left(1+{\left\Vert}  \eta_1{\right\Vert}_{C^{r+{\frac{1}{2}}}}\right)P_{S}(t).
\]
{\it The convection part $\partial_t+T_{V_1}\nabla$}:  one proceeds as in the proof Proposition \ref{sym:prop} and get
\[
S{\left(} \partial_t+T_{V_1}\cdot\nabla{\right)} \begin{pmatrix}\delta\eta \\ \delta\psi\end{pmatrix} 
={\left(} \partial_t+T_{V_1}\cdot\nabla{\right)} S\begin{pmatrix}\delta\eta \\ \delta\psi\end{pmatrix} +R
={\left(} \partial_t+T_{V_1}\cdot\nabla{\right)} S\begin{pmatrix}\Phi_1 \\ \Phi_2\end{pmatrix} +R
\]
where the remainder $R$ verifies
\[
{\left\Vert} R(t){\right\Vert}_{H^{s-{\frac{1}{2}}}\times H^{s-{\frac{3}{2}}}}{\leq} { \mathcal{F}}{\left(} M_{s,T}^1, M_{s,T}^2{\right)}\left(1+{\left\Vert}  \eta_1{\right\Vert}_{C^{r+{\frac{1}{2}}}}\right)P_{S}(t).
\]
In conclusion, we have derived that
\begin{equation}\label{contraction:eq:sym}
\left\{
\begin{aligned}
&\partial_t\Phi_1+T_{V_1}\cdot\nabla\Phi_1- T_{\gamma_1}\Phi_2=F_1+G_1,\\
&\partial_t\Phi_2+T_{V_1}\cdot\nabla\Phi_2+T_{\gamma_2}\Phi_2=F_2+G_2
\end{aligned}
\right.
\end{equation}
where for a.e. $t\in [0, T]$, 
\begin{equation}\label{contraction:F1F2}
\begin{aligned}
{\left\Vert} (F_1, F_2){\right\Vert}_{ H^{s-{\frac{3}{2}}}\times H^{s-{\frac{3}{2}}}}&{\leq} { \mathcal{F}}{\left(} M_{s,T}^1, M_{s,T}^2{\right)}\left(1+{\left\Vert}  \eta_1{\right\Vert}_{C^{r+{\frac{1}{2}}}}\right)P_{S}(t)\\
& {\leq} { \mathcal{F}}{\left(} M^1_{s, T}, M^2_{s,T}{\right)}{\left(} P_H(t)+P_S(t)Q(t){\right)}.
\end{aligned}
\end{equation}
and from \eqref{eq:difference}
\[
\begin{pmatrix}G_1 \\ G_2\end{pmatrix} =\begin{pmatrix}T_{p_1}g_1 \\ T_{q_1}(g_2-T_{B_1}g_1)\end{pmatrix}.
\]
It follows from \eqref{contraction:g1g2} that $(G_1, G_2)$ also satisfy
\begin{equation}\label{contraction:G1G2}
{\left\Vert} (G_1, G_2){\right\Vert}_{ H^{s-{\frac{3}{2}}}\times H^{s-{\frac{3}{2}}}}{\leq} { \mathcal{F}}{\left(} M^1_{s, T}, M^2_{s,T}{\right)}{\left(} P_H(t)+P_S(t)Q(t){\right)}.
\end{equation}
\subsubsection{Contraction estimates}
Put $\Phi:=\Phi_1+i\Phi_2$, then 
\begin{equation}\label{contraction:Phi}
\partial_t\Phi+T_{V_1}\cdot\nabla\Phi+iT_{\gamma_1}\Phi=F+G:=(F_1+iF_2)+(G_1+iG_2).
\end{equation}
We are now back to the situation of Proposition \ref{singleeq:Phi}: we shall conjugate \eqref{contraction:Phi} with an operator of order $s-3/2$ and then perform an $L^2$-energy estimate. As in \eqref{def:wp}, we choose 
\[
\wp_1=(\gamma_1^{(3/2)})^{2(s-{\frac{3}{2}})/3},\quad \varphi=T_{\wp_1}\Phi.
\]
After conjugating with $T_{\wp_1}$, one obtains
\begin{equation}\label{contraction:eq:varphi}
{\left(} \partial_t+T_{V_1}\cdot\nabla+iT_{\gamma_1}{\right)}\varphi=T_{\wp_1}(F+G)+H
\end{equation}
with
\[
	H:=T_{\partial_t\wp_1}\Phi+[T_{V_1}\cdot\nabla, T_{\wp_1}]\Phi+i[T_{\gamma_1}, T_{\wp_1}]\Phi.
\]
It is easy to see as in the proof of Proposition \ref{L2varphi} (using Lemma \ref{inverse}) that
\begin{equation}\label{contraction:H}
\begin{aligned}
{\left\Vert} H(t){\right\Vert}_{H^{s-{\frac{3}{2}}}}&{\leq}  { \mathcal{F}}{\left(} M^1_{s, T}, M^2_{s,T}{\right)} Q(t){\left\Vert} \Phi(t){\right\Vert}_{H^{s-{\frac{3}{2}}}}\\
&{\leq} { \mathcal{F}}{\left(} M_{s,T}^1, M_{s,T}^2{\right)} Q(t){\left[} {\left\Vert} \Phi(t){\right\Vert}_{L^2}+{\left\Vert}{\varphi}(t){\right\Vert}_{L^2}{\right]}.
\end{aligned}
\end{equation}
On the other hand, from the  estimates \eqref{contraction:F1F2}, \eqref{contraction:G1G2} for $F,~G$ we get
\begin{equation}\label{contraction:F+G}
{\left\Vert} T_{\wp_1}(F+G){\right\Vert}_{L^2}{\leq} { \mathcal{F}}{\left(} M^1_{s, T}, M^2_{s,T}{\right)}{\left(} P_H(t)+P_S(t)Q(t){\right)}.
\end{equation}
Now, using \eqref{contraction:H}, \eqref{contraction:F+G} and \eqref{energy:gamma}, \eqref{energy:V} we deduce from equation \eqref{contraction:eq:varphi} that
\begin{multline*}
\frac{\,\mathrm{d}}{{\,\mathrm{d}} t}{\left\Vert}{\varphi}(t){\right\Vert}_{L^2}^2\leq { \mathcal{F}}{\left(} M_{s,T}^1, M_{s,T}^2{\right)}\left\{ {\left[} P_H(t)+Q(t)P_S(t)+Q(t){\left\Vert} \Phi(t){\right\Vert}_{L^2}{\right]}{\left\Vert}{\varphi}(t){\right\Vert}_{L^2}+\right.\\
\left. 2Q(t){\left\Vert}{\varphi}(t){\right\Vert}^2_{L^2}\right\}.
\end{multline*}
Since 
\[
{\left\Vert} \Phi(t){\right\Vert}_{L^2}{\leq} { \mathcal{F}}{\left(} M_{s,T}^1, M_{s,T}^2{\right)} P_S(t),\quad \int_0^T Q(t)dt{\leq} 1+ Z^1_{r,T}+Z^2_{r,T},
\]
Gr\"onwall inequality then gives (see Notations \ref{MN}, \ref{quantities:contraction})
\begin{equation}\label{contraction:est:energy}
\begin{aligned}
{\left\Vert} \varphi(t){\right\Vert}_{L^2}&{\leq} { \mathcal{F}}(...){\left(} {\left\Vert} \varphi(0){\right\Vert}_{L^2}+\int_0^t{\left[} Q(m) P_S(m)+P_H(m){\right]} dm{\right)}\\
& {\leq} { \mathcal{F}}(...){\left(} {\left\Vert} \varphi(0){\right\Vert}_{L^2}+T^{\frac{1}{p'}} {\left[} (1+Z^1_{r,T}+Z^2_{r,T}) P_{S,T}+P_{H,T}{\right]} {\right)}\\
 &{\leq} { \mathcal{F}}(...){\left(} P_S(0)+T^{\frac{1}{p'}}P_T {\right)}
\end{aligned}
\end{equation}
where 
\[
{ \mathcal{F}}(...)={ \mathcal{F}}{\left(} M_{s,T}^1, M_{s,T}^2, Z^1_{r,T}, Z^2_{r,T}{\right)},~\quad \frac {1}{p}+\frac {1}{p'}=1.
\]
The next step is to go back from $\varphi$ to $(\delta \eta, \delta \psi)$. To do this, one uses again  Proposition~\ref{inverse} (and the Remark following it)  to get
	\begin{gather*}
	{\left\Vert}\delta\eta{\right\Vert}_{H^{s-1}}\leq{ \mathcal{F}}{\left(}{\left\Vert}\eta_1{\right\Vert}_{H^{s-\frac12}}{\right)}{\left[}{\left\Vert} T_\wp T_p\delta\eta{\right\Vert}_{L^2}+{\left\Vert}\delta\eta{\right\Vert}_{H^{-{\frac{1}{2}}}}{\right]},\\
	{\left\Vert}\delta\psi{\right\Vert}_{H^{s-{\frac{3}{2}}}}\leq{ \mathcal{F}}{\left(}{\left\Vert}\eta_1{\right\Vert}_{H^{s-\frac12}}{\right)}{\left[}{\left\Vert} T_\wp T_q\delta\psi{\right\Vert}_{L^2}+{\left\Vert}\delta\psi{\right\Vert}_{H^{-{\frac{1}{2}}}}{\right]}.
	\end{gather*}
Then, in view of \eqref{contraction:est:energy} it remains to estimate $\|\delta\eta\|_{H^{-{\frac{1}{2}}}}$ and $\|\delta\psi\|_{H^{-{\frac{1}{2}}}}$ by r.h.s. For $\eta$ we have
\begin{align*}
{\left\Vert} \delta\eta(t){\right\Vert}_{H^{-{\frac{1}{2}}}}&{\leq} {\left\Vert} \delta\eta(0){\right\Vert}_{H^{-{\frac{1}{2}}}}+{\left\Vert} \delta\eta(t)-\delta\eta(0){\right\Vert}_{H^{-{\frac{1}{2}}}}\\
&{\leq} {\left\Vert} \delta\eta(0){\right\Vert}_{H^{-{\frac{1}{2}}}}+{\left\Vert} \int_0^t\frac{d}{dt}\delta\eta(m)dm{\right\Vert}_{H^{-{\frac{1}{2}}}} \\
&{\leq} {\left\Vert} \delta\eta(0){\right\Vert}_{H^{-{\frac{1}{2}}}}+T\sup_{t\in [0, T]}{\left\Vert} \frac{d}{dt}\delta\eta(t){\right\Vert}_{H^{-{\frac{1}{2}}}}.
\end{align*}
The last term can be written as
\[
\frac{d}{dt}\delta\eta(t)=G(\eta_1(t))\psi_1(t)-G(\eta_2(t))\psi_2(t)=G(\eta_1)\delta\psi+[G(\eta_1(t))-G(\eta_2(t))]\psi_2(t).
\]
The Sobolev estimate for the Ditichlet-Neumann operator in Theorem \ref{DN:Sobolev} applied with $\sigma=s-3/2>1/2$ gives
\[
{\left\Vert} G(\eta_1)\delta\psi{\right\Vert}_{H^{-{\frac{1}{2}}}}{\lesssim} {\left\Vert} G(\eta_1)\delta\psi{\right\Vert}_{H^{s-\frac 52}}{\lesssim} {\left\Vert}\delta\psi{\right\Vert}_{H^{s- \frac 32}}.
\]
On the other hand, using the  shape-derivative formula  and Theorem \ref{DN:Sobolev} again, one gets
\begin{align*}
{\left\Vert} [G(\eta_1(t))-G(\eta_2(t))]\psi_2(t){\right\Vert}_{L^2}&{\leq} {\left\Vert} \int_0^1d_\eta {\left[} G{\left(}(\eta_1(t)+m\delta\eta(t){\right)}\psi_2(t){\right]}(\delta\eta(t))dm{\right\Vert}_{L^2}\\
&{\leq} {\left\Vert} \int_0^1\left\{ G(\widetilde\eta(m))\big(\widetilde B\delta\eta(t)\big)+\operatorname{div}\big( \widetilde V(m)\delta\eta(t)\big) \right\}dm{\right\Vert}_{{H^{s-2}}}\\
&{\leq} {\left\Vert} \delta\eta{\right\Vert}_{H^{s-1}}
\end{align*}
where $\widetilde \eta(m)=\eta_1+m\delta\eta,~\widetilde B(m)=B(\widetilde \eta(m), \psi_2), ~\widetilde V(m)=V(\widetilde\eta(m), \psi_2)$.\\
Summing up, we obtain
\[
{\left\Vert} \delta\eta(t){\right\Vert}_{H^{s-1}}{\leq} { \mathcal{F}}{\left(} M_{s,T}^1, M_{s,T}^2, Z_{r,T}^1, Z_{r,T}^2{\right)}{\left(} P_S(0)+T^{\frac{1}{p'}}P(t){\right)}.
\]
The quantity $\|\delta\psi\|_{H^{-{\frac{1}{2}}}}$ is treated in the same way using instead the second equation in \eqref{ww}. Therefore, we end up with the following estimate 
\[
{\left\Vert} (\delta\eta(t), \delta\psi(t)) {\right\Vert}_{H^{s-1}\times H^{s-{\frac{3}{2}}}} {\leq} { \mathcal{F}}{\left(} M_{s,T}^1, M_{s,T}^2, Z_{r,T}^1, Z_{r,T}^2{\right)}{\left(} P_S(0)+T^{\frac{1}{p'}}P(t){\right)},
\]
which implies
\begin{equation}\label{contraction:Sobolev}
P_{S,T}{\leq} { \mathcal{F}}{\left(} M_{s,T}^1, M_{s,T}^2, Z_{r,T}^1, Z_{r,T}^2{\right)}{\left(} P_H(0)+T^{\frac{1}{p'}}  P_{T}{\right)}.
\end{equation}
Observe that \eqref{contraction:Sobolev} is an a priori estimate for the Sobolev norm of the difference of solutions. To close this estimate, we seek a similar estimate in H\"older norm. For this purpose we apply  the Strichartz estimates in our companion paper \cite{NgPo} to the dispersive equation \eqref{contraction:Phi}. According to this result, for
\[
2<r<r'<s-\frac d2+\mu,
\]
and
\begin{equation}\label{mu,p}
\begin{cases}
\mu=\frac{3}{20},~p=4\quad \text{when}~d=1,\\
\mu=\frac{3}{10},~p=2\quad \text{when}~d\ge 2
\end{cases}
\end{equation}
we have 
\begin{align*}
{\left\Vert} \Phi{\right\Vert}_{L^pW^{r'-{\frac{3}{2}}}}&{\leq} C {\left\Vert} \Phi{\right\Vert}_{L^pW^{s-\frac d2-{\frac{3}{2}}+\mu}}\\
&{\leq} { \mathcal{F}}{\left(} M_{s,T}^1, Z_{r,T}^1{\right)}{\left(} {\left\Vert} F+G{\right\Vert}_{L^pH^{s-{\frac{3}{2}}}}+{\left\Vert} \Phi{\right\Vert}_{L^\infty H^{s-{\frac{3}{2}}}}{\right)},
\end{align*}
which, combined with \eqref{contraction:F1F2} and \eqref{contraction:G1G2} implies
\[
{\left\Vert} \Phi{\right\Vert}_{L^pW^{r'-{\frac{3}{2}}}}{\leq} { \mathcal{F}}{\left(} M_{s,T}^1, Z_{r,T}^1{\right)}{\left(} P_T+{\left\Vert} \Phi{\right\Vert}_{L^\infty H^{s-{\frac{3}{2}}}}{\right)}{\leq}  { \mathcal{F}}{\left(} M_{s,T}^1, Z_{r,T}^1{\right)} P_T.
\]
Then by interpolating between $r,~r'$ and using the symbolic calculus in Theorem \ref{theo:sc} one obtains for some $\delta=\delta(r,s)>0$:
\begin{equation}\label{apriori:Holder}
P_{H, T}{\leq} { \mathcal{F}}{\left(} M_{s,T}^1, M_{s,T}^2, Z_{r,T}^1, Z_{r,T}^2{\right)} T^{\delta}  P_{T}.
\end{equation}
Combining \eqref{contraction:Sobolev} and \eqref{apriori:Holder} we end up with a closed a priori estimate for the difference of two solutions of \eqref{ww} in terms of Sobolev norm and Strichartz norm:
\[
P_T{\leq} { \mathcal{F}}{\left(} M^1_{s,T}, M^2_{s,T}, Z^1_{r,T}, Z^2_{r,T}{\right)} {\left(} P_{S}(0)+T^\delta P_T{\right)}.
\]
This implies $P_{T_1}{\leq}{ \mathcal{F}}(...) P_S(0)$ for some $T_1>0$ small enough and depending only on ${ \mathcal{F}}(...)$.
Then iterating this estimate between $[T_1, 2T_1],...,[T-T_1, T]$ we obtain the following result.
\begin{theo}\label{theo:contraction}
Let $(\eta_j, \psi_j)$,~$j=1,2$ be two solutions to \eqref{ww} on $I=[0, T],~0<T{\leq} 1$ such that 
\[
(\eta_j, \psi_j)\in L^\infty(I; H^{s+{\frac{1}{2}}}({\mathbf{R}}^d)\times H^s({\mathbf{R}}^d))\cap L^p(I; W^{r+{\frac{1}{2}}}({\mathbf{R}}^d)\times W^{r,\infty}({\mathbf{R}}^d))
\]
with 
\[
s>{\frac{3}{2}}+\frac d2,~\quad 2<r<s-\frac d2+\mu;
\]
where $\mu,~p$ are given by \eqref{mu,p} and such that $\inf_{t\in [0, T]}\operatorname{dist}(\eta_j(t), \Gamma)>h>0.$ Set
\[
M^j_{s,T}:=\Vert (\eta_j, \psi_j)\Vert_{L^{\infty}([0, T]; H^{s+{\frac{1}{2}}}\times H^s)}, \quad  Z^j_{r,T}:=\Vert (\eta_j, \psi_j)\Vert_{L^p([0, T]; W^{r+{\frac{1}{2}}, \infty}\times W^{r, \infty})}.
\]
Consider the differences $\delta\eta:=\eta_1-\eta_2,~ \delta\psi:=\psi_1-\psi_2$ and their norms in Sobolev space and H\"older space:
\[
P_{T}:={\left\Vert} (\delta\eta, \delta\psi){\right\Vert}_{L^\infty(I; H^{s-1}\times H^{s-{\frac{3}{2}}})}+{\left\Vert} (\delta\eta, \delta\psi){\right\Vert}_{L^p(I; W^{r-1,\infty}\times W^{r-{\frac{3}{2}},\infty})}.
\]
Then there exists a non-decreasing  function ${ \mathcal{F}}_h:{\mathbf{R}}^+\times {\mathbf{R}}^+\to{\mathbf{R}}^+$ depending only on $d,~r,~s,~h$ such that
\[
P_T{\leq} { \mathcal{F}}_h{\left(} M^1_{s,T}, M^2_{s,T}, Z^1_{r,T}, Z^2_{r,T}{\right)}{\left\Vert} (\delta\eta, \delta\psi)\arrowvert_{t=0}{\right\Vert}_{H^{s-1}\times H^{s-{\frac{3}{2}}}}.
\]
\end{theo}
\section{Appendix: Paradifferential Calculus and technical results}\label{Appendix}
\begin{defi}
1. (Littlewood-Paley decomposition) Let~$\kappa\in C^\infty_0({\mathbf{R}}^d)$ be such that
$$
\kappa(\theta)=1\quad \text{for }{\left\vert} \theta{\right\vert}{\leq} 1.1,\qquad 
\kappa(\theta)=0\quad \text{for }{\left\vert}\theta{\right\vert}\ge 1.9.
$$
Then we define
$
\chi(\theta,\eta)=\sum_{k=0}^{+\infty} \kappa_{k-3}(\theta) \varphi_k(\eta)
$,
where
\begin{equation*}
\kappa_k(\theta)=\kappa(2^{-k}\theta)\quad\text{for }k\in {\mathbf{Z}},
\qquad \varphi_0=\kappa_0,\quad\text{ and } 
\quad \varphi_k=\kappa_k-\kappa_{k-1} \quad\text{for }k\ge 1.
\end{equation*}
Given a temperate distribution $u$ and an integer $k$ in ${\mathbf{N}}$ we also introduce $S_k u$ 
and $\Delta_k u$ by 
$S_k u=\kappa_k(D_x)u$ and $\Delta_k u=S_k u-S_{k-1}u$ for $k\ge 1$ and $\Delta_0u=S_0u$. Then we have the formal decomposition 
$$
u=\sum_{k=0}^{\infty}\Delta_k u.
$$
2. (Zygmund spaces) If~$s$ is any real number, we define the Zygmund class~$C^{s}_*({\mathbf{R}}^d)$ as the 
space of tempered distributions~$u$ such that
$$
{\left\Vert} u{\right\Vert}_{C^{s}_*}{\mathrel{:=}} \sup_q 2^{qs}{\left\Vert} \Delta_q u{\right\Vert}_{L^\infty}<+\infty.
$$
3. (H\"older spaces) For~$k\in{\mathbf{N}}$, we denote by $W^{k,\infty}({\mathbf{R}}^d)$ the usual Sobolev spaces.
For $\rho= k + \sigma$, $k\in {\mathbf{N}}, \sigma \in (0,1)$ denote 
by~$W^{\rho,\infty}({\mathbf{R}}^d)$ 
the space of functions whose derivatives up to order~$k$ are bounded and uniformly H\"older continuous with 
exponent~$\sigma$. 
\end{defi}
Let us review notations and results about Bony's paradifferential calculus (see \cite{Bony,Hormander,MePise}). Here we follow the presentation by M\'etivier in \cite{MePise} and \cite{ABZ3}, \cite{ABZ4}.
\begin{defi}
1. (Symbols) Given~$\rho\in [0, \infty)$ and~$m\in{\mathbf{R}}$,~$\Gamma_{\rho}^{m}({\mathbf{R}}^d)$ denotes the space of
locally bounded functions~$a(x,\xi)$
on~${\mathbf{R}}^d\times({\mathbf{R}}^d\setminus 0)$,
which are~$C^\infty$ with respect to~$\xi$ for~$\xi\neq 0$ and
such that, for all~$\alpha\in{\mathbf{N}}^d$ and all~$\xi\neq 0$, the function
$x\mapsto \partial_\xi^\alpha a(x,\xi)$ belongs to~$W^{\rho,\infty}({\mathbf{R}}^d)$ and there exists a constant
$C_\alpha$ such that,
\begin{equation*}
\forall{\left\vert} \xi{\right\vert}\ge {\frac{1}{2}},\quad 
{\left\Vert} \partial_\xi^\alpha a(\cdot,\xi){\right\Vert}_{W^{\rho,\infty}({\mathbf{R}}^d)}{\leq} C_\alpha
(1+{\left\vert}\xi{\right\vert})^{m-{\left\vert}\alpha{\right\vert}}.
\end{equation*}
Let $a\in \Gamma_{\rho}^{m}({\mathbf{R}}^d)$, we define the semi-norm
\begin{equation}\label{defi:norms}
M_{\rho}^{m}(a)= 
\sup_{{\left\vert}\alpha{\right\vert}{\leq} d/2+1+\rho ~}\sup_{{\left\vert}\xi{\right\vert} \ge 1/2~}
{\left\Vert} (1+{\left\vert}\xi{\right\vert})^{{\left\vert}\alpha{\right\vert}-m}\partial_\xi^\alpha a(\cdot,\xi){\right\Vert}_{W^{\rho,\infty}({\mathbf{R}}^d)}.
\end{equation}
2. (Paradifferential operators) Given a symbol~$a$, we define
the paradifferential operator~$T_a$ by
\begin{equation}\label{eq.para}
\widehat{T_a u}(\xi)=(2\pi)^{-d}\int \chi(\xi-\eta,\eta)\widehat{a}(\xi-\eta,\eta)\psi(\eta)\widehat{u}(\eta)
\, d\eta,
\end{equation}
where
$\widehat{a}(\theta,\xi)=\int e^{-ix\cdot\theta}a(x,\xi)\, dx$
is the Fourier transform of~$a$ with respect to the first variable; 
$\chi$ and~$\psi$ are two fixed~$C^\infty$ functions such that:
\begin{equation}\label{cond.psi}
\psi(\eta)=0\quad \text{for } {\left\vert}\eta{\right\vert}{\leq} \frac{1}{5},\qquad
\psi(\eta)=1\quad \text{for }{\left\vert}\eta{\right\vert}\geq \frac{1}{4},
\end{equation}
and~$\chi(\theta,\eta)$  is defined by
$
\chi(\theta,\eta)=\sum_{k=0}^{+\infty} \kappa_{k-3}(\theta) \varphi_k(\eta).
$
\end{defi}
\begin{defi}\label{defi:order}
Let~$m\in{\mathbf{R}}$.
An operator~$T$ is said to be of  order~${} m$ if, for all~$\mu\in{\mathbf{R}}$,
it is bounded from~$H^{\mu}$ to~$H^{\mu-m}$. 
\end{defi}
Symbolic calculus for paradifferential operators is summarized in the following theorem.
\begin{theo}\label{theo:sc}(Symbolic calculus)
Let~$m\in{\mathbf{R}}$ and~$\rho\in [0, \infty)$. \\
$(i)$ If~$a \in \Gamma^m_0({\mathbf{R}}^d)$, then~$T_a$ is of order~${} m$. 
Moreover, for all~$\mu\in{\mathbf{R}}$ there exists a constant~$K$ such that
\begin{equation}\label{esti:quant1}{\left\Vert} T_a {\right\Vert}_{H^{\mu}\rightarrow H^{\mu-m}}{\leq} K M_{0}^{m}(a).
\end{equation}
$(ii)$ If~$a\in \Gamma^{m}_{\rho}({\mathbf{R}}^d), b\in \Gamma^{m'}_{\rho}({\mathbf{R}}^d)$ with $\rho>0$. Then 
$T_a T_b -T_{a \sharp b}$ is of order~${} m+m'-\rho$ where
\[
a\sharp b:=\sum_{|\alpha|<\rho}\frac{(-i)^{\alpha}}{\alpha !}\partial_{\xi}^{\alpha}a(x, \xi)\partial_x^{\alpha}b(x, \xi).
\] 
Moreover, for all~$\mu\in{\mathbf{R}}$ there exists a constant~$K$ such that
\begin{equation}\label{esti:quant2}
{\left\Vert} T_a T_b  - T_{a \sharp b}   {\right\Vert}_{H^{\mu}\rightarrow H^{\mu-m-m'+\rho}}
{\leq} 
K M_{\rho}^{m}(a)M_{0}^{m'}(b)+K M_{0}^{m}(a)M_{\rho}^{m'}(b).
\end{equation}
$(iii)$ Let~$a\in \Gamma^{m}_{\rho}({\mathbf{R}}^d)$ with $\rho >0$. Denote by 
$(T_a)^*$ the adjoint operator of~$T_a$ and by~$\overline{a}$ the complex conjugate of~$a$. Then 
$(T_a)^* -T_{a^*}$ is of order~${} m-\rho$ where
\[
a^*=\sum_{|\alpha|<\rho}\frac{1}{i^{|\alpha|}\alpha!}\partial_{\xi}^{\alpha}\partial_x^{\alpha}\overline{a}.
\]
Moreover, for all~$\mu$ there exists a constant~$K$ such that
\begin{equation}\label{esti:quant3}
{\left\Vert} (T_a)^*   - T_{\overline{a}}   {\right\Vert}_{H^{\mu}\rightarrow H^{\mu-m+\rho}}{\leq} 
K M_{\rho}^{m}(a).
\end{equation}
\end{theo}
We also need the following definition for symbols with negative regularity.
\begin{defi}\label{defi:symbol<0}
For~$m\in {\mathbf{R}}$ and~$\rho\in (-\infty, 0)$,~$\Gamma^m_\rho({\mathbf{R}}^d)$ denotes the space of
distributions~$a(x,\xi)$
on~${\mathbf{R}}^d\times({\mathbf{R}}^d\setminus 0)$,
which are~$C^\infty$ with respect to~$\xi$ and 
such that, for all~$\alpha\in{\mathbf{N}}^d$ and all~$\xi\neq 0$, the function
$x\mapsto \partial_\xi^\alpha a(x,\xi)$ belongs to $C^\rho_*({\mathbf{R}}^d)$ and there exists a constant
$C_\alpha$ such that,
\begin{equation}
\forall{\left\vert} \xi{\right\vert}\ge {\frac{1}{2}},\quad {\left\Vert} \partial_\xi^\alpha a(\cdot,\xi){\right\Vert}_{C^\rho_*}{\leq} C_\alpha
(1+{\left\vert}\xi{\right\vert})^{m-{\left\vert}\alpha{\right\vert}}.
\end{equation}
For~$a\in \Gamma^m_\rho$, we define 
\begin{equation}
M_{\rho}^{m}(a)= 
\sup_{{\left\vert}\alpha{\right\vert}{\leq} 2(d +2)+\vert \rho \vert   ~}\sup_{{\left\vert}\xi{\right\vert} \ge 1/2~}
{\left\Vert} (1+{\left\vert}\xi{\right\vert})^{{\left\vert}\alpha{\right\vert}-m}\partial_\xi^\alpha a(\cdot,\xi){\right\Vert}_{C^{\rho}_*({\mathbf{R}}^d)}.
\end{equation}
\end{defi}
\begin{prop}\label{regu<0}
Let~$\rho<0$,~$m\in {\mathbf{R}}$ and~$a\in \dot{ \Gamma}^m_\rho$. Then the operator~$T_a$ is of order~$m-\rho$:
\begin{equation}\label{niSbis}
\| T_a \|_{H^s \rightarrow H^{s-(m- \rho)}}\leq C M_{\rho}^{m}(a),\qquad
\| T_a \|_{C^s_* \rightarrow C^{s-(m- \rho)}_*}\leq C M_{\rho}^{m}(a).
\end{equation}
\end{prop}
Given two functions~$a,b$ defined on~${\mathbf{R}}^d$ we define the remainder 
\begin{equation}\label{Bony:dep}
R(a,u)=au-T_a u-T_u a.
\end{equation}
We shall use frequently various estimates about paraproducts (see chapter 2 in~\cite{BCD},~\cite{BaCh} and \cite{ABZ3}) which are recalled here.
\begin{theo}\label{pproduct}
\begin{enumerate}
\item  Let~$\alpha,\beta\in {\mathbf{R}}$. If~$\alpha+\beta>0$ then
\begin{align}
&{\left\Vert} R(a,u) {\right\Vert} _{H^{\alpha + \beta-\frac{d}{2}}({\mathbf{R}}^d)}
\leq K {\left\Vert} a {\right\Vert} _{H^{\alpha}({\mathbf{R}}^d)}{\left\Vert} u{\right\Vert} _{H^{\beta}({\mathbf{R}}^d)},\label{Bony1} \\ 
&{\left\Vert} R(a,u) {\right\Vert} _{H^{\alpha + \beta}({\mathbf{R}}^d)} \leq K {\left\Vert} a {\right\Vert} _{C^{\alpha}_*({\mathbf{R}}^d)}{\left\Vert} u{\right\Vert} _{H^{\beta}({\mathbf{R}}^d)}\label{Bony2},\\
&{\left\Vert} R(a,u) {\right\Vert} _{C_*^{\alpha + \beta}({\mathbf{R}}^d)}\leq K {\left\Vert} a {\right\Vert} _{C^{\alpha}_*({\mathbf{R}}^d)}{\left\Vert} u{\right\Vert} _{C_*^{\beta}({\mathbf{R}}^d)}.\label{Bony3}
\end{align}
\item Let~$s_0,s_1,s_2$ be such that 
$s_0{\leq} s_2$ and~$s_0 < s_1 +s_2 -\frac{d}{2}$, 
then
\begin{equation}\label{boundpara}
{\left\Vert} T_a u{\right\Vert}_{H^{s_0}}{\leq} K {\left\Vert} a{\right\Vert}_{H^{s_1}}{\left\Vert} u{\right\Vert}_{H^{s_2}}.
\end{equation}
If in addition to the conditions above, $s_1+s_2>0$ then 
\begin{equation}\label{boundpara2}
{\left\Vert} au-T_ua{\right\Vert}_{H^{s_0}}{\leq} K {\left\Vert} a{\right\Vert}_{H^{s_1}}{\left\Vert} u{\right\Vert}_{H^{s_2}}.
\end{equation}
\item  Let~$m>0$ and~$s\in {\mathbf{R}}$. Then
\begin{align}
&{\left\Vert} T_a u{\right\Vert}_{H^{s-m}}{\leq} K {\left\Vert} a{\right\Vert}_{C^{-m}_*}{\left\Vert} u{\right\Vert}_{H^{s}}\label{pest1},\\ 
&{\left\Vert} T_a u{\right\Vert}_{C_*^{s-m}}{\leq} K {\left\Vert} a{\right\Vert}_{C^{-m}_*}{\left\Vert} u{\right\Vert}_{C_*^{s}}\label{pest2},\\
&{\left\Vert} T_a u{\right\Vert}_{C_*^s}{\leq} K {\left\Vert} a{\right\Vert}_{L^\infty}{\left\Vert} u{\right\Vert}_{C_*^{s}}\label{pest3}.
\end{align}
\end{enumerate}
\end{theo}
\begin{prop}\label{tame}
\begin{enumerate}
\item  If~$u_j\in H^{s_j}({\mathbf{R}}^d)$ ($j=1,2$) with $s_1+s_2> 0$ then 
\begin{equation}\label{pr}
{\left\Vert} u_1 u_2 {\right\Vert}_{H^{s_0}}{\leq} K {\left\Vert} u_1{\right\Vert}_{H^{s_1}}{\left\Vert} u_2{\right\Vert}_{H^{s_2}},
\end{equation}
if $s_0{\leq} s_j$, $j=1,2$, and $s_0< s_1+s_2-d/2$.
\item If $s\ge 0$ then 
\begin{equation}\label{tame:S}
{\left\Vert} u_1u_2{\right\Vert}_{H^s}{\leq} K({\left\Vert} u_1{\right\Vert}_{H^s}{\left\Vert} u_2{\right\Vert}_{L^{\infty}}+{\left\Vert} u_2{\right\Vert}_{H^s}{\left\Vert} u_1{\right\Vert}_{L^{\infty}}).
\end{equation}
\item If $s\ge 0$ then 
\begin{equation}\label{tame:H}
{\left\Vert} u_1u_2{\right\Vert}_{C_*^s}{\leq} K({\left\Vert} u_1{\right\Vert}_{C_*^s}{\left\Vert} u_2{\right\Vert}_{L^{\infty}}+{\left\Vert} u_2{\right\Vert}_{C_*^s}{\left\Vert} u_1{\right\Vert}_{L^{\infty}}).
\end{equation}
\item Let $\beta>\alpha>0$. Then
\begin{equation}\label{tame:H<0}
{\left\Vert} u_1u_2{\right\Vert}_{C_*^{-\alpha}}{\leq} K{\left\Vert} u_1{\right\Vert}_{C_*^{\beta}}{\left\Vert} u_2{\right\Vert}_{C_*^{-\alpha}}.
\end{equation}
\item Let~$s>d/2$ and consider~$F\in C^\infty({\mathbf{C}}^N)$ such that~$F(0)=0$. 
Then there exists a non-decreasing function~$\mathcal{F}\colon{\mathbf{R}}_+\rightarrow{\mathbf{R}}_+$ 
such that, for any~$U\in H^s({\mathbf{R}}^d)^N$,
\begin{equation}\label{F(u):H}
{\left\Vert} F(U){\right\Vert}_{H^s}{\leq} \mathcal{F}\bigl({\left\Vert} U{\right\Vert}_{L^\infty}\bigr){\left\Vert} U{\right\Vert}_{H^s}.
\end{equation}
\item Let~$s\ge 0$ and consider~$F\in C^\infty({\mathbf{C}}^N)$ such that~$F(0)=0$. 
Then there exists a non-decreasing function~$\mathcal{F}\colon{\mathbf{R}}_+\rightarrow{\mathbf{R}}_+$ 
such that, for any~$U\in C_*^s({\mathbf{R}}^d)^N$,
\begin{equation}\label{F(u):C}
{\left\Vert} F(U){\right\Vert}_{C_*^s}{\leq} \mathcal{F}\bigl({\left\Vert} U{\right\Vert}_{L^\infty}\bigr){\left\Vert} U{\right\Vert}_{C_*^s}.
\end{equation}
\end{enumerate}
\end{prop}
At last, we need some technical results on parabolic regularity:
\begin{theo}[\protect{\cite[Theorem~2.92]{BCD}}]\label{paralin}(Paralinearization)
Let $r,~\rho$ be positive real numbers and $F$ be a $C^{\infty}$ function on ${\mathbf{R}}$ such that $F(0)=0$. Assume that $\rho$ is not an integer. For any $u\in H^{\mu}({\mathbf{R}}^d)\cap C_*^{\rho}({\mathbf{R}}^d)$ we have
\[
{\left\Vert} F(u)-T_{F'(u)}u{\right\Vert}_{ H^{\mu+\rho}({\mathbf{R}}^d)}{\leq} C({\left\Vert} u{\right\Vert}_{L^{\infty}({\mathbf{R}}^d)}){\left\Vert} u{\right\Vert}_{C_*^{\rho}({\mathbf{R}}^d)}{\left\Vert} u{\right\Vert}_{H^{\mu}({\mathbf{R}}^d)}.
\]
\end{theo}
\begin{theo}[\protect{\cite[Proposition~2.18]{ABZ3}}]\label{regularitysob}
Let $\rho\in (0, 1),~J=[z_0, z_1]\subset {\mathbf{R}},~p\in \Gamma^1_{\rho}({\mathbf{R}}^d\times J),~q\in \Gamma^0_0({\mathbf{R}}^d\times J)$ with the assumption that 
\[
\Re p(z; x, \xi)\ge c|\xi|,
\]
for some constant $c>0$. Assume that $w$ solves
\[
\partial_zw+T_pw=T_qw+f,\quad w\arrowvert_{z=z_0}=w_0.
\]
Then for any $r\in{\bm{\mathrm{R}}}$, if $f\in Y^r(J)$ and $w_0\in H^r$, we have~$w\in X^r(J)$ and

\[
{\left\Vert} w{\right\Vert}_{X^r(J)}{\leq} K\left\{ {\left\Vert} w_0{\right\Vert}_{ H^r}+{\left\Vert} f{\right\Vert}_{ Y^r(J)}\right\}.
\]
for some constant $K$ depending only on $r, \rho, c,$ and $\mathcal{M}^1_{\rho}(p)$.
\end{theo}
\begin{theo}[\protect{\cite[Proposition~2.4]{ABZ4}}]\label{regularityhold}
Let $\rho\in (0, 1),~J=[z_0, z_1]\subset {\mathbf{R}},~p\in \Gamma^1_{\rho}({\mathbf{R}}^d\times J)$ with the assumption that 
\[
\Re p(z; x, \xi)\ge c|\xi|,
\]
for some constant $c>0$. Assume that $w$ solves
\[
\partial_zw+T_pw=F_1+F_2,\quad w\arrowvert_{z=z_0}=w_0.
\]
Then for any $q\in [1, \infty],~(r_0, r_1)\in {\mathbf{R}}^2$ with $r_0<r_1$, if 
\[
w\in L^{\infty}(J, C^{r_0}_*),{\hspace*{.15in}} F_1\in L^1(J, C^{r_1}_*), {\hspace*{.15in}} F_2\in L^q(J, C^{r_1-1+\frac{1}{q}+\delta}_*) ~\text{with}~\delta>0.
\]
and $w_0\in C^{r_1}_*({\mathbf{R}}^d)$, we have $w\in L^{\infty}(J, C^{r_1}_*)$ and 
\[
{\left\Vert} w{\right\Vert}_{C^0(J, C^{r_1}_*)}{\leq} K\left\{ {\left\Vert} w_0{\right\Vert}_{ C^{r_1}_*}+{\left\Vert} F_1{\right\Vert}_{ L^1(J, C^{r_1}_*)}+{\left\Vert} F_2{\right\Vert}_{ L^q(J, C^{r_1-1+\frac{1}{q}+\delta}_*)}+{\left\Vert} w{\right\Vert}_{ L^{\infty}(J, C^{r_0}_*)}\right\}.
\]
for some constant $K$ depending only on $r_0, r_1, \rho, c, \delta, q$ and $\mathcal{M}^1_{\rho}(p)$.
\end{theo}
\begin{thebibliography}{10}
\small

\bibitem{ABZ1}
Thomas Alazard, Nicolas Burq, and Claude Zuily.
\newblock On the water waves equations with surface tension.
\newblock {\em Duke Math. J.}, 158(3):413--499, 2011.

\bibitem{ABZ2}
Thomas Alazard, Nicolas Burq, and Claude Zuily.
\newblock Strichartz estimates for water waves.
\newblock {\em Ann. Sci. {\'E}c. Norm. Sup{\'e}r. (4)}, 44(5):855--903, 2011.

\bibitem{ABZ3}
Thomas Alazard, Nicolas Burq, and Claude Zuily.
\newblock On the Cauchy problem for gravity water waves.
\newblock {\em Invent.Math.}, 198(1): 71--163, 2014.

\bibitem{ABZ3'}
Thomas Alazard, Nicolas Burq and Claude Zuily. 
\newblock Cauchy theory for the gravity water waves system with nonlocalized initial data.
\newblock {\em Ann. Inst. H. Poincar\'e Anal. Non Linéaire}, 2014.

\bibitem{ABZ4}
Thomas Alazard, Nicolas Burq, and Claude Zuily.
\newblock Strichartz estimate and the Cauchy problem for the gravity water waves equations.
\newblock{ \em arXiv:1404.4276}, 2014.

\bibitem{AM}
Thomas Alazard and Guy M{\'e}tivier.
\newblock Paralinearization of the {D}irichlet to {N}eumann operator, and
  regularity of three-dimensional water waves.
\newblock {\em Comm. Partial Differential Equations}, 34(10-12):1632--1704,
  2009.

\bibitem{Alipara}
Serge Alinhac.
\newblock Paracomposition et op\'erateurs paradiff\'erentiels.
\newblock {\em Comm. Partial Differential Equations}, 11(1):87--121, 1986.

\bibitem{AliXEDP}
Serge Alinhac.
\newblock Paracomposition et application aux \'equations non-lin\'eaires.
\newblock In {\em Bony-{S}j\"ostrand-{M}eyer seminar, 1984--1985}, pages Exp.\
  No.\ 11, 11. \'Ecole Polytech., Palaiseau, 1985.

\bibitem{BaCh}
Hajer Bahouri and Jean-Yves Chemin.
\newblock \'{E}quations d'ondes quasilin\'eaires et estimations de
  {S}trichartz.
\newblock {\em Amer. J. Math.}, 121(6):1337--1377, 1999.

\bibitem{BCD}
Hajer Bahouri, Jean-Yves Chemin, and Rapha{\"e}l Danchin.
\newblock {\em Fourier analysis and nonlinear partial differential equations},
  volume 343 of {\em Grundlehren der Mathematischen Wissenschaften [Fundamental
  Principles of Mathematical Sciences]}.
\newblock Springer, Heidelberg, 2011.

\bibitem{BO}
T.~Brooke Benjamin and P.~J. Olver.
\newblock Hamiltonian structure, symmetries and conservation laws for water
  waves.
\newblock {\em J. Fluid Mech.}, 125:137--185, 1982.

\bibitem{Blair}
Matthew Blair.
\newblock Strichartz estimates for wave equations with coefficients of
  {S}obolev regularity.
\newblock {\em Comm. Partial Differential Equations}, 31(4-6):649--688, 2006.

\bibitem{Bony}
Jean-Michel Bony.
\newblock Calcul symbolique et propagation des singularit\'es pour les
  \'equations aux d\'eriv\'ees partielles non lin\'eaires.
\newblock {\em Ann. Sci. \'Ecole Norm. Sup. (4)}, 14(2):209--246, 1981.

\bibitem{BGT1}
Nicolas Burq, Patrick G\'erard, and Nikolay Tzvetkov.
\newblock Strichartz inequalities and the nonlinear {S}chr\"odinger equation on
  compact manifolds.
\newblock {\em Amer. J. Math.}, 126(3):569--605, 2004.

\bibitem{CaARMA}
R{\'e}mi Carles.
\newblock Geometric optics and instability for semi-classical {S}chr\"odinger
  equations.
\newblock {\em Arch. Ration. Mech. Anal.}, 183(3):525--553, 2007.

\bibitem{CCT}
Michael Christ, James Colliander, et Terence Tao.
\newblock Asymptotics, frequency modulation, and low regularity ill-posedness
  for canonical defocusing equations.
\newblock {\em Amer. J. Math.}, 125(6):1235--1293, 2003.

\bibitem{CMSW}
Robin~Ming Chen, Jeremy~L. Marzuola, Daniel Spirn, and J.~Douglas Wright.
\newblock On the regularity of the flow map for the gravity-capillary
  equations.
\newblock {\em J. Funct. Anal.}, 264(3):752--782, 2013.

\bibitem{CHS}
Hans Christianson, Vera~Mikyoung Hur, and Gigliola Staffilani.
\newblock Strichartz estimates for the water-wave problem with surface tension.
\newblock {\em Comm. Partial Differential Equations}, 35(12):2195--2252, 2010.

\bibitem{ChLi}
Demetrios Christodoulou and Hans Lindblad.
\newblock On the motion of the free surface of a liquid.
\newblock {\em Comm. Pure Appl. Math.}, 53(12):1536--1602, 2000.

\bibitem{CoCoGa}
Antonio C{\'o}rdoba, Diego C{\'o}rdoba, and Francisco Gancedo.
\newblock Interface evolution: water waves in 2-D.
\newblock Adv. Math., 223(1):120--173, 2010.

\bibitem{CS}
Daniel Coutand and Steve Shkoller.
\newblock Well-posedness of the free-surface incompressible {E}uler equations
  with or without surface tension.
\newblock {\em J. Amer. Math. Soc.}, 20(3):829--930 (electronic), 2007.

\bibitem{Craig1985}
Walter Craig.
\newblock {An existence theory for water waves and the Boussinesq and
  Korteweg-deVries scaling limits}.
\newblock {\em Communications in Partial Differential Equations},
  10(8):787--1003, 1985.

\bibitem{CrSu}
Walter Craig and Catherine Sulem. 
\newblock Numerical simulation of gravity waves. 
\newblock {\em J. Comput. Phys.}, 108(1):73--83, 1993.

\bibitem{CrWa}
Walter  Craig and C. Eugene Wayne.
\newblock{Mathematical aspects of surface waves on water}.
\newblock{\em Uspekhi Mat.
Nauk.}, 62(2007), 95–116.

\bibitem{Thibault} 
Thibault de Poyferré.
\newblock Blow-up conditions for gravity water-waves.
\newblock{\em  arXiv:1407.6881}, 2014.

\bibitem{NgPo}
Thibault de Poyferre and Quang-Huy Nguyen.
\newblock Strichartz estimates and local existence for the capillary water waves with non-Lipschitz initial velocity.
\newblock{\em  arXiv:1507.08918}, 2015.

\bibitem{Hormander}
Lars H{\"o}rmander.
\newblock {\em Lectures on nonlinear hyperbolic differential equations},
 {\em Math{\'e}matiques \& Applications (Berlin) [Mathematics \&
  Applications]}, volume 26.
\newblock Springer-Verlag, Berlin, 1997.

\bibitem{GMS}
Pierre Germain, Nader Masmoudi, and Jalal Shatah.
\newblock Global solutions for the gravity water waves equation in dimension 3.
\newblock {\em Annals of Mathematics}, 175(2):691--754, 2012.
 
\bibitem{GMS1}
Pierre Germain, Nader Masmoudi, and Jalal Shatah.
\newblock Global existence for capillary water waves.
\newblock{\em Comm. Pure Appl. Math.}, 68(4): 625--687, 2015.

\bibitem{HuIfTa}
John Hunter, Mihaela Ifrim, Daniel Tataru.
\newblock Two dimensional water waves in holomorphic coordinates.
\newblock{\em  arXiv:1401.1252v2}, 2014.

\bibitem{IfTa}
Mihaela Ifrim, Daniel Tataru.
\newblock Two dimensional water waves in holomorphic coordinates II: global solutions.
\newblock{\em arXiv:1404.7583}, 2014.

\bibitem{IfTa2}
Mihaela Ifrim, Daniel Tataru.
\newblock 
The lifespan of small data solutions in two dimensional capillary water waves.
\newblock{\em  arXiv:1406.5471}, 2014.

\bibitem{IoPu}
Alexandru D. Ionescu, Fabio Pusateri.
\newblock Global solutions for the gravity water waves system in 2d.
\newblock{\em Inventiones mathematicae},  Volume 199, Issue 3, pp 653--804, 2015.

\bibitem{IoPu2}
Alexandru D. Ionescu, Fabio Pusateri.
\newblock Global regularity for 2d water waves with surface tension .
\newblock{\em arXiv:1408.4428},  2015.

\bibitem{LannesJAMS}
David Lannes.
\newblock Well-posedness of the water-waves equations.
\newblock {\em J. Amer. Math. Soc.}, 18(3):605--654 (electronic), 2005.

\bibitem{LannesLivre}
David Lannes.
\newblock {\em Water waves: mathematical analysis and asymptotics.}
\newblock Mathematical Surveys and Monographs, 188. American Mathematical Society, Providence, RI, 2013. 

\bibitem{LindbladAnnals}
Hans Lindblad.
\newblock Well-posedness for the motion of an incompressible liquid with free
  surface boundary.
\newblock {\em Ann. of Math. (2)}, 162(1):109--194, 2005.

\bibitem{MePise}
Guy M{\'e}tivier.
\newblock {\em Para-differential calculus and applications to the {C}auchy
  problem for nonlinear systems}, volume~5 of {\em Centro di Ricerca Matematica
  Ennio De Giorgi (CRM) Series}.
\newblock Edizioni della Normale, Pisa, 2008.

\bibitem{MiZh}
Mei Ming and Zhifei Zhang.
\newblock Well-posedness of the water-wave problem with surface tension.
\newblock{\em J. Math. Pures Appl.} (9) 92, no. 5, 429--455, 2009. 

\bibitem{Nalimov}
V.~I. Nalimov.
\newblock The {C}auchy-{P}oisson problem.
\newblock {\em Dinamika Splo\v sn. Sredy}, (Vyp. 18 Dinamika Zidkost. so 
Svobod. Granicami):104--210, 254, 1974.

\bibitem{SZ1}
Jalal Shatah and Chongchun Zeng.
\newblock Geometry and a priori estimates for free boundary problems of the
  {E}uler equation.
\newblock {\em Comm. Pure Appl. Math.}, 61(5):698--744, 2008.

\bibitem{SZ2}
Jalal Shatah and Chongchun Zeng.
\newblock  A priori estimates for fluid interface problems.
\newblock{\em Comm. Pure Appl. Math.} 61(6): 848–876, 2008.

\bibitem{SZ3}
Jalal Shatah and Chongchun Zeng.
 \newblock Local well-posedness for fluid interface problems. 
\newblock{\em Arch. Ration. Mech. Anal.} 199(2): 653–705, 2011.

\bibitem{Smith}
Hart~F. Smith.
\newblock A parametrix construction for wave equations with {$C^{1,1}$}
  coefficients.
\newblock {\em Ann. Inst. Fourier (Grenoble)}, 48(3):797--835, 1998.

\bibitem{StTa}
Gigliola Staffilani and Daniel Tataru.
\newblock Strichartz estimates for a {S}chr\"odinger operator with nonsmooth
  coefficients.
\newblock {\em Comm. Partial Differential Equations}, 27(7-8):1337--1372, 2002.

\bibitem{TataruNS}
Daniel Tataru.
\newblock Strichartz estimates for operators with nonsmooth coefficients and
  the nonlinear wave equation.
\newblock {\em Amer. J. Math.}, 122(2):349--376, 2000.

\bibitem{TataruNSII}
Daniel Tataru.
\newblock Strichartz estimates for second order hyperbolic operators with
  nonsmooth coefficients. {II}.
\newblock {\em Amer. J. Math.}, 123(3):385--423, 2001.

\bibitem{Taylor}
Michael~E. Taylor.
\newblock {\em pseudo-differential operators and nonlinear {PDE}}, volume 100 of
  {\em Progress in Mathematics}.
\newblock Birkh\"auser Boston Inc., Boston, MA, 1991.

\bibitem{WaZh}
Chao Wang and Zhifei Zhang.
\newblock Break-down criterion for the water-wave equation.
\newblock  arXiv:1303.6029.

\bibitem{WuInvent}
Sijue Wu.
\newblock Well-posedness in {S}obolev spaces of the full water wave problem in
  2-{D}.
\newblock {\em Invent. Math.}, 130(1):39--72, 1997.

\bibitem{WuJAMS}
Sijue Wu.
\newblock Well-posedness in {S}obolev spaces of the full water wave problem in
  3-{D}.
\newblock {\em J. Amer. Math. Soc.}, 12(2):445--495, 1999.

\bibitem{Wu09}
Sijue Wu.
\newblock Almost global wellposedness of the 2-{D} full water waves problem.
\newblock {\em Invent. Math.}, 177(1):45--135, 2009.

\bibitem{Wu11}
Sijue Wu.
\newblock Global wellposedness of the 3-{D} full water waves problem.
\newblock {\em Invent. Math.}, 184(1):125--220, 2011.

\bibitem{Yosihara}
Hideaki Yosihara.
\newblock Gravity waves on the free surface of an incompressible perfect fluid
  of finite depth.
\newblock {\em Publ. Res. Inst. Math. Sci.}, 18(1):49--96, 1982.

\bibitem{Zakharov1968}
Vladimir~E. Zakharov.
\newblock Stability of periodic waves of finite amplitude on the surface of a
  deep fluid.
\newblock {\em Journal of Applied Mechanics and Technical Physics},
  9(2):190--194, 1968.

\end{thebibliography}
 
\end{document}
 


\documentclass[12pt,twoside,cd]{amsart}

\usepackage[]{geometry}
\geometry{top = 1in, bottom = 1in, left = 1in, right = 1in}
\headheight=36pt
\headsep=12pt
\footskip=24pt

\usepackage{fancyhdr}
\usepackage{marvosym}

\usepackage{amsmath,amssymb,amsthm,amsrefs,mathtools,tikz, amscd, verbatim, graphicx, xcolor, color, mathrsfs,enumerate,dsfont,extarrows,eucal,floatrow,array}
\usetikzlibrary{shapes,matrix,patterns}
\usepackage[all]{xy}
\usepackage{pdflscape}

\usepackage[english]{babel}
\usepackage{graphicx}

\newenvironment{solution}{\begin{proof}[Solution.]}\end{proof}

\newtheorem{theorem}{Theorem}[section]
\newtheorem{lemma}[theorem]{Lemma}
\newtheorem{corollary}[theorem]{Corollary}
\newtheorem{proposition}[theorem]{Proposition}

\theoremstyle{definition}
\newtheorem{remark}[theorem]{Remark}
\newtheorem{example}[theorem]{Example}
\newtheorem{definition}[theorem]{Definition}

\usepackage{chngcntr}
\counterwithin{figure}{section}

\title{Tropicalizing Spherical Embeddings}
\author{Evan D. Nash}
\address{The Ohio State University, 231 W 18th Ave, MW 216, Columbus OH, 43210}
\email{nash.228@osu.edu}

\begin{document}
\begin{abstract}
Recently, a theory for the tropicalization of a spherical homogeneous space $G/H$ was developed by Tassos Vogiannou. We extend his ideas to define the tropicalization of a spherical $G/H$-embedding. This generalizes the construction of tropicalization of toric varieties.
\end{abstract}
\maketitle
\small
\section*{Introduction}\label{intro}

Marrying algebraic geometric ideas and combinatorics is an active area of research. In recent years, the notion of tropicalization has proven to be a fruitful such tool to answer questions in algebraic geometry. Particularly, toric varieties have benefited from tropical geometric methods meshing with their inherent combinatorial structure. See for example Chapter 6 of \cite{MS} for an introduction to some examples of the utility of tropical ideas in the toric world. Toric varieties are examples of spherical varieties, which encompass a wider class of algebraic objects, among them flag varieties and symmetric varieties.
Spherical varieties also have combinatorial structure in the form of colored fans, which directly generalize the well-known polyhedral fans of toric geometry.
This theory was developed by Luna and Vust \cite{LV} in 1983.
It is a natural idea to take advantage of the similar combinatorial structure and extend the theory of tropicalization from toric varieties to the more general case.

The first steps in this direction were taken by Tassos Vogiannou in his thesis \cite{Vo}.
Among other results, Vogiannou developed a definition for the tropicalization of subvarieties of a spherical homogeneous space, which is the analogue of the dense torus orbit present in a toric variety.
In a forthcoming paper, Kiumars Kaveh and Christopher Manon extend Vogiannou's work by defining a theory of Gr\"obner bases on spherical varieties and showing that their definition agrees with Vogiannou's via a spherical fundamental theorem. 
They further consider a notion of spherical amoebas and show, with an additional assumption, that this amoeba approaches the tropicalization.
The purpose of this note is also to extend Vogiannou's work by defining the tropicalization of a general spherical embedding.
Our blueprint for this construction appeared separately in \cite{Ka} and \cite{Pay}.
These papers define the tropicalization of a toric variety by extending the tropicalization of its dense torus.
We will mimic their ideas using the shared polyhedral fan structure of toric and spherical varieties.

The layout of this article is as follows. In \textsection \ref{sphvar}, we review the basic theory of spherical varieties and \textsection \ref{HomSpaces} describes Vogiannou's definition of tropicalizing spherical homogeneous spaces.
Our construction of the tropicalization of a spherical embedding is explained in \textsection\ref{construction} and \textsection\ref{examples} contains examples.

\subsection*{Acknowledgments} This work was completed at the Fields Institute in the University of Toronto. The author is indebted to Gary Kennedy for suggesting this area of research and for providing guidance and discussions throughout its development. This paper also benefited from conversations with Kiumars Kaveh and Jenia Tevelev.

\section{Spherical Varieties}\label{sphvar}

There are a number of surveys on spherical varieties and their combinatorial structure.
Refer for example to \cite{LV}, \cite{Kn}, \cite{Pas}, or \cite{Pe} for more details on the theory discussed in this section.
Note that we use unconventional notation for some of the objects coming from the theory of spherical varieties.
There is a clash between symbols used for toric varieties and their analogs in spherical varieties; whenever possible we have favored the toric conventions as these are more widely known.
We work throughout over an algebraically closed field $k$.
Let $G$ be a connected reductive group with a Borel subgroup $B$. Let $H \leq G$ be a closed subgroup such that there is an action of $B$ on $G/H$ which has an open orbit.
In this case, we call $G/H$ a \emph{spherical homogeneous space}.
A normal $G$-variety $X$ which contains $G/H$ as an open orbit of the action of $G$ is called a \emph{spherical variety}. We sometimes say that $X$ is a \emph{$G/H$-embedding}.

Let $\mathcal{X}$ denote the group of characters $B \rightarrow k^*$.
We consider the $B$ semi-invariant rational functions on $G/H$:
\[
k(G/H)^{(B)} := {\left\lbrace {f \in k(G/H)^\ast \mid \text{there exists } \chi_f \in \mathcal{X} \text{ such that } gf = \chi_f(g)f \text{ for all } g \in B } \right\rbrace}.
\]
Here, the action of the Borel subgroup on $k(G/H)$ is given by $gf(x) = f(g^{-1}x)$, so $gf$ is only defined on those $x$ such that $g^{-1}x$ is in the domain of $f$.
This affords us a homomorphism $k(G/H)^{(B)} \rightarrow \mathcal{X}$ defined by $f \mapsto \chi_f$.
Further, the kernel of this map is the set of nonzero constant functions, which we write as $k^\ast$.
Then denote by $M$ or $M(G/H)$ the image of $k(G/H)^{(B)}/k^\ast$ in $\mathcal{X}$.
This is usually written as $\Lambda$ in the general theory of spherical varieties.
We will often identify the characters in $M$ with their associated $B$ semi-invariant rational functions in $k(G/H)^{(B)}/k^\ast$.
The lattice $M$ is finitely generated and free, so we obtain a vector space $N_\mathbb{Q}(G/H) := \text{Hom}(M, \mathbb{Q}) \cong \mathbb{Q}^m$, simply denoted $N_\mathbb{Q}$ when the underlying homogeneous space is clear.
This vector space is usually denoted $\mathcal{Q}$ in the general theory of spherical varieties.
The integer $m$ is called the \emph{rank} of the $G/H$-embedding.

We must further define the valuation cone, which will lie inside $N_\mathbb{Q}$. We consider $G$-invariant $\mathbb{Q}$-valuations $k(G/H) \rightarrow \mathbb{Q}$ which are trivial on $k^\ast$. 
By restricting such a valuation to $k(G/H)^{(B)}$, we obtain an induced map $k(G/H)^{(B)}/k^\ast \rightarrow \mathbb{Q}$, so we can identify it with a point in $N_\mathbb{Q}$.
These valuations span a cone in $N_\mathbb{Q}$ which we call the \emph{valuation cone}, denoted by $\mathcal{V}(G/H)$ or $\mathcal{V}$.
We write $\mathcal{D}(G/H)$ or $\mathcal{D}$ for the (finite) set of $B$-stable prime divisors in $G/H$. We refer to $\mathcal{D}(G/H)$ as the \emph{palette} of $G/H$ and call its elements \emph{colors}.
Every color $D$ induces a valuation $\nu_D$ on $k(G/H)$ given by a function's order of vanishing along the divisor.
We write $\rho$ for the map defined by $D \mapsto \nu_D$.

To recap, given a spherical homogeneous space $G/H$, we associate the palette $\mathcal{D}$ and a vector space $N_\mathbb{Q}$ containing the valuation cone $\mathcal{V}$. 
\begin{definition}
Let $\sigma \subseteq N_\mathbb{Q}$ be a cone and $\mathcal{F} \subseteq \mathcal{D}$. We call the pair $(\sigma,\mathcal{F})$ a \emph{colored cone} if the following properties are satisfied:
\begin{enumerate}
\item[1)] $\sigma$ is generated by $\rho(\mathcal{F})$ and finitely many elements of $\mathcal{V}$;
\item[2)] $\text{int}(\sigma) \cap \mathcal{V} \neq \emptyset$.
\end{enumerate}
We say that $(\sigma,\mathcal{F})$ is \emph{strictly convex} if in addition $\sigma$ is a strictly convex cone and $0 \notin \rho(\mathcal{F})$.
\end{definition}
\begin{definition}
A colored cone $(\tau,\mathcal{F}')$ is a  \emph{(colored) face} of a colored cone $(\sigma,\mathcal{F})$ if $\tau$ is a face of $\sigma$ satisfying condition (2) above such that $\mathcal{F}' = \mathcal{F} \cap \rho^{-1}(\tau)$. In this case we write $(\tau,\mathcal{F}') \preceq (\sigma,\mathcal{F})$ or $\tau \preceq \sigma$ if the colors are understood.
\end{definition}
\begin{definition}
A \emph{colored fan} is a finite collection $\Sigma$ of colored cones such that the following hold:
\begin{enumerate}
\item If $(\sigma,\mathcal{F}) \in \Sigma$ is a colored cone and $(\tau,\mathcal{F}')$ is a face of $(\sigma,\mathcal{F})$, then $(\tau,\mathcal{F}') \in \Sigma$.
\item Every $v \in \mathcal{V}$ is in the interior of at most one colored cone in $\Sigma$.
\end{enumerate}
We say in addition that $\Sigma$ is \emph{strictly convex} if each of its colored cones is strictly convex.
\end{definition}

For spherical varieties the notation $\mathcal{C}$ for the cone instead of $\sigma$ is more common.
With these definitions in hand, we can describe the colored fan associated to a $G/H$-embedding $X$.
Note that a $B$-stable divisor $D$ on $X$ which is not $G$-stable can be identified with a color $D \cap G/H$.
Moreover, the entire palette $\mathcal{D}(G/H)$ arises in this way.
Then for a closed $G$-orbit $\mathcal{O}$ of $X$, let $\mathcal{F} \subseteq \mathcal{D}$ consist of the $B$-stable divisors containing $\mathcal{O}$ which are not $G$-stable.
The colored cone $(\sigma,\mathcal{F})$ associated to $\mathcal{O}$ is spanned by $\rho(\mathcal{F})$ and the set of $G$-stable divisors containing $\mathcal{O}$.
Taking these colored cones over every $G$-orbit of $X$, we obtain a colored fan.
If a $G/H$-embedding $X$ has a single closed $G$-orbit and hence a single maximal colored cone, we call $X$ \emph{simple}.
A $G/H$-embedding consists of some finite number of simple embeddings glued together along $G$-orbits; this structure is reflected in the polyhedral geometry of the colored fan.
The association of a fan to a $G/H$-embedding characterizes all possible $G/H$-embeddings:
\begin{theorem}
[\cite{LV} Prop. 8.10, \cite{Kn} Thm. 3.3]
There is a bijection between simple $G/H$-embeddings and strictly convex colored cones in $N_\mathbb{Q}$ and there is a bijection between $G/H$-embeddings and strictly convex colored fans in $N_\mathbb{Q}$.
\end{theorem}
\begin{example}\label{EX}
Let $G = {\operatorname{Sl}}_2$ with $B$ the subgroup of upper triangular matrices and
\[
H = {\left\lbrace {M \in G \mid M \text{ is upper triangular with 1's on the diagonal}} \right\rbrace}.
\]
Then $G/H = \mathbb{A}^2 \setminus {\left\lbrace {0} \right\rbrace}$ where the action of $G$ is given by matrix multiplication of a column vector $[x \; y]^T$.
Every character $B \rightarrow k^\ast$ is of the form
\[
\chi_n: \left( \begin{array}{cc}
a & b \\
0 & a^{-1}
\end{array} \right)
\mapsto a^n
\]
for some $n \in \mathbb{Z}$, so $\mathcal{X} \cong \mathbb{Z}$.
Under the prescribed action of $G$, we can see that $k(G/H)^{(B)}/k^* = {\left\lbrace {y^n \mid n \in \mathbb{Z}} \right\rbrace}$ and the character associated to $y^n$ is $\chi_n$.
It follows that $M \cong \mathbb{Z}$ and hence $N_\mathbb{Q} \cong \mathbb{Q}$.

We now turn to the valuation cone $\mathcal{V}$.
Consider the following two valuations of $k(G/H)$, which are $G$-invariant:
\[
\frac{f}{g} \mapsto {{\text{mindeg } {f}}} - {{\text{mindeg } {g}}} \qquad \qquad \frac{f}{g} \mapsto \deg{g} - \deg{f}
\]
Here, mindeg denotes the minimum degree of a monomial in a polynomial in $k[x,y]$.
After restricting to $k(G/H)^{(B)} = {\left\lbrace {y^n} \right\rbrace}$, we see that the valuation on the left corresponds to sending $y^m \mapsto m$ (i.e. $\chi_1^\ast$)  
and the one on the right to $y^m \mapsto -m$ (i.e. $\chi_{-1}^\ast$).
Thus, positive multiples of these valuations induce every possible element of $N_\mathbb{Q} = \text{Hom}(M, \mathbb{Q})$ and so $\mathcal{V} = N_\mathbb{Q}$.

The only closed $B$-orbit contained in $G/H$ is the divisor $D := V(y)$, so the palette $\mathcal{D}$ consists solely of $D$.
This means that an embedding of $G/H$ can have at most one color, corresponding to the divisor where $y$ vanishes.
This divisor gives the cone spanned by $\chi_1^\ast$.

We'll finish by explicitly computing the fan associated to $\mathbb{P}^2$ with homogeneous coordinates $W$, $X$, and $Y$. 
We can realize $\mathbb{P}^2$ as an embedding of $\mathbb{A}^2 \setminus {\left\lbrace {0} \right\rbrace}$ via $[x \; y]^T \mapsto [1 : x : y]$.
There are three $G$-orbits in $\mathbb{P}^2$:
\begin{align*}
\mathbb{A}^2 \setminus {\left\lbrace {0} \right\rbrace} & := {\left\lbrace {[1 : x : y] \mid x,y \in k \text{ not both zero}} \right\rbrace} \\
V(W) & := {\left\lbrace {[0 : x : y] \mid x,y \in k \text{ not both zero}} \right\rbrace} \\
O & := {\left\lbrace {[1 : 0 : 0]} \right\rbrace}.
\end{align*}
The latter two orbits are closed, so our fan will have two maximal cones. 
The orbit $V(W)$ is itself a B-stable divisor.
This divisor is $G$-stable, so we will have a cone without color.
The function $y$ in $k(\mathbb{A}^2 \setminus {\left\lbrace {0} \right\rbrace})$ can be written as $Y/W$ along $\mathbb{P}^2$.
On $V(W)$, $Y/W$ has a pole of order $1$, so the cone associated to this orbit is the cone spanned by $\chi_{-1}^\ast$.
The other closed orbit $O$ is contained in one $B$-stable divisor as well: $V(Y)$.
This divisor is not $G$-stable, so the corresponding ray will have color.
Clearly $y$ vanishes with order 1 on $V(Y)$, so this will give the cone spanned by $\chi_1^\ast$.
This example is drawn in Table \ref{table} along with the other colored fans of $\mathbb{A}^2 \setminus {\left\lbrace {0} \right\rbrace}$.
This table also appears in \cite{Vo} except for the last column; note how the color is indicated by a bullseye.
\begin{table}[h]
\centering
\begin{tabular}{| c | c | c | c |}
\hline
Variety & Closed $G$-orbits & Colored Fan & Tropicalization \\
\hline
$\mathbb{A}^2 \setminus {\left\lbrace {0} \right\rbrace}$ & $\mathbb{A}^2 \setminus {\left\lbrace {0} \right\rbrace}$ & \begin{tikzpicture}
\draw[->,white] (0,0)--(1,0);
\draw[->,white] (0,0)--(-1,0);
\draw[fill] (0,0) circle(.05);
\end{tikzpicture} &
\begin{tikzpicture}
\draw[white,fill = white] (1,0) circle(.05);
\draw[white,fill = white] (-1,0) circle(.05);
\draw (-1,0)--(1,0);
\end{tikzpicture} \\

$\mathbb{A}^2$ & ${\left\lbrace {0} \right\rbrace}$ & \begin{tikzpicture}
\draw[red,->] (0,0)--(1,0);
\draw[->,white] (0,0)--(-1,0);
\draw[fill] (0,0) circle(.05);
\draw[red, fill= white] (.5,0) circle(.1);
\draw[red,fill= red] (.5,0) circle(.05);
\end{tikzpicture} &
\begin{tikzpicture}
\draw[white,fill = white] (-1,0) circle(.05);
\draw (-1,0)--(1,0);
\draw[red,fill = white] (1,0) circle(.1);
\draw[red,fill = red] (1,0) circle(.05);
\end{tikzpicture} \\

${\operatorname{Bl}}_0(\mathbb{A}^2)$ & $E$ & \begin{tikzpicture}
\draw[->] (0,0)--(1,0);
\draw[->,white] (0,0)--(-1,0);
\draw[fill] (0,0) circle(.05);
\end{tikzpicture} &
\begin{tikzpicture}
\draw[white] (1,0) circle(.1);
\draw[fill] (1,0) circle(.05);
\draw[white,fill = white] (-1,0) circle(.05);
\draw (-1,0)--(1,0);
\end{tikzpicture} \\

$\mathbb{P}^2 \setminus {\left\lbrace {0} \right\rbrace}$ & $V(W)$ & \begin{tikzpicture}
\draw[->,white] (0,0)--(1,0);
\draw[->] (0,0)--(-1,0);
\draw[fill] (0,0) circle(.05);
\end{tikzpicture} & 
\begin{tikzpicture}
\draw[white] (-1,0) circle(.1);
\draw[fill] (-1,0) circle(.05);
\draw[white,fill = white] (1,0) circle(.05);
\draw (-1,0)--(1,0);
\end{tikzpicture} \\

$\mathbb{P}^2$ & $V(W)$, ${\left\lbrace {0} \right\rbrace}$ & \begin{tikzpicture}
\draw[red,->] (0,0)--(1,0);
\draw[->] (0,0)--(-1,0);
\draw[fill] (0,0) circle(.05);
\draw[red, fill= white] (.5,0) circle(.1);
\draw[red,fill= red] (.5,0) circle(.05);
\end{tikzpicture} & 
\begin{tikzpicture}
\draw[fill] (-1,0) circle(.05);
\draw (-1,0)--(1,0);
\draw[red,fill = white] (1,0) circle(.1);
\draw[red,fill = red] (1,0) circle(.05);
\end{tikzpicture} \\

${\operatorname{Bl}}_0(\mathbb{P}^2)$ & $V(W)$, $E$ & \begin{tikzpicture}
\draw[->] (0,0)--(1,0);
\draw[->] (0,0)--(-1,0);
\draw[fill] (0,0) circle(.05);
\end{tikzpicture} & 
\begin{tikzpicture}
\draw[white] (1,0) circle(.1);
\draw[fill] (1,0) circle(.05);
\draw[fill] (-1,0) circle(.05);
\draw (-1,0)--(1,0);
\end{tikzpicture} \\
\hline
\end{tabular}
\caption{Colored fans and colored tropicalizations associated to $\mathbb{A}^2 \setminus {\left\lbrace {0} \right\rbrace}$. The $E$ denotes the exceptional divisor of the blowup.}
\label{table}
\end{table}
\end{example}

\section{Tropicalizing Homogeneous Spaces}\label{HomSpaces}

In his thesis \cite{Vo}, Tassos Vogiannou defines the tropicalization of a subvariety of the spherical homogeneous space $G/H$, extending the well-known theory of subvarieties of a torus.
We outline his construction here; more details and examples can be found in his thesis.
Suppose $G/H$ is a spherical homogeneous space over $k$, let $K := k((t))$ denote the Laurent series over $k$, and let $\overline{K} := \bigcup_{n \in \mathbb{N}} k((t^{1/n}))$ denote the field of Puiseaux series over $k$.
We use the valuation $\nu: \overline{K} \rightarrow \mathbb{Q}$ which gives the lowest power of $t$ appearing with nonzero coefficient.
Note that this restricts naturally to $K$ and is trivial on $k$.

We will define a map $G/H(K) \rightarrow N_\mathbb{Q}$.
Let $\gamma: {\operatorname{Spec}}{K} \rightarrow G/H$ be a $K$-point of $G/H$.
We will define a $G$-invariant discrete valuation $\nu_\gamma$ on $k(G/H)^\ast$ associated to $\gamma$.
To do this, we need to describe how $\nu_\gamma$ acts on rational functions, so let $f \in k(G/H)^\ast$ be arbitrary.
The domain of $f$ may not contain the image of $\gamma$, but we may find $g \in G$ such that the image of $\gamma$ is in the domain of $gf$.
There is a pullback map $\gamma^\ast: k(G/H) \rightarrow K$ given by evaluation at $\gamma$, so we consider $\gamma^\ast(gf) \in K$.
Then we write $\nu_\gamma(f) = \nu(\gamma^\ast(gf))$.
This is not a priori well-defined since it may depend on $g$.
To overcome this, we take $g$ so that $\nu(\gamma^\ast(gf))$ is minimized; this minimum is achieved on an open set of $G$ and we call such $g$ \emph{sufficiently general}.

Thus we have a map $G/H(K) \rightarrow {\left\lbrace {G\text{-invariant discrete valuations on } k(G/H)^\ast} \right\rbrace}$ given by $\gamma \mapsto \nu_\gamma$.
As discussed, $G$-invariant discrete valuations on $k(G/H)^\ast$ induce elements of $\mathcal{V}$, so this is really a map $G/H(K) \rightarrow \mathcal{V}$.
Further, we can extend this map so it is defined over $G/H(\overline{K})$.
Indeed, suppose $\gamma: {\operatorname{Spec}}{\overline{K}} \rightarrow G/H$ is a $\overline{K}$-point.
This induces a homomorphism of $k$-algebras $\gamma^\ast: A \rightarrow \overline{K}$ since the image of $\gamma$ must lie in some open affine ${\operatorname{Spec}}{A} \subseteq X$.
Since $G/H$ is of finite type, $A$ is finitely-generated as a $k$-algebra, and so it follows that $\gamma^\ast$ factors through $k((t^{1/n}))$ for some sufficiently large $n$.
Thus, $\gamma$ factors as ${\operatorname{Spec}}{\overline{K}} \rightarrow {\operatorname{Spec}}{k((t^{1/n}))} \rightarrow G/H$.
We can think of ${\operatorname{Spec}}{k((t^{1/n}))}$ as the spectrum of Laurent polynomials in an indeterminate variable $t^{1/n}$.
This morphism induces a valuation by the work above; dividing this valuation by $n$ gives a valuation $\nu_\gamma$ which we associate to $\gamma$.
This extension in fact gives a surjection $\text{val}: G/H(\overline{K}) \twoheadrightarrow \mathcal{V}$, which allows us to finally define the tropicalization of a subvariety of a homogeneous space.
\begin{definition}
If $Y \subseteq G/H$ is a subvariety, the \emph{tropicalization} of $Y$ is ${\operatorname{trop}}{Y} = \text{val}(Y(\overline{K}))$.
\end{definition}

In particular, note that ${\operatorname{trop}}{G/H} = \mathcal{V}(G/H)$, a fact we will use in \textsection \ref{construction}.

\section{The Construction}\label{construction}

Again let $G/H$ be a spherical homogeneous space and $X$ a $G/H$-embedding.
Let $N_\mathbb{Q}$ and $\mathcal{V} \subseteq N_\mathbb{Q}$ be the associated vector space and valuation cone. Write $\Sigma(X)$ for the colored fan associated to $X$ and denote by ${\left\lbrace {(\sigma_i,\mathcal{F}_i)} \right\rbrace}_i$ the colored cones comprising $\Sigma(X)$.
Each colored cone corresponds to a $G$-orbit $\mathcal{O}_i$ of $X$, and Corollary 2.2 in \cite{Kn} says that each orbit is a spherical $G$-variety with the action of the same Borel subgroup $B$.
A spherical $G$-variety with one orbit is a spherical homogeneous space, so it follows for all $i$ that $\mathcal{O}_i \cong G/H_i$ where $H_i$ is a closed subgroup of $G$.
As a spherical homogeneous space, an orbit $\mathcal{O}_i$ associated to $(\sigma_i,\mathcal{F}_i)$ has a valuation cone $\mathcal{V}_i := \mathcal{V}(G/H_i)$ which lies in a $\mathbb{Q}$-vector space $N_\mathbb{Q}(G/H_i)$.
Then as a set, we define ${\operatorname{trop}}{X} := \bigsqcup_i \mathcal{V}_i$.
This is similar in spirit to the construction of \cite{Ka} and \cite{Pay}; we break $X$ up into orbits and tropicalize each of them separately.
It only remains to define a topology on this space.
We will do this by defining a topology on $\bigsqcup_i N_\mathbb{Q}(G/H_i)$ and giving ${\operatorname{trop}}{X}$ the induced subspace topology.

Let $\overline{\mathbb{Q}} := \mathbb{Q} \cup {\left\lbrace {\infty} \right\rbrace}$. We fix our attention on a colored cone $(\sigma_i,\mathcal{F}_i) \in \Sigma(X)$.
If $(\sigma_j,\mathcal{F}_j) \preceq (\sigma_i,\mathcal{F}_i)$, then there is a copy of $N_\mathbb{Q}(G/H_j)$ in ${\operatorname{Hom}}{(\sigma_i^\vee \cap M, \overline{\mathbb{Q}})}$ given by considering those semigroup homomorphisms $\varphi: \sigma_i^\vee \cap M \rightarrow \overline{\mathbb{Q}}$ for which $\varphi^{-1}(\mathbb{Q}) = \sigma_j^\perp \cap M$.
More explicitly, we have that 
\[
{\operatorname{Hom}}{(\sigma_j^\perp \cap M,\mathbb{Q})} \cong N_\mathbb{Q}(G/H_j).
\]
To see this, note that $\sigma_j^\perp \cap M$ consists of those functions in $M = k(G/H)^{(B)}/k^\ast$ which do not have zeroes or poles along the orbit $G/H_j$.
These are precisely the functions in $M$ which can be restricted to $B$ semi-invariant rational functions on $G/H_j$.
Restriction thus gives a map $\sigma_j^\perp \cap M \rightarrow M(G/H_j)$.
Theorem 6.3 of \cite{Kn} shows that this map is an isomorphism, so after dualizing we have ${\operatorname{Hom}}{(\sigma_j^\perp \cap M,\mathbb{Q})} \cong N_\mathbb{Q}(G/H_j)$.
Further, homomorphisms in ${\operatorname{Hom}}{(\sigma_j^\perp \cap M,\mathbb{Q})}$ extend uniquely to homomorphisms in ${\operatorname{Hom}}{(\sigma_i^\vee \cap M,\overline{\mathbb{Q}})}$ by sending every character outside $\sigma_j^\perp \cap M$ to $\infty$.

In the toric case, it now follows that $\bigsqcup_{\sigma_j \preceq \sigma_i} N_\mathbb{Q}(G/H_j)$ is in bijective correspondence with ${\operatorname{Hom}}{(\sigma_i^\vee \cap M, \overline{\mathbb{Q}})}$.
For a general spherical variety, this is not necessarily true. 
This is because a colored cone may contain a subcone which is a face in the sense of polyhedral geometry but which does not correspond to an orbit in the spherical variety. 
To address this, we introduce the notation
\[
{\operatorname{Hom}}^\mathcal{V}{(\sigma_i^\vee \cap M, \overline{\mathbb{Q}})}
\]
to denote the homomorphisms $\varphi$ such that $\varphi^{-1}(\mathbb{Q}) = \sigma_j^\perp \cap M$ for some colored cone $(\sigma_j,\mathcal{F}_j) \preceq (\sigma_i,\mathcal{F}_i)$.

Now every homomorphism in ${\operatorname{Hom}}^\mathcal{V}{(\sigma_i^\vee \cap M,\overline{\mathbb{Q}})}$ is realized as an extension of a homomorphism $\sigma_j^\perp \cap M \rightarrow \mathbb{Q}$ where $\sigma_j \preceq \sigma_i$. 
Thus we have that $\bigsqcup_{\sigma_j \preceq \sigma_i} N_\mathbb{Q}(G/H_j)$ is in bijective correspondence with ${\operatorname{Hom}}^\mathcal{V}{(\sigma_i^\vee \cap M, \overline{\mathbb{Q}})}$.
Placing the product topology on ${\operatorname{Hom}}^\mathcal{V}{(\sigma_i^\vee \cap M, \overline{\mathbb{Q}})}$, we obtain a topology on $\bigsqcup_{\sigma_j \preceq \sigma_i} N_\mathbb{Q}(G/H_j)$.

We still must see how $\bigsqcup_{\sigma_j \preceq \sigma_i} \mathcal{V}_j$ lies in $\bigsqcup_{\sigma_j \preceq \sigma_i} N_\mathbb{Q}(G/H_j)$ under this topology.
Any homomorphism in ${\operatorname{Hom}}{(\sigma_j^\perp \cap M,\mathbb{Q})}$ which is induced by a $G$-invariant valuation is the restriction of a $G$-variant valuation in $\mathcal{V}(G/H) \subseteq {\operatorname{Hom}}{(M,\mathbb{Q})}$.
Thus, the valuation cone $\mathcal{V}_i$ is the image of the valuation cone $\mathcal{V}$ under the map ${\operatorname{Hom}}{(M,\mathbb{Q})} \rightarrow {\operatorname{Hom}}{(\sigma_j^\perp \cap M,\mathbb{Q})}$ induced by the inclusion $\sigma_j^\perp \cap M \hookrightarrow M$.
This tells us how to tropicalize a simple spherical variety, one corresponding to a single colored cone. 
Gluing together these tropicalizations along shared orbits gives a tropicalization for the entire spherical embedding.
This construction results in a topological space which is stratified by $\mathbb{Q}$-vector spaces.

Our construction as discussed thus far only recognizes the polyhedral structure of the colored fan but ignores whether or not that fan has colors.
We need this information to completely classify all spherical embeddings, so it seems useful to remember the presence of colors when we tropicalize.
We address this as follows.
In a colored fan, we can think of our palette of colors as a collection of points in $N_\mathbb{Q}$ corresponding to $B$-stable divisors.
If no colors appear in our fan, we call the spherical variety \emph{toroidal}.
If a color appears, it will lie in some number of colored cones, which is to say the associated divisor contains the orbits corresponding to those cones.
Each such orbit gives a vector space in the stratification of the tropicalization; we simply record if the color appears in the colored fan by labeling those vector spaces with that color.
We show in Example \ref{A2} two different spherical embeddings of the same homogeneous space which have different colored tropicalizations; if color is ignored the tropicalizations become the same.

\begin{remark}\label{remark} 
Toroidal varieties are in general better behaved than non-toroidal varieties with respect to tropicalization. Explicitly, Vogiannou proved that tropical compactifications in the sense of Tevelev \cite{Te} exist in toroidal varieties (cf. \cite{Vo}, \textsection 4), but they do not necessarily exist in non-toroidal varieties.
Every spherical variety is dominated by a toroidal one, however, so this is not a huge problem. 
\end{remark}

We observe finally that the map $X(\overline{K}) \rightarrow {\operatorname{trop}}{X}$ is continuous.
To see this, recall that the tropicalization map is determined by evaluation at $\overline{K}$-points.
Embedding $G/H$ in a spherical variety adds $\overline{K}$-points which are limits of $\overline{K}$-points of $G/H$.
Take a sequence in $\overline{K}(G/H)$ which has a limit point $\gamma \in G/H_j$ outside of $\overline{K}(G/H)$.
Evaluation by functions in $k(G/H)$ will not be defined at $\gamma$ since the valuation of some functions will tend to $\infty$.
But if we take the limit as the sequence approaches $\gamma$ we obtain a homomorphism in ${\operatorname{Hom}}{(\sigma_j^\perp \cap M, \overline{\mathbb{Q}})}$.
This limit agrees with the tropicalization of $\gamma$ as a point in $\bigsqcup_i \mathcal{V}(G/H_i) = {\operatorname{trop}}{X}$, so the map is continuous.

\section{Examples}\label{examples}

\begin{example}\label{toricex}
In the toric case, $G = T^n$ is a torus of dimension $n$, $H$ is trivial, and $B = G$, so there are no colors.
The $B$ semi-invariant rational functions are precisely the monomials in the variables ${\left\lbrace {x_1,\ldots,x_n} \right\rbrace}$, so $N_\mathbb{Q} \cong \mathbb{Q}^n$, spanned by cocharacters $\chi_i^*$ defined as follows:
\[
\chi^*_i(x_j) = \left\lbrace \begin{array}{ll}
1 & i = j \\
0 & \text{otherwise.}
\end{array}
\right.
\]
We can also see that the valuation cone $\mathcal{V}$ is all of $N_\mathbb{Q}$.
Indeed, consider the valuations
\[
\frac{f}{g} \mapsto \text{mindeg}_i(f) - \text{mindeg}_i(g) \qquad \qquad \frac{f}{g} \mapsto \text{deg}_i(g) - \text{deg}_i(f),
\]
where $\text{deg}_i$ and $\text{mindeg}_i$ respectively denote the degree and minimum degree in $x_i$.
These valuations are $G$-invariant and induce the cocharacters $\chi_i^*$ and $-\chi_i^\ast$, so $\mathcal{V} = N_\mathbb{Q}$.

Having realized the torus as a spherical homogeneous space, the colored fan associated to a toric variety when viewed as a spherical embedding is the same as the fan coming from the theory of toric geometry.
Our definition of the tropicalization only relies on the polyhedral structure of this fan and is identical to the process described in \cite{Ka} and \cite{Pay}.
\end{example}
\begin{example}\label{flag}
A flag variety is a spherical homogeneous space $G/P$ where $P$ is a parabolic subgroup, one which contains a Borel subgroup.
Such a homogenous space has a trivial valuation cone, so the tropicalization of a flag variety is a point under this theory.
\end{example}
\begin{example}\label{A2}
We return now to the example and notation of $\mathbb{A}^2\setminus {\left\lbrace {0} \right\rbrace}$ discussed in Example \ref{EX}.
In ${\operatorname{Bl}}_0(\mathbb{P}^2)$, there are three $G$-orbits: $G/H$, $V(W)$, and the exceptional divisor $E$. We have already seen that the valuation cone of $G/H$ is a copy of $\mathbb{Q}$, so we move on to $V(W)$ and $E$.
Both of these are copies of $\mathbb{P}^1$ and the action of $G = {\operatorname{Sl}}_2$ on both of them is given by matrix multiplication.
Since we already know how $G$ acts on these orbits, it is not strictly necessary to know the subgroup $H'$ such that $\mathbb{P}^1 = G/H'$.
In this case $G/B = \mathbb{P}^1$, so the closed orbits are flag varieties and our discussion in example \ref{flag} tells us the tropicalizations are trivial.
Let us show this explicitly.
The rational functions $k(G/B) = k(\mathbb{P}^1)$ are the rational polynomials in two variables $X$ and $Y$ which are homogeneous of the same degree in the numerator and denominator.
The action of the Borel subgroup $B$ on these functions is the same as it is on $k(G/H)$, so the only $B$ semi-invariant rational functions on $k(G/B)$ are powers of $Y$.
The only power of $Y$ in $k(\mathbb{P}^1)$ is the constant function, so $k(\mathbb{P}^1)^{(B)}$ is trivial and hence so are the associated $M$, $N_\mathbb{Q}$, and $\mathcal{V}$.

Thus, $\bigsqcup_i \mathcal{V}_i$ in this case consists of a copy of $\mathbb{Q}$ and two points.
The two points attach to $\mathbb{Q}$ by thinking of them as $\infty$ and $-\infty$.
We can think of ${\operatorname{Bl}}_0(\mathbb{P}^2)$ as the two simple spherical varieties ${\operatorname{Bl}}_0(\mathbb{A}^2)$ and $\mathbb{P}^2 \setminus {\left\lbrace {0} \right\rbrace}$ glued together along $G/H = \mathbb{A}^2 \setminus {\left\lbrace {0} \right\rbrace}$.
In ${\operatorname{Bl}}_0(\mathbb{A}^2)$, we add in limit points over 0, which correspond to an extended valuation taking $y \in k(G/H)^{(B)}$ to $\infty$, giving a copy of $\overline{\mathbb{Q}}$.
In $\mathbb{P}^2 \setminus {\left\lbrace {0} \right\rbrace}$, we add in limit points at infinity, which similarly correspond to extended valuations and we again get $\overline{\mathbb{Q}}$.
We finally glue these copies of $\overline{\mathbb{Q}}$ along their shared copy of $\mathbb{Q}$.
This is illustrated in Figure \ref{BLP2}. The gluing is reminiscent of the tropicalization of $\mathbb{P}^1$ viewed as a toric variety, as described in \textsection 6.2 of \cite{MS}.

\begin{figure}[!h]
\begin{center}
\begin{tikzpicture}[scale = 2]
\draw[thick] (-4,-.2)--(-6,-.2);
\draw[thick] (-4,.2)--(-6,.2);
\draw[fill] (-4,-.2) circle(.05);
\draw[fill] (-6,.2) circle(.05);

\draw (-2.5,.15) node[]{$\sim$};
\draw[thick,->] (-3,0)--(-2,0);

\draw[thick] (-1,0)--(1,0);
\draw[fill] (-1,0) circle(.05);
\draw[fill] (1,0) circle(.05);
\end{tikzpicture}
\caption{The colored tropicalization of ${\operatorname{Bl}}_0(\mathbb{P}^2)$}
\label{BLP2}
\end{center}
\end{figure}
If instead we had considered the $G/H$-embedding $\mathbb{P}^2$, the simple spherical varieties are $\mathbb{P}^2 \setminus {\left\lbrace {0} \right\rbrace}$ and $\mathbb{A}^2$. The former can be tropicalized as before, but $\mathbb{A}^2$ has a $G$-fixed point $[1:0:0]$ whose associated cone has color.
The point is realized as the spherical homogeneous space $G/G$, which has trivial valuation cone, just like the effective divisor in ${\operatorname{Bl}}_0(\mathbb{P}^2)$.
The gluing operation works the same as with ${\operatorname{Bl}}_0(\mathbb{P}^2)$, so topologically we again obtain a line segment.
This is shown in Figure \ref{ColP2}; the colored point associated to $[1:0:0]$ is on the right.  
\begin{figure}[!h]
\begin{center}
\begin{tikzpicture}[scale = 2]
\draw[thick] (-1,0)--(1,0);
\draw[fill] (-1,0) circle(.05);
\draw[red,fill=white] (1,0) circle(.1);
\draw[red,fill= red] (1,0) circle(.05);
\end{tikzpicture}
\caption{The colored tropicalization of $\mathbb{P}^2$}
\label{ColP2}
\end{center}
\end{figure} 
\end{example}
\begin{example}
In this example our group is $G = {\operatorname{Gl}}_2 \times {\operatorname{Gl}}_2$ and the subgroup $H$ is the diagonal so that $G/H \cong {\operatorname{Gl}}_2$. The Borel subgroup $B$ is ${\left\lbrace {(U,L) \mid U \text{ is upper triangular and } L \text{ is lower triangular}} \right\rbrace}$ and the action of $G$ on $G/H$ is given by $(g,h) \cdot X = gXh^{-1}$.
The Borel subgroup has an open orbit ${\left\lbrace {(x_{ij} \in {\operatorname{Gl}}_2 \mid x_{22} \neq 0} \right\rbrace}$, so this is a spherical homogeneous space.
We will embed $G/H$ into ${\operatorname{Bl}}_0(\mathbb{A}^4)$ by sending the coordinates $x_{ij}$ of a matrix $X$ to its image in the subvariety $D(x_{11}x_{22} - x_{12}x_{21}) \subset \mathbb{A}^4 \setminus {\left\lbrace {0} \right\rbrace}$.
Viewing ${\operatorname{Bl}}_0(\mathbb{A}^4)$ as a subvariety of $\mathbb{A}^4 \times \mathbb{P}^3$, we give $\mathbb{P}^3$ the coordinates $y_{ij}$ and denote elements by $((x_{ij}),[y_{ij}])$, so the blow-up is cut out by the equations $x_{ij}y_{k\ell} = x_{k\ell}y_{ij}$.
The action of $G$ on ${\operatorname{Bl}}_0(\mathbb{A}^4)$ is then given by matrix multiplication in both components:
\[
(g,h) \cdot ((x_{ij}),[y_{ij}]) = (g(x_{ij})h^{-1},g[y_{ij}]h^{-1}), \quad (g,h) \in G, ((x_{ij}),[y_{ij}]) \in {\operatorname{Bl}}_0(\mathbb{A}^4).
\]
In this case, the lattice of $B$ semi-invariant rational functions $M = k(G/H)^{(B)}/k^*$ on $G/H$ is spanned by $f_1 := (x_{11}x_{22} - x_{12}x_{21})/x_{22}$ and $f_2 := x_{22}$. We choose these particular generators because the associated cocharacters are cleaner for computations.
The palette $\mathcal{D}$ in this case consists of one $B$-stable divisor: $V(x_{22})$.
The vector space $N_\mathbb{Q}$ of cocharacters is two-dimensional, spanned by $\chi_1^*$ and $\chi_2^*$ defined as follows:
\[
\chi_{i}^*(f_j) = \left\lbrace \begin{array}{ll}
1 & \text{if $i = j$} \\
0 & \text{otherwise}
\end{array}
\right..
\]
The valuation cone $\mathcal{V}$ associated to $G/H$ is ${\left\lbrace {\alpha_1\chi_1^* + \alpha_2\chi_2^* \mid \alpha_1 \geq \alpha_2} \right\rbrace}$. In general, the valuation cone of ${\operatorname{Gl}}_n \times {\operatorname{Gl}}_n$ with $H$ the diagonal subgroup consists of elements of $\mathbb{Q}^n$ whose components are non-increasing. See \textsection 5.3 of \cite{Vo} for a discussion of this fact. The valuation cone and palette of $G/H$ are shown in Figure \ref{valcone}.

Let us now determine the colored fan associated to ${\operatorname{Bl}}_0(\mathbb{A}^4)$ as a ${\operatorname{Gl}}_2$-embedding. Under the prescribed action of $G$, there are four $G$-orbits: 
\begin{align*}
{\operatorname{Gl}}_2 & := {\left\lbrace {((x_{ij}),[y_{ij}]) \mid (x_{ij}) \in {\operatorname{Gl}}_2} \right\rbrace} \\
R_1 & := {\left\lbrace {((x_{ij}),[y_{ij}]) \mid (x_{ij}) \text{ has  rank } 1} \right\rbrace} \\
\mathbb{P}({\operatorname{Gl}}_2) & :=  {\left\lbrace {(0,[y_{ij}]) \mid [y_{ij}] \in {\operatorname{Gl}}_2} \right\rbrace} \\
\mathbb{P}(R_1) & := {\left\lbrace {(0,[y_{ij}]) \mid [y_{ij}] \text{ has rank } 1} \right\rbrace}
\end{align*}
Only one of these orbits is closed: $\mathbb{P}(R_1)$, so we will have one maximal colored cone.

The $B$-stable divisors containing $\mathbb{P}(R_1)$ are the exceptional divisor $E$ and $V(y_{11}y_{22} - y_{12}y_{21}) = V(\text{det}(y_{ij}))$.
Both of these are also $G$-stable, so our fan will have no colors.
Along $E$, $f_2$ clearly vanishes with order 1 and $f_1$ can be written in the form $f_1 = f_2 \cdot (y_{11}y_{22} - y_{12}y_{21})/y_{22}^2$, so it also vanishes with order 1 along $E$. Thus we obtain a ray $\sigma_{1,1}$ in the direction $(1,1)$.
Along $V(\text{det}(y_{ij}))$, $f_1$ vanishes with order 1 and $f_2$ doesn't vanish, so we obtain a ray $\sigma_{1,0}$ in the direction $(1,0)$.
Figure \ref{blowupA4} exhibits the colored cone associated to ${\operatorname{Bl}}_0(\mathbb{A}^4)$.
\begin{figure}[!h]
\begin{floatrow}
\ffigbox{\begin{tikzpicture}
\draw[fill,gray!25] (-2,-2)--(2,-2)--(2,2);
\draw[thick,->] (0,0)--(2,2);
\draw[fill] (0,0) circle(.05);
\draw[thick,->] (0,0)--(-2,-2);
\draw[red,fill=red] (-1,1) circle(.05);
\draw[red] (-1,1) circle(.1);

\draw[->,dotted] (0,0)--(2,0);
\draw[->,dotted] (0,0)--(-2,0);
\draw[->,dotted] (0,0)--(0,2);
\draw[->,dotted] (0,0)--(0,-2);

\draw (2.3,0) node {$\alpha_1$};
\draw (0,2.15) node {$\alpha_2$};
\end{tikzpicture}
\caption{The valuation cone and palette of ${\operatorname{Gl}}_2$}
\label{valcone}}

\ffigbox{\begin{tikzpicture}
\draw[fill,gray!25] (0,0)--(2,2)--(2,0);
\draw[fill] (0,0) circle(.05);

\draw[very thick,->] (0,0)--(2,0);
\draw[very thick,->] (0,0)--(2,2);

\draw[->,dotted] (0,0)--(2,0);
\draw[->,dotted] (0,0)--(-2,0);
\draw[->,dotted] (0,0)--(0,2);
\draw[->,dotted] (0,0)--(0,-2);

\draw (2.3,0) node {$\alpha_1$};
\draw (0,2.15) node {$\alpha_2$};
\end{tikzpicture}
\caption{The colored cone of ${\operatorname{Bl}}_0(\mathbb{A}^4)$.}
\label{blowupA4}}

\end{floatrow}
\end{figure}

Now we apply our construction. 
We start with $\bigsqcup N_\mathbb{Q}(G/H_i)$, where the disjoint union is over four separate colored cones corresponding to the four $G$-orbits in ${\operatorname{Bl}}_0(\mathbb{A}^4)$.
There is one colored cone of dimension zero $({\operatorname{Gl}}_2)$, two of dimension one $(R_1 \text{ and } \mathbb{P}({\operatorname{Gl}}_2))$, and one of dimension two $(\mathbb{P}(R_1))$, so $\bigsqcup N_\mathbb{Q}(G/H_i)$ consists of one copy of $\mathbb{Q}^2$, two copies of $\mathbb{Q}^1$, and one zero-dimensional vector space.
They are shown in Figure \ref{QA4}. The vertical line corresponds to the orbit $R_1$ and the slanted line to $\mathbb{P}(GL_2)$.
\begin{figure}[!h]
\begin{tikzpicture}
\draw[fill] (0,0) circle(.05);
\draw[fill,gray!25] (-2,1)--(-2,-3)--(-6,-3)--(-6,1);

\draw[thick] (-.5,.5)--(-2.5,2.5);
\draw[very thick] (0,-.75)--(0,-3);
\end{tikzpicture}
\caption{$\bigsqcup N_\mathbb{Q}(G/H_i)$}
\label{QA4}
\end{figure}

Now we consider the valuation cone $\mathcal{V}$ of each orbit.
We have already seen that the valuation cone of ${\operatorname{Gl}}_2$ is given by ${\left\lbrace {(\alpha_1,\alpha_2) \mid \alpha_1 \geq \alpha_2} \right\rbrace}$ and the valuation cone of $\mathbb{P}(R_1)$ is necessarily trivial.
The orbit $R_1$ is a spherical homogeneous space isomorphic to $V(x_{11}x_{22} - x_{12}x_{21}) \subset \mathbb{A}^4 \setminus {\left\lbrace {0} \right\rbrace}$.
The $B$ semi-invariant rational functions $k(R_1)^{(B)}$ are just the powers of $f_2$ since $f_1$ vanishes along $R_1$.
We can define valuations which consider the degree of a rational function similarly to Example \ref{EX}, so the valuation cone $\mathcal{V}(R_1)$ is a copy of $\mathbb{Q}$.
Alternatively, $\mathcal{V}(R_1)$ is the image of $\mathcal{V}({\operatorname{Gl}}_2)$ under the projection map $N_\mathbb{Q}({\operatorname{Gl}}_2) \rightarrow {\operatorname{Hom}}{(\sigma_{1,0}^\perp \cap M(G/H),\mathbb{Q})}$.
This map is defined by taking a product $f_1^{\alpha_1}f_2^{\alpha_2} \in k({\operatorname{Gl}}_2)^{(B)}/k^*$ to $f_2^{\alpha_2} \in k(R_1)^{(B)}/k^*$.
The image of $\mathcal{V}({\operatorname{Gl}}_2)$ under the resulting map $N_\mathbb{Q}({\operatorname{Gl}}_2) \rightarrow N_\mathbb{Q}(R_1)$ is all of $N_\mathbb{Q}(R_1)$, so $\mathcal{V}(R_1) = N_\mathbb{Q}(R_1)$.

Finally, the orbit $\mathbb{P}({\operatorname{Gl}}_2)$ is $D(y_{11}y_{22} - y_{12}y_{21}) \subset \mathbb{P}^3$.
The $B$ semi-invariant rational functions $k(\mathbb{P}({\operatorname{Gl}}_2))^{(B)}$ are spanned by $(y_{11}y_{22} - y_{12}y_{21})/y_{22}^2$ since they must have the same degree in the numerator and denominator.
Since $\nu(f_1) \geq \nu(f_2)$ for any $G$-invariant valuation $\nu$ on $M({\operatorname{Gl}}_2)$, we have $\nu((y_{11}y_{22} - y_{12}y_{21})/y_{22}^2) \geq 0$ for any $G$-invariant valuation $\nu$ on $M(\mathbb{P}({\operatorname{Gl}}_2))$.
Thus, $\mathcal{V}(\mathbb{P}({\operatorname{Gl}}_2))$ is a ray in $N_\mathbb{Q}(\mathbb{P}({\operatorname{Gl}}_2))$.
The union of the valuation cones and their gluing is illustrated in Figure \ref{tropA4}.
\begin{figure}[!h]
\begin{tikzpicture}
\draw[fill] (0,0) circle(.05);
\draw[fill] (-1.5,1.5) circle(.05);
\draw[fill,gray!25] (-2,1)--(-2,-3)--(-6,-3);

\draw[thick] (-6,-3)--(-2,1);
\draw[thick] (-.5,.5)--(-1.5,1.5);
\draw[very thick] (0,-.75)--(0,-3);

\draw[thick,->] (1.5,-1)--(3.5,-1);

\draw[fill,gray!25] (7,1)--(8,0)--(8,-3)--(5,-1);
\draw[fill] (8,0) circle(.05);
\draw[fill] (7,1) circle(.05);
\draw[thick] (7,1)--(5,-1);
\draw[thick] (7,1)--(8,0);
\draw[very thick] (8,-3)--(8,0);
\end{tikzpicture}
\caption{$\bigsqcup \mathcal{V}_i$ and the tropicalization of ${\operatorname{Bl}}_0(\mathbb{A}^2)$}
\label{tropA4}
\end{figure}

In Table \ref{coltrops} we exhibit several other embedddings of ${\operatorname{Gl}}_2$ along with their associated colored cones and colored tropicalizations.

\begin{table}[h]
\begin{tabular}{| >{\centering\arraybackslash} m{4cm} >{\centering\arraybackslash} m{4cm} >{\centering\arraybackslash} m{4cm} |}
\hline
Variety & Colored Fan & Colored Tropicalization \\
\hline
\raisebox{1.3cm}{${\operatorname{Gl}}_2$} \rule{0pt}{2.7cm} & \begin{tikzpicture}[scale = .5]
\draw[fill] (0,0) circle(.1);

\draw[->,dotted] (0,0)--(2,0);
\draw[->,dotted] (0,0)--(-2,0);
\draw[->,dotted] (0,0)--(0,2);
\draw[->,dotted] (0,0)--(0,-2);

\end{tikzpicture} &
\begin{tikzpicture}
\draw[fill,gray!25] (0,0)--(1,1)--(1,-1);
\draw[thick] (0,0)--(1,1);
\end{tikzpicture} \\

\raisebox{1.3cm}{$\mathbb{A}^4 \setminus {\left\lbrace {0} \right\rbrace}$} \rule{0pt}{2.7cm} & 
\begin{tikzpicture}[scale = .5]
\draw[fill] (0,0) circle(.1);
\draw[very thick,->] (0,0)--(2,0);

\draw[->,dotted] (0,0)--(2,0);
\draw[->,dotted] (0,0)--(-2,0);
\draw[->,dotted] (0,0)--(0,2);
\draw[->,dotted] (0,0)--(0,-2);

\end{tikzpicture}
& 
\begin{tikzpicture}
\draw[fill,gray!25] (0,0)--(1,1)--(1,-1);
\draw[thick] (0,0)--(1,1);
\draw[very thick] (1,1)--(1,-1);

\draw[fill = white] (1,1) circle(.1);
\end{tikzpicture} \\

\raisebox{1.3cm}{$\mathbb{A}^4$} \rule{0pt}{2.7cm} 
& 
\begin{tikzpicture}[scale = .5]
\draw[fill,red!50] (-2,2)--(0,0)--(2,0)--(2,2);
\draw[very thick,red,->] (0,0)--(-2,2);
\draw[red,fill=white] (-1,1) circle(.2);
\draw[red, fill=red] (-1,1) circle(.1);
\draw[fill] (0,0) circle(.1);

\draw[very thick,->] (0,0)--(2,0);
\draw[->,dotted] (0,0)--(-2,0);
\draw[->,dotted] (0,0)--(0,2);
\draw[->,dotted] (0,0)--(0,-2);

\end{tikzpicture}
& 
\begin{tikzpicture}
\draw[fill,gray!25] (0,0)--(1,1)--(1,-1);
\draw[thick] (0,0)--(1,1);
\draw[very thick] (1,1)--(1,-1);

\draw[red,fill = white] (1,1) circle(.1);
\draw[red,fill = red] (1,1) circle(.05);
\end{tikzpicture} \\

\raisebox{1.3cm}{${\operatorname{Bl}}_0(\mathbb{A}^4)$} \rule{0pt}{2.7cm} 
& 
\begin{tikzpicture}[scale = .5]
\draw[fill,gray!25] (0,0)--(2,0)--(2,2);

\draw[fill] (0,0) circle(.1);

\draw[very thick,->] (0,0)--(2,2);
\draw[very thick,->] (0,0)--(2,0);
\draw[->,dotted] (0,0)--(2,0);
\draw[->,dotted] (0,0)--(-2,0);
\draw[->,dotted] (0,0)--(0,2);
\draw[->,dotted] (0,0)--(0,-2);

\end{tikzpicture}
& 
\begin{tikzpicture}
\draw[fill,gray!25] (0,0)--(.5,.5)--(1,0)--(1,-1);
\draw[very thick] (1,0)--(1,-1);
\draw[thick] (0,0)--(.5,.5);
\draw[thick] (.5,.5)--(1,0);

\draw[fill] (.5,.5) circle(.05);
\draw[fill] (1,0) circle(.05);
\end{tikzpicture} \\

\raisebox{1.3cm}{$\mathbb{P}^4 \setminus {\left\lbrace {0} \right\rbrace}$} \rule{0pt}{2.7cm} 
& 
\begin{tikzpicture}[scale = .5]
\draw[fill,gray!25] (0,0)--(2,0)--(2,-2)--(-2,-2);

\draw[fill] (0,0) circle(.1);

\draw[very thick,->] (0,0)--(-2,-2);
\draw[very thick,->] (0,0)--(2,0);
\draw[->,dotted] (0,0)--(2,0);
\draw[->,dotted] (0,0)--(-2,0);
\draw[->,dotted] (0,0)--(0,2);
\draw[->,dotted] (0,0)--(0,-2);

\end{tikzpicture}
& 
\begin{tikzpicture}
\draw[fill,gray!25] (0,0)--(1,1)--(1,-1);
\draw[thick] (0,0)--(1,1);
\draw[very thick] (1,1)--(1,-1);
\draw[thick] (0,0)--(1,-1);

\draw[fill = white] (1,1) circle(.1);
\draw[fill] (0,0) circle(.05);
\draw[fill] (1,-1) circle(.05);
\end{tikzpicture} \\

\raisebox{1.3cm}{$\mathbb{P}^4$} \rule{0pt}{2.7cm} 
& 
\begin{tikzpicture}[scale = .5]
\draw[fill,red!50] (-2,2)--(0,0)--(2,0)--(2,2);
\draw[very thick,red,->] (0,0)--(-2,2);
\draw[red,fill=white] (-1,1) circle(.2);
\draw[red, fill=red] (-1,1) circle(.1);
\draw[fill,gray!25] (0,0)--(2,0)--(2,-2)--(-2,-2);

\draw[fill] (0,0) circle(.1);

\draw[very thick,->] (0,0)--(-2,-2);
\draw[very thick,->] (0,0)--(2,0);
\draw[->,dotted] (0,0)--(-2,0);
\draw[->,dotted] (0,0)--(0,2);
\draw[->,dotted] (0,0)--(0,-2);

\end{tikzpicture}
& 
\begin{tikzpicture}
\draw[fill,gray!25] (0,0)--(1,1)--(1,-1);
\draw[thick] (0,0)--(1,1);
\draw[very thick] (1,1)--(1,-1);
\draw[thick] (0,0)--(1,-1);

\draw[red,fill = white] (1,1) circle(.1);
\draw[fill] (0,0) circle(.05);
\draw[fill] (1,-1) circle(.05);
\draw[red,fill = red] (1,1) circle(.05);
\end{tikzpicture} \\

\raisebox{1.3cm}{${\operatorname{Bl}}_0(\mathbb{P}^4)$} \rule{0pt}{2.7cm}
&
\begin{tikzpicture}[scale = .5]
\draw[fill,gray!25] (0,0)--(2,0)--(2,2);
\draw[fill,gray!25] (0,0)--(2,0)--(2,-2)--(-2,-2);

\draw[fill] (0,0) circle(.1);

\draw[very thick,->] (0,0)--(2,2);
\draw[very thick,->] (0,0)--(-2,-2);
\draw[very thick,->] (0,0)--(2,0);
\draw[->,dotted] (0,0)--(2,0);
\draw[->,dotted] (0,0)--(-2,0);
\draw[->,dotted] (0,0)--(0,2);
\draw[->,dotted] (0,0)--(0,-2);

\end{tikzpicture}
& 
\begin{tikzpicture}
\draw[fill,gray!25] (0,0)--(.5,.5)--(1,0)--(1,-1);
\draw[very thick] (1,0)--(1,-1);
\draw[thick] (0,0)--(.5,.5);
\draw[thick] (.5,.5)--(1,0);
\draw[thick] (0,0)--(1,-1);

\draw[fill] (.5,.5) circle(.05);
\draw[fill] (1,0) circle(.05);
\draw[fill] (0,0) circle(.05);
\draw[fill] (1,-1) circle(.05);
\end{tikzpicture} \\
\hline
\end{tabular}
\caption{Colored fans and colored tropicalizations of ${\operatorname{Gl}}_2$-embeddings}
\label{coltrops}
\end{table}
\end{example}

\clearpage
\begin{bibdiv}
\begin{biblist}

\bib{Br}{article}{
   author={Brion, Michel},
   title={Sur la g\'eom\'etrie des vari\'et\'es sph\'eriques},
   language={French},
   journal={Comment. Math. Helv.},
   volume={66},
   date={1991},
   number={2},
   pages={237--262},
   issn={0010-2571},
   
   
}

\bib{CLS}{book}{
   author={Cox, David A.},
   author={Little, John B.},
   author={Schenck, Henry K.},
   title={Toric varieties},
   series={Graduate Studies in Mathematics},
   volume={124},
   publisher={American Mathematical Society, Providence, RI},
   date={2011},
   pages={xxiv+841},
   isbn={978-0-8218-4819-7},
   
   
}

\bib{Ka}{article}{
   author={Kajiwara, Takeshi},
   title={Tropical toric geometry},
   conference={
      title={Toric topology},
   },
   book={
      series={Contemp. Math.},
      volume={460},
      publisher={Amer. Math. Soc., Providence, RI},
   },
   date={2008},
   pages={197--207},
   
   
}

\bib{Kn}{article}{
   author={Knop, Friedrich},
   title={The Luna-Vust theory of spherical embeddings},
   conference={
      title={Proceedings of the Hyderabad Conference on Algebraic Groups
      (Hyderabad, 1989)},
   },
   book={
      publisher={Manoj Prakashan, Madras},
   },
   date={1991},
   pages={225--249},
   
}

\bib{LV}{article}{
   author={Luna, D.},
   author={Vust, Th.},
   title={Plongements d'espaces homog\`enes},
   language={French},
   journal={Comment. Math. Helv.},
   volume={58},
   date={1983},
   number={2},
   pages={186--245},
   issn={0010-2571},
   
   
}

\bib{MS}{book}{
   author={Maclagan, Diane},
   author={Sturmfels, Bernd},
   title={Introduction to tropical geometry},
   series={Graduate Studies in Mathematics},
   volume={161},
   publisher={American Mathematical Society, Providence, RI},
   date={2015},
   pages={xii+363},
   isbn={978-0-8218-5198-2},
   
}

\bib{Pas}{article}{
   author={Pasquier, Boris},
   title={Introduction to spherical varieties and description of special classes of spherical varieties},
   journal={lecture notes available at http://www.math.univ-montp2.fr/~pasquier/KIAS.pdf},
   volume={},
   date={2009},
   number={},
   pages={},
   issn={},
   
   
}

\bib{Pay}{article}{
   author={Payne, Sam},
   title={Analytification is the limit of all tropicalizations},
   journal={Math. Res. Lett.},
   volume={16},
   date={2009},
   number={3},
   pages={543--556},
   issn={1073-2780},
   
   
}

\bib{Pe}{article}{
   author={Perrin, Nicolas},
   title={On the geometry of spherical varieties},
   journal={Transform. Groups},
   volume={19},
   date={2014},
   number={1},
   pages={171--223},
   issn={1083-4362},
   
   
}

\bib{Te}{article}{
   author={Tevelev, Jenia},
   title={Compactifications of subvarieties of tori},
   journal={Amer. J. Math.},
   volume={129},
   date={2007},
   number={4},
   pages={1087--1104},
   issn={0002-9327},
   
   
}

\bib{Vo}{thesis}{
   author = {Vogiannou, Tassos},
    title = {Spherical Tropicalization},
    institution = {University of Massachusetts Amherst},
    type = {thesis},
journal = {ArXiv e-prints},
archivePrefix = {"arXiv"},
   eprint = {arXiv:1511.02203},
 primaryClass = {"math.AG"},
 keywords = {Mathematics - Algebraic Geometry},
     year = {2015},
    month = {nov},
   adsurl = {http://adsabs.harvard.edu/abs/2015arXiv151102203V},
  adsnote = {Provided by the SAO/NASA Astrophysics Data System}
}

\end{biblist}
\end{bibdiv}

\end{document}

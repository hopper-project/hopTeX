\documentclass[10pt, a4paper,twoside,reqno]{amsart}
\usepackage{a4,amssymb,amsfonts,verbatim,pstricks,pst-plot,xypic,times}
\usepackage{yfonts}

\usepackage{mltex}
\usepackage{amsmath}

\usepackage[english]{babel}
\usepackage{amssymb,latexsym}
\usepackage{amssymb}
\usepackage{latexsym}

 
 

       
       

 
 

 

 

 

 
 
 

 
 
 

\newtheorem{example}{Examples}[section]
\newtheorem{thm}{Theorem}[section]
\newtheorem{lemma}[thm]{Lemma}
\newtheorem{prop}[thm]{Proposition}
\newtheorem{cor}[thm]{Corollary}
\newtheorem{remark}[thm]{Remark}
\newtheorem{remarks}[thm]{Remarks}
\newtheorem{definition}[thm]{Definition}
\newtheorem{notation}[thm]{Notation}
\newtheorem{exabout:ample}[thm]{Example}
\newtheorem{conjecture}[thm]{Conjecture}
\newtheorem{assumption}[thm]{Assumption}

\let\La\Lambda

       
\DeclareMathAlphabet{\doba}{U}{msb}{m}{n}         

         

        

\let{\frac{\partial \partial}{\partial \let}}\na\nabla     
\let{\langle\,\langle 
\let{\right\rangle}\rangle 
\let\ti\tilde
\let\witi\widetilde

  

\setlength{\parindent}{0cm}

\frenchspacing
\parindent0cm
\parskip=.5\baselineskip
\sloppy

\begin{document}

\title[]{Complex and Lagrangian surfaces of the complex projective plane via K\"ahlerian Killing Spin$^c$ spinors}

\subjclass[2010]{53C27,\, 53C40, 53D12, 53C25}

\keywords{Spin$^c$ structures, K\"ahlerian Killing Spin$^c$ spinors, isometric immersions, complex and Lagrangian surfaces, Dirac operator}

\author[RN]{Roger Nakad}
\author[JR]{Julien Roth}

\address[R. Nakad]{Notre Dame University-Louaiz\'e, Faculty of Natural and Applied Sciences, Department of Mathematics and Statistics, P.O. Box 72, Zouk Mikael, Lebanon,}
\email{rnakad@ndu.edu.lb}

\address[J. Roth]{LAMA, Universit\'e Paris-Est Marne-la-Vall\'ee, Cit\'e Descartes, Champs sur Marne, 77454 Marne-la-Vall\'ee  cedex 2, France}
\email{julien.roth@u-pem.fr}

\maketitle
\begin{abstract} The complex projective space $\mathbb C P^2$ of complex dimension $2$ has a ${{\mathop{\rm Spin}^c}}$ structure  carrying  K\"ahlerian Killing spinors. The restriction of one of these  K\"ahlerian Killing spinors to a Lagrangian or complex surface $M^2$ characterizes  the isometric immersion of $M$ into $\mathbb C P^2$. 
\end{abstract} 

\section{Introduction}

A classical problem in Riemannian geometry is to know when a
Riemannian manifold $(M^n,g)$ can be
isometrically immersed into a fixed Riemannian manifold $(\bar{M}^{n+p},\bar{g})$. The case of space forms ${{\mathbb R}}^{n+1}$, ${{\mathbb S}}^{n+1}$ and
${{\mathbb H}}^{n+1}$ is well-known. In fact, the Gauss, Codazzi and Ricci equations
are necessary and sufficient conditions. In other ambient spaces, the Gauss, Codazzi and Ricci equations
are necessary but not sufficient in general. Some additional conditions may be required like for the case of complex space forms, products, warped products or 3-dimensional homogeneous space for instance (see \cite{Da,Da2,Kow,LO,PT,Roth3}).

\indent In low dimensions, especially for surfaces, another necessary
and sufficient condition is now well-known, namely the existence of
a special spinor field called {\it generalized Killing spinor field}
(see \cite{Fr,Mo,La,LR}). These results are the geometrically invariant versions of  previous work on the spinorial Weierstrass representation by  R. Kusner and N. Schmidt, B. Konoplechenko, I. Taimanov and many others (see \cite{KS, Ko, Ta}). This representation was expressed by T.
Friedrich \cite{Fr}) for surfaces in ${{\mathbb R}}^3$ and then extended to
other $3$-dimensonal (pseudo-)Riemannian manifolds
(\cite{Mo,Roth3,Roth4,LR2}) as well as for hypersurfaces of 4-dimensional space forms and products \cite{LR} or hypersurfaces of 2-dimensional complex space forms by the mean of ${\rm Spin}^c$ spinors \cite{NR}.

\indent More precisely, the restriction
$\varphi$ of a parallel spinor field on ${{\mathbb R}}^{n+1}$ to an oriented
Riemannian hypersurface $M^{n}$ is a solution of a generalized
Killing equation 
\begin{equation}\label{killing} \nabla_X\varphi=-\frac{1}{2}A(X))\varphi, 
\end{equation}
where
$``\cdot"$ and $\nabla$ are respectively the Clifford
multiplication and the spin connection on $M^{n}$, and $A$ is the
Weingarten tensor of the immersion. Conversely, T. Friedrich proved in
\cite{Fr} that, in the two dimensional case, if there exists a
generalized Killing spinor field satisfying Equation
\eqref{killing}, where $A$ is an arbitrary field of symmetric
endomorphisms of $TM$, then $A$ satisfies the Codazzi and
Gauss equations of hypersurface theory and is consequently the
Weingarten tensor of a local isometric immersion of $M$ into ${{\mathbb R}}^3$.
Moreover, in this case, the solution $\varphi$ of the generalized
Killing equation is equivalently a solution of the Dirac equation
\begin{equation}\label{dirac} 
D\varphi=H\varphi, 
\end{equation}
where $|\varphi|$ is constant and $H$ is a real-valued function (which is the mean curvature of the immersion in ${{\mathbb R}}^3$).

More recently, this approach was adapted by the second author, P. Bayard and M.A. Lawn in codimension two, namely, for surfaces in Riemannian 4-dimensional real space forms \cite{BLR}, and then generalized in the pseudo-Riemannian setting \cite{bay,BP} as well as for 4-dimensional products \cite{Roth}. As pointed out in \cite{RR}, this approach coincides with the Weierstrass type representation for surfaces in ${{\mathbb R}}^4$ introduced by Konopelchenko and Taimanov \cite{Ko,Ta2}.

The aim of the present article is to provide an analogue for the complex projective space ${{\mathbb C}} P^2$. The key point is that contrary to the case of hypersurfaces of ${{\mathbb C}} P^2$, which is considerered in \cite{NR}, the use of ${\rm Spin}^c$ parallel spinors is not sufficient. Indeed, for both canonical or anti-canonical ${\rm Spin}^c$ structures, the parallel spinors are always in the positive half-part of the spinor bundle. But, as proved in \cite{BLR} or \cite{Roth}, a spinor with non-vanishing positive and negative parts is required to get the integrability condition to get an immersion in the desired target space. For this reason, ${\rm Spin}^c$ parallel spinors are not adapted to our problem. Thus, we make use of real K\"ahlerian Killing spinors. Therefore, our argument holds also for the complex projective space and not for the complex hyperbolic space, since ${{\mathbb C}}{{\mathbb H}}^2$ does not carry a real or  imaginary K\"ahlerian Killing spinor.

We will focus on the case of complex and Lagrangian immersions into ${{\mathbb C}} P^2$. These two cases and especially the Lagrangian case are of particular interest in the study of surfaces in ${{\mathbb C}} P^2$ (see \cite{HR1,HR2,Ur} and references therein for instance). Namely, we prove the two following results.

The first theorem gives a spinorial characterization of complex surfaces in the complex projective space ${{\mathbb C}} P^2$.
\begin{thm}\label{thm1}
Let $(M^2,g)$ be an oriented Riemannian surface and $E$ an oriented vector bundle of rank $2$ over $M$ with scalar product $\langle\cdot,\cdot\rangle_E$ and compatible connection $\nabla^E$. We denote by $\Sigma=\Sigma M\otimes\Sigma E$ the twisted spinor bundle. Let $B:TM\times TM{\longrightarrow} E$ a bilinear symmetric map, $j:TM{\longrightarrow} TM$ a complex structure on $M$ and $t:E{\longrightarrow} E$ a complex structure on $E$. Assume moreover that $t(B(X,Y))=B(X,jY)$, $\forall X\in TM$. Then, the two following statements are equivalent
\begin{enumerate}

\item There exists a ${\rm Spin}^c$  structure on $\Sigma M\otimes\Sigma E$ with $\Omega^{M +E} (e_1, e_2)=0$ and a spinor field $\varphi\in\Gamma(\Sigma M\otimes\Sigma E)$
\begin{eqnarray}\label{partspinorcomp}
\nabla_X\varphi&=&-\frac 12\eta(X)\cdot\varphi-\frac{1}{2}X\cdot\varphi+\frac{i}{2}jX\cdot\overline{\varphi},
\end{eqnarray}
such that $\varphi^+$ and $\varphi^-$ never vanish and where $\eta$ is given by
$$
\eta(X)=\sum_{j=1}^2e_j\cdot B(e_j,X).
$$
\item There exists a local isometric {\bf complex} immersion of $(M^2,g)$ into ${{\mathbb C}} P^2$ with $E$ as normal bundle and second fundamental form $B$ such the complex structure of ${{\mathbb C}} P^2$ over $M$ is given by $j$ and $t$ (in the sense of Proposition \ref{propimmersioncp2}).
\end{enumerate}
\end{thm}
The spinors   $\varphi^+$ and $\varphi^-$ denote respectiveley the positive and negative half part of $\varphi$ defined Section $2$. They are the projections of $\varphi$ on the eigensubspaces for the eigenvalues $+1$ and $-1$ of the complex volume form and $\bar \varphi$ is defined by $\bar \varphi = \varphi_+ - \varphi_-$.

The second theorem is the analogue of Theorem \ref{thm1} for Lagrangian surfaces into ${{\mathbb C}} P^2$.
\begin{thm}\label{thm2}
Let $(M^2,g)$ be an oriented Riemannian surface and $E$ an oriented vector bundle of rank $2$ over $M$ with scalar product $\langle\cdot,\cdot\rangle_E$ and compatible connection $\nabla^E$. We denote by $\Sigma=\Sigma M\otimes\Sigma E$ the twisted spinor bundle. Let $B:TM\times TM{\longrightarrow} E$ a bilinear symmetric map, $h:TM{\longrightarrow} E$ and $s: E{\longrightarrow} TM$ the dual map of $h$. Assume moreover that $h$ and $s$ are parallel, $hs=-{\mathrm{id}}_E$ and $A_{hY}X+s(B(X,Y))=0$, for all $X\in TM$, where $A_{\nu}:TM{\longrightarrow} TM$ if defined by $g(A_{\nu}X,Y)=\langle B(X,Y),\nu\rangle_E$ for all $X,Y\in TM$ and $\nu\in E$. Then, the two following statements are equivalent
\begin{enumerate}

\item There exists a ${\rm Spin}^c$ structure on $\Sigma M\otimes\Sigma E$ with $\Omega^{M +N} (e_1, e_2)=-2$ and a spinor field $\varphi\in \Gamma(\Sigma M\otimes\Sigma E)$  satisfying for all $X\in{\mathfrak{X}}(M)$
\begin{eqnarray}\label{partspinorlagr}
\nabla_X\varphi&=&-\frac 12\eta(X)\cdot\varphi-\frac{1}{2}X\cdot\varphi+\frac{i}{2}hX\cdot\overline{\varphi},
\end{eqnarray}
such that $\varphi^+$ and $\varphi^-$ never vanish and where $\eta$ is given by
$$
\eta(X)=\sum_{j=1}^2e_j\cdot B(e_j,X).
$$
\item There exists a local isometric {\bf Lagrangian} immersion of $(M^2,g)$ into ${{\mathbb C}} P^2$ with $E$ as normal bundle and second fundamental form $B$ such that over $M$ the complex structure of ${{\mathbb C}} P^2$ is given by $h$ and $s$ (in the sense of Proposition \ref{propimmersioncp2}).
\end{enumerate}
\end{thm}

Note that in the statements of both theorem $\varphi^+$ and $\varphi^-$ denotes respectiveley the positive and negative half part of $\varphi$ defined Section \label{sec22}. They are the projections of $\varphi$ on the eigensubspaces for the eigenvalues $+1$ and $-1$ of the complex volume form. 

\section{Preliminaries and Notations}
In this section, we briefly review  some basic facts about K\"ahler geometry and ${{\mathop{\rm Spin}^c}}$ structures on manifolds and their submanifolds.
For more details we refer to  \cite{am_lectures, ballman, besse, spin, montiel, HMU, hijazi-lecture, bookspin, Ba, ginoux-these, GM,ginoux-book}. 

\subsection{Spin$^c$ structures on K\"ahler-Einstein manifolds}
Let $(M^{n}, g)$ be an $n$-dimensional closed Riemannian ${{\mathop{\rm Spin}^c}}$  manifold and denote by $\Sigma M$ its complex spinor bundle, 
which has complex rank equal to $2^{[\frac{n}{2}]}$ . The bundle $\Sigma M$ is endowed with a Clifford multiplication denoted by ``$\cdot$'' and a scalar product denoted by ${\langle\,\cdot,\, \cdot\\,\rangle}$. 

Given a ${{\mathop{\rm Spin}^c}}$ structure on $(M^{n}, g)$, one can check that the
determinant line bundle $\mathrm{det}(\Sigma M)$ has a root $L$ of index $2^{[\frac
{n}{2}]-1}$. This line bundle $L$ over $M$ is called the auxiliary line bundle associated with the
${{\mathop{\rm Spin}^c}}$ structure.
From a topological point of view, a Riemannian manifold has a ${{\mathop{\rm Spin}^c}}$ structure if and only if there exists a complex line bundle $L$ (which will be the auxiliary line bundle) on $M$ such that 
$$\omega_2 (M) = [c_1(L)]_{mod\ \ 2},$$ 
where $\omega_2(M)$ is the second Steifel-Whitney class of $M$ and $c_1(L)$ is the first Chern class of $L$. 
In the particular case, when the line bundle has a square root, i.e., $\omega_2(M)= 0$, the manifold is called a ${{\mathop{\rm Spin}}}$ manifold. In this case, we denote by $\Sigma^{'}M$  the spinor bundle or the ${{\mathop{\rm Spin}}}$ bundle. It can be chosen such that the  auxiliary line bundle is trivial.

Locally, a ${{\mathop{\rm Spin}^c}}$ structure always exists. In fact, the square root of the auxiliary line bundle
$L$ and $\Sigma^{'}M$ always exist locally. But, $\Sigma M =\Sigma^{'}M \otimes L^{\frac 12} $ is defined globally. This essentially means that, while
the spinor bundle and $L^\frac{1}{2}$ may not exist globally, their tensor product (the ${{\mathop{\rm Spin}^c}}$ bundle) is
defined globally.
Thus, the connection $\nabla$ on $\SS M$ is the twisted connection of the one on the spinor bundle (induced 
by the Levi-Civita connection) and a fixed connection $A$ on $L$. 
The ${{\mathop{\rm Spin}^c}}$ Dirac operator $D$ acting on the space of sections of $\Sigma M$ 
is defined by the composition of the connection $\nabla$ with the Clifford
multiplication. In local coordinates:
$$D =\sum_{j=1}^{n} e_j \cdot \nabla_{e_j},$$
where $\{e_j\}_{j=1,\dots, n}$ is a local orthonormal basis of $TM$. 
$D$ is a first-order elliptic operator and is formally self-adjoint with respect to the $L^2$-scalar product. 

We recall that the complex volume element $\omega_{\mathbb{C}}=\i^{[\frac{n+1}{2}]} e_1\wedge \ldots \wedge e_n$ acts as the identity on the spinor bundle if $n$ is odd. If $n$ is even,  $\omega_{\mathbb C}^2=1$. Thus, under the action of the complex volume element, the spinor bundle  decomposes into the  eigenspaces $\Sigma^{\pm} M$ corresponding to the $\pm 1$ eigenspaces, the {\it positive} (resp. {\it negative}) spinors.

Every spin manifold has a trivial ${{\mathop{\rm Spin}^c}}$ structure, by
choosing the trivial line bundle with the trivial connection whose curvature $F_A$
vanishes.  Every K\"ahler manifold $(M^{2m},g,J)$ has a canonical 
${{\mathop{\rm Spin}^c}}$ structure induced by the complex structure $J$. 
 The complexified tangent bundle decomposes into
$T^\mathbb C M = T_{1,0} M\oplus T_{0,1} M,$
 the $i$-eigenbundle (resp. $(-i)$-eigenbundle) of the complex linear extension of $J$. 
For any vector field $X$, we denote by $X^{\pm}:=\frac{1}{2}(X\mp iJX)$ its component in $T_{1,0} M$, resp. $T_{0,1} M$.
The spinor bundle of the canonical ${{\mathop{\rm Spin}^c}}$ structure is defined by 
$$\Sigma M = \Lambda^{0,*} M =\overset{m}{\underset{r=0}{\oplus}} \Lambda^r (T_{0,1}^* M),$$
and its auxiliary line bundle is  $L = (K_M)^{-1}= \Lambda^m (T_{0,1}^* M)$, where $K_M=\Lambda^{m,0}M$ is the canonical bundle of $M$. 
The line bundle $L$ has a canonical holomorphic connection, whose curvature form is given by $- i\rho$, 
where $\rho$ is the Ricci form defined, for all vector fields $X$ and $Y$, by $\rho(X, Y) = \mathrm{Ric}(JX, Y)$ and $\mathrm{Ric}$ denotes the Ricci tensor.  
Similarly, one defines  the so called anti-canonical ${{\mathop{\rm Spin}^c}}$ structure, whose spinor bundle is given by 
$\Lambda^{*, 0} M =\oplus_{r=0}^m \Lambda^r (T_{1, 0}^* M)$ and the auxiliary line bundle by $K_M$. The spinor bundle of any other ${{\mathop{\rm Spin}^c}}$ structure on $M$ can be written as:
$$\Sigma M =  \Lambda^{0, *} M \otimes \mathbb L,$$
where $\mathbb L^2 = K_M\otimes L$ and  $L$ is the auxiliary line bundle associated with this ${{\mathop{\rm Spin}^c}}$ structure. 
The K\"ahler form $\Omega$, defined as $\Omega(X,Y)=g(JX,Y)$, acts on $\Sigma M$ via Clifford multiplication 
and this action is locally  given by:
$$\Omega\cdot \psi = \frac{1}{2} \sum_{j=1}^{2m} e_j\cdot Je_j\cdot\psi,$$
for all $\psi\in\Gamma(\Sigma M)$, where $\{e_1, \dots, e_{2m}\}$ is a local orthonormal basis of $\mathrm{TM}$. Under this action, the spinor bundle decomposes as follows: 
\begin{equation}\label{decomp}
\Sigma M =\overset{m}{\underset{r=0}{\oplus}} \Sigma_r M,
\end{equation}
where $\Sigma_r M$ denotes the eigenbundle to the eigenvalue $i(2r-m)$ of $\Omega$, of complex rank $\binom{m}{k}$. 
It is easy to see that $\Sigma_r M \subset \Sigma^+ M$ (resp. $\Sigma_r M \subset \Sigma^-M$) if and only if $r$ is even (resp. $r$ is odd).
Moreover, for any $X \in \Gamma(TM)$ and $\varphi \in \Gamma(\Sigma_r M)$, we have $X^+ \cdot\varphi \in \Gamma(\Sigma_{r+1}M)$ 
and $X^-\cdot \varphi \in \Gamma(\Sigma_{r-1} M)$, with the convention ${\mathrm{\Sigma}_{{-1}} \mathrm{M}}={\mathrm{\Sigma}_{{m+1}} \mathrm{M}}=M\times\{0\}$.
Thus, for any ${{\mathop{\rm Spin}^c}}$ structure, we have 
$\Sigma_r M = \Lambda^{0, r} M\otimes \Sigma_0 M.$
Hence, $(\Sigma_0M)^2 = K_M \otimes L,$ where $L$ is the auxiliary line bundle associated with the ${{\mathop{\rm Spin}^c}}$ structure. For example, when the manifold is spin, we have $(\Sigma_0M)^2 = K_M$ \cite{hitchin, kirchberg1}.  For the canonical ${{\mathop{\rm Spin}^c}}$ structure, since $L = (K_M)^{-1}$, it follows that $\Sigma_0 M$ is trivial. This yields the existence of parallel spinors (the constant functions) lying in $\Sigma_0M$, \emph{cf.} \cite{Moro1}.

On a K\"ahler manifold $(M,g,J)$  endowed with any ${{\mathop{\rm Spin}^c}}$ structure,  a spinor of the form $\varphi_r+\varphi_{r+1}\in \Gamma(\Sigma_r M\oplus \Sigma_{r+1} M)$, for some $0\leq r\leq m$, is called a {\it   K\"ahlerian Killing ${{\mathop{\rm Spin}^c}}$ spinor} if  there exists a non-zero real constant $\alpha$, such that the following equations are satisfied, for all vector fields $X$,
 \begin{equation}\label{KKSSdefinition}
\begin{cases}
  \begin{split}
   \nabla_X\varphi_r&=  \alpha \ X^-\cdot\varphi_{r+1},\\
   \nabla_X\varphi_{r+1}&=  \alpha \  X^+\cdot\varphi_{r}.\\
  \end{split}
\end{cases}
 \end{equation}
K\"ahlerian Killing spinors lying in $\Gamma(\Sigma_{m} M\oplus \Sigma_{m+1} M) = \Gamma(\Sigma_m M)$  or in  $\Gamma(\Sigma_{-1} M\oplus \Sigma_{0} M) = \Gamma(\Sigma_0 M)$ are just parallel spinors.
In \cite{HMU}, the authors gave examples of ${{\mathop{\rm Spin}^c}}$ structures on compact K\"ahler-Einstein  manifolds of positive scalar curvature, which carry  K\"ahlerian Killing ${{\mathop{\rm Spin}^c}}$ spinors lying in $\Sigma_r M\oplus \Sigma_{r+1} M$, for $r\neq \frac{m\pm1}{2}$, in contrast to the spin case, where K\"ahlerian Killing spinors may only exist for $m$ odd in the middle of the decomposition \eqref{decomp}. We briefly describe  these ${{\mathop{\rm Spin}^c}}$ structures here. If the first Chern class $c_1(K_M)$ of the canonical bundle of
the K\"ahler $M$ is a non-zero cohomology class,  the greatest number $p\in \mathbb N^*$ such that
$\frac 1p c_1 (K_M) \in H^2 (M, \mathbb Z),$
is called the {\it Maslov index} of the K\"ahler manifold. One can thus consider a $p$-th root of the canonical bundle $K_M$, \emph{i.e.}
 a complex line bundle $\mathcal L$, such that $\mathcal L^ p = K_M$. In \cite{HMU}, O.~Hijazi, S.~Montiel and F.~Urbano proved the following:
\begin{thm}[Theorem~14, \cite{HMU}]
Let $M$ be a $2m$-dimensional K\"ahler-Einstein compact manifold with scalar
curvature $4m(m+1)$ and index $p \in  \mathbb N^*$. For each $0 \leq r \leq m+1$, there exists on $M$ a ${{\mathop{\rm Spin}^c}}$ structure with auxiliary line bundle given by $\mathcal L^q$, where  $q = \frac{p}{m+1} (2r-m-1) \in \mathbb Z$, and carrying a K\"ahlerian Killing spinor $\psi_{r-1} + \psi_r \in \Gamma(\Sigma_{r-1} M \oplus \Sigma_{r} M)$, \emph{i.e.}  for all $X \in \Gamma(TM)$, it satisfies the first order system
\begin{equation*}
\begin{cases}
\begin{split}
\nabla_X \psi_{r} = - X^+ \cdot \psi_{r-1},\\
\nabla_X \psi_{r-1}  = - X^-  \cdot \psi_{r},
\end{split}
\end{cases}
\end{equation*}
\end{thm} 
For example, if $M$ is the complex projective space $\mathbb C P^m$ of complex dimension $m$, then $p = m+1$ and $\mathcal L$ is just the tautological line bundle. We fix $0 \leq r \leq m+1$ and we endow $\mathbb C P^m$ with the ${{\mathop{\rm Spin}^c}}$ structure whose auxiliary line bundle  is given by $\mathcal L^q$ where $q = \frac{p}{m+1} (2r-m-1) = 2r-m-1 \in \mathbb Z$. For this ${{\mathop{\rm Spin}^c}}$ structure, the space of  K\"ahlerian Killing spinors   in $\Gamma(\Sigma_{r-1}M \oplus \Sigma_rM)$ has dimension $\binom{m+1}{r}$.  In this example, for $r=0$ (resp. $r = m+1$), we get the canonical (resp. anticanonical) ${{\mathop{\rm Spin}^c}}$ structure for which K\"ahlerian Killing spinors are just parallel spinors. 

\subsection{Submanifolds of Spin$^c$ manifolds.} 
\label{sec22}
Let $(M^2,g)$ be an oriented Riemannian surface, with a given ${{\mathop{\rm Spin}^c}}$ structure, and $E$ an oriented ${{\mathop{\rm Spin}^c}}$ vector bundle of rank 2 on $M$ with an Hermitian product $\langle\cdot,\cdot\rangle_E$ and a compatible connection $\nabla^E$. We consider the spinor bundle $\Sigma$ over $M$ twisted by $E$ and defined by
$\Sigma=\Sigma M\otimes\Sigma E,$
where $\Sigma M$ and $\Sigma E$ are the spinor bundles of $M$ and $E$ respectively. We endow $\Sigma$ with the spinorial connection $\nabla$ defined by
$$\nabla=\nabla^{\Sigma M}\otimes Id_{\Sigma E}+Id_{\Sigma M}\otimes\nabla^{\Sigma E},$$
where $\nabla^{\Sigma M}$ and $\nabla^{\Sigma E}$ are respectively the spinorial connections on $\Sigma M$ and $\Sigma E$.
We also define the Clifford product $\cdot$ by
$$\left\{\begin{array}{l} X\cdot\varphi=(X\cdot_{_M}\alpha)\otimes\overline\sigma\quad\text{if}\ X\in\Gamma(TM)\\ 
X\cdot\varphi=\alpha\otimes(X\cdot_{_E}\sigma)\quad\text{if}\ X\in\Gamma(E)
\end{array}
\right.$$
for all $\varphi=\alpha\otimes\sigma\in\Sigma M\otimes\Sigma E,$ where $\cdot_{_M}$ and $\cdot_{_E}$ denote the Clifford products on $\Sigma M$ and on $\Sigma E$ respectively and where $\overline{\sigma}=\sigma^+-\sigma^-$ for the natural decomposition of $\Sigma E=\Sigma^+E\oplus\Sigma^-E$. Here, $\Sigma^+E$ and $\Sigma^-E$ are the eigensubbundles (for the eigenvalue $1$ and $-1$) of $\Sigma E$ for the action of the normal volume element $\omega_{\perp}=i\xi_1\cdot_{E}\xi_2$, where $\{\xi_1,\xi_2\}$ is a local orthonormal frame of $E$. Note that $\Sigma^+M$ and $\Sigma^-M$ are defined similarly by for the tangent volume element $\omega=ie_1\cdot_{M} e_2$.  If $\{e_1,e_2\}$ is an orthonormal basis of $TM$, we define the twisted Dirac operator $D$ on $\Gamma(\Sigma)$ by
$$D\varphi=e_1\cdot\nabla_{e_1}\varphi+e_2\cdot\nabla_{e_2}\varphi.$$
We note that $\Sigma$ is also naturally equipped  with a hermitian scalar product $\langle.,.\rangle$ which is compatible with the connection $\nabla$, and thus also with a compatible real scalar product ${\mathrm{Re}} \langle.,.\rangle.$ We also note that the Clifford product $\cdot$ of vectors belonging to $TM\oplus E$ is antihermitian with respect to this hermitian product. Finally, we stress that the four subbundles $\Sigma^{\pm\pm}:=\Sigma^{\pm}M\otimes\Sigma^{\pm}E$ are orthogonal with respect to the hermitian product. We will also consider $\Sigma^+=\Sigma^{++}\oplus\Sigma^{--}$ and  $\Sigma^-=\Sigma^{+-}\oplus\Sigma^{-+}$.Throughout the paper we will assume that the hermitian product is ${{\mathbb C}}-$linear w.r.t. the first entry, and ${{\mathbb C}}-$antilinear w.r.t. the second entry.

Let $({\widetilde M}^4, \widetilde g)$  be a Riemannian ${{\mathop{\rm Spin}^c}}$ manifold and $(M^2, g)$ an oriented surface isometrically immersed into ${\widetilde M}$. We denote by $NM$ the normal bundle of $M$ into ${\widetilde M}$. As $M$ is an oriented surface, it is also ${{\mathop{\rm Spin}^c}}$. We denote by $i\widetilde F$  (resp. $iF$) the curvature $2$-form of the auxiliary line bundle $L^{\widetilde M}$  (resp. $L$) associated with the ${{\mathop{\rm Spin}^c}}$ structure on ${\widetilde M}$ (resp. $M$). Since the manifolds $M$ and ${\widetilde M}$ are  ${{\mathop{\rm Spin}^c}}$, there exists a ${{\mathop{\rm Spin}^c}}$ structure on the bundle $NM$ whose   auxiliary line bundle $L_N$  is given
 by $L_N := ({L})^{-1} \otimes {L^{\widetilde M}}_{|_M}.$ We denote by $\Sigma N$ the ${{\mathop{\rm Spin}^c}}$ bundle of $N M$ and let $\Sigma=\Sigma M\otimes \Sigma N$ the spinor bundle over $M$ twisted by $NM$ constructed as above with the associated connection and Clifford multiplication. It is a classical fact that the spinor bundle of ${\widetilde M}$ over $M$, $\Sigma{\widetilde M}_{|M}$ identifies with $\Sigma$. Moreover the connections on each bundle are linked by the so-called spinorial Gauss formula: for any $\varphi\in\Gamma(\Sigma)$ and any $X\in TM$,
\begin{eqnarray}\label{gaussspin}
\widetilde{\nabla}_X\varphi=\nabla_X\varphi+\frac{1}{2}\sum_{j=1,2}e_j\cdot II(X,e_j) \cdot\varphi
\end{eqnarray}
where $II$ is the second fundamental form, $\widetilde{\nabla}$ is the spinorial connection of $\Sigma {\widetilde M}$ and $\nabla$ is the spinoral connection of $\Sigma$ defined as above and $\{e_1,e_2\}$ is a local orthonormal frame of $TM$. Here $\cdot$ is the Clifford product on $\Sigma{\widetilde M}$ which identifies with the Clifford mulitplication on $\Sigma$.

\section{Immersed surfaces into the complex projective space}\label{sec3}

In this section, we will give the basic facts about immersed surfaces in the complex projective plane and in particular derive a sequence of necessary and sufficent conditions for the existence of such immersions.
\subsection{Compatibility equations}
 Let $(M^2,g)$ be a Riemannian surface isometrically immersed itn the $2$-dimensional complex projective space  of constant holomorphic sectional curvature $4c\,\rangle}0$. We denote by $\nabla$ the Levi-Civita connection of $(M^2,g)$, $\widetilde{g}$  the Fubini-Study metric of ${{\mathbb C}} P^2(4c)$ and $\widetilde{\nabla}$ its Levi-Civita connection. First of all, we recall that the curvature tensor of ${{\mathbb C}} P^2(4c)$ is given by
\begin{eqnarray}\label{curv}
\widetilde{R}(X,Y,Z,W)&=&c\Bigg[\left\langle X,W\right\rangle\left\langle Y,Z\right\rangle -\left\langle X,Z\right\rangle \left\langle Y,W\right\rangle+\left\langle JX,W\right\rangle\left\langle JY,Z \right\rangle\nonumber\\ &&\ \ \ \ \ \ -\left\langle JX,Z\right\rangle \left\langle JY,W\right\rangle
+2\left\langle X,JY\right\rangle \left\langle JZ,W\right\rangle\Bigg].
\end{eqnarray}
The complex structure $J$ induces the existence of the following four operators $$j:TM{\longrightarrow} TM,\ h:TM{\longrightarrow} NM,\ s:NM{\longrightarrow} TM\ \text{and}\ t:NM{\longrightarrow} NM$$ defined for any $X\in TM$ and $\xi\in NM$ by
\begin{eqnarray}\label{relationfhst}
JX=jX+hX\quad\text{and}\quad
J\xi=s\xi+t\xi.
\end{eqnarray} 
From the fact that $J^2=-\mathrm{Id}$ and $J$ is antisymmetric, we get that $j$ and $t$ are antisymmetric and we have the following relations between these four operators 
\begin{align}
&j^2X=-X-shX,& \label{relation1.1}\\
&t^2\xi=-\xi-hs\xi,& \label{relation1.2}\\
\label{relation1.3}
&js\xi+st\xi=0,&\\
\label{relation1.4}
&hjX+thX=0,&\\
\label{relation1.5}
&\widetilde{g}(hX,\xi)=-\widetilde{g}(X,s\xi),&
\end{align}
for any $X\in\Gamma(TM)$ and $\xi\in\Gamma(NM)$. Moreover, from the fact that $J$ is parallel, we have
\begin{align}
&(\nabla_Xj)Y=A_{hY}X+s(B(X,Y)),& \label{relation2.1}\\
&\nabla^{\perp}_X(hY)-h(\nabla_XY)=t(B(X,Y))-B(X,jY),& \label{relation2.2}\\
&\nabla^{\perp}(t\xi)-t(\nabla^{\perp}_X\xi)=-B(s\xi,X)-h(A_{\xi}X),& \label{relation2.3}\\
&\nabla_X(s\xi)-s(\nabla^{\perp}_X\xi)=-j(A_{\xi}X)+A_{t\xi}X,&\label{relation2.4}
\end{align}
where $B:TM\times TM{\longrightarrow} NM$ is the second fundamental form and for any $\xi\in TM$, $A_{\xi}$ is the Weingarten operator associated with $\xi$ and defined by $\widetilde{g}(A_{\xi}X,Y)=\widetilde{g}(B(X,Y),\xi)$ for any vectors $X,Y$ tangent to $M$. Finally, from \eqref{curv}, we deduce that the Gauss, Codazzi and Ricci equations are respectively given by
\begin{eqnarray}\label{gauss}
R(X,Y)Z&=&c\bigg[\left\langle Y,Z \right\rangle X-\left\langle X,Z\right\rangle Y+\left\langle jY,Z\right\rangle jX-\left\langle jX,Z\right\rangle jY\nonumber \\
&& \ \ \ \ \ \ \ \ +2\left\langle X,jY\right\rangle jZ\bigg]+A_{B(Y,Z)}X-A_{B(X,Z)}Y,\nonumber
\end{eqnarray}
\begin{eqnarray}\label{codazzi}
(\nabla_XB)(Y,Z)-(\nabla_YB)(X,Z)&=&c\bigg[ \left\langle jY,Z\right\rangle hX-\left\langle jX,Z \right\rangle hY+2\left\langle jX,Y\right\rangle hZ\bigg],\nonumber
\end{eqnarray}
\begin{eqnarray}\label{ricci}
R^{\perp}(X,Y)\xi&=&c\bigg[ \left\langle hY,\xi\right\rangle hX-\left\langle hX,\xi\right\rangle hY+2\left\langle jX,Y\right\rangle t\xi\bigg] \nonumber \\ &&+B(A_{\xi}Y,X)-B(A_{\xi}X,Y).\nonumber
\end{eqnarray}
In the local orthonormal frames $\{e_1,e_2\}$ and $\{\nu_1,\nu_2 \}$, these equations become (for $c=1$):
\begin{equation}\label{Gaussgen}K_M=1+<B_{22},B_{11}>-|B_{12}|^2+3<je_1,e_2>^2,\end{equation}
\begin{equation}\label{Riccigen}K_N=-<[S_{\nu_1},S_{\nu_2}](e_1),e_2>+h_{21}h_{12}-h_{11}h_{22}+2j_{12}t_{12},\end{equation}
\begin{equation}\label{Codazzigen}(\nabla_{e_1}B)(e_2,e_k)-(\nabla_{e_2}B)(e_1,e_k)=j_{2k} he_1-j_{1k}he_2+2j_{12} he_k\end{equation}
It is clear that equations \eqref{relation1.1} to \eqref{Codazzigen} are necessary condition for surfaces in ${{\mathbb C}} P^2(4c)$. Conversely, given $(M^2,g)$ a Reimannian surface, $E$ a $2$-dimensional vector bundle over $M$ endowed with a scalar product $\overline{g}$ and a compatible connection $\nabla^{E}$. Let $j:TM{\longrightarrow} TM,\ h:TM{\longrightarrow} E,\ s:E{\longrightarrow} TM\ \text{and}\ t:E{\longrightarrow} E$ four tensors.
\begin{definition}
We say that $(M,g,E,\overline{g},\nabla^E,B,j,h,s,t)$ satisfies the compatibility equations for ${{\mathbb C}} P^2(4c)$ if $j$ and $t$ are antisymmetric the Gauss, Codazzi and Ricci equations \eqref{Gaussgen} \eqref{Riccigen} \eqref{Codazzigen} and equations \eqref{relation1.1}-\eqref{relation2.3} are fulfilled.
\end{definition}
Now, we can state the following classical {\it Fundamental Theorem} for surfaces of ${{\mathbb C}} P^2$, which can be found for instance as a special case of the general result of P. Piccione and D. Tausk \cite{PT}.
\begin{prop}\label{propimmersioncp2}
If $(M,g,E,\overline{g},\nabla^E,B,j,h,s,t)$ satisfies the compatibility equations for ${{\mathbb C}} P^2(4c)$ then, there exists an isometric immersion $\Phi:M{\longrightarrow}{{\mathbb C}} P^2(4c)$ such that the normal bundle of $M$ for this immersion is isomorphic to $E$ and such that the second fundamental form $II$ and the normal connection $\nabla^{\perp}$  are given by $B$ and $\nabla^E$. Precisely, there exists a vector bundle isometry $\widetilde{\Phi}:E{\longrightarrow} NM$ so that
$$II=\widetilde{\Phi}\circ B,$$
$$\nabla^{\perp}\widetilde{\Phi}=\widetilde{\Phi}\nabla^E.$$
Moreover, we have 
$$J(\Phi_*X)=\Phi_*(jX)+\widetilde{\Phi}(hX),$$
$$J(\widetilde{\Phi}\xi)=\Phi_*(sX)+\widetilde{\Phi}(t\xi),$$
where $J$ is the canonical complex structure of ${{\mathbb C}} P^2(4c)$ and this isometric immersion is unique up to an isometry of ${{\mathbb C}} P^2(4c)$.

\end{prop}

\subsection{Special cases}
Two special cases are of particular interest and have been widely studied, the complex and Lagrangian surfaces. 

A surface $(M^2,g)$ of ${{\mathbb C}} P^2(4c)$ is said {\bf complex} if the tangent bundle of $M$ is stable by the complex structure of ${{\mathbb C}} P^2(4c)$, that is, $J(TM)=TM$. Note that we have necessarily $J(NM)=NM$. Hence, in that case, with the above notations, we have $h=0$, $s=0$ and so $j$ and $t$ are respectively almost complex structures on $TM$ and $NM$. The compatibility equations for complex surfaces of ${{\mathbb C}} P^2(4c)$ becomes
\begin{equation}\label{compatibilitycomplex}
\left\{\begin{array}{l}
h=0,\ s=0,\ j^2=-{\mathrm{id}}_{TM},\ t^2=-{\mathrm{id}}_{E}\\
\nabla j=0,\ \nabla^{\perp}t=0\\
t(B(X,Y))-B(X,jY)=0,\ \forall X\in TM\\
K_M=4+<B_{22},B_{11}>-|B_{12}|^2\\
K_N=-<[S_{\nu_1},S_{\nu_2}](e_1),e_2>+2\\
(\nabla_{e_1}B)(e_2,e_k)-(\nabla_{e_2}B)(e_1,e_k)=0
\end{array}\right.\end{equation}

A surface $(M^2,g)$ of ${{\mathbb C}} P^2(4c)$ is said totally real if $J(TM)$ is transversal to $TM$. In the particular case when $J(TM)=NM$, we say that $(M^2,g)$ is {\bf Lagrangian}. In that case, we have $j=0$ and $t=0$. Hence, the compatibility equations for Lagranigan surfaces of ${{\mathbb C}} P^2(4c)$ are
\begin{equation}\label{compatibilitylagrangian}
\left\{\begin{array}{l}
j=0,\ t=0,\ sh=-{\mathrm{id}}_{TM},\ hs=-{\mathrm{id}}_{E}\\
\nabla s=0,\ \nabla^{\perp}h=0\\
A_{hY}X+s(B(X,Y))=0,\ \forall X\in TM\\
K_M=1+<B_{22},B_{11}>-|B_{12}|^2\\
K_N=-<[S_{\nu_1},S_{\nu_2}](e_1),e_2>+h_{21}h_{12}-h_{11}h_{22}\\
(\nabla_{e_1}B)(e_2,e_k)-(\nabla_{e_2}B)(e_1,e_k)=0
\end{array}\right.\end{equation}

\section{Restriction of a K\"ahlerian Killing Spin$^c$ spinor and curvature computation}\label{sec4}
We consider a special ${\rm Spin}^c$ structure on ${{\mathbb C}} P^2$ carrying a (real) K\"ahlerian Killing spinor $\varphi$. For example, on can take $q= -1$ and hence $r=1$. For this structure, the curvature of the line bundle is given by $F_A(X,Y)=-2ig(JX,Y)$.  The spinor $\varphi = \varphi_0 + \varphi_1 \in \Gamma (\Sigma_0 \mathbb C P^2 \oplus \Sigma_1 \mathbb C P^2)$  satisfies the following:
$$\left\{\begin{array}{l}
\widetilde{\nabla}_X\varphi_0=-X^-\cdot\varphi_1,\\
 \widetilde{\nabla}_X\varphi_1=-X^+\cdot\varphi_0,
\end{array}
\right.$$
Thus, we have
$$\nabla_X\varphi=-\frac{1}{2}X\cdot\varphi+\frac{i}{2}JX\cdot\overline{\varphi},$$
where $\overline\varphi=\varphi_0-\varphi_1$ is the conjugate of $\varphi$ for the action of the complex volume element $\omega_4^{{\mathbb C}}=-e_1\cdot e_2\cdot e_3\cdot e_4$. Indeed, $\Sigma_0\mathbb C P^2\subset\Sigma^+\mathbb C P^2$ and $\Sigma_1\mathbb C P^2=\Sigma^-\mathbb C P^2$. Note also that such as spinor is of constant norm and each part $\varphi_0$ and $\varphi_1$ does not have any zeros. Indeed, for instance, if $\varphi_0$ vanishes at one point, then it must vanish everywhere (as it is obtained by parallel transport) and $\varphi_1$ is as a parallel spinor which is not the case for this ${\rm Spin}^c$ structure.

Now, let $M$ be a surface of ${{\mathbb C}} P^2$ with normal bundle denoted by $NM$. By the identification of the Clifford multiplications and the ${{\mathop{\rm Spin}^c}}$ Gauss formula, we have
$$\nabla_X\varphi=-\frac 12\eta(X)\cdot\varphi-\frac{1}{2}X\cdot\varphi+\frac{i}{2}JX\cdot\overline{\varphi}.$$
In intrinsic terms, it can be written as 
\begin{eqnarray}\label{partspinor}
\nabla_X\varphi&=&-\frac 12\eta(X)\cdot\varphi-\frac{1}{2}X\cdot\varphi+\frac{i}{2}jX\cdot\overline{\varphi}+\frac{i}{2}hX\cdot\overline{\varphi},
\end{eqnarray}
where $\eta$ is given by
\begin{equation}\label{relation eta B}
\eta(X)=\sum_{j=1}^2e_j\cdot B(e_j,X).
\end{equation}
Here $B$ is the second fundamental form of the immersion, and the operators $j$ and $h$ are those introduces in Section \ref{sec3}. We deduce immediately that 
$$\nabla_X\overline{\varphi}
=-\frac 12 \eta(X)\cdot\overline{\varphi}+\frac{1}{2}X\cdot\overline{\varphi}-\frac{i}{2}jX\cdot\varphi-\frac{i}{2}hX\cdot\varphi.$$

Now, let us go back to an instrinsic setting by considering $(M^2,g)$ an oriented Riemannian surface and $E$ an oriented vector bundle of rank $2$ over $M$ with scalar product $\langle\cdot,\cdot\rangle_E$ and compatible connection $\nabla^E$. We denote by $\Sigma=\Sigma M\otimes\Sigma E$ the twisted spinor bundle. Let $B:TM\times TM{\longrightarrow} E$ a bilinear symmetric map and $j:TM{\longrightarrow} TM,\ h:TM{\longrightarrow} E$ two tensors. We assume that the spinor field $\varphi\in\Gamma(\Sigma)$ satisfies 
 Equation \eqref{partspinor}. We will  compute the spinorial curvature for this spinor field $\varphi$. For this, let $\{e_1,e_2\}$ be a normal local orthonormal frame of $TM$ and $\{\nu_1,\nu_2\}$ a local orthonormal frame of $E$. We have
\begin{eqnarray*}
\nabla_{e_1}\nabla_{e_2}\varphi &=&-\frac{1}{2}\nabla_{e_1}(\eta(e_2))\cdot\varphi+\frac{1}{4}\eta(e_2)\cdot\eta(e_1)\cdot\varphi+\frac{1}{4}\eta(e_2)\cdot e_1\cdot\varphi\\
&&-\frac{i}{4}\eta(e_2)\cdot j(e_1)\cdot\overline{\varphi}-\frac{i}{4}\eta(e_2)\cdot h(e_1)\cdot\overline{\varphi}+\frac{1}{4}e_2\cdot\eta(e_1)\cdot\varphi - \frac 12 \nabla_{e_1}e_2 \cdot \varphi\\
&&+\frac{1}{4}e_2\cdot e_1\cdot\varphi-\frac{i}{4}e_2\cdot j(e_1)\cdot\overline{\varphi}-\frac{i}{4}e_2\cdot h(e_1)\cdot\overline{\varphi}\\
&&+\frac{i}{2}\nabla_{e_1}(j(e_2))\cdot\overline{\varphi}-\frac{i}{4}j(e_2)\cdot\eta(e_1)\cdot\overline{\varphi}+\frac{i}{4}j(e_2)\cdot e_1\cdot\overline{\varphi}\\
&&
+\frac{1}{4}j(e_2)\cdot j(e_1)\cdot\varphi+\frac{1}{4}j(e_2)\cdot h(e_1)\cdot\varphi+\frac{i}{2}\nabla^{\perp}_{e_1}(h(e_2))\cdot\overline{\varphi}\\
&&-\frac{i}{4}h(e_2)\cdot\eta(e_1)\cdot\overline{\varphi}+\frac{i}{4}h(e_2)\cdot e_1\cdot\overline{\varphi}+\frac{1}{4}h(e_2)\cdot j(e_1)\cdot\varphi\\
&&+\frac{1}{4} h(e_2)\cdot h(e_1)\cdot\varphi\\
&= & - \frac 12 \nabla_{e_1} (\eta(e_2))\cdot \varphi + \frac 14 \eta(e_2) \cdot\eta(e_1)\cdot \varphi + \frac 14 e_2 \cdot e_1 \cdot\varphi \\ &&  -\frac 12 \nabla_{e_1} e_2 \cdot\varphi + \frac 14  \eta(e_2) \cdot e_1 \cdot\varphi + \frac 14 e_2 \cdot\eta(e_1)\cdot \varphi \\
&& + \frac i2 \nabla_{e_1}(j(e_2)) \cdot\overline\varphi + \frac i2 \nabla_{e_1}^\perp (h(e_2))\cdot\overline\varphi
 \\
&& - \frac i4 e_2 \cdot j(e_1) \cdot\overline\varphi + \frac i4 j(e_2) \cdot e_1 \cdot\overline \varphi
\\
&&  - \frac i4 e_2 \cdot h( e_1) \cdot\overline\varphi + \frac i4 h (e_2) \cdot e_1 \cdot\overline\varphi
\\
&& + \frac 14 j (e_2) \cdot j e_1 \cdot \varphi \\
&& + \frac 14 h (e_2) \cdot h e_1 \cdot \varphi 
\\
&&+ \frac 14 \big(  j (e_2) \cdot h (e_1) \cdot \varphi + h (e_2) \cdot j e_1 \cdot\varphi\big) \\
&& - \frac i4 \big(  \eta (e_2)\cdot h (e_1)\cdot \overline\varphi + h (e_2) \cdot \eta(e_1) \cdot\overline \varphi\big)
\\
&& - \frac i4 \big(  \eta (e_2)\cdot j (e_1)\cdot \overline\varphi + j (e_2) \cdot \eta(e_1) \cdot\overline \varphi\big)
\\
&=&  I + II + III + IV+ V + VI + VII + IIX + IX + X,
\end{eqnarray*}
where 
\begin{eqnarray*}
I &=& - \frac 12 \nabla_{e_1} (\eta(e_2))\cdot \varphi + \frac 14 \eta(e_2) \cdot\eta(e_1)\cdot \varphi + \frac 14 e_2 \cdot e_1 \cdot\varphi - \frac 12 \nabla_{e_1}e_2 \cdot\varphi 
\end{eqnarray*}
\begin{eqnarray*}
II &=& + \frac i2 \nabla_{e_1}(j(e_2)) \cdot\overline\varphi + \frac i2 \nabla_{e_1}^\perp (h(e_2))\cdot\overline\varphi
\end{eqnarray*}

\begin{eqnarray*}
III &=& - \frac i4 e_2 \cdot j(e_1) \cdot\overline\varphi + \frac i4 j(e_2) \cdot e_1 \cdot\overline \varphi
\end{eqnarray*}

\begin{eqnarray*}
IV &=& - \frac i4 e_2 \cdot h( e_1) \cdot\overline\varphi + \frac i4 h (e_2) \cdot e_1 \cdot\overline\varphi
\end{eqnarray*}

\begin{eqnarray*}
V &=& + \frac 14 j (e_2) \cdot j e_1 \cdot \varphi 
\end{eqnarray*}

\begin{eqnarray*}
VI &=& + \frac 14 h (e_2) \cdot h e_1 \cdot \varphi 
\end{eqnarray*}

\begin{eqnarray*}
VII &=&+ \frac 14 \big(  j (e_2) \cdot h (e_1) \cdot \varphi + h (e_2) \cdot j e_1 \cdot\varphi\big) 
\end{eqnarray*}

\begin{eqnarray*}
IIX &= & - \frac i4 \big(  \eta (e_2)\cdot h (e_1)\cdot \overline\varphi + h (e_2) \cdot \eta(e_1) \cdot\overline \varphi\big)
 \end{eqnarray*}

\begin{eqnarray*}
IX &=& - \frac i4 \big(  \eta (e_2)\cdot j (e_1)\cdot \overline\varphi + j (e_2) \cdot \eta(e_1) \cdot\overline \varphi\big)
\end{eqnarray*}

\begin{eqnarray*}
 X &=& \frac 14  \eta (e_2) \cdot e_1\cdot\varphi + \frac 14  e_2 \cdot \eta(e_1) \cdot \varphi
\end{eqnarray*}

We point out that
\begin{eqnarray*}
\nabla_{[e_1, e_2 ]}\varphi =\underbrace{- \frac 12 \eta([e_1, e_2]) \cdot\varphi - \frac 12 [e_1, e_2] \cdot\varphi }_{\widetilde I([e_1, e_2])} + \underbrace{\frac i2 j ([e_1, e_2])\cdot\overline \varphi + \frac i2 h([e_1, e_2])\cdot\overline \varphi}_{\widetilde {II}([e_1, e_2])} .
\end{eqnarray*}
Some terms are vanishing as shown in the following lemma.
\begin{lemma}
We have  
\begin{eqnarray}
X(e_1, e_2) - X(e_2, e_1) = 0
\end{eqnarray}
\begin{eqnarray}\label{IV}
IV(e_1, e_2) - IV(e_2, e_1) = 0
\end{eqnarray}
 \begin{eqnarray}\label{}
(II+ IIX + IX) (e_1, e_2) - (II + IIX+ IX) (e_2, e_1) -  \widetilde{II}([e_1, e_2])= 0
\end{eqnarray}

\end{lemma}
{\it Proof:} \begin{enumerate}
\item Using the definition of $\eta$, we get  $-\frac 12 B(e_j, X) = e_j\cdot\eta(X) -\eta(X)\cdot e_j$. Hence
\begin{eqnarray*} 
X(e_1, e_2) - X (e_2, e_1) = -\frac 18 B(e_2, e_1) + \frac 18 B(e_1, e_2) = 0.
\end{eqnarray*}

\item \begin{eqnarray}\label{IV}
&& IV(e_1, e_2) - IV(e_2, e_1) = \nonumber \\
  &&-\frac i4 ( e_2 \cdot h( e_1) - h (e_2) \cdot e_1)\cdot\overline\varphi + \frac i4 (e_1 \cdot h( e_2) - h(e_1) \cdot e_2 ) \cdot \overline \varphi \nonumber \\ 
&=& \frac i4 (2g(e_2, h(e_1))    -2 g( h(e_2), e_1)  )\cdot\overline \varphi \nonumber\\ 
&=& 0,
\end{eqnarray}
because $X$ and $h(X)$ are orthogonal for $X \in \Gamma(TM)$. 
 \item First we have 

\begin{eqnarray}\label{II}
&& II(e_1, e_2) - II(e_2, e_1) - \widetilde{II}([e_1, e_2]) \nonumber \\
&=& \frac i2  \nabla_{e_1} (j (e_2))\cdot \overline \varphi + \frac i2  \nabla^\perp_{e_1} (h (e_2))\cdot \overline \varphi -  \frac i2  \nabla_{e_2} (j (e_1))\cdot \overline \varphi - \frac i2  \nabla^\perp_{e_2} (h (e_1))\cdot \overline \varphi\nonumber \\ && - \frac i2 j ([e_1, e_2])\cdot\overline\varphi - \frac i2 h ([e_1, e_2]) \cdot\overline\varphi \nonumber\\ 
& = &
 \frac i2 \Big ( (\nabla_{e_1}j)e_2 \cdot\overline\varphi +   (\nabla_{e_1}h)e_2 \cdot\overline\varphi - 
(\nabla_{e_2}j)e_1 \cdot\overline\varphi - (\nabla_{e_2}h)e_1 \cdot\overline\varphi \Big)
\nonumber\\ & =& 
 \frac i2 \Big (  s(B(e_1, e_2)) \cdot\overline\varphi + S_{h(e_2)} e_1 \cdot\overline \varphi + t(B(e_1, e_2))\cdot\overline\varphi -B(e_1, j(e_2)) \cdot\overline\varphi \nonumber \\ && - s(B(e_1, e_2)) \cdot\overline\varphi -S_{h(e_1)} e_2 \cdot\overline\varphi -t(B(e_1, e_2)) \cdot\overline\varphi + B(e_2, j e_1) \cdot\overline \varphi
\Big ) \nonumber \\ 
&=&  \frac i2 \Big (   S_{h(e_2)} e_1 \cdot\overline \varphi   - S_{h(e_1)} e_2 \cdot\overline\varphi    -B(e_1, j(e_2)) \cdot\overline\varphi  + B(e_2, j e_1) \cdot\overline \varphi
\Big ) 
\end{eqnarray}
Moreover, we calculate

\begin{eqnarray}\label{II'}
 && -B(e_1, j(e_2)) \cdot\overline\varphi  + B(e_2, j e_1) \cdot\overline \varphi \nonumber\\ 
&=& -g(j(e_2), e_1) B(e_1, e_1) \cdot\overline\varphi + g(e_2, j(e_1)) B(e_2, e_2) \cdot\overline\varphi
\nonumber\\ &=& 2g(j(e_1), e_2) H \cdot\overline \varphi
\end{eqnarray}

and

\begin{eqnarray}\label{II''}
&&    S_{h(e_2)} e_1 \cdot\overline \varphi   - S_{h(e_1)} e_2 \cdot\overline\varphi   \nonumber \\
&=& -<S_{h(e_1)} e_2, e_1> e_1 \cdot\overline\varphi -<S_{h(e_1)} e_2, e_2> e_2 \cdot\overline\varphi \nonumber\\ 
&&+ <S_{h(e_2)} e_1, e_1> e_1 \cdot\overline\varphi + <S_{h(e_2)} e_1, e_2> e_2 \cdot\overline\varphi 
\nonumber\\ &=& -<B(e_2, e_1), h(e_1)> e_1 \cdot\overline \varphi  - <B(e_2, e_2), h(e_1)> e_2 \cdot\overline \varphi\nonumber \\
&& + <B(e_1, e_1), h(e_2)> e_1 \cdot\overline \varphi +  <B(e_1, e_2), h(e_2)> e_2 \cdot\overline \varphi 
\end{eqnarray}
In addition we have 
\begin{eqnarray}\label{IIX}
&& IIX(e_1, e_2)- IIX(e_2, e_1)\nonumber\\
&=& \frac i4
\Big (
-e_1 \cdot B(e_1, e_2) \cdot h(e_1) -e_2 \cdot B(e_2, e_2) \cdot h(e_1) -h(e_2) \cdot e_1 \cdot B(e_1, e_1) 
-h(e_2) \cdot e_2 \cdot B(e_1, e_2)\nonumber \\ 
&& +  e_1 \cdot B(e_1, e_1) \cdot h(e_2) + e_2 \cdot B(e_1, e_2) \cdot h(e_2)  
+ h(e_1) \cdot e_1 \cdot B(e_1, e_2) + h(e_1) \cdot e_2 \cdot B(e_2, e_2) 
\Big ) \cdot \overline \varphi
\nonumber\\ 
&=&   
\frac i4 \Big (  2 <B(e_1, e_2), h(e_1)> e_1 +  2 <B(e_2, e_2), h(e_1)> e_2 \nonumber \\ && - 2 <B(e_1, e_2), h(e_2)> e_2 -  2 <B(e_1, e_1), h(e_2)> e_1 \big) \cdot\overline \varphi
 \end{eqnarray}

and 

\begin{eqnarray}\label{IX}
&& IX(e_1, e_2) - IX(e_2, e_1) \nonumber \\
&=& - \frac i4 \eta (e_2)\cdot j (e_1)\cdot \overline\varphi - \frac i4  j (e_2) \cdot \eta(e_1) \cdot\overline \varphi + \frac i4 \eta (e_1)\cdot j (e_2)\cdot \overline\varphi + \frac i4  j (e_1) \cdot \eta(e_2) \cdot\overline \varphi \nonumber  \\
&=& 
\frac i4
\Big (
-e_1 \cdot B (e_1, e_2) \cdot j(e_1) \cdot -  e_2 \cdot B (e_2, e_2) \cdot j(e_1) \cdot - j (e_2) \cdot e_1 \cdot B(e_1, e_1) \cdot \nonumber \\ 
&& - j (e_2) \cdot e_2 \cdot B(e_1, e_2) \cdot  
 + e_1 \cdot B (e_1, e_1) \cdot j(e_2) \cdot + e_2 \cdot B (e_1, e_2) \cdot j(e_2) \cdot\nonumber\\ 
&& +j (e_1) \cdot e_1 \cdot B(e_2, e_1)\cdot  + j (e_1) \cdot e_2 \cdot B(e_2, e_2) \cdot 
\Big ) \overline \varphi
 \nonumber\\ &=&  
\frac i4
\Big (
e_1 \cdot j (e_1) \cdot  B(e_1, e_1)  + e_2\cdot j (e_1) \cdot  B(e_2, e_2) -  j (e_2) \cdot e_1 \cdot  B(e_1, e_1) - j (e_2) \cdot  e_2 \cdot  B(e_1, e_2) \nonumber\\ 
&& -  e_1 \cdot j (e_2) \cdot  B(e_1, e_1) -e_2 \cdot j (e_2) \cdot  B(e_1, e_2) + j(e_1) \cdot  e_1 \cdot  B(e_1, e_2) + j(e_1) \cdot e_2 \cdot  B(e_2, e_2) \Big ) \cdot\overline\varphi    
\nonumber\\
&=& \frac i4 \big (   -2 g(j(e_1), e_2) B(e_2, e_2) + 2g(j(e_2, e_1)) B(e_1, e_1)  \big ) \cdot\overline \varphi 
\nonumber \\ &=& - i g(j(e_1, e_2)) H \cdot \overline \varphi.
\end{eqnarray}
Now, replacing  (\ref{II'}) and (\ref{II''}) in  (\ref{II}) and combining together with (\ref{IIX}) and (\ref{IX}), we get the desired result.

\end{enumerate}
\hfill $\square$\\
Now, we have this second lemma.

\begin{lemma}
We have
\begin{enumerate}
\item
\begin{eqnarray*}
V(e_1, e_2) - V(e_2, e_1) =-\frac 12\langle j(e_2),e_1)\rangle^2e_1\cdot e_2,
\end{eqnarray*}
\item
\begin{eqnarray*}
VI(e_1, e_2) - VI(e_2, e_1) =\frac 12\left[ \langle h(e_2),\nu_1\rangle\langle h(e_1),\nu_2\rangle-\langle h(e_1),\nu_1\rangle\langle h(e_2),\nu_2\rangle\right]\nu_1\cdot \nu_2.
\end{eqnarray*}
\item 
\begin{eqnarray}III(e_1, e_2) - III(e_2, e_1) = i g(e_2, je_1) \overline \varphi 
\end{eqnarray}
\item 
\begin{eqnarray}
VII(e1, e_2) - VII(e_2, e_1) =  \frac 12 \big ( j_{21}h_{11} e_1 \cdot \nu_1 + j_{21}h_{12} e_1 \cdot \nu_2 + j_{21}h_{21} e_2 \cdot \nu_1
+ j_{12}h_{22} e_2 \cdot \nu_2 \big) \cdot\varphi
\end{eqnarray}
\end{enumerate}
\end{lemma}
{\bf Proof.} 
\begin{enumerate}
\item We denote by $j_{kl} = g(je_k, e_l)$. Since $j$ is antisymmetric, we have $j_{kl} = -j_{lk}$  and so  
\begin{eqnarray}\label{V}
&& V(e_1, e_2) - V(e_2, e_1) \nonumber  \\
 &=&\frac 14  \big ( j(e_2)\cdot j(e_1) - j(e_1) \cdot j(e_2)\big) \cdot\varphi \nonumber\\ 
&=& \frac 14 (j_{21} j_{12} e_1\cdot e_2 - j_{12}j_{21} e_2\cdot e_1)\cdot\varphi \nonumber\\
& =& \frac 12 j_{21} j_{12} e_1\cdot e_2  \cdot\varphi \nonumber\\
&=& - \frac 12 g(j(e_1), e_2)^2 e_1\cdot e_2 \cdot \varphi
\end{eqnarray}

\item 
\begin{eqnarray*}
 && VI(e_1, e_2) - VI(e_2, e_1) \nonumber \\&=&  \frac 14 ( h (e_2) \cdot h (e_1) -  h (e_1) \cdot h (e_2)) \nonumber\\
&=&  \frac 14 \big( - h_{21}h_{11} + h_{21} h_{12} \nu_1 \cdot \nu_2 + h_{22} h_{11} \nu_2 \cdot \nu_1 - h_{22} h_{12} + h_{11} h_{21}\nonumber\\ && - h_{11} h_{22} \nu_1 \cdot \nu_2  - h_{12}h_{21}  -h_{11} h_{22} \nu_1 \cdot \nu_2 - h_{12} h_{21} \nu_2 \cdot \nu_1 + h_{12} h_{22}\big) \cdot\varphi 
\nonumber\\ &=& \frac 12 [g(h(e_2), \nu_1)g( h(e_1), \nu_2)    - g( h(e_1), \nu_1) g( h(e_2), \nu_2) ] \nu_1 \cdot \nu_2 \cdot \varphi
\end{eqnarray*}

\item 

\begin{eqnarray}\label{III}
III(e_1, e_2) - III(e_2, e_1)&=&- \frac i4 \big ( e_2 \cdot j(e_1) - j(e_2) \cdot e_1  -e_1 \cdot j(e_2) +j(e_1) \cdot e_2\big) \cdot\overline\varphi \nonumber \\
&=& - \frac i4 \big ( - j_{12}+ j_{21} +  j_{21} - j_{12}\big) \cdot\overline\varphi \nonumber \\
&=& i g(e_2, j (e_1)) \overline \varphi  
\end{eqnarray}

\item 
We have

\begin{eqnarray}\label{VII}
&& VII(e_1, e_2 ) - VII(e_2, e_1) \nonumber \\
&=& \frac 14 \big(  j (e_2) \cdot h (e_1)  + h (e_2) \cdot j e_1  -   j (e_1) \cdot h (e_2)  - h (e_1) \cdot j e_2  \big) \cdot \varphi \nonumber\\
&=&  
\frac 14 \big(  j_{21} e_1 \cdot (h_{11} \nu_1 + h_{12} \nu_2) + j_{12}(h_{21} \nu_1 + h_{22} \nu_2) e_2 \nonumber\\ && - j_{12} e_2 \cdot(h_{21} \nu_1 + h_{22} \nu_2) - j_{21} (h_{11} \nu_1 + h_{12} \nu_2) e_1 \big) \cdot \varphi
\nonumber\\ &=& \frac 12 \big ( j_{21}h_{11} e_1 \cdot \nu_1 + j_{21}h_{12} e_1 \cdot \nu_2 + j_{21}h_{21} e_2 \cdot \nu_1
+ j_{12}h_{22} e_2 \cdot \nu_2 \big) \cdot\varphi
\end{eqnarray}
\end{enumerate}
\hfill $\square$\\

Finally, we have this last lemma obtained by a straightforward calculation and using Lemma 3.3 of \cite{BLR}.
\begin{lemma}
\begin{eqnarray}
&& I(e_1, e_2) - I (e_2, e_1) - \widetilde I (e_1, e_2)\nonumber\\ &=& -\frac 12 \sum_{j=1}^2 e_j \cdot \big( (\nabla^{'}_{e_1} B)(e_2, e_j) ) -(\nabla^{'}_{e_2} B)(e_1, e_j) \big) \nonumber\\ && + \frac 12 g([S_{\nu_1},  S_{\nu_2}] )(e_1), e_2) \nu_1 \cdot \nu_2 \cdot\varphi \nonumber \\ && -\frac 12 e_1\cdot e_2 \cdot \varphi, 
\end{eqnarray}
where $\nabla^{'}$ is the natural connection on $T^*M \otimes T^*M \otimes E$ and $B_{kl} = B(e_k, e_l)$
\end{lemma}

\section{Lagrangian case, proof of Theorem \ref{thm2}}
Now, we have all the ingredients to prove Theorems \ref{thm1} and \ref{thm2}. We begin by the Lagragian case. Assume that $j =0$ and $ t= 0$. So we have 
\begin{eqnarray}
\mathcal{R}_{e_1, e_2} \varphi &=& \frac 12 K_M e_1 \cdot e_2 \cdot \varphi - \frac 12 K_E \nu_1 \cdot \nu_2 \cdot \varphi + \frac i2 \alpha^{M +E} (e_1, e_2) \varphi,
\end{eqnarray}
with $\alpha^{M +E} (e_1, e_2)  =0$  because $j=0$.
From the other hand, we have 

\begin{eqnarray}
&& \mathcal{R}_{e_1, e_2} \varphi \\
&=& - \frac 12 \sum_{j=1}^2 e_j \cdot( (\nabla^{`}_{e_1} B) (e_2, e_j)     -  (\nabla^{`}_{e_2} B) (e_1, e_j)    ) \cdot\varphi \nonumber \\
&& + \frac 12 (\vert B_{12}\vert^2 - <B_{11}, B_{22}>) e_1 \cdot e_2 \cdot \varphi \nonumber\\
&& + \frac 12 <[S_{\nu_1}, S_{\nu_2}](e_1), e_2> \nu_1 \cdot \nu_2 \cdot\varphi \nonumber \\
&&-\frac 12 e_1 \cdot e_2 \cdot \varphi + \frac 12  (h_{21}h_{12} - h_{11}h_{22}) \nu_1 \cdot\nu_2 \cdot\varphi
\end{eqnarray}

We get finally that $T \cdot \varphi = 0$, where $T \in (\Lambda^2 M \otimes 1 \oplus TM \otimes E \oplus 1 \otimes \Lambda^2 E)$. Thus, since $\varphi^+$ and $\varphi^-$ never vanish, we have $T =0$ by \cite{BLR}[Lemma 3.4] and it gives 

$$K_M = <B_{11}, B_{22}> - \vert B_{12}\vert^2 +1,$$
$$ K_E = - <[S_{\nu_1}, S_{\nu_2}](e_1), e_2> - (h_{21} h_{12} - h_{22}h_{11} ),$$
$$(\nabla_{e_1} B) (e_2, e_j)     -  (\nabla_{e_2} B) (e_1, e_j)   = 0, $$
which are Gauss, Ricci and Codazzi equations for a Lagrangian surface in $\mathbb C P^2$ and so the conditions \eqref{compatibilitylagrangian} are fulfuilled. Hence, by Proposition \ref{propimmersioncp2}, we conclude that there exists a Lagrangian isomertic immersion from $(M,g)$ into ${{\mathbb C}} P^2$ with $E$ as normal bundle and $B$ as second fundamental form. This proves that assertion (2) of Theorem \ref{thm2} implies assertion (1). The converse is immediate by the discussions of Sections \ref{sec3} and \ref{sec4}. Theorem \ref{thm2} is proven.

\section{Complex case, proof of Theorem \ref{thm1}}
Assume that $s =0$; $ h = 0$  so $\alpha^{M+E} (e_1, e_2) = -2$. 
We take $j e_1 = e_2$ and $t \nu_1 = \nu_2$, i.e, $g(j e_1, e_2) = g (t \nu_1, \nu_2 )=1.$ 

We calculate and we get 
\begin{eqnarray}
\mathcal{R}_{e_1, e_2} \varphi &=&- \frac 12 K_M e_1 \cdot e_2 \cdot \varphi - \frac 12 K_N \nu_1 \cdot \nu_2 \cdot \varphi + \frac i2 \alpha^{M +N} (e_1, e_2) \varphi \nonumber \\ &=&
- \frac 12 K_M e_1 \cdot e_2 \cdot \varphi - \frac 12 K_N \nu_1 \cdot \nu_2 \cdot \varphi -i  \varphi  \nonumber \\ &=& 
\overline T \cdot \varphi - i \varphi,
\end{eqnarray}
where $\overline T$ is a $2$-form.
From the other hand, we have
\begin{eqnarray}
\mathcal{R}_{e_1, e_2} \varphi &=& - e_1 \cdot e_2 \cdot\varphi + i \overline \varphi 
\nonumber \\
&& - \frac 12 \sum_{j=1}^2 e_j \cdot( (\nabla_{e_1} B) (e_2, e_j)     -  (\nabla_{e_2} B) (e_1, e_j)    ) \cdot\varphi \nonumber \\
&& + \frac 12 (\vert B_{12}\vert^2 - <B_{11}, B_{22}>) e_1 \cdot e_2 \cdot \varphi \nonumber\\
&& + \frac 12 <[S_{\nu_1}, S_{\nu_2}](e_1), e_2> \nu_1 \cdot \nu_2 \cdot\varphi \nonumber \\ 
&& = \tilde T \cdot \varphi + i \overline \varphi. 
\end{eqnarray}

Together, it gives $\overline T \cdot \varphi - \tilde T \cdot\overline \varphi - i \varphi - i \overline \varphi = 0$, which means that $\mathcal T \cdot \varphi - i \varphi - i \overline \varphi = 0$, where $\mathcal T$ is again a $2$ form given by 
\begin{eqnarray}
\mathcal T &=& - \frac 12 K_M e_1 \cdot e_2 \cdot \varphi - \frac 12 K_N \nu_1 \cdot \nu_2 \cdot \varphi\nonumber + e_1\cdot e_2 \cdot\varphi \\ 
&&  -\frac 12 (\vert B_{12}\vert^2 - <B_{11}, B_{22}>) e_1 \cdot e_2 \cdot \varphi \nonumber\\
&& - \frac 12 <[S_{\nu_1}, S_{\nu_2}](e_1), e_2> \nu_1 \cdot \nu_2 \cdot\varphi \nonumber \\ 
&& +\frac 12 \sum_{j=1}^2 e_j \cdot( (\nabla_{e_1} B) (e_2, e_j)     -  (\nabla_{e_2} B) (e_1, e_j)    ) \cdot\varphi
\end{eqnarray}

We give now the following Lemma
\begin{lemma}
Let $\mathcal T$ be a $2$ form, i.e, $\mathcal T \in \Lambda^2 M \otimes 1 \oplus \Lambda^1 M \otimes \Lambda^1 E \oplus 1 \otimes \Lambda^2 E$. Assume that 
$$\mathcal T \cdot \varphi - i \varphi - i \overline \varphi = 0, $$
and write $\mathcal T = T^t e_1 \cdot e_2  + T^n \nu_1 \cdot \nu_2 + T^m,$ where $T^m \in \Lambda^1 M \otimes \Lambda^1E$. 
Then, 
$$T^t = -1, T^n = 0, \ \ \ \ \text{and}\ \ \ T^m =  0 .$$
\end{lemma}
{\bf Proof.} 
Let $\varphi = \varphi^+ + \varphi^-,$ with
$$\varphi^+ =\varphi^{++} + \varphi^{--},$$
$$\varphi^- = \varphi^{-+} + \varphi^{+-},$$
a solution of \eqref{partspinor} with $h=0$. This means that 
 $$\nabla_X \varphi^{++} = -\frac 12 X\cdot \varphi^{-+} - \frac i2 jX\cdot \varphi^{-+}$$
  $$\nabla_X \varphi^{+-} = -\frac 12 X\cdot \varphi^{--} + \frac i2 jX\cdot \varphi^{--}$$
 $$\nabla_X \varphi^{-+} = -\frac 12 X\cdot \varphi^{++} +\frac i2 jX\cdot \varphi^{++}$$
  $$\nabla_X \varphi^{--} = -\frac 12 X\cdot \varphi^{+-} - \frac i2 jX\cdot \varphi^{+-}.$$
  For a sake of simplicity, and without lost of generality, we can restrict only $\varphi^+=\varphi^{++}$ and $\varphi^-=\varphi^{-+}$ which which have no zeros by assumption. The equation $$\mathcal T \cdot \varphi - i \varphi - i \overline \varphi = 0,$$ 
  becomes $$T^t e_1\cdot e_2\cdot (\varphi^{++} + \varphi^{-+}) + (T^n+1) \nu_1\cdot \nu_2 \cdot (\varphi^{++} + \varphi^{-+}) + T^m \cdot (\varphi^{++} + \varphi^{-+}) = i \overline \varphi = i (\varphi^{++} - \varphi^{-+})$$
 Taking  the scalar product with $\varphi^{++}$ then with $\varphi^{-+}$, we get
 $$T^t +T^n +1 = -1,$$
  $$-T^t + T^n+1 = 1,$$
  which gives $T^n =-1$, $T^t = -1$ and $T^m = 0$. These are Gauss, Codazzi and  Ricci equations and so the conditions \eqref{compatibilitycomplex} are fulfilled. There are exactly the conditions ofr a complex immersion. Hence, by Proposition \ref{propimmersioncp2}, we conclude that there exists a complex isometric immersion from $(M,g)$ into ${{\mathbb C}} P^2$ with $E$ as normal bundle and $B$ as second fundamental form. As for the Lagrangian case, this proves that assertion (2) of Theorem \ref{thm1} implies assertion (1). Here again, the converse is immediate by the discussions of Sections \ref{sec3} and \ref{sec4}, which concludes the proof of Theorem \ref{thm1}.

\section{The Dirac equation}
Let $\varphi$ be a spinor field satisfying Equation \eqref{partspinor}, then it satisfies the following Dirac equation
\begin{equation}\label{eqdirac}
D\varphi=\vec{H}\cdot\varphi-\varphi +\frac i2\beta\cdot\overline{\varphi},
\end{equation}
where $\beta$ is the 2-form defined by $ \displaystyle\beta=\sum_{i=1,2}e_i\cdot he_i=\sum_{i,j=1}^2h_{ij}e_i\cdot\xi_j$, where $h_{i,j}=\langle he_i,\xi_j\rangle$.\\
As in \cite{BLR} and \cite{Roth4}, we will show that this equation with an appropiate condition on the norm of both $\varphi^+$ and $\varphi^-$ is equivalent to Equation \eqref{partspinor}, where the tensor $B$ is expressed in terms of the spinor field $\varphi$ and such that ${\mathrm{tr}}(B)=2\vec{H}$. Moreover, from Equation \eqref{partspinor} we deduce the following conditions on the derivatives of $|\varphi^+|^2$ and $|\varphi^-|^2$. Indeed, after decomposition onto $\Sigma^+$ and $\Sigma^-$, \eqref{partspinor} becomes
$$
\nabla_X\varphi^{\pm}=-\frac12\eta(X)\cdot\varphi^{\pm}-\frac12 X\cdot\varphi^{\mp}\mp\frac i2jX\varphi^{\mp}\mp\frac i2hX\varphi^{\mp}.
$$
From this we deduce that
\begin{equation}\label{normepm}
X(|\varphi^{\pm}|^2)={\mathrm{Re}} \left\langle -\frac12 X\cdot\varphi^{\mp}\mp\frac i2jX\cdot \varphi^{\mp}\mp\frac i2hX\cdot\varphi^{\mp},\varphi^{\pm}\right\rangle
\end{equation}
Now, let $\varphi$ a spinor field solution of the Dirac equation \eqref{eqdirac} with $\varphi^+$ and $\varphi^-$ nowhere vanishing and satisying the norm condition \eqref{normepm}, we set for any vector fields $X$ and $Y$ tangent to $M$ and $\xi\in E$
\begin{eqnarray}\label{defB}
\left<B(X,Y),\xi\right>&=&\frac{1}{|\varphi^+|^2}{\mathrm{Re}}\left<X\cdot\nabla_Y\varphi^+-\frac12 \left(X+ijX+ihX\right)\cdot Y\cdot\varphi^{-},\xi\cdot\varphi^+\right>\\ \nonumber
&&+\frac{1}{|\varphi^-|^2}{\mathrm{Re}}\left<X\cdot\nabla_Y\varphi^--\frac12 \left(X-ijX-ihX\right)\cdot Y\cdot\varphi^-,\xi\cdot\varphi^+\right>
\end{eqnarray}

Then, we have the following
\begin{prop}\label{prop1}
Let $\varphi\in\Gamma(\Sigma)$ satisfying the Dirac equation \eqref{eqdirac}
$$D\varphi=\vec{H}\cdot\varphi-\varphi-\beta\cdot\overline{\varphi}$$
such that 
$$
X(|\varphi^{\pm}|^2)={\mathrm{Re}} \left\langle -\frac12 X\cdot\varphi^{\mp}\mp\frac i2jX\cdot\varphi^{\mp}\mp\frac i2hX\cdot\varphi^{\mp},\varphi^{\pm}\right\rangle
$$
then $\varphi$ is solution of Equation \eqref{partspinor}
$$\nabla_X\varphi=-\frac 12\eta(X)\cdot\varphi-\frac{1}{2}X\cdot\varphi+\frac{i}{2}jX\cdot\overline{\varphi}+\frac{i}{2}hX\cdot\overline{\varphi}$$
where $\eta$ is defined by $\displaystyle\eta(X)=-\frac{1}{2}\sum_{j=1}^2e_j\cdot B(e_j,X)$.\\
Moreover, $B$ is symmetric.
\end{prop}
The proof of this proposition will not be given, since it is completely similar to the case of Riemannian products \cite{Roth4}. Now, combining this proposition with Theorems \ref{thm1} and \ref{thm2}, we get the following corollaries. We have this first one for complex surfaces.
\begin{cor}\label{cor1}
Let $(M^2,g)$ be an oriented Riemannian surface and $E$ an oriented vector bundle of rank $2$ over $M$ with scalar product $<\cdot,\cdot>_E$ and compatible connection $\nabla^E$. We denote by $\Sigma=\Sigma M\otimes\Sigma E$ the twisted spinor bundle. Let $j$ be a complex structure on $M$ and $t$ a complex structure on $E$. Let $\vec{H}$ be a section of $E$. Then, the two following statements are equivalent
\begin{enumerate}

\item There exists a ${\rm Spin}^c$  structure on $\Sigma M\otimes\Sigma E$ with $\alpha^{M +E} (e_1, e_2)=0$ and a spinor field $\varphi$ in $\Sigma$ solution of the Dirac equation
$$D\varphi=\vec{H}\cdot\varphi-\varphi$$
such that $\varphi^+$ and $\varphi^-$ never vanish, satisfy the norm condition
$$X(|\varphi^{\pm}|^2)={\mathrm{Re}} \left\langle -\frac12 X\cdot\varphi^{\mp}\mp\frac i2jX\cdot \varphi^{\mp}\varphi^{\pm}\right\rangle$$
 and such that the maps $j$, $t$ and the tensor $B$ defined by \eqref{defB} satisfy $t(B(X,Y))=B(X,jY)$ for all $X,Y\in{\mathfrak{X}}(M)$.
 \item There exists an isometric {\bf complex} immersion of $(M^2,g)$ into ${{\mathbb C}} P^2$ with $E$ as normal bundle and mean curvature $\vec{H}$ such that over $M$ the complex strcuture of ${{\mathbb C}} P^2$ is given by $j$ and $t$ (in the sense of Proposition \ref{propimmersioncp2}).
\end{enumerate}
\end{cor}

We have this second corollary for Lagrangian surfaces.

\begin{cor}\label{cor2}
Let $(M^2,g)$ be an oriented Riemannian surface and $E$ an oriented vector bundle of rank $2$ over $M$ with scalar product $<\cdot,\cdot>_E$ and compatible connection $\nabla^E$. We denote by $\Sigma=\Sigma M\otimes\Sigma E$ the twisted spinor bundle. Let $B:TM\times TM{\longrightarrow} E$ a bilinear symmetric map, $h:TM{\longrightarrow} E$ and $s: E{\longrightarrow} TM$ the dual map of $h$. Assume that the maps $h$, $s$ are parallel and satisfy $hs=-{\mathrm{id}}_{E}$. Let $\vec{H}$ be a section of $E$. Then, the two following statements are equivalent
\begin{enumerate}

\item There exists a ${\rm Spin}^c$  structure on $\Sigma M\otimes\Sigma E$ with $\alpha^{M +E} (e_1, e_2)=-2$ and a spinor field $\varphi$ in $\Sigma$ solution of the Dirac equation
$$D\varphi=\vec{H}\cdot\varphi-\varphi+\frac i2\beta\cdot\overline{\varphi}$$
such that $\varphi^+$ and $\varphi^-$ never vanish, satisfy the norm condition
$$X(|\varphi^{\pm}|^2)={\mathrm{Re}} \left\langle -\frac12 X\cdot\varphi^{\mp}\mp\frac i2hX\cdot \varphi^{\mp}\varphi^{\pm}\right\rangle$$
 and such that  the tensor $B$ defined by \eqref{defB} satisfy $A_{hY}X+s(B(X,Y))=0$, for all $X\in TM$, where $A_{\nu}:TM{\longrightarrow} TM$ if defined by $g(A_{\nu}X,Y)=\langle B(X,Y),\nu\rangle_E$ for all $X,Y\in TM$ and $\nu\in E$.
 \item There exists an isometric {\bf Lagrangian} immersion of $(M^2,g)$ into ${{\mathbb C}} P^2$ with $E$ as normal bundle and mean curvature $\vec{H}$ such that over $M$ the complex strcuture of ${{\mathbb C}} P^2$ is given by $h$ and $s$ (in the sense of Proposition \ref{propimmersioncp2}). 
\end{enumerate}
\end{cor}
{\bf Acknowledgment.}  
The first named author would like to thank the University of Paris-Est, Marne La Vall\'ee  for its support and hospitality. Both authors are grateful to Mihaela Pilca for helpful discussions about K\"ahlerien Killing spinors.
\begin{thebibliography}{99}

\bibitem{Ba} \textsc{C. B\"ar}, {\sl Extrinsic bounds for eigenvalues of the Dirac operator}, Ann. Glob. Anal. Geom. 16 (1998) 573-596.

\bibitem{ballman} \textsc{W. Ballmann,} {\sl Lectures on K\"ahler manifolds}, ESI Lectures in Mathematics and Physics, ISBN 978-3-03719-025-8, 2006.
\bibitem{bay} \textsc{P. Bayard}, {\sl On the spinorial representation of spacelike surfaces into 4-dimensional Minkowski space}, J. Geom. Phys. \textbf{74} (2013), 289-313.
\bibitem{BLR} \textsc{P. Bayard, M.A. Lawn \& J. Roth}, {\sl Spinorial representation of surfaces in four-dimensional Space Forms}, Ann. Glob. Anal. Geom. \textbf{44 (4)} (2013), 433-453.
\bibitem{BP} \textsc{P. Bayard \& V. Patty}, {\sl Spinor representation of Lorentzian surfaces in ${{\mathbb R}}^{2,2}$}, J. Geom. Phys. \textbf{95} (2015), 74-95.
\bibitem{besse} \textsc{A. L. Besse,} {\sl Einstein Manifolds}, Ergebnisse der Mathematik (3), Springer Verlag, Berlin (1987). 

\bibitem{bookspin} \textsc{J. P. Bourguignon, O. Hijazi, J. L. Milhorat, A. Moroianu \& S. Moroianu}, {\sl A spinorial approach to Riemannian and conformal geometry}, to appear in SME Monographs in Mathematics.
\bibitem{Da} B. Daniel, \emph{ Isometric immersions into $\mathbb{S}^n \times\mathbb{R}$ and $\mathbb{H}^n\times\mathbb{R}$ and applications to minimal surfaces}, Trans. Amer. Math. Soc. \textbf{361} (2009) no 12, 6255-6282.
\bibitem{Da2} B. Daniel, \emph{ Isometric immersions into $3$-dimensional homogeneous manifolds}, Comment. Math. Helv. \textbf{82} (2007), 87-131.

\bibitem{Fr}  \textsc{T. Friedrich}, \emph{On the spinor representation of surfaces in Euclidean $3$-space}, J. Geom. Phys. (1998), 143-157.
\bibitem{ginoux-these} \textsc{N. Ginoux}, {\sl Op\'erateurs de Dirac sur les sous-vari\'et\'es},  Ph. D thesis 2002, Institut \'Elie Cartan, Nancy.
\bibitem{GM} \textsc{N. Ginoux \& B. Morel} {\sl On eigenvalue estimates for the submanifold Dirac operator}, Int. J. Math. 13 (2002), No. 5, 533-548.

\bibitem{ginoux-book} \textsc{N. Ginoux}, {\sl  The Dirac spectrum}, Lect. Notes in Math. 1976, Springer 2009.
\bibitem{HR1} \textsc{F. H\'elein \& P. Romon}, \emph{Weierstrass representation of Lagrangian surfaces in four-dimensional space using spinors and quaternions}, Comment. Math. Helv. \textbf{75} (2000), 668-680.
\bibitem{HR2} \textsc{F. H\'elein \& P. Romon}, \emph{Hamiltonian stationary tori in the complex projective plane}, Proc. London Math. Soc. \textbf{90} (2005), 472-496.
\bibitem{HMU} \textsc{O. Hijazi, S. Montiel \& F. Urbano}, {\sl ${{\mathop{\rm Spin}^c}}$ geometry of K\"{a}hler manifolds and the Hodge Laplacian on minimal Lagrangian submanifolds},  Math. Z. \textbf{253}, Number 4 (2006) 821-853. 

\bibitem{hijazi-lecture} \textsc{O. Hijazi}, {\sl Spectral properties of the Dirac operator and geometrical structures}, Proceedings of the summer school on geometric methods in quantum field theory, Villa de Leyva, Colombia, July 12-30, 1999, World Scientific 2001. 
\bibitem{Ko}  \textsc{B.G. Konopelchenko}, \emph{Weierstrass representations for surfaces in 4D spaces and their integrable deformations via DS hierarchy}, Ann. Glob. Anal. Geom. \textbf{18} (2000) 61-74. 
\bibitem{Kow} \textsc{D. Kowalczyk}, {\it Isometric immersions into products of space forms}, Geom. Dedicata \textbf{151} (2011), 1-8.

\bibitem{KS}  \textsc{R. Kusner and N. Schmidt}, \emph{The spinor representation of surfaces in space}, Preprint arxiv dg-ga/9610005 (1996).
\bibitem{La}  \textsc{M.A. Lawn}, \emph{A spinorial representation for Lorentzian surfaces in ${{\mathbb R}}^{2,1}$}, J. Geom. Phys. \textbf{58} (2008) no. 6, 683-700.
\bibitem{LO} M.-A. Lawn \& M. Ortega, \emph{A Fundamental Theorem for Hypersurfaces in Semi-Riemannian Warped Products}, J. Geom. Phys. \textbf{90} (2015), 55-70.
\bibitem{LR}  \textsc{M.A. Lawn \& J. Roth}, \emph{Spinorial characterization of surfaces in pseudo-Riemannian space forms}, Math. Phys. Anal. and Geom. \textbf{14} (2011) no. 3, 185-195.
\bibitem{LR2}  \textsc{M.A. Lawn \& J. Roth}, \emph{Isometric immersions of hypersurfaces into 4-dimensional manifolds via spinors}, Diff. Geom. Appl. \textbf{28} (2) (2010),  205-219
\bibitem{spin} \textsc{ H. B. Lawson \& M.-L. Michelson} {\sl Spin Geometry}, Princeton Mathematical Series, 38. Princeton University Press (1989).

\bibitem{montiel} \textsc{S. Montiel,} {\sl Using spinors to study submanifolds,} Roma 2004 - Nancy 2005.

\bibitem{Mo}  \textsc{B. Morel}, \emph{Surfaces in $\mathbb{S}^3$ and $\mathbb{H}^3$ via spinors}, Actes du s\'eminaire de th\'eorie spectrale, Institut Fourier, Grenoble, 23 (2005), 9-22.
\bibitem{am_lectures} \textsc{A. Moroianu,} {\sl Lectures on K{\"a}hler Geometry}, London Mathematical Society Student Text {\bf 69}, Cambridge University Press, Cambridge, 2007.

\bibitem{NR} \textsc{R. Nakad \& J. Roth}, {\sl Hypersurfaces of Spin$^c$ manifolds and Lawson Type correspondence},  Annals of Global Analysis and Geometry, Vol 42 no 3 (2012), pp 421-442. 
\bibitem{PT} \textsc{P. Piccione and D.V. Tausk}, {\sl An existence theorem for $G$-strcture preserving affine immersions}, Indiana Univ. Math. J. 57 (3) (2008), 1431-1465.
\bibitem{RR}  \textsc{P. Romon \& J. Roth}, \emph{The spinor representation formula in 3 and 4 dimensions}, Pure and Applied Differential Geometry, Proceedings of the conference PADGE 2012, Shaker Verlag, Aachen (2013) 261-282
\bibitem{Roth4}  \textsc{J. Roth}, \emph{ Spinorial characterization of surfaces into $3$-dimensional homogeneous manifolds}, J. Geom. Phys 60 (2010), 1045-106.
\bibitem{Roth3}  \textsc{J. Roth}, \emph{Isometric immersion into Lorentzian products}, Int. J. Geom. Method. Mod. Phys, 8 (2011) no.6, 1-22.
\bibitem{Roth} \textsc{J. Roth}, {\sl Spinors and isometric immersions of surfaces into 4-dimensional products}, Bulletin of the Belgian Mathematical Society - Simon Stevin Vol 21 no 4 (2014), pp 635-652.

\bibitem{Ta}  \textsc{I. Taimanov}, \emph{Surfaces of revolution in terms of solitons}, Ann. Glob. Anal. Geom. \textbf{15} (1997) 410-435.
\bibitem{Ta2}  \textsc{I. Taimanov}, \emph{Surfaces in the four-space and the Davey-Stewartson equations}, J. Geom Phys. \textbf{56} (2006) 1235-1256. 
\bibitem{Ur}  \textsc{F. Urbano}, \emph{Hamiltonian stability and index of minimal Lagrangian surfaces of the complex projective plan}, Indiana Univ. Math. J. \textbf{56} (2007), 931-946.

\bibitem{hitchin}

\textsc{N. Hitchin}, \emph{Harmonic spinors}, Adv. Math.\textbf{14} (1974), 1-55.

\bibitem{kirchberg1} \textsc{K.-D. Kirchberg}, \emph{An estimation for the first eigenvalue of the Dirac operator on closed K\"ahler manifolds ofpositive scalar curvature}, Ann. Global Anal. Geom. \textbf{4} (1986), 291-325.

\bibitem{Moro1} \textsc{A. Moroianu}, \emph{Parallel and Killing spinors on ${{\mathop{\rm Spin}^c}}$
manifolds}, Comm. Math. Phys. \textbf{187} (1997), 417-427.

\end{thebibliography}
\end{document}

\documentclass[11pt, reqno]{amsart}

\usepackage[OT2, T1]{fontenc}
\usepackage{url}
\usepackage{amsmath}
\usepackage{array}
\usepackage{amsfonts}
\usepackage{amsfonts}
\usepackage{amssymb}
\usepackage{amsthm}
\usepackage{hyperref}
\usepackage{colonequals}	
\usepackage{color}
\usepackage{mathabx}		
\usepackage{mathrsfs}
\usepackage{amscd}
\usepackage[all,cmtip]{xy}	
\usepackage{sseq}			
\usepackage{verbatim}
\usepackage{parskip}
\usepackage{microtype}
\usepackage[in]{fullpage}
\usepackage{graphicx}
\usepackage{mathrsfs} 			
\usepackage{bm} 				
\usepackage{cleveref}
\usepackage{enumitem}
\usepackage{moreenum}			
\usepackage[alphabetic,lite]{amsrefs} 	
\usepackage{pifont}
\usepackage{stmaryrd}	
\usepackage{graphicx}
\usepackage{rotating}

\DeclareSymbolFont{cyrletters}{OT2}{wncyr}{m}{n}
\DeclareMathSymbol{\Sha}{\mathalpha}{cyrletters}{"58}

\usepackage{color}

		

			
		

 	
 		
 		
 			
 		
 		
 		
 	

		
		
		
		
			
		
		
		
		
		
		
		
		
		
		
	
		
		
		
		
		
		
		
		
		

	

\providecommand{\abs}[1]{\left\lvert#1\right\rvert}
\providecommand{\norm}[1]{\lVert#1\rVert}
\providecommand{\In}[1]{\left\langle#1\right\rangle}
\providecommand{\p}[1]{\left(#1\right)}
\providecommand{\sqp}[1]{\left[#1\right]}
\providecommand{\f}[2]{\frac{#1}{#2}}
		
			
		
		
			
		
		
			
			
			
			
			
		
		
		
		
		
		
		
			
			
			
			
			
			
			
			
			
		
			
		
			
		
			
			
			
			
		
			
		
		
		
		
		
		
		
		
		
		
		
	
	
		
		
		
		
		
			
	
	
	
	
	
		
	
	
	
	
	
		
		
		
		
		
		
		
		
		
		
		
		
		
		
		
		
		
		
		
		
		
		
		
		

\theoremstyle{plain}
\newtheorem{thm}[subsection]{Theorem}
\Crefname{thm}{Theorem}{Theorems}
\newtheorem{rethm}{Theorem}
\Crefname{rethm}{Theorem}{Theorem}
\newtheorem{prop}[subsection]{Proposition}
\Crefname{prop}{Proposition}{Propositions}
\newtheorem{q}[subsection]{Question}
\Crefname{q}{Question}{Questions}
\Crefname{eg}{Example}{Examples}
\newtheorem{Problem}[subsection]{Problem}
\Crefname{Problem}{Problem}{Problems}
\newtheorem{conj}[subsection]{Conjecture}
\Crefname{conj}{Conjecture}{Conjectures}
\newtheorem{cor}[subsection]{Corollary}
\Crefname{cor}{Corollary}{Corollaries}
\newtheorem{lem}[equation]{Lemma}
\Crefname{lem}{Lemma}{Lemmas}

\newtheorem{lemma}[subsection]{Lemma}
\newtheorem{defn}[subsection]{Definition}

\theoremstyle{remark}
\newtheorem{claim}[subsubsection]{Claim}
\Crefname{claim}{Claim}{Claims}

\theoremstyle{definition}
\newtheorem{eg}[subsection]{Example}
\newtheorem{rem}[subsection]{Remark}
\Crefname{rem}{Remark}{Remarks}
\newtheorem*{rems}{Remarks}

\newtheoremstyle{subsection-tweak}
   {11pt}
   {3pt}   {}
   {}   {\bfseries}
   {}   {.5em}
   {\thmnumber{\@{#1}{}\@{#2}.}    \thmnote{~{\bfseries#3.}}}    
    
\newtheorem{innercustomconj}{Conjecture}
\newenvironment{customconj}[1]
  {\innercustomconj}
  {\endinnercustomconj}

\Crefname{innercustomconj}{Conjecture}{Conjecture}

\theoremstyle{subsection-tweak}
\newtheorem{pp}[subsection]{}

\numberwithin{equation}{subsection}

\setlength{\parindent}{0pt} 

\begin{document}
\author{K\k{e}stutis \v{C}esnavi\v{c}ius}
\title{\resizebox{\textwidth}{!}{\mbox{The $\ell$-parity conjecture over the constant quadratic extension}}}
\date{\today}
\subjclass[2010]{Primary 11G10; Secondary 11G40, 11R58}
\keywords{Parity conjecture, root number, arithmetic duality}
\address{Department of Mathematics, Massachusetts Institute of Technology, Cambridge, MA 02139, USA}
\email{kestutis@math.mit.edu}
\urladdr{http://math.mit.edu/~kestutis/}

\begin{abstract} 
For a prime $l$ and an abelian variety $A$ over a global field $K$, the $l$-parity conjecture predicts that, in accordance with the ideas of Birch and Swinnerton-Dyer, the ${\mathbb{Z}}_l$-corank of the $l{^{\infty}}$-Selmer group and the analytic rank agree modulo $2$. Assuming that $\operatorname{char} K > 0$, we prove that the $l$-parity conjecture holds for the base change of $A$ to the constant quadratic extension if $l$ is odd, coprime to $\operatorname{char} K$, and does not divide the degree of every polarization of $A$. We deduce that without the base change it holds if, in addition, the analytic rank of $A$ grows by at most $1$ in this extension. Certain elliptic curve cases of the latter result were known; the techniques that permit arbitrary dimension include the \'{e}tale cohomological interpretation of Selmer groups, the Grothendieck--Ogg--Shafarevich formula, and the study of the behavior of local root numbers in unramified extensions.
\end{abstract}

\maketitle

\section{Introduction}

\begin{pp}[The $l$-parity conjecture] {\label}{l-par}
The Birch and Swinnerton-Dyer conjecture (BSD) predicts that the completed $L$-function $L(A, s)$ of an abelian variety $A$ over a global field $K$ extends meromorphically to the whole complex plane, vanishes to the order $\operatorname{rk} A$ at $s = 1$, and satisfies the functional equation
\begin{equation}{\label}{fe}
L(A, 2 - s) = w(A) L(A, s),
\end{equation}
where $\operatorname{rk} A$ and $w(A)$ are the Mordell--Weil rank and the global root number of $A$. The vanishing assertion combines with \eqref{fe} to give ``BSD modulo $2$'', namely, the parity conjecture:
\[
(-1)^{\operatorname{rk} A} \overset{?}{=} w(A).
\]
Selmer groups tend to be easier to study than Mordell--Weil groups, so, fixing a prime $l$, one sets
\[
\operatorname{rk}_l A {\colonequals} \dim_{{\mathbb{Q}}_l} \operatorname{Hom}(\operatorname{Sel}_{l{^{\infty}}} A, {\mathbb{Q}}_l/{\mathbb{Z}}_l){\otimes}_{{\mathbb{Z}}_l} {\mathbb{Q}}_l,\ \text{where } \operatorname{Sel}_{l{^{\infty}}} A {\colonequals} \varinjlim \operatorname{Sel}_{l^n} A\text{ is the \emph{$l{^{\infty}}$-Selmer group},}
\] 
notes that the conjectured finiteness of the Shafarevich--Tate group $\Sha(A)$ implies $\operatorname{rk}_l A = \operatorname{rk} A$, and instead of the parity conjecture considers the $l$-parity conjecture:
\begin{equation}{\label}{l-par-conj}
(-1)^{\operatorname{rk}_l A} \overset{?}{=} w(A).
\end{equation}
\end{pp}

\begin{pp}[Previous investigations] {\label}{prev}
In the elliptic curve case, the $l$-parity conjecture was proved for $K = {\mathbb{Q}}$ in \cite{DD10}*{Thm.~1.4} (see \cite{Mon96}, \cite{Nek06}*{\S0.17}, \cite{Kim07} for some preceding work), for totally real number fields $K$ excluding some cases of potential complex multiplication in \cite{Nek13}*{Thm.~A}, for general number fields $K$ and curves with an $l$-isogeny in \cite{Ces12}*{Thm.~1.4} (building on \cite{DD08a}*{Thm.~2}, \cite{DD11}*{Cor.~5.8}, and \cite{CFKS10}*{proof of Thm.~2.1}), and for global fields of characteristic $l$ with $l > 3$ in \cite{TW11}*{Thm.~1}. For abelian varieties of higher dimension, the $l$-parity conjecture is largely open, but see \cite{CFKS10}*{Thm.~2.1} for a case of a suitable isogeny over number fields.
\end{pp}

\begin{pp}[The function field case] {\label}{ff-case}
Assume that $\operatorname{char} K = p$ with $p > 0$. Then the meromorphic continuation of $L(A, s)$ is known \cite{Sch82}*{Bemerkung 2) auf Seite 497}, as is the functional equation \eqref{fe} \cite{Del73}*{8.10~et~9.3}\footnote{Note the misprints: $pt{^{-1}}$ should be $(pt){^{-1}}$ in \cite{Del73}*{7.11 (iii), 7.11.8 et l'\'{e}quation suivante, 9.3.1, 9.9.1, 9.9.2}. Also note the need of including in $L(A, s)$ the factors contributed by the conductor to get the claimed form of \eqref{fe}.}. Moreover, as observed in \cite{TW11}*{Prop.~3}, \cite{KT03} proves $\operatorname{rk}_l A \le \operatorname{ord}_{s = 1} L(A, s)$ for every $l$ (for $l \neq p$, already \cite{Sch82}*{Lemma~2~i)} suffices). Also, due to \cite{KT03} (which extends the results of \cite{Sch82}), the equality in $\operatorname{rk} A \le \operatorname{rk}_l A$ for a single $l$ implies the equality for every $l$ along with the full Birch and Swinnerton-Dyer conjecture for $A$. Consequently, $\operatorname{rk}_l A = \operatorname{ord}_{s = 1} L(A, s)$ if $\operatorname{ord}_{s = 1} L(A, s) \le 1$, so the $l$-parity conjecture holds in this case. Knowing it in general amounts to knowing that the nonnegative integer $\operatorname{ord}_{s = 1} L(A, s) - \operatorname{rk}_l A$, which is expected to be $0$, is at least even. We prove this in most cases over the constant quadratic extension:
\end{pp}

\begin{thm} {\label}{main}
Let $K$ be a global function field of characteristic $p$, let ${\mathbb{F}}_q$ be its field of constants, let $l$ be a prime different from $p$, and let $A$ be an abelian variety over $K$. Suppose that there is an $n \ge 1$ such that $A_{K{\mathbb{F}}_{q^n}}$ is isogenous to an abelian variety $A{^{\prime}}$ that acquires a polarization of degree prime to $l$ over an odd degree Galois extension $K{^{\prime}}/K{\mathbb{F}}_{q^n}$ and, if $l = 2$, that the orders of the component groups of all the reductions of $A{^{\prime}}_{K{^{\prime}}}$ are odd. Then the $l$-parity conjecture holds for $A_{K{\mathbb{F}}_{q^2}}$.
\end{thm}

When referring to \Cref{main}, we adopt further notation: $S$ is the connected smooth proper curve over ${\mathbb{F}}_q$ having $K$ as its function field, ${\mathcal{A}} {\rightarrow} S$ and ${\mathcal{A}}^\vee {\rightarrow} S$ are the global N\'{e}ron models of $A$ and its dual abelian variety $A^\vee$, and, for a closed point $v\in S$, the component group of ${\mathcal{A}}_v$ is $\Phi_v {\colonequals} {\mathcal{A}}_v/{\mathcal{A}}_v^0$.

{\begin{rems} \hfill \begin{enumerate}[label={\textbf}{\thesubsection.},ref=\thesubsection]}
{\addtocounter{subsection}{1} \item} 
The restriction on $A$ for odd $l$ is mild: abelian varieties to which \Cref{main} applies for such $l$ include the principally polarized ones, in particular, Jacobians of smooth proper geometrically connected curves, and hence also elliptic curves.

{\addtocounter{subsection}{1} \item}
The component group condition for $l = 2$ concerns orders as finite \'{e}tale group schemes, i.e., it is more restrictive than insisting that the local Tamagawa factors of $A{^{\prime}}_{K{^{\prime}}}$ be odd. It holds at least if $A{^{\prime}}_{K{^{\prime}}}$ has everywhere good reduction. \Cref{l=2} isolates the difficulty in removing it.

{\addtocounter{subsection}{1} \item}
Every abelian variety over a separably closed field is isogenous to a principally polarized one. Thus, for an abelian variety $A$ over a global function field $K$, there is a finite separable extension $K{^{\prime}}/K$ such that \Cref{main} for odd $l$ applies to $A_{K{^{\prime\prime}}}$ for every finite extension~$K{^{\prime\prime}}/K{^{\prime}}$.

{\addtocounter{subsection}{1} \item}
After this paper was completed, Fabien Trihan informed us about the upcoming publication of \cite{TY14}, which will resolve the $l$-parity conjecture for all primes $l$ and all abelian varieties over global function fields. Our methods differ from those of op.~cit., and, although ours lead to a weaker result on the $l$-parity conjecture, it seems worthwhile to present them here: for instance, \Cref{st-root-ann,dual} are general results about the arithmetic of abelian varieties over global fields that seem useful beyond their role in the proof of \Cref{main}.
{\end{enumerate} \end{rems}}

Regarding the $l$-parity conjecture over $K$, we obtain

\begin{thm} {\label}{over-K} 
In \Cref{main}, if in addition the analytic rank of $A$ grows by at most one in $K{\mathbb{F}}_{q^2}/K$, then the $l$-parity conjecture holds for $A$.
\end{thm}

\begin{proof}
Since \Cref{main} proves the $l$-parity conjecture for $A_{K{\mathbb{F}}_{q^{2}}}$, \cite{DD09b}*{A.2 (2)} and \Cref{DD-app} permit passage to the quadratic twist $B$ of $A$ by $K{\mathbb{F}}_{q^{ 2}}/K$, for which the $l$-parity conjecture holds because $\operatorname{ord}_{s = 1} L(B, s) \le 1$, see \S\ref{ff-case}. 
\end{proof}

\begin{rem}
As mentioned in \S\ref{prev}, Trihan and Wuthrich have also considered the function field case of the $l$-parity conjecture in \cite{TW11}. One of their main results is a special case of \Cref{over-K}, which they prove for elliptic curves assuming in addition that, among other things, $l > 2$ and $p > 3$. 
\end{rem}

Before proceeding to outline the proof of \Cref{main} in \S\ref{outline}, we record an auxiliary 

\begin{thm}[\Cref{st-root}]{\label}{st-root-ann}
For an abelian variety $B$ over a nonarchimedean local field $k$, let $B_{k_n}$ and $a(B)$ be its base change to a degree $n$ unramified extension and conductor exponent, respectively. The local root number satisfies
\[
w(B_{k_n}) = \begin{cases} w(B),\quad \quad\  \text{ if $n$ is odd,} \\  (-1)^{a(B)},\quad \text{ if $n$ is even.}  \end{cases}
\]
\end{thm}

\begin{rem}
If $F{^{\prime}}/F$ is an everywhere unramified Galois extension of even degree of global fields, then, according to \Cref{st-root-ann}, for an abelian variety $B$ over $F$ one has
\begin{equation} {\label}{even-glob}
w(B_{F{^{\prime}}}) = (-1)^{\sum_{v\nmid \infty} a(B_{F{^{\prime}}_v})},\quad\quad \text{where $v$ runs over the finite places of $F{^{\prime}}$.}
\end{equation}
Combined with the parity conjecture, \eqref{even-glob} predicts that $\operatorname{rk} B_{F{^{\prime}}} \equiv \sum_{v\nmid \infty} a(B_{F{^{\prime}}_v}) \bmod 2$. Equation \eqref{even-glob} is also the main reason for the passage to $K{\mathbb{F}}_{q^2}$ in \Cref{main}: over $K$ itself, even with the explicit formulas in the elliptic curve case, the local root numbers are difficult to handle.
\end{rem}

\begin{pp}[An overview of the proof of \Cref{main}] {\label}{outline}
The main idea is to interpret the $l$-Selmer group as an \'{e}tale cohomology group following \cite{Ces13c}, use the Grothendieck--Ogg--Shafarevich formula together with the Hochschild--Serre spectral sequence to express the $l$-Selmer parity as a sum of local terms, and then compare place by place with the expression of the global root number as a product of local root numbers. Although the key computation contained in \S\ref{GOS-sect} is short, additional arguments are needed to augment it to a complete proof:
\begin{itemize}
\item
Due to possibly nontrivial Galois action, the Hochschild--Serre spectral sequence relates the \'{e}tale cohomological $l$-Selmer group to the Grothendieck--Ogg--Shafarevich formula only after a large constant extension. As we explain in \S\ref{self-dual}, the self-duality of Galois representations furnished by $l{^{\infty}}$-Selmer groups descends the $l$-parity conclusion to $K{\mathbb{F}}_{q^2}$.

\item
For the main idea to be relevant, in \S\ref{sel-sel} we reformulate the $l$-parity conjecture in terms of $l$-Selmer rather than $l{^{\infty}}$-Selmer groups. Special care is needed if $l = 2$, since Shafarevich--Tate groups need not be of square order even when they are finite and $A$ is principally polarized. 

\item
In the presence of local Tamagawa factors divisible by $l$, the $l$-Selmer group may differ from its \'{e}tale cohomological counterpart. Arithmetic duality results proved in \S\S\ref{ad-pre}--\ref{ade} control this difference modulo $2$ through \Cref{dual}. The $l = 2$ case again leads to complications due to the difference between alternating and skew-symmetric pairings in characteristic $2$.

\item
The comparison of the Grothendieck--Ogg--Shafarevich local terms and root numbers is possible due to \Cref{st-root-ann}, which emerged from the structure of the overall proof of \Cref{main} and, along with related local results, is treated in Appendix~\ref{app}.
\end{itemize}

Results that are auxiliary for the proof of \Cref{main} are recorded in the course of the discussion in various sections and then used in \S\ref{final} to finish the proof.
\end{pp}

\begin{pp}[Notation]{\label}{not}
For a field $F$, its algebraic closure is denoted by ${\overline}{F}$;  when needed (e.g., for forming composita), the choice of ${\overline}{F}$ is made implicitly and compatibly with overfields. If $F$ is a global field and $v$ is its place, then $F_v$ denotes the corresponding completion; if $v$ is finite, then ${\mathcal{O}}_v$ and ${\mathbb{F}}_v$ denote its ring of integers and residue field. For a prime $l$ and a torsion abelian group $G$, we denote by $G[l{^{\infty}}]$ and $G_{{\mathrm{nd}}}$ its subgroup consisting of all the elements of $l$-power order and quotient by the maximal divisible subgroup, respectively. In the $G = G[l{^{\infty}}]$ case, we say that $G$ is \emph{${\mathbb{Z}}_l$-cofinite} if $\operatorname{Hom}(G, {\mathbb{Q}}_l/{\mathbb{Z}}_l)$ is finitely generated as a ${\mathbb{Z}}_l$-module. For a prime $l$ and an abelian variety $B$ over a global field, the notation $\operatorname{rk} B$, $w(B)$, $\operatorname{rk}_l B$, $\operatorname{Sel}_{l{^{\infty}}} B$, $\Sha(B)$ introduced in \S\ref{l-par} remains in place throughout. If $B$ is an abelian variety over a local field, then $w(B)$ denotes the \emph{local} root number instead. The dual abelian variety is denoted by $B^\vee$. In the sequel, we typically write $A$ for the protagonist of \Cref{main} and use $B$ for an abelian variety in other settings. All the representations that we consider are finite dimensional.
\end{pp}

\subsection*{Acknowledgements}

I thank Bjorn Poonen for many helpful conversations and suggestions. I thank Tim Dokchitser, Vladimir Dokchitser, Fabien Trihan, and Christian Wuthrich for useful correspondence. I thank Padmavathi Srinivasan for a helpful discussion.

\section{Self-duality of $l{^{\infty}}$-Selmer groups and preliminary reductions} {\label}{self-dual}

The goal of the section is \Cref{red1}, which reduces \Cref{main} to its special case. For this reduction, we extend to global fields several results whose proofs in the number field case have been given by T.~and V.~Dokchitser; their arguments require only mild modifications, the key step being \Cref{DD}. Trihan and Wuthrich have also observed in \cite{TW11}*{proof of Prop.~4} that extensions of this sort are possible, but it seems worthwhile to indicate here the necessary changes to the proofs.

\begin{pp}[Pontryagin duals of $l{^{\infty}}$-Selmer groups] {\label}{Xl}
For an integer $m \ge 1$ and an abelian variety $B$ over a global field $F$, the finiteness of the $m$-Selmer group\footnote{The less well-known $\operatorname{char} F \mid m$ case of this finiteness follows from \cite{Mil70} and the Mordell--Weil theorem.} $\operatorname{Sel}_m B$ entails that of $\Sha(B)[m]$. Consequently, $\Sha(B)[l{^{\infty}}]$ is ${\mathbb{Z}}_l$-cofinite for every prime $l$; combining this with the Mordell--Weil theorem, we see that so is $\operatorname{Sel}_{l{^{\infty}}} B$, whose ${\mathbb{Z}}_l$-corank, $\operatorname{rk}_l B$, is also the ${\mathbb{Q}}_l$-dimension of 
\[
{\mathcal{X}}_l(B) {\colonequals} \operatorname{Hom}(\operatorname{Sel}_{l{^{\infty}}} B, {\mathbb{Q}}_l/{\mathbb{Z}}_l) {\otimes}_{{\mathbb{Z}}_l} {\mathbb{Q}}_l.
\]
If $F{^{\prime}}/F$ is finite Galois, then ${\mathcal{X}}_l(B_{F{^{\prime}}})$ is a representation of $G {\colonequals} \operatorname{Gal}(F{^{\prime}}/F)$ over ${\mathbb{Q}}_l$.

If $\Sha(B)[l{^{\infty}}]$ is finite, then ${\mathcal{X}}_l(B) \cong B(F) {\otimes}_{\mathbb{Z}} {\mathbb{Q}}_l$ and most results of this section become obvious. Their utility is in bypassing such finiteness assumptions in the proofs of further results.
\end{pp}

For proximity to \cite{DD09c}*{Thm.~2.1}, in \Cref{DD} we temporarily deviate from the notation~above.

\begin{thm} {\label}{DD}
Fix a prime $p$ and an abelian variety $A$ over a global field $K$. If $p \neq \operatorname{char} K$, then for every finite Galois $F/K$ the representation ${\mathcal{X}}_p(A_{F})$ of $G {\colonequals} \operatorname{Gal}(F/K)$ over ${\mathbb{Q}}_p$ is self-dual.
\end{thm}

\begin{proof}
The proof given by T.~and V.~Dokchitser for the corresponding statement in the number field case, \cite{DD09c}*{Thm.~2.1}, requires only minor modifications that we indicate below.

The argument of loc.~cit.~is based on \cite{DD10}*{Thm.~4.3}, which is also written in the number field context but extends to global fields with the same proof as long as one restricts to isogenies of degree prime to $\operatorname{char} K$. This restriction is needed for \cite{Mil06}*{I.(7.3.1)} to apply,\footnote{Implicitly, neither the statement nor the proof of \cite{Mil06}*{I.(7.3.1)} uses the assumption that the involved Shafarevich--Tate groups are finite; the proof of \cite{DD10}*{Thm.~4.3}, as well as the modification here, relies on this.} but not for \cite{DD10}*{Lemma 4.2}, whose proof continues to give the same conclusions for arbitrary isogenies between abelian varieties over global fields. In particular, the claims in the second paragraph of the proof of \cite{DD09c}*{Thm.~2.1} about the multiplicativity and the $p$-part of $Q(f)$ hold in this generality.

With the $p \neq \operatorname{char} K$ restriction, the proof of \cite{DD09c}*{Thm.~2.3} carries over as follows. Since $A$ and $A^\vee$ are isogenous, ${\mathcal{X}}_p(A_F) \cong {\mathcal{X}}_p(A^\vee_F)$ as ${\mathbb{Q}}_p[G]$-modules, so Zarhin's trick reduces to the principally polarized case, in which, with the above extension of \cite{DD10}*{Thm.~4.3} at hand, the only necessary modification is to make sure that $\det \phi$, and hence also $\deg f_{\phi}$, is prime to $\operatorname{char} K$. This translates into a condition on the coefficients of $\Phi$ modulo $\operatorname{char} K$, which can be satisfied simultaneously with the $p$-adic requirements because $p \neq \operatorname{char} K$.

As long as $p \neq \operatorname{char} K$, the proof of \cite{DD09c}*{Lemma 2.4} requires no modification.
\end{proof}

\begin{rem}
\Cref{DD} without the $p \neq \operatorname{char} K$ assumption and the extension of \cite{DD10}*{Thm.~4.3} to global fields are desirable results. The latter together with a modification of the proof of \cite{DD09c}*{Lemma 2.4} implies the former and follows from its current argument as long as one acquires the \cite{Mil06}*{I.(7.3.1)} input by proving that, in the notation of \cite{DD10}*{Thm.~4.3},
\[
\f{C(X/K)}{C(Y/K)}\cdot \f{\Omega_X}{\Omega_Y} = \f{z(\phi(K))}{z(\phi^t(K))} \cdot \f{\# \Sha[\phi^t]}{\#\Sha[\phi]}\quad\text{ for all $\phi$ and global fields $K$.}
\]
\end{rem}

\begin{lemma} {\label}{sel-gal}
In the setup of \S\ref{Xl}, the map ${\mathcal{X}}_l(B_{F{^{\prime}}}) {\rightarrow} {\mathcal{X}}_l(B)$ induces the second isomorphism in
\[
{\mathcal{X}}_l(B_{F{^{\prime}}})^G \cong {\mathcal{X}}_l(B_{F{^{\prime}}})_G \cong {\mathcal{X}}_l(B).
\]
\end{lemma}

\begin{proof}
The proof for elliptic curves and number fields, \cite{DD10}*{proof of Lemma 4.14}, extends: the spectral sequence $H^i(G, H^j_{\mathrm{fppf}}(F{^{\prime}}, B[l{^{\infty}}])) {\Rightarrow} H^{i + j}_{\mathrm{fppf}}(F, B[l{^{\infty}}])$ shows that $\#G$ kills the kernel and the cokernel of $H^1_{\mathrm{fppf}}(F, B[l{^{\infty}}]) {\rightarrow} H^1_{\mathrm{fppf}}(F{^{\prime}}, B[l{^{\infty}}])^G$. Hence it kills the kernel of $\operatorname{Sel}_{l{^{\infty}}} B {\rightarrow}(\operatorname{Sel}_{l{^{\infty}}} B_{F{^{\prime}}})^G$ as well, whereas $(\#G)^2$ kills the cokernel because $\#G$ kills $\operatorname{Ker}(H^1(F_v, B) {\rightarrow} H^1(F{^{\prime}}_{v{^{\prime}}}, B))$ for all places $v{^{\prime}}$ of $F{^{\prime}}$ extending a place $v$ of $F$. It remains to pass to Pontryagin duals and invert $l$.
\end{proof}

\begin{prop} {\label}{odd-gal}
For an odd degree Galois extension $F{^{\prime}}/F$ of global fields, an abelian variety $B$ over $F$, and a prime $l$ different from $\operatorname{char} F$, one has
{\begin{enumerate}[label={(\alph*)}]}
\item $ \operatorname{rk}_l B \equiv \operatorname{rk}_l B_{F{^{\prime}}} \bmod 2$, and

\item $w(B) = w(B_{F{^{\prime}}})$.
\end{enumerate}
In particular, the $l$-parity conjecture holds for $B$ if and only if it holds for $B_{F{^{\prime}}}$.
\end{prop}

\begin{proof} \hfill
{\begin{enumerate}[label={(\alph*)}]}
\item Combine the argument of \cite{DD09c}*{Cor.~2.5} with \Cref{DD} and \Cref{sel-gal}.

\item The number field proof \cite{DD09b}*{A.2 (3)} also works for global fieds. \qedhere  
\end{enumerate}
\end{proof}

\begin{rem} {\label}{DD-app}
Thanks to \Cref{DD} and \Cref{sel-gal}, all the results of \cite{DD09b}*{Appendix A} with the possible exception of \cite{DD09b}*{A.2 (5)} extend to global fields under the $p \neq \operatorname{char} K$ restriction inherited from \Cref{DD}; the same applies to \cite{DD09c}*{Cor.~2.7}. We have excluded \cite{DD09b}*{A.2~(5)} due to its reliance on \cite{Roh11a}, which is written in characteristic $0$ context.
\end{rem}

In the remainder of \S\ref{self-dual}, we discuss the relevance of the results above to \Cref{main}.

\begin{cor} {\label}{any-n}
In \Cref{main}, for every even $n$,
{\begin{enumerate}[label={(\alph*)}]}
\item {\label}{any-n-a}
$\operatorname{rk}_l A_{K{\mathbb{F}}_{q^{n}}} \equiv \operatorname{rk}_l A_{K{\mathbb{F}}_{q^2}} \bmod 2$, and

\item {\label}{any-n-b} 
$w(A_{K{\mathbb{F}}_{q^n}}) = w(A_{K{\mathbb{F}}_{q^2}})$.
\end{enumerate} 
Thus, to prove \Cref{main} it suffices to prove the $l$-parity conjecture for $A_{K{\mathbb{F}}_{q^{n}}}$ for some even $n$.
\end{cor}

\begin{proof} \hfill
{\begin{enumerate}[label={(\alph*)}]}
\item Due to \Cref{DD}, for a $1$-dimensional character $\chi$ of $\operatorname{Gal}(K{\mathbb{F}}_{q^n}/K)$, the $\chi$-isotypic and $\chi{^{-1}}$-isotypic components of ${\mathcal{X}}_l(A_{K{\mathbb{F}}_{q^n}}) {\otimes}_{{\mathbb{Q}}_l} {{\overline}{\mathbb{Q}}_l}$ have the same dimension. Therefore, the sum over all $\chi$ with $\chi^2 \neq 1$ of $\chi$-isotypic components is even dimensional, and it remains to note that by \Cref{sel-gal} the sum over all $\chi$ with $\chi^2 = 1$ of such components is ${\mathcal{X}}_l(A_{K{\mathbb{F}}_{q^2}}){\otimes}_{{\mathbb{Q}}_l} {{\overline}{\mathbb{Q}}_l}$.

\item Combine \cite{DD09b}*{A.2 (1) and (2)} with \Cref{DD-app}. \qedhere
\end{enumerate}
\end{proof}

\begin{rem}
\Cref{any-n} \ref{any-n-b} can also be deduced from \Cref{st-root}.
\end{rem}

\begin{cor} {\label}{red1}
To prove \Cref{main}, it suffices to prove the $l$-parity conjecture for $A_{K{\mathbb{F}}_{q^n}}$ for some even $n$ under the additional assumptions that $A$ has a polarization of degree prime to $l$, the $\operatorname{Gal}({\overline}{\mathbb{F}}_q/{\mathbb{F}}_q)$-action on $H^1_{\mathrm{\acute{e}t}}(S_{{\overline}{\mathbb{F}}_q}, {\mathcal{A}}[l])$ is trivial, $\Phi_v({\mathbb{F}}_v) = \Phi_v({\overline}{\mathbb{F}}_v)$ for every place $v$ of $K$, and, if $l = 2$, that also every $\#\Phi_v({\mathbb{F}}_v)$ is odd.
\end{cor}

\begin{proof}
Using \Cref{any-n} and the isogeny invariance of the $l$-parity conjecture, in \Cref{main} we replace $K$ by $K{\mathbb{F}}_{q^n}$ and $A$ by $A{^{\prime}}$ to reduce to the case when $A$ acquires a polarization of degree prime to $l$ over an odd degree Galois extension $K{^{\prime}}/K$. \Cref{odd-gal} then allows us to pass to $K{^{\prime}}$ to get the desired polarization; a further constant extension gives the remaining desiderata.
\end{proof}

\section{Replacing $l{^{\infty}}$-Selmer groups by $l$-Selmer groups} {\label}{sel-sel}

To facilitate the Grothendieck--Ogg--Shafarevich input to our proof of \Cref{main}, in \Cref{rk-sel} we (implicitly) reformulate the $l$-parity conjecture by relating the $l{^{\infty}}$-Selmer rank and the $l$-Selmer rank. A suitable polarization is handy for this---without it, controlling the parity of $\dim_{{\mathbb{F}}_l} \Sha(A)_{{\mathrm{nd}}}[l]$ would become a major concern. In fact, even with it, this parity may vary in the $l = 2$ case, as Poonen and Stoll explain in \cite{PS99}. As far as the proof of \Cref{main} is concerned, the goal of \Cref{PS} and \Cref{CT-BC} below is to overcome this difficulty by proving that the said parity is even over every quadratic extension. \Cref{PS} is a slight improvement to the main result of op.~cit. Without this improvement, in the $l = 2$ case of \Cref{main} we would be forced to restrict to principally polarized abelian varieties, which were the main focus of Poonen and Stoll.

\begin{pp}[The Cassels--Tate pairing] {\label}{CT}
For an abelian variety $B$ over a global field $F$, let 
\[
\langle\ ,\,\rangle\colon \Sha(B) \times \Sha(B^\vee) {\rightarrow} {\mathbb{Q}}/{\mathbb{Z}}
\]
be the Cassels--Tate bilinear pairing. 
For a self-dual homomorphism $\lambda\colon B {\rightarrow} B^\vee$, the pairing 
\[
\langle a, b\rangle_{\lambda} {\colonequals} \langle a,\lambda(b)\rangle\quad\text{  for $a, b \in \Sha(B)$} 
\]
is anti-symmetric \cite{PS99}*{\S6, Cor.~6}. Therefore, if $\lambda$ is in addition an isogeny, then the pairing induced by $\langle\ , \, \rangle_{\lambda}$ on $\Sha(B)[p{^{\infty}}]$ is alternating for every odd prime $p$ not dividing $\deg \lambda$. In this case, since $\In{\ ,\, }$ is nondegenerate modulo the divisible subgroups,\footnote{The first published complete proof of the fact that the left and right kernels of $\In{\ ,\, }$ are the maximal divisible subgroups of $\Sha(B)$ and $\Sha(B^\vee)$ seems to be the combination of \cite{HS05}*{Thm.~0.2}, \cite{HS05e}, and \cite{GA09}*{Thm.~1.2} (although loc.~cit.~neglects the prime to the characteristic torsion subgroups, the MathSciNet review of op.~cit.~remarks that the proof of \cite{HS05}*{Thm.~0.2} extends and yields the corresponding claim for these subgroups).} for every $p\nmid 2\deg {\lambda}$ we have
\begin{equation}{\label}{odd-sha}
\dim_{{\mathbb{F}}_p} \Sha(B)_{{\mathrm{nd}}}[p] \equiv 0 \bmod 2.
\end{equation}
If ${\lambda}$ is a self-dual isogeny of odd degree, then the validity of \eqref{odd-sha} for $p = 2$ is governed by $\In{c, c}_{\lambda} \in \{ 0, \f{1}{2}\}$, where $c = (\Sha(\lambda)[2]){^{-1}}( c_{\lambda}) \in \Sha(B)[2]$ and $c_{\lambda} \in \Sha(B^\vee)[2] \subset H^1(k, B^\vee)[2]$ is the image of $\lambda \in (\operatorname{NS} B)(F)$ as in \cite{PS99}*{\S4, Cor.~2}. Poonen and Stoll observed this in \cite{PS99}*{\S6,~Thm.~8} when $\lambda$ is a principal polarization, and the general case follows from their argument:
\end{pp}

\begin{thm} {\label}{PS}
In the setup of \S\ref{CT} with a self-dual isogeny $\lambda$ of odd degree $d$, the claims of \cite{PS99}*{\S6,~Thm.~8} continue to hold if everywhere in loc.~cit.~one replaces $\Sha_{{\mathrm{nd}}}$ by its subgroup $\Sha_{{\mathrm{nd}}}(d{^{\prime}})$ consisting of the elements of order prime to $d$. In particular, 
\[
\dim_{{\mathbb{F}}_2} \Sha(B)_{{\mathrm{nd}}}[2] \equiv 0 \bmod 2\quad \text{ if and only if }\quad \In{c, c}_{\lambda} = 0. 
\]
\end{thm}

\begin{proof}
The only necessary change to the proof of loc.~cit.~is the indicated replacement.
\end{proof}

\begin{rem}
Under the assumptions of \Cref{PS}, \cite{PS99}*{\S6, Cor.~9} continues to hold, too, with the same modification: one replaces $\Sha$ and $\Sha_{{\mathrm{nd}}}$ by $\Sha(d{^{\prime}})$ and $\Sha_{{\mathrm{nd}}}(d{^{\prime}})$ throughout.
\end{rem}

\begin{lemma} {\label}{CT-BC}
For an extension $F{^{\prime}}/F$ of global fields, there is a commutative diagram
\[
\xymatrix{
\Sha(B) \times \Sha (B^\vee) \ar[d]^-\operatorname{Res} \ar[r]^-{\langle\, ,\, \rangle} & {\mathbb{Q}}/{\mathbb{Z}} \ar[d]^-{[F{^{\prime}} : F]} \\
\Sha(B_{F{^{\prime}}}) \times \Sha (B_{F{^{\prime}}}^\vee) \ar[r]^-{\langle\, ,\, \rangle} & {\mathbb{Q}}/{\mathbb{Z}}.
}
\]
\end{lemma}

\begin{proof}
Combine the definition of the pairings, \cite{PS99}*{\S3.1}, with the well-known commutativity of
\[
\xymatrix{
\operatorname{Br}(F_v) \ar[d]_\operatorname{Res}\ar[r]^-{\operatorname{inv}_v} & {\mathbb{Q}}/{\mathbb{Z}} \ar[d]^-{[F_{v{^{\prime}}}{^{\prime}} : F_v]} \\
\operatorname{Br}(F{^{\prime}}_{v{^{\prime}}})  \ar[r]^-{\operatorname{inv}_{v{^{\prime}}}} & {\mathbb{Q}}/{\mathbb{Z}}
}
\]
for an extension $F{^{\prime}}_{v{^{\prime}}}/F_v$ of local fields.
\end{proof}

\begin{rem}
Bjorn Poonen told us \Cref{CT-BC}, which has also been observed by Adam~Morgan.
\end{rem}

\begin{prop} {\label}{rk-sel}
In \Cref{main}, if $A$ has a polarization of degree prime to $l$, then
\[
\operatorname{rk}_l A_{K{\mathbb{F}}_{q^2}} \equiv \dim_{{\mathbb{F}}_l} \operatorname{Sel}_l A_{K{\mathbb{F}}_{q^2}} - \dim_{{\mathbb{F}}_l} A[l](K{\mathbb{F}}_{q^2}) \bmod 2.
\]
\end{prop}

\begin{proof}
For any prime $l$ and an abelian variety $B$ over a global field $F$, one has
\[
\operatorname{rk}_l B = \operatorname{rk} B + \dim_{{\mathbb{F}}_l} \Sha(B)[l] - \dim_{{\mathbb{F}}_l} \Sha(B)_{{\mathrm{nd}}}[l] = \dim_{{\mathbb{F}}_l} \operatorname{Sel}_l B - \dim_{{\mathbb{F}}_l} B[l](F) - \dim_{{\mathbb{F}}_l} \Sha(B)_{{\mathrm{nd}}}[l].
\]
For $A_{K{\mathbb{F}}_{q^2}}$, in addition, $\dim_{{\mathbb{F}}_l} \Sha(A_{K{\mathbb{F}}_{q^2}})_{{\mathrm{nd}}}[l]$ is even by \eqref{odd-sha}, \Cref{PS}, and \Cref{CT-BC}.
\end{proof}

\begin{rem}
No generality is gained by requiring a self-dual isogeny instead of a polarization in \Cref{rk-sel} (or \Cref{main}): for $n \in {\mathbb{Z}}_{> 0}$, if an abelian variety $B$ over a field $F$ has a self-dual isogeny  of degree prime to $n$, then it also has a polarization of degree prime to $n$. Indeed, one knows that the degree function\footnote{The degree of a self-dual homomorphism that is not an isogeny is defined to be $0$.} $\deg\colon (\operatorname{NS} B)(F) {\rightarrow} {\mathbb{Z}}$ is a polynomial with rational coefficients on the lattice $(\operatorname{NS} B)(F)$; 
consequently, $\deg$ modulo $n$ is translation-invariant with respect to a sublattice, and it remains to note that the cone of polarizations spans $(\operatorname{NS} B)(F)$.
\end{rem}

\section{The Grothendieck--Ogg--Shafarevich argument}{\label}{GOS-sect}

Equation \eqref{GOS-eq}, which follows from the Grothendieck--Ogg--Shafarevich theorem, lies at the heart of our proof of \Cref{main}. Its link to the $l$-parity conjecture is this: the right hand side has to do with root numbers thanks to \Cref{st-root}, whereas $H^1_{\mathrm{\acute{e}t}}(S, {\mathcal{A}}[l])$ is intimately related to $\operatorname{Sel}_l A$ due to the results of \cite{Ces13c}.

\begin{prop}{\label}{GOS}
In \Cref{main}, suppose that there is an isogeny $A {\rightarrow} A^\vee$ of degree prime to $l$ and the $\operatorname{Gal}({\overline}{\mathbb{F}}_q/{\mathbb{F}}_q)$-action on $H^1_{\mathrm{\acute{e}t}}(S_{{\overline}{\mathbb{F}}_q}, {\mathcal{A}}[l])$ is trivial. Letting $v$ range over the places of $K$, we have
\begin{equation}{\label}{GOS-eq}
\dim_{{\mathbb{F}}_l} H^1_{\mathrm{\acute{e}t}}(S, {\mathcal{A}}[l]) - \dim_{{\mathbb{F}}_l} H^0_{\mathrm{\acute{e}t}}(S, {\mathcal{A}}[l])\quad\quad \equiv \sum_{v\text{ inert in } K{\mathbb{F}}_{q^2}} a(A[l]_{K_v}) \mod 2,
\end{equation}
where the Artin conductors $a(A[l]_{K_v})$ are defined by \eqref{art-cond} (with $k = K_v$ and $N = 0$).
\end{prop}

\begin{proof}
Since ${\mathbb{F}}_q$ is of cohomological dimension $1$, the Hochschild--Serre spectral sequence
\[
E_2^{ij} = H^i(\operatorname{Gal}({\overline}{\mathbb{F}}_q/{\mathbb{F}}_q), H^j_{\mathrm{\acute{e}t}}(S_{{\overline}{\mathbb{F}}_q}, {\mathcal{A}}[l])) {\Rightarrow} H^{i + j}_{\mathrm{\acute{e}t}}(S, {\mathcal{A}}[l])
\]
degenerates on the $E_2$-page, and hence gives rise to a short exact sequence
\[
0 {\rightarrow} H^1(\operatorname{Gal}({\overline}{\mathbb{F}}_q/{\mathbb{F}}_q), A[l](K{\overline}{\mathbb{F}}_q)) {\rightarrow} H^1_{\mathrm{\acute{e}t}}(S, {\mathcal{A}}[l]) {\rightarrow} H^1_{\mathrm{\acute{e}t}}(S_{{\overline}{\mathbb{F}}_q}, {\mathcal{A}}[l]) {\rightarrow} 0.
\]
Moreover, $\dim_{{\mathbb{F}}_l} H^1(\operatorname{Gal}({\overline}{\mathbb{F}}_q/{\mathbb{F}}_q), A[l](K{\overline}{\mathbb{F}}_q)) = \dim_{{\mathbb{F}}_l} H^0(\operatorname{Gal}({\overline}{\mathbb{F}}_q/{\mathbb{F}}_q), A[l](K{\overline}{\mathbb{F}}_q)) = \dim_{{\mathbb{F}}_l} A[l](K)$, so the left hand side of \eqref{GOS-eq} equals $\dim_{{\mathbb{F}}_l} H^1_{\mathrm{\acute{e}t}}(S_{{\overline}{\mathbb{F}}_q}, {\mathcal{A}}[l])$. 

The Grothendieck--Ogg--Shafarevich formula \cite{Ray65}*{Thm.~1} gives
\begin{equation} {\label}{GOS-quot}
\sum_{i = 0}^2  \dim_{{\mathbb{F}}_l} H^i_{\mathrm{\acute{e}t}}(S_{{\overline}{\mathbb{F}}_q}, {\mathcal{A}}[l])\quad\quad \equiv \sum_{\text{closed points }s \in S_{{\overline}{\mathbb{F}}_q}} a(A[l]_{K_{v(s)}}) \mod 2,
\end{equation}
where $v(s)$ is the place of $K$ lying below $s$ and the right hand side local terms (which at first sight differ from those of loc.~cit.) are explained by
\begin{enumerate}[label={(\arabic*)}]
\item {\label}{one}
Since $S_{{\overline}{\mathbb{F}}_q} {\rightarrow} S$ is pro-(finite \'{e}tale), ${\mathcal{A}}[l]_{{\overline}{\mathbb{F}}_q} {\rightarrow} S_{{\overline}{\mathbb{F}}_q}$ inherits the N\'{e}ron property from ${\mathcal{A}}[l] {\rightarrow} S$ (see \cite{Ces13c}*{B.5} for the N\'{e}ron property of the latter). Hence, \cite{Ray65}*{(1 ter)} identifies the original local terms as Artin~conductors. 
\end{enumerate}

To arrive at \eqref{GOS-eq} it remains to analyze \eqref{GOS-quot} further:

\begin{enumerate}[label={(\arabic*)}] \addtocounter{enumi}{1}
\item
The right hand side local terms that arise from a fixed place $v$ all equal $a(A[l]_{K_v})$ and there is an even number of them if and only if $v$ is split in $K{\mathbb{F}}_{q^2}$.

\item {\label}{three}
Let $j\colon U {\hookrightarrow} S_{{\overline}{\mathbb{F}}_q}$ be a nonempty open subscheme for which ${\mathcal{A}}_U$ is an abelian scheme. On the small \'{e}tale sites, $j_* ({\mathcal{A}}[l]_U) \cong {\mathcal{A}}[l]_{S_{{\overline}{\mathbb{F}}_q}}$ due to the N\'{e}ron property of the latter, and similarly $j_* ({\mathcal{A}}^\vee[l]_U)  \cong {\mathcal{A}}^\vee[l]_{S_{{\overline}{\mathbb{F}}_q}}$. By Cartier--Nishi duality \cite{Oda69}*{Thm.~1.1}, ${\mathcal{A}}[l]_U$ and ${\mathcal{A}}^\vee[l]_U$ are Cartier dual, so $H^2(S_{{\overline}{\mathbb{F}}_q}, {\mathcal{A}}[l])^* \cong A^\vee[l](K{\overline}{\mathbb{F}}_q)$ by \cite{Mil80}*{V.2.2 (b)}\footnote{Loc.~cit.~rests on \cite{Mil80}*{V.2.1 (b)}, Step $1$ of the proof of which uses \cite{Mil80}*{V.1.14}, which relies on the proof of \cite{Mil80}*{V.1.13}, for which there is an erratum \cite{Mil80e}. This need not concern us: the \cite{Mil80}*{V.2.1 (b)} input can be replaced by \cite{Fu11}*{8.5.3} (or by \cite{SGA4III}*{XVIII.3.2.6} if one also uses \cite{Mil80}*{Step $3$ of the proof of V.2.1 (b)}).}. The isogeny assumption gives $A^\vee[l](K{\overline}{\mathbb{F}}_q) \cong A[l](K{\overline}{\mathbb{F}}_q)$. In conclusion, $\#H^2(S_{{\overline}{\mathbb{F}}_q}, {\mathcal{A}}[l]) = \#H^0(S_{{\overline}{\mathbb{F}}_q}, {\mathcal{A}}[l])$. \qedhere
\end{enumerate}
\end{proof}

\begin{lemma} {\label}{cond-comp}
Let $k$ be a nonarchimedean local field, let ${\mathbb{F}}$ be its residue field, let $B {\rightarrow} \operatorname{Spec} k$ be an abelian variety, let $a(B)$ and ${\mathcal{B}}_{\mathbb{F}}$ be its conductor exponent and the special fiber of the N\'{e}ron model, and let $\Phi$ be the component group scheme of ${\mathcal{B}}_{\mathbb{F}}$. For every prime $l$ different from $\operatorname{char} {\mathbb{F}}$, one has
\[
a(B) = a(B[l]) + \dim_{{\mathbb{F}}_l} \Phi[l]({\overline}{\mathbb{F}}),
\]
where the Artin conductor $a(B[l])$ is defined by \eqref{art-cond} (with $N = 0$).
\end{lemma}

\begin{proof}
Let $I$ and $(V_l B)^{\mathrm{ss}}$ be the inertia and the semisimplification of the $l$-adic Tate module. Then 
\[\begin{aligned}
a(B) &= \operatorname{Sw}((V_l B)^{\mathrm{ss}}) + \dim_{{\mathbb{Q}}_l} V_l B - \dim_{{\mathbb{Q}}_l} (V_l B)^{I}, \\
a(B[l]) &= \operatorname{Sw}(B[l]) + \dim_{{\mathbb{F}}_l}  B[l] - \dim_{{\mathbb{F}}_l} B[l]^{I}.
\end{aligned}\]
 Since the Swan conductor is additive and compatible with reduction mod $l$, it remains to note that 
\[
\dim_{{\mathbb{F}}_l} B[l]^{I} = \dim_{{\mathbb{Q}}_l} (V_l B)^{I} + \dim_{{\mathbb{F}}_l} \Phi[l]({\overline}{\mathbb{F}})
\]
due to the identification of $B[l]^{I}$ and $(V_l B)^{I}$ with ${\mathcal{B}}_{\mathbb{F}}[l]$ and $V_l ({\mathcal{B}}_{\mathbb{F}})$ explained in \cite{ST68}*{Lemma~2}.
\end{proof}

\begin{cor} {\label}{main-part}
In \Cref{main}, if $A$ has a polarization of degree prime to $l$ and the $\operatorname{Gal}({\overline}{\mathbb{F}}_q/{\mathbb{F}}_{q^2})$-action on $H^1_{\mathrm{\acute{e}t}}(S_{{\overline}{\mathbb{F}}_q}, {\mathcal{A}}[l])$ is trivial, then, letting $w$ range over the places of $K{\mathbb{F}}_{q^2}$, we have
\begin{equation} {\label}{part-eq}
(-1)^{\operatorname{rk}_l A_{K{\mathbb{F}}_{q^2}}} = w(A_{K{\mathbb{F}}_{q^2}}) \cdot (-1)^{\dim_{{\mathbb{F}}_l} \operatorname{Sel}_l A_{K{\mathbb{F}}_{q^2}} - \dim_{{\mathbb{F}}_l} H^1_{\mathrm{\acute{e}t}}(S_{{\mathbb{F}}_{q^2}},\, {\mathcal{A}}[l]) - \sum_w \dim_{{\mathbb{F}}_l} \Phi_{v(w)}[l]({\overline}{\mathbb{F}}_q)},
\end{equation}
where $v(w)$ denotes the place of $K$ below $w$. If, in addition, $\Phi_v[l] = 0$ for all places $v$ of $K$, then the $l$-parity conjecture holds for $A_{K{\mathbb{F}}_{q^2}}$.
\end{cor}

\begin{proof}
\Cref{rk-sel} together with \Cref{GOS} (applied over $K{\mathbb{F}}_{q^2}$) gives
\[
(-1)^{\operatorname{rk}_l A_{K{\mathbb{F}}_{q^2}} - \dim_{{\mathbb{F}}_l} \operatorname{Sel}_l A_{K{\mathbb{F}}_{q^2}} + \dim_{{\mathbb{F}}_l} H^1_{\mathrm{\acute{e}t}}(S_{{\mathbb{F}}_{q^2}},\, {\mathcal{A}}[l])} = (-1)^{\sum_{w\text{ inert in } K{\mathbb{F}}_{q^4}} a(A[l]_{K_{v(w)}})}.
\]
A $v$ that is split (resp.,~inert) in $K{\mathbb{F}}_{q^2}$ contributes summands in pairs (resp.,~stays inert in $K{\mathbb{F}}_{q^4}$), so
\[
\sum_{w\text{ inert in } K{\mathbb{F}}_{q^4}} a(A[l]_{K_{v(w)}}) \equiv \sum_{v\text{ inert in } K{\mathbb{F}}_{q^2}} a(A[l]_{K_{v}}) \bmod 2.
\]
\Cref{cond-comp} and \Cref{st-root} give the remaining input to obtain \eqref{part-eq}, namely, they give the~equality
\[
(-1)^{\sum_{v\text{ inert in } K{\mathbb{F}}_{q^2}} (a(A[l]_{K_{v}}) + \dim_{{\mathbb{F}}_l} \Phi_{v}[l]({\overline}{\mathbb{F}}_q))} = w(A_{K{\mathbb{F}}_{q^2}}).
\] 
For the last sentence, view Selmer groups as \'{e}tale cohomology groups: \cite{Ces13c}*{5.4~(c)} together with the smoothness of ${\mathcal{A}}[l]$ gives $\operatorname{Sel}_l A_{K{\mathbb{F}}_{q^2}} = H^1_{\mathrm{\acute{e}t}}(S_{{\mathbb{F}}_{q^2}}, {\mathcal{A}}[l])$ inside $H^1(K{\mathbb{F}}_{q^2}, A[l])$ if the local Tamagawa factors of $A_{K{\mathbb{F}}_{q^2}}$ are prime to $l$ (in particular, if $\Phi_v[l] = 0$ for all $v$).
\end{proof}

The combination of \Cref{red1,main-part} proves \Cref{main} for $l = 2$. Our main goal in \S\S\ref{ad-pre}--\ref{ade} is to lift the $\Phi_v[l] = 0$ restriction for odd $l$: arithmetic duality techniques will help to analyze the right hand side exponent of \eqref{part-eq} through \Cref{dual}.

\section{Arithmetic duality generalities: comparing Selmer sizes modulo squares} {\label}{ad-pre}

The main goal of this section is to prove \Cref{beyond-KMR}, which in \S\ref{ade} will specialize to the arithmetic duality input needed for the proof of \Cref{main}. \Cref{beyond-KMR} generalizes \cite{KMR13}*{Thm.~3.9} to the case of commutative self-dual finite group schemes over global fields from the case of self-dual $2$-dimensional ${\mathbb{F}}_p$-vector space group schemes over number fields. Although its proof is loosely modeled on that of loc.~cit., modifications to loc.~cit.~are necessary due to the $\operatorname{char} F \mid \# {\mathcal{G}}$ possibility when various cohomology groups are no longer finite. The simpler $\operatorname{char} F \nmid \#{\mathcal{G}}$ case of \Cref{beyond-KMR} suffices for the proof of \Cref{main}, but it seems unnatural to confine the general techniques in this way. Consequently, \Cref{dual} does not exclude the more subtle $\operatorname{char} F \mid n$ cases.

In the buildup to \Cref{beyond-KMR} we follow an axiomatic approach by introducing further assumptions as we need them. This way, in \Cref{beyond-MR} we arrive at a generalization of \cite{MR07}*{Prop.~1.3 (i)} that removes the self-duality, ${\mathbb{F}}_p$-vector space, and number field assumptions from loc.~cit.

In this section and \S\ref{ade} all the cohomology groups  are fppf. Identifications with \'{e}tale or Galois cohomology are implicit. Likewise implicit is the Tate modification: we write $H^i({\mathbb{R}}, -)$ for ${\widehat}{H}^i({\mathbb{R}}, -)$.

\begin{pp}[The basic setup] {\label}{setup}
Let $F$ be a global field. If $\operatorname{char} F = 0$, let $S$ be the spectrum of the ring of integers of $F$; if $\operatorname{char} F > 0$, let $S$ be the connected smooth proper curve over a finite field such that the function field of $S$ is $F$. Let $U \subset S$ be a nonempty open subscheme. We denote by $v$ a place of $F$ and identify the nonarchimedean $v$ with the closed points of $S$; writing $v\not \in U$ signifies that $v$ does not correspond to a closed point of $U$ (and hence could be archimedean). 

Let ${\mathcal{G}} {\rightarrow} U$ and ${\mathcal{H}} {\rightarrow} U$ be commutative finite flat group schemes, and suppose that there is a perfect bilinear pairing ${\mathcal{G}} \times_U {\mathcal{H}} {\xrightarrow}{b} {\mathbb{G}}_m$ that identifies ${\mathcal{G}}$ and ${\mathcal{H}}$ as Cartier duals. The cohomology groups $H^1(U, {\mathcal{G}})$ and $H^1(U, {\mathcal{H}})$ are ``cut out by local conditions'', i.e., as noted in \cite{Ces13c}*{4.3}, the squares
\[
\xymatrix{
H^1(U, {\mathcal{G}}) \ar[d] \ar@{^(->}[r] & H^1(F, {\mathcal{G}})  \ar[d]    	& & H^1(U, {\mathcal{H}}) \ar[d] \ar@{^(->}[r] & H^1(F, {\mathcal{H}})  \ar[d]    \\
\prod\limits_{v\in U} H^1({\mathcal{O}}_v, {\mathcal{G}}) \ar@{^(->}[r] & \prod\limits_{v\in U} H^1(F_v, {\mathcal{G}}), & & \prod\limits_{v\in U} H^1({\mathcal{O}}_v, {\mathcal{H}}) \ar@{^(->}[r] & \prod\limits_{v\in U} H^1(F_v, {\mathcal{H}})
}
\]
are Cartesian. The main result of this section, \Cref{beyond-KMR}, investigates further subgroups cut out by also imposing local conditions at all $v\not\in U$. Its proof hinges on, among other things, the Tate--Shatz local duality \cite{Sha64}*{Duality theorem on p.~411} (alternatively, \cite{Mil06}*{I.2.3,~I.2.13~(a),~III.6.10}), which says that for every place $v$ and integer $i$ the cup product pairing
\begin{equation}{\label}{Shatz}
\xymatrix{
H^i(F_v, {\mathcal{G}}) \times H^{2 - i}(F_v, {\mathcal{H}}) \ar[r] &H^2(F_v, {\mathbb{G}}_m) \ar@{^(->}[r]^-{\operatorname{inv}_v} &{\mathbb{Q}}/{\mathbb{Z}}
}
\end{equation}
that uses $b$ identifies $H^i(F_v, {\mathcal{G}})$ and $H^{2-i}(F_v, {\mathcal{H}})$ as Pontryagin duals of each other. The Pontryagin duality in question is that of Hausdorff locally compact abelian topological groups---see \cite{Sha64}*{esp.~II.\S4} for the definition and the properties of the topology on the cohomology groups. These groups are finite and discrete if $\operatorname{char} F_v \nmid \#{\mathcal{G}}$; this case suffices for the proof of \Cref{main}.
\end{pp}

\begin{eg} {\label}{main-eg}
Our main case of interest in the setup of \S\ref{setup} is when $B {\rightarrow} \operatorname{Spec} F$ and $B^\vee {\rightarrow} \operatorname{Spec} F$ are dual abelian varieties, ${\mathcal{B}} {\rightarrow} S$ and ${\mathcal{B}}^\vee {\rightarrow} S$ are their N\'{e}ron models, $U \subset S$ is such that ${\mathcal{B}}_U {\rightarrow} U$ and ${\mathcal{B}}_U^\vee {\rightarrow} U$ are abelian schemes, and ${\mathcal{G}} = {\mathcal{B}}[n]_U$, ${\mathcal{H}} = {\mathcal{B}}^\vee[n]_U$ for some $n \in {\mathbb{Z}}_{> 0}$. Cartier--Nishi duality \cite{Oda69}*{Thm.~1.1} supplies the pairing $b$ in this case.
\end{eg}

\begin{lemma} {\label}{im-orth}
In the setup of \S\ref{setup}, for every integer $i$ the images of the pullback maps
\[
 H^i(U, {\mathcal{G}}) {\xrightarrow}{\operatorname{loc}^i({\mathcal{G}})} \textstyle\bigoplus_{v\not\in U} H^i(F_v, {\mathcal{G}}) \quad \text{and} \quad H^{2 - i}(U, {\mathcal{H}}) {\xrightarrow}{\operatorname{loc}^{2 - i}({\mathcal{H}})} \bigoplus_{v \not\in U} H^{2 - i}(F_v, {\mathcal{H}})
\]
are orthogonal complements under the sum of the pairings \eqref{Shatz}.
\end{lemma}

\begin{proof}
In the $\#{\mathcal{G}} \in \Gamma(U, {\mathcal{O}}_U^\times)$ case, the Poitou--Tate sequence gives the claim once one explicates its morphisms and interprets the global cohomology groups as Galois cohomology with restricted ramification (for this interpretation consult, e.g., \cite{Mil06}*{II.2.9}). To treat the general case we will use an extension of the Poitou--Tate sequence, namely, the compactly supported flat cohomology~sequence
\[
\dotsb {\rightarrow} H^i_c(U, {\mathcal{G}}) {\rightarrow} H^i(U, {\mathcal{G}}) {\xrightarrow}{\operatorname{loc}^i({\mathcal{G}})} \textstyle\bigoplus_{v \not\in U} H^i(F_v, {\mathcal{G}}) {\xrightarrow}{\delta_{c}^i({\mathcal{G}})} H^{i + 1}_c(U, {\mathcal{G}}) {\rightarrow} \dotsb
\]
of \cite{Mil06}*{III.0.4 (a)}. Due to its exactness and the perfectness of the pairings in 
\begin{equation}\tag{$\dagger$}\begin{aligned}{\label}{pair}
\xymatrix{
H^i(U, {\mathcal{G}}) \ar[d]_-{\operatorname{loc}^i({\mathcal{G}})} \ar@{}[r]|-{\bigtimes} & H^{3- i}_c(U, {\mathcal{H}}) \ar[rrr]^{\text{\cite{Mil06}*{III.3.2 and III.8.2}}} &&& H^3_c(U, {\mathbb{G}}_m) \ar[r]^-\operatorname{tr} &{\mathbb{Q}}/{\mathbb{Z}} \ar@{=}[d] \\
\bigoplus_{v\not\in U} H^i(F_v, {\mathcal{G}}) \ar@{}[r]|-{\bigtimes} & \bigoplus_{v\not\in U} H^{2 - i}(F_v, {\mathcal{H}}) \ar[u]^-{\delta^{2 - i}_c({\mathcal{H}})} \ar[rrr]^{\sum_v \eqref{Shatz}} &&& \bigoplus_{v\not\in U} H^2(F_v, {\mathbb{G}}_m) \ar[u]^-{\delta^2_c({\mathbb{G}}_m)} \ar[r]^-{\sum_v \operatorname{inv}_v} & {\mathbb{Q}}/{\mathbb{Z}},
}
\end{aligned}\end{equation}
it remains to establish the commutativity of \eqref{pair} (and of its analogue that has ${\mathcal{G}}$ and ${\mathcal{H}}$ interchanged and $i$ replaced by $2 - i$). The commutativity of its right square results from \cite{Mil06}*{II.\S3,~(b)~on~p.~176}.

Due to the definition of the top pairing, the agreement of the cup product \eqref{Shatz} with the $\operatorname{Ext}$-product as in \cite{GH71}*{3.1}, and the agreement of the $\operatorname{Ext}$-product with the Yoneda edge product as in \cite{GH70}*{4.5}, the commutativity of the left square of \eqref{pair} reduces to that of \begin{equation}\tag{\ddag}\begin{aligned}{\label}{pair-pair}
\xymatrix{
\operatorname{Ext}^i_U({\mathcal{H}}, {\mathbb{G}}_m) \ar[d] \ar@{}[r]|-{\bigtimes} & H^{3- i}_c(U, {\mathcal{H}}) \ar[rr]^{\text{\cite{Mil06}*{III.0.4 (e)}}} && H^3_c(U, {\mathbb{G}}_m)  \\
\bigoplus_{v\not\in U} \operatorname{Ext}^i_{F_v}({\mathcal{H}}, {\mathbb{G}}_m) \ar@{}[r]|-{\bigtimes} & \bigoplus_{v\not\in U} H^{2 - i}(F_v, {\mathcal{H}}) \ar[u]^-{\delta^{2 - i}_c({\mathcal{H}})} \ar[rr] && \bigoplus_{v\not\in U} H^2(F_v, {\mathbb{G}}_m) \ar[u]^-{\delta^2_c({\mathbb{G}}_m)}.
}
\end{aligned}\end{equation}
This commutativity results from the definitions and the inspection of the proof of loc.~cit.:~if one fixes injective resolutions ${\mathcal{H}} {\rightarrow} I^\bullet({\mathcal{H}})$ and ${\mathbb{G}}_m {\rightarrow} I^\bullet({\mathbb{G}}_m)$ over $U$ and interprets elements of $\operatorname{Ext}^i({\mathcal{H}}, {\mathbb{G}}_m)$ and $\bigoplus_{v\not\in U} H^{2 - i}(F_v, {\mathcal{H}})$ as homotopy classes of maps $I^\bullet({\mathcal{H}}) {\xrightarrow}{a} I^\bullet({\mathbb{G}}_m)[i]$ and ${\mathbb{Z}} {\xrightarrow}{d} \Gamma\p{\bigoplus_{v\not\in U} F_v, I^\bullet({\mathcal{H}})|_{\bigoplus_{v\not \in U} F_v}}[2- i]$, then both ways to pair $a$ and $d$ in \eqref{pair-pair} result in the element of $H^3_c(U, {\mathbb{G}}_m)$ that is represented by the homotopy class of
\[
\xymatrix{
{\mathbb{Z}} \ar[rrrrr]^-{\p{0,\, \Gamma(\bigoplus_{v\not\in U} F_v,\, a|_{\bigoplus_{v\not\in U} F_v})[2 - i]\, \circ\, d}} &&&&&\textstyle \Gamma\p{U, I^\bullet({\mathbb{G}}_m)}[3] \oplus \Gamma\p{\bigoplus_{v\not\in U} F_v, I^\bullet({\mathbb{G}}_m)|_{\bigoplus_{v\not \in U} F_v}}[2].
} \qedhere
\]
\end{proof}

\begin{pp}[Local conditions at $v\not\in U$] {\label}{not-in-U}
In the setup of \S\ref{setup}, suppose that for every $j \in \{1, 2\}$ and $v \not \in U$ one has open compact subgroups 
\[
\operatorname{Sel}^j({\mathcal{G}}_{F_v}) \subset H^1(F_v, {\mathcal{G}}) \quad \text{and} \quad \operatorname{Sel}^j({\mathcal{H}}_{F_v}) \subset H^1(F_v, {\mathcal{H}})
\]
that are orthogonal complements under \eqref{Shatz}. For $j \in \{1, 2\}$, define the Selmer groups by requiring
\[\begin{aligned}
0 {\rightarrow} &\operatorname{Sel}^j({\mathcal{G}}) {\rightarrow} H^1(U, {\mathcal{G}}) {\rightarrow} \textstyle\bigoplus_{v\not\in U} H^1(F_v, {\mathcal{G}})/\operatorname{Sel}^j({\mathcal{G}}_{F_v}),\\
0 {\rightarrow} &\operatorname{Sel}^j({\mathcal{H}}) {\rightarrow} H^1(U, {\mathcal{H}}) {\rightarrow} \textstyle\bigoplus_{v\not\in U} H^1(F_v, {\mathcal{H}})/\operatorname{Sel}^j({\mathcal{H}}_{F_v}).
\end{aligned}\]
to be exact. For $v\not\in U$, \cite{HR79}*{24.10} gives further orthogonal complements 
\[
\operatorname{Sel}^{1 + 2}({\mathcal{G}}_{F_v}) {\colonequals} \operatorname{Sel}^1({\mathcal{G}}_{F_v}) + \operatorname{Sel}^2({\mathcal{G}}_{F_v})\quad \text{and} \quad \operatorname{Sel}^{1 \cap 2}({\mathcal{H}}_{F_v}) {\colonequals} \operatorname{Sel}^1({\mathcal{H}}_{F_v}) \cap \operatorname{Sel}^2({\mathcal{H}}_{F_v}),
\]
and one defines the corresponding Selmer groups by the exactness of 
\[\begin{aligned}
0 {\rightarrow} &\operatorname{Sel}^{1 + 2}({\mathcal{G}}) {\rightarrow} H^1(U, {\mathcal{G}}) {\rightarrow} \textstyle\bigoplus_{v\not\in U} H^1(F_v, {\mathcal{G}})/\operatorname{Sel}^{1 + 2}({\mathcal{G}}_{F_v}),\\
0 {\rightarrow} &\operatorname{Sel}^{1 \cap 2}({\mathcal{H}}) {\rightarrow} H^1(U, {\mathcal{H}}) {\rightarrow} \textstyle\bigoplus_{v\not\in U} H^1(F_v, {\mathcal{H}})/\operatorname{Sel}^{1 \cap 2}({\mathcal{H}}_{F_v}).
\end{aligned}\]
\end{pp}

\begin{eg} {\label}{eg-loc-c}
In \Cref{main-eg} the local conditions of most interest to us arise when one takes the images of the local Kummer maps for $\operatorname{Sel}^1$ and, under appropriate restrictions, their fppf cohomological counterparts $H^1({\mathcal{O}}_v, {\mathcal{B}}[n])$ and $H^1({\mathcal{O}}_v, {\mathcal{B}}^\vee[n])$ for $\operatorname{Sel}^2$. In \S\ref{ade} we will justify that the results of \S\ref{ad-pre} can be applied in this setting to compare $\#\operatorname{Sel}_n B$ and $\#H^1(S, {\mathcal{B}}[n])$ modulo squares.
\end{eg}

\begin{prop} {\label}{beyond-MR}
In the setup of \S\ref{not-in-U}, 
\[
\#\p{\f{\operatorname{Sel}^{1 + 2}({\mathcal{G}})}{\operatorname{Sel}^1({\mathcal{G}})}} \cdot \#\p{\f{\operatorname{Sel}^1({\mathcal{H}})}{\operatorname{Sel}^{1 \cap 2}({\mathcal{H}})}} = \prod_{v\not\in U} \#\p{\f{\operatorname{Sel}^{1 + 2}({\mathcal{G}}_{F_v})}{\operatorname{Sel}^1({\mathcal{G}}_{F_v})}} = \prod_{v\not\in U} \#\p{\f{\operatorname{Sel}^1({\mathcal{H}}_{F_v})}{\operatorname{Sel}^{1\cap 2}({\mathcal{H}}_{F_v})}},
\]
where all the appearing cardinalities are finite.
\end{prop}

\begin{proof}
Due to \Cref{im-orth} and the choice of $\operatorname{Sel}^1({\mathcal{G}}_{F_v})$ and $\operatorname{Sel}^1({\mathcal{H}}_{F_v})$, \cite{HR79}*{24.10} shows that
\[
\operatorname{Im}(\operatorname{loc}^1({\mathcal{G}}))\,\textstyle{+} \bigoplus_{v\not\in U} \operatorname{Sel}^1({\mathcal{G}}_{F_v})  \subset \bigoplus_{v\not\in U} H^1(F_v, {\mathcal{G}})  \quad\text{and}\quad \operatorname{Im}(\operatorname{loc}^1({\mathcal{H}})|_{\operatorname{Sel}^1({\mathcal{H}})}) \subset \bigoplus_{v\not\in U} H^1(F_v, {\mathcal{H}})
\]  
are orthogonal complements. Therefore, so are
\[
\textstyle{} \f{H^1(U, {\mathcal{G}})}{\operatorname{Sel}^1({\mathcal{G}})} \subset \bigoplus_{v\not\in U} \f{H^1(F_v, {\mathcal{G}})}{\operatorname{Sel}^1({\mathcal{G}}_{F_v})}  \quad\text{and}\quad \operatorname{Im}(\operatorname{loc}^1({\mathcal{H}})|_{\operatorname{Sel}^1({\mathcal{H}})}) \subset \bigoplus_{v\not\in U} \operatorname{Sel}^1({\mathcal{H}}_{F_v}).
\]
Likewise,
\[
\textstyle{} \bigoplus_{v\not\in U}\f{\operatorname{Sel}^{1+ 2}({\mathcal{G}}_{F_v})}{\operatorname{Sel}^1({\mathcal{G}}_{F_v})} \subset \bigoplus_{v\not\in U} \f{H^1(F_v, {\mathcal{G}})}{\operatorname{Sel}^1({\mathcal{G}}_{F_v})}  \quad\text{and}\quad \bigoplus_{v\not\in U} \operatorname{Sel}^{1\cap 2}({\mathcal{H}}_{F_v}) \subset \bigoplus_{v\not\in U} \operatorname{Sel}^1({\mathcal{H}}_{F_v})
\]
are also orthogonal complements. By combining the last two claims we deduce that
\[
\textstyle{} \f{\operatorname{Sel}^{1 + 2}({\mathcal{G}})}{\operatorname{Sel}^1({\mathcal{G}})}  \subset \bigoplus_{v\not\in U} \f{H^1(F_v, {\mathcal{G}})}{\operatorname{Sel}^1({\mathcal{G}}_{F_v})}  \quad\text{and}\quad \operatorname{Im}(\operatorname{loc}^1({\mathcal{H}})|_{\operatorname{Sel}^1({\mathcal{H}})}) + \bigoplus_{v\not\in U} \operatorname{Sel}^{1\cap 2}({\mathcal{H}}_{F_v}) \subset \bigoplus_{v\not\in U} \operatorname{Sel}^1({\mathcal{H}}_{F_v})
\]
are orthogonal complements, too, and hence so are
\[
\textstyle{} \f{\operatorname{Sel}^{1 + 2}({\mathcal{G}})}{\operatorname{Sel}^1({\mathcal{G}})}  \subset \bigoplus_{v\not\in U} \f{\operatorname{Sel}^{1+ 2}({\mathcal{G}}_{F_v})}{\operatorname{Sel}^1({\mathcal{G}}_{F_v})}  \quad\text{and}\quad \f{\operatorname{Sel}^1({\mathcal{H}})}{\operatorname{Sel}^{1\cap 2}({\mathcal{H}})} \subset \bigoplus_{v\not\in U} \f{\operatorname{Sel}^1({\mathcal{H}}_{F_v})}{\operatorname{Sel}^{1 \cap 2}({\mathcal{H}}_{F_v})}.
\]
Both $\operatorname{Sel}^1({\mathcal{H}}_{F_v})$ and $\operatorname{Sel}^{1 \cap 2}({\mathcal{H}}_{F_v})$ are open and compact, so the groups in the last display are finite and the conclusion follows from the fact that dual finite abelian groups have the same cardinality.
\end{proof}

\begin{pp}[Breaking the symmetry] {\label}{break}
In the setup of \S\ref{not-in-U}, suppose that there exists an isomorphism $\theta\colon {\mathcal{G}} {\rightarrow} {\mathcal{H}}$ of $U$-group schemes such that the bilinear pairings
\begin{equation} {\label}{anti-L}
b_{F_v}(-, \theta_{F_v}(\cdot))\colon {\mathcal{G}}_{F_v} \times_{F_v} {\mathcal{G}}_{F_v} {\rightarrow} {\mathbb{G}}_m \quad\quad\quad \text{for $v \not\in U$}
\end{equation}
are antisymmetric. The cup product and \eqref{anti-L} induce the top pairing in the commutative diagram
\begin{equation}\begin{aligned}{\label}{pair-comp}
\xymatrix{
\ar@{}[r]|-{\In{\ ,\ }_v:}&H^1(F_v, {\mathcal{G}}) \times H^1(F_v, {\mathcal{G}}) \ar[r] \ar[d]^-{\wr}_-{\operatorname{id} \times H^1(F_v, \theta)} &H^2(F_v, {\mathbb{G}}_m) \ar@{^(->}[r]^-{\operatorname{inv}_v} \ar@{=}[d] &{\mathbb{Q}}/{\mathbb{Z}} \ar@{=}[d]\\
&H^1(F_v, {\mathcal{G}}) \times H^1(F_v, {\mathcal{H}}) \ar[r]^-{\eqref{Shatz}} &H^2(F_v, {\mathbb{G}}_m) \ar@{^(->}[r]^-{\operatorname{inv}_v} &{\mathbb{Q}}/{\mathbb{Z}}
}
\end{aligned}\end{equation}
for $v\not\in U$. The pairing $\In{\ ,\, }_v$ inherits nondegeneracy and bilinearity from \eqref{Shatz}. The antisymmetry of \eqref{anti-L} implies the symmetry of $\In{\ ,\, }_v$, as can be seen by using \v{C}ech cohomology (which in our setting agrees with the derived functor cohomology due to \cite{Sha64}*{Thm.~1}).

Furthermore, suppose that the isomorphism $H^1(F_v, \theta)$  identifies
\[
\operatorname{Sel}^j({\mathcal{G}}_{F_v}) \subset H^1(F_v, {\mathcal{G}}) \quad \text{with} \quad \operatorname{Sel}^j({\mathcal{H}}_{F_v}) \subset H^1(F_v, {\mathcal{H}}) \quad\quad \text{for every $v\not\in U$ and $j \in \{1, 2\}$,}
\]
so, in particular, $\operatorname{Sel}^j({\mathcal{G}}_{F_v})$ is a maximal isotropic subgroup (a Lagrangian in other terminology) for $\In{\ ,\, }_v$. Consequently, the isomorphism $H^1(U, \theta)$ identifies 
\[
\operatorname{Sel}^j({\mathcal{G}}) \subset H^1(U, {\mathcal{G}})\quad  \text{with}\quad  \operatorname{Sel}^j({\mathcal{H}}) \subset H^1(U, {\mathcal{H}}).
\]
\end{pp}

\begin{pp}[Quadratic forms] {\label}{quad}
In the setup of \S\ref{break}, suppose further that there exist quadratic forms 
\[
q_v \colon H^1(F_v, {\mathcal{G}}) {\rightarrow} {\mathbb{Q}}/{\mathbb{Z}} \quad\quad \text{ for $v\not\in U$}
\]
such that for each $q_v$ the associated bilinear pairing is $\In{\ ,\, }_v$; concretely, each $q_v$ is required to satisfy $q_v(ax) = a^2 q_v(x)$ for $a \in {\mathbb{Z}}$ and $q_v(x + y) - q_v(x) - q_v(y) = \In{x, y}_v$. Suppose also that 
\begin{equation}{\label}{glob-quad} \tag{$\bigstar$}
\textstyle\sum_{v\not\in U} q_v(x_v) = 0 \quad \text{for every $x \in H^1(U, {\mathcal{G}})$ with pullbacks $x_v \in H^1(F_v, {\mathcal{G}})$} 
\end{equation}
and that each $\operatorname{Sel}^j({\mathcal{G}}_{F_v}) \subset H^1(F_v, {\mathcal{G}})$ is maximal isotropic for $q_v$; the latter requirement means that $q_v(\operatorname{Sel}^j({\mathcal{G}}_{F_v})) = 0$ in addition to the maximal isotropy of $\operatorname{Sel}^j({\mathcal{G}}_{F_v})$ for $\In{\ , \,}_v$.

Quadratic forms come into play only when $\#{\mathcal{G}}$ is even: for odd $\#{\mathcal{G}}$, the equality $\In{x, x}_v = 2q_v(x)$ determines $q_v$, which satisfies the requirements (for \eqref{glob-quad}, use the reciprocity for the Brauer group). 
\end{pp}

\begin{thm} {\label}{beyond-KMR}
In the setup of \S\ref{quad}, suppose (for simplicity) that $\operatorname{Sel}^j({\mathcal{G}})$ is finite for $j \in \{ 1, 2\}$.~Then 
\begin{equation}{\label}{beyond-KMR-eq}
\f{\# \operatorname{Sel}^1({\mathcal{G}})}{\#\operatorname{Sel}^2({\mathcal{G}})} \equiv \prod_{v\not\in U} \#\p{ \f{\operatorname{Sel}^1({\mathcal{G}}_{F_v})}{\operatorname{Sel}^1({\mathcal{G}}_{F_v}) \cap \operatorname{Sel}^2({\mathcal{G}}_{F_v})}  } \bmod {\mathbb{Q}}^{\times 2}.
\end{equation}
\end{thm}

\begin{proof}
We will combine \Cref{beyond-MR} with \Cref{XYZ}, which is a variant of \cite{KMR13}*{Lemma~2.3}. Before stating \Cref{XYZ}, we introduce
\[
(V, q) = \p{\textstyle\bigoplus_{v \not\in U} H^1(F_v, {\mathcal{G}}),\, \sum_{v\not\in U} q_v}, \quad \text{whose associated bilinear form is} \quad \In{\ ,\, } {\colonequals} \textstyle\sum_{v\not\in U} \In{\ ,\, }_v.
\] 
Since $\In{\ ,\, }$ is continuous and nondegenerate, it exhibits $V$ as its own Pontryagin dual. The subgroups
\[
X = \textstyle\bigoplus_{v \not\in U} \operatorname{Sel}^1({\mathcal{G}}_{F_v}), \quad\quad\quad\quad Y = \textstyle\bigoplus_{v \not\in U} \operatorname{Sel}^2({\mathcal{G}}_{F_v}),\quad\quad\quad\quad Z = \operatorname{Im}(\operatorname{loc}^1({\mathcal{G}})),
\]
are maximal isotropic for $q$: $X$ and $Y$ due to \S\ref{quad}, $Z$ due to \eqref{pair-comp}, \Cref{im-orth}, and \eqref{glob-quad}.

Since $\operatorname{Sel}^1({\mathcal{G}})$ is finite, so is its quotient $X \cap Z$, and likewise for $Y \cap Z$. In $V$, both $X$ and $X\cap Y$ are open and compact, so $\f{X + Y}{Y} \cong \f{X}{X\cap Y}$ is finite. Thus, $(X + Y)\cap Z$ is finite, too, since~$\f{(X + Y)\cap Z}{Y \cap Z} {\hookrightarrow} \f{X + Y}{Y}$.

\begin{lem} {\label}{XYZ}
$\#\p{(X + Y)\cap Z} \equiv \#\p{X\cap Z + Y \cap Z} \bmod {\mathbb{Q}}^{\times 2}$.
\end{lem}

\begin{proof}
The proof of \cite{KMR13}*{Lemma 2.3} extends; we outline this extension.

Once $\f{(X + Y)\cap Z}{X \cap Z + Y \cap Z}$ is shown to carry a nondegenerate alternating bilinear pairing, a well-known \cite{Dav10}*{proof of Lemma 4.2} implies that $\f{(X + Y)\cap Z}{X \cap Z + Y \cap Z}$ is a square of another finite abelian group, and hence is of square order, as desired. The bilinear pairing $[\ ,\, ]\colon (X + Y) \cap Z {\rightarrow} {\mathbb{Q}}/{\mathbb{Z}}$ well-defined by
\[
[x + y, x{^{\prime}} + y{^{\prime}}] {\colonequals} \In{x, y{^{\prime}}}\quad \text{for} \quad x + y,\, x{^{\prime}} + y{^{\prime}} \in (X + Y)\cap Z \quad  \text{with}\quad x, x{^{\prime}} \in X \quad \text{and} \quad y, y{^{\prime}} \in Y
\]
is alternating due to the isotropy of $Z$, $X$, and $Y$: indeed, $\In{x, y} = q(x + y) - q(x) - q(y) = 0$. Due to its resulting antisymmetry, its right and left kernels agree; in particular, this common kernel $K$ contains $X \cap Z + Y \cap Z$. Once we argue the reverse inclusion, as we do below, the nondegeneracy of the alternating bilinear pairing on $\f{(X + Y)\cap Z}{X \cap Z + Y \cap Z}$ induced by $[\ ,\, ]$, and hence also the conclusion,~follows.

For $x + y$ as above, $x + y \in K$ implies $x \in ((X + Y)\cap Z + X \cap Y)^\perp$, where the orthogonal complement is taken in $(V, \In{\ ,\,})$. Since the appearing subgroups are closed, \cite{HR79}*{24.10} gives
\[
((X + Y)\cap Z + X \cap Y)^\perp = ((X \cap Y) + Z) \cap (X + Y) = (X + Y)\cap Z + X \cap Y,
\]
so the freedom of adjusting $x$ and $y$ by opposite elements of $X \cap Y$ allows us to assume that $x\in (X +Y) \cap Z \subset Z$. Then $y \in Z$ as well, which leads to the sought $x + y \in X \cap Z + Y \cap Z$.
\end{proof}

According to \Cref{beyond-MR}, the right hand side of \eqref{beyond-KMR-eq} equals $\#\p{\f{\operatorname{Sel}^{1 + 2}({\mathcal{G}})}{\operatorname{Sel}^{1\cap 2}({\mathcal{G}})}}$, which in turn equals $\#\p{\f{(X + Y) \cap Z}{X\cap Y\cap Z}}$. Moreover,
\[
\#\p{\f{(X + Y) \cap Z}{X\cap Y\cap Z}} \overset{\ref{XYZ}}{\equiv} \#(X\cap Z + Y \cap Z) \cdot \#(X\cap Y \cap Z) \equiv \f{\#(X \cap Z)}{\#(Y \cap Z)} \mod {\mathbb{Q}}^{\times 2},
\]
and it remains to observe that $\f{\#(X \cap Z)}{\#(Y \cap Z)} = \f{\#\operatorname{Sel}^1({\mathcal{G}})}{\#\operatorname{Sel}^2({\mathcal{G}})}$.
\end{proof}

\begin{rem}
If $\operatorname{char} F \nmid \#{\mathcal{G}}$, then \cite{Mil06}*{II.2.13 (a)} implies the assumed finiteness of $\operatorname{Sel}^j({\mathcal{G}})$.
\end{rem}

\section{Comparing Selmer sizes modulo squares in the main case of interest} {\label}{ade}

In this section we specialize the results of \S\ref{ad-pre} to the setup of \Cref{main-eg}, which we enforce and recall: ${\mathcal{B}}$ and ${\mathcal{B}}^\vee$ are global N\'{e}ron models of dual abelian varieties $B$ and $B^\vee$ that have good reduction at all the points of $U$, and ${\mathcal{G}} = {\mathcal{B}}[n]_U$, ${\mathcal{H}} = {\mathcal{B}}^\vee[n]_U$ for some $n \in {\mathbb{Z}}_{> 0}$. Our main task is to justify that under suitable restrictions on $B$ various general assumptions made in \S\ref{ad-pre} are met if the local conditions are chosen as in \Cref{eg-loc-c} (we recall the choices in \Cref{1-2}). This justification leads to \Cref{dual}, which in \S\ref{final} will finish the proof of \Cref{main}.

\begin{prop} {\label}{1-2}
The following subgroups suit \S\ref{not-in-U}, i.e., are open compact orthogonal complements.
{\begin{enumerate}[label={(\alph*)}]}
\item {\label}{1-2-a}
For $j = 1$ and $v\not\in U$, 
\[
\operatorname{Sel}^1({\mathcal{G}}_{F_v}) = B(F_v)/nB(F_v) \quad \text{ and }\quad \operatorname{Sel}^1({\mathcal{H}}_{F_v}) = B^\vee(F_v)/nB^\vee(F_v).
\]
With these choices, $\operatorname{Sel}^1({\mathcal{G}}) = \operatorname{Sel}_n(B)$ and $\operatorname{Sel}^1({\mathcal{H}}) = \operatorname{Sel}_n (B^\vee)$; both Selmer groups are~finite.

\item {\label}{1-2-b}
For $j = 2$ and $v\not\in U$,
\[\begin{aligned}
&\operatorname{Sel}^2({\mathcal{G}}_{F_v}) = H^1({\mathcal{O}}_v, {\mathcal{B}}[n])  \quad\ \, \text{ and }\quad \operatorname{Sel}^2({\mathcal{H}}_{F_v}) = H^1({\mathcal{O}}_v, {\mathcal{B}}^{\vee}[n])\quad \quad \quad\ \text{if $v \nmid \infty$}, \\
&\operatorname{Sel}^2({\mathcal{G}}_{F_v}) = B(F_v)/nB(F_v)  \quad \text{ and }\quad \operatorname{Sel}^2({\mathcal{H}}_{F_v}) = B^\vee(F_v)/nB^\vee(F_v)\quad\quad \text{if $v \mid \infty$}
\end{aligned}\]
under the assumption that $B$ (and hence also $B^\vee$) has semiabelian reduction at all $v\in S\setminus U$ with $\operatorname{char} {\mathbb{F}}_v \mid n$. With these choices, $\operatorname{Sel}^2({\mathcal{G}})$ is the subgroup $H^1(S, {\mathcal{B}}[n]){^{\prime}} \subset H^1(S, {\mathcal{B}}[n])$ consisting of the elements whose restrictions to every $H^1(F_v, B[n])$ with $v\mid \infty$ lie in $B(F_v)/nB(F_v)$, and similarly for $\operatorname{Sel}^2({\mathcal{H}})$; both $\operatorname{Sel}^2({\mathcal{G}})$ and $\operatorname{Sel}^2({\mathcal{H}})$ are finite.
\end{enumerate} 
\end{prop}

\begin{proof} \hfill
{\begin{enumerate}[label={(\alph*)}]}
\item 
The orthogonal complement claim is a well-known important step of the proof of Tate local duality, compare \cite{Mil06}*{proof of III.7.8}. The continuity of the homomorphisms in
\[
B(F_v) {\xrightarrow}{\delta} H^1(F_v, B[n]) {\rightarrow} H^1(F_v, B)[n] {\rightarrow} 0
\]
is another step of that proof. This continuity, compactness of $B(F_v)$, and discreteness of $H^1(F_v, B)$ show that $\delta(B(F_v))$, which equals $B(F_v)/nB(F_v)$, is compact and open; likewise for $B^\vee(F_v)/nB^\vee(F_v)$.

The identification $\operatorname{Sel}^1({\mathcal{G}}) = \operatorname{Sel}_n B$ results from the definition of the $n$-Selmer group and \cite{Ces13c}*{2.5 (d) and 4.2}; likewise for $\operatorname{Sel}^1({\mathcal{H}})$. The claimed finiteness is known (compare~\S\ref{Xl}).

\item 
The $v\mid \infty$ case follows from \ref{1-2-a}. Below we analyze the case when $v\in S \setminus U$ (i.e., when $v\nmid \infty$). For such $v$ the given $\operatorname{Sel}^2$ are subgroups due to \cite{Ces13c}*{A.5 and B.4}.

Loc.~cit.~also shows that ${\mathcal{B}}[n]_{{\mathcal{O}}_v} {\rightarrow} \operatorname{Spec} {\mathcal{O}}_v$ is affine, quasi-finite, and faithfully flat. A torsor under ${\mathcal{B}}[n]_{{\mathcal{O}}_v}$ inherits these properties, and hence, thanks to \cite{EGAIV4}*{18.5.11 a)${\Rightarrow}$c)}, trivializes over the ring of integers ${\mathcal{O}}_v{^{\prime}}$ of a finite extension $F_v{^{\prime}}/F_v$. 
Consequently, 
\[
H^1({\mathcal{O}}_v, {\mathcal{B}}[n]) = \textstyle\varinjlim_{F_v{^{\prime}}} \widecheck{H}^1({\mathcal{O}}_v{^{\prime}}/{\mathcal{O}}_v, {\mathcal{B}}[n]), \quad \text{where one restricts to the $F_v{^{\prime}}$ inside ${\overline}{F}_v$.}
\]
Since $\widecheck{H}^1({\mathcal{O}}_v{^{\prime}}/{\mathcal{O}}_v, {\mathcal{B}}[n]) {\rightarrow} H^1({\mathcal{O}}_v, {\mathcal{B}}[n])$ includes the torsors that trivialize over ${\mathcal{O}}_v{^{\prime}}$, it is injective; therefore, so is $\widecheck{H}^1({\mathcal{O}}_v{^{\prime}}/{\mathcal{O}}_v, {\mathcal{B}}[n]) {\xrightarrow}{a} \widecheck{H}^1(F_v{^{\prime}}/F_v, B[n])$. Once we show that $\operatorname{Im} a$ is open, the openness of $H^1({\mathcal{O}}_v, {\mathcal{B}}[n])$ will follow: for each $F_v{^{\prime}}$ the preimage of $H^1({\mathcal{O}}_v, {\mathcal{B}}[n])$ in $\widecheck{H}^1(F_v{^{\prime}}/F_v, B[n])$ will contain the open subgroup $\operatorname{Im} a$, and hence will be open. For the openness of $\operatorname{Im} a$, construct the \v{C}ech complex for ${\mathcal{B}}[n]$ and ${\mathcal{O}}_v{^{\prime}}/{\mathcal{O}}_v$ as in \cite{Gro68}*{la~preuve~de~11.4}, i.e., as a complex of affine finite type ${\mathcal{O}}_v$-group schemes that are restrictions of scalars of ${\mathcal{B}}[n]$ from iterated ${\mathcal{O}}_v$-tensor products of copies of ${\mathcal{O}}_v{^{\prime}}$ back to ${\mathcal{O}}_v$. On ${\mathcal{O}}_v$- and $F_v$-points this complex evaluates to complexes used to compute $\widecheck{H}^i({\mathcal{O}}_v{^{\prime}}/{\mathcal{O}}_v, {\mathcal{B}}[n])$ and $\widecheck{H}^i(F_v{^{\prime}}/F_v, B[n])$. Therefore, letting $Z^1$ denote the affine finite type ${\mathcal{O}}_v$-group scheme of $1$-cocycles, we deduce the openness of $\operatorname{Im} a$ from that of $Z^1({\mathcal{O}}_v) \subset Z^1(F_v)$ and $Z^1(F_v) {\twoheadrightarrow} \widecheck{H}^1(F_v{^{\prime}}/F_v, B[n])$. 

Due to \ref{1-2-a} and the preceding paragraph, the intersection of $B(F_v)/nB(F_v)$ and $H^1({\mathcal{O}}_v, {\mathcal{B}}[n])$ is open. By \cite{Ces13c}*{2.5~(a)}, it is also of finite index in both subgroups. Therefore, the compactness of $B(F_v)/nB(F_v)$ proved in \ref{1-2-a} implies the compactness of $H^1({\mathcal{O}}_v, {\mathcal{B}}[n])$.

The compactness and openness of $H^1({\mathcal{O}}_v, {\mathcal{B}}^\vee[n])$ follow for reasons of symmetry. 

The orthogonal complement claim is essentially \cite{McC86}*{4.14}, but we must check that \eqref{Shatz} agrees with the pairing used in loc.~cit. More precisely, due to the perfectness of the pairings~in 
\begin{equation}\begin{aligned}{\label}{flat-ort}
\xymatrix{
H^2_{{\mathbb{F}}_v}({\mathcal{O}}_v, {\mathcal{B}}[n]) \ar@{}[r]|-{\bigtimes} & H^1({\mathcal{O}}_v, {\mathcal{B}}^\vee[n]) \ar[d]\ar[rr]^-{\text{\cite{McC86}*{4.14}}} && H^3_{{\mathbb{F}}_v}({\mathcal{O}}_v, {\mathbb{G}}_m) \cong {\mathbb{Q}}/{\mathbb{Z}} \\
H^1(F_v, B[n]) \ar[u]\ar@{}[r]|-{\bigtimes} & H^1(F_v, B^\vee[n]) \ar[rr]^-{\eqref{Shatz}} && H^2(F_v, {\mathbb{G}}_m) \cong {\mathbb{Q}}/{\mathbb{Z}} \ar@<15pt>[u]_-{\wr}  
}
\end{aligned}\end{equation}
and the exactness of the cohomology with supports sequence
\[
\dotsb {\rightarrow} H^m_{{\mathbb{F}}_v}({\mathcal{O}}_v, {\mathcal{B}}[n]) {\rightarrow} H^m({\mathcal{O}}_v, {\mathcal{B}}[n]) {\rightarrow} H^m(F_v, B[n]) {\rightarrow} H^{m + 1}_{{\mathbb{F}}_v}({\mathcal{O}}_v, {\mathcal{B}}[n]) {\rightarrow} \dotsb
\]
of \cite{Mil06}*{III.0.3 (c)}, it suffices to prove that \eqref{flat-ort} and its analogue for ${\mathcal{B}}^\vee[n]$ commute. Let $i\colon \operatorname{Spec} {\mathbb{F}}_v {\hookrightarrow} \operatorname{Spec} {\mathcal{O}}_v$ and $j\colon \operatorname{Spec} F_v {\hookrightarrow} \operatorname{Spec} {\mathcal{O}}_v$ be the indicated immersions. Both pairings in \eqref{flat-ort} are Yoneda edge products---the top one due to its definition and the bottom one due to the observations made in the proof of \Cref{im-orth}---so the commutativity of \eqref{flat-ort} will follow from that of
\begin{equation}\begin{aligned}{\label}{ext-fest}
\xymatrix{
\operatorname{Ext}^2(i_* \underline{\mathbb{Z}}_{{\mathbb{F}}_v} , {\mathcal{B}}[n]) \ar@{}[r]|-{\bigtimes} & \operatorname{Ext}^1({\mathcal{B}}[n], {\mathbb{G}}_m) \ar@{=}[d]\ar[rr]^-{\text{\cite{McC86}*{4.14}}} && \operatorname{Ext}^3(i_*\underline{\mathbb{Z}}_{{\mathbb{F}}_v}, {\mathbb{G}}_m)  \\
\operatorname{Ext}^1(j_! \underline{\mathbb{Z}}_{F_v} , {\mathcal{B}}[n]) \ar[u]\ar@{}[r]|-{\bigtimes} & \operatorname{Ext}^1({\mathcal{B}}[n], {\mathbb{G}}_m) \ar[d]\ar[rr] && \operatorname{Ext}^2(j_!\underline{\mathbb{Z}}_{F_v}, {\mathbb{G}}_m) \ar[u]_{\wr}  \\
\operatorname{Ext}^1(\underline{\mathbb{Z}}_{F_v}, B[n]) \ar@{}[u]|-{\reflectbox{\rotatebox[origin=c]{90}{\text{\scalebox{1.5}{$\cong$}}}}}\ar@{}[r]|-{\bigtimes} & \operatorname{Ext}^1(B[n], {\mathbb{G}}_m) \ar[rr]^-{\eqref{Shatz}} && \operatorname{Ext}^2(\underline{\mathbb{Z}}_{F_v}, {\mathbb{G}}_m), \ar@{}[u]|-{\reflectbox{\rotatebox[origin=c]{90}{\text{\scalebox{1.5}{ $\cong$}}
}}}
}
\end{aligned}\end{equation}
where the identifications arise from the $j_! \dashv j^*$ adjunction as in \cite{Mil06}*{proof of III.0.3 (b)}. To see the commutativity of the bottom part of \eqref{ext-fest}, replace ${\mathcal{B}}[n]$ and ${\mathbb{G}}_m$ by injective resolutions over $\operatorname{Spec} {\mathcal{O}}_v$, interpret elements of $\operatorname{Ext}$ groups as homotopy classes of maps (compare with the proof of \Cref{im-orth} for this), and use the $j_! \dashv j^*$ adjunction together with the fact that $j^*$ preserves injectives. 
To see the commutativity of the upper part, observe that in the derived category the upper vertical arrows correspond to precomposition with the first morphism of the distinguished triangle $i_* \underline{\mathbb{Z}}_{{\mathbb{F}}_v}[-1] {\rightarrow} j_! \underline{\mathbb{Z}}_{F_v} {\rightarrow} \underline{\mathbb{Z}}_{{\mathcal{O}}_v} {\rightarrow} i_* \underline{\mathbb{Z}}_{{\mathbb{F}}_v}$.

The identification $\operatorname{Sel}^2({\mathcal{G}}) = H^1(S, {\mathcal{B}}[n]){^{\prime}}$ results from \cite{Ces13c}*{4.2 and B.4}, and its finiteness follows from the comparison with $\operatorname{Sel}_n B$ described in \cite{Ces13c}*{5.5}; likewise for $\operatorname{Sel}^2({\mathcal{H}})$.
\qedhere
\end{enumerate}
\end{proof}

\begin{rem} {\label}{rem-pr}
In many cases $H^1(S, {\mathcal{B}}[n]){^{\prime}} = H^1(S, {\mathcal{B}}[n])$: for instance, this happens if $n$ is odd or if $B(F_v)$ is connected for every real $v$ (which implies the same for $B^\vee$, cf.~\cite{GH81}*{\S1}) because then $H^1(F_v, B[n]) = 0$ for all $v\mid \infty$, as loc.~cit.~proves. We resort to the somewhat artificial $H^1(S, {\mathcal{B}}[n]){^{\prime}}$ to make our duality results apply even when $H^1(F_v, B[n]) \neq 0$ for some $v\mid \infty$.
\end{rem}

\begin{prop} {\label}{anti-pf}
Let $\theta\colon {\mathcal{B}}[n]_U {\rightarrow} {\mathcal{B}}^\vee[n]_U$ be the isomorphism induced by a self-dual isogeny ${\widetilde}{\theta}$ of degree prime to $n$, and let $b$ be the duality pairing of \S\ref{setup} for ${\mathcal{G}} = {\mathcal{B}}[n]_U$ and~${\mathcal{H}} = {\mathcal{B}}^\vee[n]_U$.
{\begin{enumerate}[label={(\alph*)}]} 
\item {\label}{anti-pf-a}
The pairing $b(-, \theta(\cdot))\colon {\mathcal{B}}[n]_U \times_U {\mathcal{B}}[n]_U {\rightarrow} {\mathbb{G}}_m$ is antisymmetric.

\item
The morphisms induced by ${\widetilde}{\theta}$ identify $B(F_v)/nB(F_v)$ with $B^\vee(F_v)/nB^\vee(F_v)$ for all $v$ and $H^1({\mathcal{O}}_v, {\mathcal{B}}[n])$ with $H^1({\mathcal{O}}_v, {\mathcal{B}}^\vee[n])$ for $v\nmid \infty$. 
\end{enumerate}
In particular, $\theta$ meets the assumptions of \S\ref{break} if the $\operatorname{Sel}^j$ are chosen as in \Cref{1-2}.
\end{prop}

\begin{proof}
Only \ref{anti-pf-a} requires proof. Let $b^\vee$ be the analogue of $b$ for $B^\vee$, use the identification $B^{\vee \vee} = B$, and apply \cite{Oda69}*{Thm.~1.1} and \cite{Oda69}*{Cor.~1.3 (ii)} with ${\lambda} = [n]_{{\mathcal{B}}_U}$, ${\lambda}{^{\prime}} = [n]_{{\mathcal{B}}_U^\vee}$, and ${\alpha} = {\beta} = {\widetilde}{\theta}$ to obtain respective equalities in the desired
\[
b(-, \theta(\cdot)) = b^\vee(\theta(-), \cdot) = -b(\cdot, \theta(-)). \qedhere
\]
\end{proof}

\begin{pp}[Suitable quadratic forms $q_v$] {\label}{qv-main}
Suppose that the assumptions of \Cref{1-2}~\ref{1-2-b} are met and that there is a self-dual isogeny ${\widetilde}{\theta}{^{\prime}} \colon B {\rightarrow} B^\vee$ of degree prime to $n$. Consider the self-dual isogeny ${\widetilde}{\theta} {\colonequals} \begin{cases} 2{\widetilde}{\theta}{^{\prime}}, \text{ if $n$ is odd,} \\ {\widetilde}{\theta}{^{\prime}},\ \, \text{ if $n$ is even,} \end{cases}$ which also has degree prime to $n$. The self-dual isogeny $\lambda {\colonequals} n{\widetilde}{\theta}$ comes from a symmetric line bundle ${\mathscr{L}}$ on $B$ due to \cite{PR12}*{Rem.~4.5}, so the results of \cite{PR12}*{\S4} apply.  In particular, for $v \not \in U$, we can use the pullback ${\mathscr{L}}_v$ of ${\mathscr{L}}$ to $B_{F_v}$ to define the quadratic form 
\[
\xymatrix{
q_v\colon H^1(F_v, B[n]) \ar@{^(->}[r] &H^1(F_v, B[{\lambda}]) \ar[rrr]^-{-\text{\cite{PR12}*{Cor.~4.7}}} &&&H^2(F_v, {\mathbb{G}}_m) \ar@{^(->}[r]^-{\operatorname{inv}_v} &{\mathbb{Q}}/{\mathbb{Z}},
}
\]
where we take the negative of the quadratic form $H^1(F_v, B[{\lambda}]) {\rightarrow} H^2(F_v, {\mathbb{G}}_m)$ provided by loc.~cit.
\end{pp}

\begin{prop} {\label}{qv-grand}
In the setup of \S\ref{qv-main},
{\begin{enumerate}[label={(\alph*)}]}
\item {\label}{qv-grand-a}
The bilinear pairing associated to $q_v$ is $\In{\ ,\, }_v$ given by \S\ref{break} via \Cref{anti-pf} applied to ${\widetilde}{\theta}$;

\item {\label}{qv-grand-b}
$q_v(B(F_v)/nB(F_v)) = 0$ for every $v\not\in U$;

\item {\label}{qv-grand-c}
$\sum_{v\not\in U} q_v(x_v) = 0$ for every $x \in H^1(U, {\mathcal{B}}[n])$ with pullbacks $x_v \in H^1(F_v, B[n])$;

\item {\label}{qv-grand-d}
$q_v(H^1({\mathcal{O}}_v, {\mathcal{B}}[n])) = 0$ for every $v \in S \setminus U$ for which
\begin{enumerate}[label={(\roman*)}]
\item {\label}{ass-i}
$B$ has semiabelian reduction at $v$ if $\operatorname{char} {\mathbb{F}}_v \mid n$, and

\item {\label}{ass-ii}
The local Tamagawa factor $\#\Phi_v({\mathbb{F}}_v)$ is odd if $n$ is even (here $\Phi_v = {\mathcal{B}}_{{\mathbb{F}}_v}/{\mathcal{B}}_{{\mathbb{F}}_v}^0$).
\end{enumerate}
\end{enumerate}
In particular, if \ref{ass-i}--\ref{ass-ii} hold for every $v\in S\setminus U$, then the $q_v$ meet the assumptions of \S\ref{quad}.
\end{prop}

\begin{proof} \hfill
{\begin{enumerate}[label={(\alph*)}]}
\item 
By \cite{PR12}*{Cor.~4.7}, the bilinear pairing associated to $q_v$ is the restriction to $H^1(F_v, B[n])$ of the cup product pairing $H^1(F_v, B[{\lambda}]) \times H^1(F_v, B[{\lambda}]) {\rightarrow} {\mathbb{Q}}/{\mathbb{Z}}$ that uses Cartier duality $b_{B[{\lambda}]}\colon B[{\lambda}] \times B[{\lambda}] {\rightarrow} {\mathbb{G}}_m$. Due to the naturality of the cup product, it remains to show that
\[
\xymatrix{
B[n] \times B[n] \ar[r]^-{\operatorname{id} \times \theta_F} \ar@{^(->}[d] & B[n] \times B^\vee[n] \ar@<-4pt>[d]^-{b_F} \\
B[{\lambda}] \times B[{\lambda}] \ar[r]^-{b_{B[{\lambda}]}} & {\mathbb{G}}_m
}
\]
commutes. For this, apply \cite{Oda69}*{Cor.~1.3 (ii)} with ${\alpha} = \operatorname{id}_B$, ${\beta} = {\widetilde}{\theta}$, ${\lambda} = [n]_B$, and ${\lambda}{^{\prime}} = \lambda$.

\item
This follows from \cite{PR12}*{Prop.~4.9} because the inclusion $H^1(F_v, B[n]) {\hookrightarrow} H^1(F_v, B[{\lambda}])$ maps the subgroup $B(F_v)/nB(F_v)$ into $B^\vee(F_v)/{\lambda} B(F_v)$ via the homomorphism induced by~$\theta$.

\item
By \cite{Ces13c}*{4.2 and 2.5 (d)}, for every $x\in H^1(U, {\mathcal{B}}[n]) \subset H^1(F, B[{\lambda}])$ and every $v \in U$ one has $x_v \in B(F_v)/nB(F_v) \subset B^\vee(F_v)/{\lambda} B(F_v)$. Therefore, \cite{PR12}*{Thm.~4.14 (a)} gives the~claim.

\item
Set $n{^{\prime}} {\colonequals} n$ if $n$ is odd, and $n{^{\prime}} {\colonequals} \#\Phi_v({\mathbb{F}}_v)$ if $n$ is even. By \cite{Ces13c}*{2.5~(a)},
\[
n{^{\prime}} x \in B(F_v)/nB(F_v)\quad \text{for every}\quad x \in H^1({\mathcal{O}}_v, {\mathcal{B}}[n]).
\] 
Thus, \ref{qv-grand-b} gives $n^{\prime 2} q_v(x) = 0$. On the other hand, $\In{\ ,\,}_v$ vanishes on $H^1({\mathcal{O}}_v, {\mathcal{B}}[n])$ due to \ref{qv-grand-a} and \S\ref{break} via \Cref{anti-pf}, so $2 q_v(x) = \In{x, x}_v = 0$. Since $n{^{\prime}}$ is odd, we get $q_v(x) = 0$. \qedhere
\end{enumerate}
\end{proof}

\begin{thm} {\label}{dual}
Let $n$ be a positive integer, $B$ an abelian variety over a global field $F$, and ${\mathcal{B}} {\rightarrow} S$ its global N\'{e}ron model. For $v\nmid \infty$, let $\Phi_v$ denote the component group scheme of ${\mathcal{B}}_{{\mathbb{F}}_v}$. Suppose that 
\begin{enumerate}[label={(\roman*)}]
\item
$B$ has a self-dual isogeny ${\widetilde}{\theta}{^{\prime}}$ of degree prime to $n$,

\item 
$B$ has semiabelian reduction at every nonarchimedean place $v$ of $F$ with $\operatorname{char} {\mathbb{F}}_v \mid n$, and

\item {\label}{dual-iii}
If $n$ is even, then $\#\Phi_v({\mathbb{F}}_v)$ is odd for every nonarchimedean $v$.
\end{enumerate}
Let $H^1(S, {\mathcal{B}}[n]){^{\prime}} \subset H^1(S, {\mathcal{B}}[n])$ be the subgroup of the elements whose restrictions to every $H^1(F_v, B[n])$ with $v\mid \infty$ lie in $B(F_v)/nB(F_v)$ (see \Cref{rem-pr} for some cases when $H^1(S, {\mathcal{B}}[n]){^{\prime}} = H^1(S, {\mathcal{B}}[n])$).~Then
\[
\f{\#\operatorname{Sel}_n B}{\#H^1(S, {\mathcal{B}}[n]){^{\prime}}} \equiv \prod_{v\nmid \infty} \f{\#\Phi_v({\mathbb{F}}_v)}{\#(n\Phi_v)({\mathbb{F}}_v)} \bmod {\mathbb{Q}}^{\times 2}.
\]
\end{thm}

\begin{proof}
\Cref{1-2,anti-pf,qv-grand} let us to apply \Cref{beyond-KMR}, which gives the conclusion because
\[
\#\p{\f{B(F_v)/nB(F_v)}{H^1({\mathcal{O}}_v, {\mathcal{B}}[n]) \cap (B(F_v)/nB(F_v))}} = \f{\#\Phi_v({\mathbb{F}}_v)}{\#(n\Phi_v)({\mathbb{F}}_v)} \quad \text{for $v \nmid \infty$}
\]
due to \cite{Ces13c}*{2.5 (a)}.
\end{proof}

\begin{rem} {\label}{l=2}
Since \Cref{beyond-KMR} is general, one could remove the assumption \ref{dual-iii} from \Cref{dual} by proving \Cref{qv-grand} \ref{qv-grand-d} without its assumption \ref{ass-ii}. This would also remove the additional assumption in the $l = 2$ case from \Cref{main}.
\end{rem}

\section{The proof of \Cref{main}} {\label}{final}

Let $w$ denote a place of $K{\mathbb{F}}_{q^2}$, and let $v(w)$ denote the place of $K$ lying below $w$. \Cref{red1,main-part} reduce \Cref{main} to the congruence
\begin{equation}{\label}{final-eq}\tag{$\maltese$}
\dim_{{\mathbb{F}}_l} \operatorname{Sel}_l A_{K{\mathbb{F}}_{q^2}} - \dim_{{\mathbb{F}}_l} H^1_{\mathrm{\acute{e}t}}(S_{{\mathbb{F}}_{q^2}},\, {\mathcal{A}}[l]) \equiv \textstyle\sum_w \dim_{{\mathbb{F}}_l} \Phi_{v(w)}[l]({\mathbb{F}}_{v(w)}) \bmod 2
\end{equation}
that needs to be established under the additional assumptions that $A$ has a polarization of degree prime to $l$, one has $\Phi_v({\overline}{\mathbb{F}}_v) = \Phi_v({\mathbb{F}}_v)$ for all $v$, and, if $l = 2$, that also every $\#\Phi_v({\mathbb{F}}_v)$ is odd. Under these assumptions, \eqref{final-eq} follows from \Cref{dual} applied to $B = A_{K{\mathbb{F}}_{q^2}}$ and $n = l$. \qed

\appendix
\section{Local root numbers in unramified extensions} {\label}{app}

\Cref{st-root}, which is the goal of this appendix, details the behavior of the local root number of an abelian variety upon an unramified extension of degree $n$ of the nonarchimedean local base field. In fact, this behavior manifests itself for a wider class of representations than those coming from abelian varieties, as we observe in \Cref{st-WD}. To summarize, for representations in this class the local root number ``stabilizes'' upon unramified base change of sufficiently divisible degree to a value determined by the parity of the conductor.\footnote{We do not use conductor ideals, so `conductor' abbreviates what some authors call `conductor exponent'.} Such behavior, which is crucial for our proof of \Cref{main}, seems not to have been pointed out previously.

\begin{pp}[The setup]
In this appendix, let $k$ be a nonarchimedean local field, and let ${\mathfrak{o}}$, ${\mathbb{F}}$, $p$, and $k^s$ be its ring of integers, residue field, residue characteristic, and a choice of a separable closure. The unramified subextension of $k^s/k$ of degree $n$ and its ring of integers are denoted by $k_n$ and ${\mathfrak{o}}_n$. The Weil group and its inertia subgroup are denoted by $W(k^s/k)$ and $I$. A geometric Frobenius in $W(k^s/k)$ is denoted by ${\mathrm{Frob}}_k$. For a field $F$ with $\operatorname{char} F \neq p$, we let $\abs{\cdot}_k\colon W(k^s/k) {\rightarrow} F^\times$ be the unramified character characterized by $\abs{{\mathrm{Frob}}_k}_k = (\#{\mathbb{F}}){^{-1}}$; for an integer $m$ and a representation $V$ of $W(k^s/k)$ over $F$, we set $V(m) {\colonequals} V {\otimes} \abs{\cdot}_k^m$, where the second factor denotes $F$ on which $W(k^s/k)$ acts through the $m{^{\mathrm{th}}}$ power of $\abs{\cdot}_k$.
\end{pp}

\begin{pp}[${\epsilon}$-factors of Weil--Deligne representations] {\label}{eps-WD}
For a field $F$ with $\operatorname{char} F \neq p$, a \emph{Weil--Deligne representation} of $W(k^s/k)$ over $F$ is a pair $\rho{^{\prime}} = (\rho, N)$ that consists of
\begin{itemize}
\item A finite dimensional representation $\rho$ of $W(k^s/k)$ over $F$ such that its restriction  to an open subgroup of $I$ is trivial, and

\item A $W(k^s/k)$-homomorphism $N\colon \rho {\rightarrow} \rho(-1)$. 
\end{itemize}
Subject to the choices of a nontrivial additive character $\psi\colon k {\rightarrow} F^\times$ and a nonzero $F$-valued Haar measure $dx$ on $(k, +)$, the ${\epsilon}$-factor of $\rho{^{\prime}}$ is defined by
\begin{equation}{\label}{eps-WD-def}
{\epsilon}(\rho{^{\prime}}, \psi, dx) {\colonequals} {\epsilon}_0(\rho, \psi, dx) \det(-{\mathrm{Frob}}_k\, |\, (\operatorname{Ker} N)^{I}){^{-1}},
\end{equation}
where for the appearing ${\epsilon}_0$-factor as well as the definitions of an additive character and an $F$-valued Haar measure we refer to \cite{Del73}*{\S6} (or to \cite{Ces13a}*{1.1 and \S\S2.3--4}). The \emph{Artin conductor} of $\rho{^{\prime}}$ is
\begin{equation}{\label}{art-cond}
a(\rho{^{\prime}}) {\colonequals} \operatorname{Sw} \rho + \dim_F \rho - \dim_F (\operatorname{Ker} N)^{I};
\end{equation}
for the definition of the Swan conductor $\operatorname{Sw} \rho$, see \cite{Del73}*{\S6.2} or \cite{Ces13a}*{\S2.9}.
\end{pp}

\begin{prop} {\label}{eps-gr}
In the setup of \S\ref{eps-WD}, for the restriction $\rho{^{\prime}}|_{k_n}$ of $\rho{^{\prime}}$ to $W(k^s/k_n)$ one has
\[
{\epsilon}(\rho{^{\prime}}|_{k_n}, \psi \circ \operatorname{Tr}_{k_n/k}, dx_{n}) = \begin{cases} {\epsilon}(\rho{^{\prime}}, \psi, dx)^n,\quad\quad\quad\quad\, \text{if $n$ is odd,}  \\ (-1)^{a(\rho{^{\prime}})} {\epsilon}(\rho{^{\prime}}, \psi, dx)^n, \, \text{ if $n$ is even.} \end{cases}
\]
Here $dx_n$ denotes the Haar measure on $(k_n, +)$ characterized by $\int_{{\mathfrak{o}}_{n}} dx_{n} = (\int_{\mathfrak{o}} dx)^n$. 
\end{prop}

\begin{proof}
For the ${\epsilon}_0$-factor appearing in \eqref{eps-WD-def}, the inductivity in degree $0$ gives
\begin{equation}{\label}{A}
{\epsilon}_0(\rho|_{k_n}, \psi \circ \operatorname{Tr}_{k_n/k}, dx_n) = {\epsilon}_0({\textbf}{1}_{k_n}, \psi \circ \operatorname{Tr}_{k_n/k}, dx_n)^{\dim_F \rho} \cdot \f{{\epsilon}_0((\operatorname{Ind}_{k_n}^k {\textbf}{1}_{k_n}) {\otimes} \rho, \psi, dx)}{{\epsilon}_0(\operatorname{Ind}_{k_n}^k {\textbf}{1}_{k_n} , \psi, dx)^{\dim_F \rho}}.
\end{equation}
Since $\operatorname{Ind}_{k_n}^k {\textbf}{1}_{k_n}$ is unramified, \cite{Del73}*{5.5.3} (or \cite{Ces13a}*{3.2.2} for general $F$) simplifies the fraction~to
\begin{equation}{\label}{B}
\det(\operatorname{Ind}_{k_n}^k {\textbf}{1}_{k_n})({\mathrm{Frob}}_k)^{\operatorname{Sw} \rho} \cdot \f{{\epsilon}_0(\rho, \psi, dx)^n}{{\epsilon}_0({\textbf}{1}_{k} , \psi, dx)^{n\dim_F \rho}} = (-1)^{(n - 1) \operatorname{Sw} \rho} \cdot \f{{\epsilon}_0(\rho, \psi, dx)^n}{{\epsilon}_0({\textbf}{1}_{k} , \psi, dx)^{n\dim_F \rho}}.
\end{equation}
Let $n(\psi)$ denote the largest integer $n$ such that $\psi|_{\pi^{-n}{\mathfrak{o}}} = 1$, where $\pi \in {\mathfrak{o}}$ is a uniformizer. Since $k_n/k$ is unramified, $n(\psi \circ \operatorname{Tr}_{k_n/k}) = n(\psi)$ by \cite{Del73}*{\S4.11}; we use this in the computation
\begin{equation}{\label}{C}
{\epsilon}_0({\textbf}{1}_{k} , \psi, dx)^{n} = \p{-(\#{\mathbb{F}})^{n(\psi)} \cdot \int_{\mathfrak{o}} dx}^n = (-1)^{n - 1}{\epsilon}_0({\textbf}{1}_{k_n}, \psi \circ \operatorname{Tr}_{k_n/k}, dx_n).
\end{equation}
The equations \eqref{A}, \eqref{B}, and \eqref{C} combine to give
\begin{equation}{\label}{D}
{\epsilon}_0(\rho|_{k_n}, \psi \circ \operatorname{Tr}_{k_n/k}, dx_n) = (-1)^{(n - 1)(\operatorname{Sw} \rho + \dim_F \rho)} {\epsilon}_0(\rho, \psi, dx)^n.
\end{equation}
It remains to put \eqref{D} together with the evident
\[
\det(-{\mathrm{Frob}}_{k_n}\, |\, (\operatorname{Ker} N)^{I}){^{-1}} = (-1)^{-(n - 1)\dim_F (\operatorname{Ker} N)^{I}} \det(-{\mathrm{Frob}}_{k}\, |\, (\operatorname{Ker} N)^{I})^{-n}. \qedhere
\]
\end{proof}

\begin{pp}[Root numbers] {\label}{root}
Assume that $F = {\mathbb{C}}$ and $\int_{\mathfrak{o}} dx \in {\mathbb{R}}^+$ in \S\ref{eps-WD}. The \emph{root number} of $\rho{^{\prime}}$ is
\[
w(\rho{^{\prime}}, \psi) {\colonequals} \f{{\epsilon}(\rho{^{\prime}}, \psi, dx)}{\abs{{\epsilon}(\rho{^{\prime}}, \psi, dx)}}.
\]
It does not depend on the choice of $dx$ as long as $\int_{\mathfrak{o}} dx \in {\mathbb{R}}^+$. If $\det \rho$ is ${\mathbb{R}}^+$-valued, then $w(\rho{^{\prime}}, \psi)$ does not depend on the choice of $\psi$ either, thanks to the formula \cite{Del73}*{5.4}. In this case we abbreviate $w(\rho{^{\prime}}, \psi)$ to $w(\rho{^{\prime}})$. If $B$ is an abelian variety over $k$, then
\[
\textstyle{\bigwedge^{2g}} H^1_{\mathrm{\acute{e}t}}(B, {\mathbb{Q}}_l) \cong H^{2g}_{\mathrm{\acute{e}t}}(B, {\mathbb{Q}}_l) \cong {\mathbb{Q}}_l(-g),
\]
so the independence of $\psi$ is witnessed if $\rho{^{\prime}}$ is the complex Weil--Deligne representation $\sigma_B{^{\prime}}$ that one associates to $H^1_{\mathrm{\acute{e}t}}(B, {\mathbb{Q}}_l) \cong (V_l B)^*$ for a prime $l$ different from $p$ using Grothendieck's quasi-unipotence theorem and an embedding $\iota\colon {\mathbb{Q}}_l {\hookrightarrow} {\mathbb{C}}$. By \cite{Sab07}*{1.15}, the isomorphism class of $\sigma_B{^{\prime}}$ does not depend on $l$ and $\iota$,\footnote{Loc.~cit.~does not use its additional $\operatorname{char} k = 0$ assumption in the proof. Also, we bypass this issue by analyzing the right hand side of \eqref{l-par-conj} through the second case of \Cref{st-root}, the proof of which works for every $l$ and $\iota$.} and hence neither does the root number of $B$ defined by $w(B) {\colonequals} w(\sigma_B{^{\prime}})$. Due to the Weil pairing, the presence of a polarization of $B$, and \cite{Del73}*{5.7.1}, $w(B) \in \{ \pm 1\}$.
\end{pp}

\begin{cor}{\label}{st-WD}
For a Weil--Deligne representation $\rho{^{\prime}}$ of $W(k^s/k)$ over ${\mathbb{C}}$ such that $\det\rho$ is ${\mathbb{R}}^+$-valued and $w(\rho{^{\prime}})$ is an $m{^{\mathrm{th}}}$ root of unity,
\[
w(\rho{^{\prime}}|_{k_n}) = (-1)^{a(\rho{^{\prime}})}
\]
for every even $n$ divisible by $m$.
\end{cor}

\begin{proof}
Combine \Cref{eps-gr} with the discussion in \S\ref{root}.
\end{proof}

\begin{cor} {\label}{st-root}
Let $B$ be an abelian variety over $k$, and let $a(B)$ be its conductor exponent. Then
\[
w(B_{k_n}) = \begin{cases} w(B),\quad \quad\  \text{ if $n$ is odd,} \\  (-1)^{a(B)},\quad \text{ if $n$ is even.}  \end{cases}
\]
\end{cor}

\begin{proof}
Combine \Cref{eps-gr}, the discussion in \S\ref{root}, and the equality $a(B) = a(\sigma_B{^{\prime}})$ that results from the definitions (for which one can consult \cite{Ser70}*{\S2}).
\end{proof}

\begin{rem}
For elliptic curves, excluding the troublesome additive reduction case if $p \le 3$, one can also prove \Cref{st-root} by the means of explicit case-by-case formulae for $w(B)$ and $a(B)$.
\end{rem}

 

\begin{bibdiv}
\begin{biblist}

\bibselect{bibliography}

\end{biblist}
\end{bibdiv}

\end{document} 

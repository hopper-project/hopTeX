\documentclass[12pt,reqno]{amsart}

  
\usepackage[usenames]{color}
\usepackage[latin1]{inputenc} 

\usepackage[english]{babel} 

\usepackage{enumerate}
\usepackage{amssymb, amsmath}
\usepackage{geometry,graphicx}
\usepackage{hyperref}

\theoremstyle{plain}
\newtheorem{theorem}{Theorem}[section] 

\newtheorem{lemma}[theorem]{Lemma} 
\newtheorem{corollary}[theorem]{Corollary}
\newtheorem{proposition}[theorem]{Proposition}

\theoremstyle{definition}
\newtheorem{remark}[theorem]{Remark}
\newtheorem{remarks}[theorem]{Remarks}
\newtheorem{definition}[theorem]{Definition}
\newtheorem{example}[theorem]{Example}
\newtheorem{examples}[theorem]{Examples}

\geometry{a4paper,twoside,top=3.5cm,bottom=3.5cm,left=3cm,right=3cm,headsep=1cm}

  

\begin{document}

\title{On congruences beetwen newforms with different sign at a Steinberg prime}

\author{Luis Dieulefait}
\thanks{The first author is partially supported by MICINN grants MTM2012-33830 and by an ICREA Academia
Research Prize.}
 \author{Eduardo Soto}
\thanks{The second author is partially supported by MICINN grant MTM2013-45075-P.}

\address{L. Dieulefait\\ Departament d'Àlgebra i Geometria\\ Universitat de Barcelona\\ Barcelona, Spain}
\email{ldieulefait@ub.edu}

\address{E. Soto\\
Departament d'Àlgebra i Geometria\\ Universitat de Barcelona\\ Barcelona, Spain} 
\email{eduard.soto@ub.com}

\maketitle

\begin{abstract}
Let $f$ be a newform of weight $2$ on $\Gamma_0(N)$ with Fourier $q$-expansion $f(q)=q+\sum_{n\geq 2} a_n q^n$, where $\Gamma_0(N)$ denotes the group of invertible matrices with integer coefficients, upper triangular mod $N$. Let $p$ be a prime dividing $N$ once, $p\parallel N$, a Steinberg prime. Then, it is well known that $a_p\in\{1,-1\}$. We denote by $K_f$ the field of coefficients of $f$. Let $\lambda$ be a finite place in $K_f$ not dividing $2N$ and assume that the mod $\lambda$ Galois representation attached to $f$ is irreducible. In this paper we will give necessary and sufficient conditions for the existence of another newform $f'(q)=q+\sum_{n\geq 2} a'_n q^n$ of weight $2$ on $\Gamma_0(N)$ and a finite place $\lambda'$ of $K_{f'}$ such that $a_p=-a'_p$ and the Galois representations $\bar\rho_{f,\lambda}$ and $\bar\rho_{f',\lambda'}$ are isomorphic.
\keywords{Galois representation \and congruence mod $\ell$ \and Steinberg prime \and modular forms}
\end{abstract}
\subsection*{Acknowledgments}
The authors are grateful to K. Ribet for providing so much valuable feedback. The second author wants to thank X. Guitart for many stimulating conversations. 

\section{Introduction}
\label{intro}
Let $f,f'$ be a pair of newforms of weight $2$ of level $N$ and $N'$ respectively and trivial nebentypus. Consider the composite field $L=K_f \cdot K_f'$, that is the smallest field extension of $\mathbb Q$ containing all coefficients of the $q$-expansions of $f$ and $f'$ at infinity, then $L$ is a number field. Let $\mathfrak l$ be a prime of $L$ and let $\ell$ be the rational prime satisfying $\ell\mathbb Z=\mathfrak l\cap \mathbb Z$. We will be interested in pairs of newforms $f$ and $f'$ for which
\begin{equation}\label{cong'}
a_n \equiv a'_n \pmod {\mathfrak l}\qquad \text{for every integer $n$ coprime to $\ell NN'$}.
\end{equation}
The Fourier coefficients of a newform are completely determined by the $a_p$ coefficients of prime subindex. It's easy to see then that 

 \eqref{cong'} is equivalent to 
\begin{equation}\label{cong}
a_p \equiv a'_p \pmod {\mathfrak l}\qquad \text{for every prime $p\nmid \ell NN'$}.
\end{equation}
Let $\lambda=\mathfrak l \cap \mathcal O_{K_f}$ and $\lambda'=\mathfrak l \cap \mathcal O_{K_f'}$ and assume that the Galois representations $\bar\rho_{f,\lambda}$, $\bar\rho_{f',\lambda'}$ attached to $f$ and $f'$ are irreducible. Then $f$, $f'$ satisfy \eqref{cong'} if and only if $\bar\rho_{f,\lambda}$ and $\bar\rho_{f',\lambda'}$ are isomorphic. In general, it is not an easy problem to  find for a given newform $f$ another newform $f'$ satisfying \eqref{cong'}, neither proving the existence of such a newform $f'$. Ribet's level raising \cite{RibetRai} and level lowering \cite{RibetLow} theorems are very powerful in this context. In this article we will consider a newform $f$ of weight $2$ on $\Gamma_0(N)$ together with a prime $\lambda\nmid 2N$ of $K_f$ and will give necessary and sufficient conditions for the existence of another newform $f'$ of weight $2$ on $\Gamma_0(N)$ and a prime $\lambda'$ of $K_{f'}$ such that  
\begin{itemize}
\item $\bar\rho_{f,\lambda}$ and $\bar\rho_{f',\lambda'}$ are isomorphic and
\item $a_p=-a_p$ for a prime $p$ dividing $N$ once.
\end{itemize}

\section{mod $\lambda$ Galois representations attached to a newform}
From now on let's fix an odd prime $\ell$ and an immersion $\overline{\mathbb Q}{\lhook\joinrel\longrightarrow} \overline{\mathbb Q}_\ell$.\footnote{In particular, for every number field $K$ we have fixed a prime $\lambda$ over $\ell$ and a completion  $K_{\ell}\subset \overline{\mathbb Q}_\ell$ of $K$ with respect to $\lambda$.} Let $\overline{\mathbb F}_\ell$ denote the residue field of the ring of integers of $\overline{\mathbb Q}_\ell$, which is indeed an algebraic closure of $\mathbb F_\ell$.
Let $\ell$ be a rational prime and let $f$ be a newform on $\Gamma_0(N)$ with $q$-expansion at infinity
$$
f(z) = q + \sum_{n\geq 2} a_n q^n.
$$
We denote by $K_f$  the number field $\mathbb Q(\{a_n\}_n)$ of coefficients of $f$ and by $\mathcal O_f$ its ring of integers. Consider the $\ell$-adic Galois representation attached to $f$ by Deligne
$$
\rho_{f,\ell}: Gal(\overline{\mathbb Q}\mid \mathbb Q)\rightarrow GL_2(\mathcal O_{f,\ell})
$$
where $K_{f,\ell}$ denotes the completion of $K_f$ with respect to (the fixed prime $\lambda$ above) $\ell$ and $\mathcal O_{f,\ell}$ denotes its ring of integers. Recall that $\mathcal O_{f,\ell}$ is a local ring whose residue field $\mathcal O_{f,\ell}/\mathfrak m_{f,\ell}$ is a finite extension of $\mathbb F_\ell$ contained in $\overline{\mathbb F}_\ell$. Reducing mod $\mathfrak m_{f,\ell}$, we obtain the mod $\ell$ Galois representation
$$
\bar\rho_{f,\ell}:Gal(\overline{\mathbb Q}\mid\mathbb Q)\longrightarrow GL_2(\mathcal O_{f,\ell}/\mathfrak m_{f,\ell}){\lhook\joinrel\longrightarrow} GL_2(\overline{\mathbb F}_\ell)
$$
attached to $f$ (well defined up to semi-simplification), satisfying
$$
\text{tr}\,\bar\rho_{f,\ell} (Frob_p) \equiv a_p\qquad \det\bar\rho_{f,\ell}(Frob_p)\equiv p \pmod {\mathfrak m_{f,\ell}}
$$
for every rational prime $p\nmid \ell N$. 
\section{ Lowering and raising the levels}
We state here Ribet's theorems. As we announced in the introduction they are very powerful and are the main tools needed in the proof of our theorem.
\begin{theorem}[Ribet's level lowering theorem]
Let $f$ be a newform of weight $2$ on $\Gamma_0(N)$, let $p$ be a prime dividing $N$ once Assume that $\ell\nmid2N$ and that the mod $\ell$ Galois representation 
$$
\bar\rho_{f,\ell}:Gal(\overline{\mathbb Q}\mid \mathbb Q)\longrightarrow GL_2(\overline{\mathbb F}_\ell)
$$
is irreducible and unramified at $p$. Then there exists a newform $f'$ of weight $2$ on $\Gamma_0(N/p)$ such that $\bar\rho_{f',\ell}$ is isomorphic to $\bar\rho_{f,\ell}$.
\end{theorem}
\begin{proof}
See \cite{RibetLow}.
\end{proof}

\begin{theorem}[Ribet's level raising theorem]
Let $f$ be a newform of weight $2$ on $\Gamma_0(N)$ such that the mod $\ell$ Galois representation
$$
\bar\rho_{f,\ell}: Gal(\overline{\mathbb Q}\mid \mathbb Q)\longrightarrow GL_2(\overline{\mathbb F}_\ell)
$$
is irreducible. Let $p\nmid \ell N$ be a prime satisfying
$$
\emph{\text{tr}}\,\bar\rho_{f,\ell}(Frob_p) \equiv (-1)^k(p+1) \pmod \ell
$$
for some $k\in\{0,1\}$. 
Then there exists a newform $f'$ of weight $2$ on $\Gamma_0(pN)$ such that $\bar\rho_{f,\ell}$ is isomorphic to $\bar\rho_{f',\ell}$. In particular the $p$-th Fourier coefficient $a'_p$ of $f'$ equals $(-1)^k$. Moreover, if $p+1\equiv 0\pmod \ell$ then there are at least two such newforms $f'$: one for each coefficient $a'_p\in\{\pm1\}$.
\end{theorem}
\begin{proof}
See in \cite{RibetRai} theorem 1 and remarks in section $3$.
\end{proof}

\section{Local decomposition at Steinberg primes and proof of the theorem}
In this article we consider modular forms $f$ of weight $2$ on $\Gamma_0(N)$ and we work with primes $p$ dividing $N$ once. Recall that in this case the local type of (the automorphic form attached to) $f$ at $p$ is Steinberg.
 
Here we state a useful lemma related to the local behavior of the mod $\ell$ Galois representations at a Steinberg prime.

\begin{lemma}[Langlands]\label{locallemma}
Let $f=q+\sum_{n\geq 2}a_n q^n $ be a newform of weight $2$ on $\Gamma_0(N)$ and let $p\parallel N$, $p$ and $\ell$ coprime. Then one has that
$$
\bar\rho_{f,\ell}\vert_{D_p} \sim 
\left(
\begin{array}{cc}
\bar\chi\cdot\bar\varepsilon_\ell  &*\\
0	&\bar\chi
\end{array}
\right)
$$
where $\bar\chi:D_p\rightarrow \mathbb F^*_\ell$ denotes the unramified character that maps $\text{Frob}_p$ to $a_p$ and $\bar\varepsilon_\ell$ denotes the $\ell$-adic cyclotomic character.\footnote{Recall that when $f$ is of weight $2$ on $\Gamma_0(N)$ and $p\parallel N$ then $a_p\in \{\pm 1\}$. 
}

\end{lemma}
\begin{proof}
See \cite{LW} proposition 2.8.
\end{proof}
\begin{theorem}[Main result]\label{THEtheorem}
Let $f$ be a newform of weight $2$ on $\Gamma_0(N)$ and let $p\neq \ell$ be a prime dividing $N$ once.
Assume that the mod $\ell$ Galois representation $\bar\rho_{f,\ell}$ is absolutely irreducible, $\ell \neq 2$. Then the following are equivalent:
\begin{enumerate}[(a)]
\item $\bar{\rho}_{f,\ell}$ is unramified at $p$ and 
$$
p \equiv -1 \pmod \ell.
$$
\item There is a newform $f'$ of weight $2$ on $\Gamma_0(N)$ such that $\bar\rho_{f',\ell}$ is isomorphic to $\bar\rho_{f,\ell}$ and 
$$
a_p = -a'_p
$$
where $a_p$ (resp. $a'_p$) is the $p$-th Fourier coefficient of $f$ (resp. $f'$).
\end{enumerate}
\end{theorem}
\begin{proof}
\emph{b implies a}

Let's write $\bar\rho = \bar\rho_{f,\ell}$ and $\bar{\rho}'=\bar\rho_{f',\ell}$ for simplicity. Let $D_p\subseteq Gal(\overline{\mathbb Q}\mid\mathbb Q)$ be a decomposition group of $p$. Since $\{a_p,a'_p\} = \{1, -1\}$, we may assume without lost of generality that  $\bar\rho$, $\bar\rho'$ act locally at $p$ as
$$
\bar\rho \vert_{D_p}\sim
\left(
\begin{array}{cc}
\bar\varepsilon_\ell  & *\\
0 		  &1
\end{array}
\right)
$$
and 
$$
\bar\rho' \vert_{D_p}\sim
\left(
\begin{array}{cc}
\bar\chi\cdot\bar\varepsilon_\ell  & *\\
0 		  &\bar\chi
\end{array}
\right)
$$
due to lemma \ref{locallemma}.
Since $\bar\rho$ and $\bar\rho'$ are isomorphic so are their local behaviors. Thus, specializing at  a Frobenius map
$$
\left(
\begin{array}{cc}
p  & *\\
0 		  &1
\end{array}
\right)
\sim
\left(
\begin{array}{cc}
-p  & *\\
0 		  &-1
\end{array}
\right)\pmod \ell.
$$
Eigenvalues must coincide and $\ell>2$ so
$$
p\equiv -1 \pmod \ell
$$
To see that $\bar\rho$ is unramified at $p$ notice that 
$$
\chi \varepsilon_\ell\not\equiv \varepsilon_\ell.
$$
Thus, $\bar\rho\vert_{D_p}\simeq \bar\varepsilon_\ell \oplus\bar\chi\bar\varepsilon_\ell$ and $\bar\rho$ is unramified since $\varepsilon_\ell$ and $\chi$ are so.

\emph{a implies b}\\
Ribet's lowering level theorem applies to the modular representation $\bar\rho_{f,\ell}$ of level $N$. Thus, there exists a newform $g$ of weight $2$ on $\Gamma_0(N')$ such that $\bar\rho_{g,\ell}\sim \bar\rho_{f,\ell}$. Moreover, we have that
\begin{align*}
\text{tr} \bar\rho_{f,\ell}(\text{Frob}_p)&\equiv a_p\cdot (p+1)\\
							&\equiv 0  \pmod \ell
\end{align*}
by lemma \ref{locallemma}.
Now we can apply Ribet's raising level theorem to $\bar\rho_{g,\ell}$ and there exists a newform $f'$ on $\Gamma_0(N)$ such that 
$$
\bar\rho_{f',\ell}\sim\bar\rho_{f,\ell}.
$$  
By \S $3$ page $9$ of \cite{RibetRai}, when both conditions 
$$
\text{tr}\bar \rho_{g,\mathfrak l} ( Frob_p )\equiv \pm (p+1) \pmod \ell
$$
are satisfied Ribet's proof allows us to choose the $a_p$ coefficient of $f'$. We shall choose $f'$ such that $a'_p = - a_p$ and the implication holds.\qed
\end{proof}
\section{An example}
In this section we are going to give an example of mod $5$ Galois representation to which our theorem applies. We will use many well-known properties of elliptic curves without proof.

Let $\ell$ be a prime, $n>0$ an integer and $E$ an elliptic curve over $\mathbb Q$. The $\ell^n$-th torsion group $E[\ell^n]$ of $E$ has a natural structure of free $\mathbb Z/\ell^n\mathbb Z$-module of rank $2$. 
The action of the Galois group $G=Gal(\overline{\mathbb Q}\mid\mathbb Q)$ is compatible with the $\mathbb Z/\ell^ n \mathbb Z$ module structure of  $E[\ell^n]$, so that $E[\ell^n]$ has a natural structure of $\mathbb \mathbb Z/\ell^n\mathbb Z[G]$-module.  
That is, the action induces a group homomorphism
$$
\bar\rho_{E,\ell^n}: Gal(\overline{\mathbb Q}\mid \mathbb Q)\longrightarrow Aut_{\mathbb Z/\ell^n\mathbb Z} (E[\ell^n])\simeq GL_2(\mathbb Z/\ell^n\mathbb Z)
$$
where the isomorphism depends on the choice of a basis in $E[\ell^n]$. The case $n=1$ is of special interest and is known as the mod $\ell$ Galois representation attached to $E$. The Tate module $\mathcal T_\ell\, E$ of $E$ at $\ell$ is a free $\mathbb Z_\ell$-module of rank $2$. The morphisms $\{\bar\rho_{E,\ell^n}\}$ induce a group morphism
$$
\rho_{E,\ell}:Gal(\overline{\mathbb Q}\mid \mathbb Q)\longrightarrow Aut (\mathcal T_\ell \, E)\simeq GL_2(\mathbb Z_\ell)
$$
known as the $\ell$-adic Galois representation attached to $E$. The mod $\ell$ Galois representation $\bar\rho_{E,\ell}$ attached to  $E$ can be recovered from $\rho_{E,\ell}$ by taking reduction mod $\ell\mathbb Z_\ell$.

Well-known modularity
 theorems (Wiles, Taylor-Wiles, Breuil-Conrad-Diamond-Taylor theorems) state that such a $\ell$-adic Galois representation is isomorphic to the Galois representation $\rho_{f_E,\ell}$ attached
 to some newform $f_E$ of weight $2$ on $\Gamma_0(N)$ for $N$ equal to the conductor of $E$ and $K_f=\mathbb Q$.
 Moreover the $p$-th Fourier coefficient of 
 $f$ coincides with the $c_p$ coefficient of $E$ (defined below) for every prime $p$. In this section we will apply theorem \ref{THEtheorem} to the mod $5$ Galois representation $\bar\rho_{E,5}\simeq \bar\rho_{f_E,5}$ attached to  the following elliptic curve given by a (global minimal) Weierstras equation.
$$
E:ZY^2 +XYZ + YZ^2 = X^3 + X^2Z -614X Z^2-5501Z^3.
$$
Its discriminant is
$$
\Delta = 2^5\cdot19^5\cdot 37\\
$$
For every prime $p$ let $\tilde E_p$ denote the curve obtained by reducing mod $p$ a Global minimal Weierstrass model of $E$. As usual, we consider the value
$$
c_p=p+1 -\#\tilde E_p
$$
for every prime $p$.
One can check that 
$$
\begin{cases}
\text{$E_2$ has a node whose tangent lines are defined over $\mathbb F_2$,}\\
\text{$E_p$ has a node whose tangent lines have slopes in $\mathbb F_{p^2}\setminus \mathbb F_p$  for $p\in \{19,37\}$,}\\
\text{$E_p$ is an elliptic curve over $\mathbb F_p$, otherwise.}
\end{cases}
$$
Thus $E$ has conductor $N= 2\cdot 19\cdot 37=1406$, $c_2 = -c_{19}=-c_{37}=1$ and there is a newform $f=q+\sum_{n\geq 2} a_n q^n$ of weight $2$ on $\Gamma_0(1406)$ such that 
\begin{itemize}
\item $\rho_{E,5}$ and $\rho_{f,5}$ are isomorphic.
\item
$c_p = a_p$ for every prime $p$.
\end{itemize}
Let's see now that $\bar\rho_{f,5}$ satisfies the hypothesis of the theorem \ref{THEtheorem} for $p=19$. Since
\begin{align*}
E_3&=\{[x:y:z]\in \mathbb P^2_{\mathbb F_3}: zy^2 + xyz+ yz^2 = x^3+x^2z+xz^2+z^3\}\\
	&=\{[0:1:0], [2:0:1]\}
\end{align*}
then $c_3= 2$ and the characteristic polynomial of $\bar\rho_{f,5}(Frob_3)$ is congruent to
$$
P(X) = X^2- 2X + 3   \pmod 5.
$$
Since the discriminant of $P(x)$ is $2\pmod 5$ and $2$ is not a square in $\mathbb F_5$, then $P(x)$ is irreducible over $\mathbb F_5$. In particular $\bar\rho_{f,5}:Gal(\overline{\mathbb Q}\mid\mathbb Q)\longrightarrow GL_2(\mathbb F_5)$ is irreducible. It's well known that such a representation is irreducible if and only if it is absolutely irreducible
Hence, $\bar\rho_{f,5}$ is absolutely irreducible. 

On the other hand, it's well known (see \cite{DDT} proposition 2.12) that for a prime $p\neq \ell$ dividing once the conductor of an elliptic curve $E$, $\bar\rho_{E,\ell}$ is  unramified at $p$ if and only if $\ell\mid v_{p}(\Delta)$. Since
$
v_{19}(\Delta)=5
$
thus $\bar\rho_{f,5}$ is also unramified at $19$. Hence theorem \ref{THEtheorem} applies to $\bar\rho_{E,5}$ and there exists another newform $f'$ of weight $2$ on $\Gamma_0(1406)$ 
such that $\bar\rho_{f',\ell}$ is isomorphic to $\bar\rho_{E,5}$. Thus, 
$$
c_p \equiv a_p \pmod {\mathfrak m_{f',\ell}}.
\qquad\text{for every $p\nmid 1406\cdot 5$}.
$$
and the $19$th Fourier coefficient $a'_{19}$ of $f'$ satisfies
$$
a'_{19}=-c_{19}=1.
$$
Theorem \ref{THEtheorem} will allow us to determine $a'_2$ and $a'_{37}$ as well. If we assume that $a'_2=-a_2$, since $\bar\rho_{f,5}$ and $\bar\rho_{f',\ell}$ are isomorphic and $\bar\rho_{f,\ell}$ is irreducible theorem \ref{THEtheorem} applies with $\ell=5$ and $p=2$ so we would conclude that $2\equiv -1\pmod 5$. Hence, $a'_2\neq -a_2$. Since $a'_2\in\{\pm 1\}$ then $a'_2 = a_2=1$. Similarly one can prove that $a'_{37}= a_{37}=-1$.

In this particular case one can prove that $f'$ corresponds to the newform attached to some elliptic curve $E'$ of conductor $N$. 
For that purpose, we search for $f'$ at some table of newforms of weight $2$ on $\Gamma_0(1406)$ (see \cite{Stein}). One can check that there are $16$ of them up to Galois conjugation. 
One determines by computing $c_3=2$ and $(c_2,c_{19}, c_{37})=(1,-1,-1)$ coefficients that our initial curve $E$ corresponds to the $8$th newform of \cite{Stein}. 
To determine $f'$ first recall that due to theorem \ref{THEtheorem} we have that $a'_2=a'_{19}=-a'_{37}=1$ and we can eliminate all but $4$th, $5$th and $6$th newforms. 
In this case it is enough to check that the only one satisfying $a_3\equiv c_3=2 \pmod 5$ is the $4$th newform. 
Thus, in this case $f'$ is unique. Moreover, $f'$ corresponds to a rational elliptic curve $E'$ (well defined up to isogeny) since $[K_{f'}:\mathbb Q]=1$.  

Finally, we will give a minimal Weierstrass equation of $E'$. To do so we compute the set of isogeny classes of elliptic curves with conductor $1406$. There are $8$ isogeny classes (see \cite{Cremona2006}) and three of them (labeled by $e$, $f$ and $g$) satisfy $a_2=a_{19}=-a_{37}=1$. Hence, the set of isogeny classes of elliptic curves $\{e,f,g\}$ corresponds to the set of conjugacy classes of newforms $\{4,5,6\}$ but the isogeny class labeled by $g$ is the only one that satisfies $c'_3\equiv 2 \pmod 3$. It is due to Cremona that the elliptic curve given by the Weierstrass equation
$$
E': Y^2 Z + XYZ + YZ^2 = X^3- X^2 Z-1191X Z^2 +507615 Z^3.
$$
is a representative of the isogeny class $g$. So, we have been able to explicit all the modular forms whose existence was predicted by theorem \ref{THEtheorem}.
\begin{remark}
It's not true in general that the modular form $f'$ obtained in our theorem is unique. 
On the other hand, we have found examples of pairs $(f,f')$ of modular forms as in theorem \ref{THEtheorem} for which $[K_{f'}:\mathbb Q]>[K_{f}:\mathbb Q]=1$ or examples for which $K_f$, $K_{f'}\neq \mathbb Q$.
\end{remark}

 \begin{thebibliography}{AM} 

 \bibitem{Cremona2006} Cremona, John:
 \emph{The elliptic curve database for conductors to 130000.}
 Algorithmic number theory, 
 11--29, Lecture Notes in Comput. Sci., 4076, Springer, Berlin  (2006).  
 
\bibitem{DDT} Darmon, Henri ;  Diamond, Fred ;  Taylor, Richard:
\emph{Fermat's last theorem. Elliptic curves, modular forms and Fermat's last theorem
 (Hong Kong,
 1993)}, 
 2--140, Int. Press, Cambridge, MA  (1997). 

 
 \bibitem{LW} Loeffler, David ;  Weinstein, Jared:
 \emph{On the computation of local components of a newform.}
 Math. Comp.  \textbf{81}, no. 278, 1179--1200  (2012).
              
\bibitem{RibetRai} Ribet, Kenneth A.:  \emph{Raising the levels of modular representations.}
 Séminaire de Théorie des Nombres, Paris 1987--88, 
 259--271, Progr. Math. \textbf{81}, Birkh\"auser Boston, Boston, MA,  (1990). 

\bibitem{RibetLow}  Ribet, Kenneth A.:\emph{Lowering the levels of modular representations without multiplicity
 one.}
 Internat. Math. Res. Notices,  no. \textbf{2}, 15--19 (1991).

\bibitem{Stein} Stein, William:  \emph{Arithmetic data about every weight 2 newform on $\Gamma_0(N)$.}\\
\href{http://modular.math.washington.edu/Tables/arith_of_factors/data/}{http://modular.math.washington.edu/Tables/arith\_of\_factors/data/}

\end{thebibliography}

\end{document}


\documentclass[12pt,a4paper,reqno]{amsart} 
\pagestyle{plain}
\usepackage{amssymb}
\usepackage{latexsym}
\usepackage{amsmath}
\usepackage{mathrsfs}
\usepackage{cite}

\usepackage{calc}                   
                    

     

\setcounter{section}{\value{section}-1}   

\numberwithin{equation}{section}          

\newtheorem{thm}{Theorem}
\numberwithin{thm}{section}
\newtheorem*{tom}{Theorem \ref{hypoellthm}$'$}

\newtheorem{prop}[thm]{Proposition}
\newtheorem{cor}[thm]{Corollary}
\newtheorem{lemma}[thm]{Lemma}

\theoremstyle{definition}

\newtheorem{defn}[thm]{Definition}
\newtheorem{example}[thm]{Example}

\theoremstyle{remark}

\newtheorem{rem}[thm]{Remark}              
                                           
                       
                       

                      

\author{Sandro Coriasco}

\address{Department of mathematics, Turin's University, Italy}

\email{sandro.coriasco@unito.it}

\author{Karoline Johansson}

\address{Department of Mathematics and Systems Engineering,
V{\"a}xj{\"o} University, Sweden}

\email{karoline.johansson@vxu.se}

\author{Joachim Toft}

\address{Department of Mathematics and Systems Engineering,
V{\"a}xj{\"o} University, Sweden}

\email{joachim.toft@vxu.se}

\title{Wave-front sets of Banach function types}

\keywords{Wave-front, Fourier, Banach, modulation, micro-local}

\subjclass[2000]{35A18,35S30,42B05,35H10}

\frenchspacing

\begin{document}

\begin{abstract}
Let $\omega ,\omega _0$ be appropriate weight functions and $\mathscr B$ be
an invarian BF-space.
We introduce the wave-front set, ${\operatorname{WF}}_{{\mathscr F\! \mathscr B}(\omega )}(f)$ of the distribution $f$
with respect to weighted Fourier Banach space ${\mathscr F\! \mathscr B}(\omega )$. We prove
that usual mapping properties for pseudo-differential operators
${\operatorname{Op}} _t (a)$ with symbols $a$ in $S^{(\omega _0)}_{\rho ,0}$ hold for
such wave-front sets. In particular we prove
$\displaystyle
{\operatorname{WF}} _{{\mathscr F\! \mathscr B}(\omega /\omega _0)}({\operatorname{Op}} _t (a) f)\subseteq {\operatorname{WF}} _{{\mathscr F\! \mathscr B}(\omega
)}(f)$ and ${\operatorname{WF}} _{{\mathscr F\! \mathscr B}(\omega
)}(f)
\subseteq {\operatorname{WF}} _{{\mathscr F\! \mathscr B}(\omega /\omega
_0)}({\operatorname{Op}} _t (a) f){{\textstyle{\, \bigcup \, }}} {\operatorname{Char}} (a)$.
Here ${\operatorname{Char}} (a)$ is the set of characteristic points of $a$.
\end{abstract}

\maketitle

\section{Introduction}\label{sec0}

\par

In this paper we introduce
wave-front sets with respect to Fourier images of translation
invariant BF-spaces. The family of such
wave-front sets contains the wave-front sets of Sobolev type,
introduced by H{\"o}rmander in \cite {Hrm-nonlin}, the
classical wave-front sets (cf. Sections 8.1
and 8.2 in \cite {Ho1}), and wave-front sets of Fourier Lebesgue
types, introduced in \cite{PTT1}. Roughly speaking, for any
given distribution $f$ and for appropriate Banach (or Frech\'et) space
$\mathcal B$ of tempered distributions, the wave-front set
${\operatorname{WF}}  _{\mathcal B}(f)$ of $f$ consists of all pairs
$(x_0,\xi _0)$ in ${\mathbf R^{d}}\times ({\mathbf R^{d}}{\setminus 0})$ such that no
localizations of the distribution at $x_0$ belongs to $\mathcal B$ in
the direction $\xi _0$.

\par

We also establish mapping properties for a quite general class of
pseudo-differential operators on such wave-front sets, 
and show that the micro-local analysis in \cite{PTT1} in background
of Fourier Lebesgue spaces can be further generalized. It follows that our approach
gives rise to flexible micro-local analysis tools which fit well to
the most common approach developed in e.g. \cite {Ho1,
Hrm-nonlin}. In particular, we prove that
usual mapping properties, which are valid for classical wave-front
sets (cf. Chapters VIII and XVIII in \cite{Ho1}), also hold for
wave-front sets of Fourier Banach types.
For example, we show
\begin{equation}\label{eq:inclusions}
\begin{aligned}
{\operatorname{WF}} _{{\mathscr F\! \mathscr B}(\omega /\omega _0)}({\operatorname{Op}} _t (a) &f)\subseteq {\operatorname{WF}} _{{\mathscr F\! \mathscr B}(\omega
)}(f)\\
&
\subseteq {\operatorname{WF}} _{{\mathscr F\! \mathscr B}(\omega /\omega
_0)}({\operatorname{Op}} _t (a) f){{\textstyle{\, \bigcup \, }}} {\operatorname{Char}} (a).
\end{aligned}
\end{equation}
That is, any operator
${\operatorname{Op}} (a)$ shrinks the wave-front sets and opposite embeddings can be
obtained by including ${\operatorname{Char}} (a)$, the set of characteristic points of
the operator symbol $a$.

\par

The symbol classes for the pseudo-differential operators are denoted by
$S_{\rho ,\delta}^{(\omega _0)}({\mathbf R^{{2d}}})$, the set of all
smooth functions $a$ on ${\mathbf R^{{2d}}}$ such that $a/\omega _0\in S^0_{\rho
,\delta}({\mathbf R^{{2d}}})$. Here $\rho ,\delta \in \mathbf R$ and $\omega_0$ is
an appropriate smooth function on ${\mathbf R^{{2d}}}$. We note that $S_{\rho ,\delta}^{(\omega _0)}({\mathbf R^{{2d}}})$ agrees
with the H{\"o}rmander class $S^r_{\rho ,\delta}({\mathbf R^{{2d}}})$ when
$\omega _0(x, \xi )={\langle \xi\rangle} ^r$, where $r\in \mathbf R$ and ${\langle \xi\rangle}
=(1+|\xi |^2)^{1/2}$.

\par

The set of characteristic points ${\operatorname{Char}} (a)$ of $a\in
S_{\rho ,\delta}^{(\omega )}$ is the same as in \cite{PTT1}, and
depends on the choices of
$\rho$, $\delta$ and $\omega$ (see Definition \ref{defchar} and
Proposition \ref{psiecharequiv}).
We recall that this set is smaller than the set of characteristic
points given by \cite{Ho1}. It is empty when $a$ satisfies a local
ellipticity condition
with respect to $\omega$, which is fulfilled for any hypoelliptic
partial differential operator with constant coefficients
(cf. \cite{PTT1}). As a consequence of \eqref{eq:inclusions}, it follows that such hypoelliptic
operators preserve the wave-front sets, as expected (cf. Example
3.9 in \cite{PTT1}).

\medspace

Information on regularity in background of wave-front sets of
Fourier Banach types might be more detailed compared to classical
wave-front sets, because of our choices of different weight functions
$\omega$ and Banach spaces when defining our Fourier Banach space
${\mathscr F\! \mathscr B}(\omega )({\mathbf R^{d}})$. For example, the space ${\mathscr F\! \mathscr B}(\omega)=\mathscr
FL^1_{(\omega )}({\mathbf R^{d}})$, with $\omega (x,\xi )={\langle \xi\rangle} ^N$ for some
integer $N\ge 0$, is locally close to $C^N({\mathbf R^{d}})$ (cf. the Introduction of
\cite{PTT1}). Consequently, the wave-front set with respect to $\mathscr FL^1_{(\omega )}$ can be
used to investigate a sort of regularity which is close to smoothness
of order $N$.

\medspace

Furthermore, we are able to apply our results on pseudo-differential
operators in context of modulation space theory, when discussing
mapping properties of pseudo-differential operators with respect to
wave-front sets.  
The modulation spaces were introduced by Feichtinger in \cite{F1}, and
the theory was developed in
\cite{Feichtinger3, Feichtinger4, Feichtinger5, Grochenig0a}. The
modulation space $M(\omega, \mathscr B )$, where $\omega$ is
a weight function (or time-frequency shift) on phase space ${\mathbf R^{{2d}}}$,
appears as the set of temperated (ultra-)distributions
whose short-time Fourier transform belong to the weighted Banach space
$\mathscr B(\omega )$.
These types of modulation spaces contains the (classical) modulation
spaces $M^{p,q}_{(\omega)}({\mathbf R^{{2d}}})$ as well as the space
$W^{p,q}_{(\omega)}({\mathbf R^{{2d}}})$ related to the Wiener amalgam spaces,
by choosing $\mathscr B =L^{p,q}_1({\mathbf R^{{2d}}})$ and $\mathscr B
=L^{p,q}_2({\mathbf R^{{2d}}})$ respectively (see Remark \ref{Modamalgam}).
In the last part of the paper we define wave-front sets with respect to
weighted modulation spaces, and prove that they coincide with the
wave-front sets of Fourier Banach types.

\par

Parallel to this development, modulation spaces have been
incorporated into the calculus of pseudo-differential operators, in
the sense of the study of continuity of (classical)
pseudo-differential operators acting on modulation spaces (cf.
\cite{Tachizawa1,Czaja,Pilipovic2,Pilipovic3,Teofanov1,Teofanov2}),
and the study of operators of non-classical type, where modulation
spaces are used as symbol classes. We refer to \cite{Gro-book, Grochenig2, Grochenig0,
Grochenig1b, Grochenig1c, Herau1, HTW, Pilipovic2,
Sjostrand1, Sjostrand2, Toft2, Toft35, To8, Toft4} for more facts about pseudo-differential operators in background of modulation space theory.

\medspace

The paper is organized as follows. In Section
\ref{sec1} we recall the definition and basic properties for
pseudo-differential operators, translation invariant Banach function
spaces (BF-spaces) and (weighted) Fourier Banach
spaces. Here we also define sets of characteristic points for a broad
class of pseudo-differential operators.
In Section \ref{sec2} we prove some properties for the sets of
characteristic points, which shows that our definition coincide with
the sets of characteristic points defined in \cite{PTT1}. These sets
might be smaller than characteristic sets in
\cite{Ho1} (cf. \cite[Example 3.11]{PTT1}).

\par 

In Section \ref{sec3} we define wave-front sets with respect to
(weighted) Fourier Banach spaces, and prove some of their
main properties. Thereafter, in Section \ref{sec4} we
show how these wave-front sets are propagated under the action
of pseudo-differential operators. In particular, we prove \eqref{eq:inclusions},
when $\omega _0$ and $\omega$ are appropriate weights and
$a$ belongs to $S^{(\omega _0)}_{\rho ,0}$ with
$\rho >0$.

\par

In Section \ref{sec5} we consider wave-front sets obtained from
sequences of Fourier Banach spaces. These types of wave-front sets
contain the classical ones (with respect to smoothness), and the mapping
properties for pseudo-differential operators also hold in this context
(cf. Section 18.1 in \cite{Ho1}).

\par

Finally, Section \ref{sec6} is devoted to study the definition and basic properties of
wave-front sets with respect to modulation spaces. We prove that they can be identified with certain
wave-front sets of Fourier Banach types.

\par

\section{Preliminaries}\label{sec1}

\par

In this section we recall some notation and basic results. The proofs
are in general omitted. In what
follows we let $\Gamma$ denote an open cone in ${\mathbf R^{d}}{\setminus 0}$. If
$\xi \in {\mathbf R^{d}}{\setminus 0}$ is fixed, then an open cone which contains
$\xi $ is sometimes denoted by $\Gamma_\xi$.

\par

Assume that $\omega, v\in L^\infty _{loc}({\mathbf R^{d}})$ are positive
functions. Then $\omega$ is called $v$-moderate if
\begin{equation}\label{moderate}
\omega (x+y) \leq C\omega (x)v(y)
\end{equation}
for some constant $C$ which is independent of $x,y\in {\mathbf R^{d}}$. If $v$
in \eqref{moderate} can be chosen as a polynomial, then $\omega$ is
called polynomially moderate. We let $\mathscr P({\mathbf R^{d}})$ be the set
of all polynomially moderated functions on ${\mathbf R^{d}}$. We say that $v$ is
\emph{submultiplicative} when \eqref{moderate} holds with $\omega =v$.
Throughout we assume that the submultiplicative weights are even.
If $\omega (x,\xi )\in \mathscr P({\mathbf R^{{2d}}})$ is constant with respect
to the $x$-variable ($\xi$-variable), then we sometimes write $\omega
(\xi )$ ($\omega (x)$) instead of $\omega (x,\xi )$. In this case we
consider $\omega$ as an element in $\mathscr P({\mathbf R^{{2d}}})$ or in
$\mathscr P({\mathbf R^{d}})$ depending on the situation.

\par

We also need to consider classes of weight functions, related to
$\mathscr P$. More precisely, we let $\mathscr P_0({\mathbf R^{d}})$ be the set
of all $\omega \in \mathscr P({\mathbf R^{d}})\bigcap C^\infty ({\mathbf R^{d}})$ such
that $\partial ^\alpha \omega /\omega \in L^\infty$ for all
multi-indices $\alpha$.
For each $\omega \in \mathscr P({\mathbf R^{d}})$, there is an equivalent
weight $\omega _0\in \mathscr P_0({\mathbf R^{d}})$, that is,
$C^{-1}\omega _0\le \omega \le C\omega _0$ holds for some constant
$C$ (cf. \cite[Lemma 1.2]{To8}).

\par

Assume that $\rho ,\delta \in \mathbf R$. Then we let $\mathscr
P_{\rho ,\delta}({\mathbf R^{{2d}}})$ be the set of all $\omega (x,\xi )$ in
$\mathscr P({\mathbf R^{{2d}}})\cap C^\infty ({\mathbf R^{{2d}}})$ such that
$$
{\langle \xi\rangle} ^{\rho |\beta |-\delta |\alpha |}\frac {(\partial ^\alpha
_x\partial ^\beta _\xi \omega )(x,\xi )}{\omega (x,\xi )}\in L^\infty
({\mathbf R^{{2d}}}),
$$
for every multi-indices $\alpha$ and $\beta$. Note that in contrast to
$\mathscr P_0$, we do not have an equivalence between $\mathscr
P_{\rho ,\delta}$ and $\mathscr P$ when $\rho >0$. On the other hand,
if $s\in \mathbf R$ and $\rho \in [0,1]$, then $\mathscr P_{\rho
,\delta} ({\mathbf R^{{2d}}})$ contains $\omega (x,\xi )={\langle \xi\rangle} ^s$, which
are one of the most important classes in the applications.

\par

For any weight $\omega$ in $\mathscr P({\mathbf R^{d}})$ or in $\mathscr P_{\rho
,\delta}({\mathbf R^{d}})$, we let $L^p_{(\omega )}({\mathbf R^{d}})$ be the set of all
$f\in L^1_{loc}({\mathbf R^{d}})$ such that $f\cdot \omega \in L^p({\mathbf R^{d}})$.

\medspace

The Fourier transform $\mathscr F$ is the linear and continuous
mapping on $\mathscr S'({\mathbf R^{d}})$ which takes the form
$$
(\mathscr Ff)(\xi )= \widehat f(\xi ) \equiv (2\pi )^{-d/2}\int _{{\mathbf R^{{d}}}} f(x)e^{-i{\langle x,\xi\rangle} }\, dx
$$
when $f\in L^1({\mathbf R^{d}})$. We recall that $\mathscr F$ is a homeomorphism
on $\mathscr S'({\mathbf R^{d}})$ which restricts to a homeomorphism on $\mathscr
S({\mathbf R^{d}})$ and to a unitary operator on $L^2({\mathbf R^{d}})$.

\medspace

Next we recall the definition of Banach function spaces.

\par

\begin{defn}\label{BFspaces}
Assume that $\mathscr B$ is a Banach space of complex-valued
measurable functions on ${\mathbf R^{d}}$ and that $v \in \mathscr P({\mathbf R^{{d}}})$
is submultiplicative.
Then $\mathscr B$ is called a \emph{(translation) invariant
BF-space on ${\mathbf R^{d}}$} (with respect to $v$), if there is a constant $C$
such that the following conditions are fulfilled:
\begin{enumerate}
\item $\mathscr S({\mathbf R^{d}})\subseteq \mathscr
B\subseteq \mathscr S'({\mathbf R^{d}})$ (continuous embeddings);

{\vspace{0.1cm}}

\item if $x\in {\mathbf R^{d}}$ and $f\in \mathscr B$, then $f(\cdot -x)\in
\mathscr B$, and
\begin{equation}\label{translmultprop1}
{\Vert {f(\cdot -x)}\Vert _{{\mathscr B}}}\le Cv(x){\Vert {f}\Vert _{{\mathscr B}}}\text ;
\end{equation}

{\vspace{0.1cm}}

\item if $f,g\in L^1_{loc}({\mathbf R^{d}})$ satisfy $g\in \mathscr B$  and $|f|
\le |g|$ almost everywhere, then $f\in \mathscr B$ and
$$
{\Vert f\Vert _{{\mathscr B}}}\le C{\Vert g\Vert _{{\mathscr B}}}\text .
$$
\end{enumerate}
\end{defn}

\par

Assume that $\mathscr B$ is a translation invariant BF-space. If $f\in
\mathscr B$ and $h\in L^\infty$, then it follows from (3) in
Definition \ref{BFspaces} that $f\cdot h\in \mathscr B$ and
\begin{equation}\label{multprop}
{\Vert {f\cdot h}\Vert _{{\mathscr B}}}\le C{\Vert f\Vert _{{\mathscr B}}}{\Vert h\Vert _{{L^\infty}}}.
\end{equation}

\par

\begin{rem}\label{newbfspaces}
Assume that $\omega _0,v,v_0\in \mathscr P({\mathbf R^{d}})$ are such $v$ and
$v_0$ are submultiplicative,
$\omega _0$ is $v_0$-moderate, and assume that $\mathscr B$ is a
translation-invariant BF-space on ${\mathbf R^{d}}$ with respect to $v$. Also
let $\mathscr B_0$ be the Banach space which consists of all $f\in
L^1_{loc}({\mathbf R^{d}})$ such that ${\Vert f\Vert _{{\mathscr B_0}}}\equiv {\Vert {f\, \omega _0
}\Vert _{{\mathscr B}}}$ is finite. Then $\mathscr B_0$ is a translation
invariant BF-space with respect to $v_0v$.
\end{rem}

\par

\begin{rem}\label{BFemb}
Let $\mathscr B$ be an invariant BF-space. Then it is easy to find
Sobolev type spaces which are continuously embedded in $\mathscr
B$. In fact, for each $p\in [1,\infty ]$ and integer $N\ge 0$, let
$Q^p_N({\mathbf R^{d}})$ be the set of all $f\in L^p({\mathbf R^{d}})$ such that ${\Vert f\Vert _{{Q^p_N}}}<\infty$, where
$$
{\Vert f\Vert _{{Q^p_N}}}\equiv \sum _{|\alpha +\beta |\le N}{\Vert {x^\alpha
D^\beta f}\Vert _{{L^p}}}.
$$
Then for each $p$ fixed, the topology for $\mathscr S({\mathbf R^{d}})$ can be
defined by the semi-norms $f\mapsto {\Vert f\Vert _{{Q^p_N}}}$, for $N=0,1,\dots
$.

\par

A combination of this fact and (1) and (3) in Definition
\ref{BFspaces} now shows that for each $p\in [1,\infty ]$ and each
translation invariant BF-space $\mathscr B$, there is an integer $N\ge
0$ such that $Q^p_N({\mathbf R^{d}})\subseteq \mathscr B$. Moreover, let
$L^\infty _N({\mathbf R^{d}})$ be the set of all $f\in L^\infty _{loc}({\mathbf R^{d}})$
such that $f\, {\langle {\, \cdot \, }\rangle} ^N\in L^\infty$. Then, since any
element in $L^\infty _N$ can be majorized with an element in
$Q^\infty _N$, it follows from (3) in Definition \ref{BFspaces} that
$L^\infty _N\subseteq \mathscr B$, provided $N$ is chosen large
enough. This proves the assertion.
\end{rem}

\par

For future references we note that if $\mathscr B$ is a translation
invariant BF-space with respect to the submultiplicative weight $v$ on
${\mathbf R^{d}}$, then the convolution map $*$ on $\mathscr S({\mathbf R^{d}})$ extends
uniquely to a continuous mapping from $\mathscr B\times L^1_{(v)}({\mathbf R^{d}})$, and for some constant $C$ it holds
\begin{equation}\label{propupps}
{\Vert {{\varphi} *f}\Vert _{{\mathscr B}}}\le C{\Vert {\varphi}\Vert _{{L^1_{(v)}}}}{\Vert f\Vert _{{\mathscr B}}},\quad
{\varphi} \in L^1_{(v)}({\mathbf R^{d}}),\ f\in \mathscr B.
\end{equation}
In fact, if $f\in \mathscr B$ and $g$ is a step function, then $f*g$ is
well-defined and belongs to $\mathscr B$ in view of the definitions, and
Minkowski's inequality gives
\begin{multline*}
{\Vert {f*g}\Vert _{{\mathscr B}}} =\Big \Vert \int f({\, \cdot \, } -y)g(y)\, dy \Big \Vert
_{\mathscr B}
\\[1ex]
\le \int {\Vert { f({\, \cdot \, }-y)}\Vert _{{\mathscr B}}}|g(y)|\, dy \le C\int {\Vert {
f}\Vert _{{\mathscr B}}}|g(y)v(y)|\, dy = C{\Vert { f}\Vert _{{\mathscr B}}}{\Vert g\Vert _{{L^1_{(v)}}}}.
\end{multline*}

\par

Now assume that $g\in C_0^\infty$. Then $f*g$ is well-defined as an
element in $\mathscr S' \cap C^\infty$, and by approximating $g$ with
step functions and using \eqref{propupps} it follows that $f*g\in
\mathscr B$ and that \eqref{propupps} holds also in this case.
The assertion now follows from this fact and a simple argument of
approximations, using the fact that $C^\infty _0$ is dense in
$L^1_{(v)}$.

\par

For each translation invariant BF-space $\mathscr B$ on ${\mathbf R^{d}}$, and
each pair of vector spaces $(V_1,V_2)$ such that $V_1\oplus V_2={\mathbf R^{d}}$, we define the projection spaces $\mathscr B_1$ and
$\mathscr B_2$ of $\mathscr B$ by the formulae
\begin{alignat}{2}
&\mathscr{B}_1 &  &\equiv {\{ \, {f\in \mathscr
S'(V_1)}\, ;\, {f\otimes {\varphi} \in \mathscr B \text{ for every } {\varphi} \in 
\mathscr S (V_2)}\, \} }\label{B1def}
\intertext{and}
&\mathscr{B}_2 &  &\equiv {\{ \, {f\in \mathscr
S'(V_2)}\, ;\, {{\varphi} \otimes f \in \mathscr B \text{ for every } {\varphi} \in 
\mathscr S (V_1)}\, \} }.\label{B2def}
\end{alignat}

\par

\begin{prop}\label{propbnoll}
Let $\mathscr{B}$ be a translation
invariant BF-space on ${\mathbf R^{d}}$, and let $\mathscr B_1$ and $\mathscr
B_2$ be the same as in \eqref{B1def} and \eqref{B2def}. Then 
\begin{alignat}{2}
&\mathscr{B}_1 &  &= {\{ \, {f\in \mathscr
S'(V_1)}\, ;\, {f\otimes {\varphi} \in \mathscr B \text{ for some } {\varphi} \in 
\mathscr S (V_2){\setminus 0}}\, \} }\tag*{(\ref{B1def})$'$}
\intertext{and}
&\mathscr{B}_2 &  &= {\{ \, {f\in \mathscr
S'(V_2)}\, ;\, {{\varphi} \otimes f \in \mathscr B \text{ for some } {\varphi} \in 
\mathscr S (V_1){\setminus 0}}\, \} }.\tag*{(\ref{B2def})$'$}
\end{alignat}

\par

In particular, if ${\varphi} _j\in \mathscr S (V_j){\setminus 0}$ for $j=1,2$ are
fixed and $f_1\in \mathscr{S}'(V_1)$ and $f\in \mathscr{S}'(V_2)$,
then $\mathscr B_1$ and $\mathscr B_2$ are translation
invariant BF-spaces under the norms
$$
{\Vert f\Vert _{{\mathscr B_1}}}\equiv {\Vert {f\otimes {\varphi} _1}\Vert _{{\mathscr B}}}\quad
\text{and}\quad {\Vert f\Vert _{{\mathscr B_2}}}\equiv {\Vert {{\varphi} _2\otimes
f}\Vert _{{\mathscr B}}}
$$
respectively.
\end{prop}

\par

\begin{proof}
We only prove \eqref{B2def}$'$. The other equality follows by similar
arguments and is left for the reader. We may assume that $V_j={\mathbf R^{{d_j}}}$ with $d_1+d_2=d$.

\par

Let $\mathscr B_0$ be the right-hand side of \eqref{B2def}$'$. Then it
is obvious that $\mathscr B_2\subseteq \mathscr B_0$. We have to prove
the opposite inclusion.

\par

Therefore, assume that $f\in \mathscr B_0$, and choose ${\varphi}_0 \in
\mathscr S({\mathbf R^{{d_1}}}){\setminus 0}$ such that ${\varphi} _0\otimes f \in
\mathscr{B}$. Also let ${\varphi} \in \mathscr S({\mathbf R^{{d_1}}})$ be
arbitrary. We shall prove that ${\varphi} \otimes f \in
\mathscr{B}$.

\par

Let $Q \subseteq {\mathbf R^{{d_1}}}$ be an open ball and $c>0$
be chosen such that $|{\varphi} _0(x)| >c$ when $x\in Q$. Also
let the lattice $\Lambda \subseteq {\mathbf R^{{d_1}}}$ and ${\varphi} _1\in
C^{\infty}_0(Q)$ be such that $0\leq {\varphi} _1\leq 1$ and
$$
\sum_{\{x_j\}\in \Lambda} {\varphi} _1({\, \cdot \, } -x_j)=1.
$$
Then ${\varphi} _1\leq C|{\varphi} _0|$, for some constant $C>0$, which gives
$$
{\Vert {{\varphi} _1\otimes f}\Vert _{{\mathscr B}}}\le C{\Vert {{\varphi}
_0\otimes f}\Vert _{{\mathscr B}}}<\infty .
$$
This in turn gives
\begin{multline}\label{chi0est1}
{\Vert {{\varphi} \otimes f}\Vert _{{\mathscr B}}} \le \sum {\Vert {({\varphi} _1({\, \cdot \, } -x_j){\varphi}
)\otimes f}\Vert _{{\mathscr B}}}
\\[1ex]
\le \sum  v(x_j,0){\Vert {({\varphi} _1{\varphi} ({\, \cdot \, } +x_j))\otimes f}\Vert _{{\mathscr B}}} 
\\[1ex]
\le C
\Big (\sum  v(x_j,0){\Vert {{\varphi} ({\, \cdot \, } +x_j)}\Vert _{{L^\infty (Q)}}}\Big ) {\Vert {{\varphi} _1\otimes f}\Vert _{{\mathscr B}}}.
\end{multline}
Since $v\in \mathscr P$ and ${\varphi} \in \mathscr S$, it follows that the
sum in the right-hand side of \eqref{chi0est1} is finite. Hence $f\in
\mathscr B_2$, and the proof is complete.
\end{proof}

\par

\begin{rem}
We note that the last sum in \eqref{chi0est1} is the norm
$$
{\Vert {\varphi}\Vert _{{W_{(v)}}}} \equiv
\sum  v(x_j,0){\Vert {{\varphi} ({\, \cdot \, } +x_j)}\Vert _{{L^\infty (Q)}}}
$$ 
for the weighted Wiener space
$$
W_{(v)}({\mathbf R^{d}}) ={\{ \, {f\in L^\infty _{loc}({\mathbf R^{d}})}\, ;\, {{\Vert f\Vert _{{W_{(v)}}}}<\infty }\, \} }
$$
(cf. \cite{Gro-book}). The results in Proposition \ref{propbnoll} can therefore be improved in such way that we may replace $\mathscr S$ by $W_{(v)}$ in \eqref{B1def}, \eqref{B2def}, \eqref{B1def}$'$ and \eqref{B2def}$'$.
\end{rem}

\par

Assume that $\mathscr B$ is a translation invariant BF-space on ${\mathbf R^{d}}$, and that $\omega \in \mathscr P({\mathbf R^{d}})$. Then we let ${\mathscr F\! \mathscr B}
{(\omega )}$ be the set of all $f\in \mathscr S'({\mathbf R^{d}})$ such that
$\xi \mapsto \widehat f(\xi )\omega (x,\xi )$ belongs to $\mathscr
B$. It follows that ${\mathscr F\! \mathscr B} {(\omega )}$ is a Banach space under the
norm
\begin{equation}\label{FLnorm}
{\Vert f\Vert _{{{\mathscr F\! \mathscr B} {(\omega )}}}}\equiv {\Vert {\widehat
f\, \omega}\Vert _{{\mathscr B}}}.
\end{equation}

\par

\begin{rem} \label{whyomega}
In many situations it is convenient to permit an $x$
dependency for the weight $\omega$ in the definition of Fourier Banach spaces.
More precisely, for each $\omega \in \mathscr P({\mathbf R^{{2d}}})$ and each translation invariant BF-space $\mathscr B$ on ${\mathbf R^{d}}$, we let $\mathscr {FB}{(\omega )}$ be the set of all $f\in
\mathscr S'({\mathbf R^{d}})$ such that
$$
{\Vert f\Vert _{{\mathscr {FB}{(\omega )}}}}
= {\Vert f\Vert _{{\mathscr {FB}{(\omega),x}}}}
\equiv {\Vert {\widehat f\, \omega (x,{\, \cdot \, } )}\Vert _{{\mathscr B}}}
$$
is finite. Since $\omega$ is
$v$-moderate for some $v\in \mathscr P({\mathbf R^{{2d}}})$ it follows that
different choices of $x$ give rise to equivalent norms.
Therefore the condition
${\Vert f\Vert _{{{\mathscr F\! \mathscr B}(\omega )}}}<\infty$ is
independent of $x$, and  it follows that ${\mathscr F\! \mathscr B}(\omega )({\mathbf R^{d}})$ is  independent of $x$ although ${\Vert {\, \cdot \, }\Vert _{{{\mathscr F\! \mathscr B}(\omega )}}}$ might depend on $x$.
\end{rem}

\par

Recall that a topological vector space $V\subseteq \mathscr
D'(X)$ is called \emph{local} if $V\subseteq V_{loc}$. Here
$X\subseteq {\mathbf R^{d}}$ is open, and $V_{loc}$ consists of all $f\in
\mathscr D'(X)$ such that ${\varphi} f \in V$ for every ${\varphi} \in C_0^\infty
(X)$. For future references we note that if $\mathscr B$ is a
translation invariant BF-space on ${\mathbf R^{d}}$ and $\omega \in \mathscr
P({\mathbf R^{{2d}}})$, then it follows from \eqref{propupps} that ${\mathscr F\! \mathscr B}(\omega
)$ is a local space, i.{\,}e.
\begin{equation}\label{Blocal}
{\mathscr F\! \mathscr B} (\omega )\subseteq {\mathscr F\! \mathscr B} (\omega )_{loc}\equiv ({\mathscr F\! \mathscr B} (\omega ))_{loc}.
\end{equation}

\medspace

We need to recall some facts from Chapter XVIII in \cite {Ho1}
concerning pseudo-differential operators. Let $a\in
\mathscr S({\mathbf R^{{2d}}})$, and $t\in \mathbf R$ be fixed. Then
the pseudo-differential operator ${\operatorname{Op}} _t(a)$ is the linear and
continuous operator on $\mathscr S({\mathbf R^{d}})$, defined by the formula
\begin{equation}\label{e0.5}
({\operatorname{Op}} _t(a)f)(x)
=
(2\pi ) ^{-d}\iint a((1-t)x+ty,\xi )f(y)e^{i{\langle {x-y},\xi\rangle} }\,
dyd\xi .
\end{equation}
For general $a\in \mathscr S'({\mathbf R^{{2d}}})$, the pseudo-differential
operator ${\operatorname{Op}} _t(a)$ is defined as the continuous operator from
$\mathscr S({\mathbf R^{d}})$ to $\mathscr S'({\mathbf R^{d}})$ with distribution
kernel
\begin{equation}\label{weylkernel}
K_{t,a}(x,y)=(2\pi )^{-d/2}(\mathscr F_2^{-1}a)((1-t)x+ty,x-y).
\end{equation}
Here $\mathscr F_2F$ is the partial Fourier transform of
$F(x,y)\in \mathscr S'({\mathbf R^{{2d}}})$ with respect to the $y$-variable.
This definition makes sense, since the mappings $\mathscr F_2$ and
$$
F(x,y)\mapsto F((1-t)x+ty,x-y)
$$
are homeomorphisms on $\mathscr
S'({\mathbf R^{{2d}}})$. We also note that the latter definition of ${\operatorname{Op}} _t(a)$
agrees with the operator in \eqref{e0.5} when $a\in \mathscr S({\mathbf R^{{2d}}})$. If  $t=0$, then ${\operatorname{Op}} _t(a)$ agrees with the Kohn-Nirenberg
representation ${\operatorname{Op}} (a)=a(x,D)$.

\par

If $a\in \mathscr S'({\mathbf R^{{2d}}})$ and $s,t\in
\mathbf R$, then there is a unique $b\in \mathscr S'({\mathbf R^{{2d}}})$ such
that ${\operatorname{Op}} _s(a)={\operatorname{Op}} _t(b)$. By straight-forward applications of
Fourier's inversion  formula, it follows that
\begin{equation}\label{pseudorelation}
{\operatorname{Op}} _s(a)={\operatorname{Op}} _t(b) \quad \Longleftrightarrow \quad b(x,\xi
)=e^{i(t-s){\langle {D_x},{D_\xi}\rangle}}a(x,\xi ).
\end{equation}
(Cf. Section 18.5 in \cite{Ho1}.)

\par

Next we discuss symbol classes which we use. Let $r,
\rho ,\delta \in \mathbf R$ be fixed. Then we recall from
\cite{Ho1} that $S^r_{\rho ,\delta}({\mathbf R^{{2d}}})$ is the set of all $a\in
C^\infty ({\mathbf R^{{2d}}})$ such that for each pairs of multi-indices
$\alpha$ and $\beta$, there is a constant $C_{\alpha ,\beta}$ such
that
$$
|\partial _x^\alpha \partial _\xi ^\beta a(x,\xi )|\le
C_{\alpha ,\beta }{\langle \xi\rangle} ^{r-\rho |\beta |+\delta |\alpha |}.
$$
Usually we assume that $0\le \delta \le \rho \le 1$, $0<\rho$ and
$\delta <1$.

\par

More generally, assume that $\omega \in \mathscr P_{\rho ,\delta } ({\mathbf R^{{2d}}})$. Then we recall from the introduction that
$S_{\rho ,\delta}^{(\omega )}({\mathbf R^{{2d}}})$ consists of all $a\in
C^\infty ({\mathbf R^{{2d}}})$ such that
\begin{equation}\label{Somegadef}
|\partial _x^\alpha \partial _\xi ^\beta a(x,\xi )|\le
C_{\alpha ,\beta }\omega (x,\xi ){\langle \xi\rangle} ^{-\rho |\beta
|+\delta |\alpha |}.
\end{equation}
We note that $S_{\rho ,\delta}^{(\omega )}({\mathbf R^{{2d}}})=S(\omega ,g_{\rho
,\delta})$, when $g=g_{\rho 
,\delta}$ is the Riemannian metric on ${\mathbf R^{{2d}}}$, defined by the formula
$$
\big (g_{\rho ,\delta }\big )_{(y,\eta )}(x,\xi ) = {\langle \eta\rangle}
^{2\delta}|x|^2 + {\langle \eta\rangle} ^{-2\rho}|\xi |^2
$$
(cf. Section 18.4--18.6 in \cite{Ho1}). Furthermore, $S^{(\omega
)}_{\rho ,\delta} =S^r_{\rho ,\delta}$ when $\omega (x,\xi )={\langle \xi\rangle}
^r$, as remarked in the introduction.

\par

The following result shows that pseudo-differential operators with
symbols in $S^{(\omega )}_{\rho ,\delta}$ behave well. We refer to
\cite {Ho1} or \cite{PTT1} for the proof.

\par

\begin{prop}\label{p5.4}
Let $\rho ,\delta \in [0,1]$ be such that $0\le \delta \le
\rho \le 1$ and $\delta <1$, and let $\omega \in \mathscr P_{\rho ,\delta}
({\mathbf R^{{2d}}})$. If $a\in S^{(\omega )}_{\rho ,\delta} ({\mathbf R^{{2d}}})$, then ${\operatorname{Op}} _t(a)$ is continuous on $\mathscr S({\mathbf R^{d}})$ and
extends uniquely to a continuous operator on $\mathscr S'({\mathbf R^{d}})$.
\end{prop}

\par

We also need to define the set of characteristic points of a symbol
$a\in S^{(\omega )}_{\rho ,\delta} ({\mathbf R^{{2d}}})$, when $\omega \in
\mathscr P_{\rho ,\delta}({\mathbf R^{{2d}}})$ and $0\le \delta < \rho \le
1$. In Section \ref{sec2} we show that this definition is equivalent
to Definition 1.3 in \cite{PTT1}. We remark that our sets of
characteristic points are smaller than the corresponding sets in \cite
{Ho1}. (Cf. \cite[Definition 18.1.5]{Ho1} and Remark
\ref{compchar} in Section \ref{sec2}).

\par

\begin{defn}\label{defchar}
Assume that  $0 \leq \delta < \rho \leq 1$, $\omega _0 \in
\mathscr{P}_{\rho,\delta}({\mathbf R^{{2d}}})$ and $a\in
S^{(\omega_0)}_{\rho,\delta}({\mathbf R^{{2d}}})$. Then $a$ is called
\emph{$\psi$-invertible} with respect to $\omega _0$ at the
point $(x_0,\xi_0)\in {\mathbf R^{d}}\times ({\mathbf R^{d}}{\setminus 0})$, if
there exist a neighbourhood $X$ of $x_0$, an open conical
neighbourhood $\Gamma$ of $\xi_0$ and positive constants $R$ and $C$ such that
\begin{equation}\label{nydef}
|a(x,\xi)|\geq C\omega _0(x,\xi),
\end{equation}
for $x\in X$, $\xi\in \Gamma$ and $|\xi|\geq R$.

\par

The point $(x_0,\xi_0)$ is called \emph{characteristic} for $a$ with
respect to $\omega _0$ if $a$ is \emph{not} $\psi$-invertible 
with respect to $\omega _0$ at $(x_0,\xi_0)$. The set of characteristic points (the
characteristic set), for $a$ with respect to $\omega _0$ is denoted
${\operatorname{Char}}(a)={\operatorname{Char}}_{(\omega _0)}(a)$.
\end{defn}

\par

We note that $(x_0,\xi _0)\notin {\operatorname{Char}}_{(\omega _0)}(a)$ means that $a$
is elliptic near $x_0$ in the direction $\xi _0$. Since the case $\omega _0=1$ in Definition \ref{defchar} is especially important we also make the following definition. We say that $c\in
S^{0}_{\rho,\delta}({\mathbf R^{{2d}}})$ is
\emph{$\psi$-invertible} at $(x_0,\xi_0) \in {\mathbf R^{d}}\times ({\mathbf R^{d}}{\setminus 0})$, if $(x_0,\xi _0)\notin {\operatorname{Char}} _{(\omega _0)}(c)$ with $\omega _0=1$. That is,  there exist a neighbourhood $X$ of $x_0$, an open conical
neighbourhood $\Gamma$ of $\xi_0$ and $R >0$ such that \eqref{nydef} holds for $a=c$ and $\omega _0=1$,
for some constant $C>0$ which is independent of $x\in X$
and $\xi\in \Gamma$ such that $|\xi|\geq R$.

\par

It will also be convenient to have the following definition of different types of cutoff functions.

\par

\begin{defn}\label{cuttdef}
Let $X\subseteq {\mathbf R^{d}}$ be open, $\Gamma \subseteq {\mathbf R^{d}}{\setminus 0}$ be an open cone, $x_0\in X$ and let $\xi _0\in \Gamma $. 

\begin{enumerate}
\item A smooth function ${\varphi}$ on ${\mathbf R^{d}}$ is called a cutoff function with respect to $x_0$ and $X$, if $0\le {\varphi} \le 1$, ${\varphi} \in C_0^\infty (X)$ and ${\varphi} =1$ in an open neighbourhood of $x_0$. The set of cutoff functions with respect to $x_0$ and $X$ is denoted by $\mathscr C_{x_0}(X)$;

{\vspace{0.1cm}}

\item  A smooth function $\psi$ on ${\mathbf R^{d}}$ is called a directional cutoff function with respect to $\xi_0$ and $\Gamma$, if there is a constant $R>0$ and open conical neighbourhood $\Gamma _1$ of $\xi _0$ such that the following is true:
\begin{itemize}
\item $0\le \psi \le 1$ and ${\operatorname{supp}} \psi \subseteq \Gamma$;

{\vspace{0.1cm}}

\item  $\psi (t\xi )=\psi (\xi )$ when $t\ge 1$ and $|\xi |\ge R$;

{\vspace{0.1cm}}

\item $\psi (\xi )=1$ when $\xi \in \Gamma _1$ and $|\xi |\ge R$.
\end{itemize}

The set of directional cutoff functions with respect to $\xi _0$ and $\Gamma$ is denoted by $\mathscr C^{{\operatorname{dir}}}  _{\xi _0}(\Gamma )$.
\end{enumerate}

\end{defn}

\par

\begin{rem}\label{psiinvremark}
We note that if ${\varphi} \in \mathscr C_{x_0}(X)$ and $\psi \in \mathscr
C^{{\operatorname{dir}}}  _{\xi _0}(\Gamma )$ for some $(x_0,\xi _0)\in {\mathbf R^{d}}\times ({\mathbf R^{d}}{\setminus 0})$, then $c\equiv {\varphi} \otimes \psi$ belongs to $S^0_{1,0}({\mathbf R^{{2d}}})$ and is $\psi$-invertible at $(x_0,\xi _0)$.
\end{rem}

\par

\section{Pseudo-differential calculus with symbols
in $S^{(\omega )}_{\rho ,\delta}$}\label{sec2}

\par

In this section we make a review of basic results for
pseudo-differential operators with symbols in classes of the form
$S^{(\omega )}_{\rho ,\delta}({\mathbf R^{{2d}}})$, when $0\le \delta <\rho \le
1$ and $\omega \in \mathscr P_{\rho ,\delta}({\mathbf R^{{2d}}})$. For the
standard properties in the pseudo-differential calculus we only state
the results and refer to \cite{Ho1} for the proofs. Though there are
similar stated and
proved properties concerning sets of characteristic points, we include
proofs of these properties in order to being more self-contained.

\par

We start with the following result concerning compositions and
invariance properties for pseudo-differential operators. Here we let
$$
\sigma _s(x,\xi )={\langle \xi\rangle} ^s,
$$
where ${\langle \xi\rangle} =(1+|\xi
|^2)^{1/2}$ as usual. We also recall that $S^{-\infty }_{\rho ,\delta
}=S^{-\infty }_{1,0}$ consists of all $a\in C^\infty ({\mathbf R^{{2d}}})$ such
that for each $N\in \mathbf R$ and multi-index $\alpha$, there is a
constant $C_{N,\alpha}$ such that
$$
|\partial ^\alpha a(x,\xi )|\le C_{N,\alpha}{\langle \xi\rangle} ^{-N}.
$$

\par

\begin{prop}\label{pseudocomp}
Let  $0\le \delta <\rho \le 1$, $\mu =\rho -\delta >0$ and
$\omega ,\omega _1,\omega _2\in \mathscr P_{\rho ,\delta }({\mathbf R^{{2d}}})$. Also let $\{ m_j\} _{j=0}^{\infty}$ be a sequence of real
numbers such that $m_j\to -\infty$ as $j\to \infty$. Then the
following is true:
\begin{enumerate}
\item if $a_1\in S^{(\omega _1)}_{\rho ,\delta }({\mathbf R^{{2d}}})$ and
$a_2\in S^{(\omega _2)}_{\rho ,\delta }({\mathbf R^{{2d}}})$, then ${\operatorname{Op}}
(a_1)\circ {\operatorname{Op}} (a_2)={\operatorname{Op}} (c)$, for some $c\in S^{(\omega
_1\omega _2)}_{\rho ,\delta }({\mathbf R^{{2d}}})$. Furthermore,
\begin{equation}\label{symbcomp}
c(x,\xi )-\sum _{|\alpha |<N}\frac {i^{|\alpha |}(D^{\alpha}_\xi
a_1)(x,\xi )(D^{\alpha}_x a_2)(x,\xi )}{\alpha
!}\in S^{(\omega
_1\omega _2\sigma _{-N\mu })}_{\rho ,\delta }({\mathbf R^{{2d}}})
\end{equation}
for every $N\ge0$;

{\vspace{0.1cm}}

\item if $M=\sup _{k\ge 0}(m_k)$,
$M_j=\sup _{k\ge j}(m_k)$ and $a_j\in S^{(\omega \sigma _{m_j})}_{\rho
,\delta }({\mathbf R^{{2d}}})$, then it exists $a\in S^{(\omega \sigma
_{M})}_{\rho ,\delta }({\mathbf R^{{2d}}})$ such that
\begin{equation}\label{asymptexp}
a(x,\xi )-\sum _{|\alpha |<N}a_j(x,\xi )\in S^{(\omega 
\sigma _{M_N})}_{\rho ,\delta }({\mathbf R^{{2d}}});
\end{equation}
for every $N\ge 0$;

{\vspace{0.1cm}}

\item if $a,b\in \mathscr S'({\mathbf R^{{2d}}})$ and $s,t\in \mathbf R$ are
such that ${\operatorname{Op}} _s(a)={\operatorname{Op}} _t(b)$, then $a\in S^{(\omega )}_{\rho
,\delta }({\mathbf R^{{2d}}})$, if and only if $b\in S^{(\omega )}_{\rho ,\delta
}({\mathbf R^{{2d}}})$, and
\begin{equation}\label{pseudorel2}
b(x,\xi )-\sum _{k<N}\frac {(i(t-s){\langle {D_x},{D_\xi}\rangle})^{k}a(x,\xi )}{k
!}\in S^{(\omega \sigma _{-N\mu })}_{\rho ,\delta }({\mathbf R^{{2d}}})
\end{equation}
for every $N\ge 0$.
\end{enumerate}
\end{prop}

\par

As usual we write
\begin{equation}\tag*{(\ref{asymptexp})$'$}
a\sim \sum a_j
\end{equation}
when \eqref{asymptexp} is fulfilled for every $N\ge 0$. In particular
it follows from \eqref{symbcomp} and \eqref{pseudorel2} that
\begin{equation}\tag*{(\ref{symbcomp})$'$}
c\sim \sum \frac {i^{|\alpha |}(D^{\alpha}_\xi
a_1)(D^{\alpha}_x a_2)}{\alpha !}
\end{equation}
when ${\operatorname{Op}} (a_1)\circ {\operatorname{Op}} (a_2)={\operatorname{Op}} (c)$, and
\begin{equation}\tag*{(\ref{pseudorel2})$'$}
b\sim \sum \frac {(i(t-s){\langle {D_x},{D_\xi}\rangle})^{k}a}{k !}
\end{equation}
when ${\operatorname{Op}} _s(a)={\operatorname{Op}} _t(b)$.

\par

In the following proposition we show that the set of characteristic
points for a pseudo-differential operator is independent of the choice
of pseudo-differential calculus.

\par

\begin{prop}\label{anvandningkar}
Assume that $s,t\in \mathbf R$, $0\leq \delta < \rho\leq 1$,
$\omega_0\in\mathscr{P}_{\rho,\delta}$ and that $a,b\in
S^{(\omega_0)}_{\rho,\delta}({\mathbf R^{{2d}}})$ satisfy ${\operatorname{Op}}_s(a)={\operatorname{Op}}_t(b)$. Then
\begin{equation}\label{char}
{\operatorname{Char}}_{(\omega_0)}(a)={\operatorname{Char}}_{(\omega_0)}(b).
\end{equation}
\end{prop}

\par

\begin{proof}
Let $\mu$ and $\sigma _s$ be the same as in the proof of Proposition
\ref{pseudocomp}. By Proposition \ref{pseudocomp} (3) we have
$$
b=a+h,
$$
for some $h\in S^{(\omega_0\sigma_{-\mu})}_{\rho,\delta}$.

\par

Assume that $(x_0,\xi_0)\notin {\operatorname{Char}}_{(\omega_0)}(a)$. By the
definitions, there is a negihbourhood $X$ of $x_0$, an open conical
negihbourhood $\Gamma$ of $\xi_0$, $C>0$ and $R >0$ such that
$$
|a(x,\xi )| \ge C\omega _0(x,\xi )\quad \text{and}\quad |h(x,\xi )|\le
C\omega _0(x,\xi )/2,
$$
as $x\in X$, $\xi \in \Gamma$ and $|\xi |\ge R$. This gives
$$
|b(x,\xi )| \ge C\omega _0(x,\xi )/2,\quad \text{when}\quad x\in X,\
\xi \in \Gamma ,\ |\xi |\ge R,
$$
and it follows that $(x_0,\xi_0)\notin {\operatorname{Char}}_{(\omega_0)}(b)$. Hence
${\operatorname{Char}}_{(\omega_0)}(b)\subseteq {\operatorname{Char}}_{(\omega_0)}(a)$. By symmetry, 
the opposite inclusion also holds. Hence
${\operatorname{Char}}_{(\omega_0)}(a) = {\operatorname{Char}}_{(\omega_0)}(b)$, and the proof is
complete.
\end{proof}

\par

The following proposition shows different aspects of set of
characteristic points, and is important when investigating wave-front
properties for pseudo-differential operators. In particular it shows
that ${\operatorname{Op}} (a)$ satisfy certain invertibility properties outside the
set of characteristic points for $a$. More precisely,  
outside ${\operatorname{Char}} _{(\omega _0)}(a)$, we prove that
\begin{equation}\label{invexp}
{\operatorname{Op}} (b){\operatorname{Op}} (a) ={\operatorname{Op}} (c)+{\operatorname{Op}} (h),
\end{equation}
for some convenient $b$, $c$ and $h$ which take the role of inverse,
identity symbol and smoothing remainder respectively.

\par

\begin{prop}\label{psiecharequiv}
Let  $0\le \delta <\rho \le 1$, $\omega _0\in \mathscr
P_{\rho,\delta}({\mathbf R^{{2d}}})$, $a\in S^{(\omega _0)}_{\rho,\delta}({\mathbf R^{{2d}}})$, $(x_0,\xi _0)\in {\mathbf R^{d}} \times ({\mathbf R^{d}}{\setminus 0})$, and let
$\mu =\rho -\delta$. Then the following conditions are equivalent:
\begin{enumerate}
\item $(x_0,\xi _0)\notin {\operatorname{Char}} _{(\omega _0)} (a)$;

{\vspace{0.1cm}}

\item there is an element $c\in S^0_{\rho ,\delta}$ which is $\psi$-invertible at $(x_0,\xi _0)$, and an element $b\in S^{(1/\omega _0)}_{\rho,\delta}$ such that $ab=c$;

{\vspace{0.1cm}}
 
\item there is an element $c\in S^0_{\rho ,\delta}$ which is $\psi$-invertible at $(x_0,\xi _0)$, and elements 
$h\in S^{-\mu}_{\rho,\delta}$ and $b\in S^{(1/\omega _0)}_{\rho,\delta}$ such that \eqref{invexp} holds;

{\vspace{0.1cm}}

\item for each neighbourhood $X$ of $x_0$ and conical neighbourhood $\Gamma $ of $\xi _0$, there is an element $c={\varphi} \otimes \psi$ where ${\varphi} \in \mathscr C_{x_0}(X)$ and $\psi _{\xi _0}^{{\operatorname{dir}}} (\Gamma )$, and elements 
$h\in \mathscr S$ and $b\in S^{(1/\omega _0)}_{\rho,\delta}$ such that \eqref{invexp} holds. Furthermore, 
the supports of $b$ and $h$ are contained in $X\times {\mathbf R^{d}}$.
\end{enumerate}
\end{prop}

\par

For the proof we note that $\mu $ in Proposition
\ref{psiecharequiv} is positive, which in turn implies that $\cap _{j\ge 0}S^{(\omega _0\sigma _{-j\mu })}
({\mathbf R^{{2d}}})$ agrees with $S^{-\infty}({\mathbf R^{{2d}}})$.

\par

\begin{proof}
The equivalence between (1) and (2) follows by letting
$b(x,\xi )={\varphi} (x)\psi (\xi )/a(x,\xi )$ for some appropriate  ${\varphi} \in \mathscr C_{x_0}({\mathbf R^{d}})$ and $\psi \in \mathscr C_{\xi _0}^{{\operatorname{dir}}} ({\mathbf R^{d}} {\setminus 0})$.

\par

(4) $\Rightarrow$ (3) is obvious in view of Remark \ref{psiinvremark}. Assume that (3) holds. We shall
prove that (1) holds, and since $|b|\leq C/\omega _0$, it suffices to
prove that
\begin{align}
|a(x,\xi )b(x,\xi )| \geq 1/2\label{abinvertible}
\intertext{when}
(x,\xi) \in X\times \Gamma,\ |\xi |\ge R\label{abinvcond}
\end{align}
holds for some conical neighbourhood $\Gamma$ of $\xi _0$, some open
neighbourhood $X$ of $x_0$ and some $R>0$.

\par

By Proposition \ref{pseudocomp} (1) it follows that $ab=c+h$ for
some $h\in S ^{-\mu}_{\rho,\delta}$. By choosing $R$ large enough
and $\Gamma$ sufficiently small conical neighbourhood of $\xi _0$, it
follows that $c(x,\xi )=1$ and $|h(x,\xi )|\leq 1/2$ when
\eqref{abinvcond} holds. This gives \eqref{abinvertible}, and (1)
follows.

\par

It remains to prove that (1) implies (4). Therefore assume that (1) holds, and choose an open neighbourhood $X$ of $x_0$, an open conical neighbourhood $\Gamma $ of $\xi _0$ and $R>0$ such that \eqref{nydef} holds when $(x,\xi )\in X\times \Gamma$ and $|\xi |>R$. Also let ${\varphi} _j\in \mathscr C_{x_0}(X)$ and $\psi _j\in \mathscr C_{\xi _0}^{{\operatorname{dir}}} (\Gamma )$ for $j=1,2,3$ be such that ${\varphi} _j=1$ on ${\operatorname{supp}} {\varphi} _{j+1}$, $\psi _j=1$ on ${\operatorname{supp}} \psi _{j+1}$ when $j=1,2$, and $\psi _j(\xi )=0$ when $|\xi |\le R$. We also set $c_j={\varphi} _j\otimes \psi _j$ when $j\le 2$ and $c_j=c_2$ when $j \ge 3$.

\par

If $b_1(x,\xi )={\varphi} _1(x)\psi _1(\xi )/a(x,\xi
)\in S ^{(1/\omega _0)}_{\rho,\delta}$, then the symbol of ${\operatorname{Op}}
(b_1){\operatorname{Op}} (a)$ is equal to  $c_1\mod (S ^{-\mu }_{\rho,\delta})$. Hence
\begin{equation}\label{opcompmod}
{\operatorname{Op}} (b_j){\operatorname{Op}} (a)={\operatorname{Op}} (c_j)+{\operatorname{Op}} (h_j)
\end{equation}
holds for $j=1$ and some $h_1\in S ^{-\mu }_{\rho,\delta}$.

\par

For $j\geq 2$ we now define $\widetilde b_j\in S ^{(1/\omega _0
)}_{\rho,\delta}$ by the Neumann serie
$$
{\operatorname{Op}} (\widetilde b_j) = \sum _{k=0}^{j-1}(-1)^k{\operatorname{Op}} (\widetilde r_k),
$$
where ${\operatorname{Op}} (\widetilde r_k)={\operatorname{Op}} (h_1)^k{\operatorname{Op}} (b_1)\in {\operatorname{Op}} (S ^{(\sigma
_{-k\mu}/\omega _0)}_{\rho,\delta})$. Then \eqref{opcompmod} gives
\begin{multline*}
{\operatorname{Op}} (\widetilde b_j){\operatorname{Op}} (a) = \sum _{k=0}^{j-1}(-1)^k{\operatorname{Op}} (h_1)^k{\operatorname{Op}}
(b_1){\operatorname{Op}} (a)
\\[1ex]
=\sum _{k=0}^{j-1}(-1)^k{\operatorname{Op}} (h_1)^k({\operatorname{Op}} (c_1)+{\operatorname{Op}} (h_1)).
\end{multline*}
That is
\begin{equation}\label{expan1}
{\operatorname{Op}} (\widetilde b_j){\operatorname{Op}} (a) = {\operatorname{Op}} (c_1)+{\operatorname{Op}} (\widetilde h_{1,j})+{\operatorname{Op}}
(\widetilde h_{2,j}),
\end{equation}
where
\begin{align}
{\operatorname{Op}} (\widetilde h_{1,j}) &= (-1)^{j-1}{\operatorname{Op}} (h_1)^{j}\in {\operatorname{Op}} (S
^{-j\mu}_{\rho,\delta})\label{hjtilde}
\intertext{and}
{\operatorname{Op}} (\widetilde h_{2,j}) &= -\sum _{k=1}^{j-1}(-1)^k{\operatorname{Op}} (h_1)^k{\operatorname{Op}}
(1-c_1)\in {\operatorname{Op}} (S^{-\mu}_{\rho,\delta}).\notag
\end{align}

\par

By Proposition \ref{pseudocomp} (1) and asymptotic expansions it
follows that
\begin{multline}\label{h2tilde}
{\operatorname{Op}} (\widetilde h_{2,j}) = -\sum _{k=1}^{j-1}(-1)^k{\operatorname{Op}}
(1-c_1){\operatorname{Op}} (h_1)^k
\\[1ex]
+ {\operatorname{Op}} (\widetilde h_{3,j}) + {\operatorname{Op}} (\widetilde
h_{4,j}),
\end{multline}
for some $\widetilde h_{3,j}\in S ^{-\mu}_{\rho,\delta}$ which
is equal to zero in ${\operatorname{supp}} c_1$ and $\widetilde h_{4,j}\in
S^{-j\mu}_{\rho,\delta}$. Now let $b_j$ and $r_k$ be defined by
the formulae
\begin{align*}
{\operatorname{Op}} (b_j) &= {\operatorname{Op}}
(c_2){\operatorname{Op}} (\widetilde b_j)\in {\operatorname{Op}} (S ^{(1/\omega _0)}_{\rho,\delta})
\\[1ex]
{\operatorname{Op}} (r_k) &= {\operatorname{Op}} (c_2){\operatorname{Op}} (\widetilde r_k)\in {\operatorname{Op}} (S ^{(\sigma
_{-k\mu }/\omega _0)}_{\rho,\delta}).
\end{align*}
Then
$$
{\operatorname{Op}} (b_j) = \sum _{k=0}^{j-1}(-1)^k{\operatorname{Op}} (r_k)
$$
and \eqref{expan1}--\eqref{h2tilde} give
\begin{multline*}
{\operatorname{Op}} (b_j){\operatorname{Op}} (a) = {\operatorname{Op}} (c_2){\operatorname{Op}} (c_1)+{\operatorname{Op}}
(c_2){\operatorname{Op}} (\widetilde h_{1,j})
\\[1ex]
-\sum _{k=1}^{j-1}(-1)^k{\operatorname{Op}} (c_2){\operatorname{Op}}
(1-c_1){\operatorname{Op}} (h_1)^k + {\operatorname{Op}} (c_2){\operatorname{Op}} (\widetilde h_{3,j}) +
{\operatorname{Op}} (c_2){\operatorname{Op}} (\widetilde h_{4,j}).
\end{multline*}
Since $c_1=1$ and $\widetilde h_{3,j}=0$ on ${\operatorname{supp}} c_2$, it
follows that
\begin{align*}
{\operatorname{Op}} (c_2){\operatorname{Op}} (c_1) &= {\operatorname{Op}} (c_2)\mod {\operatorname{Op}} (S^{-\infty}),
\\[1ex]
{\operatorname{Op}} (c_2){\operatorname{Op}} (\widetilde h_{1,j}) &\in {\operatorname{Op}} (S^{-j\mu}_{\rho,\delta}),
\\[1ex]
\sum _{k=1}^{j-1}(-1)^k{\operatorname{Op}} (c_2){\operatorname{Op}} (1-c_1){\operatorname{Op}} (h_1)^k &\in  {\operatorname{Op}}
(S^{-\infty}),
\\[1ex]
{\operatorname{Op}} (c_2){\operatorname{Op}} (\widetilde h_{3,j}) &\in {\operatorname{Op}} (S^{-\infty})
\intertext{and}
{\operatorname{Op}} (c_2){\operatorname{Op}} (\widetilde h_{4,j}) &\in {\operatorname{Op}} (S^{-j\mu}_{\rho,\delta}).
\end{align*}
Hence, \eqref{opcompmod} follows for some $h_j\in
S^{-j\mu}_{\rho,\delta}$.

\par

By choosing $b_0\in S ^{(1/\omega )}_{\rho,\delta}$ such that
$$
b_0\sim \sum r_k,
$$
it follows that ${\operatorname{Op}} (b_0){\operatorname{Op}} (a) ={\operatorname{Op}} (c_2)+{\operatorname{Op}} (h_0)$, with
$$
h_0\in S ^{-\infty}.
$$
The assertion (4) now follows by letting
\begin{align*}
b(x,\xi )={\varphi} _3(x)b_0(x,\xi ),\quad  c(x,\xi ) &= {\varphi} _3(x)c_2(x,\xi ),
\\[1ex]
\text{and}\quad h(x,\xi ) &= {\varphi} _3(x) h_0(x,\xi ),
\end{align*}
and using the fact that if ${\varphi} _3 \in C_0^\infty ({\mathbf R^{d}})$ and $h_0\in
S^{-\infty}({\mathbf R^{{2d}}})$, then ${\varphi} _3 (x)h_0(x,\xi )\in \mathscr S({\mathbf R^{{2d}}})$.
The proof is complete.
\end{proof}

\par

\begin{rem}\label{compchar}
By Proposition \ref{psiecharequiv} it follows that Definition 1.3 in
\cite {PTT1} is equivalent to Definition \ref{defchar}. We also remark
that if $a$ is an appropriate symbol, and ${\operatorname{Char}} '(a)$ the set of
characteristic points for $a$ in the sense of \cite[Definition
18.1.5]{Ho1}, then ${\operatorname{Char}} _{(\omega _0)}(a)\subseteq {\operatorname{Char}}
'(a)$. Furthermore, strict embedding might occur, especially for
symbols to hypoelliptic partial operators with constant coefficients, which are not elliptic (cf.  Example 3.11 in \cite{PTT1}).
\end{rem}

\par

\section{Wave front sets with respect to Fourier
Banach spaces}\label{sec3}

\par

In this section we define wave-front sets with respect to Fourier
Banach spaces, and show some basic properties. 

\par

Let $\omega \in \mathscr P({\mathbf R^{{2d}}})$, $\Gamma \subseteq {\mathbf R^{d}}{\setminus 0}$ be an open cone and let $\mathscr B$ be a translation
invariant BF-space on ${\mathbf R^{d}}$. For any $f\in
\mathscr E'({\mathbf R^{d}})$, let
\begin{equation}\label{skoff}
|f|_{{\mathscr F\! \mathscr B} (\omega,\Gamma)}=|f|_{{\mathscr F\! \mathscr B}(\omega,\Gamma)_x}\equiv {\Vert {\widehat{f}\omega(x,{\, \cdot \, } )\chi_{\Gamma} }\Vert _{{\mathscr{B}}}}.
\end{equation}
We note that $\widehat f\omega(x,{\, \cdot \, } )\chi_{\Gamma}\in \mathscr
B_{loc}$ for every $f\in \mathscr E'$. If $\widehat
f\omega(x,{\, \cdot \, } )\chi_{\Gamma}\notin \mathscr B$, then we set $|f|_{{\mathscr F\! \mathscr B}
(\omega,\Gamma)}=+\infty$. Hence  $|{\, \cdot \, } |_{{\mathscr F\! \mathscr B} (\omega,\Gamma)}$
defines a semi-norm on $\mathscr E'$ which might attain the value
$+\infty$. Since $\omega $ is $v$-moderate for some $v \in \mathscr
P({\mathbf R^{{2d}}})$, it follows that different $x \in {\mathbf R^{d}}$ gives rise to
equivalent semi-norms. Furthermore, if
$\Gamma ={\mathbf R^{d}}{\setminus 0}$ and $f\in \mathscr {FB}{(\omega )}\cap \mathscr E'$, then
$|f|_{{\mathscr F\! \mathscr B} (\omega,\Gamma)}$ agrees with ${\Vert f\Vert _{{{\mathscr F\! \mathscr B} {(\omega)}}}}$. For simplicity we write $|f|_{{\mathscr F\! \mathscr B} (\Gamma)}$ instead of $|f|_{{\mathscr F\! \mathscr B}
(\omega,\Gamma)}$ when $\omega =1$.

\par

For the sake of notational convenience we set
\begin{equation} \label{notconv}
|{\, \cdot \, } |_{\mathcal B(\Gamma )}=|{\, \cdot \, } |_{{\mathscr F\! \mathscr B}(\omega,\Gamma)_x}, \quad
\mbox{when}
\quad
\mathcal B={\mathscr F\! \mathscr B} (\omega).
\end{equation}
We let $ \Theta _{\mathcal B}(f)=\Theta _{{\mathscr F\! \mathscr B}(\omega)} (f)$ be the set of all $ \xi \in {\mathbf R^{d}}{\setminus 0} $ such that
$|f|_{\mathcal B(\Gamma )} < \infty$, for some $
\Gamma = \Gamma _{\xi}$.  We also let $\Sigma _{\mathcal B} (f)$
be the complement of $ \Theta_{\mathcal
B} (f)$ in ${\mathbf R^{d}}{\setminus 0} $. Then
$\Theta _{ \mathcal B} (f)$ and $\Sigma _{\mathcal
B} (f)$ are open respectively
closed subsets in ${\mathbf R^{d}}{\setminus 0}$, which are independent of
the choice of $ x \in {\mathbf R^{d}}$ in \eqref{skoff}.

\par

\begin{defn}\label{wave-frontdef}
Let  $\mathscr B$ be a translation invariant BF-space on ${\mathbf R^{d}}$, $\omega \in \mathscr P({\mathbf R^{{2d}}})$, $\mathcal
B$ be as in \eqref{notconv}, and let
$X$ be an open subset of ${\mathbf R^{d}}$.
The wave-front set of
$f\in \mathscr D'(X)$,
$
{\operatorname{WF}} _{\mathcal B}(f)  \equiv  {\operatorname{WF}} _{{\mathscr F\! \mathscr B}(\omega )}(f)
$
with respect to $\mathcal B$ consists of all pairs $(x_0,\xi_0)$ in
$X\times ({\mathbf R^{d}} {\setminus 0})$ such that
$
\xi _0 \in  \Sigma _{\mathcal B} ({\varphi} f)
$
holds for each ${\varphi} \in C_0^\infty (X)$ such that ${\varphi} (x_0)\neq
0$.
\end{defn}

\par

We note that ${\operatorname{WF}}  _{\mathcal B}(f)$ in Definition \ref{wave-frontdef}
is a closed set in $X\times
({\mathbf R^{d}}{\setminus 0})$, since it is obvious that its complement is
open. We also note that if $ x_0\in {\mathbf R^{d}}$ is fixed and $\omega _0(\xi
)=\omega (x_0,\xi )$, then ${\operatorname{WF}} _{{\mathscr F\! \mathscr B}(\omega )} (f)={\operatorname{WF}} _{{\mathscr F\! \mathscr B}(\omega _0)}(f)$, since $\Sigma _{\mathcal B}$ is independent of $x_0$.

\par

The following theorem shows that wave-front sets with respect to
${\mathscr F\! \mathscr B}(\omega )$ satisfy appropriate micro-local
properties. It also shows that such wave-front sets decreases when the local Fourier BF-spaces increases, or when the weight $\omega$ decreases.

\par

\begin{thm}\label{theta-sigma-propertiesAA}
Let $X\subseteq {\mathbf R^{d}}$ be open, $\mathscr B_1,\mathscr B_2$ be translation invariant BF-spaces, ${\varphi} \in C^\infty({\mathbf R^{{d}}})$, $\omega _1,\omega _2\in
\mathscr{P}({\mathbf R^{{2d}}})$ and $f\in \mathscr{D}'(X)$. If ${\mathscr F\! \mathscr B} _1(\omega _1)_{loc}\subseteq {\mathscr F\! \mathscr B} _2(\omega _2)_{loc}$, then
\begin{equation*}
{\operatorname{WF}} _{{\mathscr F\! \mathscr B} _2(\omega _2)}(\varphi f)\subseteq {\operatorname{WF}} _{{\mathscr F\! \mathscr B} _1(\omega
_1)}(f).
\end{equation*}
\end{thm}

\par

\begin{proof}
It suffices to prove
\begin{equation}\label{chi-subsetAA}
\Sigma_{ {\mathcal B _2} } ({\varphi} f) \subseteq
\Sigma_{\mathcal B_1} (f).
\end{equation}
when $\mathcal B_j = {\mathscr F\! \mathscr B} _j(\omega _j)$, $ {\varphi} \in  \mathscr S({\mathbf R^{d}})$ and $f\in
\mathscr E'({\mathbf R^{d}})$, since the statement only involve local assertions. The local properties and Remark \ref{newbfspaces} also imply that it is no restriction to assume that $\omega _1 =\omega _2 = 1$.

\par

Let $ \xi_0 \in \Theta_{\mathcal B_2} (f)$,
and choose open cones $\Gamma _1$ and $\Gamma_2$ in ${\mathbf R^{d}}$ such that
$\overline {\Gamma _2}\subseteq \Gamma _1$. Since $f$ has compact
support, it follows that $|\widehat f(\xi )|\le
C{\langle \xi\rangle} ^{N_0}$ for some positive constants $C$ and $N_0$. The result
therefore follows if we prove that for each $N$, there are constants
$C_N$ such that
\begin{multline}\label{cuttoff1}
|{\varphi} f|_{ {\mathcal B _2} (\Gamma _2)}\le C_N \Big (|
f|_{\mathcal B_1(\Gamma _1)} +\sup _{\xi \in {\mathbf R^{d}}} \big ( |\widehat f(\xi )|{\langle \xi\rangle} ^{-N} \big )
\Big )
\\[1ex]
\text{when}\quad \overline \Gamma _2\subseteq \Gamma
_1\quad \text{and}\quad N=1,2,\dots .
\end{multline}

\par

By using the fact that $\omega$ is $v$-moderate for some $v\in
\mathscr P({\mathbf R^{d}}) $, and letting $F(\xi )=|\widehat f(\xi )|$ and $\psi (\xi )=|\widehat {\varphi} (\xi )|$, it
follows that $\psi$ turns rapidly to zero at infinity and
\begin{multline*}
|{\varphi} f| _{ {\mathcal B _2}(\Gamma _2)} = |\varphi f|_{{\mathscr F\! \mathscr B} _2(\Gamma_2)}
=
\|\mathscr{F}(\varphi f) \chi_{\Gamma _2}\|_{\mathscr{B}_2}
\\[1ex]
\leq
C \Big \| \Big ( \int_{{\mathbf R^{{d}}}} \widehat {\varphi} ({\, \cdot \, } - \eta )\widehat
f(\eta )\, d\eta \Big ) \chi_{\Gamma_2} \Big \|_{\mathscr{B}_2}
\leq
C(J_1 + J_2)
\end{multline*}
for some positive constant $C$, where
\begin{align}
J_1
&=
\Big \| \Big ( \int _{\Gamma _1} \widehat {\varphi} ({\, \cdot \, } - \eta)\widehat
f(\eta )\, d\eta \Big ) \chi _{\Gamma_2} \Big \| _{\mathscr{B}_2}\label{J1def}
\intertext{and}
J_2
&=
\Big \| \Big ( \int _{\complement \Gamma _1}\widehat {\varphi} ({\, \cdot \, } -
\eta)\widehat f (\eta )\, d\eta \Big )\chi _{\Gamma _2} \Big
\| _{\mathscr{B} _2}\label{J2def}
\end{align}
and $\chi_{\Gamma_2}$ is the characteristic function of
$\Gamma_2$. First we estimate $J_1$. By (3) in Definition
\ref{BFspaces} and \eqref{propupps}, it follows for some constants
$C_1,\dots ,C_5$ that
\begin{multline}\label{J1comp}
J_1\leq  C_1 \Big \| \int _{\Gamma _1}\widehat {\varphi} ({\, \cdot \, } -\eta )
\widehat f(\eta )\, d\eta \Big \| _{\mathscr{B}_2}
=
C_1 {\Vert {\widehat {\varphi} * (\chi_{\Gamma_1}\widehat f) }\Vert _{{\mathscr{B}_2}}}
\\[1ex]
=
C_2 {\Vert {{\varphi}  \mathscr F^{-1}(\chi_{\Gamma_1}\widehat f) }\Vert _{{{\mathscr F\! \mathscr B} _2}}}
\le
C_3 {\Vert {{\varphi}  \mathscr F^{-1}(\chi_{\Gamma_1}\widehat f) }\Vert _{{{\mathscr F\! \mathscr B} _1}}}
\\[1ex]
=
C_4 {\Vert {\widehat {\varphi} * (\chi_{\Gamma_1}\widehat f) }\Vert _{{\mathscr{B}_1}}}
\leq
C_5{\Vert {\widehat {\varphi} \|_{L^1_{(v)}}\|\chi_{\Gamma_1}\widehat f
}\Vert _{{\mathscr{B}_1}}}
=
C_{\psi}|f|_{{\mathscr F\! \mathscr B} _1(\Gamma_1)},
\end{multline}
where $C_{\psi}  = C_5{\Vert {\widehat {\varphi}}\Vert _{{L^1_{(v)}}}}<\infty$, since
$\widehat {\varphi}$ turns rapidly
to zero at infinity. In the second inequality we have used the fact
that $({\mathscr F\! \mathscr B} _1)_{loc}\subseteq ({\mathscr F\! \mathscr B} _2)_{loc}$.

\par

In order to estimate $J_2$, we note that the conditions $\xi \in
\Gamma _2$, $\eta \notin \Gamma _1$ and the fact that $\overline
{\Gamma _2}\subseteq \Gamma _1$ imply that  $|\xi -\eta |>c\max
(|\xi|,|\eta |)$ for some constant $c>0$, since this is true when
$1=|\xi |\ge |\eta|$. We also note that if $N_1$ is large enough, then
${\langle {\, \cdot \, }\rangle} ^{-N_1}\in \mathscr B_2$, because $\mathscr S$ is
continuously embedded in $\mathscr B_2$. Since $\psi$ turns rapidly to
zero at infinity, it follows that for each $N_0> d+N_1$ and $N\in
\mathbf N$ such that $N > N_0$, it holds
\begin{multline}\label{J2comp}
J_2
\leq
C_1\Big \| \Big (\int_{\complement\Gamma _1}{\langle {{\, \cdot \, } -
\eta}\rangle}^{-(2N_0+N)} F(\eta)\, d\eta \Big )\chi_{\Gamma_2}\Big \|
_{\mathscr{B}_2}
\\[1ex]
\leq
C_2\Big \| \Big (\int_{\complement \Gamma_1} {\langle {\, \cdot \, }\rangle}^{-N_0} {\langle \eta\rangle} ^{-N_0}({\langle \eta\rangle}^{-N}F(\eta ))\, d\eta \Big
)\chi_{\Gamma_2}\Big \| _{\mathscr{B}_2}
\\[1ex]
\leq
C_2\int_{\complement \Gamma_1}\|{\langle {\, \cdot \, }\rangle}
^{-N_0}\chi_{\Gamma_2}\|_{\mathscr{B}_2} {\langle \eta\rangle} ^{-N_0}(|{\langle \eta\rangle}
^{-N}F(\eta )|)\, d\eta
\\[1ex]
\leq
 C \sup_{\eta \in {\mathbf R^{{d}}}}|{\langle \eta\rangle}^{-N}F(\eta )|,
\end{multline}
for some constants $C_1$ and $C_2 > 0$, where $C=C_2{\Vert {{\langle {\, \cdot \, }\rangle}
^{-{N_0}}}\Vert _{{\mathscr B _2}}}{\Vert {{\langle {\, \cdot \, }\rangle} ^{-{N_0}}}\Vert _{{L^1}}}<\infty$. This
proves \eqref{cuttoff1}, and the result follows.
\end{proof}

\par

\section{Mapping properties for pseudo-differential
operators on wave-front sets}\label{sec4}

\par

In this section we establish mapping properties for
pseudo-differential operators on wave-front sets of Fourier Banach
types. More precisely, we prove the following result (cf. \eqref{eq:inclusions}):

\par

\begin{thm}\label{mainthm2}
Let $\rho >0$, $\omega \in \mathscr P({\mathbf R^{{2d}}})$, $\omega _0 \in \mathscr P_{\rho ,0}({\mathbf R^{{2d}}})$, $a\in S^{(\omega _0)}_{\rho ,0} ({\mathbf R^{{2d}}})$, and $f\in \mathscr
S'({\mathbf R^{d}})$. Also let $\mathscr B$ be a translation
invariant BF-space on ${\mathbf R^{d}}$.
Then
\begin{multline}\label{wavefrontemb1}
{\operatorname{WF}} _{{\mathscr F\! \mathscr B} (\omega /\omega _0)} ({\operatorname{Op}} (a)f) \subseteq
{\operatorname{WF}} _{{\mathscr F\! \mathscr B} (\omega )} (f)\\[1 ex]
\subseteq {\operatorname{WF}} _{{\mathscr F\! \mathscr B} (\omega /\omega _0)} ({\operatorname{Op}} (a)f){{\textstyle{\, \bigcup \, }}}
{\operatorname{Char}} _{(\omega _0 )}(a).
\end{multline}
\end{thm}

\par

We shall mainly follow the proof of Theorem 3.1 in \cite{PTT1}. The following restatement of Proposition 3.2 in \cite{PTT1} shows that  $ (x_0, \xi) \not\in
{\operatorname{WF}} _{{\mathscr F\! \mathscr B} (\omega /\omega _0)} ({\operatorname{Op}} (a)f) $ when $ x_0 \not\in {\operatorname{supp}} f $.

\par

\begin{prop}\label{propmain1AA}
Let $\omega \in \mathscr P({\mathbf R^{{2d}}})$,  $\omega _0 \in \mathscr
P_{\rho ,\delta} ({\mathbf R^{{2d}}})$,
$0\le \delta \le \rho$, $0<\rho$, $\delta <1$, and let $a\in
S^{(\omega _0 )}_{\rho ,\delta}({\mathbf R^{{2d}}})$. Also let $\mathscr B$ be a translation invariant BF-space, and let the operator
$L_a$ on $\mathscr S'({\mathbf R^{d}})$ be defined by the formula
\begin{equation}\label{Ladef}
(L_af)(x) \equiv  {\varphi} _1(x)({\operatorname{Op}} (a)({\varphi} _2f))(x), \quad f\in \mathscr
S'({\mathbf R^{d}}),
\end{equation}
where ${\varphi} _1\in
C_0^{\infty}({\mathbf R^{d}})$ and ${\varphi} _2 \in S_{0,0}^0({\mathbf R^{d}})$ are such that
$$
{\operatorname{supp}} {\varphi} _1\bigcap {\operatorname{supp}} {\varphi} _2=\emptyset .
$$
Then the kernel of $L_a$ belongs to
$\mathscr S({\mathbf R^{{2d}}})$. In particular, the following is true:
\begin{enumerate}
\item $L_a ={\operatorname{Op}} (a_0)$ for some $a_0\in \mathscr S({\mathbf R^{{2d}}})$;

{\vspace{0.1cm}}

\item ${\operatorname{WF}} _{{\mathscr F\! \mathscr B} (\omega /\omega _0)}(L_af)=\emptyset$.
\end{enumerate}
\end{prop}

\par

Next we consider properties of the wave-front set of ${\operatorname{Op}} (a)f$ at a
fixed point when $f$ is concentrated to that point.

\par

\begin{prop}\label{keyprop2AA}
Let $\rho$, $\omega$, $\omega _0$, $a$ and $\mathscr B$ be as in Theorem \ref{mainthm2}. Also let 
$f\in \mathscr E'({\mathbf R^{d}})$. Then the following is true:
\begin{enumerate}
\item if $\Gamma _1,\Gamma _2\subseteq {\mathbf R^{d}}{\setminus 0}$ are open cones such that $\overline{\Gamma _2}\subseteq \Gamma _1$, and
$|f|_{{\mathscr F\! \mathscr B} (\omega ,\Gamma _1)}<\infty$, then $|{\operatorname{Op}} (a)f|_{{\mathscr F\! \mathscr B} (\omega /\omega _0,\Gamma _2)}<\infty$;

{\vspace{0.1cm}}

\item ${\operatorname{WF}}  _{{\mathscr F\! \mathscr B} (\omega /\omega _0)}({\operatorname{Op}} (a)f)\subseteq
{\operatorname{WF}} _{{\mathscr F\! \mathscr B} (\omega )}(f)$.
\end{enumerate}
\end{prop}

\par

We note that ${\operatorname{Op}} (a)f$ in Proposition \ref{keyprop2AA} makes sense as
an element in $\mathscr S'({\mathbf R^{d}})$, by Proposition \ref{p5.4}.

\par

\begin{proof}
We shall mainly follow the proof of Proposition 3.3 in
\cite{PTT1}. We may assume that $\omega (x,\xi )=\omega (\xi )$, $\omega
_0(x,\xi )=\omega _0(\xi )$, and that ${\operatorname{supp}} a\subseteq K\times {\mathbf R^{d}}$
for some compact set $K\subseteq {\mathbf R^{d}}$, since the
statements only involve local assertions.

\par

Let  $F(\xi )=|\widehat f(\xi )\omega (\xi )|$, and let $\mathscr
F_1a$ denote the partial Fourier transform of $a(x, \xi)$ with respect
to the $x$ variable. By straightforward computation, for arbitrary $N$ we have
\begin{equation}\label{estpseudo}
|\mathscr{F}({\operatorname{Op}}(a)f)(\xi)\omega(\xi)/\omega _0(\xi)|
\leq
 C \int_{{\mathbf R^{{d}}}} {\langle {\xi-\eta}\rangle}^{-N} F(\eta)\, d\eta ,
\end{equation}
for some constant $C$ (cf. (3.6) and (3.8) in \cite{PTT1}).

\par

We have to estimate
$$
|({\operatorname{Op}}(a)f)|_{{\mathscr F\! \mathscr B} (\omega/\omega _0,\Gamma_2)}
=
\| \mathscr{F}({\operatorname{Op}}(a)f)\omega/\omega _0 \chi_{\Gamma_2}\| _{\mathscr{B}}.
$$
By \eqref{estpseudo} we get
\begin{multline*}
\|\mathscr{F}({\operatorname{Op}}(a)f)\omega/\omega _0 \chi_{\Gamma_2}\|_{\mathscr{B}}
\leq
C\Big \| \Big ( \int {\langle {\cdot - \eta}\rangle}^{-N}F(\eta ) \, d\eta
\Big ) \chi _{\Gamma _2}\Big \| _{\mathscr{B}}
\cr
\leq
C(J_1 + J_2),
\end{multline*}
where $C$ is a constant and
\begin{align*}
J_1 &= \Big \| \Big ( \int _{\Gamma_1}{\langle {\cdot - \eta}\rangle}^{-N}F(\eta
) \, d\eta \Big ) \chi _{\Gamma_2 }\Big \| _{\mathscr{B}}
\intertext{and}
J_2 &= \Big \| \Big ( \int _{\complement \Gamma_1}{\langle {\cdot -
\eta}\rangle}^{-N}F(\eta ) \, d\eta \Big ) \chi _{\Gamma_2 }\Big \|
_{\mathscr{B}}.
\end{align*}

\par

In order to estimate $J_1$ and $J_2$ we argue as in the proof of
\eqref{cuttoff1}. More precisely, by \eqref{propupps} we get
\begin{multline*}
J_1 \leq
\Big \| \int_{\Gamma_1}{\langle {{\, \cdot \, }  - \eta}\rangle}^{-N}F(\eta ) \, d\eta \Big
\|_{\mathscr{B}}
=
\|{\langle {\, \cdot \, }\rangle}^{-N}* (\chi_{\Gamma_1}F)\| _{\mathscr{B}}
\\[1ex]
\leq
C\| {\langle {\, \cdot \, }\rangle} ^{-N}\|_{L^1_{(v)}}\|\chi_{\Gamma_1}F\|_{\mathscr{B}}
<
\infty.
\end{multline*}

\par

Next we estimate $J_2$. Since $\overline {\Gamma_2} \subseteq \Gamma_1$, we get
$$
|\xi -\eta |\ge c\max ( |\xi |,|\eta |),\quad \text{when}\quad \xi \in
\Gamma _2,\ \text{and}\ \eta \in \complement \Gamma _1,
$$
for some constant $c>0$. (Cf. the proof of Proposition 3.3.)

\par

Since $f$ has compact support, it follows that $F(\eta )\le C{\langle \eta\rangle} ^{t_1}$ for some constant $C$. By combining these estimates we
obtain
\begin{multline*}
J_2
\leq
\Big \| \Big (\int _{\complement \Gamma_1}F(\eta){\langle {{\, \cdot \, } -
\eta}\rangle}^{-N} \, d\eta \Big ) \chi _{\Gamma _2}\Big \| _{\mathscr{B}}
\\[1ex]
\leq
C\Big \| \Big ( \int_{\complement \Gamma_1}{\langle {\eta}\rangle}^{t_1}{\langle {\, \cdot \, }\rangle} ^{-N/2} {\langle \eta\rangle}^{-N/2} \, d\eta \Big ) \chi_{\Gamma_2}\Big \|
_{\mathscr{B}}
\\[1ex]
\leq
C\|{\langle {\, \cdot \, }\rangle} ^{-N/2}\chi_{\Gamma _2}\| _{\mathscr{B}} \int
_{\complement \Gamma_1} {\langle \eta\rangle}^{-N/2+t_1} \, d \eta.
\end{multline*}
Hence, if we choose $N$ sufficiently large, it follows that the
right-hand side is finite. This proves (1).

\par

The assertion (2) follows immediately from (1) and the
definitions. The proof is complete.
\end{proof}

\par

\begin{proof}[Proof of Theorem \ref{mainthm2}]
By Proposition \ref{pseudocomp} it is no restriction to assume that
$t=0$.
We start to prove the first inclusion in \eqref{wavefrontemb1}.
Assume that $(x_0,\xi _0)\notin {\operatorname{WF}} _{{\mathscr F\! \mathscr B} {(\omega)}}(f)$, let $\chi \in C_0^\infty ({\mathbf R^{d}})$ be such that $\chi =1$
in a neighborhood of $x_0$, and set $\chi _1=1-\chi$ and $a_0(x,\xi
)=\chi (x)a(x,\xi )$. Then it follows from Proposition
\ref{propmain1AA} that
$$
(x_0,\xi _0)\notin {\operatorname{WF}} _{{\mathscr F\! \mathscr B} {(\omega /\omega _0)}}({\operatorname{Op}} (a)(\chi
_1f)).
$$
Furthermore, by Proposition \ref{keyprop2AA} we get
\begin{equation*}
(x_0,\xi _0)\notin {\operatorname{WF}} _{{\mathscr F\! \mathscr B} {(\omega /\omega _0
)}}({\operatorname{Op}} (a_0)(\chi f)),
\end{equation*}
which implies that
$$
(x_0,\xi _0)\notin {\operatorname{WF}} _{{\mathscr F\! \mathscr B} {(\omega /\omega _0 )}}({\operatorname{Op}} (a)(\chi f)),
$$
since ${\operatorname{Op}} (a)(\chi f)$ is equal to ${\operatorname{Op}} (a_0)(\chi f)$ near
$x_0$. The result is now a consequence of the inclusion
\begin{multline*}
{\operatorname{WF}} _{{\mathscr F\! \mathscr B} {(\omega /\omega _0)}}({\operatorname{Op}} (a)f)
\\[1ex]
\subseteq {\operatorname{WF}} _{{\mathscr F\! \mathscr B} {(\omega /\omega _0 )}}({\operatorname{Op}} (a)(\chi f)){{\textstyle{\, \bigcup \, }}} {\operatorname{WF}} _{{\mathscr F\! \mathscr B}
{(\omega /\omega _0 )}}({\operatorname{Op}} (a)(\chi _1f)).
\end{multline*}

\par

It remains to prove the last inclusion in \eqref{wavefrontemb1}. By
Proposition \ref{propmain1AA} it follows that it is no restriction to
assume that $f$ has compact support. Assume that
$$
(x_0,\xi _0)\notin  {\operatorname{WF}} _{{\mathscr F\! \mathscr B} (\omega /\omega _0)}({\operatorname{Op}}
(a)f){{\textstyle{\, \bigcup \, }}} {\operatorname{Char}} _{(\omega _0)}(a),
$$
and choose $b$, $c$ and $h$ as in Proposition \ref{psiecharequiv} (4). We
shall prove that $(x_0,\xi _0)\notin  {\operatorname{WF}}  _{{\mathscr F\! \mathscr B} (\omega )}(f)$. Since
$$
f = {\operatorname{Op}} (1-c)f + {\operatorname{Op}} (b){\operatorname{Op}} (a)f-{\operatorname{Op}} (h)f,
$$
the result follows if we prove
$$
(x_0,\xi _0)\notin \mathsf S_1{{\textstyle{\, \bigcup \, }}} \mathsf S_2{{\textstyle{\, \bigcup \, }}} \mathsf
S_3,
$$
where
\begin{align*}
\mathsf S_1 &= {\operatorname{WF}} _{{\mathscr F\! \mathscr B} (\omega )}({\operatorname{Op}} (1-c)f),\quad
\mathsf S_2 = {\operatorname{WF}} _{{\mathscr F\! \mathscr B} (\omega )}({\operatorname{Op}} (b){\operatorname{Op}} (a)f)
\\[1ex]
\text{and}\quad \mathsf S_3 &= {\operatorname{WF}} _{{\mathscr F\! \mathscr B} (\omega )}({\operatorname{Op}}
(h)f).
\end{align*}

\par

We start to consider $\mathsf S_2$. By the first embedding in
\eqref{wavefrontemb1} it follows that
$$
\mathsf S_2 = {\operatorname{WF}} _{{\mathscr F\! \mathscr B} (\omega )}({\operatorname{Op}} (b){\operatorname{Op}} (a)f)
\subseteq {\operatorname{WF}}  _{{\mathscr F\! \mathscr B} (\omega /\omega _0)}({\operatorname{Op}} (a)f).
$$
Since we have assumed that $(x_0,\xi _0)\notin {\operatorname{WF}} _{{\mathscr F\! \mathscr B} (\omega /\omega _0)}({\operatorname{Op}}
(a)f)$, it follows that $(x_0,\xi _0)\notin \mathsf S_2$.

\par

Next we consider $\mathsf S_3$. Since $h\in \mathscr S$, it follows that ${\operatorname{Op}} (h)f\in \mathscr S$. Hence $S_3$ is empty.

\par

Finally we consider $\mathsf S_1$. By the assumptions it follows
that $c_0=1-c$ is zero in $\Gamma$, and by replacing $ \Gamma $ with a
smaller cone, if necessary, we may assume that $ c_0 = 0 $ in a
conical neighborhood of $ \Gamma$. Hence, if $\Gamma \equiv \Gamma _1$, $\Gamma _2$, $J_1$ and $J_2$ are the
same as in the
proof of Proposition \ref{keyprop2AA}, then it follows from that proof
and the fact that $c_0(x,\xi )\in S^0_{\rho ,0}$ is compactly
supported in the $x$-variable, that $J_1<+\infty$, and that for each $N\ge
0$, there are constants $C_N$ and $C_N'$ such that
\begin{multline}\label{estagain4}
| {\operatorname{Op}} (c_0)f |_{{\mathscr F\! \mathscr B} (\omega /\omega _0,\Gamma _2)} \le
C_N (J_1+J_2)
\\[1ex]
\le C_N' \Big (J_1+ \Big \Vert \int _{\complement
\Gamma _1}{\langle {\, \cdot \, }\rangle} ^{-N}{\langle \eta\rangle} ^{-N}\, d\eta \, \chi _{\Gamma _2} 
\Big \Vert _{\mathscr B} \Big ).
\end{multline}
By choosing $N$ large enough, it follows that
$$
|{\operatorname{Op}} (c_0)f |_{{\mathscr F\! \mathscr B}(\omega/\omega _0,\Gamma _2)}
< \infty .
$$
This proves that $(x_0,\xi _0)\notin \mathsf S_1$, and
the proof is complete.
\end{proof}

\par

\begin{rem}\label{rho0}
We note that the statements in Theorems \ref{mainthm2} are not true if $\omega _0=1$ and the assumption $\rho >0$ is replaced by
$\rho =0$. (Cf. Remark 3.7 in \cite{PTT1}.)
\end{rem}

\medspace

Next we apply Theorem \ref{mainthm2} on
operators which are elliptic with respect to $S^{(\omega _0)}_{\rho
,\delta}({\mathbf R^{{2d}}})$, where $\omega _0\in \mathscr P_{\rho,\delta}({\mathbf R^{{2d}}})$. More
precisely, assume that $0\le \delta <\rho \le1$ and $a\in S^{(\omega
_0)}_{\rho ,\delta}({\mathbf R^{{2d}}})$. Then $a$ and
${\operatorname{Op}} (a)$ are called (locally) \emph{elliptic} with respect to
$S^{(\omega _0)}_{\rho ,\delta}({\mathbf R^{{2d}}})$ or $\omega _0$,
if for each compact set $K\subseteq {\mathbf R^{d}}$, there are positive
constants $c$ and $R$ such that
$$
|a(x,\xi )| \ge c\omega _0(x,\xi ),\quad x\in K,\ |\xi |\ge R.
$$
Since $|a(x,\xi )|\le C\omega _0(x,\xi )$, it follows from the
definitions that for each multi-index $\alpha$, there are constants
$C_{\alpha,\beta}$ such that
\begin{equation*}
|\partial ^\alpha _x\partial ^\beta _\xi a(x,\xi )| \le C_{\alpha
,\beta}|a(x,\xi )|{\langle \xi\rangle} ^{-\rho |\beta |+\delta |\alpha |},\quad
x\in K,\   |\xi |>R,
\end{equation*}
when $a$ is elliptic. (See e.{\,}g. \cite {Ho1,BBR}.) 

\par

It immediately follows from the
definitions that ${\operatorname{Char}} _{(\omega _0)}(a)=\emptyset$ when $a$ is
elliptic with respect to $\omega _0$. The following result is now an
immediate consequence of Theorem \ref{mainthm2}.

\par

\begin{thm}\label{hypoellthm}
Let $\omega \in \mathscr P({\mathbf R^{{2d}}})$, $\omega _0\in \mathscr P_{\rho ,0}({\mathbf R^{{2d}}})$, $\rho >0$, and
let $a\in S^{(\omega _0)} _{\rho ,0} ({\mathbf R^{{2d}}})$ be elliptic with respect to $\omega _0$. Also let $\mathscr B$
be a translation invariant BF-space.
If $f\in \mathscr S'({\mathbf R^{d}})$, then
$$
{\operatorname{WF}} _{{\mathscr F\! \mathscr B} (\omega /\omega _0)} ({\operatorname{Op}} (a)f)= {\operatorname{WF}} _{{\mathscr F\! \mathscr B} (\omega )} (f).
$$
\end{thm}

\par

\section{Wave-front sets of sup and inf types and
pseudo-differential operators}\label{sec5}

\par

In this section we put the micro-local analysis in a more general
context compared to previous sections,
and define wave-front sets with respect to sequences of
Fourier BF-spaces.

\par

Let $\omega _j \in \mathscr
P({\mathbf R^{{2d}}})$ and $\mathscr B_j$ be translation invariant BF-space on
${\mathbf R^{d}}$ when $j$ belongs to some index set $J$, and consider the array
of spaces, given by
\begin{equation}\label{notconvsequences}
(\mathcal B_j) \equiv (\mathcal B_j)_{j\in J},\quad
\text{where}\quad \mathcal B_j={\mathscr F\! \mathscr B} _j {(\omega
_j)}, \quad j \in J.
\end{equation}

\par

If $f\in \mathscr S'({\mathbf R^{d}})$, and $(\mathcal B_j)$ is given by
\eqref{notconvsequences}, then we let
$\Theta_{(\mathcal B_j) }^{\sup}(f)$ be the set of all $\xi \in {\mathbf R^{d}}{\setminus 0}$ such that for some $\Gamma = \Gamma _{\xi}$ and \emph{each}
$j\in J$ it holds $|f|_{\mathcal B_j(\Gamma )} < \infty$. We
also let $\Theta _{(\mathcal B_j) }^{\inf}(f)$ be the set of all $
\xi \in {\mathbf R^{d}}{\setminus 0} $ such that for some $\Gamma = \Gamma
_{\xi}$ and \emph{some} $j\in J$ it holds $|f|_{\mathcal B_j(\Gamma)}
<\infty$. Finally we let $\Sigma _{(\mathcal B_j) }^{\sup} (f)$ and
$\Sigma _{(\mathcal B_j) }^{\inf} (f)$ be the complements in ${\mathbf R^{d}}{\setminus 0} $ of $\Theta_{(\mathcal B_j) }^{\sup}(f)$ and $\Theta
_{(\mathcal B_j) }^{\inf} (f)$ respectively.

\par

\begin{defn}\label{defsuperposWF}
Let $J$ be an index set, $\mathscr B_j$ be translation invariant
BF-space on ${\mathbf R^{d}}$, $\omega _j\in \mathscr P({\mathbf R^{{2d}}})$ when $j\in
J$, $(\mathcal B_j)$ be as in
\eqref{notconvsequences}, and let $X$ be an open subset of ${\mathbf R^{d}}$.
\begin{enumerate}
\item The wave-front set of $f\in \mathscr D'(X)$,
${\operatorname{WF}} ^{\, \sup} _{(\mathcal B_j) }(f) = {\operatorname{WF}} ^{\, \sup} _{({\mathscr F\! \mathscr B}
_j(\omega _j) )}(f)$,
of \emph{sup-type} with respect to $(\mathcal B_j) $,
consists of all pairs
$(x_0,\xi_0)$ in $X\times ({\mathbf R^{d}} {\setminus 0})$ such that
$
\xi _0 \in  \Sigma ^{\sup} _{(\mathcal B_j)} ({\varphi} f)
$
holds for each ${\varphi} \in C_0^\infty (X)$ such that ${\varphi} (x_0)\neq
0$;

{\vspace{0.1cm}}

\item The wave-front set of $f\in \mathscr D'(X)$,
${\operatorname{WF}} ^{\, \inf} _{(\mathcal B_j) }(f) = {\operatorname{WF}} ^{\, \inf} _{({\mathscr F\! \mathscr B}
_j(\omega _j) )}(f)$, of \emph{inf-type} with respect to $(\mathcal
B_j)$, consists of all pairs
$(x_0,\xi_0)$ in $X\times ({\mathbf R^{d}} {\setminus 0})$ such that
$
\xi _0 \in  \Sigma ^{\inf} _{(\mathcal B_j)} ({\varphi} f)
$
holds for each ${\varphi} \in C_0^\infty (X)$ such that ${\varphi} (x_0)\neq
0$.
\end{enumerate}
\end{defn}

\par

\begin{rem}\label{remstandWF}
Let $\omega _j(x,\xi ) = {\langle \xi\rangle} ^{-j}$ for $j\in J=\mathbf
N_0$ and $\mathscr B_j=L^{q_j}$, where $q_j\in [1,\infty ]$. Then it
follows that ${\operatorname{WF}} _{(\mathcal B_j)}^{\, \sup}(f)$ in
Definition \ref{defsuperposWF} is equal to the standard wave front
set ${\operatorname{WF}} (f)$ in Chapter VIII in \cite{Ho1}.
\end{rem}

\par

The following result follows immediately from Theorem \ref{mainthm2}
and its proof. We omit the details.

\par

\begin{tom}
Let $\rho >0$, $\omega _j\in \mathscr P({\mathbf R^{{2d}}})$ for $j\in J$, $\omega _0 \in \mathscr P_{\rho ,0}({\mathbf R^{{2d}}})$, $a\in S^{(\omega _0)}_{\rho ,0} ({\mathbf R^{{2d}}})$ and $f\in \mathscr
S'({\mathbf R^{d}})$. Also let $\mathscr B_j$ be a translation
invariant BF-space on ${\mathbf R^{d}}$ for every $j$. Then
\begin{multline}\tag*{(\ref{wavefrontemb1})$'$}
{\operatorname{WF}} ^{\, \sup}_{({\mathscr F\! \mathscr B} _j(\omega _j/\omega _0))} ({\operatorname{Op}} (a)f) \subseteq
{\operatorname{WF}} ^{\, \sup}_{({\mathscr F\! \mathscr B}_j (\omega _j))} (f)
\\[1ex]
\subseteq {\operatorname{WF}} ^{\, \sup}_{({\mathscr F\! \mathscr B} _j(\omega _j/\omega _0))} ({\operatorname{Op}}
(a)f){{\textstyle{\, \bigcup \, }}} {\operatorname{Char}} _{(\omega _0)}(a),
\end{multline}
and
\begin{multline}\tag*{(\ref{wavefrontemb1})$''$}
{\operatorname{WF}} ^{\, \inf}_{({\mathscr F\! \mathscr B} _j(\omega _j/\omega _0))} ({\operatorname{Op}} (a)f) \subseteq
{\operatorname{WF}} ^{\, \inf}_{({\mathscr F\! \mathscr B}_j (\omega _j))} (f)
\\[1ex]
\subseteq {\operatorname{WF}} ^{\, \inf}_{({\mathscr F\! \mathscr B} _j(\omega _j/\omega _0))} ({\operatorname{Op}} (a)f){{\textstyle{\, \bigcup \, }}}
{\operatorname{Char}} _{(\omega _0)}(a).
\end{multline}
\end{tom}

\par

The following generalization of Theorem \ref{hypoellthm} is an
immediate consequence of Theorem \ref{mainthm2}$'$, since ${\operatorname{Char}}
_{(\omega _0)}(a)=\emptyset$, when $a$ is elliptic with respect to
$\omega_0$.

\par

\begin{tom}

Let $\rho >0$, $\omega _j\in \mathscr P({\mathbf R^{{2d}}})$ for $j\in J$, $\omega _0 \in \mathscr P_{\rho ,0}({\mathbf R^{{2d}}})$ and let $a\in S^{(\omega _0)}_{\rho ,0} ({\mathbf R^{{2d}}})$ be elliptic
with respect to $\omega _0$.  Also let $\mathscr B_j$ be a translation
invariant BF-space on ${\mathbf R^{d}}$ for every $j$. If $f\in \mathscr S'({\mathbf R^{d}})$, then
$$
{\operatorname{WF}} ^{\, \sup}_{({\mathscr F\! \mathscr B} _j(\omega _j/\omega _0))} ({\operatorname{Op}} (a)f) =
{\operatorname{WF}} ^{\, \sup}_{({\mathscr F\! \mathscr B}_j (\omega _j))} (f)
$$
and 
$$
{\operatorname{WF}} ^{\, \inf}_{({\mathscr F\! \mathscr B} _j(\omega _j/\omega _0))} ({\operatorname{Op}} (a)f) =
{\operatorname{WF}} ^{\, \inf}_{({\mathscr F\! \mathscr B}_j (\omega _j))} (f).
$$

\end{tom}

\par

\begin{rem}\label{remhormWFsets}
We note that many properties valid for the wave-front sets of Fourier
Banach type also hold for wave-front sets in the present
section. For example, the conclusions in Remark \ref{rho0}
hold for wave-front sets of sup- and inf-types.
\end{rem}

\par

Finally we remark that there are some technical generalizations of
Theorem \ref{mainthm2} which involve pseudo-differential operators
with symbols in $S^{(\omega _0)}_{\rho ,\delta} ({\mathbf R^{{2d}}})$ with $0\le
\delta <\rho \le 1$. From these generalizations it follows that
$$
{\operatorname{WF}} ({\operatorname{Op}} (a)f) \subseteq {\operatorname{WF}}(f) \subseteq {\operatorname{WF}} ({\operatorname{Op}} (a)f){{\textstyle{\, \bigcup \, }}} {\operatorname{Char}}
_{(\omega _0)}(a),
$$
when $0\le \delta <\rho \le 1$, $\omega _0\in \mathscr P_{\rho ,\delta
}({\mathbf R^{{2d}}})$, $a\in S^{(\omega _0)}_{\rho ,\delta}({\mathbf R^{{2d}}})$ and
$f\in \mathscr S'({\mathbf R^{d}})$. (Cf. Theorem 5.3$'$ and Theorem 5.5 in
\cite{PTT1}.)

\par

\section{Wave front sets with respect to modulation spaces}\label{sec6}

\par

In this section we define wave-front sets with respect to modulation
spaces, and show that they coincide with wave-front sets of Fourier
Banach types. In particular, all micro-local properties for
pseudo-differential operators in the previous sections carry over to
wave-front sets of modulation space types.

\par

We start with defining general types of modulation spaces. Let
(the window) $\phi \in \mathscr S'({\mathbf R^{d}}){\setminus 0}$ be
fixed, and let $f\in \mathscr S'({\mathbf R^{d}})$. Then the short-time Fourier
transform $V_\phi f$ is the element in $\mathscr S'({\mathbf R^{{2d}}})$,
defined by the formula
$$ 
(V_{\phi} f)(x,\xi) \equiv \mathscr{F}(f\cdot
\overline{\phi(\cdot-x)})(\xi).
$$
We usually assume that $\phi \in \mathscr{S}({\mathbf R^{{d}}})$, and in
this case the short-time Fourier transform $(V_{\phi}f)$ takes
the form 
$$ 
(V_{\phi} f)(x,\xi)
=
(2\pi)^{-d/2}\int_{{\mathbf R^{{d}}}} f(y)\overline{\phi(y-x)}e^{-i{\langle y,\xi\rangle}
}\, dy,
$$ 
when $f\in \mathscr{S}({\mathbf R^{{d}}})$.

\par

Now let $\mathscr{B}$ be a translation invariant BF-space on
${\mathbf R^{{2d}}}$, with respect to $v\in \mathscr P({\mathbf R^{{2d}}})$. Also let
$\phi \in \mathscr{S}({\mathbf R^{{d}}}){\setminus {0}}$ and $\omega \in
\mathscr{P}({\mathbf R^{{2d}}})$ be such that $\omega$ is $v$-moderate. Then the
modulation space $M(\omega)=M(\omega ,\mathscr B)$ consists of all $f\in
\mathscr{S}'({\mathbf R^{{d}}})$ such that $V_{\phi}f\cdot \omega \in
\mathscr{B}$. We note that $M(\omega ,\mathscr B)$ is a Banach space
with the norm
\begin{equation}\label{modnorm}
\|f\|_{M(\omega ,\mathscr B)} \equiv \|V_{\phi} f
\omega\|_{\mathscr{B}}
\end{equation}
(cf. \cite{Feichtinger3}).

\par

\begin{rem}\label{Modamalgam}
Assume that $p,q\in [1,\infty]$, $\omega\in \mathscr P ({\mathbf R^{{2d}}})$ and
let $L^{p,q}_{1}({\mathbf R^{{2d}}})$ and $L^{p,q}_{2}({\mathbf R^{{2d}}})$ be the sets of
all $F\in  L^1_{loc} ({\mathbf R^{{2d}}})$ such that
\begin{equation*}
\|F\|_{L^{p,q}_1}  \equiv \Big ( \int \Big( \int |F(x,\xi)|^p\,
dx\Big )^{q/p}\,d\xi \Big )^{1/q}
<\infty
\end{equation*}
and
\begin{equation*}
\|F\|_{L^{p,q}_2} \equiv \Big ( \int \Big ( \int |F(x,\xi)|^p\,
d\xi \Big )^{q/p}\, dx\Big )^{1/q}<\infty ,
\end{equation*}
respectively (with obvious modifications when $p=\infty$ or
$q=\infty$). Then $M(\omega ,\mathscr B)$ is equal to the usual
modulation space $M^{p,q}_{(\omega)}({\mathbf R^{{d}}})$ when
$\mathscr{B}=L^{p,q}_1({\mathbf R^{{2d}}})$. If instead
$\mathscr{B}=L^{p,q}_2({\mathbf R^{{2d}}})$, then $M(\omega ,\mathscr B)$ is
equal to the space $W^{p,q}_{(\omega)}({\mathbf R^{{d}}})$, related to
Wiener-amalgam spaces.
\end{rem}

\par

In the following proposition we list some important properties for
modulation spaces. We refer to \cite{Gro-book} for the proof.

\par

\begin{prop}\label{modproperties}
Assume that $\mathscr B$ is a translation invariant BF-space on ${\mathbf R^{{2d}}} $with respect to $v\in \mathscr P({\mathbf R^{{2d}}})$, and that
$\omega _0, v_0\in\mathscr{P}({\mathbf R^{{2d}}})$ are such that $\omega$
is $v$-moderate. Then the following is true:
\begin{enumerate}
\item if $\phi\in M^1_{(v_0v)}({\mathbf R^{{d}}}){\setminus {0}}$, then $f\in M(\omega
,\mathscr B)$ if and only if $V_\phi f \omega \in \mathscr
B$. Furthermore, \eqref{modnorm} defines a norm on $M(\omega
,\mathscr B)$, and different choices of $\phi$ gives rise to equivalent
norms;

{\vspace{0.1cm}}

\item $ M^1_{(v_0v)}({\mathbf R^{d}})\subseteq M(\omega ,\mathscr{B})\subseteq
M^{\infty}_{(1/(v_0v))}({\mathbf R^{d}})$.
\end{enumerate}
\end{prop}

\par

The following generalization of Theorem 2.1 in \cite{RSTT} shows that
modulation spaces are locally the same as translation invariant
Fourier BF-spaces. We recall that if ${\varphi} \in \mathscr S  ({\mathbf R^{d}}){\setminus 0}$ and $\mathscr B$ is a translation invariant BF-space  on ${\mathbf R^{{2d}}}$, then it follows from Proposition \ref{propbnoll} that
\begin{equation}\label{B0def}
\mathscr{B}_0 \equiv {\{ \, {f\in \mathscr
S'({\mathbf R^{d}})}\, ;\, {{\varphi} \otimes f \in \mathscr B}\, \} }
\end{equation}
is a translation invariant BF-space on ${\mathbf R^{d}}$ which is independent of
the choice of ${\varphi}$.

\par

\begin{prop}\label{propekvnorm}
Let ${\varphi} \in C_0^\infty ({\mathbf R^{d}}){\setminus 0}$,
$\mathscr B$ be a translation invariant BF-space  on ${\mathbf R^{{2d}}}$, and
let $\mathscr B_0$ be as in \eqref{B0def}. Also let $\omega \in
\mathscr P({\mathbf R^{{2d}}})$, and $\omega _0(\xi )=\omega (x_0,\xi )$,
for some fixed $x_0\in {\mathbf R^{d}}$. Then
$$
M(\omega ,\mathscr B)\cap \mathscr E'({\mathbf R^{d}}) = {\mathscr F\! \mathscr B} _0(\omega _0)\cap
\mathscr E'({\mathbf R^{d}}).
$$
Furthermore, if $K\subseteq {\mathbf R^{d}}$ is compact, then 
\begin{equation}\label{locest1}
C^{-1}{\Vert {f}\Vert _{{{\mathscr F\! \mathscr B} _0(\omega _0)}}}\le {\Vert f\Vert _{{M(\omega ,\mathscr B)}}}\le
C{\Vert {f}\Vert _{{{\mathscr F\! \mathscr B} _0(\omega _0)}}},\quad f\in \mathscr E'(K),
\end{equation}
for some constant $C$ which only depends on $K$.
\end{prop}

\par

We need the following lemma for the proof.

\par

\begin{lemma}\label{korttidslemma}
Assume that $f\in \mathscr{E}'({\mathbf R^{{d}}})$. Then the following is true:
\begin{enumerate}
\item \label{kortl1} if $\phi\in C^{\infty}_0({\mathbf R^{{d}}})$, then there
exists $0\le {\varphi} \in C^{\infty}_0({\mathbf R^{{d}}})$ such that
\begin{equation}\label{stft-ft}
(V_{\phi}f)(x,\xi )={\varphi} (x)(\widehat {f}*(\mathscr {F}(\overline
{\phi(\cdot-x)})))(\xi )\, \text ;
\end{equation}

{\vspace{0.1cm}}

\item \label{kortl2} if ${\varphi} \in C^{\infty}_0({\mathbf R^{{d}}})$, then there
exists $\phi \in C^{\infty}_0({\mathbf R^{{d}}})$ such that
\begin{equation}\label{ft-stft}
({\varphi} \otimes \widehat{f}) (x,\xi) ={\varphi} (x)V_{\phi}f(x,\xi ).
\end{equation}
\end{enumerate}
\end{lemma}

\par

\begin{proof}
(1) Let ${\varphi} \in C_0^\infty$ be equal to $(2\pi )^{d/2}$ in a
compact set containing the support of the map $x\mapsto V_\phi f(x,\xi
)$. Then (1) is a straight-forward consequence of Fourier's
inversion formula.

\par

The assertion (2) follows by choosing $\phi \in C_0^\infty$ such that
$\phi =1$ on ${\operatorname{supp}} f -{\operatorname{supp}} {\varphi}$. 
\end{proof}

\par

\begin{proof}[Proof of Proposition \ref{propekvnorm}]
We may assume that $\omega =\omega _0=1$ in view of Remark
\ref{newbfspaces}. Assume that $f\in
\mathscr E'$ and ${\varphi} \in C_0^\infty ({\mathbf R^{d}}){\setminus 0}$. From
\eqref{kortl2} of Lemma \ref{korttidslemma} it follows that there
exists $\phi \in C^{\infty}_0$ such that
\begin{equation*}
\| f\|_{M(\mathscr{B})}= \| V_{\phi}f \| _{\mathscr{B}}=
\|\varphi \otimes \widehat{f}\| _{\mathscr{B}}= {\Vert {\widehat f}\Vert _{{\mathscr
B_0}}},
\end{equation*}
and \eqref{locest1} follows. The proof is complete.
\end{proof}

\medspace

Let $\mathscr B$ be a translation invariant BF-space on ${\mathbf R^{{2d}}}$, $\phi \in \mathscr{S}({\mathbf R^{{d}}} ){\setminus {0}}$ be fixed,
$\omega \in \mathscr{P}({\mathbf R^{{2d}}})$, $\Gamma \subseteq {\mathbf R^{{d}}}{\setminus {0}}$
be an open cone, and let $\chi_{\Gamma}(x,\xi)=\chi_{\Gamma}(\xi)$ be
the characteristic function of $\Gamma$. For any $f\in
\mathscr{S}'({\mathbf R^{{d}}})$ we set
\begin{multline}\label{modseminorm}
|f|_{\mathcal B(\Gamma )} =
|f|_{{M^{{}}(\omega _{{}},{{\Gamma}},\mathscr{B})}}  =|f|_{{M^{{\phi}}(\omega _{{}},{{\Gamma}},\mathscr{B})}}
\equiv
\|(V_{\phi}f) \omega\chi_{\Gamma}\|_{\mathscr{B}}
\\[1ex]
\text{when}\quad \mathcal B=M{(\omega ,\mathscr B)}.
\end{multline}
We note that $|{\, \cdot \, } |_{\mathcal B(\Gamma )}$ defines a semi-norm
on $\mathscr S'$ which might attain the value $+\infty$. If $\Gamma
={\mathbf R^{d}}{\setminus 0}$, then $|f|_{\mathcal B(\Gamma )} = {\Vert f\Vert _{{M{(\omega ,\mathscr B)}}}}$.

\par

Let $\mathscr B$ be a translation invariant BF-space on ${\mathbf R^{{2d}}}$,  $\omega \in \mathscr P({\mathbf R^{{2d}}})$, $f\in \mathscr D'(X)$, and let
$\mathcal B=M{(\omega ,\mathscr B)}$. Then $\Theta _{\mathcal B}(f)$, $\Sigma
_{\mathcal B}(f)$ and the wave-front set ${\operatorname{WF}}
_{\mathcal B}(f)$ of $f$ with respect to the modulation space
$\mathcal B$ are defined in the same way as in Section
\ref{sec3}, after replacing the semi-norms of Fourier Banach types in
\eqref{notconv} with the semi-norms in \eqref{modseminorm}.

\par

In Theorem \ref{WFidentity} below we prove that wave-front sets of Fourier BF-spaces
and modulation space types agree with each others. As a
first step we prove that
${\operatorname{WF}}_{{M^{{\phi}}(\omega _{{}},\mathscr{B}^{{}})}}(f)$ is independent of $\phi$ in \eqref{modseminorm}.

\par

\begin{prop}\label{fonsteroberoende}
Let $X\subseteq {\mathbf R^{d}}$ be open,
$f\in \mathscr{D}'(X)$ and $\omega\in \mathscr{P}({\mathbf R^{{2d}}})$. Then ${\operatorname{WF}}_{{M^{{\phi}}(\omega _{{}},\mathscr{B}^{{}})}}(f)$ is
independent of the window function $\phi \in \mathscr S ({\mathbf R^{{d}}})
{\setminus {0}}$.
\end{prop}

\par

We need some preparation for the proof, and start with the following
lemma. We omit the proof (the result can be found in \cite{CG1}).

\par

\begin{lemma}\label{STFTdecay}
Let $f\in \mathscr E'({\mathbf R^{d}})$ and $\phi \in \mathscr S ({\mathbf R^{d}})$. Then
for some constant $N_0$ and every $N\geq 0$, there are constants $C_N$
such that
$$
|V_{\phi} f(x,\xi)| \leq C_N {\langle x\rangle} ^{-N}{\langle \xi\rangle}^{N_0}.
$$
\end{lemma} 

\par

The following result can be found in \cite{Gro-book}. Here $\widehat
*$ is the twisted convolution, given by the formula
$$
(F\, \widehat *\, G)(x,\xi )=(2\pi )^{-d/2}\iint F(x-y,\xi -\eta
)G(y,\eta )e^{-i{\langle {x-y},\eta\rangle}}\, dyd\eta ,
$$
when $F,G\in \mathscr S({\mathbf R^{{2d}}})$. The definition of $\widehat *$
extends in such way that one may permit one  of $F$ and $G$ to belong
to $\mathscr S'({\mathbf R^{{2d}}})$, and in this case it follows that $F \,
\widehat * \, G$ belongs to $\mathscr S'\cap C^\infty$.

\par

\begin{lemma}\label{stftproperties}
Let $f\in \mathscr S'({\mathbf R^{d}})$ and $\phi _j\in
\mathscr S({\mathbf R^{d}})$ for $j=1,2,3$. Then
$$
(V_{\phi _1}f)\widehat*(V_{\phi _2}\phi _3) = (\phi
_3,\phi _1)_{L^2}\cdot V_{\phi _2}f.
$$
\end{lemma}

\par

\begin{proof}[Proof of Proposition \ref{fonsteroberoende}]
We assume that $f\in \mathscr{E}'({\mathbf R^{d}})$ and that $\omega(x,\xi) =\omega(\xi)$, since the statements only involve local assertions. Assume that $\phi, \phi_1 \in \mathscr S({\mathbf R^{d}}){\setminus 0}$ and let $\Gamma _1$ and $\Gamma_2$ be open cones in ${\mathbf R^{d}}$ such that $\overline{\Gamma_2}\subseteq \Gamma_1$. The assertion follows if we prove that 
\begin{equation}\label{modseminormineq}
|f|_{{M^{{\phi}}(\omega _{{}},{{\Gamma_2}},\mathscr{B})}} \leq C(|f|_{{M^{{\phi_1}}(\omega _{{}},{{\Gamma_1}},\mathscr{B})}} + 1)
\end{equation}
for some constant $C$.

\par

When proving \eqref{modseminormineq} we shall mainly follow the proof
of \eqref{cuttoff1}. Let
$v \in \mathscr P$ be chosen such that $\omega$ is $v$-moderate, and let
$$
\Omega_1 = \{ (x, \xi) ; \xi\in \Gamma_1 \} \subseteq {\mathbf R^{{2d}}} \qquad
\text{and} \qquad \Omega_2 = \complement \Omega_1 \subseteq {\mathbf R^{{2d}}},
$$
with characteristic functions $\chi_1$ and $\chi_2$ respectively. Also
set
$$
F _k(x, \xi) = |V_{\phi_1}f(x, \xi)\omega(\xi)\chi _k(x, \xi)| \qquad
\text{and} \qquad G = |V_{\phi}\phi_1(x, \xi)v(\xi)|.
$$
By Lemma \ref{stftproperties}, and the fact that $\omega$ is $v$-moderate we get
$$
|V_{\phi}f(x, \xi)\omega(x, \xi)| \leq  C((F_1 + F_2) * G)(x, \xi),
$$
for some constant $C$, which implies that
\begin{equation}\label{uppskattning}
|f|_{{M^{{\phi}}(\omega _{{}},{{\Gamma_2}},\mathscr{B})}} \leq C (J _1 + J _2), 
\end{equation}
where
$$
J_k = {\Vert {(F_k * G)\chi_{\Gamma_2}}\Vert _{{\mathscr B}}}
$$
and $\chi_{\Gamma_2}(x,\xi)=\chi_{\Gamma_2}(\xi)$ is the characteristic function of $\Gamma_2$.
Since $G$ turns rapidly to zero at infinity, \eqref{propupps} gives
\begin{equation}\label{del1}
J_1 \leq {\Vert {F_1*G}\Vert _{{\mathscr B}}} \leq {\Vert G\Vert _{{L^1_{(v)}}}} {\Vert {F_1}\Vert _{{\mathscr B}}}=C|f|_{{M^{{\phi_1}}(\omega _{{}},{{\Gamma_1}},\mathscr{B})}}, 
\end{equation}
where $C = {\Vert {G}\Vert _{{L^1_{(v)}}}}$.

\par

Next we consider $J_2$. Since, for each $N \geq 0$, there are constants $C_N$ such that
$$
F_2(x, \xi)=0,\qquad \text{and} \qquad {\langle {\xi-\eta}\rangle}^{-2N}\leq C_N {\langle \xi\rangle}^{-N}{\langle \eta\rangle}^{-N}
$$
when $\xi\in \Gamma_2$ and $\eta \in \complement \Gamma_1$, it follows from Lemma \ref{STFTdecay} and the computations in \eqref{J2comp} that
$$
(F_2 * G)(x, \xi) \leq C_N {\langle x\rangle}^{-N}{\langle \xi\rangle}^{-N},\qquad  \xi\in \Gamma_2.
$$
Consequently, $J_2 < \infty$. The estimate \eqref{modseminormineq} is now a consequence of
\eqref{uppskattning}, \eqref{del1} and the fact that $J_2 < \infty$.
This completes the proof.
\end{proof}

\par

Since ${\operatorname{WF}}_{M^\phi(\omega,\mathscr B)} (f)$ is independent of $\phi$ we usually omit $\phi$ and write ${\operatorname{WF}}_{M(\omega,\mathscr B)} (f)$ instead. We are now able to prove the following.

\par

\begin{prop}\label{theta-sigma}
Assume that $\mathscr B$ is a translation invariant BF-space on ${\mathbf R^{{2d}}}$, $\mathscr B_0$ is given by \eqref{B0def}, $\phi\in \mathscr S ({\mathbf R^{d}}) {\setminus {0}}$ and
$\omega \in \mathscr P({\mathbf R^{{2d}}})$. Also assume that $f\in
\mathscr E'({\mathbf R^{{ d}}})$. Then
\begin{equation}\label{fy-invariant2}
\Theta_{{M^{{\phi}}(\omega _{{}},\mathscr{B}^{{}})}} (f) = \Theta_{\mathscr{FB}_0
(\omega )} (f)\quad \text{and}\quad
\Sigma_{{M^{{\phi}}(\omega _{{}},\mathscr{B}^{{}})}}  (f) = \Sigma_{\mathscr{FB}_0(\omega )} (f).
\end{equation}
\end{prop}

\par

\begin{proof}
We may assume that $\omega =1$ in view of Lemma \ref{newbfspaces}. Let $\Gamma _1,\Gamma _2$ be open cones in ${\mathbf R^{d}}{\setminus 0}$ such that $\overline {\Gamma _2}\subseteq \Gamma _1$, let $\chi _{\Gamma _2}(x,\xi )=\chi _{\Gamma _2}(\xi )$ be the characteristic function of $\Gamma _2$, and let ${\varphi}$ and $\phi$ be chosen such that (1) in Lemma \ref{korttidslemma} is fulfilled.

\par

By \eqref{stft-ft} it follows that
$$
|V_\phi f(x,\xi )| \le {\varphi} (x)(|\widehat f|*|\mathscr F\check \phi |)(\xi ).
$$
This gives
\begin{multline*}
|f|_{M^\phi (\Gamma _2,\mathscr B)} = |V_\phi f\chi _{\Gamma _2}|_{\mathscr B}
\le C | {\varphi} \otimes \big (  (|\widehat f|*|\mathscr F\check \phi |) \chi _{\Gamma _2}\big )|_{\mathscr B}
\\[1ex]
= C | (|\widehat f|*|\mathscr F\check \phi |) \chi _{\Gamma _2} |_{\mathscr B_0}
\le C(J_1+J_2),
\end{multline*}
for some constant $C$, where $J_1$ and $J_2$ are the same as in \eqref{J1def} and \eqref{J2def} with $\mathscr B_2=\mathscr B_0$, $\psi =|\mathscr F\check \phi |$ and $F=|\widehat f|$.

\par

A combination of the latter estimate, \eqref{J1comp} and \eqref{J2comp} now gives that for each $N\ge 0$, there is a constant $C_N$ such that
$$
|f|_{M^\phi (\Gamma _2,\mathscr B)}  \le C_N\Big ( |f|_{{\mathscr F\! \mathscr B}_0} +\sup _\xi |\widehat f(\xi ){\langle \xi\rangle} ^{-N}|\Big ).
$$
Hence, by choosing $N$ large enough it follows that $|f|_{M^\phi (\Gamma _2,\mathscr B)}$ is finite when $ |f|_{{\mathscr F\! \mathscr B}_0}<\infty$. Consequently,
\begin{equation}\label{thetaFBMB}
\Theta_{\mathscr{FB}_0} (f)\subseteq \Theta_{M(\mathscr B)} (f).
\end{equation}

\par

In order to get a reversed inclusion we choose ${\varphi}$ and $\phi$ such that Lemma \ref{korttidslemma} (2) is fulfilled. Then \eqref{ft-stft} gives
\begin{multline*}
|f|_{{\mathscr F\! \mathscr B} _0(\Gamma )} = \| {\varphi} \otimes (\widehat{f} \, \chi _\Gamma ) \| _{\mathscr B}
=\| ({\varphi} \otimes 1)(V_{\phi}f \, \chi _\Gamma ) \| _{\mathscr B} 
\\[1ex]
\le C_1{\Vert {\varphi}\Vert _{{L^\infty}}} {\Vert {V_{\phi}f \, \chi _\Gamma}\Vert _{{\mathscr B}}}
= C_2 |f|_{M^\phi (\Gamma ,\mathscr B)},
\end{multline*}
for some constants $C_1, C_2 >0.$ This proves that \eqref{thetaFBMB} holds with reversed inclusion. The proof is complete.
\end{proof}

\par

The following result is now an immediate consequence of Proposition \ref{theta-sigma}.

\par

\begin{thm}\label{WFidentity}
Assume that $\mathscr B$ is a translation invariant BF-space on ${\mathbf R^{{2d}}}$, $\mathscr B_0$ is given by \eqref{B0def}, $\omega \in \mathscr P({\mathbf R^{{2d}}})$, $X\subseteq {\mathbf R^{d}}$ is open and that $f\in \mathscr D'(X)$. Then
$$
{\operatorname{WF}}_{{\mathscr F\! \mathscr B} _0(\omega )}(f) = {\operatorname{WF}} _{M(\omega ,\mathscr B)}(f).
$$
\end{thm}

\par

\vspace{2cm}

\begin{thebibliography}{150}

\bibitem{BaC}
W. Baoxiang, H. Chunyan \emph{Frequency-uniform decomposition
method for the generalized BO, KdV and NLS equations}, {J.
Differential Equations}, \textbf{239} (2007), 213--250.

\bibitem{BBR} P. Boggiatto, E. Buzano, L. Rodino \emph{Global
Hypoellipticity and Spectral Theory},  Mathematical Research, 92,
Akademie Verlag, Berlin, 1996.

\bibitem{CG1} E.~{C}ordero, K.~{G}r{\"o}chenig,
\emph{{T}ime-frequency analysis of localization operators},
 {J}. {F}unct. {A}nal., \textbf{205(1)} (2003), 107--131.

\bibitem{Czaja} W. Czaja, Z. Rzeszotnik
\emph{Pseudodifferential operators and Gabor frames: spectral
asymptotics}, {Math. Nachr.} \textbf{233-234} (2002), 77--88.

\bibitem{F1}  H.~G.~Feichtinger \emph{Modulation spaces on locally
compact abelian groups. Technical report}, {University of
Vienna}, Vienna, 1983; also in: M. Krishna, R. Radha,
S. Thangavelu (Eds) Wavelets and their applications, Allied
Publishers Private Limited, NewDehli Mumbai Kolkata Chennai Hagpur
Ahmedabad Bangalore Hyderbad Lucknow, 2003, pp. 99--140.

\bibitem{F2} \bysame \emph{Wiener amalgams over Euclidean spaces and some of their applications},
in: Function spaces (Edwardsville, IL, 1990), Lect. Notes in pure and
appl. math., 136, Marcel Dekker, New York, 1992, pp. 123–137.

\bibitem{Feichtinger3}  {H. G. Feichtinger and K. H. Gr{\"o}chenig}
\emph{Banach spaces related to integrable group representations and
their atomic decompositions, I}, J. Funct. Anal., \textbf{86}
(1989), 307--340.

\bibitem{Feichtinger4} \bysame \emph{Banach spaces related to
integrable group representations and their atomic decompositions, II},
Monatsh. Math., \textbf{108} (1989), 129--148.

\bibitem{Feichtinger5} \bysame \emph{Gabor frames and time-frequency
analysis of distributions}, {J. Functional
Anal.,} \textbf {146} (1997), 464--495.

\bibitem{Feichtinger6} \bysame \emph{Modulation spaces: Looking back and ahead},
Sampl. Theory Signal Image Process. \textbf{5} (2006), 109--140.

\bibitem{Fo}  {G. B. Folland} \emph
{Harmonic analysis in phase space}, {Princeton U. P., Princeton},
1989.

\bibitem{Grobner} P. Gr{\"o}bner \emph{Banachr{\"a}ume Glatter
Funktionen und Zerlegungsmethoden}, Thesis, University of Vienna,
Vienna, 1992.

\bibitem{Grochenig0a} {K. H. Gr{\"o}chenig} \emph {Describing
functions: atomic decompositions versus frames},
{Monatsh. Math.},\textbf{112} (1991), 1--42.

\bibitem{Gro-book} K. Gr\"{o}chenig, \newblock \textit{Foundations of
Time-Frequency Analysis},
\newblock Birkh\"auser, Boston, 2001.

\bibitem{Grochenig2} \bysame \emph{Composition and spectral invariance
of pseudodifferential operators on modulation spaces}, J. Anal.
Math., \textbf{98} (2006), 65--82.

\bibitem{Grochenig0}  {K. H. Gr{\"o}chenig and C. Heil} \emph
{Modulation spaces and pseudo-differential operators}, Integral
Equations Operator Theory, \textbf{34} (1999), 439--457.

\bibitem{Grochenig1b}  \bysame \emph {Modulation spaces as symbol
classes for pseudodifferential operators {\rm {in: M. Krishna,
R. Radha, S. Thangavelu (Eds) Wavelets and their applications}}},
Allied Publishers Private Limited, NewDehli Mumbai Kolkata Chennai
Hagpur Ahmedabad Bangalore Hyderbad Lucknow, 2003, pp. 151--170.

\bibitem{Grochenig1c} \bysame \emph{Counterexamples for boundedness of
pseudodifferential operators}, Osaka  J. Math., \textbf{41} (2004),
681--691.

\bibitem{Grochenig2a} K. Gr{\"o}chenig, M. Leinert \emph{Wiener's lemma
for twisted convolution and Gabor frames}, J. Amer. Math. Soc.,
\textbf{17} (2004), 1--18.

\bibitem{Herau1} F. H{\' e}rau \emph{Melin--H{\"o}rmander inequality
in a Wiener type pseudo-differential algebra}, Ark. Mat., \textbf{39}
(2001), 311--38.

\bibitem{HTW} A. Holst, J. Toft, P. Wahlberg \emph{Weyl product
algebras and modulation spaces}, {J. Funct. Anal.}, \textbf{251}
(2007), 463--491.

\bibitem{Ho1}  L. H{\"o}rmander \emph{The Analysis of Linear
Partial Differential Operators}, vol {I--III},
Springer-Verlag, Berlin Heidelberg NewYork Tokyo, 1983, 1985.

\bibitem{Hrm-nonlin} \bysame \emph{Lectures on Nonlinear Hyperbolic
Differential Equations}, Springer-Verlag, Berlin, 1997.

\bibitem{Okoudjou} K. Okoudjou \emph{Embeddings of some classical
Banach spaces into modulation spaces}, {Proc. Amer. Math. Soc.},
\textbf{132} (2004), 1639--1647.

\bibitem{Pilipovic2} {S. Pilipovi\'c, N. Teofanov} \emph{On a symbol
class of Elliptic Pseudodifferential Operators}, {Bull. Acad.
Serbe Sci. Arts}, \textbf {27} (2002), 57--68.

\bibitem{Pilipovic3} \bysame \emph{Pseudodifferential operators on
ultra-modulation spaces}, J. Funct. Anal., \textbf{208} (2004),
194--228.

\bibitem{PTT1} {S. Pilipovi\'c, N. Teofanov, J. Toft},
\emph{Micro-local analysis in Fourier Lebesgue and modulation
spaces. Part I}, preprint, in  arXiv:0804.1730, 2008.

\bibitem{PTT2} {S. Pilipovi\'c, N. Teofanov, J. Toft},
\emph{Micro-local analysis in Fourier Lebesgue and modulation
spaces. Part II}, preprint, in arXiv:0805.4476, 2008.

\bibitem{RSTT} M. Ruzhansky, m. Sugimoto, N. Tomita, J. Toft
\emph{Changes of variables in modulation and Wiener amalgam spaces},
Preprint, 2008, Available at arXiv:0803.3485v1.

\bibitem{Sjostrand1}  {J. Sj{\"o}strand} \emph{An algebra of
pseudodifferential operators}, {Math. Res. L.}, \textbf 1 (1994),
185--192.

\bibitem{Sjostrand2} \bysame \emph{Wiener type algebras of
pseudodifferential operators}, S\'eminaire Equations aux D\'eriv\'ees
Partielles, Ecole Polytechnique, 1994/1995, {Expos\'e n$^{\circ}$ IV.}

\bibitem{Sugimoto1} {M. Sugimoto, N. Tomita} \emph{The dilation
property of modulation spaces and their inclusion relation with Besov
Spaces}, {J. Funct. Anal. (1)}, \textbf{248} (2007),
79--106.

\bibitem{Tachizawa1}  {K. Tachizawa} \emph{The boundedness of
pseudo-differential operators on modulation spaces},
Math. Nachr., \textbf{168} (1994), 263--277.

\bibitem{Teofanov1}  {N. Teofanov} \emph{Ultramodulation spaces and
pseudodifferential operators}, {Endowment Andrejevi\'c}, Beograd,
2003.

\bibitem{Teofanov2} \bysame \emph{Modulation spaces, Gelfand-Shilov
spaces and pseudodifferential operators}, Sampl. Theory Signal
Image Process, \textbf{5} (2006), 225--242.

\bibitem{Toft2} J. Toft \emph{Continuity properties for
modulation spaces with applications to pseudo-differential calculus,
I}, {J. Funct. Anal.}, \textbf{207} (2004),
399--429.

\bibitem{Toft35} \bysame \emph{Convolution and embeddings for
weighted modulation spaces {\rm {in: P. Boggiatto, R. Ashino,
M. W. Wong (Eds)}} Advances in Pseudo-Differential Operators,}
Operator Theory: Advances and Applications \textbf{155},
Birkh{\"a}user Verlag, Basel 2004, pp. 165--186.

\bibitem{To8} \bysame \emph{Continuity
properties for modulation spaces with applications to
pseudo-differential calculus, II}, {Ann. Global Anal. Geom.},
\textbf{26} (2004), 73--106.

\bibitem{To9} \bysame \emph{Continuity and Schatten-von Neumann
Properties for Pseudo-Differential Operators and Toeplitz
operators on Modulation Spaces}, The Erwin Schr{\"o}dinger
International Institute for Mathematical Physics, Preprint ESI
\textbf{1732} (2005).

\bibitem{Toft4} \bysame \emph{Continuity and Schatten
properties for pseudo-differential operators on modulation spaces {\rm
{in: J. Toft, M. W. Wong, H. Zhu (Eds) Modern Trends in
Pseudo-Differential Operators,}}} Operator Theory: Advances and
Applications \textbf{172}, Birkh{\"a}user Verlag, Basel, 2007,
pp. 173--206.

\bibitem{Wong} M. W. Wong \newblock \textit{An Introduction To
Pseudodifferential Operators} 2nd Edition, World Scientific, 1999.
\end{thebibliography}

\end{document}

\begin{equation}\label{Somegadef}
|\partial ^\alpha _x\partial ^\beta _\xi a(x,\xi )|\le C_{\alpha
,\beta}\omega (x,\xi ){\langle \xi\rangle} ^{-\rho |\beta |}.
\end{equation}
Here ${\langle \xi\rangle} =(1+|\xi |^2)^{1/2}$. In the case $\omega (x,\xi
)={\langle \xi\rangle} ^r$, then $S_{\rho ,\delta}^{(\omega )}$ agrees with the
H{\"o}rmander class $S^r_{\rho ,0}$ in \cite {Ho1}.

\par

Obviously, any such pseudo-differential operator is continuous from
$\mathscr S({\mathbf R^{d}})$ to $\mathscr S'({\mathbf R^{d}})$. Furthermore, as
consequence of the general theory on pseudo-differential operators it
follows that if in addition $\rho \in [0,1]$, then ${\operatorname{Op}} (a)$ is
continuous on $\mathscr S({\mathbf R^{d}})$ and extends uniquely to a continuous
operator on $\mathscr S'({\mathbf R^{d}})$ (cf. Proposition 18.5.10 and Theorem
18.6.2 in \cite{Ho1}).

\par

As a consequence of Theorem 18.1.6 and its proof in \cite{Ho1}, it
follows that the first embedding in
\begin{equation}\label{intremb}
{\operatorname{WF}}({\operatorname{Op}} (a)f)\subseteq {\operatorname{WF}}(f)\subseteq {\operatorname{WF}}({\operatorname{Op}} (a)f){{\textstyle{\, \bigcup \, }}} {\operatorname{Char}} (a)
\end{equation}

\begin{proof}
We may assume that $\omega _1=\omega _2=1$ in view of Remark \ref{newbfspaces}.

Assume that $\xi_0 \in \Theta_{\mathscr F\! \mathscr B}(f)$, and let $\Gamma_1$
and $\Gamma_2$ be open cones in ${\mathbf R^{{d}}}$ such that
$\overline{\Gamma_2}\subseteq \Gamma_1$. Since $f$ has compact support, it follows that
$|\widehat{f}(\xi)| \leq C{\langle \xi\rangle}^{N_0}$ for some
positive constants $C$ and $N_0$. Therefore, the result follows by
proving that for each $N$, there are constants $C_N$ such that
\begin{equation}\label{cuttoff1}
|\phi f|_{{\mathscr F\! \mathscr B} _2(\Gamma _2)}
\leq
C_N (|f|_{{\mathscr F\! \mathscr B} _1(\Gamma_1)}
+ \sup_{\xi}
(|\widehat{f}(\xi)|{\langle \xi\rangle}^{-N}))
\end{equation}
when $\overline{\Gamma}_2 \subseteq\Gamma_1$ and $N = 1, 2,\dots$.

\par

First we consider the case $\mathscr B_1=\mathscr B_2=\mathscr B$.
By letting $F(\xi) = |
\widehat{f}(\xi)|$ and  $\psi(\xi) =
|\widehat{\varphi}(\xi)|$, it follows that $\psi$ turns
rapidly to zero at infinity and that
\begin{multline*}
|\varphi f|_{{\mathscr F\! \mathscr B} (\omega_0,\Gamma_2)}
=
\|\mathscr{F}(\varphi f) \omega_0\chi_{\Gamma_2}\|_{\mathscr{B}}
\\[1ex]
\leq
C \Big \| \Big ( \int_{{\mathbf R^{{d}}}}\psi({\, \cdot \, } - \eta)F(\eta )\, d\eta \Big )
\chi_{\Gamma_2} \Big \|_{\mathscr{B}}
\leq
C(J_1 + J_2)
\end{multline*}
for some positive constant $C$, where
$$
J_1
=
\Big \| \Big ( \int_{\Gamma_1}\psi({\, \cdot \, } - \eta)F(\eta )\, d\eta \ \Big
) \chi_{\Gamma_2}\Big \|_{\mathscr{B}}
$$
and
$$
J_2
=
\Big \| \Big (\int_{\complement \Gamma_1}\psi({\, \cdot \, } - \eta)F(\eta )\,
d\eta \Big )\chi_{\Gamma_2} \Big \|_{\mathscr{B}}
$$
and $\chi_{\Gamma_2}$ is the characteristic function of
$\Gamma_2$. First we estimate $J_1$. By (3) in Definition
\ref{BFspaces} and \eqref{propupps}, it follows that
\begin{multline*}
J_1\leq  C_1 \Big \| \int_{\Gamma_1}\psi(\cdot-\eta)F(\eta)\, d\eta
\Big \| _{\mathscr{B}}
\\[1ex]
=
C_1 {\Vert {\psi * (\chi_{\Gamma_1}F) }\Vert _{{\mathscr{B}}}}
\leq
C_2{\Vert {\psi\|_{L^1_{(v)}}\|\chi_{\Gamma_1}F }\Vert _{{\mathscr{B}}}}
=
C_{\psi}|f |_{{\mathscr F\! \mathscr B} (\omega_0,\Gamma_1)},
\end{multline*}
for some constants $C_1$ and $C_2$, where $C_{\psi}  = C_2{\Vert {\psi}\Vert _{{L^1_{(v)}}}}<\infty$, since $\psi$ turns rapidly to zero at
infinity.

\par

In order to estimate $J_2$, we note that the conditions $\xi \in
\Gamma _2$, $\eta \notin \Gamma _1$ and the fact that $\overline
{\Gamma _2}\subseteq \Gamma _1$ imply that  $|\xi -\eta |>c\max
(|\xi|,|\eta |)$ for some constant $c>0$, since this is true when
$1=|\xi |\ge |\eta|$. We also note that if $N_1$ is large enough, then
${\langle {\, \cdot \, }\rangle} ^{-N_1}\in \mathscr B$, because $\mathscr S$ is
continuously embedded in $\mathscr B$. Since $\psi$ turns rapidly to
zero at infinity, it follows that for each $N_0> d+N_1$ and $N\in
\mathbf N$ such that $N > N_0$, it holds
\begin{multline*}
J_2
\leq
C_1\Big \| \Big (\int_{\complement\Gamma _1}{\langle {{\, \cdot \, } -
\eta}\rangle}^{-(2N_0+N)} F(\eta)\, d\eta \Big )\chi_{\Gamma_2}\Big \|
_{\mathscr{B}}
\\[1ex]
\leq
C_2\Big \| \Big (\int_{\complement \Gamma_1} {\langle {\, \cdot \, }\rangle}^{-N_0} {\langle \eta\rangle} ^{-N_0}({\langle \eta\rangle}^{-N}F(\eta ))\, d\eta \Big
)\chi_{\Gamma_2}\Big \| _{\mathscr{B}}
\\[1ex]
\leq
C_2\int_{\complement \Gamma_1}\|{\langle {\, \cdot \, }\rangle}
^{-N_0}\chi_{\Gamma_2}\|_{\mathscr{B}} {\langle \eta\rangle} ^{-N_0}(|{\langle \eta\rangle}
^{-N}F(\eta )|)\, d\eta
\\[1ex]
\leq
 C \sup_{\eta \in {\mathbf R^{{d}}}}|{\langle \eta\rangle}^{-N}F(\eta )|,
\end{multline*}
for some constants $C_1$ and $C_2 > 0$, where $C=C_2{\Vert {{\langle {\, \cdot \, }\rangle}
^{-{N_0}}}\Vert _{{\mathscr B}}}{\Vert {{\langle {\, \cdot \, }\rangle} ^{-{N_0}}}\Vert _{{L^1}}}<\infty$. This
proves \eqref{chi-subsetAA}, and the result follows.
\end{proof}

\begin{prop}\label{theta-sigma-propertiesAA}
Assume that $\varphi \in \mathscr{S}({\mathbf R^{{d}}})$, $\omega \in
\mathscr{P}({\mathbf R^{{2d}}})$ and $f\in \mathscr{E}'({\mathbf R^{{d}}})$. Then
\begin{equation}\label{chi-subsetAA}
\Sigma_{{\mathscr F\! \mathscr B} (\omega)}(\varphi f)\subseteq \Sigma_{{\mathscr F\! \mathscr B} (\omega)}(f).
\end{equation}
\end{prop}

\par

\begin{proof}
Assume that $\xi_0 \in \Theta_{{\mathscr F\! \mathscr B} (\omega_0)}(f)$, and let $\Gamma_1$
and $\Gamma_2$ be open cones in ${\mathbf R^{{d}}}$ such that
$\overline{\Gamma_2}\subseteq \Gamma_1$. Also let $\omega_0(\xi) =
\omega(0,\xi)$. Since $f$ has compact support, it follows that
$|\widehat{f}(\xi)\omega_0(\xi)| \leq C{\langle \xi\rangle}^{N_0}$ for some
positive constants $C$ and $N_0$. Therefore, the result follows by
proving that for each $N$, there are constants $C_N$ such that
\begin{equation}\label{cuttoff1}
|\varphi f|_{{\mathscr F\! \mathscr B} (\omega_0,\Gamma_2)}
\leq
C_N (|f|_{{\mathscr F\! \mathscr B} (\omega_0,\Gamma_1)}
+ \sup_{\xi}
(|\widehat{f}(\xi)\omega_0(\xi)|{\langle \xi\rangle}^{-N}))
\end{equation}
when $\overline{\Gamma}_2 \subseteq\Gamma_1$ and $N = 1, 2,\cdots$.

\par

By using the fact that $\omega_0$ is $v_0$-moderate for some $v_0\in
\mathscr{P}({\mathbf R^{{d}}})$ and letting $F(\xi) = |
\widehat{f}(\xi)|\omega_0(\xi)$ and  $\psi(\xi) =
|\widehat{\varphi}(\xi)|v_0(\xi)$, it follows that $\psi$ turns
rapidly to zero at infinity and that
\begin{multline*}
|\varphi f|_{{\mathscr F\! \mathscr B} (\omega_0,\Gamma_2)}
=
\|\mathscr{F}(\varphi f) \omega_0\chi_{\Gamma_2}\|_{\mathscr{B}}
\\[1ex]
\leq
C \Big \| \Big ( \int_{{\mathbf R^{{d}}}}\psi({\, \cdot \, } - \eta)F(\eta )\, d\eta \Big )
\chi_{\Gamma_2} \Big \|_{\mathscr{B}}
\leq
C(J_1 + J_2)
\end{multline*}
for some positive constant $C$, where
$$
J_1
=
\Big \| \Big ( \int_{\Gamma_1}\psi({\, \cdot \, } - \eta)F(\eta )\, d\eta \ \Big
) \chi_{\Gamma_2}\Big \|_{\mathscr{B}}
$$
and
$$
J_2
=
\Big \| \Big (\int_{\complement \Gamma_1}\psi({\, \cdot \, } - \eta)F(\eta )\,
d\eta \Big )\chi_{\Gamma_2} \Big \|_{\mathscr{B}}
$$
and $\chi_{\Gamma_2}$ is the characteristic function of
$\Gamma_2$. First we estimate $J_1$. By (3) in Definition
\ref{BFspaces} and \eqref{propupps}, it follows that
\begin{multline*}
J_1\leq  C_1 \Big \| \int_{\Gamma_1}\psi(\cdot-\eta)F(\eta)\, d\eta
\Big \| _{\mathscr{B}}
\\[1ex]
=
C_1 {\Vert {\psi * (\chi_{\Gamma_1}F) }\Vert _{{\mathscr{B}}}}
\leq
C_2{\Vert {\psi\|_{L^1_{(v)}}\|\chi_{\Gamma_1}F }\Vert _{{\mathscr{B}}}}
=
C_{\psi}|f |_{{\mathscr F\! \mathscr B} (\omega_0,\Gamma_1)},
\end{multline*}
for some constants $C_1$ and $C_2$, where $C_{\psi}  = C_2{\Vert {\psi}\Vert _{{L^1_{(v)}}}}<\infty$, since $\psi$ turns rapidly to zero at
infinity.

\par

In order to estimate $J_2$, we note that the conditions $\xi \in
\Gamma _2$, $\eta \notin \Gamma _1$ and the fact that $\overline
{\Gamma _2}\subseteq \Gamma _1$ imply that  $|\xi -\eta |>c\max
(|\xi|,|\eta |)$ for some constant $c>0$, since this is true when
$1=|\xi |\ge |\eta|$. We also note that if $N_1$ is large enough, then
${\langle {\, \cdot \, }\rangle} ^{-N_1}\in \mathscr B$, because $\mathscr S$ is
continuously embedded in $\mathscr B$. Since $\psi$ turns rapidly to
zero at infinity, it follows that for each $N_0> d+N_1$ and $N\in
\mathbf N$ such that $N > N_0$, it holds
\begin{multline*}
J_2
\leq
C_1\Big \| \Big (\int_{\complement\Gamma _1}{\langle {{\, \cdot \, } -
\eta}\rangle}^{-(2N_0+N)} F(\eta)\, d\eta \Big )\chi_{\Gamma_2}\Big \|
_{\mathscr{B}}
\\[1ex]
\leq
C_2\Big \| \Big (\int_{\complement \Gamma_1} {\langle {\, \cdot \, }\rangle}^{-N_0} {\langle \eta\rangle} ^{-N_0}({\langle \eta\rangle}^{-N}F(\eta ))\, d\eta \Big
)\chi_{\Gamma_2}\Big \| _{\mathscr{B}}
\\[1ex]
\leq
C_2\int_{\complement \Gamma_1}\|{\langle {\, \cdot \, }\rangle}
^{-N_0}\chi_{\Gamma_2}\|_{\mathscr{B}} {\langle \eta\rangle} ^{-N_0}(|{\langle \eta\rangle}
^{-N}F(\eta )|)\, d\eta
\\[1ex]
\leq
 C \sup_{\eta \in {\mathbf R^{{d}}}}|{\langle \eta\rangle}^{-N}F(\eta )|,
\end{multline*}
for some constants $C_1$ and $C_2 > 0$, where $C=C_2{\Vert {{\langle {\, \cdot \, }\rangle}
^{-{N_0}}}\Vert _{{\mathscr B}}}{\Vert {{\langle {\, \cdot \, }\rangle} ^{-{N_0}}}\Vert _{{L^1}}}<\infty$. This
proves \eqref{chi-subsetAA}, and the result follows.
\end{proof}


\documentclass[final]{siamltex}

\usepackage{amssymb,amsmath}

        

\begin{document}

\title{Birth of the shape of a reachable set\thanks{This
        work was partially supported by the Russian Foundation for Basic Research (grants  11-08-00435, 13-08-00441).}}

\author{Elena Goncharova\thanks{Institute for System Dynamics and Control Theory, Siberian Branch of Russian Academy of Sciences,
        134, Lermontov st., 664033 Irkutsk, Russia
        ({\tt goncha@icc.ru}).}
        \and Alexander Ovseevich\thanks{Institute for Problems in Mechanics, Russian Academy of Sciences, 101/1, Vernadsky av.,
        119526 Moscow, Russia ({\tt ovseev@ipmnet.ru}).}}

\maketitle
\begin{abstract}
We study reachable sets of linear dynamical systems with geometric constraints
on control. The main result is the existence of a limit shape of the reachable
sets as the terminal time tends to zero. Here, a shape of a set stands for the
set regarded up to an invertible linear transformation.
\end{abstract}

\begin{keywords}
Linear dynamical system, reachable sets.
\end{keywords}

\begin{AMS}
93B03, 34H05
\end{AMS}

\pagestyle{myheadings} \thispagestyle{plain} \markboth{E. GONCHAROVA AND A. OVSEEVICH}{BIRTH OF THE
SHAPE}

\noindent
\section{Problem statement}
Consider the following linear 
control system:
\begin{equation}\label{syst}
\dot x=Ax+Bu,\; x\in \mathbb{V}={{\mathbf R}}^n,\; u\in U\subset \mathbb{U}={{\mathbf R}}^m,
\end{equation}
where $U$ is a central symmetric convex  body in ${{\mathbf R}}^m$. \iffalse, i.e., $U$ is
a convex compact set with the non-empty interior, and $U=-U$. \fi Given $T>0$,
we get the reachable set ${\cal D}(T)$ being the set of the ends $\{x(T)\}$ at
time $T$ of all admissible trajectories of system {(\ref{{syst}})} such that
$x(0)=0.$ We study the asymptotic behavior of the reachable sets ${\cal D}(T)$
as $T\to 0$. The problem is not as trivial as it may seem, since the obvious
answer ${\cal D}(T)\to\{0\}$ is not good enough. Of course, ${\cal D}(T)$ is
small with a small $T$. Still we can look at it through a kind of microscope
and see fine details of its development, in particular, its shape, where a
shape of a set is the set regarded up to an invertible linear transformation.
For instance, consider a rather trivial system {(\ref{{syst}})}, where $A=0$,
$B=1$, and $U$ is the unit ball $B_1$ in ${{\mathbf R}}^n={{\mathbf R}}^m$. Then ${\cal D}(T)=TB_1$,
and the reachable set has the shape of a ball at any time $T>0$.
\iffalse , where the shape ${\mathop{\rm Sh}\nolimits}\Omega$ of any subset
$\Omega\subset{{\mathbf R}}^n$ is by definition the orbit of $\Omega$ under natural action
of the general linear group ${\mathop{\rm GL}\nolimits}_n({{\mathbf R}})$. \fi In our example,  for all $T>0$,
the sets ${\cal D}(T)$ have the same shape, so that
there exists a limit of the shapes as $T\rightarrow  0$, and this limit is not zero, but the shape of a ball. 

In this paper we will show that, for any completely controllable linear system, shapes of the
reachable sets converge as time tends to zero. We will give an estimate for the rate of convergence.
The paper can be regarded as the $T \to 0$ counterpart of the results \cite{ovs} on the large-time
dynamics of the reachable sets, where the existence of a limit shape
has been established.

\section{Shapes of reachable sets}\label{shapes section}

\iffalse Assume for simplicity that the set $U$ of admissible controls is a central symmetric
(compact) convex body in ${{\mathbf R}}^m$ with the zero center. Then, the controllability condition ensures
that the reachable sets  ${\cal D} (T)$ are central symmetric convex bodies in ${{\mathbf R}}^n.$ \fi

There is a well developed mathematical concept of shapes, see \cite{ovs}.
Consider the metric space $\mathbb{B}$ of central symmetric convex bodies with
the Banach-Mazur distance $\rho $:
\begin{eqnarray*}\label{B-M}
\rho (\Omega _1,\Omega _2)=\log (t(\Omega _1,\Omega _2)t(\Omega _2,\Omega _1)), \; t(\Omega_1 ,
\Omega_2 ) = \inf \{t \geq 1: t\Omega_1\supset \Omega_2 \}.
\end{eqnarray*}
The general linear group ${\mathop{\rm GL}\nolimits}(\mathbb{V})$ naturally acts on the space
$\mathbb{B}$ by isometries. The factorspace $\mathbb{S}$ is called the space of
shapes of central symmetric convex bodies, where the shape ${\mathop{\rm Sh}\nolimits}\Omega \in
\mathbb{S}$ of a convex body $\Omega \in \mathbb{B}$ is the orbit ${\mathop{\rm Sh}\nolimits}\Omega =
\{g \Omega : {\rm det}\, g \neq 0\}$ of the point $\Omega$ with respect to the
action of ${\mathop{\rm GL}\nolimits}(\mathbb{V})$. The Banach-Mazur factormetric $$\rho ({\mathop{\rm Sh}\nolimits}\Omega
_1,{\mathop{\rm Sh}\nolimits}\Omega _2)=\inf_{g\in GL(\mathbb{V})}\rho (g\Omega _1,\Omega _2)$$ makes
$\mathbb{S}$ into a compact metric space.   \iffalse The convergence of convex
bodies may be also understood in the sense of convergence of their support
functions. Remind that the support function of a convex compact set $\Omega$ is
given by formula: $H_\Omega(\xi)=\sup_{x\in\Omega}\langle x,\xi\rangle$, where
$\xi\in{\mathbb V}^*$, and uniquely defines the set $\Omega$. The equivalence
of the two definitions of convergence of convex bodies \iffalse --- in the
terms of convergence of their support functions and in the sense of the
Banach-Mazur metric --- \fi is established by the following easy lemma
\cite{fig-periodic}:
\begin{lemma} \label{fig lemma}
A sequence ${\Omega}_i \in {\mathbb B}$ converges to $\Omega \in {\mathbb B}$ in the sense of the
Banach-Mazur metric if and only if the corresponding sequence of the support functions
$H_i(\xi)=H_{\Omega _i}(\xi)$ converges to the support function $H_\Omega(\xi) $ pointwise and is
uniformly bounded on the unit sphere in the dual space ${\mathbb V}^{*}$.
\end{lemma}
\fi If $\rho (\Omega _1(T),\Omega _2(T)) \to 0$ as $T\to 0$, 
the convex bodies $\Omega_1$ and $\Omega_2$ are asymptotically equal, and we
write $\Omega _1(T) \sim \Omega _2(T)$. The asymptotic equivalence of shapes is
defined similarly.

\section{Main result: Autonomous case}

We assume that system {(\ref{{syst}})} is time-invariant and the Kalman
controllability condition holds. The Kalman condition ensures that the
reachable sets ${\cal D} (T)$ to system {(\ref{{syst}})} are central symmetric
convex bodies in ${{\mathbf R}}^n$. In what follows, the convergence of 
shapes of reachable sets is understood in the sense of the Banach-Mazur metric.
\begin{theorem}\label{thmain} \iffalse Assume that {(\ref{{syst}})} is an autonomous system such that $U$ is
central symmetric, and the Kalman controllability holds.  Then, \fi The shapes ${\mathop{\rm Sh}\nolimits}{\cal D} (T)$
have a limit ${\mathop{\rm Sh}\nolimits}_0 $ as $T\to 0$. The Banach--Mazur distance $\rho({\mathop{\rm Sh}\nolimits}{\cal D} (T),{\mathop{\rm Sh}\nolimits}_0 )$ is
$O(T)$.
\end{theorem}

\noindent This means that there exists a time independent convex body $\Omega$ such that 
$${\cal D} (T)\sim C(T)\Omega,$$ where $C(T)$ is a matrix function. The Banach-Mazur distance between the left- and
the right-hand sides of the latter formula is $O(T)$.

The proof of Theorem~\ref{thmain} is based on two easy lemmas, and the use of
the Brunovsky normal form of a controllable system.
\begin{lemma} \label{linear_feedback}
Consider the linear system
\begin{equation}\label{syst2}
\dot x={\widetilde A}x+{\widetilde B}u, 
\; u\in U,
\end{equation}
obtained from system {\rm {(\ref{{syst}})}} by adding a linear feedback, that is, ${\widetilde A}=A+BC$
and ${\widetilde B}=B$. Then the Banach--Mazur distance $\rho({\cal D} (T),\widetilde{\cal D} (T))$
is $O({T})$ as $T\to0$, where ${\cal D} (T)$ and $\widetilde{\cal D} (T)$ are the reachable sets to
systems {\rm {(\ref{{syst}})}} and {\rm {(\ref{{syst2}})}}, respectively.
\end{lemma}

\begin{lemma} \label{gauge}
Consider the linear system
\begin{equation}\label{syst22}
\dot x={\widetilde A}x+{\widetilde B}u, 
\; u\in U, 
\end{equation}
obtained from system {\rm {(\ref{{syst}})}} by a gauge transformation, where ${\widetilde A}=C^{-1}AC$,
${\widetilde B}=C^{-1}B$, and $C$ is an invertible matrix.  Then ${\mathop{\rm Sh}\nolimits}\widetilde{\cal D} (T)={\mathop{\rm Sh}\nolimits}{\cal
D} (T)$, where $\widetilde{\cal D} (T)$ is the reachable set to system {\rm {(\ref{{syst22}})}}.
\end{lemma}

\iffalse Lemma~\ref{gauge} is obvious. We postpone proving the Lemma
\ref{linear_feedback} for a moment, and \fi

From the Lemmas it follows that applying gauge transformations coupled with
adding a linear feedback do not affect the validity of Theorem~\ref{thmain}.
However, by means of these transformations, a general system {(\ref{{syst}})} can
be reduced to the Brunovsky normal form~\cite{Brun}. \iffalse, where the
matrices $A$ and $B$ are the direct sums $A=\oplus A_i$, $B=\oplus B_i$, and
the matrices $A_i$ and $B_i$ of sizes $n_i\times n_i$ and $n_i\times 1$,
respectively, take the form
\begin{equation*}\label{syst3}
A_i=\left(\begin{array}{cccc}
  0 & 1 &  &  \\
   & 0 & \ddots &  \\
   &  & \ddots & 1 \\
   &  &  & 0 \\
\end{array}\right), \quad
B_i=\left(\begin{array}{c}
  0 \\
  0 \\
  \vdots \\
  1 \\
\end{array}\right).
\end{equation*}
\fi One can show that shapes ${\mathop{\rm Sh}\nolimits}{\cal D} (T)$ of reachable sets of the
Brunovsky system do not depend on $T$.

The Brunovsky classification is known to be closely related to the Grothendieck
theorem \cite{Groth} on decomposition of vector bundles on ${\mathbf P}^1$ into
a sum  of linear bundles.

\iffalse

We can relate to the Brunovsky system {(\ref{{syst}})}, {(\ref{{syst3}})} a distinguished   matrix function
$\delta=\oplus \,\delta_i$, where
\begin{equation}\label{delta}\delta_i(T)={\mathop{\rm diag}\nolimits}(T^{-n_i},T^{-n_i+1},\dots,T^{-1})\end{equation}
such that
\begin{equation}\label{prop}
\delta A\delta^{-1}=T^{-1}A,\quad \delta B=T^{-1}B.
\end{equation}
This immediately implies that for $T$ fixed and  $y=\delta x$, we have
\begin{equation}\label{y}
\dot y=T^{-1}\left(Ay+Bu\right).
\end{equation}
Equation {(\ref{{y}})} reveals the geometric meaning of the matrix $\delta(T)$: The
corresponding gauge transformation is equivalent to the passage to the new time
scale  $t\mapsto t/T$ in  {(\ref{{syst}})}, {(\ref{{syst3}})}. Since the gauge
transformations do not change shapes of the reachable sets,
we conclude that the shapes ${\mathop{\rm Sh}\nolimits}{\cal D} (T)$ of the reachable sets to the
Brunovsky system do not depend on $T$.

It remains to prove Lemma~\ref{linear_feedback}. Consider a trajectory $t\mapsto x(t)$ of system
{(\ref{{syst}})}, and the corresponding trajectory $\widetilde x(t)$ of {(\ref{{syst2}})}. We have
\begin{eqnarray}&&\dot x(t)=A x(t)+Bu(t),\nonumber\\
&&\dot{\widetilde x}(t)=A{\widetilde x}(t)+B(u(t)+C{\widetilde x}(t))\nonumber\end{eqnarray} It is
clear that $C{\widetilde x}(t)=O(t)$, and therefore for all $t\leq T$ the control vector $\widetilde
u(t)=u(t)+C{\widetilde x}(t)$  belongs to the set $(1+\varepsilon)U$, where $\varepsilon=O(T)$. This
means that $\widetilde{\cal D} (T)\subset (1+\varepsilon){\cal D} (T)$, where $\varepsilon=O(T)$.
Since the relation between systems {(\ref{{syst}})} and {(\ref{{syst2}})} is symmetric, we similarly have
that ${\cal D} (T)\subset (1+\varepsilon)\widetilde{\cal D} (T)$. But this implies
Lemma~\ref{linear_feedback} in view of the definition of the Banach--Mazur distance~{(\ref{{B-M}})}.

\fi

\section{Non-autonomous case} \iffalse In fact, the same phenomenon of the existence of a limit shape takes place
in the non-autonomous case. We should just assume a kind of genericity
condition generalizing the Kalman one we operated with in the time-invariant
case. \fi

Consider system {(\ref{{syst}})}, where now the data $A$, $B$, and $U$ are
$C^\infty$-functions of  time $t\geq0$. By a standard trick we make
{(\ref{{syst}})} into the time-invariant system
\begin{eqnarray*}\label{timeinv}
&&\dot\tau=1,\\ &&\dot x={ A(\tau)}x+{ B(\tau)}u.
\end{eqnarray*}
Consider the Lie algebra ${\mathcal L}$ generated by the vector fields $\left(
1,
  { A(\tau)}x \right)$ and $\left( 0,
  { B(\tau)}u \right)$ in ${{\mathbf R}}\times\mathbb{V}={{\mathbf R}}^{n+1}$, where $u\in{{\mathbf R}}^m$ is a constant vector. Define
  ${\mathcal L}(\tau,x)$ as the set of the values at $(\tau,x)$ of all vector fields from ${\mathcal L}$.

Assume that the following Kalman type condition holds:
\begin{quote}
For each $(\tau,x)\in{{\mathbf R}}\times\mathbb{V}$ the set ${\mathcal L}(\tau,x)$ coincides with the entire
tangent space ${{\mathbf R}}^{n+1}$. In other words,
\begin{equation}\label{ext_kalman}
\dim{\mathcal L}(\tau,x)=n+1.
\end{equation}
\end{quote}
\iffalse In the time-invariant case this assumption coincides with the Kalman
controllability condition. \fi

\begin{theorem}\label{thmain2} \iffalse Assume that {\rm {(\ref{{syst}})}} is a non-autonomous system such that $U$ is
central symmetric, and the genericity condition {\rm {(\ref{{ext_kalman}})}} holds. Then, the \fi Let
$\mathcal{D}(T)$ be the reachable set to a non-autonomous system of the form {\rm {(\ref{{syst}})}} and
the genericity condition {\rm {(\ref{{ext_kalman}})}} hold. The shapes ${\mathop{\rm Sh}\nolimits}{\cal D} (T)$ have a limit
${\mathop{\rm Sh}\nolimits}_0 $ as $T\to0$. Moreover, the Banach--Mazur distance $\rho({\mathop{\rm Sh}\nolimits}{\cal D} (T),{\mathop{\rm Sh}\nolimits}_0 )$ is
$O({T}).$
\end{theorem}

The proof is based on a reduction to the case $A=0$ and a subsequent study of
the filtration $F_k^*=\{\xi \in \mathbb{V}^*:{ B(t)}^*\xi=O(t^k)\}$ of the dual
space $\mathbb{V}^*$.

\iffalse

\begin{proof}
One can easily reduce system {(\ref{{syst}})} to the case $A=0$. Indeed, make a gauge transformation $x=Cy,$ where $\dot
C=AC,\,C(0)=1.$ The Cauchy problem is solvable. Then, $\dot y=C^{-1}Bu,$ and the shapes of the reachable sets to the new
system are the same as those of the old one.

For the new system $\dot y={\widetilde B}u$ the condition {(\ref{{ext_kalman}})} takes the form:
\begin{quote}
For any (constant) vector  $\xi\in\mathbb{V}^*={{\mathbf R}}^n$ the function  $t\mapsto{\widetilde B}^*\xi$ is
not flat at any time instant $\tau$: a higher derivative does not vanish at $\tau$.
\end{quote}

We associate a flag in $\mathbb{V}^*$ to the matrix function ${\widetilde B}$. With an integer
$k\geq0$ we associate the set $F_k^*=\{\xi:{\widetilde B}^*(t)\xi=O(t^k)\}$ of vectors
$\xi\in\mathbb{V}^*={{\mathbf R}}^n$. It is obvious that $F_k^*$ form a decreasing sequence of subspaces of
$\mathbb{V}^*$ such that $F_0^*=\mathbb{V}^*$ and $F_\infty^*=0.$ The latter equality is a
restatement of the nonflatness condition {(\ref{{ext_kalman}})}.

Consider the graded space ${\mathop{\rm Gr}\nolimits}\mathbb{V}^*=\oplus_{k=0}^\infty F_{k}^*/F_{k+1}^*$, and choose an
isomorphism $\phi:\mathbb{V}^*\simeq{\mathop{\rm Gr}\nolimits}\mathbb{V}^*$ such that for any $j$ the subspace $F_j^*$ maps
to $\oplus_{k=j}^{\infty} F_{k}^*/F_{k+1}^*$ in such a way that the induced map $F_j^*/F_{j+1}^*\to
F_j^*/F_{j+1}^*$ is identical. In other words, for any $\xi\in \mathbb{V}^*$  we have a unique
representation of the form $\xi=\sum_{i\in I} \xi_i,$ where $\xi_i
$ belongs to the subspace $V_{k(i)}=\phi^{-1}\left(F_{k(i)}^*/F_{k(i)+1}^*\right)$ in $F_{k(i)}^*$.
Here, the set $I$ of indices can be identified with the set $\{k(i): i\in I\}$ of jumps in the
filtration $F^*$, i.e. the values of $k$ such that $F_{k}^*\neq F_{k+1}^*$.

We define a generalization $\Delta^*(T)$ of the transposed matrix $\delta(T)$ from {(\ref{{delta}})} as
follows:
$$\Delta^*(T)\xi=\sum_{\in I}T^{-k(i)-1}\xi_i.$$ In other words, $\Delta^*(T)$ is equal to $T^{-k(i)-1}$ on $V_{k(i)}$. If
$\xi_i\neq0$, then $\xi_i\in F_{k(i)}^*\setminus F_{k(i)+1}^*$, and we have that ${\widetilde
B}^*(t)\xi_i=t^{k(i)}\eta_i(t),$ where $\eta_i(t)$ is $C^\infty$-smooth in $t$, and
$\eta_i(0)\neq0.$ Thus, we have the $C^\infty$-smooth matrix function ${\widetilde
B}_i^*(t)\xi=\eta_i(t),$ and the decomposition ${\widetilde B}^*(t)=\sum t^{k(i)}{\widetilde
B}_i^*(t)$. Therefore, $${\widetilde B}^*(t)\Delta^*(T)=\frac1T\sum
\left(\frac{t}{T}\right)^{k(i)}{\widetilde B}_i^*(t).$$ Since each ${\widetilde
B}_i^*(t)={\widetilde B}_i^*(0)+O(t),$ the left-hand side coincides with $\frac1T{\widetilde
B}^*\left(\frac{t}{T}\right)+O\left(\frac{t}{T}\right)$ so that
\begin{equation}\label{deltas}
{\widetilde B}^*(t)\Delta^*(T)=\frac1T{\widetilde
B}^*\left(\frac{t}{T}\right)+O\left(\frac{t}{T}\right).
\end{equation}
The support function of the reachable set ${\mathcal D}(T)$ has the form $\displaystyle H_{{\mathcal
D}(T)}(\xi)=\int_0^T h_t({\widetilde B}^*(t)\xi),$ where $h_t$ is the support function of the set
$U_t$ of controls at time $t$. Thus, the support function of the normalized reachable set
$\widetilde{\mathcal D}(T)=\Delta(T){\mathcal D}(T)$ is given by
$$H_{\widetilde{\mathcal D}(T)}(\xi)=\int_0^T h_t({\widetilde B}^*(t)\Delta^*(T)\xi)dt,$$ where $\Delta(T)$ is by definition
the adjoint of the already introduced operator $\Delta^*(T)$. Because of {(\ref{{deltas}})} the integral
can be rewritten as
$$H_{{\mathcal
D}(T)}(\xi)=\int_0^1 h_{\tau T}({\widetilde B}^*(\tau)\xi+O(\tau T))d\tau,$$ where $\tau=t/T$. The
latter integral equals $\displaystyle \int_0^1 h_{0}({\widetilde B}^*(\tau)\xi))d\tau+O(T)$, and
surely converges as $T\to 0$ to $\displaystyle\int_0^1 h_{0}({\widetilde B}^*(\tau)\xi))d\tau$. If
$\xi\neq0$ the function $\tau\mapsto{\widetilde B}^*(\tau)\xi$ does not vanish identically in any
open interval. Thus, the latter integral is positive, what means that it defines the support
function of a convex body $\Omega$,
$$H_\Omega(\xi)=\int_0^1 h_{0}({\widetilde B}^*(\tau)\xi))d\tau,$$
 and we conclude by invoking Lemma \ref{fig lemma} that the shapes ${\mathop{\rm Sh}\nolimits} {\mathcal D}(T)={\mathop{\rm Sh}\nolimits}
\widetilde{\mathcal D}(T)$ tend to ${\mathop{\rm Sh}\nolimits}_0={\mathop{\rm Sh}\nolimits}\Omega$ as $T\to 0$.
\end{proof}
\fi

\begin{thebibliography}{9}

\bibitem{ovs}  {\it Ovseevich~A.I.} Asymptotic behavior of attainable and
superattainable sets // Proceedings of the Conference on Modeling, Estimation
and Filtering of Systems with Uncertainty, Sopron, Hungary, 1990,
Birkha\"{u}ser, Basel, Switzerland. 1991. P. 324--333.

\bibitem{fig-periodic} {\it Figurina T.Yu., Ovseevich A.I.}  Asymptotic behavior
of attainable sets of linear periodic control systems
// J. Optim. Theory Appl. 1999. Vol. 100. No 2. P. 349--364.

\bibitem{Brun} {\it Brunovsky~P.} A classification of linear controllable systems. Kibernetika 6 (1970), 176-188.

\bibitem{Groth} {\it Grothendieck A.} Sur la classification des fibr\'es holomorphes sur la sph\'ere de Riemann,  Amer. J. Math., 79, 121-138 (1957)

\end{thebibliography}

\end{document}


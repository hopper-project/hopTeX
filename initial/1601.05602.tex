
\documentclass[11pt,oneside,english]{amsart}
\usepackage[T1]{fontenc}
\usepackage[latin9]{inputenc}
\usepackage{geometry}
\geometry{verbose,tmargin=3cm,bmargin=3cm,lmargin=2.5cm,rmargin=2.5cm,footskip=1cm}
\usepackage{babel}
\usepackage{amsthm}
\usepackage{amsbsy}
\usepackage{amssymb}
\usepackage{graphicx}
\usepackage{graphics}
\usepackage[unicode=true,pdfusetitle,
 bookmarks=true,bookmarksnumbered=false,bookmarksopen=false,
 breaklinks=false,pdfborder={0 0 1},backref=false,colorlinks=false]
 {hyperref}

\makeatletter

\providecommand{\tabularnewline}{\\}

\numberwithin{equation}{section}
\numberwithin{figure}{section}
 \theoremstyle{definition}
 \newtheorem{defn}{\protect\definitionname}
\theoremstyle{plain}
\newtheorem{thm}{\protect\theoremname}
\newtheorem{prop}[thm]{Proposition}
  \theoremstyle{plain}
  \newtheorem{lem}[thm]{\protect\lemmaname}
  \theoremstyle{remark}
  \newtheorem*{rem*}{\protect\remarkname}
  \newtheorem{qn}{Question}
  \theoremstyle{plain}
  \newtheorem{cor}[thm]{\protect\corollaryname}

\makeatother

  \providecommand{\corollaryname}{Corollary}
  \providecommand{\definitionname}{Definition}
  \providecommand{\lemmaname}{Lemma}
  \providecommand{\remarkname}{Remark}
\providecommand{\theoremname}{Theorem}

\begin{document}

\title{Algebraic torsion for contact manifolds with convex boundary}
\author{Andr\'as Juh\'asz}\address{Mathematical Institute, University of Oxford, Andrew Wiles Building,
Radcliffe Observatory Quarter, Woodstock Road, Oxford, OX2 6GG, UK}\email{juhasza@maths.ox.ac.uk}\thanks{AJ was supported by a Royal Society Research Fellowship.}

\author{Sungkyung Kang}\address{Mathematical Institute, University of Oxford, Andrew Wiles Building,
Radcliffe Observatory Quarter, Woodstock Road, Oxford, OX2 6GG, UK}\email{sungkyung.kang@maths.ox.ac.uk}
\subjclass[2010]{57M27; 57R58}\keywords{Contact structure; Algebraic torsion; Heegaard Floer homology}

\date{\today}
\begin{abstract}
We extend the Heegaard Floer homological definition of algebraic torsion (${\mathit{AT}}$) for closed contact 3-manifolds
due to Kutluhan et al.~to contact 3-manifolds with
convex boundary. We show that the ${\mathit{AT}}$ of a codimension zero contact submanifold
bounds the ${\mathit{AT}}$ of the ambient manifold from above.
As the neighborhood of an overtwisted disk has algebraic torsion zero, we obtain
that overtwisted contact structures have vanishing ${\mathit{AT}}$.
We also prove that the ${\mathit{AT}}$ of a small perturbation of a
$2\pi$ Giroux torsion domain has ${\mathit{AT}}$ at most one,
hence any contact structure with positive Giroux torsion has ${\mathit{AT}}$ at most one
(and, in particular, a vanishing contact invariant).
\end{abstract}

\maketitle

\section{Introduction}

Algebraic torsion of closed contact $(2n-1)$-manifolds was defined by
Latschev and Wendl~\cite{key-11} via symplectic field theory.
It is an invariant with values in ${\mathbb{N}} \cup \{\infty\}$ whose finiteness
gives obstructions to the existence of symplectic fillings and exact symplectic cobordisms.
They also showed that the order of algebraic torsion is zero if and only if the
contact homology is trivial -- in particular, if the contact structure is overtwisted --
and it has order at most one in the presence of positive Giroux torsion.
Note that the analytical foundations of symplectic field theory are still under
development. Hence, in the appendix, Hutchings provided a similar numerical invariant
for contact 3-manifolds via embedded contact homology, however, it is currently unknown
whether this is independent of the contact form.

Using the isomorphism between ${\mathit{ECH}}$ and Heegaard Floer homology, Kutluhan et al.~\cite{key-10}
defined a Heegaard Floer homological analogue of algebraic torsion for closed contact 3-manifolds.
Their definition uses open book decompositions, and gives a refinement of the
Ozsv\'ath-Szab\'o contact invariant~$c(\xi)$. Using the fact that an overtwisted contact structure
is supported by a non right-veering monodromy, they proved that ${\mathit{AT}}(M,\xi) = 0$ if
$\xi$ is overtwisted.

In this paper, we extend ${\mathit{AT}}$ to contact manifolds with convex boundary,
following the definition of Kutluhan et al.~in the closed case.
The definition is in terms of a partial open book decomposition of
the underlying sutured manifold supporting the contact structure,
and an arc basis of the page. This data gives rise to a filtration of
the sutured Floer boundary map, and the algebraic torsion is the first
page of the associated spectral sequence where the distinguished generator
representing the contact invariant vanishes, or $\infty$ otherwise. Then
we take the minimum over all arc bases and partial open books.

Our first main result is that the algebraic torsion of a codimension zero contact submanifold
gives an upper bound on the algebraic torsion of the ambient manifold.

\begin{thm} \label{thm:ineq}
Let $(M,\xi)$ be a contact $3$-manifold with convex boundary,
and $(N,\xi|_N)$ is a codimension zero submanifold of~$\text{Int}(M)$ with convex boundary.
Then
\[
\mbox{AT}(N,\xi|_N) \ge {\mathit{AT}}(M,\xi).
\]
\end{thm}

We will prove this result in Section~\ref{sec:ineq}.
As a corollary, we prove that if a contact manifold with convex boundary
is overtwisted, then it has algebraic torsion zero. This follows
immediately from a simple computation that a neighborhood of an overtwisted
disk has algebraic torsion zero.

In Section~\ref{sec:gt}, we carry out a computation that shows that the algebraic torsion
of a slight enlargement of a Giroux $2\pi$-torsion $T^2 \times I$
has algebraic torsion at most one. In particular, every contact manifold
with positive Giroux torsion has vanishing Ozsv\'ath-Szab\'o invariant,
which was proved by Ghiggini et al.~\cite{key-5}, modulo the issue of
defining canonical orientation systems in sutured Floer homology.
Together with Theorem~\ref{thm:ineq}, we obtain the following corollary.

\begin{thm} \label{thm:gt}
If a contact $3$-manifold $(M,\xi)$ with convex boundary has Giroux $2\pi$-torsion,
then
\[
{\mathit{AT}}(M,\xi) \le 1.
\]
\end{thm}

This was shown in the closed case by Latschev and Wendl~\cite[Theorem~2]{key-11}
via symplectic field theory, and conjectured in the Heegaard Floer setting
in the closed case by Kutluhan et al.~\cite[Question~3.2]{key-10}.
More generally, they asked whether the presence of planar $k$-torsion (see~\cite[Section~3.1]{key-11}
for a definition) implies that the order of the algebraic torsion is at most~$k$.

\subsection*{Acknowledgement} We would like to thank Andy Wand for helpful discussions,
and Paolo Ghiggini and Chris Wendl for comments on earlier versions of this paper.

\section{Algebraic torsion for manifolds with boundary}

We first recall the Heegaard Floer homological definition of algebraic torsion for closed contact
$3$-manifolds due to Kutluhan et al.~\cite{key-10}. Let $(M,\xi)$ be a closed contact 3-manifold.
By the Giroux correspondence theorem \cite{key-9}, the contact structure $\xi$ is supported by
some open book decomposition of $M$, which is well-defined up to positive stabilizations.
Choosing any compatible open book $(S,\phi)$ and an arc basis $\underline{\mathbf{a}}$
on $S$ gives an induced Heegaard diagram $(\Sigma,{\boldsymbol{\alpha}},{\boldsymbol{\beta}})$
of $M$. Here, an arc basis is a set ${\underline{\mathbf{a}}}$ of
arcs on $S$ with endpoints on $\partial S$ such that it forms a basis of $H_1(S,\partial S)$.
We obtain~${\underline{\mathbf{b}}}$ by isotoping ${\underline{\mathbf{a}}}$ such that the endpoints of ${\underline{\mathbf{a}}}$ are moved in the positive
direction along $\partial S$ and each component of ${\underline{\mathbf{a}}}$ intersects the corresponding component
of ${\underline{\mathbf{b}}}$ positively in a single point.
Then we set $\Sigma = (S \times \{1/2\}) \cup_{\partial S} (-S \times \{0\})$ and let
\[
\begin{split}
{\boldsymbol{\alpha}} := \{(a \times \{1/2\}) \cup (a \times \{0\}) \,\colon\, a \in \underline{\mathbf{a}}\}, \\
{\boldsymbol{\beta}} := \{(b \times \{1/2\}) \cup (\phi(b) \times \{0\}) \,\colon\, b \in \underline{\mathbf{a}}\}.
\end{split}
\]

We say that a domain $D$ in the diagram $({\Sigma},{\boldsymbol{\alpha}},{\boldsymbol{\beta}})$ connects $\mathbf{x}$, $\mathbf{y} \in {\mathbb{T}}_{\alpha} \cap {\mathbb{T}}_{\beta}$
if $\partial(\partial D\cap{\boldsymbol{\alpha}})=\mathbf{x}-\mathbf{y}$
and $\partial(\partial D \cap {\boldsymbol{\beta}}) = {\mathbf{y}} - {\mathbf{x}}$, and we denote by $D({\mathbf{x}},{\mathbf{y}})$ the set of such domains.
Using this Heegaard diagram, Kutluhan at al.~\cite{key-10} define a function~$J_+$,
which assigns an integer to every domain $D \in D({\mathbf{x}},{\mathbf{y}})$, as follows:
\[
J_{+}(D)=n_{\mathbf{x}}(D)+n_{\mathbf{y}}(D)-e(D)+|\mathbf{x}|-|\mathbf{y}|.
\]
Here, $n_{\mathbf{x}}(D)$ is the sum over all $p\in\mathbf{x}$ of the averages of the coefficients
of $D$ at the four regions around $p$, the term
$e(D)$ is the Euler measure of $D$, and $|{\mathbf{x}}|$, $|{\mathbf{y}}|$ are the number
of cycles in the elements of $S_n$ associated with ${\mathbf{x}}$ and ${\mathbf{y}}$, respectively.
When $D$ is a domain of Maslov index $1$, the equality
$e(D)=1-n_{\mathbf{x}}(D)-n_{\mathbf{y}}(D)$ holds by the work of Lipshitz~\cite{key-6},
so the formula becomes
\[
J_{+}(D)=2(n_{\mathbf{x}}(D)+n_{\mathbf{y}}(D))-1+|\mathbf{x}|-|\mathbf{y}|.
\]
 For any topological Whitney disk $C \in \pi_2({\mathbf{x}},{\mathbf{y}})$, we can define $J_{+}(C)$
as the value $J_{+}({\mathcal{D}}(C))$, where ${\mathcal{D}}(C)$ is the domain of $C$.
The function $J_{+}$ is additive in the sense that
\[
J_{+}(D_{1} + D_{2}) = J_{+}(D_{1})+J_{+}(D_{2})
\]
for every $D_1 \in {\mathcal{D}}({\mathbf{x}},{\mathbf{y}})$ and $D_2 \in {\mathcal{D}}({\mathbf{y}},{\mathbf{z}})$.
Also, $J_{+}(C)$ is always a nonnegative even integer for any J-holomorphic
disk $C$.
Hence, we have a splitting
\[
\widehat{\partial}=\partial_{0}+\partial_{1}+\partial_{2}+\cdots
\]
of the Floer differential $\widehat{\partial}$, where $\partial_{i}$ is defined by counting all J-holomorphic disks
$C$ satisfying $\mu(C)=1$ and $J_{+}(C)=2i$. As shown in~\cite{key-10}, this gives a sequence of chain complexes
defined recursively by
\[
E_{n}^{(S,\phi,\underline{\mathbf{a}})}=H_{\ast}\left(E_{n-1}^{(S,\phi,\underline{\mathbf{a}})},\partial_{n-1}\right),
\]
where $E_0 = {\mathbb{Z}}_2\langle\, {\mathbb{T}}_{\alpha} \cap {\mathbb{T}}_{\beta} \,\rangle$.
Now the algebraic torsion of $(M,\xi)$ can be defined in the following way.
\begin{defn}
Let $(M,\xi)$ be a compact contact $3$-manifold. For an open
book decomposition $(S,\phi)$ supporting $\xi$ and an arc basis ${\underline{\mathbf{a}}}$ on~$S$, denote the
induced spectral sequence defined above by $E_{n}^{(S,\phi,\underline{\mathbf{a}})}$.
We say that ${\mathit{AT}}(S,\phi,{\underline{\mathbf{a}}})=k$ if the contact
class ${\mathit{EH}}(\xi)\in E_{0}^{(S,\phi,{\underline{\mathbf{a}}})}$, represented by the generator $({\underline{\mathbf{a}}} \cap {\underline{\mathbf{b}}}) \times \{1/2\} \in {\mathbb{T}}_{\alpha} \cap {\mathbb{T}}_{\beta}$,
is nonzero in $E_{k}^{(S,\phi,{\underline{\mathbf{a}}})}$, and zero in~$E_{k+1}^{(S,\phi,{\underline{\mathbf{a}}})}$.
Then we define
\[
{\mathit{AT}}(M,\xi)=\min\left\{\, {\mathit{AT}}(S,\phi,\underline{\mathbf{a}})\,\colon\,
(S,\phi)\mbox{ is an open book supporting } \xi \mbox{ and ${\underline{\mathbf{a}}} \subset S$ is an arc basis}\,\right\} .
\]

Before extending this definition to manifolds with boundary,
we first review the definition of partial open book decompositions, introduced
by Honda, Kazez, and Mati\'c~\cite{key-2}.
We follow the treatment of Etg\"u and Ozbagci~\cite{key-4}. An abstract partial open book decomposition
is a triple $\mbox{\ensuremath{\mathcal{P}}}=(S,P,h)$, where \end{defn}
\begin{itemize}
\item $S$ is a compact oriented connected surface with nonempty boundary,
\item $P=P_{1}\cup \dots\cup P_{r}$ is a proper subsurface of $S$ such
that $S$ is obtained from $\overline{S \setminus P}$ by successively attaching
$1$-handles $P_{1},\dots,P_{r}$,
\item $h:P\rightarrow S$ is an embedding such that $h|_{A}=\mbox{Id}_A$,
where $A=\partial P\cap\partial S$.
\end{itemize}
Given a partial open book decomposition $(S,P,h)$, we associate to it a
sutured $3$-manifold $(M,\Gamma)$, as follows.
Let $H = S \times [-1,0]/\sim$, where $(x,t) \sim (x,t')$ for every $x \in \partial S$ and $t$, $t' \in [-1,0]$.
Furthermore, let $N = P \times I/ \sim$, where $(x,t) \sim (x,t')$ for every $x \in A$ and $t$, $t' \in I$.
We obtain the manifold~$M$ by gluing $(x,0) \in \partial N$ to $(x,0) \in \partial H$
and $(x,1) \in \partial N$ to $(h(x),-1) \in \partial H$ for every $x \in P$.
The sutures are defined as
\[
\Gamma = ({\overline}{\partial S \setminus \partial P}) \times \{0\} \cup -({\overline}{\partial P \setminus \partial S}) \times \{1/2\}.
\]
Then
\[
\Sigma=(P\times\{0\}\cup-S\times\{-1\})/\sim.
\]
is a Heegaard surface for $(M,\Gamma)$.

Let $\xi$ be a contact structure on $M$ such that $\partial M$ is convex with dividing set $\Gamma$.
Similarly to the original Giroux correspondence, we say that~$\xi$ is compatible with
the partial open book decomposition $(S,P,h)$ if
\begin{itemize}
\item $\xi$ is tight on the handlebodies $H$ and $N$,
\item $\partial H$ is a convex surface with dividing set $\partial S \times\{0\}$,
\item $\partial N$ is a convex surface with dividing set $\partial P\times\{1/2\}$.
\end{itemize}
Then the relative Giroux correspondence theorem says that $\xi$ is
uniquely determined up to contact isotopy, and given such a contact structure $\xi$,
any two partial open book decompositions compatible with $\xi$ are
related by positive stabilizations.

We now extend the definition of algebraic torsion to manifolds
with boundary. Suppose that a contact $3$-manifold $(M,\xi)$
with convex boundary $\partial M$ and dividing set $\Gamma$ is given.
Then $(M,\Gamma)$ is a balanced sutured manifold if $M$ has no closed components.
Indeed, every convex surface has a non-empty dividing set, and
$\chi(R_+(\Gamma)) = \chi(R_-(\Gamma))$ by \cite[Proposition~3.5]{polytope}.
Then we have a compatible partial
open book decomposition $\mbox{\ensuremath{\mathcal{P}}}=(S,P,h)$.
An arc basis for $(S,P,h)$ is a set ${\underline{\mathbf{a}}}$ of properly embedded
arcs in~$P$ with endpoints on~$A$ such that $S \setminus {\underline{\mathbf{a}}}$
deformation retracts onto ${\overline}{S \setminus P}$.
Similarly to the closed case, a partial open book decomposition of $M$,
together with a choice of arc basis $\underline{\mathbf{a}}$, gives
a sutured Heegaard diagram $({\Sigma},{\boldsymbol{\alpha}},{\boldsymbol{\beta}})$ of $M$.

The differential of the sutured Floer chain counts the number of J-holomorphic
curves $C$ with $\mu(C)=1$, modulo the ${\mathbb{R}}$-action, that do not intersect the suture
$\Gamma = \partial\Sigma$. For any topological Whitney disk $C$ from
$\mathbf{x}\in {\mathbb{T}}_{\alpha} \cap {\mathbb{T}}_{\beta}$ to $\mathbf{y}\in {\mathbb{T}}_{\alpha} \cap {\mathbb{T}}_{\beta}$
that does not intersect $\partial {\Sigma}$, we define the number $J_{+}(C)$
as in the case when $M$ is closed by
\[
J_{+}(C)=n_{\mathbf{x}}(D)+n_{\mathbf{y}}(D)-e(D)+|{\mathbf{x}}|-|{\mathbf{y}}|,
\]
where, $D={\mathcal{D}}(C)$ is the domain of $C$.
Since the equality $e(D)=1-n_{\mathbf{x}}(D)-n_{\mathbf{y}}(D)$
for $\mu(D)=1$ still holds in the sutured case, we get that
\begin{equation} \label{eqn:J}
J_{+}(C)=2(n_{\mathbf{x}}(D)+n_{\mathbf{y}}(D))-1+|{\mathbf{x}}|-|{\mathbf{y}}|
\end{equation}
when $\mu(C)=1$.
As in the closed case, the function $J_+$ is clearly additive.
We claim that $J_{+}(C)$ is always a nonnegative integer for disks $C$ with
$\mu(C)=1$.

\begin{lem} \label{lem:jplus}
For any J-holomorphic disk $C \in \pi_2({\mathbf{x}},{\mathbf{y}})$
such that $\mu(C)=1$, we have
\[
J_{+}(C) \in {\mathbb{Z}}_{\ge 0}.
\]
\end{lem}

\begin{proof}
Let $\mathbf{x}=\{x_{i}\}$ and $\mathbf{y}=\{y_{i}\}$,
where $x_{i} \in \alpha_{i} \cap \beta_{\sigma_{\mathbf{x}}(i)}$ and
$y_{i} \in \alpha_{i} \cap \beta_{\sigma_{\mathbf{y}}(i)}$, and write
\[
S = \left\{\, i\,\colon\,\sigma_{\mathbf{x}}(i)\ne\sigma_{\mathbf{y}}(i) \,\right\}.
\]
If $\sigma = \sigma_{\mathbf{x}} \circ \sigma_{\mathbf{y}}^{-1}$, then $\sigma(i) = i$ for every $i \not\in S$.
So $\sigma$ is a product of some disjoint non-trivial cycles whose lengths sum up to at most $|S|$.
Hence, $\sigma$ can be written as the product of at most $|S|-1$ transpositions.
In other words, $\sigma_{\mathbf{x}}$ can be obtained from $\sigma_{\mathbf{y}}$ by post-composing it with at most $|S|-1$
transpositions. If $\pi$ is a permutation and $\tau = (ij)$ is a transposition, then
\[
|\tau \circ \pi| - |\pi| =
\begin{cases}
$-1$ & \mbox{if $i$ and $j$ lie in different cycles,} \\
0 & \mbox{if $f(i) = j$ or $f(j) = i$,} \\
1 & \mbox{otherwise.}
\end{cases}
\]
Applying this repeatedly, we obtain that
\[
||\sigma_{\mathbf{x}}|-|\sigma_{\mathbf{y}}|| \le |S|-1.
\]
Let $D={\mathcal{D}}(C)$ be the domain of $C$, this is non-negative as $C$ is holomorphic.
For every $i \in S$, we have $x_i \neq y_i$,
so $D$ has to have a non-zero coefficient at both $x_i$ and $y_i$,
contributing at least $1/4$ to $n_{\mathbf{x}}(D)$ and $n_{\mathbf{y}}(D)$, respectively.
Therefore, we have
\[
J_{+}(C)\ge  2\left(\frac14 |S| + \frac14 |S|\right) -1 - (|S| - 1) = 0.
\]

To see that $J_+(C)$ is an integer, note that the points $x_i$ and $y_i$
divide ${\alpha}_i$ into two arcs. The multiplicities of $\partial D \cap {\alpha}_i$
along these arcs differ by one as $D$ connects ${\mathbf{x}}$ to ${\mathbf{y}}$; denote these multiplicities
by $m$ and $m+1$. Then the multiplicities of $D$ around $x_i$ are $a$, $a+m$, $b$, $b+m+1$,
and around $y_i$ they are $c$, $c+m$, $d$, $d+m+1$. It follows that
\[
2\left(n_{x_i}(D)+n_{y_i}(D)\right) = 2\left(\frac14 (2a+2b+2m+1) + \frac14(2c+2d+2m+1) \right) = a+b+c+d+2m+1 \in {\mathbb{Z}}.
\]
This concludes the proof of the lemma.
\end{proof}

By the above lemma, we can split the sutured Floer differential $\widehat{\partial}$
as
\[
\widehat{\partial}=\partial_{0}+\partial_{1/2}+\partial_{1}+\cdots,
\]
 where $\partial_{r}$ counts J-holomorphic curves $C$ with $\mu(C)=1$
and $J_{+}(C)=2r$.
\begin{lem}
Let $E_{0}=CF(\Sigma,{\boldsymbol{\alpha}},{\boldsymbol{\beta}})$.
For each half-integer $r \ge 0$,
\[
\left(E_{r+1/2}=H_{\ast}(E_{r},\partial_{r}),\partial_{r+1/2}\right)
\]
is a chain complex. \end{lem}
\begin{proof}
The proof is very similar to the case when $M$ is closed.
Let $\mathbf{x}$, $\mathbf{y}\in\mathbb{T}_{\boldsymbol{\alpha}}\cap\mathbb{T}_{\boldsymbol{\beta}}$
and a homotopy class $\phi \in \pi_2(\mathbf{x},\mathbf{y})$
satisfying $\mu(\phi)=2$ be given. The function $J_{+}$ defines a partition
\[
\widetilde{\mathcal{M}}(\mbox{\ensuremath{\phi}})=\coprod_{r\in\frac{1}{2}\mathbb{Z}}\widetilde{\mathcal{M}}(\phi)_{r}
\]
of the moduli space $\widetilde{\mathcal{M}}(\phi)$, where $\widetilde{\mathcal{M}}(\phi)_{r}$ consists of J-holomorphic
disks $C\in\widetilde{\mathcal{M}}(\phi)$ satisfying $J_{+}(C) = 2r$.
Since the value of $J_+$ depends only on the domain of
a given topological Whitney disk, this extends to a partition of the compatification
\[
\overline{\mathcal{M}(\phi)}=\coprod_{r\in\frac{1}{2}\mathbb{Z}}\overline{\mathcal{M}(\phi)}_{r}.
\]
 By the additivity of $J_{+}$ and the nonnegativity of $J_{+}(C)$
for $\mu(C)=1$, any broken flowline in $\partial\overline{\mathcal{M}(\phi)}$
is a sum of two disks with nonnegative $J_{+}$. Thus we get that
\begin{eqnarray*}
\sum_{0\le i\le2r+1}\partial_{i}\partial_{2r+1-i}\mathbf{x} & = & \sum_{{\mathbf{y}},{\mathbf{z}} \in {\mathbb{T}}_{\alpha} \cap {\mathbb{T}}_{\beta}} \sum_{\substack{\psi \in \pi_2(\mathbf{x},\mathbf{z})\\
\mu(\psi)=1
}}\sum_{\substack{ \psi' \in \pi_2(\mathbf{z},\mathbf{y})\\
\mu(\psi')=1\\
J_{+}(\psi)+J_{+}(\psi') = 2r+1
}}\#\widetilde{\mathcal{M}}(\psi)\cdot\#\widetilde{\mathcal{M}}(\psi')\cdot\mathbf{y}\\
 & = &  \sum_{\mathbf{y} \in {\mathbb{T}}_{\alpha} \cap {\mathbb{T}}_{\beta}} \sum_{\substack{\phi \in \pi_2(\mathbf{x},\mathbf{y})\\
\mu(\phi)=2
}}\#\partial\overline{\mathcal{M}(\phi)}_{2r+1}\cdot\mathbf{y}=0.
\end{eqnarray*}
 Since $\partial_{s}=0$ in $E_{r+1/2}$ for all $s\le r$, we get that
$\partial_{r+1/2}^{2}=0$ in $E_{r+1/2}$. The result follows.
\end{proof}

Hence we can define the algebraic torsion of $(M,\xi)$ in the following way.
\begin{defn}
For a contact $3$-manifold $(M,\xi)$ with
convex boundary, a compatible partial open book decomposition $\mathcal{P}$,
and a choice of arc basis $\underline{\mathbf{a}}$, denote the induced
spectral sequence by $E_{n}^{(\mathcal{P},\underline{\mathbf{a}})}$.
We say that $AT(\mathcal{P},\underline{\mathbf{a}})=k$ if the contact
class $EH(\xi)\in E_{0}^{(\mathcal{P},\underline{\mathbf{a}})}$ is
nonzero in~$E_{k}^{(\mathcal{P},\underline{\mathbf{a}})}$ and zero
in $E_{k+\frac{1}{2}}^{(\mathcal{P},\underline{\mathbf{a}})}$. Then
we define
\[
{\mathit{AT}}(M,\xi)=\min\left\{ AT(\mathcal{P},\underline{\mathbf{a}})\,\colon\,\mathcal{P}\mbox{ is a compatible partial open book and }\underline{\mathbf{a}}\mbox{ is an arc basis}\right\} .
\]
\end{defn}

Note that, in the above definition, the algebraic torsion ${\mathit{AT}}(M,\xi)$
is defined to be some nonnegative half-integer. However, in the next
section, we shall see that it is actually an integer by proving that
the values of $J_{+}$ are always even.

\begin{rem*}
Given a closed contact $3$-manifold $(M,\xi)$,
the contact manifold $(M(1),\xi)$ is obtained by removing
a tight contact ball from $M$. The suture on $\partial M(1)\simeq S^{2}$
is a single curve. Then it follows from the above definition that
$\mbox{AT}(M(1),\xi)=\mbox{AT}(M,\xi)$.
\end{rem*}

\section{Inequality of algebraic torsions} \label{sec:ineq}

The goal of this section is to prove Theorem~\ref{thm:ineq} from the introduction.
Let $(M,\xi)$ be a contact $3$-manifold with convex boundary,
and let $N$ be a codimension zero submanifold of~$\text{Int}(M)$, also with convex
boundary. We can suppose that $M \setminus N$ has no isolated components; i.e., every
component of $M \setminus N$ intersects~$\partial M$. Indeed, removing a tight contact 
ball from each isolated components leaves ${\mathit{AT}}$ unchanged. 

We now briefly recall the construction of the contact gluing map $\Phi$ on sutured Floer homology, defined
by Honda, Kazez, and Mati\'c~\cite{key-2}. Choose a tubular neighborhood
$U\simeq\partial N\times\mathbb{R}$ of $\partial N$, on which the
contact structure $\xi$ becomes $\mathbb{R}$-invariant, and write $N' = M \setminus (N \cup U)$.
Let $\Sigma_{N'}$ be a Heegaard surface
compatible with $\xi|_{N'}$, and let~$\Sigma_{U}$ be a Heegaard
surface compatible with $\xi|_{U}$. Then, for any sutured Heegaard
diagram ${\mathcal{H}} = (\Sigma,{\boldsymbol{\beta}},{\boldsymbol{\alpha}})$
of $N$ that is contact-compatible near $\partial N$ in
the sense of Honda et al.~\cite{key-2}, $\Sigma\cup\Sigma_{U}\cup\Sigma_{N'}$
is a Heegaard surface for~$M$, and we can complete ${\boldsymbol{\alpha}}$ and ${\boldsymbol{\beta}}$ to attaching
sets of $M$ by adding ${\boldsymbol{\alpha}}'$ and ${\boldsymbol{\beta}}'$ compatible
with $\xi|_{N' \cup U}$. We write
\[
{\mathcal{H}}' = (\Sigma\cup\Sigma_{U}\cup\Sigma_{N'},{\boldsymbol{\beta}}\cup{\boldsymbol{\beta}}',{\boldsymbol{\alpha}}\cup{\boldsymbol{\alpha}}').
\]
Then the map
\begin{eqnarray*}
\Phi:{\mathit{CF}}({\mathcal{H}}) & \rightarrow & {\mathit{CF}}({\mathcal{H}}'),\\
{\mathbf{y}} & \mapsto & ({\mathbf{y}},{\mathbf{x}}')
\end{eqnarray*}
is a chain map, where ${\mathbf{x}}' \in {\mathbb{T}}_{{\boldsymbol{\alpha}}' \cap {\boldsymbol{\beta}}'}$ is the canonical representative of the contact class $EH(\xi|_{N' \cup U})$.

Suppose that we choose the diagram $(\Sigma,{\boldsymbol{\beta}},{\boldsymbol{\alpha}})$ to be
the one induced from a partial open book decomposition of $(N,\xi|_{N},\Gamma_{N})$.
Then, since $\xi|_{N}$ and $\xi|_{U}$ come from the same contact
structure $\xi$, it is automatically contact-compatible near $\partial N'$,
which in turn implies that the map $\Phi$ above is well-defined when
we use this diagram. Also, it obviously maps the contact class of
$N$ to the contact class of $M$. Note that, given a partial open
book decomposition $\mathcal{P}_{N}=(S_{N},P_{N},h_{N})$ and an arc
basis $\underline{\mathbf{a}}_{N}$ of $N$, this construction of
Honda et al.~actually gives a partial open book $\mathcal{P}=(S,P,h)$ and
an arc basis $\underline{\mathbf{a}}$ that extend $\mathcal{P}_{N}$
and $\underline{\mathbf{a}}_{N}$, respectively.

\begin{lem}
The map $\Phi$ induces a chain map
\[
\Phi_r \colon E_{r}^{(\mathcal{P}_{N},\underline{\mathbf{a}}_{N})}\rightarrow E_{r}^{(\mathcal{P},\underline{\mathbf{a}})}
\]
for every nonnegative half-integer $r$.
\end{lem}

\begin{proof}
Let ${\mathbf{y}}$, ${\mathbf{z}} \in {\mathbb{T}}_{\alpha} \cap {\mathbb{T}}_{\beta}$. Any holomorphic disk~$C$ from
$(\mathbf{y},\mathbf{x}^{\prime})$ to $(\mathbf{z},\mathbf{x}^{\prime})$ in ${\mathit{CF}}({\mathcal{H}}')$
is actually a holomorphic disk from $\mathbf{y}$ to $\mathbf{z}$
in ${\mathit{CF}}({\mathcal{H}})$; i.e., its domain $D:={\mathcal{D}}(C)$ is zero outside $\Sigma$, see~\cite{key-2}.
Since the Euler measure and the point measure of $D$ depend only on the non-zero coefficients,
the Maslov index of $C$ in ${\mathcal{H}}$ and in ${\mathcal{H}}'$ are the same. Suppose that $\mu(C) = 1$.
Then, in ${\mathcal{H}}'$, we have
\begin{eqnarray*}
J_{+}(C) & = & 2(n_{(\mathbf{y},\mathbf{x}^{\prime})}(D)+n_{(\mathbf{z},\mathbf{x}^{\prime})}(D))-1+|(\mathbf{y},\mathbf{x}^{\prime})|-|(\mathbf{z},\mathbf{x}^{\prime})|\\
 & = & 2(n_{\mathbf{y}}(D)+n_{\mathbf{z}}(D))-1+|\mathbf{y}|+|\mathbf{x}^{\prime}|-|\mathbf{z}|-|\mathbf{x}^{\prime}|\\
 & = & 2(n_{\mathbf{y}}(D)+n_{\mathbf{z}}(D))-1+|\mathbf{y}|-|\mathbf{z}|.
\end{eqnarray*}
This is the same as the value of $J_{+}(C)$ in ${\mathcal{H}}$.
Hence $\Phi$ preserves the $J_{+}$ filtration.

Now we induct on $r$. When $r=0$, since $\Phi$ preserves $J_{+}$,
it must commute with $\partial_{0}$. Suppose that
\[
\Phi_r  \colon E_{r}^{(\mathcal{P}_{N},\underline{\mathbf{a}_{N}})}\rightarrow E_{r}^{(\mathcal{P},\underline{\mathbf{a}})}
\]
is a chain map for some nonnegative half-integer $r$. Since $\Phi$
preserves $J_{+}$, the coefficients of $\partial_{r+1/2}$ in $E_{r}^{(\mathcal{P},\underline{\mathbf{a}})}$
are the same as the corresponding coefficients of $\partial_{r+1/2}$
in $E_{r}^{(\mathcal{P}_{N},\underline{\mathbf{a}}_{N})}$. Hence
$\Phi$ commutes with $\partial_{r+1/2}$. This completes the proof.
\end{proof}

\begin{proof}[Proof of Theorem~\ref{thm:ineq}]
Since $\Phi$ sends the contact element to the contact
element,
\[
{\mathit{AT}}(\mathcal{P}_{N},\underline{\mathbf{a}}_{N})\ge {\mathit{AT}}(\mathcal{P},\underline{\mathbf{a}})\ge {\mathit{AT}}(M,\xi).
\]
Taking the minimum of over all possible choices of $(\mathcal{P}_{N},\underline{\mathbf{a}}_{N})$,
we get that
\[
{\mathit{AT}}(N,\xi|_{N})\ge {\mathit{AT}}(M,\xi).
\]
\end{proof}

The proof of Lemma~\ref{lem:jplus} can also be used to prove that the algebraic
torsion of a manifold with boundary is always an integer.

\begin{lem}
Let $(M,\xi)$ be a balanced sutured contact $3$-manifold
with convex boundary. Then $\mbox{AT}(M,\xi)\in\mathbb{Z}_{\ge0}$.
\end{lem}

\begin{proof}
Choose a closed contact $3$-manifold $(Y,\xi_Y)$
into which $(M,\xi)$ embeds as a contact submanifold. We may
assume that $M \subset Y(1)$. Let $\mathcal{P}_{M}$ be a compatible
partial open book decomposition of $(M,\xi)$, and let $\underline{\mathbf{a}}_{M}$
be an arc basis. Then, we have an open book decomposition $\mathcal{P}$
and an arc basis $\underline{\mathbf{a}}$ of $Y(1)$ that extend
$\mathcal{P}_{M}$ and $\underline{\mathbf{a}}_{M}$, respectively.
Let ${\mathcal{H}}_M = (\Sigma_{M},{\boldsymbol{\alpha}}_{M},{\boldsymbol{\beta}}_{M})$
and $(\Sigma,{\boldsymbol{\alpha}},{\boldsymbol{\beta}})$ be the induced sutured Heegaard diagrams, and let
$\Phi \colon {\mathit{CF}}({\mathcal{H}}_M) \rightarrow {\mathit{CF}}({\mathcal{H}})$ be the gluing map. Then, for any $\mathbf{x}$, $\mathbf{y} \in \mathbb{T}_{\alpha} \cap\mathbb{T}_{\beta}$,
a J-holomorphic disk $C$ from $\mathbf{x}$ to $\mathbf{y}$ is actually
a J-holomorphic disk from $\Phi(\mathbf{x})$ to $\Phi(\mathbf{y})$,
with the same $J_{+}$ value.

Now, capping the boundary of $\Sigma$ gives a Heegaard diagram ${\mathcal{H}}_0 = (\Sigma_{0}=\Sigma/\partial\Sigma,{\boldsymbol{\alpha}},{\boldsymbol{\beta}})$.
Then we have an isomorphism
\[
{\mathit{CF}}({\mathcal{H}}) \rightarrow {\mathit{CF}}({\mathcal{H}}'),
\]
and any J-holomorphic disk in ${\mathcal{H}}$ is also a J-holomorphic disk in ${\mathcal{H}}_0$.
Since ${\mathcal{H}}_0$ is constructed from an open book decomposition of the closed manifold $(Y,\xi_Y)$,
we know from the work of Kutluhan et al.~\cite{key-10} that any J-holomorphic disk inside it
has nonnegative and even $J_{+}$ values. Hence, it follows that any
J-holomorphic disk inside ${\mathcal{H}}_M$ also has nonnegative and even $J_{+}$ values. The result follows.
\end{proof}

\section{Calculation of upper bounds to some algebraic torsions} \label{sec:gt}

Let $(M,\xi)$ be a contact $3$-manifold with convex boundary. Suppose that $(M,\xi)$
is overtwisted. Then, by definition, it contains an embedded overtwisted
disk $\Delta$. This has a standard neighborhood; i.e., there exists
a neighborhood $U\supset\Delta$ such that $(U,\xi|_{U})$ is contactomorphic
to a neighborhood of the disk $\Delta_{std}=\{z=0,\rho\le\pi\}$ inside
the standard overtwisted contact structure on $\mathbb{R}^{3}$, which is
defined as follows~\cite{key-8}:
\[
\xi_{OT}=\ker(\cos\rho \, dz+\rho\sin\rho\, d\phi).
\]
 Inside $U$, we can perturb $\Delta$ to a convex surface $D$. Take
a neighborhood $V=D\times [-1,1]$ such that $\xi|_{\text{Int}(M)}$ is $\mathbb{R}$-invariant.
After rounding its edges, we obtain an open subset $V_{0}\simeq D^3$
such that the dividing set $\Gamma_{V_{0}}$ on $\partial V_{0}$
is given by three disjoint curves. Honda, Kazez, and Mati\'c~\cite[Example~1]{key-1} gave a partial open
book decomposition of $N=\overline{V_{0}}$, and the corresponding Heegaard diagram is shown in Figure~\ref{fig:ot}.
\begin{figure}
\includegraphics{overtwisted.pdf}
\caption{A sutured Heegaard diagram arising from a partial open book decomposition of a neighborhood of an overtwisted disk.
We obtain the Heegaard surface by identifying the two bold horizontal arcs.}
\label{fig:ot}
\end{figure}
This diagram can be used to show that ${\mathit{AT}}(M,\xi)=0$,
which was proven by Kutluhan et al.~\cite{key-10} in the closed case using the fact that an overtwisted
contact structure admits an open book whose monodromy is not right-veering.

\begin{prop} \label{prop:ot}
If~$N$ is the standard neighborhood of an overtiwsted disk in the contact manifold $(M,\xi)$ as above,
then
\[
{\mathit{AT}}(N,\xi|_{N})=0.
\]
\end{prop}

\begin{proof}
Honda et al.~\cite[Example~1]{key-1} computed that $c(N,\xi|_{N}) = 0$;
we extend their proof. Consider the partial open book decomposition
of $(N,\xi)$ shown in Figure~\ref{fig:ot}.
The contact element $EH(N,\xi|_{N})$ is represented by the point~$x$,
which is zero in homology because $\partial y = x$. The only
J-holomophic curve from $y$ to $x$ is the bigon, which
satisfies $J_{+}=0$. Hence $AT(N,\xi|_{N})\le0$.
\end{proof}

\begin{thm} \label{thm:OT}
If the contact manifold $(M,\xi)$ with convex boundary is overtwisted, then $AT(M,\xi)=0$.
\end{thm}

\begin{proof}
We have $AT(M,\xi)\le AT(N,\xi|_{N})=0$ by Theorem~\ref{thm:ineq} and Proposition~\ref{prop:ot}.
\end{proof}

Now we consider the case when $(M,\xi)$ has Giroux $2\pi$-torsion.
Recall that a contact manifold $(M,\xi)$ has $2\pi$-torsion if it
admits an embedding
\[
(M_{2\pi},\eta_{2\pi})=(T^{2}\times[0,1],\ker(\cos(2\pi t)\,dx-\sin(2\pi t)\,dy))\hookrightarrow(M,\xi).
\]
The boundary of $(M_{2\pi},\eta_{2\pi})$ is not convex. However, as in \cite[Lemma~5]{key-5},
if it embeds in $(M,\xi)$, then there exist small $\epsilon_0$, $\epsilon_1 >0$
such that the slightly extended domain
\[
(M',\eta')= \left(T^{2}\times[-\epsilon_0,1+\epsilon_1],\ker(\cos(2\pi t)\,dx-\sin(2\pi t)\,dy) \right)
\]
also embeds inside $(M,\xi)$ such that $T^2 \times \{-\epsilon_0\}$ and $T^2 \times \{\epsilon_1\}$
are pre-Lagrangian tori with integer slopes $s_0$ and $s_1$ that form a basis of $H_1(T^2)$.
By the work of Ghiggini~\cite{key-3}, we can perturb $\partial M'$
to get a new contact submanifold $\widetilde{M}$
such that $\partial\widetilde{M}$ is convex, and the slopes of the
dividing sets are $s_0$ and $s_1$. After a change of coordinates in $\widetilde{M}$,
we can assume these slopes are $0$ and $\infty$.

The contact manifold $\widetilde{M}$ is non-minimally-twisting and consists of five basic slices,
which means that we can construct a partial open book decomposition
of it by attaching four bypasses to a partial open book diagram of
a basic slice, which can be found in Examples~4, 5 and~6 of~\cite{key-1}.
The diagram we get is shown in Figure~\ref{fig:gt}.
\begin{figure}
\resizebox{.7\textwidth}{!}{\includegraphics{munji-4-crop.pdf}}
\caption{A sutured diagram arising from a partial open book decomposition of a neighborhood of a Giroux torsion domain.
The opposite green arcs in the boundary are identified.}
\label{fig:gt}
\end{figure}

\begin{figure}
\resizebox{.7\textwidth}{!}{\includegraphics{munji-5-crop.pdf}}
\caption{We apply the Sarkar-Wang algorithm by isotoping the red curves along the dashed arcs.}
\label{fig:dashed}
\end{figure}

Applying the Sarkar-Wang algorithm~\cite{key-12} to this diagram along the dotted arcs in Figure~\ref{fig:dashed}
gives the one in Figure~\ref{fig:SW}. It is easy to check that every
region that does not intersect the boundary is either a bigon or a quadrilateral.
\begin{figure}
\resizebox{.7\textwidth}{!}{\includegraphics{munji-6-crop.pdf}}
\caption{The diagram after applying the Sarkar-Wang algorithm.}
\label{fig:SW}
\end{figure}

\begin{figure}
\resizebox{.7\textwidth}{!}{\includegraphics{munji-7-crop.pdf}}
\caption{The quadrilaterals and bigons relevant to the computation are shaded.}
\label{fig:color}
\end{figure}

The contact element ${\mathit{EH}}(\widetilde{M})$ is represented by the
unordered tuple $(x_{1},y_{1},z_{1},w_{1}, v_1)$. We now directly prove
that the contact invariant of $\widetilde{M}$ is zero and calculate
its algebraic torsion with respect to the given diagram, thus giving
an upper bound on ${\mathit{AT}}(\widetilde{M})$.

If $Q$ is a quadrilateral component of ${\Sigma} \setminus ({\boldsymbol{\alpha}} \cup {\boldsymbol{\beta}})$ disjoint from $\partial {\Sigma}$
with corners $c_1,c_2,c_3,c_4 \in {\boldsymbol{\alpha}} \cap {\boldsymbol{\beta}}$,
then we say that $c_{1}$, $c_{3}$ are its from-corners and $c_{2}$, $c_{4}$
are its to-corners if
\[
\partial(\partial Q \cap {\boldsymbol{\alpha}}) = c_1 + c_3 - c_2 - c_4.
\]
For any generator $(c_1,c_3,\dots) \in {\mathbb{T}}_{\alpha} \cap {\mathbb{T}}_{\beta}$,
the coefficient of $(c_{2},c_{4},\dots)$ in the boundary $\partial(c_{1},c_{3},\dots)$
is the number of such quadrilaterals.

Since the only quadrilateral whose to-corners
are in $\{x_{1},y_{1},z_{1},w_{1},v_1\}$ is $y_{1}y_{2}z_{1}z_{2}$,
we get that
\[
\partial(x_{1},y_{2},z_{2},w_{1},v_1)=(x_{1},y_{1},z_{1},w_{1},v_1)+(x_{1},y_{3},z_{2},w_{1},v_1),
\]
where the last term comes from the bigon $y_{2}y_{3}$. This quadrilateral and bigon are  shaded
grey in Figure~\ref{fig:SW}.

The only quadrilateral whose to-corners are in $\{x_{1},y_{2},z_{2},w_{1},v_1\}$
is $x_{1}x_{2}y_{3}y_{4}$, and we have that
\[
\partial(x_{2},y_{4},z_{2},w_{1},v_1)=(x_{1},y_{3},z_{2},w_{1},v_1)+(x_{3},y_{4},z_{2},w_{1},v_1),
\]
where the last term comes from the bigon $x_{2}x_{3}$.
This quadrilateral and bigon are  shaded pink in Figure~\ref{fig:SW}.

The only quadrilateral whose to-corners are in $\{x_{3},y_{4},z_{2},w_{1},v_1\}$
is $x_{3}x_{4}w_{1}w_{2}$, and we have that
\[
\partial(x_{4},y_{4},z_{2},w_{2},v_1)=(x_{3},y_{4},z_{2},w_{1},v_1)+(x_{5},y_{4},z_{2},w_{2},v_1),
\]
where the last term comes from the bigon $x_{4}x_{5}$.
This quadrilateral and bigon are shaded light blue in Figure~\ref{fig:SW}.

The only quadrilateral whose to-corners are in $\{x_{5},y_{4},z_{2},w_{2},v_1\}$
is $x_{5}x_{6}z_{2}z_{3}$, shaded green in Figure~\ref{fig:SW}, and we have that
\[
\partial(x_{6},y_{4},z_{3},w_{2},v_1)=(x_{5},y_{4},z_{2},w_{2},v_1)+(x_9,y_1,z_3,w_2,v_1),
\]
where the last term comes from the quadrilateral $x_6x_9y_4y_1$.

The only quadrilateral whose to-corners are in $\{x_9,y_1,z_3,w_2,v_1\}$
is $y_1y_{15}z_{3}z_{2}$, and we have that
\[
\partial(x_{9},y_{15},z_{2},w_{2},v_1)=(x_9,y_1,z_3,w_2,v_1)+(x_9,y_{14},z_2,w_2,v_1),
\]
where the last term comes from the bigon $y_{15}y_{14}$.
This quadrilateral and bigon are shaded yellow in Figure~\ref{fig:SW}.

The only quadrilateral whose to-corners are in $\{x_9,y_{14},z_2,w_2,v_1\}$
is $y_{14}y_{13}v_1v_2$, and we have that
\[
\partial(x_{9},y_{13},z_{2},w_{2},v_2)=(x_9,y_{14},z_2,w_2,v_1)+(x_9,y_{12},z_2,w_2,v_2),
\]
where the last term comes from the bigon $y_{13}y_{12}$.
This quadrilateral and bigon are shaded blue in Figure~\ref{fig:SW}.

Finally, the only quadrilateral whose to-corners are in $\{x_9,y_{12},z_2,w_2,v_2\}$
is $y_{12}y_{11}w_2w_3$, shown in red, and we have that
\[
\partial(x_{9},y_{11},z_{2},w_{3},v_2)=(x_9,y_{12},z_2,w_2,v_2).
\]

Hence,
\[
\begin{split}
\partial((x_{1},y_{2},z_{2},w_{1},v_1)-(x_{2},y_{4},z_{2},w_{1},v_1)+(x_{4},y_{4},z_{2},w_{2},v_1)-(x_{6},y_{4},z_{3},w_{2},v_1)\\
+ (x_{9},y_{15},z_{2},w_{2},v_1) - (x_{9},y_{13},z_{2},w_{2},v_2) + (x_{9},y_{11},z_{2},w_{3},v_2))=
(x_{1},y_{1},z_{1},w_{1}),
\end{split}
\]
whose homology class is ${\mathit{EH}}(\widetilde{M})$. Thus $c(\widetilde{M})=0$, so the algebraic torsion of $\widetilde{M}$
is finite.

\begin{rem*}
This result, together with the gluing map of \cite{key-2}, gives
an explicit computational proof of the fact that the contact invariant
of any contact $3$-manifold with Giroux $2\pi$-torsion vanishes,
which was proven in the closed case by Ghiggini, Honda, and Van Horn-Morris~\cite{key-5} modulo canonical orientation
systems in ${\mathit{SFH}}$.
\end{rem*}

\begin{rem*}
In \cite[Example 6-(c)]{key-1}, Honda et al.~show that if we only attach four bypasses to a basic slice;
i.e., if the contact structure is minimally twisting,
then the contact invariant is non-zero because it embeds in the unique Stein
fillable contact structure on~$T^{3}$, which already has non-zero
contact invariant. This can also be shown explicitly using a computation analogous to, but simpler
than the one above. Hence, it is necessary to enlarge the Giroux $2\pi$-torsion domain a bit
to obtain vanishing of the contact element.
\end{rem*}

\begin{prop} \label{prop:gt}
For the perturbed Giroux $2\pi$-torsion domain $\widetilde{M}$, we have
\[
{\mathit{AT}}(\widetilde{M})\le 1.
\]
\end{prop}

\begin{proof}
The complete list of the $J$-holomorphic disks used in the calculations above and
the values used to compute their $J_{+}$ are given in the table below.
If we label the ${\alpha}$- and ${\beta}$-curves such that $x_1 \in {\alpha}_1 \cap {\beta}_1$,
$y_1 \in {\alpha}_2 \cap {\beta}_2$, $z_1 \in {\alpha}_3 \cap {\beta}_3$, $w_1 \in {\alpha}_4 \cap {\beta}_4$,
and $v_1 \in {\alpha}_5 \cap {\beta}_5$, then 
\[
\begin{split}
x_1, x_9, y_4, y_{16} \in {\beta}_1, \\
x_6, y_1, z_2 \in {\beta}_2, \\
z_1, x_2, x_3, y_2, y_3, y_{11}, w_2 \in {\beta}_3, \\
x_4, x_5, y_{14}, y_{15}, z_3, w_1, v_2 \in {\beta}_4, \\
y_{12}, y_{13}, w_3, v_1 \in {\beta}_5. 
\end{split}
\]
Furthermore,
$x_i \in {\alpha}_1$, $y_i \in {\alpha}_2$, $z_i \in {\alpha}_3$, $w_i \in {\alpha}_4$, and $v_i \in {\alpha}_5$
for any~$i$. Note that if there is a bigon connecting ${\mathbf{x}}$, ${\mathbf{y}} \in {\mathbb{T}}_{\alpha} \cap {\mathbb{T}}_{\beta}$,
then $|{\mathbf{x}}| = |{\mathbf{y}}|$.
Using this, 
\[
\begin{split}
|(x_1,y_1,z_1,w_1,v_1)| = |(1)(2)(3)(4)(5)| = 5, \\
|(x_1,y_2,z_2,w_1,v_1)| = |(1)(23)(4)(5)| = 4 = |(x_1,y_3,z_2,w_1,v_1)|, \\
|(x_2,y_4,z_2,w_1,v_1)| = |(132)(4)(5)| = 3 = |(x_3,y_4,z_2,w_1,v_1)|, \\
|(x_4,y_4,z_2,w_2,v_1)| = |(1432)(5)| = 2 = |(x_5,y_4,z_2,w_2,v_1)|, \\
|(x_6,y_4,z_3,w_2,v_1)| = |(12)(34)(5)| =  3, \\
|(x_9,y_1,z_3,w_2,v_1)| = |(1)(2)(34)(5)| = 4, \\
|(x_9,y_{15},z_2,w_2,v_1)| = |(1)(243)(5)| = 3 = |(x_9,y_{14},z_2,w_2,v_1)| \\
|(x_9,y_{14},z_2,w_2,v_2)| = |(1)(2543)| = 2 = |(x_9,y_{12},z_2,w_2,v_2)|, \\
|(x_9,y_{11},z_2,w_3,v_2)| = |(1)(23)(45)| = 3. 
\end{split}
\]

From the table below, we see that every $J$-holomorphic disk $\phi$ used
to compute the differential satisfies $J_{+}(\phi) \le 2$; cf.~Equation~\ref{eqn:J}. 

\begin{center}
\begin{tabular}{|c|c|c|c|c|}
\hline
Type & Name & $2(n_{\mathbf{x}}+n_{\mathbf{y}})$ & $|\mathbf{x}|-|\mathbf{y}|$ & $J_{+}$\tabularnewline
\hline
\hline
quadrilateral & $y_{1}y_{2}z_{1}z_{2}$ & 2 & -1 & 0\tabularnewline
\hline
quadrilateral & $x_{1}x_{2}y_{3}y_{4}$ & 2 & -1 & 0\tabularnewline
\hline
quadrilateral & $x_{3}x_{4}w_{1}w_{2}$ & 2 & -1 & 0\tabularnewline
\hline
quadrilateral & $x_{5}x_{6}z_{2}z_{3}$ & 2 & 1 & 2\tabularnewline
\hline
quadrilateral & $x_6 x_9 y_4 y_1$ & 2 & -1 & 0 \tabularnewline
\hline
quadrilateral & $y_1 y_{15} z_3 z_2$ & 2 & -1 & 0 \tabularnewline
\hline
quadrilateral & $y_{14} y_{13} v_1 v_2$ & 2 & -1 & 0 \tabularnewline
\hline
quadrilateral & $y_{12} y_{11} w_2 w_3$ & 2 & 1 & 2 \tabularnewline
\hline
bigon & $y_{2}y_{3}$ & 1 & 0 & 0\tabularnewline
\hline
bigon & $x_{2}x_{3}$ & 1 & 0 & 0\tabularnewline
\hline
bigon & $x_{4}x_{5}$ & 1 & 0 & 0\tabularnewline
\hline
bigon & $y_{15}y_{14}$ & 1 & 0 & 0\tabularnewline
\hline
bigon & $y_{13}y_{12}$ & 1 & 0 & 0\tabularnewline
\hline
\end{tabular}
\par\end{center}

Hence, we have
${\mathit{EH}}(\widetilde{M})\in \mbox{Im}(\partial_{0}+\partial_{1})$. Therefore,
${\mathit{EH}}(\widetilde{M})$ vanishes in $E_{5/2}$; i.e., ${\mathit{AT}}(\widetilde{M})\le 1$.
\end{proof}

As an immediate corollary, we obtain Theorem~\ref{thm:gt} from the introduction.

\begin{cor}
If a contact $3$-manifold $(M,\xi)$ with convex boundary has Giroux $2\pi$-torsion,
then
\[
{\mathit{AT}}(M,\xi) \le 1.
\]
\end{cor}

\begin{proof}
If the Giroux $2\pi$-torsion domain $M_{2\pi}$ embeds in $M$, then
the perturbed domain $\widetilde{M}$ also embeds in $M$, by the
argument outlined at the beginning of this section. Then Theorem~\ref{thm:ineq}
and Proposition~\ref{prop:gt} imply that
\[
{\mathit{AT}}(M,\xi)\le {\mathit{AT}}(\widetilde{M}) \le 1.
\]
\end{proof}

\section{Open questions}

There are some questions that naturally arise in the discussions above.
First, as in the case of closed contact $3$-manifolds, we would like to
know how the algebraic torsion ${\mathit{AT}}(\mathcal{P},\underline{\mathbf{a}})$
depends on the choice of partial open book decomposition~$\mathcal{P}$ and
arc system~$\underline{\mathbf{a}}$.

\begin{rem*}
Given two possible choices of $(\mathcal{P},\underline{\mathbf{a}})$
and $(\mathcal{P}^{\prime},\underline{\mathbf{b}})$ for a given
contact $3$-manifold $(M,\xi)$ with convex boundary, it is natural to ask whether
${\mathit{AT}}(\mathcal{P},\underline{\mathbf{a}})={\mathit{AT}}(\mathcal{P}^{\prime},\underline{\mathbf{b}})$.
In the closed case, according to Wand~\cite{key-13}, the number ${\mathit{AT}}(S,\phi,\underline{\mathbf{a}})$
does not depend on the isotopy class of the arc basis $\underline{\mathbf{a}}$,
but if two arc bases differ by an arc-slide, the corresponding values
of ${\mathit{AT}}$ might not be the same. Since our definition of ${\mathit{AT}}$ is a direct
generalization of the original one, the same holds in our case.
\end{rem*}

Now, given the inequality ${\mathit{AT}}(N,\xi|_{N}) \ge {\mathit{AT}}(M,\xi)$,
whenever $(N,\xi|_{N})$ is a compact codimension zero submanifold
of $(M,\xi)$ with convex boundary, we are led to the following question.

\begin{qn} \label{qn:1}
If a balanced sutured contact $3$-manifold $(N,\xi)$ satisfies
${\mathit{AT}}(M,\xi_{M}) \le k$ for every closed contact $3$-manifold
$(M,\xi_{M})$ in which $(N,\xi)$ embeds, do we have ${\mathit{AT}}(N,\xi)\le k$?
\end{qn}

An affirmative answer to Question~\ref{qn:1} would imply that the inequality
${\mathit{AT}}(N,\xi|_{N}) \le {\mathit{AT}}(M,\xi)$ is sharp
and cannot be improved without giving extra conditions even when~$M$
is assumed to be closed. We can ask the following question
regarding the algebraic torsion of planar torsion domains.

\begin{qn} \label{qn:2}
If $(M,\xi)$ is a planar torsion domain of order $k$, as
defined by Wendl~\cite{Wendl},
then do we have ${\mathit{AT}}(M,\xi) \le k$?
\end{qn}

Note that Proposition~\ref{prop:gt} covers the case $k = 1$.
An affirmative answer to Question~\ref{qn:2}, together with Theorem~\ref{thm:ineq}, would
imply that the algebraic torsion of any closed contact $3$-manifold
is bounded above by its planar torsion, which would answer \cite[Question~3.2]{key-10}.

Finally, probably the most interesting question in this area is whether
the converse of Theorem~\ref{thm:OT} holds.

\begin{qn}
If ${\mathit{AT}}(M,\xi) = 0$, then does this imply that $\xi$ is overtwisted? 
\end{qn}

\begin{thebibliography}{Juh�sz}

\bibitem[El]{key-8} Ya.~Eliashberg, \emph{Classification
of overtwisted contact structures on 3-manifolds}, Invent.
Math. \textbf{98} (1989), no.~3, 623--637.

\bibitem[EO]{key-4} T.~Etg\"u and B.~Ozbagci, \emph{Partial
open book decompositions and the contact class in sutured Floer homology},
Turkish J. Math. \textbf{33} (2009), 295--312.

\bibitem[Gh]{key-3} P.~Ghiggini, \emph{Tight contact
structures on Seifert manifolds over $T^2$ with one singular fibre},
Algebr. Geom. Topol. \textbf{5} (2005), no. 2, 785--833.

\bibitem[GHH]{key-5} P.~Ghiggini, K.~Honda, and J.~Van Horn-Morris,
\emph{The vanishing of the contact invariant in the presence
of torsion}, arXiv:0706.1602 (2007).

\bibitem[Gi]{key-9} E.~Giroux, \emph{G\'eom\'etrie
de contact: de la dimension trois vers les dimensions sup\'erieures,}
Proceedings of the International Congress of Mathematicians, Vol. II
(Beijing, 2002), Higher Ed. Press, 2002, pp. 405--414.

\bibitem[HKM1]{key-2} K.~Honda, W.~H.~Kazez, and G.~Mati\'c,
\emph{Contact structures, sutured Floer homology and TQFT},
arXiv:0807.2431 (2008).

\bibitem[HKM2]{key-1}K.~Honda, W.~H.~Kazez, and G.~Mati\'{c},
\emph{The contact invariant in sutured Floer homology},
Invent. Math. \textbf{176} (2009), no.~3, 637--676.

\bibitem[Ju1]{key-7} A.~Juh\'asz, \emph{Holomorphic
discs and sutured manifolds}, Algebr. Geom. Topol. \textbf{6} (2006),
no.~3, 1429--1457.

\bibitem[Ju2]{polytope} A.~Juh\'asz, \emph{The sutured Floer polytope}, Geom. Topol. \textbf{14} (2010),
1303--1354.

\bibitem[KMVW]{key-10} C.~Kutluhan, G.~Mati\'c, J.~Van Horn-Morris, and A.~Wand,
\emph{Algebraic torsion via Heegaard Floer homology}, arXiv:1503.01685 (2015).

\bibitem[LWH]{key-11} J.~Latschev, C.~Wendl, and M.~Hutchings,
\emph{Algebraic torsion in contact manifolds},
Geom. Funct. Anal. \textbf{21} (2011), no.~5, 1144--1195.

\bibitem[Li]{key-6} R.~Lipshitz, \emph{A cylindrical
reformulation of Heegaard Floer homology}, Geom. Topol. \textbf{10} (2006),
no. 2, 955--1096.

\bibitem[SW]{key-12} S.~Sarkar and J.~Wang, \emph{An
algorithm for computing some Heegaard Floer homologies.},
Ann. of Math. \textbf{171} (2010), 1213--1236.

\bibitem[Wa]{key-13} A.~Wand, private communication.

\bibitem[We]{Wendl} C.~Wendl, \emph{A hierarchy of local symplectic filling obstructions for contact 3-manifolds},
Duke Math J. \textbf{162} (2013), no. 12, 2197--2283.
\end{thebibliography}

\end{document}


\documentclass[12pt]{amsart}

\usepackage{t1enc}
\usepackage[latin2]{inputenc}
\usepackage[english]{babel}
\usepackage{amssymb}
\usepackage{amsthm}
\usepackage{amsmath}
\usepackage{verbatim}
\usepackage{textcomp}
\usepackage{a4wide}

\usepackage[dvips]{graphicx}

\theoremstyle{plain}
\newtheorem{theorem}{Theorem}[section]
\newtheorem{lemma}[theorem]{Lemma}
\newtheorem{proposition}[theorem]{Proposition}
\newtheorem{corollary}[theorem]{Corollary}
\newtheorem*{problem}{Problem}
\newtheorem*{conjecture}{Conjecture}
\theoremstyle{definition}
\newtheorem{definition}[theorem]{Definition}
\newtheorem{construction}[theorem]{Construction}   
\newtheorem{algorithm}[theorem]{Algorithm}   
\newtheorem{notation}[theorem]{Notation}   
\newtheorem{example}[theorem]{Example}
\newtheorem{examples}[theorem]{Examples}
\newtheorem{remark}[theorem]{Remark}

\date{{December 6, 2013}\\
\small Mathematics Subject Classifications: 03C15, 37B05, 05C55}

\begin{document}

\keywords{finite flow, Ramsey, amenable, measure concentration}

\title{Topological dynamics of unordered Ramsey structures}

\author{Moritz M\"{u}ller}
\address{Kurt G\"odel Research Center (KGRC), Vienna, Austria.}
\email{moritz.mueller@univie.ac.at}

\author{Andr\'as Pongr\'acz}
\address{Laboratoire d'Informatique de l'\'Ecole Polytechnique (LIX), Palaiseau, France. }
\email{andras.pongracz@lix.polytechnique.fr}
\thanks{The second author has received funding from the European Research Council under the European Community's Seventh Framework Programme (FP7/2007-2013 Grant Agreement no. 257039).}

\begin{abstract}
In this paper we investigate the connections between Ramsey properties of Fra\"{\i}ss\'{e} 
classes ${\mathcal K}$ and the universal minimal flow $M(G_{\mathcal K})$ of the automorphism group $G_{\mathcal K}$
of their Fra\"{\i}ss\'{e} limits. As an extension of a result of Kechris, Pestov and Todorcevic \cite{kpt} 
we show that if the class ${\mathcal K}$ has finite Ramsey degree 
for embeddings, then this degree equals the size of $M(G_{\mathcal K})$. 
We give a partial answer to a question of Angel, 
Kechris and Lyons \cite{akl} showing that 
if ${\mathcal K}$ is a relational Ramsey class and $G_{\mathcal K}$ is amenable, then $M(G_{\mathcal K})$ admits a unique
invariant Borel probability measure that is concentrated on a unique generic orbit.
\end{abstract}

 \maketitle

\section{Introduction}

With a Fra\"iss\'e class of finite structures ${\mathcal K}$ one can associate in a natural way a topological group $G_{\mathcal K}$, namely, the automorphism group of the Fra\"iss\'e limit of~${\mathcal K}$. 
For example, the
Fra\"iss\'e limit of finite dimensional vector spaces over a 
fixed finite field $F$ is the $\aleph_0$-dimensional vector space $V_{\infty,F}$ over 
$F$ with automorphism group $\textup{GL}(V_{\infty,F})$.
The groups of the form $G_{\mathcal K}$ are precisely the Polish groups that are non-archimedian in the sense 
that they have a basis at the identity consisting of open subgroups (\cite{kb}). 

In \cite{kpt} Kechris, Pestov and Todorcevic  developed a ``duality theory'' \cite[$\S4$(A)]{kechris} 
linking finite combinatorics of ${\mathcal K}$ with topological dynamics of $G_{\mathcal K}$, more precisely, it links 
combinatorial properties of ${\mathcal K}$ with properties of the universal minimal $G_{\mathcal K}$-flow $M(G_{\mathcal K})$. For groups 
of the form $G_{\mathcal K}$ the flow $M(G_{\mathcal K})$ is an inverse limit of metrizable $G_{\mathcal K}$-flows (cf.~\cite[T1.5]{kpt}), 
and in many interesting cases is metrizable itself. If so, $M(G_{\mathcal K})$ 
either has the size of the continuum or else is finite \cite[$\S 1$(E)]{kpt}.
An extreme case is that $M(G_{\mathcal K})$ is a single point, that is, $G_{\mathcal K}$ is extremely amenable.
It is shown in \cite{kpt} that for ordered ${\mathcal K}$ this happens if and only if ${\mathcal K}$ is Ramsey.
For example, $V_{\infty,F}$ together with the so-called ``canonical order'' has an extremely amenable automorphism group.

We give a characterization of $M(G_{\mathcal K})$ having an arbitrary finite cardinality in terms of Ramsey properties of ${\mathcal K}$. 
Namely, we use Fouch\'e's Ramsey degrees \cite{fouche,fouche2,fouche3} and show that  $M(G_{\mathcal K})$ has finite 
size $d$ if and only if ${\mathcal K}$ has Ramsey degree $d$ (Theorem~\ref{thm:char}). We do not 
assume ${\mathcal K}$ to be ordered, but use Ramsey degrees {\em for embeddings} 
instead (see~e.g.\ \cite{nesetril,bodirsky}). These coincide with the usual Ramsey degrees on rigid 
structures, so our characterization generalizes the mentioned result of~\cite{kpt} and so does its proof. As a corollary we get (Corollary~\ref{cor:asymptotic})
that Ramsey degrees for embeddings are asymptotic in the sense that 
all structures in ${\mathcal K}$ have degree at most $d$ if all {\em large enough} structures have degree at most $d$ (i.e.\ every structure 
embeds into one of degree at most $d$).

Given an appropriate (unordered) class ${\mathcal K}$ one can first produce a so-called reasonable order expansion ${\mathcal K}^*$ whose Fra\"iss\'e 
limit expands the limit of ${\mathcal K}$ by a (linear) order $<^*$. The group $G_{\mathcal K}$ acts naturally on 
orders and one gets a $G_{\mathcal K}$-flow $X_{{\mathcal K}^*}$ as the orbit closure $\overline{G_{\mathcal K}\cdot <^*}$. Again, as shown in \cite{kpt}, 
minimality of this flow corresponds to a combinatorial property of ${\mathcal K}^*$ called the ordering property (cf.~\cite{nesetril}), 
and indeed $X_{{\mathcal K}^*}$ is $M(G_{\mathcal K})$ if and only if ${\mathcal K}^*$ additionally is Ramsey.\footnote{See \cite{nguyen2} for a discussion of how to characterize universality alone.} 
Moreover, the Ramsey degree of $A\in{\mathcal K}$ equals the number of non-isomorphic order expansions it 
has in ${\mathcal K}^*$(\cite[$\S10$]{kpt},\cite[$\S 4$]{nguyen}).

For example, the universal minimal $\textup{GL}(V_{\infty,F})$-flow is the orbit closure of the canonical order. 
This canonical order is forgetful in the sense that any finite dimensional $F$-vector space gets up to isomorphism 
only one order expansion, so Ramsey degrees are 1 in this case. The Ramsey degrees {\em for embeddings} 
on the other hand are unbounded (cf.~Corollary~\ref{cor:emb2}). In general, the relationship between the two degrees is not trivial. 
We show that if a Ramsey class in a relational language has finite Ramsey degree for embeddings, then this degree must be 
a power of 2 (Theorem~\ref{thm:mystery}).
 

Recently, Angel, Kechris and Lyons \cite{akl} extended the duality theory to other important properties of $M(G_{\mathcal K})$, 
namely whether or not there is a (unique) $G_{\mathcal K}$-invariant Borel probability measure on $M(G_{\mathcal K})$. 
In this case, the group $G_{\mathcal K}$ is called amenable (uniquely ergodic), and this happens if and 
only if all minimal $G_{\mathcal K}$-flows admit such a (unique)
measure (\cite[P8.1]{akl}). 
For example, $\textup{GL}(V_{\infty,F})$ is uniquely ergodic.

The $G_{\mathcal K}$-flows $X_{{\mathcal K}^*}$ have a generic (i.e.\ comeager) orbit $G_{\mathcal K}\cdot <^*$ which is in 
fact dense $G_\delta$ \cite[P14.3]{akl}. In many examples, 
a
$G_{\mathcal K}$-invariant measure on $M(G_{\mathcal K})$, if exists, 
turns out to be concentrated on this generic orbit. However, answering a question in~\cite[Q15.3]{akl}, 
Zucker \cite[T1.2]{zucker} showed that the measure on $M(\textup{GL}(V_{\infty,F}))$ is {\em not} concentrated on the generic orbit.

We show that such counterexamples rely on the language containing function symbols. More precisely,
we show that if ${\mathcal K}$ is Ramsey over a relational language and $G_{\mathcal K}$ is amenable, then $G_{\mathcal K}$ is uniquely ergodic 
and the unique $G_{\mathcal K}$-invariant 
Borel probability measure on $M(G_{\mathcal K})$ is indeed concentrated on a dense $G_\delta$ orbit (Theorem~\ref{thm:concentr}).

\section{Preliminaries}

\subsection{Notation} For $k\in {\mathbb N}$ we let $[k]$ denote $\{0,\ldots,k-1\}$ and understand $[0]=\emptyset$. 
If $X,Y$ are sets, $f$ a function from $X$ to $Y$, $n\in {\mathbb N}$ and $Z\subseteq X^n$ we write $f(Z)$ for the 
set $\{f(\bar x)\mid \bar x\in Z\}$ where $f(\bar x)$ denotes the tuple $ (f(x_0),\ldots, f(x_{n-1}))$ 
for $\bar x=(x_0,\ldots, x_{n-1})\in X^n$.
For $X_0\subseteq X$ we let $f\upharpoonright X_0$ denote the restriction of $f$ to $X_0$; for a relation $Z$ 
as above, $Z\upharpoonright X_0$ denotes $Z\cap(X_0^n)$. The identity on $X$ is denoted by $\operatorname{id}_X$.

 
\subsection{Fra\"iss\'e theory}\label{sec:fraisse}
Fix a countable language $L$. We let $A,B,\ldots$ range over ($L$-)struc\-tures. The distinction between 
structures and their universes are blurred notationally. We speak of {\em relational} structures and classes of structures if the underlying language $L$ is relational. We write $A\le B$ to indicate that there exists an embedding from $A$ into $B$, and we let $B^A$ denote the set of embeddings from $A$ into $B$. 

The {\em age} $\operatorname{Age}(F)$ of  a structure $F$ is the class of finitely generated structures which embed into $F$. A structure $F$ is {\em locally finite} if its finitely generated substructures are finite. 
For $A\in\operatorname{Age}(F)$ we call $F$  {\em $A$-homogeneous} if for all $a,a'\in F^A$ there is $g\in\textup{Aut}(F)$ such that $g\circ a=a'$. If $F$ is $A$-homogeneous for all $A\in\operatorname{Age}(F)$, it is {\em (ultra-)homogeneous}. 

A structure $F$ is {\em Fra\"iss\'e} if it is countably infinite, locally finite and homogeneous. The age ${\mathcal K}:=\operatorname{Age}(F)$ of a Fra\"iss\'e structure $F$ 
\begin{itemize}\itemsep=0pt
\item[--] is {\em hereditary:} for all $A,B$, if $A\le B$ and $B\in{\mathcal K}$, then $A\in{\mathcal K}$; 
\item[--] has {\em joint embedding:} for all $A,B\in{\mathcal K}$ there is $C\in{\mathcal K}$ such that both $A\le C$ and $B\le C$; 
\item[--] has {\em amalgamation:} for all $ A, B_0, B_1\in{\mathcal K}$ and $a_0\in B_0^A, a_1\in  B_1^A$ there are $ C\in{\mathcal K}$ and $b_0\in C^{B_0},b_1\in  C^{ B_1}$ such that $b_0\circ a_0=b_1\circ a_1$.
\end{itemize}
A class ${\mathcal K}$ of finite structures that has these three properties and for every $n\in{\mathbb N}$ contains a structure (with universe) of size at least $n$, is a {\em Fra\"iss\'e class}. The following 
is well-known~\cite[T4.4.4]{zieglerbuch}:

\begin{theorem}[Fra\"iss\'e 1954] \label{thm:fraisse}
For every Fra\"iss\'e class ${\mathcal K}$ there exists a Fra\"iss\'e structure $F$ with age ${\mathcal K}$. 
\end{theorem}

A standard back-and-forth argument shows that the structure $F$ in Theorem~\ref{thm:fraisse} is unique up isomorphism; 
it is called the {\em Fra\"iss\'e limit of ${\mathcal K}$} and denoted by $\operatorname{Flim}({\mathcal K})$. 

We mention some standard examples:

\begin{examples}\label{exas:fraisseclasses} The Fra\"iss\'e limit of the class of linear orderings is the rational order $(\mathbb Q,<)$. 
The Fra\"iss\'e limit of the class of finite Boolean algebras is the countable atomless Boolean algebra $B_\infty$. 
The Fra\"iss\'e limit of the class of finite graphs is the random graph $R$. The Fra\"iss\'e limit of 
the class of finite vector spaces over a fixed finite field $F$ is the vector space $V_{\infty,F}$ of 
dimension $\aleph_0$ over $F$.
\end{examples}

We refer to \cite{cameron,cherlin,mcph} as surveys on homogeneous structures.

\subsection{Ramsey degrees} Write ${B \choose A}$ for the set of substructures of $B$ which are isomorphic to $A$. 
Note that ${B'\choose A}\subseteq{C\choose A}$ whenever $B'\in{C\choose B}$.
If $k,d\in{\mathbb N}$ then $C\to (B)^A_{k,d}$ means that for every colouring $\chi:{C\choose A}\to [k]$ 
there exists $B'\in{C\choose B}$ such that $|\chi({B'\choose A})|\le d$. The {\em Ramsey degree} of $A$ in
 a class 
of structures ${\mathcal K}$ is the least $d\in{\mathbb N}$ such that for all $B\in {\mathcal K}$ and $k\ge 2$ there is $C\in{\mathcal K}$ 
such that $C\to (B)^A_{k,d}$ -- provided that such a $d$ exists; otherwise it is $\infty$. 
Taking the supremum over $A\in{\mathcal K}$ gives the Ramsey degree of ${\mathcal K}$, and 
the Ramsey degree of a structure $F$ is understood to be the Ramsey degree of $\operatorname{Age}(F)$; if 
this degree is 1, then ${\mathcal K}$ resp. $F$ are simply called {\em Ramsey}. 

\begin{examples} $(\mathbb Q,<)$, $B_\infty$ and $V_{\infty,F}$ are Ramsey~\cite{kpt}. 
The random graph $R$ has Ramsey degree $\infty$; indeed, a finite graph $G$ has Ramsey 
degree $|G|!/|\operatorname{Aut}(G)|$ in the class of finite graphs~\cite[$\S10$]{kpt}.
\end{examples}

Ramsey degrees have been introduced by Fouch\'e in \cite{fouche}. We refer to 
the surveys \cite{grs, nesetrilsurvey} on Ramsey theory.

\subsection{Topological dynamics} With a Fra\"iss\'e class ${\mathcal K}$ we associate the topological group
$$
G_{\mathcal K}:=\operatorname{Aut}(\operatorname{Flim}({\mathcal K})),
$$
the identity having basic neighborhoods
$$
G_{(A)}:=\{g\in G_{\mathcal K}\mid g\upharpoonright A=\textup{id}_A\}
$$
for all finite substructures $A$ of $\operatorname{Flim}({\mathcal K})$. For any topological group $G$ a {\em $G$-flow} is a continuous action 
$a:G\times X\to X$ of $G$ on a compact Hausdorff space $X$. 
When the action is understood we shall refer to $X$ as a $G$-flow 
and write $g\cdot x$ or $gx$ for $a(g,x)$. For $Y\subseteq X$ we
write $G\cdot Y:=\bigcup_{g\in G}gY=\bigcup_{y\in Y}Gy$ where $Gy:=\{gy\mid g\in G\}$ denotes the {\em orbit of $y$} and 
$gY:=\{gy\mid y\in Y\}$.

\begin{example}\label{exa:lo} Let $G=\operatorname{Aut}(F)$ for a countable structure $F$. 
The {\em space of linear orders (on $F$)} is 
$\textit{LO}:=\{R\subseteq F^2\mid R \textrm{ is a linear order on } F\}$ with topology given by basic open sets
$\{R\mid \ R_0 \subseteq R\}$ for $R_0$ a linear order on a finite 
subset $A$ of $F$. This space is compact and Hausdorff, and a $G$-flow with 
respect to $(g,R)\mapsto g(R)$, the {\em logic action} of $G$ on $\textit{LO}$.
\end{example}

A subset $Y\subseteq X$ is {\em $G$-invariant} if $G\cdot Y\subseteq Y$. 
Closed $G$-invariant subsets $Y$ are $G$-flows with respect to the restriction 
of the action. Such $G$-flows are {\em subflows} of $X$. 
The flow $X$ is {\em minimal} if $X$ and $\emptyset$ are its only subflows, 
that is, if and only if every orbit is dense. By Zorn's lemma, every $G$-flow contains a minimal subflow.
A {\em homomorphism (isomorphism)} of a $G$-flow $X$ into another $Y$ is a continuous (bijective) 
$G$-map $\pi:X\to Y$; being a $G$-map means that $\pi(g\cdot x)=g\cdot \pi(x)$ for all $g\in G,x\in X$.

The following is well-known (cf.~\cite[$\S 3$]{usp}). 

\begin{theorem} For every  Hausdorff  topological group $G$ there exists a minimal $G$-flow $M(G)$ which is {\em universal} in the sense that for every minimal $G$-flow $Y$ there is a homomorphism from $X$ into $Y$. Any two universal minimal $G$-flows are isomorphic. 
\end{theorem}

An interesting case is that $|M(G)|=1$, equivalently, every $G$-flow $X$ has a fixed point, i.e. 
an $x\in X$ such that $G\cdot x=\{x\}$. In this case $G$ is called {\em extremely amenable}. 
Being {\em amenable} means that there exists a (Borel probability) measure $\mu$ on $M(G)$ which 
is {\em $G$-invariant} (i.e. $\mu(X)=\mu(g\cdot X)$ for every Borel $X\subset M(G)$ and $g\in G$). 
If there is exactly one such measure then
$G$ is {\em uniquely ergodic}. It is shown in \cite[P8.1]{akl} that for a uniquely ergodic  $G$ 
in fact every minimal $G$-flow has a unique $G$-invariant measure. 

We refer to \cite[$\S 1$]{kpt} for a survey on universal minimal flows.

\subsection{Duality theory} Let $<$ be a binary relation symbol.  
A class ${\mathcal K}^*$ of finite $L\cup\{<\}$-structures is {\em ordered} if each of its members 
has the form $(A,<^A)$ for a (linear) order $<^A$ (on $A$) and some finite $L$-structure $A$; the 
order $<^A$ is called a {\em ${\mathcal K}^*$-admissible} one (cf.~\cite{nesetril}). 

The following is \cite[T4.8]{kpt}.

\begin{theorem}\label{thm:extremeamenability} Assume that ${\mathcal K}^*$ is an ordered Fra\"iss\'e class.
Then $G_{{\mathcal K}^*}$ is extremely amenable if and  only if ${\mathcal K}^*$ is Ramsey.
\end{theorem}

Let ${\mathcal K}:=\{A\mid (A,<^A)\in {\mathcal K}^*\}$ be the {\em $L$-reduct} of ${\mathcal K}^*$; ${\mathcal K}^*$ is {\em reasonable} if 
for all $A,B\in{\mathcal K}$, all $a\in B^A$ and all ${\mathcal K}^*$-admissible orders $<^A$ on $A$ 
there is a ${\mathcal K}^*$-admissible order $<^B$ on $B$ such that $a(<^A)\subseteq <^B$, i.e. $a\in (B,<^B)^{(A,<^A)}$. 

\begin{lemma}\label{lem:reason}
Let ${\mathcal K}$ be a Fra\"iss\'e class and let $F=\operatorname{Flim}({\mathcal K})$. Then ${\mathcal K}^*=\operatorname{Age}(F, R)$ is reasonable for every order $R$ on $F$.
\end{lemma}
\begin{proof}
Let $A,B\in{\mathcal K}$, $a\in B^A$ and $<^A$ be a ${\mathcal K}^*$-admissible order on $A$. Let $a_0\in (F, R)^{(A, <^A)}$ and $b\in F^B$. In particular, $a_0\in F^A$ and $b\circ a \in F^A$, and then by homogeneity of $F$ there exists an $\alpha\in\operatorname{Aut}(F)$ such that $\alpha\circ b\circ a = a_0$. We define 
$$<^B := (b^{-1}\circ \alpha^{-1})(R\upharpoonright (\alpha\circ b)(B))$$

We need to show that $a^{-1}(<^B\upharpoonright a(A)) = <^A$. We have that 
\begin{multline*}
a^{-1}(<^B\upharpoonright a(A)) = a^{-1}((b^{-1}\circ \alpha^{-1})(R\upharpoonright (\alpha\circ b)(B))\upharpoonright a(A)) =\\ a^{-1}((b^{-1}\circ \alpha^{-1})(R\upharpoonright (\alpha\circ b)(a(A)))) = a_0^{-1}(R\upharpoonright a_0(A))= <^A
\end{multline*}

The last equality holds as $a_0\in (F, R)^{(A, <^A)}$.
\end{proof}

The following is \cite[P5.2,~T10.8]{kpt}. Recall that $\textit{LO}$ denotes the space of orders (Example~\ref{exa:lo}).

\begin{theorem}\label{thm:minflow} Let ${\mathcal K}^*$ be a reasonable ordered Fra\"iss\'e class in the language $L\cup\{<\}$ 
and ${\mathcal K}$ its $L$-reduct. 
\begin{enumerate}
\item Then ${\mathcal K}$ is  Fra\"iss\'e and $\operatorname{Flim}({\mathcal K}^*)=(\operatorname{Flim}({\mathcal K}),<^*)$ for some linear order $<^*$. 
\item Let $X_{{\mathcal K}^*}:=\overline{G_{\mathcal K}\cdot <^*}$ be the orbit closure of $<^*$ in the logic action of $G_{\mathcal K}$ on  
$\textit{LO}$. Then $X_{{\mathcal K}^*}$ is the universal minimal $G_{\mathcal K}$-flow if and only if ${\mathcal K}^*$ is Ramsey and has the ordering property.
\end{enumerate}
\end{theorem}

That ${\mathcal K}^*$  has the {\em ordering property} means that for all $A\in{\mathcal K}$ there is a $B\in{\mathcal K}$ such that
$(A,<^A)\le(B,<^B)$ for all ${\mathcal K}^*$-admissible orders $<^A$ on $A$ and $<^B$ on $B$.

In \cite {akl} Kechris et al. showed that a certain quantitative version of the ordering property characterizes unique ergodicity 
for so-called Hrushovski classes. Here, we shall only need the following \cite[P9.2]{akl}.

\begin{proposition} \label{prop:randomorder} Let ${\mathcal K}^*$ be a reasonable ordered Fra\"iss\'e 
class which is Ramsey and satisfies the ordering property, and let ${\mathcal K}$ be its $L$-reduct. 
Then $G_{\mathcal K}$ is amenable (uniquely ergodic)
if and only if there exists a
consistent random ${\mathcal K}^*$-admissible ordering $(R_A)_{A\in{\mathcal K}}$ (and for every other consistent 
random ${\mathcal K}^*$-admissible ordering $(R'_A)_{A\in{\mathcal K}}$ we 
have that $R_A$ and $R'_A$ have the same distribution for every $A\in{\mathcal K}$). 

Indeed, if  $(R_A)_{A\in{\mathcal K}}$ is a random ${\mathcal K}^*$-admissible ordering, then
there is a $G_{\mathcal K}$-invariant Borel probability measure $\mu$ on $X_{{\mathcal K}^*}$ 
such that for every $A\in{\mathcal K}$ and every 
${\mathcal K}^*$-admissible ordering $<$ on $A$ we have\footnote{Given a random variable we always use $\Pr$ to 
denote the probability measure of its underlying probability space.} 
$\mu(U(<))=\Pr[R_A=<]$ where $$U(<):=\{R\in X_{{\mathcal K}^*}\mid R\upharpoonright A=<\}.
$$
\end{proposition}

A {\em random ${\mathcal K}^*$-admissible ordering} is a family $(R_A)_{A\in{\mathcal K}}$ of random variables 
such that each $R_A$ takes values in the set of ${\mathcal K}^*$-admissible 
orders on~$A$. It is {\em consistent} if 
for all $A,B\in{\mathcal K}$ and $a\in B^A$ the random variables $a^{-1}(R_B\upharpoonright {\mathrm{im}}(a))$ and $R_A$ have the same distribution.

\begin{examples}\label{exas} In \cite[$\S6$]{kpt} the reader can find constructions of reasonable 
ordered Fra\"iss\'e classes ${\mathcal K}^*$ whose reduct ${\mathcal K}$ is any of the classes mentioned in Example~\ref{exas:fraisseclasses}; 
in all these cases ${\mathcal K}^*$ is Ramsey and has the ordering property. By Theorem~\ref{thm:extremeamenability} one sees
that the automorphism groups of $(\mathbb Q, <)$ and of certain ordered versions of $B_\infty,R,V_{\infty,F}$ are 
extremely amenable~\cite{kpt}. Theorem~\ref{thm:minflow} allows  to calculate the universal minimal flows of the automorphism 
groups of $B_\infty,R$ and $V_{\infty,F}$. $\operatorname{Aut}(B_\infty)$ is not amenable, while $\operatorname{Aut}(R)$ and $\operatorname{Aut}(V_{\infty,F})$ are uniquely ergodic~\cite{akl}.
\end{examples}

\section{Automorphism groups with finite universal minimal flows}

Theorem~\ref{thm:extremeamenability} characterizes the condition that the universal minimal flow has size 1. In this section we provide a similar characterization for the condition that it has an arbitrary finite size.
To this end we consider Ramsey degrees {\em for embeddings}. The main result in this section reads:

\begin{theorem}\label{thm:char}
Let $d\in{\mathbb N}$ and ${\mathcal K}$ be a Fra\"iss\'e class.
The following are equivalent.
\begin{enumerate}
\item $M(G_{\mathcal K})$ has size at most $d$;
\item  the Ramsey degree for embeddings of ${\mathcal K}$ is at most $d$.
\end{enumerate}
\end{theorem}

We start with some preliminary observations concerning finite universal minimal flows in Section~\ref{sec:finiteflow}. 
In Section~\ref{sec:remb} we define Ramsey degrees for embeddings and discuss their relationship to Ramsey 
degrees. The results proved in Sections~\ref{sec:finiteflow} and \ref{sec:remb} are mainly folklore.
In Section~\ref{sec:proof} we prove the result above and in 
Section~\ref{sec:cor}  we note some corollaries.

\subsection{Finite universal minimal flows}\label{sec:finiteflow}

\begin{lemma}\label{lem:orb}
 Let $G$ be a topological Hausdorff group and $d\in\mathbb N$. 
Then $M(G)$ has size at most $d$ if and only if every nonempty $G$-flow has an orbit of size at most $d$.
\end{lemma}

\begin{proof}
Assume that $|M(G)|\le d$, and let $X$ be a nonempty $G$-flow. Then there is a minimal subflow $X'$ of $X$ and 
a homomorphism $\pi$ of $M(G)$ onto $X'$. Thus $|X'|\leq d$.

Conversely, if every nonempty $G$-flow has an orbit of size at most $d$, then so does $M(G)$. Since $M(G)$ is 
minimal, this orbit is dense in $M(G)$, so it is equal to $M(G)$ by finiteness.  
\end{proof}

\begin{lemma}\label{lem:todorpart}
 Let $G$ be a topological Hausdorff group and $H$ an extremely amenable closed subgroup of $G$ with finite index. Then $H$ is a 
normal clopen subgroup of $G$ and $M(G)$ is isomorphic to the action of $G$ on $G/H$ by left multiplication.
\end{lemma}

\begin{proof} Clearly, a closed subgroup of finite index is open. We first show that $G/H$ is the universal minimal $G$-flow.
Since $H$ is open $G/H$ is discrete, and as $|G:H|$ is finite, $G/H$ is compact. Hence, $G/H$ is a $G$-flow. 
It is minimal, because $G$ acts 
transitively on $G/H$. If $Y$ is an arbitrary $G$-flow, 
then its restriction to $H$ is an $H$-flow, so it has a fixed point $y\in Y$. Then $gH\mapsto gy$ is
a homomorphism from $G/H$ into $Y$. 

As $gHg^{-1}$ is a closed subgroup of finite index for every $g\in G$, 
so is $H'=H\cap gHg^{-1}$. As above, we see that $G/H'$ is a minimal $G$-flow. By universality of $G/H$ 
there exists a surjection from $G/H$ onto $G/H'$, so $|G:H'|\le |G:H|$. Thus $H= gHg^{-1}$ for every $g\in G$, 
that is, $H$ is normal.
\end{proof}

\begin{proposition}\label{prop:todor}  Let $G$ be a topological Hausdorff group and $d\in\mathbb N$. Then 
$M(G)$ has size~$d$ if and only if 
$G$ has an extremely amenable, open, normal subgroup of index $d$.
\end{proposition}

\begin{proof}
The backward direction follows from  Lemma~\ref{lem:todorpart}. Conversely, assume that  $X:=M(G)$ has size $d$.
For $x\in X$ let $H_x\leq G$ be the stabilizer of $x$.
Then there is a bijection between the set of left cosets of $H_x$ and the orbit $G\cdot x$. Since $G\cdot x$ is finite, 
$G\cdot x=\overline{G\cdot x}$, so $G\cdot x=X$ by minimality.
Hence, $|G:H_x|=|X|=d$.  As $H_x$ is closed and of finite index, so is 
$N:=\bigcap_{x\in X}H_x$, and hence $N$ is clopen.
Since $N$ is the pointwise stabilizer of~$X$, it is normal. 
 

Let $Y$ be a minimal $N$-flow. 
Let $\tau: G/N\rightarrow G$ be a function with $\tau(hN)\in hN$.
Define $a: G \times G/N \rightarrow N$ by setting
$$
a(g, hN):= \tau(hN)^{-1}\cdot g^{-1}\cdot \tau(ghN).
$$
A straightforward calculation shows that $a$ satisfies the so-called cocycle identity, that is, for all $g_1,g_2,h\in G$ 
\begin{equation}\label{eq:cocycle}
a(g_1 g_2, hN) = a(g_2, hN)\cdot a(g_1, g_2 hN). 
\end{equation}

We can construct an action of $G$ on $(G/N \times Y)$ by
$$
(g,(hN, y)) \mapsto (ghN, a(g, hN)^{-1}\cdot y).
$$
That this indeed defines a group action follows directly from~\eqref{eq:cocycle}.
The action is continuous and $(G/N \times Y)$ is compact, so $(G/N \times Y)$ is a $G$-flow.

Let $h\in G,y\in Y$ be arbitrary.
We show that
\begin{equation}\label{eq:section}
Y(h,y):=\{a(n, hN)^{-1}\cdot y\mid n\in N\}\textup{ is dense in } Y.
\end{equation}
Indeed, as $N$ is normal, we have $Y(h,y)=\tau(hN)^{-1} \cdot N \cdot \tau(hN) \cdot y = N\cdot y$. Since $Y$ is a minimal $N$-flow, the orbit $N\cdot y$ is dense in $Y$. 

The orbit $G\cdot (hN,y)$ contains $N\cdot (ghN,y')$ for every $g\in G$ and $y':=a(g, hN)^{-1}\cdot y$. But
$N\cdot (ghN,y')=\{(nghN, a(n, hN)^{-1}\cdot y')\mid n\in N\}=\{ghN\}\times Y(h,y')$, where the last equality 
holds because $N$ is normal. So the orbit $G\cdot (hN,y)$ contains $\bigcup_{g\in G}(\{gN\}\times Y(h,y_g))$ for certain $y_g$'s, 
and this set is dense in $(G/N\times Y)$ by~\eqref{eq:section}. Thus $(G/N\times Y)$ is a minimal $G$-flow.

By the universality of $X$ there exists a surjection from $X$ onto  $(G/N \times Y)$. 
In particular, $|G/N\times Y|\le d$. By definition  of $N$ we have $|G:N|\ge d$, 
so $|Y|=1$, $|G/N|=d$.
This means $N$ is extremely amenable and has index $d$ in $G$.
\end{proof}

\begin{example}\label{ex:flowd}
For $d\in \mathbb{N}$ let $G^*$ be the automorphism group of $(\mathbb Q, <, 0,1, \ldots, d-1)$, the structure  
with universe $\mathbb Q$ that interprets for all $i\in[d]$ a constant by $i$ and a 
binary relation symbol $<$ by the rational order. Let $G$ be the group 
generated by $G^*$ and the permutation $\alpha=(0\ 1 \ldots\ d-1)$. 
This is a closed subgroup of the group of all permutations of $\mathbb Q$, so $G=G_{\mathcal K}$ for some Fra\"iss\'e class ${\mathcal K}$ 
(see e.g.~\cite{kb}). 
Since $\alpha$ commutes with $G$, $G^*$ is normal in~$G$. Moreover, $G^*$ has index $d$ in $G$, and it follows from
\cite[L13]{bpt} (see also \cite[P24]{bodpin}) that~$G^*$ is extremely amenable. 
By Lemma~\ref{lem:todorpart}, $|M(G)|=|G/G^*|=d$.
\end{example}

\begin{example} 
Let $G$ be the automorphism group of $(\mathbb Q,E_d,<)$ where $<$ is the rational order and $E_d$ is an equivalence relation with $d$ classes each of which 
is dense in $(\mathbb Q,<)$. Let $H$ be the subgroup of $G$ consisting of those automorphisms that preserve each of the classes.
It is shown in \cite[T8.4]{kpt} that $H$ is extremely amenable and of index $d!$ in $G$.
By Lemma~\ref{lem:todorpart}, $|M(G)|=|G/H|=d!$.
\end{example}

\subsection{Ramsey degrees for embeddings}\label{sec:remb}
Let $k,d\in{\mathbb N}$ and ${\mathcal K}$ be a class of finite structures. Then $C\hookrightarrow (B)^A_{k,d}$ means that for every colouring $\chi:C^A\to [k]$ there exists a $b\in C^B$ such that $|\chi(b\circ B^A)|\le d$.  Naturally here, $b\circ B^A$ denotes $\{b\circ a\mid a\in B^A\}$.
The {\em Ramsey degree for embeddings} of $A$ in ${\mathcal K}$ is the least $d\in{\mathbb N}$ such 
that for all $B\in {\mathcal K}$ and $k\ge 2$ there is a $C\in{\mathcal K}$ such that $C\hookrightarrow (B)^A_{k,d}$ --
provided that such a $d$ exists; otherwise it is $\infty$. 
Taking the supremum over $A\in{\mathcal K}$ gives the Ramsey degree for embeddings of ${\mathcal K}$. If this degree is~1 we call ${\mathcal K}$ {\em Ramsey for embeddings}.

\begin{lemma}\label{lem:infinite2} Let $d\in{\mathbb N}$, ${\mathcal K}$ be a Fra\"iss\'e class, $F=\operatorname{Flim}({\mathcal K})$ and $A\in{\mathcal K}$. 
The Ramsey degree for embeddings of $A$ in ${\mathcal K}$ is at most $d$ if and only if $F\hookrightarrow (B)^A_{k,d}$ 
for all $B\in{\mathcal K}$ and~$k\ge 2$.
\end{lemma}

\begin{proof} Assume that the Ramsey degree for embeddings of $A$ in ${\mathcal K}$ is at most $d$. Let $B\in{\mathcal K}, k\ge 2$ and 
$\chi:F^A\to[k]$. We are looking for $b'\in F^B$ such that $|\chi(b'\circ B^A)|\le d$.
Choose $C\in{\mathcal K}$ such that $C\hookrightarrow (B)_{k,d}^A$. Choose a $c\in F^C$ and let 
$\chi':C^A\to[k]$ map $a\in C^A$ to $\chi(c\circ a)$. By $C\hookrightarrow (B)_{k,d}^A$ there is a $b\in C^B$ 
such that $|\chi'(b\circ B^A)|\le d$, i.e. $|\chi(c\circ b\circ B^A)|\le d$. Then $b':=c\circ b\in F^B$ is as desired.

Assume  that there is an $A\in{\mathcal K}$ whose Ramsey degree for embeddings is bigger than $d$. Choose  $B\in{\mathcal K}, k\ge 2$ such 
that for every finite substructure $C$ of $F$ there is a colouring 
$\chi:C^A\to[k]$ which is {\em good for $C$}, i.e.\ $|\chi(b\circ B^A)|> d$ for all $b\in C^B$. 
The set $G(C):=\{\chi\in [k]^{F^A}\mid \chi\upharpoonright C^A\textrm{ is good for } C\}$ is nonempty and closed in $[k]^{F^A}$ carrying 
the product topology with $[k]$ being discrete. Given finitely many such sets $G(C_1),\ldots,G(C_n)$ their intersection 
contains the nonempty set $G(C)$ where $C$ is the substructure generated by $C_1\cup\ldots\cup C_n$ in $F$ 
(note that $C$ is finite by local finiteness of $F$). Since $[k]^{F^A}$ is compact, 
$\bigcap_CG(C)\neq\emptyset$ where $C$ ranges over the finite substructures of $F$. 
Any $\chi\in \bigcap_CG(C)$ is good for $F$, so $F\not\hookrightarrow (B)^A_{k,d}$. 
\end{proof}

We shall need the following result of Ne\v{s}et\v{r}il~\cite[T3.2]{nesetril}. 
We include the short proof.

\begin{lemma}\label{lem:Ramam} Let ${\mathcal K}$ be a hereditary class of finite structures with joint embedding. If ${\mathcal K}$ is Ramsey for embeddings, then it has amalgamation.
\end{lemma}

\begin{proof} Let $A,B_0,B_1\in{\mathcal K}$ and $a_0\in B_0^A, a_1\in B_1^A$. Let $B\in{\mathcal K}$ and $b_0\in B^{B_0},b_1\in B^{B_1}$. Choose $C\in {\mathcal K}$ with $C\hookrightarrow(B)_{4,1}^{A}$. We claim that there exist $e_0\in C^{B_0},e_1\in C^{B_1}$ such that $e_0\circ a_0=e_1\circ a_1$. Consider the following colouring $\chi:C^A\to P(\{0,1\})$: for $a\in C^A$ the colour $\chi(a)\subseteq \{0,1\}$ contains $i\in\{0,1\}$ if and only if there exists an $e\in C^{B_i}$ such that $e\circ a_i=a$. Choose $b\in C^B$ such that $\chi(b\circ B^A)$ contains precisely one colour. 
Then this colour is $\{0,1\}$, because for $i\in\{0,1\}$ we have $i\in\chi(b\circ b_i \circ a_i)$ and $b\circ b_i \circ a_i\in b\circ B^A$. Let $a\in B^A$. Then $\chi(b\circ a)=\{0,1\}$, thus there are $e_0\in C^{B_0},e_1\in C^{B_1}$ such that $e_0\circ a_0=a=e_1\circ a_1$.
\end{proof}

\begin{remark}\label{rem:rigiddegrees} Clearly, $C\hookrightarrow (B)^A_{k,d}$ is equivalent to $C\rightarrow (B)^A_{k,d}$ when $A$ is rigid (i.e. $\operatorname{Aut}(A)=\{\operatorname{id}_A\}$). In particular, the Ramsey degree and the Ramsey degree for embeddings coincide for rigid structures. The following proposition generalizes this observation.
\end{remark}

\begin{proposition}\label{prop:emb} Let $d\in{\mathbb N}$, and let ${\mathcal K}$ be a class of finite structures. Let $A\in{\mathcal K}$ and $\ell=|\operatorname{Aut}(A)|$. 
The Ramsey degree for embeddings of $A$ in ${\mathcal K}$ is at most $d\cdot\ell$ if and only if 
the Ramsey degree of $A$ in ${\mathcal K}$ is at most $d$. 
\end{proposition}

\begin{proof} First assume that the Ramsey degree for embeddings of $A$ in ${\mathcal K}$ is at most $d\cdot\ell$. Let $B\in{\mathcal K}$ and $k\ge 2$. We are looking for a $C\in{\mathcal K}$ such that $C\to(B)_{k,d}^A$. By assumption we find some $C\in{\mathcal K}$ with $C\hookrightarrow(B)^A_{k,d\cdot\ell}$ and we claim that this $C$ is as desired. Let a colouring $\chi:{C\choose A}\to[k]$ be given.
For every $A'\in {C\choose A}$ there are precisely $\ell$ embeddings $a^{A'}_0,\ldots, a^{A'}_{\ell-1}\in C^A$ with image $A'$. Define $\chi':C^A\to[k]\times[\ell]$ to map $a\in C^A$ to $(i,j)$ for $i:=\chi({\mathrm{im}}(a))$ and $j$ such that
$a=a_j^{{\mathrm{im}}(a)}$. Since  $C\hookrightarrow(B)^A_{k,d\cdot\ell}$ there is $b\in C^B$ such that $|\chi'(b\circ B^A)|\le d\cdot\ell$. 
Observe that $(i,j)\in\chi'(b\circ B^A)$ implies $\{i\}\times[\ell]\subseteq\chi'(b\circ B^A)$. Hence, there are (not necessarily distinct) $i_0,\ldots,i_{d-1}\in[k]$ such that $\chi'(b\circ B^A)=\{i_0,\ldots,i_{d-1}\}\times[\ell]$.
Clearly, ${\mathrm{im}}(b)\in{C\choose B}$ and we claim that $\chi({{\mathrm{im}}(b)\choose A})\subseteq \{i_0,\ldots,i_{d-1}\}$. Indeed, for $A'\in{{\mathrm{im}}(b)\choose A}$ there is an $a\in B^A$ such that ${\mathrm{im}}(b\circ a)=A'$, namely $a:= b^{-1}\circ a'$ for some isomorphism $a':A\rightarrow A'$. As ${{\mathrm{im}}(b)\choose A}\subseteq {C\choose A}$ we find $j\in[\ell]$ such that $a_j^{A'}=b\circ a$. Then $\chi'(b\circ a)=(\chi(A'),j)$, and in particular $\chi(A')\in \{i_0,\ldots,i_{d-1}\}$.

Conversely, assume that the Ramsey degree of $A$ in ${\mathcal K}$ is at most $d$. Let $B\in{\mathcal K}$ and $k\ge 2$ be given. 
By assumption there exists a $C\in{\mathcal K}$ such that $C\to(B)^A_{k^\ell,d}$. We claim that $C\hookrightarrow(B)_{k,d\cdot\ell}^A$.
Let $\chi:C^A\to[k]$ be a colouring and define $\chi':{C\choose A}\to[k]^{\ell}$ by setting
$\chi'(A'):=(\chi(a_0^{A'}),\ldots,\chi(a_{\ell-1}^{A'}))$ for $A'\in{C\choose A}$; here, for $A'\in{C\choose A}$ we let $a_0^{A'},\ldots,a_{\ell-1}^{A'}$ enumerate the embeddings in $C^A$ with image $A'$. Since $C\rightarrow(B)_{k^{\ell},d}^A$ there exists $B'\in{C\choose B}$ and $(i^{0}_0,\ldots, i^0_{\ell-1}),\ldots,(i^{d-1}_0,\ldots, i^{d-1}_{\ell-1})\in[k]^{\ell}$ 
such that $\chi'({B'\choose A})\subseteq\{(i^\nu_0,\ldots,i^\nu_{\ell-1})\mid \nu\in[d]\}$. 
Choose $b \in C^B $ with image $B'$. We claim that $\chi(b\circ B^A)\subseteq\{i^{\nu}_j\mid \nu\in[d],j\in[\ell]\}$.
Let $a\in B^A$. Then $b\circ a\in C^A$ and ${\mathrm{im}}(b\circ a)\in {B'\choose A}\subseteq {C\choose A}$. Choose $j\in[\ell]$ such that
$b\circ a=a_j^{{\mathrm{im}}(b\circ a)}$. Let $\nu\in[d]$ be such that 
$\chi'({\mathrm{im}}(b\circ a))=(\chi(a_0^{{\mathrm{im}}(b\circ a)}),\ldots,\chi(a_{\ell-1}^{{\mathrm{im}}(b\circ a)}))=
(i^\nu_0,\ldots,i^\nu_{\ell-1})$. Hence, $\chi(b\circ a)=\chi(a_{j}^{{\mathrm{im}}(b\circ a)})=i^{\nu}_j$.
\end{proof}

\begin{corollary}\label{cor:emb2} Let ${\mathcal K}$ be a class of finite structures and $A\in {\mathcal K}$. Then
the Ramsey degree of $A$ in ${\mathcal K}$ is 1 if and only if the Ramsey degree for embeddings of $A$ in ${\mathcal K}$ is $|\operatorname{Aut}(A)|$. 
\end{corollary}

\begin{proof}
By Proposition~\ref{prop:emb} is suffices to show that the Ramsey degree for embeddings of $A$ in ${\mathcal K}$ is at least 
$\ell:=|\operatorname{Aut}(A)|$. Let $C\in {\mathcal K}$ be arbitrary. Using the notation from the previous proof, let $\chi:C^A\to[\ell]$ map 
$a\in C^A$ to the $j<\ell$ such that $a=a_j^{{\mathrm{im}}(a)}$. Then for $B:= A$ and 
every $b\in C^B$ we have $\chi(b\circ C^B)=[\ell]$.
\end{proof}

Our main result concerning the relationship of Ramsey degrees and Ramsey degrees for embeddings is the following.

\begin{theorem}\label{thm:mystery}
 Let ${\mathcal K}$ be a relational Fra\"iss\'e class which is Ramsey. Then the 
Ramsey degree for embeddings of ${\mathcal K}$ is infinite or a finite power of 2.
\end{theorem}

We refer to Examples~\ref{exas:rationals} for some natural 
examples of relational Fra\"iss\'e classes which are Ramsey and have infinite Ramsey degree for embeddings.
We prove Theorem~\ref{thm:mystery} in Section~\ref{sec:cor2}.

\subsection{Proof of Theorem~\ref{thm:char}} \label{sec:proof}
Theorem~\ref{thm:char} is a consequence of the following 
two propositions which in fact establish something stronger.

We say that a class of finite structures $\mathcal D$ is {\em cofinal in} another 
such class ${\mathcal K}$ if for all $A\in{\mathcal K}$ there exists $B\in\mathcal D$ such that $A\le B$.

\begin{proposition}\label{prop:ramtoam} Let $d\in{\mathbb N}$ and ${\mathcal K}$ be a Fra\"iss\'e class.
Assume that the class of structures with Ramsey degree  for embeddings  at most $d$ in ${\mathcal K}$ is cofinal in ${\mathcal K}$.
Then $M(G_{\mathcal K})$ has size at most $d$.
\end{proposition}

\begin{proof}
Write $G:=G_{\mathcal K}$ and $F:=\operatorname{Flim}({\mathcal K})$. Let $A\in{\mathcal K}, a_0\in F^A$ and write $A_0:={\mathrm{im}}(a_0)$. 
Consider the map $\Phi: G\rightarrow F^A$, $g\mapsto g\circ a_0$ . 
By homogeneity of $F$, $\Phi$ is surjective. We have for all $g,h\in G$
$$g\circ a_0=h\circ a_0\Longleftrightarrow gG_{(A_0)}=hG_{(A_0)}.$$
Hence, $\Phi$ induces a bijection ${\mathrm{e}}$ from $G/G_{(A_0)}$ onto $F^A$. Observe that 
\begin{equation}\label{eq:e}
g\circ {\mathrm{e}}(hG_{(A_0)})=g\circ (h \circ a_0)= (gh)\circ a_0={\mathrm{e}}((gh)G_{(A_0)}). 
\end{equation}

\noindent{\em Claim 1.} Assume that $A$ has Ramsey degree  for embeddings at most $d$ in ${\mathcal K}$. 
Let $k\in{\mathbb N}$ and $f:G\to [k]$ be constant on each $gG_{(A_0)}\subseteq G$ for $g\in G$. Then, for every finite $H\subseteq G$ there exists $g\in G$ such that $|f(gH)|\le d$.\medskip

\noindent{\em Proof of Claim 1:} The function $f$ induces a function $\tilde f$ from 
$G/G_{(A_0)}$ to $[k]$. Note that $\tilde f\circ {\mathrm{e}}^{-1}:F^A\to [k]$. 
There is a finite substructure $B\subseteq F$ such that  
\begin{equation}\label{eq:B}
\{{\mathrm{e}}(hG_{(A_0)})\mid h\in H\}\subseteq B^A.
\end{equation}
By Lemma~\ref{lem:infinite2} there is $b\in F^B$ such that $|(\tilde f\circ {\mathrm{e}}^{-1})(b\circ B^A)|\le d$.
By homogeneity of~$F$ there is a $g\in G$ such that $g\circ \textup{id}_B=b$.
We show that $g$ is as desired, namely $f(gh)\in (\tilde f\circ {\mathrm{e}}^{-1})(b\circ B^A)$ for every $h\in H$: 
\begin{align*}
f(gh)=\tilde f((gh)G_{(A_0)})&=\tilde f\circ {\mathrm{e}}^{-1} ({\mathrm{e}}((gh)G_{(A_0)}))
=\tilde f\circ {\mathrm{e}}^{-1} (g \circ {\mathrm{e}}(hG_{(A_0)})  )
\end{align*}
where the last equality follows from \eqref{eq:e}. By \eqref{eq:B} we have 
$g\circ {\mathrm{e}}(hG_{(A_0)})\in g\circ B^A= b\circ B^A$, and our claim follows.\hfill$\dashv$\medskip

For $n\in{\mathbb N},n\ge 1,$ consider $\mathbb R^n$ with the Euclidian norm $\|\cdot\|$. For $\varepsilon>0$ and $x\in\mathbb R^n$ let $$B_{\varepsilon}(x) :=\{y\in \mathbb R^n\mid \|x-y\|<\varepsilon\}.$$

As a topological group $G$ carries its {\em left} uniformity, that is, the uniformity with basic entourages $\{(g,h)\mid g^{-1}h\in G_{(A)}\}$ for $A\in\operatorname{Age}(M), A\subseteq M$.
\medskip

\noindent{\em Claim 2.} Let $n$ be a positive integer, $f:G\to \mathbb R^n$ be left uniformly continuous and bounded, $H\subseteq G$ be finite and $\varepsilon$ be a positive real.
Then there are $g\in G$ and $h_0,\ldots,h_{d-1}\in H$ such that 
\begin{equation}\label{eq:claim2}
\textstyle
f(gH)\subseteq \bigcup_{\nu<d}B_{\varepsilon}(f(gh_\nu)).
\end{equation}

\noindent{\em Proof of Claim 2:} By left uniform continuity of $f$ there is a finite substructure  
$A'\subseteq F$ such that $\|f(g)- f(g')\|<\varepsilon/6$ for all $g,g'\in G$ with $gG_{(A')}=g'G_{(A')}$. 
By our cofinality assumption, there exist $A''\in{\mathcal K}$ and $a'\in (A'')^{A'}$ such that $A''$ 
has Ramsey degree for embeddings at most $d$ in ${\mathcal K}$. Since $F$ is homogeneous, there is an embedding $a''\in F^{A''}$ such that $a''\circ a'=\operatorname{id}_{A'}$. Hence, the image $A$ of $a''$ has Ramsey degree for embeddings at most $d$ in ${\mathcal K}$, and $A'\subseteq A\subseteq F$. Thus $G_{(A)}\subseteq G_{(A')}$, so
for all $g,g'\in G$ with $gG_{(A)}=g'G_{(A)}$
\begin{equation}\label{eq:A}
\|f(g)- f(g')\|<\varepsilon/6.
\end{equation}

We claim that there exists a function $\tilde f:G\to \mathbb R^n$ such that
\begin{enumerate}\itemsep=0pt
\item[(a)] ${\mathrm{im}}(\tilde f)$ is finite;
\item[(b)] $\tilde f$ is constant on $gG_{(A)}$ for every $g\in G$;
\item[(c)] $\|f(g)-\tilde f(g)\|<\varepsilon/2$ for every $g\in G$.
\end{enumerate}
By (a) and (b) we can apply Claim 1 and obtain some $g\in G$ such that $|\tilde f(gH)|\le d$. Choose $h_0,\ldots, h_{d-1}\in H$
such that $\tilde f(gH)=\{\tilde f(gh_\nu) \mid \nu<d\}$. To verify~\eqref{eq:claim2}, let $h\in H$ be given.
We have to show that there exists $\nu<d$ such that  
$\|f(gh)-f(gh_\nu)\|<\varepsilon$. Indeed, this holds for $\nu<d$ such that $\tilde f(gh)=\tilde f(gh_\nu)$, because by (c) 
we have both $\|f(gh)-\tilde f(gh_\nu)\|=\|f(gh)-\tilde f(gh)\|<\varepsilon/2$ and
$\|\tilde f(gh_\nu)-f(gh_\nu)\|<\varepsilon/2$.

Thus, we are left to find $\tilde f$ with properties (a)-(c).

As $f$ is bounded, its image is contained in a compact subset of $\mathbb R^n$. Choose finitely many points
$y_\nu\in\mathbb R^n, \nu<k',$ such that this compact set is covered by $\bigcup_{\nu<k'} B_{\varepsilon/6}(y_\nu)$. 
Assume that precisely the first $k\le k'$ balls 
$B_{\varepsilon/6}(y_\nu)$ contain a point from the image of $f$. For $\nu<k$ choose $\widehat{\nu}\in G$ such 
that $f(\widehat{\nu})\in B_{\varepsilon/6}(y_\nu)$. Then 
$\bigcup_{\nu<k}B_{\varepsilon/3}(f(\widehat{\nu}))$ covers the image of~$f$. Hence, for every $g\in G$ we can choose
$\nu_g<k$ such that 
\begin{equation}\label{eq:nug}
\|f(g)-f(\widehat{\nu_g})\|<\varepsilon/3. 
\end{equation}
Let $c:G\to G$ be a selector for the partition $\{gG_{(A)}\mid g\in G\}$ of $G$, that is, for all $g,g'\in G$
 we have $c(g)\in gG_{(A)}$, and $c(g)=c(g')$ if and only if $gG_{(A)}=g'G_{(A)}$.
Define 
$$
\tilde f(g):=f(\widehat{\nu_{c(g)}}).
$$ 
Then $\tilde f$ satisfies (a) and (b). 
For all $g\in G$ we have $c(g)\in gG_{(A)}$, so $gG_{(A)}=c(g)G_{(A)}$ and thus 
$\|f(g)-f(c(g))\|<\varepsilon/6$ by \eqref{eq:A}. As
$\|f(c(g))-f(\widehat{\nu_{c(g)}})\|<\varepsilon/3$ by \eqref{eq:nug}, we conclude 
that $\tilde f$ satisfies  (c).\hfill$\dashv$\medskip

We aim to show that every $G$-flow has an orbit of size at most $d$ (Lemma~\ref{lem:orb}). So let $X$ be a $G$-flow. We are looking for some $x_0\in X$ such that
\begin{equation}\label{eq:goal}
|G\cdot x_0|\le d.
\end{equation}

Recall that the compact Hausdorff space $X$ carries a unique uniformity compatible with its topology. 
Suppose $f$ is a uniformly continuous function from $X$ into $\mathbb R^{n}$ for some $n\ge 1$. For each $x\in X$
define the function $f_x:G\to  \mathbb R^{n}$ by 
$$
f_x(g):=f(g^{-1}\cdot x).
$$ 
Then $f_x$ is left uniformly continuous. This follows from the well-known fact that
for every $x\in X$ the map $g\mapsto g^{-1}\cdot x$ is left uniformly continuous (see~e.g.~\cite[L2.1.5]{pestov}). 

With a triple $(H,f,\varepsilon)$ for a finite subset $H\subseteq G$, and a bounded, 
uniformly continuous function $f:X\to\mathbb R^{n}$, and a real $\varepsilon>0$ we associate the set
$$\textstyle
Y(H,f,\varepsilon):=\left\{x\in X\mid \exists h_0,\ldots, h_{d-1} \in H: f_x(H)\subseteq \bigcup_{\nu<d}\overline{B_{\varepsilon}(f_x(h_\nu))}\right\}.
$$
Since $H$ is finite, $Y(H,f,\varepsilon)$ is a finite union of closed sets of the form
$\{x\in X\mid f_x(H)\subseteq C\}$ for $C\subseteq \mathbb R^n$ closed, and consequently, 
$Y(H,f,\varepsilon)$ is closed.

\medskip

\noindent{\em Claim 3.} The family of closed sets $Y(H,f,\varepsilon)$ with $H,f,\varepsilon$ as above has the finite intersection property.\medskip

\noindent{\em Proof of Claim 3:}
For $j<\ell$ let $H_j\subseteq G$ be finite, $\varepsilon_j>0$ and $f^j:X\to \mathbb R^{n_j}$ for $n_j\ge 1$.
Set $H:=\bigcup_{j<\ell}H_j,\varepsilon:=\min_{j<\ell}\varepsilon_j, n:=\sum_{j<\ell}n_j$ and
define $f:X\to\mathbb R^n$ by $f(x):=f^0(x)*\cdots *f^{\ell-1}(x)$ where $*$ denotes concatenation. Then $f$ is uniformly continuous and bounded.

Let $x\in X$ be arbitrary. Since $f_x:G\to\mathbb R^n$ is left uniformly continuous, Claim~2 applies, and there exist
$g\in G$ and $h_0,\ldots,h_{d-1}\in H$ such that
$f_x(gH)\subseteq \bigcup_{\nu<d}B_{\varepsilon}(f_x(gh_\nu))$. In other words,
\begin{equation}\label{eq:cl41}
\forall h\in H\ \exists \nu<d:\ f(h^{-1}g^{-1}x)\in B_{\varepsilon}(f(h_\nu^{-1}g^{-1}x)).
\end{equation}

Any $y\in \mathbb R^n$ can be written as $y[0]*\cdots *y[\ell-1]$, where $y[j]\in\mathbb R^{n_j}$ for all 
$j<\ell$. In this notation, $f_x(g)[j]=f^j_x(g)$ for all $g\in G,x\in X, j<\ell$.
Clearly, $f_x(g)\in B_\varepsilon(y)$ implies $f_x(g)[j]\in B_\varepsilon(y[j])$ for all $y\in \mathbb R^n, j<\ell$.
Writing $x_0:=g^{-1}x$, ~\eqref{eq:cl41} yields: 
\begin{equation*}
\forall j<\ell\ \forall h\in H_j\ \exists \nu<d:\ f(h^{-1}g^{-1}x)[j]=f^j_{x_0}(h)\in B_{\varepsilon}(f^j_{x_0}(h_\nu)).
\end{equation*}
Since $\varepsilon\le \varepsilon_j$ we obtain
$$\textstyle
\forall j<\ell:\ f^j_{x_0}(H_j)\subseteq\bigcup_{\nu<d}B_{\varepsilon_j}(f^j_{x_0}(h_\nu)).
$$
Thus,  $x_0\in \bigcap_{j<\ell}Y(H_j,f^j,\varepsilon_j)\neq\emptyset$.\hfill$\dashv$\medskip

By Claim 3 and since $X$ is compact, there exists an $x_0$ in the intersection of all the sets $Y(H,f,\varepsilon)$, 
$(H,f,\varepsilon)$ a triple as above. We claim that $x_0$ satisfies \eqref{eq:goal}.
Assume otherwise that there  are $g_0,\ldots,g_{d}\in G$ such that $g_0x_0,\ldots,g_{d}x_0$ are pairwise distinct. Choose 
 $f:X\to[0,1]\subseteq \mathbb R^1$ uniformly continuous such that $f(g_\nu x_0)=\nu/d$ for all 
$\nu\le d$. 
Then $x_0\notin Y(\{g^{-1}_\nu\mid \nu\le d\}, f,\varepsilon)$ for a small enough $\varepsilon>0$, a contradiction.\end{proof}

\begin{proposition} \label{prop:amtoram}
Let $d\in{\mathbb N}$, $F$ be countable and locally finite, $G:=\operatorname{Aut}(F)$ and $A\in\operatorname{Age}(F)$ such that
$F$ is $A$-homogeneous. If $M(G)$ has size at most $d$, then $F\hookrightarrow (B)^A_{k,d}$ for all $B\in\operatorname{Age}(F)$ and $k\ge 2$.
\end{proposition}

\begin{proof} Assume that $|M(G)|\le d$, and let $B\in\operatorname{Age}(F),k\ge 2$ and $\chi_0:F^A\to [k]$ be a colouring. 
Note that $[k]^{F^A}$ is compact Hausdorff in the product topology with $[k]$ being discrete. 
The group $G$  acts continuously on $[k]^{F^A}$ by shift
$(g,\chi)\mapsto g\cdot \chi$, where $g\cdot\chi$ colours $a\in F^A$ by $\chi(g^{-1}\circ a)$. 
Consider the orbit closure $\overline{G\cdot \chi_0}$ of $\chi_0$. 
By Lemma~\ref{lem:orb}, 
the induced action of $G$ on $\overline{G\cdot \chi_0}$ has an orbit of size at most $d$, that is,
there exist $\chi_1\in \overline{G\cdot \chi_0}$ and $\psi_0,\ldots,\psi_{d-1}\in \overline{G\cdot \chi_0}$ such that
$G\cdot \chi_1=\{\psi_i\mid i<d\}$. 

Let $b\in F^B$. Observe that $b\circ B^A$ is a finite subset of $F^A$.
Since $\chi_1\in \overline{G\cdot \chi_0}$, there exists a 
$g\in G$ such that $g\cdot \chi_0$ and $\chi_1$ agree on $b\circ B^A$. Note that $g^{-1}\circ b\in F^{B}$, so we are left to show that $|\chi_0(g^{-1}\circ b\circ  B^A)|\le d$. We fix some $a_0\in F^A$, and claim that
for all $a\in g^{-1}\circ b\circ  B^A$ there exists a $\nu<d$ such that $\chi_0(a)=\psi_\nu(a_0)$. 
To see this, let $a\in g^{-1}\circ b\circ B^A\subseteq F^A$ and choose $h\in G$ 
such that $h\circ a_0=a$. Such an $h$ exists since $F$ is $A$-homogeneous. Then
$$
\chi_0(a)= (g\cdot\chi_0)(g\circ a)=\chi_1(g\circ a)= \chi_1((gh)\circ a_0) =((gh)^{-1}\cdot \chi_1)(a_0),
$$ 
where the second equality follows from  $g\circ a \in b\circ B^A$ and the choice of $g$.
As $(gh)^{-1}\cdot \chi_1\in G\cdot \chi_1$, and by choice of $\chi_1$, there exists $\nu<d$ 
such that $(gh)^{-1}\cdot \chi_1=\psi_\nu$. Thus 
$\chi_0(a)=\psi_\nu(a_0)$ as claimed.\end{proof}

\begin{proof}[Proof of Theorem~\ref{thm:char}.]
(1) $\Rightarrow$ (2). Write $F=\operatorname{Flim}({\mathcal K})$ and let $A\in{\mathcal K}=\operatorname{Age}(F)$. Then $F$ and $A$ satisfy the assumptions of Proposition~\ref{prop:amtoram}, so $F\hookrightarrow(B)^A_{k,d}$ for all $B\in{\mathcal K}$ and $k\ge 2$. Now apply Lemma~\ref{lem:infinite2}.

(2) $\Rightarrow$ (1). By Proposition~\ref{prop:ramtoam}.
\end{proof}

\subsection{Corollaries}\label{sec:cor}

\begin{corollary}\label{cor:asymptotic} Let $d\in{\mathbb N}$ and  ${\mathcal K}$ be a Fra\"iss\'e class.
The following are equivalent.
\begin{enumerate}\itemsep=0pt
 \item The class of structures with Ramsey degree for embeddings at most $d$ in ${\mathcal K}$ is cofinal in ${\mathcal K}$.
\item ${\mathcal K}$ has Ramsey degree for embeddings at most $d$.
\end{enumerate}
\end{corollary}

\begin{proof}
Assume (1). By Proposition~\ref{prop:ramtoam} we have $|M(G_{\mathcal K})|\le d$. 
As $F:=\operatorname{Flim}({\mathcal K})$ is Fra\"iss\'e, Proposition~\ref{prop:amtoram} implies $F\hookrightarrow (B)^A_{k,d}$ for all
$A,B\in{\mathcal K}$. Then Lemma~\ref{lem:infinite2} implies~(2).
\end{proof}

It is noted in \cite[$\S 1$(D)]{kpt} that a separable metrizable group $G$ is extremely amenable, i.e.
$M(G)$ has size 1, if and only if every metrizable $G$-flow has a fixed point. In this context it might be of interest to note:

\begin{corollary}  Let $d\in{\mathbb N}$ and ${\mathcal K}$ be a Fra\"iss\'e class.
The following are equivalent.
\begin{enumerate}\itemsep=0pt
\item $M(G_{\mathcal K})$ has size at most $d$.
\item Every continuous action of $G_{\mathcal K}$ on the Cantor space has an orbit of size at most $d$.
\end{enumerate}
\end{corollary}

\begin{proof}
(1) implies (2)  by Lemma~\ref{lem:orb}. Conversely, assume (2).
Let $A\in{\mathcal K}$ be arbitary and write $F:=\operatorname{Flim}({\mathcal K})$. Then $F$ and $A$ satisfy the assumptions of
Proposition~\ref{prop:amtoram}. In the proof of this proposition 
we only require the following for $G_{\mathcal K}$: for all $k\ge 2$
and all $\chi_0\in[k]^{F^A}$, the shift action of $G_{\mathcal K}$ restricted to $\overline{G_{\mathcal K}\cdot\chi_0}$ has a small orbit.
But $[k]^{F^A}$ is homeomorphic to the Cantor space and the restricted shift is a continuous action on this space.
Thus (2) suffices to carry out this proof and we conclude that $F\hookrightarrow(B)^A_{k,d}$ for all $B\in{\mathcal K}=\operatorname{Age}(F)$. 
By Lemma~\ref{lem:infinite2} 
every $A\in{\mathcal K}$ has Ramsey degree for embeddings at most~$d$ in~${\mathcal K}$. Then 
Proposition~\ref{prop:ramtoam} implies~(1).
\end{proof}

\section{Measure concentration}

We say that a probability measure is {\em concentrated on} any set of measure 1.  In this section we prove the following.

\begin{theorem}\label{thm:concentr}
Let ${\mathcal K}$ be a relational Fra\"iss\'e class which is Ramsey.
If $G_{\mathcal K}$ is amenable, then it is uniquely ergodic and the (unique) $G_{\mathcal K}$-invariant Borel 
probability measure on $M(G_{\mathcal K})$ is concentrated on a (unique) dense $G_\delta$ orbit.
\end{theorem}

In Section~\ref{sec:forget} we construct a forgetful order expansion using the Ramsey property, in Section~\ref{sec:concentr} 
we prove Theorem~\ref{thm:concentr}, and the final Section~\ref{sec:cor2} contains some observations 
concerning the $\omega$-categorical case and a proof of (a stronger version of) Theorem~\ref{thm:mystery}.

\subsection{Forgetful order expansions}\label{sec:forget}

An ordered Fra\"iss\'e class ${\mathcal K}^*$ in the language $L\cup\{<\}$ is called {\em forgetful} if for all $A,B\in{\mathcal K}$ and ${\mathcal K}^*$-admissible 
orderings $<^A, <^B$ on $A,B$ respectively, $(A,<^A)\cong (B,<^B)$ whenever $A\cong B$; here ${\mathcal K}$ 
denotes the $L$-reduct of ${\mathcal K}^*$.

For example, the orderings of $B_\infty$ and $V_{\infty,F}$ mentioned in Example~\ref{exas} have forgetful
 ages (see~\cite[$\S6$]{kpt} for details). The following is easy to see (cf.~\cite[P5.6]{kpt}).

\begin{lemma}\label{lem:forgetramsey} Let ${\mathcal K}^*$ be a forgetful ordered Fra\"iss\'e class in the language $L\cup\{<\}$ and ${\mathcal K}$ 
its $L$-reduct. Then ${\mathcal K}^*$ has the ordering property, and  ${\mathcal K}^*$ is Ramsey if and only if so is ${\mathcal K}$.
\end{lemma}

Before showing that the Ramsey property ensures the existence of reasonable forgetful expansions, we present a well-known technical lemma. Informally, this technical lemma guarantees a monochromatic copy of a given $B$ when copies of several different $A_i$ are coloured simultaneously.

\begin{lemma}\label{lem:ramseyref}
Let ${\mathcal K}$ be a Ramsey class. Let $n\in \mathbb{N}$, $k_0, \ldots, k_{n-1}\in\mathbb{N}$, $A_0, \ldots, A_{n-1} , B\in {\mathcal K}$. 
Then there exists a $C\in {\mathcal K}$ with the following property: for any family of
colourings $\chi_i: \binom{C}{A_i}\rightarrow [k_i]$, $i\in [n]$, there
 exists a $B'\in \binom{C}{B}$ such that 
$\chi_i\upharpoonright\binom{B'}{A_i}$ 
is constant for all $i\in [n]$.
\end{lemma}
\begin{proof}
Let $C_0:= B$, and for every $0<i\le n$ choose $C_i\in{\mathcal K}$ such that
$C_i\to (C_{i-1})_{k_{i-1},1}^{A_{i-1}}$. Let $C:= C_n$. Then by using the definition of the $C_i$ and a 
straightforward induction on $j\in [n]$ we obtain that there is a $C'_{n-1-j}\in \binom{C}{C_{n-1-j}}$ such that 
$\chi_{n-1-j}\upharpoonright \binom{C'_{n-1-j}}{A_{n-1-j}}$ 
is constant for all $i\in [n]\setminus [n-1-j]$. Setting $j= n-1$ yields $B'$ as in the statement. 
\end{proof}

\begin{lemma}\label{lem:forget}
Let ${\mathcal K}$ be a Fra\"{\i}ss\'{e} class in the language $L$. If ${\mathcal K}$ is Ramsey, then there exists a 
reasonable, forgetful ordered Fra\"{\i}ss\'{e} class ${\mathcal K}^*$ in the language $L\cup \{<\}$ with $L$-reduct ${\mathcal K}$.
\end{lemma}
\begin{proof} Let $F:=\operatorname{Flim}({\mathcal K})$ and consider the space {\em LO} of linear orders on $F$ (cf.~Example~\ref{exa:lo}). 
Let $(A,B)$ range over pairs with $A\in{\mathcal K}$ and $B\subseteq F$.
Call $R\in\textit{LO}$ order forgetful for $(A,B)$ if 
$(A',R\upharpoonright A')\cong(A'',R\upharpoonright A'')$ for all $A',A''\in{B\choose A}$. 

\medskip

\noindent{\em Claim.} If $n\ge 1$ and $(A_0,B_0), \ldots,(A_{n-1},B_{n-1})$ are pairs as above with all $B_i\subseteq F$ finite, 
then there exists $R\in\textit{LO}$ 
that is order forgetful for every $(A_i,B_i), i\in [n]$.
\medskip

\noindent{\em Proof of Claim:} Choose $B\subseteq F$ finite 
such that $\bigcup_{i\in [n]}B_i\subseteq B$. It suffices to find
an order which is order forgetful for every $(A_i, B), i\in [n]$. Fix an arbitrary order $R\in\textit{LO}$. 
For $i\in [n]$ let $\chi_i$ colour each  $A_i'\in{F \choose A_i}$ by the isomorphism 
type of $(A_i',R\upharpoonright A_i')$, and let $k_i\in{\mathbb N}$ be the number of colours 
of $\chi_i$. 
By Lemma~\ref{lem:ramseyref} and homogeneity of $F$ there exist $B'\subseteq F$ and $g\in \operatorname{Aut}(F)$ such that 
$g(B')=B$ and each $\chi_i$ is constant on $\binom{B'}{A_i}$. 
By definition of the $\chi_i$ this means that $R$ is order 
forgetful for $(A_i, B')$ for all $i\in [n]$. Hence, $g(R)$ is order forgetful for all $(A_i, B), i\in [n]$.
\hfill$\dashv$\medskip

For every $A\in{\mathcal K}$ and $B\subseteq F$ finite, the set of orders that are order forgetful for $(A,B)$ is 
closed in $\textit{LO}$. By the claim and compactness, there exists $R\in\textit{LO}$ which is order forgetful 
for all pairs $(A,B)$ such that $A\in{\mathcal K}$ and $B\subseteq F$ is finite.
Then $R$ is order forgetful for $(A,F)$ for every $A\in{\mathcal K}$. Equivalently, ${\mathcal K}^*:=\operatorname{Age}(F,R)$ is forgetful. 
To see that ${\mathcal K}^*$ is Fra\"iss\'e, observe that ${\mathcal K}^*$ is hereditary and has joint embedding. As ${\mathcal K}^*$ is 
Ramsey by Lemma~\ref{lem:forgetramsey}, it has amalgamation by Lemma~\ref{lem:Ramam} (and Remark~\ref{rem:rigiddegrees}; 
note that ${\mathcal K}^*$ is rigid because it is ordered). According to Lemma~\ref{lem:reason}, ${\mathcal K}^*$ is reasonable.
\end{proof}

\begin{examples}\label{exas:rationals}

The structures $F_1:=(\mathbb{Q}, \operatorname{Betw}), F_2:=(\mathbb{Q}, \operatorname{Cycl}), F_3:=(\mathbb{Q}, \operatorname{Sep})$ and 
$F_4:=(\mathbb{Q}, =)$ are Ramsey (see \cite{cameron} for definitions). 
If $<$ is the rational order, then ${\mathcal K}_i^*:=\operatorname{Age}((F_i,<))$ is forgetful with reduct ${\mathcal K}_i:=\operatorname{Age}(F_i)$.
By Lemmas~\ref{lem:reason},~\ref{lem:forgetramsey} and Theorem~\ref{thm:minflow}, $M(G_{{\mathcal K}_i})$ is 
$\overline{G_{{\mathcal K}_i}\cdot <}$. Then $M(G_{{\mathcal K}_1})$
is the 2-element discrete space, Hence, by Theorem~\ref{thm:char}, 
${\mathcal K}_1$ has Ramsey degree for embeddings 2. 
Theorem~\ref{thm:minflow} also allows to 
explicitly describe $M(G_{{\mathcal K}_i})$ for $i=2,3,4$ and these have the size of the continuum. 
 Hence, ${\mathcal K}_2,{\mathcal K}_3$ and ${\mathcal K}_4$ have infinite Ramsey degree for embeddings.
\end{examples}

\begin{examples}
Let ${\mathcal K}$ be a Fra\"iss\'e class of digraphs such that there is a directed cycle in~${\mathcal K}$. 
Then there does not exist a forgetful ordered Fra\"iss\'e class with $L$-reduct ${\mathcal K}$:
by forgetfullness, every directed edge in any $A\in {\mathcal K}$ would be ordered in the same way and then
a directed cycle contradicts transitivity of the order. 
For example, this applies to the age of the universal homogeneous digraph, 
the random tournament and the local order (see~\cite{mcph}). 
\end{examples}

\subsection{Proof of Theorem~\ref{thm:concentr}}\label{sec:concentr}

Let $F:=\operatorname{Flim}({\mathcal K})$ and $L$ denote the relational language of~${\mathcal K}$. Since ${\mathcal K}$ is assumed to be Ramsey, Lemma~\ref{lem:forget} applies and
there is a reasonable forgetful ordered Fra\"iss\'e class ${\mathcal K}^*$ in the language $L\cup\{<\}$
with $L$-reduct ${\mathcal K}$. By Lemma~\ref{lem:forgetramsey} and Theorem~\ref{thm:minflow}, 
 $\operatorname{Flim}({\mathcal K}^*)=(F,<^*)$ for some order~$<^*$, and $X_{{\mathcal K}^*}=\overline{G_{\mathcal K}\cdot <^*}$ is the universal minimal flow of~$G_{\mathcal K}$.

Assume that $G_{\mathcal K}$ is amenable. In order to verify that $G_{\mathcal K}$ is uniquely ergodic, it 
suffices by Proposition~\ref{prop:randomorder} to show that for every consistent random ordering $(R_A)_{A\in{\mathcal K}}$ we
 have that each random variable $R_A$ is uniformly distributed. By forgetfulness, for any two ${\mathcal K}^*$-admissible
 orderings $<,<'$ on $A$ there is an $\alpha\in\operatorname{Aut}(A)$ such that $\alpha(<)=<'$, and 
then $\Pr[R_A=<']=\Pr[\alpha^{-1}\circ R_A=<]=\Pr[R_A=<]$ where the latter equality follows from
$(R_A)_{A}$ being consistent. 

Let $\mu$ denote the unique $G_{\mathcal K}$-invariant Borel probability measure  on $X_{{\mathcal K}^*}$. Recall 
the notation $U(<)$ from Proposition~\ref{prop:randomorder}. By this result, $U(<)$ and $U(<')$ 
have the same $\mu$-measure whenever $<$ and $<'$ are  ${\mathcal K}^*$-admissible orderings of the same finite subset of~$F$.

An order $R\in X_{{\mathcal K}^*}$ 
is outside $G_{\mathcal K}\cdot <^*$ if and only if $(F,R)\not\cong(F,<^*)$, if 
and only if $(F,R)$ is not homogeneous (cf.~Section~\ref{sec:fraisse}), if and only if there exist a finite $A\subseteq F$, some $(B,<^B)\in{\mathcal K}^*$ and
$a\in (B,<^B)^{(A,<^*\upharpoonright A)}$ such that $R$ is {\em bad for} $(B,<^B,a)$, meaning that 
there is no $b\in (F,R)^{(B,<^B)}$ with $b\circ a= \operatorname{id}_A$. 
As the language of $F$ is relational, we may assume that $B={\mathrm{im}}(a)\cup\{p\}$ with $p\in F\setminus {\mathrm{im}}(a)$. 

Observe that the set of orders $R\in X_{{\mathcal K}^*}$ which are bad for $(B,<^B,a)$
is closed. Hence, $X_{{\mathcal K}^*}\setminus G_{\mathcal K}\cdot <^*$ is  $F_\sigma$, so 
$G_{\mathcal K}\cdot <^*$ is a dense $G_\delta$ orbit in $X_{{\mathcal K}^*}$ (see also \cite[14.3]{akl}). Since $X_{{\mathcal K}^*}$ is a Baire space, $G_{\mathcal K}\cdot <^*$ is clearly unique with this property. We prove that $\mu(G_{\mathcal K}\cdot <^*)=1$.
It suffices to show that for each $(B,<^B,a)$ with $B={\mathrm{im}}(a)\dot\cup\{p\}$ as above, the set
$\mathcal B:=\{R\in X_{{\mathcal K}^*}\mid  R \text{ is bad for }(B,<^B,a)\}$ has $\mu$-measure 0.

We construct a sequence $(\mathcal U_n)_{n\in\mathbb N}$ such that for all $n\in\mathbb N$
\begin{enumerate}
\item[(a)] $\mathcal U_n$ is a cover of $\mathcal B$, i.e.\ $\mathcal B\subseteq \bigcup\mathcal U_n$;
\item[(b)] every $U\in\mathcal U_n$ equals some $U(<')$ such that $<'\supseteq <^*\upharpoonright A$ is a ${\mathcal K}^*$-admissible order with 
$|{\textrm{dom}}(<')|=|A|+n$; 
\item[(c)] $\textstyle\mu(\bigcup \mathcal U_{n+1})\leq \frac{|A|+n}{|A|+n+1}\cdot\mu(\bigcup\mathcal U_n).$
\end{enumerate}
Here, ${\textrm{dom}}(<')$ is the set linearly orderd by $<'$; note that (b) implies ${\textrm{dom}}(<')\supseteq A$.

This finishes the proof: by (a) and (c) we have for all $n\in\mathbb N$
$$\textstyle
\mu(\mathcal B)\le \mu(\bigcup\mathcal U_n)\le 
\prod_{m<n}\frac{|A|+m}{|A|+m+1}\cdot \mu(\bigcup\mathcal U_0)=\mu(\bigcup\mathcal U_0)\cdot \frac{|A|}{|A|+n}\to_n 0.
$$

 Set $\mathcal U_0:=\{U(<^*\upharpoonright A)\}$ and assume that 
$\mathcal U_n$ is already defined. It suffices to find for every $U(<')\in\mathcal U_n$
some $p'\notin{\textrm{dom}}(<')$ and a family $(<_i)_{i\in I}$ such that
\begin{enumerate}
\item[(a')] $\bigcup_{i\in I}U(<_i)\cap\mathcal B=U(<')\cap\mathcal B$;
\item[(b')]  for every $i\in I$, $<_i\supseteq <'$  is a ${\mathcal K}^*$-admissible order with ${\textrm{dom}}(<_i)={\textrm{dom}}(<')\cup\{p'\}$;
\item[(c')] $\mu(\bigcup_{i\in I}U(<_i))\le\frac{|A|+n}{|A|+n+1}\cdot\mu(U(<'))$.
\end{enumerate}

Write $A':={\textrm{dom}}(<')$, and choose $R\in \mathcal B\cap U(<')$. Since 
$R\in \overline{G_{\mathcal K}\cdot <^*}$ there is a $g\in G_{\mathcal K}$ such that 
\begin{equation}\label{eq:Rg}\textstyle
 g(<^*)\upharpoonright A'= R\upharpoonright A'=<'.
\end{equation}
In particular, 
$g(<^*)\upharpoonright A=<'\upharpoonright A=<^*\upharpoonright A$ and
$(A,<^*\upharpoonright A)$ is a substructure of $(F,g(<^*))$. Since
$(F,g(<^*))$ is isomorphic to $(F,<^*)$, it is homogeneous, so
there exists an embedding $b\in (F,g(<^*))^{(B,<^B)}$ with $b\circ a=\operatorname{id}_A$. 
We set $p':=b(p)$ and claim that $p'\notin A'$. Otherwise, ${\mathrm{im}}(b)\subseteq A'$, so 
$b\in (F,R)^{(B,<^B)}$ by~\eqref{eq:Rg}, and this contradicts $R$ being bad for $(B, <^*, a)$.

Let $<_0,\ldots,<_{s-1}$ list the ${\mathcal K}^*$-admissible orders on $A'\cup\{p'\}$ 
extending $<'$, and note that $s\le |A'|+1$. Let $I\subseteq[s]$ consist of those $i<s$ such that 
$U(<_i)\cap\mathcal B\neq\emptyset$. Then (a') and (b') follow, and
we are left to verify (c'). The sets $U(<_i),i<s,$ partition $U(<')$ and, as already noted, 
have pairwise equal $\mu$-probability, so $\mu(U(<_i))=\mu(U(<'))/s$. Thus
\begin{equation}\label{eq:I}\textstyle
\mu(\bigcup_{i\in I}U(<_i))=|I|/s\cdot\mu(U(<')). 
\end{equation}

There exists $i_0<s$ such that $<_{i_0}=g(<^*)\upharpoonright (A'\cup\{p'\})$. Since 
$b\in (F,g(<^*))^{(B,<^B)}$ has  ${\mathrm{im}}(b)\subseteq A'\cup\{p'\}$, 
we have that $b\in (F,S)^{(B,<^B)}$ for every $S\in U(<_{i_0})$. Hence, no such $S$ is bad for $(B,<^B,a)$, that is, 
$U(<_{i_0})\cap\mathcal B=\emptyset$, so $i_0\notin I$. Thus 
$|I|<s$. Since $|A'|=|A|+n$, we have $s\le |A|+n+1$, so $|I|/s\le (s-1)/s\le (|A|+n)/(|A|+n+1)$. Hence, (c') follows from~\eqref{eq:I}.

\subsection{The $\omega$-categorical case}\label{sec:cor2}
Of particular interest are Fra\"iss\'e classes
${\mathcal K}$ which have an $\omega$-categorical Fra\"iss\'e 
limit $F:=\operatorname{Flim}({\mathcal K})$. By the theorem of Ryll-Nardzewski (see e.g.\ \cite[T4.3.1]{zieglerbuch}) this happens e.g.\ if 
the language $L$ of ${\mathcal K}$ is finite and relational (cf.~\cite[T4.4.7]{zieglerbuch}), and 
is equivalent to $G_{\mathcal K}$ being {\em oligomorphic}: 
for every $n\in\mathbb N$, $G_{\mathcal K}$ has only finitely 
many $n$-orbits. An {\em $n$-orbit of $G_{\mathcal K}$} is an orbit of the {\em diagonal action} of $G_{\mathcal K}$ on $F^n$ given by 
$g\cdot \bar a=g\cdot (a_0,\ldots, a_{n-1}):= g(\bar a)=(g(a_0),\ldots, g(a_{n-1}))$. 

\begin{lemma}\label{lem:forgetoligo}
 Let ${\mathcal K}^*$ be a reasonable ordered Fra\"iss\'e class in the 
language $L\cup\{<\}$ with $L$-reduct ${\mathcal K}$. Then $G_{{\mathcal K}^*}$ is oligomorphic if and only if so is $G_{\mathcal K}$.
\end{lemma}

\begin{proof}
Let $F=\operatorname{Flim}({\mathcal K})$. By Theorem~\ref{thm:minflow} we have that $\operatorname{Flim}({\mathcal K}^*)=(F, <^*)$ for 
some order~$<^*$ on $F$. As $G_{{\mathcal K}^*}$ is a subgroup of $G_{\mathcal K}$, it suffices to show that
 every orbit $T\subseteq F^n$ of~$G_{\mathcal K}$ that consists of tuples with all different entries 
is the union of finitely many $n$-orbits of~$G_{{\mathcal K}^*}$. Let $\bar s=(s_1, \ldots, s_n)$ and $\bar t=(t_1, \ldots, t_n)$
 be tuples in $T$ such that the unique extension of the partial isomorphism $s_1\mapsto t_1, \ldots, s_n\mapsto t_n$ to 
the substructures in~$F$ generated by $\bar s$ and $\bar t$ is a partial isomorphism of $F^*$. Then by homogeneity of $F^*$
 we have that $\bar s$ and $\bar t$ are in the same $n$-orbit of $G_{{\mathcal K}^*}$. As there are finitely many ways to define
 a (${\mathcal K}^*$-admissible) order on the structure generated by a tuple in $T$, the claim follows.
\end{proof}

\begin{lemma}\label{lem:normalfinite} 
Let ${\mathcal K}^*$ be a reasonable ordered Fra\"iss\'e class in the 
language $L\cup\{<\}$ with $L$-reduct ${\mathcal K}$. Assume that $G_{{\mathcal K}^*}$ is oligomorphic.
If $G_{{\mathcal K}^*}$ is normal in~$G_{\mathcal K}$, then it has finite index in $G_{\mathcal K}$.
\end{lemma}

\begin{proof}
By reasonability $\operatorname{Flim}({\mathcal K}^*)=(\operatorname{Flim}({\mathcal K}),<^*)$ for some order $<^*$.
Consider the logic action of $G_{\mathcal K}$ on $\textit{LO}$ (Example~\ref{exa:lo}). Then $G_{{\mathcal K}^*}$ is the stabilizer of $<^*$.
Hence, $|G_{\mathcal K}:G_{{\mathcal K}^*}|=|G_{\mathcal K}\cdot <^*|$ and it suffices to show that $G_{\mathcal K}\cdot <^*$ is finite. 
If $G_{{\mathcal K}^*}$ is normal, then it fixes every $R\in G_{\mathcal K}\cdot <^*$. Thus every such $R$ is a union of 2-orbits.
As $G_{{\mathcal K}^*}$ is oligomorphic, there are only finitely many such $R$. 
\end{proof}

We use the following mode of speech from~\cite{akl}: let ${\mathcal K}$ be a Fra\"iss\'e class in the language~$L$; 
a {\em companion of ${\mathcal K}$} is a reasonable ordered Fra\"iss\'e class ${\mathcal K}^*$
in the language $L\cup\{<\}$ which is Ramsey, has the ordering property and has $L$-reduct ${\mathcal K}$.
Note:

\begin{proposition}\label{prop:ramseycompanion}
If a Fra\"iss\'e class is Ramsey, then it has a companion.
\end{proposition}

\begin{proof}
By Lemmas~\ref{lem:forget} and~\ref{lem:forgetramsey}. 
\end{proof}

\begin{proposition}\label{prop:boolean}
Let ${\mathcal K}$ be a relational Fra\"iss\'e class that has a companion.
If $M(G_{\mathcal K})$ is finite, then $|M(G_{\mathcal K})|$ is a power of 2.
\end{proposition}

 
\begin{proof}
Let $L$ denote the relational language of ${\mathcal K}$ and let ${\mathcal K}^*$ be a companion of ${\mathcal K}$.
By Theorem~\ref{thm:minflow} we have that $F^*:=\operatorname{Flim}({\mathcal K}^*)=(F,<^*)$ for $F:=\operatorname{Flim}({\mathcal K})$,  and that
$M(G_{\mathcal K})$ is $\overline{G_{\mathcal K}\cdot <^*}$. Assume that $M(G_{\mathcal K})$ is finite.
Then $G_{\mathcal K}\cdot <^*$ is finite, and since $G_{{\mathcal K}^*}$ is the stabilizer of $<^*$ in 
the logic action of $G_{\mathcal K}$ on $\textit{LO}$, 
$G_{{\mathcal K}^*}$ has finite index in $G_{\mathcal K}$. 
By Theorem~\ref{thm:extremeamenability}, $G_{{\mathcal K}^*}$ is extremely amenable. 
By Lemma~\ref{lem:todorpart}, $G_{{\mathcal K}^*}$ is normal in $G_{\mathcal K}$ and $|M(G_{\mathcal K})|=|G_{\mathcal K}:G_{{\mathcal K}^*}|$.

Consider the diagonal actions of $G_{\mathcal K}$ and $G_{{\mathcal K}^*}$ on $F^2$. We claim that for every $g\in G_{\mathcal K}$ and 
every 2-orbit $S$ of $G_{{\mathcal K}^*}$ the set $g\cdot S\subseteq F^2$ is also a 2-orbit of $G_{{\mathcal K}^*}$. 
Indeed, normality implies that two pairs in the same 2-orbit of $G_{{\mathcal K}^*}$
are mapped by $g$ to two pairs which are also in the same 2-orbit of $G_{{\mathcal K}^*}$, so there exists 
 a 2-orbit $T$ with $g\cdot S\subseteq T$. 
Reasoning analoguously for $g^{-1}$ and $T$ we obtain $g^{-1}\cdot T\subseteq S$, so $g\cdot S=T$.

Call a  2-orbit $S$ of $G_{{\mathcal K}^*}$ {\em black} if $a<^*b$ for all $(a,b)\in S$, and {\em white} if 
$b<^*a$ for all $(a,b)\in S$; orbits which are neither black nor white contain only pairs $(a,b)$ with $a=b$.
Let $S$ be black or white. For every $g\in G_{\mathcal K}$, also $g(S)$ is black or white, and if $g(S)$
has the same colour as $S$, then $g(S)=S$. Indeed, as $g\in G_{\mathcal K}$, $g\upharpoonright\{a,b\}$ preserves 
all relations from $L$, and as $g(S)$ has the same colour as $S$, it also preserves $<^*$. Hence, for every $(a,b)\in S$,
$g\upharpoonright\{a,b\}$ is a partial isomorphism of $F^*$, so it extends to some $h\in G_{{\mathcal K}^*}$ 
by homogeneity. Thus $g\cdot (a,b)=h\cdot (a,b)$, so $g\cdot (a,b)\in S$ and $g(S)=S$ follows.

We claim that $g^2\in G_{{\mathcal K}^*}$ for every $g\in G_{\mathcal K}$. Seeking for contradiction, assume that 
there is an $(a,b)\in F^2$ such that $a<^*b$ is not equivalent to
$g^2(a)<^*g^2(b)$. Then there is a  black or white 2-orbit~$S$ of $G_{{\mathcal K}^*}$ such that $g^2(S)$ has a different colour.
The colour of $g(S)$ equals  that of $S$ or $g^2(S)$, and consequently, $S=g(S)$ or $g(S)=g^2(S)$. 
The first case $S=g(S)$ is impossible, because it implies $S=g^2(S)$. The second case $g(S)=g^2(S)$ is also
impossible, because it implies the first via
$g(S)=g^{-1}(g^2(S))=g^{-1}(g(S))=S$.

It follows that  $G_{\mathcal K}/G_{{\mathcal K}^*}$  is an elementary abelian 2-group, i.e., it is the direct product of copies of the 2-element group.
\end{proof}

\begin{theorem}\label{prop:power2}
Let ${\mathcal K}$ be a relational Fra\"iss\'e class with companion ${\mathcal K}^*$. Assume that $G_{\mathcal K}$ is oligomorphic.
Then the following are equivalent.
\begin{enumerate}
\item $|G_{\mathcal K}:G_{{\mathcal K}^*}|$ is finite.
\item $|G_{\mathcal K}:G_{{\mathcal K}^*}|$ is a finite power of 2.
\item $M(G_{\mathcal K})$ is finite.
\item $|M(G_{\mathcal K})|$ is a finite power of 2.
\item $G_{{\mathcal K}^*}$ is normal in $G_{\mathcal K}$.
\end{enumerate}
\end{theorem}
 
\begin{proof} 
By Theorem~\ref{thm:minflow} we have that $F^*:=\operatorname{Flim}({\mathcal K}^*)=(F,<^*)$ for $F:=\operatorname{Flim}({\mathcal K})$, and that
$M(G_{\mathcal K})$ is $\overline{G_{\mathcal K}\cdot <^*}$. Then $G_{{\mathcal K}^*}$ 
is oligomorphic by Lemma~\ref{lem:forgetoligo}, and extremely amenable by
Theorem~\ref{thm:extremeamenability}.
In a Hausdorff space a finite set equals its closure.
As the elements of $G_{\mathcal K}\cdot <^*$ are in a one-to-
one correspondence with $G_{\mathcal K}/G_{{\mathcal K}^*}$, we obtain $(1)\Leftrightarrow(3)$ and $(2)\Leftrightarrow(4)$. 
Proposition~\ref{prop:boolean} implies $(3)\Leftrightarrow(4)$, thus the first four items are equivalent. $(5)\Rightarrow(1)$ 
follows from Lemma~\ref{lem:normalfinite}, and Lemma~\ref{lem:todorpart} implies $(1)\Rightarrow(5)$.
\end{proof}

\begin{corollary}\label{cor:mystery2}
 Let ${\mathcal K}$ be a relational Fra\"iss\'e class that has a companion. Then the 
Ramsey degree for embeddings of ${\mathcal K}$ is infinite or a finite power of 2.
\end{corollary}

\begin{proof} 
By  Proposition~\ref{prop:boolean} and Theorem~\ref{thm:char}.
\end{proof}

\begin{proof}[Proof of Theorem~\ref{thm:mystery}.] By Proposition~\ref{prop:ramseycompanion} and Corollary~\ref{cor:mystery2}.
\end{proof}

\section{Acknowledgements} The authors are indebted to Todor Tsankov for the elegant proof of Proposition~\ref{prop:todor} which is shorter and more general than their original one, and to Manuel Bodirsky, Lionel Nguyen van Th\'{e} and Lyubomyr Zdomskyy for their many comments on the manuscript.

\bibliographystyle{acm}
\begin{thebibliography}{10}

\bibitem{akl}
{\sc Angel, O., Kechris, A.~S., and Lyons, R.}
\newblock Random orderings and unique ergodicity of automorphism groups.
\newblock {\em Journal of the European Mathematical Society\/} (2012).
\newblock to appear; preprint available at arXiv:1208.2389v1 [math.DS].

\bibitem{kb}
{\sc Becker, H., and Kechris, A.~S.}
\newblock {\em The Descriptive Set Theory of Polish Group Actions}, vol.~232 of
  {\em London Mathematical Society, Lecture Note Series}.
\newblock Cambridge University Press, 1996.

\bibitem{bodirsky}
{\sc Bodirsky, M.}
\newblock New {R}amsey classes from old.
\newblock preprint arXiv:1204.3258 [math.LO], 2012.

\bibitem{bodpin}
{\sc Bodirsky, M., and Pinsker, M.}
\newblock Reducts of {R}amsey structures.
\newblock In {\em Model Theoretic Methods in Finite Combinatorics}, vol.~558 of
  {\em Contemporary Mathematics}. American Mathematical Society, 2011,
  pp.~489--519.

\bibitem{bpt}
{\sc Bodirsky, M., Pinsker, M., and Tsankov, T.}
\newblock Decidability of definability.
\newblock In {\em Proceedings of the 26th Annual IEEE Symposium on Logic in
  Computer Science (LICS'11)}. IEEE Computer Society, 2011, pp.~321--328.

\bibitem{cameron}
{\sc Cameron, P.~J.}
\newblock {\em Aspects of infinite permutation groups}, vol.~339 of {\em London
  Mathematical Society, Lecture Note Series}.
\newblock Cambridge University Press, 2007.

\bibitem{cherlin}
{\sc Cherlin, G.}
\newblock {\em Two problems on homogeneous structures revisited}, vol.~558 of
  {\em Model Theoretic Methods in Finite Combinatorics, Contemporary
  Mathematics, AMS}.
\newblock 2011.

\bibitem{fouche}
{\sc Fouch\'e, W.~L.}
\newblock Symmetry and the {R}amsey degree of posets.
\newblock {\em Discrete Mathematics\/} (1997), 309--315.

\bibitem{fouche2}
{\sc Fouch\'e, W.~L.}
\newblock Symmetries and {R}amsey properties of trees.
\newblock {\em Discrete Mathematics 197/198\/} (1999), 325--330.

\bibitem{fouche3}
{\sc Fouch\'e, W.~L.}
\newblock Symmetry and the {R}amsey degree of finite relational structures.
\newblock {\em Journal of Combinatorial Theory A 85\/} (1999), 135--147.

\bibitem{grs}
{\sc Graham, R.~L., Rothschild, B.~L., and Spencer, J.~H.}
\newblock {\em Ramsey theory}.
\newblock Wiley-Interscience Series in Discrete Mathematics and Optimization.
  John Wiley \& Sons, Inc., New York, 1990.
\newblock Second edition.

\bibitem{kechris}
{\sc Kechris, A.~S.}
\newblock Dynamics of non-archimedean polish groups.
\newblock {\em Proceedings of the European Congress of Mathematics, Krakow,
  Poland\/} (2012).
\newblock to appear.

\bibitem{kpt}
{\sc Kechris, A.~S., Pestov, V., and Todorcevic, S.}
\newblock Fraiss\'e limits, {R}amsey theory, and topological dynamics of
  automorphism groups.
\newblock {\em Geometric and Functional Analysis 15}, 1 (2005), 106--189.

\bibitem{mcph}
{\sc Macpherson, H.~D.}
\newblock A survey of homogeneous structures.
\newblock {\em Discrete Mathematics 311\/} (2011), 1599--1634.

\bibitem{nesetrilsurvey}
{\sc Ne\v{s}et\v{r}il, J.}
\newblock Ramsey theory.
\newblock {\em Handbook of Combinatorics\/} (1995), 1331--1403.

\bibitem{nesetril}
{\sc Ne\v{s}et\v{r}il, J.}
\newblock Ramsey classes and homogeneous structures.
\newblock {\em Combinatorics, Probability \& Computing 14}, 1-2 (2005),
  171--189.

\bibitem{pestov}
{\sc Pestov, V.}
\newblock {\em Dynamics of Infinite-Dimensional Groups and Ramsey-Type
  Phenomena}.
\newblock Publicacoes dos Col\'oquios de Matem\'atica, IMPA, Rio de Janeiro,
  2005.

\bibitem{zieglerbuch}
{\sc Tent, K., and Ziegler, M.}
\newblock {\em A course in Model theory}.
\newblock Lecture notes in Logic. Cambridge University Press, 2012.

\bibitem{usp}
{\sc Uspenskij, V.}
\newblock Compactifications of topological groups.
\newblock {\em Proceedings of the Ninth Prague Topological Symposium\/} (2001),
  331--346.

\bibitem{nguyen}
{\sc van Th\'{e}, L.~N.}
\newblock More on the {K}echris-{P}estov-{T}odorcevic correspondence:
  precompact expansions.
\newblock {\em Fundamentae Mathematicae 222\/} (2013), 19--47.

\bibitem{nguyen2}
{\sc van Th\'{e}, L.~N.}
\newblock Universal flows of closed subgroups of ${S}_\infty$.
\newblock {\em Asymptotic Geometric Analysis, Fields Institute Communications
  68\/} (2013), 229--245.

\bibitem{zucker}
{\sc Zucker, A.}
\newblock Amenability and unique ergodicity of automorphism groups of
  {F}ra\"{\i}ss\'{e} structures.
\newblock preprint, arXiv:1304.2839 [math.LO], 2013.

\end{thebibliography}

\end{document}


\documentclass[12pt,letterpaper,american,english]{amsart}
\usepackage{charter}
\usepackage{helvet}
\usepackage{courier}
\usepackage[T1]{fontenc}
\usepackage[utf8x]{inputenc}
\usepackage{mathrsfs}
\usepackage{amsthm}
\usepackage{amsbsy}
\usepackage{amstext}
\usepackage{amssymb}
\usepackage[all]{xy}

\makeatletter

\pdfpageheight\paperheight
\pdfpagewidth\paperwidth

\numberwithin{equation}{section}
\numberwithin{figure}{section}
\theoremstyle{plain}
\newtheorem{thm}{\protect{Theorem}}[section]
  \theoremstyle{definition}
  \newtheorem{defn}[thm]{\protect{Definition}}
  \theoremstyle{remark}
  \newtheorem*{rem*}{\protect{Remark}}
  \theoremstyle{remark}
  \newtheorem{rem}[thm]{\protect{Remark}}
  \theoremstyle{definition}
  \newtheorem{example}[thm]{\protect{Example}}
  \theoremstyle{plain}
  \newtheorem{cor}[thm]{\protect{Corollary}}
  \theoremstyle{plain}
  \newtheorem{prop}[thm]{\protect{Proposition}}
  \theoremstyle{plain}
  \newtheorem{lem}[thm]{\protect{Lemma}}
  \theoremstyle{definition}
  \newtheorem{condition}[thm]{\protect{Condition}}

\usepackage{verbatim} 
\usepackage{eucal}
\usepackage{amsthm}
\usepackage[all]{xy}
\usepackage{manfnt}
\usepackage{float}
\usepackage{xr}
\SelectTips{cm}{}
\xyoption{line}
\usepackage{hyperref}
\makeatletter 
\makeatletter 
\emergencystretch=100pt

\mathcode`\:="603A 
                   
\hyphenation{morph-ism}
\hyphenation{homo-morph-ism}
\hyphenation{iso-morph-ism}
\hyphenation{homo-topy}
\hyphenation{homeo-morph-ism}
\hyphenation{endo-morph-ism}
\hyphenation{homo-top-ic}
\hyphenation{null-homo-topy}
\hyphenation{null-homo-top-ic}
\hyphenation{trans-for-ma-tion}
\DeclareSymbolFont{rsfs}{U}{rsfs}{m}{n}
\DeclareSymbolFontAlphabet{\mathrf}{rsfs}

\makeatother

\usepackage{babel}
  \addto\captionsamerican{}
  \addto\captionsamerican{}
  \addto\captionsamerican{}
  \addto\captionsamerican{}
  \addto\captionsamerican{}
  \addto\captionsamerican{}
  \addto\captionsamerican{}
  \addto\captionsamerican{}
  \addto\captionsenglish{}
  \addto\captionsenglish{}
  \addto\captionsenglish{}
  \addto\captionsenglish{}
  \addto\captionsenglish{}
  \addto\captionsenglish{}
  \addto\captionsenglish{}
  \addto\captionsenglish{}
  \providecommand{Condition}{Condition}
  \providecommand{Corollary}{Corollary}
  \providecommand{Definition}{Definition}
  \providecommand{Example}{Example}
  \providecommand{Lemma}{Lemma}
  \providecommand{Proposition}{Proposition}
  \providecommand{Remark}{Remark}
\providecommand{Theorem}{Theorem}

\begin{document}

\title{$\mathfrak{S}$-coalgebras determine fundamental groups}

\author{Justin R. Smith}

\subjclass[2000]{Primary 18G55; Secondary 55U40}

\keywords{operads, cofree coalgebras}

\curraddr{Department of Mathematics\\
Drexel University\\
Philadelphia,~PA 19104}

\email{jsmith@drexel.edu}

\urladdr{http://vorpal.math.drexel.edu}

\maketitle
\global\long
\global\long
\global\long
 \global\long
\global\long
\global\long
\global\long
\global\long
\global\long
\global\long
\global\long
\global\long
\global\long
\global\long
\global\long
\global\long
\global\long
\global\long
\global\long
\global\long
\global\long
\global\long
\global\long
\global\long
\global\long
\global\long
\global\long
\global\long
\global\long
\global\long
\global\long
\global\long
\global\long
\global\long
\global\long
\global\long
\global\long
\global\long
\global\long
\global\long
\global\long
\global\long
\global\long
\global\long
\global\long
\global\long
\global\long
\global\long
\global\long
\global\long
\global\long
\global\long

\global\long
\global\long

\global\long

\global\long
\global\long

\global\long
\global\long
\global\long

\global\long

\global\long

\global\long
\global\long

\global\long
 

\global\long

\global\long

\global\long

\global\long

\global\long

\global\long

\global\long

\newdir{ >}{{}*!/-5pt/@{>}}

\global\long
 

\global\long

\global\long

\global\long

\global\long

\global\long

\global\long

\global\long

\global\long

\global\long

\global\long

\global\long

\global\long

\global\long

\global\long

\global\long

\global\long

\global\long

\global\long

\global\long

\global\long

\global\long

\global\long

\global\long
 

\date{\today}
\begin{abstract}
In this paper, we extend earlier work by showing that if $X$ and
$Y$ are simplicial complexes (i.e. simplicial sets whose nondegenerate
simplices are determined by their vertices), an \emph{isomorphism}
${\mathcal{N}(X)}\cong{\mathcal{N}(Y)}$ of ${\mathfrak{S}}$-coalgebras implies that the 3-skeleton
of $X$ is weakly equivalent to the 3-skeleton of $Y$, also implying
that $\pi_{1}(X)=\pi_{1}(Y)$.
\end{abstract}

\section{Introduction}

In \cite{Smith:1994}, the author defined the functor ${\mathcal{C}(* )}$ on
simplicial sets --- essentially the chain complex equipped with the
structure of a coalgebra over an operad ${\mathfrak{S}}$. This coalgebra structure
determined all Steenrod and other cohomology operations. Since these
coalgebras are not \emph{nilpotent}\footnote{In a nilpotent coalgebra, iterated coproducts of elements ``peter
out'' after a finite number of steps. See\foreignlanguage{american}{
\cite[chapter~3]{operad-book} for the precise definition.}}\emph{ }they have a kind of ``transcendental'' structure that contains
much more information.

In section~\ref{sec:The-functor-cfn}, we define a variant of the
${\mathcal{C}(* )}$-functor, named ${\mathcal{N}(*)}$. It is defined for simplicial complexes
--- semi-simplicial sets whose simplices are uniquely determined by
their vertices. The script-N emphasizes that its underlying chain-complex
is \emph{normalized} and ${\mathcal{C}(* )}$ can be views as an extension of
${\mathcal{N}(X)}$ to general simplicial sets (see \cite{smith-cellular}).

In \cite{smith-cellular}, we showed that if $X$ and $Y$ are pointed,
reduced simplicial sets, then a quasi-isomorphism ${\mathcal{C}(X )}\to{\mathcal{C}(Y )}$
induces one of their ${\mathbb{Z}}$-completions ${\mathbb{Z}}_{\infty}X\to{\mathbb{Z}}_{\infty}Y$.
It follows that that the ${\mathcal{C}(* )}$-functor determine a \emph{nilpotent
}space's weak homotopy type.

In the present paper, we extend this by showing:

\medskip{}

Corollary. \ref{cor:cellular-determines-pi1}. \emph{If $X$ and $Y$
are simplicial complexes with the property that there exists an isomorphism
\[
g:{\mathcal{N}(X)}\to{\mathcal{N}(Y)}
\]
then their 3-skeleta are isomorphic and 
\[
\pi_{1}(X)\cong\pi_{1}(Y)
\]
}

\medskip{}

This implies that the functors ${\mathcal{C}(* )}$ and ${\mathcal{N}(*)}$ encapsulate
``non-abelian'' information about a simplicial set --- such as its
(possibly non-nilpotent) fundamental group. The requirement that $g$
be an \emph{isomorphism} is stronger than needed for this (but \emph{quasi}-isomorphism
is not enough). 

The proof actually requires $X$ and $Y$ to be simplicial \emph{complexes}
rather than general simplicial sets. The author conjectures that the
${\mathcal{C}(* )}$-functor determines the integral homotopy type of an arbitrary
simplicial set.

Since the transcendental portion of ${\mathcal{C}(X )}$ can be mapped to a power
series ring (see the proof of lemma~\ref{lem:diagonals-linearly-independent}),
the analysis of this data may require methods of analysis and algebraic
geometry.

\section{Definitions and assumptions}
\begin{defn}
\label{defr:chaincat}Let ${\mathbf{Ch}}$ denote the category of ${\mathbb{Z}}$-graded
${\mathbb{Z}}$-free chain complexes and let ${{\mathbf{Ch}}_{0}}\subset{\mathbf{Ch}}$
denote the subcategory of chain complexes concentrated in positive
dimensions. 

If $c\in{\mathbf{Ch}}$, 
\[
C^{\otimes n}=\underbrace{C\otimes_{\mathbb{Z}}\otimes\cdots\otimes_\intsC}_{n\,\text{times }}
\]

\end{defn}
We also have categories of spaces:
\begin{defn}
\label{def:simplicial-set-complex}Let ${\mathbf{SS}}$ denote the category
of simplicial sets and ${\mathbf{SC}}$ that of simplicial \emph{complexes.}
A simplicial \emph{complex} is a simplicial set without degeneracies
(i.e., a semi-simplicial set) with the property that simplices are
uniquely determined by their vertices.\end{defn}
\begin{rem*}
Following \cite{may-finite}, we can define a simplicial set to have
\emph{Property~A} if every face of a nondegenerate simplex is nondegenerate.
Theorem~12.4.4 of \cite{may-finite} proves that simplicial sets
with property~A have \emph{second subdivisions} that are simplicial
complexes. The bar-resolution ${\mathrm{R}S_{2 }}$ is an example of a simplicial
set that does \emph{not} have property~A.

On the other hand, it is well-known that \emph{all} topological spaces
are weakly homotopy equivalent to simplicial complexes --- see, for
example, Theorem~2C.5 and Proposition~4.13 of \cite{hatcher-alg-top}.
\end{rem*}
We make extensive use of the Koszul Convention (see~\cite{Gugenheim:1960})
regarding signs in homological calculations:
\begin{defn}
\label{def:koszul-1} If $f:C_{1}\to D_{1}$, $g:C_{2}\to D_{2}$
are maps, and $a\otimes b\in C_{1}\otimes C_{2}$ (where $a$ is a
homogeneous element), then $(f\otimes g)(a\otimes b)$ is defined
to be $(-1)^{\deg(g)\cdot\deg(a)}f(a)\otimes g(b)$. \end{defn}
\begin{rem}
If $f_{i}$, $g_{i}$ are maps, it isn't hard to verify that the Koszul
convention implies that $(f_{1}\otimes g_{1})\circ(f_{2}\otimes g_{2})=(-1)^{\deg(f_{2})\cdot\deg(g_{1})}(f_{1}\circ f_{2}\otimes g_{1}\circ g_{2})$.\end{rem}
\begin{defn}
\label{def:homcomplex-1}Given chain-complexes $A,B\in{\mathbf{Ch}}$
define
\[
{\mathrm{Hom}_{\mathbb{Z}}}(A,B)
\]
to be the chain-complex of graded ${\mathbb{Z}}$-morphisms where the degree
of an element $x\in{\mathrm{Hom}_{\mathbb{Z}}}(A,B)$ is its degree as a map and with differential
\[
\partial f=f\circ\partial_{A}-(-1)^{\deg f}\partial_{B}\circ f
\]
As a ${\mathbb{Z}}$-module ${\mathrm{Hom}_{\mathbb{Z}}}(A,B)_{k}=\prod_{j}{\mathrm{Hom}_{\mathbb{Z}}}(A_{j},B_{j+k})$.\end{defn}
\begin{rem*}
Given $A,B\in\mathbf{Ch}^{S_{n}}$, we can define ${\mathrm{Hom}_{{\mathbb{Z}} S_{n }}}(A,B)$
in a corresponding way.\end{rem*}
\begin{defn}
\label{def:tmap} Let $\alpha_{i}$, $i=1,\dots,n$ be a sequence
of nonnegative integers whose sum is $|\alpha|$. Define a set-mapping
 of symmetric groups 
\[
{{\mathrm{T}_{{{{{{{\alpha}}}_{1},\dots,{{{\alpha}}}_{{{n}}}}}}}}}:S_{n}\to S_{|\alpha|}
\]
 as follows:
\begin{enumerate}
\item for $i$ between 1 and $n$, let $L_{i}$ denote the length-$\alpha_{i}$
integer sequence: 
\item ,where $A_{i}=\sum_{j=1}^{i-1}\alpha_{j}$ --- so, for instance, the
concatenation of all of the $L_{i}$ is the sequence of integers from
1 to $|\alpha|$; 
\item ${{\mathrm{T}_{{{{{{{\alpha}}}_{1},\dots,{{{\alpha}}}_{{{n}}}}}}}}}(\sigma)$ is the permutation on the integers $1,\dots,|\alpha|$
that permutes the blocks $\{L_{i}\}$ via $\sigma$. In other words,
 $\sigma$ s the permutation 
\[
\left(\begin{array}{ccc}
1 & \dots & n\\
\sigma(1) & \dots & \sigma(n)
\end{array}\right)
\]
 then ${{\mathrm{T}_{{{{{{{\alpha}}}_{1},\dots,{{{\alpha}}}_{{{n}}}}}}}}}(\sigma)$ is the permutation defined by writing
\[
\left(\begin{array}{ccc}
L_{1} & \dots & L_{n}\\
L_{\sigma(1)} & \dots & L_{\sigma(n)}
\end{array}\right)
\]
 and regarding the upper and lower rows as sequences length $|\alpha|$. 
\end{enumerate}
\end{defn}
\begin{rem*}
Do not confuse the $T$-maps defined here with the transposition map
for tensor products of chain-complexes. We will use the special notation
$T_{i}$ to represent $T_{1,\dots,2,\dots,1}$, where the 2 occurs
in the $i^{\mathrm{th}}$ position. The two notations don't conflict
since the old notation is never used in the case when $n=1$. Here
is an example of the computation of ${\mathrm{T}_{{2,1,3}}}((1,3,2))={\mathrm{T}_{{2,1,3}}}\left(\begin{array}{ccc}
1 & 2 & 3\\
3 & 1 & 2
\end{array}\right)$:$L_{1}=\{1\}2$, $L_{2}=\{3\}$, $L_{3}=\{4,5,6\}$. The permutation
maps the ordered set $\{1,2,3\}$ to $\{3,1,2\}$, so we carry out
the corresponding mapping of the sequences $\{L_{1},L_{2},L_{3}\}$
to get $\left(\begin{array}{ccc}
L_{1} & L_{2} & L_{3}\\
L_{3} & L_{1} & L_{2}
\end{array}\right)=\left(\begin{array}{ccc}
\{1,2\} & \{3\} & \{4,5,6\}\\
\{4,5,6\} & \{1,2\} & \{3\}
\end{array}\right)=\left(\begin{array}{cccccc}
1 & 2 & 3 & 4 & 5 & 6\\
4 & 5 & 6 & 1 & 2 & 3
\end{array}\right)$ (or $((1,4)(2,5)(3,6))$, in cycle notation).\end{rem*}
\begin{defn}
\label{def:operad} A sequence of differential graded $\mathbb{Z}$-free
modules, $\{\mathcal{V}_{i}\}$, will be said to form an \emph{operad}
if:
\begin{enumerate}
\item there exists a \emph{unit map} (defined by the commutative diagrams
below) 
\[
\eta:\mathbb{Z}\to\mathcal{V}_{1}
\]

\item for all $i>1$, $\mathcal{V}_{i}$ is equipped with a left action
of $S_{i}$, the symmetric group. 
\item for all $k\ge1$, and $i_{s}\ge0$ there are maps 
\[
\gamma:\mathcal{V}_{i_{1}}\otimes\cdots\otimes\mathcal{V}_{i_{k}}\otimes\mathcal{V}_{k}\to\mathcal{V}_{i}
\]
 where $i=\sum_{j=1}^{k}i_{j}$.

The $\gamma$-maps must satisfy the conditions:

\end{enumerate}
\end{defn}
\begin{description}
\item [{Associativity}] the following diagrams commute, where $\sum j_{t}=j$,
$\sum i_{s}=i$, and $g_{\alpha}=\sum_{\ell=1}^{\alpha}j_{\ell}$
and $h_{s}=\sum_{\beta=g_{s-1}+1}^{g_{s}}i_{\beta}$: \foreignlanguage{american}{
\[
{\makeatletter \xydef@\xymatrixcolsep@{{50pt}} \makeatother }{\makeatletter \xydef@\xymatrixrowsep@{{15pt}} \makeatother }\xymatrix{{\left(\bigotimes_{s=1}^{j}\mathcal{V}_{i_{s}}\right)\otimes\left(\bigotimes_{t=1}^{k}\mathcal{V}_{j_{t}}\right)\otimes\mathcal{V}_{k}}\ar[r]^{{\qquad\text{Id}\otimes\gamma}}\ar[dd]_{\text{shuffle}} & {\left(\bigotimes_{s=1}^{j}\mathcal{V}_{i_{s}}\right)\otimes\mathcal{V}_{j}}\ar[d]^{\gamma}\\
 & {\mathcal{V}_{i}}\\
{\left(\left(\bigotimes_{q=1}^{j_{t}}\mathcal{V}\right)\otimes\bigotimes_{t=1}^{k}\mathcal{V}_{j_{t}}\right)\otimes\mathcal{V}_{k}}\ar[r]_{{\qquad(\otimes_{t}\gamma)\otimes\text{Id}}} & {\left(\bigotimes_{t=1}^{k}\mathcal{V}_{h_{k}}\right)\otimes\mathcal{V}_{k}}\ar[u]_{\gamma}
}
\]
}
\item [{Units}] the diagrams  \[\begin{array}{cc}\xymatrix{{{\mathbb{Z}}^{k}\otimes\mathcal{V}_{k}}\ar[r]^{\cong}\ar[d]_{{\eta}^{k}\otimes\text{Id}}&{\mathcal{V}_{k}}\\
{{\mathcal{V}_{1}}^{k}\otimes{\mathcal{V}_{k}}}\ar[ur]_{\gamma}&}&\xymatrix{{\mathcal{V}_{k}\otimes{\mathbb{Z}}}\ar[r]^{\cong}\ar[d]_{\text{Id}\otimes\eta}&{\mathcal{V}_{k}}\\
{\mathcal{V}_{k}\otimes\mathcal{V}_{1}}\ar[ur]_{\gamma}&}\end{array}\] commute.
\item [{Equivariance}] the diagrams  \[\xymatrix@C+20pt{{\mathcal{V}_{j_{1}}\otimes\cdots\otimes\mathcal{V}_{j_{k}}\otimes\mathcal{V}_{k}}\ar[r]^-{\gamma}\ar[d]_{\sigma^{-1}\otimes\sigma}&{\mathcal{V}_{j}}\ar[d]^{{\mathrm{T}_{{j_{1},\dots,j_{k}}}}(\sigma)}\\
{\mathcal{V}_{j_{\sigma(1)}}\otimes\cdots\otimes\mathcal{V}_{j_{\sigma(k)}}\otimes\mathcal{V}_{k}}\ar[r]_-{\gamma}&{\mathcal{V}_{j}}}\]commute, where $\sigma\in S_{k}$, and the $\sigma^{-1}$ on the left
permutes the factors $\{\mathcal{V}_{j_{i}}\}$ and the $\sigma$
on the right simply acts on $\mathcal{V}_{k}$. See \ref{def:tmap}
for a definition of ${\mathrm{T}_{{j_{1},\dots,j_{k}}}}(\sigma)$. \[\xymatrix@C+20pt{{\mathcal{V}_{j_{1}}\otimes\cdots\otimes\mathcal{V}_{j_{k}}\otimes\mathcal{V}_{k}}\ar[r]^-{\gamma}\ar[d]_{\tau_{1}\otimes\cdots\tau_{k}\otimes\text{Id}}&{\mathcal{V}_{j}}\ar[d]^-{\tau_{1}\oplus\cdots\oplus\tau_{k}}\\
{\mathcal{V}_{j_{\sigma(1)}}\otimes\cdots\otimes\mathcal{V}_{j_{\sigma(k)}}\otimes\mathcal{V}_{k}}\ar[r]_-{\gamma}&{\mathcal{V}_{j}}}\] where $\tau_{s}\in S_{j_{s}}$ and $\tau_{1}\oplus\cdots\oplus\tau_{k}\in S_{j}$
is the block sum.\end{description}
\begin{rem*}
The alert reader will notice a discrepancy between our definition
of operad and that in \cite{Kriz-May} (on which it was based). The
difference is due to our using operads as parameters for systems of
\emph{maps}, rather than $n$-ary operations. We, consequently, compose
elements of an operad as one composes \emph{maps}, i.e. the second
operand is to the \emph{left} of the first. This is also why the symmetric
groups act on the \emph{left} rather than on the right. \end{rem*}
\begin{defn}
\label{def:unitaloperad}An operad, $\mathcal{V}$, will be called
\emph{unital} if $\mathcal{V}$ has a $0$-component $\mathcal{V}_{0}={\mathbb{Z}}$,
concentrated in dimension $0$ and augmentations
\[
\epsilon_{n}:\mathcal{V}_{0}\otimes\cdots\otimes\mathcal{V}_{0}\otimes\mathcal{V}_{n}=\mathcal{V}_{n}\to\mathcal{V}_{0}={\mathbb{Z}}
\]
 induced by their structure maps.\end{defn}
\begin{rem*}
The term ``unital operad'' is used in different ways by different
authors. We use it in the sense of Kriz and May in \cite{Kriz-May},
meaning the operad has a $0$-component that acts like an arity-lowering
augmentation under compositions. 
\end{rem*}
We will frequently want to think of operads in other terms:
\begin{defn}
\label{def:operadcomps} Let $\mathcal{V}$ be an operad, as defined
above. Given $i\le k_{1}>0$, define the $i^{\mathrm{th}}$ \emph{composition}

\[
\circ_{i}:\mathcal{V}_{k_{2}}\otimes\mathcal{V}_{k_{1}}\to\mathcal{V}_{k_{1}+k_{2}-1}
\]
 as the composite
\begin{multline*}
\underbrace{{\mathbb{Z}}\otimes\cdots\otimes{\mathbb{Z}}\otimes\mathcal{V}_{k_{2}}\otimes{\mathbb{Z}}\otimes\cdots\otimes{\mathbb{Z}}}_{\text{\ensuremath{i^{\text{th}}}factor}}\otimes\mathcal{V}_{k_{1}}\\
\to\underbrace{\mathcal{V}_{1}\otimes\cdots\otimes\mathcal{V}_{1}\otimes\mathcal{V}_{k_{2}}\otimes\mathcal{V}_{1}\otimes\cdots\otimes\mathcal{V}_{1}}_{\text{\ensuremath{i^{\text{th}}}factor}}\otimes\mathcal{V}_{k_{1}}\to\mathcal{V}_{k_{1}+k_{2}-1}
\end{multline*}
 where the final map on the right is $\gamma$. 

These compositions satisfy the following conditions, for all $a\in\mathscr{U}_{n}$,
$b\in\mathscr{U}_{m}$, and $c\in\mathscr{U}_{t}$:
\begin{description}
\item [{Associativity}] $(a\circ_{i}b)\circ_{j}c=a\circ_{i+j-1}(b\circ_{j}c)$
\item [{Commutativity}] $a\circ_{i+m-1}(b\circ_{j}c)=(-1)^{mn}b\circ_{j}(a\circ_{i}c)$ 
\item [{Equivariance}] $a\circ_{\sigma(i)}(\sigma\cdot b)={{{\mathrm{T}_{{\underbrace{{\scriptstyle {1,\dots,n,\dots,1}}}_{{i^{\mathrm{th}}\ \mathrm{position}}}}}}}}(\sigma)\cdot(a\circ_{i}b)$ 
\end{description}
\end{defn}
\begin{rem*}
I am indebted to Jim Stasheff for pointing out to me that operads
were originally defined this way and called \emph{composition algebras.}
Given this definition of operad, we recover the $\gamma$ map in definition~\ref{def:operad}
by setting: 
\[
\gamma(u_{i_{1}}\otimes\cdots\otimes u_{i_{k}}\otimes u_{k})=u_{i_{1}}\circ_{1}\cdots\circ_{k-1}u_{i_{k}}\circ_{k}u_{k}
\]
 (where the implied parentheses associate to the right). It is left
to the reader to verify that the two definitions are equivalent (the
commutativity condition, here, is a special case of the equivariance
condition). Given a \emph{unital} operad, we can use the augmentation
maps to recover the composition operations.
\end{rem*}
A simple example of an operad is:
\begin{example}
\label{example:frakS0}For each $n\ge0$, $X$, the operad ${\mathfrak{S}}_{0}$
has ${\mathfrak{S}}_{0}(n)={\mathbb{Z}} S_{n}$, concentrated in dimension $0$,
with structure-map induced by
\begin{eqnarray*}
\gamma_{\alpha_{1},\dots,\alpha_{n}}:S_{\alpha_{1}}\times\cdots\times S_{\alpha_{n}}\times S_{n} & \to & S_{\alpha_{1}+\cdots+\alpha_{n}}\\
\sigma_{\alpha_{1}}\times\cdots\times\sigma_{\alpha_{n}}\times\sigma_{n} & \mapsto & {{\mathrm{T}_{{{{{{{\alpha}}}_{1},\dots,{{{\alpha}}}_{{{n}}}}}}}}}(\sigma_{n})\circ(\sigma_{\alpha_{1}}\oplus\cdots\oplus\sigma_{\alpha_{n}})
\end{eqnarray*}
In other words, each of the $S_{\alpha_{i}}$ permutes elements within
the subsequence $\{\alpha_{1}+\cdots+\alpha_{i-1}+1,\dots,\alpha_{1}+\cdots+\alpha_{i}\}$
of the sequence $\{1,\dots,\alpha_{1}+\cdots+\alpha_{n}\}$ and $S_{n}$
permutes these $n$ blocks. 
\end{example}
For the purposes of this paper, the main example of an operad is
\begin{defn}
\label{def:coend}Given any $C\in{\mathbf{Ch}}$, the associated \emph{coendomorphism
operad}, ${\mathrm{CoEnd}}(C)$ is defined by
\[
{\mathrm{CoEnd}}(C)(n)={\mathrm{Hom}_{\mathbb{Z}}}(C,C^{\otimes n})
\]
 Its structure map
\begin{multline*}
\gamma_{\alpha_{1},\dots,\alpha_{n}}:{\mathrm{Hom}_{\mathbb{Z}}}(C,C^{\otimes n})\otimes{\mathrm{Hom}_{\mathbb{Z}}}(C,C^{\otimes\alpha_{1}})\otimes\cdots\otimes{\mathrm{Hom}_{\mathbb{Z}}}(C,C^{\otimes\alpha_{n}})\to\\
{\mathrm{Hom}_{\mathbb{Z}}}(C,C^{\otimes\alpha_{1}+\cdots+\alpha_{n}})
\end{multline*}
simply composes a map in ${\mathrm{Hom}_{\mathbb{Z}}}(C,C^{\otimes n})$ with maps of each
of the $n$ factors of $C$. 

This is a non-unital operad, but if $C\in{\mathbf{Ch}}$ has an augmentation
map $\varepsilon:C\to{\mathbb{Z}}$ then we can  regard $\epsilon$ as the
only element of ${\mathrm{Hom}_{\mathbb{Z}}}(C,C^{\otimes n})={\mathrm{Hom}_{\mathbb{Z}}}(C,C^{\otimes0})={\mathrm{Hom}_{\mathbb{Z}}}(C,{\mathbb{Z}})$.
\end{defn}
Morphisms of operads are defined in the obvious way:
\begin{defn}
\label{def:operadmorphism} Given two operads $\mathcal{V}$ and $\mathcal{W}$,
a \emph{morphism} 
\[
f:\mathcal{V}\to\mathcal{W}
\]
 is a sequence of chain-maps 
\[
f_{i}:\mathcal{V}_{i}\to\mathcal{W}_{i}
\]
 commuting with all the diagrams in \ref{def:operad}.
\end{defn}
Verification that this satisfies the required identities is left to
the reader as an exercise.
\begin{defn}
\label{def:sfrakfirstmention}Let ${\mathfrak{S}}$ denote the operad defined
in \cite{Smith:1994} --- where ${\mathfrak{S}}_{n}={\mathrm{R}S_{n }}$ is the normalized
bar-resolution of ${\mathbb{Z}}$ over ${{\mathbb{Z}} S_{n }}$ for all $n>0$. This
is similar to the Barratt-Eccles operad defined in \cite{Barratt-Eccles-operad},
except that the latter is composed of \emph{unnormalized} bar-resolutions.
See \cite{Smith:1994} or appendix~A of \cite{smith-cellular}, for
the details. 

Appendix~A of \cite{smith-cellular} contains explicit computations
of some composition-operations in ${\mathfrak{S}}$.
\end{defn}
Now we are ready to define the all-important concept of \emph{coalgebras}
over an operad:
\begin{defn}
\label{def:coalgebra-over-operad-1}A chain-complex $C$ is a \emph{coalgebra
over the operad} $\mathcal{V}$ if there exists a morphism of operads
\[
\mathcal{V}\to{\mathrm{CoEnd}}(C)
\]
\end{defn}
\begin{rem*}
A coalgebra, $C$, over an operad, $\mathcal{V}$, is a sequence of
maps 
\[
f_{n}:\mathcal{V}_{n}\otimes C\to C^{\otimes n}
\]
 for all $n>0$, where $f_{n}$ is ${{\mathbb{Z}} S_{n }}$-equivariant and $S_{n}$
acts by permuting factors of $C^{\otimes n}$. The maps, $\{f_{n}\}$,
are related in the sense that they fit into commutative diagrams:
\begin{equation}
{\makeatletter \xydef@\xymatrixcolsep@{{20pt}} \makeatother }\xymatrix{{\mathscr{\mathcal{V}}_{n}\otimes\mathscr{\mathcal{V}}_{m}\otimes C}\ar[r]^{\circ_{i}} & {\mathscr{\mathcal{V}}_{n+m-1}\otimes C}\ar[r]^{f_{n+m-1}} & {C^{\otimes n+m-1}}\\
{\mathscr{\mathcal{V}}_{n}\otimes\mathscr{\mathcal{V}}_{m}\otimes C}\ar[r]_{1\otimes f_{m}}\ar@{=}[u] & {\mathscr{\mathcal{V}}_{n}\otimes C^{\otimes m}}\ar[r]_{Z_{i-1}\qquad\quad} & {C^{i-1}\otimes\mathscr{\mathcal{V}}_{n}\otimes C\otimes C^{\otimes m-i}}\ar[u]_{1\otimes\dots\otimes f_{n}\otimes\dots\otimes1}
}
\label{dia:coalgebra-over-operad}
\end{equation}
 for all $n,m\ge1$ and $1\le i\le m$. Here $Z_{i-1}:\mathcal{V}_{n}\otimes C^{m}\to C^{\otimes i-1}\otimes\mathcal{V}_{n}\otimes C\otimes C^{\otimes m-i}$
is the map that shuffles the factor $\mathcal{V}_{n}$ to the right
of $i-1$ factors of $C$. In other words: The abstract composition-operations
in $\mathcal{V}$ exactly correspond to compositions of maps in $\{{\mathrm{Hom}_{\mathbb{Z}}}(C,C^{\otimes n})\}$.
We exploit this behavior in applications of coalgebras over operads,
using an explicit knowledge of the algebraic structure of $\mathcal{V}$.
\end{rem*}
The structure of a coalgebra over an operad can also be described
in several equivalent ways:
\begin{enumerate}
\item $f_{n}:\mathcal{V}(n)\otimes C\to C^{\otimes n}$
\item $g:C\to\prod_{n=0}^{\infty}{\mathrm{Hom}_{{\mathbb{Z}} S_{n }}}(\mathcal{V}(n),C^{\otimes n})$
\end{enumerate}
where both satisfy identities that describe how composites of these
maps are compatible with the operad-structure.
\begin{defn}
\label{def:coalgebra-over-operad}A chain-complex $C$ is a \emph{coalgebra
over the operad} $\mathcal{V}$ if there exists a morphism of operads
\[
\mathcal{V}\to{\mathrm{CoEnd}}(C)
\]
\end{defn}
\begin{rem*}
The structure of a coalgebra over an operad can be described in several
equivalent ways:
\begin{enumerate}
\item $f_{n}:\mathcal{V}(n)\otimes C\to C^{\otimes n}$
\item $g:C\to\prod_{n=0}^{\infty}{\mathrm{Hom}_{{\mathbb{Z}} S_{n }}}(\mathcal{V}(n),C^{\otimes n})$
\end{enumerate}
\end{rem*}
where both satisfy identities that describe how composites of these
maps are compatible with the operad-structure.
\begin{defn}
\label{def:s-coalgebra-morphism}Using the second description,
\[
\alpha_{C}:C\to\prod_{n=1}^{\infty}{\mathrm{Hom}_{{\mathbb{Z}} S_{n }}}({\mathrm{R}S_{n }},C^{\otimes n})
\]
an ${\mathfrak{S}}$-coalgebra morphism 
\[
f:C\to D
\]
is a chain-map that makes the diagram
\[
\xymatrix{{{\lceilC\rceil}}\ar[d]_{{\lceilf\rceil}}\ar[r]^{\alpha_{C}\qquad\qquad\quad} & {\prod_{n=1}^{\infty}{\mathrm{Hom}_{{\mathbb{Z}} S_{n }}}({\mathrm{R}S_{n }},{\lceilC\rceil}^{\otimes n})}\ar[d]^{\prod_{n=1}^{\infty}{\mathrm{Hom}_{{\mathbb{Z}} S_{n }}}(1,{\lceilf\rceil}^{\otimes n})}\\
{{\lceilD\rceil}}\ar[r]_{\alpha_{D}\qquad\qquad\quad} & {\prod_{n=1}^{\infty}{\mathrm{Hom}_{{\mathbb{Z}} S_{n }}}({\mathrm{R}S_{n }},{\lceilD\rceil}^{\otimes n})}
}
\]
commute, where ${\lceil*\rceil}$ is the forgetful functor that turns
a coalgebra into a chain-complex.

We also need
\end{defn}

\section{morphisms of ${\mathfrak{S}}$-coalgebras\label{sec:morphisms}}

Proposition~\ref{pro:simplicespropertyS} proves that if $e_{n}=\underbrace{[(1,2)|\cdots|(1,2)]}_{n\text{ terms}}\in{\mathrm{R}S_{2 }}$
and $x\in{\mathcal{N}(X)}$ is the image of a $k$-simplex, then
\[
f_{2}(e_{k}\otimes x)=\xi_{k}\cdot x\otimes x
\]
where $\xi_{k}=(-1)^{k(k-1)/2}$.
\begin{defn}
\label{def:gamma-m-map}If \foreignlanguage{american}{$k,m$ are positive
integers, $C$ is a chain-complex, and $F_{2,m}=e_{m}$ and $F_{k,m}=\underbrace{e_{m}\circ_{1}\cdots\circ_{1}e_{m}}_{k-1\text{ iterations}}\in RS_{k}$
--- compositions in the operad ${\mathfrak{S}}$ --- set 
\[
\rho_{m}=(\xi_{m}\cdot E_{2,m},\xi_{m}^{2}\cdot E_{3,m},\xi_{m}^{3}\cdot E_{4,m},\dots)\in\prod_{n=2}^{\infty}RS_{n}
\]
with $\xi_{m}=(-1)^{m(m-1)/2}$ and define 
\[
\gamma_{m}:\prod_{n=2}^{\infty}{\mathrm{Hom}_{{\mathbb{Z}} S_{n }}}({\mathrm{R}S_{n }},C_{m}^{\otimes n})\to\prod_{n=2}^{\infty}C^{\otimes n}
\]
via evaluation on $\rho_{m}$.}
\end{defn}
We have
\begin{cor}
\label{cor:simplex-image}If $X$ is a simplicial set and $c\in{\mathcal{C}(X )}$
is an element generated by an $n$-simplex, then the image of $c$
under the composite
\[
{\mathcal{N}(X)}_{n}\xrightarrow{\alpha_{{\mathcal{C}(X )}}}\prod_{k=1}^{\infty}{\mathrm{Hom}_{{\mathbb{Z}} S_{k }}}({\mathrm{R}S_{k }},{\mathcal{N}(X)}^{\otimes k})\xrightarrow{\gamma_{n}}\prod_{k=1}^{\infty}{\mathcal{N}(X)}^{\otimes k}
\]
 is 
\[
e(c)=(c,c\otimes c,\dots)
\]
\end{cor}
\begin{proof}
This follows immediately from proposition~\ref{pro:simplicespropertyS}
and the fact that operad-compositions map to compositions of coproducts.
\end{proof}
Lemma~\ref{lem:diagonals-linearly-independent} implies that
\begin{cor}
\label{cor:n-simplices-map-to-simplices}Let $X$ be a simplicial
set and suppose 
\[
f:{\mathcal{N}^{n}}={\mathcal{N}({\Delta^{n}})}\to{\mathcal{N}(X)}
\]
 is a ${\mathfrak{S}}$-coalgebra morphism. Then the image of the generator $\Delta^{n}\in{\mathcal{N}({\Delta^{n}})}$
is a generator of ${\mathcal{N}(X)}$ defined by an $n$-simplex of $X$.\end{cor}
\begin{proof}
Suppose 
\[
f(\Delta^{n})=\sum_{k=1}^{t}c_{k}\cdot\sigma_{k}^{n}\in{\mathcal{N}(X)}
\]
where the $\sigma_{k}^{n}$ are images of $n$-simplices of $X$.
If $f(\Delta^{n})$ is not equal to one of the $\sigma_{k}^{n}$,
lemma~\ref{lem:diagonals-linearly-independent} implies that its
image is linearly independent of the $\sigma_{k}^{n}$, a \emph{contradiction.}
The statement about sub-simplices follows from the main statement.
\end{proof}
We also conclude that:
\begin{cor}
\label{cor:automorphisms-trivial}If $f:{\mathcal{N}({\Delta^{n}})}\to{\mathcal{N}({\Delta^{n}})}$
is 
\begin{enumerate}
\item an isomorphism of ${\mathfrak{S}}$-algebras in dimension $n$ and 
\item an endomorphism in lower dimensions 
\end{enumerate}
then $f$ must be an isomorphism. If $n\le3$, then $f$ must also
be the identity map.\end{cor}
\begin{rem*}
The final statement actually works for some larger values of $n$,
but the arguments become vastly more complicated (requiring the use
of higher coproducts). It would have extraordinary implications if
it were true for \emph{all} $n$.\end{rem*}
\begin{proof}
We first show that $f$ must be an isomorphism. We are given that
$f$ is an isomorphism in dimension $n$. We use downward induction
on dimension to show that it is an isomorphism in lower dimensions:

Suppose $f$ is an isomorphism in dimension $k$ and $\Delta^{k}\subset\Delta^{n}$
is a simplex. The boundary of $\Delta^{k}$ is a linear combination
of $k+1$ distinct faces which corollary\foreignlanguage{american}{~\ref{cor:n-simplices-map-to-simplices}
implies must map to $k-1$-dimensional simplices with the \emph{same}
coefficients (of $\pm1$). The Pigeonhole Principal and the fact that
$f$ is a \emph{chain-map} imply that all of the $k+1$ distinct faces
of $f(\Delta^{k})$ must be in the image of $f$ so that $f$ induces
a 1-1 correspondence between faces of $\Delta^{k}$ and those of $f(\Delta^{k})$.
It follows that $f|{\mathcal{N}({\Delta^{k}})}$ is an isomorphism in dimension
$k-1$. Since $\Delta^{k}$ was arbitrary, it follows that $f$ is
an isomorphism in dimension $k-1$.}

It follows that $f$ is actually an \emph{automorphism} of ${\mathcal{N}({\Delta^{n}})}$.
Now we assume that $n\le3$ and show that $f$ is the \emph{identity
map:}

If $n=1$ then corollary~\ref{cor:n-simplices-map-to-simplices}
implies that $f|{\mathcal{N}({\Delta^{1}})}_{1}=1$. Since the $0$-simplices
must map to $0$-simplices (by corollary~\ref{cor:n-simplices-map-to-simplices})
with a $+1$ sign it follows that the only possible non-identity automorphism
of ${\mathcal{N}({\Delta^{1}})}$ swaps the ends of $\Delta^{1}$ --- but this
would violate the condition that $f$ is a chain-map.

In dimension 2, let $\Delta^{2}$ be a $2$-simplex. Similar reasoning
to that used in the one-dimensional case implies that a non-identity
automorphism of ${\mathcal{N}({\Delta^{2}})}$ would (at most) involve permuting
some of its faces. Since
\[
\partial\Delta^{2}=F_{0}\Delta^{2}-F_{1}\Delta^{2}+F_{2}\Delta^{2}
\]
the only non-identity permutation compatible with the boundary map
swaps $F_{0}\Delta^{2}$ and $F_{2}\Delta^{2}$. But the coproduct
of $\Delta^{2}$ is given by
\[
[\,]_{2}\otimes\Delta^{2}\mapsto\Delta^{2}\otimes F_{0}F_{1}\Delta^{2}+F_{2}\Delta^{2}\otimes F_{0}\Delta^{2}+F_{1}F_{2}\Delta^{2}\otimes\Delta^{2}
\]
(see proposition~\ref{prop:e1timesdelta2}) where $[\,]$ is the
0-dimensional generator of ${\mathrm{R}S_{2 }}$ --- the bar-resolution of ${\mathbb{Z}}$
over ${{\mathbb{Z}} S_{{_{2}} }}$. It follows that swapping $F_{0}\Delta^{2}$ and
$F_{2}\Delta^{2}$ would violate the condition that $f$ must preserve
coproducts. The case where $n=1$ implies that the vertices cannot
be permuted.

When $n=3$, we have 
\[
\partial\Delta^{3}=F_{0}\Delta^{3}-F_{1}\Delta^{3}+F_{2}\Delta^{3}-F_{3}\Delta^{3}
\]
so, in principal, we might be able to swap $F_{0}\Delta^{3}$ and
$F_{2}\Delta^{3}$ or $F_{1}\Delta^{3}$ and $ $$F_{3}\Delta^{3}$.
The coproduct does not rule any of these actions out since it involves
multiple face-operations. The first ``higher coproduct'' does, however
--- see \ref{prop:e1timesdelta3}:\foreignlanguage{english}{
\begin{align}
f_{2}([(1,2)]\otimes\Delta^{3}) & =F_{1}F_{2}\Delta^{3}\otimes\Delta^{3}-F_{2}\Delta^{3}\otimes F_{0}\Delta^{3}\nonumber \\
 & +\Delta^{3}\otimes F_{0}F_{1}\Delta^{3}-\Delta^{3}\otimes F_{0}F_{3}\Delta^{3}\nonumber \\
 & -F_{1}\Delta^{3}\otimes F_{3}\Delta^{3}-\Delta^{3}\otimes F_{2}F_{3}\Delta^{3}\label{eq:e1timesdelta3-1}
\end{align}
The two terms with two-dimensional factors are $-F_{2}\Delta^{3}\otimes F_{0}\Delta^{3}$
and $-F_{1}\Delta^{3}\otimes F_{3}\Delta^{3}$ and these would be
altered by the permutation mentioned above. It follows that the only
automorphism of ${\mathcal{N}({\Delta^{3}})}$ is the identity map. The lower-dimensional
cases imply that the 1-simplices and vertices cannot be permuted either.}
\end{proof}
A similar line of reasoning implies that:
\begin{cor}
\label{cor:cf-gives-simplices}Let $X$ be a simplicial complex and
let 
\[
f:{\mathcal{N}({\Delta^{n}})}\to{\mathcal{N}(X)}
\]
map $\Delta^{n}$ to a simplex $\sigma\in{\mathcal{C}(X )}$ defined by the inclusion
$\iota:\Delta^{n}\to X$. Then
\[
f({\mathcal{N}({\Delta^{n}})})\subset{\mathcal{N}({\iota})}({\mathcal{N}({\Delta^{n}})})
\]
so that $f=\alpha\circ{\mathcal{N}({\iota})}$, where $\alpha:{\mathcal{N}({\Delta^{n}})}\to{\mathcal{N}({\Delta^{n}})}$
is an automorphism. If $n\le3$, then $f={\mathcal{N}({\iota})}$.\end{cor}
\begin{proof}
Since $X$ is a simplicial complex, the map $\iota$ is an inclusion.

Suppose $\Delta^{k}\subset\Delta^{n}$ and $f({\mathcal{N}({\Delta^{k}})})_{k}\subset{\mathcal{N}({\Delta^{k}})}_{k}$.
Since the boundary of $\Delta^{k}$ is an alternating sum of $k+1$
faces, and since they must map to $k-1$-dimensional simplices of
${\mathcal{N}({f(\Delta^{k})})}$ with the same signs (so no cancellations can
take place) we must have $f(F_{i}\Delta^{k})\subset{\mathcal{N}({f(\Delta^{k})})}$
and the conclusion follows by downward induction on dimension. The
final statements follow immediately from corollary~\ref{cor:automorphisms-trivial}.
\end{proof}

\section{The functor ${\operatorname{hom}_{\mathbf{n}}(\bigstar,*)}$}

We define a complement to the ${\mathcal{N}(*)}$-functor: 
\begin{defn}
\label{def:fc}Define a functor
\[
{\operatorname{hom}_{\mathbf{n}}(\bigstar,*)}:{\mathrf I_{0}}\to{\mathbf{SS}}
\]
to the category of semi-simplicial sets, as follows:

If $C\in{\mathrf I_{0}}$, define the $n$-simplices of ${\operatorname{hom}_{\mathbf{n}}(\bigstar,C)}$ to be
the $\mathfrak{S}$-coalgebra morphisms
\[
{\mathcal{N}^{n}}\to C
\]
where ${\mathcal{N}^{n}}={\mathcal{N}({\Delta^{n}})}$ is the normalized chain-complex of
the standard $n$-simplex, equipped with the ${\mathfrak{S}}$-coalgebra structure
defined in theorem~\ref{thm:ns-construct}.

Face-operations are duals of coface-operations
\[
d_{i}:[0,\dots,i-1,i+1,\dots n]\to[0,\dots,n]
\]
with $i=0,\dots,n$ and vertex $i$ in the target is \emph{not} in
the image of $d_{i}$.\end{defn}
\begin{rem*}
Compare this to the functor ${\mathrm{hom}(\bigstar,*)}$ defined in \cite{smith-cellular}.
The subscript $\mathbf{n}$ emphasizes that we do not take \emph{degeneracies}
into account.\end{rem*}
\begin{prop}
\label{prop:ux-map}If $X$ is a simplicial complex (i.e., its simplices
are uniquely determined by their vertices) there exists a natural
map
\[
u_{X}:X\to{\operatorname{hom}_{\mathbf{n}}(\bigstar,{{\mathcal{N}(X)}})}
\]
\end{prop}
\begin{proof}
To prove the first statement, note that any simplex $\Delta^{k}$
in $X$ comes equipped with a canonical inclusion
\[
\iota:\Delta^{k}\to X
\]
The corresponding order-preserving map of vertices induces an ${\mathfrak{S}}$-coalgebra
morphism 
\[
{\mathcal{N}({\iota})}:{\mathcal{N}({\Delta^{k}})}={\mathcal{N}^{k}}\to{\mathcal{N}(X)}
\]
so $u_{X}$ is defined by
\[
\Delta^{k}\mapsto{\mathcal{N}({\iota})}
\]
It is not hard to see that this operation respects face-operations.\end{proof}
\begin{thm}
\label{thm:simplicial-complexes-determined}If $X\in{\mathbf{SC}}$ is a
simplicial complex then the canonical map
\[
u_{X}:X\to{\operatorname{hom}_{\mathbf{n}}(\bigstar,{{\mathcal{N}(X)}})}
\]
defined in proposition~\ref{prop:ux-map} is an isomorphism of the
3-skeleton.\end{thm}
\begin{proof}
This follows immediately from corollary~\ref{cor:n-simplices-map-to-simplices},
which implies that simplices map to simplices and corollary~\ref{cor:cf-gives-simplices},
which implies that these maps are \emph{unique. }\end{proof}
\begin{cor}
\label{cor:cellular-determines-pi1}If $X$ and $Y$ are simplicial
complexes with the property that there exists an isomorphism 
\[
{\mathcal{N}(X)}\to{\mathcal{N}(Y)}
\]
then their 3-skeleta are weakly equivalent and 
\[
\pi_{1}(X)\cong\pi_{1}(Y)
\]
\end{cor}
\begin{proof}
Any morphism $g:{\mathcal{N}(X)}\to{\mathcal{N}(Y)}$ induces a morphism of simplicial
sets
\[
{\mathrm{hom}(\bigstar,g)}:{\operatorname{hom}_{\mathbf{n}}(\bigstar,{{\mathcal{N}(X)}})}\to{\operatorname{hom}_{\mathbf{n}}(\bigstar,{{\mathcal{N}(Y)}})}
\]
and this is an isomorphism (and homeomorphism) of simplicial complexes
if $g$ is an isomorphism. The conclusion follows from theorem~\ref{thm:simplicial-complexes-determined}
which implies that the canonical maps
\begin{align*}
u_{X}:X\to & {\operatorname{hom}_{\mathbf{n}}(\bigstar,{{\mathcal{N}(X)}})}\\
u_{Y}:Y\to & {\operatorname{hom}_{\mathbf{n}}(\bigstar,{{\mathcal{N}(Y)}})}
\end{align*}
are isomorphisms of the 3-skeleta, and the fact that fundamental groups
are determined by the 2-skeleta.
\end{proof}
\appendix

\section{The functor ${\mathcal{N}(*)}$\label{sec:The-functor-cfn}}

We begin with the elementary but powerful Cartan Theory of Constructions,
originally described in \cite{Cartan3,Cartan4,Cartan5,Cartan6}:
\begin{lem}
\label{lem:cartanconstruction}Let $M_{i}$, $i=1,2$ be DGA-modules,
where:
\begin{enumerate}
\item $M_{1}=A_{1}\otimes N_{1}$, where $N_{1}$ is ${\mathbb{Z}}$-free and
$A_{1}$ is a DGA-algebra (so $M_{1}$, merely regarded as a DGA-algebra,
is free on a basis equal to a ${\mathbb{Z}}$-basis of $N_{1}$)
\item $M_{2}$ is a left DGA-module over a DGA-algebra $A_{2}$, possessing 

\begin{enumerate}
\item a sub-DG-module, $N_{2}\subset M_{2}$, such that $\partial_{M_{2}}|N_{2}$
is injective, 
\item a contracting chain-homotopy $\varphi:M_{2}\to M_{2}$ whose image
lies in $N_{2}\subset M_{2}$.
\end{enumerate}
\end{enumerate}

Suppose we are given a chain-map $f_{0}:M_{1}\to M_{2}$ in dimension
$0$ with $f_{0}(N_{1})\subseteq N_{2}$ and want to extend it to
a chain-map from $M_{1}$ to $M_{2}$, subject to the conditions:
\begin{itemize}
\item $f(N_{1})\subseteq N_{2}$
\item $f(a\otimes n)=g(a)\cdot f(n)$, where $g:A_{1}\to A_{2}$ is some
morphism of DG-modules such that $a\otimes n\mapsto g(a)\cdot f(n)$
 is a chain-map.
\end{itemize}

Then the extension $f:M_{1}\to M_{2}$ exists and is unique.

\end{lem}
\begin{rem*}
In applications of this result, the morphism $g$ will often be a
morphism of DGA-algebras, but this is not necessary.

The \emph{existence} of $f$ immediately follows from basic homological
algebra; the interesting aspect of it is its \emph{uniqueness} (not
merely uniqueness up to a chain-homotopy). We will use it repeatedly
to prove associativity conditions by showing that two apparently different
maps satisfying the hypotheses must be \emph{identical}.

The Theory of Constructions formed the cornerstone of Henri Cartan's
elegant computations of the homology and cohomology of Eilenberg-MacLane
spaces in \cite{Cartan11}.\end{rem*}
\begin{proof}
The uniqueness of $f$ follows by induction and the facts that: 
\begin{enumerate}
\item $f$ is determined by its values on $N_{1}$ 
\item the image of the contracting chain-homotopy, $\varphi$, lies in $N_{2}\subset M_{2}$.
\item the boundary map of $M_{2}$ is \emph{injective} on $N_{2}$ (which
implies that there is a \emph{unique} lift of $f$ into the next higher
dimension).
\end{enumerate}
\end{proof}
Now construct a contracting cochain on the normalized chain-complex
of a standard simplex:
\begin{defn}
\label{def:simplex-contracting-cochain}Let $\Delta^{k}$ be a standard
$k$-simplex with vertices $\{[0],\dots,[k]\}$ and $j$-faces $\{[i_{0},\dots,i_{j}]\}$
with $i_{0}<\cdots<i_{j}$ and let $s^{k}$ denote its normalized
chain-complex with boundary map $\partial$. This is equipped with
an augmentation
\[
\epsilon:s^{k}\to{\mathbb{Z}}
\]
that maps all vertices to $1\in{\mathbb{Z}}$ and all other simplices to
$0$. Let 
\[
\iota_{k}:{\mathbb{Z}}\to s^{k}
\]
 denote the map sending $1\in{\mathbb{Z}}$ to the image of the vertex $[n]$.
Then we have a contracting cochain\textit{\emph{
\begin{equation}
\varphi_{k}([i_{0},\dots,i_{t}]=\left\{ \begin{array}{cc}
(-1)^{t+1}[i_{0},\dots,i_{t},k] & \mathrm{if}\, i_{t}\ne k\\
0 & \mathrm{if}\, i_{t}=k
\end{array}\right.\label{eq:simplex-contracting-cochain}
\end{equation}
and $1-\iota_{k}\circ\epsilon=\partial\circ\varphi_{k}+\varphi_{k}\circ\partial$.}}\end{defn}
\begin{thm}
\label{thm:ns-construct}The normalized chain-complex of $[i_{0},\dots,i_{k}]=\Delta^{k}$
has a ${\mathfrak{S}}$-coalgebra structure that is natural with respect to order-preserving
mappings of vertex-sets
\[
[i_{0},\dots,i_{k}]\to[j_{0},\dots,j_{\ell}]
\]
with $j_{0}\le\cdots\le j_{\ell}$ and $\ell\ge k$. This ${\mathfrak{S}}$-coalgebra
is denoted ${\mathcal{N}^{k}}$.

If $X$ is a simplicial complex (a semi-simplicial set whose simplices
are uniquely determined by their vertices), then the normalized chain-complex
of $X$ has a natural ${\mathfrak{S}}$-coalgebra structure 
\[
{\mathcal{N}(X)}={\varinjlim\,}{\mathcal{N}^{k}}
\]
 for $\Delta^{n}\in\boldsymbol{\Delta}\downarrow X$ --- the simplex
category of $X$. \end{thm}
\begin{rem*}
The author has a Common LISP program for computing $f_{n}(x\otimes C(\Delta^{k}))$
--- the number of terms is exponential in the dimension of $x$. 

Compare this with the functor ${\mathcal{C}(* )}$ defined in \cite{Smith:1994}
and \cite{smith-cellular}. For simplicial complexes, ${\mathcal{C}(X )}={\mathcal{N}(X)}$.\end{rem*}
\begin{proof}
\textit{\emph{If $C=s^{k}=C(\Delta^{k})$ --- the (unnormalized) chain
complex --- we can define a corresponding contracting homotopy on
$C^{\otimes n}$ via 
\begin{align*}
\Phi= & \varphi_{k}\otimes1\otimes\cdots\otimes1+\iota_{k}\circ\epsilon\otimes\varphi_{k}\otimes\cdots\otimes1+\\
 & \cdots+\iota_{k}\circ\epsilon\otimes\cdots\otimes\iota_{k}\circ\epsilon\otimes\varphi_{k}
\end{align*}
where $\varphi_{k}$, $\iota_{k}$, and $\epsilon$ are as in definition~\ref{def:simplex-contracting-cochain}.
}}Above dimension $0$, $\Phi$ is effectively equal to $\varphi_{k}\otimes1\otimes\cdots\otimes1$\textit{\emph{.
Now set $M_{2}=C^{\otimes n}$ and $N_{2}={\operatorname{im}}(\Phi)$. In dimension
$0$, we define $f_{n}$ for all $n$ via:
\[
f_{n}(A\otimes[0])=\left\{ \begin{array}{ll}
[0]\otimes\cdots\otimes[0] & \mathrm{if}\, A=[\,]\\
0 & \mathrm{if}\,\dim A>0
\end{array}\right.
\]
This clearly makes $s^{0}$ a coalgebra over ${\mathfrak{S}}$.}}

\textit{\emph{Suppose that the $f_{n}$ are defined below dimension
$k$. Then ${\mathcal{C}({\partial\Delta^{k}} )}$ is well-defined and satisfies
the conclusions of this theorem. We define $f_{n}(a[a_{1}|\dots|a_{j}]\otimes[0,\dots,k])$
by induction on $j$, requiring that:}}
\begin{condition}[Invariant Condition]
\label{cond:invariant-condition}
\begin{equation}
f_{n}(A(S_{n},1)\otimes s^{k})\subseteq[i_{1},\dots,k]\otimes\mathrm{other}\,\mathrm{terms}\label{eq:invariant-condition}
\end{equation}
 

--- in other words, the \emph{leftmost factor} must be in ${\operatorname{im}}\varphi_{k}$.
This is the same as the leftmost factors being ``rearward'' faces
of $\Delta^{k}$.
\end{condition}
Now we set
\begin{eqnarray}
f_{n}(A\otimes s^{k}) & = & \Phi\circ f_{n}(\partial A\otimes s^{k})\nonumber \\
 & + & (-1)^{\dim A}\Phi\circ f_{n}(A\otimes\partial s^{k})\label{eq:high-diag-comp}
\end{eqnarray}
 where $A\in A(S_{n},1)\subset{\mathrm{R}S_{n }}$ and the term $f_{n}(A\otimes\partial s^{k})$
refers to the coalgebra structure of ${\mathcal{C}({\partial\Delta^{k}} )}$.

The term $f_{n}(A\otimes\partial s^{k})$ is defined by induction
and diagram~\ref{dia:coalgebra-over-operad} commutes for it. The
term $f_{n}(\partial A\otimes s^{k})$ is defined by induction on
the dimension of $A$ and diagram~\ref{dia:coalgebra-over-operad}
for it as well.

The composite maps in both branches of diagram~\ref{dia:coalgebra-over-operad}
satisfy condition~\ref{cond:invariant-condition} since:
\begin{enumerate}
\item any composite of $f_{n}$-maps will continue to satisfy condition~\ref{cond:invariant-condition}.
\item $\circ_{i}(1\otimes A(S_{n},1)\otimes\cdots\otimes A(S_{m},1))\subseteq1\otimes A(S_{n+m-1},1)$
so that composing an $f_{n}$-map with $\circ_{i}$ results in a map
that still satisfies condition~\ref{cond:invariant-condition}.
\item the diagram commutes in lower dimensions (by induction on $k$)
\end{enumerate}
Lemma~\ref{lem:cartanconstruction} implies that the composites one
gets by following the two branches of diagram~\ref{dia:coalgebra-over-operad}
must be \emph{equal, }so the diagram commutes.

We ultimately get an expression for $f_{n}(x\otimes[0,\dots,k])$
as a sum of tensor-products of sub-simplices of $[0,\dots,k]$ ---
given as ordered lists of vertices.

We claim that this ${\mathfrak{S}}$-coalgebra structure is natural with respect
to ordered mappings of vertices. This follows from the fact that the
only significant property that the vertex $k$ \emph{has} in equation~\ref{eq:simplex-contracting-cochain},
condition~\ref{cond:invariant-condition} and equation~\ref{eq:high-diag-comp}
is that it is the \emph{highest numbered} vertex. 
\end{proof}
We conclude this section some computations of higher coproducts:
\begin{example}
\label{example:e1timesdelta2}If $[0,1,2]=\Delta^{2}$ is a $2$-simplex,
then

\begin{equation}
f_{2}([\,]\otimes\Delta^{2})=\Delta^{2}\otimes F_{0}F_{1}\Delta^{2}+F_{2}\Delta^{2}\otimes F_{0}\Delta^{2}+F_{1}F_{2}\Delta^{2}\otimes\Delta^{2}\label{eq:delta-2-coproduct-1}
\end{equation}
--- the standard (Alexander-Whitney) coproduct --- and

\begin{align*}
f_{2}([(1,2)]\otimes\Delta^{2})= & [0,1,2]\otimes[1,2]-[0,2]\otimes[0,1,2]\\
 & -[0,1,2]\otimes[0,1]
\end{align*}
or, in face-operations

\begin{align}
f_{2}([(1,2)]\otimes\Delta^{2})= & \Delta^{2}\otimes F_{0}\Delta^{2}-F_{1}\Delta^{2}\otimes\Delta^{2}\label{eq:e1timesdelta2-1}\\
 & -\Delta^{2}\otimes F_{2}\Delta^{2}\nonumber 
\end{align}
\end{example}
\begin{proof}
If we write $\Delta^{2}=[0,1,2]$, we get
\[
f_{2}([\,]\otimes\Delta^{2})=[0,1,2]\otimes[2]+[0,1]\otimes[1,2]+[0]\otimes[0,1,2]
\]

To compute $f_{2}([(1,2)]\otimes\Delta^{2})$ we have a version of
equation~\ref{eq:high-diag-comp}:
\begin{align*}
f(e_{1}\otimes\Delta^{2}) & =\Phi_{2}(f_{2}(\partial e_{1}\otimes\Delta^{2})-\Phi_{2}f_{2}(e_{1}\otimes\partial\Delta^{2})\\
 & =-\Phi_{2}(f_{2}((1,2)\cdot[\,]\otimes\Delta^{2})+\Phi_{2}(f_{2}([\,]\otimes\Delta^{2})-\Phi_{2}f_{2}(e_{1}\otimes\partial\Delta^{2})
\end{align*}
Now 
\begin{align*}
\Phi_{2}(1,2)\cdot(f_{2}([\,]\otimes\Delta^{2})= & (\varphi_{2}\otimes1)\bigl([2]\otimes[0,1,2]-[1,2]\otimes[0,1]\\
 & +[0,1,2]\otimes[0]\bigr)\\
 & +(i\circ\epsilon\otimes\varphi_{2})\bigl([2]\otimes[0,1,2]\\
 & -[1,2]\otimes[0,1]+[0,1,2]\otimes[0]\bigr)\\
= & 0
\end{align*}
and
\begin{align*}
\Phi_{2}(f_{2}([\,]\otimes\Delta^{2})= & (\varphi_{2}\otimes1)\bigl([0,1,2]\otimes[2]+[0,1]\otimes[1,2]\\
 & +[0]\otimes[0,1,2]\bigr)\\
= & [0,1,2]\otimes[1,2]-[0,2]\otimes[0,1,2]
\end{align*}
In addition, proposition~\ref{pro:simplicespropertyS} implies that
\begin{align*}
f_{2}(e_{1}\otimes\partial\Delta^{2})= & [1,2]\otimes[1,2]-[0,2]\otimes[0,2]\\
 & +[0,1]\otimes[0,1]
\end{align*}
so that
\[
\Phi_{2}f_{2}(e_{1}\otimes\partial\Delta^{2})=[0,1,2]\otimes[0,1]
\]

We conclude that
\begin{align*}
f_{2}([(1,2)]\otimes\Delta^{2})= & [0,1,2]\otimes[1,2]-[0,2]\otimes[0,1,2]\\
 & -[0,1,2]\otimes[0,1]
\end{align*}
 which implies equation~\ref{eq:e1timesdelta2-1}.
\end{proof}
We end this section with computations of some ``higher coproducts.''
We have a ${{\mathbb{Z}} S_{2 }}$-equivariant chain-map
\[
f_{2}({\mathrm{R}S_{2 }}\otimes C)\to C\otimes C
\]

\begin{prop}
\label{prop:e1timesdelta2}If $\Delta^{2}$ is a $2$-simplex, then:

\begin{equation}
f_{2}([\,]\otimes\Delta^{2})=\Delta^{2}\otimes F_{0}F_{1}\Delta^{2}+F_{2}\Delta^{2}\otimes F_{0}\Delta^{2}+F_{1}F_{2}\Delta^{2}\otimes\Delta^{2}\label{eq:delta-2-coproduct}
\end{equation}
Here $e_{0}=[\,]$ is the 0-dimensional generator of ${\mathrm{R}S_{2 }}$ and
this is just the standard (Alexander-Whitney) coproduct. 

In addition, we have: 
\begin{align}
f_{2}([(1,2)]\otimes\Delta^{2})= & \Delta^{2}\otimes F_{0}\Delta^{2}-F_{1}\Delta^{2}\otimes\Delta^{2}\label{eq:e1timesdelta2}\\
 & -\Delta^{2}\otimes F_{2}\Delta^{2}\nonumber 
\end{align}
\end{prop}
\begin{proof}
If we write $\Delta^{2}=[0,1,2]$, we get
\[
f_{2}([\,]\otimes\Delta^{2})=[0,1,2]\otimes[2]+[0,1]\otimes[1,2]+[0]\otimes[0,1,2]
\]

To compute $f_{2}([(1,2)]\otimes\Delta^{2})$ we have a version of
equation~\ref{eq:hdiag-comp}:
\begin{align*}
f_{2}(e_{1}\otimes\Delta^{2}) & =\Phi_{2}(f_{2}(\partial e_{1}\otimes\Delta^{2})-\Phi_{2}f_{2}(e_{1}\otimes\partial\Delta^{2})\\
 & =-\Phi_{2}(f_{2}((1,2)\cdot[\,]\otimes\Delta^{2})+\Phi_{2}(f_{2}([\,]\otimes\Delta^{2})-\Phi_{2}f_{2}(e_{1}\otimes\partial\Delta^{2})
\end{align*}
Now 
\begin{align*}
\Phi_{2}(1,2)\cdot(f_{2}([\,]\otimes\Delta^{2})= & (\varphi_{2}\otimes1)([2]\otimes[0,1,2]-[1,2]\otimes[0,1]+[0,1,2]\otimes[0])\\
 & +(i\circ\epsilon\otimes\varphi_{2})([2]\otimes[0,1,2]-[1,2]\otimes[0,1]+[0,1,2]\otimes[0])\\
= & 0
\end{align*}
and
\begin{align*}
\Phi_{2}(f_{2}([\,]\otimes\Delta^{2}) & =(\varphi_{2}\otimes1)\left([0,1,2]\otimes[2]+[0,1]\otimes[1,2]+[0]\otimes[0,1,2]\right)\\
 & =[0,1,2]\otimes[1,2]-[0,2]\otimes[0,1,2]
\end{align*}
In addition, proposition~\ref{pro:simplicespropertyS} implies that
\[
f_{2}(e_{1}\otimes\partial\Delta^{2})=[1,2]\otimes[1,2]-[0,2]\otimes[0,2]+[0,1]\otimes[0,1]
\]
so that
\[
\Phi_{2}f_{2}(e_{1}\otimes\partial\Delta^{2})=[0,1,2]\otimes[0,1]
\]

We conclude that
\begin{align*}
f_{2}([(1,2)]_{2}\otimes\Delta^{2})= & [0,1,2]\otimes[1,2]-[0,2]\otimes[0,1,2]\\
 & -[0,1,2]\otimes[0,1]
\end{align*}
which implies equation~\ref{eq:e1timesdelta2}.
\end{proof}
We continue this computation one dimension higher:
\begin{prop}
\label{prop:e1timesdelta3}If $\Delta^{3}$ is a $3$-simplex, then:
\foreignlanguage{english}{
\begin{align}
f_{2}([(1,2)]\otimes\Delta^{3}) & =F_{1}F_{2}\Delta^{3}\otimes\Delta^{3}-F_{2}\Delta^{3}\otimes F_{0}\Delta^{3}\nonumber \\
 & +\Delta^{3}\otimes F_{0}F_{1}\Delta^{3}-\Delta^{3}\otimes F_{0}F_{3}\Delta^{3}\nonumber \\
 & -F_{1}\Delta^{3}\otimes F_{3}\Delta^{3}-\Delta^{3}\otimes F_{2}F_{3}\Delta^{3}\label{eq:e1timesdelta3}
\end{align}
}\end{prop}
\begin{proof}
As before, $\Delta^{3}=[0,1,2,3]$, and we have 
\begin{align*}
f_{2}(e_{1}\otimes\Delta^{3}) & =\Phi_{3}(f_{2}(\partial e_{1}\otimes\Delta^{3})-\Phi_{3}f_{2}(e_{1}\otimes\partial\Delta^{3})\\
 & =-\Phi_{3}(f_{2}((1,2)\cdot[\,]\otimes\Delta^{3})+\Phi_{3}(f_{2}([\,]\otimes\Delta^{3})-\Phi_{3}f_{2}(e_{1}\otimes\partial\Delta^{3})
\end{align*}
and 
\[
\Phi_{3}(f_{2}((1,2)\cdot[\,]\otimes\Delta^{3})=0
\]
 We also conclude \foreignlanguage{english}{
\begin{align*}
\Phi_{3}(f_{2}([\,]\otimes\Delta^{3})= & [0,3]\otimes\Delta^{3}-[0,1,3]\otimes[1,2,3]\\
 & +\Delta^{3}\otimes[2,3]
\end{align*}
Now
\[
\partial\Delta^{3}=[1,2,3]-[0,2,3]+[0,1,3]-[0,1,2]
\]
and equation~\ref{eq:e1timesdelta2} implies that
\begin{align*}
f_{2}(e_{1}\otimes\partial\Delta^{3}) & =[1,2,3]\otimes[2,3]-[1,3]\otimes[1,2,3]\\
 & -[1,2,3]\otimes[1,2]\\
 & -[0,2,3]\otimes[2,3]+[0,3]\otimes[0,2,3]\\
 & +[0,2,3]\otimes[0,2]\\
 & +[0,1,3]\otimes[1,3]-[0,3]\otimes[0,1,3]\\
 & -[0,1,3]\otimes[0,1]\\
 & -[0,1,2]\otimes[1,2]+[0,2]\otimes[0,1,2]\\
 & +[0,1,2]\otimes[0,1]
\end{align*}
\emph{Most} of these terms die when one applies $\Phi_{3}$:
\begin{align*}
\Phi_{3}f_{2}(e_{1}\otimes\partial\Delta^{3}) & =\Delta^{3}\otimes[1,2]+[0,2,3]\otimes[0,1,2]\\
 & -\Delta^{3}\otimes[0,1]
\end{align*}
We conclude that
\begin{align*}
f_{2}(e_{1}\otimes\Delta^{3}) & =[0,3]\otimes\Delta^{3}-[0,1,3]\otimes[1,2,3]\\
 & +\Delta^{3}\otimes[2,3]-\Delta^{3}\otimes[1,2]\\
 & -[0,2,3]\otimes[0,1,2]-\Delta^{3}\otimes[0,1]
\end{align*}
which implies equation~\ref{eq:e1timesdelta3}. }
\end{proof}
With this in mind, note that images of simplices in ${\mathcal{N}(*)}$ have
an interesting property:
\begin{prop}
\label{pro:simplicespropertyS}Let $X$ be a simplicial set with $C={\mathcal{N}(X)}$
and with coalgebra structure 
\[
f_{n}:RS_{n}\otimes{\mathcal{N}(X)}\to{\mathcal{N}(X)}^{\otimes n}
\]
and suppose $RS_{2}$ is generated in dimension $n$ by $e_{n}=\underbrace{[(1,2)|\cdots|(1,2)]}_{n\text{ terms}}$.
If $x\in C$ is the image of a $k$-simplex, then
\[
f_{2}(e_{k}\otimes x)=\xi_{k}\cdot x\otimes x
\]
where $\xi_{k}=(-1)^{k(k-1)/2}$.\end{prop}
\begin{rem*}
This is just a chain-level statement that the Steenrod operation $\operatorname{Sq}^{0}$
acts trivially on mod-$2$ cohomology. A weaker form of this result
appeared in \cite{Davis:mco}.\end{rem*}
\begin{proof}
Recall that $({\mathrm{R}S_{2 }})_{n}={\mathbb{Z}}[{\mathbb{Z}}_{2}]$ generated by $ $$e_{n}=[\underbrace{(1,2)|\cdots|(1,2)}_{n\text{ factors}}]$.
Let $T$ be the generator of ${\mathbb{Z}}_{2}$ --- acting on $C\otimes C$
by swapping the copies of $C$.

We assume that $f_{2}(e_{i}\otimes C(\Delta^{j}))\subset C(\Delta^{j})\otimes C(\Delta^{j})$
so that 
\begin{equation}
i>j\implies f_{2}(e_{i}\otimes C(\Delta^{j}))=0\label{eq:big-diag-condition}
\end{equation}

\end{proof}
As in section~4 of \cite{Smith:1994}, if $e_{0}=[\,]\in{\mathrm{R}S_{2 }}$ is
the $0$-dimensional generator, we define
\[
f_{2}:{\mathrm{R}S_{2 }}\otimes C\to C\otimes C
\]
 inductively by
\begin{eqnarray}
f_{2}(e_{0}\otimes[i]) & = & [i]\otimes[i]\nonumber \\
f_{2}(e_{0}\otimes[0,\dots,k]) & = & \sum_{i=0}^{k}[0,\dots,i]\otimes[i,\dots,k]\label{eq:big-diag1}
\end{eqnarray}
Let $\sigma=\Delta^{k}$ and inductively define
\begin{align}
f_{2}(e_{k}\otimes\sigma) & =\Phi_{k}(f_{2}(\partial e_{k}\otimes\sigma)+(-1)^{k}\Phi_{k}f_{2}(e_{k}\otimes\partial\sigma)\nonumber \\
 & =\Phi_{k}(f_{2}(\partial e_{k}\otimes\sigma)\label{eq:hdiag-comp}
\end{align}
because of equation~\ref{eq:big-diag-condition}. 
\begin{proof}
Expanding $\Phi_{k}$, we get
\begin{align}
f_{2}(e_{k}\otimes\sigma) & =(\varphi_{k}\otimes1)(f_{2}(\partial e_{k}\otimes\sigma))+(i\circ\epsilon\otimes\varphi_{k})f_{2}(\partial e_{k}\otimes\sigma)\nonumber \\
 & =(\varphi_{k}\otimes1)(f_{2}(\partial e_{k}\otimes\sigma))\label{eq:big-diag2}
\end{align}
 because $\varphi_{k}^{2}=0$ and $\varphi_{k}\circ i\circ\epsilon=0$. 

Noting that $\partial e_{k}=(1+(-1)^{k}T)e_{k-1}\in{\mathrm{R}S_{2 }}$, we get
\begin{align*}
f_{2}(e_{k}\otimes\sigma) & =(\varphi_{k}\otimes1)(f_{2}(e_{k-1}\otimes\sigma)+(-1)^{k}(\varphi_{k}\otimes1)\cdot T\cdot f_{2}(e_{k-1}\otimes\sigma)\\
 & =(-1)^{k}(\varphi_{k}\otimes1)\cdot T\cdot f_{2}(e_{k-1}\otimes\sigma)
\end{align*}
again, because $\varphi_{k}^{2}=0$ and $\varphi_{k}\circ\iota_{k}\circ\epsilon=0$.
We continue, using equation~\ref{eq:big-diag2} to compute $f(e_{k-1}\otimes\sigma)$:
\begin{align*}
f_{2}(e_{k}\otimes\sigma)= & (-1)^{k}(\varphi_{k}\otimes1)\cdot T\cdot f_{2}(e_{k-1}\otimes\sigma)\\
= & (-1)^{k}(\varphi_{k}\otimes1)\cdot T\cdot(\varphi_{k}\otimes1)\biggl(f_{2}(\partial e_{k-1}\otimes\sigma)\\
 & +(-1)^{k-1}f_{2}(e_{k-1}\otimes\partial\sigma)\biggr)\\
= & (-1)^{k}\varphi_{k}\otimes\varphi_{k}\cdot T\cdot\biggl(f_{2}(\partial e_{k-1}\otimes\sigma)\\
 & +(-1)^{k-1}f_{2}(e_{k-1}\otimes\partial\sigma)\biggr)
\end{align*}
If $k-1=0$, then the left term vanishes. If $k-1=1$ so $\partial e_{k-1}$
is $0$-dimensional then equation~\ref{eq:big-diag1} gives $f(\partial e_{1}\otimes\sigma)$
and this vanishes when plugged into $\varphi_{k}\otimes\varphi_{k}$.
If $k-1>1$, then $f_{2}(\partial e_{k-1}\otimes\sigma)$ is in the
image of $\varphi_{k}$, so it vanishes when plugged into $\varphi_{k}\otimes\varphi_{k}$.

In \emph{all} cases, we can write
\begin{align*}
f_{2}(e_{k}\otimes\sigma) & =(-1)^{k}\varphi_{k}\otimes\varphi_{k}\cdot T\cdot(-1)^{k-1}f_{2}(e_{k-1}\otimes\partial\sigma)\\
 & =-\varphi_{k}\otimes\varphi_{k}\cdot T\cdot f_{2}(e_{k-1}\otimes\partial\sigma)
\end{align*}
If $f_{2}(e_{k-1}\otimes\Delta^{k-1})=\xi_{k-1}\Delta^{k-1}\otimes\Delta^{k-1}$
(the inductive hypothesis), then 
\begin{multline*}
f_{2}(e_{k-1}\otimes\partial\sigma)=\\
\sum_{i=0}^{k}\xi_{k-1}\cdot(-1)^{i}[0,\dots,i-1,i+1,\dots k]\otimes[0,\dots,i-1,i+1,\dots k]
\end{multline*}
and the only term that does not get annihilated by $\varphi_{k}\otimes\varphi_{k}$
is 
\[
(-1)^{k}[0,\dots,k-1]\otimes[0,\dots,k-1]
\]
 (see equation~\foreignlanguage{english}{\ref{eq:simplex-contracting-cochain}}).
We get
\begin{align*}
f_{2}(e_{k}\otimes\sigma) & =\xi_{k-1}\cdot\varphi_{k}\otimes\varphi_{k}\cdot T\cdot(-1)^{k-1}[0,\dots,k-1]\otimes[0,\dots,k-1]\\
 & =\xi_{k-1}\cdot\varphi_{k}\otimes\varphi_{k}(-1)^{(k-1)^{2}+k-1}[0,\dots,k-1]\otimes[0,\dots,k-1]\\
 & =\xi_{k-1}\cdot(-1)^{(k-1)^{2}+2(k-1)}\varphi[0,\dots,k-1]\otimes\varphi[0,\dots,k-1]\\
 & =\xi_{k-1}\cdot(-1)^{k-1}[0,\dots,k]\otimes[0,\dots,k]\\
 & =\xi_{k}\cdot[0,\dots,k]\otimes[0,\dots,k]
\end{align*}
where the sign-changes are due to the Koszul Convention. We conclude
that $\xi_{k}=(-1)^{k-1}\xi_{k-1}$.
\end{proof}

\section{Proof of lemma~\ref{lem:diagonals-linearly-independent}}
\begin{lem}
\label{lem:diagonals-linearly-independent}Let $C$ be a free abelian
group, let 
\[
\hat{C}={\mathbb{Z}}\oplus\prod_{i=1}^{\infty}C^{\otimes i}
\]

Let $e:C\to\hat{C}$ be the function that sends $c\in C$ to
\[
(1,c,c\otimes c,c\otimes c\otimes c,\dots)\in\hat{C}
\]
For any integer $t>1$ and any set $\{c_{1},\dots,c_{t}\}\in C$ of
distinct, nonzero elements, the elements 
\[
\{e(c_{1}),\dots,e(c_{t})\}\in{\mathbb{Q}}\otimes_{\mathbb{Z}}\hat{C}
\]
are linearly independent over ${\mathbb{Q}}$. It follows that $e$ defines
an injective function
\[
\bar{e}:{\mathbb{Z}}[C]\to\hat{C}
\]
\end{lem}
\begin{proof}
We will construct a vector-space morphism
\begin{equation}
f:{\mathbb{Q}}\otimes_{\mathbb{Z}}\hat{C}\to V\label{eq:diagonals-linearly-independent1}
\end{equation}
such that the images, $\{f(e(c_{i}))\}$, are linearly independent.
We begin with the ``truncation morphism''
\[
r_{t}:\hat{C}\to{\mathbb{Z}}\oplus\bigoplus_{i=1}^{t-1}C^{\otimes i}=\hat{C}_{t-1}
\]
which maps $C^{\otimes1}$ isomorphically. If $\{b_{i}\}$ is a ${\mathbb{Z}}$-basis
for $C$, we define a vector-space morphism 
\[
g:\hat{C}_{t-1}\otimes_{\mathbb{Z}}{\mathbb{Q}}\to{\mathbb{Q}}[X_{1},X_{2},\dots]
\]
by setting
\[
g(c)=\sum_{\alpha}z_{\alpha}X_{\alpha}
\]
where $c=\sum_{\alpha}z_{\alpha}b_{\alpha}\in C\otimes_{\mathbb{Z}}{\mathbb{Q}}$,
and extend this to $\hat{C}_{t-1}\otimes_{\mathbb{Z}}{\mathbb{Q}}$ via 
\[
g(c_{1}\otimes\cdots\otimes c_{j})=g(c_{1})\cdots g(c_{j})\in{\mathbb{Q}}[X_{1},X_{2},\dots]
\]
The map in equation~\ref{eq:diagonals-linearly-independent1} is
just the composite
\[
\hat{C}\otimes_{\mathbb{Z}}{\mathbb{Q}}\xrightarrow{r_{t-1}\otimes1}\hat{C}_{t-1}\otimes_{\mathbb{Z}}{\mathbb{Q}}\xrightarrow{g}{\mathbb{Q}}[X_{1},X_{2},\dots]
\]
It is not hard to see that 
\[
p_{i}=f(e(c_{i}))=1+f(c_{i})+\cdots+f(c_{i})^{t-1}\in{\mathbb{Q}}[X_{1},X_{2},\dots]
\]
 for $i=1,\dots,t$. Since the $f(c_{i})$ are \emph{linear} in the
indeterminates $X_{i}$, the degree-$j$ component (in the indeterminates)
of $f(e(c_{i}))$ is precisely $f(c_{i})^{j}$. It follows that a
linear dependence-relation
\[
\sum_{i=1}^{t}\alpha_{i}\cdot p_{i}=0
\]
with $\alpha_{i}\in{\mathbb{Q}}$, holds if and only if
\[
\sum_{i=1}^{t}\alpha_{i}\cdot f(c_{i})^{j}=0
\]
 for all $j=0,\dots,t-1$. This is equivalent to $\det M=0$, where
\[
M=\left[\begin{array}{cccc}
1 & 1 & \cdots & 1\\
f(c_{1}) & f(c_{2}) & \cdots & f(c_{t})\\
\vdots & \vdots & \ddots & \vdots\\
f(c_{1})^{t-1} & f(c_{2})^{t-1} & \cdots & f(c_{t})^{t-1}
\end{array}\right]
\]
 Since $M$ is the transpose of the Vandermonde matrix, we get
\[
\det M=\prod_{1\le i<j\le t}(f(c_{i})-f(c_{j}))
\]
Since $f|C\otimes_{\mathbb{Z}}{\mathbb{Q}}\subset\hat{C}\otimes_{\mathbb{Z}}{\mathbb{Q}}$
is \emph{injective,} it follows that this \emph{only} vanishes if
there exist $i$ and $j$ with $i\ne j$ and $c_{i}=c_{j}$. The second
conclusion follows.
\end{proof}
\bibliographystyle{amsplain}

\providecommand{\bysame}{\leavevmode\hbox to3em{\hrulefill}\thinspace}
\providecommand{\MR}{\relax\ifhmode\unskip\space\fi MR }
\providecommand{\MRhref}[2]{  \href{http://www.ams.org/mathscinet-getitem?mr=#1}{#2}
}
\providecommand{\href}[2]{#2}
\begin{thebibliography}{10}

\bibitem{Barratt-Eccles-operad}
M.~Barratt and P.~Eccles, \emph{On {$\Gamma_{+}$}-structures. {I}. {A} free
  group functor for stable homotopy theory}, Topology (1974), 25--45.

\bibitem{Cartan4}
H.~Cartan, \emph{{C}onstructions multiplicatives}, S{\'e}minaire Henri Cartan,
  vol.~7, Secr{\'e}tariat math{\'e}matique, Paris, 1954-1955, exp. 4,
  \verb,<http://www.numdam.org/item?id=SHC_1954-1955__7_1_A4_0>,, pp.~1--6.

\bibitem{Cartan5}
\bysame, \emph{Constructions multiplicatives it{\'e}r{\'e}es cohomologie},
  S{\'e}minaire Henri Cartan, vol.~7, Secr{\'e}tariat math{\'e}matique, Paris,
  1954-1955, pp.~1--6.

\bibitem{Cartan11}
\bysame, \emph{D{\'e}termination des alg{\`e}bres {$H_*(\pi,n;\mathbb{Z})$}},
  S{\'e}minaire Henri Cartan, vol.~7, Secr{\'e}tariat math{\'e}matique, Paris,
  1954-1955, exp. 11,
  \verb,<http://www.numdam.org/item?id=SHC_1954-1955__7_1_A11_0>,, pp.~1--24.

\bibitem{Cartan3}
\bysame, \emph{{DGA}-modules (suite). {N}otion de {C}onstruction},
  S{\'e}minaire Henri Cartan, vol.~7, Secr{\'e}tariat math{\'e}matique, Paris,
  1954-1955, exp. 3,
  \verb,<http://www.numdam.org/item?id=SHC_1954-1955__7_1_A2_0>,, pp.~1--11.

\bibitem{Cartan6}
\bysame, \emph{Op{\'e}rations dans les constructions acycliques}, S{\'e}minaire
  Henri Cartan, vol.~7, Secr{\'e}tariat math{\'e}matique, Paris, 1954-1955,
  pp.~1--11.

\bibitem{Davis:mco}
James~F. Davis, \emph{Higher diagonal approximations and skeletons of {$K(\pi,
  1)$}'s}, Lecture Notes in Mathematics, vol. 1126, Springer-Verlag, 1983,
  pp.~51--61.

\bibitem{Gugenheim:1960}
V.~K. A.~M. Gugenheim, \emph{On a theorem of {E.} {H.} {B}rown}, Illinois J. of
  Math. \textbf{4} (1960), 292--311.

\bibitem{Kriz-May}
I.~Kriz and J.~P. May, \emph{Operads, algebras, modules and motives},
  Ast{\'e}risque, vol. 233, Soci{\'e}t{\'e} {M}ath{\'e}matique de {France},
  1995.

\bibitem{operad-book}
Martin Markl, Steve Shnider, and Jim Stasheff, \emph{Operads in {A}lgebra,
  {T}opology and {P}hysics}, Mathematical Surveys and Monographs, vol.~96,
  American Mathematical Society, May 2002.

\bibitem{may-finite}
J.~Peter May, \emph{Finite spaces and larger contexts}, Available from
  http;//math.uchicago.edu/~may/FINITE/FINITEBOOK/FiniteAugBOOK.pdf‎.

\bibitem{smith-cellular}
Justin~R. Smith, \emph{Cellular coalgebras over the operad $\mathfrak{S}$
  {I.}}, arXiv:1304.6328 [math.AT].

\bibitem{Smith:1994}
\bysame, \emph{Iterating the cobar construction}, vol. 109, Memoirs of the A.
  M. S., no. 524, American Mathematical Society, Providence, Rhode Island, May
  1994.

\end{thebibliography}

    \end{document}


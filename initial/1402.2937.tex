

\usepackage[english]{babel}
\usepackage{amssymb,amsfonts,amsxtra,amsmath}
\usepackage{dsfont,mathrsfs}
\usepackage{tikz}

\usepackage[all]{xypic}
\xyoption{dvips}
\usepackage{url}
\usepackage{stmaryrd}
\usepackage[colorlinks,plainpages,backref]{hyperref}
\usepackage[neveradjust]{paralist}

\theoremstyle{definition}
\newtheorem{ntn}{Notation}
\newtheorem{dfn}[ntn]{Definition}
\theoremstyle{plain}
\newtheorem{lem}[ntn]{Lemma}
\newtheorem{prp}[ntn]{Proposition}
\newtheorem{thm}[ntn]{Theorem}
\newtheorem{cor}[ntn]{Corollary}
\newtheorem{cnj}[ntn]{Conjecture}
\newtheorem{alg}[ntn]{Algorithm}
\theoremstyle{remark}
\newtheorem{cst}[ntn]{Construction}
\newtheorem{exa}[ntn]{Example}
\newtheorem{que}[ntn]{Question}
\newtheorem{rmk}[ntn]{Remark}
\numberwithin{equation}{section}

\DeclareFontFamily{OT1}{rsfs}{}
\DeclareFontShape{OT1}{rsfs}{n}{it}{<-> rsfs10}{}
\DeclareMathAlphabet{\mathscr}{OT1}{rsfs}{n}{it}

\begin{document}

\title[Free divisors with only normal crossings]{Quasihomogeneous free divisors with only normal crossings in codimension one}

\author[X.~Liao]{Xia Liao}
\address{X.~Liao\\
Department of Mathematics\\
University of Kaiserslautern\\
67663 Kaiserslautern\\
Germany}
\email{\href{mailto:liao@mathematik.uni-kl.de}{liao@mathematik.uni-kl.de}}

\author[M.~Schulze]{Mathias Schulze}
\address{M.~Schulze\\
Department of Mathematics\\
University of Kaiserslautern\\
67663 Kaiserslautern\\
Germany}
\email{\href{mailto:mschulze@mathematik.uni-kl.de}{mschulze@mathematik.uni-kl.de}}

\thanks{The research leading to these results has received funding from the People Programme (Marie Curie Actions) of the European Union's Seventh Framework Programme (FP7/2007-2013) under REA grant agreement n\textsuperscript{o} PCIG12-GA-2012-334355.}

\date{\today}

\subjclass{32S25 (Primary) 16W25 (Secondary)}

\keywords{free divisor, normal crossing, quasihomogeneity, logarithmic, derivation, Lie algebra}

\begin{abstract}
We prove that any divisor as in the title must be normal crossing.
\end{abstract}

\maketitle
\tableofcontents

\section*{Introduction}

Free divisors occur classically as discriminants in singularity theory, for instance in the base space of versal deformations of isolated hypersurface (or complete intersection) singularities.
While there is an abundance of specific situations in which certain hypersurfaces are free divisors, it is rather unclear under what geometric conditions a general hypersurface can be a free divisor.
Since free divisors have purely one-codimensional singular locus, it is for instance natural to ask what singularities can occur in a free divisor in codimension one.
The simplest case to look at is that of free divisors with only normal crossings in codimension one.
However, besides the normal crossing divisor itself, no such object is known.
This raises the following

\begin{que}[{Faber~\cite[Que.~60]{Fab12}}]\label{41}
Is a free divisor normal crossing if it is normal crossing in codimension one?
\end{que}

In a particular case this question is answered.

\begin{thm}[{Granger--Schulze~\cite[Thm.~1.8]{GS11}}]\label{39}
A free divisor with smooth normalization is normal crossing if and only if it is normal crossing in codimension one.
\end{thm}

For free divisors which are normal crossing in codimension one, we shall prove that quasihomogeneity implies smoothness of the normalization.

\begin{thm}\label{0}
A quasihomogeneous free divisor $D$ is normal crossing if and only if it is normal crossing in codimension one.
\end{thm}

Although we do not know how to remove it, the additional homogeneity hypothesis can be justified as follows:
By Granger--Schulze~\cite[Thm.~1.1]{GS11} and Faber~\cite[Rmk.~54]{Fab12}, free divisors which are normal crossing in codimension one are Euler homogeneous.
This type of homogeneity is weaker then quasihomogeneity.
However the difference between the two is rather subtle and relates to approximation theorems (see \cite{Sai71} or \cite[\S3, Thm.~4]{HM89}).

There are many examples of reducible free divisors such as free hyperplane arrangements (see for example \cite[\S4]{OT92}) or free discriminants in prehomogeneous vector spaces (see \cite{GMS11}).
On the other hand, it is empirically known that irreducible free divisors are rare objects (see \cite{ST12}).
Our result supports this observation by deducing reducibility from local reducibility in codimension one (under the stated additional hypotheses).

The question of existence of negative degree derivations is subject to several open problems such as the Halperin conjecture (see \cite{Hau02}) or, for instance, a conjecture of Wahl (see \cite[Conj.~1.4]{Wah83}).
In the course of the proof of Theorem~\ref{0}, we show that the hypotheses imply that all irreducible components of $D$ have a basis of homogeneous logarithmic derivations of non-positive weighted degree (see Proposition~\ref{20}).
This basis generates a finite-dimensional complex Lie algebra ${\mathfrak{d}}$ (see \eqref{46}).
Substituting this property for the normal crossing hypothesis, we prove the following intermediary result by studying the Lie algebra representation of ${\mathfrak{d}}$ on the space of lowest weight variables (see \S\ref{17}).

\begin{thm}\label{50}
Any quasihomogeneous free divisor $D$ admitting a basis of homogeneous logarithmic derivations of non-positive weighted degree has a smooth irreducible component.
\end{thm}

We shall now give a summary of our approach:
We first pass to the irreducible and non suspended case (see Proposition~\ref{42} and Remark~\ref{47}).

In \S\ref{22}, we consider a free divisor germ $D$ which is normal crossing in codimension one.
Combining results from Granger--Schulze~\cite{GS11} and Mond--Pellikaan~\cite{MP89}, we prove that the ideal of submaximal minors of a Saito matrix coincides with the Jacobian ideal (see Proposition~\ref{34}).

In \S\ref{27}, we assume moreover that $D$ is quasihomogeneous with respect to strictly positive weights.
Then a degree argument using Saito's criterion shows that there is a basis of logarithmic derivations of non-positive weighted degree (see Proposition~\ref{20}).
Here the irreducibility of $D$ is needed (see Lemma~\ref{43}).
For all following arguments the normal crossing hypothesis in codimension one can be replaced by the existence of such a basis.

In \S\ref{51} we consider the (finite-dimensional) complex Lie algebra ${\mathfrak{d}}$ generated by the above basis.
Dropping the Euler derivation defining the quasihomogeneity, we pass to the Lie subalgebra ${\mathfrak{a}}\subset{\mathfrak{d}}$ annihilating a weighted homogeneous defining equation of $D$.

In \S\ref{17}, we study the representation ${\mathfrak{a}}'$ of ${\mathfrak{a}}$ on the vector space ${\mathfrak{m}}'$ of minimal weight variables which corresponds to a block of a Saito matrix (see \eqref{30}).
In case ${\mathfrak{a}}'$ is solvable, Lie's theorem and Saito's criterion enforce smoothness of $D$ (see Proposition~\ref{31}).
In case ${\mathfrak{a}}'$ is not solvable, we describe a change of coordinates and weights that makes the dimension of ${\mathfrak{m}}'$ drop.
Iterating, we arrive at a solvable ${\mathfrak{a}}'$ returning to the previous case (see Proposition~\ref{33}).

\subsection*{Acknowledgments}

The first author would like to thank the Department of Mathematics at
the University of Kaiserslautern for providing a pleasant working
environment during his postdoctoral stay.

\section{Normal crossings in codimension one}\label{22}

Let $X:=({\mathds{C}}^{n+1},0)$ be the germ of complex $(n+1)$-space and let $D$ be the germ of a reduced divisor in $X$. 
Recall that there are two exact sequences of $\mathscr{O}_X$-modules:
\begin{gather}
{\SelectTips{cm}{}\xymatrix}{
0 \ar[r] &  \operatorname{Der}_X(-\log D) \ar[r] & \operatorname{Der}_{\mathds{C}}({{\mathscr}{O}}_X) \ar[r] & {{\mathscr}{J}}_D \ar[r] &  0,
}\label{13}\\
{\SelectTips{cm}{}\xymatrix}{
0 \ar[r] &  \Omega^p_X \ar[r] &  \Omega^p_X(\log D) \ar[r]^-{\rho_D^p} & \omega^{p-1}_D \ar[r] &  0.
}\label{12}
\end{gather}
We point out that ${{\mathscr}{J}}_{D}$ in \eqref{13} is the Jacobian ideal of $D$. 
If $D$ is defined by an equation $f\in{{\mathscr}{O}}_X$, then ${{\mathscr}{J}}_{D}$ is the ideal in ${{\mathscr}{O}}_D$ generated by the partial derivatives $\frac{{\partial} f}{{\partial} x_1},\dots,\frac{{\partial} f}{{\partial} x_{n+1}}$ with respect to a coordinate system $x_1,\dots,x_{n+1}$. 
In \eqref{12}, $\omega^{p-1}_D$ is the image of $\Omega_X^p(\log D)$ in ${{\mathscr}{M}}_D\otimes_{{{\mathscr}{O}}_D}\Omega^{p-1}_D$ under the Saito's logarithmic residue map $\rho_D^p$ (see \cite[\S2]{Sai80}), where ${{\mathscr}{M}}_D$ is the ring of meromorphic functions on $D$. 
Aleksandrov~\cite[\S4, Thm.]{Ale90} proved that $\omega^p_D$ agrees with the module of $p$th regular differential forms on $D$ (see \cite{Bar78}). 
For further properties of the logarithmic residue, we refer to \cite{GS11}.

Let us assume that $D$ is normal crossing in codimension one which means that, outside of an analytic subset codimension at least two, a representative of $D$ is locally isomorphic to a normal crossing divisor.
Then we know by \cite[Thm ~1.1]{GS11}, that $\omega^0_{D}$ can be identified with the ring of weakly holomorphic functions on $D$. 
Let $\pi\colon\tilde{D} \to D$ be the normalization map and let ${{\mathscr}{C}}_D$ denote be the conductor ideal. 
This ring, in turn, can be identified with $\pi_*{{\mathscr}{O}}_{\tilde{D}}$ where ${{\mathscr}{O}}_{\tilde{D}}={\widetilde}{{{\mathscr}{O}}_D}$ is the integral closure of ${{\mathscr}{O}}_D$ in ${{\mathscr}{M}}_D$. 
Combining these facts, \eqref{12} for $p=1$ turns into a short exact sequence
\begin{equation}\label{9}
{\SelectTips{cm}{}\xymatrix}{
0 \ar[r] &  \Omega^1_X \ar[r] &  \Omega^1_X(\log D) \ar[r]^-{\rho_D^1} &  \pi_*{{\mathscr}{O}}_{\tilde{D}} \ar[r] &  0
}
\end{equation}
whenever $D$ is normal crossing in codimension one. 
The dual of the inclusion map in \eqref{9} is the inclusion map in \eqref{13}.
Note $\pi_*{{\mathscr}{O}}_{\tilde{D}}$ is finite as ${{\mathscr}{O}}_D$- and hence also as ${{\mathscr}{O}}_X$-module. 

Let us further assume that $D$ is a free divisor germ. 
This means that $\Omega^1_X(\log D)$ is a locally free sheaf of rank $n+1=\dim X$. 
Since \eqref{9} is a free ${{\mathscr}{O}}_X$-resolution of $\pi_*{{\mathscr}{O}}_{\tilde D}$ of length one, $\pi_*{{\mathscr}{O}}_{\tilde D}$ is Cohen--Macaulay by the Auslander--Buchsbaum formula. 
Then \cite[Proof of Thm.~3.4]{MP89} applies to show that the first Fitting ideal of the ${{\mathscr}{O}}_X$-module $\pi_*{{\mathscr}{O}}_{\tilde D}$ equals the preimage ${{\mathscr}{C}}'_D$ of the conductor ideal ${{\mathscr}{C}}_D$ in ${{\mathscr}{O}}_X$. 
By \cite[Thm.~1.6]{GS11} (or \cite[Proof of Thm.~3.4]{MP89}), ${{\mathscr}{C}}_D={{\mathscr}{J}}_D$ and hence ${{\mathscr}{C}}'_D$ equals the preimage ${{\mathscr}{J}}'_D$ of the Jacobian ideal ${{\mathscr}{J}}_D$ in ${{\mathscr}{O}}_X$.

\begin{rmk}\label{48}
Assume that $D$ is a free divisor germ and let ${\underline}\delta=(\delta_1,\dots,\delta_{n+1})$ and $(\omega_1,\dots,\omega_{n+1})$ be bases of $\operatorname{Der}_X(-\log D)$ and $\Omega_X^1(\log D)$ respectively. 
Then there are expansions 
\begin{equation}
\delta_j=\sum_{i=1}^{n+1}a_{i,j}\frac{\partial}{{\partial} x_i},\quad dx_i=\sum_{j=1}^{n+1}\lambda_{j,i}\omega_i,\quad a_{i,j}=\lambda_{j,i}\in{{\mathscr}{O}}_X.\label{44}
\end{equation}
With this notation, the so-called Saito matrix  $A=(a_{i,j})$ represents the inclusion map in \eqref{13} while $\Lambda=A^t=(\lambda_{i,j})$ represents that in \eqref{9}. 
Saito's freeness criterion (see \cite[(1.8) Thm.~ii)]{Sai80}) states that
\[
f=\det(A)=\det(\Lambda)\in{{\mathscr}{O}}_X
\]
is a defining equation for $D$.
\end{rmk}

The preceding discussion proves the following

\begin{prp}\label{34}
Let $D$ be a free divisor germ in $X$ which is normal crossing in codimension one. 
Then the submaximal minors of any Saito matrix $A$ generate the preimage ${{\mathscr}{J}}'_D$ of the Jacobian ideal ${{\mathscr}{J}}_D$ of $D$ in ${{\mathscr}{O}}_{X}$.\qed
\end{prp}

The following result justifies that we may assume that $D$ is irreducible.

\begin{prp}\label{42}
The combination of freeness and normal crossing in codimension one descends to all irreducible components of a divisor germ.
\end{prp}

\begin{proof}
For any free divisor germ $D$, the normal crossing hypothesis is equivalent to the ideal ${{\mathscr}{J}}_f$ generated by the partial derivatives of $f$ being radical by \cite[Thm.~1.6, Rmk.~1.7]{GS11}.
By \cite[2.2.(i)]{Fab12}, the combination of freeness and radical ${{\mathscr}{J}}_f$ descends to all irreducible components of $D$.
\end{proof}

Denote by $M_{i,j}$ the $(n\times n)$-minor obtained from $\Lambda$ by deleting the $i$th row and $j$th column.
We shall need the following 

\begin{lem}\label{43}
Let $D$ be an irreducible free divisor germ satisfying 
\begin{equation}\label{18}
\operatorname{Der}_X(-\log D)\subset{\mathfrak{m}}_X\operatorname{Der}_{\mathds{C}}({{\mathscr}{O}}_X).
\end{equation}
Then $M_{i,j}\ne0$ for all $i,j=1,\dots,n+1$.
\end{lem}

\begin{proof}
Let $\omega_1,\dots,\omega_{n+1}$ be a basis of $\Omega_X^1(\log D)$ and denote by $g_i$ the image of $\omega_i$ under the residue map $\rho_D^1$ in \eqref{9}.
By Cramer's rule (see \cite[Lem.~3.3]{MP89}), we have
\[
M_{i,j}g_k=\pm M_{k,j}g_i
\]
in ${{\mathscr}{O}}_{\tilde D}$ which is a domain by irreducibility of $D$.
Assuming that $M_{i,j}=0$ for some $i,j\in\{1,\dots,n+1\}$, this implies that $M_{k,j}g_i=0$ for all $k=1,\dots,n+1$.
So either $M_{k,j}$ or $g_i$ must be zero.
Assuming first that $g_i=0$, $\omega_i\in\Omega_X^1$ by the residue sequence \eqref{9}.
But then duality gives $1={{\left\langle{\delta_i,\omega_i}\right\rangle}}\in{\mathfrak{m}}_X$ contradicting to \eqref{18}.
Therefore, we must have $M_{k,j}=0$ in ${{\mathscr}{O}}_D$ or, in other words, $f$ divides $M_{k,j}$ in ${{\mathscr}{O}}_X$ for all $k=1,\dots,n+1$.
In terms of \eqref{44}, condition \eqref{18} means that $\lambda_{j,i}\in{\mathfrak{m}}_X$ and hence
\[
\det\Lambda=\sum_{k=1}^{n+1}\pm\lambda_{k,j}M_{k,j}\in{\mathfrak{m}}_Xf
\]
contradicting Saito's criterion (see Remark~\ref{48}).
\end{proof}

\begin{rmk}\label{45}
By \cite[(3.6) Proof]{Sai80}, there is a product structure
\begin{equation}\label{46}
(D,X)\cong (D'\times ({\mathds{C}}^r,0),X)
\end{equation}
for some $r\in\{0,\dots,n\}$ such that the divisor germ $D'$ in $X'=({\mathds{C}}^{n+1-r},0)$ satisfies \eqref{18}.
Moreover, $D'$ is unique up to isomorphism by the cancellation lemma for analytic spaces (see \cite[Lem.~1.5]{Eph78}).
We say that $D$ is suspended if $r>0$.
\end{rmk}

\section{Quasihomogeneous divisors}\label{27}

From now we assume that $D$ is quasihomogeneous.
This means, by definition, that there is a coordinate system $x_1,\dots,x_n$ with weights $w_i=\deg(x_i)\in{\mathds{Q}}_+$, $i=1,\dots,n$, such that $D$ is defined by a weighted homogeneous polynomial $f$ of degree $d\in{\mathds{Q}}_+$. 
We shall order the weights increasingly by
\begin{equation}\label{26}
0<w_1\le w_2\le\cdots\le w_{n+1}.
\end{equation}
and denote by
\begin{equation}\label{14}
\chi:=\sum_{i=1}^{n+1}w_ix_i\frac{\partial}{{\partial} x_i}
\end{equation}
the corresponding (logarithmic standard) Euler derivation. 
More intrinsically, a general $\chi\in\operatorname{Der}_{\mathds{C}}({{\mathscr}{O}}_X)$ is a standard Euler derivation if there is a regular system of parameters $x_1,\dots,x_n$ of ${{\mathscr}{O}}_X$ consisting of eigenvectors of $\chi$ with eigenvalues in ${\mathds{Q}}_+$.
It is logarithmic along $D$ if the ideal of $D$ is generated by some eigenvector $f$ of $\chi$.
In these terms, quasihomogeneity of $D$ is equivalent to the existence of a logarithmic standard Euler derivation $\chi$ along $D$.

Weighted homogeneity of degree $e$ of an element $0\ne g\in{{\mathscr}{O}}_X$ can be expressed by
\begin{equation}\label{15}
[\chi,g]=\chi(g)=eg.
\end{equation}
In particular, quasihomogeneity of $D$ descends to any irreducible component of $D$ using a fixed $\chi$.
The Euler derivation $\chi$ turns $\operatorname{Der}_{\mathds{C}}({{\mathscr}{O}}_X)$ into a graded ${{\mathscr}{O}}_X$-module (see \cite[(2.2)~Satz]{SW73}).
More explicitly, a derivation $0\ne\delta\in\operatorname{Der}_{\mathds{C}}({{\mathscr}{O}}_X)$ is weighted homogeneous of degree $e$, if in the expansion $\delta=\sum_{i=1}^{n+1}a_i\frac{\partial}{{\partial} x_i}$, $\deg(a_i)-w_i=e$ for all $i=1,\dots,n+1$. 
In other words, the weighted degree of $\delta(g)$ is $e+\deg(g)$ for any weighted homogeneous polynomial $g$. 
For example, the Euler derivation in \eqref{14} is weighted homogeneous of degree $0$.
Analogously to \eqref{15}, the weighted homogeneity of $\delta$ can be expressed by
\begin{equation}\label{19}
[\chi,\delta]=e\delta.
\end{equation}
Indeed, for any weighted homogeneous polynomial $g$, we have
\begin{equation}\label{16}
[\chi,\delta](g)=\chi(\delta(g))-\delta(\chi(g))=\left(\deg(\delta(g))-\deg(g)\right)\delta(g)=\deg(\delta)\delta(g).
\end{equation}
Due to \eqref{15} and \eqref{19}, we prefer to speak of $\chi$-weights, $\chi$-homogeneity and the $\chi$-degree 
\[
\deg(-)=\deg_\chi(-)
\]
independently of the choice of coordinate system.
For notational convenience, we set $\deg(0):=-\infty$.
The $f$ chosen above is simply a non zero $\chi$-homogeneous element of minimal $\chi$-degree of the ideal of $D$.
In particular, $f$ is uniquely determined by $\chi$ up to a constant factor.

\begin{ntn}\label{25}
For any ${\mathds{C}}[\chi]$-module $M$, we denote by $M_w$ (resp.~$M_{\le w}$, etc.) the ${\mathds{C}}[\chi]$-submodule generated by all $\chi$-homogeneous elements $m\in M$ of $\chi$-degree $\deg(m)=w$ (resp.~ $\deg(m)\le w$, etc.).
\end{ntn}

We denote by $\operatorname{Der}(-\log f)$ the submodule of $\operatorname{Der}(-\log D)$ which annihilates $f$.
Because $f$ is uniquely determined by $\chi$ up to a constant factor, the same holds true for $\operatorname{Der}(-\log f)$.
Note that 
\begin{equation}\label{28}
\operatorname{Der}(-\log D)={{\mathscr}{O}}_X\chi\oplus\operatorname{Der}(-\log f)
\end{equation}
as ${{\mathscr}{O}}_X$-modules.
Indeed, for any $\delta\in\operatorname{Der}(-\log D)$, we have $\delta-a\chi\in\operatorname{Der}(-\log f)$ where $a=\frac{\delta(f)}{d\cdot f}\in{\mathds{C}}$.

\section{Lie algebras of logarithmic derivations}\label{51}

It is known \cite[(1.5) ii)]{Sai80} that $\operatorname{Der}_X(-\log D)$ is closed under the bracket operation $[-,-]$.
In particular, $\chi$ acts on $\operatorname{Der}_X(-\log D)$ turning it into a graded ${{\mathscr}{O}}_X$-module.
By \eqref{16} with $g=f$, the same applies to $\operatorname{Der}_X(-\log f)$.
By the Jacobi identity
\[
[\chi,[\delta,\eta]]=[[\chi,\delta],\eta]+[\delta,[\chi,\eta]]
\]
we have
\begin{equation}\label{21}
\deg([\delta,\eta])=\deg(\delta)+\deg(\eta)
\end{equation}
if $[\delta,\eta]\ne0$ and hence $\operatorname{Der}_X(-\log D)_{\le0}$ is a finite dimensional complex Lie algebra.
On the other hand, let $\Delta=(\delta_1,\dots,\delta_r)$ be a minimal set of $\chi$-homogeneous generators of $\chi$-degree
\[
\deg(\delta_i)=d_i.
\]
Then $\Delta$ generates a (possibly infinite) complex Lie algebra
\begin{equation}\label{40}
{\mathfrak{d}}={\mathfrak{d}}_{\Delta}\subset\operatorname{Der}_X(-\log D)
\end{equation}
with a Lie subalgebra
\begin{equation}\label{24}
{\mathfrak{a}}={\mathfrak{a}}_{\Delta}:=\{\delta\in{\mathfrak{d}}\mid \delta(f)=0\}\subset{\mathfrak{d}}
\end{equation}
of elements annihilating $f$.
By \eqref{28}, we have
\begin{equation}\label{23}
{\mathfrak{d}}={\mathds{C}}\chi\ltimes{\mathfrak{a}}.
\end{equation}
We shall be interested in the condition
\begin{equation}\label{36}
{\mathfrak{d}}\subset\operatorname{Der}_X(-\log D)_{\le0}
\end{equation}
which implies in particular that ${\mathfrak{d}}$ is a finite dimensional complex Lie algebra.
By \eqref{21}, \eqref{36} is equivalent to $\deg(\delta_i)\le0$ for all $i=1,\dots,r$.

Let us further assume that $D$ is a free divisor germ.
By \eqref{15}, there is a $\chi$-homogeneous basis $\Delta=(\delta_1,\dots,\delta_{n+1})$ of $\operatorname{Der}_X(-\log D)$ such that $\delta_1=\chi$.
It follows that the $\chi$-degree of the entry $\lambda_{i,j}$ of $\Lambda$ equals 
\begin{equation}\label{11}
\deg(\lambda_{i,j})=d_i + w_j.
\end{equation}

\begin{exa}
Let $D$ be the divisor in ${\mathds{C}}^2$ defined by $y^2-x^3=0$. 
This divisor is free and quasihomogeneous at $(0,0)$ with respect to the weight $\deg(x)=2$, $\deg(y)=3$. 
One can choose $\delta_1=2x \frac{\partial}{\partial x} + 3 y \frac{\partial}{\partial y}$ and $\delta_2=2y\frac{\partial}{\partial x} + 3x^2\frac{\partial}{\partial y}$ as a free basis of $\operatorname{Der}_X(-\log D)$. 
Then the matrix $\Lambda$ takes the form:
\begin{equation*}
\Lambda=
\begin{pmatrix}
2x & 3y \\
2y & 3x^2
\end{pmatrix}
\end{equation*}
It is clear that $d_1=0$ and $d_2=1$ in this example.
\end{exa}

\begin{rmk}\label{47}
By \cite[Lem.~2.2.(iv)]{CMN96}, the following properties hold equivalently for $D$ and $D'$ in \eqref{46}: irreducibility, smoothness, normal crossing (in codimension one), freeness, quasihomogeneity, and property \eqref{36}.
\end{rmk}

\begin{prp}\label{20}
Let $D$ be an irreducible quasihomogeneous free divisor germ which is normal crossing in codimension one.
Then $\operatorname{Der}_X(-\log D)$ admits a $\chi$-homogeneous basis whose elements have non positive $\chi$-degree.
In other words, \eqref{36} holds.
\end{prp}

\begin{proof}
By Remark~\ref{47}, we may hence assume that \eqref{18} holds true and hence $\deg(M_{i,j})\ne-\infty$ by Lemma~\ref{43}.
Using \eqref{11} and Saito's criterion (see Remark~\ref{48}), we can therefore compute
\begin{align*}
\deg(M_{i,j}) 
&=\sum_{k=1}^{n+1}(d_k+w_k)-d_i-w_j \\
&=d-d_i-w_j \\
&=\deg\left(\frac{\partial f}{\partial x_j}\right)-d_i.
\end{align*}
Setting $j=n+1$ minimizes $\deg\left(\frac{\partial f}{\partial x_j}\right)$.
Since 
\[
{{\left\langle{M_{i,j}\mid i,j=1,\dots,n+1}\right\rangle}}={{\left\langle{\frac{\partial f}{\partial x_k}\mid k=1,\dots,n+1}\right\rangle}}
\]
by Proposition~\ref{34}, it follows that $d_i\le 0$ for all $i=1,\dots,n+1$.
\end{proof}

\section{Representation on lowest weight variables}\label{17}

From now on we assume that $D$ is not suspended, that is, \eqref{18} holds true.
For notational convenience, we shall abbreviate 
\[
{\mathfrak{m}}:={\mathfrak{m}}_X.
\]
Then ${\mathfrak{m}}$ and hence, by the Leibniz rule, also ${\mathfrak{m}}^2$ is a ${\mathfrak{d}}$-module.
Thus,
\[
\pi\colon{\mathfrak{m}}{\twoheadrightarrow}{\mathfrak{m}}/{\mathfrak{m}}^2=:\bar{\mathfrak{m}}
\]
is a ${\mathfrak{d}}$-module homomorphism. 
Since $\chi\in{\mathfrak{d}}$, this induces a ${\mathfrak{d}}_0$-module homomorphism
\[
\pi_w\colon{\mathfrak{m}}_w{\twoheadrightarrow}\bar{\mathfrak{m}}_w
\]
for any $w\in{\mathds{Q}}_+$.

From now on we shall assume \eqref{36}.
Then ${\mathfrak{m}}_{\le w}$ is a ${\mathfrak{d}}$-module and ${\mathfrak{m}}_w\subset{\mathfrak{m}}_{\le w}$ is a ${\mathfrak{d}}_0$-submodule (see Notation~\ref{25}).
Note that ${\mathfrak{m}}_{\le w}$ is finite dimensional by \eqref{26}.
In particular, the space
\[
{\mathfrak{m}}'={\mathfrak{m}}_\chi:={\mathfrak{m}}_{\le w_1}={\mathfrak{m}}_{w_1}
\]
of $\chi$-homogeneous variables of minimal $\chi$-weight is a ${\mathfrak{d}}$-module.
It follows that there is a commutative diagram of ${\mathfrak{d}}$-modules
\begin{equation}\label{35}
\xymat@C=10pt{
&{\mathfrak{m}}\ar@{->>}^-\pi[dr]\\
{\mathfrak{m}}'\ar@{^(->}[ur]\ar@{^(->}[rr]&&\bar{\mathfrak{m}}.
}
\end{equation}
For any Lie subalgebra ${\mathfrak{l}}\subset{\mathfrak{d}}$ we denote by ${\mathfrak{l}}'$ and $\bar{\mathfrak{l}}$ the image of ${\mathfrak{l}}$ in ${\mathfrak{gl}}_{\mathds{C}}({\mathfrak{m}}')$ and ${\mathfrak{gl}}_{\mathds{C}}(\bar{\mathfrak{m}})$ respectively.

Choose coordinates such that $x_1,\dots,x_k$ is a basis of ${\mathfrak{m}}'$.
Then any $\delta\in{\mathfrak{d}}$ can be written as $\delta=\sum_{i=1}^{n+1}a_i\frac{\partial}{{\partial} x_i}$ where $a_1,\dots,a_k$ must be linear functions of $x_1,\dots,x_k$ and $\delta'=\sum_{i=1}^{k}a_i\frac{\partial}{{\partial} x_i}\in{\mathfrak{d}}'$.
If $\delta\in{\mathfrak{m}}\operatorname{Der}_X(-\log D)$ then $\delta\in{\mathfrak{m}}^2\operatorname{Der}_{\mathds{C}}({{\mathscr}{O}}_X)$ by \eqref{18}.
But then the linear coefficients of $\delta'$ and hence $\delta'$ itself must be zero.
This shows that 
\begin{equation}\label{52}
{\mathfrak{d}}{\twoheadrightarrow}\operatorname{Der}_X(-\log D)/{\mathfrak{m}}\operatorname{Der}_X(-\log D){\twoheadrightarrow}{\mathfrak{d}}'.
\end{equation}
In particular, the choice of $\chi$ uniquely determines
\[
{\mathfrak{d}}'={\mathfrak{d}}'_\chi,\quad{\mathfrak{a}}'={\mathfrak{a}}'_\chi.
\]

In case $D$ is a free divisor germ, the preceding discussion shows that ${\mathfrak{d}}'$ corresponds to a block of the Saito matrix with linear entries.
Indeed, by \eqref{52}, $\Delta$ can be reordered such that $\delta_1',\dots,\delta_\ell'$ becomes a vector space basis of ${\mathfrak{d}}'$.
Then the transposed Saito matrix can be visualized as follows:
\begin{equation}\label{30}
\Lambda=
\left[
\begin{array}{c|c}
{\mathfrak{d}}' & * \\
\hline
0 & * 
\end{array}
\right]=
\left[
\begin{array}{c|c}
w_1x_1\cdots w_kx_k & * \\
\hline
{\mathfrak{a}}' & * \\
\hline
0 & * 
\end{array}
\right]
\end{equation}

\begin{prp}\label{31}
Let $D$ be a quasihomogeneous free divisor germ satisfying \eqref{36} and \eqref{18}.
If ${\mathfrak{a}}'$ is solvable then $D$ has a smooth irreducible component $D'$.
In particular, this is the case if $k=\dim_{\mathds{C}}({\mathfrak{m}}')=1$.
\end{prp}

\begin{proof}
By Lie's theorem, we may assume that $x_1$ is a common eigenvector for ${\mathfrak{a}}'$.
This means that the first column of $\Lambda$ in \eqref{30} consists of constant multiples of $x_1$ and hence $x_1\vert\det(\Lambda)$.
By Saito's criterion (see Remark~\ref{48}), $f=\det(\Lambda)$ is a defining equation for $D$ and hence $D'=V(x_1)$ is a smooth irreducible component.
\end{proof}

\begin{lem}\label{32}
Let ${\mathfrak{s}}'\subset{\mathfrak{d}}'$ be a semisimple Lie subalgebra.
\begin{enumerate}[(a)]
\item\label{32a}
Any semisimple Lie subalgebra ${\mathfrak{s}}'\subset{\mathfrak{d}}'$ admits a lift ${\mathfrak{s}}\subset{\mathfrak{a}}_0$.
In particular, \eqref{35} becomes a diagram of ${\mathfrak{s}}$-modules.
\item\label{32b} Then there is a ${\mathfrak{s}}$-module splitting of $\pi_w$ for each $w\in{\mathds{Q}}_+$.
\end{enumerate}
\end{lem}

\begin{proof}\
\begin{asparaenum}[(a)]
\item With \eqref{23}, we also have ${\mathfrak{d}}'={\mathds{C}}\chi'\ltimes{\mathfrak{a}}'$ and hence ${\mathfrak{s}}'=[{\mathfrak{s}}',{\mathfrak{s}}']\subset{\mathfrak{a}}'$.
Then ${\mathfrak{s}}'\cap r({\mathfrak{a}}')=0$ being a solvable ideal in ${\mathfrak{s}}'$, it follows that ${\mathfrak{s}}'\subset{\mathfrak{a}}'/r({\mathfrak{a}}')$.
By semisimplicity of ${\mathfrak{a}}_0/r({\mathfrak{a}}_0)$ the surjection ${\mathfrak{a}}_0/r({\mathfrak{a}}_0){\twoheadrightarrow}{\mathfrak{a}}'/r({\mathfrak{a}}')$ splits.
The image of ${\mathfrak{s}}'$ under this splitting composed with a Levi splitting ${\mathfrak{a}}_0/r({\mathfrak{a}}_0){\hookrightarrow}{\mathfrak{a}}_0$ is the desired lift ${\mathfrak{s}}$.
\item Because ${\mathfrak{s}}$ is semisimple and $\pi_w$ is surjective, the desired splitting exists due to complete reducibility. 
\end{asparaenum}
\end{proof}

Speaking in terms of variables, Lemma~\ref{32}.\eqref{32b} associates to each $x_i$ a new variable $x_i'\in {\mathfrak{m}}_{w_i}$ namely the image of $\bar x_i$ under the splitting map. 
Note that $\pi_{w_1}$ is an isomorphism, so $x_i=x_i'$ for all $x_i\in{\mathfrak{m}}'$. 
In other words, ${{\left\langle{x_1',\dots,x_k'}\right\rangle}}_{\mathds{C}}={{\left\langle{x_1,\dots,x_k}\right\rangle}}_{\mathds{C}}={\mathfrak{m}}'$ is the space of variables of lowest $\chi$ degree in both coordinate systems.
By definition of ${\mathfrak{m}}_{w_i}$, we have $\chi(x_i')=w_ix_i'$ for all $i=1,\dots,n+1$.
Thus,
\[
\chi=\sum_{i=1}^{n+1}w_ix_i'\frac{\partial}{{\partial} x_i'}
\]
remains a standard Euler derivation and $f$ remains a polynomial in the new coordinate system.
However, an additional feature of the coordinates $x_1',\dots,x_{n+1}'$ is that ${\mathfrak{s}}$ acts on the coordinate vector space ${{\left\langle{x_1',\dots,x_{n+1}'}\right\rangle}}_{\mathds{C}}$.
This is what we mean by saying that ${\mathfrak{s}}$ acts linearly in terms of the coordinates $x_1',\dots,x_{n+1}'$.

\begin{prp}\label{33}
Let $D$ be a quasihomogeneous divisor germ satisfying \eqref{36} and \eqref{18}.
Then ${\mathfrak{a}}'$ is solvable for a suitable choice of $\chi$.
\end{prp}

\begin{proof}
By quasihomogeneity of $D$ there exists a logarithmic standard Euler derivation $\chi$ along $D$.
In case the corresponding ${\mathfrak{a}}'$ is not solvable, we will describe a procedure that gradually modifies $\chi$. 
Iterating the procedure, we will reach a new $\chi$ for which ${\mathfrak{a}}'$ is solvable.

Suppose ${\mathfrak{a}}'$ is not solvable.
Then ${\mathfrak{a}}'$ must contain a semisimple Lie subalgebra ${\mathfrak{s}}'$. 
By part \eqref{32a} of Lemma~\ref{32}, such an ${\mathfrak{s}}'$ can be lifted to a semisimple Lie subalgebra ${\mathfrak{s}}\subset{\mathfrak{a}}_0$. 
Using part \eqref{32b}, we find new coordinates $x_1',\dots,x_{n+1}'$ in terms of which ${\mathfrak{s}}$ acts linearly.
By the structure theory of complex semisimple Lie algebras, ${\mathfrak{s}}$ contains a Lie subalgebra isomorphic to ${\mathfrak{sl}}_2({\mathds{C}})$ and we may assume that ${\mathfrak{s}}\cong{\mathfrak{sl}}_2({\mathds{C}})$.
Then ${\mathfrak{s}}$ is then generated by an ${\mathfrak{sl}}_2$-triple $h,v_+,v_-\in{\mathfrak{a}}_0$.
By the linearity of the action of ${\mathfrak{s}}$ and since ${\mathfrak{s}}\subset{\mathfrak{a}}_0$, $h$ is a semisimple linear operator on ${{\left\langle{x_1',\dots,x_{n+1}'}\right\rangle}}_{\mathds{C}}$ commuting with $\chi$. 
Hence, after a linear change of the coordinates $x_1',\dots,x_{n+1}'$ both $\chi$ and $h$ become diagonal.
This means that besides the standard Euler derivation $\chi$, we now have another Euler-like derivation $h$, in the sense that every $x_i'$ is an eigenvector of $h$.
However, there are negative eigenvalues for $h$ according to the representation theory of ${\mathfrak{sl}}_2$. 
By construction \eqref{35} is a commutative diagram of ${\mathfrak{s}}$-modules.
Hence, by the injectivity of the bottom map in this diagram, ${\mathfrak{m}}'$ is a non-trivial ${\mathfrak{s}}$-module. 
So again by the representation theory of ${\mathfrak{sl}}_2$, there exists a variable $x_r'=x_r\in{\mathfrak{m}}'$ of $h$-weight $h(x_r')/x_r'<0$. 

Consider now $\tilde\chi=\chi+\epsilon h$ where $\epsilon\in{\mathds{Q}}_+$ and denote by $\tilde w_i:=\tilde\chi(x_i')/x_i'=\deg_{\tilde\chi}(x_i')$ the corresponding $\tilde\chi$-weight of $x_i'$.
Choosing $\epsilon\ll1$ we have $\tilde w_i>0$ and ${\mathfrak{m}}_{\tilde\chi}'\subset{\mathfrak{m}}_{\chi}'$.
In particular, $\tilde\chi$ is again a standard Euler derivation and logarithmic by construction.
In fact $\tilde\chi(f)=\chi(f)=d\cdot f$ since $h\in{\mathfrak{s}}\subset{\mathfrak{a}}_0$. 
By the representation theory of ${\mathfrak{sl}}_2$, $v_+(x_r')\in{\mathfrak{m}}_\chi'\setminus{\mathfrak{m}}_{\tilde\chi}'$ and hence ${\mathfrak{m}}_{\tilde\chi}'\subsetneq{\mathfrak{m}}_\chi'$ is a strict inclusion of finite dimensional vector spaces.

Repeatedly passing from $\chi$ to $\tilde\chi$ is the procedure announced in the beginning of the proof. 
By the dimension drop of ${\mathfrak{m}}'$, the process terminates after finite steps. 
In the extreme case where $\dim_{\mathds{C}}({\mathfrak{m}}')$ drops to $1$, ${\mathfrak{a}}'\subset{\mathfrak{gl}}_{\mathds{C}}({\mathfrak{m}}')$ is at most $1$-dimensional and must be solvable.
\end{proof}

\section{Proof of the main theorems}

\begin{proof}[Proof of Theorem~\ref{50}]
By Remark~\ref{47}, we may assume that \eqref{18} holds.
Then the claim follows from Propositions~\ref{31} and \ref{33}
\end{proof}

\begin{proof}[Proof of Theorem~\ref{0}]
By Remark~\ref{47}, we may assume that \eqref{18} holds.
By Theorem~\ref{39} and Proposition~\ref{42}, it suffices to prove that $D$ is smooth in case $D$ is irreducible.
This follows by applying Proposition~\ref{20} and Theorem~\ref{50}.
\end{proof}

\bibliographystyle{amsalpha}
\bibliography{fdnc}
\end{document}


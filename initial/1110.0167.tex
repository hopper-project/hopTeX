\documentclass[12pt, reqno]{amsart}

\usepackage[T2A]{fontenc}
\usepackage[cp1251]{inputenc}
\usepackage[english]{babel}
\usepackage{amsfonts,amssymb, amsthm} 

\theoremstyle{plain}

\newtheorem{condition}{Condition}
\newtheorem{lemma}{Lemma}
\newtheorem{theorem}{Theorem}
\newtheorem{proposition}{Proposition}
\newtheorem{remark}{Remark}
\newtheorem{col}{Colorary}

\author{Nikita\,V.~Artamonov}
\email{nikita.artamonov@gmail.com}

\title[On exponential decay rate of semigroup]{On exponential decay rate of semigroup associated with 
second order linear differential equation in Hilbert space with strong damping operator}

\keywords{$C_0$-semigroup of operators,  generator of $C_0$-semigroup,
accretive operator, sectorial operator, spectrum}

\thanks{This paper is supported by the Russian Foundation of Basic Research (project No
11-01-00790)}

\begin{document}

\begin{abstract}
We obtain  estimate of the exponential decay rate of
semigroup associated with second order linear differential
equation $u''+Du'+Au=0$ in Hilbert space. We assume that $A$
is a selfadjoint positive definite operator,  $D$ is an accretive
sectorial operator and ${\operatorname{Re}} D\geq\delta A$, $\delta>0$.
We obtain a location of the spectrum of a pencil associated with
linear differential equation.

\end{abstract}

\maketitle

\textbf{1.}  In Hilbert space $H, (\cdot,\cdot)$ we consider a second order linear differential equation
 \begin{equation}\label{SeconfOrderDiffEqn}
  u''(t)+Du'(t)+Au(t)=0,
 \end{equation}
here $u(t)$ is a vector-value function on semi-axis ${{\mathbb R}}_+=[0,+\infty)$. Many evolution equations arising in mechanics
 can be reduced to the equation \eqref{SeconfOrderDiffEqn} in an appropriate
space (see, for example,  \cite{CramerLatushkin03, Huang97, JacobTrunk09}). In this case $A$ represents 
potential energy and $D$ represents dissipation ($D$ is a damping operator). We will assume that
\begin{condition}\label{ConditionA}
$A$ is a selfadjoit positive definite operator with dense domain ${{\mathcal D}}(A)$. Let
 \[
     a_0=\inf_{x\in{{\mathcal D}}(A),\|x\|=1}(A x,x)=
     \inf_{x\in{{\mathcal D}}(A^{1/2}),\|x\|=1}(A^{1/2} x,A^{1/2}x)>0.
 \]
\end{condition}
By $H_s$ we denote a collection of Hilbert spaces generated by $A^{1/2}$, i.e. for $s\geq0$ the space $H_s$ is the
domain ${{\mathcal D}}(A^{s/2})$ endowed with the norm $\|x\|_s=\|A^{s/2}x\|$, for $s<0$ the space $H_s$ is the completion of $H$
with respect to the norm $\|\cdot\|_s$. By definition $H_0=H,H_1={{\mathcal D}}(A^{1/2}),H_2={{\mathcal D}}(A)$ and 
$H_s\hookrightarrow H_r$ for $s>r$. Since $|(x,y)|=|(A^{-s/2}x,A^{s/2}y)|\leq\|x\|_{-s}\cdot\|y\|_s$  for $s>0$ and for all 
$x\in H, y\in H_s$, then the sesquilinear  form $(x,y)$ can be extended by continuity to a sesquilinear form 
$(x,y)_{-s,s}$ on $H_{-s}\times H_s$. Therefore we can regard the space $H_{-s}$ as a dual of $H_s$ ($H_{-s}=H_s^*$), 
the duality is determined by  the sesquilinear form $(x,y)_{-s,s}$ (duality with respect to the pivot space $H$). By ${{\mathcal{L} }}(X,Y)$
we will denote a space of bounded operators acting from a space $X$ into a space $Y$. Operator $A$ can be regarded
as a bounded operator acting in the collection of Hilbert spaces: $A\in{{\mathcal{L} }}(H_s,H_{s-2})$ $\forall s\in{{\mathbb R}}$.

Following \cite{JacobTrunk09} we will assume that
\begin{condition}\label{ConditionD}
$D\in{{\mathcal{L} }}(H_1,H_{-1})$ is an accretive sectorial operator, i.e. ${\operatorname{Re}}(Dx,x)_{-1,1}\geq0$ and 
$|{\operatorname{Im}}(Dx,x)_{-1,1}|\leq\nu {\operatorname{Re}}(Dx,x)_{-1,1}$ for all $x\in H_1$  and some $\nu>0$.
\end{condition}
Denote (infimum with respect to $x\in H_1,x\ne0$)
 \[
    \alpha=\inf\frac{{\operatorname{Re}}(Dx,x)_{-1,1}}{\|x\|^2_{-1}},\;
    \beta=\inf\frac{{\operatorname{Re}}(Dx,x)_{-1,1}}{\|x\|^2},\;
    \delta=\inf\frac{{\operatorname{Re}}(Dx,x)_{-1,1}}{\|x\|^2_{1}}.
 \]
Inequality $\|x\|^2_1\geq a_0\|x\|^2\geq a_0^2\|x\|^2_{-1}$ ($\forall x\in H_1$)  implies, that $$\alpha\geq a_0\beta\geq a_0^2\delta\geq0.$$
By $\|D\|=\sup_{x\in H_1,\|x\|_1=1}\|Dx\|_{-1}$ we denote a norm of the operator  $D$. Note, that the operator $D\in{{\mathcal{L} }}(H_1,H_{-1})$ 
is accretive (sectorial) in the sense of the condition \ref{ConditionD} iff the operator $A^{-1/2}DA^{-1/2}\in{{\mathcal{L} }}(H)$ 
is accretive (sectorial). 

With the linear differential equation \eqref{SeconfOrderDiffEqn} we associate a quadratic operator
pencil \cite{HrynivShkalikov04,JacobTrunk09}
 \[
    L(\lambda)=\lambda^2 {{\mathcal J}}+\lambda D+A,
 \]
here ${{\mathcal J}}:H_1\hookrightarrow H_{-1}$ is an embedding operator, $\lambda\in{{\mathbb C}}$ is a spectral parameter.
We regard the pencil as an operator-function $L(\lambda)\in{{\mathcal{L} }}(H_1,H_{-1})$. As usual  one can define a resolvent set
  \[
     \rho(L)=\{\lambda\in{{\mathbb C}}\;:\;\exists L^{-1}(\lambda)\in{{\mathcal{L} }}(H_{-1},H_1)\}
  \]
and a spectrum $\sigma(L)={{\mathbb C}}\backslash\rho(L)$ of the pencil $L(\lambda)$.

The second order differential equation \eqref{SeconfOrderDiffEqn} can be linearized as a first order
differential equation \cite{HrynivShkalikov04,Huang97,JacobTrunk09}
\begin{equation}\label{FirstOrderSystem}
 w'(t)={{\mathcal{T} }} w(t), \quad w(t)=\begin{pmatrix} u' & u \end{pmatrix}^\top
\end{equation} 
in "energy" space ${{\mathfrak H}}=H\times H_1$ with the matrix operator 
 \[
    {{\mathcal{T} }}=\begin{pmatrix}
     -D & -A \\ {{I}} & 0
    \end{pmatrix},\;
    {{\mathcal D}}({{\mathcal{T} }})=\left\{\begin{pmatrix} w_1 \\ w_2 \end{pmatrix}\in H_1\times H_1\,:\,
    D w_1+A w_2\in H
    \right\}
 \]
Since ${\operatorname{Re}}({{\mathcal{T} }} w,w)_{{\mathfrak H}}=-{\operatorname{Re}}(Dw_1,w_1)_{-1,1}\leq0$ and $0\in\rho({{\mathcal{T} }})$, then ($-{{\mathcal{T} }}$) is a maximal
accretive operator and, therefore, ${{\mathcal{T} }}$ is a generator of $C_0$-semigroup  $\exp(t{{\mathcal{T} }})$ of contractions
\cite{EngelNagel2000}. In \cite{HrynivShkalikov04} it is shown that if the conditions \ref{ConditionA} and
\ref{ConditionD} hold and $\beta>0$, then the operator ${{\mathcal{T} }}$ is a generator of exponentially decaying semigroup. 
In \cite{HrynivShkalikov99}, in particular, it is  proved that, under the conditions \ref{ConditionA},
\ref{ConditionD} and $\delta>0$, the operator ${{\mathcal{T} }}$ is a generator of analytic semigroup.

In papers \cite{BatkaiEngel04, Huang97} for the case $\beta>0$ was obtained results on the exponential decay rate of the
semigroup  $\exp(t{{\mathcal{T} }})$ and on location of the spectrum of the pencil $L(\lambda)$. In paper \cite{JacobTrunk09} was obtained
results on analyticity of the semigroup $\exp(t{{\mathcal{T} }})$ and on the location of the spectrum of the pencil $L(\lambda)$.
In present paper using another technique for the cases $\delta>0$ we obtain an estimate for the exponential decay rate of
the semigroup generated by ${{\mathcal{T} }}$.

\textbf{2.} In the space ${{\mathfrak H}}$ with respect to the given inner product the operator $(-{{\mathcal{T} }})$ is neither uniformly accretive nor sectorial.
For $\theta\geq0$  introduce a collection of sesquilinear forms
 \begin{multline*}
 [w,v]_\theta=(w_1,v_1)+\theta(w_1,v_1)_{-1}+(w_2,v_2)_1+\theta (w_2,v_2)+\theta(Dw_2,Dv_2)_{-1}+\\
 \theta(Dw_2,v_1)_{-1}+\theta(w_1,Dv_2)_{-1},\quad w=\begin{pmatrix} w_1 \\ w_2 \end{pmatrix},\,
  v=\begin{pmatrix} v_1 \\ v_2 \end{pmatrix}\in{{\mathfrak H}},
 \end{multline*}
here $(\cdot,\cdot)_{s}=(A^{s/2}\cdot,A^{s/2}\cdot)$ is an inner product in the space $H_s$. 
Since
 \[
   |w|^2_\theta=[w,w]_\theta=\|w_1\|^2+\|w_2\|^2_1+\theta\|w_2\|^2+\|w_1+Dw_2\|^2_{-1},
 \]
then $[\cdot,\cdot]_\theta$ is an inner product in ${{\mathfrak H}}$ topologically equivalent to
the given one. Obviously $[\cdot,\cdot]_0=(\cdot,\cdot)_{{\mathfrak H}}$.
\begin{proposition}\label{QuaraticFormEstimate}
Let the conditions \ref{ConditionA} and \ref{ConditionD} are satisfied and $\delta>0$. 
Then for arbitrary $\theta>0$ and $0\leq b\leq\sqrt{\theta}$ for all 
$w=(w_1,w_2)^\top\in{{\mathcal D}}({{\mathcal{T} }})$ the following inequalities
 \[
    {\operatorname{Re}}[{{\mathcal{T} }} w,w]_\theta\leq -\omega_\theta|w|^2_\theta,\quad 
    \bigl|{\operatorname{Im}}[{{\mathcal{T} }} w,w]_\theta\bigr|\leq M_{\theta,b}\bigl|{\operatorname{Re}}[{{\mathcal{T} }} w,w]_\theta\bigr|+b|w|^2_\theta,
 \] 
hold, where
 \begin{gather*}
   \frac{1}{\omega_\theta} =\frac{1}{\beta}+\frac{\|D\|^2}{2\delta}+\frac12\left(\frac{\theta}{\alpha}+\frac{1}{\theta\delta}\right)+
   \frac12\sqrt{\left(\frac{\theta}{\alpha}+\frac{1}{\theta\delta}+\frac{\|D\|^2}{\delta}\right)^2-\frac{4}{\alpha\delta}}>0, \\
   M_{\theta,b} = \nu+\frac{2}{\delta(b+\sqrt{b^2+4\theta})}+\frac{\sqrt{\theta}-b}{\beta}.
 \end{gather*}
\end{proposition}
The resolvent set of the operator ${{\mathcal{T} }}$ is non-empty, therefore 
\begin{col}
Under the conditions of the proposition \ref{QuaraticFormEstimate}
 \[
    \sigma({{\mathcal{T} }})\subset\{\lambda\in{{\mathbb C}}\;:\; {\operatorname{Re}}\lambda\leq -\omega_\theta, 
     |{\operatorname{Im}}\lambda|\leq M_{\theta,b}|{\operatorname{Re}}\lambda|+b\}.
 \]
\end{col}
Putting $b=0$ we obtain
\begin{col}
Under the conditions of the proposition \ref{QuaraticFormEstimate}
for all $\theta>0$
 \[
    \sigma({{\mathcal{T} }})\subset\left\{\lambda\in{{\mathbb C}}\;:\;{\operatorname{Re}}\lambda\leq -\omega_\theta, 
    |{\operatorname{Im}}\lambda|\leq  \left(\nu+\frac{1}{\delta\sqrt{\theta}}+\frac{\sqrt{\theta}}{\beta}\right)
    |{\operatorname{Re}}\lambda|\right\}
 \]
\end{col}
It's easy to prove \cite{JacobTrunk09}, that $\sigma(L)=\sigma({{\mathcal{T} }})$ .
\begin{theorem}\label{MainResult}
Let the conditions \ref{ConditionA} and \ref{ConditionD} are satisfied and $\delta>0$.
Then the operator ${{\mathcal{T} }}$ is a generator of the (analytic) semigroup $\exp(t{{\mathcal{T} }})$ in the space 
${{\mathfrak H}}$ with exponential decay rate 
\[
    \omega =\left(\frac{1}{\beta}+\frac{2}{\sqrt{\alpha\delta}}+\frac{\|D\|^2}{2\delta}+
   \sqrt{\frac{4\|D\|^2}{\delta\sqrt{\alpha\delta}}+\frac{\|D\|^4}{\delta^4}}\right)^{-1}>0,
 \]
i.e. for all $t\geq0$ the inequality $\|\exp(t{{\mathcal{T} }})\|_{{\mathfrak H}}\leq {\operatorname{const}}\cdot\exp(-\omega t)$ holds. 
For all $b\geq0$
 \[
     \sigma(L)=\sigma({{\mathcal{T} }})\subset\{\lambda\in\mathbb{C}\;|\; {\operatorname{Re}}\lambda\leq -\omega,
    |{\operatorname{Im}}\lambda|\leq M_b|{\operatorname{Re}}\lambda|+b\}
 \]
where  $M_b=\min_{\theta\geq b^2}M_{\theta,b}$. 
\end{theorem}
\begin{col}
Under the conditions of the theorem \ref{MainResult} for all $(u_1,u_0)^\top\in{{\mathcal D}}({{\mathcal{T} }})$ there exists a
unique solution $u(t)$ of the Cauchy problem for the differential equation
\eqref{SeconfOrderDiffEqn} with initial conditions $u(0)=u_0$, $u'(0)=u_1$
and 
 \[
    \|u(t)\|^2_1+\|u'(t)\|^2\leq{\operatorname{const}}\cdot\exp(-2\omega t)(\|u_0\|^2_1+\|u_1\|^2).
 \]
\end{col}
Putting $b=0$ we have
\begin{col}
Under the conditions of the theorem \ref{MainResult}
 \[
     \sigma(L)=\sigma({{\mathcal{T} }})\subset\{\lambda\in\mathbb{C}\;|\; {\operatorname{Re}}\lambda\leq -\omega,
    |{\operatorname{Im}}\lambda|\leq \left(\nu+\frac{2}{\sqrt{\delta\beta}}\right)|{\operatorname{Re}}\lambda|\}.
 \]
\end{col}
\begin{remark}
In \cite{JacobTrunk09} under the condition of the theorem \ref{MainResult} was obtained the following location
of the pencil's spectrum
 \begin{align*}
     \sigma(L)=\sigma({{\mathcal{T} }})&\subset\{\lambda\in{{\mathbb C}}\;|\; {\operatorname{Re}}\lambda\leq0,\;
    |{\operatorname{Im}}\lambda|\leq \nu|{\operatorname{Re}}\lambda|+\delta^{-1}\}\\
     \sigma(L)=\sigma({{\mathcal{T} }})&\subset\left\{\lambda\in{{\mathbb C}}\;|\;\delta\leq\frac{|{\operatorname{Re}}\lambda|}{a_0^{-2}+|\lambda|^{-2}}\right\}
 \end{align*}
\end{remark}

\textbf{3.} With the operator ${{\mathcal{T} }}$ in the space ${{\mathfrak H}}$ endowed with the inner
product $[\cdot,\cdot]_\theta$ we can associate a linearization of the pencil
$L(\lambda)$ under the form $\mathbf{L}(\lambda)=\lambda \mathbf{Q}-\mathbf{T}$,
where
 \[ 
   \mathbf{Q}=\begin{pmatrix} {{I}}+\theta A^{-1} & \theta A^{-1}D \\ \theta D^* A^{-1} & A+\theta{{I}}+\theta D^* A^{-1}D \end{pmatrix}
   \quad
   \mathbf{T}=\begin{pmatrix} -D & -A-\theta{{I}} \\ A+\theta{{I}} &   -\theta D^* \end{pmatrix}.
 \]
The linearization  $\mathbf{L}(\lambda)$ can be regarded as an operator-function
$\mathbf{L}(\lambda)\in{{\mathcal{L} }}(H_1\times H_1,H_{-1}\times H_{-1})$.
 
\textbf{Acknowledgement} The author thanks prof. A.A.~Shkalikov and prof. C.~Trunk 
for fruitful discussions.

\begin{thebibliography}{99}

\bibitem{HrynivShkalikov99}
R.~O.~Hryniv, A.~A.~Skalikov, \textit{Operator models in elastisity theory and hydrodynamics and
associated analytic semigroups.}, Mosc. Univ. Math. Bull., 54(5), 1999, 1--10.

\bibitem{HrynivShkalikov04}
 R.~O.~Hryniv, A.~A.~Shkalikov, \textit{ Exponential decay of solution energy for equations associated with
some operator models of mechanics.}, Fuct. Anal. Appl. 83(3), 2004, 163--172.

\bibitem{BatkaiEngel04}
A.~B\'{a}tkai, K.~Engel, \textit{Exponential decay of $2\times2$ operator matrix semigroups.},
 J. Comput. Anal. Appl., 6(2), 2004, 153--163.

\bibitem{CramerLatushkin03}
D.~Cramer, Yu.~Latushkin, \textit{ Gearhart--Pr\"{u}ss theorem in stability
for wave equations: a survay.}, Lect. Notes Pure Appl. Math., 
234, 2003, 105--119

\bibitem{EngelNagel2000}
K.~Engel, R.~Nagel, One-Parameter Semigroups for Linear Evolution
Equations.  Graduate Texts in Mathematics., Springer--Verlag,
Berlin–-Heildelberg–-New York, 2000

\bibitem{Huang97}
S.--Z.~Huang, \textit{On energy decay rate of linear damped elastic systems.},
Tuebinger Ber. Funktionalanal., 6, 1997, 65-97

\bibitem{JacobTrunk09}
B.\,~Jacob , C.\,~Trunk, \textit{Spectrum and analyticity of semigroups
arising in elasticity theory and hydromechanics.},
Semigroup Forum., 79(1), 2009, 79--100

\end{thebibliography}

\end{document}


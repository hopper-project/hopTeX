\documentclass{amsart}
\usepackage{color}
\usepackage{epsfig}
\usepackage{float}
\usepackage{graphicx}
\newtheorem{theorem}{Theorem}[section]
\newtheorem{corollary}[theorem]{Corollary}
\newtheorem{lemma}[theorem]{Lemma}
\newtheorem{proposition}[theorem]{Proposition}
\theoremstyle{definition}
\newtheorem{definition}[theorem]{Definition}

\newtheorem{exercise}[theorem]{Exercise}
\theoremstyle{remark}
\newtheorem{remark}[theorem]{Remark}
\newtheorem{example}[theorem]{Example}
\numberwithin{equation}{section}

\begin{document}

\title[Boundary Schwarz lemma   and its applications]
{Boundary Schwarz lemma over octonions for slice regular functions and its applications}

\author[X. P. Wang]{Xieping Wang}
\address{Xieping Wang, Department of Mathematics, University of Science and
Technology of China, Hefei 230026, China}

\email{pwx$\symbol{64}$mail.ustc.edu.cn}

\thanks{This work was supported by the NNSF  of China (11071230), RFDP (20123402110068).}

\keywords{Octonions, Slice regular functions, Boundary Schwarz lemma, Landau-Toeplitz type theorem, Cauchy estimate.}
\subjclass[2010]{30G35. 32A26}

\begin{abstract}
In this paper we prove a boundary Schwarz lemma for slice regular self-mappings of the open unit ball  of the octonionic space. This result generalizes our recent result for quaternionic slice regular functions as well as the classical result  for holomorphic functions. It  involves  a Lie bracket and  an associator, which reflect the non-commutative and non-associative feature of octonions.
As applications, we obtain two Landau-Toeplitz type theorems for slice regular functions with respect to regular diameter and slice diameter, respectively. A Cauchy type estimate is also discussed.
\end{abstract}

\maketitle

\section{Introduction}
A promising theory of quaternion-valued functions of one quaternionic variable, now called slice regular functions, was initially introduced by Gentili and Struppa in \cite{GS1, GS2} and has been extensively studied over the past few years. It turns out to be significantly different from that of regular functions in the sense of Cauchy-Fueter and has  elegant applications to the functional calculus for noncommutative operators \cite{Co2}, Schur analysis \cite{ACS} and the construction and classification of orthogonal complex structures on dense open subsets of $\mathbb R^4$ \cite{GSS2014}.  Meanwhile, the theory of quaternionic slice regular functions has been extended to octonions in \cite{GS50}. The related theory of slice monogenic functions on domains in the paravector space $\mathbb R^{n+1}$ with values in the Clifford algebra $\mathbb R_n$ was introduced in \cite{Co6,Co3}. For the detailed up-to-date theory, we refer the reader to the monographs \cite{GSS, Co2} and the references therein. More recently, a connection between monogenic
and generalized slice monogenic functions is investigated in \cite{CLSS} by means of  Radon and dual Radon transform. These function theories were also unified and generalized in \cite{Ghiloni1} by means of the concept of slice functions on the so-called quadratic cone of a real alternative *-algebra, based on a  slight modification of a well-known construction due to Fueter. The theory of slice regular functions on  alternative *-algebras over $\mathbb R$ is by now well-developed through a series of papers \cite{Ghiloni5, Ghiloni3,Ghiloni4,Ghiloni6,Ghiloni7} mainly due to Ghiloni and Perotti after their seminal work \cite{Ghiloni1}.

Exactly as the quaternions $\mathbb H$ being the only real associative division algebra of dimension greater than 2, the theory of quaternionic slice regular functions should be the most beautiful one  among these function theories mentioned above. This is indeed the case. Such a special class of functions enjoys many nice properties similar to those of classical holomorphic functions of one complex variable.
Among them, we particularly mention the open mapping theorem, which is by now  known to hold only for slice regular functions on symmetric slice domains in $\mathbb H$ and allows us to prove the Koebe type one-quarter theorem for slice regular extension to the  quaternionic ball of a univalent holomorphic function on the unit disc of the complex plane, see Theorem \cite[Theorem 4.9]{RW2} for details. Furthermore, from  the analytical perspective, only for slice regular functions, the regular product and regular quotient have  an intimate connection with the usual pointwise product and quotient. It is exactly this connection which plays a crucial role in many arguments, see the monograph \cite{GSS} and the recent paper \cite{WR} for more details.

Now a rather natural question arises of whether the nice properties enjoyed by quaternionic slice regular functions can be proved for  slice regular functions on domains over the octonions $\mathbb O$,  the only   \textit{division} algebra among alternative algebras over $\mathbb R$ of dimension greater than 4. In this paper, we will mainly focus on the boundary behavior of octonionic slice regular functions, analogous to that of holomorphic functions. The latter  is of classical interest. For instance, Fatou proved that every bounded holomorphic function  on the unit disc $\mathbb D\subset \mathbb C$ admits the non-tangential limit at almost every point of $\partial\mathbb D$ \cite{Fatou}, and Stein made a far-reaching generalization of Fatou's theorem by proving that each bounded holomorphic functions on a bounded domain in $\mathbb C^n$ with $C^2$ boundary has the admissible limit at almost every boundary point \cite{Stein}.

In this paper we first prove a boundary Schwarz lemma for slice regular self-mappings of the open unit ball $\mathbb B:=\{w\in \mathbb O: |w|<1\}$. To state its precise content, we first introduce some necessary notations. For a given element $\xi=x+y I\in\mathbb O$ with
$I$ being an element of the unit 6-dimensional of purely imaginary octonions
\begin{equation}\label{Pure-sphere}
\mathbb S=\big\{w \in \mathbb O:w^2 =-1\big\},
\end{equation}
we denote by $\mathbb S_{\xi}$ the associated 6-dimensional sphere (reduces to the point $\xi$ when $\xi$ is real):
$$\mathbb S_{\xi}:=x+y\, \mathbb S=\big\{x+y J: J\in\mathbb S\big\}.$$
It is well known  that $\mathbb S_{\xi}$ is exactly the conjugacy class of $\xi$ (cf. \cite[Proposition 2, Corollary 2.1]{Serodio}). For any three octonions $u, v, w\in\mathbb O$, the \textit{Lie bracket} of $u,v $ and the \textit{associator} of  $u, v, w$ are respectively defined to be
$$[u, v]:=uv-vu,\qquad [u, v, w]:=(uv)w-u(vw).$$
We also denote by $\langle$ , $\rangle$ the standard Euclidean inner product on $\mathbb O\cong\mathbb R^8$. Now our first main result can be stated as follows:

\begin{theorem}[Schwarz]\label{BSL}
Let $\xi\in \partial\mathbb B$ and $f$ be a slice regular function on $\mathbb B\cup\mathbb S_{\xi}$ such that $f(\mathbb B)\subseteq\mathbb B$ and $f(\xi)\in \partial\mathbb B$. Then
\begin{enumerate}
  \item [(i)] it holds that
  \begin{equation}\label{Schwarz ineq1}
\begin{split}
\frac{\partial |f|}{\partial \xi}(\xi)&
=\overline{\xi}\Big(f(\xi)\overline{f'(\xi)}+\big[\bar{\xi}, f(\xi)\overline{R_{\bar{\xi}}R_{\xi}f(\xi)}\,\big]+2\big[\xi, f(\xi), R_{\bar{\xi}}R_{\xi}f(\xi)\big]\Big)
\\
&\geq\frac{\big|1-\big\langle f(0), f(\xi)\big\rangle\big|^2}{1-|f(0)|^2},
\end{split}
\end{equation}
where $\frac{\partial |f|}{\partial \xi}(\xi)$ is the directional derivative of $|f|$ along the direction $\xi$  at the boundary point $\xi\in \partial\mathbb B$;

  \item [(ii)] if further  $f(0)=0$ and $f(\xi)=\xi$, then
\begin{equation}\label{Schwarz ineq2}
\frac{\partial |f|}{\partial \xi}(\xi)= f'(\xi)-\big[\xi, R_{\bar{\xi}}R_{\xi}f(\xi)\,\big] \geq\frac{2}{1+{\rm{Re}}f'(0)}.
\end{equation}
Moreover, equality holds for  the inequality in {\rm{(\ref{Schwarz ineq2})}} if and only if $f$ is of the form
\begin{equation}\label{f-expression}
f(w)=w\big(1-wa\bar{\xi}\,\big)^{-\ast}\ast\big(w-a\xi\big)\bar{\xi}
\end{equation}
for some constant $a\in [-1,1)$.
\end{enumerate}
\end{theorem}

For the precise definitions of $R_{\bar{\xi}}R_{\xi}f(\xi)$ and $\ast$-product appeared in Theorem \ref{BSL}, see (\ref{de:Rf1}), (\ref{de:Rf2}) and Sect. 2 below. It turns out that  $R_{\bar{\xi}}R_{\xi}f(\xi)$ is intimately related to  the second coefficient in a new series expansion of slice regular function $f$, see \cite[Theorem 4.1]{Stop3} for the quaternionic case and \cite[Theorem 5.4]{Ghiloni3} for the real alternative *-algebra case.

Although the key ingredient in proving Theorem $\ref{BSL}$ is still a careful consideration of the geometrical information of $f$ at its prescribed contact point $\xi$ (i.e. $f(\xi)\in \partial\mathbb B$), two crucial difficulties arise in the octonionic setting. One is that the case of $\xi$ being a contact point of $f$ (i.e. $f(\xi)\in\partial\mathbb B$) can not reduce to the case of $\xi$ being a boundary fixed point of $f$ (i.e. $f(\xi)=\xi\in\partial\mathbb B$); the other is that, because of the lack of associativity in  $\mathbb O$, there is in general no nice connection between the regular product and the usual pointwise product unlike in the quaternionic setting. The peculiarities of the non-associative setting produce a completely new phenomenon, called the \textit{camshaft effect} in \cite{Ghiloni2}: an isolated zero of a slice regular function $f$ is not necessarily a zero for the regular product $f\ast g$ of $f$ with another slice  regular function $g$. Therefore, the method used in our recent work \cite{WR} fails in the present setting to get some satisfactory and even sharp estimate similar as the one in \cite[Theorem 2.4, Ineq.(2.6)]{WR}. Fortunately, we can come up with an effective approach to overcome partially such a technical difficulty.

Let $f$ be as described in Theorem $\ref{BSL}$. Notice that   the directional derivative $\frac{\partial f}{\partial \xi}(\xi)$ of $f$ along the direction $\xi$  at the boundary point $\xi\in \partial\mathbb B$ satisfies that
$$\frac{\partial f}{\partial \xi}(\xi)=\xi f'(\xi),$$
thus the obvious inequality
$$\Big|\frac{\partial f}{\partial \xi}(\xi)\Big|\geq \frac{\partial |f|}{\partial \xi}(\xi)$$ results in:

\begin{corollary}
Let $\xi\in \partial\mathbb B$ and $f$ be a slice regular function on $\mathbb B\cup\{\xi\}$ such that $f(\mathbb B)\subseteq\mathbb B$, $f(0)=0$ and $f(\xi)=\xi$. Then
\begin{equation*}
|f'(\xi)|\geq\frac{2}{1+{\rm{Re}}f'(0)}.
\end{equation*}
Moreover, equality holds for the last  inequality if and only if $f$ is of the form
\begin{equation*}
f(w)=w\big(1-wa\bar{\xi}\,\big)^{-\ast}\ast\big(w-a\xi\big)\bar{\xi}
\end{equation*}
for some constant $a\in [-1,1)$.
\end{corollary}

Next we shall give some applications of Theorem \ref{BSL} to the study of the rigidity of slice regular functions. We first recall the definition of \textit{regular diameter}, a suitable tool to measure the image of the open unit ball $\mathbb B$ of the octonionic space $\mathbb O$ through a slice  regular function.

\begin{definition}
Let $f$ be a  slice  regular function on $\mathbb B$ with Taylor expansion
$$f(w)=\sum\limits_{n=0}^{\infty}w^na_n.$$ For each $r\in(0, 1)$, the \textit{regular diameter} of the image of $r\mathbb B$ under $f$ is defined to be
\begin{equation}\label{regular-diam}
\widetilde{d}\big(f(r\mathbb B)\big):=\max_{u,v\in \overline{\mathbb B}}\max_{|w|\leq r}|f_u(w)-f_v(w)|,
\end{equation}
where
$$f_u(w):=\sum\limits_{n=0}^{\infty}w^n(u^na_n), \qquad f_v(w):=\sum\limits_{n=0}^{\infty}w^n(v^na_n).$$
The \textit{regular diameter} of the image of $\mathbb B$ under $f$ is defined to be
\begin{equation}\label{regular-diam}
\widetilde{d}\big(f(\mathbb B)\big):=\lim_{r\rightarrow 1^-}\widetilde{d}\big(f(r\mathbb B)\big).
\end{equation}
\end{definition}

As a first application of Theorem \ref{BSL}, we have the following Landau-Toeplitz type theorem  for octonionic slice  regular functions, whose quaternionic version was proved in \cite{Gen-Sar}.
\begin{theorem}\label{regular diam-LT}
Let $f$ be a slice  regular function on $\mathbb B$ such that
$$\widetilde{d}\big(f(\mathbb B)\big)=2.$$
Then
\begin{equation}\label{regular-diam01}
\widetilde{d}\big(f(r\mathbb B)\big)\leq 2r
\end{equation}
for each $r\in(0, 1)$, and
\begin{equation}\label{regular-diam02}
 |f'(0)|\leq 1.
\end{equation}
Moreover, equality holds in $(\ref{regular-diam01})$ for some $r_0\in(0,1)$, or in $(\ref{regular-diam02})$, if and only if $f$ is an affine function
$$f(w)=f(0)+wf'(0).$$
\end{theorem}

Let $E, \Omega$ be two subsets of $\mathbb O$ and $f: \Omega\rightarrow \mathbb O$ a function. We denote by ${\rm{diam}}\, E=\sup_{z, w\in E}|z-w|$ the Euclidean diameter of $E$ and define the \textit{slice diameter} of the image of $\Omega$ under $f$ to be
\begin{equation}\label{slice-diam}
\widehat{d}\big(f(\Omega)\big):=\sup_{I\in\mathbb S}{\rm{diam}}\, f(\Omega_I),
\end{equation}
where $\Omega_I$ denotes the intersection $\Omega\cap\mathbb C_I$, and $\mathbb S$ is the same as in (\ref{Pure-sphere}).
Thus we have another version of Landau-Toeplitz type theorem with respect to slice diameter.
\begin{theorem}\label{slice diam-LT}
Let $f$ be a slice  regular function on $\mathbb B$ such that
$$\widehat{d}\big(f(\mathbb B)\big)=2.$$
Then
\begin{equation}\label{slice-diam01}
{\rm{diam}}\,\big(f(r\mathbb B_I)\big)\leq 2r
\end{equation}
for each $r\in(0, 1)$ and each $I\in\mathbb S$, and
\begin{equation}\label{slice-diam02}
 |f'(0)|\leq 1.
\end{equation}
Moreover, equality holds in $(\ref{slice-diam01})$ for some $r_0\in(0,1)$ and $I_0\in\mathbb S$, or in $(\ref{slice-diam02})$, if and only if $f$ is an affine function
$$f(w)=f(0)+wf'(0).$$
\end{theorem}

As a second application of Theorem \ref{BSL}, we have the following Cauchy type estimate, which is an analogue of an old result due to Poukka (see \cite{Poukka}):

\begin{theorem}\label{Poukka}
Let $f$ be a bounded  slice regular function on $\mathbb B$ and $d:={\rm{Diam}}\, f(\mathbb B)$ the Euclidean diameter of the image set $f(\mathbb B)$. Then the inequality
\begin{equation}\label{Cauchy type}
\frac{|f^{(n)}(0)|}{n!}\leq \frac12d
\end{equation}
holds for every positive  integer $n\in \mathbb N$. Moreover, equality holds in $(\ref{Cauchy type})$ for some $n_0$ if and only if
$$f(w)=f(0)+\frac12 w^{n_0}d\,e^{I\theta}$$
for some $I\in\mathbb S$ and some $\theta \in \mathbb R$.
\end{theorem}
It is well worth remarking here that inequality $(\ref{Cauchy type})$ easily follows from the classical result and the splitting lemma for slice regular functions. The point here is to prove the last statement in the theorem.

The remaining part of this paper is organized as follows. In Sect. $2$, we set up basic notations and give some preliminary results from the theory of octonionic slice regular functions. In Sect. 3, we first establish some useful lemmas and use them to prove Theorem \ref{BSL}. Sect. 4 is devoted to the detailed proofs of Theorems \ref{regular diam-LT}, \ref{slice diam-LT} and \ref{Poukka}.

\section{Preliminaries}
We recall in this section some necessary definitions and preliminary results on octonionic slice regular functions. Let $\mathbb O$ denote the non-commutative and non-associative division algebra of octonions (also called Cayley numbers). A simple way to describe its construction is to consider a basis
$\mathcal{E}=\{e_0=1, e_1,\ldots, e_6, e_7\}$ of $\mathbb R^8$ and relations
\begin{equation}\label{Generation rule01}
e_ie_j=-\delta_{ij}+\psi_{ijk}e_k, \quad i, j, k=1,2,\ldots,7,
\end{equation}
where $\delta_{ij}$ is the Kronecker delta, and $\psi_{ijk}$ is totally antisymmetric in $i, j, k$, non-zero and equal to one for the seven combinations in the following set
$$\Sigma=\big\{(1, 2, 3), (1, 4, 5), (2, 4, 6), (3, 4, 7), (5, 3, 6), (6, 1, 7), (7, 2, 5)\big\}$$
so that every element in $\mathbb O$ can be uniquely written as $w=x_0+\sum_{k=1}^7x_ke_k$, with $x_k (k=1, 2, 3, 4)$ being real numbers. The full multiplication table is conveniently encoded in a 7-point projective plane, the so-called Fano mnemonic graph, shown in Fig. \ref{figure 1} below. In the Fano mnemonic graph, the vertices are labeled by $1, \ldots, 7$
instead of $e_1, \ldots, e_7$. Each of the 7 oriented lines gives a quaternionic triple. The
product of any two imaginary units is given by the third unit on the unique line
connecting them, with the sign determined by the relative orientation.

\begin{figure}[H]\label{figure 1}
\centering\includegraphics[width=4cm]
{Fano_mnemonic1.png}
\caption{Fano Mnemonic}
\end{figure}

Alternatively, $\mathbb O$ can be obtained from the quaternions $\mathbb H$ by the well-known \textit{Cayley-Dickson process}, which goes as follows. Let $\{1, e_1, e_2, e_3:=e_1e_2\}$ denote a real basis of $\mathbb H$. Each element $w\in\mathbb O$ can be written as $w=w_1+w_2e_4$, where $w_1, w_2\in\mathbb H$ and $e_4$ is a fixed imaginary unit of $\mathbb O$. The addition on $\mathbb O$ is defined componentwisely and the product is defined by
\begin{equation}\label{Generation rule02}
zw=(z_1+z_2e_4)(w_1+w_2e_4):=z_1w_1-\overline{w}_2z_2+(z_2\overline{w}_1+w_2z_1)e_4
\end{equation}
for all $z=z_1+z_2e_4$, $w=w_1+w_2e_4\in\mathbb O$. Set $e_5:=e_1e_4$, $e_6:=e_2e_4$, $e_7:=e_3e_4=(e_1e_2)e_4$. Then $\{1, e_1, e_2,\ldots, e_7\}$ forms a real basis of $\mathbb O$, and one can easily verify that the product rule given by (\ref{Generation rule02}) is the same as the one in (\ref{Generation rule01}), and hence these two approaches indeed yield the same algebra $\mathbb O$.

For each $w=x_0+\sum_{k=1}^7x_ke_k\in\mathbb O$, the real number $x_0$ is called the \textit{real part} of $w$, and is denoted by ${\rm{Re}}(w)$, while $\sum_{k=1}^7x_ke_k$ is called the \textit{imaginary part} of $w$  and is denoted by ${\rm{Im}}(w)$. Moreover, we can define in a natural fashion the \textit{conjugate} $\overline{w}:=x_0-\sum_{k=1}^7x_ke_k\in\mathbb O$, and the \textit{squared norm} $|w|^2:=w\overline{w}=\overline{w}w=\sum_{k=0}^7x_k^2$ (and by the Artin's theorem below, $|zw|=|z||w|$ for any $z,w\in\mathbb O$), which is induced by the standard Euclidean inner product on $\mathbb O\cong \mathbb R^8$ given by
\begin{equation}\label{inner product on O}
\langle z, w\rangle=\textrm{Re}(z\overline{w})=\frac12(z\overline{w}+w\overline{z}), \qquad \forall\, z,w\in\mathbb O.
\end{equation}
Also,
\begin{equation}\label{inner and norm on O}
\langle z, w\rangle=\frac12\big(|z+w|^2-|z|^2-|w|^2\big), \qquad \forall\, z,w\in\mathbb O.
\end{equation}
The \textit{associator} of three octonions $u, v, w\in\mathbb O$ is defined to be
$$[u, v, w]:=(uv)w-u(vw),$$
which is \textit{totally antisymmetric}
in its arguments $u, v, w\in\mathbb O$ and has \textit{no real part}, i.e.
\begin{equation}\label{Real part free}
{\rm{Re}}\,[u, v, w]=0.
\end{equation}
Although the associator does not vanish in
general, the octonions do satisfy a weak form of associativity known as \textit{alternativity},
namely the so-called Moufang identities (see e.g. \cite[p. 120]{Harvey} for the proofs):
\begin{equation}\label{Monfang}
(uvu)w=u(v(uw)),\quad w(uvu)=((wu)v)u, \quad u(vw)u=(uv)(wu).
\end{equation}
The underlying reason for this is the so-called \textit{Artin's theorem}, which states that \textit{in every alternative algebra, the subalgebra generated by two elements is always associative} (cf. \cite[p. 18]{Okubo}).

Each $w\in \mathbb O$ can be also expressed as $w = x + yI$, where $x, y \in \mathbb R$ and
$$I=\dfrac{{\rm{Im}}\, (w)}{|{\rm{Im}}\, (w)|}$$
 if ${\rm{Im}}(w)\neq 0$, otherwise we take $I$ arbitrarily such that $I^2=-1$.
Then $I $ is an element of the unit 6-dimensional sphere of purely imaginary octonions
$$\mathbb S=\big\{w \in \mathbb O:w^2 =-1\big\}.$$
For any two elements $I, J\in\mathbb S$, we define the \textit{wedge product} of $I$ and $J$ as
$$I\wedge J:=\frac12[I, J]=\frac12(IJ-JI),$$
which satisfies that
\begin{equation}\label{relation-inner-wedge}
IJ=-\langle I,J\rangle+I\wedge J,
\end{equation}
in view of (\ref{inner product on O}).
For every $I \in \mathbb S $ we will denote by $\mathbb C_I$ the plane $ \mathbb R \oplus I\mathbb R $, isomorphic to $ \mathbb C$, and, if $\Omega \subseteq \mathbb O$, by $\Omega_I$ the intersection $ \Omega \cap \mathbb C_I $. Also, we will denote by $\mathbb B$ the open unit ball centred at the origin, i.e.
$$\mathbb B=\big\{w \in \mathbb O:|w|<1\big\}.$$

We can now recall the definition of slice regularity.

\begin{definition} \label{de: regular} Let $\Omega$ be a domain in $\mathbb O$. A function $f :\Omega \rightarrow \mathbb O$ is called \emph{slice regular} if, for all $ I \in \mathbb S$, its restriction $f_I$ to $\Omega_I$ is \emph{holomorphic}, i.e., it has continuous partial derivatives and satisfies
$$\bar{\partial}_I f(x+yI):=\frac{1}{2}\left(\frac{\partial}{\partial x}+I\frac{\partial}{\partial y}\right)f_I (x+yI)=0$$
for all $x+yI\in \Omega_I $.
\end{definition}

A wide class of examples of slice regular functions is given by polynomials and power series. Indeed, a function $f$ is slice regular on an  open   ball $B(0,R)=\big\{w \in \mathbb O: |w|<R\big\}$ if and only if $f$ admits a power series expansion
\begin{equation}\label{Taylor expansion on ball}
f(w)=\sum_{n=0}^{\infty}w^na_n,
\end{equation}
which converges absolutely and uniformly on every compact subset of $B(0,R)$ (see \cite{GS50}).
As shown in \cite{CGSS}, the natural   domains of definition  of quaternionic slice regular functions are the  so-called  symmetric slice domains, which play
for quaternionic slice regular functions the role played by domains of holomorphy for holomorphic functions of several complex variables. This is also the case for octonionic slice regular functions.

\begin{definition} \label{de: domain}
Let $\Omega$ be a domain in $\mathbb O $.

1. $\Omega$ is called a \textit{slice domain}  if it intersects the real axis and if  for every $I \in \mathbb S $, $\Omega_I$  is a domain in $ \mathbb C_I $.

2. $\Omega$ is called an \textit{axially symmetric domain} if for every point  $x + yI \in \Omega$, with  $x,y \in \mathbb R $ and $I\in \mathbb S$, the entire two-dimensional sphere $x + y\mathbb S$ is contained in $\Omega $.
\end{definition}

A domain in $\mathbb O$ is called a \textit{symmetric slice domain} if it is not only a slice domain, but also an axially symmetric domain. By the very definition, an open ball $B(0,R)$ is  a typical  symmetric slice domain.  For slice regular functions a natural definition of slice derivative is given as follows:
\begin{definition} \label{de: derivative}
Let $f :\Omega \rightarrow \mathbb O$  be a slice regular function. For each $I\in\mathbb S$, the $I$- derivative of $f$ at $w=x+yI$
is defined by
$$\partial_I f(x+yI):=\frac{1}{2}\left(\frac{\partial}{\partial x}-I\frac{\partial}{\partial y}\right)f_I (x+yI)$$
on $\Omega_I$. The \textit{slice derivative} of $f$ is the function $f'$ defined by $\partial_I f$ on $\Omega_I$ for all $I\in\mathbb S$.
 \end{definition}

From the very definition and  Artin's theorem for alternative algebras mentioned as before, the slice derivative of a slice regular function $f:\Omega \rightarrow \mathbb O$ is still slice regular so that we can iterate the differentiation to obtain the $n$-th
slice derivative
$$\partial^{n}_I f(w)=\frac{\partial^{n} f}{\partial x^{n}}(w),\quad\,\forall \,\, n\in \mathbb N,$$
where $w=x+yI\in \Omega$. In what follows, for the sake of simplicity, we will denote the $n$-th slice derivative by $f^{(n)}$ for every $n\in \mathbb N$.
Now it is easy to see that the coefficient $a_n$ appeared in (\ref{Taylor expansion on ball}) is exactly $f^{(n)}(0)/n!$. We will also omit the term `slice' when referring to slice regular functions.

Since  slice regularity does not keep under pointwise product of two regular functions, a new multiplication operation, called the regular product (or $\ast$-product), will be defined by means of a suitable modification of the usual one subject to the non-commutativity and non-associativity of $\mathbb O$, based on the following splitting lemma (Compare to \cite[Lemma 2.7]{GS50} and \cite[Lemma 2.4]{Ghiloni3}), which is more convenient for our subsequent arguments.

\begin{lemma}[Splitting Lemma]\label{eq:Splitting}
Let $f$ be a regular function on a  domain $\Omega\subseteq \mathbb O$. Then for any $I, J, K\in\mathbb S$ with  $I, J, IJ, K$ being mutually perpendicular with respect to the standard Euclidean inner product on $\mathbb O$, there exist four holomorphic functions $F_k: \Omega_I\rightarrow\mathbb C_I$, $k=1,2,3,4$ such that
\begin{equation}\label{splitting relation}
f_I(z)=F_1(z)+F_2(z)J+\big(F_3(z)+\overline{F_4(z)}J\big)K
\end{equation}
for all $z\in\Omega_I$.
\end{lemma}

\begin{proof}
The well-known Cayley-Dickson process guarantees the existence of four functions $F_k: \Omega_I\rightarrow\mathbb C_I$, $k=1,2,3,4$ such that  equality (\ref{splitting relation}) holds for all $z\in\Omega_I$. Now it remains to verify the holomorphy of these four functions $F_k, k=1,2,3,4$. By (\ref{Generation rule02}), for every $z\in\Omega_I$,
\begin{equation}\label{holomorphy-verification}
\begin{split}
\bar{\partial}_I f(z)
&=\bar{\partial}_IF_1(z)+\bar{\partial}_IF_2(z)J
+\Big(\big(F_3(z)+\overline{F_4(z)}J \big)\bar{\partial}_{I }\Big)K
\\
&=\bar{\partial}_IF_1(z)+\bar{\partial}_IF_2(z)J
+\big(\bar{\partial}_IF_3(z)+\partial_I\overline{F_4(z)}J\big)K.
\end{split}
\end{equation}
Thus $\bar{\partial}_I f(z)=0$ implies $\bar{\partial}_IF_k(z)=0$, proving the holomorphy of these four functions $F_k, k=1,2,3,4$. The proof is complete.
\end{proof}

Let $\Omega\subseteq \mathbb O$ be a symmetric slice domain and $I, J,  K\in\mathbb S$ be such that  $I, J, IJ, K$ are mutually perpendicular with respect to the standard Euclidean inner product on $\mathbb O$. Let $f$ and $g$ be two regular functions on   $\Omega\subseteq \mathbb O$. Then the splitting lemma  above guarantees the existence of eight holomorphic functions $F_k, G_k: \Omega_I\rightarrow \mathbb C_I$, $k=1,2,3,4$ such that for all $z\in\Omega_I$,
\begin{equation*}\label{splitting of f}
f_I(z)=F_1(z)+F_2(z)J+\big(F_3(z)+\overline{F_4(z)}J\big)K
\end{equation*}
and
\begin{equation*}\label{splitting of g}
g_I(z)=G_1(z)+G_2(z)J+\big(G_3(z)+\overline{G_4(z)}J\big)K
\end{equation*}
Following the approach in \cite{CGSS}, we define the function $f_I\ast g_I:\Omega_I\rightarrow\mathbb O$ as
\begin{equation}\label{def-regular product}
f_I\ast g_I(z):=H_1(z)+H_2(z)J+\big(H_3(z)+\overline{H_4(z)}J\big)K,
\end{equation}
where
$$H_1(z)=F_1(z)G_1(z)-F_2(z)\overline{G_2(\overline{z})}
-F_3(z)\overline{G_3(\overline{z})}-F_4(z)\overline{G_4(\bar{z})}, $$
$$H_2(z)=F_1(z)G_2(z)+F_2(z)\overline{G_1(\bar{z})}
+\overline{F_3(\overline{z})}\,\overline{G_4(\overline{z})}
-\overline{F_4(\overline{z})}\,\overline{G_3(\overline{z})}, $$
$$H_3(z)=F_1(z)G_3(z)-\overline{F_2(\overline{z})}\,\overline{G_4(\overline{z})}
+F_3(z)\overline{G_1(\overline{z})}+
\overline{F_4(\overline{z})}\,\overline{G_2(\overline{z})}, $$
$$H_4(z)=F_1(z)G_4(z)+\overline{F_2(\bar{z})}\,\overline{G_3(\overline{z})}
-\overline{F_3(\bar{z})}\,\overline{G_2(\overline{z})}+F_4(z)\overline{G_1(\overline{z})}. $$
Then $f_I\ast g_I(z)$ is obviously holomorphic satisfying $f_I\ast g_I(z)=f(z)g(z)$ (independent of the choice of $I\in\mathbb S$) for all $z\in\Omega\cap\mathbb R$. Therefore, $f_I\ast g_I$ admits a \textit{unique} regular extension to $\Omega$, independent of the choice of $I\in\mathbb S$,  via the formula (2) in \cite[Proposition 6]{Ghiloni1}, analogous to  \cite[Lemma 4.4]{CGSS}. We denote  by ${\rm{ext}}(f_I\ast g_I)$ this unique regular extension of $f_I\ast g_I$.

\begin{definition}\label{R-product}
Let $f$ and $g$ be two regular functions on  a symmetric slice domain $\Omega\subseteq \mathbb O$. Then the regular function
$$f\ast g(w):={\rm{ext}}(f_I\ast g_I)(w)$$
defined as the extension of (\ref{def-regular product}) is called the \textit{regular product} (or \textit{$\ast$-product}) of $f$ and $g$.
\end{definition}

\begin{remark}\label{remark on R-product01}
In the special case that $\Omega=B(0,R)$, there is a more direct way of defining the regular product.
Let $f$, $g:B(0,R)\rightarrow \mathbb O$ be two regular functions and let
\begin{equation*}\label{def of regular product}
f(w)=\sum\limits_{n=0}^{\infty}w^na_n,\qquad g(w)=\sum\limits_{n=0}^{\infty}w^nb_n
\end{equation*}
be their power series expansions. The regular product  of $f$ and $g$ given in Definition (\ref{R-product}) is coherent with  the one given by
$$f\ast g(w):=\sum\limits_{n=0}^{\infty}w^n\bigg(\sum\limits_{k=0}^n a_kb_{n-k}\bigg).$$
This follows from the identity principle and the fact that these two products defined in  these two ways are exactly the usual pointwise product  when they are restricted to $B(0,R)\cap\mathbb R$, as one can patiently verify.
\end{remark}

\begin{remark}\label{remark on R-product02}
When $\Omega\subseteq \mathbb O$ is a symmetric slice domain, the same reason as in the preceding remark also shows that the product defined in \cite[Definition 9]{Ghiloni1} coincides with the one in Definition \ref{R-product} provided all the considered functions are  regular.
\end{remark}

\begin{remark}\label{remark on R-product03}
Notice that the regular product is obviously distributive, but in general non-associative and non-commutative, since the underlying algebra $\mathbb O$ is non-associative and non-commutative. However, it is associative in some special cases. For instance, let $f$ and $g$ be two regular functions on  a symmetric slice domain $\Omega\subseteq \mathbb O$ and satisfy the additional condition that $f(\Omega_I)\subseteq\mathbb C_I$ and $g(\Omega_I)\subseteq\mathbb C_I$ for some $I\in\mathbb S$, then it follows from Artin's theorem for alternative algebras, Remark \ref{remark on R-product01} and the identity principle that
$$f\ast g=g\ast f$$
and
$$(f\ast g)\ast h=f\ast(g\ast h)$$
for every regular function $h$ on $\Omega$.
\end{remark}

Again let $\Omega\subseteq \mathbb O$ be a symmetric slice domain and $I, J, K\in\mathbb S$ be such that  $I, J, IJ, K$ are mutually perpendicular with respect to the standard Euclidean inner product on $\mathbb O$. Let $f$ be a regular function on   $\Omega\subseteq \mathbb O$. Then the splitting lemma (Lemma \ref{eq:Splitting}) guarantees the existence of four holomorphic functions $F_k: \Omega_I\rightarrow \mathbb C_I$, $k=1,2,3,4$ such that for all $z\in\Omega_I$,
\begin{equation*}\label{splitting of f}
f_I(z)=F_1(z)+F_2(z)J+\big(F_3(z)+\overline{F_4(z)}J\big)K.
\end{equation*}
We define two functions $f^c_I, f^s_I:\Omega_I\rightarrow \mathbb O$ as
\begin{equation}\label{con-def}
 f^c_I(z):=\overline{F_1(\bar{z})}-F_2(z)J-\big(F_3(z)+\overline{F_4(z)}J\big)K,
\end{equation}
and
\begin{equation}\label{sym-def}
f^s_I(z):= f_I\ast f^c_I(z)
=\sum_{k=1}^{4}F_k(z)\overline{F_k(\bar{z})}= f^c_I\ast f_I(z).
\end{equation}
Here the second equality in (\ref{sym-def}) follows from (\ref{def-regular product}).
Then both $f^c_I(z)$ and $f^s_I(z)$ are obviously holomorphic satisfying $f^c_I(z)=\overline{f(z)}$ and $f^s_I(z)=|f(z)|^2$ (independent of the choice of $I\in\mathbb S$) for all $z\in\Omega\cap\mathbb R$. Therefore, they  admit respectively a \textit{unique} regular extension to $\Omega$, independent of the choice of $I\in\mathbb S$. We denote them  by ${\rm{ext}}(f^c_I)$ and ${\rm{ext}}(f^s_I)$, respectively.

\begin{definition}\label{con-sym-def}
Let $f$  be a regular function on  a symmetric slice domain $\Omega\subseteq \mathbb O$. Then the regular function
$$f^c(w):={\rm{ext}}(f^c_I)(w)$$
defined as the extension of (\ref{con-def}) is called the \textit{regular conjugate} of $f$, and the regular function
$$f^s(w):={\rm{ext}}(f^s_I)(w)=f\ast f^c(w)=f^c\ast f(w)$$
is called the \textit{symmetrization} of $f$.
\end{definition}

\begin{remark}\label{remark on con-sym01}
In the special case that $\Omega=B(0,R)$, there is an equivalent way of defining the regular conjugate and the symmetrization of regular functions, which goes as follows.
Let $f:B(0,R)\rightarrow \mathbb O$ be a regular function with the  power series expansion
\begin{equation*}\label{equi-def}
f(w)=\sum\limits_{n=0}^{\infty}w^na_n.
\end{equation*}
Then the regular conjugate and the symmetrization of $f$ are respectively given by
$$f^c(w)=\sum\limits_{n=0}^{\infty}w^n\overline{a}_n,$$
and
$$f^s(w)=f\ast f^c(w)=f^c\ast f(w)=\sum\limits_{n=0}^{\infty}w^n\bigg(\sum\limits_{k=0}^n a_k\overline{a}_{n-k}\bigg).$$
One can easily verify that these two definitions are the same as those in Definition \ref{con-sym-def}.
\end{remark}

\begin{remark}\label{remark on con-sym02}
From (\ref{sym-def}) one immediately deduces that the  symmetrization $f^s$ of every regular function $f$ on a symmetric slice domain  $\Omega\subseteq \mathbb O$ is slice preserving, i.e. $f^s(\Omega_I)\subseteq \mathbb C_I$ for every $I\in\mathbb S$.
\end{remark}

Both the regular conjugate and  the symmetrization are well-behaved with respect to the  regular product.

\begin{proposition}\label{beh-con-sym}
Let $f$ and $g$ be two regular functions on  a symmetric slice domain $\Omega\subseteq \mathbb O$. Then $(f^c)^c=f$, $(f\ast g)^c=g^c\ast f^c$ and
$$(f\ast g)^s=f^sg^s=g^sf^s.$$
\end{proposition}

\begin{proof}
The results follow from the very definition and a direct verification.
\end{proof}

Now we can use the notion of regular conjugate and symmetrization introduced above to define the regular reciprocal of a regular function:

\begin{definition}\label{def-regular reciprocal}
Let $f$  be a non-identically vanishing  regular function on  a symmetric slice domain $\Omega\subseteq \mathbb O$ and $\mathcal{Z}_{f^s}$   the set of zeros of its symmetrization $f^s$. We define the \textit{regular reciprocal} of $f$ as the function $f^{-\ast}:\Omega\setminus \mathcal{Z}_{f^s}\rightarrow \mathbb O$ given by
\begin{equation}\label{exp-regular reciprocal}
f^{-\ast}(w):=f^s(w)^{-1}f^c(w).
\end{equation}
\end{definition}

The function $f^{-\ast}$ defined in (\ref{exp-regular reciprocal}) deserves the name of regular reciprocal of $f$ due to the following:

\begin{proposition}
Let $f$  be a  non-identically vanishing regular function on  a symmetric slice domain $\Omega\subseteq \mathbb O$ and $\mathcal{Z}_{f^s}$   the set of zeros of its symmetrization $f^s$. Then
$$f^{-\ast}\ast f=f\ast f^{-\ast}=1$$
on $\Omega\setminus \mathcal{Z}_{f^s}$.
\end{proposition}

We conclude this section with the following simple proposition.
\begin{proposition}
Let $f$ and $g$  be two non-identically vanishing regular function on  a symmetric slice domain $\Omega\subseteq \mathbb O$. Then
$$(f\ast g)^{-\ast}=g^{-\ast}\ast f^{-\ast}$$
on $\Omega\setminus (\mathcal{Z}_{f^s}\cup \mathcal{Z}_{g^s})$.
\end{proposition}

\begin{proof}
The result follows from Remarks \ref{remark on R-product03} and \ref{remark on con-sym02}, together with Proposition \ref{beh-con-sym}:
$$(f\ast g)^{-\ast}=(f^sg^s)^{-1}(g^c\ast f^c)
=\big((g^s)^{-1}g^c\big)\ast\big((f^s)^{-1}f^c\big)=g^{-\ast}\ast f^{-\ast}.$$
\end{proof}

\section{Proof of Theorem \ref{BSL}}
In this section, we shall give a proof of Theorem \ref{BSL}. Before presenting the details, we need some auxiliary results.

Let $\Omega\subseteq\mathbb O$ be a symmetric slice domain. For each regular function $f:\Omega\rightarrow \mathbb O$ and each $\xi\in\Omega$, an argument similar to the one in the proof \cite[Proposition 3.17]{GSS}, together with the splitting lemma (Lemma \ref{eq:Splitting}) and Artin's theorem for alternative algebras guarantees  the existence of a unique regular function on $\Omega$, denoted by $R_{\xi}f$, such that
\begin{equation}\label{de:Rf1}
f(w)-f(\xi)=(w-\xi)\ast R_{\xi}f(w), \qquad \forall \, w\in\Omega.
\end{equation}
Applying the same procedure to $R_{\xi}f$ at the point $\overline{\xi}$ yields
\begin{equation}\label{de:Rf2}
\begin{split}
f(w)=&f(\xi)+(w-\xi)\ast \Big(R_{\xi}f(\bar{\xi})+(w-\bar{\xi}\,)\ast R_{\overline{\xi}}R_{\xi}f(w)\Big) \\
=&f(\xi)+(w-\xi)R_{\xi}f(\bar{\xi})+\Delta_{\xi}(w)R_{\overline{\xi}}R_{\xi}f(w), \qquad \forall \, w\in\Omega,
\end{split}
\end{equation}
where $\Delta_{\xi}(w):=(w-\xi)\ast(w-\bar{\xi}\,)=w^2+2w{\rm{Re}}(\xi)+|\xi|^2$ is called the \textit{characteristic polynomial} of $\xi$ or the \textit{symmetrization} of $w-\xi$, and the second equality in (\ref{de:Rf2}) follows from Remark \ref{remark on R-product03}.

From the very definition and (\ref{de:Rf1}), one can see that $R_{\xi}f(\bar{\xi})$ is exactly the \textit{sphere derivative} $\partial_sf(\xi)$  of $f$ at the point $\xi$:
$$\partial_sf(\xi):=\big(2Im(\xi)\big)^{-1}\big(f(\xi)-f(\overline{\xi})\big)=R_{\xi}f(\bar{\xi}).$$
In addition, for every $v\in\partial \mathbb B$ and every $t\in\mathbb R$ small enough, replacing $w$ by $\xi+tv$ in (\ref{de:Rf2}) yields
$$f(\xi+tv)-f(\xi)=tv\partial_sf(\xi)+t\big(tv^2+(\xi v-v\bar{\xi}\,)\big)R_{\overline{\xi}}R_{\xi}f(\xi+tv),$$
from which the following lemma immediately follows.
\begin{lemma}\label{D-derivative}
Let $f$ be a regular function on a  symmetric slice domain $\Omega\subseteq \mathbb O$ and $\xi\in\Omega$. Then for every $v\in\partial \mathbb B$, the directional derivative of $f$ along $v$ is given by
\begin{equation}\label{eq:D-derivative}
\frac{\partial f}{\partial v}(\xi):=\lim\limits_{\mathbb R\ni t\rightarrow 0}\frac{f(\xi+tv)-f(\xi)}{t}=v\partial_sf(\xi)+(\xi v-v\overline{\xi}\,)R_{\overline{\xi}}R_{\xi}f(\xi).
\end{equation}
In particular, it holds that
\begin{equation}\label{derivative-relation}
f'(\xi)=\partial_sf(\xi)+2{\rm{Im}}(\xi)R_{\overline{\xi}}R_{\xi}f(\xi).
\end{equation}
\end{lemma}

For each $\alpha\in\partial \mathbb B$, we consider two multipliers $\mathcal{L}_{\alpha}$ and $\mathcal{R}_{\alpha}$ on the octonionic space $\big(\mathbb O, \langle \,,\,\rangle\big)$ associated with $\alpha$, induced respectively by left and right multiplications, i.e.
$$\mathcal{L}_{\alpha}(w)=\alpha w,\qquad \mathcal{R}_{\alpha}(w)= w\alpha, \qquad \forall\, w\in\mathbb O.$$
Here the inner product $\langle \,\,,\,\rangle$ is same as the one in $(\ref{inner product on O})$.
Clearly, $\mathcal{L}_{\alpha}$ and $\mathcal{R}_{\alpha}$ are two $\mathbb R$-linear bijections with inverses $\mathcal{L}_{\alpha^{-1}}$ and $\mathcal{R}_{\alpha^{-1}}$, respectively. Moreover, they are two unitary operators on $\big(\mathbb O, \langle \,,\,\rangle\big)$ in virtue of equality $(\ref{inner and norm on O})$. Therefore, we have the following simple lemma.

\begin{lemma}\label{unitary multipliers}
For each $\alpha\in\partial \mathbb B$, $\mathcal{L}_{\alpha}$ and $\mathcal{R}_{\alpha}$ are two unitary operators on the octonionic space $\big(\mathbb O, \langle \,,\,\rangle\big)$.
\end{lemma}

Also, the following lemma is needed in the proof of Theorem \ref{BSL}.

\begin{lemma}\label{lem: moudulus inequality}
Let $f$ be a   regular self-mapping of the open unit ball $\mathbb B$. Then
\begin{equation}\label{moudulus inequality}
\frac{1-|f(w)|^2}{1-|w|^2}\geq\frac{\big|1-\big\langle f(0), f(w)\big\rangle\big|^2}{1-|f(0)|^2},\qquad \forall\, w\in\mathbb O.
\end{equation}
\end{lemma}

\begin{proof}
For every $w\in\mathbb O$, let $I\in\mathbb S$ be such that $w\in \mathbb B_I$. Then by the splitting lemma (Lemma \ref{eq:Splitting}), we can find $J$ and $K$ in $\mathbb S$, such that $I, J, IJ, K$ are mutually perpendicular with respect to the standard Euclidean inner product on $\mathbb O$ and there are  four holomorphic functions
$F_k: \mathbb B_I\rightarrow\mathbb B_I$, $k=1,2,3,4$,  such that
\begin{equation}\label{splitting}
f_I(z)=F_1(z)+F_2(z)J+\big(F_3(z)+\overline{F_4(z)}J\big)K, \qquad \forall\, z\in\mathbb B_I.
\end{equation}
Let  $B^4\subset\mathbb C_I^4$ be the open unit ball. We consider the holomorphic mapping $F:\mathbb B_I\rightarrow B^4$ given by
$$F(z):=\big(F_1(z), F_2(z), F_3(z), F_4(z)\big), \qquad \forall\, z\in\mathbb B_I,$$
in virtue of the fact that
$$|F(z)|^2=\sum\limits_{k=1}^4|F_k(z)|^2=|f(z)|^2<1$$
for all $z\in\mathbb B_I$. Now it follows from the classical Schwarz-Pick lemma (see e.g. \cite[Theorem 8.1.4]{Rudin}) that
\begin{equation}\label{norm-Schwarz}
 \frac{\big|1-\big\langle F(0), F(z)\big\rangle_{\mathbb C_{I}^4}\big|^2}{\big(1-|F(0)|^2\big)\big(1-|F(z)|^2\big)}\leq\frac{1}{1-|z|^2},
 \qquad \forall\, z\in\mathbb B_{I},
\end{equation}
which $\langle \,\, ,\,\rangle_{\mathbb C_{I}^4}$ denotes the standard Hermitian inner product on $\mathbb C_{I}^4$, i.e.
$$\langle \alpha, \beta\rangle_{\mathbb C_{I}^4}=\sum_{k=1}^4\alpha_k \overline{\beta}_k, \qquad \forall\, \alpha, \beta\in \mathbb C_{I}^4.$$
Once notice that $\big\langle f(0), f(w)\big\rangle={\rm{Re}}\big(\langle F(0), F(z)\rangle_{\mathbb C_{I}^4}\big)\in\mathbb R$, inequality  (\ref{moudulus inequality}) immediately follows from (\ref{norm-Schwarz}).
\end{proof}

Now we come to prove Theorem $\ref{BSL}$.

\begin{proof}[Proof of Theorem $\ref{BSL}$]
The argument is essentially the same as the one in the proof of \cite[Theorem 2.4]{WR}.
First, it follows from inequality  (\ref{moudulus inequality}) that the directional derivative of $|f|^2$ along $\xi$ at the boundary point $\xi\in\partial\mathbb B$ satisfies that
\begin{equation}\label{radial der}
\frac{\partial |f|^2}{\partial \xi}(\xi)\geq 2\frac{\big|1-\big\langle f(0), f(\xi)\big\rangle\big|^2}{1-|f(0)|^2}.
\end{equation}
However,
\begin{equation}\label{tangent der}
\frac{\partial |f|^2}{\partial \tau}(\xi)=0, \qquad \forall\, \,\tau\in T_{\xi}(\partial \mathbb B)\cong \mathbb R^7.
\end{equation}
Indeed, for any unit tangent vector $\tau\in T_{\xi}(\partial \mathbb B)$, take a smooth curve
$\gamma:(-1,1)\rightarrow\overline{\mathbb B}$ such that
$$\gamma(0)=\xi,\quad \gamma'(0)=\tau.$$
By definition we have
$$\frac{\partial |f|^2}{\partial \tau}(\xi)=\left.\bigg(\frac{d}{dt}\big|f(\gamma(t))\big|^2\bigg)\right|_{t=0}=0,$$
since the function $|f(\gamma(t))\big|^2$ in $t$ attains its maximum at the point $t=0$.

In view of Lemma \ref{D-derivative}, we have
$$\frac{\partial f}{\partial v}(\xi)=v\partial_sf(\xi)+(\xi v-v\overline{\xi}\,)R_{\overline{\xi}}R_{\xi}f(\xi),\qquad \forall\,v\in\partial \mathbb B,$$
from which and Lemma \ref{unitary multipliers} it follows that
\begin{equation}\label{der-relation1}
\begin{split}
\frac{\partial |f|^2}{\partial v}(\xi)
&=2\Big\langle\frac{\partial f}{\partial v}(\xi),f(\xi)\Big\rangle\\
&=2\Big\langle v\partial_sf(\xi)+(\xi v-v\overline{\xi}\,)R_{\bar{\xi}}R_{\xi}f(\xi),f(\xi)\Big\rangle\\
&=2\Big\langle v,f(\xi)\overline{\partial_sf(\xi)}+
\bar{\xi}\Big(f(\xi)\overline{R_{\bar{\xi}}R_{\xi}f(\xi)}\Big)-
\Big(f(\xi)\overline{R_{\bar{\xi}}R_{\xi}f(\xi)}\Big)\xi\Big\rangle\\
&=:2\big(A+B\big),
\end{split}
\end{equation}
where
\begin{equation}\label{exp-A}
\begin{split}
A&=\Big\langle v,\,f(\xi)\Big(\overline{\partial_sf(\xi)
 -\bar{\xi}R_{\bar{\xi}}R_{\xi}f(\xi)}\Big)\Big\rangle \\
&=\Big\langle v,\,f(\xi)\Big(\overline{f'(\xi)-\xi R_{\bar{\xi}}R_{\xi}f(\xi)}\Big)\Big\rangle,
\end{split}
\end{equation}
and
\begin{equation}\label{exp-B}
B=\Big\langle v,\,\bar{\xi}\Big(f(\xi)\overline{R_{\bar{\xi}}R_{\xi}f(\xi)}\Big)-\big[f(\xi), \overline{R_{\bar{\xi}}R_{\xi}f(\xi)}, \xi\big]\Big\rangle.
\end{equation}
The second equality in (\ref{exp-A}) follows from  equality (\ref{derivative-relation}).
Substituting the following simple equalities
\begin{equation*}
\begin{split}
f(\xi)\Big(\overline{R_{\bar{\xi}}R_{\xi}f(\xi)}\bar{\xi}\Big)
&=\Big(f(\xi)\overline{R_{\bar{\xi}}R_{\xi}f(\xi)}\Big)
\bar{\xi}-\big[f(\xi), \overline{R_{\bar{\xi}}R_{\xi}f(\xi)},\bar{\xi}\,\big]\\
&=\Big(f(\xi)\overline{R_{\bar{\xi}}R_{\xi}f(\xi)}\Big)
\bar{\xi}+\big[f(\xi), \overline{R_{\bar{\xi}}R_{\xi}f(\xi)},\xi\,\big]
\end{split}
\end{equation*}
into  the second equality in (\ref{exp-A}) yields
\begin{equation}\label{exp-A01}
\begin{split}
A&=\Big\langle v,\,f(\xi)\overline{f'(\xi)}-\Big(f(\xi)\overline{ R_{\bar{\xi}}R_{\xi}f(\xi)}\Big)\bar{\xi}-\big[f(\xi), \overline{R_{\bar{\xi}}R_{\xi}f(\xi)},\xi\,\big]\Big\rangle.
\end{split}
\end{equation}
Substituting (\ref{exp-B}) and (\ref{exp-A01}) into (\ref{der-relation1}) gives
\begin{equation}\label{der-relation2}
\begin{split}
\frac{\partial |f|^2}{\partial v}(\xi)
&=2\Big\langle v,f(\xi)\overline{f'(\xi)}+\big[\bar{\xi}, f(\xi)\overline{R_{\bar{\xi}}R_{\xi}f(\xi)}\,\big]-2\big[f(\xi), \overline{R_{\bar{\xi}}R_{\xi}f(\xi)}, \xi\big]\Big\rangle\\
&=2\Big\langle v,f(\xi)\overline{f'(\xi)}+\big[\bar{\xi}, f(\xi)\overline{R_{\bar{\xi}}R_{\xi}f(\xi)}\,\big]+2\big[\xi, f(\xi), R_{\bar{\xi}}R_{\xi}f(\xi)\big]
\Big\rangle.
\end{split}
\end{equation}
Now it follows from $(\ref{tangent der})$ and $(\ref{der-relation2})$ that
$$f(\xi)\overline{f'(\xi)}+\big[\bar{\xi}, f(\xi)\overline{R_{\bar{\xi}}R_{\xi}f(\xi)}\,\big]+2\big[\xi, f(\xi), R_{\bar{\xi}}R_{\xi}f(\xi)\big]\perp
 T_{\xi}(\partial \mathbb B)$$
so that in view of $(\ref{radial der})$ and $(\ref{der-relation2})$,
\begin{equation*}
\begin{split}
\frac{\partial |f|}{\partial \xi}(\xi)
&=\overline{\xi}\Big(f(\xi)\overline{f'(\xi)}+\big[\bar{\xi}, f(\xi)\overline{R_{\bar{\xi}}R_{\xi}f(\xi)}\,\big]+2\big[\xi, f(\xi), R_{\bar{\xi}}R_{\xi}f(\xi)\big]\Big)
\\
&\geq \frac{\big|1-\big\langle f(0), f(\xi)\big\rangle\big|^2}{1-|f(0)|^2},
\end{split}
\end{equation*}
which completes the proof of $(\ref{Schwarz ineq1})$.

To prove $(\ref{Schwarz ineq2})$, notice first that the first equality in $(\ref{Schwarz ineq2})$ directly follows from  $(\ref{Schwarz ineq1})$. It remains to prove the following inequality
$$\frac{\partial |f|}{\partial \xi}(\xi)\geq\frac{2}{1+{\rm{Re}}f'(0)}.$$
To this end, let $I\in\mathbb S$ be such that $\xi\in \partial\mathbb B\cap \mathbb C_I$. Then by the splitting lemma (Lemma \ref{eq:Splitting}), we can find $J$ and $K$ in $\mathbb S$, such that $I, J, IJ, K$ are mutually perpendicular and if $\mathcal{H}$ is the subspace of $\mathbb O$ generated by $\{1, I, J, IJ\}$, then there are two regular functions
$F: \mathbb B\cap\mathcal{H}\rightarrow\mathbb B\cap\mathcal{H}$ and $G: \mathbb B\cap\mathcal{H}\rightarrow\mathbb B\cap\mathcal{H}K$  such that
$$f(w)=F(w)+G(w),\qquad \forall\, w\in\mathbb B\cap\mathcal{H}.$$
 Then for any $w\in\mathbb B\cap\mathcal{H}$, we have
\begin{equation}\label{squared-norm}
|f(w)|^2=|F(w)|^2+|G(w)|^2
\end{equation}
and
$$f'(w)=F'(w)+G'(w).$$
Moreover,
$$F(\xi)=\xi, \qquad G(\xi)=0, \qquad {\rm{Re}}\,f'(0)={\rm{Re}}\,F'(0).$$
Now it follows from \cite[Corollary 2.6]{WR} that
$$\frac{\partial |f|}{\partial \eta}(\eta)
=\frac{\partial |F|}{\partial \eta}(\eta)
=F'(\xi)-\big[\xi,R_{\bar{\xi}}R_{\xi}F(\xi)\big]
\geq\frac{2}{1+{\rm{Re}}F'(0)}=\frac{2}{1+{\rm{Re}}f'(0)}.$$
If equality holds for inequality in (\ref{Schwarz ineq2}), then it again follows from \cite[Corollary 2.6]{WR} that
\begin{equation}\label{F-expression}
F(w)=w\big(1-wa\bar{\xi}\,\big)^{-\ast}\ast\big(w-a\xi\big)\bar{\xi}
\qquad \forall\, w\in\mathbb B\cap\mathcal{H},
\end{equation}
for some constant $a\in [-1,1)$. Furthermore, it follows from equality in (\ref{squared-norm}) that
$$|G(w)|^2=|f(w)|^2-|F(w)|^2\leq 1-|F(w)|^2, \qquad \forall\, w\in\mathbb B\cap\mathcal{H},$$
which together with (\ref{F-expression}) implies that $G\equiv0$, in virtue of the maximum principle (Theorem \ref{MP} below),
and hence
$$f(w)={\rm{ext}}\,F(w)=w\big(1-wa\bar{\xi}\,\big)^{-\ast}\ast\big(w-a\xi\big)\bar{\xi},\qquad \forall\, w\in\mathbb B.$$
Therefore, the equality in inequality (\ref{Schwarz ineq2}) can hold only for regular self-mappings  of  the form (\ref{f-expression}), and a direct calculation shows that it does indeed hold for all such regular self-mappings. Now  the proof is complete.
\end{proof}

\begin{remark}\label{Non-real}
It is worth remarking here that, as in the quaternionic setting, the Lie bracket
in (\ref{Schwarz ineq2}) does not vanish
and $f'(\xi)$ may not be a real number, in general; An explicit counterexample can be found in \cite[Example 4.5]{WR}.
\end{remark}

The regular functions of the form (\ref{f-expression}) are indeed self-mappings of the open unit ball $\mathbb B\subset\mathbb O$, due to the following result:

\begin{proposition}\label{convex combination identity}
Let $f$ be a regular function on a symmetric slice domain  $\Omega\subseteq \mathbb O$ such that $f(\Omega_I)\subseteq \mathbb C_I $ for some $I\in \mathbb S $.  Then
the convex combination identity
\begin{equation}\label{convex combination identity01}
\big|f(x+yJ)\big|^2 =\frac{1+\langle I,J\rangle}{2}\big|f(x+yI)\big|^2+
\frac{1-\langle I,J\rangle}{2}\big|f(x-yI)\big|^2
\end{equation}
holds for every $ x+yJ \in \Omega$.
\end{proposition}

\begin{proof}
The idea is essentially the same as in \cite{RW2}. Let $I\in \mathbb S$ be   as described in  the proposition. First, it is easy to verify that for every $J\in \mathbb S$, the set $\big\{1, I, I\wedge J, I(I\wedge J)\big\}$ is an orthogonal set of $\mathbb O\simeq\mathbb R^8$. By the representation formula for regular functions (cf. \cite[Proposition 6]{Ghiloni1}),
\begin{equation}\label{repesentaion}
f(w)=\frac{1}{2}\big(f(z)+f(\bar z)\big)-\frac{1}{2}J\Big(I\big(f(z)-f(\bar z)\big)\Big)
\end{equation}
for every  $ w=x+yJ \in \Omega$ with $z=x+yI$ and $\bar z=x-yI$.

Taking modulus on both sides of (\ref{repesentaion}) and applying Lemma \ref{unitary multipliers} to obtain
\begin{equation}\label{norm-repesentaion}
\begin{split}
|f(w)|^2
=&\frac14\Big(\big|f(z)+f(\bar z)\big|^2+\big|f(z)-f(\bar z)\big|^2\Big)\\
&-\frac12\Big\langle f(z)+f(\bar z), \, J\Big(I\big(f(z)-f(\bar z)\big)\Big)\Big\rangle\\
=&\frac12\big(|f(z)|^2+|f(\bar z)|^2\big)+
\frac12\Big\langle \big(f(z)+f(\bar z)\big)\Big(\big(\,\overline{f(z)}-\overline{f(\bar z)}\,\big)I\Big), \,J\Big\rangle\\
=&: \frac12\big(|f(z)|^2+|f(\bar z)|^2\big)+\frac{1}{2}A.
\end{split}
\end{equation}
By assumption,  $f(\Omega_I)\subseteq \mathbb C_I $. This together with   Artin's theorem for alternative algebras and Lemma \ref{unitary multipliers} implies
\begin{equation}\label{B-computation01}
\begin{split}
A&=\Big\langle \big(f(z)+f(\bar z)\big)\big(\,\overline{f(z)}-\overline{f(\bar z)}\,\big)I, \,J\Big\rangle\\
&=-\Big\langle \big(f(z)+f(\bar z)\big)\big(\,\overline{f(z)}-\overline{f(\bar z)}\,\big), \,J I\Big\rangle.\\
\end{split}
\end{equation}
Recalling equality in (\ref{relation-inner-wedge}), an orthogonality consideration gives
\begin{equation}\label{B-computation02}
\begin{split}
A&=\langle I, J\rangle\Big\langle \big(f(z)+f(\bar z)\big)\big(\,\overline{f(z)}-\overline{f(\bar z)}\,\big), \,1\Big\rangle\\
&=\langle I, J\rangle\Big\langle f(z)+f(\bar z), \,\overline{f(z)}-\overline{f(\bar z)}\Big\rangle\\
&=\langle I, J\rangle \big(|f(z)|^2-|f(\bar z)|^2\big).
\end{split}
\end{equation}
Now the desired equality (\ref{convex combination identity01}) immediately follows by substituting  (\ref{B-computation02}) into (\ref{norm-repesentaion}). The proof is complete.
 \end{proof}

\section{Proofs of Theorems \ref{regular diam-LT}, \ref{slice diam-LT} and \ref{Poukka}}
We begin with a definition of regular diameter, which is intimately related to a new regular composition (cf. \cite{RW}).

\begin{definition}\label{regular-composition}
Let $u\in \mathbb O$ and $f:\mathbb B\rightarrow \mathbb O$   a regular function with  Taylor expansion
$$f(w)=\sum\limits_{n=0}^{\infty}w^na_n.$$
We define the regular composition of $f$ with the regular function $w\mapsto wu$  to be
$$f_u(w):=\sum\limits_{n=0}^{\infty}(wu)^{\ast n}\ast a_n=\sum\limits_{n=0}^{\infty}w^n(u^na_n).$$
\end{definition}
If $|u|=1$, the radius of convergence of the series expansion for $f_u$ is the same as that for $f$. Moreover, if $u$ and $w_0$ lie in the same plane $\mathbb C_I$, then $u$ and $w_0$ commute, and hence $f_u(w_0)=f(uw_0)$. In particular,  if $u\in \mathbb R$, then $f_u(w)=f(uw)$ for every $w\in\mathbb B$.

\begin{definition}\label{regular-diameter}
Let $f$ be a regular function on $\mathbb B$ with   Taylor expansion
$$f(w)=\sum\limits_{n=0}^{\infty}w^na_n.$$ For each $r\in(0, 1)$, the \textit{regular diameter} of the image of $r\mathbb B$ under $f$ is defined to be
\begin{equation}\label{regular-diameter01}
\widetilde{d}\big(f(r\mathbb B)\big):=\max_{u,v\in \overline{\mathbb B}}\max_{|w|\leq r}|f_u(w)-f_v(w)|.
\end{equation}
The \textit{regular diameter} of the image of $\mathbb B$ under $f$ is defined to be
\begin{equation}\label{regular-diameter02}
\widetilde{d}\big(f(\mathbb B)\big):=\lim_{r\rightarrow 1^-}\widetilde{d}\big(f(r\mathbb B)\big).
\end{equation}
\end{definition}

Clearly, $\widetilde{d}\big(f(r\mathbb B)\big)$ is an increasing function of $r\in(0,1)$; hence the limit in (\ref{regular-diameter02}) always exists. Therefore, $\widetilde{d}\big(f(\mathbb B)\big)$ is well-defined. Moreover, in view of the following maximum principle for slice regular functions, $\widetilde{d}\big(f(r\mathbb B)\big)/2r$ is an increasing function of $r\in(0,1)$ as well (see (\ref{diameter-quo01}) below).

\begin{theorem}[The maximum principle]\label{MP}
Let $f:\Omega\rightarrow \mathbb O$ be a regular function on a slice domain $\Omega\subseteq\mathbb O$. If there exist a $I\in\mathbb S$ such that the restriction $|f_I|$  of $|f|$ to $\Omega_I$ attains a local maximum at some point $w_0\in \Omega_I$, then $f$ is constant.
\end{theorem}

\begin{proof}
 We can split $f_{I}$ as $$f(z)=F_1(z)+F_2(z)J+\big(F_3(z)+\overline{F_4(z)}J\big)K, \qquad \forall\, z\in\Omega_I,$$
where $F_k: \Omega_I\rightarrow \mathbb C_I$, $k=1,2,3,4$, are four holomorphic functions, and $I, J, K$ enjoy the same property as   in the proof of Lemma \ref{lem: moudulus inequality}. Then the holomorphic mapping $F:\Omega_{I}\rightarrow \mathbb C_{I}^4$ given by
$$F(z):=\big(F_1(z), F_2(z), F_3(z), F_4(z)\big)$$
 satisfies that
$$|F(z)|^2=\sum\limits_{k=1}^4|F_k(z)|^2=|f(z)|^2$$
for all $z\in\Omega_{I}$. By assumption, $|F|$ attains a local maximum at the point $w_0\in \Omega_{I}$. Thus from the maximum principle for holomorphic mappings (cf. \cite[Theorem 2.8.3]{Klimek}) it immediately follows that $F$ is constant on $\Omega_{I}$, and $f$ is constant there as well, and in turn on $\Omega$ by the identity principle.
\end{proof}

We also have the following results.

\begin{proposition}\label{regular-diameter-relation}
Let $f$ be a regular function on $\mathbb B$. Then
\begin{equation}\label{regular-diameter-relation01}
 {\rm{diam}}\, f(\mathbb B)\leq \widetilde{d}\big(f(\mathbb B)\big)\leq 2\,{\rm{diam}}\, f(\mathbb B).
\end{equation}
\end{proposition}

\begin{lemma}\label{Constant lemma}
Let  $g$ be a regular function on $\mathbb B$ such that for each $w\in\mathbb B\setminus\{0\}$,
$$\big\langle I_w, g(w)\big\rangle=0, $$
where $I_{w}={\rm{Im}}\,w/|{\rm{Im}}\,w|$ is the pure imaginary unit identified by $w$. Then $g$ is a real constant function.
\end{lemma}

The proofs of Proposition \ref{regular-diameter-relation}  and Lemma \ref{Constant lemma} are completely the same as those of \cite[Propositions 3.8 and 3.4]{Gen-Sar}, and so we omit them. We next prove the following simple lemma, which will be needed in the proof of Theorem \ref{regular diam-LT}.

\begin{lemma}\label{associator}
For any three octonions $u, v, w\in \mathbb O$, it holds that
\begin{equation}\label{associator01}
 \big\langle u,\, [u, v, w]\big\rangle=0.
\end{equation}
\end{lemma}

\begin{proof}
First, we prove that
$$ \big\langle I,\, [I, v, w]\big\rangle=0$$
for every $I\in\mathbb S$, $v, w\in \mathbb O$. Indeed,
\begin{equation*}
\begin{split}
\big\langle I,\, [I, v, w]\big\rangle
&=\big\langle I,\, (Iv)w\big\rangle-\big\langle I,\,  I(vw)\big\rangle
\\
&=-\big\langle 1, \big((Iv)w\big)I \big\rangle
-\big\langle 1, vw\big\rangle \ \ \,\quad \qquad \qquad \mbox{by Lemma \ref{unitary multipliers}}
\\
&=-\big\langle 1, [Iv, w, I]+ (Iv)(w I) \big\rangle
-\big\langle 1, vw\big\rangle
\\
&=-\big\langle 1, (Iv)(wI) \big\rangle
-\big\langle 1, vw\big\rangle \ \ \   \quad \qquad \qquad \mbox{by (\ref{Real part free})}
\\
&=-\big\langle 1, I(vw) I \big\rangle
-\big\langle 1, vw\big\rangle \ \ \, \qquad \qquad \qquad \mbox{by (\ref{Monfang})}
\\
&=\big\langle 1, vw\big\rangle
-\big\langle 1, vw\big\rangle \ \,\quad\qquad\qquad \qquad \qquad \mbox{by Lemma \ref{unitary multipliers}}
\\
&=0.
\end{split}
\end{equation*}
For each $u\in\mathbb O$. We write $u=x+yI$, where $x, y\in\mathbb R$ and $I\in\mathbb S$, then
\begin{equation*}
\begin{split}
\big\langle u,\, [u, v, w]\big\rangle
&= \big\langle u,\, [yI, v, w]\big\rangle
\\
&= y\big\langle x+yI, \, [yI, v, w]\big\rangle
\\
&= y^2\big\langle I,\, [I, v, w]\big\rangle
\\
&=0,
\end{split}
\end{equation*}
which completes the proof.
\end{proof}

Now we are in a position to prove Theorem \ref{regular diam-LT}.

\begin{proof}[Proof of Theorem $\ref{regular diam-LT}$]
The proof is partly the same as that of \cite[Theorem 3.9]{Gen-Sar}, the main difference being that we use some extra technical treatments together with Theorem \ref{BSL}, instead of \cite[Proposition 3.2]{Gen-Sar}, which is not enough for our purpose because of the non-associativity of octonions.

We first prove inequality (\ref{regular-diam01}). To this end, we take $u, v\in\overline{\mathbb B}$ and consider the following auxiliary function
$$g_{u,v}(w)=\frac12 w^{-1}\big(f_u(w)-f_u(w)\big).$$
Then $g_{u,v}$ is regular on $\mathbb B$ with
\begin{equation}\label{g-value}
g_{u,v}(0)=\frac12 (u-v)f'(0).
\end{equation}
Applying the maximum principle (Theorem \ref{MP}) to the  regular function $g_{u,v}$
 yields that for each $r\in(0,1)$, we can write
$$\max_{|w|\leq r}|g_{u,v}(w)|=\max_{|w|\leq r}\frac{|f_u(w)-f_v(w)|}{2|w|}
=\frac1{2r}\max_{|w|\leq r} |f_u(w)-f_v(w)|,$$
which implies that
\begin{equation}\label{diameter-quo01}
\frac{\widetilde{d}\big(f(r\mathbb B)\big)}{2r}
=\frac1{2r}\max_{u,v\in \overline{\mathbb B}}\max_{|w|\leq r}|f_u(w)-f_v(w)|
=\max_{u,v\in \overline{\mathbb B}}\max_{|w|\leq r}|g_{u,v}(w)|.
\end{equation}
Therefore, $\widetilde{d}\big(f(r\mathbb B)\big)/2r$ is an increasing function of $r\in(0,1)$ and so always not more than
$$\lim_{r\rightarrow1^-}\frac{\widetilde{d}\big(f(r\mathbb B)\big)}{2r}
=\frac12 \widetilde{d}\big(f(\mathbb B)\big)=1.$$

This means that
\begin{equation}\label{regular-diam10}
\widetilde{d}\big(f(r\mathbb B)\big)\leq 2r
\end{equation}
for each $r\in(0, 1)$, proving inequality (\ref{regular-diam01}). To prove inequality (\ref{regular-diam02}), consider the odd part of $f$,
$$f_{odd}(w)=\frac12\big(f(w)-f(-w)\big),$$
which is regular on $\mathbb B$ satisfying both $f_{odd}(0)=0$ and
$$|f_{odd}(w)|=\frac12|f(w)-f(-w)|\leq\frac12 \widetilde{d}\big(f(\mathbb B)\big)=1$$
for all $w\in\mathbb B$. Thus it follows from the Schwarz  lemma that
\begin{equation}\label{modulus of der}
|f'(0)|=|f'_{odd}(0)|\leq 1.
\end{equation}

Now we come to prove the last assertion in the theorem. Obviously, if $f(w)=f(0)+wf'(0)$ with $|f'(0)|=1$,  equality holds both in (\ref{regular-diam01}) and (\ref{regular-diam02}). Conversely, suppose that equality holds in (\ref{regular-diam02}), i.e. $|f'(0)|=1$. Thus $|f'_{odd}(0)|=1$, and then again by the Schwarz lemma,
$$f_{odd}(w)=wf'(0).$$
We next claim that in this case $\widetilde{d}\big(f(r\mathbb B)\big)=2r$
for each $r\in(0, 1)$. Indeed, from (\ref{g-value}) and (\ref{diameter-quo01}) it follows that for each $r\in(0, 1)$,
$$\frac{\widetilde{d}\big(f(r\mathbb B)\big)}{2r}\geq
\max_{u,v\in \overline{\mathbb B}}|g_{u,v}(0)|=\frac12 \max_{u,v\in \overline{\mathbb B}}|u-v||f'(0)|=1,$$
which together with (\ref{regular-diam10}) implies that
$$\widetilde{d}\big(f(r\mathbb B)\big)=2r$$
for each $r\in(0, 1)$, as claimed.

 Take $\xi\in\mathbb B\setminus\{0\}$ with $0<|w|=:r<1$ and set
 \begin{equation}\label{h-definition}
   h_{\xi}(w)=\frac12 \Big(f(w)-f(-\xi)\Big).
 \end{equation}
 Thus $h_{\xi}$ is regular on $\mathbb B$ satisfying
 $$ h_{\xi}(\xi)=\frac12 \Big(f(\xi)-f(-\xi)\Big)=f_{odd}(\xi)=\xi f'(0).$$
 Moreover, from the very definition (\ref{h-definition}) and Proposition \ref{regular-diameter-relation} it follows that
 \begin{equation}\label{h-condition}
\max_{|w|\leq r}|h_{\xi}(w)|
=\frac12 \max_{|w|\leq r}\big |f(w)-f(-\xi)\big|
\leq\frac12 {\rm{diam}}\, f(r\mathbb B)
\leq\frac12 \widetilde{d}\big(f(r\mathbb B)\big)
=r
=|h_{\xi}(\xi)|.
\end{equation}
Therefore, the regular function $h_{\xi}$ satisfies all the assumptions given in Theorem \ref{BSL}, and hence
\begin{equation}
\begin{split}
\frac{\partial |h_{\xi}|}{\partial \xi}(\xi)
&=\overline{\xi}\Big(h_{\xi}(\xi)\overline{h_{\xi}'(\xi)}+\big[\bar{\xi}, h_{\xi}(\xi)\overline{R_{\bar{\xi}}R_{\xi}h_{\xi}(\xi)}\,\big]+2\big[\xi, h_{\xi}(\xi), R_{\bar{\xi}}R_{\xi}h_{\xi}(\xi)\big]\Big)
\\
&>0.
\end{split}
\end{equation}
In particular,
\begin{equation}\label{Key relation01}
\begin{split}
0&=\Big\langle I_{\xi},\, \overline{\xi}\Big(h_{\xi}(\xi)\overline{h_{\xi}'(\xi)}+\big[\bar{\xi}, h_{\xi}(\xi)\overline{R_{\bar{\xi}}R_{\xi}h_{\xi}(\xi)}\,\big]+2\big[\xi, h_{\xi}(\xi), R_{\bar{\xi}}R_{\xi}h_{\xi}(\xi)\big]\Big)\Big\rangle
\\
&=\Big\langle I_{\xi}\xi,\,  h_{\xi}(\xi)\overline{h_{\xi}'(\xi)}+\big[\bar{\xi}, h_{\xi}(\xi)\overline{R_{\bar{\xi}}R_{\xi}h_{\xi}(\xi)}\,\big]+2\big[\xi, h_{\xi}(\xi), R_{\bar{\xi}}R_{\xi}h_{\xi}(\xi)\big] \Big\rangle
\\
&=\Big\langle I_{\xi}\xi,\,  h_{\xi}(\xi)\overline{h_{\xi}'(\xi)}\Big\rangle.
\end{split}
\end{equation}
Here $I_{\xi}={\rm{Im}}\,\xi/|{\rm{Im}}\,\xi|$ is the pure imaginary unit identified by $\xi$, the second equality follows from  Lemma \ref{unitary multipliers}, and the last one follows from Lemma \ref{associator} and its proof.

Substituting the values of $h_{\xi}(\xi)$ and $h'_{\xi}(\xi)$ into the preceding inequalities yields that
\begin{equation}
\begin{split}
0&=\frac1 {r^2}\Big\langle \xi I_{\xi},\,  \big(\xi f'(0)\big)\overline{f'(\xi)}\Big\rangle
\\
&=\frac1 {r^2}\Big\langle \xi I_{\xi},\, \big[\xi, f'(0), \overline{f'(\xi)}\,\big]+\xi \big(f'(0)\overline{f'(\xi)}\,\big)\Big\rangle
\\
&=\frac1 {r^2}\Big\langle \xi I_{\xi},\,\xi \big(f'(0)\overline{f'(\xi)}\,\big)\Big\rangle
\\
&=\Big\langle I_{\xi},\, f'(0)\overline{f'(\xi)}\, \Big\rangle
\\
&=-\Big\langle I_{\xi},\, f'(\xi)\overline{f'(0)}\, \Big\rangle.
\end{split}
\end{equation}
Here we have again used Lemmas \ref{unitary multipliers} and \ref{associator}. Therefore, for each $\xi\in\mathbb B\setminus\{0\}$,
\begin{equation}\label{desried result}
\begin{split}
\Big\langle I_{\xi},\, f'(\xi)\ast\overline{f'(0)}\Big\rangle
&=\sum_{n=1}^{\infty} n \Big\langle I_{\xi},\, \xi^{n-1}\big(a_n\overline{f'(0)}\,\big)\Big\rangle
\\
&=\sum_{n=1}^{\infty} n \Big\langle I_{\xi},\, \big(\xi^{n-1}a_n\big)\overline{f'(0)}-\big[\xi^{n-1}, a_n, \overline{f'(0)}\,\big]\Big\rangle
\\
&=\sum_{n=1}^{\infty} n \Big\langle I_{\xi},\, \big(\xi^{n-1}a_n\big)\overline{f'(0)}\Big\rangle
\\
&=\Big\langle I_{\xi},\, f'(\xi)\overline{f'(0)}\, \Big\rangle
\\
&=0.
\end{split}
\end{equation}
Thus by Lemma \ref{Constant lemma}, the regular function
$$\xi\mapsto f'(\xi)\ast \overline{f'(0)}$$
must be a real constant function $|f'(0)|^2=1$, and hence $f'(\xi)\equiv f'(0)$. Consequently, $f$ is of the desired form
$$f(w)=f(0)+wf'(0).$$

Now to complete the proof, it suffices to show that how equality in (\ref{regular-diam01}) for some $r_0\in(0,1)$ implies equality in (\ref{regular-diam02}). This part is completely the same as that in the proof of \cite[Theorem 3.9]{Gen-Sar} and so we omit it.
\end{proof}

\begin{proof}[Proof of Theorem $\ref{slice diam-LT}$]
The proof of this theorem is similar to that of Theorem \ref{regular diam-LT}. The only difference  is that, instead of Theorem \ref{BSL}, we use the following simple observation.
With the regular function $h_{\xi}$ constructed in (\ref{h-definition})  and the function $f$ in this theorem in mind, if $|f'(0)|=1$, then $f_{odd}(w)=wf'(0)$ and ${\rm{diam}}\,f(r\mathbb B_I)=2r$
for each $r\in(0, 1)$ and each $I\in\mathbb S$, and hence as in (\ref{h-condition}) we have \begin{equation}\label{h-weak-condition}
  \max_{w\in r\overline{\mathbb B}_{I_{\xi}}}|h_{\xi}(w)|
  =\frac12 \max_{w\in r\overline{\mathbb B}_{I_{\xi}}}\big |f(w)-f(-\xi)\big|\leq\frac12 {\rm{diam}}\, f(r\mathbb B_{I_{\xi}})=r=|h_{\xi}(\xi)|.
\end{equation}
Thus as in the proof of Theorem \ref{BSL}, we deduce that the directional derivative of $|h_{\xi}|^2$ along the direction $v_0:=I_{\xi}\xi\in T_{\xi}\big(\partial (r\mathbb B_{I_{\xi}})\big)$ at the point $\xi \in \partial (r\mathbb B_{I_{\xi}})$ vanishes, i.e.
$$\frac{\partial |h_{\xi}|^2}{\partial v_0}(\xi)=0.$$
This together with (\ref{der-relation2}) with $f$ replaced by $h_{\xi}$ and $v$ by $v_0$ implies
\begin{equation}\label{Key relation02}
\begin{split}
0&=\Big\langle I_{\xi}\xi,\,  h_{\xi}(\xi)\overline{h_{\xi}'(\xi)}+\big[\bar{\xi}, h_{\xi}(\xi)\overline{R_{\bar{\xi}}R_{\xi}h_{\xi}(\xi)}\,\big]+2\big[\xi, h_{\xi}(\xi), R_{\bar{\xi}}R_{\xi}h_{\xi}(\xi)\big] \Big\rangle
\\
&=\Big\langle I_{\xi}\xi,\,  h_{\xi}(\xi)\overline{h_{\xi}'(\xi)}\Big\rangle,
\end{split}
\end{equation}
which is $(\ref{Key relation01})$ except the first equality there and is sufficient  to obtain $(\ref{desried result})$ and in turn the desired result. This completes the proof.
\end{proof}

\begin{proof}[Proof of Theorem \ref{Poukka}]
The argument is standard (cf. \cite[p.149, Theorem 9.1]{BMMPR}). Write $a_k:=f^{(k)}(0)/k!$, so that
$$f(w)=\sum_{k=0}^{\infty}w^ka_k.$$
Fix a positive integer  $n$ and a $I\in\mathbb S$, consider  the regular function on $\mathbb B$ given by
\begin{equation}\label{g-definition00}
g(w)=\sum_{k=0}^{\infty}w^k\big((1-e^{k\pi I/n})a_k\big).
\end{equation}
Notice that $g_I(z)=f_I(z)-f_I(ze^{\pi I/n})$ holds for all $z\in \mathbb B_I$. Thus together with Lemma \ref{unitary multipliers}, the absolute and locally uniform convergence of the power series in $(\ref{g-definition00})$ implies that for each $r\in(0,1)$,
\begin{equation}\label{Area-consideration00}
\begin{split}
d^2&\geq\frac{1}{2\pi}\int_{-\pi}^{\pi}\big|g(re^{I\theta})\big|^{2}d\theta
\\
&=\frac{1}{2\pi}\sum_{k, l=0}^{\infty} r^{k+l}\int_{-\pi}^{\pi}\Big\langle e^{kI\theta}\big((1-e^{k\pi I/n})a_k\big),\, e^{lI\theta}\big((1-e^{l\pi I/n})a_l\big)\Big\rangle \,d\theta
\\
&=\frac{1}{2\pi}\sum_{k, l=0}^{\infty} r^{k+l}\int_{-\pi}^{\pi}\Big\langle e^{(k-l)I\theta}\big((1-e^{k\pi I/n})a_k\big),\, \big((1-e^{l\pi I/n})a_l\big)\Big\rangle\, d\theta
\\
&=\frac{1}{2\pi}\sum_{k, l=0}^{\infty} r^{k+l}\bigg\langle \bigg(\int_{-\pi}^{\pi}e^{(k-l)I\theta}\,d\theta\bigg)\big((1-e^{k\pi I/n})a_k\big),\, \big((1-e^{l\pi I/n})a_l\big)\bigg\rangle
\\
&=\sum_{k=0}^{\infty}\big|1-e^{k\pi I/n}\big|^2|a_k|^2r^{2k}.
\end{split}
\end{equation}
Thus by Lebesgue's monotone convergence theorem,
\begin{equation}\label{Area-consideration}
\sum_{k=0}^{\infty}\big|1-e^{k\pi I/n}\big|^2|a_k|^2=\lim\limits_{r\rightarrow 1^-} \sum_{k=0}^{\infty}\big|1-e^{k\pi I/n}\big|^2|a_k|^2r^{2k}\leq d^2.
\end{equation}
In particular,  $$|a_n|\leq \frac d2,$$
which is precisely inequality $(\ref{Cauchy type})$.

If equality holds in $(\ref{Cauchy type})$ for some $n_0$, then (\ref{Area-consideration}) with $n$ replaced by $n_0$ implies that
$$(1-e^{k\pi I/n_0})a_k=0$$
for all $k\neq n_0$. In particular, $a_k=0$ whenever $k$ is not a multiple of $n_0$. Thus
\begin{equation}\label{fh-relation}
f(w)=h(w^{n_0}),
\end{equation}
where
$$h(w):=\sum_{k=0}^{\infty}w^ka_{kn_0},$$
which satisfies that
\begin{equation}\label{h-definition00}
h'(0)=a_{n_0}\qquad \mbox{and} \qquad {\rm{Diam}}\, h(\mathbb B)={\rm{Diam}}\, f(\mathbb B)=d.
\end{equation}
Suppose that $d>0$. By the very definition,
$$\widehat{d}\big(h(\mathbb B)\big)\leq {\rm{diam}}\,h(\mathbb B)=d. $$
This together with Theorem \ref{slice diam-LT} implies
$$\frac{d}2=|a_{n_0}|=|h'(0)|\leq\frac12 \widehat{d}\big(h(\mathbb B)\big)\leq \frac{d}2.$$
Consequently,
$$ |h'(0)|=\frac12 \widehat{d}\big(h(\mathbb B)\big)=\frac{d}2.$$
It immediately follows from Theorem \ref{slice diam-LT} that
$$h(w)=h(0)+wh'(0),$$
which implies that $f$ is of the desired form. The proof is complete.
\end{proof}

\bibliographystyle{amsplain}
\begin{thebibliography}{99}
\bibitem{ACS} D. Alpay, F. Colombo, I. Sabadini, \textit{Slice Hyperholomorphic Schur Analysis}, preprint 2015, available at: \verb"https://www.mate.polimi.it/biblioteca/add/quaderni/qdd209.pdf"

\bibitem{BMMPR} R.B. Burckel, D.E. Marshall, D. Minda, P. Poggi-Corradini, T.J. Ransford, \textit{Area, capacity and diameter versions of Schwarz's lemma}. Conform. Geom. Dyn. \textbf{12} (2008), 133--152.

\bibitem{CGSS} F. Colombo, G. Gentili, I. Sabadini, D.C. Struppa, \textit{Extension results for slice regular functions of a quaternionic variable}. Adv. Math. \textbf{222} (2009), 1793--1808.

\bibitem{CLSS} F. Colombo, R. L\'{a}vi\v{c}ka, I. Sabadini, V. Sou\v{c}ek, \textit{The Radon transform between monogenic and generalized slice monogenic functions}. Math. Ann. \textbf{363 } (2015),   733--752.
    
\bibitem{Co3} F. Colombo, I. Sabadini, D.C. Struppa,\textit{ An extension theorem for slice monogenic functions and some of its consequences}. Israel J. Math. \textbf{177} (2010), 369--389.

\bibitem{Co2} F. Colombo, I. Sabadini, D.C. Struppa, \textit{Noncommutative functional calculus. Theory and applications of slice hyperholomorphic functions.} Progress in Mathematics, vol. 289. Birkh\"auser/Springer, Basel, 2011.
\bibitem{Co6} F. Colombo, I. Sabadini, D.C. Struppa, \textit{Slice monogenic functions}. Israel J. Math. \textbf{171} (2009), 385--403.

\bibitem{Fatou} P. Fatou, \textit{S\'{e}ries trigonom\'{e}triques et s\'{e}ries de Taylor}, Acta Math. \textbf{30} (1906), 335--400.

\bibitem{GSS2014} G. Gentili, S. Salamon, C. Stoppato, \textit{Twistor transforms of quaternionic functions and orthogonal complex structures}. J. Eur. Math. Soc. \textbf{16} (2014),  2323--2353.

\bibitem{Gen-Sar} G. Gentili, G. Sarfatti, \textit{Landau-Toeplitz theorems for slice regular functions over quaternions}. Pacific J. Math. \textbf{265} (2013),  381--404.

\bibitem{GSS} G. Gentili, C. Stoppato, D.C. Struppa, \textit{Regular functions of a quaternionic variable}. Springer Monographs in Mathematics, Springer, Berlin-Heidelberg, 2013.
\bibitem{GS1} G. Gentili, D.C. Struppa,\textit{ A new approach to Cullen-regular functions of a quaternionic variable}. C. R. Math. Acad. Sci. Paris, \textbf{342}  (2006), 741--744.
\bibitem{GS2} G. Gentili, D.C. Struppa, \textit{A new theory of regular functions of a quaternionic variable}. Adv. Math.  \textbf{216}  (2007), 279--301.

\bibitem{GS50} G. Gentili, D.C. Struppa, \textit{Regular functions on the space of Cayley numbers}. Rocky Mt. J. Math. \textbf{40} (2010), 225--241.
\bibitem{GV} G. Gentili, F. Vlacci, \textit{Rigidity for regular functions over Hamilton and Cayley numbers and a boundary Schwarz Lemma}. Indag. Math. (N.S.) \textbf{19} (2008), 535--545.
\bibitem{Ghiloni1} R. Ghiloni, A. Perotti, \textit{Slice regular functions on real alternative algebras}. Adv. Math. \textbf{226} (2011), 1662--1691.
\bibitem{Ghiloni2} R. Ghiloni, A. Perotti, \textit{Zeros of regular functions of quaternionic and octonionic variable: a division lemma and the
cam-shaft effect}, Ann. Mat. Pura Appl. \textbf{190} (2011), 539--551.

\bibitem{Ghiloni5} R. Ghiloni, A. Perotti, \textit{Volume Cauchy formulas for slice functions on real associative *-algebras}. Complex Var. Elliptic Equ. \textit{58} (2013),  1701--1714.

\bibitem{Ghiloni3} R. Ghiloni, A. Perotti, \textit{Power and spherical series over real alternative $\ast$-algebras}, Indiana Univ. Math. Journal, \textbf{63} (2014),  495--532.

\bibitem{Ghiloni4} R. Ghiloni, A. Perotti, \textit{Global differential equations for slice regular functions}. Math. Nachr. \textbf{287} (2014), 561--573.

\bibitem{Ghiloni6} R. Ghiloni, V. Recupero, \textit{Semigroups over real alternative *-algebras: generation theorems and spherical sectorial operators}. Trans. Amer. Math. Soc., doi:10.1090/tran/6399, 2014.
\bibitem{Ghiloni7} R. Ghiloni, A. Perotti, C. Stoppato, \textit{The algebra of slice functions}. Trans. Amer. Math. Soc., in press.

\bibitem{Harvey} F.R. Harvey, \textit{Spinors and Calibrations}. Perspectives in Mathematics, Vol. 9, Academic Press, San Diego 1990.

\bibitem{Klimek} M. Klimek, \textit{Pluripotential Theory}. Oxford University Press, 1991.

\bibitem{Okubo} S. Okubo, \textit{Introduction to Octonion and Other Non-Associative Algebras in Physics}. Cambridge University Press, 1995.

\bibitem{Poukka} K.A. Poukka,  \textit{\"{U}ber die gr\"{o}{\ss}te Schwankung einer analytischen Funktion auf einer Kreisperipherie}. Arch. der Math. und Physik. \textbf{12} (1907), 251--254.

\bibitem{RW} G. Ren, X. wang, \textit{Slice composition operators}.  Complex Var. Elliptic Equ. 2016, to appear.

\bibitem{RW2} G. Ren, X. Wang, \textit{The growth and distortion  theorems for slice monogenic functions}. submitted. See also:  arXiv:1410.4369v2.

\bibitem{Rudin} W. Rudin, \textit{Function Theory in the Unit Ball of $\mathbb C^n$}, Springer-Verlag, New York, 1987.

\bibitem{Serodio} R.  Ser\^{o}dio, \textit{On octonionic polynomials}. Adv. Appl. Clifford Alg. \textbf{17} (2007), 245--258.

\bibitem{Stein} E.M. Stein, \textit{Boundary Behavior of Holomorphic Functions of Several Complex Variables}, Princeton University Press, Princeton, N.J., 1972.

\bibitem{Stop3} C. Stoppato, \textit{A new series expansion for slice regular functions}. Adv. Math. \textbf{231} (2012), 1401--1416.

\bibitem{WR}  X. Wang, G. Ren, \textit{Julia theory for slice regular functions}. Trans. Amer. Math. Soc. 2015, in press. See also: arXiv:1502.02368v2.

\end{thebibliography}
\end{document}


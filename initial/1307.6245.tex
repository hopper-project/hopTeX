
\usepackage{amsmath, amsthm, amsfonts, amssymb, mathrsfs}

\begin{document}

\theoremstyle{plain}
\newtheorem{define}{Definition}
\newtheorem{theorem}{Theorem}
\newtheorem{lemma}[theorem]{Lemma}
\newtheorem{criterion}[theorem]{Criterion}
\newtheorem{proposition}[theorem]{Proposition}
\newtheorem{corollary}[theorem]{Corollary}

\newtheorem{example}[theorem]{Example}
\theoremstyle{remark}
\newtheorem{remark}{Remark}

\title[Root system]{Root system of singular perturbations of the harmonic oscillator type operators}

\author{Boris Mityagin}
\address[Boris Mityagin]{
Department of Mathematics
The Ohio State University
231 West 18th Ave.
Columbus, OH 43210, USA}
\email{mityagin.1@osu.edu}

\author{Petr Siegl}
\address[Petr Siegl]{Mathematishes Institut, Universit\"at Bern, Sidlerstrasse 5, 3006 Bern, Switzerland \& On leave from Nuclear Physics Institute ASCR, 25068 \v Re\v z, Czech Republic}
\email{petr.siegl@math.unibe.ch}

\subjclass[2010]{47A55, 47A70, 34L10}

\keywords{non-self-adjoint operators, harmonic oscillator, Riesz basis, quadratic forms, singular potentials}

\date{23rd July 2013}

\begin{abstract}
We analyze perturbations of the harmonic oscillator type operators in a Hilbert space ${{\mathcal{H}}}$, {{\emph{i.e.}}}~of the self-adjoint operator with simple positive eigenvalues $\mu_k$ satisfying $\mu_{k+1}-\mu_k \geq \Delta >0$. Perturbations are considered in the sense of quadratic forms. Under a local subordination assumption, the eigenvalues of the perturbed operator become eventually simple and the root system forms a Riesz basis. 
\end{abstract}

\thanks{
P.S. appreciates the kind hospitality and support of OSU allowing his stays there in November 2012 and July 2013. P.S. acknowledges the SCIEX Programme, the work has been conducted within the SCIEX-NMS Fellowship, project 11.263.
}

\maketitle

\section{Introduction}

This paper deals with the spectrum and eigensystem of perturbations of a self-adjoint operator $A$ in a Hilbert space ${{\mathcal{H}}}$. $A$ is of the one dimensional harmonic oscillator type,
{{\emph{i.e.}}}~its eigenvalues are simple, positive and satisfy 
\begin{equation}\label{ass.A}
A \psi_n = \mu_n \psi_n, \quad 
\mu_1 >0, 
\quad
\mu_{n+1} -\mu_n \geq  \Delta, \ \forall n \in {\mathbb{N}}, \ \Delta> 0;
\end{equation}
see also Remark \ref{rem.sim.as}.
The perturbations are not assumed to be symmetric, therefore
the studied operator $T$ is generically non-self-adjoint (and non-normal), 
hence the spectrum typically does not remain real and the basis property of eigensystem is no longer guaranteed.

The main aim is to extend (to cover in particular the $\delta$ potential) the results of \cite{Adduci-2012-10, Adduci-2012-73,Agranovich-1994-28} on sufficient conditions on perturbations guaranteeing that the eigensystem of the perturbed operator forms a Riesz basis. Problems of this type are studied in many works, both classical ones as \cite{DS3,Kato-1966,Markus-1988} and more recent ones, for instance, \cite{Agranovich-1994-28, Shkalikov-2010-269, Wyss-2010-258, Xu-2005-210, Zwart-2010-249}. 

The essential issue in the analysis is that the gaps between the eigenvalues of the unperturbed operator $A$ do not grow. Assuming that the gaps grow, {{\emph{i.e.}}}~$\mu_{n+1}-\mu_n \rightarrow + \infty $ and the perturbation $B$ is bounded, Kato proved, {\emph{cf.}}~\cite[Thm. V.4.15a, Lem. V.4.17a]{Kato-1966}, that the system of eigenfunctions of $A+B$, plus possibly finite number of associated functions, forms a Riesz basis. The analogous classical theorem allowing also unbounded perturbations can be found in \cite[Thm. XIX.2.7]{DS3}; nevertheless, the growth condition of the gaps is preserved. Constant gaps are allowed in \cite{Agranovich-1994-28}; however, only bounded perturbations satisfy the assumption in \cite{Agranovich-1994-28} and the result is weaker, since only the Riesz basis with brackets is claimed.

Adduci and Mityagin overcome the problem of the non-growing gaps in the study of the harmonic oscillator, {\emph{cf.}}~\cite{Adduci-2012-10}, by 

\noindent
a) using the Hilbert transform as the important technical tool,

\noindent
b) replacing the condition of the boundedness of perturbation $B$ by 
\begin{equation}\label{trad.ass.B}
\|B \psi_n \| \rightarrow 0 \ {\rm as} \  n \rightarrow \infty,
\end{equation}

\noindent
c) using the following result of Kato; 
\begin{criterion}[{\cite[Lem.V.4.17a]{Kato-1966}}]\label{lem.Kato}
Let $\{P_j\}_{j\in{\mathbb{N}}_0}$ be a complete family of orthogonal projections in a Hilbert space ${{\mathcal{H}}}$, and let $\{Q_j\}_{j\in{\mathbb{N}}_0}$ be a family of (not necessarily orthogonal) projections such that $Q_jQ_k = \delta_{jk}Q_j$. Assume that
\begin{equation*}
\begin{aligned}
{\rm Rank} \, P_0 &= {\rm Rank} \, Q_0 < \infty, \\
\sum_{j=1}^{\infty} \|P_j(Q_j-P_j) u\|^2 &\leq c_0 \|u\|^2 \ {\rm for \ every \ } u \in {{\mathcal{H}}},
\end{aligned}
\end{equation*}
where $c_0$ is a constant smaller than $1$. Then there is a $W \in {\mathscr{B}(\mathcal{H})}$ with $W^{-1} \in {\mathscr{B}(\mathcal{H})}$ such that $Q_j=W^{-1} P_j W$ for $j \in {\mathbb{N}}_0$.
\end{criterion}

The condition \eqref{trad.ass.B} has been called by Shkalikov a \emph{local subordinate condition}, see the discussion in \cite{Shkalikov-2012-18} and also \cite[Sec.1]{Adduci-2012-73} for some explanations on this wording. 
Criterion \ref{lem.Kato} is a substitution of the often used Bari--Markus criterion, which is given with more restrictive conditions 
\begin{equation*}
\begin{aligned}
&\sum_{j=0}^\infty \| Q_j -P_j\|^2 < \infty \\
&{\rm Rank} \, P_j = {\rm Rank} \, Q_j < \infty, \quad j=0,1,\dots,
\end{aligned}
\end{equation*}
see {{\emph{e.g.}}}~\cite[Chap.6, Sec. 5.3, Thm. 5.2]{Gohberg-1969} or \cite{Markus-1988}.

For the harmonic oscillator in $L^2({{\mathbb{R}}})$, the property of Hermite functions, 
\begin{equation*}
\max_{x \in {{\mathbb{R}}}} |h_n(x)| \leq C (1+n)^{-1/12},
\end{equation*}
can be used to show that \eqref{trad.ass.B} is satisfied for $B$ being, for instance, a multiplication operator by $v \in L^2({{\mathbb{R}}})$ what is consistently used in \cite{Adduci-2012-10}.

The results of \cite{Adduci-2012-10} for the harmonic oscillator have been extended in \cite{Adduci-2012-73} to the abstract setting with the possibility of the controlled condensation of eigenvalues, {{\emph{i.e.}}}~$\mu_{n+1} -\mu_n \geq \kappa n^{\omega-1}$ with fixed $\kappa >0$ and $\omega >1/2$, or the finite clustering of eigenvalues, {{\emph{i.e.}}}~there exist fixed values $q>0$ and $\delta>0$ such that $\mu_{n+q}-\mu_n \geq \delta$ for all $n$. In the latter case, similar results have been obtained in \cite{Shkalikov-2010-269} using different methods. However, the $\delta$ potential is not covered by the assumption \eqref{trad.ass.B} which is essential in \cite{Adduci-2012-10, Adduci-2012-73, Shkalikov-2010-269}.

In this paper, we consider perturbations of $A$ in the sense of quadratic forms. (Such a setting has been considered in \cite{Agranovich-1994-28} under the form $p$-subordination assumption, {\emph{cf.}}~\eqref{form.p.sub} below.)
At first we define the quadratic form $t:=a+b$, where $a$ corresponds to $A$ and $b$ is the perturbation. The perturbed operator $T$ is associated with the form $t$, see Section \ref{sec.op.def} for details. Such a framework is one way how to include singular perturbations, {\emph{cf.}}~\cite[Chap.VI.3.-4.]{Kato-1966} or \cite[\S.1.]{Simon-1971-21} in self-adjoint setting. Our main example is the harmonic oscillator in $L^2({{\mathbb{R}}})$ perturbed by the $\delta$ potential with complex coupling.  
We remark that the form $b$ does not need to be closed and therefore it does not need to represent an operator in a considered Hilbert space ${{\mathcal{H}}}$, distributional potentials are typical cases. 

A straightforward reformulation of the condition \eqref{trad.ass.B}, coming from \cite{Adduci-2012-10, Adduci-2012-73, Shkalikov-2010-269}, would be 
\begin{equation*}\label{trad.ass.B.ref}
\|B\psi_n\|^2 = \sum_{m=1}^{\infty} |b(\psi_n, \psi_m)|^2 \rightarrow 0 \ {\rm as} \ n \rightarrow \infty.
\end{equation*}
Nevertheless, the analysis of the harmonic oscillator perturbed by the $\delta$ potential, {{\emph{i.e.}}}~$b(\phi,\psi)=\phi(0) \overline{\psi(0)} $, reveals that the condition \eqref{trad.ass.B.ref} is not satisfied, {\emph{cf.}}~\eqref{hn.expl}--\eqref{hn.bound} in Section \ref{sec.ex}.

Our results are obtained under the assumption
\begin{equation}\label{ass.b}
\forall m,n \in {\mathbb{N}}, \ \ |b(\psi_m,\psi_n)| \leq \frac{M_b}{m^{\alpha} n^{\alpha}},  \ \alpha >0, \ M_b >0.
\end{equation}
This extends the previously considered classes of perturbations. For the harmonic oscillator particularly, it means a step towards singular potentials including the mentioned $\delta$. Moreover,  this paper yields a partially new version of the proof of the main result in \cite{Adduci-2012-10} for some cases. More precisely, unlike in \cite{Adduci-2012-10, Adduci-2012-73}, where the important technical tool was the Hilbert transform, only the Schur test is used here.

We remark that a non-symmetric situation, {{\emph{i.e.}}}~$|b(\psi_m,\psi_n)| \leq M_b n^{-\alpha}m^{-\beta}$, $\alpha, \beta >0$ can be analysed in the same way, only a straightforward modifications in the proofs are needed. Moreover, the connection to the previous work \cite{Adduci-2012-10} 
is explained in the last example of Section \ref{sec.ex}, {\emph{cf.}}~\eqref{V.form}--\eqref{V.form.bound}.

This paper as well as mentioned previous works aims to find sufficient conditions for the Riesz basisness of the eigensystem. However, the negative results, {{\emph{i.e.}}}~the fact that the eigensystem does is not a Riesz basis (or even a basis), have been obtain particularly for complex oscillators in \cite{Davies-1999-200, Davies-2000-32, Davies-2004-70, Henry-2013}, and just recently in \cite{Mityagin-2013-prep}.

The paper is organised as follows. In Section \ref{sec.op.def}, we define the operator $T$ and recall some known facts. The main results on the localization of the spectrum and Riesz basisness of the eigensystem are contained and proven in Section \ref{sec.main.res}. In Section \ref{sec.tech.lem}, we collect technical lemmas used in the proofs of main results. Section \ref{sec.ex} consists of examples and conclusions and discussion are contained in the final Section \ref{sec.concl}.

\section{Definition of the operator and preliminaries}
\label{sec.op.def}

The definition of the operator $T$ is based on the first representation theorem \cite[Thm.VI.2.1]{Kato-1966} that provides the unique correspondence between the $m$-sectorial operator $T$ and the closed sectorial form $t$.
The detailed definition of the operator $T$ can be also found in \cite[Sec.2.]{Agranovich-1994-28}.

The self-adjoint operator $A$ is associated, via the second representation theorem \cite[Thm.VI.2.23]{Kato-1966}, with a quadratic form 
\begin{equation*}
a(\psi,\psi)  := \|A^{1/2} \psi\|^2, \ \ 
{{\operatorname{Dom}}}(a)  := {{\operatorname{Dom}}}(A^{1/2}).
\end{equation*}
We consider perturbations by a form $b$ satisfying the condition \eqref{ass.b}.
It follows that $b$ is a form $p$-subordinated perturbation of $a$, {{\emph{i.e.}}}~ 
there exist $0 \leq p<1$ and $C>0$ such that  
\begin{equation}\label{form.p.sub}
\forall f \in {{\operatorname{Dom}}}(a), \quad |b(f,f)| \leq C \left( a(f,f) \right )^p \|f\|^{2(1-p)},
\end{equation}
see Lemma \ref{lem.sub}. The form $p$-subordination implies the form relative boundedness of $b$ with respect to $a$ with the bound $0$. 
The perturbed operator $T$ is defined as the operator associated, via the first representation theorem \cite[Thm.VI.2.1]{Kato-1966}, with a sectorial form
\begin{equation*}
t:= a + b,  \ \ 
{{\operatorname{Dom}}}(t)  = {{\operatorname{Dom}}}(a).
\end{equation*}
The domains of $t$ and $a$ are the same due to the $p$-subordination, nevertheless, ${{\operatorname{Dom}}}(T)$ and ${{\operatorname{Dom}}}(A)$ are typically different. 
The form relative boundedness with the bound 0 together with \cite[Thm.VI.3.4]{Kato-1966} imply that $T$ has a compact resolvent. 

The definition of $T$ can be also reformulated as
\begin{equation*}
T = A^{1/2}(I+B(0))A^{1/2}.
\end{equation*}
Here $B(z)$, $z \in {{\mathbb{C}}}$, is the operator uniquely determined by the bounded form $b((z-A)^{-1/2} \cdot,(\overline{z}-A)^{-1/2} \cdot)$, {{\emph{i.e.}}}~$\langle B(z)f,g \rangle = b((z-A)^{-1/2} f,(\overline{z}-A)^{-1/2} g) $ for all $f,g \in {{\mathcal{H}}}$. The square root of $z-A$ is defined as 
\begin{equation*}
(z-A)^{-1/2} f := \sum_{k \in {\mathbb{N}}} (z-\mu_k)^{-1/2} c_k \psi_k,
\end{equation*}
for $f = \sum_{k \in {\mathbb{N}}} c_k \psi_k$ and $w^s := |w|^s e^{{{\rm i}} s \arg w}$, $-\pi < \arg w \leq \pi$, for $0\neq w \in {{\mathbb{C}}}$ and $s\in {{\mathbb{R}}}$.
For all $z \in \rho(A)$, 
\begin{equation*}
z-T = (z-A)^{1/2}(I+B(z))(z-A)^{1/2}.
\end{equation*}
This relation yields a suitable representation of the resolvent of $T$, {{\emph{i.e.}}}~
\begin{equation}\label{Tz.res.dec}
(z-T)^{-1} = (z-A)^{-1/2}(I+B(z))^{-1}(z-A)^{-1/2},
\end{equation}
provided $I+B(z)$ is invertible and $z \in \rho(A)$. Formulas of this type are also derived in~\cite[Lem.1.]{Agranovich-1994-28}, \cite[Chap.VI.3.1.]{Kato-1966}.

\begin{remark}\label{rem.sim.as}
We have started with the operator $A$ with eigenvalues satisfying $\mu_{n+1} -\mu_n \geq \Delta$, $\Delta >0$. However, to simplify all formulas, we will assume that $\Delta=1$ and $\mu_1 \geq 1$ in the sequel.
(This can be always achieved by considering $\Delta^{-1}(A + c I)$ with suitably chosen $c \in {{\mathbb{R}}}_+$.) Eigenvalues $\mu$ then satisfy
\begin{equation}\label{mu.sim}
\mu_k \geq k.
\end{equation}

\end{remark}

\section{Main results}
\label{sec.main.res}

Set 
\begin{equation}\label{Pin.def}
\begin{aligned}
\Pi_0&:=\{ z \in {{\mathbb{C}}}: - h < {\operatorname{Re}} z <  (N+3/2), |{\operatorname{Im}} z| <  h\} \\
\Pi_k&:=\{ z \in {{\mathbb{C}}}: |z-\mu_k| < 1/2 \}, \ \ \Gamma_k:=\partial \Pi_k,\\
\Pi&:=\Pi_0 \bigcup_{j>N+1}\Pi_j,
\end{aligned}
\end{equation}
where $N \in {\mathbb{N}}$ and $h>1$ are determined in the following way. The aim is to localize the spectrum of $T$. We will succeed if we guarantee that $\|B(z)\|\leq 1/2$ for $z$ outside of $\Pi$. 

Let $N\equiv N(\alpha)$ be an integer such that
\begin{equation}\label{ass.N}
M_b \, C(2\alpha) \, \sigma_{2\alpha}(n) \leq  \frac{1}{2}, \ \forall n>N,
\end{equation}
where $C(\alpha),$ $\sigma_{\alpha}(n)$ are introduced in Lemma \ref{lem.sum.est.2} below. $h>1$ is selected such that 
\begin{equation}\label{ass.h}
2 M_b 
\left(
\frac{1}{h}\sum_{k=1}^{N+2} \frac{1}{k^{2\alpha} } 
+
D(2\alpha) \tau_{2\alpha}(h)
\right)
\leq \frac{1}{2},
\end{equation}
where $D(\alpha),$ $\tau_{\alpha}(h)$ are introduced in Lemma \ref{lem.h}.

\begin{proposition}\label{prop.loc}
Let conditions \eqref{ass.A}, \eqref{ass.b} hold and let $N$ and $h$ satisfy the conditions \eqref{ass.N} and \eqref{ass.h}, respectively. Then the eigenvalues of $T$ are contained in the interior of $\Pi$, {\emph{cf.}}~\eqref{Pin.def}. Moreover, Riesz projections 
\begin{equation}\label{Pn.SN.def}
\begin{aligned}
S_{N+1}&:=\frac{1}{2\pi {{\rm i}}} \int_{\Gamma_0}(z-T)^{-1} {{{\rm d}}} z, \\
P_j & := \frac{1}{2\pi {{\rm i}}} \int_{\Gamma_j}(z-T)^{-1} {{{\rm d}}} z \ {\rm for} \ \ j>N+1,
\end{aligned}
\end{equation}
are well-defined and 
\begin{equation*}
{\rm Rank} \, S_{N+1} =  N+1, \quad {\rm Rank} \, P_{j} =1 \ {\rm for } \ j>N+1.
\end{equation*}

\end{proposition}
\begin{proof}
At first we show that $(z-T)^{-1}$ is bounded for every $z \notin \Pi_0 \cup_{j>N+1}\Pi_j$. Using the resolvent factorization \eqref{Tz.res.dec}, it suffices to prove that $\|B(z)\| \leq 1/2$. Let $f = \sum_{j=1}^{\infty} f_j \psi_j \in {{\mathcal{H}}}$, then

\begin{equation}\label{loc.est.1}
\begin{aligned}
\|B(z)f\|^2  &= 
\sum_{k=1}^{\infty} |\langle B(z) f, \psi_k \rangle|^2 
= 
\sum_{k=1}^{\infty} 
\left| 
\sum_{j=1}^{\infty} \frac{f_j b(\psi_j,\psi_k) }{(z-\mu_j)^{\frac{1}{2}} (\overline{z}-\mu_k)^{\frac{1}{2}}}
\right|^2 
\\
&\leq 
M_b^2 \sum_{k=1}^{\infty} 
\frac{1}{k^{2\alpha}|\mu_k-z|}
\left(
\sum_{j=1}^{\infty} \frac{|f_j|}{ j^{\alpha}|\mu_j-z|^{\frac{1}{2}}}
\right)^2 
\\
&
\leq 
M_b^2 
\left(
\sum_{k=1}^{\infty} 
\frac{1}{k^{2\alpha}|\mu_k -z|}
\right)^2
\|f\|^2.
\end{aligned}
\end{equation}

Let ${\operatorname{Re}} z \in [(\mu_n-1/2),(\mu_n+1/2)]$ and $z \notin \Pi_n$, $n \geq N$. We apply inequalities from Lemma \ref{lem.sum.est.2} and we obtain
\begin{equation}\label{loc.est.2}
\|B(z)\| 
\leq
M_b \, C(2\alpha) \, \sigma_{2\alpha} (n)
\leq \frac{1}{2},
\end{equation}
for $n >N$, where $N$ is chosen above, {\emph{cf.}}~\eqref{ass.N}. 

The next step is estimate outside of $\Pi_0$.
If ${\operatorname{Re}} z = - h$, then
\begin{equation*}
 \|B(z)\|
\leq 
M_b 
\sum_{k=1}^{\infty} \frac{1}{ k^{2\alpha}(k+h) }
\leq 
M_b \, D(2\alpha) \, \tau_{2\alpha}(h)
\leq
\frac{1}{2},
\end{equation*}
for $h$ selected as above, {\emph{cf.}}~\eqref{ass.h}.

If ${\operatorname{Re}} z \in [ - h, (N+3/2)]$, then $|{\operatorname{Im}} z| \geq h$. We denote $z= s+{{\rm i}} h$, then
\begin{equation*}
\begin{aligned}
\|B(z)\| 
&\leq 
M_b 
\sum_{k=1}^{\infty} \frac{1}{k^{2\alpha} \sqrt{(\mu_k-s)^2 +h^2 }}
\leq 
2  M_b
\sum_{k=1}^{\infty} \frac{1}{k^{2\alpha} (|\mu_k-s| +h) }
\\
&
\leq 
2  M_b
\left(
\sum_{k=1}^{N+2} \frac{1}{h k^{2\alpha}} +
\sum_{k=N+3}^{\infty} \frac{1}{k^{2\alpha} (k- N-2 +h) }
\right),
\\
&
\leq 
2  M_b
\left(
\frac{1}{h} \sum_{k=1}^{N+2} \frac{1}{k^{2\alpha}} +
D(2\alpha) \tau_{2\alpha}(h)
\right)
\leq
\frac{1}{2}
,
\end{aligned}
\end{equation*}
where we use \eqref{mu.sim}, \eqref{ass.h}, and inequalities 
$$(a+b)/2 \leq \sqrt{a^2 + b^2}, \qquad -h \leq s  \leq N+3/2.$$

The standard argument, based on \cite[Lem.VII.6.7]{DS1}, shows that 
\begin{equation*}
{\rm Trace} \, \frac{1}{2\pi {{\rm i}}} \int_{\Gamma_n} (z-A)^{-1/2}(I+t B(z))^{-1} (z-A)^{-1/2} {{{\rm d}}} z, \ 0 \leq t \leq 1,   
\end{equation*}
is a continuous integer valued function. Therefore it is constant and the second part of the claim follows.
\end{proof}

The obvious corollary is that the eigenvalues $\lambda_n$ of $T$ become eventually simple (for $n > N+1$) and localized around those of the unperturbed operator $A$, while the first part of the spectrum is localized in $\Pi_0$; it is important that there is only a finite number of eigenvalues in $\Pi_0$. The latter also means that the eigensystem of $T$ contains at most finite number of root vectors associated to different eigenvalues $\{\lambda_n\}_{n=1}^{N_0}$, $N_0 \leq N+1$, in $\Pi_0$ with algebraic multiplicities $\{m_n\}_{n=1}^{N_0}$, $\sum_{n=1}^{N_0} m_n = N+1$, and the rest, $\{\lambda_n\}_{n>N+1}$, consists of eigenvectors $\{\phi_n\}$ related to eigenvalues in $\cup_{j>N}\Pi_j$ of both algebraic and geometric multiplicity one. 

\begin{remark}
Proposition \ref{prop.loc} serves to define the Riesz projections $S_{N+1}$ and $P_j$ that are further analysed in Theorem \ref{thm.RB}. If we wish to localize the eigenvalues of $T$ more precisely, we can modify $\Pi_k$, $k>N$, to be circles with radii $r_k \to 0$ instead of $1/2$. Then the straightforward modification of estimates \eqref{loc.est.1}, \eqref{loc.est.2}, and Lemma \ref{lem.sum.est.2} yields that $r_k$ can decrease as $r_k \sim k^{-2\alpha} \log k$ or $r_k \sim k^{-1}$ depending on $2\alpha \leq 1$ or $2\alpha >1$.
\end{remark}

To formulate the main result we denote $\{P_n^0\}$ the one dimensional spectral projections of $A$ related to eigenvalues $\{\mu_n\}$, {{\emph{i.e.}}}~ 
\begin{equation*}\label{sp.proj.A}
P_n^0  := \frac{1}{2\pi {{\rm i}}} \int_{|z-\mu_n|=\Delta/2}(z-A)^{-1} {{{\rm d}}} z, \quad n \in {\mathbb{N}}.
\end{equation*}

\begin{theorem}\label{thm.RB}
Let conditions \eqref{ass.A}, \eqref{ass.b} hold. 
Then there exists a bounded operator $W$ with bounded inverse such that projectors $\{P_n\}$ and $S_{N+1}$, {\emph{cf.}}~\eqref{Pn.SN.def}, satisfy 
$$
W P_n W^{-1} = P_n^0
$$
for all $n > N+1$ and 
$$
W S_{N+1} W^{-1} = \sum_{n=1}^{N+1} P_n^0.
$$
Hence, $\{S_{N+1},P_{N+2},P_{N+3},\dots\}$ is a Riesz system of projectors.
\end{theorem}
\begin{remark}
Projectors $P_n$ are one-dimensional for $n>N+1$ and $S_{N+1}$ has rank $N+1$. Therefore the system of root vectors of $T$ contains a Riesz basis $\{f_n\}_{n=1}^{\infty}$ with $f_n = \phi_n$ for $n>N+1$.
\end{remark}

\begin{proof}
The proof is based on Criterion \ref{lem.Kato}. The spectral projections $P_n^0$ of $A$ form a complete family of orthogonal projections, since $A$ is self-adjoint with discrete spectrum. 

In order to apply Criterion \ref{lem.Kato}, we have to find $N_* > N+2$, $N_* \in {\mathbb{N}}$, such that
\begin{equation*}
\forall f \in {{\mathcal{H}}}, \ \ \sum_{n \geq N_*}^{\infty} \|P_n^0(P_n-P_n^0)f\|^2 \leq \frac{1}{2} \|f\|^2,
\end{equation*}
so we take (in the notation of Criterion \ref{lem.Kato}) $P_0:= \sum_{j=1}^{N_*-1} P_j^0$ and $Q_0= S_{N_*-1}$. The latter has the same rank as $P_0$, {\emph{cf.}}~Proposition \ref{prop.loc}.

If $n > N + 2$ and $z\in \Gamma_n$, then we have
\begin{equation}
\label{Pn.est.1}
\begin{aligned}
&\|P_n^0(P_n-P_n^0)f\|^2 \leq \frac{1}{4\pi^2} \left\|\int_{\Gamma_n} ((z-T)^{-1} - (z-A)^{-1} )f {{{\rm d}}} z \right\|^2 
\\
& \leq 
\frac{1}{4\pi^2}
\left( \int_{\Gamma_n} \|(z-A)^{-\frac{1}{2}} \| \|(I+B(z))^{-1}\| \|B(z)(z-A)^{-\frac{1}{2}} f\| |{{{\rm d}}} z | \right)^2
\\
& \leq \frac{2}{\pi^2} \left( \int_{\Gamma_n} \|B(z)(z-A)^{-\frac{1}{2}} f\| |{{{\rm d}}} z| \right)^2,
\end{aligned}
\end{equation}
where we use the factorization of resolvent \eqref{Tz.res.dec} and the bound $\|B(z)\| \leq 1/2$ for $z \in \Gamma_n$, $n>N$, {\emph{cf.}}~the proof of Proposition \ref{prop.loc}.
Decomposing $f = \sum_{j=1}^{\infty} f_j \psi_j \in {{\mathcal{H}}}$ we obtain
\begin{equation*}
\begin{aligned}
 \|B(z)(z-A)^{-\frac{1}{2}} f\|^2 &= 
\sum_{k=1}^{\infty} 
\left| 
\left\langle 
B(z) (z-A)^{-\frac{1}{2}} 
\sum_{j=1}^{\infty}f_j \psi_j,
\psi_k
\right\rangle
\right|^2
\\
& \leq 
\sum_{k=1}^{\infty} \frac{1}{|\mu_k-z|} 
\left| 
\sum_{j=1}^{\infty} \frac{f_j}{z-\mu_j} b(\psi_j,\psi_k) \right|^2
\\
&
\leq  M_b^2
\sum_{k=1}^{\infty} \frac{1}{k^{2\alpha}|\mu_k-z|} 
\left(
\sum_{j=1}^{\infty} \frac{|f_j|}{j^{\alpha}|\mu_j-z|}  
\right)^2.
\end{aligned}
\end{equation*}
For $n >N +2$, we select $z_n^* \in \Gamma_n$ for which the maximum of the integrand in \eqref{Pn.est.1} is attained; notice that $\{z_n^*\}$ depends on $f$. Then we can continue estimates in \eqref{Pn.est.1},
\begin{equation}\label{Pn.est.2}
\begin{aligned}
\|P_n^0(P_n-P_n^0)f\|^2 
\leq 
2 M_b^2  
\sum_{k=1}^{\infty} \frac{1}{k^{2\alpha}|\mu_k-z_n^*|} 
\left( 
\sum_{j=1}^{\infty} \frac{|f_j|}{j^{\alpha}|\mu_j-z_n^*|}  
\right)^2.
\end{aligned}
\end{equation}
We apply Lemma \ref{lem.sum.est.2} on the first sum in \eqref{Pn.est.2},
\begin{equation*}
\begin{aligned}
\|P_n^0(P_n-P_n^0)f\|^2 
\leq 
2 M_b^2 C(2\alpha) \sigma_{2\alpha}(n)
\left( 
\sum_{j=1}^{\infty} \frac{|f_j|}{j^{\alpha}|\mu_j-z_n^*|}  
\right)^2.
\end{aligned}
\end{equation*}

The final step is estimating the sum of $\|P_n^0(P_n-P_n^0)f\|^2$, starting at some $N_1 > N+2$, $N_1 \in {\mathbb{N}}$. We fix $\omega$, $0< \omega <\alpha$, and assume that $N_1$ so large that $\sigma_{\alpha}(n) \leq \sigma_{\alpha}(N_1)$ for $n \geq N_1$; notice that $\sigma_{2\alpha}(n) \leq \sigma_{2\alpha}(N_1)$ is valid as well.
Then, leaving aside the constant,
\begin{equation*}
\begin{aligned}
&\sum_{n=N_1}^{\infty} 
\sigma_{2\alpha}(n)
\left( 
\sum_{j=1}^{\infty} \frac{|f_j|}{j^{\alpha}|\mu_j-z_n^*|} 
\right)^2
\leq  
\sigma_{2\alpha}(N_1) N_1^{2\omega} 
\sum_{n= N_1}^{\infty} 
\left( 
\sum_{j=1}^{\infty} \frac{|f_j|}{n^{\omega}j^{\alpha}|\mu_j-z_n^*|} 
\right)^2
\\
&
=
\sigma_{2\alpha}(N_1) N_1^{2\omega} 
\|\mathcal{M} \tilde{f}\|^2_{\ell^2({\mathbb{N}})}
,
\end{aligned}
\end{equation*}
where $\mathcal{M}$ is an operator acting in $\ell^2({\mathbb{N}})$ with matrix elements 
\begin{equation*}
\begin{aligned}
\mathcal{M}_{nj}&= 0, & n < N_1, \\
\mathcal{M}_{nj}&=
\frac{1}{n^{\omega}j^{\alpha}|\mu_j-z_n^*|}, & n \geq N_1.
\end{aligned}
\end{equation*}
and $\tilde{f}=\{|f_n|\}_{n\in{\mathbb{N}}} \in \ell^2({\mathbb{N}})$.

We intend to bound $\|\mathcal{M}\|$ using the Schur test, {\emph{cf.}}~\cite{Schur-1911-140} or \cite[Sec. 3]{Dym-2003-210}, \cite[Thm. 5.2]{Halmos-1978-96}. To this end we estimate the following sums by applying Lemma \ref{lem.sum.est.2}
\begin{equation*}
\begin{aligned}
& \sum_{j=1}^{\infty}|\mathcal{M}_{nj}| 
=
\sum_{j=1}^{\infty} \frac{1}{n^{\omega}j^{\alpha}|\mu_j-z_n^*|}
\leq 
\frac{1}{n^{\omega}} C(\alpha) \sigma_{\alpha}(n)
\leq 
C(\alpha) \frac{\sigma_{\alpha}(N_1)}{N_1^{\omega}},
\\
&\sum_{n=1}^{\infty}|\mathcal{M}_{nj}| 
\leq
\frac{1}{j^{\alpha}}
C(\omega)
\sigma_{\omega}(j)
\leq
\frac{C(\omega)}{\omega e}
,
\end{aligned}
\end{equation*}
where we use \eqref{log.est} and that $N_1$ is such that $\sigma_{\alpha}(n) \leq \sigma_{\alpha}(N_1)$. 
The Schur test then yields
\begin{equation*}
\|\mathcal{M}\|^2
\leq 
 \frac{C(\alpha) C(\omega)}{\omega e}   
\frac{\sigma_{\alpha}(N_1)}{N_1^{\omega}}.
\end{equation*}
Therefore, since $\omega < \alpha$, 
\begin{equation*}
\begin{aligned}
\sum_{n\geq N_1}^{\infty} \|P_n^0(P_n-P_n^0)f\|^2 
\leq  
\frac{2 M_b^2 C(2\alpha) C(\alpha) C(\omega)}{\omega e}   
\sigma_{2\alpha}(N_1) \sigma_{\alpha}(N_1) N_1^{\omega} 
\|f\|^2,
\end{aligned}
\end{equation*}
which proves the existence of sought $N_*$.

\end{proof}

\section{Technical lemmas}
\label{sec.tech.lem}

We collect technical results, mainly estimates on the sums appearing in the proofs of main results. At first we explain in details the form subordination of the perturbation.

\begin{lemma}\label{lem.sub}
Let $A$ and $b$ satisfy \eqref{ass.A} and \eqref{ass.b}.
Then there exist $0 \leq p <1$ and $C>0$ such that  
\begin{equation*}
\forall f \in {{\operatorname{Dom}}}(a), \ |b(f,f)| \leq C \left( a(f,f) \right)^p \|f\|^{2(1-p)},
\end{equation*}
{{\emph{i.e.}}}~$b$ is $p$-subordinated to $a$. Moreover, for $\alpha \leq 1/2$, $p$ can be selected as $1-2\alpha + \tau$, with $\tau >0$ arbitrarily small. If $\alpha>1/2$, then $b$ is bounded.
\end{lemma}
\begin{proof}
Writing $f= \sum_{j=1}^{\infty} f_j \psi_j$, we get ($0\leq\beta < 1/2$)
\begin{equation*}
\begin{aligned}
|b(f,f)| & = 
\left| 
\sum_{j,k=1}^{\infty} \overline{f}_j f_k b(\psi_j,\psi_k)  \right| 
\leq 
M_b 
\left(
\sum_{j=1}^{\infty} \frac{|f_j|}{j^{\alpha}} 
\right)^2
= 
M_b 
\left(
\sum_{j=1}^{\infty} |f_j|j^{\beta} \frac{1}{j^{\alpha+\beta}}  
\right)^2
\\ 
& 
\leq  
M_b \|A^{\beta} f\|^2 
\sum_{j=1}^{\infty} \frac{1}{j^{2(\alpha+\beta)}}
\leq 
M_b (a(f,f))^{2\beta} \|f\|^{2(1-2\beta)} 
\sum_{j=1}^{\infty} \frac{1}{j^{2(\alpha+\beta)}}, 
\end{aligned}
\end{equation*}
where we used the H\"older inequality in the last step.
For $\alpha \leq 1/2$, we select $\beta=1/2-\alpha + \tau/2$, $\tau >0$ and we receive the claim with $p=1-2\alpha + \tau$. For $\alpha > 1/2$, we can take $\beta=0$ and therefore $b$ is bounded.
\end{proof}

\begin{lemma}\label{lem.sum.est.2}
Let $n \in {\mathbb{N}}$, $n>1$, $ {\operatorname{Re}} z_n \in [\mu_n-1/2,\mu_n+1/2]$, and $z_n \notin \Pi_n$, where $\Pi_n$ is defined in \eqref{Pin.def} and $\gamma >0$. Then
\begin{equation*}
\sum_{k =1}^{\infty} \frac{1}{k^{\gamma}|\mu_k-z_n| } 
\leq
C(\gamma)
\sigma_{\gamma}(n),
\end{equation*}
where $C(\gamma)$ does not depend on $n$ and
\begin{equation*}
\sigma_{\gamma}(n):=
\begin{cases}
n^{-\gamma} \log n, & \gamma \leq 1, 
\\
n^{-1},  & \gamma > 1.
\end{cases}
\end{equation*}

\end{lemma}
\begin{proof}
Using $|\mu_k-z_n|\geq |k-n|/2$ for $k\neq n$, we obtain
\begin{equation}
\label{eq.lem.1}
\sum_{k=1}^{\infty} \frac{1}{k^{\gamma}|\mu_k-z_n| } \leq 
2
\left( \sum_{k=1}^{n-1} \frac{1}{k^{\gamma}(n-k) } +
\frac{1}{n^{\gamma}}
+ \sum_{k=n+1}^{\infty} \frac{1}{k^{\gamma}(n-k) }
\right)
.
\end{equation}
If $f$ is a convex non-negative function in interval $[1,p]$, then
\begin{equation*}
\int_1^p f(x) {{{\rm d}}} x \leq \sum_{i=1}^p f(i) \leq f(1) + f(p) + \int_1^p f(x) {{{\rm d}}} x.
\end{equation*}
Therefore the first term on the right in \eqref{eq.lem.1} can be estimated as
\begin{equation*}
\begin{aligned}
&\sum_{k=1}^{n-1} \frac{1}{k^{\gamma}(n-k) } 
\leq 
\frac{1}{n-1} + \frac{1}{(n-1)^{\gamma}} + \int_{1}^{n-1} \frac{{{{\rm d}}} x}{x^{\gamma} (n-x) }.
\end{aligned}
\end{equation*}
Splitting the integral we obtain
\begin{equation*}
\begin{aligned}
\int_{n/2}^{n-1} \frac{{{{\rm d}}} x}{x^{\gamma} (n-x) }
&\leq 
\frac{2^{\gamma}}{n^{\gamma}} \int_{n/2}^{n-1} \frac{{{{\rm d}}} x}{n-x}
\leq
2^{\gamma} \frac{\log n}{n^{\gamma}},
\\
\int_1^{n/2} \frac{{{{\rm d}}} x}{x^{\gamma} (n-x) }
&\leq 
\frac{2}{n} \int_1^{n/2} \frac{{{{\rm d}}} x}{x^{\gamma}},
\end{aligned}
\end{equation*}
where depending on $\gamma$
\begin{equation*}
\frac{2}{n} \int_1^{n/2} \frac{{{{\rm d}}} x}{x^{\gamma}} \leq 
\begin{cases}
2^{\gamma}(1-\gamma)^{-1} n^{-\gamma}, & \gamma <1, \\
2 n^{-1} \log n, & \gamma =1, \\
2 (\gamma-1)^{-1} n^{-1}, & \gamma >1.
\end{cases}
\end{equation*}
The third term in \eqref{eq.lem.1} is split as well
\begin{equation*}
\sum_{k=n+1}^{\infty} \frac{1}{k^{\gamma}(n-k) } = 
\frac{1}{(n+1)^{\gamma}} +
\sum_{k=n+2}^{2n} \frac{1}{k^{\gamma}(k-n) } + \sum_{k=2n+1}^{\infty} \frac{1}{k^{\gamma}(k-n) }
\end{equation*}
and estimated as
\begin{equation*}
\begin{aligned}
&\sum_{k=n+2}^{2n} \frac{1}{k^{\gamma}(k-n) } 
\leq 
\frac{1}{n^{\gamma}} \sum_{k=n+2}^{2n} \frac{1}{k-n } 
\leq 
\frac{1}{n^{\gamma}} \int_{n+1}^{2n}  \frac{{{{\rm d}}} x}{x-n} 
\leq  \frac{\log n}{n^{\gamma}}, 
\\
&\sum_{k=2n+1}^{\infty} \frac{1}{k^{\gamma}(k-n)} 
 \leq
2\sum_{k=2n+1}^{\infty} \frac{1}{k^{\gamma+1}} 
\leq  
2 \int_{2n}^{\infty}  \frac{{{{\rm d}}} x}{x^{\gamma +1}}
\leq 
\frac{2^{1-\gamma}}{\gamma}\frac{1}{n^{\gamma}}
,
\end{aligned}
\end{equation*}
where we used $k-n >k/2$ in the second estimate.

Combing all the inequalities and using for $\gamma = 1+\beta >1$ that 
\begin{equation}\label{log.est}
\frac{\log n}{n^{\beta}} \leq \max_{x \geq 0} \frac{x} {e^{\beta x}} = \frac{1}{\beta e}, 
\end{equation}
we obtain the claim.
\end{proof}

\begin{lemma}\label{lem.h}
Let $h>1$. Then
\begin{equation*}
\sum_{k=1}^{\infty} \frac{1}{ k^{\gamma}(k+h) }
\leq 
D(\gamma) \tau_{\gamma}(h),
\end{equation*}
where $D(\gamma)$ does not depend on $h$ and
\begin{equation*}
\tau_{\gamma}(h):=
\begin{cases}
h^{-\gamma}, & {\rm if} \ \gamma < 1, \\
h^{-1} \log h , & {\rm if} \ \gamma = 1, \\
h^{-1}, & {\rm if} \ \gamma > 1.
\end{cases}
\end{equation*}
\end{lemma}
\begin{proof}
Proof is analogous to the one of Lemma \ref{lem.sum.est.2}.
\end{proof}

\section{Examples}
\label{sec.ex}

The first example is the harmonic oscillator in $L^2({{\mathbb{R}}})$ perturbed by the $\delta$ potential placed in $x_0$ with coupling $\nu \in {{\mathbb{C}}}$, more precisely
\begin{equation}
\label{ho.def}
\begin{aligned}
A &= -\frac{{{{\rm d}}}^2}{{{{\rm d}}} x^2} + x^2,
&{{\operatorname{Dom}}}(A) & = \{ \psi \in W^{2,2}({{\mathbb{R}}}): x^2 \psi \in L^2({{\mathbb{R}}})\}, \\
a(\psi,\psi) & = \|\psi'\|^2 + \|x \psi\|^2, 
&{{\operatorname{Dom}}}(a) & = \{\psi \in W^{1,2}({{\mathbb{R}}}): x \psi \in L^2({{\mathbb{R}}}) \}, \\
b_1(\psi,\psi) & = \nu |\psi(x_0)|^2, 
&{{\operatorname{Dom}}}(b_1) & = W^{1,2}({{\mathbb{R}}}), \quad \nu \in {{\mathbb{C}}}.
\end{aligned}
\end{equation}
Eigenvalues of $A$ are $\mu_n=2n+1,$ $n \in {\mathbb{N}}$, and eigenfunctions are Hermite functions
\begin{equation}
\label{hn.expl}
h_n(x) = \frac{1}{(2^n n! \sqrt{\pi})^{1/2}}e^{-\frac{x^2}{2} }H_n(x), \quad n=0,1,2,\dots.
\end{equation}
If $x_0 =0$, the values $h_n(0)$ read, {\emph{cf.}}~\cite[Eq.22.4.8, Eq.22.2.14]{Abramowitz-1964-55},
\begin{equation*}
\begin{aligned}
h_{2n-1}(0) &= 0, \ \ 
h_{2n}(0) & = \frac{(-1)^n ((2n)!)^{\frac{1}{2}}}{ \pi^{\frac{1}{4}} 2^n n! }.
\end{aligned}
\end{equation*}
Using the Stirling formula for the factorial,
$$n! = \sqrt{2\pi n} 
\left(
\frac{n}{e}
\right)^n e^{\lambda_n}, \quad \frac{1}{12n+1} < \lambda_n < \frac{1}{12n}, $$
{\emph{cf.}}~\cite[Eq.6.1.38]{Abramowitz-1964-55}, \cite[\S 2.9]{Feller-1968}, yields
\begin{equation}\label{hn.bound}
|h_{2n}(0)| =  \frac{1+\mathcal{O}(\frac{1}{n})}{\pi^{\frac{1}{2}} n^{\frac{1}{4}}},
\end{equation}
hence the perturbation $b_1$ satisfies assumption \eqref{ass.b} if $x_0=0$. 

Further analysis, {\emph{cf.}}~\cite[p.700]{Askey-1965-87} and further references in 
\cite{Adduci-2012-10}, shows that
\begin{equation}\label{hn.est}
h_n(x) \leq C
\begin{cases}
(N^{\frac{1}{2}} (N^{\frac{1}{2}} - x ))^{-\frac{1}{4}}, 
& 0 \leq x \leq N^{\frac{1}{2}} - N^{-\frac{1}{6}},  
\\
N^{-1/12}, 
& N^{\frac{1}{2}} - N^{-\frac{1}{6}} \leq x \leq N^{\frac{1}{2}} + N^{-\frac{1}{6}},
\\ 
\frac{e^{- \xi N^{\frac{1}{4} } (x-N^{\frac{1}{2}})^{\frac{3}{2}}}}
{(N^{\frac{1}{2}} (x -N^{\frac{1}{2}}))^{\frac{1}{4}}}
,
& N^{\frac{1}{2}} + N^{-\frac{1}{6}} \leq x \leq (2N)^{\frac{1}{2}},
\\
e^{- \xi x^2}, & x \geq (2N)^{\frac{1}{2}},
\end{cases}
\end{equation}
where $N=2n+1$ and $\xi >0$. Therefore $|h_n(x_0)| \leq \tilde C n^{-\frac{1}{4}}$ if $2 N \geq x_0^2$  and we can apply our results.

The second example is again the harmonic oscillator, but we consider the infinite number of $\delta$ potentials.
\begin{example}
Let $A$ be the harmonic oscillator, {\emph{cf.}}~\eqref{ho.def}, and let 
\begin{equation}
b_2(\phi,\psi) = \sum_{k =-\infty}^{\infty} \nu_k \, \overline{\phi(p_k)} \, \psi(p_k),
\end{equation}
where points $\{p_k\}_{k \in {{\mathbb{Z}}}}$ and coupling constants $\{\nu_k\}_{k\in {{\mathbb{Z}}}}$ satisfy $p_k ={\operatorname{sgn}} k \, |k|^{\gamma}$, $0< \gamma \leq 1$, and $\nu_0 \neq 0$, $|\nu_k| = |k|^{-\beta}$, $\beta \geq 0$, respectively. Then $b_2$ satisfies the condition \eqref{ass.b}, if $\beta + \gamma >1$.
Therefore the statement of Theorem \ref{thm.RB} holds.
\end{example}
\begin{proof} 
We intend to determine the relation between $\beta$ and $\gamma$ guaranteeing that $b_2$ satisfies the assumption \eqref{ass.b}. In the first step, we use H\"older inequality and the fact that Hermite functions are either even or odd and obtain
\begin{equation*}
|b_2(h_m,h_n)| \leq 
2 \left(
\sum_{k =0}^{\infty} |\nu_k| |h_m(p_k)|^2
\right)^{1/2}
\left(
\sum_{k =0}^{\infty} |\nu_k| |h_n(p_k)|^2
\right)^{1/2}.
\end{equation*}
The estimates are based on the behaviour of Hermite functions \eqref{hn.est} and are divided into four parts.
For $ k \leq K_1$ with $K_1 \in {\mathbb{N}}$ such that $(N^{1/2} - N^{-1/6})^{1/\gamma}-1 \leq K_1 \leq (N^{1/2} - N^{-1/6})^{1/\gamma}$, we have
\begin{equation*}
\sum_{k=1}^{K_1} \frac{N^{-1/4}}{k^{\beta}(N^{1/2}-k^{\gamma})^{1/2}  } 
\leq 
 \int_1^{K_1} \frac{N^{-1/4} {{{\rm d}}} x}{x^\beta (N^{1/2}-x^\gamma)^{1/2} }
 +  \frac{2N^{-1/4}}{p_*^\beta (N^{1/2}-p_*^{\gamma})^{1/2}  },
\end{equation*} 
where $1 \leq p_* \leq K_1$ is such that the maximum of $x^{-\beta}(N^{1/2}-x^{\gamma})^{-1/2}$ is attained at $x=p_*$.  By a very crude estimate without searching for the actual value of $p_*$, the second term is $O(N^{-1/6})$. The substitution $x^{\gamma} = N^{1/2} y$ in the integral leads to
\begin{equation*}
\int_1^{K_1} \frac{ N^{-1/4}{{{\rm d}}} x}{x^\beta (N^{1/2}-x^\gamma)^{1/2} }
\leq 
\frac{N^{ \frac{1-\beta -\gamma}{2\gamma} } }{\gamma} 
\int_{N^{-1/2}}^{  1} 
\frac{ y^{ \frac{1-\beta}{\gamma} -1 } {{{\rm d}}} y}{ (1-y)^{1/2}}.
\end{equation*}
For $N \to + \infty$ the integral is $\mathcal{O}(1)$ for $0< \beta < 1$, $\mathcal{O}(\log N)$ for $\beta =1$, and $\mathcal{O}(N^{\frac{\beta-1}{2\gamma}})$ for $\beta >1$.
Taking into account the behaviour of the prefactor,  
assumption \eqref{ass.b} is satisfied if $\beta > 1-\gamma$.

In the second part, we have 
\begin{equation*}
\begin{aligned}
\sum_{k=K_1+1}^{K_2} \frac{N^{-1/6}}{k^{\beta} }
&\leq \frac{N^{-1/6}}{(K_1+1)^{\beta}} + \int_{K_1+1}^{K_2} \frac{N^{-1/6} {{{\rm d}}} x}{x^\beta},
\end{aligned}
\end{equation*} 
where $(N^{1/2} + N^{-1/6})^{1/\gamma}-1 \leq K_2 \leq (N^{1/2} + N^{-1/6})^{1/\gamma}$. The first term is $\mathcal{O}(N^{\frac{-\gamma-3\beta}{6 \gamma}})$, giving no condition on $\beta$. For $\beta \neq 1$, the integral can be estimated as
\begin{equation*}
\begin{aligned}
\int_{K_1+1}^{K_2} \frac{N^{-1/6} {{{\rm d}}} x}{x^\beta}
& \leq
\int_{(N^{1/2}-N^{-1/6})^{1/\gamma} }^{(N^{1/2}+N^{-1/6})^{1/\gamma} } \frac{N^{-1/6} {{{\rm d}}} x}{x^\beta} 
\\
&\leq \frac{N^{-1/6}}{|1-\beta|}
\left|
(N^{1/2}+N^{-1/6})^{\frac{1-\beta}{\gamma}} - (N^{1/2}-N^{-1/6})^{\frac{1-\beta}{\gamma}}
\right|
\\
& = \frac{N^{\frac{3-3\beta-\gamma}{6 \gamma}}}{|1-\beta|} 
\left|
\frac{2(1-\beta)}{\gamma}N^{-2/3} + \mathcal{O}(N^{-4/3})
\right|  
= \mathcal{O}(N^{\frac{3-3\beta-5\gamma}{6 \gamma}})
\end{aligned}
\end{equation*}
For $\beta =1$ the integral is $o(N^{-1/6})$. Therefore the condition on $\beta$ reads $\beta > 1 - 5\gamma/3$, which is less restrictive than $\beta >1-\gamma$ from the first part. Using analogous estimates, it can be verified that the conditions on $\beta$ from the third and fourth part are also weaker. Therefore, we will make the assumption $\beta > 1-\gamma$ to guarantee that the condition \eqref{ass.b} holds.
\end{proof}

In the last example we consider a form $b_3$ generated by a function potential $V$
\begin{equation}\label{V.form}
b_3(\psi,\psi):=\langle V \psi, \psi \rangle.
\end{equation}
This is in fact the case studied in \cite{Adduci-2012-10}. We assume that $V$ is in
\begin{equation*}
 L(p,\tau):= \{\phi:(1+|x|^2)^{\tau/2}|\phi(x)| \in L^p({{\mathbb{R}}}) \}.
\end{equation*}
If $p=4$, then $\|V h_n \| \leq C n^{\tau/2-1/8} \log(n+2)$. For $p\neq 4$, we have
$\|V h_n \| \leq C (n+1)^{\tau/2+t(p)}$, where
\begin{equation*}
t(p) = 
\begin{cases}
-\frac{1}{12}(2-\frac{2}{p}), & 2 \leq p <4, \\
-\frac{1}{2p}, & 4<p<\infty,
\end{cases}
\end{equation*}
see \cite[Eq.(2), Lem.5.2]{Adduci-2012-10} for details. 
If the condition $\tau/2-1/8<0$ or $\tau/2 +t(p)<0$ holds, $b_3$ satisfies the condition \eqref{ass.b}, since
\begin{equation}\label{V.form.bound}
|b_3(h_m,h_n)| = |\langle V h_m, h_n \rangle| 
\leq 
\min (\|V h_m\|,\|V h_n\|)
\leq
\|V h_m\|^{1/2} \|V h_n\|^{1/2}
.
\end{equation}

\section{Conclusions}
\label{sec.concl}

The positive results, {{\emph{i.e.}}}~the claims that the eigensystem of $T$ contains a Riesz basis, are obtained if \emph{the local subordination} condition \eqref{ass.b} (or \eqref{trad.ass.B} in \cite{Adduci-2012-10, Adduci-2012-73, Shkalikov-2010-269}) is satisfied. 
Previous works \cite{Markus-1988}, \cite{Agranovich-1994-28}, \cite{Wyss-2010-258} use the usual subordination in sense of operators or forms, {\emph{cf.}}~\eqref{form.p.sub}, but there, there are no claims on the Riesz basisness for the perturbations of the harmonic oscillator (except a weaker result of Riesz basis of subspaces for bounded perturbations, {\emph{cf.}}~\cite{Agranovich-1994-28}). The example 
$$-\frac{{{{\rm d}}}^2}{{{{\rm d}}} x^2} + x^2 + 2 {{\rm i}} x, $$
analysed in \cite{Mityagin-2013-prep}, shows that the subordination does not help to claim even the basisness of eigensystem, since in this case the perturbation $2{{\rm i}} x$ is subordinated, however, the norms of Riesz projections $P_n$ grows in $n$, namely
$$\lim_{n \to + \infty} \frac{1}{\sqrt{n}} \log \|P_n\| = 2^{3/2}.$$
In fact, various rates of spectral projection growth can be obtained by subordinated perturbations of anharmonic oscillators, {\emph{cf.}}~\cite{Mityagin-2013-prep}. Therefore, in order to obtain the positive results, a finer condition on the perturbation than the subordination should be found.

{\footnotesize
\bibliographystyle{acm}
\bibliography{C:/Data/00Synchronized/references}
}

\end{document}


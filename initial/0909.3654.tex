\documentclass[12pt]{amsart}
\usepackage[dvips]{graphicx}
\usepackage{amsmath,graphics}
\usepackage{amscd}
\usepackage{amsfonts,amssymb,xypic}
\theoremstyle{plain}
\newtheorem{theorem}{Theorem}
\newtheorem{thm}{Theorem}
\newtheorem{lem}[thm]{Lemma}
\newtheorem{lemma}[thm]{Lemma}
\newtheorem{cor}[thm]{Corollary}
\newtheorem{prop}[thm]{Proposition}
\newtheorem{conj}[thm]{Conjecture}
\newtheorem{question}[thm]{Question}
 \theoremstyle{remark}
 \newtheorem{remark}[thm]{Remark}
\textwidth 6in    
\oddsidemargin.25in    
\evensidemargin.25in     
\marginparwidth=.85in

\hyphenation{metabel-ian}

\begin{document}
\title[Metabelian SL$(n,{{\mathbb C}})$ representations of knot groups II]{Metabelian SL$(n,{{\mathbb C}})$ representations of knot groups II:  fixed points}

\author{Hans U. Boden}
\address{Department of Mathematics, McMaster University,
Hamilton, Ontario} \email{boden@mcmaster.ca}
\thanks{The first named author was supported by a grant from the Natural Sciences and Engineering Research Council of Canada.}

\author{Stefan Friedl}
\address{
University of Warwick, Coventry, UK}
\email{sfriedl@gmail.com}

\subjclass[2000]{Primary: 57M25, Secondary: 20C15}
\keywords{Metabelian representation, knot group, character variety,
 group action, fixed point}

\date{\today}
\begin{abstract}
Given a knot $K$ in an integral homology sphere ${\Sigma}$ with exterior $N_K$,
there is a natural action of the cyclic group ${{\mathbb Z}}/n$ on the space of ${{\rm SL}}(n,{{\mathbb C}})$ representations
of the knot group $\pi_1(N_K)$, and this induces an action on the ${{\rm SL}}(n,{{\mathbb C}})$ character variety.
We identify the fixed points of this action in terms of characters of metabelian representations,
and we apply this to show that the twisted Alexander polynomial $\Delta^{\alpha}_{K,1}(t)$
associated to an irreducible metabelian ${{\rm SL}}(n,{{\mathbb C}})$
representation ${\alpha}$ is actually a polynomial in $t^n$.

 \end{abstract}
\maketitle

\section{Introduction}

Suppose $K$ is a knot. Throughout this paper we will always understand this to mean that $K$ is  an oriented simple closed curve in an integral homology 3-sphere ${\Sigma}$.
We write $N_K={\Sigma}^3\smallsetminus \tau(K),$ where $\tau(K)$ denotes an open tubular neighborhood of $K$.

The study of metabelian representations and metabelian quotients of knot groups goes back to the pioneering work of
Neuwirth \cite{Ne65},  de Rham \cite{dRh68}, Burde \cite{Bu67}  and Fox \cite{Fo70}
(see also \cite[Section~14]{BZ03}).
The theory was further developed
by many authors, including Hartley \cite{Ha79,Ha83},  Livingston \cite{Li95}, Letsche \cite{Le00}, Lin \cite{Lin01}, Nagasato
\cite{Na07} and  Jebali \cite{Je08}.
In \cite{BF08} we proved a classification theorem for irreducible metabelian representations,
and in this paper we continue  our study of metabelian representations of knot groups.

We begin by introducing some terminology.
Given a topological space $M$, let $R_n(M)$ be the space
 of ${{\rm SL}}(n,{{\mathbb C}})$ representations of $\pi_1(M)$   and
 $X_n(M)$ the associated character variety.
 We use $\xi_{\alpha}$ to denote the character of the representation
 ${\alpha} {\colon\thinspace} \pi_1(M) \to {{\rm SL}}(n,{{\mathbb C}})$. We will often make use of the important fact
 that two irreducible representations
 determine the same character if and only if they are conjugate (see \cite[Corollary 1.33]{LM85}).

Now suppose $K$ is a knot.
There is an action  of the group ${{\mathbb Z}}/n$ on the representation variety
$R_n(N_K)$ given by twisting by the $n$--th roots of unity ${\omega}^k= e^{2 \pi ik/n} \in U(1)$.
(This is a special case of the more general twisting operation described in \cite[Ch. 5]{LM85}.)
More precisely, we write ${{\mathbb Z}}/n = \langle {\sigma} \mid {\sigma}^n=1\rangle$ and
set
$({\sigma} \cdot {\alpha})(g) = {\omega}^{{\varepsilon}(g)} {\alpha}(g)$ for each  $g \in \pi_1(N_K)$,
where ${\varepsilon} {\colon\thinspace} \pi_1(N_K) \to H_1(N_K) = {{\mathbb Z}}$ is determined by the given orientation of the knot.

This constructs an  action of ${{\mathbb Z}}/n$ on $R_n(N_K)$
which, in turn, descends to an action
on the character variety $X_n(N_K)$.
Our main result identifies the fixed points of
${{\mathbb Z}}/n$ in $X_n^*(N_K)$, the irreducible characters, as those associated to
metabelian representations.

\begin{thm}\label{thm1}
The character $\xi_{\alpha}$ of an irreducible representation  ${\alpha} {\colon\thinspace} \pi_1(N_K) \to {{\rm SL}}(n,{{\mathbb C}})$ is
fixed under the ${{\mathbb Z}}/n$ action if and only if ${\alpha}$ is metabelian.
\end{thm}

In proving this result, we actually characterize the entire fixed point set  $X_n(N_K)^{{{\mathbb Z}}/n}$
in terms of characters $\xi_{\alpha}$ of the metabelian representations ${\alpha}={\alpha}_{(n,\chi)}$
described in Subsection \ref{sec2-3} (see  Theorem \ref{thm4}).
When $n=2,$ it turns out that every metabelian ${{\rm SL}}(2,{{\mathbb C}})$ representation is dihedral and in this case
Theorem \ref{thm1} was first proved by F. Nagasato and Y. Yamaguchi (cf. \cite[Proposition 4.8]{NY08}).

As an application of Theorem \ref{thm1}, we prove a
result about the twisted Alexander polynomials associated  to metabelian representations.
This result was first shown by C. Herald, P. Kirk and C. Livingston in \cite{HKL08} using completely
different methods. Our approach is elementary and quite natural, and it is explained in Section  \ref{section:twialex}, where we apply it to give an answer
to a question raised by Hirasawa and Murasugi in \cite{HM09}.

\bigskip \noindent
{\bf Acknowledgments. \ } The authors would like to thank
Steven Boyer, Christopher Herald, Michael Heusener, Paul Kirk, Charles Livingston, Andrew Nicas and Adam Sikora for
generously sharing their knowledge, wisdom, and insight.
We would also like to thank Fumikazu Nagasato and Yoshikazu Yamaguchi for communicating
the results of their paper to us.

\section{The classification of metabelian representations of knot groups}

In this section we recall some results from \cite{BF08}
regarding the classification of metabelian  representations of knot groups.

\subsection{Preliminaries}
Given a group $\pi$, we shall write  $\pi^{(n)}$ for the $n$--th term of the
derived series of $\pi$. These subgroups are defined inductively by setting
$\pi^{(0)}=\pi$ and $\pi^{(i+1)}=[\pi^{(i)},\pi^{(i)}]$.
The group $\pi$ is called \emph{metabelian} if $\pi^{(2)}=\{e\}$.

Suppose $V$ is a finite dimensional vector space over ${{\mathbb C}}$.
A  representation $\varrho {\colon\thinspace} \pi\to {\operatorname{Aut}}(V)$
is called \emph{metabelian}
 if $\varrho$ factors through $\pi/\pi^{(2)}$. The representation $\varrho$ is
 called \emph{reducible}
if there exists a proper subspace $U\subset V$ invariant under $\varrho({\gamma})$ for all ${\gamma}\in  \pi.$
Otherwise $\varrho$ is called \emph{irreducible} or \emph{simple}.
If $\varrho$ is the direct sum of simple representations, then $\varrho$ is called \emph{semisimple}.

Two representations
$\varrho_1 {\colon\thinspace} \pi \to {\operatorname{Aut}}(V)$ and $\varrho_2 {\colon\thinspace} \pi \to {\operatorname{Aut}}(W)$ are called \emph{isomorphic}
if there exists an isomorphism $\phi {\colon\thinspace} V\to W$ such that $\phi^{-1} \circ \varrho_1(g)\circ \phi=\varrho_2(g)$ for all $g\in \pi$.

\subsection{Metabelian quotients of knot groups}\label{section:metabk}

Let $K\subset {\Sigma}^3$ be a knot in an integral homology 3-sphere.
In the following we denote by $\widetilde{N}_K$
the infinite cyclic cover of $N_K$ corresponding to the abelianization $\pi_1(N_K)\to H_1(N_K) \cong {{\mathbb Z}}$.
Therefore $\pi_1(\widetilde{N}_K)=\pi_1(N_K)^{(1)}$ and
$$ H_1(N_K;{{\mathbb Z}}[t^{\pm 1}])=H_1(\widetilde{N}_K) \cong \pi_1(N_K)^{(1)}/\pi_1(N_K)^{(2)}.$$
The ${{\mathbb Z}}[t^{\pm 1}]$--module structure is
given on the right hand side
by $t^n\cdot g:=\mu^{-n}g\mu^n$, where $\mu$ is a meridian
of $K$.

For a knot $K$, we set $\pi:=\pi_1(N_K)$ and consider the short exact sequence
$$ 1\to \pi^{(1)}/\pi^{(2)}\to \pi/\pi^{(2)}\to\pi/\pi^{(1)}\to 1. $$
Since $\pi/\pi^{(1)}=H_1(N_K)\cong {{\mathbb Z}}$, this sequence splits and we get isomorphisms
$$ \begin{array}{rcccl} \pi/\pi^{(2)}
&\cong & \pi/\pi^{(1)} \ltimes \pi^{(1)}/\pi^{(2)}
&\cong &{{\mathbb Z}} \ltimes \pi^{(1)}/\pi^{(2)}   \cong {{\mathbb Z}} \ltimes H_1(N_K;{{\mathbb Z}}[t^{\pm 1}]) \\
    g &\mapsto &(\mu^{{\varepsilon}(g)},\mu^{-{\varepsilon}(g)}g) &\mapsto &({\varepsilon}(g),\mu^{-{\varepsilon}(g)}g), \end{array}  $$
where the semidirect products are taken with respect to the ${{\mathbb Z}}$ actions defined by
letting $n \in {{\mathbb Z}}$ act by conjugation by $\mu^n$ on $\pi^{(1)}/\pi^{(2)}$ and by multiplication
by $t^n$ on $H_1(N_K; {{\mathbb Z}}[t^{\pm1}])$.

\subsection{Irreducible metabelian ${{\rm SL}}(n, {{\mathbb C}})$ representations of knot groups}
\label{sec2-3}
Let $K$ be a knot. We write $H=H_1(N_K;{{\mathbb Z}}[t^{\pm 1}])$.
The discussion of the previous section shows that irreducible metabelian ${{\rm SL}}(n,{{\mathbb C}})$ representations of $\pi_1(N_K)$ correspond precisely to the irreducible  ${{\rm SL}}(n,{{\mathbb C}})$ representations of ${{\mathbb Z}}\ltimes H$.

Let
$\chi {\colon\thinspace} H\to {{\mathbb C}}^*$ be a character which factors through $H/(t^n-1)$ and suppose $z\in S^1$ with $z^n=(-1)^{n+1}$. Then it follows from
\cite[Section~3]{BF08} that, for $(j, h) \in {{\mathbb Z}}\ltimes H,$ setting
$$  {\alpha}_{(\chi,z)} (j,h) =
 \begin{pmatrix}
 0& &\dots &z \\
 z&0&\dots &0 \\
\vdots &\ddots &\ddots&\vdots \\
     0&\dots &z &0 \end{pmatrix}^j
     \begin{pmatrix} \chi(h) &0&\dots &0 \\
 0&\chi(th) &\dots &0 \\
\vdots &&\ddots &\vdots \\ 0&0&\dots &\chi(t^{n-1}h) \end{pmatrix}
$$
defines an  ${{\rm SL}}(n,{{\mathbb C}})$ representation whose isomorphism type of this representation does not depend on the choice of $z$.
In our notation we will not normally
distinguish between metabelian representations
of $\pi_1(N_K)$ and representations of $ {{\mathbb Z}} \ltimes H$.

In the following we say that a character $\chi {\colon\thinspace} H\to {{\mathbb C}}^*$ has \emph{order $n$} if
it factors through $H/(t^n-1)$, but not through $H/(t^\ell-1)$ for any $\ell < n$.
 Given a character  $\chi {\colon\thinspace} H\to {{\mathbb C}}^*$, let $t^i\chi$ be the character defined by $(t^i\chi)(h)=\chi(t^ih)$.
Any character $\chi {\colon\thinspace} H\to {{\mathbb C}}^*$ which factors through $H/(t^n-1)$ must have order $k$ for some
divisor $k$ of $n$.
The following is a combination of \cite[Lemma~2.2]{BF08} and  \cite[Theorem~3.3]{BF08}.

\begin{thm} \label{thm2}
Suppose  $\chi {\colon\thinspace} H  \to {{\mathbb C}}^*$ is a character that factors through $H/(t^n-1)$.
\begin{enumerate}
\item[(i)] ${\alpha}_{(n,\chi)}{\colon\thinspace} {{\mathbb Z}}\ltimes H\to {{\rm SL}}(n,{{\mathbb C}})$ is irreducible if and only if the character $\chi$ has order $n$.
\item[(ii)] Given two characters $\chi,\chi'{\colon\thinspace} H\to {{\mathbb C}}^*$ of order $n$, the representations
${\alpha}_{(n,\chi)}$ and ${\alpha}_{(n,\chi')}$ are conjugate if and only if $\chi=t^k\chi'$ for some $k$.
\item[(iii)] For any irreducible representation  ${\alpha} {\colon\thinspace}{{\mathbb Z}}\ltimes H\to {{\rm SL}}(n,{{\mathbb C}})$
 there exists a character $\chi {\colon\thinspace} H\to {{\mathbb C}}^*$ of order $n$ such that
${\alpha}$ is conjugate to ${\alpha}_{(n,\chi)}$.
\end{enumerate}
\end{thm}

\section{Main results}
\subsection{Metabelian characters as fixed points}

Set ${\omega} = e^{2 \pi i/n}$ and recall the action
of the cyclic group ${{\mathbb Z}}/n =\langle {\sigma} \mid {\sigma}^n=1\rangle$ on
 representations ${\alpha} {\colon\thinspace} \pi_1(N_K) \to {{\rm SL}}(n,{{\mathbb C}})$ obtained by
  setting $({\sigma} \cdot {\alpha})(g)= {\omega}^{{\varepsilon}(g)} {\alpha}(g)$ for all $g \in \pi_1(N_K),$
where ${\varepsilon} {\colon\thinspace} \pi_1(N_K) \to H_1(N_K)={{\mathbb Z}}$.

We begin with the following  lemma.

\begin{lemma}\label{lem3}
Suppose ${\alpha} {\colon\thinspace} \pi_1(N_K) \to {{\rm SL}}(n,{{\mathbb C}})$ is a representation
whose associated character $\xi_{\alpha} \in X_n(N_K)$ is a fixed point
of the ${{\mathbb Z}}/n$ action. Then up to conjugation,
we have
\begin{equation} \label{eq2}
{\alpha}(\mu) =
\begin{pmatrix} 0& &\dots &z \\
z&0&\dots &0 \\
\vdots &\ddots &\ddots&\vdots \\
     0&\dots &z &0 \end{pmatrix},
     \end{equation}
for some (in fact any) $z \in U(1)$ such that $z^{n}=(-1)^{n+1}.$\end{lemma}

\begin{proof}
Let $c(t) = \det ({\alpha}(\mu)-tI)$ denote the characteristic polynomial
of ${\alpha}(\mu),$ which we can write as
$$c(t) = (-1)^n t^n + c_{n-1} t^{n-1} + \cdots + c_1 t +1.$$
Note that $c(t)$ is determined by the character $\xi_{\alpha} \in X_n(N_K)$,
and so assuming $\xi_{\alpha}$ is a fixed point of the ${{\mathbb Z}}/n$ action, we conclude that
${\alpha}(\mu)$ and ${\omega}^k {\alpha}(\mu)$ have the same characteristic polynomials for all $k$.
In particular,
\begin{eqnarray*}
c(t)&=&\det({\alpha}(\mu)-tI)\\
&=&\det ({\omega}^{-1} {\alpha}(\mu)-tI) \\
&=& \det ({\omega}^{-1} {\alpha}(\mu)-({\omega}^{-1} {\omega})tI)\\
&=&\det ({\omega}^{-1} I) \det ({\alpha}(\mu)- {\omega} tI) \\
&=&  \det ({\alpha}(\mu)-t {\omega} I) = c({\omega} t).
\end{eqnarray*}
However, ${\omega}^k \neq 1$ unless $n | k$, and this implies
$0=c_{n-1}=c_{n-2} = \cdots = c_1$ and $c(t) = (-1)^n t^n +1.$
In particular the matrix ${\alpha}(\mu)$ and the matrix appearing in
Equation (\ref{eq2}) have the same set of $n$ distinct eigenvalues. This implies that the two matrices are conjugate.
\end{proof}

In order to prove Theorem \ref{thm1}, we establish the following more
general result.
\begin{thm} \label{thm4}
The fixed point set  of the ${{\mathbb Z}}/n$ action on $X_n(N_K)$
consists of characters $\xi_{\alpha}$ of the metabelian representations
${\alpha} = {\alpha}_{(n,\chi)}$ described in Section \ref{sec2-3}. In other words,
$$X_n(N_K)^{{{\mathbb Z}}/n} = \{ \xi_{\alpha} \mid {\alpha} ={\alpha}_{(n,\chi)} \text{ for } \chi {\colon\thinspace} H_1(N_K; {{\mathbb Z}}[t^{\pm 1}]) \to {{\mathbb C}}^*\}.$$
\end{thm}

Notice that Theorem~\ref{thm1} can be viewed as the
special case of Theorem \ref{thm4} where
${\alpha}$ is irreducible. (Recall that irreducible representations are conjugate if and only if they define the same character.)
Notice further that not every reducible metabelian
representation is of the form ${\alpha}_{(n,\chi)}$.

\begin{proof}

We first show that if ${\alpha} {\colon\thinspace} \pi_1(N_K) \to {{\rm SL}}(n,{{\mathbb C}})$ is given as ${\alpha} = {\alpha}_{(n,\chi)},$
then ${\sigma} \cdot {\alpha}$ is conjugate to ${\alpha}$. This of course implies that $\xi_{\alpha} = \xi_{{\sigma} \cdot {\alpha}}$.

Assume then that ${\alpha} = {\alpha}_{(n,\chi)}.$
Then we have $${\alpha}(\mu) =\begin{pmatrix}
0& \dots& &z \\
z& 0&\dots&0 \\
\vdots &\ddots &\ddots&\vdots \\
     0&\dots &z &0 \end{pmatrix},$$
     where $z$ satisfies $z^{n}= (-1)^{n+1}.$
Further, ${\alpha}(g)$ is diagonal for all $g\in [\pi_1(N_K), \pi_1(N_K)]$.
By definition of ${\sigma} \cdot {\alpha},$ we  see that
$$({\sigma}\cdot{\alpha}) (\mu) = {\omega} {\alpha}(\mu)=
\begin{pmatrix} 0& \dots &&{\omega} z \\
{\omega} z &0&\dots &0 \\
\vdots &\ddots &\ddots&\vdots \\
     0&\dots &{\omega} z &0 \end{pmatrix}$$
     and that $({\sigma} \cdot {\alpha})(g) = {\alpha}(g)$
     for all $g \in [\pi_1(N_K), \pi_1(N_K)]$.
It follows easily  from Theorem \ref{thm2} (2)
that ${\sigma} \cdot {\alpha}$ and ${\alpha}_{(n,\chi)}$ are conjugate;
however it is easy to see this directly too.
Simply take   $$P = \begin{pmatrix} 1&&& 0 \\
 &{\omega} \\
&&\ddots  \\
     0&&&{\omega}^{n-1} \end{pmatrix},$$
     and compute that
     ${\sigma} \cdot {\alpha} =P {\alpha} P^{-1}$ as claimed.

We now show the other implication, namely that each point
$\xi \in X_n(N_K)^{{{\mathbb Z}}/n}$ in the fixed point set
can be represented as the character $\xi=\xi_{\alpha}$
of a metabelian representation ${\alpha} = {\alpha}_{(n,\chi)},$
where $\chi {\colon\thinspace} H_1(N_K; {{\mathbb Z}}[t^{\pm 1}]) \to {{\mathbb C}}^*$
is a character that factors through $H_1(N_K; {{\mathbb Z}}[t^{\pm 1}])/(t^n-1)$,
hence has order $k$ for some $k$ dividing $n$.
(Note that Theorem \ref{thm2} (1) tells us that
${\alpha}_{(n,\chi)}$ is irreducible if and only
if $\chi$ has order $n$.)

By the general results on representation spaces and character varieties (see \cite{LM85}),
it follows that every point in the character variety $X_n(N_K)$ can be represented
as $\xi_{\alpha}$ for some semisimple representation ${\alpha} {\colon\thinspace} \pi_1(N_K) \to {{\rm SL}}(n,{{\mathbb C}})$.
Further, two semisimple representations ${\alpha}_1$ and ${\alpha}_2$
determine the same character if and only if ${\alpha}_1$ is conjugate to ${\alpha}_2.$
(This is evident from the fact that the orbits of the semisimple representations under conjugation
are closed.)

Given $\xi \in X_n(N_K)^{{{\mathbb Z}}/n}$,
we can therefore suppose that $\xi=\xi_{\alpha}$ for some semisimple representation ${\alpha}$.
Clearly ${\sigma} \cdot {\alpha}$ is also semisimple, and
since $\xi_{\alpha} = \xi_{{\sigma} \cdot {\alpha}}$, we conclude from the above that ${\alpha}$ and ${\sigma} \cdot {\alpha}$
are conjugate representations. This means  that there exists a matrix $A  \in {{\rm SL}}(n,{{\mathbb C}})$
such that $A {\alpha} A^{-1} = {\sigma} \cdot {\alpha}$,
in other words, for all $g\in \pi_1(N_K),$ we have
\begin{equation} \label{eq2} A {\alpha}(g) A^{-1} = {\omega}^{{\varepsilon}(g)} {\alpha}(g).\end{equation}
Lemma \ref{lem3} implies
${\alpha}(\mu)$ is conjugate to  the matrix in Equation (\ref{eq2}).
It is convenient to conjugate ${\alpha}$ so that ${\alpha}(\mu)$ is diagonal,
meaning that
$${\alpha}(\mu)= \begin{pmatrix} z&&& 0 \\
 &{\omega} z  \\
&&\ddots  \\
     0&&&{\omega}^{n-1} z \end{pmatrix},
$$
where $z$ satisfies $z^n=(-1)^{n+1}$.

We now apply (\ref{eq2}) to the meridian to conclude that
$$A {\alpha}(\mu) = {\omega} {\alpha}(\mu) A,$$
 which implies
$A = (a_{ij})$ satisfies $a_{ij}=0$ unless $j = i+1 \mod(n).$
Thus, we see that
$$A = \begin{pmatrix} 0&{\lambda}_1&0& \dots &0 \\
0&0&{\lambda}_2&\dots &0 \\
 \vdots& \vdots & \ddots& \ddots &\vdots \\
 0&0&\dots&0&{\lambda}_{n-1}\\
 {\lambda}_n & 0 & \dots &0 &0
\end{pmatrix}$$
for some ${\lambda}_1, \ldots, {\lambda}_n$ satisfying ${\lambda}_1 \cdots {\lambda}_n = (-1)^{n+1}.$

It is completely straightforward to see that the characteristic polynomial of
$A$ is given by
$$\det (A-tI) =
(-1)^n(t^n - (-1)^{n+1}).$$
From this, we conclude that $A$ has as its eigenvalues the
$n$ distinct $n$--th roots of $(-1)^{n+1}.$ In particular, the subset of ${{\rm SL}}(n,{{\mathbb C}})$
of matrices that commute with $A$ is just a copy of the unique maximal torus
$T_A \cong ({{\mathbb C}}^*)^{n-1}$ containing $A$.

For any $g \in [\pi_1(N_K), \pi_1(N_K)]$, we have
${\alpha}(g) = ({\sigma} \cdot {\alpha})(g)$. Thus it follows that
$A {\alpha}(g) A^{-1} = {\alpha}(g),$ and this implies that ${\alpha}(g) \in T_A$ for
all $g \in [\pi_1(N_K), \pi_1(N_K)]$. This shows that
the restriction of ${\alpha}$ to the commutator subgroup
$[\pi_1(N_K), \pi_1(N_K)]$ is abelian, and we conclude from this that
${\alpha}$ is indeed metabelian. Notice that this, and an application of  Theorem \ref{thm2} (3),
completes the proof in the case ${\alpha}$ is irreducible.

In the general case, it follows from the discussion in Section  \ref{section:metabk} that
${\alpha}$ factors through ${{\mathbb Z}}\ltimes H_1(N_K;{{\mathbb Z}}[t^{\pm 1}])$.
Let $H =  H_1(N_K;{{\mathbb Z}}[t^{\pm 1}])$.
Given a character $\chi {\colon\thinspace} H\to{{\mathbb C}}^*$ we define the associated weight space $V_\chi$ by setting
$$V_{\chi}=\{ v\in {{\mathbb C}}^n \, |\, \chi(h)\cdot v={\alpha}(h)v \text{ for all }h\in H\}.$$
Recall that  $A\cdot {\alpha}(h)\cdot A^{-1}={\alpha}(h)$ for any $h\in H$. It is straightforward so show that $A$ restricts to an automorphism of $V_\chi$.
Since $H$ is abelian there exists at least one character $\chi{\colon\thinspace} H\to {{\mathbb C}}^*$ such that
$V_{\chi}$ is non--trivial. For any $i$ we denote by $t^i\chi$ the character given by $(t^i\chi)(h)=\chi(t^i h), h\in H$.

Note that $A$ has $n$ distinct eigenvalues and therefore is diagonalizable.
Since $A$ restricts to an automorphism of $V_\chi$, there is an
eigenvector $v$ of $A$ which lies in $V_{\chi}$. Let ${\lambda}$ be the corresponding eigenvalue. By the proof of \cite[Theorem~2.3]{BF08},
the map ${\alpha}(\mu)$ induces an isomorphism $V_{\chi}\to V_{t\chi}$.
We now calculate
\[ A\cdot {\alpha}(\mu)  v=(A{\alpha}(\mu)A^{-1})\cdot Av={\omega} {\alpha}(\mu) \cdot {\lambda} v={\lambda} {\omega} \cdot {\alpha}(\mu)v,\]
i.e.  ${\alpha}(\mu) v \in V_{t\chi}$ is an eigenvector of $A$ with eigenvalue ${\omega} {\lambda}$.

Iterating this argument, we see that ${\alpha}(\mu)^iv$ lies in $V_{t^i\chi}$ and is an eigenvector of $A$ with eigenvalue ${\omega}^i {\lambda}$.
Since ${\omega}$ is a primitive $n$--th root of unity, the eigenvalues ${\lambda},{\omega} {\lambda},\dots,{\omega}^{n-1}{\lambda}$ are all distinct, and this implies that
the corresponding eigenvectors $v, {\alpha}(\mu)v, \dots, {\alpha}(\mu)^{n-1}v$  form a basis for ${{\mathbb C}}^n$.

Let $m$ be the order of $\chi$, i.e. $m$ is the minimal number such that $\chi=t^m\chi$.
By the above we see that ${{\mathbb C}}^n$ is generated by $V_{\chi}, V_{t \chi}, \dots,V_{t^m\chi}$.
Since the characters $\chi, t\chi, \dots,t^m\chi$ are pairwise distinct, it follows  that  ${{\mathbb C}}^n$ is given as the direct sum $V_{\chi}\oplus V_{t\chi}\oplus \dots \oplus V_{t^{m-1}\chi}$.

We write $k=\dim_{{\mathbb C}}(V_{\chi})$ and note that $n=km$. We note further that ${\alpha}(\mu)^m$ has   eigenvalues given by the set
\begin{equation}\label{eq3}
\{z^m,z^me^{2\pi i/k},\dots,z^me^{2\pi i(k-1)/k}\},
\end{equation}
and each eigenvalue has multiplicity $m$.
Clearly ${\alpha}(\mu)^m$ restricts to an automorphism of $V_{t^i\chi}$ for $i=0,\dots,m -1$, and
equally clearly we see that the restrictions all give conjugate representations.
This implies that the restriction of ${\alpha}(\mu)^m$ to $V_{\chi}$ has  eigenvalues in the
set (\ref{eq3}) above, each occurring with multiplicity $1$.
In particular we can find a basis
 $\{v_1,\dots,v_k\}$  for $V_\chi$ in which the matrix of ${\alpha}(\mu)^m$ has the form
 $${\alpha}(\mu^m) =\begin{pmatrix}
0& \dots& &z^m \\
z^m& 0&\dots&0 \\
\vdots &\ddots &\ddots&\vdots \\
     0&\dots &z^m &0 \end{pmatrix}.$$
It is now straightforward to verify that with respect to the ordered basis
$$\left\{ \begin{array}{cccc}
v_1,&z^{-1}{\alpha}(\mu)v_1,&\dots,&z^{-(m-1)}{\alpha}(\mu)^{m-1}v_1,\\
v_2,&z^{-1}{\alpha}(\mu)v_2,&\dots,&z^{-(m-1)}{\alpha}(\mu)^{m-1}v_2,\\
\vdots &\vdots &\dots& \vdots \\
v_k,&z^{-1}{\alpha}(\mu)w_k,&\dots,&z^{-(m-1)}{\alpha}(\mu)^{m-1}v_k\end{array}\right\} ,$$
${\alpha}$ is given by ${\alpha}(n,\chi)$.
\end{proof}

\subsection{Application to twisted Alexander polynomials}\label{section:twialex}
As an application, we now prove the following result regarding twisted Alexander polynomials of knots
corresponding to metabelian representations.
In the following, we use  $\Delta^{\alpha}_{K,i}(t)$ to denote the $i$--th twisted Alexander
polynomial for a given representation ${\alpha} {\colon\thinspace} \pi_1(N_K) \to {{\rm SL}}(n,{{\mathbb C}})$ as presented
in \cite{FV09}.

\begin{prop} \label{prop5}
Let ${\alpha}$ be  a metabelian  representation of the form
${\alpha}={\alpha}_{(n,\chi)} {\colon\thinspace} \pi_1(N_K) \to {{\rm SL}}(n,{{\mathbb C}})$. Then
\[  \Delta^{\alpha}_{K,0}(t) =\left\{ \begin{array}{rl} 1-t^n, &\mbox{ if $\chi$ is trivial}, \\ 1,& \mbox{ otherwise.}\end{array} \right.\]
Furthermore
the twisted Alexander polynomial
 $\Delta^{\alpha}_{K,1}(t)$ is actually a polynomial
in $t^n.$
\end{prop}

\begin{remark}
In their paper \cite{HKL08},  C. Herald, P. Kirk, and C. Livingston
prove the same result using an entirely different approach (cf. p.~10 of \cite{HKL08}).
We also point out that Proposition \ref{prop5} gives a positive answer to Conjecture A from a
recent paper by M. Hirasawa and K. Murasugi (see  \cite{HM09}).
\end{remark}

\begin{proof}
The proof of the first statement is not difficult. It is immediate
when $\chi$ is trivial, and it follows by a direct calculation when
 $\chi$ is non--trivial.

We now turn to the proof of the second statement.
For ${\theta} \in U(1)$ and any representation
${\beta}:�\pi_1(N_K) \to {{\rm GL}}(n,{{\mathbb C}})$, define the ${\theta}$-twist of ${\beta}$
to be the representation sending $g \in \pi_1(N_K)$ to
${\theta}^{{\varepsilon}(g)} {\beta}(g),$ where ${\varepsilon}{\colon\thinspace} \pi_1(N_K) \to {{\mathbb Z}}$ is
 determined by the orientation of $K$.
We denote the newly obtained representation
by ${\beta}_{\theta} {\colon\thinspace} \pi_1(N_K) \to {{\rm GL}}(n,{{\mathbb C}})$.
Note that in case ${\alpha}:�\pi_1(N_K) \to {{\rm SL}}(n,{{\mathbb C}})$ and
${\theta}= e^{2 \pi ik/n}$ is an $n$-th root of unity,
${\alpha}_{\theta}$ is again an ${{\rm SL}}(n,{{\mathbb C}})$ representation.
The proof of the proposition relies on the
formula
\begin{equation} \label{eq4}
\Delta_{K,1}^{{\beta}_{\theta}}(t) = \Delta_{K,1}^{\beta}({\theta} t).
\end{equation}
This formula is well-known and follows directly from the definition of the twisted Alexander
polynomial.
Equation (\ref{eq4}) combines with Theorem \ref{thm1} to complete the proof, as we now explain.
Take ${\omega} = e^{2 \pi i/n}$. If ${\alpha}={\alpha}_{(n,\chi)}$ is metabelian,
then Theorem \ref{thm1} shows that its conjugacy class is fixed under the ${{\mathbb Z}}/n$
action. In particular, since ${\alpha}$ and ${\alpha}_{\omega}$ are
conjugate, Equation (\ref{eq4}) shows that
$$\Delta^{\alpha}_{K,1}(t) =�\Delta^{{\alpha}_{\omega}}_{K,1}(t)=�\Delta^{\alpha}_{K,1}({\omega} t).$$
Expanding
$\Delta^{\alpha}_{K,1}(t)= \sum a_i t^i$
and using the fact that $t^k = ({\omega} t)^k$ if and only if $k$ is a multiple
of $n$,
this shows that $a_k =0$ unless $k$ is a
multiple of $n$ and this completes the proof.
\end{proof}

\begin{thebibliography}{100000}

\bibitem[BF08]{BF08}
H. U. Boden and S. Friedl,
{\em Metabelian ${{\rm SL}}(n,{{\mathbb C}})$ representations of knot groups,} Pacific J. Math.  {\bf 238} (2008), 7--25.

\bibitem[Bu67]{Bu67}
G. Burde,  {\em Darstellungen von Knotengruppen},  Math. Ann.  173  (1967) 24--33

\bibitem[Bu90]{Bu90}
G. Burde, {Knots,} {\em $SU(2)$-representation spaces for two-bridge knot groups,}
Math. Ann. {\bf 288} (1990), no. 1, 103--119

\bibitem[BZ03]{BZ03}
G. Burde and H. Zieschang, {\em Knots}, Second edition. de Gruyter Studies in Mathematics, 5. Walter de Gruyter \& Co., Berlin, 2003.

\bibitem[dRh68]{dRh68}
G. de Rham, {\em Introduction aux polyn\^omes d'un n{\oe}ud}, Enseignement Math. (2)  13, 187--194 (1968).

\bibitem[Fo70]{Fo70}
R. H. Fox, {\em Metacyclic invariants of knots and links},  Canad. J. Math.  22  1970 193--201.

\bibitem[FV09]{FV09} S. Friedl and S. Vidussi,
{\em A survey of twisted Alexander polynomials}, 2009 preprint, {math.GT/0905.0591}.
\bibitem[Ha79]{Ha79}
R. Hartley, {\em Metabelian representations of knot groups},  Pacific J. Math.  82  (1979), no. 1, 93--104.
\bibitem[Ha83]{Ha83}
R. Hartley, {\em Lifting group homomorphisms},  Pacific J. Math.  105  (1983), no. 2, 311--320.

\bibitem[HKL08]{HKL08}
C. Herald, P. Kirk, and C. Livingston,
{\em Metabelian representations, twisted Alexander polynomials, knot slicing,
and mutation},  2008 preprint, to appear in Math. Z.
{math.GT/0804.1355}.
\bibitem[HM09]{HM09}
M. Hirasawa and K. Murasugi, {\em  Twisted Alexander polynomials of 2-bridge knots associated to metabelian representations}, 2009 preprint {math.GT  0903.1689}

\bibitem[Je08]{Je08}
H. Jebali, {\em Module d'Alexander et repr\'esentations m\'etab\'eliennes}, Annales de la facult\'e des sciences de Toulouse S\'er. 6, 17 no. 4 (2008), p. 751-764.

\bibitem[Le00]{Le00}
C. Letsche, {\em An obstruction to slicing knots using the eta invariant},
Math. Proc. Cambridge Philos. Soc. 128 (2000), no. 2, 301--319.
\bibitem[Lin01]{Lin01}
X. S. Lin, {\em Representations of knot groups and twisted Alexander polynomials},  Acta Math. Sin. (Engl. Ser.)  17  (2001),  no. 3, 361--380.

\bibitem[Li95]{Li95}
C. Livingston, {\em  Lifting representations of knot groups},  J. Knot Theory Ramifications  4  (1995),  no. 2, 225--234.

\bibitem[LM85]{LM85}
A. Lubotzky and A. R. Magid, {\em Varieties of representations of finitely generated groups}, Memoirs of the Amer. Math. Soc. Vol.  58, No. 336 (1985).
\bibitem[Na07]{Na07}
F. Nagasato,
{\em Finiteness of a section of the ${{\rm SL}}(2,{{\mathbb C}})$--character variety of knot groups},  Kobe J. Math., {\bf 24} no. 2 (2007) 125--136.
\bibitem[NY08]{NY08}
F. Nagasato and Y. Yamaguchi,
{\em On the geometry of a certain slice of the character variety of a knot group}, 2008 preprint,
{math.GT/0807.0714}.
\bibitem[Ne65]{Ne65}
L. P. Neuwirth, {\em Knot groups}, Annals of Mathematics Studies, No. 56 Princeton University Press, Princeton, N.J. (1965)

\end{thebibliography}
\end{document}


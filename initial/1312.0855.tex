\documentclass{amsart}
\usepackage{amsfonts,amsmath,amstext,amsthm,amssymb}

\newtheorem*{theorem}{Theorem}
\newtheorem{lemma}{Lemma}

\numberwithin{equation}{section}

\begin{document}
\author{G. G\'at, U. Goginava,  and G. Karagulyan}

\title{On everywhere divergence of the strong $\Phi$-means of Walsh-Fourier series}
\address{G. G\'at, Institute of Mathematics and Computer Science, College of
Ny\'\i regyh\'aza, P.O. Box 166, Nyiregyh\'aza, H-4400 Hungary}
\email{gatgy@nyf.hu}
\address{U. Goginava, Department of Mathematics, Faculty of Exact and
Natural Sciences, Tbilisi State University, Chavchavadze str. 1, Tbilisi
0128, Georgia}
\email{zazagoginava@gmail.com}

\address{G. Karagulyan, Institute of Mathematics of Armenian National Academy
of Science, Bughramian Ave. 24b, 375019, Yerevan, Armenia}
\email{g.karagulyan@yahoo.com}
\subjclass[2010]{42C10, 42A24}\keywords{ Walsh series, strong summability, everywhere divergent Walsh-Fourier series}
\thanks{The first author is supported by project T\'AMOP-4.2.2.A-11/1/KONV-2012-0051}

\begin{abstract}
Almost everywhere strong  exponential summability of Fourier series in Walsh and trigonometric systems established by Rodin in 1990. We prove, that if the growth of a function $\Phi(t):[0,\infty)\to[0,\infty)$ is bigger than the exponent, then the strong $\Phi$-summability of a Walsh-Fourier series can fail everywhere. The analogous theorem for trigonometric system was proved before by one of the author of this paper.
\end{abstract}
\maketitle
\section{Introduction}
In the study of almost everywhere convergence and summability  of Fourier series the trigonometric and Walsh systems have many common properties. Kolmogorov \cite{Kol} in 1926 gave a first example of everywhere divergent trigonometric Fourier series. Existence of almost everywhere divergent Walsh-Fourier series first proved by Stein \cite{Ste}. Then Schipp in \cite{Sch1} constructed an example of everywhere divergent Walsh-Fourier series. A significant complement to these divergence theorems are investigations on almost everywhere summability of Fourier series.

Let $\Phi(t):[0,\infty)\to[0,\infty)$, $\Phi(0)=0$, be an increasing continuous function. A numerical series with partial sums $s_1,s_2,\ldots $ is said to be (strong) $\Phi$-summable to a number $s$, if
{\ifx01
\begin{equation*} \label{0-1}
\lim_{n\to\infty}\frac{1}{n}\sum_{k=1}^n\Phi\left(|s_k-s|\right)=0.
 \end{equation*}\fi  
\ifx11\begin{equation}\label{0-1}
\lim_{n\to\infty}\frac{1}{n}\sum_{k=1}^n\Phi\left(|s_k-s|\right)=0.
\end{equation}\fi   
\ifx21\begin{align*}\label{0-1}
\lim_{n\to\infty}\frac{1}{n}\sum_{k=1}^n\Phi\left(|s_k-s|\right)=0.
\end{align*}\fi   
\ifx31\begin{align}\label{0-1}
\lim_{n\to\infty}\frac{1}{n}\sum_{k=1}^n\Phi\left(|s_k-s|\right)=0.
\end{align}\fi    
\ifx41\begin{gather*}\label{0-1}
\lim_{n\to\infty}\frac{1}{n}\sum_{k=1}^n\Phi\left(|s_k-s|\right)=0.
\end{gather*}\fi  
\ifx51\begin{gather}\label{0-1}
\lim_{n\to\infty}\frac{1}{n}\sum_{k=1}^n\Phi\left(|s_k-s|\right)=0.
\end{gather}\fi   
\ifx61\begin{multline*}\label{0-1}
\lim_{n\to\infty}\frac{1}{n}\sum_{k=1}^n\Phi\left(|s_k-s|\right)=0.
\end{multline*}\fi  
\ifx71\begin{multline}\label{0-1}
\lim_{n\to\infty}\frac{1}{n}\sum_{k=1}^n\Phi\left(|s_k-s|\right)=0.
\end{multline}\fi  
\ifx81\begin{multline*}\begin{split}\label{0-1}
\lim_{n\to\infty}\frac{1}{n}\sum_{k=1}^n\Phi\left(|s_k-s|\right)=0.
\end{split}\end{multline*}\fi
\ifx91\begin{multline}\begin{split}\label{0-1}
\lim_{n\to\infty}\frac{1}{n}\sum_{k=1}^n\Phi\left(|s_k-s|\right)=0.
\end{split}\end{multline}\fi
}
We note that the condition {(\ref{{0-1}})} is as strong as rapidly growing is $\Phi$, and in the case of $\Phi(t)=t^p$, $p>0$, the condition {(\ref{{0-1}})} coincides with $H^p$-summability, well known in the theory of Fourier series. Marcinkiewicz-Zygmund in \cite{Mar}, \cite{Zyg} established almost everywhere $H^p$-summability for arbitrary trigonometric Fourier series (ordinary and conjugate). K.~I.~Oskolkov in \cite{Osk} proved a.e. $\Phi$-summability for trigonometric Fourier series if $\Phi(t)=O(t/\log\log t)$. Then V.~Rodin \cite{Rod1} established the analogous with $\Phi$ satisfying the condition
{\ifx01
\begin{equation*} \label{0-2}
\limsup_{t\to\infty}\frac{\log\Phi(t)}{t}<\infty,
 \end{equation*}\fi  
\ifx11\begin{equation}\label{0-2}
\limsup_{t\to\infty}\frac{\log\Phi(t)}{t}<\infty,
\end{equation}\fi   
\ifx21\begin{align*}\label{0-2}
\limsup_{t\to\infty}\frac{\log\Phi(t)}{t}<\infty,
\end{align*}\fi   
\ifx31\begin{align}\label{0-2}
\limsup_{t\to\infty}\frac{\log\Phi(t)}{t}<\infty,
\end{align}\fi    
\ifx41\begin{gather*}\label{0-2}
\limsup_{t\to\infty}\frac{\log\Phi(t)}{t}<\infty,
\end{gather*}\fi  
\ifx51\begin{gather}\label{0-2}
\limsup_{t\to\infty}\frac{\log\Phi(t)}{t}<\infty,
\end{gather}\fi   
\ifx61\begin{multline*}\label{0-2}
\limsup_{t\to\infty}\frac{\log\Phi(t)}{t}<\infty,
\end{multline*}\fi  
\ifx71\begin{multline}\label{0-2}
\limsup_{t\to\infty}\frac{\log\Phi(t)}{t}<\infty,
\end{multline}\fi  
\ifx81\begin{multline*}\begin{split}\label{0-2}
\limsup_{t\to\infty}\frac{\log\Phi(t)}{t}<\infty,
\end{split}\end{multline*}\fi
\ifx91\begin{multline}\begin{split}\label{0-2}
\limsup_{t\to\infty}\frac{\log\Phi(t)}{t}<\infty,
\end{split}\end{multline}\fi
}
which is equivalent to the bound $\Phi(t)<\exp(Ct)$ with some $C>0$. Moreover, Rodin invented an interesting  property, that is almost everywhere ${{\rm BMO\,}}$-boundedness of Fourier series, and the  a.e. $\Phi$-summability immediately follows from this results, applying John-Nirenberg theorem. G.~A.~Karagulyan in \cite{Kar1,Kar2} proved that the condition {(\ref{{0-2}})} is sharp for a.e. $\Phi$-summability for Fourier series. That is if
{\ifx01
\begin{equation*} \label{0-3}
\limsup_{t\to\infty}\frac{\log\Phi(t)}{t}=\infty,
 \end{equation*}\fi  
\ifx11\begin{equation}\label{0-3}
\limsup_{t\to\infty}\frac{\log\Phi(t)}{t}=\infty,
\end{equation}\fi   
\ifx21\begin{align*}\label{0-3}
\limsup_{t\to\infty}\frac{\log\Phi(t)}{t}=\infty,
\end{align*}\fi   
\ifx31\begin{align}\label{0-3}
\limsup_{t\to\infty}\frac{\log\Phi(t)}{t}=\infty,
\end{align}\fi    
\ifx41\begin{gather*}\label{0-3}
\limsup_{t\to\infty}\frac{\log\Phi(t)}{t}=\infty,
\end{gather*}\fi  
\ifx51\begin{gather}\label{0-3}
\limsup_{t\to\infty}\frac{\log\Phi(t)}{t}=\infty,
\end{gather}\fi   
\ifx61\begin{multline*}\label{0-3}
\limsup_{t\to\infty}\frac{\log\Phi(t)}{t}=\infty,
\end{multline*}\fi  
\ifx71\begin{multline}\label{0-3}
\limsup_{t\to\infty}\frac{\log\Phi(t)}{t}=\infty,
\end{multline}\fi  
\ifx81\begin{multline*}\begin{split}\label{0-3}
\limsup_{t\to\infty}\frac{\log\Phi(t)}{t}=\infty,
\end{split}\end{multline*}\fi
\ifx91\begin{multline}\begin{split}\label{0-3}
\limsup_{t\to\infty}\frac{\log\Phi(t)}{t}=\infty,
\end{split}\end{multline}\fi
}
then there exists an integrable function $f\in L^1(0,2\pi)$ such that
{\ifx00
\begin{equation*} 
\limsup_{n\to\infty}\frac{1}{n}\sum_{k=1}^n\Phi(|S_k(x,f)|)=\infty,\quad \limsup_{n\to\infty}\frac{1}{n}\sum_{k=1}^n\Phi(|\tilde S_k(x,f)|)=\infty,
 \end{equation*}\fi  
\ifx10\begin{equation}
\limsup_{n\to\infty}\frac{1}{n}\sum_{k=1}^n\Phi(|S_k(x,f)|)=\infty,\quad \limsup_{n\to\infty}\frac{1}{n}\sum_{k=1}^n\Phi(|\tilde S_k(x,f)|)=\infty,
\end{equation}\fi   
\ifx20\begin{align*}
\limsup_{n\to\infty}\frac{1}{n}\sum_{k=1}^n\Phi(|S_k(x,f)|)=\infty,\quad \limsup_{n\to\infty}\frac{1}{n}\sum_{k=1}^n\Phi(|\tilde S_k(x,f)|)=\infty,
\end{align*}\fi   
\ifx30\begin{align}
\limsup_{n\to\infty}\frac{1}{n}\sum_{k=1}^n\Phi(|S_k(x,f)|)=\infty,\quad \limsup_{n\to\infty}\frac{1}{n}\sum_{k=1}^n\Phi(|\tilde S_k(x,f)|)=\infty,
\end{align}\fi    
\ifx40\begin{gather*}
\limsup_{n\to\infty}\frac{1}{n}\sum_{k=1}^n\Phi(|S_k(x,f)|)=\infty,\quad \limsup_{n\to\infty}\frac{1}{n}\sum_{k=1}^n\Phi(|\tilde S_k(x,f)|)=\infty,
\end{gather*}\fi  
\ifx50\begin{gather}
\limsup_{n\to\infty}\frac{1}{n}\sum_{k=1}^n\Phi(|S_k(x,f)|)=\infty,\quad \limsup_{n\to\infty}\frac{1}{n}\sum_{k=1}^n\Phi(|\tilde S_k(x,f)|)=\infty,
\end{gather}\fi   
\ifx60\begin{multline*}
\limsup_{n\to\infty}\frac{1}{n}\sum_{k=1}^n\Phi(|S_k(x,f)|)=\infty,\quad \limsup_{n\to\infty}\frac{1}{n}\sum_{k=1}^n\Phi(|\tilde S_k(x,f)|)=\infty,
\end{multline*}\fi  
\ifx70\begin{multline}
\limsup_{n\to\infty}\frac{1}{n}\sum_{k=1}^n\Phi(|S_k(x,f)|)=\infty,\quad \limsup_{n\to\infty}\frac{1}{n}\sum_{k=1}^n\Phi(|\tilde S_k(x,f)|)=\infty,
\end{multline}\fi  
\ifx80\begin{multline*}\begin{split}
\limsup_{n\to\infty}\frac{1}{n}\sum_{k=1}^n\Phi(|S_k(x,f)|)=\infty,\quad \limsup_{n\to\infty}\frac{1}{n}\sum_{k=1}^n\Phi(|\tilde S_k(x,f)|)=\infty,
\end{split}\end{multline*}\fi
\ifx90\begin{multline}\begin{split}
\limsup_{n\to\infty}\frac{1}{n}\sum_{k=1}^n\Phi(|S_k(x,f)|)=\infty,\quad \limsup_{n\to\infty}\frac{1}{n}\sum_{k=1}^n\Phi(|\tilde S_k(x,f)|)=\infty,
\end{split}\end{multline}\fi
}
holds everywhere on ${\ensuremath{\mathbb R}}$, where $S_k(x,f)$ and $\tilde S_k(x,f)$ are the ordinary and conjugate partial sums of Fourier series of $f(x)$.

Analogous problems are considered also for Walsh series. Almost everywhere $H^p$-summability of Walsh-Fourier series with $p>0$ proved by F.~Schipp in \cite{Sch2}. Almost everywhere $\Phi$-summability with condition {(\ref{{0-2}})} proved by V.~Rodin \cite{Rod2}.
\begin{theorem}[Rodin]If $\Phi(t):[0,\infty)\to[0,\infty)$, $\Phi(0)=0$, is an increasing continuous function satisfying {(\ref{{0-2}})}, then  the partial sums of Walsh-Fourier series of any function $f\in L^1[0,1)$ satisfy the condition
 {\ifx00
\begin{equation*} 
\lim_{n\to\infty}\frac{1}{n}\sum_{k=1}^n\Phi\left(|S_k(x,f)-f(x)|\right)=0
 \end{equation*}\fi  
\ifx10\begin{equation}
\lim_{n\to\infty}\frac{1}{n}\sum_{k=1}^n\Phi\left(|S_k(x,f)-f(x)|\right)=0
\end{equation}\fi   
\ifx20\begin{align*}
\lim_{n\to\infty}\frac{1}{n}\sum_{k=1}^n\Phi\left(|S_k(x,f)-f(x)|\right)=0
\end{align*}\fi   
\ifx30\begin{align}
\lim_{n\to\infty}\frac{1}{n}\sum_{k=1}^n\Phi\left(|S_k(x,f)-f(x)|\right)=0
\end{align}\fi    
\ifx40\begin{gather*}
\lim_{n\to\infty}\frac{1}{n}\sum_{k=1}^n\Phi\left(|S_k(x,f)-f(x)|\right)=0
\end{gather*}\fi  
\ifx50\begin{gather}
\lim_{n\to\infty}\frac{1}{n}\sum_{k=1}^n\Phi\left(|S_k(x,f)-f(x)|\right)=0
\end{gather}\fi   
\ifx60\begin{multline*}
\lim_{n\to\infty}\frac{1}{n}\sum_{k=1}^n\Phi\left(|S_k(x,f)-f(x)|\right)=0
\end{multline*}\fi  
\ifx70\begin{multline}
\lim_{n\to\infty}\frac{1}{n}\sum_{k=1}^n\Phi\left(|S_k(x,f)-f(x)|\right)=0
\end{multline}\fi  
\ifx80\begin{multline*}\begin{split}
\lim_{n\to\infty}\frac{1}{n}\sum_{k=1}^n\Phi\left(|S_k(x,f)-f(x)|\right)=0
\end{split}\end{multline*}\fi
\ifx90\begin{multline}\begin{split}
\lim_{n\to\infty}\frac{1}{n}\sum_{k=1}^n\Phi\left(|S_k(x,f)-f(x)|\right)=0
\end{split}\end{multline}\fi
}
almost everywhere on $[0,1)$.
\end{theorem}
In this theorem and everywhere bellow the notation $S_k(x,f)$ stands for the partial sums of Walsh-Fourier series of  $f\in L^1[0,1)$.
In the present paper we establish, that, as in trigonometric case \cite{Kar2}, the bound {(\ref{{0-2}})} is sharp for a.e. $\Phi$-summability of Walsh-Fourier series. Moreover, we prove
\begin{theorem}
If an increasing function $\Phi(t):[0,\infty)\to[0,\infty)$ satisfies the condition {(\ref{{0-3}})}, then there exists a function $f\in L^1[0,1)$ such that
{\ifx01
\begin{equation*} \label{0-4}
\limsup_{n\to\infty}\frac{1}{n}\sum_{k=1}^n\Phi\left(|S_k(x,f)|\right)=\infty
 \end{equation*}\fi  
\ifx11\begin{equation}\label{0-4}
\limsup_{n\to\infty}\frac{1}{n}\sum_{k=1}^n\Phi\left(|S_k(x,f)|\right)=\infty
\end{equation}\fi   
\ifx21\begin{align*}\label{0-4}
\limsup_{n\to\infty}\frac{1}{n}\sum_{k=1}^n\Phi\left(|S_k(x,f)|\right)=\infty
\end{align*}\fi   
\ifx31\begin{align}\label{0-4}
\limsup_{n\to\infty}\frac{1}{n}\sum_{k=1}^n\Phi\left(|S_k(x,f)|\right)=\infty
\end{align}\fi    
\ifx41\begin{gather*}\label{0-4}
\limsup_{n\to\infty}\frac{1}{n}\sum_{k=1}^n\Phi\left(|S_k(x,f)|\right)=\infty
\end{gather*}\fi  
\ifx51\begin{gather}\label{0-4}
\limsup_{n\to\infty}\frac{1}{n}\sum_{k=1}^n\Phi\left(|S_k(x,f)|\right)=\infty
\end{gather}\fi   
\ifx61\begin{multline*}\label{0-4}
\limsup_{n\to\infty}\frac{1}{n}\sum_{k=1}^n\Phi\left(|S_k(x,f)|\right)=\infty
\end{multline*}\fi  
\ifx71\begin{multline}\label{0-4}
\limsup_{n\to\infty}\frac{1}{n}\sum_{k=1}^n\Phi\left(|S_k(x,f)|\right)=\infty
\end{multline}\fi  
\ifx81\begin{multline*}\begin{split}\label{0-4}
\limsup_{n\to\infty}\frac{1}{n}\sum_{k=1}^n\Phi\left(|S_k(x,f)|\right)=\infty
\end{split}\end{multline*}\fi
\ifx91\begin{multline}\begin{split}\label{0-4}
\limsup_{n\to\infty}\frac{1}{n}\sum_{k=1}^n\Phi\left(|S_k(x,f)|\right)=\infty
\end{split}\end{multline}\fi
}
holds everywhere on $[0,1)$.
\end{theorem}
It is clear this theorem generalizes Schipp's theorem on everywhere divergence of Walsh-Fourier series.  S.~V.~Bochkarev in \cite{Boch} has constructed a function $f\in L^1[0,1)$ such that
{\ifx01
\begin{equation*} \label{0-5}
\limsup_{n\to\infty }\frac{|S_n(x,f)|}{\omega_n}=\infty
 \end{equation*}\fi  
\ifx11\begin{equation}\label{0-5}
\limsup_{n\to\infty }\frac{|S_n(x,f)|}{\omega_n}=\infty
\end{equation}\fi   
\ifx21\begin{align*}\label{0-5}
\limsup_{n\to\infty }\frac{|S_n(x,f)|}{\omega_n}=\infty
\end{align*}\fi   
\ifx31\begin{align}\label{0-5}
\limsup_{n\to\infty }\frac{|S_n(x,f)|}{\omega_n}=\infty
\end{align}\fi    
\ifx41\begin{gather*}\label{0-5}
\limsup_{n\to\infty }\frac{|S_n(x,f)|}{\omega_n}=\infty
\end{gather*}\fi  
\ifx51\begin{gather}\label{0-5}
\limsup_{n\to\infty }\frac{|S_n(x,f)|}{\omega_n}=\infty
\end{gather}\fi   
\ifx61\begin{multline*}\label{0-5}
\limsup_{n\to\infty }\frac{|S_n(x,f)|}{\omega_n}=\infty
\end{multline*}\fi  
\ifx71\begin{multline}\label{0-5}
\limsup_{n\to\infty }\frac{|S_n(x,f)|}{\omega_n}=\infty
\end{multline}\fi  
\ifx81\begin{multline*}\begin{split}\label{0-5}
\limsup_{n\to\infty }\frac{|S_n(x,f)|}{\omega_n}=\infty
\end{split}\end{multline*}\fi
\ifx91\begin{multline}\begin{split}\label{0-5}
\limsup_{n\to\infty }\frac{|S_n(x,f)|}{\omega_n}=\infty
\end{split}\end{multline}\fi
}
everywhere on $[0,1)$, where $\omega_n=o(\sqrt {\log n })$. It is easy to observe, that this theorem implies the existence of a function $f\in L^1[0,1)$ satisfying {(\ref{{0-4}})} whenever
{\ifx00
\begin{equation*} 
\limsup_{t\to\infty}\frac{\log\Phi(t)}{t^2}=\infty,
 \end{equation*}\fi  
\ifx10\begin{equation}
\limsup_{t\to\infty}\frac{\log\Phi(t)}{t^2}=\infty,
\end{equation}\fi   
\ifx20\begin{align*}
\limsup_{t\to\infty}\frac{\log\Phi(t)}{t^2}=\infty,
\end{align*}\fi   
\ifx30\begin{align}
\limsup_{t\to\infty}\frac{\log\Phi(t)}{t^2}=\infty,
\end{align}\fi    
\ifx40\begin{gather*}
\limsup_{t\to\infty}\frac{\log\Phi(t)}{t^2}=\infty,
\end{gather*}\fi  
\ifx50\begin{gather}
\limsup_{t\to\infty}\frac{\log\Phi(t)}{t^2}=\infty,
\end{gather}\fi   
\ifx60\begin{multline*}
\limsup_{t\to\infty}\frac{\log\Phi(t)}{t^2}=\infty,
\end{multline*}\fi  
\ifx70\begin{multline}
\limsup_{t\to\infty}\frac{\log\Phi(t)}{t^2}=\infty,
\end{multline}\fi  
\ifx80\begin{multline*}\begin{split}
\limsup_{t\to\infty}\frac{\log\Phi(t)}{t^2}=\infty,
\end{split}\end{multline*}\fi
\ifx90\begin{multline}\begin{split}
\limsup_{t\to\infty}\frac{\log\Phi(t)}{t^2}=\infty,
\end{split}\end{multline}\fi
}
instead of the condition {(\ref{{0-3}})}, which was the best bound before. A divergence theorem like {(\ref{{0-5}})} with $\omega_n=o(\sqrt {\log n/\log\log n })$ for the trigonometric Fourier series established by S.~V.~Konyagin in \cite{Kon}.

We note also, that the problem of uniformly $\Phi$-summability of trigonometric Fourier series, when $f(x)$ is a continuous function considered by V.~Totik \cite{Tot1, Tot2}. He proved that the condition {(\ref{{0-2}})} is necessary and sufficient for uniformly  $\Phi$-summability of Fourier series of continuous functions. For the Walsh series the analogous problem is considered by S.~Fridli and F.~Schipp \cite{FrSc1,FrSc2}, V.~Rodin \cite{Rod2},  U.~Goginava and L.~Gogoladze \cite{GoGo}.
\section{Proof of theorem}
 Recall the definitions of Rademacher and Walsh functions (see \cite{GES} or \cite{SWS}). We consider the function
{\ifx00
\begin{equation*} 
r_0(x)=\left\{
\begin{array}{rcl}
1, &\hbox{ if }& x\in[0,1/2),\\
-1, &\hbox{ if }& x\in[1/2,1),
\end{array}
\right.
 \end{equation*}\fi  
\ifx10\begin{equation}
r_0(x)=\left\{
\begin{array}{rcl}
1, &\hbox{ if }& x\in[0,1/2),\\
-1, &\hbox{ if }& x\in[1/2,1),
\end{array}
\right.
\end{equation}\fi   
\ifx20\begin{align*}
r_0(x)=\left\{
\begin{array}{rcl}
1, &\hbox{ if }& x\in[0,1/2),\\
-1, &\hbox{ if }& x\in[1/2,1),
\end{array}
\right.
\end{align*}\fi   
\ifx30\begin{align}
r_0(x)=\left\{
\begin{array}{rcl}
1, &\hbox{ if }& x\in[0,1/2),\\
-1, &\hbox{ if }& x\in[1/2,1),
\end{array}
\right.
\end{align}\fi    
\ifx40\begin{gather*}
r_0(x)=\left\{
\begin{array}{rcl}
1, &\hbox{ if }& x\in[0,1/2),\\
-1, &\hbox{ if }& x\in[1/2,1),
\end{array}
\right.
\end{gather*}\fi  
\ifx50\begin{gather}
r_0(x)=\left\{
\begin{array}{rcl}
1, &\hbox{ if }& x\in[0,1/2),\\
-1, &\hbox{ if }& x\in[1/2,1),
\end{array}
\right.
\end{gather}\fi   
\ifx60\begin{multline*}
r_0(x)=\left\{
\begin{array}{rcl}
1, &\hbox{ if }& x\in[0,1/2),\\
-1, &\hbox{ if }& x\in[1/2,1),
\end{array}
\right.
\end{multline*}\fi  
\ifx70\begin{multline}
r_0(x)=\left\{
\begin{array}{rcl}
1, &\hbox{ if }& x\in[0,1/2),\\
-1, &\hbox{ if }& x\in[1/2,1),
\end{array}
\right.
\end{multline}\fi  
\ifx80\begin{multline*}\begin{split}
r_0(x)=\left\{
\begin{array}{rcl}
1, &\hbox{ if }& x\in[0,1/2),\\
-1, &\hbox{ if }& x\in[1/2,1),
\end{array}
\right.
\end{split}\end{multline*}\fi
\ifx90\begin{multline}\begin{split}
r_0(x)=\left\{
\begin{array}{rcl}
1, &\hbox{ if }& x\in[0,1/2),\\
-1, &\hbox{ if }& x\in[1/2,1),
\end{array}
\right.
\end{split}\end{multline}\fi
}
periodically continued over the real line. The Rademacher functions are defined by $r_k(x)=r_0(2^kx)$, $k=0,1,2,\ldots$.
Walsh system is obtained by all possible products of Rademacher functions. We shall consider the Paley ordering of Walsh system.
We set $w_0(x)\equiv 1$. To define $w_n(x)$ when $n\ge 1$ we write $n$ in dyadic form
{\ifx01
\begin{equation*} \label{a35}
n=\sum_{j=0}^k\varepsilon_j2^j,
 \end{equation*}\fi  
\ifx11\begin{equation}\label{a35}
n=\sum_{j=0}^k\varepsilon_j2^j,
\end{equation}\fi   
\ifx21\begin{align*}\label{a35}
n=\sum_{j=0}^k\varepsilon_j2^j,
\end{align*}\fi   
\ifx31\begin{align}\label{a35}
n=\sum_{j=0}^k\varepsilon_j2^j,
\end{align}\fi    
\ifx41\begin{gather*}\label{a35}
n=\sum_{j=0}^k\varepsilon_j2^j,
\end{gather*}\fi  
\ifx51\begin{gather}\label{a35}
n=\sum_{j=0}^k\varepsilon_j2^j,
\end{gather}\fi   
\ifx61\begin{multline*}\label{a35}
n=\sum_{j=0}^k\varepsilon_j2^j,
\end{multline*}\fi  
\ifx71\begin{multline}\label{a35}
n=\sum_{j=0}^k\varepsilon_j2^j,
\end{multline}\fi  
\ifx81\begin{multline*}\begin{split}\label{a35}
n=\sum_{j=0}^k\varepsilon_j2^j,
\end{split}\end{multline*}\fi
\ifx91\begin{multline}\begin{split}\label{a35}
n=\sum_{j=0}^k\varepsilon_j2^j,
\end{split}\end{multline}\fi
}
where $\varepsilon_k=1$ and $\varepsilon_j=0$ or $1$ if $j=0,1,\ldots,k-1$, and set
{\ifx00
\begin{equation*} 
w_n(x)=\prod_{j=0}^k(r_j(x))^{\varepsilon_j}.
 \end{equation*}\fi  
\ifx10\begin{equation}
w_n(x)=\prod_{j=0}^k(r_j(x))^{\varepsilon_j}.
\end{equation}\fi   
\ifx20\begin{align*}
w_n(x)=\prod_{j=0}^k(r_j(x))^{\varepsilon_j}.
\end{align*}\fi   
\ifx30\begin{align}
w_n(x)=\prod_{j=0}^k(r_j(x))^{\varepsilon_j}.
\end{align}\fi    
\ifx40\begin{gather*}
w_n(x)=\prod_{j=0}^k(r_j(x))^{\varepsilon_j}.
\end{gather*}\fi  
\ifx50\begin{gather}
w_n(x)=\prod_{j=0}^k(r_j(x))^{\varepsilon_j}.
\end{gather}\fi   
\ifx60\begin{multline*}
w_n(x)=\prod_{j=0}^k(r_j(x))^{\varepsilon_j}.
\end{multline*}\fi  
\ifx70\begin{multline}
w_n(x)=\prod_{j=0}^k(r_j(x))^{\varepsilon_j}.
\end{multline}\fi  
\ifx80\begin{multline*}\begin{split}
w_n(x)=\prod_{j=0}^k(r_j(x))^{\varepsilon_j}.
\end{split}\end{multline*}\fi
\ifx90\begin{multline}\begin{split}
w_n(x)=\prod_{j=0}^k(r_j(x))^{\varepsilon_j}.
\end{split}\end{multline}\fi
}
The partial sums of Walsh-Fourier series of a function $f\in L^1[0,1)$ have a formula
{\ifx00
\begin{equation*} 
S_n(x,f)=\int_0^1f(t)D_n(x\oplus t)dt,
 \end{equation*}\fi  
\ifx10\begin{equation}
S_n(x,f)=\int_0^1f(t)D_n(x\oplus t)dt,
\end{equation}\fi   
\ifx20\begin{align*}
S_n(x,f)=\int_0^1f(t)D_n(x\oplus t)dt,
\end{align*}\fi   
\ifx30\begin{align}
S_n(x,f)=\int_0^1f(t)D_n(x\oplus t)dt,
\end{align}\fi    
\ifx40\begin{gather*}
S_n(x,f)=\int_0^1f(t)D_n(x\oplus t)dt,
\end{gather*}\fi  
\ifx50\begin{gather}
S_n(x,f)=\int_0^1f(t)D_n(x\oplus t)dt,
\end{gather}\fi   
\ifx60\begin{multline*}
S_n(x,f)=\int_0^1f(t)D_n(x\oplus t)dt,
\end{multline*}\fi  
\ifx70\begin{multline}
S_n(x,f)=\int_0^1f(t)D_n(x\oplus t)dt,
\end{multline}\fi  
\ifx80\begin{multline*}\begin{split}
S_n(x,f)=\int_0^1f(t)D_n(x\oplus t)dt,
\end{split}\end{multline*}\fi
\ifx90\begin{multline}\begin{split}
S_n(x,f)=\int_0^1f(t)D_n(x\oplus t)dt,
\end{split}\end{multline}\fi
}
where $D_n(x)$ is the Dirichlet kernel and $\oplus$ denotes the dyadic addition. We note that
{\ifx00
\begin{equation*} 
D_{2^k}(x)=\left\{
\begin{array}{rcl}
2^k, &\hbox{ if }& x\in[0,2^{-k}),\\
0, &\hbox{ if }& x\in[2^{-k},1).
\end{array}
\right.
 \end{equation*}\fi  
\ifx10\begin{equation}
D_{2^k}(x)=\left\{
\begin{array}{rcl}
2^k, &\hbox{ if }& x\in[0,2^{-k}),\\
0, &\hbox{ if }& x\in[2^{-k},1).
\end{array}
\right.
\end{equation}\fi   
\ifx20\begin{align*}
D_{2^k}(x)=\left\{
\begin{array}{rcl}
2^k, &\hbox{ if }& x\in[0,2^{-k}),\\
0, &\hbox{ if }& x\in[2^{-k},1).
\end{array}
\right.
\end{align*}\fi   
\ifx30\begin{align}
D_{2^k}(x)=\left\{
\begin{array}{rcl}
2^k, &\hbox{ if }& x\in[0,2^{-k}),\\
0, &\hbox{ if }& x\in[2^{-k},1).
\end{array}
\right.
\end{align}\fi    
\ifx40\begin{gather*}
D_{2^k}(x)=\left\{
\begin{array}{rcl}
2^k, &\hbox{ if }& x\in[0,2^{-k}),\\
0, &\hbox{ if }& x\in[2^{-k},1).
\end{array}
\right.
\end{gather*}\fi  
\ifx50\begin{gather}
D_{2^k}(x)=\left\{
\begin{array}{rcl}
2^k, &\hbox{ if }& x\in[0,2^{-k}),\\
0, &\hbox{ if }& x\in[2^{-k},1).
\end{array}
\right.
\end{gather}\fi   
\ifx60\begin{multline*}
D_{2^k}(x)=\left\{
\begin{array}{rcl}
2^k, &\hbox{ if }& x\in[0,2^{-k}),\\
0, &\hbox{ if }& x\in[2^{-k},1).
\end{array}
\right.
\end{multline*}\fi  
\ifx70\begin{multline}
D_{2^k}(x)=\left\{
\begin{array}{rcl}
2^k, &\hbox{ if }& x\in[0,2^{-k}),\\
0, &\hbox{ if }& x\in[2^{-k},1).
\end{array}
\right.
\end{multline}\fi  
\ifx80\begin{multline*}\begin{split}
D_{2^k}(x)=\left\{
\begin{array}{rcl}
2^k, &\hbox{ if }& x\in[0,2^{-k}),\\
0, &\hbox{ if }& x\in[2^{-k},1).
\end{array}
\right.
\end{split}\end{multline*}\fi
\ifx90\begin{multline}\begin{split}
D_{2^k}(x)=\left\{
\begin{array}{rcl}
2^k, &\hbox{ if }& x\in[0,2^{-k}),\\
0, &\hbox{ if }& x\in[2^{-k},1).
\end{array}
\right.
\end{split}\end{multline}\fi
}
Dirichlet kernel can be expressed by modified Dirichlet kernel $D_n^*(x)$ by
{\ifx00
\begin{equation*} 
D_n(x)=w_n(x)D_n^*(x).
 \end{equation*}\fi  
\ifx10\begin{equation}
D_n(x)=w_n(x)D_n^*(x).
\end{equation}\fi   
\ifx20\begin{align*}
D_n(x)=w_n(x)D_n^*(x).
\end{align*}\fi   
\ifx30\begin{align}
D_n(x)=w_n(x)D_n^*(x).
\end{align}\fi    
\ifx40\begin{gather*}
D_n(x)=w_n(x)D_n^*(x).
\end{gather*}\fi  
\ifx50\begin{gather}
D_n(x)=w_n(x)D_n^*(x).
\end{gather}\fi   
\ifx60\begin{multline*}
D_n(x)=w_n(x)D_n^*(x).
\end{multline*}\fi  
\ifx70\begin{multline}
D_n(x)=w_n(x)D_n^*(x).
\end{multline}\fi  
\ifx80\begin{multline*}\begin{split}
D_n(x)=w_n(x)D_n^*(x).
\end{split}\end{multline*}\fi
\ifx90\begin{multline}\begin{split}
D_n(x)=w_n(x)D_n^*(x).
\end{split}\end{multline}\fi
}
If $n\in {\ensuremath{\mathbb N}}$ has the form {(\ref{{a35}})}, then we have
{\ifx00
\begin{equation*} 
D_n^*(x)=\sum_{j=0}^k\varepsilon_jD_{2^j}^*(x)=\sum_{j=0}^k\varepsilon_jr_j(x)D_{2^j}(x).
 \end{equation*}\fi  
\ifx10\begin{equation}
D_n^*(x)=\sum_{j=0}^k\varepsilon_jD_{2^j}^*(x)=\sum_{j=0}^k\varepsilon_jr_j(x)D_{2^j}(x).
\end{equation}\fi   
\ifx20\begin{align*}
D_n^*(x)=\sum_{j=0}^k\varepsilon_jD_{2^j}^*(x)=\sum_{j=0}^k\varepsilon_jr_j(x)D_{2^j}(x).
\end{align*}\fi   
\ifx30\begin{align}
D_n^*(x)=\sum_{j=0}^k\varepsilon_jD_{2^j}^*(x)=\sum_{j=0}^k\varepsilon_jr_j(x)D_{2^j}(x).
\end{align}\fi    
\ifx40\begin{gather*}
D_n^*(x)=\sum_{j=0}^k\varepsilon_jD_{2^j}^*(x)=\sum_{j=0}^k\varepsilon_jr_j(x)D_{2^j}(x).
\end{gather*}\fi  
\ifx50\begin{gather}
D_n^*(x)=\sum_{j=0}^k\varepsilon_jD_{2^j}^*(x)=\sum_{j=0}^k\varepsilon_jr_j(x)D_{2^j}(x).
\end{gather}\fi   
\ifx60\begin{multline*}
D_n^*(x)=\sum_{j=0}^k\varepsilon_jD_{2^j}^*(x)=\sum_{j=0}^k\varepsilon_jr_j(x)D_{2^j}(x).
\end{multline*}\fi  
\ifx70\begin{multline}
D_n^*(x)=\sum_{j=0}^k\varepsilon_jD_{2^j}^*(x)=\sum_{j=0}^k\varepsilon_jr_j(x)D_{2^j}(x).
\end{multline}\fi  
\ifx80\begin{multline*}\begin{split}
D_n^*(x)=\sum_{j=0}^k\varepsilon_jD_{2^j}^*(x)=\sum_{j=0}^k\varepsilon_jr_j(x)D_{2^j}(x).
\end{split}\end{multline*}\fi
\ifx90\begin{multline}\begin{split}
D_n^*(x)=\sum_{j=0}^k\varepsilon_jD_{2^j}^*(x)=\sum_{j=0}^k\varepsilon_jr_j(x)D_{2^j}(x).
\end{split}\end{multline}\fi
}
We shall write $a\lesssim b$, if $a<c\cdot b$ and $c>0$ is an absolute constant. The notation ${\ensuremath{\mathbb I}}_E$  stands for the indicator function of a set $E$.  An interval is said to be a set of the form $[a,b)$. For a dyadic interval $\delta$ we denote by $\delta^+$ and $\delta^-$ left and right halves of $\delta$. We denote the spectrum of a Walsh polynomial $P(x)=\sum_{k=0}^m a_kw_k(x)$  by
{\ifx00
\begin{equation*} 
{{\rm sp\,}} P(x)=\{k\in {\ensuremath{\mathbb N}}\cup {0}:\, a_k\neq 0\}.
 \end{equation*}\fi  
\ifx10\begin{equation}
{{\rm sp\,}} P(x)=\{k\in {\ensuremath{\mathbb N}}\cup {0}:\, a_k\neq 0\}.
\end{equation}\fi   
\ifx20\begin{align*}
{{\rm sp\,}} P(x)=\{k\in {\ensuremath{\mathbb N}}\cup {0}:\, a_k\neq 0\}.
\end{align*}\fi   
\ifx30\begin{align}
{{\rm sp\,}} P(x)=\{k\in {\ensuremath{\mathbb N}}\cup {0}:\, a_k\neq 0\}.
\end{align}\fi    
\ifx40\begin{gather*}
{{\rm sp\,}} P(x)=\{k\in {\ensuremath{\mathbb N}}\cup {0}:\, a_k\neq 0\}.
\end{gather*}\fi  
\ifx50\begin{gather}
{{\rm sp\,}} P(x)=\{k\in {\ensuremath{\mathbb N}}\cup {0}:\, a_k\neq 0\}.
\end{gather}\fi   
\ifx60\begin{multline*}
{{\rm sp\,}} P(x)=\{k\in {\ensuremath{\mathbb N}}\cup {0}:\, a_k\neq 0\}.
\end{multline*}\fi  
\ifx70\begin{multline}
{{\rm sp\,}} P(x)=\{k\in {\ensuremath{\mathbb N}}\cup {0}:\, a_k\neq 0\}.
\end{multline}\fi  
\ifx80\begin{multline*}\begin{split}
{{\rm sp\,}} P(x)=\{k\in {\ensuremath{\mathbb N}}\cup {0}:\, a_k\neq 0\}.
\end{split}\end{multline*}\fi
\ifx90\begin{multline}\begin{split}
{{\rm sp\,}} P(x)=\{k\in {\ensuremath{\mathbb N}}\cup {0}:\, a_k\neq 0\}.
\end{split}\end{multline}\fi
}

In the proof of following lemma we use a well known inequality
{\ifx01
\begin{equation*} \label{a25}
\left|\left\{x\in (0,1):\, \left|\sum_{k=1}^na_kr_k(x)\right|\le \lambda\right\}\right|\ge 1-2\exp\left( -\lambda^2/4\sum_{k=1}^na_k^2\right),\quad \lambda>0,
 \end{equation*}\fi  
\ifx11\begin{equation}\label{a25}
\left|\left\{x\in (0,1):\, \left|\sum_{k=1}^na_kr_k(x)\right|\le \lambda\right\}\right|\ge 1-2\exp\left( -\lambda^2/4\sum_{k=1}^na_k^2\right),\quad \lambda>0,
\end{equation}\fi   
\ifx21\begin{align*}\label{a25}
\left|\left\{x\in (0,1):\, \left|\sum_{k=1}^na_kr_k(x)\right|\le \lambda\right\}\right|\ge 1-2\exp\left( -\lambda^2/4\sum_{k=1}^na_k^2\right),\quad \lambda>0,
\end{align*}\fi   
\ifx31\begin{align}\label{a25}
\left|\left\{x\in (0,1):\, \left|\sum_{k=1}^na_kr_k(x)\right|\le \lambda\right\}\right|\ge 1-2\exp\left( -\lambda^2/4\sum_{k=1}^na_k^2\right),\quad \lambda>0,
\end{align}\fi    
\ifx41\begin{gather*}\label{a25}
\left|\left\{x\in (0,1):\, \left|\sum_{k=1}^na_kr_k(x)\right|\le \lambda\right\}\right|\ge 1-2\exp\left( -\lambda^2/4\sum_{k=1}^na_k^2\right),\quad \lambda>0,
\end{gather*}\fi  
\ifx51\begin{gather}\label{a25}
\left|\left\{x\in (0,1):\, \left|\sum_{k=1}^na_kr_k(x)\right|\le \lambda\right\}\right|\ge 1-2\exp\left( -\lambda^2/4\sum_{k=1}^na_k^2\right),\quad \lambda>0,
\end{gather}\fi   
\ifx61\begin{multline*}\label{a25}
\left|\left\{x\in (0,1):\, \left|\sum_{k=1}^na_kr_k(x)\right|\le \lambda\right\}\right|\ge 1-2\exp\left( -\lambda^2/4\sum_{k=1}^na_k^2\right),\quad \lambda>0,
\end{multline*}\fi  
\ifx71\begin{multline}\label{a25}
\left|\left\{x\in (0,1):\, \left|\sum_{k=1}^na_kr_k(x)\right|\le \lambda\right\}\right|\ge 1-2\exp\left( -\lambda^2/4\sum_{k=1}^na_k^2\right),\quad \lambda>0,
\end{multline}\fi  
\ifx81\begin{multline*}\begin{split}\label{a25}
\left|\left\{x\in (0,1):\, \left|\sum_{k=1}^na_kr_k(x)\right|\le \lambda\right\}\right|\ge 1-2\exp\left( -\lambda^2/4\sum_{k=1}^na_k^2\right),\quad \lambda>0,
\end{split}\end{multline*}\fi
\ifx91\begin{multline}\begin{split}\label{a25}
\left|\left\{x\in (0,1):\, \left|\sum_{k=1}^na_kr_k(x)\right|\le \lambda\right\}\right|\ge 1-2\exp\left( -\lambda^2/4\sum_{k=1}^na_k^2\right),\quad \lambda>0,
\end{split}\end{multline}\fi
}
for Rademacher polynomials (see for example \cite{KaSa}, chap. 2, theorem 5).
\begin{lemma}\label{L2}
If $n\in {\ensuremath{\mathbb N}}$, $n>50$, then there exists a set $E_n\subset [0,1)$, which is a union of some dyadic intervals of the length $2^{-n}$, satisfies
the inequality
{\ifx01
\begin{equation*} \label{a27}
|E_n|>1-2\exp(-n/36),
 \end{equation*}\fi  
\ifx11\begin{equation}\label{a27}
|E_n|>1-2\exp(-n/36),
\end{equation}\fi   
\ifx21\begin{align*}\label{a27}
|E_n|>1-2\exp(-n/36),
\end{align*}\fi   
\ifx31\begin{align}\label{a27}
|E_n|>1-2\exp(-n/36),
\end{align}\fi    
\ifx41\begin{gather*}\label{a27}
|E_n|>1-2\exp(-n/36),
\end{gather*}\fi  
\ifx51\begin{gather}\label{a27}
|E_n|>1-2\exp(-n/36),
\end{gather}\fi   
\ifx61\begin{multline*}\label{a27}
|E_n|>1-2\exp(-n/36),
\end{multline*}\fi  
\ifx71\begin{multline}\label{a27}
|E_n|>1-2\exp(-n/36),
\end{multline}\fi  
\ifx81\begin{multline*}\begin{split}\label{a27}
|E_n|>1-2\exp(-n/36),
\end{split}\end{multline*}\fi
\ifx91\begin{multline}\begin{split}\label{a27}
|E_n|>1-2\exp(-n/36),
\end{split}\end{multline}\fi
}
and for any $x\in E_n$ there exists an integer $m=m(x)<2^n$ such that 
{\ifx01
\begin{equation*} \label{a20}
\int_0^x D_m^*(x\oplus t)dt\ge \frac{n}{30}.
 \end{equation*}\fi  
\ifx11\begin{equation}\label{a20}
\int_0^x D_m^*(x\oplus t)dt\ge \frac{n}{30}.
\end{equation}\fi   
\ifx21\begin{align*}\label{a20}
\int_0^x D_m^*(x\oplus t)dt\ge \frac{n}{30}.
\end{align*}\fi   
\ifx31\begin{align}\label{a20}
\int_0^x D_m^*(x\oplus t)dt\ge \frac{n}{30}.
\end{align}\fi    
\ifx41\begin{gather*}\label{a20}
\int_0^x D_m^*(x\oplus t)dt\ge \frac{n}{30}.
\end{gather*}\fi  
\ifx51\begin{gather}\label{a20}
\int_0^x D_m^*(x\oplus t)dt\ge \frac{n}{30}.
\end{gather}\fi   
\ifx61\begin{multline*}\label{a20}
\int_0^x D_m^*(x\oplus t)dt\ge \frac{n}{30}.
\end{multline*}\fi  
\ifx71\begin{multline}\label{a20}
\int_0^x D_m^*(x\oplus t)dt\ge \frac{n}{30}.
\end{multline}\fi  
\ifx81\begin{multline*}\begin{split}\label{a20}
\int_0^x D_m^*(x\oplus t)dt\ge \frac{n}{30}.
\end{split}\end{multline*}\fi
\ifx91\begin{multline}\begin{split}\label{a20}
\int_0^x D_m^*(x\oplus t)dt\ge \frac{n}{30}.
\end{split}\end{multline}\fi
}
\end{lemma}
\begin{proof}
We define
{\ifx01
\begin{equation*} \label{a26}
E_n=\left\{x\in [0,1):\, \left|\sum_{j=1}^nr_j(x)r_{j+1}(x)\right|<\frac{n}{3}\right\}
 \end{equation*}\fi  
\ifx11\begin{equation}\label{a26}
E_n=\left\{x\in [0,1):\, \left|\sum_{j=1}^nr_j(x)r_{j+1}(x)\right|<\frac{n}{3}\right\}
\end{equation}\fi   
\ifx21\begin{align*}\label{a26}
E_n=\left\{x\in [0,1):\, \left|\sum_{j=1}^nr_j(x)r_{j+1}(x)\right|<\frac{n}{3}\right\}
\end{align*}\fi   
\ifx31\begin{align}\label{a26}
E_n=\left\{x\in [0,1):\, \left|\sum_{j=1}^nr_j(x)r_{j+1}(x)\right|<\frac{n}{3}\right\}
\end{align}\fi    
\ifx41\begin{gather*}\label{a26}
E_n=\left\{x\in [0,1):\, \left|\sum_{j=1}^nr_j(x)r_{j+1}(x)\right|<\frac{n}{3}\right\}
\end{gather*}\fi  
\ifx51\begin{gather}\label{a26}
E_n=\left\{x\in [0,1):\, \left|\sum_{j=1}^nr_j(x)r_{j+1}(x)\right|<\frac{n}{3}\right\}
\end{gather}\fi   
\ifx61\begin{multline*}\label{a26}
E_n=\left\{x\in [0,1):\, \left|\sum_{j=1}^nr_j(x)r_{j+1}(x)\right|<\frac{n}{3}\right\}
\end{multline*}\fi  
\ifx71\begin{multline}\label{a26}
E_n=\left\{x\in [0,1):\, \left|\sum_{j=1}^nr_j(x)r_{j+1}(x)\right|<\frac{n}{3}\right\}
\end{multline}\fi  
\ifx81\begin{multline*}\begin{split}\label{a26}
E_n=\left\{x\in [0,1):\, \left|\sum_{j=1}^nr_j(x)r_{j+1}(x)\right|<\frac{n}{3}\right\}
\end{split}\end{multline*}\fi
\ifx91\begin{multline}\begin{split}\label{a26}
E_n=\left\{x\in [0,1):\, \left|\sum_{j=1}^nr_j(x)r_{j+1}(x)\right|<\frac{n}{3}\right\}
\end{split}\end{multline}\fi
}
Since $\phi_j(x)=r_j(x)r_{j+1}(x)$, $j=1,2,\ldots, n$ are independent functions, taking values $\pm 1$ equally, the inequality
{(\ref{{a25}})} holds for $\phi_j(x)$ functions too. Applying {(\ref{{a25}})} in {(\ref{{a26}})} we will get the bound {(\ref{{a27}})}. Observe that for a fixed $x\in E_n$ we have
{\ifx01
\begin{equation*} \label{a17}
\#\{j\in{\ensuremath{\mathbb N}}:\,1\le j\le n:\, r_j(x)r_{j+1}(x)=-1\}>n/3,
 \end{equation*}\fi  
\ifx11\begin{equation}\label{a17}
\#\{j\in{\ensuremath{\mathbb N}}:\,1\le j\le n:\, r_j(x)r_{j+1}(x)=-1\}>n/3,
\end{equation}\fi   
\ifx21\begin{align*}\label{a17}
\#\{j\in{\ensuremath{\mathbb N}}:\,1\le j\le n:\, r_j(x)r_{j+1}(x)=-1\}>n/3,
\end{align*}\fi   
\ifx31\begin{align}\label{a17}
\#\{j\in{\ensuremath{\mathbb N}}:\,1\le j\le n:\, r_j(x)r_{j+1}(x)=-1\}>n/3,
\end{align}\fi    
\ifx41\begin{gather*}\label{a17}
\#\{j\in{\ensuremath{\mathbb N}}:\,1\le j\le n:\, r_j(x)r_{j+1}(x)=-1\}>n/3,
\end{gather*}\fi  
\ifx51\begin{gather}\label{a17}
\#\{j\in{\ensuremath{\mathbb N}}:\,1\le j\le n:\, r_j(x)r_{j+1}(x)=-1\}>n/3,
\end{gather}\fi   
\ifx61\begin{multline*}\label{a17}
\#\{j\in{\ensuremath{\mathbb N}}:\,1\le j\le n:\, r_j(x)r_{j+1}(x)=-1\}>n/3,
\end{multline*}\fi  
\ifx71\begin{multline}\label{a17}
\#\{j\in{\ensuremath{\mathbb N}}:\,1\le j\le n:\, r_j(x)r_{j+1}(x)=-1\}>n/3,
\end{multline}\fi  
\ifx81\begin{multline*}\begin{split}\label{a17}
\#\{j\in{\ensuremath{\mathbb N}}:\,1\le j\le n:\, r_j(x)r_{j+1}(x)=-1\}>n/3,
\end{split}\end{multline*}\fi
\ifx91\begin{multline}\begin{split}\label{a17}
\#\{j\in{\ensuremath{\mathbb N}}:\,1\le j\le n:\, r_j(x)r_{j+1}(x)=-1\}>n/3,
\end{split}\end{multline}\fi
}
where $\#A$ denotes the cardinality of a set $A$. On the other hand the value in {(\ref{{a17}})} characterizes the number of  sign changes in the sequence $r_1(x),r_2(x),\ldots ,r_{n+1}(x)$. Using this fact, we may fix integers $1\le k_1<k_2<\ldots<k_\nu\le n$, such that
{\ifx01
\begin{equation*} \label{a18}
r_{k_i}(x)=1,\quad r_{k_i+1}(x)=-1,\quad i=1,2,\ldots , \nu,\quad \nu\ge \frac{n}{6}-1.
 \end{equation*}\fi  
\ifx11\begin{equation}\label{a18}
r_{k_i}(x)=1,\quad r_{k_i+1}(x)=-1,\quad i=1,2,\ldots , \nu,\quad \nu\ge \frac{n}{6}-1.
\end{equation}\fi   
\ifx21\begin{align*}\label{a18}
r_{k_i}(x)=1,\quad r_{k_i+1}(x)=-1,\quad i=1,2,\ldots , \nu,\quad \nu\ge \frac{n}{6}-1.
\end{align*}\fi   
\ifx31\begin{align}\label{a18}
r_{k_i}(x)=1,\quad r_{k_i+1}(x)=-1,\quad i=1,2,\ldots , \nu,\quad \nu\ge \frac{n}{6}-1.
\end{align}\fi    
\ifx41\begin{gather*}\label{a18}
r_{k_i}(x)=1,\quad r_{k_i+1}(x)=-1,\quad i=1,2,\ldots , \nu,\quad \nu\ge \frac{n}{6}-1.
\end{gather*}\fi  
\ifx51\begin{gather}\label{a18}
r_{k_i}(x)=1,\quad r_{k_i+1}(x)=-1,\quad i=1,2,\ldots , \nu,\quad \nu\ge \frac{n}{6}-1.
\end{gather}\fi   
\ifx61\begin{multline*}\label{a18}
r_{k_i}(x)=1,\quad r_{k_i+1}(x)=-1,\quad i=1,2,\ldots , \nu,\quad \nu\ge \frac{n}{6}-1.
\end{multline*}\fi  
\ifx71\begin{multline}\label{a18}
r_{k_i}(x)=1,\quad r_{k_i+1}(x)=-1,\quad i=1,2,\ldots , \nu,\quad \nu\ge \frac{n}{6}-1.
\end{multline}\fi  
\ifx81\begin{multline*}\begin{split}\label{a18}
r_{k_i}(x)=1,\quad r_{k_i+1}(x)=-1,\quad i=1,2,\ldots , \nu,\quad \nu\ge \frac{n}{6}-1.
\end{split}\end{multline*}\fi
\ifx91\begin{multline}\begin{split}\label{a18}
r_{k_i}(x)=1,\quad r_{k_i+1}(x)=-1,\quad i=1,2,\ldots , \nu,\quad \nu\ge \frac{n}{6}-1.
\end{split}\end{multline}\fi
}
 Suppose $\delta_j$ is the dyadic interval of the length $2^{-j}$ containing the point $x$. Observe that {(\ref{{a18}})} is equivalent to the condition
{\ifx01
\begin{equation*} \label{a28}
x\in  \left(\left(\delta_{k_j}\right)^+\right)^-.
 \end{equation*}\fi  
\ifx11\begin{equation}\label{a28}
x\in  \left(\left(\delta_{k_j}\right)^+\right)^-.
\end{equation}\fi   
\ifx21\begin{align*}\label{a28}
x\in  \left(\left(\delta_{k_j}\right)^+\right)^-.
\end{align*}\fi   
\ifx31\begin{align}\label{a28}
x\in  \left(\left(\delta_{k_j}\right)^+\right)^-.
\end{align}\fi    
\ifx41\begin{gather*}\label{a28}
x\in  \left(\left(\delta_{k_j}\right)^+\right)^-.
\end{gather*}\fi  
\ifx51\begin{gather}\label{a28}
x\in  \left(\left(\delta_{k_j}\right)^+\right)^-.
\end{gather}\fi   
\ifx61\begin{multline*}\label{a28}
x\in  \left(\left(\delta_{k_j}\right)^+\right)^-.
\end{multline*}\fi  
\ifx71\begin{multline}\label{a28}
x\in  \left(\left(\delta_{k_j}\right)^+\right)^-.
\end{multline}\fi  
\ifx81\begin{multline*}\begin{split}\label{a28}
x\in  \left(\left(\delta_{k_j}\right)^+\right)^-.
\end{split}\end{multline*}\fi
\ifx91\begin{multline}\begin{split}\label{a28}
x\in  \left(\left(\delta_{k_j}\right)^+\right)^-.
\end{split}\end{multline}\fi
}
This implies
{\ifx03
\begin{equation*} 
&\left(\left(\delta_{k_j}\right)^+\right)^+\subset [0,x),\label{a30}\\
&r_{k_j}(x\oplus t)= 1,\quad t\in\delta_{k_j}\cap[0,x).\label{a29}
 \end{equation*}\fi  
\ifx13\begin{equation}
&\left(\left(\delta_{k_j}\right)^+\right)^+\subset [0,x),\label{a30}\\
&r_{k_j}(x\oplus t)= 1,\quad t\in\delta_{k_j}\cap[0,x).\label{a29}
\end{equation}\fi   
\ifx23\begin{align*}
&\left(\left(\delta_{k_j}\right)^+\right)^+\subset [0,x),\label{a30}\\
&r_{k_j}(x\oplus t)= 1,\quad t\in\delta_{k_j}\cap[0,x).\label{a29}
\end{align*}\fi   
\ifx33\begin{align}
&\left(\left(\delta_{k_j}\right)^+\right)^+\subset [0,x),\label{a30}\\
&r_{k_j}(x\oplus t)= 1,\quad t\in\delta_{k_j}\cap[0,x).\label{a29}
\end{align}\fi    
\ifx43\begin{gather*}
&\left(\left(\delta_{k_j}\right)^+\right)^+\subset [0,x),\label{a30}\\
&r_{k_j}(x\oplus t)= 1,\quad t\in\delta_{k_j}\cap[0,x).\label{a29}
\end{gather*}\fi  
\ifx53\begin{gather}
&\left(\left(\delta_{k_j}\right)^+\right)^+\subset [0,x),\label{a30}\\
&r_{k_j}(x\oplus t)= 1,\quad t\in\delta_{k_j}\cap[0,x).\label{a29}
\end{gather}\fi   
\ifx63\begin{multline*}
&\left(\left(\delta_{k_j}\right)^+\right)^+\subset [0,x),\label{a30}\\
&r_{k_j}(x\oplus t)= 1,\quad t\in\delta_{k_j}\cap[0,x).\label{a29}
\end{multline*}\fi  
\ifx73\begin{multline}
&\left(\left(\delta_{k_j}\right)^+\right)^+\subset [0,x),\label{a30}\\
&r_{k_j}(x\oplus t)= 1,\quad t\in\delta_{k_j}\cap[0,x).\label{a29}
\end{multline}\fi  
\ifx83\begin{multline*}\begin{split}
&\left(\left(\delta_{k_j}\right)^+\right)^+\subset [0,x),\label{a30}\\
&r_{k_j}(x\oplus t)= 1,\quad t\in\delta_{k_j}\cap[0,x).\label{a29}
\end{split}\end{multline*}\fi
\ifx93\begin{multline}\begin{split}
&\left(\left(\delta_{k_j}\right)^+\right)^+\subset [0,x),\label{a30}\\
&r_{k_j}(x\oplus t)= 1,\quad t\in\delta_{k_j}\cap[0,x).\label{a29}
\end{split}\end{multline}\fi
}
Now consider the integer
{\ifx00
\begin{equation*} 
m=2^{k_1}+2^{k_2}+\ldots+2^{k_\nu}.
 \end{equation*}\fi  
\ifx10\begin{equation}
m=2^{k_1}+2^{k_2}+\ldots+2^{k_\nu}.
\end{equation}\fi   
\ifx20\begin{align*}
m=2^{k_1}+2^{k_2}+\ldots+2^{k_\nu}.
\end{align*}\fi   
\ifx30\begin{align}
m=2^{k_1}+2^{k_2}+\ldots+2^{k_\nu}.
\end{align}\fi    
\ifx40\begin{gather*}
m=2^{k_1}+2^{k_2}+\ldots+2^{k_\nu}.
\end{gather*}\fi  
\ifx50\begin{gather}
m=2^{k_1}+2^{k_2}+\ldots+2^{k_\nu}.
\end{gather}\fi   
\ifx60\begin{multline*}
m=2^{k_1}+2^{k_2}+\ldots+2^{k_\nu}.
\end{multline*}\fi  
\ifx70\begin{multline}
m=2^{k_1}+2^{k_2}+\ldots+2^{k_\nu}.
\end{multline}\fi  
\ifx80\begin{multline*}\begin{split}
m=2^{k_1}+2^{k_2}+\ldots+2^{k_\nu}.
\end{split}\end{multline*}\fi
\ifx90\begin{multline}\begin{split}
m=2^{k_1}+2^{k_2}+\ldots+2^{k_\nu}.
\end{split}\end{multline}\fi
}
Using {(\ref{{a30}})} and {(\ref{{a29}})}, we obtain
{\ifx08
\begin{equation*} 
\int_0^x D_m^*(x\oplus t)dt&=\sum_{j=1}^\nu \int_0^x r_{k_j}(x\oplus t)D_{2^{k_j}}(x\oplus t)dt\\
&=\sum_{j=1}^\nu 2^{k_j}\int_{\delta_{k_j}\cap [0,x)} r_{k_j}(x\oplus t)dt\\
&\ge \sum_{j=1}^\nu 2^{k_j}\int_{\left(\left(\delta_{k_j}\right)^+\right)^+} r_{k_j}(x\oplus t)dt\\
&=\sum_{j=1}^\nu 2^{k_j-2}|\delta_{k_j}|=\frac{\nu}{4}>\frac{n}{30}.
 \end{equation*}\fi  
\ifx18\begin{equation}
\int_0^x D_m^*(x\oplus t)dt&=\sum_{j=1}^\nu \int_0^x r_{k_j}(x\oplus t)D_{2^{k_j}}(x\oplus t)dt\\
&=\sum_{j=1}^\nu 2^{k_j}\int_{\delta_{k_j}\cap [0,x)} r_{k_j}(x\oplus t)dt\\
&\ge \sum_{j=1}^\nu 2^{k_j}\int_{\left(\left(\delta_{k_j}\right)^+\right)^+} r_{k_j}(x\oplus t)dt\\
&=\sum_{j=1}^\nu 2^{k_j-2}|\delta_{k_j}|=\frac{\nu}{4}>\frac{n}{30}.
\end{equation}\fi   
\ifx28\begin{align*}
\int_0^x D_m^*(x\oplus t)dt&=\sum_{j=1}^\nu \int_0^x r_{k_j}(x\oplus t)D_{2^{k_j}}(x\oplus t)dt\\
&=\sum_{j=1}^\nu 2^{k_j}\int_{\delta_{k_j}\cap [0,x)} r_{k_j}(x\oplus t)dt\\
&\ge \sum_{j=1}^\nu 2^{k_j}\int_{\left(\left(\delta_{k_j}\right)^+\right)^+} r_{k_j}(x\oplus t)dt\\
&=\sum_{j=1}^\nu 2^{k_j-2}|\delta_{k_j}|=\frac{\nu}{4}>\frac{n}{30}.
\end{align*}\fi   
\ifx38\begin{align}
\int_0^x D_m^*(x\oplus t)dt&=\sum_{j=1}^\nu \int_0^x r_{k_j}(x\oplus t)D_{2^{k_j}}(x\oplus t)dt\\
&=\sum_{j=1}^\nu 2^{k_j}\int_{\delta_{k_j}\cap [0,x)} r_{k_j}(x\oplus t)dt\\
&\ge \sum_{j=1}^\nu 2^{k_j}\int_{\left(\left(\delta_{k_j}\right)^+\right)^+} r_{k_j}(x\oplus t)dt\\
&=\sum_{j=1}^\nu 2^{k_j-2}|\delta_{k_j}|=\frac{\nu}{4}>\frac{n}{30}.
\end{align}\fi    
\ifx48\begin{gather*}
\int_0^x D_m^*(x\oplus t)dt&=\sum_{j=1}^\nu \int_0^x r_{k_j}(x\oplus t)D_{2^{k_j}}(x\oplus t)dt\\
&=\sum_{j=1}^\nu 2^{k_j}\int_{\delta_{k_j}\cap [0,x)} r_{k_j}(x\oplus t)dt\\
&\ge \sum_{j=1}^\nu 2^{k_j}\int_{\left(\left(\delta_{k_j}\right)^+\right)^+} r_{k_j}(x\oplus t)dt\\
&=\sum_{j=1}^\nu 2^{k_j-2}|\delta_{k_j}|=\frac{\nu}{4}>\frac{n}{30}.
\end{gather*}\fi  
\ifx58\begin{gather}
\int_0^x D_m^*(x\oplus t)dt&=\sum_{j=1}^\nu \int_0^x r_{k_j}(x\oplus t)D_{2^{k_j}}(x\oplus t)dt\\
&=\sum_{j=1}^\nu 2^{k_j}\int_{\delta_{k_j}\cap [0,x)} r_{k_j}(x\oplus t)dt\\
&\ge \sum_{j=1}^\nu 2^{k_j}\int_{\left(\left(\delta_{k_j}\right)^+\right)^+} r_{k_j}(x\oplus t)dt\\
&=\sum_{j=1}^\nu 2^{k_j-2}|\delta_{k_j}|=\frac{\nu}{4}>\frac{n}{30}.
\end{gather}\fi   
\ifx68\begin{multline*}
\int_0^x D_m^*(x\oplus t)dt&=\sum_{j=1}^\nu \int_0^x r_{k_j}(x\oplus t)D_{2^{k_j}}(x\oplus t)dt\\
&=\sum_{j=1}^\nu 2^{k_j}\int_{\delta_{k_j}\cap [0,x)} r_{k_j}(x\oplus t)dt\\
&\ge \sum_{j=1}^\nu 2^{k_j}\int_{\left(\left(\delta_{k_j}\right)^+\right)^+} r_{k_j}(x\oplus t)dt\\
&=\sum_{j=1}^\nu 2^{k_j-2}|\delta_{k_j}|=\frac{\nu}{4}>\frac{n}{30}.
\end{multline*}\fi  
\ifx78\begin{multline}
\int_0^x D_m^*(x\oplus t)dt&=\sum_{j=1}^\nu \int_0^x r_{k_j}(x\oplus t)D_{2^{k_j}}(x\oplus t)dt\\
&=\sum_{j=1}^\nu 2^{k_j}\int_{\delta_{k_j}\cap [0,x)} r_{k_j}(x\oplus t)dt\\
&\ge \sum_{j=1}^\nu 2^{k_j}\int_{\left(\left(\delta_{k_j}\right)^+\right)^+} r_{k_j}(x\oplus t)dt\\
&=\sum_{j=1}^\nu 2^{k_j-2}|\delta_{k_j}|=\frac{\nu}{4}>\frac{n}{30}.
\end{multline}\fi  
\ifx88\begin{multline*}\begin{split}
\int_0^x D_m^*(x\oplus t)dt&=\sum_{j=1}^\nu \int_0^x r_{k_j}(x\oplus t)D_{2^{k_j}}(x\oplus t)dt\\
&=\sum_{j=1}^\nu 2^{k_j}\int_{\delta_{k_j}\cap [0,x)} r_{k_j}(x\oplus t)dt\\
&\ge \sum_{j=1}^\nu 2^{k_j}\int_{\left(\left(\delta_{k_j}\right)^+\right)^+} r_{k_j}(x\oplus t)dt\\
&=\sum_{j=1}^\nu 2^{k_j-2}|\delta_{k_j}|=\frac{\nu}{4}>\frac{n}{30}.
\end{split}\end{multline*}\fi
\ifx98\begin{multline}\begin{split}
\int_0^x D_m^*(x\oplus t)dt&=\sum_{j=1}^\nu \int_0^x r_{k_j}(x\oplus t)D_{2^{k_j}}(x\oplus t)dt\\
&=\sum_{j=1}^\nu 2^{k_j}\int_{\delta_{k_j}\cap [0,x)} r_{k_j}(x\oplus t)dt\\
&\ge \sum_{j=1}^\nu 2^{k_j}\int_{\left(\left(\delta_{k_j}\right)^+\right)^+} r_{k_j}(x\oplus t)dt\\
&=\sum_{j=1}^\nu 2^{k_j-2}|\delta_{k_j}|=\frac{\nu}{4}>\frac{n}{30}.
\end{split}\end{multline}\fi
}
\end{proof}
\begin{lemma}\label{L1}
For any integer $n>n_0$ there exists a Walsh polynomial $f(x)=f_n(x)$ such that
{\ifx03
\begin{equation*} 
&\|f\|_1\le 4,\quad {{\rm sp\,}} f(x)\subset [p(n),q(n)],\\
\sup_{N\in [p(n),2q(n)]}&\frac{\#\{k\in {\ensuremath{\mathbb N}}:\,1\le k\le N,\, |S_k(x,f)|>n/40 \}}{N}\gtrsim 2^{-2n},\label{a32}
 \end{equation*}\fi  
\ifx13\begin{equation}
&\|f\|_1\le 4,\quad {{\rm sp\,}} f(x)\subset [p(n),q(n)],\\
\sup_{N\in [p(n),2q(n)]}&\frac{\#\{k\in {\ensuremath{\mathbb N}}:\,1\le k\le N,\, |S_k(x,f)|>n/40 \}}{N}\gtrsim 2^{-2n},\label{a32}
\end{equation}\fi   
\ifx23\begin{align*}
&\|f\|_1\le 4,\quad {{\rm sp\,}} f(x)\subset [p(n),q(n)],\\
\sup_{N\in [p(n),2q(n)]}&\frac{\#\{k\in {\ensuremath{\mathbb N}}:\,1\le k\le N,\, |S_k(x,f)|>n/40 \}}{N}\gtrsim 2^{-2n},\label{a32}
\end{align*}\fi   
\ifx33\begin{align}
&\|f\|_1\le 4,\quad {{\rm sp\,}} f(x)\subset [p(n),q(n)],\\
\sup_{N\in [p(n),2q(n)]}&\frac{\#\{k\in {\ensuremath{\mathbb N}}:\,1\le k\le N,\, |S_k(x,f)|>n/40 \}}{N}\gtrsim 2^{-2n},\label{a32}
\end{align}\fi    
\ifx43\begin{gather*}
&\|f\|_1\le 4,\quad {{\rm sp\,}} f(x)\subset [p(n),q(n)],\\
\sup_{N\in [p(n),2q(n)]}&\frac{\#\{k\in {\ensuremath{\mathbb N}}:\,1\le k\le N,\, |S_k(x,f)|>n/40 \}}{N}\gtrsim 2^{-2n},\label{a32}
\end{gather*}\fi  
\ifx53\begin{gather}
&\|f\|_1\le 4,\quad {{\rm sp\,}} f(x)\subset [p(n),q(n)],\\
\sup_{N\in [p(n),2q(n)]}&\frac{\#\{k\in {\ensuremath{\mathbb N}}:\,1\le k\le N,\, |S_k(x,f)|>n/40 \}}{N}\gtrsim 2^{-2n},\label{a32}
\end{gather}\fi   
\ifx63\begin{multline*}
&\|f\|_1\le 4,\quad {{\rm sp\,}} f(x)\subset [p(n),q(n)],\\
\sup_{N\in [p(n),2q(n)]}&\frac{\#\{k\in {\ensuremath{\mathbb N}}:\,1\le k\le N,\, |S_k(x,f)|>n/40 \}}{N}\gtrsim 2^{-2n},\label{a32}
\end{multline*}\fi  
\ifx73\begin{multline}
&\|f\|_1\le 4,\quad {{\rm sp\,}} f(x)\subset [p(n),q(n)],\\
\sup_{N\in [p(n),2q(n)]}&\frac{\#\{k\in {\ensuremath{\mathbb N}}:\,1\le k\le N,\, |S_k(x,f)|>n/40 \}}{N}\gtrsim 2^{-2n},\label{a32}
\end{multline}\fi  
\ifx83\begin{multline*}\begin{split}
&\|f\|_1\le 4,\quad {{\rm sp\,}} f(x)\subset [p(n),q(n)],\\
\sup_{N\in [p(n),2q(n)]}&\frac{\#\{k\in {\ensuremath{\mathbb N}}:\,1\le k\le N,\, |S_k(x,f)|>n/40 \}}{N}\gtrsim 2^{-2n},\label{a32}
\end{split}\end{multline*}\fi
\ifx93\begin{multline}\begin{split}
&\|f\|_1\le 4,\quad {{\rm sp\,}} f(x)\subset [p(n),q(n)],\\
\sup_{N\in [p(n),2q(n)]}&\frac{\#\{k\in {\ensuremath{\mathbb N}}:\,1\le k\le N,\, |S_k(x,f)|>n/40 \}}{N}\gtrsim 2^{-2n},\label{a32}
\end{split}\end{multline}\fi
}
where $p(n)$, $q(n)$ are positive integers, and $n_0$ is an absolute constant.
\end{lemma}
\begin{proof}
We define
{\ifx00
\begin{equation*} 
\theta_k=\frac{k-1}{2^n}+\frac{k-1}{4^n}\in \Delta_k=\left[\frac{k-1}{2^n},\frac{k}{2^n}\right),\quad k=1,2,\ldots ,2^n.
 \end{equation*}\fi  
\ifx10\begin{equation}
\theta_k=\frac{k-1}{2^n}+\frac{k-1}{4^n}\in \Delta_k=\left[\frac{k-1}{2^n},\frac{k}{2^n}\right),\quad k=1,2,\ldots ,2^n.
\end{equation}\fi   
\ifx20\begin{align*}
\theta_k=\frac{k-1}{2^n}+\frac{k-1}{4^n}\in \Delta_k=\left[\frac{k-1}{2^n},\frac{k}{2^n}\right),\quad k=1,2,\ldots ,2^n.
\end{align*}\fi   
\ifx30\begin{align}
\theta_k=\frac{k-1}{2^n}+\frac{k-1}{4^n}\in \Delta_k=\left[\frac{k-1}{2^n},\frac{k}{2^n}\right),\quad k=1,2,\ldots ,2^n.
\end{align}\fi    
\ifx40\begin{gather*}
\theta_k=\frac{k-1}{2^n}+\frac{k-1}{4^n}\in \Delta_k=\left[\frac{k-1}{2^n},\frac{k}{2^n}\right),\quad k=1,2,\ldots ,2^n.
\end{gather*}\fi  
\ifx50\begin{gather}
\theta_k=\frac{k-1}{2^n}+\frac{k-1}{4^n}\in \Delta_k=\left[\frac{k-1}{2^n},\frac{k}{2^n}\right),\quad k=1,2,\ldots ,2^n.
\end{gather}\fi   
\ifx60\begin{multline*}
\theta_k=\frac{k-1}{2^n}+\frac{k-1}{4^n}\in \Delta_k=\left[\frac{k-1}{2^n},\frac{k}{2^n}\right),\quad k=1,2,\ldots ,2^n.
\end{multline*}\fi  
\ifx70\begin{multline}
\theta_k=\frac{k-1}{2^n}+\frac{k-1}{4^n}\in \Delta_k=\left[\frac{k-1}{2^n},\frac{k}{2^n}\right),\quad k=1,2,\ldots ,2^n.
\end{multline}\fi  
\ifx80\begin{multline*}\begin{split}
\theta_k=\frac{k-1}{2^n}+\frac{k-1}{4^n}\in \Delta_k=\left[\frac{k-1}{2^n},\frac{k}{2^n}\right),\quad k=1,2,\ldots ,2^n.
\end{split}\end{multline*}\fi
\ifx90\begin{multline}\begin{split}
\theta_k=\frac{k-1}{2^n}+\frac{k-1}{4^n}\in \Delta_k=\left[\frac{k-1}{2^n},\frac{k}{2^n}\right),\quad k=1,2,\ldots ,2^n.
\end{split}\end{multline}\fi
}
Let $E_n$ be the set obtained in {Lemma \ref{{L2}}}. We define $f(x)$ by
{\ifx01
\begin{equation*} \label{a11}
f(x)=2^{\gamma}\cdot {\ensuremath{\mathbb I}}_{(E_n)^c}(x)r_n(x)
+\frac{1}{2^{n}}\sum_{j=1}^{2^n}\bigg(D_{u_{2^n}}(x\oplus\theta_j)-D_{u_j}(x\oplus\theta_j)\bigg),
 \end{equation*}\fi  
\ifx11\begin{equation}\label{a11}
f(x)=2^{\gamma}\cdot {\ensuremath{\mathbb I}}_{(E_n)^c}(x)r_n(x)
+\frac{1}{2^{n}}\sum_{j=1}^{2^n}\bigg(D_{u_{2^n}}(x\oplus\theta_j)-D_{u_j}(x\oplus\theta_j)\bigg),
\end{equation}\fi   
\ifx21\begin{align*}\label{a11}
f(x)=2^{\gamma}\cdot {\ensuremath{\mathbb I}}_{(E_n)^c}(x)r_n(x)
+\frac{1}{2^{n}}\sum_{j=1}^{2^n}\bigg(D_{u_{2^n}}(x\oplus\theta_j)-D_{u_j}(x\oplus\theta_j)\bigg),
\end{align*}\fi   
\ifx31\begin{align}\label{a11}
f(x)=2^{\gamma}\cdot {\ensuremath{\mathbb I}}_{(E_n)^c}(x)r_n(x)
+\frac{1}{2^{n}}\sum_{j=1}^{2^n}\bigg(D_{u_{2^n}}(x\oplus\theta_j)-D_{u_j}(x\oplus\theta_j)\bigg),
\end{align}\fi    
\ifx41\begin{gather*}\label{a11}
f(x)=2^{\gamma}\cdot {\ensuremath{\mathbb I}}_{(E_n)^c}(x)r_n(x)
+\frac{1}{2^{n}}\sum_{j=1}^{2^n}\bigg(D_{u_{2^n}}(x\oplus\theta_j)-D_{u_j}(x\oplus\theta_j)\bigg),
\end{gather*}\fi  
\ifx51\begin{gather}\label{a11}
f(x)=2^{\gamma}\cdot {\ensuremath{\mathbb I}}_{(E_n)^c}(x)r_n(x)
+\frac{1}{2^{n}}\sum_{j=1}^{2^n}\bigg(D_{u_{2^n}}(x\oplus\theta_j)-D_{u_j}(x\oplus\theta_j)\bigg),
\end{gather}\fi   
\ifx61\begin{multline*}\label{a11}
f(x)=2^{\gamma}\cdot {\ensuremath{\mathbb I}}_{(E_n)^c}(x)r_n(x)
+\frac{1}{2^{n}}\sum_{j=1}^{2^n}\bigg(D_{u_{2^n}}(x\oplus\theta_j)-D_{u_j}(x\oplus\theta_j)\bigg),
\end{multline*}\fi  
\ifx71\begin{multline}\label{a11}
f(x)=2^{\gamma}\cdot {\ensuremath{\mathbb I}}_{(E_n)^c}(x)r_n(x)
+\frac{1}{2^{n}}\sum_{j=1}^{2^n}\bigg(D_{u_{2^n}}(x\oplus\theta_j)-D_{u_j}(x\oplus\theta_j)\bigg),
\end{multline}\fi  
\ifx81\begin{multline*}\begin{split}\label{a11}
f(x)=2^{\gamma}\cdot {\ensuremath{\mathbb I}}_{(E_n)^c}(x)r_n(x)
+\frac{1}{2^{n}}\sum_{j=1}^{2^n}\bigg(D_{u_{2^n}}(x\oplus\theta_j)-D_{u_j}(x\oplus\theta_j)\bigg),
\end{split}\end{multline*}\fi
\ifx91\begin{multline}\begin{split}\label{a11}
f(x)=2^{\gamma}\cdot {\ensuremath{\mathbb I}}_{(E_n)^c}(x)r_n(x)
+\frac{1}{2^{n}}\sum_{j=1}^{2^n}\bigg(D_{u_{2^n}}(x\oplus\theta_j)-D_{u_j}(x\oplus\theta_j)\bigg),
\end{split}\end{multline}\fi
}
where
{\ifx03
\begin{equation*} 
&\gamma= \big[\log_2(\exp(n/36))\big],\label{a36}\\
&u_j=2^{10 (j+n)},\quad j=1,2,\ldots , 2^n.\label{a2}
 \end{equation*}\fi  
\ifx13\begin{equation}
&\gamma= \big[\log_2(\exp(n/36))\big],\label{a36}\\
&u_j=2^{10 (j+n)},\quad j=1,2,\ldots , 2^n.\label{a2}
\end{equation}\fi   
\ifx23\begin{align*}
&\gamma= \big[\log_2(\exp(n/36))\big],\label{a36}\\
&u_j=2^{10 (j+n)},\quad j=1,2,\ldots , 2^n.\label{a2}
\end{align*}\fi   
\ifx33\begin{align}
&\gamma= \big[\log_2(\exp(n/36))\big],\label{a36}\\
&u_j=2^{10 (j+n)},\quad j=1,2,\ldots , 2^n.\label{a2}
\end{align}\fi    
\ifx43\begin{gather*}
&\gamma= \big[\log_2(\exp(n/36))\big],\label{a36}\\
&u_j=2^{10 (j+n)},\quad j=1,2,\ldots , 2^n.\label{a2}
\end{gather*}\fi  
\ifx53\begin{gather}
&\gamma= \big[\log_2(\exp(n/36))\big],\label{a36}\\
&u_j=2^{10 (j+n)},\quad j=1,2,\ldots , 2^n.\label{a2}
\end{gather}\fi   
\ifx63\begin{multline*}
&\gamma= \big[\log_2(\exp(n/36))\big],\label{a36}\\
&u_j=2^{10 (j+n)},\quad j=1,2,\ldots , 2^n.\label{a2}
\end{multline*}\fi  
\ifx73\begin{multline}
&\gamma= \big[\log_2(\exp(n/36))\big],\label{a36}\\
&u_j=2^{10 (j+n)},\quad j=1,2,\ldots , 2^n.\label{a2}
\end{multline}\fi  
\ifx83\begin{multline*}\begin{split}
&\gamma= \big[\log_2(\exp(n/36))\big],\label{a36}\\
&u_j=2^{10 (j+n)},\quad j=1,2,\ldots , 2^n.\label{a2}
\end{split}\end{multline*}\fi
\ifx93\begin{multline}\begin{split}
&\gamma= \big[\log_2(\exp(n/36))\big],\label{a36}\\
&u_j=2^{10 (j+n)},\quad j=1,2,\ldots , 2^n.\label{a2}
\end{split}\end{multline}\fi
}
We have
{\ifx02
\begin{equation*} 
&{{\rm sp\,}}\left({\ensuremath{\mathbb I}}_{(E_n)^c}(x)r_n(x)\right)\subset [2^n,2^{n+1}),\\
&{{\rm sp\,}}\left(D_{u_{2^n}}(x\oplus\theta_j)-D_{u_j}(x\oplus\theta_j)\right)\subset (u_j,u_{2^n}]\subset [2^n,u_{2^n}],
 \end{equation*}\fi  
\ifx12\begin{equation}
&{{\rm sp\,}}\left({\ensuremath{\mathbb I}}_{(E_n)^c}(x)r_n(x)\right)\subset [2^n,2^{n+1}),\\
&{{\rm sp\,}}\left(D_{u_{2^n}}(x\oplus\theta_j)-D_{u_j}(x\oplus\theta_j)\right)\subset (u_j,u_{2^n}]\subset [2^n,u_{2^n}],
\end{equation}\fi   
\ifx22\begin{align*}
&{{\rm sp\,}}\left({\ensuremath{\mathbb I}}_{(E_n)^c}(x)r_n(x)\right)\subset [2^n,2^{n+1}),\\
&{{\rm sp\,}}\left(D_{u_{2^n}}(x\oplus\theta_j)-D_{u_j}(x\oplus\theta_j)\right)\subset (u_j,u_{2^n}]\subset [2^n,u_{2^n}],
\end{align*}\fi   
\ifx32\begin{align}
&{{\rm sp\,}}\left({\ensuremath{\mathbb I}}_{(E_n)^c}(x)r_n(x)\right)\subset [2^n,2^{n+1}),\\
&{{\rm sp\,}}\left(D_{u_{2^n}}(x\oplus\theta_j)-D_{u_j}(x\oplus\theta_j)\right)\subset (u_j,u_{2^n}]\subset [2^n,u_{2^n}],
\end{align}\fi    
\ifx42\begin{gather*}
&{{\rm sp\,}}\left({\ensuremath{\mathbb I}}_{(E_n)^c}(x)r_n(x)\right)\subset [2^n,2^{n+1}),\\
&{{\rm sp\,}}\left(D_{u_{2^n}}(x\oplus\theta_j)-D_{u_j}(x\oplus\theta_j)\right)\subset (u_j,u_{2^n}]\subset [2^n,u_{2^n}],
\end{gather*}\fi  
\ifx52\begin{gather}
&{{\rm sp\,}}\left({\ensuremath{\mathbb I}}_{(E_n)^c}(x)r_n(x)\right)\subset [2^n,2^{n+1}),\\
&{{\rm sp\,}}\left(D_{u_{2^n}}(x\oplus\theta_j)-D_{u_j}(x\oplus\theta_j)\right)\subset (u_j,u_{2^n}]\subset [2^n,u_{2^n}],
\end{gather}\fi   
\ifx62\begin{multline*}
&{{\rm sp\,}}\left({\ensuremath{\mathbb I}}_{(E_n)^c}(x)r_n(x)\right)\subset [2^n,2^{n+1}),\\
&{{\rm sp\,}}\left(D_{u_{2^n}}(x\oplus\theta_j)-D_{u_j}(x\oplus\theta_j)\right)\subset (u_j,u_{2^n}]\subset [2^n,u_{2^n}],
\end{multline*}\fi  
\ifx72\begin{multline}
&{{\rm sp\,}}\left({\ensuremath{\mathbb I}}_{(E_n)^c}(x)r_n(x)\right)\subset [2^n,2^{n+1}),\\
&{{\rm sp\,}}\left(D_{u_{2^n}}(x\oplus\theta_j)-D_{u_j}(x\oplus\theta_j)\right)\subset (u_j,u_{2^n}]\subset [2^n,u_{2^n}],
\end{multline}\fi  
\ifx82\begin{multline*}\begin{split}
&{{\rm sp\,}}\left({\ensuremath{\mathbb I}}_{(E_n)^c}(x)r_n(x)\right)\subset [2^n,2^{n+1}),\\
&{{\rm sp\,}}\left(D_{u_{2^n}}(x\oplus\theta_j)-D_{u_j}(x\oplus\theta_j)\right)\subset (u_j,u_{2^n}]\subset [2^n,u_{2^n}],
\end{split}\end{multline*}\fi
\ifx92\begin{multline}\begin{split}
&{{\rm sp\,}}\left({\ensuremath{\mathbb I}}_{(E_n)^c}(x)r_n(x)\right)\subset [2^n,2^{n+1}),\\
&{{\rm sp\,}}\left(D_{u_{2^n}}(x\oplus\theta_j)-D_{u_j}(x\oplus\theta_j)\right)\subset (u_j,u_{2^n}]\subset [2^n,u_{2^n}],
\end{split}\end{multline}\fi
}
and therefore
{\ifx00
\begin{equation*} 
{{\rm sp\,}} f(x)\subset [p(n),q(n)],\quad p(n)=2^n,\quad q(n)=u_{2^n}.
 \end{equation*}\fi  
\ifx10\begin{equation}
{{\rm sp\,}} f(x)\subset [p(n),q(n)],\quad p(n)=2^n,\quad q(n)=u_{2^n}.
\end{equation}\fi   
\ifx20\begin{align*}
{{\rm sp\,}} f(x)\subset [p(n),q(n)],\quad p(n)=2^n,\quad q(n)=u_{2^n}.
\end{align*}\fi   
\ifx30\begin{align}
{{\rm sp\,}} f(x)\subset [p(n),q(n)],\quad p(n)=2^n,\quad q(n)=u_{2^n}.
\end{align}\fi    
\ifx40\begin{gather*}
{{\rm sp\,}} f(x)\subset [p(n),q(n)],\quad p(n)=2^n,\quad q(n)=u_{2^n}.
\end{gather*}\fi  
\ifx50\begin{gather}
{{\rm sp\,}} f(x)\subset [p(n),q(n)],\quad p(n)=2^n,\quad q(n)=u_{2^n}.
\end{gather}\fi   
\ifx60\begin{multline*}
{{\rm sp\,}} f(x)\subset [p(n),q(n)],\quad p(n)=2^n,\quad q(n)=u_{2^n}.
\end{multline*}\fi  
\ifx70\begin{multline}
{{\rm sp\,}} f(x)\subset [p(n),q(n)],\quad p(n)=2^n,\quad q(n)=u_{2^n}.
\end{multline}\fi  
\ifx80\begin{multline*}\begin{split}
{{\rm sp\,}} f(x)\subset [p(n),q(n)],\quad p(n)=2^n,\quad q(n)=u_{2^n}.
\end{split}\end{multline*}\fi
\ifx90\begin{multline}\begin{split}
{{\rm sp\,}} f(x)\subset [p(n),q(n)],\quad p(n)=2^n,\quad q(n)=u_{2^n}.
\end{split}\end{multline}\fi
}
Using {(\ref{{a27}})} and {(\ref{{a36}})}, we obtain
{\ifx00
\begin{equation*} 
\|f\|_1\le 2^\gamma(1-|E_n|)+2\le \exp(n/36)\cdot 2\exp(-n/36)+2=  4.
 \end{equation*}\fi  
\ifx10\begin{equation}
\|f\|_1\le 2^\gamma(1-|E_n|)+2\le \exp(n/36)\cdot 2\exp(-n/36)+2=  4.
\end{equation}\fi   
\ifx20\begin{align*}
\|f\|_1\le 2^\gamma(1-|E_n|)+2\le \exp(n/36)\cdot 2\exp(-n/36)+2=  4.
\end{align*}\fi   
\ifx30\begin{align}
\|f\|_1\le 2^\gamma(1-|E_n|)+2\le \exp(n/36)\cdot 2\exp(-n/36)+2=  4.
\end{align}\fi    
\ifx40\begin{gather*}
\|f\|_1\le 2^\gamma(1-|E_n|)+2\le \exp(n/36)\cdot 2\exp(-n/36)+2=  4.
\end{gather*}\fi  
\ifx50\begin{gather}
\|f\|_1\le 2^\gamma(1-|E_n|)+2\le \exp(n/36)\cdot 2\exp(-n/36)+2=  4.
\end{gather}\fi   
\ifx60\begin{multline*}
\|f\|_1\le 2^\gamma(1-|E_n|)+2\le \exp(n/36)\cdot 2\exp(-n/36)+2=  4.
\end{multline*}\fi  
\ifx70\begin{multline}
\|f\|_1\le 2^\gamma(1-|E_n|)+2\le \exp(n/36)\cdot 2\exp(-n/36)+2=  4.
\end{multline}\fi  
\ifx80\begin{multline*}\begin{split}
\|f\|_1\le 2^\gamma(1-|E_n|)+2\le \exp(n/36)\cdot 2\exp(-n/36)+2=  4.
\end{split}\end{multline*}\fi
\ifx90\begin{multline}\begin{split}
\|f\|_1\le 2^\gamma(1-|E_n|)+2\le \exp(n/36)\cdot 2\exp(-n/36)+2=  4.
\end{split}\end{multline}\fi
}
From the expression {(\ref{{a11}})} it follows that any value taken by $f(x)$ is either $0$ or a sum of different numbers of the form
$\pm 2^k$ with $k\ge \gamma$. This implies
{\ifx00
\begin{equation*} 
|f(x)|\ge 2^\gamma\ge \frac{\exp(n/36)}{2}>\frac{n}{40},\quad n>n_0=150,
 \end{equation*}\fi  
\ifx10\begin{equation}
|f(x)|\ge 2^\gamma\ge \frac{\exp(n/36)}{2}>\frac{n}{40},\quad n>n_0=150,
\end{equation}\fi   
\ifx20\begin{align*}
|f(x)|\ge 2^\gamma\ge \frac{\exp(n/36)}{2}>\frac{n}{40},\quad n>n_0=150,
\end{align*}\fi   
\ifx30\begin{align}
|f(x)|\ge 2^\gamma\ge \frac{\exp(n/36)}{2}>\frac{n}{40},\quad n>n_0=150,
\end{align}\fi    
\ifx40\begin{gather*}
|f(x)|\ge 2^\gamma\ge \frac{\exp(n/36)}{2}>\frac{n}{40},\quad n>n_0=150,
\end{gather*}\fi  
\ifx50\begin{gather}
|f(x)|\ge 2^\gamma\ge \frac{\exp(n/36)}{2}>\frac{n}{40},\quad n>n_0=150,
\end{gather}\fi   
\ifx60\begin{multline*}
|f(x)|\ge 2^\gamma\ge \frac{\exp(n/36)}{2}>\frac{n}{40},\quad n>n_0=150,
\end{multline*}\fi  
\ifx70\begin{multline}
|f(x)|\ge 2^\gamma\ge \frac{\exp(n/36)}{2}>\frac{n}{40},\quad n>n_0=150,
\end{multline}\fi  
\ifx80\begin{multline*}\begin{split}
|f(x)|\ge 2^\gamma\ge \frac{\exp(n/36)}{2}>\frac{n}{40},\quad n>n_0=150,
\end{split}\end{multline*}\fi
\ifx90\begin{multline}\begin{split}
|f(x)|\ge 2^\gamma\ge \frac{\exp(n/36)}{2}>\frac{n}{40},\quad n>n_0=150,
\end{split}\end{multline}\fi
}
whenever
{\ifx01
\begin{equation*} \label{a31}
x\in {{\rm supp\,}} f=(E_n)^c\bigcup\left(\bigcup_{j=1}^{2^n-1}(\theta_j\oplus {{\rm supp\,}} D_{u_j})\right).
 \end{equation*}\fi  
\ifx11\begin{equation}\label{a31}
x\in {{\rm supp\,}} f=(E_n)^c\bigcup\left(\bigcup_{j=1}^{2^n-1}(\theta_j\oplus {{\rm supp\,}} D_{u_j})\right).
\end{equation}\fi   
\ifx21\begin{align*}\label{a31}
x\in {{\rm supp\,}} f=(E_n)^c\bigcup\left(\bigcup_{j=1}^{2^n-1}(\theta_j\oplus {{\rm supp\,}} D_{u_j})\right).
\end{align*}\fi   
\ifx31\begin{align}\label{a31}
x\in {{\rm supp\,}} f=(E_n)^c\bigcup\left(\bigcup_{j=1}^{2^n-1}(\theta_j\oplus {{\rm supp\,}} D_{u_j})\right).
\end{align}\fi    
\ifx41\begin{gather*}\label{a31}
x\in {{\rm supp\,}} f=(E_n)^c\bigcup\left(\bigcup_{j=1}^{2^n-1}(\theta_j\oplus {{\rm supp\,}} D_{u_j})\right).
\end{gather*}\fi  
\ifx51\begin{gather}\label{a31}
x\in {{\rm supp\,}} f=(E_n)^c\bigcup\left(\bigcup_{j=1}^{2^n-1}(\theta_j\oplus {{\rm supp\,}} D_{u_j})\right).
\end{gather}\fi   
\ifx61\begin{multline*}\label{a31}
x\in {{\rm supp\,}} f=(E_n)^c\bigcup\left(\bigcup_{j=1}^{2^n-1}(\theta_j\oplus {{\rm supp\,}} D_{u_j})\right).
\end{multline*}\fi  
\ifx71\begin{multline}\label{a31}
x\in {{\rm supp\,}} f=(E_n)^c\bigcup\left(\bigcup_{j=1}^{2^n-1}(\theta_j\oplus {{\rm supp\,}} D_{u_j})\right).
\end{multline}\fi  
\ifx81\begin{multline*}\begin{split}\label{a31}
x\in {{\rm supp\,}} f=(E_n)^c\bigcup\left(\bigcup_{j=1}^{2^n-1}(\theta_j\oplus {{\rm supp\,}} D_{u_j})\right).
\end{split}\end{multline*}\fi
\ifx91\begin{multline}\begin{split}\label{a31}
x\in {{\rm supp\,}} f=(E_n)^c\bigcup\left(\bigcup_{j=1}^{2^n-1}(\theta_j\oplus {{\rm supp\,}} D_{u_j})\right).
\end{split}\end{multline}\fi
}
On the other hand if $l\ge q(n)$ and $x$ satisfies {(\ref{{a31}})}, then we have
{\ifx00
\begin{equation*} 
|S_l(x,f)|=|f(x)|> \frac{n}{40}.
 \end{equation*}\fi  
\ifx10\begin{equation}
|S_l(x,f)|=|f(x)|> \frac{n}{40}.
\end{equation}\fi   
\ifx20\begin{align*}
|S_l(x,f)|=|f(x)|> \frac{n}{40}.
\end{align*}\fi   
\ifx30\begin{align}
|S_l(x,f)|=|f(x)|> \frac{n}{40}.
\end{align}\fi    
\ifx40\begin{gather*}
|S_l(x,f)|=|f(x)|> \frac{n}{40}.
\end{gather*}\fi  
\ifx50\begin{gather}
|S_l(x,f)|=|f(x)|> \frac{n}{40}.
\end{gather}\fi   
\ifx60\begin{multline*}
|S_l(x,f)|=|f(x)|> \frac{n}{40}.
\end{multline*}\fi  
\ifx70\begin{multline}
|S_l(x,f)|=|f(x)|> \frac{n}{40}.
\end{multline}\fi  
\ifx80\begin{multline*}\begin{split}
|S_l(x,f)|=|f(x)|> \frac{n}{40}.
\end{split}\end{multline*}\fi
\ifx90\begin{multline}\begin{split}
|S_l(x,f)|=|f(x)|> \frac{n}{40}.
\end{split}\end{multline}\fi
}
Thus we obtain
{\ifx00
\begin{equation*} 
\frac{\#\{k\in {\ensuremath{\mathbb N}}:\,1\le k\le 2q(n),\, |S_k(x,f)|>n/40 \}}{2q(n)}\ge \frac{1}{2}> 2^{-2n},
 \end{equation*}\fi  
\ifx10\begin{equation}
\frac{\#\{k\in {\ensuremath{\mathbb N}}:\,1\le k\le 2q(n),\, |S_k(x,f)|>n/40 \}}{2q(n)}\ge \frac{1}{2}> 2^{-2n},
\end{equation}\fi   
\ifx20\begin{align*}
\frac{\#\{k\in {\ensuremath{\mathbb N}}:\,1\le k\le 2q(n),\, |S_k(x,f)|>n/40 \}}{2q(n)}\ge \frac{1}{2}> 2^{-2n},
\end{align*}\fi   
\ifx30\begin{align}
\frac{\#\{k\in {\ensuremath{\mathbb N}}:\,1\le k\le 2q(n),\, |S_k(x,f)|>n/40 \}}{2q(n)}\ge \frac{1}{2}> 2^{-2n},
\end{align}\fi    
\ifx40\begin{gather*}
\frac{\#\{k\in {\ensuremath{\mathbb N}}:\,1\le k\le 2q(n),\, |S_k(x,f)|>n/40 \}}{2q(n)}\ge \frac{1}{2}> 2^{-2n},
\end{gather*}\fi  
\ifx50\begin{gather}
\frac{\#\{k\in {\ensuremath{\mathbb N}}:\,1\le k\le 2q(n),\, |S_k(x,f)|>n/40 \}}{2q(n)}\ge \frac{1}{2}> 2^{-2n},
\end{gather}\fi   
\ifx60\begin{multline*}
\frac{\#\{k\in {\ensuremath{\mathbb N}}:\,1\le k\le 2q(n),\, |S_k(x,f)|>n/40 \}}{2q(n)}\ge \frac{1}{2}> 2^{-2n},
\end{multline*}\fi  
\ifx70\begin{multline}
\frac{\#\{k\in {\ensuremath{\mathbb N}}:\,1\le k\le 2q(n),\, |S_k(x,f)|>n/40 \}}{2q(n)}\ge \frac{1}{2}> 2^{-2n},
\end{multline}\fi  
\ifx80\begin{multline*}\begin{split}
\frac{\#\{k\in {\ensuremath{\mathbb N}}:\,1\le k\le 2q(n),\, |S_k(x,f)|>n/40 \}}{2q(n)}\ge \frac{1}{2}> 2^{-2n},
\end{split}\end{multline*}\fi
\ifx90\begin{multline}\begin{split}
\frac{\#\{k\in {\ensuremath{\mathbb N}}:\,1\le k\le 2q(n),\, |S_k(x,f)|>n/40 \}}{2q(n)}\ge \frac{1}{2}> 2^{-2n},
\end{split}\end{multline}\fi
}
which implies {(\ref{{a32}})}. Now consider the case when {(\ref{{a31}})} doesn't hold. We may suppose that
{\ifx01
\begin{equation*} \label{a22}
x\in \Delta_k\setminus {{\rm supp\,}} f,\quad 1\le k\le 2^n.
 \end{equation*}\fi  
\ifx11\begin{equation}\label{a22}
x\in \Delta_k\setminus {{\rm supp\,}} f,\quad 1\le k\le 2^n.
\end{equation}\fi   
\ifx21\begin{align*}\label{a22}
x\in \Delta_k\setminus {{\rm supp\,}} f,\quad 1\le k\le 2^n.
\end{align*}\fi   
\ifx31\begin{align}\label{a22}
x\in \Delta_k\setminus {{\rm supp\,}} f,\quad 1\le k\le 2^n.
\end{align}\fi    
\ifx41\begin{gather*}\label{a22}
x\in \Delta_k\setminus {{\rm supp\,}} f,\quad 1\le k\le 2^n.
\end{gather*}\fi  
\ifx51\begin{gather}\label{a22}
x\in \Delta_k\setminus {{\rm supp\,}} f,\quad 1\le k\le 2^n.
\end{gather}\fi   
\ifx61\begin{multline*}\label{a22}
x\in \Delta_k\setminus {{\rm supp\,}} f,\quad 1\le k\le 2^n.
\end{multline*}\fi  
\ifx71\begin{multline}\label{a22}
x\in \Delta_k\setminus {{\rm supp\,}} f,\quad 1\le k\le 2^n.
\end{multline}\fi  
\ifx81\begin{multline*}\begin{split}\label{a22}
x\in \Delta_k\setminus {{\rm supp\,}} f,\quad 1\le k\le 2^n.
\end{split}\end{multline*}\fi
\ifx91\begin{multline}\begin{split}\label{a22}
x\in \Delta_k\setminus {{\rm supp\,}} f,\quad 1\le k\le 2^n.
\end{split}\end{multline}\fi
}
According to {Lemma \ref{{L2}}}, there exists an integer $m=m(x)<2^n$ satisfying one of the inequality {(\ref{{a20}})}. First we suppose it satisfies the first one.
Together with $m$ we consider
{\ifx00
\begin{equation*} 
p=p(x)=m(x)(1+2^n)<2^{2n}.
 \end{equation*}\fi  
\ifx10\begin{equation}
p=p(x)=m(x)(1+2^n)<2^{2n}.
\end{equation}\fi   
\ifx20\begin{align*}
p=p(x)=m(x)(1+2^n)<2^{2n}.
\end{align*}\fi   
\ifx30\begin{align}
p=p(x)=m(x)(1+2^n)<2^{2n}.
\end{align}\fi    
\ifx40\begin{gather*}
p=p(x)=m(x)(1+2^n)<2^{2n}.
\end{gather*}\fi  
\ifx50\begin{gather}
p=p(x)=m(x)(1+2^n)<2^{2n}.
\end{gather}\fi   
\ifx60\begin{multline*}
p=p(x)=m(x)(1+2^n)<2^{2n}.
\end{multline*}\fi  
\ifx70\begin{multline}
p=p(x)=m(x)(1+2^n)<2^{2n}.
\end{multline}\fi  
\ifx80\begin{multline*}\begin{split}
p=p(x)=m(x)(1+2^n)<2^{2n}.
\end{split}\end{multline*}\fi
\ifx90\begin{multline}\begin{split}
p=p(x)=m(x)(1+2^n)<2^{2n}.
\end{split}\end{multline}\fi
}
Using the definition of $\theta_j$, observe, that
{\ifx02
\begin{equation*} 
&w_m(\theta_k)=w_m\left(\frac{k-1}{2^n}\right),\\
&w_{m\cdot 2^n}(\theta_k)=w_{m\cdot 2^n}\left(\frac{k-1}{4^n}\right)=w_m\left(\frac{k-1}{2^n}\right),
 \end{equation*}\fi  
\ifx12\begin{equation}
&w_m(\theta_k)=w_m\left(\frac{k-1}{2^n}\right),\\
&w_{m\cdot 2^n}(\theta_k)=w_{m\cdot 2^n}\left(\frac{k-1}{4^n}\right)=w_m\left(\frac{k-1}{2^n}\right),
\end{equation}\fi   
\ifx22\begin{align*}
&w_m(\theta_k)=w_m\left(\frac{k-1}{2^n}\right),\\
&w_{m\cdot 2^n}(\theta_k)=w_{m\cdot 2^n}\left(\frac{k-1}{4^n}\right)=w_m\left(\frac{k-1}{2^n}\right),
\end{align*}\fi   
\ifx32\begin{align}
&w_m(\theta_k)=w_m\left(\frac{k-1}{2^n}\right),\\
&w_{m\cdot 2^n}(\theta_k)=w_{m\cdot 2^n}\left(\frac{k-1}{4^n}\right)=w_m\left(\frac{k-1}{2^n}\right),
\end{align}\fi    
\ifx42\begin{gather*}
&w_m(\theta_k)=w_m\left(\frac{k-1}{2^n}\right),\\
&w_{m\cdot 2^n}(\theta_k)=w_{m\cdot 2^n}\left(\frac{k-1}{4^n}\right)=w_m\left(\frac{k-1}{2^n}\right),
\end{gather*}\fi  
\ifx52\begin{gather}
&w_m(\theta_k)=w_m\left(\frac{k-1}{2^n}\right),\\
&w_{m\cdot 2^n}(\theta_k)=w_{m\cdot 2^n}\left(\frac{k-1}{4^n}\right)=w_m\left(\frac{k-1}{2^n}\right),
\end{gather}\fi   
\ifx62\begin{multline*}
&w_m(\theta_k)=w_m\left(\frac{k-1}{2^n}\right),\\
&w_{m\cdot 2^n}(\theta_k)=w_{m\cdot 2^n}\left(\frac{k-1}{4^n}\right)=w_m\left(\frac{k-1}{2^n}\right),
\end{multline*}\fi  
\ifx72\begin{multline}
&w_m(\theta_k)=w_m\left(\frac{k-1}{2^n}\right),\\
&w_{m\cdot 2^n}(\theta_k)=w_{m\cdot 2^n}\left(\frac{k-1}{4^n}\right)=w_m\left(\frac{k-1}{2^n}\right),
\end{multline}\fi  
\ifx82\begin{multline*}\begin{split}
&w_m(\theta_k)=w_m\left(\frac{k-1}{2^n}\right),\\
&w_{m\cdot 2^n}(\theta_k)=w_{m\cdot 2^n}\left(\frac{k-1}{4^n}\right)=w_m\left(\frac{k-1}{2^n}\right),
\end{split}\end{multline*}\fi
\ifx92\begin{multline}\begin{split}
&w_m(\theta_k)=w_m\left(\frac{k-1}{2^n}\right),\\
&w_{m\cdot 2^n}(\theta_k)=w_{m\cdot 2^n}\left(\frac{k-1}{4^n}\right)=w_m\left(\frac{k-1}{2^n}\right),
\end{split}\end{multline}\fi
}
and therefore we get
{\ifx01
\begin{equation*} \label{a13}
w_p(\theta_k)=w_m(\theta_k)w_{m\cdot 2^n}(\theta_k)=1,\quad k=1,2,\ldots, 2^n.
 \end{equation*}\fi  
\ifx11\begin{equation}\label{a13}
w_p(\theta_k)=w_m(\theta_k)w_{m\cdot 2^n}(\theta_k)=1,\quad k=1,2,\ldots, 2^n.
\end{equation}\fi   
\ifx21\begin{align*}\label{a13}
w_p(\theta_k)=w_m(\theta_k)w_{m\cdot 2^n}(\theta_k)=1,\quad k=1,2,\ldots, 2^n.
\end{align*}\fi   
\ifx31\begin{align}\label{a13}
w_p(\theta_k)=w_m(\theta_k)w_{m\cdot 2^n}(\theta_k)=1,\quad k=1,2,\ldots, 2^n.
\end{align}\fi    
\ifx41\begin{gather*}\label{a13}
w_p(\theta_k)=w_m(\theta_k)w_{m\cdot 2^n}(\theta_k)=1,\quad k=1,2,\ldots, 2^n.
\end{gather*}\fi  
\ifx51\begin{gather}\label{a13}
w_p(\theta_k)=w_m(\theta_k)w_{m\cdot 2^n}(\theta_k)=1,\quad k=1,2,\ldots, 2^n.
\end{gather}\fi   
\ifx61\begin{multline*}\label{a13}
w_p(\theta_k)=w_m(\theta_k)w_{m\cdot 2^n}(\theta_k)=1,\quad k=1,2,\ldots, 2^n.
\end{multline*}\fi  
\ifx71\begin{multline}\label{a13}
w_p(\theta_k)=w_m(\theta_k)w_{m\cdot 2^n}(\theta_k)=1,\quad k=1,2,\ldots, 2^n.
\end{multline}\fi  
\ifx81\begin{multline*}\begin{split}\label{a13}
w_p(\theta_k)=w_m(\theta_k)w_{m\cdot 2^n}(\theta_k)=1,\quad k=1,2,\ldots, 2^n.
\end{split}\end{multline*}\fi
\ifx91\begin{multline}\begin{split}\label{a13}
w_p(\theta_k)=w_m(\theta_k)w_{m\cdot 2^n}(\theta_k)=1,\quad k=1,2,\ldots, 2^n.
\end{split}\end{multline}\fi
}
Define
{\ifx01
\begin{equation*} \label{a6}
L(x)=\{l\in{\ensuremath{\mathbb N}}:\, l=p+\mu\cdot2^{2n}, \mu \in {\ensuremath{\mathbb N}}\}.
 \end{equation*}\fi  
\ifx11\begin{equation}\label{a6}
L(x)=\{l\in{\ensuremath{\mathbb N}}:\, l=p+\mu\cdot2^{2n}, \mu \in {\ensuremath{\mathbb N}}\}.
\end{equation}\fi   
\ifx21\begin{align*}\label{a6}
L(x)=\{l\in{\ensuremath{\mathbb N}}:\, l=p+\mu\cdot2^{2n}, \mu \in {\ensuremath{\mathbb N}}\}.
\end{align*}\fi   
\ifx31\begin{align}\label{a6}
L(x)=\{l\in{\ensuremath{\mathbb N}}:\, l=p+\mu\cdot2^{2n}, \mu \in {\ensuremath{\mathbb N}}\}.
\end{align}\fi    
\ifx41\begin{gather*}\label{a6}
L(x)=\{l\in{\ensuremath{\mathbb N}}:\, l=p+\mu\cdot2^{2n}, \mu \in {\ensuremath{\mathbb N}}\}.
\end{gather*}\fi  
\ifx51\begin{gather}\label{a6}
L(x)=\{l\in{\ensuremath{\mathbb N}}:\, l=p+\mu\cdot2^{2n}, \mu \in {\ensuremath{\mathbb N}}\}.
\end{gather}\fi   
\ifx61\begin{multline*}\label{a6}
L(x)=\{l\in{\ensuremath{\mathbb N}}:\, l=p+\mu\cdot2^{2n}, \mu \in {\ensuremath{\mathbb N}}\}.
\end{multline*}\fi  
\ifx71\begin{multline}\label{a6}
L(x)=\{l\in{\ensuremath{\mathbb N}}:\, l=p+\mu\cdot2^{2n}, \mu \in {\ensuremath{\mathbb N}}\}.
\end{multline}\fi  
\ifx81\begin{multline*}\begin{split}\label{a6}
L(x)=\{l\in{\ensuremath{\mathbb N}}:\, l=p+\mu\cdot2^{2n}, \mu \in {\ensuremath{\mathbb N}}\}.
\end{split}\end{multline*}\fi
\ifx91\begin{multline}\begin{split}\label{a6}
L(x)=\{l\in{\ensuremath{\mathbb N}}:\, l=p+\mu\cdot2^{2n}, \mu \in {\ensuremath{\mathbb N}}\}.
\end{split}\end{multline}\fi
}
Once again using the definition of $\theta_k$ as well as {(\ref{{a13}})}, we conclude
{\ifx01
\begin{equation*} \label{a14}
w_l(\theta_k)=w_p(\theta_k)w_{\mu\cdot 2^{2n}}(\theta_k)=1,\quad k=1,2,\ldots,2^n,\quad l\in L(x).
 \end{equation*}\fi  
\ifx11\begin{equation}\label{a14}
w_l(\theta_k)=w_p(\theta_k)w_{\mu\cdot 2^{2n}}(\theta_k)=1,\quad k=1,2,\ldots,2^n,\quad l\in L(x).
\end{equation}\fi   
\ifx21\begin{align*}\label{a14}
w_l(\theta_k)=w_p(\theta_k)w_{\mu\cdot 2^{2n}}(\theta_k)=1,\quad k=1,2,\ldots,2^n,\quad l\in L(x).
\end{align*}\fi   
\ifx31\begin{align}\label{a14}
w_l(\theta_k)=w_p(\theta_k)w_{\mu\cdot 2^{2n}}(\theta_k)=1,\quad k=1,2,\ldots,2^n,\quad l\in L(x).
\end{align}\fi    
\ifx41\begin{gather*}\label{a14}
w_l(\theta_k)=w_p(\theta_k)w_{\mu\cdot 2^{2n}}(\theta_k)=1,\quad k=1,2,\ldots,2^n,\quad l\in L(x).
\end{gather*}\fi  
\ifx51\begin{gather}\label{a14}
w_l(\theta_k)=w_p(\theta_k)w_{\mu\cdot 2^{2n}}(\theta_k)=1,\quad k=1,2,\ldots,2^n,\quad l\in L(x).
\end{gather}\fi   
\ifx61\begin{multline*}\label{a14}
w_l(\theta_k)=w_p(\theta_k)w_{\mu\cdot 2^{2n}}(\theta_k)=1,\quad k=1,2,\ldots,2^n,\quad l\in L(x).
\end{multline*}\fi  
\ifx71\begin{multline}\label{a14}
w_l(\theta_k)=w_p(\theta_k)w_{\mu\cdot 2^{2n}}(\theta_k)=1,\quad k=1,2,\ldots,2^n,\quad l\in L(x).
\end{multline}\fi  
\ifx81\begin{multline*}\begin{split}\label{a14}
w_l(\theta_k)=w_p(\theta_k)w_{\mu\cdot 2^{2n}}(\theta_k)=1,\quad k=1,2,\ldots,2^n,\quad l\in L(x).
\end{split}\end{multline*}\fi
\ifx91\begin{multline}\begin{split}\label{a14}
w_l(\theta_k)=w_p(\theta_k)w_{\mu\cdot 2^{2n}}(\theta_k)=1,\quad k=1,2,\ldots,2^n,\quad l\in L(x).
\end{split}\end{multline}\fi
}
Suppose
{\ifx01
\begin{equation*} \label{a16}
l\in L(x)\cap [u_{k-1},u_k),\quad k\le 2^n.
 \end{equation*}\fi  
\ifx11\begin{equation}\label{a16}
l\in L(x)\cap [u_{k-1},u_k),\quad k\le 2^n.
\end{equation}\fi   
\ifx21\begin{align*}\label{a16}
l\in L(x)\cap [u_{k-1},u_k),\quad k\le 2^n.
\end{align*}\fi   
\ifx31\begin{align}\label{a16}
l\in L(x)\cap [u_{k-1},u_k),\quad k\le 2^n.
\end{align}\fi    
\ifx41\begin{gather*}\label{a16}
l\in L(x)\cap [u_{k-1},u_k),\quad k\le 2^n.
\end{gather*}\fi  
\ifx51\begin{gather}\label{a16}
l\in L(x)\cap [u_{k-1},u_k),\quad k\le 2^n.
\end{gather}\fi   
\ifx61\begin{multline*}\label{a16}
l\in L(x)\cap [u_{k-1},u_k),\quad k\le 2^n.
\end{multline*}\fi  
\ifx71\begin{multline}\label{a16}
l\in L(x)\cap [u_{k-1},u_k),\quad k\le 2^n.
\end{multline}\fi  
\ifx81\begin{multline*}\begin{split}\label{a16}
l\in L(x)\cap [u_{k-1},u_k),\quad k\le 2^n.
\end{split}\end{multline*}\fi
\ifx91\begin{multline}\begin{split}\label{a16}
l\in L(x)\cap [u_{k-1},u_k),\quad k\le 2^n.
\end{split}\end{multline}\fi
}
Since $x$ is taken outside of ${{\rm supp\,}} f$, we have
{\ifx09
\begin{equation*} \label{a3}
S_l(x,f)&=\frac{1}{2^{n}}\left(\sum_{j=1}^{k-1}D_l(x\oplus\theta_j)-\sum_{j=1}^{k-1}D_{u_j}(x\oplus\theta_j)\right)\\
&=\frac{1}{2^{n}}\sum_{j=1}^{k-1}D_l(x\oplus\theta_j).
 \end{equation*}\fi  
\ifx19\begin{equation}\label{a3}
S_l(x,f)&=\frac{1}{2^{n}}\left(\sum_{j=1}^{k-1}D_l(x\oplus\theta_j)-\sum_{j=1}^{k-1}D_{u_j}(x\oplus\theta_j)\right)\\
&=\frac{1}{2^{n}}\sum_{j=1}^{k-1}D_l(x\oplus\theta_j).
\end{equation}\fi   
\ifx29\begin{align*}\label{a3}
S_l(x,f)&=\frac{1}{2^{n}}\left(\sum_{j=1}^{k-1}D_l(x\oplus\theta_j)-\sum_{j=1}^{k-1}D_{u_j}(x\oplus\theta_j)\right)\\
&=\frac{1}{2^{n}}\sum_{j=1}^{k-1}D_l(x\oplus\theta_j).
\end{align*}\fi   
\ifx39\begin{align}\label{a3}
S_l(x,f)&=\frac{1}{2^{n}}\left(\sum_{j=1}^{k-1}D_l(x\oplus\theta_j)-\sum_{j=1}^{k-1}D_{u_j}(x\oplus\theta_j)\right)\\
&=\frac{1}{2^{n}}\sum_{j=1}^{k-1}D_l(x\oplus\theta_j).
\end{align}\fi    
\ifx49\begin{gather*}\label{a3}
S_l(x,f)&=\frac{1}{2^{n}}\left(\sum_{j=1}^{k-1}D_l(x\oplus\theta_j)-\sum_{j=1}^{k-1}D_{u_j}(x\oplus\theta_j)\right)\\
&=\frac{1}{2^{n}}\sum_{j=1}^{k-1}D_l(x\oplus\theta_j).
\end{gather*}\fi  
\ifx59\begin{gather}\label{a3}
S_l(x,f)&=\frac{1}{2^{n}}\left(\sum_{j=1}^{k-1}D_l(x\oplus\theta_j)-\sum_{j=1}^{k-1}D_{u_j}(x\oplus\theta_j)\right)\\
&=\frac{1}{2^{n}}\sum_{j=1}^{k-1}D_l(x\oplus\theta_j).
\end{gather}\fi   
\ifx69\begin{multline*}\label{a3}
S_l(x,f)&=\frac{1}{2^{n}}\left(\sum_{j=1}^{k-1}D_l(x\oplus\theta_j)-\sum_{j=1}^{k-1}D_{u_j}(x\oplus\theta_j)\right)\\
&=\frac{1}{2^{n}}\sum_{j=1}^{k-1}D_l(x\oplus\theta_j).
\end{multline*}\fi  
\ifx79\begin{multline}\label{a3}
S_l(x,f)&=\frac{1}{2^{n}}\left(\sum_{j=1}^{k-1}D_l(x\oplus\theta_j)-\sum_{j=1}^{k-1}D_{u_j}(x\oplus\theta_j)\right)\\
&=\frac{1}{2^{n}}\sum_{j=1}^{k-1}D_l(x\oplus\theta_j).
\end{multline}\fi  
\ifx89\begin{multline*}\begin{split}\label{a3}
S_l(x,f)&=\frac{1}{2^{n}}\left(\sum_{j=1}^{k-1}D_l(x\oplus\theta_j)-\sum_{j=1}^{k-1}D_{u_j}(x\oplus\theta_j)\right)\\
&=\frac{1}{2^{n}}\sum_{j=1}^{k-1}D_l(x\oplus\theta_j).
\end{split}\end{multline*}\fi
\ifx99\begin{multline}\begin{split}\label{a3}
S_l(x,f)&=\frac{1}{2^{n}}\left(\sum_{j=1}^{k-1}D_l(x\oplus\theta_j)-\sum_{j=1}^{k-1}D_{u_j}(x\oplus\theta_j)\right)\\
&=\frac{1}{2^{n}}\sum_{j=1}^{k-1}D_l(x\oplus\theta_j).
\end{split}\end{multline}\fi
}
On the other hand by {(\ref{{a14}})} we get
{\ifx09
\begin{equation*} \label{a12}
\frac{1}{2^{n}}\left|\sum_{j=1}^{k-1}D_l(x\oplus\theta_j)\right|
&=\frac{1}{2^{n}}\left|\sum_{j=1}^{k-1}w_l\left(\theta_j\right)D_l^*(x\oplus\theta_j)\right|\\
&=\frac{1}{2^{n}}\left|\sum_{j=1}^{k-1}D_l^*(x\oplus\theta_j)\right|.
 \end{equation*}\fi  
\ifx19\begin{equation}\label{a12}
\frac{1}{2^{n}}\left|\sum_{j=1}^{k-1}D_l(x\oplus\theta_j)\right|
&=\frac{1}{2^{n}}\left|\sum_{j=1}^{k-1}w_l\left(\theta_j\right)D_l^*(x\oplus\theta_j)\right|\\
&=\frac{1}{2^{n}}\left|\sum_{j=1}^{k-1}D_l^*(x\oplus\theta_j)\right|.
\end{equation}\fi   
\ifx29\begin{align*}\label{a12}
\frac{1}{2^{n}}\left|\sum_{j=1}^{k-1}D_l(x\oplus\theta_j)\right|
&=\frac{1}{2^{n}}\left|\sum_{j=1}^{k-1}w_l\left(\theta_j\right)D_l^*(x\oplus\theta_j)\right|\\
&=\frac{1}{2^{n}}\left|\sum_{j=1}^{k-1}D_l^*(x\oplus\theta_j)\right|.
\end{align*}\fi   
\ifx39\begin{align}\label{a12}
\frac{1}{2^{n}}\left|\sum_{j=1}^{k-1}D_l(x\oplus\theta_j)\right|
&=\frac{1}{2^{n}}\left|\sum_{j=1}^{k-1}w_l\left(\theta_j\right)D_l^*(x\oplus\theta_j)\right|\\
&=\frac{1}{2^{n}}\left|\sum_{j=1}^{k-1}D_l^*(x\oplus\theta_j)\right|.
\end{align}\fi    
\ifx49\begin{gather*}\label{a12}
\frac{1}{2^{n}}\left|\sum_{j=1}^{k-1}D_l(x\oplus\theta_j)\right|
&=\frac{1}{2^{n}}\left|\sum_{j=1}^{k-1}w_l\left(\theta_j\right)D_l^*(x\oplus\theta_j)\right|\\
&=\frac{1}{2^{n}}\left|\sum_{j=1}^{k-1}D_l^*(x\oplus\theta_j)\right|.
\end{gather*}\fi  
\ifx59\begin{gather}\label{a12}
\frac{1}{2^{n}}\left|\sum_{j=1}^{k-1}D_l(x\oplus\theta_j)\right|
&=\frac{1}{2^{n}}\left|\sum_{j=1}^{k-1}w_l\left(\theta_j\right)D_l^*(x\oplus\theta_j)\right|\\
&=\frac{1}{2^{n}}\left|\sum_{j=1}^{k-1}D_l^*(x\oplus\theta_j)\right|.
\end{gather}\fi   
\ifx69\begin{multline*}\label{a12}
\frac{1}{2^{n}}\left|\sum_{j=1}^{k-1}D_l(x\oplus\theta_j)\right|
&=\frac{1}{2^{n}}\left|\sum_{j=1}^{k-1}w_l\left(\theta_j\right)D_l^*(x\oplus\theta_j)\right|\\
&=\frac{1}{2^{n}}\left|\sum_{j=1}^{k-1}D_l^*(x\oplus\theta_j)\right|.
\end{multline*}\fi  
\ifx79\begin{multline}\label{a12}
\frac{1}{2^{n}}\left|\sum_{j=1}^{k-1}D_l(x\oplus\theta_j)\right|
&=\frac{1}{2^{n}}\left|\sum_{j=1}^{k-1}w_l\left(\theta_j\right)D_l^*(x\oplus\theta_j)\right|\\
&=\frac{1}{2^{n}}\left|\sum_{j=1}^{k-1}D_l^*(x\oplus\theta_j)\right|.
\end{multline}\fi  
\ifx89\begin{multline*}\begin{split}\label{a12}
\frac{1}{2^{n}}\left|\sum_{j=1}^{k-1}D_l(x\oplus\theta_j)\right|
&=\frac{1}{2^{n}}\left|\sum_{j=1}^{k-1}w_l\left(\theta_j\right)D_l^*(x\oplus\theta_j)\right|\\
&=\frac{1}{2^{n}}\left|\sum_{j=1}^{k-1}D_l^*(x\oplus\theta_j)\right|.
\end{split}\end{multline*}\fi
\ifx99\begin{multline}\begin{split}\label{a12}
\frac{1}{2^{n}}\left|\sum_{j=1}^{k-1}D_l(x\oplus\theta_j)\right|
&=\frac{1}{2^{n}}\left|\sum_{j=1}^{k-1}w_l\left(\theta_j\right)D_l^*(x\oplus\theta_j)\right|\\
&=\frac{1}{2^{n}}\left|\sum_{j=1}^{k-1}D_l^*(x\oplus\theta_j)\right|.
\end{split}\end{multline}\fi
}
Using the definition of $D_l^*(x)$, observe that
{\ifx00
\begin{equation*} 
D_l^*(x)=D_p^*(x)+D_{\mu\cdot 2^{2n}}^*(x)=D_{m}^*(x)+D_{m\cdot 2^n}^*(x)+D_{\mu\cdot 2^{2n}}^*(x).
 \end{equation*}\fi  
\ifx10\begin{equation}
D_l^*(x)=D_p^*(x)+D_{\mu\cdot 2^{2n}}^*(x)=D_{m}^*(x)+D_{m\cdot 2^n}^*(x)+D_{\mu\cdot 2^{2n}}^*(x).
\end{equation}\fi   
\ifx20\begin{align*}
D_l^*(x)=D_p^*(x)+D_{\mu\cdot 2^{2n}}^*(x)=D_{m}^*(x)+D_{m\cdot 2^n}^*(x)+D_{\mu\cdot 2^{2n}}^*(x).
\end{align*}\fi   
\ifx30\begin{align}
D_l^*(x)=D_p^*(x)+D_{\mu\cdot 2^{2n}}^*(x)=D_{m}^*(x)+D_{m\cdot 2^n}^*(x)+D_{\mu\cdot 2^{2n}}^*(x).
\end{align}\fi    
\ifx40\begin{gather*}
D_l^*(x)=D_p^*(x)+D_{\mu\cdot 2^{2n}}^*(x)=D_{m}^*(x)+D_{m\cdot 2^n}^*(x)+D_{\mu\cdot 2^{2n}}^*(x).
\end{gather*}\fi  
\ifx50\begin{gather}
D_l^*(x)=D_p^*(x)+D_{\mu\cdot 2^{2n}}^*(x)=D_{m}^*(x)+D_{m\cdot 2^n}^*(x)+D_{\mu\cdot 2^{2n}}^*(x).
\end{gather}\fi   
\ifx60\begin{multline*}
D_l^*(x)=D_p^*(x)+D_{\mu\cdot 2^{2n}}^*(x)=D_{m}^*(x)+D_{m\cdot 2^n}^*(x)+D_{\mu\cdot 2^{2n}}^*(x).
\end{multline*}\fi  
\ifx70\begin{multline}
D_l^*(x)=D_p^*(x)+D_{\mu\cdot 2^{2n}}^*(x)=D_{m}^*(x)+D_{m\cdot 2^n}^*(x)+D_{\mu\cdot 2^{2n}}^*(x).
\end{multline}\fi  
\ifx80\begin{multline*}\begin{split}
D_l^*(x)=D_p^*(x)+D_{\mu\cdot 2^{2n}}^*(x)=D_{m}^*(x)+D_{m\cdot 2^n}^*(x)+D_{\mu\cdot 2^{2n}}^*(x).
\end{split}\end{multline*}\fi
\ifx90\begin{multline}\begin{split}
D_l^*(x)=D_p^*(x)+D_{\mu\cdot 2^{2n}}^*(x)=D_{m}^*(x)+D_{m\cdot 2^n}^*(x)+D_{\mu\cdot 2^{2n}}^*(x).
\end{split}\end{multline}\fi
}
Since the supports of the functions $D_{m\cdot 2^n}^*(t)$ and $D_{\mu\cdot 2^{2n}}^*(t)$ are in $\Delta_1$, we conclude
{\ifx01
\begin{equation*} \label{a24}
D_l^*(x\oplus\theta_j)=D_m^*(x\oplus\theta_j),\quad x\in \Delta_k, \quad j\neq  k.
 \end{equation*}\fi  
\ifx11\begin{equation}\label{a24}
D_l^*(x\oplus\theta_j)=D_m^*(x\oplus\theta_j),\quad x\in \Delta_k, \quad j\neq  k.
\end{equation}\fi   
\ifx21\begin{align*}\label{a24}
D_l^*(x\oplus\theta_j)=D_m^*(x\oplus\theta_j),\quad x\in \Delta_k, \quad j\neq  k.
\end{align*}\fi   
\ifx31\begin{align}\label{a24}
D_l^*(x\oplus\theta_j)=D_m^*(x\oplus\theta_j),\quad x\in \Delta_k, \quad j\neq  k.
\end{align}\fi    
\ifx41\begin{gather*}\label{a24}
D_l^*(x\oplus\theta_j)=D_m^*(x\oplus\theta_j),\quad x\in \Delta_k, \quad j\neq  k.
\end{gather*}\fi  
\ifx51\begin{gather}\label{a24}
D_l^*(x\oplus\theta_j)=D_m^*(x\oplus\theta_j),\quad x\in \Delta_k, \quad j\neq  k.
\end{gather}\fi   
\ifx61\begin{multline*}\label{a24}
D_l^*(x\oplus\theta_j)=D_m^*(x\oplus\theta_j),\quad x\in \Delta_k, \quad j\neq  k.
\end{multline*}\fi  
\ifx71\begin{multline}\label{a24}
D_l^*(x\oplus\theta_j)=D_m^*(x\oplus\theta_j),\quad x\in \Delta_k, \quad j\neq  k.
\end{multline}\fi  
\ifx81\begin{multline*}\begin{split}\label{a24}
D_l^*(x\oplus\theta_j)=D_m^*(x\oplus\theta_j),\quad x\in \Delta_k, \quad j\neq  k.
\end{split}\end{multline*}\fi
\ifx91\begin{multline}\begin{split}\label{a24}
D_l^*(x\oplus\theta_j)=D_m^*(x\oplus\theta_j),\quad x\in \Delta_k, \quad j\neq  k.
\end{split}\end{multline}\fi
}
Thus, applying {Lemma \ref{{L2}}} and  {(\ref{{a12}})}, we obtain the bound
{\ifx09
\begin{equation*} \label{a15}
\frac{1}{2^{n}}\left|\sum_{j=1}^{k-1}D_l(x\oplus\theta_j)\right|=&\frac{1}{2^{n}}\left|\sum_{j=1}^{k-1}D_m^*(x\oplus\theta_j)\right|\\
\ge &\int_0^xD_m^*(x\oplus t)dt-1>\frac{n}{30}-1>\frac{n}{40},\quad n>n_0=150,
 \end{equation*}\fi  
\ifx19\begin{equation}\label{a15}
\frac{1}{2^{n}}\left|\sum_{j=1}^{k-1}D_l(x\oplus\theta_j)\right|=&\frac{1}{2^{n}}\left|\sum_{j=1}^{k-1}D_m^*(x\oplus\theta_j)\right|\\
\ge &\int_0^xD_m^*(x\oplus t)dt-1>\frac{n}{30}-1>\frac{n}{40},\quad n>n_0=150,
\end{equation}\fi   
\ifx29\begin{align*}\label{a15}
\frac{1}{2^{n}}\left|\sum_{j=1}^{k-1}D_l(x\oplus\theta_j)\right|=&\frac{1}{2^{n}}\left|\sum_{j=1}^{k-1}D_m^*(x\oplus\theta_j)\right|\\
\ge &\int_0^xD_m^*(x\oplus t)dt-1>\frac{n}{30}-1>\frac{n}{40},\quad n>n_0=150,
\end{align*}\fi   
\ifx39\begin{align}\label{a15}
\frac{1}{2^{n}}\left|\sum_{j=1}^{k-1}D_l(x\oplus\theta_j)\right|=&\frac{1}{2^{n}}\left|\sum_{j=1}^{k-1}D_m^*(x\oplus\theta_j)\right|\\
\ge &\int_0^xD_m^*(x\oplus t)dt-1>\frac{n}{30}-1>\frac{n}{40},\quad n>n_0=150,
\end{align}\fi    
\ifx49\begin{gather*}\label{a15}
\frac{1}{2^{n}}\left|\sum_{j=1}^{k-1}D_l(x\oplus\theta_j)\right|=&\frac{1}{2^{n}}\left|\sum_{j=1}^{k-1}D_m^*(x\oplus\theta_j)\right|\\
\ge &\int_0^xD_m^*(x\oplus t)dt-1>\frac{n}{30}-1>\frac{n}{40},\quad n>n_0=150,
\end{gather*}\fi  
\ifx59\begin{gather}\label{a15}
\frac{1}{2^{n}}\left|\sum_{j=1}^{k-1}D_l(x\oplus\theta_j)\right|=&\frac{1}{2^{n}}\left|\sum_{j=1}^{k-1}D_m^*(x\oplus\theta_j)\right|\\
\ge &\int_0^xD_m^*(x\oplus t)dt-1>\frac{n}{30}-1>\frac{n}{40},\quad n>n_0=150,
\end{gather}\fi   
\ifx69\begin{multline*}\label{a15}
\frac{1}{2^{n}}\left|\sum_{j=1}^{k-1}D_l(x\oplus\theta_j)\right|=&\frac{1}{2^{n}}\left|\sum_{j=1}^{k-1}D_m^*(x\oplus\theta_j)\right|\\
\ge &\int_0^xD_m^*(x\oplus t)dt-1>\frac{n}{30}-1>\frac{n}{40},\quad n>n_0=150,
\end{multline*}\fi  
\ifx79\begin{multline}\label{a15}
\frac{1}{2^{n}}\left|\sum_{j=1}^{k-1}D_l(x\oplus\theta_j)\right|=&\frac{1}{2^{n}}\left|\sum_{j=1}^{k-1}D_m^*(x\oplus\theta_j)\right|\\
\ge &\int_0^xD_m^*(x\oplus t)dt-1>\frac{n}{30}-1>\frac{n}{40},\quad n>n_0=150,
\end{multline}\fi  
\ifx89\begin{multline*}\begin{split}\label{a15}
\frac{1}{2^{n}}\left|\sum_{j=1}^{k-1}D_l(x\oplus\theta_j)\right|=&\frac{1}{2^{n}}\left|\sum_{j=1}^{k-1}D_m^*(x\oplus\theta_j)\right|\\
\ge &\int_0^xD_m^*(x\oplus t)dt-1>\frac{n}{30}-1>\frac{n}{40},\quad n>n_0=150,
\end{split}\end{multline*}\fi
\ifx99\begin{multline}\begin{split}\label{a15}
\frac{1}{2^{n}}\left|\sum_{j=1}^{k-1}D_l(x\oplus\theta_j)\right|=&\frac{1}{2^{n}}\left|\sum_{j=1}^{k-1}D_m^*(x\oplus\theta_j)\right|\\
\ge &\int_0^xD_m^*(x\oplus t)dt-1>\frac{n}{30}-1>\frac{n}{40},\quad n>n_0=150,
\end{split}\end{multline}\fi
}
which holds whenever $l$ satisfies {(\ref{{a16}})}.
Taking into account of {(\ref{{a3}})} and {(\ref{{a15}})}, we get
{\ifx00
\begin{equation*} 
\frac{\#\{l\in {\ensuremath{\mathbb N}}:\,1\le l\le u_k,\, |S_l(x,f)|>n/40 \}}{u_k}\ge \frac{\#\left(L(x)\cap [u_{k-1},u_k)\right)}{u_k}\gtrsim 2^{-2n},
 \end{equation*}\fi  
\ifx10\begin{equation}
\frac{\#\{l\in {\ensuremath{\mathbb N}}:\,1\le l\le u_k,\, |S_l(x,f)|>n/40 \}}{u_k}\ge \frac{\#\left(L(x)\cap [u_{k-1},u_k)\right)}{u_k}\gtrsim 2^{-2n},
\end{equation}\fi   
\ifx20\begin{align*}
\frac{\#\{l\in {\ensuremath{\mathbb N}}:\,1\le l\le u_k,\, |S_l(x,f)|>n/40 \}}{u_k}\ge \frac{\#\left(L(x)\cap [u_{k-1},u_k)\right)}{u_k}\gtrsim 2^{-2n},
\end{align*}\fi   
\ifx30\begin{align}
\frac{\#\{l\in {\ensuremath{\mathbb N}}:\,1\le l\le u_k,\, |S_l(x,f)|>n/40 \}}{u_k}\ge \frac{\#\left(L(x)\cap [u_{k-1},u_k)\right)}{u_k}\gtrsim 2^{-2n},
\end{align}\fi    
\ifx40\begin{gather*}
\frac{\#\{l\in {\ensuremath{\mathbb N}}:\,1\le l\le u_k,\, |S_l(x,f)|>n/40 \}}{u_k}\ge \frac{\#\left(L(x)\cap [u_{k-1},u_k)\right)}{u_k}\gtrsim 2^{-2n},
\end{gather*}\fi  
\ifx50\begin{gather}
\frac{\#\{l\in {\ensuremath{\mathbb N}}:\,1\le l\le u_k,\, |S_l(x,f)|>n/40 \}}{u_k}\ge \frac{\#\left(L(x)\cap [u_{k-1},u_k)\right)}{u_k}\gtrsim 2^{-2n},
\end{gather}\fi   
\ifx60\begin{multline*}
\frac{\#\{l\in {\ensuremath{\mathbb N}}:\,1\le l\le u_k,\, |S_l(x,f)|>n/40 \}}{u_k}\ge \frac{\#\left(L(x)\cap [u_{k-1},u_k)\right)}{u_k}\gtrsim 2^{-2n},
\end{multline*}\fi  
\ifx70\begin{multline}
\frac{\#\{l\in {\ensuremath{\mathbb N}}:\,1\le l\le u_k,\, |S_l(x,f)|>n/40 \}}{u_k}\ge \frac{\#\left(L(x)\cap [u_{k-1},u_k)\right)}{u_k}\gtrsim 2^{-2n},
\end{multline}\fi  
\ifx80\begin{multline*}\begin{split}
\frac{\#\{l\in {\ensuremath{\mathbb N}}:\,1\le l\le u_k,\, |S_l(x,f)|>n/40 \}}{u_k}\ge \frac{\#\left(L(x)\cap [u_{k-1},u_k)\right)}{u_k}\gtrsim 2^{-2n},
\end{split}\end{multline*}\fi
\ifx90\begin{multline}\begin{split}
\frac{\#\{l\in {\ensuremath{\mathbb N}}:\,1\le l\le u_k,\, |S_l(x,f)|>n/40 \}}{u_k}\ge \frac{\#\left(L(x)\cap [u_{k-1},u_k)\right)}{u_k}\gtrsim 2^{-2n},
\end{split}\end{multline}\fi
}
which completes the proof of lemma.
\end{proof}
\begin{proof}[Proof of theorem]
We may choose numbers $\{n_k\}_{k=1}^\infty$ and $\{\alpha_k\}_{k=1}^\infty$ such that
{\ifx03
\begin{equation*} 
&p(n_{k+1})>2q(n_k),\label{c2}\\
&\Phi\left(\frac{n_k}{50\cdot 2^k}\right)>\exp(2n_k),\label{c3}\\
&n_{k+1}>800k2^kq(n_k),\label{c4}
 \end{equation*}\fi  
\ifx13\begin{equation}
&p(n_{k+1})>2q(n_k),\label{c2}\\
&\Phi\left(\frac{n_k}{50\cdot 2^k}\right)>\exp(2n_k),\label{c3}\\
&n_{k+1}>800k2^kq(n_k),\label{c4}
\end{equation}\fi   
\ifx23\begin{align*}
&p(n_{k+1})>2q(n_k),\label{c2}\\
&\Phi\left(\frac{n_k}{50\cdot 2^k}\right)>\exp(2n_k),\label{c3}\\
&n_{k+1}>800k2^kq(n_k),\label{c4}
\end{align*}\fi   
\ifx33\begin{align}
&p(n_{k+1})>2q(n_k),\label{c2}\\
&\Phi\left(\frac{n_k}{50\cdot 2^k}\right)>\exp(2n_k),\label{c3}\\
&n_{k+1}>800k2^kq(n_k),\label{c4}
\end{align}\fi    
\ifx43\begin{gather*}
&p(n_{k+1})>2q(n_k),\label{c2}\\
&\Phi\left(\frac{n_k}{50\cdot 2^k}\right)>\exp(2n_k),\label{c3}\\
&n_{k+1}>800k2^kq(n_k),\label{c4}
\end{gather*}\fi  
\ifx53\begin{gather}
&p(n_{k+1})>2q(n_k),\label{c2}\\
&\Phi\left(\frac{n_k}{50\cdot 2^k}\right)>\exp(2n_k),\label{c3}\\
&n_{k+1}>800k2^kq(n_k),\label{c4}
\end{gather}\fi   
\ifx63\begin{multline*}
&p(n_{k+1})>2q(n_k),\label{c2}\\
&\Phi\left(\frac{n_k}{50\cdot 2^k}\right)>\exp(2n_k),\label{c3}\\
&n_{k+1}>800k2^kq(n_k),\label{c4}
\end{multline*}\fi  
\ifx73\begin{multline}
&p(n_{k+1})>2q(n_k),\label{c2}\\
&\Phi\left(\frac{n_k}{50\cdot 2^k}\right)>\exp(2n_k),\label{c3}\\
&n_{k+1}>800k2^kq(n_k),\label{c4}
\end{multline}\fi  
\ifx83\begin{multline*}\begin{split}
&p(n_{k+1})>2q(n_k),\label{c2}\\
&\Phi\left(\frac{n_k}{50\cdot 2^k}\right)>\exp(2n_k),\label{c3}\\
&n_{k+1}>800k2^kq(n_k),\label{c4}
\end{split}\end{multline*}\fi
\ifx93\begin{multline}\begin{split}
&p(n_{k+1})>2q(n_k),\label{c2}\\
&\Phi\left(\frac{n_k}{50\cdot 2^k}\right)>\exp(2n_k),\label{c3}\\
&n_{k+1}>800k2^kq(n_k),\label{c4}
\end{split}\end{multline}\fi
}
where $p(n)$ and $q(n)$ are the sequences determined in {Lemma \ref{{L1}}}. We just note that  {(\ref{{c3}})} may guarantee by using {(\ref{{0-3}})}. Applying {Lemma \ref{{L1}}}, we get polynomials $g_k(x)=f_{n_k}(x)$, which satisfy {(\ref{{a32}})} for any $x\in [0,1)$. We have
{\ifx00
\begin{equation*} 
f(x)=\sum_{k=1}^\infty2^{-k}g_k(x)\in L^1[0,1).
 \end{equation*}\fi  
\ifx10\begin{equation}
f(x)=\sum_{k=1}^\infty2^{-k}g_k(x)\in L^1[0,1).
\end{equation}\fi   
\ifx20\begin{align*}
f(x)=\sum_{k=1}^\infty2^{-k}g_k(x)\in L^1[0,1).
\end{align*}\fi   
\ifx30\begin{align}
f(x)=\sum_{k=1}^\infty2^{-k}g_k(x)\in L^1[0,1).
\end{align}\fi    
\ifx40\begin{gather*}
f(x)=\sum_{k=1}^\infty2^{-k}g_k(x)\in L^1[0,1).
\end{gather*}\fi  
\ifx50\begin{gather}
f(x)=\sum_{k=1}^\infty2^{-k}g_k(x)\in L^1[0,1).
\end{gather}\fi   
\ifx60\begin{multline*}
f(x)=\sum_{k=1}^\infty2^{-k}g_k(x)\in L^1[0,1).
\end{multline*}\fi  
\ifx70\begin{multline}
f(x)=\sum_{k=1}^\infty2^{-k}g_k(x)\in L^1[0,1).
\end{multline}\fi  
\ifx80\begin{multline*}\begin{split}
f(x)=\sum_{k=1}^\infty2^{-k}g_k(x)\in L^1[0,1).
\end{split}\end{multline*}\fi
\ifx90\begin{multline}\begin{split}
f(x)=\sum_{k=1}^\infty2^{-k}g_k(x)\in L^1[0,1).
\end{split}\end{multline}\fi
}
The condition {(\ref{{c2}})} provides increasing spectrums of these polynomials. Thus, if $p(n_k)<l\le q(n_k)$, then we have
{\ifx09
\begin{equation*} \label{c5}
|S_l(x,f)|=&\left|\sum_{j=1}^\infty2^{-j}S_l(x,g_j)\right|=\left|\sum_{j=1}^{k-1}2^{-j}g_j(x)+2^{-k}S_l(x,g_k)\right|\\
\ge&2^{-k}|S_l(x,g_k)|-4(k-1)q(n_{k-1}).
 \end{equation*}\fi  
\ifx19\begin{equation}\label{c5}
|S_l(x,f)|=&\left|\sum_{j=1}^\infty2^{-j}S_l(x,g_j)\right|=\left|\sum_{j=1}^{k-1}2^{-j}g_j(x)+2^{-k}S_l(x,g_k)\right|\\
\ge&2^{-k}|S_l(x,g_k)|-4(k-1)q(n_{k-1}).
\end{equation}\fi   
\ifx29\begin{align*}\label{c5}
|S_l(x,f)|=&\left|\sum_{j=1}^\infty2^{-j}S_l(x,g_j)\right|=\left|\sum_{j=1}^{k-1}2^{-j}g_j(x)+2^{-k}S_l(x,g_k)\right|\\
\ge&2^{-k}|S_l(x,g_k)|-4(k-1)q(n_{k-1}).
\end{align*}\fi   
\ifx39\begin{align}\label{c5}
|S_l(x,f)|=&\left|\sum_{j=1}^\infty2^{-j}S_l(x,g_j)\right|=\left|\sum_{j=1}^{k-1}2^{-j}g_j(x)+2^{-k}S_l(x,g_k)\right|\\
\ge&2^{-k}|S_l(x,g_k)|-4(k-1)q(n_{k-1}).
\end{align}\fi    
\ifx49\begin{gather*}\label{c5}
|S_l(x,f)|=&\left|\sum_{j=1}^\infty2^{-j}S_l(x,g_j)\right|=\left|\sum_{j=1}^{k-1}2^{-j}g_j(x)+2^{-k}S_l(x,g_k)\right|\\
\ge&2^{-k}|S_l(x,g_k)|-4(k-1)q(n_{k-1}).
\end{gather*}\fi  
\ifx59\begin{gather}\label{c5}
|S_l(x,f)|=&\left|\sum_{j=1}^\infty2^{-j}S_l(x,g_j)\right|=\left|\sum_{j=1}^{k-1}2^{-j}g_j(x)+2^{-k}S_l(x,g_k)\right|\\
\ge&2^{-k}|S_l(x,g_k)|-4(k-1)q(n_{k-1}).
\end{gather}\fi   
\ifx69\begin{multline*}\label{c5}
|S_l(x,f)|=&\left|\sum_{j=1}^\infty2^{-j}S_l(x,g_j)\right|=\left|\sum_{j=1}^{k-1}2^{-j}g_j(x)+2^{-k}S_l(x,g_k)\right|\\
\ge&2^{-k}|S_l(x,g_k)|-4(k-1)q(n_{k-1}).
\end{multline*}\fi  
\ifx79\begin{multline}\label{c5}
|S_l(x,f)|=&\left|\sum_{j=1}^\infty2^{-j}S_l(x,g_j)\right|=\left|\sum_{j=1}^{k-1}2^{-j}g_j(x)+2^{-k}S_l(x,g_k)\right|\\
\ge&2^{-k}|S_l(x,g_k)|-4(k-1)q(n_{k-1}).
\end{multline}\fi  
\ifx89\begin{multline*}\begin{split}\label{c5}
|S_l(x,f)|=&\left|\sum_{j=1}^\infty2^{-j}S_l(x,g_j)\right|=\left|\sum_{j=1}^{k-1}2^{-j}g_j(x)+2^{-k}S_l(x,g_k)\right|\\
\ge&2^{-k}|S_l(x,g_k)|-4(k-1)q(n_{k-1}).
\end{split}\end{multline*}\fi
\ifx99\begin{multline}\begin{split}\label{c5}
|S_l(x,f)|=&\left|\sum_{j=1}^\infty2^{-j}S_l(x,g_j)\right|=\left|\sum_{j=1}^{k-1}2^{-j}g_j(x)+2^{-k}S_l(x,g_k)\right|\\
\ge&2^{-k}|S_l(x,g_k)|-4(k-1)q(n_{k-1}).
\end{split}\end{multline}\fi
}
 Applying {Lemma \ref{{L1}}}, for any $x\in[0,1)$ we may find a number $N_k\in [p(n_k),2q(n_k)]$ such  that
{\ifx00
\begin{equation*} 
\#\{l\in {\ensuremath{\mathbb N}}:\,p(n_k)<l\le N_k,\, |S_l(x,g_k)|>n_k/40 \}\gtrsim \frac{N_k}{2^{2n_k}}.
 \end{equation*}\fi  
\ifx10\begin{equation}
\#\{l\in {\ensuremath{\mathbb N}}:\,p(n_k)<l\le N_k,\, |S_l(x,g_k)|>n_k/40 \}\gtrsim \frac{N_k}{2^{2n_k}}.
\end{equation}\fi   
\ifx20\begin{align*}
\#\{l\in {\ensuremath{\mathbb N}}:\,p(n_k)<l\le N_k,\, |S_l(x,g_k)|>n_k/40 \}\gtrsim \frac{N_k}{2^{2n_k}}.
\end{align*}\fi   
\ifx30\begin{align}
\#\{l\in {\ensuremath{\mathbb N}}:\,p(n_k)<l\le N_k,\, |S_l(x,g_k)|>n_k/40 \}\gtrsim \frac{N_k}{2^{2n_k}}.
\end{align}\fi    
\ifx40\begin{gather*}
\#\{l\in {\ensuremath{\mathbb N}}:\,p(n_k)<l\le N_k,\, |S_l(x,g_k)|>n_k/40 \}\gtrsim \frac{N_k}{2^{2n_k}}.
\end{gather*}\fi  
\ifx50\begin{gather}
\#\{l\in {\ensuremath{\mathbb N}}:\,p(n_k)<l\le N_k,\, |S_l(x,g_k)|>n_k/40 \}\gtrsim \frac{N_k}{2^{2n_k}}.
\end{gather}\fi   
\ifx60\begin{multline*}
\#\{l\in {\ensuremath{\mathbb N}}:\,p(n_k)<l\le N_k,\, |S_l(x,g_k)|>n_k/40 \}\gtrsim \frac{N_k}{2^{2n_k}}.
\end{multline*}\fi  
\ifx70\begin{multline}
\#\{l\in {\ensuremath{\mathbb N}}:\,p(n_k)<l\le N_k,\, |S_l(x,g_k)|>n_k/40 \}\gtrsim \frac{N_k}{2^{2n_k}}.
\end{multline}\fi  
\ifx80\begin{multline*}\begin{split}
\#\{l\in {\ensuremath{\mathbb N}}:\,p(n_k)<l\le N_k,\, |S_l(x,g_k)|>n_k/40 \}\gtrsim \frac{N_k}{2^{2n_k}}.
\end{split}\end{multline*}\fi
\ifx90\begin{multline}\begin{split}
\#\{l\in {\ensuremath{\mathbb N}}:\,p(n_k)<l\le N_k,\, |S_l(x,g_k)|>n_k/40 \}\gtrsim \frac{N_k}{2^{2n_k}}.
\end{split}\end{multline}\fi
}
Thus, using also {(\ref{{c4}})} and {(\ref{{c5}})}, we conclude
{\ifx00
\begin{equation*} 
\#\{l\in {\ensuremath{\mathbb N}}:\,p(n_k)<l\le N_k,\, |S_l(x,f)|> n_k/50\cdot  2^k \}\gtrsim \frac{N_k}{2^{2n_k}}
 \end{equation*}\fi  
\ifx10\begin{equation}
\#\{l\in {\ensuremath{\mathbb N}}:\,p(n_k)<l\le N_k,\, |S_l(x,f)|> n_k/50\cdot  2^k \}\gtrsim \frac{N_k}{2^{2n_k}}
\end{equation}\fi   
\ifx20\begin{align*}
\#\{l\in {\ensuremath{\mathbb N}}:\,p(n_k)<l\le N_k,\, |S_l(x,f)|> n_k/50\cdot  2^k \}\gtrsim \frac{N_k}{2^{2n_k}}
\end{align*}\fi   
\ifx30\begin{align}
\#\{l\in {\ensuremath{\mathbb N}}:\,p(n_k)<l\le N_k,\, |S_l(x,f)|> n_k/50\cdot  2^k \}\gtrsim \frac{N_k}{2^{2n_k}}
\end{align}\fi    
\ifx40\begin{gather*}
\#\{l\in {\ensuremath{\mathbb N}}:\,p(n_k)<l\le N_k,\, |S_l(x,f)|> n_k/50\cdot  2^k \}\gtrsim \frac{N_k}{2^{2n_k}}
\end{gather*}\fi  
\ifx50\begin{gather}
\#\{l\in {\ensuremath{\mathbb N}}:\,p(n_k)<l\le N_k,\, |S_l(x,f)|> n_k/50\cdot  2^k \}\gtrsim \frac{N_k}{2^{2n_k}}
\end{gather}\fi   
\ifx60\begin{multline*}
\#\{l\in {\ensuremath{\mathbb N}}:\,p(n_k)<l\le N_k,\, |S_l(x,f)|> n_k/50\cdot  2^k \}\gtrsim \frac{N_k}{2^{2n_k}}
\end{multline*}\fi  
\ifx70\begin{multline}
\#\{l\in {\ensuremath{\mathbb N}}:\,p(n_k)<l\le N_k,\, |S_l(x,f)|> n_k/50\cdot  2^k \}\gtrsim \frac{N_k}{2^{2n_k}}
\end{multline}\fi  
\ifx80\begin{multline*}\begin{split}
\#\{l\in {\ensuremath{\mathbb N}}:\,p(n_k)<l\le N_k,\, |S_l(x,f)|> n_k/50\cdot  2^k \}\gtrsim \frac{N_k}{2^{2n_k}}
\end{split}\end{multline*}\fi
\ifx90\begin{multline}\begin{split}
\#\{l\in {\ensuremath{\mathbb N}}:\,p(n_k)<l\le N_k,\, |S_l(x,f)|> n_k/50\cdot  2^k \}\gtrsim \frac{N_k}{2^{2n_k}}
\end{split}\end{multline}\fi
}
and finally,  using {(\ref{{c3}})} we obtain
{\ifx00
\begin{equation*} 
\frac{1}{N_k}\sum_{j=1}^{N_k}\Phi(|S_j(x,f)|)\gtrsim\frac{1}{N_k}\cdot\frac{N_k}{2^{2n_k}}\cdot\Phi\left( \frac{n_k}{50\cdot  2^k }\right)\ge \left(\frac{e}{2}\right)^{2n_k},\quad k=1,2,\ldots.
 \end{equation*}\fi  
\ifx10\begin{equation}
\frac{1}{N_k}\sum_{j=1}^{N_k}\Phi(|S_j(x,f)|)\gtrsim\frac{1}{N_k}\cdot\frac{N_k}{2^{2n_k}}\cdot\Phi\left( \frac{n_k}{50\cdot  2^k }\right)\ge \left(\frac{e}{2}\right)^{2n_k},\quad k=1,2,\ldots.
\end{equation}\fi   
\ifx20\begin{align*}
\frac{1}{N_k}\sum_{j=1}^{N_k}\Phi(|S_j(x,f)|)\gtrsim\frac{1}{N_k}\cdot\frac{N_k}{2^{2n_k}}\cdot\Phi\left( \frac{n_k}{50\cdot  2^k }\right)\ge \left(\frac{e}{2}\right)^{2n_k},\quad k=1,2,\ldots.
\end{align*}\fi   
\ifx30\begin{align}
\frac{1}{N_k}\sum_{j=1}^{N_k}\Phi(|S_j(x,f)|)\gtrsim\frac{1}{N_k}\cdot\frac{N_k}{2^{2n_k}}\cdot\Phi\left( \frac{n_k}{50\cdot  2^k }\right)\ge \left(\frac{e}{2}\right)^{2n_k},\quad k=1,2,\ldots.
\end{align}\fi    
\ifx40\begin{gather*}
\frac{1}{N_k}\sum_{j=1}^{N_k}\Phi(|S_j(x,f)|)\gtrsim\frac{1}{N_k}\cdot\frac{N_k}{2^{2n_k}}\cdot\Phi\left( \frac{n_k}{50\cdot  2^k }\right)\ge \left(\frac{e}{2}\right)^{2n_k},\quad k=1,2,\ldots.
\end{gather*}\fi  
\ifx50\begin{gather}
\frac{1}{N_k}\sum_{j=1}^{N_k}\Phi(|S_j(x,f)|)\gtrsim\frac{1}{N_k}\cdot\frac{N_k}{2^{2n_k}}\cdot\Phi\left( \frac{n_k}{50\cdot  2^k }\right)\ge \left(\frac{e}{2}\right)^{2n_k},\quad k=1,2,\ldots.
\end{gather}\fi   
\ifx60\begin{multline*}
\frac{1}{N_k}\sum_{j=1}^{N_k}\Phi(|S_j(x,f)|)\gtrsim\frac{1}{N_k}\cdot\frac{N_k}{2^{2n_k}}\cdot\Phi\left( \frac{n_k}{50\cdot  2^k }\right)\ge \left(\frac{e}{2}\right)^{2n_k},\quad k=1,2,\ldots.
\end{multline*}\fi  
\ifx70\begin{multline}
\frac{1}{N_k}\sum_{j=1}^{N_k}\Phi(|S_j(x,f)|)\gtrsim\frac{1}{N_k}\cdot\frac{N_k}{2^{2n_k}}\cdot\Phi\left( \frac{n_k}{50\cdot  2^k }\right)\ge \left(\frac{e}{2}\right)^{2n_k},\quad k=1,2,\ldots.
\end{multline}\fi  
\ifx80\begin{multline*}\begin{split}
\frac{1}{N_k}\sum_{j=1}^{N_k}\Phi(|S_j(x,f)|)\gtrsim\frac{1}{N_k}\cdot\frac{N_k}{2^{2n_k}}\cdot\Phi\left( \frac{n_k}{50\cdot  2^k }\right)\ge \left(\frac{e}{2}\right)^{2n_k},\quad k=1,2,\ldots.
\end{split}\end{multline*}\fi
\ifx90\begin{multline}\begin{split}
\frac{1}{N_k}\sum_{j=1}^{N_k}\Phi(|S_j(x,f)|)\gtrsim\frac{1}{N_k}\cdot\frac{N_k}{2^{2n_k}}\cdot\Phi\left( \frac{n_k}{50\cdot  2^k }\right)\ge \left(\frac{e}{2}\right)^{2n_k},\quad k=1,2,\ldots.
\end{split}\end{multline}\fi
}
This implies the divergence of  $\Phi$-means at a point  $x\in[0,1)$ taken arbitrarily, which completes the proof of the theorem.
\end{proof}
\begin{thebibliography}{99}
\bibitem{Boch}
S. V. Bochkarev, Everywhere divergent Fourier series with respect to the Walsh system and with respect to multiplicative systems, Russian Math. Surveys, 59(2004), no 1, 103--124.
\bibitem{GoGo}
U. Goginava and L. Gogoladze, Strong approximation by Marcinkiewicz means of two-dimensional Walsh-Fourier series, Constr. Approx., 35 (2012), no.
1, 1--19.
\bibitem{GES}
B. I. Golubov, A. V. Efimov and V. A. Skvortsov. Series and transformations of Walsh, Moscow, 1987 (Russian); English translation, Kluwer
Academic, Dordrecht, 1991.
\bibitem{FrSc1}
S. Fridli and F. Schipp, Strong summability and Sidon type inequality, Acta Sci. Math. (Szeged), 60 (1985), 277--289.
\bibitem{FrSc2}
S. Fridli and F. Schipp, Strong approximation via Sidon type inequalities, J. Approx. Theory, 94 (1998), 263--284.
\bibitem{Kar1}
G. A. Karagulyan, On the divergence of strong $\Phi$-means of Fourier series, Izv. Acad. Sci. of Armenia, 26(1991), no 2, 159--162.
\bibitem{Kar2}
G. A. Karagulyan, Everywhere divergence $\Phi$-means of Fourier series, Math. Notes, 80(2006), no 1--2, 47--56.
\bibitem{KaSa}
B. S. Kashin and A. A. Saakyan, Orthogonal series. Translated from the Russian by Ralph P. Boas. Translation edited by Ben Silver. Translations of Mathematical Monographs, 75. American Mathematical Society, Providence, RI, 1989.

\bibitem{Kol}
A. N. Kolmogoroff, Une s\'{e}rie de Fourier-Lebesque divergente presgue partout,
Comp. Rend., 183(1926), no 4,  1327--1328.
\bibitem{Kon}
S. V. Konyagin, On everywhere divergence of trigonometric Fourier series, Sb. Math., 191(2000), no 1,  97--120.
\bibitem{Mar} J.  Marcinkiewicz, Sur la sommabilit\'{e} forte des s\'{e}ries de Fourier,
J. Lond. Math. Soc., 14(1939), 162--168.
\bibitem{Osk}
K. I. Oskolkov, On strong summability of Fourier series, Trudy Mat. Inst. Steklov. 172 (1985), 280--290.
\bibitem{Rod1}
V. A. Rodin, ${{\rm BMO\,}}$-strong means of Fourier series, Functional Analysis and Its Applications
23(1989), no 2, 145--147.
\bibitem{Rod2}
V. A. Rodin, The space ${{\rm BMO\,}}$ and strong means of Walsh-Fourier series, Mathematics of the USSR-Sbornik, 74(1993), no 1, 203--218.
\bibitem{Ste}
E. M. Stein, On the limits of sequences of operators, Annals Math., 74(1961), no 2, 140--170.
\bibitem{Sch1}
F. Schipp, \"{U}ber die Divergenz der Walsh-Fourierreihen, Ann Univ. Sci. Budapest, Sec. Math., 12(1969),  49--62.
\bibitem{Sch2}
F. Schipp, \"{U}ber die Summation von Walsh-Fourierreihen,  Acta Sci. Math.(Szeged),
30(1969), 77--87.
\bibitem{SWS}
F. Schipp, W. R. Wade,  P. Simon and J. P\'{a}l, Walsh Series, an Introduction to Dyadic Harmonic Analysis, Adam Hilger, Bristol, New York, 1990.
\bibitem{Tot1} V. Totik, Notes on Fourier series strong approximations,
J. Approx. Theory, 43(1985), 105--111.
\bibitem{Tot2} V. Totik, On the strong approximation of Fourier series,
Acta Math. Acad. Sci. Hung., 1980, 1-2, 157--172.
\bibitem{Zyg} A.  Zygmund, On the convergence and summability of power series on
the circle of convergence, Proc. Lond. Math. Soc., 47(1941), 326--350.
\end{thebibliography}
\end{document} 

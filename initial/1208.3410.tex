\documentclass[12pt]{amsart}

\usepackage{amstext}
\usepackage{amsthm}
\usepackage{amsmath}
\usepackage{amssymb}
\usepackage{latexsym}
\usepackage{amsfonts}
\usepackage{graphicx}

\usepackage[pagebackref,hypertexnames=false, colorlinks, citecolor=red, linkcolor=red]{hyperref} 
\usepackage{amsrefs}

\bibliographystyle{plain}

\setlength{\evensidemargin}{0in}
\setlength{\oddsidemargin}{0in}
\setlength{\topmargin}{-.5in}
\setlength{\textheight}{9in}
\setlength{\textwidth}{6.5in}

\begin{document}

\newenvironment{entry}
{\begin{list}{X}  {      \setlength{\labelwidth}{55pt}      \setlength{\leftmargin}{\labelwidth}

\numberwithin{equation}{section}

\newtheorem{thm}{Theorem}[section]
\newtheorem{lm}[thm]{Lemma}
\newtheorem{cor}[thm]{Corollary}
\newtheorem{conj}[thm]{Conjecture}
\newtheorem{prob}[thm]{Problem}
\newtheorem{prop}[thm]{Proposition}
\newtheorem*{prop*}{Proposition}

\theoremstyle{definition}
\newtheorem{df}[thm]{Definition}
\newtheorem*{df*}{Definition}

\theoremstyle{remark}
\newtheorem{rem}[thm]{Remark}
\newtheorem*{rem*}{Remark}

\title{Corona Solutions Depending Smoothly on Corona Data}

\author[S. Treil]{Sergei Treil$^\dagger$}
\address{Sergei Treil, Department of Mathematics\\ Brown University\\ 151 Thayer Street\\ Providence, RI USA 02906}
\email{treil@math.brown.edu}
\thanks{Research supported in part by a National Science Foundation DMS grant \#0800876.}

\author[B. D. Wick]{Brett D. Wick$^\ddagger$}
\address{Brett D. Wick, School of Mathematics\\ Georgia Institute of Technology\\ 686 Cherry Street\\ Atlanta, GA USA 30332-0160}
\email{wick@math.gatech.edu}
\thanks{Research supported in part by a National Science Foundation DMS grants \# 1001098 and \# 0955432.}

\thanks{This note began at the workshop ``The Corona Problem: Connections Between Operator Theory, Function Theory and Geometry'''.  The authors thank the Fields Institute for the pleasant working conditions.}

\keywords{Corona Problem, Bezout Equation}

\begin{abstract} 
In this note we show that if the Corona data depends continuously (smoothly) on a parameter, the solutions of the corresponding Bezout equations can be chosen to have the same smoothness in the parameter. 
\end{abstract}

\maketitle

\section{Introduction and Main Results}

\subsection{Background}
Let $H^\infty:=H^\infty\left({\mathbb{D}}\right)$ denote the collection of all bounded analytic functions on the unit disc ${\mathbb{D}}$.  The classical Carleson Corona Theorem, see \cite{Carl}, states that 
if functions
\(f_{j}\in H^{\infty}\) are such that
\(\sum_{j=1}^{N}{\ensuremath{\left\vert{f_{j}}\right\vert}}^{2}\geq\delta^2>0\), then  there exists
functions $g_{j}\in H^{\infty}$ such that
$\sum_{j=1}^{N}g_{j}f_{j}=1$.   This is equivalent to the fact that the unit disc ${\mathbb{D}}$ is dense in the
maximal ideal space of the algebra $H^\infty$, but the importance of 
the
Corona Theorem goes much beyond the theory of maximal ideals of
$H^\infty$.  The Corona Theorem, and especially its generalization, the so called
Matrix (Operator) Corona Theorem play an important role in 
operator
theory (such as the angles between invariant subspaces, unconditionally convergent
spectral decompositions, computation of spectrum, etc.). 

The Vector Valued Corona Theorem says that if  $f=(f_k)_{k=1}^\infty\in H^\infty_{\ell^2}$ is a bounded 
analytic function whose values are in $\ell^2$ such that 
\begin{equation}
\label{C}
\tag{C}
1\ge \sum_{k=1}^\infty {\ensuremath{\left\vert{f_k(z)}\right\vert}}^2\ge\delta^2  >0, \qquad \forall z\in{\mathbb{D}}, 
\end{equation}
then there exists $g\in H^\infty_{\ell^2}$ solving the Bezout equation, 
\begin{equation}
\label{B}
\tag{B}
g^T(z)f(z):=\sum_{k=1}^\infty g_k(z) f_k(z)\equiv 1\quad\forall z\in{\mathbb{D}}.
\end{equation}
Moreover, the function $g$ can be chosen to satisfy (for  $\delta\le1/2$)
\begin{align}
\label{CorEst-01}
\|g(z)\|{_{ {}_{\scriptstyle {\ell^2}}}} \le C\frac{1}{\delta^2}\log\frac1{\delta} \qquad \forall z\in {\mathbb{D}},
\end{align}
with $C$ an absolute constant.  Note also that the above condition \eqref{C} is necessary for the existence of the function $g$ appearing in \eqref{B}.  Condition \eqref{C} is usually called the  Carleson Corona Condition (or simply the Corona Condition).  Condition \eqref{B} is named so after the Bezout equation. 

The main result of this paper is that if the Corona data $f$ depends smoothly on a parameter $s$, then one can find a solution $g$ also depending smoothly on the parameter $s$. 

\subsection{Preliminaries} 
To give the precise statements we  need to introduce some notation. 
\label{s:prelim}
\subsubsection{Function spaces \(C^r\left(K;H^\infty\right) \) and \(H^\infty\left(C^r(K)\right)\)}
Let $K$ be a compact Hausdorff space. We say that a function $f:{\mathbb{D}}\times K\to {\mathbb{C}}$ belongs to $C\left(K;H^\infty\right)$ if $f({\,\cdot\,} , s)\in H^\infty$ for any $s\in K$ and the function $s\mapsto f ({\,\cdot\,}, s)$ is a continuous function on $K$ (with values in the Banach space $H^\infty$). Since $K$ is compact, the above continuity simply means that the function $f$ is uniformly continuous in the variable $s\in K$ on ${\mathbb{D}}\times K$. Thus the space $C\left(K;H^\infty\right)$ can be identified with the space of functions $f:{\mathbb{D}}\times K\to {\mathbb{C}}$ which are analytic in the variable $z\in{\mathbb{D}}$ and uniformly continuous on ${\mathbb{D}}\times K$ in the variable $s\in K$.  In other words, for any ${\varepsilon}>0$ there exists $\delta>0$ that for all $z\in {\mathbb{D}}$ and for all $s, s'\in K$ the inequality $|f(z, s)-f(z, s')|<{\varepsilon}$ holds whenever $|s-s'|<\delta$. 

We can also define the space $H^\infty\left(C(K)\right)$ as the space of functions $f:{\mathbb{D}}\times K\to {\mathbb{C}}$ such that the function $z\mapsto f(z, {\,\cdot\,})$ is a bounded analytic function with values in $C(K)$. Since for functions with values in a Banach space weak and strong analyticity coincide, see  for example \cite{Nik-book-v1}*{Theorem 3.11.4}, the space $H^\infty\left(C(K)\right)$ can be identified with the collection of bounded functions $f:{\mathbb{D}}\times K\to {\mathbb{C}}$ which are analytic in the variable $z\in {\mathbb{D}}$ and continuous in the variable $s\in K$. 

If \(K\) has a smooth structure, for example if \(K\) is a compact subset of ${\mathbb{R}}^d$ or a smooth manifold such that $K ={\operatorname{Clos}} \left({\operatorname{Int}} K\right)$, we can define smooth versions of these function spaces.  The spaces $C^r\left(K;H^\infty\right)$, $r\in {\mathbb{N}}\cup\infty$ is the space of functions $f:{\mathbb{D}}\times K\to {\mathbb{C}}$ such that the vector-valued function $s\mapsto f({\,\cdot\,}, s)\in H^\infty$  has all its partial derivatives (in the strong sense) up to order $r$ existing on ${\operatorname{Int}} K$  and are continuous up to the boundary.  We will identify the function with the derivative of order $0$, so the function itself is continuous up to the boundary. Similarly, one can say that $C^r\left(H^\infty\right)$ consists of functions $f:{\mathbb{D}}\times K\to {\mathbb{C}}$ which are analytic in the variable $z\in {\mathbb{D}}$ and all the partial derivatives in the variable $s\in K$ up to order $r$ exist and are uniformly continuous in $s$ on ${\mathbb{D}}\times {\operatorname{Int}} K$. 

We will also consider vector-valued versions $C^r \left(K; H^\infty_{\ell^2}\right)$ of these spaces, consisting of all functions $f : {\mathbb{D}}\times K \to \ell^2$ such that the function $s \mapsto f({\,\cdot\,}, s) $ is a $C^r$ function with values in the vector-valued space $H^\infty_{\ell^2}$, i.e. $H^\infty$ with values in  $\ell^2$. 

We can also define the space of functions $H^\infty(C^r(K))$, which is the space of functions $f:{\mathbb{D}}\times K\to {\mathbb{C}}$ which are analytic in the variable $z\in {\mathbb{D}}$ and such that all partial derivatives in $s$ up to the order $r$ exist  on ${\mathbb{D}}\times {\operatorname{Int}} K$, are  continuous and bounded there, and for all $z\in {\mathbb{D}}$ extend continuously to the boundary of $K$.   Again, in a similar manner, we also consider the vector-valued version $H^\infty \left(C^r_{\ell^2}(K)\right) $ of such spaces, consisting of all functions $f :{\mathbb{D}}\times K \to \ell^2$ such that the function $z\mapsto f(z, {\,\cdot\,})$ is a bounded analytic function with values in the vector-valued space $C^r_{\ell^2}(K)$, i.e., $C^r(K)$ with values in $\ell^2$. 

Note, that the only difference between the spaces $C^r\left(K;H^\infty\right)$ and $H^\infty\left(C^r(K)\right)$ is that in the former we require the uniform continuity in $s$ on ${\mathbb{D}}\times K$ of the derivatives in $s$ of order $r$, and in the latter not.  

\subsubsection{Domains with uniformly continuous path metric}
\label{s:PathMetrUinfCont}
For an open connected subset $\Omega$ of a smooth manifold in addition to the metric $\rho$ inherited from the manifold we can define the \emph{path metric} $\rho{_{\scriptstyle \text{\rm p}}}(x,y)$ as the infimum of the lengths of the paths in $\Omega$ connecting $x$ and $y$. Clearly $\rho\le \rho{_{\scriptstyle \text{\rm p}}}$, and $\rho{_{\scriptstyle \text{\rm p}}}$ is continuous with respect to $\rho$. 

We are interested in domains where the path metric is \emph{uniformly continuous}, because   for such domains 
$ H^\infty\left(C^r(K)\right)\subset C^{r-1}\left(K;H^\infty\right)$.   An example of a domain with uniformly continuous path metric is the so-called weakly uniform domains. 
We say that a domain $\Omega$ (in ${\mathbb{R}}^d$ or in a smooth manifold) is \emph{weakly uniform}  if there exists $M<\infty$ such that  any two points $x_1, x_2\in \Omega$ can be connected by a smooth path in $\Omega$ of length at most $M{\operatorname{dist}}(x_1, x_2)$. 
For example, a bounded domain with a smooth boundary is definitely a weakly uniform domain.  If ${\operatorname{Int}} K$ is a weakly uniform domain, then an easy application of the Mean Value Theorem shows that $ H^\infty\left(C^r(K)\right)\subset C^{r-1}\left(K;H^\infty\right)$. 

\subsection{Main Results}
The first theorem gives us smooth dependence of the solution of the Corona Bezout equation on a parameter.  
If $r\ge 1$ we assume that  $K$ is a compact subset of ${\mathbb{R}}^d$ or of a smooth manifold, such that $K={\operatorname{Clos}}\left({\operatorname{Int}} K\right)$. For $r=0$ we only assume that $K$ is a compact Hausdorff space. 
\begin{thm}
\label{t:main-01}
Let $r\in{\mathbb{Z}}_+$ and let $f=(f_k)_{k=1}^\infty\in C^r \left(K;H^\infty_{\ell^2}\right)$  be such that 
\[
1\ge \sum_{k=1}^\infty \left|f_k(z, s)\right|^2\ge\delta^2 >0\qquad \forall (z,s)\in {\mathbb{D}}\times K. 
\] 
Then there exist $g=(g_k)_{k=1}^\infty\in C^r\left(K;H^\infty_{\ell^2}\right)$ such that 
\[
g^T(z,s)f(z,s) := \sum_{k=1}^\infty f_k(z,s) g_k(z,s) \equiv 1\qquad \forall (z,s)\in {\mathbb{D}}\times K. 
\]
Moreover,
\begin{align}
\label{SolEst-01}
\left\|g(z, s)\right\|_{\ell^2}  & := \left(\sum_{k=1}^\infty |g_k(z, s)|^2 \right)^{1/2}  \le 2 C(\delta) <\infty,  \\
\intertext{where $C(\delta) $ is the estimate in the Corona Theorem for $H^\infty$, and}
\label{SolEst-02}
\left\|g\right\|{_{ {}_{\scriptstyle {C^{\alpha}\left(K;H^\infty_{\ell^2}\right)}}}} & \le C\left(\alpha, \omega_f^{-1}\left(\frac{1}{2C(\delta)}\right),  K\right) \left\|f\right\|{_{ {}_{\scriptstyle {C^{\alpha}\left(K;H^\infty_{\ell^2}\right)}}}} \qquad \forall \alpha\in{\mathbb{Z}}_+, \ \alpha \le r;
\end{align}
here $\omega_f$ is the modulus of continuity of $f$, and \(\omega_f^{-1}\) is its inverse. 
\end{thm}
Note that for $r=0$ the estimate \eqref{SolEst-02} is superfluous since it follows immediately from \eqref{SolEst-01}. 

The next theorem can be interpreted as the Corona Theorem for the algebra $H^\infty\left(C^r(K)\right)$, $r\ge 1$. We assume here that $K$ is a subset of a smooth manifold, $K={\operatorname{Clos}}\left({\operatorname{Int}} K\right)$ such that the path metric on $K$ is uniformly continuous (with the respect to the metric inherited from the manifold). 

\begin{thm}
\label{t:main-02}
Let $r\in{\mathbb{N}}\cup\infty$ and let $f =(f_k)_{k=1}^\infty \in H^\infty\left(C^r_{\ell^2}(K)\right)$ be such that 
\[
1\ge \sum_{k=1}^\infty \left|f_k(z, s)\right|^2\ge\delta^2 >0\qquad \forall (z,s)\in {\mathbb{D}}\times K. 
\] 
Then there exist $g=(g_k)_{k=1}^\infty \in H^\infty\left(C^r_{\ell^2}(K)\right)$ such that 
\[
g^T(z,s) f(z,s) := \sum_{k=1}^\infty f_k(z,s) g_k(z,s) \equiv 1\qquad \forall (z,s)\in {\mathbb{D}}\times K. 
\]
Moreover,
\begin{align}
\label{SolEst-03}
\left\|g(z, s)\right\|_{\ell^2} 
& \le 2C(\delta) <\infty \qquad \forall (z,s)\in {\mathbb{D}}\times K,  \\
\intertext{where $C(\delta) $ is the estimate in the Corona Theorem for $H^\infty$, and}
\label{SolEst-04}
\left\|g\right\|{_{ {}_{\scriptstyle {H^\infty\left(C_{\ell^2}^\alpha(K)\right)}}}}  
& \le C\left(\delta, \alpha, K,  \left\| f\right\|{_{ {}_{\scriptstyle {C^1_{\ell^2}(K)}}}}\right)  \left\|f\right\|{_{ {}_{\scriptstyle {H^\infty\left(C_{\ell^2}^\alpha(K)\right)}}}} \qquad \forall \alpha\in{\mathbb{Z}}_+, \ \alpha \le r.
\end{align}
\end{thm}
\section{The Proofs}
\label{s:proof}

\begin{proof}[Proof of Theorem \ref{t:main-01}]
Let $f$ and $g$ be the column vectors with entries $f_k$ and $g_k$ respectively. By the Carleson Corona Theorem for infinitely many functions there exists a constant $C(\delta)$ such that for any $f\in H^\infty_{\ell^2}$ satisfying 
\[
1\ge \left|f(z)\right|\ge\delta>0\quad\forall z\in{\mathbb{D}},
\]
there exists $g\in H^\infty_{\ell^2}$, $\left\|g\right\|{_{ {}_{\scriptstyle {H^\infty_{\ell^2}}}}} \le C(\delta)$ such that $g^T(z) f(z) \equiv 1$ for all $z\in {\mathbb{D}}$.  

Here is the main idea of how the proof will proceed.  We will use this above for each function $f({\,\cdot\,}, s)$.  However, since $s\mapsto f({\,\cdot\,}, s)$ is continuous we will be able to construct a perturbation of the solutions that remains close to $1$ on some open neighborhood.  Then, using a partition of unity argument, we can construct the desired function, which is again close to $1$, but now for all $s$ and all $z$. 

Since $s\mapsto f({\,\cdot\,}, s)$ is continuous, for each point $s\in K$ let $U_s$ be a neighborhood of $s$ such that for all $s'\in U_s$ 
\[
\left\|f({\,\cdot\,}, s) - f({\,\cdot\,}, s')\right\|{_{ {}_{\scriptstyle {H^\infty_{\ell^2}}}}} \le \frac{1}{2C(\delta)}.
\] 
If $g_s({\,\cdot\,}) \in H^\infty_{\ell^2}$ solves the Bezout equation $g_s^T({\,\cdot\,}) f({\,\cdot\,}, s)\equiv 1$ satisfying $\|g_s({\,\cdot\,})\|{_{ {}_{\scriptstyle {H^\infty_{\ell^2}}}}} \le C(\delta)$, then clearly for $s'\in U_s$ 
\begin{align}
\label{pert-01}
\left\| 1-g_s^T ({\,\cdot\,}) f({\,\cdot\,} , s') \right\|_{H^\infty} \le \frac{1}{2}.
\end{align}

Since, by assumption, $K$ is compact, we can take a finite cover $U_k := U_{s_k}$, $k=1, 2, \ldots, N$ such that $K \subset \bigcup_{k=1}^N U_k$, and let $\eta_k$ be a partition of unity subordinated to this covering. This means that $0\le \eta_k\le 1$, ${\operatorname{supp}} \eta_k \subset U_k$ and $\sum_1^N \eta_k(s) \equiv 1$.   Note, that if $K$ has a smooth structure then one can take $\eta_k\in C^\infty(K)$. Moreover, it is possible to construct a subcover $U_{k}$ and the $C^\infty(K)$ partition of unity $\eta_k$ such that for all $\alpha\in {\mathbb{N}}$, 
\begin{align}
\label{part-est-01}
\sum_{k=1}^N \left\| \eta_k\right\|{_{ {}_{\scriptstyle {C^\alpha(K)}}}} \le C\left(\alpha, \omega_f\left(\frac{1}{2C(\delta)}\right), K\right)
\end{align}

For each $k$ let $ g_{s_k}({\,\cdot\,})$ be a solution of the Bezout equation 
\[
 g_{s_k}^T ({\,\cdot\,}) f({\,\cdot\,}, s_k) \equiv 1
\]
satisfying $\left\|g_{s_k}({\,\cdot\,})\right\|{_{ {}_{\scriptstyle {H^\infty_{\ell^2}}}}}\le C(\delta)$. Define
\[
 {\widetilde} g(z, s):= \sum_{k=1}^N \eta_k(s) g_{s_k}(z). 
\]
Then \eqref{pert-01} clearly implies that for all $s'\in K$
\begin{align}
\label{pert-02}
\left\|  1 - {\widetilde} g^T({\,\cdot\,}, s') f({\,\cdot\,}, s') \right\|_{H^\infty}\le \frac{1}{2}. 
\end{align}
Indeed, for $k$ such that $\eta_k(s')>0$ we have $s'\in U_{s_k}$, so the estimate \eqref{pert-01} holds  for $s=s_k$. 
But for any $s'$ the function  ${\widetilde} g({\,\cdot\,}, s')$ is a convex combination of such $g_{s_k}({\,\cdot\,})$, so we get \eqref{pert-02} as a convex combination of estimates \eqref{pert-01}.  

Inequality \eqref{pert-02} implies that the (scalar) function ${\widetilde} g^T({\,\cdot\,}, s') f({\,\cdot\,}, s')$ is invertible in $H^\infty$ and that $\left\| ( {\widetilde} g^T({\,\cdot\,}, s') f({\,\cdot\,}, s'))^{-1} \right\|_{H^\infty} \le 2$. Therefore the function $ g := ({\widetilde} g^T f)^{-1} {\widetilde} g$ solves the Bezout equation $g^T f\equiv 1$ and satisfies \eqref{SolEst-01}. Computing the derivatives of $g$ in $s$ and taking into account \eqref{part-est-01} we easily get \eqref{SolEst-02}. 
\end{proof}

\begin{proof}[Proof of Theorem \ref{t:main-02}]
As we mentioned above in Section \ref{s:PathMetrUinfCont}, if the path metric $\rho{_{\scriptstyle \text{\rm p}}}$ is uniformly continuous 
then 
\[
H^\infty\left(C^r(K)\right)\subset C^{r-1}\left(K;H^\infty\right),  
\] 
and the same for the $\ell^2$-valued case. In particular, $ H^\infty\left(C^r_{\ell^2}(K)\right)\subset C\left(K;H^\infty_{\ell^2}\right)$ for all $r\ge 1$. 

Since $s\mapsto f({\,\cdot\,}, s)$ is continuous, we can construct the partition of unity $\eta_k$ as in the proof of Theorem \ref{t:main-01} above. Note that if the path metric $\rho{_{\scriptstyle \text{\rm p}}}$ is uniformly continuous, the modulus of continuity $\omega_f$ can be estimated  via the modulus of continuity of $\rho{_{\scriptstyle \text{\rm p}}}$ and 
\[
\omega_f \left(\frac{1}{2C(\delta)}\right) \le 
\omega_{\rho{_{\scriptstyle \text{\rm p}}}}\left(\frac{1}{2C(\delta)}\right)  \left\|f\right\|{_{ {}_{\scriptstyle {C^1_{\ell^2}}}}}.
\] 
where $\omega_{\rho{_{\scriptstyle \text{\rm p}}}}$ is the modulus of continuity of $\rho{_{\scriptstyle \text{\rm p}}}$.  The rest follows as in the proof of Theorem \ref{t:main-01}. 
\end{proof}

\section{Some Remarks}
The method of the proof allows us to get smooth dependence on the parameter for the Corona type problems in  very general situations, like in the matrix and operator Corona Problem, as well as in the Corona Problem for multipliers of the reproducing kernel Hilbert spaces.  Below, we state some sample results. The proofs are similar to what was presented in Section \ref{s:proof} and are left as an exercise for the reader.  

Let $X$ and $Y$ be Banach spaces, and let ${\mathfrak{A}}={\mathfrak{A}}\left(\Omega; B(X,Y)\right)$ and ${\mathfrak{B}}= {\mathfrak{B}}\left(\Omega;B(Y,X)\right)$ be some Banach spaces of functions  on some set $\Omega$ with values in $B(X,Y)$ (the space of bounded operators from $X$ to $Y$) and in $B(Y,X)$ respectively.

As in Section \ref{s:prelim} we can introduce the spaces $C^r\left(K;{\mathfrak{A}}\right)$, $C^r\left(K;{\mathfrak{B}}\right)$, where $K$ is a compact set.  The space $C^r\left(K;{\mathfrak{A}}\right)$ consists of all functions $f:\Omega\times K \to B(X, Y)$ such that the function $s\mapsto f({\,\cdot\,}, s)$ is an $r$ times continuously differentiable function on $K$, and similarly for $C^r\left(K;{\mathfrak{B}}\right)$. As in Section \ref{s:prelim} we assume that  $K$ is a Hausdorff compact space if $r=0$, and if $r\ge1$ we assume that $K$ is a compact subset of a smooth manifold such that $K= {\operatorname{Clos}}\left({\operatorname{Int}} K\right)$. 

We also assume that \({\mathfrak{B}}{\mathfrak{A}}{\mathfrak{B}} \subset {\mathfrak{B}}\), i.e.~that \({\mathfrak{B}}{\mathfrak{A}}\) belongs to the multiplier algebra of \({\mathfrak{B}}\). This implies that for \(f\in{\mathfrak{A}}\) and \(g, h\in {\mathfrak{B}}\)
\begin{align}
\label{mult-01}
\left\| g f h \right\|{_{ {}_{\scriptstyle {\mathfrak{B}}}}} 
\le C({\mathfrak{A}}, {\mathfrak{B}}) \left\|g\right\|{_{ {}_{\scriptstyle {\mathfrak{B}}}}} \left\|f\right\|{_{ {}_{\scriptstyle {\mathfrak{A}}}}} \left\|h\right\|{_{ {}_{\scriptstyle {\mathfrak{B}}}}}  .
\end{align}

A typical example here would be multiplier spaces of (vector-valued) function spaces. Let for example \({\mathcal{X}}\) and \({\mathcal{Y}}\) be  spaces of functions  on \(\Omega\) with values in \(X\) and \(Y\) respectively. Define  \({\mathfrak{A}}\) and \({\mathfrak{B}}\) as the multiplier function spaces between \({\mathcal{X}}\) and \({\mathcal{Y}}\). Namely, \( {\mathfrak{A}} \) is the space of operator-valued functions \( {\varphi} :\Omega \to B(X, Y) \) such that the map \( f\mapsto {\varphi} f\) is a bounded operator from \({\mathcal{X}}\) to \({\mathcal{Y}}\); similarly, the space \({\mathfrak{B}}\) will be the collection of functions \({\varphi}:\Omega \to B(Y,X) \) such that the map \( f\mapsto {\varphi} f\) is a bounded operator from  \({\mathcal{Y}}\) to \({\mathcal{X}}\).  

For a concrete example, if \({\mathcal{X}}=H^2(X)\) and \({\mathcal{Y}}= H^2(Y)\) then \({\mathfrak{A}} = H^\infty \left(B(X,Y)\right) \) and \({\mathfrak{B}} = H^\infty\left(B(Y, X)\right)\) and we are in the situation of the operator corona. This situation with \(X=\ell^2\) and \(Y = {\mathbb{C}}\) gives us the settings of Theorem \ref{t:main-01}.   Another example would be the spaces of multipliers of Bergman or Dirichlet type spaces (in the disc or more general domains in \({\mathbb{C}}\) or \({\mathbb{C}}^n\)). But in general we do not need to assume that the spaces \({\mathfrak{A}}\) and \({\mathfrak{B}}\) are multiplier spaces.  

\begin{thm}
\label{t:main-03}
Under the above assumptions, let \(f\in C^r\left(K;{\mathfrak{A}}\right) \) be such that \(\|f({\,\cdot\,},s)\|{_{ {}_{\scriptstyle {\mathfrak{A}}}}}\le 1\) for all \(s\in K\).  Suppose that for all \(s\in K\) there exists \(g_s\in {\mathfrak{B}}\) such that 
\[
g_s (z) f(z, s) \equiv I\quad\forall z\in\Omega, \qquad \|g_s\|{_{ {}_{\scriptstyle {\mathfrak{B}}}}}\le C_0. 
\]
Then there exists \(g\in C^r(K;{\mathfrak{A}}) \) such that 
\[
g(z, s) f(z, s)\equiv I\quad \forall(z,s)\in\Omega\times K, \qquad \text{and} \qquad \| g({\,\cdot\,}, s)\|{_{ {}_{\scriptstyle {\mathfrak{B}}}}} \le 2C_0. 
\]
Moreover, for all \(\alpha\in {\mathbb{Z}}_+\), \(\alpha\le r\) we have the estimate 
\[
\| g\|{_{ {}_{\scriptstyle {C^\alpha\left(K;{\mathfrak{B}}\right)}}}} \le  C \left( \alpha, \omega_f^{-1}\left(\frac{1}{2C_0 C({\mathfrak{A}}, {\mathfrak{B}})}\right) , K \right) 
\|f\|{_{ {}_{\scriptstyle {C^\alpha\left(K, {\mathfrak{B}}\right)}}}}.
\]
Here, again, \(\omega_f\) is the modulus of continuity of \(f\) and \( C({\mathfrak{A}}, {\mathfrak{B}}) \) is the constant from \eqref{mult-01}. 
\end{thm}

\begin{rem*}
Note that in the above theorem we do not assume any kind of ``Corona Theorem'' for the spaces \({\mathfrak{A}}\) and \({\mathfrak{B}}\); we just postulate the solvability of the Bezout equations for each \(s\in K\). 
\end{rem*}

The proof of Theorem \ref{t:main-03} goes along the lines of the proof of Theorem \ref{t:main-01}. We construct a cover \(U_s\), \(s\in K\) of \(K\) such that 
\[
\left\|f({\,\cdot\,}, s) - f({\,\cdot\,}, s')\right\|{_{ {}_{\scriptstyle {\mathfrak{A}}}}} \le \frac{1}{2C_0C({\mathfrak{A}}, {\mathfrak{B}})} \qquad \forall s'\in U_s, 
\] 
take a finite subcover and construct a partition of unity \(\eta_k=\eta_{s_k}\), then finally construct the function \({\widetilde} g\), etc. 

Estimate \eqref{mult-01} implies that instead of \eqref{pert-02} we have the estimate
\[
\| I - {\widetilde} g({\,\cdot\,}, s') f({\,\cdot\,}, s') \|{_{ {}_{\scriptstyle {{\mathcal{M}}({\mathfrak{A}})}}}} \le \frac12, 
\]
where \({\mathcal{M}}({\mathfrak{A}})\) is the (left) multiplier algebra of \({\mathfrak{A}}\). But that implies that 
\[
\| \left( {\widetilde} g({\,\cdot\,}, s') f({\,\cdot\,}, s') \right)^{-1} \|{_{ {}_{\scriptstyle {{\mathcal{M}}({\mathfrak{A}})}}}} \le 2
\]
so all the estimates in Theorem \ref{t:main-03} follow. 

\begin{bibdiv}
\begin{biblist}

\bib{Carl}{article}{
   author={Carleson, Lennart},
   title={Interpolations by bounded analytic functions and the corona
   problem},
   journal={Ann. of Math. (2)},
   volume={76},
   date={1962},
   pages={547--559}
}

\bib{Garnett}{book}{
   author={Garnett, John B.},
   title={Bounded analytic functions},
   series={Graduate Texts in Mathematics},
   volume={236},
   edition={1},
   publisher={Springer},
   place={New York},
   date={2007},
   pages={xiv+459}
}

\bib{Nik-book-v1}{book}{
  title = {Operators, functions, and systems: an easy reading. {V}ol. 1},
  publisher = {American Mathematical Society},
  year = {2002},
  author = {Nikolski, Nikolai K.},
  volume = {92},
  pages = {xiv+461},
  series = {Mathematical Surveys and Monographs},
  address = {Providence, RI},
  note = {Hardy, Hankel, and Toeplitz, Translated from the French by Andreas
        Hartmann},
  isbn = {0-8218-1083-9}
}

\bib{Nik}{book}{
   author={Nikol{\cprime}ski{\u\i}, N. K.},
   title={Treatise on the shift operator},
   series={Grundlehren der Mathematischen Wissenschaften [Fundamental
   Principles of Mathematical Sciences]},
   volume={273},
   note={Spectral function theory;
   With an appendix by S. V. Hru\v s\v cev [S. V. Khrushch\"ev] and V. V.
   Peller;
   Translated from the Russian by Jaak Peetre},
   publisher={Springer-Verlag},
   place={Berlin},
   date={1986},
   pages={xii+491}
}

\bib{Fuhr}{article}{
   author={Fuhrmann, Paul A.},
   title={On the corona theorem and its application to spectral problems in
   Hilbert space},
   journal={Trans. Amer. Math. Soc.},
   volume={132},
   date={1968},
   pages={55--66}
}

\bib{Rosen}{article}{
   author={Rosenblum, Marvin},
   title={A corona theorem for countably many functions},
   journal={Integral Equations Operator Theory},
   volume={3},
   date={1980},
   number={1},
   pages={125--137}
}

\bib{Tolo}{article}{
   author={Tolokonnikov, V. A.},
   title={Estimates in the Carleson corona theorem, ideals of the algebra
   $H^{\infty }$, a problem of Sz.-Nagy},
   language={Russian, with English summary},
   note={Investigations on linear operators and the theory of functions,
   XI},
   journal={Zap. Nauchn. Sem. Leningrad. Otdel. Mat. Inst. Steklov. (LOMI)},
   volume={113},
   date={1981},
   pages={178--198, 267}
}

\bib{MR2449054}{article}{
   author={Treil, Sergei},
   author={Wick, Brett D.},
   title={Analytic projections, corona problem and geometry of holomorphic
   vector bundles},
   journal={J. Amer. Math. Soc.},
   volume={22},
   date={2009},
   number={1},
   pages={55--76}
}
		

\bib{MR2158178}{article}{
   author={Treil, Sergei},
   author={Wick, Brett D.},
   title={The matrix-valued $H^p$ corona problem in the disk and
   polydisk},
   journal={J. Funct. Anal.},
   volume={226},
   date={2005},
   number={1},
   pages={138--172}
}
		

\bib{Uchiyama}{article}{
   author={Uchiyama, A.},
   title={Corona Theorems for Countably Many Functions and Estimates for their Solutions},
   year={1980},
   status={preprint}
}

\end{biblist}
\end{bibdiv}

\end{document}

****************************************

The end of the paper, do not need anything after that for the proof?

*****************************************

In \cite{J} P.~Jones give an integral formula giving the a bounded on the boundary solution of the ${\ensuremath{\bar{\partial}}}$-equation
\[
{\ensuremath{\bar{\partial}}} u = \mu
\]
for a Carleson measure $\mu$. After an easy modification this result can be used to obtain a solution  depending smoothly on a parameter $s$, provided the measure $\mu$ depends smoothly on $s$  (the P.~Jones' solution involves the variation $|\mu|$, so one have to be a bit careful about smooth dependence here). 

Then we use as a starting point a modification of the original Carleson's solution of the Corona Problem presented in the Garnett's book \cite[Ch. VIII, s.~5]{Garnett}. We show that the right sides of the corresponding ${\ensuremath{\bar{\partial}}}$-equations can be made smoothly depending on the corona data, and then use P.~Jones' formula to get the solution depending smoothly on the data. 

The following theorem was obtained in \cite{Garnett} and smoothing out the characteristic function of the corresponding domain. 

\begin{thm}[Garnett, {\protect\cite[Chapt.~VIII, Theorem 5.1]{Garnett}}]
\label{CContour_Garnett}
Let $\delta>0$.  If $f(z)\in H^\infty({\mathbb{D}})$ with ${\ensuremath{\left\|{f}\right\|}}_\infty\leq 1$, then there exists a $\psi(z)\in C^\infty({\mathbb{D}})$ such that
\begin{itemize}
\item[(a)] $0\leq\psi(z)\leq 1$;
\item[(b)] $\psi(z)=1$ if ${\ensuremath{\left\vert{f(z)}\right\vert}}\geq\delta$;
\item[(c)] $\psi(z)=0$ if ${\ensuremath{\left\vert{f(z)}\right\vert}}<\epsilon(\delta)<\delta$;
\item[(d)] The measure ${\ensuremath{\left\vert{{\overline\partial}_z\psi(z)}\right\vert}}dxdy$ is Carleson with the Carleson norm at least $A(\delta)$. 
\end{itemize}
The constants $\epsilon(\delta)>0$ and $A(\delta)$ depend only on the parameter $\delta$.
\end{thm}

We will need the following modification of this theorem, that gives us a smooth dependence on the parameter $s$. 

\begin{thm}
\label{SmoothCContour}
Let $\delta>0$.  If $ f\in C(K, H^\infty)$ with $\sup_{z\in{\mathbb{D}}}\sup_{s\in[0,1]}{\ensuremath{\left\vert{f(z;s)}\right\vert}}\leq 1$, then there exists a function $(z,s)\mapsto \psi(z,s)\in C(K;C^\infty({\mathbb{D}}))$, $z\in {\mathbb{D}}$, $s\in K$ such that
\begin{itemize}
\item[(a)] $0\leq\psi(z;s)\leq 1$;
\item[(b)] $\psi(z;s)=1$ if ${\ensuremath{\left\vert{f(z;s)}\right\vert}}\geq\delta$;
\item[(c)] $\psi(z;s)=0$ if ${\ensuremath{\left\vert{f(z;s)}\right\vert}}<\epsilon(\delta)<\delta$;
\item[(d)] The measures ${\ensuremath{\left\vert{{\overline\partial}_z\psi(z, s)}\right\vert}}dxdy$, $s\in K$ is Carleson with the Carleson norm at least $A(\delta)$
\end{itemize}
The constants $\epsilon(\delta)>0$ and $A(\delta)$ depend only on the parameter $\delta$.

Moreover, if $f\in C^r(K;H^\infty)$ then $\psi\in C^r(K;C^\infty({\mathbb{D}}))$ and
\[
\| \psi \|{_{ {}_{\scriptstyle {C^r(K;C^1({\mathbb{D}}))}}}} \le C(K, \delta) \|f\|{_{ {}_{\scriptstyle {C^r(K; H^\infty({\mathbb{D}}))}}}}. 
\]
Similarly, if $f\in H^\infty(C^r(K))$ then $\psi \in C^\infty({\mathbb{D}}; C^r(K))$ and
\[
\| \psi \|{_{ {}_{\scriptstyle {C^1({\mathbb{D}};C^r(K))}}}} \le C(K, \delta) \|f\|{_{ {}_{\scriptstyle {H^\infty(C^r(K))}}}}
\]
\end{thm}

Let $\delta>0$ be given and $\epsilon(\delta)$ from Theorem \ref{CContour_Garnett} and suppose that $f, \tilde{f}\in H^\infty({\mathbb{D}})$ with ${\ensuremath{\left\|{f-\tilde{f}}\right\|}}_\infty<\min\left\{\delta,\frac{\epsilon(\delta)}{2}\right\}$.  Then we claim that if $\psi$ is the function from Theorem \ref{CContour_Garnett} associated to the function $f$, then that same function also works for $\tilde{f}$, but with (slightly) worse constants.  Parts (a) and (d) are of course unchanged, and it remains to handle (b) and (c).  To see part (b), note that by the triangle inequality we have
\begin{equation*}
{\ensuremath{\left\vert{\tilde{f}(z)}\right\vert}} \leq {\ensuremath{\left\vert{f(z)}\right\vert}}+{\ensuremath{\left\vert{f(z)-\tilde{f}(z)}\right\vert}}\leq\delta+{\ensuremath{\left\vert{f(z)}\right\vert}},
\end{equation*}
which implies
$$
\left\{z\in{\mathbb{D}}: {\ensuremath{\left\vert{\tilde{f}(z)}\right\vert}}\geq2\delta\right\}\subset\left\{z\in{\mathbb{D}}: {\ensuremath{\left\vert{f(z)}\right\vert}}\geq\delta\right\}.
$$
Thus, on the set where ${\ensuremath{\left\vert{\tilde{f}(z)}\right\vert}}\geq 2\delta$ then $\psi(z)=1$ as well.  For part (c), note that
$$
{\ensuremath{\left\vert{f(z)}\right\vert}}\leq{\ensuremath{\left\vert{\tilde{f}(z)}\right\vert}}+{\ensuremath{\left\vert{f(z)-\tilde{f}(z)}\right\vert}}\leq\frac{\epsilon(\delta)}{2}+{\ensuremath{\left\vert{\tilde{f}(z)}\right\vert}}.
$$
This implies that
$$
\left\{z\in{\mathbb{D}}: {\ensuremath{\left\vert{\tilde{f}(z)}\right\vert}}\leq\frac{\epsilon(\delta)}{2}\right\}\subset\left\{z\in{\mathbb{D}}: {\ensuremath{\left\vert{f(z)}\right\vert}}\leq\epsilon(\delta)\right\}.
$$
So we have that $\psi(z)=0$ on the set ${\ensuremath{\left\vert{\tilde{f}(z)}\right\vert}}<\frac{\epsilon(\delta)}{2}$.  These computations prove the claim that $\psi$ also works for the function $\tilde{f}$, but with slightly worse constants.

For a function $f\in H^\infty({\mathbb{D}};C^{\infty}([0,1])$, we fix an $s\in[0,1]$.  By Theorem \ref{CContour_Garnett}, there exists a corresponding function $\psi_s$.  Now the observation above, gives that there exists an open set $U_s$ such that if $s'\in U_s$, then the function $\psi_s$ works for both $f(z;s)$ and $f(z;s')$.  Then the sets $\{U_s\}_{s\in[0,1]}$ form an open cover of $[0,1]$ (in the relative topology) and so by compactness there exists a finite number $N$ and points $s_1,\ldots, s_N\in[0,1]$ such that $U_j:=U_{s_j}$ is a finite open cover.  To each $s_j$ we also have the corresponding function $\psi_{j}:=\psi_{s_j}$ obtained by applying Theorem \ref{CContour_Garnett} to the function $f(z;s_j)$.

Let $\sigma$ be a finite subset $\{1,\ldots, N\}$.  Suppose that $s\in \cap_{j\in\sigma} U_j$, then we have that for all $j\in\sigma$ that $\psi_j$ satisfies the conclusions of the Theorem \ref{CContour_Garnett} for the function $f(z;s)$.  So in fact any convex combination of functions $\psi_j$ will work.

Now take a smooth partition of unity $\{\eta_j(s)\}_{j=1}^N$ subordinated to the collection of open sets $\{U_j\}_{j=1}^N$ such that 
\begin{itemize}
\item[(i)] $0\leq\eta_j(s)\leq 1$ for all $s\in[0,1]$ and $1\leq j\leq N$;
\item[(ii)] $\sum_{j=1}^N \eta_j(s)=1$;
\item[(iii)] $\operatorname{supp}\,\eta_j$ is compactly contained in $U_j$ for $1\leq j\leq N$.
\end{itemize}
Now define
\begin{equation}
\label{Smooth_CContour_Garnett}
\psi(z;s):=\sum_{j=1}^N \psi_j(z)\eta_j(s).
\end{equation}
We then have the function $\psi(z;s)$ is smooth in both $z$ and $s$ and satisfies the conclusions of Theorem \ref{CContour_Garnett}.  More precisely, 

\begin{proof}
It is clear that $\psi(z;s)$ is smooth on ${\mathbb{D}}\times[0,1]$.  To see part (a), note that $0\leq\eta_j(s)\leq 1$ and $0\leq\psi_j(z)\leq1$, and so the lower bound trivially holds, while for the upper bound we use
$$
\psi(z;s)=\sum_{j=1}^N \psi_j(z)\eta_j(s)\leq\sum_{j=1}^N \eta_j(s)=1.
$$
We now turn to parts (b) and (c) since they are handled similarly.  Fix $s$ and let $\sigma(s)$ be the subset of $\{1,\ldots,N\}$ such that $\eta_k(s)\neq 0$.  Note that if $k\in\sigma(s)$, then we have that $s\in U_k$, and so the functions $\psi_k$ will satisfy the conclusions of Theorem \ref{CContour_Garnett} for the function $f(z;s)$.  Now if ${\ensuremath{\left\vert{f(z;s)}\right\vert}}\geq\delta$, then we have that $\psi_k(z)=1$ for $k\in\sigma(s)$.  Thus, we have
$$
\psi(z;s)=\sum_{j=1}^{N}\psi_j(z)\eta_j(s)=\sum_{k\in\sigma(s)}\psi_k(z)\eta_k(s)=\sum_{k\in\sigma(s)}\eta_k(s)=1.
$$
Similarly, if ${\ensuremath{\left\vert{f(z;s)}\right\vert}}<\epsilon(\delta)$, then we have that $\psi_k(z)=0$ for $k\in\sigma(s)$ and so
$$
\psi(z;s)=\sum_{j=1}^{N}\psi_j(z)\eta_j(s)=\sum_{k\in\sigma(s)}\psi_k(z)\eta_k(s)=0.
$$

Finally, for part (d), note that we have
$$
{\overline\partial}_z\psi(z;s)=\sum_{j=1}^N \eta_j(s){\overline\partial}_z\psi_j(z).
$$
Using this we have for $S=\{z=re^{i\theta}:\theta_0<\theta<\theta_0+\ell, 1-\ell<r<1\}$ that 
$$
\int_{S}{\ensuremath{\left\vert{{\overline\partial}_z\psi(z;s)}\right\vert}}dxdy\leq \sum_{j=1}^N\eta_j(s)\int_{S}{\ensuremath{\left\vert{{\overline\partial}_z\psi_j(z)}\right\vert}}dxdy\leq A(\delta)\ell \sum_{j=1}^N\eta_j(s)=A(\delta)\ell.
$$

\end{proof}

\begin{rem}
For simplicity, we worked with $H^\infty({\mathbb{D}}; C^{\infty}([0,1]))$.  For an arbitrary compact subset $K\subset{\mathbb{C}}^n$, the construction can be done in an analogous manner.
\end{rem}

\section{The Corona Theorem with Smooth Data}

As is well known, to solve the Corona problem one (typically) uses ${\overline\partial}$-equations.  Our tool to solve the ${\overline\partial}$-equations and problems encountered is the following theorem of Jones.  This theorem allows us to provide a construction solution to the resulting ${\overline\partial}$-problems and will allow for smooth dependence upon the parameters in question.

\begin{thm}[Jones, \cite{J}]
\label{continuousJones'}
Let $\mu $ be a complex $H^2(\mathbb{D})$ Carleson measure on $\mathbb{D}$. Then with $S\left( \mu \right)(z) $ given by
\begin{equation}
\label{Jonessolutionop}
S\left( \mu \right) \left( z\right) :=\int_{\mathbb{D}}K\left(\sigma, z, \zeta\right) d\mu \left( \zeta \right)
\end{equation}where $\sigma:=\frac{\left\vert \mu \right\vert }{\left\Vert \mu \right\Vert
_{CM(H^2)}}$ and\begin{equation*}
K\left( \sigma ,z,\zeta \right) := \frac{2i}{\pi }\frac{1-\left\vert
\zeta \right\vert ^{2}}{\left( z-\zeta \right) \left( 1-\overline{\zeta }z\right) }\exp \left\{\int_{\left\vert \omega \right\vert \geq
\left\vert \zeta \right\vert }\left( -\frac{1+\overline{\omega }z}{1-\overline{\omega }z}+\frac{1+\overline{\omega }\zeta }{1-\overline{\omega }\zeta }\right) d\sigma \left( \omega \right) \right\} ,
\end{equation*}we have that:
\begin{enumerate}
\item $S\left( \mu \right) \in L_{loc}^{1}\left( \mathbb{D}\right) $.

\item $\overline{\partial}S\left(\mu\right) =\mu $ in
the sense of distributions.

\item $\int_{\mathbb{D}}\left\vert K\left( \frac{\left\vert \mu
\right\vert }{\left\Vert \mu \right\Vert _{CM(H^2)}},x,\zeta \right) \right\vert
d\left\vert \mu \right\vert \left( \zeta \right) \lesssim \left\Vert \mu
\right\Vert _{CM(H^2)}$ for all $x\in \mathbb{T}=\partial \mathbb{D}$,\\
\item[] so $\left\Vert S\left( \mu \right) \right\Vert _{L^{\infty }\left( \mathbb{T}\right) }\lesssim\left\Vert \mu \right\Vert _{CM(H^2)}$.
\end{enumerate}
\end{thm}
We remark that Jones only stated his Theorem in the case of the upper half plane, and the statement above is the standard translation of his result to the unit disc.  See Xiao, \cite{X} for a proof of this Theorem in this case.

Suppose now that we have $f_1,\ldots, f_n\in H^\infty\left({\mathbb{D}};C^{\infty}[0,1]\right)$ with ${\ensuremath{\left\vert{f_k(z;s)}\right\vert}}\leq 1$ for $1,\ldots, k$ and
$$
\max_{1\leq k\leq n}{\ensuremath{\left\vert{f_k(z;s)}\right\vert}}\geq\delta\quad \forall z\in{\mathbb{D}}\quad s\in[0,1].
$$

For $k=1,\ldots, n$, apply Theorem \ref{SmoothCContour} to each $f_k(z;s)$ and find smooth functions $\psi_k(z;s)$ satisfying:
\begin{itemize}
\item[(a)] $0\leq\psi_k(z;s)\leq 1$;
\item[(b)] $\psi_k(z;s)=1$ if ${\ensuremath{\left\vert{f_k(z;s)}\right\vert}}\geq\delta$;
\item[(c)] $\psi_k(z;s)=0$ if ${\ensuremath{\left\vert{f_k(z;s)}\right\vert}}<\epsilon(\delta)$;
\item[(d)] $\int_{S}{\ensuremath{\left\vert{{\overline\partial}_z\psi_k(z;s)}\right\vert}}dxdy\leq A(\delta)\ell$ for every Carleson box $S$ given by $$
S=\left\{re^{i\theta}:\theta_0<\theta<\theta_0+\ell, 1-\ell<r<1\right\}.$$
\end{itemize}
We use these functions $\{\psi_k\}$ to solve the Corona problem for the data $f_k$.  Now define
$$
\varphi_{k}(z;s)=\frac{\psi_{k}(z;s)}{f_k(z;s)\sum_{l=1}^n \psi_{l}(z;s)}
$$
and note that 
\begin{itemize}
\item[(i)] $\sum_{k=1}^n f_k(z;s)\varphi_{k}(z;s)=1$ for all $z\in{\mathbb{D}}$ and $s\in[0,1]$;
\item[(ii)] ${\ensuremath{\left\vert{\varphi_k(z;s)}\right\vert}}\leq \epsilon(\delta)^{-1}$ for all $z\in{\mathbb{D}}$, $s\in[0,1]$ and $1\leq k\leq n$.
\end{itemize}
It is immediate to see that (i) holds and to see (ii), one uses property (c) of the function $\psi_k$.  Since ${\ensuremath{\left\vert{f_k(z;s)}\right\vert}}<\epsilon(\delta)$, then we have that $\psi_k(z;s)=0$ and so $\varphi_k(z;s)=0$.  And, if ${\ensuremath{\left\vert{f_k(z;s)}\right\vert}}\geq\epsilon(\delta)$, then we have
$$
{\ensuremath{\left\vert{\varphi_k(z;s)}\right\vert}}={\ensuremath{\left\vert{\frac{\psi_{k}(z;s)}{f_k(z;s)\sum_{l=1}^n \psi_{l}(z;s)}}\right\vert}}\leq \frac{1}{\epsilon(\delta)}\frac{\psi_{k}(z;s)}{\sum_{l=1}^n \psi_{l}(z;s)}\leq\frac{1}{\epsilon(\delta)}.
$$
We set
$$
g_j(z;s)=\varphi_j(z;s)+\sum_{k=1}^n a_{j,k}(z;s) f_k(z;s)
$$
where the functions $a_{j,k}(z;s)$ are to be determined, though we will require that $a_{j,k}(z;s)=-a_{k,j}(z;s)$.  Note that this alternating condition implies that
$$
\sum_{j=1}^n f_j(z;s)g_j(z;s)=\sum_{j=1}^n f_j(z;s)\varphi_j(z;s)+\sum_{j=1}^n\sum_{k=1}^n a_{j,k}(z;s) f_j(z;s)f_k(z;s)=1.
$$
To have the alternating characterisitic of $a_{j,k}$ we set $a_{j,k}(z;s)=b_{j,k}(z;s)-b_{k,j}(z;s)$ for some yet to be determined functions.  We will chose the functions $b_{j,k}$ to be solutions to the following $\overline{\partial}$-problem:
\begin{equation}
\label{dbar}
\overline{\partial}_z b_{j,k}(z;s)=\varphi_j(z;s)\overline{\partial}_z\varphi_k(z;s).
\end{equation}
It is well known that this choice of $b_{j,k}$ forces $g_j(z;s)$ to be analytic in the disc.  Indeed, we have (suppressing the dependence on $(z;s)$ for simplicity)
\begin{eqnarray*}
\overline{\partial}_z g_j & = & \overline{\partial}_z \varphi_j+\sum_{k=1}^n f_k\overline{\partial}_za_{j,k}\\
& = & \overline{\partial}_z \varphi_j+\sum_{k=1}^n f_k\left(\overline{\partial}_zb_{j,k}-\overline{\partial}_zb_{k,j}\right)\\
& = & \overline{\partial}_z \varphi_j+\sum_{k=1}^n f_k\left(\varphi_j\overline{\partial}_z\varphi_k-\varphi_k\overline{\partial}_z\varphi_j\right)\\
& = &  \overline{\partial}_z \varphi_j+\varphi_j\overline{\partial}_z\left(\sum_{k=1}^n f_k\varphi_k\right)-\overline{\partial}_z\varphi_j\sum_{k=1}^nf_k\varphi_k\\
& = & \overline{\partial}_z \varphi_j+\varphi_j\overline{\partial}_z1-\overline{\partial}_z\varphi_j1=0.
\end{eqnarray*}
If we have that $b_{j,k}$ are bounded, then, we will also have that $g_j$ are bounded as well.

So, we need to guarantee that the solutions of \eqref{dbar} are bounded, which can be accomplished by Theorem \ref{continuousJones'}.  However, to apply Theorem \ref{continuousJones'}, we need to know that the right-hand side of \eqref{dbar} generates a Carleson measure for $H^2$.  But, this follows easily from the properties of the functions $\psi_k(z;s)$.  To see this, we compute to find that
$$
\varphi_j{\overline\partial}_z\varphi_k=\varphi_j\frac{\sum_{l=1}^n \left(\psi_l{\overline\partial}_z\psi_k-\psi_k{\overline\partial}_z\psi_l\right)}{f_k\left(\sum_{l=1}^n\psi_l\right)^2}.
$$
However, by property (ii) above, property (a), (b) of $\psi_k$, and the fact that ${\ensuremath{\left\vert{f_k(z;s)}\right\vert}}\geq \delta$, we have
$$
{\ensuremath{\left\vert{\varphi_j{\overline\partial}_z\varphi_k}\right\vert}}\lesssim \frac{1}{\delta\epsilon(\delta)}\sum_{l=1}^n{\ensuremath{\left\vert{{\overline\partial}_z\psi_l}\right\vert}}.
$$ 
However, we have by property (d) that the Carleson measure condition for $\varphi_j{\overline\partial}_z\varphi_k$ holds since for a Carleson box $S$ given by $S=\{re^{i\theta}:\theta_0<\theta<\theta_0+\ell, 1-\ell<r<1\}$, we have
$$
\int_{S} {\ensuremath{\left\vert{\varphi_j(z;s){\overline\partial}_z\varphi_k(z;s)}\right\vert}}dxdy\lesssim \frac{1}{\delta\epsilon(\delta)}\sum_{j=1}^n\int_{S}{\ensuremath{\left\vert{{\overline\partial}_z\psi_j(z;s)}\right\vert}}dxdy\lesssim  \frac{1}{\delta\epsilon(\delta)}\sum_{j=1}^n A(\delta)\ell=nC(\delta)\ell.
$$
So by Theorem \ref{continuousJones'}, we have that $b_{j,k}(z;s)$ is bounded by $C(n,\delta)$, and so we have that $g_j(z;s)$ is bounded by $C(n,\delta)$.

However, from the Jones construction, and the explicit formula for $g_j$ (technically, the $b_{j,k}$) we can see that the solutions depend continuously on the data $f_k$.

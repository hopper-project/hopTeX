\documentclass[12pt]{amsart}
\usepackage{amsmath, amssymb, amsfonts}
\usepackage[all]{xypic}
\usepackage[pdftex]{graphicx}

\paperheight=297mm
\paperwidth=210mm
\setlength{\oddsidemargin}{0pt}
\setlength{\evensidemargin}{0pt} \setlength{\headheight}{12pt}
\setlength{\footskip}{36pt}
\setlength{\hoffset}{0in}\setlength{\voffset}{-24pt}
\setlength{\topmargin}{0pt}
\setlength{\headsep}{12pt} \setlength{\marginparwidth}{0pt}
\setlength{\marginparpush}{0pt} \setlength{\textwidth}{210mm}
\addtolength{\textwidth}{-2in} \setlength{\textheight}{297mm}
\addtolength{\textheight}{-24pt}
\addtolength{\textheight}{-2in}

\theoremstyle{plain}
\newtheorem{theorem}{Theorem}[section]
\newtheorem{lemma}[theorem]{Lemma}
\newtheorem{corollary}[theorem]{Corollary}
\newtheorem{prop}[theorem]{Proposition}
\newtheorem{hyp}[theorem]{Hypothesis}
\newtheorem{algorithm}[theorem]{Algorithm}
\newtheorem{conj}[theorem]{Conjecture}

\theoremstyle{remark}
\newtheorem{case}{Case}
\newtheorem{subcase}{Subcase}[case]
\newtheorem{subsubcase}{Subsubcase}[subcase]
\newtheorem{note}[theorem]{Note}
\newtheorem{remark}[theorem]{Remark}
\newtheorem{question}[theorem]{Question}
\newtheorem{remarks}[theorem]{Remarks}
\newtheorem{example}[theorem]{Example}
\newtheorem{examples}[theorem]{Examples}
\newtheorem*{note*}{Note}
\newtheorem*{remark*}{Remark}
\newtheorem*{example*}{Example}

\theoremstyle{definition}
\newtheorem*{definition*}{Definition}
\newtheorem*{hypothesis*}{Hypothesis}
\newtheorem*{assumptions*}{Assumptions}
\newtheorem{definition}[theorem]{Definition}
\newtheorem{notation}[theorem]{Notation}

\numberwithin{equation}{section}

\title[Hybrid Iwasawa algebras and the equivariant Iwasawa main conjecture]
    {Hybrid Iwasawa algebras,\\ the equivariant Iwasawa main conjecture\\ and applications}

\author{Henri Johnston}
\address{
Department of Mathematics\\
University of Exeter\\
Exeter\\
EX4 4QF\\
U.K.
}
\email{H.Johnston@exeter.ac.uk}
\urladdr{http://emps.exeter.ac.uk/mathematics/staff/hj241}

\author{Andreas Nickel}
\address{
Universit\"{a}t Bielefeld\\
Fakult\"{a}t f\"{u}r Mathematik\\
Postfach 100131\\
Universit\"{a}tsstr. 25\\
33501 Bielefeld\\
Germany}
\email{anickel3@math.uni-bielefeld.de}
\urladdr{http://www.math.uni-bielefeld.de/$\sim$anickel3/english.html}

\subjclass[2010]{11R23, 11R42}
\keywords{Iwasawa main conjecture, Iwasawa algebra, Brumer's conjecture, Stark's conjectures, equivariant $L$-values, class groups, annihilation}
\date{Version of 30th October 2014}
\begin{document}

\maketitle

\begin{abstract}
Let $p$ be an odd prime.
We give an unconditional proof of the equivariant Iwasawa main conjecture for totally real fields
for an infinite class of one-dimensional non-abelian $p$-adic Lie extensions.
We then show that the $p$-primary parts of refinements of the
(imprimitive, not necessarily abelian) Brumer and Brumer-Stark conjectures are
implied by the equivariant Iwasawa main conjecture.
Crucially, neither result depends on the vanishing of the relevant Iwasawa $\mu$-invariant.
\end{abstract}

\section{Introduction}

This article consists of two main parts.
First, we give an unconditional proof of the
equivariant Iwasawa main conjecture (EIMC) for totally real fields
for an infinite class of one-dimensional non-abelian $p$-adic Lie extensions.
This is achieved by carefully studying the structure of certain noncommutative Iwasawa algebras and is mainly algebraic in nature.
Second, by using new arithmetic ideas we show that the EIMC implies the $p$-primary parts of refinements of the
imprimitive Brumer and Brumer-Stark conjectures.
These latter conjectures are classical when the relevant Galois group is abelian, and have been generalised to 
the non-abelian case by the second named author \cite{MR2976321} (independently,
they have been formulated in even greater generality by Burns \cite{MR2845620}).
By combining the two main results, we give unconditional proofs of 
the non-abelian Brumer and Brumer-Stark conjectures in many cases.
We stress that neither of the main results depends on the vanishing of the relevant Iwasawa $\mu$-invariant.\\

Let $p$ be an odd prime.
Let $K$ be a totally real number field and let $K_{\infty}$ be the cyclotomic ${\mathbb{Z}}_{p}$-extension of $K$.
An admissible $p$-adic Lie extension $\mathcal{L}$ of $K$ is a Galois extension $\mathcal{L}$ of $K$ such that 
(i) $\mathcal{L}/K$ is unramified outside a finite set of primes $S$ of $K$,
(ii) $\mathcal{L}$ is totally real, (iii) $\mathcal{G} := {\mathrm{Gal}}(\mathcal{L}/K)$ is a compact 
$p$-adic Lie group, and (iv) $\mathcal{L}$ contains $K_{\infty}$. 
Let $M_{S}^{\mathrm{ab}}(p)$ be the maximal abelian $p$-extension of $\mathcal{L}$ unramified outside the set of primes above $S$.
Let $\Lambda(\mathcal{G}):={\mathbb{Z}}_{p}[[\mathcal{G}]]$ denote the Iwasawa algebra of $\mathcal{G}$ over ${\mathbb{Z}}_{p}$
and let $X_{S}$ denote the (left) $\Lambda(\mathcal{G})$-module ${\mathrm{Gal}}(M_{S}^{\mathrm{ab}}(p) / \mathcal{L})$. 
Roughly speaking, the equivariant Iwasawa main conjecture (EIMC) relates $X_{S}$ to special values of Artin $L$-functions
via $p$-adic $L$-functions. This relationship can be expressed as the existence of a certain element in an algebraic $K$-group;
it is also conjectured that this element is unique.
 
There are at least three different versions of the EIMC. 
The first is due to Ritter and Weiss and deals with the case where $\mathcal{G}$ is a one-dimensional $p$-adic Lie group \cite{MR2114937},
and was proven  under a certain `$\mu=0$' hypothesis in a series of articles culminating in \cite{MR2813337}.
The second version follows the framework of Coates, Fukaya, Kato, Sujatha and Venjakob \cite{MR2217048} and 
was proven by Kakde \cite{MR3091976}, again assuming $\mu=0$.
This version is for $\mathcal{G}$ of arbitrary (finite) dimension and Kakde's proof uses a strategy of Burns and Kato to reduce to the one-dimensional case
(see Burns \cite{burns-mc}). Finally, Greither and Popescu \cite{GrP-EIMC} have formulated and proven an EIMC via the Tate module of a certain Iwasawa-theoretic abstract $1$-motive, but they restricted their formulation to one-dimensional abelian extensions and the formulation itself
requires a $\mu=0$ hypothesis.
In \cite{MR3072281}, the second named author generalised this formulation (again assuming $\mu=0$) to the one-dimensional non-abelian case, and in the situation that all three formulations make sense (i.e.\ $\mathcal{G}$ is a one-dimensional $p$-adic Lie group and $\mu=0$), 
he showed that they are in fact all equivalent.
Venjakob \cite{MR3068897} has also compared the work of Ritter and Weiss to that of Kakde.

The classical Iwasawa $\mu=0$ conjecture (at $p$) is the assertion that for every number field $F$, the
Galois group of the maximal unramified abelian $p$-extension of $F_{\infty}$ is finitely generated as a ${\mathbb{Z}}_{p}$-module.
This conjecture was proven by Ferrero and Washington \cite{MR528968} in the case that $F/{\mathbb{Q}}$ is abelian, but little progress has been made since.
The $\mu=0$ hypothesis for an admissible $p$-adic Lie extension $\mathcal{L}/K$ discussed in the paragraph above 
is implied by the classical Iwasawa $\mu=0$ conjecture (at $p$) for a certain finite extension of $K$.
It follows from the result of Ferrero and Washington that the EIMC holds unconditionally when $\mathcal{L}/K$ 
is an admissible pro-$p$ extension and $K$ is abelian over ${\mathbb{Q}}$.

We wish to prove the EIMC in cases in which the $\mu=0$ hypothesis is not known. 
As a consequence, we must restrict to the case in which $\mathcal{G}$ is one-dimensional because the
`$\mathfrak{M}_{H}(G)$-conjecture' is required to even formulate the EIMC. 
This conjecture is known unconditionally in the one-dimensional case but in the more general case it is 
presently only known to hold under the $\mu=0$ hypothesis (see \S \ref{subsec:higher-rk}).
We remark that in the one-dimensional case, the $\mu=0$ hypothesis is equivalent to 
the Iwasawa module $X_{S}$ being finitely generated as a ${\mathbb{Z}}_{p}$-module; 
a detailed discussion of the relation to the classical Iwasawa $\mu=0$ conjecture in this setting is given in Remark \ref{rmk:mu=0}.

The main result of the first half of this article is the unconditional proof of the EIMC for an infinite class of admissible extensions with Galois group $\mathcal{G}$
a one-dimensional non-abelian $p$-adic Lie group and for which the $\mu=0$ hypothesis is not known.
A key ingredient is a result of Ritter and Weiss \cite{MR2114937} which, roughly speaking,
says that a version of the EIMC `over maximal orders' (or `character by character') holds without any $\mu=0$ hypothesis.
The proof uses Brauer induction to reduce to the abelian case, which is essentially equivalent to
the Iwasawa main conjecture for totally real fields proven by Wiles \cite{MR1053488}. 
Another key ingredient is the notion of `hybrid Iwasawa algebras' which is an adaptation of 
the notion of `hybrid $p$-adic group rings' first introduced by the present authors in \cite{hybrid-ETNC}.
Let $p$ be a prime and $G$ be a finite group with normal subgroup $N$.
The group ring ${\mathbb{Z}}_{p}[G]$ is said to be `$N$-hybrid' if ${\mathbb{Z}}_{p}[G]$ is isomorphic to the direct product of ${\mathbb{Z}}_{p}[G/N]$ and a maximal order.
Now suppose that $p$ is odd and that $\mathcal{G}$ is a one-dimensional $p$-adic Lie group with a finite normal subgroup $N$.
Then the Iwasawa algebra $\Lambda(\mathcal{G})$ is said to be `$N$-hybrid' 
if it decomposes into a direct product of $\Lambda(\mathcal{G}/N)$ and a maximal order.
By using the maximal order variant of the EIMC and certain functoriality properties, we show that the EIMC for the 
full extension (corresponding to $\mathcal{G}$) is equivalent to the EIMC for the sub-extension corresponding to $\mathcal{G}/N$.
There are many cases in which $\mu=0$ is not known for the full extension, but is known for the sub-extension,
and thus we obtain new unconditional results. 
However, we first need explicit criteria for $\Lambda(\mathcal{G})$ to be $N$-hybrid.
Since $\mathcal{G}$ is one-dimensional it decomposes as a semidirect product $\mathcal{G} \simeq H \rtimes \Gamma$
where $H$ is finite and $\Gamma$ is isomorphic to ${\mathbb{Z}}_{p}$. 
Moreover, if $N$ is a finite normal subgroup of $\mathcal{G}$ then it must in fact be a (normal) subgroup of $H$.
We show that $\Lambda(\mathcal{G})$ is $N$-hybrid if and only if ${\mathbb{Z}}_{p}[H]$ is $N$-hybrid;
this is easy to see in the case that $\mathcal{G}$ is a direct product $H \times \Gamma$, 
but much more difficult in the general case. 
In \cite{hybrid-ETNC} we gave explicit criteria for ${\mathbb{Z}}_{p}[H]$ to be $N$-hybrid in terms of the degrees of the complex irreducible characters of $H$,
and thus the same criteria can be used to determine whether $\Lambda(\mathcal{G})$ is $N$-hybrid. 
With applications in the second part of the article in mind, we also study the behavior of $N$-hybrid
$p$-adic group rings ${\mathbb{Z}}_{p}[G]$ when we change the group $G$ and its normal subgroup $N$.
(As discussed in Remark \ref{rmk:applications-to-ETNC}, these new results on hybrid $p$-adic group rings also have applications 
to the equivariant Tamagawa number conjecture.)
\\ 

Now let $L/K$ be a finite Galois CM-extension of number fields with Galois group $G$. 
To each finite set $S$ of places of $K$ containing all the archimedean places, one can associate a so-called `Stickelberger element'
$\theta_{S}(L/K)$ in the centre of the complex group algebra ${\mathbb{C}}[G]$. 
This element is defined via values at $s=0$ of $S$-truncated Artin $L$-functions attached to the complex characters of $G$.
Let $\mu_{L}$ and ${\mathrm{cl}}_{L}$ denote the roots of unity and the class group of $L$, respectively.
Assume further that $S$ contains the set $S_{\mathrm{ram}}$ of all finite primes of $K$ that ramify in $L/K$. 
Then it was independently shown in \cite{MR524276, MR579702, MR525346} that for abelian $G$ we have
\[
    {\mathrm{Ann}}_{{\mathbb{Z}}[G]} (\mu_{L}) \theta_{S}(L/K) \subseteq {\mathbb{Z}}[G],
\]
where ${\mathrm{Ann}}_{{\mathbb{Z}}[G]}(M)$ denotes the annihilator ideal of a (left) ${\mathbb{Z}}[G]$-module $M$.
Now Brumer's conjecture asserts that in fact
\[
{\mathrm{Ann}}_{{\mathbb{Z}}[G]} (\mu_{L}) \theta_{S}(L/K) \subseteq {\mathrm{Ann}}_{{\mathbb{Z}}[G]}({\mathrm{cl}}_{L}).
\]
The Brumer-Stark conjecture is a refinement of Brumer's conjecture that not only asserts that the class of a given ideal 
is annihilated in ${\mathrm{cl}}_{L}$, so it becomes a principal ideal, but also gives information about a generator of that ideal. 
The second named author \cite{MR2976321} introduced non-abelian generalisations of Brumer's conjecture, 
the Brumer-Stark conjecture and of the so-called strong Brumer-Stark property.
The extension $L/K$ satisfies the latter property if certain Stickelberger elements are contained in
the (noncommutative) Fitting invariants of corresponding ray class groups; however, this property
does not hold in general, even in the case that $G$ is abelian,
as follows from results of Greither and Kurihara \cite{MR2443336}. 
If this property does hold, however, it also implies the validity of the (non-abelian) Brumer-Stark conjecture and Brumer's conjecture. 
The main result of the second half of this article is the proof that the EIMC implies 
the $p$-part of a dual version of the strong Brumer-Stark property under the mild restrictions
that $S$ contains all the $p$-adic places of $K$ and that a certain identity between complex and $p$-adic Artin $L$-functions at $s=0$ holds. 
This identity is satisfied when $G$ is monomial (i.e.\ every irreducible complex character of $G$ is induced from a linear character of a subgroup), 
and is conjecturally always true.

When the classical Iwasawa $\mu$-invariant vanishes, the above result on the dual version 
of the strong Brumer-Stark property has been already established by the second named author \cite{MR3072281}
which in turn was a non-abelian generalisation of work of Greither and Popescu \cite{GrP-EIMC}.
The main reason why the vanishing of $\mu$ was required is, of course, to ensure the validity of the EIMC. 
However, both articles use a formulation of the EIMC via the Tate module of a certain Iwasawa-theoretic abstract
$1$-motive which requires the vanishing of $\mu$ even for its formulation. This formulation is inspired by Deligne's theory
of $1$-motives \cite{MR0498552} and previous work of Greither and Popescu \cite{MR2899958}
on the Galois module structure of $p$-adic realisations of Picard $1$-motives.
Our new approach is different to, though partly inspired by, the approaches in \cite{GrP-EIMC, MR3072281}.
More precisely, we reinterpret certain well-known exact sequences involving ray class groups in terms of
\'{e}tale cohomology. Taking direct limits along the cyclotomic ${\mathbb{Z}}_{p}$-extension of $L$,
this allows us to establish a concrete link between the canonical complex
occurring in the EIMC and certain ray class groups. The theory of noncommutative Fitting invariants
then plays a crucial role in the Iwasawa descent.\\

This article is organised as follows. In \S \ref{sec:hybrid-group-rings} we review some material on hybrid
$p$-adic group rings, and show how new examples of such group rings can be obtained from existing examples.
In \S \ref{sec:hybrid-Iwasawa-alg} we generalise this notion to Iwasawa algebras and study its structure and basic properties.
In particular, we give explicit criteria for an Iwasawa algebra to be $N$-hybrid.
This enables us to give many examples of one-dimensional non-abelian $p$-adic Lie groups $\mathcal{G}$
such that the Iwasawa algebra $\Lambda(\mathcal{G})$ is $N$-hybrid for a non-trivial finite normal subgroup
$N$ of $\mathcal{G}$. 
In \S \ref{sec:EIMC} we give a slight reformulation of the EIMC that is convenient for our purposes. 
We also recall the functorial properties of the EIMC and reinterpret the maximal order variant of the EIMC of Ritter and Weiss \cite{MR2114937}.
The algebraic preparations of \S \ref{sec:hybrid-Iwasawa-alg} then permit us to verify the EIMC for many one-dimensional non-abelian
$p$-adic Lie extensions without assuming the $\mu=0$ hypothesis. These three sections form the first half of this article.

The second half begins with a review of noncommutative Fitting invariants in \S \ref{sec:noncommFitt}.
These were introduced by the second named author \cite{MR2609173} and were further developed by both the present authors in \cite{MR3092262}.
In \S \ref{sec:Brumer-Stark} we first reinterpret certain exact sequences involving ray class groups in terms of \'{e}tale cohomology.
We then recall the statements of the (non-abelian) Brumer and Brumer-Stark conjectures and show that a dual
version of the strong Brumer-Stark property implies both of these conjectures.
The bulk of the second half of this article is then devoted to the proof of its main result, 
namely that the EIMC implies that a certain $p$-adic Stickelberger element lies in the noncommutative Fitting
invariant of the Pontryagin dual of a certain ray class group. When the underlying Galois group $G$
is monomial then, roughly speaking,
the $p$-adic Stickelberger element coincides with the usual (complex) Stickelberger element, and
thus the dual of the strong Brumer-Stark property holds. 
As a consequence, we obtain many unconditional results
when we combine this with our results of \S \ref{sec:EIMC} on the EIMC.

\subsection*{Acknowledgements}
It is a pleasure to thank Werner Bley, Ted Chinburg, Takako Fukaya, Lennart Gehrmann, Cornelius Greither, Annette Huber-Klawitter, 
Mahesh Kakde, Kazuya Kato, Daniel Macias Castillo, Cristian Popescu, J\"urgen Ritter, Sujatha, Otmar Venjakob, Al Weiss and Malte Witte 
for helpful discussions and correspondence.
The second named author acknowledges financial support provided by the DFG within the Collaborative Research Center 701
`Spectral Structures and Topological Methods in Mathematics'.

\subsection*{Notation and conventions}
All rings are assumed to have an identity element and all modules are assumed
to be left modules unless otherwise  stated. We fix the following notation:

\medskip

\begin{tabular}{ll}
$S_{n}$ & the symmetric group on $n$ letters\\
$A_{n}$ & the alternating group on $n$ letters\\
$C_{n}$ & the cyclic group of order $n$\\
$D_{2n}$ & the dihedral group of order $2n$\\
$Q_{8}$ & the quaternion group of order $8$\\
$V_{4}$ & the subgroup of $A_{4}$ generated by double transpositions\\
${\mathbb{F}}_{q}$ & the finite field with $q$ elements, where $q$ is a prime power\\
${\mathrm{Aff}}(q)$ & the affine group isomorphic to ${\mathbb{F}}_{q} \rtimes {\mathbb{F}}_{q}^{\times}$ defined in Example \ref{ex:affine}\\
$v_{\mathfrak{p}}(x)$ & the $\mathfrak{p}$-adic valuation of $x$\\
$R^{\times}$ & the group of units of a ring $R$\\
$\zeta(R)$ & the centre of a ring $R$\\
$M_{m \times n} (R)$ & the set of all $m \times n$ matrices with entries in a ring $R$\\
$\zeta_{n}$ & a primitive $n$th root of unity\\
$K_{\infty}$ & the cyclotomic ${\mathbb{Z}}_{p}$-extension of the number field $K$\\
$K^{+}$ & the maximal totally real subfield of $K$\\
$\mu_{K}$ & the roots of unity of a field $K$\\
${\mathrm{cl}}_{K}$ & the class group of a number field $K$ \\
$K^{c}$ & an algebraic closure of a field $K$ \\
${\mathrm{Irr}}_{F}(G)$ & the set of $F$-irreducible characters of the (pro)-finite group $G$\\
& (with open kernel) where $F$ is a field of characteristic $0$

\end{tabular}

\section{Hybrid $p$-adic group rings} \label{sec:hybrid-group-rings}

We recall material on hybrid $p$-adic group rings from \cite[\S 2]{hybrid-ETNC}
and prove new results which provide many new examples.
We shall sometimes abuse notation by using the symbol $\oplus$ to denote the direct product of rings or orders.

\subsection{Background material}
Let $p$ be a prime and let $G$ be a finite group.
For a normal subgroup $N \unlhd G$, let $e_{N} = |N|^{-1}\sum_{\sigma \in N} \sigma$
be the associated central trace idempotent in the group algebra ${\mathbb{Q}}_{p}[G]$.
Then there is a ring isomorphism ${\mathbb{Z}}_{p}[G]e_{N} \simeq {\mathbb{Z}}_{p}[G/N]$.
We now specialise \cite[Definition 2.5]{hybrid-ETNC} to the case of $p$-adic group rings
(we shall not need the more general case of $N$-hybrid orders).

\begin{definition}
Let $N \unlhd G$. We say that the $p$-adic group ring ${\mathbb{Z}}_{p}[G]$ is \emph{$N$-hybrid}
if (i) $e_{N} \in {\mathbb{Z}}_{p}[G]$ (i.e. $p \nmid |N|$) and (ii) ${\mathbb{Z}}_{p}[G](1-e_{N})$ is a maximal
${\mathbb{Z}}_{p}$-order in ${\mathbb{Q}}_{p}[G](1-e_{N})$.
\end{definition}

\begin{remark}
The group ring ${\mathbb{Z}}_{p}[G]$ is itself maximal if and only if $p$ does not divide $|G|$
if and only if ${\mathbb{Z}}_{p}[G]$ is $G$-hybrid. Moreover, ${\mathbb{Z}}_{p}[G]$ is always $\{1\}$-hybrid.
\end{remark}

For every field $F$ of characteristic $0$ and every finite group $G$, we denote by ${\mathrm{Irr}}_{F}(G)$
the set of $F$-irreducible characters of $G$.
Let ${\mathbb{Q}}_{p}^{c}$ be an algebraic closure of ${\mathbb{Q}}_{p}$.
If $p$ is a prime and $x$ is a rational number, we let $v_{p}(x)$ denote its $p$-adic valuation.

\begin{prop}[{\cite[Proposition 2.7]{hybrid-ETNC}}]\label{prop:hybrid-criterion-groupring}
The group ring ${\mathbb{Z}}_{p}[G]$ is $N$-hybrid if and only if
for every $\chi \in {\mathrm{Irr}}_{{\mathbb{Q}}_{p}^{c}}(G)$ such that $N \not \leq \ker \chi$
we have $v_{p}(\chi(1))=v_{p}(|G|)$.
\end{prop}

\begin{remark}
In the language of modular representation theory, when $v_{p}(\chi(1))=v_{p}(|G|)$
we say that ``$\chi$ belongs to a $p$-block of defect zero''.
\end{remark}

\subsection{Frobenius groups}\label{subsec:frobenius-groups}
We recall the definition and some basic facts about Frobenius groups and then use them to
provide many examples of hybrid group rings.
For further results and examples, we refer the reader to \cite[\S 2.3]{hybrid-ETNC}.

\begin{definition}
A \emph{Frobenius group} is a finite group $G$ with a proper non-trivial subgroup $H$
such that $H \cap gHg^{-1}=\{ 1 \}$ for all $g \in G-H$,
in which case $H$ is called a \emph{Frobenius complement}.
\end{definition}

\begin{theorem}\label{thm:frob-kernel}
A Frobenius group $G$ contains a unique normal subgroup $N$, known as the Frobenius kernel, such that
$G$ is a semidirect product $N \rtimes H$. Moreover:
\begin{enumerate}
\item $|N|$ and $[G:N]=|H|$ are relatively prime.
\item The Frobenius kernel $N$ is nilpotent.
\item If $K \unlhd G $ then either $K \unlhd N$ or $N \unlhd K$.
\item If $\chi \in {\mathrm{Irr}}_{\mathbb{C}}(G)$ such that  $N \not \leq \ker \chi$ then $\chi= {\mathrm{ind}}_{N}^{G}(\psi)$ for some $1 \neq \psi \in {\mathrm{Irr}}_{\mathbb{C}}(N)$.
\end{enumerate}
\end{theorem}

\begin{proof}
For (i) and (iv) see \cite[\S 14A]{MR632548}.
For (ii) see \cite[10.5.6]{MR1357169} and for (iii) see  \cite[Exercise 7, \S 8.5]{MR1357169}.
\end{proof}

\begin{prop}[{\cite[Proposition 2.13]{hybrid-ETNC}}]\label{prop:frob-N-hybrid}
Let $G$ be a Frobenius group with Frobenius kernel $N$.
Then for every prime $p$ not dividing $|N|$, the group ring ${\mathbb{Z}}_{p}[G]$ is $N$-hybrid.
\end{prop}

\begin{proof}
We repeat the short argument for the convenience of the reader.
Let $\chi \in {\mathrm{Irr}}_{{\mathbb{Q}}_{p}^{c}}(G)$ such that $N \not \leq \ker \chi$.
Then by Theorem \ref{thm:frob-kernel} (iv) $\chi$ is induced from a nontrivial irreducible character
of $N$ and so $\chi(1)$ is divisible by $[G:N]$.
However, $|N|$ and $[G:N]$ are relatively prime by Theorem \ref{thm:frob-kernel} (i) and so
Proposition \ref{prop:hybrid-criterion-groupring} now gives the desired result.
\end{proof}

We now give some examples, the first two of which were also given in \cite[\S 2.3]{hybrid-ETNC}.

\begin{example}\label{ex:metacyclic}
Let $p<q$ be distinct primes and assume that $p \mid (q-1)$.
Then there is an embedding $C_{p} \hookrightarrow {\mathrm{Aut}}(C_{q})$ and so there is
a fixed-point-free action of $C_{p}$ on $C_{q}$.
Hence the corresponding semidirect product $G = C_{q} \rtimes C_{p}$ is a Frobenius group
(see \cite[Theorem 2.12]{hybrid-ETNC} or \cite[\S 4.6]{MR2599132}, for example), and so ${\mathbb{Z}}_{p}[G]$ is $N$-hybrid with $N = C_{q}$.
\end{example}

\begin{example}\label{ex:affine}
Let $q$ be a prime power and let ${\mathbb{F}}_{q}$ be the finite field with $q$ elements.
The group ${\mathrm{Aff}}(q)$ of affine transformations on ${\mathbb{F}}_{q}$ is the group of transformations
of the form $x \mapsto ax +b$ with $a \in {\mathbb{F}}_{q}^{\times}$ and $b \in {\mathbb{F}}_{q}$.
Let $G={\mathrm{Aff}}(q)$ and let $N=\{ x \mapsto x+b \mid b \in {\mathbb{F}}_{q} \}$.
Then $G$ is a Frobenius group with Frobenius kernel $N \simeq {\mathbb{F}}_{q}$ and is isomorphic to
the semidirect product ${\mathbb{F}}_{q} \rtimes {\mathbb{F}}_{q}^{\times}$ with the natural action.
Moreover, $G/N \simeq {\mathbb{F}}_{q}^{\times} \simeq C_{q-1}$ and $G$ has precisely one
non-linear irreducible complex character, which is rational-valued and of degree $q-1$.
Hence for every prime $p$ not dividing $q$, we have that ${\mathbb{Z}}_{p}[G]$ is $N$-hybrid
and is isomorphic to ${\mathbb{Z}}_{p}[C_{q-1}] \oplus M_{(q-1) \times (q-1)}({\mathbb{Z}}_{p})$.
Note that in particular ${\mathrm{Aff}}(3) \simeq S_{3}$ and ${\mathrm{Aff}}(4) \simeq A_{4}$.
Thus ${\mathbb{Z}}_{2}[S_{3}] \simeq {\mathbb{Z}}_{2}[C_{2}] \oplus M_{2 \times 2}({\mathbb{Z}}_{2})$
and ${\mathbb{Z}}_{3}[A_{4}] \simeq {\mathbb{Z}}_{3}[C_{3}] \oplus M_{3 \times 3}({\mathbb{Z}}_{3})$.
\end{example}

\begin{example}\label{ex:dicyclic-complement}
Let $p$ be an odd prime and let ${\mathrm{Dic}}_{p} := \langle a,b \mid a^{2p}=1, a^{p}=b^{2}, b^{-1}ab = a^{-1} \rangle$ be the dicyclic group of order $4p$.
We recall a construction given in \cite[Chapter 14]{MR1828640}.
For every positive integer $n$ we let $\zeta_{n}$ denote a primitive $n$th root of unity.
There is an embedding $\iota$ of ${\mathrm{Dic}}_{p}$ into the subring $S_{p} := {\mathbb{Z}}[\frac{1}{2p}, \zeta_{2p},j]$ of the real quaternions;
here, $j^{2} = -1$ and $j \zeta_{2p} = \zeta_{2p}^{-1} j$.
We put $R_{p} := {\mathbb{Z}}[\frac{1}{2p}, \zeta_{p} + \zeta_{p}^{-1}] \subseteq S_{p}$.
Let $t$ and $k_{i}$, $1\leq i \leq t$ be positive integers and let $\mathfrak{p}_{i}$, $1 \leq i \leq t$ be maximal ideals of $R_{p}$.
Then for each $i$ there is a $R_{p} / \mathfrak{p}_{i}^{k_{i}}$-algebra isomorphism
$S_{p} / \mathfrak{p}_{i}^{k_{i}} S_{p} \simeq M_{2 \times 2} (R_{p}/\mathfrak{p}_{i}^{k_{i}})$ which
induces a fixed-point-free action of ${\mathrm{Dic}}_{p}$ on $N(\mathfrak{p}_{i}^{k_{i}}) := (R_{p} / \mathfrak{p}_{i}^{k_{i}})^{2}$ via $\iota$.
Thus $G := N \rtimes {\mathrm{Dic}}_{p}$ with $N := \prod_{i=1}^{t} N(\mathfrak{p}_{i}^{k_{i}})$ is a Frobenius group with Frobenius complement ${\mathrm{Dic}}_{p}$.
In fact, every Frobenius group with Frobenius complement isomorphic to ${\mathrm{Dic}}_{p}$
is of this type (see \cite[Theorem 14.4]{MR1828640}).
In particular, ${\mathbb{Z}}_{p}[G]$ is $N$-hybrid.
\end{example}

\begin{example} \label{ex:non-abelian-kernel}
Let $p$ and $q$ be primes, $f$ and $n$ positive integers such that $q>n>1$ and $q \mid (p^{f}-1)$.
Let $N$ be the subgroup of ${\mathrm{GL}}_{n}({\mathbb{F}}_{p^{f}})$ comprising upper triangular matrices with all diagonal
entries equal to $1$.
There are pairwise distinct $b_{j} \in {\mathbb{F}}_{p^{f}}$, $1 \leq j \leq n$ such that $b_{j}^{q} = 1$.
Let $h$ be the diagonal matrix with entries $b_{1}, \dots, b_{n}$ and set $H:= \langle h \rangle$.
Then $G := N \rtimes H$ is a Frobenius group  of order $q p^{fn(n-1)/2}$ and so ${\mathbb{Z}}_{p}[G]$ is $N$-hybrid.
Moreover, the Frobenius kernel $N$ has nilpotency class $n-1$ (see \cite[Example 16.8b]{MR1645304})
and hence is complicated if $n$ is large.
\end{example}

Recall that a character of a finite group is called \emph{monomial} if it
is induced from a linear character of a subgroup.
A finite group is called monomial if each of its (complex) irreducible characters is monomial.
In particular, nilpotent groups are monomial (see \cite[Theorem 11.3]{MR632548}).
We state the following lemma for later use.

\begin{lemma}\label{lem:monomial-Frobenius}
Let $G \simeq N \rtimes H$ be a Frobenius group.
Then $G$ is monomial if and only if its Frobenius complement $H$ is monomial.
\end{lemma}

\begin{proof}
Suppose $H$ is monomial.
Let $\chi \in {\mathrm{Irr}}_{\mathbb{C}}(G)$.
If $N \leq \ker \chi$ then $\chi$ is inflated from some $\varphi \in {\mathrm{Irr}}_{\mathbb{C}}(G/N)$.
Otherwise $N \nleq \ker \chi$ and so $\chi$ is induced
from some $\psi \in {\mathrm{Irr}}_{\mathbb{C}}(N)$ by Theorem \ref{thm:frob-kernel} (iv).
The Frobenius complement $H \simeq G/N$ is monomial by assumption, and
the Frobenius kernel $N$ is nilpotent by Theorem \ref{thm:frob-kernel} (ii) and thus is monomial.
However, induction is transitive and inflation commutes with induction (as in \cite[Theorem 4.2 (3)]{MR1984740}, for example)
so that in both cases $\chi$ is induced from a linear character. Therefore $G$ is monomial.
The converse follows from the fact that any quotient of a monomial group is monomial
(this can easily be proved using that inflation commutes with induction; also see \cite[Chapter 2, \S 4]{MR655785}).
\end{proof}

\subsection{New $p$-adic hybrid group rings from old}
We now use character theory to show how new examples of hybrid $p$-adic group rings can be obtained from existing examples.

For a field $F$ of characteristic $0$, a finite group $G$ and (virtual) $F$-characters $\chi$ and $\psi$ of $G$,
we let $\langle \chi, \psi \rangle$ denote the usual inner product.

\begin{remark}
For any finite group $G$ with subgroup $H$, any prime $p$,
and any $\chi \in {\mathrm{Irr}}_{{\mathbb{Q}}_{p}^{c}}(G)$, we have $H \leq \ker\chi$ if and only if $\langle {\mathrm{res}}^{G}_{H} \chi, \psi \rangle = 0$
for every non-trivial $\psi \in {\mathrm{Irr}}_{{\mathbb{Q}}_{p}^{c}}(H)$. We shall use this easy observation several times (for different choices of
$G$, $H$ and $\chi$) in the proofs of Lemma \ref{lem:kernel-basechange} and Propositions \ref{prop:hybrid-basechange-down} and
\ref{prop:hybrid-basechange-up} below. 
\end{remark}

\begin{lemma}\label{lem:kernel-basechange}
Let $G$ be a finite group with subgroups $N \leq H \unlhd  G$.
Let $p$ be a prime.
Fix $\chi \in {\mathrm{Irr}}_{{\mathbb{Q}}_{p}^{c}}(G)$ and let $\eta \in {\mathrm{Irr}}_{{\mathbb{Q}}_{p}^{c}}(H)$ be an irreducible constituent of ${\mathrm{res}}^{G}_{H} \chi$.
Then:
\begin{enumerate}
\item If $N \leq \ker \chi$, then $N \leq \ker \eta$.
\item Assume in addition that $N \unlhd G$. Then $N \leq \ker \chi$ if and only if $N \leq \ker \eta$.
\end{enumerate}
\end{lemma}

\begin{proof}
As $H \unlhd G$, we have a natural right action of $G$ on ${\mathrm{Irr}}_{{\mathbb{Q}}_{p}^{c}}(H)$;
namely, for each $g \in G$, $h \in H$ we have $\eta^{g}(h) = \eta(g^{-1}hg)$.
Let $G_{\eta} = \{g \in G \mid \eta^{g} = \eta \}$ be the stabiliser of $\eta$ in $G$ and
let $R_{\eta}$ be a set of right coset representatives of $G_{\eta}$ in $G$.
Then by Clifford theory (see \cite[Proposition 11.4]{MR632548}) we have
${\mathrm{res}}^{G}_{H} \chi = e \sum_{g \in R_{\eta}} \eta^{g}$ for some positive integer $e$, and in particular
\begin{equation}\label{eqn:Clifford-theory}
\chi(1) = e [G:G_{\eta}] \eta(1).
\end{equation}
Let $\psi \in {\mathrm{Irr}}_{{\mathbb{Q}}_{p}^{c}}(N)$. Then we have
\begin{equation}\label{eqn:res-non-neg}
\langle {\mathrm{res}}^{G}_{N} \chi, \psi \rangle = \langle {\mathrm{res}}^{H}_{N} ({\mathrm{res}}^{G}_{H} \chi), \psi \rangle
= e \sum_{g \in R_{\eta}} \langle {\mathrm{res}}^{H}_{N} \eta^{g}, \psi \rangle \geq 0,
\end{equation}
since $\langle {\mathrm{res}}^{H}_{N} \eta^{g}, \psi \rangle \geq 0$ for every $g \in R_{\eta}$.

Suppose that $N \leq \ker \chi$.
Then $\langle {\mathrm{res}}^{G}_{N} \chi, \psi \rangle = 0$ for every non-trivial $\psi \in {\mathrm{Irr}}_{{\mathbb{Q}}_{p}^{c}}(N)$ and so by \eqref{eqn:res-non-neg} we have $\langle {\mathrm{res}}^{H}_{N} \eta, \psi \rangle = 0$. Hence $N \leq \ker \eta$, proving (i).

Now assume that $N \unlhd G$ and suppose conversely that $N \leq \ker \eta$.
Then $\langle {\mathrm{res}}^{H}_{N} \eta, \psi \rangle = 0$ for every non-trivial $\psi \in {\mathrm{Irr}}_{{\mathbb{Q}}_{p}^{c}}(N)$,
and for every $g \in R_{\eta}$ we have
\[
    \langle {\mathrm{res}}^{H}_{N} \eta^{g}, \psi \rangle = \langle {\mathrm{res}}^{H}_{N} \eta, \psi^{g^{-1}} \rangle = 0.
\]
Thus by \eqref{eqn:res-non-neg} we have $\langle {\mathrm{res}}^{G}_{N} \chi, \psi \rangle = 0$
for every non-trivial $\psi \in {\mathrm{Irr}}_{{\mathbb{Q}}_{p}^{c}}(N)$ and so  $N \leq \ker \chi$.
\end{proof}

The following proposition is a generalisation of \cite[Proposition 2.8 (iv)]{hybrid-ETNC}

\begin{prop}\label{prop:hybrid-basechange-down}
Let $G$ be a finite group with normal subgroups $N, H \unlhd  G$.
Let $K$ be a normal subgroup of $H$ such that $K \leq N$. Let $p$ be a prime.
If  ${\mathbb{Z}}_{p}[G]$ is $N$-hybrid then ${\mathbb{Z}}_{p}[H]$ is $K$-hybrid.
\end{prop}

\begin{proof}
Fix $\eta \in {\mathrm{Irr}}_{{\mathbb{Q}}_{p}^{c}}(H)$ such that $K \not\leq \ker \eta$.
By Proposition \ref{prop:hybrid-criterion-groupring} we have to show that $v_{p}(\eta(1)) = v_{p}(|H|)$.
Let $\chi \in {\mathrm{Irr}}_{{\mathbb{Q}}_{p}^{c}}(G)$ be any irreducible constituent of ${\mathrm{ind}}^{G}_{H}  \eta$.
Then by Frobenius reciprocity we have  $\langle\chi, {\mathrm{ind}}^{G}_{H}  \eta \rangle = \langle {\mathrm{res}}^{G}_{H} \chi, \eta \rangle \neq 0$.
By Lemma \ref{lem:kernel-basechange} (i) we have $K \not\leq \ker \chi$ and a fortiori $N \not\leq \ker \chi$.
Then again by Proposition \ref{prop:hybrid-criterion-groupring} we find that
$v_{p}(\chi(1)) = v_{p}(|G|)$ as ${\mathbb{Z}}_{p}[G]$ is $N$-hybrid by assumption.
This and equation \eqref{eqn:Clifford-theory} imply $v_{p}(e \eta(1)) = v_{p}(|G_{\eta}|) = v_{p}([G_{\eta}:H] \cdot |H|)$.
But $\eta(1)$ divides $|H|$ whereas $e$ divides $[G_{\eta}:H]$ by \cite[Theorem 21.3]{MR1645304}, so we must have
$v_{p}(e) = v_{p}([G_{\eta}:H])$ and $v_{p}(\eta(1)) = v_{p}(|H|)$, as desired.
\end{proof}

The following proposition is a generalisation of \cite[Lemma 2.9]{hybrid-ETNC}.

\begin{prop} \label{prop:hybrid-basechange-up}
Let $G$ be a finite group with normal subgroups $N \unlhd H \unlhd G$ such that $N \unlhd G$.
Let $p$ be a prime and assume that $p \nmid [G:H]$.
Then ${\mathbb{Z}}_{p}[G]$ is $N$-hybrid if and only if ${\mathbb{Z}}_{p}[H]$ is $N$-hybrid.
\end{prop}

\begin{proof}
If ${\mathbb{Z}}_{p}[G]$ is $N$-hybrid then ${\mathbb{Z}}_{p}[H]$ is $N$-hybrid by Proposition \ref{prop:hybrid-basechange-down} with $K=N$.
Assume conversely that $p \nmid [G:H]$ and that ${\mathbb{Z}}_{p}[H]$ is $N$-hybrid.
Let $\chi \in {\mathrm{Irr}}_{{\mathbb{Q}}_{p}^{c}}(G)$ such that $N \not\leq \ker \chi$ and let $\eta$ be an irreducible constituent of ${\mathrm{res}}^{G}_{H} \chi$.
Then $N \not\leq \ker \eta$ by Lemma \ref{lem:kernel-basechange} (ii) and $\eta(1)$ divides $\chi(1)$ by \eqref{eqn:Clifford-theory}.
However, we have $v_{p}(\eta(1)) = v_{p}(|H|) = v_{p}(|G|)$ by assumption, and so we must also have that $v_{p}(\chi(1)) = v_{p}(|G|)$.
Thus ${\mathbb{Z}}_{p}[G]$ is $N$-hybrid by Proposition \ref{prop:hybrid-criterion-groupring}.
\end{proof}

\begin{example}\label{ex:S4-A4-V4}
Let $p=3$, $G=S_{4}$, $H=A_{4}$ and $N = V_{4}$. Then the hypotheses of Proposition \ref{prop:hybrid-basechange-up} are satisfied.
Hence ${\mathbb{Z}}_{3}[S_{4}]$ is $V_{4}$-hybrid if and only if ${\mathbb{Z}}_{3}[A_{4}]$ is $V_{4}$-hybrid.
In fact, ${\mathbb{Z}}_{3}[A_{4}]$ is indeed $V_{4}$-hybrid since $A_{4}$ is a Frobenius group with Frobenius kernel $V_{4}$
(see Example \ref{ex:affine}) and so ${\mathbb{Z}}_{3}[S_{4}]$ is also $V_{4}$-hybrid.
However, $S_{4}$ is \emph{not} a Frobenius group (see \cite[Example 2.18]{hybrid-ETNC}).
Thus Proposition \ref{prop:hybrid-basechange-up} can be used to give examples which do not come directly from
Proposition \ref{prop:frob-N-hybrid}.
\end{example}

\begin{example} \label{ex:affine-Frobenius}
Let $q = \ell^{n}$ be a prime power and let $\phi: {\mathbb{F}}_{q} \rightarrow {\mathbb{F}}_{q}$, $x \mapsto x^{\ell}$ be the Frobenius automorphism.
Each $c \in {\mathbb{F}}_{q}{^{\times}}$ defines a map $m_{c}: {\mathbb{F}}_{q} \rightarrow {\mathbb{F}}_{q}$, $x \mapsto c \cdot x$.
We may consider $\phi$ and $m_{c}$ as elements of ${\mathrm{GL}}_{n}({\mathbb{F}}_{\ell})$. Then $\phi m_{c} \phi^{-1} = m_{c^{\ell}}$
and we may form the semidirect product ${\mathbb{F}}_{q}{^{\times}} \rtimes \langle \phi \rangle$ inside ${\mathrm{GL}}_{n}({\mathbb{F}}_{\ell})$.
Moreover, the action of ${\mathbb{F}}_{q}{^{\times}} \rtimes \langle \phi \rangle$ on ${\mathbb{F}}_{q}$ gives a semidirect product
$G := {\mathbb{F}}_{q} \rtimes ({\mathbb{F}}_{q}{^{\times}} \rtimes \langle \phi \rangle)$.
Then the group ${\mathrm{Aff}}(q) \simeq {\mathbb{F}}_{q} \rtimes {\mathbb{F}}_{q}^{\times}$ of affine transformations on ${\mathbb{F}}_{q}$
naturally identifies with a normal subgroup of $G$, and $N = {\mathbb{F}}_{q}$ is normal in both $G$ and ${\mathrm{Aff}}(q)$.
However, ${\mathbb{Z}}_{p}[{\mathrm{Aff}}(q)]$ is $N$-hybrid for every prime $p \neq  \ell$ by Example \ref{ex:affine}.
If we further suppose that $p$ does not divide $n = [G : {\mathrm{Aff}}(q)]$, then ${\mathbb{Z}}_{p}[G]$
is $N$-hybrid by Proposition \ref{prop:hybrid-basechange-up}.
Note that this recovers Example \ref{ex:S4-A4-V4} since $G \simeq S_{4}$ when $\ell=n=2$.
\end{example}

\begin{remark}\label{rmk:hybrid-Galois-basechange}
One of the main reasons for interest in Propositions \ref{prop:hybrid-basechange-down} and \ref{prop:hybrid-basechange-up}
comes from base change in Galois theory.
(In the following, the subgroup $K$ of Proposition \ref{prop:hybrid-basechange-down} will be denoted $N'$
and $K$ will instead denote a field. The symbols $G'$, $N'$, etc.\ do \emph{not} denote commutator subgroups.)
Assume that $L/K$ is a finite Galois extension with Galois group $G$.
Let $p$ be a prime and suppose that ${\mathbb{Z}}_{p}[G]$ is $N$-hybrid for some normal subgroup $N$ of $G$.
Put $F := L^{N}$.
Now let $K'/K$ be a finite abelian extension of $K$ and put $F' = FK'$ and $L' = LK'$.
Then $H := {\mathrm{Gal}}(L'/K')$ naturally identifies with a subgroup of $G$.
Similarly, $N' := {\mathrm{Gal}}(L'/F')$ is normal in $H$ and identifies with a subgroup of $N$.
The fixed field $L^{H}$ is a subfield of $K'$ and thus $L^{H}/K$ is a Galois extension, since $K'/K$ is abelian.
Hence $H$ is normal in $G$ and we conclude by Proposition \ref{prop:hybrid-basechange-down}
that ${\mathbb{Z}}_{p}[H]$ is $N'$-hybrid. Finally, let $G' := {\mathrm{Gal}}(L'/K)$; if $p \nmid [K':K]$ then
 ${\mathbb{Z}}_{p}[G']$ is also $N'$-hybrid by Proposition \ref{prop:hybrid-basechange-up}.
The situation is illustrated by the following field diagram.
\[
\xymatrix@1@!0@W=14pt@H=14pt{
& & L'  \ar@{-}[dll] \ar@{-}[dd]_{N'} \ar@/^1pc/@{-}[dddd]^{H} \ar@{-}@/_5pc/[dddddll] _{G'} \\
L \ar@{-}[dd]^{N}  \ar@/_1pc/@{-}[dddd]_{G} & &   \\
& & F' \ar@{-}[dll] \ar@{-}[dd]  \\
F \ar@{-}[dd] & & \\
& & K' \ar@{-}[dll] \\
K & &
}
\]
This will be applied in the proof of Corollary \ref{cor:pth-root-base-change}, which itself will be applied in the case that
$K'/K$ is a subextension of $K(\zeta_{p})/K$, where $\zeta_{p}$ is a primitive $p$th root of unity.
\end{remark}

\begin{remark}\label{rmk:applications-to-ETNC}
Burns and Flach \cite{MR1884523} formulated
the equivariant Tamagawa number conjecture (ETNC)
for any motive over ${\mathbb{Q}}$ with the action of a semisimple ${\mathbb{Q}}$-algebra, 
describing the leading term at $s=0$ of an equivariant motivic $L$-function in terms of certain 
cohomological Euler characteristics. 
The present authors introduced hybrid $p$-adic groups rings in \cite{hybrid-ETNC} and used them to prove many new cases of the 
$p$-part of the ETNC for Tate motives; 
the same methods can also be applied to several related conjectures.
Thus the new results on $p$-adic group rings given here combined with the results of \cite{hybrid-ETNC}
give unconditional proofs of the $p$-part of the ETNC for Tate motives and related conjectures in many new cases.
\end{remark}

\section{Hybrid Iwasawa algebras}\label{sec:hybrid-Iwasawa-alg}

\subsection{Iwasawa algebras of one-dimensional $p$-adic Lie groups}\label{subsec:Iwasawa-algebras}
Let $p$ be an odd prime and let $\mathcal{G}$ be a profinite group containing a finite normal subgroup $H$ such that $\mathcal{G}/H \simeq \Gamma$ where $\Gamma$ is a pro-$p$-group isomorphic to ${\mathbb{Z}}_{p}$;
thus $\mathcal{G}$ can be written as a semidirect product $H \rtimes \Gamma$ and is a one-dimensional $p$-adic Lie group.
The Iwasawa algebra of $\mathcal{G}$ is
\[
\Lambda(\mathcal{G}) := {\mathbb{Z}}_{p}[[\mathcal{G}]] = \varprojlim {\mathbb{Z}}_{p}[\mathcal{G}/\mathcal{N}],
\]
where the inverse limit is taken over all open normal subgroups $\mathcal{N}$ of $\mathcal{G}$.
If $F$ is a finite field extension of ${\mathbb{Q}}_{p}$  with ring of integers $\mathcal{O}=\mathcal{O}_{F}$,
we put $\Lambda^{\mathcal{O}}(\mathcal{G}) := \mathcal{O} \otimes_{{\mathbb{Z}}_{p}} \Lambda(\mathcal{G}) = \mathcal{O}[[\mathcal{G}]]$.
We fix a topological generator $\gamma$ of $\Gamma$.
Since any homomorphism $\Gamma \rightarrow {\mathrm{Aut}}(H)$ must have open kernel, we may choose a natural number $n$ such that $\gamma^{p^n}$ is central in $\mathcal{G}$; we fix such an $n$.
 As $\Gamma_{0} := \Gamma^{p^n} \simeq {\mathbb{Z}}_{p}$, there is a ring isomorphism
$R:=\mathcal{O}[[\Gamma_{0}]] \simeq \mathcal{O}[[T]]$ induced by $\gamma^{p^n} \mapsto 1+T$
where $\mathcal{O}[[T]]$ denotes the power series ring in one variable over $\mathcal{O}$.
If we view $\Lambda^{\mathcal{O}}(\mathcal{G})$ as an $R$-module (or indeed as a left $R[H]$-module), there is a decomposition
\begin{equation}\label{eq:Lambda-R-decomp}
\Lambda^{\mathcal{O}}(\mathcal{G}) = \bigoplus_{i=0}^{p^n-1} R[H] \gamma^{i}.
\end{equation}
Hence $\Lambda^{\mathcal{O}}(\mathcal{G})$ is finitely generated as an $R$-module and is an $R$-order in the separable $E:=Quot(R)$-algebra
$\mathcal{Q}^{F} (\mathcal{G})$, the total ring of fractions of $\Lambda^{\mathcal{O}}(\mathcal{G})$, obtained
from $\Lambda^{\mathcal{O}}(\mathcal{G})$ by adjoining inverses of all central regular elements.
Note that $\mathcal{Q}^{F} (\mathcal{G}) =  E \otimes_{R} \Lambda^{\mathcal{O}}(\mathcal{G})$ and that by
\cite[Lemma 1]{MR2114937} we have $\mathcal{Q}^{F} (\mathcal{G}) = F \otimes_{{\mathbb{Q}}_{p}} \mathcal{Q}(\mathcal{G})$,
where $\mathcal{Q}(\mathcal{G}) := \mathcal{Q}^{{\mathbb{Q}}_{p}}(\mathcal{G})$.

\subsection{Characters and central primitive idempotents} \label{subsec:idempotents}
For any field $K$ of characteristic $0$ let ${\mathrm{Irr}}_{K}(\mathcal{G})$ the set of $k$-irreducible characters of $\mathcal{G}$ with open kernel.
Fix a character $\chi \in {\mathrm{Irr}}_{{\mathbb{Q}}_{p}^{c}}(\mathcal{G})$ and let $\eta$ be an irreducible constituent of
${\mathrm{res}}^{\mathcal{G}}_{H} \chi$.
Then $\mathcal{G}$ acts on $\eta$ as $\eta^{g}(h) = \eta(g^{-1}hg)$
for $g \in \mathcal{G}$, $h \in H$, and following \cite[\S 2]{MR2114937} we set
\[
St(\eta) := \{g \in \mathcal{G}: \eta^g = \eta \}, \quad e(\eta) := \frac{\eta(1)}{|H|} \sum_{h \in H} \eta(h^{-1}) h,
\quad e_{\chi} := \sum_{\eta \mid {\mathrm{res}}^{\mathcal{G}}_{H} \chi} e(\eta).
\]
By \cite[Corollary to Proposition 6]{MR2114937} $e_{\chi}$ is a primitive central idempotent of
$\mathcal{Q}^{c}(\mathcal{G}) := {\mathbb{Q}}_{p}^{c} \otimes_{{\mathbb{Q}}_{p}} \mathcal{Q}(\mathcal{G})$.
In fact, every primitive central idempotent of $\mathcal{Q}^{c}(\mathcal{G})$ is of this form
and $e_{\chi} = e_{\chi'}$ if and only if $\chi = \chi' \otimes \rho$ for some character $\rho$ of $\mathcal{G}$ of type $W$
(i.e.~${\mathrm{res}}^{\mathcal{G}}_{H} \rho = 1$).
The irreducible constituents of ${\mathrm{res}}^{\mathcal{G}}_{H} \chi$ are precisely the conjugates of $\eta$
under the action of $\mathcal{G}$, each occurring with the same multiplicity $z_{\chi}$ by \cite[Proposition 11.4]{MR632548}. By \cite[Lemma 4]{MR2114937} we have $z_{\chi}=1$ and thus we also have equalities
\begin{equation}\label{eq:idem-sum}
{\mathrm{res}}^{\mathcal{G}}_{H} \chi = \sum_{i=0}^{w_{\chi}-1} \eta^{\gamma^{i}},
\quad
e_{\chi} = \sum_{i=0}^{w_{\chi}-1} e(\eta^{\gamma^{i}}) = \frac{\chi(1)}{|H|w_{\chi}}\sum_{h \in H} \chi(h^{-1})h,
\end{equation}
where $w_{\chi} := [\mathcal{G} : St(\eta)]$.
Note that  $\chi(1) = w_{\chi} \eta(1)$ and that
$w_{\chi}$ is a power of $p$ since $H$ is a subgroup of $St(\eta)$.

\subsection{Working over sufficiently large $p$-adic fields}\label{subsec:sufficiently-large}
We now specialise to the case where $F/{\mathbb{Q}}_{p}$ is a finite extension over which both
characters $\chi$ and $\eta$ have realisations.
Let $V_{\chi}$ denote a realisation of $\chi$ over $F$.	
By \cite[Proposition 5]{MR2114937}, there exists a unique element
$\gamma_{\chi} \in \zeta(\mathcal{Q}^{c}(\mathcal{G})e_{\chi})$ such that $\gamma_{\chi}$
acts trivially on $V_{\chi}$ and $\gamma_{\chi}= gc$ where $g \in \mathcal{G}$ with $(g \bmod H) = \gamma^{w_{\chi}}$
and $c \in ({\mathbb{Q}}_{p}^{c}[H]e_{\chi})^{\times}$. Moreover, $\gamma_{\chi}= gc=cg$.

\begin{lemma}\label{lem:c-in-F[H]}
In fact  $c \in (F[H]e_{\chi})^{\times}$ and so
$\gamma_{\chi} \in \zeta(\mathcal{Q}^{F}(\mathcal{G})e_{\chi})$.
\end{lemma}

\begin{proof}
We recall the definition of $\gamma_{\chi} := gc$ given in the proof of \cite[Proposition 5]{MR2114937}, the only difference
being that there $V_{\chi}$ is defined over ${\mathbb{Q}}_{p}^{c}$ rather than $F$.
Choose $g \in \mathcal{G}$ such that $(g \bmod H) = \gamma^{w_{\chi}}$.
Then $g \in St(\eta^{\gamma^{i}})$ for each $i$,
and it acts on $V_{\chi} = \oplus_{i=0}^{w_{\chi}-1} e(\eta^{\gamma^{i}}) V_{\chi}$ componentwise.
Since, by \cite[Lemma 4]{MR2114937}, each $e(\eta^{\gamma^{i}}) V_{\chi}$ is $H$-irreducible,
\[
g^{-1} \mid_{e(\eta^{\gamma^{i}})V_{\chi}} =: c(\eta^{\gamma^{i}}) \in F[H]e(\eta^{\gamma^{i}}) \simeq {\mathrm{End}}_{F}(e(\eta^{\gamma^{i}})V_{\chi})
\simeq M_{\eta(1) \times \eta(1)}(F).
\]
Now we set $c:=\sum_{i=0}^{w_{\chi}-1} c(\eta^{\gamma^{i}})$, and we see that $c \in (F[H]e_{\chi})^{\times}$.
\end{proof}

By \cite[Proposition 5]{MR2114937}, the element $\gamma_{\chi}$
generates a procyclic $p$-subgroup $\Gamma_{\chi}$ of $(\mathcal{Q}^{F}(\mathcal{G})e_{\chi})^{\times}$.
Let $\Lambda^{\mathcal{O}}(\Gamma_{\chi})$ be the integral domain $\mathcal{O}[[\Gamma_{\chi}]]$
with field of fractions $\mathcal{Q}^{F}(\Gamma_{\chi})$.

\begin{lemma}\label{lem:unique-max-order-in-centre}
$\Lambda^{\mathcal{O}}(\Gamma_{\chi})$ is the unique maximal $R$-order in $\zeta(\mathcal{Q}^{F}(\mathcal{G})e_{\chi})\simeq \mathcal{Q}^{F}(\Gamma_{\chi})$.
\end{lemma}

\begin{proof}
By \cite[Proposition 6]{MR2114937}, $\mathcal{Q}^{F}(\Gamma_{\chi})$ is contained in $\mathcal{Q}^{F}(\mathcal{G})e_{\chi}$,
and $\gamma_{\chi} \in \mathcal{Q}^{F}(\Gamma_{\chi})$ induces an isomorphism
$\mathcal{Q}^{F}(\Gamma_{\chi}) \stackrel{\simeq}{\longrightarrow} \zeta(\mathcal{Q}^{F}(\mathcal{G})e_{\chi})$.
Therefore $\Lambda^{\mathcal{O}}(\Gamma_{\chi})$ is an $R$-order in $\zeta(\mathcal{Q}^{F}(\mathcal{G})e_{\chi})$.
Moreover, $\Lambda^{\mathcal{O}}(\Gamma_{\chi})$ is a maximal $R$-order since there is an isomorphism of commutative rings
 $\Lambda^{\mathcal{O}}(\Gamma_{\chi}) = \mathcal{O}[[\Gamma_{\chi}]] \simeq \mathcal{O}[[T]]$ and $\mathcal{O}[[T]]$ is integrally closed. Uniqueness follows from commutativity of $\mathcal{Q}^{F}(\Gamma_{\chi})$.
\end{proof}

\subsection{Maximal order $e_{\chi}$-components of Iwasawa algebras}
We give criteria for `$e_{\chi}$-components' of Iwasawa algebras of one-dimensional $p$-adic Lie groups to be maximal orders
in the case that $F/{\mathbb{Q}}_{p}$ is a sufficiently large finite extension. Moreover,
we give an explicit description of such components.

\begin{prop}\label{prop:chi-comp-max-order}
Let  $\chi \in {\mathrm{Irr}}_{{\mathbb{Q}}_{p}^{c}}(\mathcal{G})$ and let $\eta$ be an irreducible constituent of
${\mathrm{res}}^{\mathcal{G}}_{H} \chi$.
Let $F/{\mathbb{Q}}_{p}$ be a finite extension over which both characters $\chi$ and $\eta$ have realisations
and let $\mathcal{O}=\mathcal{O}_{F}$ be its ring of integers.
Suppose that $v_{p}(\eta(1))=v_{p}(|H|)$.
Then $e_{\chi} \in \Lambda^{\mathcal{O}}(\mathcal{G})$,
$\zeta(\Lambda^{\mathcal{O}}(\mathcal{G})e_{\chi}) = \Lambda^{\mathcal{O}}(\Gamma_{\chi})$
and there is an isomorphism of $R:=\mathcal{O}[[\Gamma_{0}]]$-orders
\[
\Lambda^{\mathcal{O}}(\mathcal{G})e_{\chi} \simeq M_{\chi(1) \times \chi(1)}(\Lambda^{\mathcal{O}}(\Gamma_{\chi})).
\]
Moreover, these are maximal $R$-orders and as rings are isomorphic to $M_{\chi(1) \times \chi(1)}(\mathcal{O}[[T]])$.
\end{prop}

\begin{proof}
Suppose that $v_{p}(\eta(1))=v_{p}(|H|)$. Then $v_{p}(\eta^{\gamma^{i}}(1)) =v_{p}(|H|)$ for each $i$. This has two consequences.
First, $e(\eta^{\gamma^{i}}) \in \mathcal{O}[H] \subseteq \Lambda^{\mathcal{O}}(\mathcal{G})$ for each $i$
and so the description of $e_{\chi}$ in \eqref{eq:idem-sum} shows that $e_{\chi} \in \Lambda^{\mathcal{O}}(\mathcal{G})$.
Second, since $\eta$ can be realised over $F$,
a standard result on ``$p$-blocks of defect zero'' (see \cite[Proposition 46 (b)]{MR0450380}, for example) shows that for each $i$ we have
an isomorphism of $\mathcal{O}$-orders
\begin{equation}\label{eq:matrix-order-key-point}
\mathcal{O}[H]e(\eta^{\gamma^{i}}) \simeq M_{\eta(1) \times \eta(1)}(\mathcal{O}).
\end{equation}

Let $M_{\chi}$ be an $\mathcal{O}$-lattice on $V_{\chi}$ that is stable under the action of $\mathcal{G}$.
In particular, $M_{\chi}$ is an $\mathcal{O}[H]$-module and since $e(\eta^{\gamma^{i}}) \in \mathcal{O}[H]$ for each $i$,
we have $M_{\chi} = \oplus_{i=0}^{w_{\chi}-1}e(\eta^{\gamma^{i}})M_{\chi}$.
Now recall the proof of Lemma \ref{lem:c-in-F[H]}.
The element $g^{-1}$ acts on $e(\eta^{\gamma^{i}})M_{\chi}$ for each $i$ and so we see that in fact
\[
c(\eta^{\gamma^{i}}) =
g^{-1} \mid_{e(\eta^{\gamma^{i}})M_{\chi}} \in \mathcal{O}[H]e(\eta^{\gamma^{i}}) \simeq {\mathrm{End}}_{\mathcal{O}}(e(\eta^{\gamma^{i}})M_{\chi})
\]
where the isomorphism follows from \eqref{eq:matrix-order-key-point}.
Hence $c \in \mathcal{O}[H]e_{\chi} \subseteq \Lambda^{\mathcal{O}}(\mathcal{G})e_{\chi}$ and
since $ge_{\chi} \in  \Lambda^{\mathcal{O}}(\mathcal{G})e_{\chi}$ we have
$\gamma_{\chi} := gc\in \zeta(\Lambda^{\mathcal{O}}(\mathcal{G})e_{\chi})$.
Thus $\Lambda^{\mathcal{O}}(\Gamma_{\chi}) \subseteq \zeta(\Lambda^{\mathcal{O}}(\mathcal{G})e_{\chi})$.
However, $\Lambda^{\mathcal{O}}(\Gamma_{\chi})$ is maximal by Lemma \ref{lem:unique-max-order-in-centre}
and therefore $\zeta(\Lambda^{\mathcal{O}}(\mathcal{G})e_{\chi}) = \Lambda^{\mathcal{O}}(\Gamma_{\chi})$.

Extending scalars in \eqref{eq:matrix-order-key-point} gives an isomorphism of $R$-orders $R[H]e(\eta) \simeq M_{\eta(1) \times \eta(1)}(R)$.
Let $e(\eta) = f_{1} + \cdots + f_{\eta(1)}$ be a decomposition of $e(\eta)$
into a sum of orthogonal indecomposable idempotents of $R[H]e(\eta)$.
Observe that $f_{k,i}:=\gamma^{-i}f_{k}\gamma^{i}$ is also an indecomposable idempotent
for every $0 \leq i < p^{n}$ and $1 \leq k \leq \eta(1)$.
Moreover, each $f_{k,i}$ belongs to $R[H]e(\eta^{\gamma^{i}})$ since
$\gamma^{-i}e(\eta)\gamma^{i}=e(\eta^{\gamma^{i}})$ and $H$ is normal in $\mathcal{G}$.
Thus $e(\eta^{\gamma^{i}}) = f_{1,i} + \cdots + f_{\eta(1),i}$ is a decomposition of $e(\eta^{\gamma^{i}})$
into a sum of orthogonal indecomposable idempotents of $R[H]e(\eta^{\gamma^{i}})$.
Note that $e(\eta) = e(\eta^{\gamma^{i}})$ if and only if  $w_{\chi}$ divides $i$.
Hence $R[H]e_{\chi} = \oplus_{i=0}^{w_{\chi}-1} R[H] e(\eta^{\gamma^{i}})$ is an $R$-suborder of
$\Lambda^{\mathcal{O}}(\mathcal{G})e_{\chi}$ and $e_{\chi} = \sum_{k=1}^{\eta(1)} \sum_{i=0}^{w_{\chi}-1} f_{k,i}$
is a decomposition of $e_{\chi}$ into orthogonal idempotents.
By considering the appropriate `elementary matrices' in $R[H]e(\eta) \simeq  M_{\eta(1) \times \eta(1)}(R)$
together with powers of $\gamma$, it is straightforward to see that there is a subset $S$ of
$(\Lambda^{\mathcal{O}}(\mathcal{G})e_{\chi})^{\times}$
such that each element of $I:=\{ f_{k,i} \}_{k=1,i=0}^{k=\eta(1), i=w_{\chi}-1}$ is a conjugate of $f_{1}=f_{1,0}$
by some element of $S$. Furthermore, $|I|=\eta(1)w_{\chi}=\chi(1)$.
Therefore by \cite[\S 46, Exercise 2]{MR892316} we have a ring isomorphism
\begin{equation}\label{eq:is-matrix-ring-chi1}
\Lambda^{\mathcal{O}}(\mathcal{G})e_{\chi} \simeq M_{\chi(1) \times \chi(1)}(f_{1}\Lambda^{\mathcal{O}}(\mathcal{G})e_{\chi}f_{1}).
\end{equation}

As shown in the proof of \cite[Proposition 6 (2)]{MR2114937},
the dimension of $\mathcal{Q}^{F}(\mathcal{G})e_{\chi}$ over
$\zeta(\mathcal{Q}^{F}(\mathcal{G})e_{\chi}) \simeq \mathcal{Q}^{F}(\Gamma_{\chi})$ is $\chi(1)^{2}$.
Since $e_{\chi}$ is primitive we have $\mathcal{Q}^{F}(\mathcal{G})e_{\chi} \simeq M_{m \times m}(D)$ for some $m$ and some skewfield $D$
with $\zeta(D)=\mathcal{Q}^{F}(\Gamma_{\chi})$.
Moreover, $\chi(1)=ms$ where $[D: \mathcal{Q}^{F}(\Gamma_{\chi})]=s^{2}$.
However, \eqref{eq:is-matrix-ring-chi1} shows that $m \geq \chi(1)$ and so in fact $\chi(1)=m$
and
\begin{equation}\label{eq:is-matrix-algebra-chi1}
\mathcal{Q}^{F}(\mathcal{G})e_{\chi} \simeq M_{\chi(1) \times \chi(1)}(\zeta(\mathcal{Q}^{F}(\mathcal{G})e_{\chi})) \simeq M_{\chi(1) \times \chi(1)}(\mathcal{Q}^{F}(\Gamma_{\chi})).
\end{equation}
Combining \eqref{eq:is-matrix-ring-chi1}, \eqref{eq:is-matrix-algebra-chi1} and the fact that
$\zeta(\Lambda^{\mathcal{O}}(\mathcal{G})e_{\chi}) = \Lambda^{\mathcal{O}}(\Gamma_{\chi})$
therefore shows that there are $R$-order isomorphisms
\[
\Lambda^{\mathcal{O}}(\mathcal{G})e_{\chi} \simeq M_{\chi(1) \times \chi(1)}(\zeta(\Lambda^{\mathcal{O}}(\mathcal{G})e_{\chi}))
\simeq M_{\chi(1) \times \chi(1)}(\Lambda^{\mathcal{O}}(\Gamma_{\chi})).
\]

Lemma \ref{lem:unique-max-order-in-centre} and \cite[Theorem 8.7]{MR1972204} show that
$M_{\chi(1) \times \chi(1)}(\Lambda^{\mathcal{O}}(\Gamma_{\chi}))$ is a maximal $R$-order.
The last claim follows from the ring isomorphism $\Lambda^{\mathcal{O}}(\Gamma_{\chi}) = \mathcal{O}[[\Gamma_{\chi}]] \simeq \mathcal{O}[[T]]$.
\end{proof}

\subsection{Central idempotents and Galois actions}\label{subset:idem-gal-actions}
Fix a character $\chi \in {\mathrm{Irr}}_{{\mathbb{Q}}_{p}^{c}}(\mathcal{G})$ and let $\eta$ be an irreducible constituent of
${\mathrm{res}}^{\mathcal{G}}_{H} \chi$.
We define two fields
\[
K_{\chi} := {\mathbb{Q}}_{p}(\chi(h) \mid h \in H) \subseteq {\mathbb{Q}}_{p}(\eta) := {\mathbb{Q}}_{p}(\eta(h) \mid h \in H)
\]
and remark that the containment is not an equality in general.
Note that $K_{\chi} = K_{\chi \otimes \rho}$ whenever $\rho$ is of type $W$ (i.e.~${\mathrm{res}}^{\mathcal{G}}_{H} \rho = 1$).
Since $H$ is normal in $\mathcal{G}$, we have ${\mathbb{Q}}_{p}(\eta^{g}) = {\mathbb{Q}}_{p}(\eta)$ for every $g \in \mathcal{G}$ and thus
${\mathbb{Q}}_{p}(\eta)$ does not depend on the particular choice $\eta$ of irreducible constituent of ${\mathrm{res}}^{\mathcal{G}}_{H}\chi$.
We let $\sigma \in {\mathrm{Gal}}({\mathbb{Q}}_{p}^{c}/{\mathbb{Q}}_{p})$ act on $\chi$ by ${}^{\sigma}\chi(g) = \sigma(\chi(g))$ for all $g \in \mathcal{G}$
and similarly on characters of $H$. Note that the actions on ${\mathrm{res}}^{\mathcal{G}}_{H}\chi$ and $\eta$ factor through
${\mathrm{Gal}}(K_{\chi}/{\mathbb{Q}}_{p})$ and ${\mathrm{Gal}}({\mathbb{Q}}_{p}(\eta)/{\mathbb{Q}}_{p})$, respectively.
Moreover, for $\sigma \in {\mathrm{Gal}}({\mathbb{Q}}_{p}^{c}/{\mathbb{Q}}_{p})$ we have $\sigma(e_{\chi})=e_{({}^{\sigma}\chi)}$,
and $\sigma(e_{\chi})=e_{\chi}$ if and only if ${\mathrm{res}}^{\mathcal{G}}_{H}\chi = {}^{\sigma}({\mathrm{res}}^{\mathcal{G}}_{H}\chi)$.
Hence the action of ${\mathrm{Gal}}({\mathbb{Q}}_{p}^{c}/{\mathbb{Q}}_{p})$ on $e_{\chi}$ factors through ${\mathrm{Gal}}(K_{\chi}/{\mathbb{Q}}_{p})$ and we define
\[
\varepsilon_{\chi} := \sum_{\sigma \in {\mathrm{Gal}}(K_{\chi} / {\mathbb{Q}}_{p})} \sigma(e_{\chi}).
\]
We remark that $\varepsilon_{\chi}$ is a primitive central idempotent of $\mathcal{Q}(\mathcal{G})$.
Finally, we define an equivalence relation on ${\mathrm{Irr}}_{{\mathbb{Q}}_{p}^{c}}(\mathcal{G})$ as follows:
$\chi \sim \chi'$ if and only if there exists $\sigma \in {\mathrm{Gal}}(K_{\chi}/{\mathbb{Q}}_{p})$ such that $\sigma(e_{\chi})=e_{\chi'}$.
Note that the number of equivalence classes is finite and
that we have the decomposition $1 = \sum_{\chi/\sim} \varepsilon_{\chi}$ in $\mathcal{Q}(\mathcal{G})$.

\subsection{Maximal order $\varepsilon_{\chi}$-components of Iwasawa algebras}
We give criteria for `$\varepsilon_{\chi}$-components' of Iwasawa algebras of one-dimensional $p$-adic Lie groups to be maximal orders in the case that $F={\mathbb{Q}}_{p}$;
moreover, we give a somewhat explicit description of such components.

\begin{prop}\label{prop:max-part-of-hybrid-matrix-over-comm-ring}
Let $\chi \in {\mathrm{Irr}}_{{\mathbb{Q}}_{p}^{c}}(\mathcal{G})$ and let $\eta$ be an irreducible constituent of
${\mathrm{res}}^{\mathcal{G}}_{H} \chi$.
If $v_{p}(\eta(1))=v_{p}(|H|)$ then $\varepsilon_{\chi} \in \Lambda(\mathcal{G})$ and $\varepsilon_{\chi}\Lambda(\mathcal{G})$
is a maximal ${\mathbb{Z}}_{p}[[\Gamma_{0}]]$-order.
Moreover, $\varepsilon_{\chi}\Lambda(\mathcal{G}) \simeq M_{\chi(1) \times \chi(1)}(S_{\chi})$
for some local integrally closed domain $S_{\chi}$.
\end{prop}

\begin{remark}
The condition that $v_{p}(\eta(1))=v_{p}(|H|)$ is independent of choice of irreducible consistent $\eta$ of ${\mathrm{res}}^{\mathcal{G}}_{H} \chi$
and thus only depends on $\chi$. Moreover, if the condition holds for $\chi \in {\mathrm{Irr}}_{{\mathbb{Q}}_{p}^{c}}(\mathcal{G})$, then
it also holds for ${}^{\sigma}\chi$ for all $\sigma \in {\mathrm{Gal}}({\mathbb{Q}}_{p}^{c}/{\mathbb{Q}}_{p})$ (recall \S \ref{subset:idem-gal-actions}).
\end{remark}

\begin{proof}[Proof of Proposition \ref{prop:max-part-of-hybrid-matrix-over-comm-ring}]
Assume the notation and hypotheses of Proposition \ref{prop:chi-comp-max-order}.
In particular, Proposition \ref{prop:chi-comp-max-order} shows that $e_{\chi} \in \Lambda^{\mathcal{O}}(\mathcal{G})$.
Hence $\varepsilon_{\chi} \in \mathcal{Q}(\mathcal{G}) \cap \Lambda^{\mathcal{O}}(\mathcal{G}) = \Lambda(\mathcal{G})$.
Now
\[
 \bigoplus_{\sigma \in {\mathrm{Gal}}(K_{\chi}/{\mathbb{Q}}_{p})} \Lambda^{\mathcal{O}}(\mathcal{G})\sigma(e_{\chi})
= \Lambda^{\mathcal{O}}(\mathcal{G}) \varepsilon_{\chi}
= \mathcal{O} \otimes_{{\mathbb{Z}}_{p}} \Lambda(\mathcal{G})  \varepsilon_{\chi}
=\mathcal{O}[[\Gamma_{0}]] \otimes_{{\mathbb{Z}}_{p}[[\Gamma_{0}]]} \Lambda(\mathcal{G}) \varepsilon_{\chi},
\]
which is a maximal $\mathcal{O}[[\Gamma_{0}]]$-order by Proposition \ref{prop:chi-comp-max-order}.
Hence we have that $\Lambda(\mathcal{G}) \varepsilon_{\chi}$ is a maximal ${\mathbb{Z}}_{p}[[\Gamma_{0}]]$-order.

Set $\Lambda = \Lambda(\mathcal{G})\varepsilon_{\chi}$ and $S=\zeta(\Lambda)$.
Note that $S$ is contained in $\zeta(\mathcal{Q}(\mathcal{G})\varepsilon_{\chi})$, which is a field
since $\varepsilon_{\chi}$ is a primitive central idempotent of $\mathcal{Q}(\mathcal{G})$.
Thus $S$ is an integral domain.
Moreover, $S$ is semiperfect by \cite[Example (23.3)]{MR1838439} and thus is a finite direct product of local rings by
\cite[Theorem (23.11)]{MR1838439}; therefore $S$ is in fact a local integral domain.
Furthermore,
$S$ must be integrally closed since $\Lambda(\mathcal{G})\varepsilon_{\chi}$ is maximal.

Now set $\Lambda' = \Lambda^{\mathcal{O}}(\mathcal{G})\varepsilon_{\chi} = \mathcal{O} \otimes_{{\mathbb{Z}}_{p}} \Lambda$.
It is clear that $\mathcal{O} \otimes_{{\mathbb{Z}}_{p}} S \subseteq S':=\zeta(\Lambda')$.
A standard argument gives the reverse containment; here we give a slightly modified version of the proof of
\cite[Theorem 7.6]{MR1972204}.
Let $b_{1}, \ldots, b_{m}$ be a ${\mathbb{Z}}_{p}$-basis of $\mathcal{O}$. Let $x \in S'$.
Then we can write $x= \sum_{i} b_{i} \otimes \lambda_{i}$ for some $\lambda_{i} \in \Lambda$
and in particular, $x$ commutes with $1 \otimes \lambda$ for all $\lambda \in \Lambda$.
Hence $\sum_{i} b_{i} \otimes (\lambda \lambda_{i} - \lambda_{i} \lambda) = 0$
and so $\lambda \lambda_{i} = \lambda_{i} \lambda$ for all $\lambda \in \Lambda$,
that is, $\lambda_{i} \in \zeta(\Lambda)=S$.
Therefore $S' = \mathcal{O} \otimes_{{\mathbb{Z}}_{p}} S$. This gives
\begin{equation}\label{eqn:extend-by-central-scalars}
\Lambda'  = \mathcal{O} \otimes_{{\mathbb{Z}}_{p}} \Lambda \simeq (\mathcal{O} \otimes_{{\mathbb{Z}}_{p}} S) \otimes_{S} \Lambda \simeq S' \otimes_{S} \Lambda.
\end{equation}

Let $S'' :=e_{\chi}S'=\zeta(\Lambda' e_{\chi})$.
By Proposition \ref{prop:chi-comp-max-order} we have $S'' \simeq \mathcal{O}[[\Gamma_{\chi}]]$,
which is a local integrally closed domain.
Let $k$ and $k''$ be the residue fields of $S$ and $S''$, respectively.
Then $k''/k$ is a finite extension of finite fields of characteristic $p$.
By Proposition \ref{prop:chi-comp-max-order} and \eqref{eqn:extend-by-central-scalars} we have
\begin{equation}\label{eqn:defn-of-Gamma}
\Lambda'' := S'' \otimes_{S} \Lambda  \simeq e_{\chi}(S' \otimes_{S} \Lambda)
\simeq e_{\chi}\Lambda' \simeq M_{\chi(1) \times \chi(1)}(S'').
\end{equation}
Now set $\overline{\Lambda} := k \otimes_{S} \Lambda$ and observe that
\[
k'' \otimes_{k} \overline{\Lambda} = k'' \otimes_{k} (k \otimes_{S} \Lambda) \simeq k'' \otimes_{S} \Lambda
\simeq k'' \otimes_{S''} (S'' \otimes_{S} \Lambda) \simeq k'' \otimes_{S''} \Lambda'' \simeq M_{\chi(1) \times \chi(1)}(k'').
\]
Therefore \cite[Theorem 7.18]{MR1972204} shows that $\overline{\Lambda}$ is a separable $k$-algebra.
Now \cite[Theorem 4.7]{MR0121392} shows that $\Lambda$ is separable over $S$.
Hence $\Lambda$ is an Azumaya algebra (i.e.\ separable over its centre) and so defines a class $[\Lambda]$
in the Brauer group ${\mathrm{Br}}(S)$.
Now \cite[Corollary 6.2]{MR0121392} shows that the natural homomorphism
${\mathrm{Br}}(S) \rightarrow {\mathrm{Br}}(k)$ given by $[X] \mapsto [k \otimes_{S} X]$ is in fact a monomorphism.
However, ${\mathrm{Br}}(k)$ is trivial by Wedderburn's theorem that every finite division ring is a field.
Therefore $[\Lambda]$ is the trivial element of ${\mathrm{Br}}(S)$ and so $\Lambda$ is a matrix ring over its centre $S$.
Now \eqref{eqn:defn-of-Gamma} shows that in fact $\Lambda \simeq M_{\chi(1) \times \chi(1)}(S)$.
\end{proof}

The following proposition is an adaptation of \cite[Lemma 2.1]{hybrid-ETNC}.

\begin{prop}\label{prop:idempotent-implies-defect-zero}
Let $\chi \in {\mathrm{Irr}}_{{\mathbb{Q}}_{p}^{c}}(\mathcal{G})$.
If $\varepsilon_{\chi} \in \Lambda(\mathcal{G})$ then $v_{p}(\eta(1))=v_{p}(|H|)$ for every
irreducible constituent $\eta$ of ${\mathrm{res}}^{\mathcal{G}}_{H} \chi$.
\end{prop}

\begin{proof}
Fix an irreducible constituent $\eta$ of ${\mathrm{res}}^{\mathcal{G}}_{H} \chi$ and recall the notation of \S \ref{subset:idem-gal-actions}.
We may write $\varepsilon_{\chi} = \frac{\eta(1)}{|H|} \sum_{h \in H} \beta(h^{-1}) h$, where
\[
\beta := \sum_{\sigma \in {\mathrm{Gal}}(K_{\chi} / {\mathbb{Q}}_{p})} \sigma \left( \sum_{i=0}^{w_{\chi}-1} \eta^{\gamma^{i}} \right)
\]
is a ${\mathbb{Q}}_{p}$-valued character of $H$.
Recall that an element $h \in H$ is said to be $p$-singular if its order is divisible by $p$.
Write $\varepsilon_{\chi} = \sum_{h \in H} a_{h} h$ with $a_{h} \in {\mathbb{Z}}_{p}$ for $h \in H$.
Then \cite[Proposition 5]{MR1261587} shows that $a_{h}=0$ for every $p$-singular $h \in H$
(alternatively, one can use \cite[Proposition 3]{MR1261587} and that $\varepsilon_{\chi}$ is central).
Hence the character $\beta$ vanishes on $p$-singular elements.
Let $P$ be a Sylow $p$-subgroup of $H$.
Then $\beta$ vanishes on $P- \{1\}$, so the multiplicity of the trivial character of $P$ in the restriction ${\mathrm{res}}^{H}_{P} \beta$ is
\[
\langle {\mathrm{res}}^{H}_{P} \beta , 1_{P} \rangle = \beta(1) |P|^{-1} = [K_{\chi} : {\mathbb{Q}}_{p}] w_{\chi} \eta(1)|P|^{-1}.
\]
Now fix $0 \leq i < w_{\chi}$. Then $P' := \gamma^{i} P \gamma^{-i}$ is also a Sylow $p$-subgroup of $H$. Hence we may write $P' = hPh^{-1}$ for some $h \in H$.
We have
$\langle {\mathrm{res}}^{H}_{P} \eta^{\gamma^{i}} , 1_{P} \rangle = \langle {\mathrm{res}}^{H}_{P} \eta^{h} , 1_{P} \rangle = \langle {\mathrm{res}}^{H}_{P} \eta , 1_{P} \rangle$,
and thus
\[
\langle {\mathrm{res}}^{H}_{P} \beta, 1_{P} \rangle
= \sum_{\sigma \in {\mathrm{Gal}}(K_{\chi} / {\mathbb{Q}}_{p})} \sum_{i=0}^{w_{\chi}-1} \langle ^{\hat{\sigma}}({\mathrm{res}}^{H}_{P}\eta^{\gamma^{i}}), 1_{P} \rangle
= [K_{\chi} : {\mathbb{Q}}_{p}] w_{\chi}  \langle {\mathrm{res}}^{H}_{P} \eta , 1_{P} \rangle,
\]
where $\hat \sigma \in {\mathrm{Gal}}({\mathbb{Q}}_{p}(\eta) / {\mathbb{Q}}_{p})$ denotes any lift of $\sigma$.
Therefore $\eta(1) = |P| \langle {\mathrm{res}}^{H}_{P} \eta , 1_{P} \rangle$, and so $v_{p}(\eta(1))=v_{p}(|H|)$.
\end{proof}

\begin{prop}\label{prop:varep-comp-TFAE}
Let $\chi \in {\mathrm{Irr}}_{{\mathbb{Q}}_{p}^{c}}(\mathcal{G})$ and let $\eta$ be an irreducible constituent of
${\mathrm{res}}^{\mathcal{G}}_{H} \chi$. Then the following are equivalent:
\begin{enumerate}
\item $v_{p}(\eta(1))=v_{p}(|H|)$,
\item $\varepsilon_{\chi} \in \Lambda(\mathcal{G})$,
\item $\varepsilon_{\chi} \in \Lambda(\mathcal{G})$ and $\varepsilon_{\chi}\Lambda(\mathcal{G})$
is a maximal ${\mathbb{Z}}_{p}[[\Gamma_{0}]]$-order.
\end{enumerate}
Moreover, if these equivalent conditions hold then
$\varepsilon_{\chi}\Lambda(\mathcal{G}) \simeq M_{\chi(1) \times \chi(1)}(S_{\chi})$
for some local integrally closed domain $S_{\chi}$.
\end{prop}

\begin{proof}
This is the combination of Propositions \ref{prop:max-part-of-hybrid-matrix-over-comm-ring} and \ref{prop:idempotent-implies-defect-zero}.
\end{proof}

\subsection{Hybrid Iwasawa algebras}
Let $p$ be an odd prime and let $\mathcal{G} = H \rtimes \Gamma$ be a one-dimensional $p$-adic Lie group.
For a finite normal subgroup $N$ of $\mathcal{G}$, let $e_{N} = |N|^{-1}\sum_{\sigma \in N} \sigma$
be the associated central trace idempotent in the  algebra $\mathcal{Q}(\mathcal{G})$.
Then there is a ring isomorphism $\Lambda(\mathcal{G})e_{N} \simeq \Lambda(\mathcal{G}/N)$.
In particular, if $\mathcal{G}'$ is the commutator subgroup of $\mathcal{G}$ and
$\mathcal{G}^{\mathrm{ab}}=\mathcal{G}/\mathcal{G}'$ is the maximal abelian quotient,
then $\Lambda(\mathcal{G})e_{\mathcal{G}'} \simeq \Lambda(\mathcal{G}^{\mathrm{ab}})$.
Note that $\mathcal{G}/H$ is abelian and thus $\mathcal{G}'$ is a subgroup of $H$ and in particular is finite.

\begin{definition}\label{def:max-comm-hybrid}
Let $N$ be a finite normal subgroup of $\mathcal{G}$.
We say that the Iwasawa algebra $\Lambda(\mathcal{G})$ is \emph{$N$-hybrid}
if (i) $e_{N} \in \Lambda(\mathcal{G})$ (i.e.\ $p \nmid |N|$) and
(ii) $\Lambda(\mathcal{G})(1-e_{N})$ is a maximal ${\mathbb{Z}}_{p}[[\Gamma_{0}]]$-order in
$\mathcal{Q}(\mathcal{G})(1-e_{N})$.
\end{definition}

\begin{remark}\label{rmk:p-not-divide-H-max-order}
The Iwasawa algebra $\Lambda(\mathcal{G})$ is itself a maximal order if and only if $p$ does not divide $|H|$ if and only if 
$\Lambda(\mathcal{G})$ is $H$-hybrid. Moreover, $\Lambda(\mathcal{G})$ is always $\{ 1 \}$-hybrid.
\end{remark}

\begin{lemma}\label{lem:kernel-criterion}
Let $\chi \in {\mathrm{Irr}}_{{\mathbb{Q}}_{p}^{c}}(\mathcal{G})$ and let $\eta$ be an irreducible constituent of ${\mathrm{res}}^{\mathcal{G}}_{H} \chi$.
Let $N$ be a finite normal subgroup of $\mathcal{G}$.
Then $N$ is in fact a normal subgroup of $H$.
Moreover, $N \leq \ker(\chi)$ if and only if $N \leq \ker(\eta)$.
\end{lemma}

\begin{proof}
The canonical projection $\mathcal{G} \twoheadrightarrow \mathcal{G} / H \simeq \mathbb Z_{p}$ maps $N$ onto $NH /H \simeq N / H\cap N$.
However, the only finite (normal) subgroup of $\mathbb{Z}_{p}$ is the trivial group;
hence $H \cap N = N$ and thus $N$ is a (normal) subgroup of $H$.
As $\chi$ factors through a finite quotient of $\mathcal{G}$, the second claim follows from
Lemma \ref{lem:kernel-basechange} (ii).
\end{proof}

\begin{lemma}\label{lem:exists-chi-irr-constit-eta}
Let $\eta \in {\mathrm{Irr}}_{{\mathbb{Q}}_{p}^{c}}(H)$. Then there exists $\chi \in {\mathrm{Irr}}_{{\mathbb{Q}}_{p}^{c}}(\mathcal{G})$ such that $\eta$ is an irreducible constituent of
${\mathrm{res}}^{\mathcal{G}}_{H} \chi$.
\end{lemma}

\begin{proof}
Let $G=\mathcal{G}/\Gamma_{0} \simeq H \rtimes \Gamma/\Gamma_{0}$ (recall that $\Gamma_{0}$ is central in $\mathcal{G}$).
Let $\overline{\chi}$ be an irreducible constituent of ${\mathrm{ind}}^{G}_{H} \eta$.
Then by Frobenius reciprocity $\langle {\mathrm{res}}^{G}_{H} \overline{\chi}, \eta \rangle = \langle \overline{\chi}, {\mathrm{ind}}_{H}^{G} \eta \rangle \neq 0$.
Now take $\chi = {\mathrm{infl}}^{\mathcal{G}}_{G} \overline{\chi}$.
\end{proof}

\begin{theorem}\label{thm:hybrid-criterion}
Let $p$ be an odd prime and let $\mathcal{G} = H \rtimes \Gamma$ be a one-dimensional $p$-adic Lie group
with a finite normal subgroup $N$.
Then $N$ is in fact a normal subgroup of $H$.
Moreover, $\Lambda(\mathcal{G}):={\mathbb{Z}}_{p}[[\mathcal{G}]]$ is $N$-hybrid if and only if ${\mathbb{Z}}_{p}[H]$ is $N$-hybrid.
\end{theorem}

\begin{proof}
The first assertion is contained in Lemma \ref{lem:kernel-criterion}.

Suppose that ${\mathbb{Z}}_{p}[H]$ is $N$-hybrid.
Let $\chi \in {\mathrm{Irr}}_{{\mathbb{Q}}_{p}^{c}}(\mathcal{G})$ such that $N \not \leq \ker(\chi)$.
Let $\eta$ be an irreducible constituent of ${\mathrm{res}}^{\mathcal{G}}_{H} \chi$.
Then $N \not \leq \ker(\eta)$ by Lemma \ref{lem:kernel-criterion} and so
$v_{p}(\eta(1))=v_{p}(|H|)$ by Proposition \ref{prop:hybrid-criterion-groupring}.
Hence $\varepsilon_{\chi} \in \Lambda(\mathcal{G})$ and $\varepsilon_{\chi}\Lambda(\mathcal{G})$
is a maximal ${\mathbb{Z}}_{p}[[\Gamma_{0}]]$-order by Proposition \ref{prop:varep-comp-TFAE}.
Now by Lemma \ref{lem:kernel-criterion} we have
\begin{equation}\label{eq:1-en-decomp}
 1-e_{N}
= \sum_{\stackrel{\eta \in {\mathrm{Irr}}_{{\mathbb{Q}}_{p}^{c}}(H)}{N \not \leq \ker(\eta)}} e(\eta)
= \sum_{\stackrel{\chi \in {\mathrm{Irr}}_{{\mathbb{Q}}_{p}^{c}}(\mathcal{G})/\sim}{N \not \leq \ker(\chi)}} \varepsilon_{\chi}
\end{equation}
where $\sim$ denotes the equivalence relation defined in \S \ref{subset:idem-gal-actions}.
Therefore $1-e_{N} \in \Lambda(\mathcal{G})$ and
 $\Lambda(\mathcal{G})(1-e_{N})$ is a maximal ${\mathbb{Z}}_{p}[[\Gamma_{0}]]$-order
as it is the direct product of such orders. Hence $\Lambda(\mathcal{G})$ is $N$-hybrid.

Suppose conversely that $\Lambda(\mathcal{G})$ is $N$-hybrid.
Let $\eta \in {\mathrm{Irr}}_{{\mathbb{Q}}_{p}^{c}}(H)$ such that $N \not \leq \ker(\eta)$.
By Lemma \ref{lem:exists-chi-irr-constit-eta} there exists $\chi \in {\mathrm{Irr}}_{{\mathbb{Q}}_{p}^{c}}(\mathcal{G})$ such that $\eta$ is an irreducible
constituent of ${\mathrm{res}}^{\mathcal{G}}_{H} \chi$ and by Lemma \ref{lem:kernel-criterion}
we have $N \not \leq \ker(\chi)$.
Since $1-e_{N} \in \Lambda(\mathcal{G})$ and $\Lambda(\mathcal{G})(1-e_{N})$
is a maximal ${\mathbb{Z}}_{p}[[\Gamma_{0}]]$-order, it follows from
\eqref{eq:1-en-decomp} that in particular $\varepsilon_{\chi} \in \Lambda(\mathcal{G})$.
Thus Proposition \ref{prop:varep-comp-TFAE} gives $v_{p}(\eta(1))=v_{p}(|H|)$.
Therefore ${\mathbb{Z}}_{p}[H]$ is $N$-hybrid by Proposition \ref{prop:hybrid-criterion-groupring}.
\end{proof}

\begin{corollary}\label{cor:hybrid-implies-nr-surjective}
Let $p$ be an odd prime and let $\mathcal{G} = H \rtimes \Gamma$ be a one-dimensional $p$-adic Lie group
with a finite normal subgroup $N$.
If $\Lambda(\mathcal{G}) := {\mathbb{Z}}_{p}[[\mathcal{G}]]$ is $N$-hybrid then
$\Lambda(\mathcal{G})(1-e_{N})$ is isomorphic to a direct product of matrix rings over integrally closed commutative local rings.
\end{corollary}

\begin{proof}
This follows from Proposition \ref{prop:varep-comp-TFAE} and the proof of Theorem \ref{thm:hybrid-criterion} above.
\end{proof}

\begin{prop}\label{prop:frob-N-iwasawa-hybrid}
Let $p$ be an odd prime and $\mathcal{G} = H \rtimes \Gamma$ be a one-dimensional $p$-adic Lie group.
Suppose that $H$ is a Frobenius group with Frobenius kernel $N$.
If $p$ does not divide $|N|$, then $\Lambda(\mathcal{G}):={\mathbb{Z}}_{p}[[\mathcal{G}]]$ is $N$-hybrid.
\end{prop}

\begin{proof}
Let $\gamma$ be a topological generator of $\Gamma$.
Then $N$ is a normal subgroup of $H$ and so $\gamma N \gamma^{-1}$ is also a normal subgroup of $H$
since $\gamma H = H \gamma$. Thus $N$ and $\gamma N \gamma^{-1}$ are normal subgroups of $H$ of equal order and so
Theorem \ref{thm:frob-kernel} (iii) implies that they are in fact equal. Hence $N$ is normal in $\mathcal{G}$.
The claim is now an immediate consequence of Proposition \ref{prop:hybrid-criterion-groupring}
and Theorem \ref{thm:hybrid-criterion}.
\end{proof}

\begin{example}
Let $q$ be a prime power and let $H={\mathrm{Aff}}(q)$ be the Frobenius group with Frobenius kernel $N$ defined in Example \ref{ex:affine}.
Let $p$ be an odd prime not dividing $q$ and let $\mathcal{G} = H \rtimes \Gamma$ (any choice of semidirect product).
Then $p$ does not divide $|N|$ and so
$\Lambda(\mathcal{G}) := {\mathbb{Z}}_{p}[[\mathcal{G}]]$ is $N$-hybrid  by Proposition \ref{prop:frob-N-iwasawa-hybrid}.
We consider its structure in more detail.
Let $\eta$ be the unique non-linear irreducible character of $H$.
Choose $\chi \in {\mathrm{Irr}}_{{\mathbb{Q}}_{p}^{c}}(\mathcal{G})$
such that $\eta$ appears as an irreducible constituent of ${\mathrm{res}}^{\mathcal{G}}_{H} \chi$ (this is possible by Lemma \ref{lem:exists-chi-irr-constit-eta}.)
As $\eta$ is the only non-linear irreducible character of $H$, we must have $\eta^{g} = \eta$ for every $g \in \mathcal{G}$,
i.e., $St(\eta) = \mathcal{G}$. Consequently, we have $w_{\chi} = 1$ and thus $\chi(1) = \eta(1) = q-1$.
Since both $\eta$ and $\chi$ have realisations over ${\mathbb{Q}}$ and hence over ${\mathbb{Q}}_{p}$,
applying Proposition \ref{prop:chi-comp-max-order} therefore shows that there is a ring isomorphism
$\Lambda(\mathcal{G}) \simeq {\mathbb{Z}}_{p}[[C_{q-1} \rtimes \Gamma]] \oplus M_{(q-1) \times (q-1)}({\mathbb{Z}}_{p}[[T]])$.
\end{example}

\begin{example}\label{ex:S4-V4}
Let $p=3$, $H=S_{4}$ and $N=V_{4}$.
Recall from Example \ref{ex:S4-A4-V4} that ${\mathbb{Z}}_{3}[S_{4}]$ is $V_{4}$-hybrid and $S_{4}$ is \emph{not} a Frobenius group.
Let $\mathcal{G} = H \rtimes \Gamma$ (any choice of semidirect product).
As each automorphism of $S_{4}$ is inner, $N$ is normal in $\mathcal{G}$
and so Theorem \ref{thm:hybrid-criterion} shows that $\Lambda(\mathcal{G}):={\mathbb{Z}}_{3}[[\mathcal{G}]]$ is $N$-hybrid.
We consider its structure in more detail.
Note that $H/N \simeq S_{3}$ and the only two complex irreducible characters $\eta$ and $\eta'$ of $H$
not inflated from characters of $S_{3}$ are of degree $3$ and have realisations over ${\mathbb{Q}}$ and hence over ${\mathbb{Q}}_{3}$.
Choose characters $\chi, \chi' \in {\mathrm{Irr}}_{{\mathbb{Q}}_{3}^{c}}(\mathcal{G})$ with the property that $\eta$ and $\eta'$ appear as irreducible constituents of ${\mathrm{res}}^{\mathcal{G}}_{H} \chi$ and ${\mathrm{res}}^{\mathcal{G}}_{H} \chi'$, respectively.
Again, as each automorphism of $S_{4}$ is inner, we have
$St(\eta) = St(\eta') = \mathcal{G}$ and thus $w_{\chi} = w_{\chi'} = 1$.
Therefore Proposition \ref{prop:chi-comp-max-order} yields a ring isomorphism
$\Lambda(\mathcal{G}) \simeq {\mathbb{Z}}_{3}[[S_{3} \rtimes \Gamma]] \oplus M_{3 \times 3}({\mathbb{Z}}_{3}[[T]]) \oplus  M_{3 \times 3}({\mathbb{Z}}_{3}[[T']]))$.
\end{example}

\subsection{Iwasawa algebras and commutator subgroups}
Let $p$ be prime (not necessarily odd) and let $\mathcal{G}=H \rtimes \Gamma$ be a one-dimensional $p$-adic Lie group.

\begin{prop}[{\cite[Proposition 4.5]{MR3092262}}]\label{prop:niceIwasawa-algebras}
The Iwasawa algebra $\Lambda(\mathcal{G})={\mathbb{Z}}_{p}[[\mathcal{G}]]$ is a direct product of matrix rings over commutative rings
if and only if $p$ does not divide the order of the commutator subgroup $\mathcal{G}'$ of $\mathcal{G}$.
\end{prop}

\begin{corollary}\label{cor:no-skewfields}
If $p$ does not divide the order of $\mathcal{G}'$ then no skewfields appear in the Wedderburn decomposition of $\mathcal{Q}(\mathcal{G})$.
\end{corollary}

\begin{remark}
Note that $\mathcal{G}'$ is a normal subgroup of $H$.
If $\Lambda(\mathcal{G})$ is $\mathcal{G}'$-hybrid then $p$ does not divide the order of $\mathcal{G}'$.
However, the converse does not hold in general.
\end{remark}

\subsection{Hybrid algebras in Iwasawa theory}\label{subsec:algebras-in-Iwasawa-thy}
Let $p$ be an odd prime.
We denote the cyclotomic ${\mathbb{Z}}_{p}$-extension of a number field $K$ by $K_{\infty}$ and let $K_{m}$ be its $m$th layer.
We put $\Gamma_{K} := {\mathrm{Gal}}(K_{\infty}/K)$
and choose a topological generator $\gamma_{K}$. In Iwasawa theory one is often concerned with the
following situation. Let $L/K$ be a finite Galois extension of number fields with Galois group $G$.
We put $H := {\mathrm{Gal}}(L_{\infty}/K_{\infty})$ and $\mathcal{G} := {\mathrm{Gal}}(L_{\infty}/K)$.
Then $H$ naturally identifies with a normal subgroup of $G$ and $G/H$ is cyclic of $p$-power order
(the field $L^{H}$ equals $L \cap K_{\infty}$ and thus identifies with $K_{m}$ for some $m< \infty$).
Moreover,  the argument given in \cite[\S 1]{MR2114937} shows that the short exact sequence
\[
0 \longrightarrow H \longrightarrow \mathcal{G} \longrightarrow \Gamma_{K} \longrightarrow 0
\]
splits and so we obtain a semidirect product $\mathcal{G} \simeq H \rtimes \Gamma$ where $\Gamma \simeq {\mathbb{Z}}_{p}$.

\begin{prop}\label{prop:hybrid-codescent}
Keep the above notation and suppose that ${\mathbb{Z}}_{p}[G]$ is $N$-hybrid.
Then $N$ naturally identifies with a normal subgroup of $\mathcal{G}$,
which is also a normal subgroup of $H$.
Moreover, both ${\mathbb{Z}}_{p}[H]$ and $\Lambda(\mathcal{G}):={\mathbb{Z}}_{p}[[\mathcal{G}]]$ are also $N$-hybrid.
\end{prop}

\begin{proof}
Let $F = L^{N}$.
As $e_{N}$ lies in ${\mathbb{Z}}_{p}[G]$, we have that $p \nmid |N|$.
Hence $N$ naturally identifies with ${\mathrm{Gal}}(L_{\infty}/F_{\infty})$, which is a normal subgroup of both 
$H$ and $\mathcal{G}$
since $F_{\infty}/K$ is a Galois extension.
Thus ${\mathbb{Z}}_{p}[H]$ is $N$-hybrid by Proposition \ref{prop:hybrid-basechange-down} and so
$\Lambda(\mathcal{G})$ is also $N$-hybrid by Theorem \ref{thm:hybrid-criterion}.
\end{proof}

\begin{example} \label{ex:S4-V4-II}
Let $p=3$ and suppose that $G = {\mathrm{Gal}}(L/K) \simeq S_{4}$.
Then we also have ${\mathrm{Gal}}(L_{\infty}/K_{\infty}) \simeq S_{4}$, since $S_{4}$ has no abelian quotient of $3$-power order.
As a consequence, we have $\mathcal{G} = {\mathrm{Gal}}(L_{\infty}/K) \simeq S_{4} \times \Gamma_K$.
Using the notation of Example \ref{ex:S4-V4}, this yields a ring isomorphism
$\Lambda(\mathcal{G}) \simeq {\mathbb{Z}}_{3}[[S_{3} \times \Gamma_{K}]]
\oplus M_{3 \times 3}({\mathbb{Z}}_{3}[[T]]) \oplus  M_{3 \times 3}({\mathbb{Z}}_{3}[[T']]))$.
\end{example}

\begin{example}
Assume that $G = N \rtimes V$ is a Frobenius group and that $p \nmid |N|$.
Then by Proposition \ref{prop:frob-N-hybrid} the group ring ${\mathbb{Z}}_{p}[G]$ is $N$-hybrid.
It is straightfoward to check that $H = {\mathrm{Gal}}(L_{\infty}/K_{\infty}) \simeq N \rtimes U$ is a Frobenius group with $U \leq V$.
Let $F = L^{N}$.
If we assume that $V$ is abelian, then ${\mathrm{Gal}}(F_{\infty}/K) \simeq \Gamma_K \times U$ is also abelian
and we have an isomorphism
$\Lambda(\mathcal{G}) \simeq \Lambda(\Gamma_{K} \times U) \oplus (1-e_{N}) \mathfrak{M}(G)$
where $\mathfrak{M}(G)$ is a maximal order containing $\Lambda(\mathcal{G})$.
\end{example}

\section{The equivariant Iwasawa main conjecture} \label{sec:EIMC}

\subsection{Algebraic $K$-theory}\label{subsec:K-theory}
Let $R$ be a noetherian integral domain with field of fractions $E$.
Let $A$ be a finite-dimensional semisimple $E$-algebra and let $\mathfrak{A}$ be an $R$-order in $A$.
Let ${\mathrm{PMod}}(\mathfrak{A})$ denote the category of finitely generated projective (left) $\mathfrak{A}$-modules.
We write $K_{0}(\mathfrak{A})$ for the Grothendieck group of ${\mathrm{PMod}}(\mathfrak{A})$ (see \cite[\S 38]{MR892316})
and $K_{1}(\mathfrak{A})$ for the Whitehead group (see \cite[\S 40]{MR892316}).
Let $K_{0}(\mathfrak{A}, A)$ denote the relative algebraic $K$-group associated to the ring homomorphism
$\mathfrak{A} \hookrightarrow A$.
We recall that $K_{0}(\mathfrak{A}, A)$ is an abelian group with generators $[X,g,Y]$ where
$X$ and $Y$ are finitely generated projective $\mathfrak{A}$-modules
and $g:E \otimes_{R} X \rightarrow E \otimes_{R} Y$ is an isomorphism of $A$-modules;
for a full description in terms of generators and relations, we refer the reader to \cite[p.\ 215]{MR0245634}.
Moreover, there is a long exact sequence of relative $K$-theory (see \cite[Chapter 15]{MR0245634})
\begin{equation}\label{eqn:long-exact-seq}
K_{1}(\mathfrak{A}) \longrightarrow K_{1}(A) \stackrel{\partial}{\longrightarrow} K_{0}(\mathfrak{A}, A)
\stackrel{\rho}{\longrightarrow} K_{0}(\mathfrak{A}) \longrightarrow K_{0}(A).
\end{equation}
The reduced norm map ${\mathrm{nr}} = {\mathrm{nr}}_{A}: A \rightarrow \zeta(A)$ is defined componentwise on the Wedderburn decomposition of $A$
and extends to matrix rings over $A$ (see \cite[\S 7D]{MR632548}); thus
it induces a map $K_{1}(A) \longrightarrow \zeta(A)^{\times}$, which we also denote by ${\mathrm{nr}}$.

Let $\mathcal C^{b} ({\mathrm{PMod}} (\mathfrak{A}))$ be the category of bounded complexes of finitely generated projective $\mathfrak{A}$-modules.
Then $K_{0}(\mathfrak{A}, A)$ identifies with the Grothendieck group whose generators are $[C^{\bullet}]$, where $C^{\bullet}$
is an object of the category $\mathcal C^{b}{_{\mathrm{tor}}}({\mathrm{PMod}}(\mathfrak{A}))$ of bounded complexes of finitely generated projective $\mathfrak{A}$-modules whose cohomologies are $R$-torsion, and the relations are as follows: $[C^{\bullet}] = 0$ if $C^{\bullet}$ is acyclic, and
$[C_{2}^{\bullet}] = [C_{1}^{\bullet}] + [C_{3}^{\bullet}]$ for every short exact sequence
\begin{equation}\label{eq:SES-of-complexes}
0 \longrightarrow C_{1}^{\bullet} \longrightarrow C_{2}^{\bullet} \longrightarrow C_{3}^{\bullet} \longrightarrow 0
\end{equation}
in $\mathcal C^{b}{_{\mathrm{tor}}}({\mathrm{PMod}}(\mathfrak{A}))$ (see \cite[Chapter 2]{MR3076731} or \cite[\S 2]{zbMATH06148871}, for example).

Let $\mathcal{D} (\mathfrak{A})$ be the derived category of $\mathfrak{A}$-modules.
A complex of $\mathfrak{A}$-modules is said to be perfect if it is
isomorphic in $\mathcal{D} (\mathfrak{A})$ to an element of $\mathcal C^b({\mathrm{PMod}} (\mathfrak{A}))$.
We denote the full triangulated subcategory of
$\mathcal{D} (\mathfrak{A})$ comprising perfect complexes by $\mathcal{D}^{\mathrm{perf}} (\mathfrak{A})$,
and the full triangulated subcategory
comprising perfect complexes whose cohomologies are $R$-torsion by $\mathcal{D}^{\mathrm{perf}}{_{\mathrm{tor}}} (\mathfrak{A})$.
Then any object of $\mathcal{D}^{\mathrm{perf}}{_{\mathrm{tor}}} (\mathfrak{A})$ defines an element in
$K_{0}(\mathfrak{A}, A)$.
In particular, a finitely generated $R$-torsion $\mathfrak{A}$-module $M$ of finite projective dimension 
considered as a complex concentrated in degree $0$ defines an element $[M] \in K_{0}(\mathfrak{A}, A)$.

We now specialise to the situation of \S \ref{subsec:Iwasawa-algebras}.
Let $p$ be an odd prime and let $\mathcal{G} \simeq H \rtimes \Gamma$ be a one-dimensional $p$-adic Lie group.
Let $A = \mathcal{Q}(\mathcal{G})$, $\mathfrak{A} = \Lambda(\mathcal{G})={\mathbb{Z}}_{p}[[\mathcal{G}]]$ and $R={\mathbb{Z}}_{p}[[\Gamma_{0}]]$,
where $\Gamma_{0}$ is an open subgroup of $\Gamma$ that is central in $\mathcal{G}$.
Then \cite[Corollary 3.8]{MR3034286}  (take $\mathcal{O}={\mathbb{Z}}_{p}$ and $G=\mathcal{G}$ and note that
$\mathcal{O}[[G]]_{S^{*}}=\mathcal{Q}(\mathcal{G})$ since $\mathcal{G}$ is one-dimensional)
shows that the map $\partial$ in \eqref{eqn:long-exact-seq}
is surjective
(one can also give a slight modification of the proof of either \cite[Proposition 3.4]{MR2217048} or \cite[Lemma 1.5]{MR2819672});
thus the sequence
\begin{equation}\label{eqn:Iwasawa-K-sequence}
K_{1}(\Lambda(\mathcal{G})) \longrightarrow K_{1}(\mathcal{Q}(\mathcal{G})) \stackrel{\partial}{\longrightarrow}
K_{0}(\Lambda(\mathcal{G}),\mathcal{Q}(\mathcal{G})) \longrightarrow 0
\end{equation}
is exact.

\subsection{Admissible extensions and the $\mu=0$ hypothesis}\label{subsec:admiss-and-mu}

\begin{definition}\label{def:one-dim-adm}
Let $p$ be an odd prime and let $K$ be a totally real number field.
An admissible one-dimensional $p$-adic Lie extension $\mathcal{L}$ of $K$ is a Galois extension $\mathcal{L}$ of $K$
such that (i) $\mathcal{L}$ is totally real,
(ii) $\mathcal{L}$ contains the cyclotomic ${\mathbb{Z}}_{p}$-extension $K_{\infty}$ of $K$, and
(iii) $[\mathcal{L} : K_{\infty}]$ is finite.
\end{definition}

Let $\mathcal{L}/K$ be an admissible one-dimensional $p$-adic Lie extension with Galois group $\mathcal{G}$.
Let $H={\mathrm{Gal}}(\mathcal{L}/K_{\infty})$ and let $\Gamma_{K}={\mathrm{Gal}}(K_{\infty}/K)$.
Then we have a short exact sequence
\[
0 \longrightarrow H \longrightarrow \mathcal{G} \longrightarrow \Gamma_{K} \longrightarrow 0
\]
and as in \S \ref{subsec:algebras-in-Iwasawa-thy} we have a semidirect product decomposition
$\mathcal{G} \simeq H \rtimes \Gamma$ where $\Gamma \simeq {\mathbb{Z}}_{p}$.

Let $S_{\infty}$ be the set of archimedean places of $K$ and let $S_{p}$ be the set of places of $K$ above $p$.
Let $S_{\mathrm{ram}}=S_{\mathrm{ram}}(\mathcal{L}/K)$ be the (finite) set of places of $K$ that ramify in $\mathcal{L}/K$;
note that  $S_{p} \subseteq S_{\mathrm{ram}}$.
Let $S$ be a finite set of places of $K$ containing $S_{\mathrm{ram}} \cup S_{\infty}$.
Let $M_{S}^{\mathrm{ab}}(p)$ be the maximal abelian pro-$p$-extension of $\mathcal{L}$ unramified outside $S$,
and denote the $\Lambda(\mathcal{G})$-module ${\mathrm{Gal}}(M_{S}^{\mathrm{ab}}(p)/\mathcal{L})$ by $X_{S}$.

\begin{definition}\label{def:mu=0-hypothesis}
We say that $\mathcal{L}/K$ satisfies the $\mu=0$ hypothesis if $X_{S}$ is finitely generated as a ${\mathbb{Z}}_{p}$-module.
\end{definition}

\begin{remark}\label{rmk:mu=0}
The classical Iwasawa $\mu=0$ conjecture (at $p$) is the assertion that for every number field $F$, the
Galois group of the maximal unramified abelian $p$-extension of $F_{\infty}$ is a finitely generated ${\mathbb{Z}}_{p}$-module.
This conjecture has been proven by Ferrero and Washington \cite{MR528968} in the case that $F/{\mathbb{Q}}$ is abelian.
Now let $\mathcal{L}/K$ be an admissible one-dimensional $p$-adic Lie extension
and let $L$ be a finite Galois extension of $K$ such that $L_{\infty}=\mathcal{L}$.
Let $E$ be an intermediate field of $L/K$ such that $L/E$ is of $p$-power degree.
Then \cite[Theorem 11.3.8]{MR2392026} says that $\mathcal{L}/K$
satisfies the $\mu=0$ hypothesis if and only if $E_{\infty}/K$ does.
Finally, let $\zeta_{p}$ denote a primitive $p$th root of unity.
Then by \cite[Corollary 11.4.4]{MR2392026} Iwasawa's conjecture for $E(\zeta_{p})$ implies the $\mu=0$ hypothesis for 
$E_{\infty}(\zeta_{p})^{+}/K$ and thus for $E_{\infty}/K$ and $\mathcal{L}/K$.
\end{remark}

\subsection{A reformulation of the equivariant Iwasawa main conjecture} \label{subsec:EIMC-reformulation}
We give a slight reformulation of the equivariant Iwasawa main conjecture for totally real fields.

Let $\mathcal{L}/K$ be an admissible one-dimensional $p$-adic Lie extension.
We assume the notation and setting of \S \ref{subsec:admiss-and-mu}.
However, we do \emph{not} assume the $\mu=0$ hypothesis for $\mathcal{L}/K$ except where explicitly stated.
Let $C_{S}^{\bullet}(\mathcal{L}/K)$ be the canonical complex
\[
    C_{S}^{\bullet}(\mathcal{L}/K) := R{\mathrm{Hom}}(R\Gamma_{\mathrm{\acute{e}t}}({\mathrm{Spec}}(\mathcal{O}_{\mathcal{L},S}), {\mathbb{Q}}_{p} / {\mathbb{Z}}_{p}), {\mathbb{Q}}_{p} / {\mathbb{Z}}_{p}).
\]
Here, $\mathcal{O}_{\mathcal{L},S}$ denotes the ring of integers $\mathcal{O}_{\mathcal{L}}$ in $\mathcal{L}$ localised at all primes above those in $S$
and
${\mathbb{Q}}_{p} / {\mathbb{Z}}_{p}$ denotes the constant sheaf of the abelian group ${\mathbb{Q}}_{p} / {\mathbb{Z}}_{p}$ on the \'{e}tale site
of ${\mathrm{Spec}}(\mathcal{O}_{\mathcal{L},S})$.
The only non-trivial cohomology groups occur in degree $-1$ and $0$ and we have
\[
H^{-1}(C_{S}^{\bullet}(\mathcal{L}/K)) \simeq X_{S}, \qquad H^{0}(C_{S}^{\bullet}(\mathcal{L}/K)) \simeq {\mathbb{Z}}_{p}.
\]
It follows from \cite[Proposition 1.6.5]{MR2276851} that $C_{S}^{\bullet}(\mathcal{L}/K)$ belongs to $\mathcal{D}^{\mathrm{perf}}{_{\mathrm{tor}}}(\Lambda(\mathcal{G}))$.
In particular, $C_{S}^{\bullet}(\mathcal{L}/K)$ defines a class $[C_{S}^{\bullet}(\mathcal{L}/K)]$ in $K_{0}(\Lambda(\mathcal{G}), \mathcal{Q}(\mathcal{G}))$.
Note that $C_{S}^{\bullet}(\mathcal{L}/K)$ and the complex used
by Ritter and Weiss (as constructed in \cite{MR2114937}) become isomorphic in $\mathcal{D}(\Lambda(\mathcal{G}))$ by
\cite[Theorem 2.4]{MR3072281} (see also \cite{MR3068897} for more on this topic).
Hence it makes no essential difference which of these complexes we use.

Recall the notation and hypotheses of \S \ref{subsec:idempotents} and \S \ref{subsec:sufficiently-large}.
In particular, $F$ is a sufficiently large finite extension of ${\mathbb{Q}}_{p}$.
Let $\chi_{\mathrm{cyc}}$ be the $p$-adic cyclotomic character
\[
\chi_{\mathrm{cyc}}: {\mathrm{Gal}}(\mathcal{L}(\zeta_{p})/K) \longrightarrow {\mathbb{Z}}_{p}^{\times},
\]
defined by $\sigma(\zeta) = \zeta^{\chi_{\mathrm{cyc}}(\sigma)}$ for any $\sigma \in {\mathrm{Gal}}(\mathcal{L}(\zeta_{p})/K)$ and any $p$-power root of unity $\zeta$.
Let $\omega$ and $\kappa$ denote the composition of $\chi_{\mathrm{cyc}}$ with the projections onto the first and second factors of the canonical decomposition ${\mathbb{Z}}_{p}^{\times} = \mu_{p-1} \times (1+p{\mathbb{Z}}_{p})$, respectively;
thus $\omega$ is the Teichm\"{u}ller character.
We note that $\kappa$ factors through $\Gamma_{K}$ 
(and thus also through $\mathcal{G}$) and by abuse of notation we also 
use $\kappa$ to denote the associated maps with these domains.
We put $u := \kappa(\gamma_{K})$.
For $r \in {\mathbb{N}}_{0}$ divisible by $p-1$ 
(or more generally divisible by the degree $[\mathcal{L}(\zeta_{p}) : \mathcal{L}]$), 
up to the natural inclusion map of codomains, 
we have $\chi_{\mathrm{cyc}}^{r}=\kappa^{r}$. 

For $\chi \in {\mathrm{Irr}}_{{\mathbb{Q}}_{p}^{c}}(\mathcal{G})$ and $r \in {\mathbb{Z}}$ we define maps
\[
j_{\chi}^{r}: \zeta(\mathcal{Q}^{F} (\mathcal{G})) \twoheadrightarrow \zeta(\mathcal{Q}^{F} (\mathcal{G})e_{\chi}) \simeq \mathcal{Q}^{F}(\Gamma_{\chi}) \rightarrow  \mathcal{Q}^{F}(\Gamma_{K}),
\]
where the last arrow is induced by mapping $\gamma_{\chi}$ to 
$(u^{r}\gamma_{K})^{w_{\chi}}$.
Note that $j_{\chi} := j_{\chi}^{0}$ agrees with the corresponding map $j_{\chi}$ in \cite{MR2114937}.
It follows from \cite[Proposition 6 (3)]{MR2114937} that for every matrix $\Theta \in M_{n \times n} (\mathcal{Q}(\mathcal{G}))$ we have
\begin{equation} \label{eqn:jchi-det}
j_{\chi} ({\mathrm{nr}}(\Theta)) = \mathrm{det}_{\mathcal{Q}^{F}(\Gamma_{K})} (\Theta \mid {\mathrm{Hom}}_{F[H]}(V_{\chi},  \mathcal{Q}^{F}(\mathcal{G})^n)).
\end{equation}
Here, $\Theta$ acts on $f \in {\mathrm{Hom}}_{F[H]}(V_{\chi},  \mathcal{Q}^{F}(\mathcal{G})^{n})$ via right multiplication,
and $\gamma_{K}$ acts on the left via $(\gamma_{K} f)(v) = \gamma \cdot f(\gamma^{-1} v)$ for all $v \in V_{\chi}$,
where $\gamma$ is the unique lift of $\gamma_{K}$ to $\Gamma \leq \mathcal{G}$.
Hence the map
\begin{eqnarray*}
{\mathrm{Det}}(~)(\chi): K_{1}(\mathcal{Q}(\mathcal{G})) & \rightarrow & \mathcal{Q}^{F}(\Gamma_{K})^{\times} \\
 {[P,\alpha]}& \mapsto & \mathrm{det}_{\mathcal{Q}^{F}(\Gamma_{K})} (\alpha \mid {\mathrm{Hom}}_{F[H]}(V_{\chi},  F \otimes_{{\mathbb{Q}}_{p}} P)),
\end{eqnarray*}
where $P$ is a projective $\mathcal{Q}(\mathcal{G})$-module and $\alpha$ a $\mathcal{Q}(\mathcal{G})$-automorphism of $P$, is just $j_{\chi} \circ {\mathrm{nr}}$ (see \cite[\S 3, p.558]{MR2114937}).
If $\rho$ is a character of $\mathcal{G}$ of type $W$ (i.e.~${\mathrm{res}}^{\mathcal{G}}_H \rho = 1$)
then we denote by
$\rho^{\sharp}$ the automorphism of the field $\mathcal{Q}^{c}(\Gamma_{K})$ induced by
$\rho^{\sharp}(\gamma_{K}) = \rho(\gamma_{K}) \gamma_{K}$. 
Moreover, we denote the additive group generated by all ${\mathbb{Q}}_{p}^{c}$-valued
characters of $\mathcal{G}$ with open kernel by $R_p(\mathcal{G})$; finally, 
${\mathrm{Hom}}^{\ast}(R_{p}( \mathcal{G}), \mathcal{Q}^{c}(\Gamma_{K})^{\times})$
is the group of all homomorphisms 
$f: R_p(\mathcal{G}) \rightarrow \mathcal{Q}^{c}(\Gamma_{K}){^{\times}}$ satisfying
\[
\begin{array}{ll}
f(\chi \otimes \rho) = \rho^{\sharp}(f(\chi)) & \mbox{ for all characters } \rho \mbox{ of type } W \mbox{ and}\\
f({}^{\sigma}\chi) = \sigma(f(\chi)) & \mbox{ for all Galois automorphisms } \sigma \in {\mathrm{Gal}}({\mathbb{Q}}_{p}^{c}/{\mathbb{Q}}_{p}).
\end{array}
\]
By \cite[Proof of Theorem 8]{MR2114937} we have an isomorphism
\begin{eqnarray*}
\zeta(\mathcal{Q}(\mathcal{G})){^{\times}} & \simeq & 
{\mathrm{Hom}}^{\ast}(R_{p}(\mathcal{G}), \mathcal{Q}^{c}(\Gamma_{K})^{\times})\\
x & \mapsto & [\chi \mapsto j_{\chi}(x)].
\end{eqnarray*}
By \cite[Theorem 8]{MR2114937} the map $\Theta \mapsto [\chi \mapsto {\mathrm{Det}}(\Theta)(\chi)]$
defines a homomorphism
\[
{\mathrm{Det}}: K_{1}(\mathcal{Q}(\mathcal{G})) \rightarrow {\mathrm{Hom}}^{\ast}(R_p(\mathcal{G}), \mathcal{Q}^{c}(\Gamma_{K}){^{\times}})
\]
such that we obtain a commutative triangle
\begin{equation} \label{eqn:Det_triangle}
\xymatrix{
& K_{1}(\mathcal{Q}(\mathcal{G})) \ar[dl]_{\mathrm{nr}} \ar[dr]^{\mathrm{Det}} &\\
{\zeta(\mathcal{Q}(\mathcal{G}))^{\times}} \ar[rr]^{\sim} & & {{\mathrm{Hom}}^{\ast}(R_p( \mathcal{G}), \mathcal{Q}^{c}(\Gamma_{K})^{\times})}.}
\end{equation}

Each topological generator $\gamma_{K}$ of  $\Gamma_{K}$ permits the definition of a 
power series $G_{\chi,S}(T) \in {\mathbb{Q}}_{p}^{c} \otimes_{{\mathbb{Q}}_{p}} Quot({\mathbb{Z}}_{p}[[T]])$ 
by starting out from the Deligne-Ribet power series for linear characters of open subgroups 
of $\mathcal{G}$ (see \cite{MR579702}; also see \cite{ MR525346, MR524276}) 
and then extending to the general case by using Brauer induction (see \cite{MR692344}).
One then has an equality
\[
L_{p,S}(1-s,\chi) = \frac{G_{\chi,S}(u^s-1)}{H_{\chi}(u^s-1)},
\]
where $L_{p,S}(s,\chi)$ denotes the `$S$-truncated $p$-adic Artin $L$-function' attached to $\chi$ constructed by Greenberg \cite{MR692344},
and where, for irreducible $\chi$, one has
\[
H_{\chi}(T) = \left\{\begin{array}{ll} \chi(\gamma_{K})(1+T)-1 & \mbox{ if }  H \subseteq \ker \chi\\
1 & \mbox{ otherwise.}  \end{array}\right.
\]
Now \cite[Proposition 11]{MR2114937} implies that
\[
L_{K,S} : \chi \mapsto \frac{G_{\chi,S}(\gamma_{K}-1)}{H_{\chi}(\gamma_{K}-1)}
\]
is independent of the topological generator $\gamma_{K}$ and lies in 
${\mathrm{Hom}}^{\ast}(R_{p}( \mathcal{G}), \mathcal{Q}^{c}(\Gamma_{K})^{\times})$.
Diagram \eqref{eqn:Det_triangle} implies that there is a unique element 
$\Phi_{S} = \Phi_{S}(\mathcal{L}/K) \in \zeta(\mathcal{Q}(\mathcal{G}))^{\times}$
such that
\[
j_{\chi}(\Phi_{S}) = L_{K,S}(\chi)
\]
for every $\chi \in {\mathrm{Irr}}_{{\mathbb{Q}}_{p}^{c}}(\mathcal{G})$.
It is now clear that the following is a reformulation of the EIMC without its uniqueness statement.

\begin{conj}[EIMC]\label{conj:EIMC}
There exists $\zeta_{S} \in K_{1}(\mathcal{Q}(\mathcal{G}))$ such that $\partial(\zeta_{S}) = -[C_{S}^{\bullet}(\mathcal{L}/K)]$
and ${\mathrm{nr}}(\zeta_{S}) = \Phi_{S}$.
\end{conj}

It can be shown that the truth of Conjecture \ref{conj:EIMC} is independent of the choice of $S$, 
provided $S$ is finite and contains $S_{\mathrm{ram}} \cup S_{\infty}$.
The following theorem has been shown independently by Ritter and Weiss \cite{MR2813337} and Kakde \cite{MR3091976}.

\begin{theorem}\label{thm:EIMC-with-mu}
If $\mathcal{L}/K$ satisfies the $\mu=0$ hypothesis then the EIMC holds for $\mathcal{L}/K$.
\end{theorem}

\begin{corollary}\label{cor:EIMC-unconditional}
Let $\mathcal{P}$ be a Sylow $p$-subgroup of $\mathcal{G}$.
If $\mathcal{L}^{\mathcal{P}}/{\mathbb{Q}}$ is abelian then $\mathcal{P}$ is normal in $\mathcal{G}$ (and thus is unique),
and the EIMC holds for $\mathcal{L}/K$.
\end{corollary}

\begin{proof}
The first claim is clear.
Let $E=\mathcal{L}^{\mathcal{P}}$ and let $L$ be a finite Galois extension of $K$ such that $L_{\infty}=\mathcal{L}$.
Then $L/E$ is a finite Galois extension of $p$-power degree.
Moreover, $E/{\mathbb{Q}}$ is a (finite) abelian extension by hypothesis and so $E(\zeta_{p})/{\mathbb{Q}}$ is also abelian.
Therefore the $\mu=0$ hypothesis for $\mathcal{L}/K$ holds by the results discussed in Remark \ref{rmk:mu=0}.
\end{proof}

We shall also consider the EIMC with its uniqueness statement.

\begin{conj}[EIMC with uniqueness]\label{conj:EIMC-unique}
There exists a unique $\zeta_{S} \in K_{1}(\mathcal{Q}(\mathcal{G}))$ such that $\partial(\zeta_{S}) = -[C_{S}^{\bullet}(\mathcal{L}/K)]$
and ${\mathrm{nr}}(\zeta_{S}) = \Phi_{S}$.
\end{conj}

\begin{remark}\label{rmk:SK1}
Let $SK_{1}(\mathcal{Q}(\mathcal{G})) = \ker({\mathrm{nr}}: K_{1}(\mathcal{Q}(\mathcal{G})) \longrightarrow \zeta(\mathcal{Q}(\mathcal{G}))^{\times})$.
If $SK_{1}(\mathcal{Q}(\mathcal{G}))$ vanishes then it is clear that the uniqueness statement of the EIMC follows from
its existence statement.
Moreover, $SK_{1}(\mathcal{Q}(\mathcal{G}))$ vanishes if no skewfields appear in the Wedderburn decomposition of
$\mathcal{Q}(\mathcal{G})$; in particular, this is the case if $\mathcal{G}$ is abelian or, more generally, if $p$ does not
divide the order of the commutator subgroup $\mathcal{G}'$ of $\mathcal{G}$ (see Corollary \ref{cor:no-skewfields}).
As noted in \cite[Remark E]{MR2114937} (also see \cite[Remark 3.5]{burns-mc}), a conjecture of Suslin implies that $SK_{1}(\mathcal{Q}(\mathcal{G}))$ in fact always vanishes.
\end{remark}

\subsection{The interpolation property}\label{subsec:interpolation-property}
We fix a choice of field isomorphism $\iota:{\mathbb{C}}_{p} \rightarrow {\mathbb{C}}$.
Recall the assumption that $S \subseteq S_{\mathrm{ram}} \cup S_{\infty}$ so that in particular $S_{p} \subseteq S$.
Let $\chi \in {\mathrm{Irr}}_{{\mathbb{Q}}_{p}^{c}}(\mathcal{G})$.
Let $L_{S}(s,\iota \circ \chi)$ denote the $S$-truncated Artin $L$-function attached to $\iota \circ \chi \in {\mathrm{Irr}}_{\mathbb{C}}(\mathcal{G})$.
For $r \in {\mathbb{Z}}$ with $r \geq 1$ let $L_{S}(1-r, \chi) = \iota^{-1}(L_{S}(s,\iota \circ \chi))$, which is in fact independent of the choice of $\iota$.
If $\chi$ is linear then for  $r \geq 1$ we have
\begin{equation} \label{eqn:interpolation-property}
L_{p,S}(1-r, \chi) = L_{S}(1-r,\chi\omega^{-r}).
\end{equation}
Using Brauer induction, \eqref{eqn:interpolation-property} can be extended to the case where $\chi$ is non-linear provided $r \geq 2$.
However, if $\chi$ is non-linear and $r=1$,
this argument fails due to the potential presence of trivial zeros.
Nevertheless, it seems plausible that the identity
\begin{equation}\label{eqn:values-padic-complex}
        L_{p,S}(0, \chi) = L_S(0, \chi \omega^{-1})
\end{equation}
holds in general.
As both sides are well-behaved with respect to direct sum, inflation and induction of characters, one can show that
\eqref{eqn:values-padic-complex} does hold when $\chi$ is a monomial character
(also see the discussion in \cite[\S 2]{MR656068}).
From recent work of Burns \cite[Theorem 5.2 (i)]{burns-p-adic} it follows that the left hand side of \eqref{eqn:values-padic-complex}
vanishes whenever the right hand side does.

\subsection{Relation to the framework of \cite{MR2217048}}
We now discuss Conjecture \ref{conj:EIMC} within the framework of the theory of \cite[\S 3]{MR2217048};
this section may be skipped if the reader is only interested in the formulation of \S \ref{subsec:EIMC-reformulation}.
Let
\[
\pi: \mathcal{G} \rightarrow {\mathrm{GL}}_{n}(\mathcal{O})
\]
be a continuous homomorphism, where $\mathcal{O}=\mathcal{O}_{F}$ denotes the ring of integers of $F$
and $n$ is some integer greater or equal to $1$.
There is a ring homomorphism
\begin{equation} \label{eqn:first_Phi}
\Phi_{\pi}: \Lambda(\mathcal{G}) \rightarrow M_{n\times n}(\Lambda^{\mathcal{O}}(\Gamma_{K}))
\end{equation}
induced by the continuous group homomorphism
\begin{eqnarray*}
\mathcal{G} & \rightarrow & (M_{n \times n}(\mathcal{O}) \otimes_{{\mathbb{Z}}_p} \Lambda(\Gamma_{K})){^{\times}} = {\mathrm{GL}}_{n}(\Lambda^{\mathcal{O}}(\Gamma_{K}))\\
\sigma & \mapsto & \pi(\sigma) \otimes \overline{\sigma},
\end{eqnarray*}
where $\overline{\sigma}$ denotes the image of $\sigma$ in $\mathcal{G} / H = \Gamma_{K}$. 
By \cite[Lemma 3.3]{MR2217048} the
homomorphism \eqref{eqn:first_Phi} extends to a ring homomorphism
\[
\Phi_{\pi}: \mathcal{Q}(\mathcal{G}) \rightarrow M_{n\times n}(\mathcal{Q}^{F}(\Gamma_{K}))
\]
and this in turn induces a homomorphism
\[
\Phi_{\pi}': K_{1}(\mathcal{Q}(\mathcal{G})) \rightarrow 
K_{1}(M_{n\times n}(\mathcal{Q}^{F}(\Gamma_{K}))) = \mathcal{Q}^{F}(\Gamma_{K}){^{\times}}.
\]
Let ${\mathrm{aug}}: \Lambda^{\mathcal{O}}(\Gamma_{K}) \twoheadrightarrow \mathcal{O}$ be the augmentation map and put $\mathfrak{p} = \ker({\mathrm{aug}})$.
Writing $\Lambda^{\mathcal{O}}(\Gamma_{K})_{\mathfrak{p}}$ for the localisation of $\Lambda^{\mathcal{O}}(\Gamma_{K})$ at $\mathfrak{p}$, it is clear that ${\mathrm{aug}}$ naturally extends to a homomorphism ${\mathrm{aug}}: \Lambda^{\mathcal{O}}(\Gamma_{K})_{\mathfrak{p}} \rightarrow F$.
One defines an evaluation map
\begin{equation} \label{eqn:evaluation-map}
\begin{array}{rcl}
\phi: \mathcal{Q}^{F}(\Gamma_{K}) & \rightarrow & F \cup \{\infty\}\\
x & \mapsto & \left\{ \begin{array}{ll} {\mathrm{aug}} (x) & \mbox{ if } x \in \Lambda^{\mathcal{O}}(\Gamma_{K})_{\mathfrak{p}}\\
\infty & \mbox{ otherwise}. \end{array} \right.
\end{array}
\end{equation}
It is straightforward to show that for $r \in {\mathbb{Z}}$ we have
\begin{equation}\label{eq:PhiS-jr-p-adic}
\phi(j_{\chi}^{r}(\Phi_{S})) = L_{p,S}(1-r, \chi). 
\end{equation}

If $\zeta$ is an element of $K_{1}(\mathcal{Q}(\mathcal{G}))$, we define $\zeta(\pi)$ to be $\phi(\Phi_{\pi}'(\zeta))$.
Conjecture \ref{conj:EIMC} now implies that there is an element $\zeta_{S} \in K_{1}(\mathcal{Q}(\mathcal{G}))$ such that
$\partial(\zeta_{S}) = -[C_{S}^{\bullet}(\mathcal{L}/K)]$ and for each $r \geq 1$ divisible by $p-1$
and every irreducible Artin representation $\pi_{\chi}$ of $\mathcal{G}$ with character $\chi$ we have
\[
\zeta_{S}(\pi_{\chi}\kappa^{r}) = \phi(j_{\chi}^{r}(\Phi_{S})) = L_{p,S}(1-r, \chi) = L_{S}(1-r,\chi),
\]
where the first equality follows from \cite[Lemma 2.3]{MR2822866} (for the last equality see \S \ref{subsec:interpolation-property}).

\subsection{A maximal order variant of the EIMC}

We shall prove the EIMC in many cases in which the $\mu=0$ hypothesis is not known;
in some of these cases we shall also prove the EIMC with uniqueness.

The following key result of Ritter and Weiss can be seen as a `maximal order variant' of Conjecture \ref{conj:EIMC}; 
crucially, it does not require the $\mu=0$ hypothesis.
We assume the setup and notation of \S \ref{subsec:EIMC-reformulation}.

\begin{theorem}\label{thm:EIMC-MaxOrd}
Let $\mathfrak{M}(\mathcal{G})$ be a maximal ${\mathbb{Z}}_{p}[[\Gamma_{0}]]$-order 
such that $\Lambda(\mathcal{G}) \subseteq \mathfrak{M}(\mathcal{G}) \subseteq \mathcal{Q}(\mathcal{G})$.
Choose $x_{S} \in K_{1}(\mathcal{Q}(\mathcal{G}))$ such that $\partial(x_{S}) = -[C_{S}^{\bullet}(\mathcal{L}/K)]$.
Then ${\mathrm{nr}}(x_{S})\Phi_{S}^{-1} \in \zeta(\mathfrak{M}(\mathcal{G})){^{\times}}$.
\end{theorem}

\begin{proof}
By \cite[Theorem 16]{MR2114937} we know that
${\mathrm{Det}}(x_{S})L_{K,S}^{-1} \in {\mathrm{Hom}}^{\ast}(R_{p}(\mathcal{G}), \Lambda^{c}(\Gamma_{K})^{\times})$,
where $\Lambda^{c}(\Gamma_{K}) := {\mathbb{Z}}_{p}^{c} \otimes_{{\mathbb{Z}}_{p}} \Lambda(\Gamma_{K})$ and ${\mathbb{Z}}_{p}^{c}$ denotes the integral closure
of ${\mathbb{Z}}_{p}$ in ${\mathbb{Q}}_{p}^{c}$.
Moreover, ${\mathrm{Hom}}^{\ast}(R_{p}(\mathcal{G}), \Lambda^{c}(\Gamma_{K})^{\times})$ identifies with $\zeta(\mathfrak{M}(\mathcal{G})){^{\times}}$
under the isomorphism in diagram \eqref{eqn:Det_triangle} as is explained in \cite[Remark H]{MR2114937}. Thus ${\mathrm{nr}}(x_{S}) \Phi_{S}^{-1}$ lies in $\zeta(\mathfrak{M}(\mathcal{G})){^{\times}}$.
\end{proof}

\begin{corollary} \label{cor:EIMC-MaxOrd}
Let $\mathfrak{M}(\mathcal{G})$ be a maximal ${\mathbb{Z}}_{p}[[\Gamma_{0}]]$-order such that 
$\Lambda(\mathcal{G}) \subseteq \mathfrak{M}(\mathcal{G}) \subseteq \mathcal{Q}(\mathcal{G})$
and let $e \in \mathfrak{M}(\mathcal{G})$ be a central idempotent. Suppose that the reduced norm map
\begin{equation} \label{eqn:nr-surjective-hypothesis}
{\mathrm{nr}}: K_{1}(e \mathfrak{M}(\mathcal{G})) \longrightarrow \zeta(e \mathfrak{M}(\mathcal{G})){^{\times}}
\end{equation}
is surjective.
Then there exists $y_{S} \in K_{1}(e \mathcal{Q}(\mathcal{G}))$ such that ${\mathrm{nr}}(y_{S}) = e \Phi_{S}$
and $y_{S}$ maps to $[e \mathfrak{M}(\mathcal{G}) \otimes^{\mathbb{L}}_{\Lambda(\mathcal{G})} C_{S}^{\bullet}(\mathcal{L}/K)]$ under
$K_{1}(e \mathcal{Q}(\mathcal{G})) \rightarrow K_{0}(e \mathfrak{M}(\mathcal{G}), e \mathcal{Q}(\mathcal{G}))$.
\end{corollary}

\begin{proof}
Choose $x_{S} \in K_{1}(\mathcal{Q}(\mathcal{G}))$ as in Theorem \ref{thm:EIMC-MaxOrd}. By assumption, there is
$z_{S} \in K_{1}(e \mathfrak{M}(\mathcal{G}))$ such that ${\mathrm{nr}}(z_{S}) = e {\mathrm{nr}}(x_{S}) \Phi_{S}^{-1}$.
Let $z'_{S}$ be the image of $z_{S}$ in $K_{1}(e \mathcal{Q}(\mathcal{G}))$.
 Then $y_{S} := e x_{S} (z'_{S})^{-1}$ has the desired properties.
\end{proof}

\begin{remark}\label{rmk:nr-surjective}
It is not clear whether the map \eqref{eqn:nr-surjective-hypothesis} is always surjective. This map is surjective if no skewfields
occur in the Wedderburn decomposition of $e \mathcal{Q}(\mathcal{G})$, and thus one can always take $e = e_{\mathcal{G}'}$,
where $\mathcal{G'}$ is the commutator subgroup of $\mathcal{G}$ (note that $\mathcal{G}' \leq H$).
If $p$ does not divide the order of $\mathcal{G}'$, then by Corollary \ref{cor:no-skewfields}
one can take an arbitrary $e$ (in particular, $e=1$ is possible).
If $\Lambda(\mathcal{G})$ is $N$-hybrid then Corollary \ref{cor:hybrid-implies-nr-surjective} shows that one can take $e=1-e_{N}$.
\end{remark}

\begin{theorem}\label{thm:EIMC-for-p-not-dividing-ordH}
Let $\mathcal{L}/K$ be an admissible one-dimensional $p$-adic Lie extension with Galois group $\mathcal{G} \simeq  H \rtimes \Gamma$.
If $p \nmid |H|$ then the EIMC with uniqueness holds for $\mathcal{L}/K$.
\end{theorem}

\begin{remark}
Theorem \ref{thm:EIMC-for-p-not-dividing-ordH} does not require the $\mu=0$ hypothesis.
The statement is the same as that of \cite[Example 2]{MR2205173},
except that the proof of this latter result does require the $\mu=0$ hypothesis which is not explicitly stated (see \cite[p.\ 48]{MR2242618}).
\end{remark}

\begin{proof}[Proof of Theorem \ref{thm:EIMC-for-p-not-dividing-ordH}]
Since the commutator subgroup $\mathcal{G}'$ is a subgroup of $H$,
Corollary \ref{cor:no-skewfields} shows that the Wedderburn decomposition of $\mathcal{Q}(\mathcal{G})$ contains no skewfields.
This has two consequences. First, uniqueness follows from Remark \ref{rmk:SK1}.
Second, by Remark \ref{rmk:nr-surjective} the hypotheses of Corollary \ref{cor:EIMC-MaxOrd} are satisfied for every choice of $e$.
However, Remark \ref{rmk:p-not-divide-H-max-order} shows that $\Lambda(\mathcal{G})$ is in fact a maximal order since $p \nmid |H|$.
Therefore the EIMC for $\mathcal{L}/K$ follows from Corollary \ref{cor:EIMC-MaxOrd} with $e=1$.
\end{proof}

\subsection{Functorialities}
Let $\mathcal{L}/K$ be an admissible one-dimensional $p$-adic Lie extension with Galois group $\mathcal{G}$.
Let $N$ be a finite normal subgroup of $\mathcal{G}$ and let $\mathcal{H}$ be an open subgroup of $\mathcal{G}$.
There are canonical maps
    \begin{eqnarray*}
        {\mathrm{quot}}^{\mathcal{G}}_{\mathcal{G}/N}: & K_{0}(\Lambda(\mathcal{G}), \mathcal{Q}(\mathcal{G})) \longrightarrow &
            K_{0}(\Lambda(\mathcal{G}/N), \mathcal{Q}(\mathcal{G}/N)),\\
            {\mathrm{res}}^{\mathcal{G}}_{\mathcal{H}}: & K_{0}(\Lambda(\mathcal{G}), \mathcal{Q}(\mathcal{G})) \longrightarrow &
            K_{0}(\Lambda(\mathcal{H}), \mathcal{Q}(\mathcal{H}))
    \end{eqnarray*}
    induced from scalar extension along $\Lambda(\mathcal{G}) \longrightarrow \Lambda(\mathcal{G}/N)$ and restriction of scalars
    along $\Lambda(\mathcal{H}) \hookrightarrow \Lambda(\mathcal{G})$. Similarly, we have maps (see \cite[\S 3]{MR2114937})
    \begin{eqnarray*}
        {\mathrm{quot}}^{\mathcal{G}}_{\mathcal{G}/N}: & {\mathrm{Hom}}^{\ast}(R_{p}(\mathcal{G}), \mathcal{Q}^{c}(\Gamma_{K}){^{\times}}) \longrightarrow &
            {\mathrm{Hom}}^{\ast}(R_{p}(\mathcal{G}/N), \mathcal{Q}^{c}(\Gamma_{K}){^{\times}}),\\
        {\mathrm{res}}^{\mathcal{G}}_{\mathcal{H}}: & {\mathrm{Hom}}^{\ast}(R_{p}(\mathcal{G}), \mathcal{Q}^{c}(\Gamma_{K}){^{\times}}) \longrightarrow &
            {\mathrm{Hom}}^{\ast}(R_{p}(\mathcal{H}), \mathcal{Q}^{c}(\Gamma_{K'}){^{\times}}),
    \end{eqnarray*}
    where $K' := \mathcal{L}^{\mathcal{H}}$; here for $f \in {\mathrm{Hom}}^{\ast}(R_{p}(\mathcal{G}), \mathcal{Q}^{c}(\Gamma_{K}){^{\times}})$
    we have $({\mathrm{quot}}^{\mathcal{G}}_{\mathcal{G}/N} f)(\chi) = f({\mathrm{infl}}^{\mathcal{G}}_{\mathcal{G}/N} \chi)$ and
    $({\mathrm{res}}^{\mathcal{G}}_{\mathcal{H}} f)(\chi') = f({\mathrm{ind}}^{\mathcal{G}}_{\mathcal{H}} \chi')$ for $\chi \in R_{p}(\mathcal{G}/N)$
    and $\chi' \in R_{p}(\mathcal{H})$. Then diagram \eqref{eqn:Det_triangle} induces canonical maps
    \begin{eqnarray*}
        {\mathrm{quot}}^{\mathcal{G}}_{\mathcal{G}/N}: & \zeta(\mathcal{Q}(\mathcal{G})){^{\times}} \longrightarrow &
            \zeta(\mathcal{Q}(\mathcal{G}/N)){^{\times}},\\
        {\mathrm{res}}^{\mathcal{G}}_{\mathcal{H}}: & \zeta(\mathcal{Q}(\mathcal{G}))^{\times} \longrightarrow &
            \zeta(\mathcal{Q}(\mathcal{H})){^{\times}}.
    \end{eqnarray*}
The first map is easily seen to be induced by the canonical projection $\mathcal{G} \twoheadrightarrow \mathcal{G}/N$.

The following proposition is an obvious reformulation of \cite[Proposition 12]{MR2114937}
(note that the proof of \cite[Proposition 12 (1)(a)]{MR2114937} uses a result which assumes Leopoldt's conjecture;
a direct proof without this assumption is given in \cite[Appendix]{MR2813337}); also see
\cite[Proposition 1.6.5]{MR2276851}.

\begin{prop} \label{prop:funtorialities}
Let $\mathcal{L}/K$ be an admissible one-dimensional $p$-adic Lie extension with Galois group $\mathcal{G}$.
Then the following statements hold.
\begin{enumerate}
\item
Let $N$ be a finite normal subgroup of $\mathcal{G}$ and put $\mathcal{L}' := \mathcal{L}^{N}$.
Then
\[
{\mathrm{quot}}^{\mathcal{G}}_{\mathcal{G}/N}([C_{S}^{\bullet}(\mathcal{L}/K)]) = [C_{S}^{\bullet}(\mathcal{L}'/K)], \quad
{\mathrm{quot}}^{\mathcal{G}}_{\mathcal{G}/N}(\Phi_{S}(\mathcal{L}/K)) = \Phi_{S}(\mathcal{L}'/K).
\]
In particular, if the EIMC holds for $\mathcal{L}/K$, then it holds for $\mathcal{L}'/K$.
\item
Let $\mathcal{H}$ be an open subgroup of $\mathcal{G}$ and put $K' := \mathcal{L}^{\mathcal{H}}$. Then
\[
{\mathrm{res}}^{\mathcal{G}}_{\mathcal{H}}([C_{S}^{\bullet}(\mathcal{L}/K)]) = [C_{S}^{\bullet}(\mathcal{L}/K')], \quad
{\mathrm{res}}^{\mathcal{G}}_{\mathcal{H}}(\Phi_{S}(\mathcal{L}/K)) = \Phi_{S}(\mathcal{L}/K').
\]
In particular, if the EIMC holds for $\mathcal{L}/K$, then it holds for $\mathcal{L}/K'$.
\end{enumerate}
\end{prop}

\subsection{The EIMC over hybrid Iwasawa algebras}
We show how hybrid Iwasawa algebras can be used to `break up' certain cases of the EIMC.

\begin{theorem}\label{thm:EIMC-break-down}
Let $\mathcal{L}/K$ be an admissible one-dimensional $p$-adic Lie extension with Galois group $\mathcal{G}$.
Suppose that $\Lambda(\mathcal{G})$ is $N$-hybrid for some
finite normal subgroup $N$ of $\mathcal{G}$.
Let $\overline{\mathcal{P}}$ be a Sylow $p$-subgroup of $\overline{\mathcal{G}}:={\mathrm{Gal}}(\mathcal{L}^{N}/K) \simeq \mathcal{G}/N$.
Then the following statements hold.
\begin{enumerate}
\item The EIMC holds for $\mathcal{L}/K$ if and only if it holds for $\mathcal{L}^{N}/K$.
\item The EIMC with uniqueness holds for $\mathcal{L}/K$ if and only if it holds for $\mathcal{L}^{N}/K$.
\item If $(\mathcal{L}^{N})^{\overline{\mathcal{P}}} / {\mathbb{Q}}$ is abelian, then the EIMC holds for $\mathcal{L}/K$.
\item If $\mathcal{L}^{N} / {\mathbb{Q}}$ is abelian, then the EIMC with uniqueness holds for $\mathcal{L}/K$.
\end{enumerate}
\end{theorem}

\begin{remark}
Under the hypotheses in (iii), $\overline{\mathcal{P}}$ is necessarily normal in $\overline{\mathcal{G}}$
and thus is its unique Sylow $p$-subgroup.
\end{remark}

\begin{proof}[Proof of Theorem \ref{thm:EIMC-break-down}]
By assumption $\Lambda(\mathcal{G})$ decomposes into $\Lambda(\mathcal{G}) e_{N} \oplus \mathfrak{M}(\mathcal{G}) (1 - e_{N})$
for some maximal order $\mathfrak{M}(\mathcal{G})$. This induces a decomposition of relative $K$-groups
\begin{eqnarray*}
K_{0}(\Lambda(\mathcal{G}), \mathcal{Q}(\mathcal{G}))
& \simeq & K_{0}(\Lambda(\mathcal{G}) e_{N}, \mathcal{Q}(\mathcal{G}) e_{N}) \times K_{0}(\mathfrak{M}(\mathcal{G}) (1 - e_{N}), \mathcal{Q}(\mathcal{G}) (1-e_{N}))\\
 & \simeq & K_{0}(\Lambda(\mathcal{G}/N), \mathcal{Q}(\mathcal{G}/N)) \times K_{0}(\mathfrak{M}(\mathcal{G}) (1 - e_{N}), \mathcal{Q}(\mathcal{G}) (1-e_{N}))
\end{eqnarray*}
which maps $[C_{S}^{\bullet}(\mathcal{L}/K)]$ to the pair
$([C_{S}^{\bullet}(\mathcal{L}^{N}/K)], [\mathfrak{M}(\mathcal{G}) (1 - e_{N}) \otimes_{\Lambda(\mathcal{G})}^{\mathbb{L}} C_{S}^{\bullet}(\mathcal{L}/K)])$
by Proposition \ref{prop:funtorialities} (i).
Similarly, we have a decomposition
\begin{eqnarray*}
\zeta(\mathcal{Q}(\mathcal{G}))^{\times} & \simeq & \zeta(\mathcal{Q}(\mathcal{G}) e_{N})^{\times} \times \zeta(\mathcal{Q}(\mathcal{G})(1-e_{N}))^{\times} \\
& \simeq & \zeta(\mathcal{Q}(\mathcal{G}/N))^{\times} \times \zeta(\mathcal{Q}(\mathcal{G})(1-e_{N}))^{\times}
\end{eqnarray*}
which maps $\Phi_{S}(\mathcal{L}/K)$ to the pair $(\Phi_{S}(\mathcal{L}^{N}/K), \Phi_{S}(\mathcal{L}/K) (1-e_{N}))$.
Hence (i) follows from Corollary \ref{cor:EIMC-MaxOrd} and Remark \ref{rmk:nr-surjective}.

Part (ii) follows from (i) once
\[
SK_{1}((1-e_{N})\mathcal{Q}(\mathcal{G})) := 
\ker({\mathrm{nr}}: K_{1}((1-e_{N})\mathcal{Q}(\mathcal{G})) \longrightarrow \zeta((1-e_{N})\mathcal{Q}(\mathcal{G}))^{\times})
\]
is shown to vanish.
This is indeed the case since Corollary \ref{cor:hybrid-implies-nr-surjective} implies that $\mathcal{Q}(\mathcal{G})(1-e_{N})$ is a direct product
of matrix rings over fields.
Part (iii) follows from part (i) and Corollary \ref{cor:EIMC-unconditional}.
Finally, part (iv) follows from parts (ii) and (iii) and Remark \ref{rmk:SK1}.
\end{proof}

The following theorem is useful in applications to proving results about finite Galois extensions of number fields.

\begin{theorem}\label{thm:EIMC-start-with-num-fields}
Let $L/K$ be a finite Galois extension of totally real number fields with Galois group $G$.
Let $p$ be an odd prime and let $L_{\infty}$ be the cyclotomic ${\mathbb{Z}}_{p}$-extension of $L$.
Let $P$ be a Sylow $p$-subgroup of $G$. Then the following statements hold.
\begin{enumerate}
\item $L_{\infty}/K$ is an admissible one-dimensional $p$-adic Lie extension.
\item If $p \nmid |G|$ then the EIMC with uniqueness holds for $L_{\infty}/K$.
\item If $L^{P}/{\mathbb{Q}}$ is abelian then the EIMC holds for $L_{\infty}/K$.
\end{enumerate}
Suppose further that ${\mathbb{Z}}_{p}[G]$ is $N$-hybrid.
Let $\overline{P}$ be a Sylow $p$-subgroup of $\overline{G}:={\mathrm{Gal}}(L^{N}/K) \simeq G/N$.
Then the following statements hold.
\begin{enumerate}
\setcounter{enumi}{3}
\item If $L^{N}/{\mathbb{Q}}$ is abelian then the EIMC with uniqueness holds for $L_{\infty}/K$.
\item If $(L^{N})^{\overline{P}}/{\mathbb{Q}}$ is abelian then the EIMC holds for $L_{\infty}/K$.
\end{enumerate}
\end{theorem}

\begin{remark}
Under the hypothesis in (iii), $P$ is necessarily normal in $G$ and thus is its unique
Sylow $p$-subgroup. In part (v), we have $(L^{N})^{\overline{P}}=L^{NP}=L^{N \rtimes P}$.
\end{remark}

\begin{proof}[Proof of Theorem \ref{thm:EIMC-start-with-num-fields}]
For part (i) it is trivial to check that the conditions of Definition \ref{def:one-dim-adm} are satisfied.
We adopt the setup and notation of \S \ref{subsec:algebras-in-Iwasawa-thy}.
Since $H$ identifies with a subgroup  of $G$, part (ii) follows from Theorem \ref{thm:EIMC-for-p-not-dividing-ordH}
(in fact $H$ identifies with $G$ in this case.)
Let $\mathcal{P}$ be a Sylow $p$-subgroup of $\mathcal{G}$.
Since $\Gamma_{L}:={\mathrm{Gal}}(L_{\infty}/L)$ is an open normal pro-$p$ subgroup of $\mathcal{G}$
we have $\Gamma_{L} \leq \mathcal{P}$ and $\mathcal{P}$ maps to $P$ under the natural 
projection $\mathcal{G} \rightarrow \mathcal{G}/\Gamma_{L} \simeq G$.
Hence $L^{P}=L_{\infty}^{\mathcal{P}}$ and so part (iii) follows from Corollary \ref{cor:EIMC-unconditional}.

Now suppose that ${\mathbb{Z}}_{p}[G]$ is $N$-hybrid.
Proposition \ref{prop:hybrid-codescent} says that $N$ identifies with a normal subgroup of $\mathcal{G}$ which is also a normal subgroup of $H$;
moreover, both ${\mathbb{Z}}_{p}[H]$ and $\Lambda(\mathcal{G})$ are $N$-hybrid. 
Note that $(L_{\infty})^{N}=(L^{N})_{\infty}$.
If $L^{N}/{\mathbb{Q}}$ is abelian then $L_{\infty}^{N}/{\mathbb{Q}}$ is also abelian and so part (iv) follows from 
Theorem \ref{thm:EIMC-break-down} (iv).
Part (v) now follows from (iii) and Theorem \ref{thm:EIMC-break-down} (i).
\end{proof}

\begin{corollary}\label{cor:EIMC-Frobenius}
Let $L/K$ be a finite Galois extension of totally real number fields with Galois group $G$.
Suppose that $G = N \rtimes V$ is a Frobenius group with Frobenius kernel $N$ and abelian Frobenius complement $V$.
Further suppose that $L^{N}/{\mathbb{Q}}$ is abelian (in particular, this is the case when $K={\mathbb{Q}}$).
Let $p$ be an odd prime and let $L_{\infty}$ be the cyclotomic ${\mathbb{Z}}_{p}$-extension of $L$.
Then the following statements hold.
\begin{enumerate}
\item If $p \nmid |N|$ then the EIMC with uniqueness holds for $L_{\infty}/K$.
\item If $N$ is a $p$-group then the EIMC holds for $L_{\infty}/K$.
\item If $N$ is an $\ell$-group for any prime $\ell$ then the EIMC holds for $L_{\infty}/K$.\\
(This includes the cases $\ell=2$ and $\ell=p$.)
\end{enumerate}
In particular, (iii) holds in the following cases:
\begin{itemize}
\item $G \simeq {\mathrm{Aff}}(q)$, where $q$ is a prime power (see Example \ref{ex:affine}),
\item $G \simeq C_{\ell} \rtimes C_{q}$, where $q<\ell$ are distinct primes such that $q \mid (\ell-1)$ and $C_{q}$ acts on $C_{\ell}$ via an embedding $C_{q} \hookrightarrow {\mathrm{Aut}}(C_{\ell})$ (see Example \ref{ex:metacyclic}),
\item $G$ is isomorphic to any of the Frobenius groups constructed in Example \ref{ex:non-abelian-kernel}.
\end{itemize}
\end{corollary}

\begin{proof}
Part (i) follows from Proposition \ref{prop:frob-N-hybrid} and Theorem \ref{thm:EIMC-start-with-num-fields} (iv).
Part (ii) follows from  Theorem \ref{thm:EIMC-start-with-num-fields} (iii) with $P=N$.
Part (iii) is just the combination of (i) and (ii).
\end{proof}

\begin{remark}
Among all groups of order $\leq 10^{6}$ there are $568,220$ metabelian Frobenius groups
(see \cite[Remark 11.13 (A)]{MR1600514}), which in particular have abelian Frobenius complement
and so satisfy the hypotheses of Corollary \ref{cor:EIMC-Frobenius}.
\end{remark}

\begin{example} \label{ex:EIMC-dicyclic}
Let $p$ be an odd prime.
Let $V = {\mathrm{Dic}}_{p}$ be dicyclic of order $4p$ and $G = N \rtimes V$ be a Frobenius group as in Example \ref{ex:dicyclic-complement}.
Let $L/K$ be a finite Galois extension of totally real number fields with Galois group ${\mathrm{Gal}}(L/K) \simeq G$.
The unique Sylow $p$-subgroup $P$ of ${\mathrm{Dic}}_{p}$ coincides with the commutator subgroup
and thus $L^{N \rtimes P}/K$ is cyclic of order $4$.
If we further assume that $L^{N \rtimes P}/{\mathbb{Q}}$ is abelian (for instance, assume that $K={\mathbb{Q}}$),
then Theorem \ref{thm:EIMC-start-with-num-fields} (v) implies that the EIMC holds for $L_{\infty}/K$,
where $L_{\infty}$ is the cyclotomic ${\mathbb{Z}}_{p}$-extension of $L$.
Note that this cannot be deduced from Corollary \ref{cor:EIMC-Frobenius}
because although $G$ is a Frobenius group, it has a non-abelian Frobenius complement.
\end{example}

\begin{example} \label{ex:EIMC-modified-affine}
Let $q=\ell^{n}$ be a prime power and consider the group ${\mathbb{F}}_{q} \rtimes ({\mathbb{F}}_{q}^{\times} \rtimes \langle \phi \rangle)$
of Example \ref{ex:affine-Frobenius}. Let $p \mid (q-1)$ be an odd prime that does not divide $n$.
Then the group $\mu_{p}({\mathbb{F}}_{q})$ of $p$-power roots of unity in ${\mathbb{F}}_{q}$ is non-trivial and we put
$G := {\mathbb{F}}_{q} \rtimes (\mu_{p}({\mathbb{F}}_{q}) \rtimes \langle \phi \rangle)$ and $U := {\mathbb{F}}_{q} \rtimes \mu_{p}({\mathbb{F}}_{q}) \unlhd G$.
Then Example \ref{ex:affine-Frobenius} and Proposition \ref{prop:hybrid-basechange-up} imply that ${\mathbb{Z}}_{p}[G]$ is
${\mathbb{F}}_{q}$-hybrid.
Now assume that $L/K$ is a Galois extension of totally real number fields with ${\mathrm{Gal}}(L/K) \simeq G$
and let $L_{\infty}$ be the cyclotomic ${\mathbb{Z}}_{p}$-extension of $L$.
If $L^{U}/{\mathbb{Q}}$ is abelian (for instance if $K={\mathbb{Q}}$), then Theorem \ref{thm:EIMC-start-with-num-fields} (v) implies that the EIMC holds
for $L_{\infty}/K$.
\end{example}

\begin{example}\label{ex:EIMC-S4}
Let $L/K$ be a finite Galois extension of totally real number fields with Galois group ${\mathrm{Gal}}(L/K) \simeq S_{4}$.
Let $p$ be an odd prime and let $L_{\infty}$ be the cyclotomic ${\mathbb{Z}}_{p}$-extension of $L$.
Then Theorem \ref{thm:EIMC-start-with-num-fields} (ii) shows that  the EIMC with uniqueness holds for $L_{\infty}/K$ when $p>3$.
Now further assume that $L^{A_{4}}/{\mathbb{Q}}$ is abelian (in particular, this is the case when $K={\mathbb{Q}}$) and consider the case $p=3$.
The group ring ${\mathbb{Z}}_{3}[S_{4}]$ is $V_{4}$-hybrid as shown in Example \ref{ex:S4-A4-V4}.
Moreover, the Sylow $3$-subgroup of $S_{4}/V_{4} \simeq S_{3}$ is $A_{3} \simeq C_{3}$ and we have
$(L^{V_{4}})^{A_{3}}=L^{A_{4}}$, so the EIMC for $L_{\infty}/K$ follows from Theorem \ref{thm:EIMC-start-with-num-fields} (v).
Let $\mathcal{G}={\mathrm{Gal}}(L_{\infty}/K)$.
Then as shown in Example \ref{ex:S4-V4-II} we have $\mathcal{G} = S_{4} \times \Gamma_{K}$,
and so no skewfields occur in the Wedderburn decomposition of $\mathcal{Q}(\mathcal{G})$.
Therefore, the EIMC with uniqueness also holds for $L_{\infty}/K$ when $p=3$.
\end{example}

\begin{corollary}\label{cor:pth-root-base-change}
Let $L/K$ be a finite Galois extension of totally real number fields with Galois group $G$.
Let $p$ be an odd prime and let $L_{\infty}$ be the cyclotomic ${\mathbb{Z}}_{p}$-extension of $L$.
Let $N$ be a normal subgroup of $G$ and let $\overline{P}$ be a Sylow $p$-subgroup of $\overline{G}:={\mathrm{Gal}}(L^{N}/K) \simeq G/N$.
Suppose that ${\mathbb{Z}}_{p}[G]$ is $N$-hybrid and that $(L^{N})^{\overline{P}}/{\mathbb{Q}}$ is abelian.
Let $K'/K$ be a field extension such that $K'$ is totally real, $K'/{\mathbb{Q}}$ is abelian and $p \nmid [K':K] < \infty$.
Let $L'_{\infty}=L_{\infty}K'$.
Then the EIMC holds for both $L_{\infty}/K$ and $L_{\infty}'/K$.
\end{corollary}

\begin{remark}\label{rmk:base-change-in-particular}
The hypothesis that $(L^{N})^{\overline{P}}/{\mathbb{Q}}$ is abelian forces $K/{\mathbb{Q}}$ to be abelian, and thus one can
take $K'$ to be the compositum of $K$ with another finite abelian extension $K''/{\mathbb{Q}}$ such that $p \nmid [K'':{\mathbb{Q}}]$.
In particular, Corollary \ref{cor:pth-root-base-change} can be applied with $K'=K(\zeta_{p})^{+}$
and $L_{\infty}'=L_{\infty}(\zeta_{p})^{+}$ (here $F^{+}$ denotes the maximal totally real subfield of $F$).
\end{remark}

\begin{proof}[Proof of Corollary \ref{cor:pth-root-base-change}]
The EIMC holds for $L_{\infty}/K$ by Theorem \ref{thm:EIMC-start-with-num-fields} (v).
Let $F=L^{N}$ and put $F'=FK'$ and $L'=LK'$.
Let $G'={\mathrm{Gal}}(L'/K)$ and $N'={\mathrm{Gal}}(L'/F')$.
Then ${\mathbb{Z}}_{p}[G']$ is $N'$-hybrid by Remark \ref{rmk:hybrid-Galois-basechange}.
Let $\overline{P}'$ be a Sylow $p$-subgroup of $\overline{G'}:=G'/N'$.
Then
\[
((L')^{N'})^{\overline{P}'} = (F')^{\overline{P}'} = F^{\overline{P}} K'= (L^{N})^{\overline{P}}K',
\]
which is an abelian extension of ${\mathbb{Q}}$
as it is the compositum of two such extensions.
Therefore the EIMC holds for $L_{\infty}'/K$ by Theorem \ref{thm:EIMC-start-with-num-fields} (v).
\end{proof}

\subsection{Remarks on the higher dimension case} \label{subsec:higher-rk}
In \cite{MR3091976}, Kakde proved a more general version of the EIMC for admissible
$p$-adic Lie extensions of arbitrary (finite) dimension under a suitable version of the $\mu=0$ hypothesis.
This used a strategy of Burns and Kato to reduce the proof to the one-dimensional case discussed above (see \cite{burns-mc}).
We briefly discuss some of the obstacles to generalising the approach of this article to prove higher dimension cases
of the EIMC when a suitable $\mu=0$ hypothesis is not known.
The most serious is that a certain `$\mathfrak{M}_{H}(G)$-conjecture' is required to even formulate
the higher dimension version of the EIMC, and that this is presently only known to hold under a suitable $\mu=0$ hypothesis
(see \cite[p.\ 5]{zbMATH06148870} and \cite{MR2905532}).
Another problem is that a higher dimension version of Theorem \ref{thm:EIMC-MaxOrd} (the `maximal order variant of the EIMC')
has not been proven unconditionally.
Finally, it is not entirely clear how the notion of a hybrid Iwasawa algebra generalises to the higher dimension case.

\section{Noncommutative Fitting invariants} \label{sec:noncommFitt}

\subsection{Denominator ideals}\label{subsec:denom-ideals}
Let $R$ be a noetherian integrally closed domain with field of fractions $E$.
Let $A$ be a finite-dimensional separable $E$-algebra and let $\mathfrak{A}$ be an $R$-order in $A$.
We choose a maximal order $\mathfrak{A}'$ containing $\mathfrak{A}$.
Following \cite[\S 3.6]{MR3092262}, for every matrix $H \in M_{b \times b} (\mathfrak{A})$ there is a generalised adjoint matrix
$H^{\ast} \in M_{b\times b}(\mathfrak{A}')$ such that $H^{\ast} H = H H^{\ast} = {\mathrm{nr}} (H) \cdot 1_{b \times b}$
(note that the conventions in \cite[\S 3.6]{MR3092262} slightly differ from those in  \cite{MR2609173}).
If $\tilde{H} \in M_{b \times b} (\mathfrak{A})$ is a second matrix, then $(H \tilde{H})^{\ast} = \tilde{H}^{\ast} H^{\ast}$.
We define
\begin{eqnarray*}
\mathcal{H}(\mathfrak{A}) & := &
\{ x \in \zeta(\mathfrak{A}) \mid xH^{\ast} \in M_{n \times n}(\mathfrak{A}) \, \forall H \in M_{n \times n}(\mathfrak{A}) \, \forall n \in {\mathbb{N}} \},\\
\mathcal{I}(\mathfrak{A}) & := &
\langle {\mathrm{nr}}(H) \mid H \in  M_{n \times n}(\mathfrak{A}), \,  n \in {\mathbb{N}}\rangle_{\zeta(\mathfrak{A})}.
\end{eqnarray*}
Since $x \cdot {\mathrm{nr}}(H) = x H^{\ast} H$, in particular we have
\[
\mathcal{H}(\mathfrak{A}) \cdot \mathcal I(\mathfrak{A}) = \mathcal{H}(\mathfrak{A}) \subseteq \zeta(\mathfrak{A}).
\]
Hence $\mathcal{H}(\mathfrak{A})$ is an ideal in the commutative $R$-order $\mathcal{I}(\mathfrak{A})$.
We will refer to $\mathcal{H}(\mathfrak{A})$ as the \emph{denominator ideal} of the $R$-order $\mathfrak{A}$.
If $p$ is a prime and $G$ is a finite group, we put
\begin{eqnarray*}
\mathcal{I}(G) := \mathcal{I}({\mathbb{Z}}[G]), & &  \mathcal{I}_{p}(G) := \mathcal{I}({\mathbb{Z}}_{p}[G]), \\
\mathcal{H}(G) := \mathcal{H}({\mathbb{Z}}[G]), & &  \mathcal{H}_{p}(G) := \mathcal{H}({\mathbb{Z}}_{p}[G]).
\end{eqnarray*}
The importance of the $\zeta(\mathfrak{A})$-module $\mathcal{H}(\mathfrak{A})$
comes from its relation to noncommutative Fitting invariants, which we introduce now.

\subsection{Noncommutative Fitting invariants}
For further details on the following material we refer the reader to \cite{MR2609173} and \cite{MR3092262}.
Let $A$ be a finite-dimensional separable algebra over a field $E$ and $\mathfrak{A}$ be an $R$-order in $A$, where $R$ is an integrally closed complete commutative noetherian local domain with field of fractions $E$.
For example, if $p$ is a prime we can take $\mathfrak{A}$ to be a $p$-adic group ring ${\mathbb{Z}}_{p}[G]$
where $G$ is a finite group or to be a completed group algebra ${\mathbb{Z}}_{p}[[\mathcal{G}]]$ where $\mathcal{G}$
is a one-dimensional $p$-adic Lie group.

Let $N$ and $M$ be two $\zeta(\mathfrak{A})$-submodules of an $R$-torsionfree $\zeta(\mathfrak{A})$-module.
Then $N$ and $M$ are called \emph{${\mathrm{nr}}(\mathfrak{A})$-equivalent} if there exists a positive integer $n$ and a matrix $U \in {\mathrm{GL}}_{n}(\mathfrak{A})$
such that $N = {\mathrm{nr}}(U) \cdot M$.
We denote the corresponding equivalence class by $[N]_{{\mathrm{nr}}(\mathfrak{A})}$.
We say that $N$ is
${\mathrm{nr}}(\mathfrak{A})$-contained in $M$ (and write $[N]_{{\mathrm{nr}}(\mathfrak{A})} \subseteq [M]_{{\mathrm{nr}}(\mathfrak{A})}$)
if for all $N' \in [N]_{{\mathrm{nr}}(\mathfrak{A})}$ there exists $M' \in [M]_{{\mathrm{nr}}(\mathfrak{A})}$
such that $N' \subseteq M'$. Note that it suffices to check this property for one $N_{0} \in [N]_{{\mathrm{nr}}(\mathfrak{A})}$.
    
We will say that $x$ is contained in $[N]_{{\mathrm{nr}}(\mathfrak{A})}$ (and write $x \in [N]_{{\mathrm{nr}}(\mathfrak{A})}$) if there is $N_{0} \in [N]_{{\mathrm{nr}}(\mathfrak{A})}$ such that $x \in N_{0}$.

Now let $M$ be a (left) $\mathfrak{A}$-module with finite presentation
\begin{equation} \label{eqn:finite_presentation}
\mathfrak{A}^a \stackrel{h}{\longrightarrow} \mathfrak{A}^b \longrightarrow M \longrightarrow 0.
\end{equation}
We identify the homomorphism $h$ with the corresponding matrix in $M_{a \times b}(\mathfrak{A})$ and define
$S(h) = S_b(h)$ to be the set of all $b \times b$ submatrices of $h$ if $a \geq b$. In the case $a=b$
we call \eqref{eqn:finite_presentation} a \emph{quadratic presentation}.
The \emph{Fitting invariant} of $h$ over $\mathfrak{A}$ is defined to be
\[
{\mathrm{Fitt}}_{\mathfrak{A}}(h) = \left\{ \begin{array}{lll} [0]_{{\mathrm{nr}}(\mathfrak{A})} & \mbox{ if } & a<b \\
\left[\langle {\mathrm{nr}}(H) \mid H \in S(h)\rangle_{\zeta(\mathfrak{A})}\right]_{{\mathrm{nr}}(\mathfrak{A})} & \mbox{ if } & a \geq b. \end{array} \right.
\]
We call ${\mathrm{Fitt}}_{\mathfrak{A}}(h)$ a Fitting invariant of $M$ over $\mathfrak{A}$.
One defines ${\mathrm{Fitt}}_{\mathfrak{A}}^{\max}(M)$ to be the unique Fitting invariant of $M$ over $\mathfrak{A}$ which is maximal among all Fitting invariants of $M$ with respect to the partial order ``$\subseteq$''.
If $M$ admits a quadratic presentation $h$,
we set 
\begin{equation}\label{eq:quad-fitt-defn}
{\mathrm{Fitt}}_{\mathfrak{A}}(M) := {\mathrm{Fitt}}_{\mathfrak{A}}(h),
\end{equation}
which can be shown to be independent of the chosen quadratic presentation.

Now let $C^{\bullet} \in \mathcal{D}^{\mathrm{perf}}{_{\mathrm{tor}}}(\mathfrak{A})$ and recall from \S \ref{subsec:K-theory}
that $C^{\bullet}$ defines an element $[C^{\bullet}]$ in the relative algebraic $K$-group $K_{0}(\mathfrak{A},A)$.
Recall the long exact sequence of $K$-theory \eqref{eqn:long-exact-seq}.
If $\rho([C^{\bullet}])=0$ (this is necessarily the case in the situation of \eqref{eqn:Iwasawa-K-sequence}, 
for example), we choose  $x \in K_{1}(A)$ such that $\partial(x) = [C^{\bullet}]$ and define
\begin{equation}\label{eqn:fitt-of-complex}
{\mathrm{Fitt}}_{\mathfrak{A}}(C^{\bullet}) := \left[\langle {\mathrm{nr}}(x) \rangle_{\zeta(\mathfrak{A})}\right]_{{\mathrm{nr}}(\mathfrak{A})}.
\end{equation}
Note that this is well-defined by the exactness of \eqref{eqn:long-exact-seq}.
Let $C^{\bullet}_{i} \in \mathcal{D}^{\mathrm{perf}}{_{\mathrm{tor}}}(\mathfrak{A})$ such that $\rho([C_{i}^{\bullet}])=0$ for $i=1,2,3$.
Then if $[C_{2}^{\bullet}] = [C_{1}^{\bullet}] + [C_{3}^{\bullet}]$ in $K_{0}(\mathfrak{A},A)$
(this is the case in the situation of  \eqref{eq:SES-of-complexes}, for example)
it is straightforward to show that 
\begin{equation}\label{eqn:fitt-of-sum-of-complexes-in-rel-K-zero}
{\mathrm{Fitt}}_{\mathfrak{A}}(C_{2}^{\bullet}) = {\mathrm{Fitt}}_{\mathfrak{A}}(C_{1}^{\bullet}) \cdot {\mathrm{Fitt}}_{\mathfrak{A}}(C_{3}^{\bullet}).
\end{equation}
To put this in context, 
we note that if $C^{\bullet}$ is isomorphic in $\mathcal{D}(\mathfrak{A})$ to a complex $P^{-1} \rightarrow P^{0}$ concentrated in
degree $-1$ and $0$ such that $P^{-1}$ and $P^{0}$ are both finitely generated $R$-torsion $\mathfrak{A}$-modules
of projective dimension at most one, then
\begin{equation}\label{eq:rel-fitt-eq}
{\mathrm{Fitt}}_{\mathfrak{A}}(C^{\bullet}) = {\mathrm{Fitt}}_{\mathfrak{A}}(P^{0} : P^{-1}), 
\end{equation}
where the righthand side denotes the relative Fitting invariant of \cite[Definition 3.6]{MR2609173}.

\begin{remark}\label{rmk:quad-pres-rho-is-zero}
Let $M$ be a finitely generated $R$-torsion $\mathfrak{A}$-module of projective dimension at most one.
Then it is straightforward to show that $M$ admits a quadratic presentation if and only if $\rho([M])=0$
(see \cite[p.\ 2764]{MR2609173}). 
\end{remark}

\begin{remark}\label{rmk:quad-complex-defns-coincide}
Let $M$ be a finitely generated $R$-torsion $\mathfrak{A}$-module of projective dimension at most one 
and assume that $M$ admits a quadratic presentation. 
Then one can consider $M$ as a complex concentrated in degree $0$ defining an element of 
$\mathcal{D}^{\mathrm{perf}}{_{\mathrm{tor}}}(\mathfrak{A})$, and one can show that the definitions of ${\mathrm{Fitt}}_{\mathfrak{A}}(M)$
given by \eqref{eq:quad-fitt-defn} and \eqref{eqn:fitt-of-complex} coincide in this situation. 
\end{remark}

Noncommutative Fitting invariants provide a powerful tool for computing annihilators; for the following result
see \cite[Theorem 3.3]{MR3092262} or \cite[Theorem 4.2]{MR2609173}.

\begin{theorem}\label{thm:Fitt-annihilation}
If $M$ is a finitely presented $\mathfrak{A}$-module, then
\[
\mathcal{H}(\mathfrak{A}) \cdot {\mathrm{Fitt}}_{\mathfrak{A}}^{\max}(M) \subseteq {\mathrm{Ann}}_{\zeta(\mathfrak{A})}(M).
\]
\end{theorem}

We list some properties of noncommutative Fitting invariants which we will use later.

\begin{lemma}\label{lem:Fitting-properties}
Let $M$ and $M'$ be finitely presented $\mathfrak{A}$-modules and let $e \in A$ be a central idempotent.
Then the following statements hold.
\begin{enumerate}
\item
If $M \twoheadrightarrow M'$ is a surjection then ${\mathrm{Fitt}}_{\mathfrak{A}}^{\max}(M) \subseteq {\mathrm{Fitt}}_{\mathfrak{A}}^{\max}(M')$.
\item 
If $M$ and $M'$ admit quadratic presentations then so does $M \oplus M'$ and we have an equality
$
{\mathrm{Fitt}}_{\mathfrak{A}}(M) \cdot {\mathrm{Fitt}}_{\mathfrak{A}}(M') =  {\mathrm{Fitt}}_{\mathfrak{A}}(M \oplus M')
$.
\item
We have an inclusion $e {\mathrm{Fitt}}_{\mathfrak{A}}^{\max}(M) \subseteq {\mathrm{Fitt}}_{\mathfrak{A} e}^{\max}(\mathfrak{A} e \otimes_{\mathfrak{A}} M)$
with equality if $e \in \mathfrak{A}$.
\end{enumerate}
\end{lemma}

\begin{proof}
For (i) see \cite[Theorem 3.1 (i)]{MR3092262}.
Part (ii) is a special case of \cite[Theorem 3.1 (iii)]{MR3092262}.
The first claim of part (iii) is \cite[Theorem 3.1 (vi)]{MR3092262}, and the second claim follows easily from the definition of Fitting invariants
and the decomposition $\mathfrak{A} = \mathfrak{A} e \oplus \mathfrak{A} (1-e)$.
\end{proof}

\begin{lemma}\label{lem:fitt-eq-complex-quad}
Let $A$ and $B$ be finitely generated $R$-torsion $\mathfrak{A}$-modules 
of projective dimension at most one and with quadratic presentations.
Let $A \rightarrow B$ be a complex concentrated in degrees $-1$ and $0$. 
Then recalling Remark \ref{rmk:quad-complex-defns-coincide} we have
\[
{\mathrm{Fitt}}_{\mathfrak{A}}(B:A) = {\mathrm{Fitt}}_{\mathfrak{A}}(A \rightarrow B) = {\mathrm{Fitt}}_{\mathfrak{A}}^{-1}(A) \cdot {\mathrm{Fitt}}_{\mathfrak{A}}(B).
\] 
\end{lemma}

\begin{proof}
The first equality follows from \eqref{eq:rel-fitt-eq}.
We consider $A$ and $B$ as complexes concentrated in degree $0$.
Then we have a short exact sequence of complexes
\[
0 \longrightarrow B \longrightarrow (A \rightarrow B) \longrightarrow A[1] \longrightarrow 0,
\] 
where $A[1]$ is concentrated in degree $-1$. 
Hence by \eqref{eq:SES-of-complexes} we have 
\[
[A \rightarrow B] = [B] + [A[1]] = [B] - [A],
\]
in $K_{0}(\mathfrak{A},A)$ and so the desired result now follows from \eqref{eqn:fitt-of-sum-of-complexes-in-rel-K-zero}.  
\end{proof}

\subsection{Noncommutative Fitting invariants over Iwasawa algebras}\label{subsec:noncomm-fitt-over-Iwasawa}
We now consider the main case of interest in this article, i.e., $p$ is an odd prime,
$\mathcal{G} = H \rtimes \Gamma$ is a one-dimensional $p$-adic Lie group and $\mathfrak{A} := \Lambda(\mathcal{G})$ is the Iwasawa algebra of
$\mathcal{G}$.
Recall from \eqref{eq:Lambda-R-decomp} that $\Lambda(\mathcal{G})$ is a free
$R = {\mathbb{Z}}_{p}[[\Gamma_{0}]]$-order in $A = \mathcal{Q}(\mathcal{G})$.
Also recall that Proposition \ref{prop:niceIwasawa-algebras} (i.e.\ {\cite[Proposition 4.5]{MR3092262})
determines all Iwasawa algebras that are direct products of matrix rings over commutative rings.

Now let $\Gamma' \simeq {\mathbb{Z}}_{p}$ be a normal subgroup of $\mathcal{G}$ such that $\Gamma' \cap H = 1$.
Then $\Gamma'$ is open in $\mathcal{G}$ and we set $G := \mathcal{G} / \Gamma'$.
Thus every irreducible character $\chi$ of $G$ may be viewed as an irreducible character of $\mathcal{G}$ with open kernel.

\begin{prop}[{\cite[Theorem 6.4]{MR2609173}}]\label{prop:Fitting-descent}
Let $M$ be a finitely presented $\Lambda(\mathcal{G})$-module and let $\lambda \in {\mathrm{Fitt}}_{\Lambda(\mathcal{G})}^{\max}(M)$.
Then
\[
     \sum_{\chi \in {\mathrm{Irr}}_{{\mathbb{Q}}_{p}^{c}}(G)} \phi(j_{\chi}(\lambda)) e(\chi) \in {\mathrm{Fitt}}_{{\mathbb{Z}}_{p}[G]}^{\max}(M_{\Gamma'}),
\]
where $\phi$ denotes the evaluation map \eqref{eqn:evaluation-map} and $e(\chi) := \chi(1) |G|^{-1} \sum_{g \in G} \chi(g^{-1}) g$.
\end{prop}

\section{The Brumer-Stark conjecture and generalisations} \label{sec:Brumer-Stark}

\subsection{Equivariant $L$-values} \label{subsec:L-values}
Let $L/K$ be a finite Galois extension of number fields with Galois group $G$.
For each place $v$ of $K$ we fix a place $w$ of $L$ above $v$ and write $G_{w}$ and $I_{w}$ for the decomposition
group and inertia subgroup of $L/K$ at $w$, respectively.
When $w$ is a finite place, we  choose a lift $\phi_{w} \in G_{w}$ of the Frobenius automorphism at $w$;
moreover, we write $\mathfrak{P}_{w}$ for the associated prime ideal in $L$ and ${\mathrm{ord}}_{w}$ for the associated valuation.

Let $S$ be a finite set of places of $K$ containing the set $S_{\infty}$ of archimedean places.
For $\chi \in {\mathrm{Irr}}_{\mathbb{C}}(G)$, we denote the $S$-truncated Artin $L$-function attached to $\chi$ and $S$ by $L_{S}(s,\chi)$.
Recall that there is a canonical isomorphism
$\zeta({\mathbb{C}}[G]) \simeq \prod_{\chi \in {\mathrm{Irr}}_{\mathbb{C}} (G)} {\mathbb{C}}$.
We define the equivariant $S$-truncated Artin $L$-function to be the meromorphic $\zeta({\mathbb{C}}[G])$-valued function
\[
L_{S}(s) := (L_{S}(s,\chi))_{\chi \in {\mathrm{Irr}}_{\mathbb{C}} (G)}.
\]
For $\chi \in {\mathrm{Irr}}_{\mathbb{C}}(G)$, let $V_{\chi}$ be a left ${\mathbb{C}}[G]$-module with character $\chi$.
If $T$ is a second finite set of places of $K$ such that $S \cap T = \emptyset$, we define
\[
\delta_{T}(s,\chi) = \prod_{v \in T} \det(1 - N(v)^{1-s} \phi_{w}^{-1} \mid V_{\chi}^{I_{w}}) \quad  \textrm{ and }  \quad
\delta_{T}(s) := (\delta_{T}(s,\chi))_{\chi\in {\mathrm{Irr}}_{\mathbb{C}} (G)}.
\]
We put
\[
\Theta_{S,T}(s) := \delta_{T}(s) \cdot L_{S}(s)^{\sharp},
\]
where we denote by $^{\sharp}: {\mathbb{C}}[G] \to {\mathbb{C}}[G]$ the anti-involution induced by $g \mapsto g^{-1}$.
These functions are the so-called $(S,T)$-modified $G$-equivariant $L$-functions and we define Stickelberger elements
\[
\theta_{S}^{T}(L/K) = \theta_{S}^{T} := \Theta_{S,T}(0) \in \zeta({\mathbb{Q}}[G]).
\]
Note that a priori we only have $\theta_{S}^{T} \in \zeta({\mathbb{C}}[G])$, but by a result of Siegel \cite{MR0285488} we know that $\theta_{S}^{T}$
in fact belongs to $\zeta({\mathbb{Q}}[G])$.
If $T$ is empty, we will abbreviate $\theta_{S}^{T}$ to $\theta_{S}$.

Fix an odd prime $p$ and a choice of isomorphism $\iota : {\mathbb{C}}_{p} \rightarrow {\mathbb{C}}$.
Then the image of $\theta_{S}^{T}$ under the canonical maps 
\begin{equation}\label{eqn:embeddings-Q-Qp-Cp}
\zeta({\mathbb{Q}}[G]) \hookrightarrow \zeta({\mathbb{Q}}_{p}[G]) \hookrightarrow \zeta({\mathbb{C}}_{p}[G]) \cong \textstyle{\prod_{\chi \in {\mathrm{Irr}}_{{\mathbb{C}}_{p}}}} {\mathbb{C}}_{p} 
\end{equation}
is given by $(\iota^{-1}(L_{S}(0,\iota \circ \chi))_{\chi \in {\mathrm{Irr}}_{{\mathbb{C}}_{p}}(G)}$ and this is independent of the choice of $\iota$ (compare with the discussion of \S \ref{subsec:interpolation-property}).
We shall henceforth consider $\theta_{S}^{T}$ as an element of $\zeta({\mathbb{Q}}_{p}[G])$ or $\zeta({\mathbb{C}}_{p}[G])$ via \eqref{eqn:embeddings-Q-Qp-Cp}.
Moreover, we shall often drop $\iota$ and $\iota^{-1}$ from the notation.
 
Now suppose further that $L/K$ is a CM-extension and let $j \in G$ denote complex conjugation.
Recall that $\chi \in {\mathrm{Irr}}_{\mathbb{C}}(G)$ or ${\mathrm{Irr}}_{{\mathbb{C}}_{p}}(G)$ is \emph{even} when $\chi(j) = \chi(1)$, 
and is \emph{odd} when $\chi(j) = -\chi(1)$.
Let $S_{p}$ denote the set of places of $K$ above $p$.
When $S_{p} \cup S_{\infty} \subseteq S$, we define $p$-adic Stickelberger elements by
\begin{eqnarray*}
\theta_{p,S}^{T}(L/K)  =  \theta_{p,S}^{T} & := & (\theta_{p,S,\chi}^{T})_{\chi \in {\mathrm{Irr}}_{{\mathbb{C}}_{p}} (G)},\\
\theta_{p,S,\chi}^{T} & := & \left\{ \begin{array}{ll} 0 & \mbox{ if } \chi \mbox{ is even}\\
                                        \delta_{T}(0,\chi) \cdot L_{p,S}(0,\check{\chi} \omega) & \mbox{ if } \chi \mbox{ is odd},
                                        \end{array} \right.
\end{eqnarray*}
where $\check \chi$ denotes the character contragredient to $\chi$.

\begin{lemma} \label{lem:p-adic-vs-complex-Stickelberger}
Let $p$ be an odd prime and let $L/K$ be a finite Galois CM-extension of number fields.
If ${\mathrm{Gal}}(L^{+}/K)$ is monomial and $ S_{p} \cup S_{\infty} \subseteq S$, then
$\theta_{p,S}^{T} = \theta_{S}^{T}$.
\end{lemma}

\begin{proof}
Let $\psi \in {\mathrm{Irr}}_{\mathbb{C}}({\mathrm{Gal}}(L/K))$. 
If $\psi$ is the trivial character then \cite[Chapter I, Proposition 3.4]{MR782485} shows that the order of vanishing of
$L_{S}(s,\psi)$ at $s=0$ is $|S|-1 \geq 1$, and so $L_{S}(s,\psi)=0$.
If $\psi$ is even and non-trivial the argument given in \cite[top of p.\ 71]{MR782485} again shows that $L_{S}(s,\psi)=0$.
Hence we have
$\theta_{p,S}^{T} = \theta_{S}^{T}$ whenever \eqref{eqn:values-padic-complex} holds for all even 
$\chi \in {\mathrm{Irr}}_{\mathbb{C}}({\mathrm{Gal}}(L/K))$ (note $\omega^{-1}$ is odd),
and so the desired result follows from the discussion in \S \ref{subsec:interpolation-property}.
\end{proof}

\subsection{Ray class groups}
For any set $S$ of places of $K$, we write $S(L)$ for the set of places of $L$ which lie above those in $S$.
Now let $T$ and $S$ be as in \S \ref{subsec:L-values}.
We write ${\mathrm{cl}}_{L}^{T}$ for the ray class group of $L$ associated to the ray
$\mathfrak{M}_{L}^{T} := \prod_{w \in T(L)} \mathfrak{P}_w$ and $\mathcal{O}_{L,S}$ for the ring of $S(L)$-integers in $L$.
Let $\mathcal{O}_{L} := \mathcal{O}_{L, S_{\infty}}$ be the ring of integers in $L$.
Let $S_{f}$ be the set of all finite primes in $S$;
then there is a natural map ${\mathbb{Z}} S_{f}(L) \to {\mathrm{cl}}_{L}^{T}$ which sends each place $w \in S_{f}(L)$
to the corresponding class $[\mathfrak{P}_w] \in {\mathrm{cl}}_{L}^{T}$. We denote the cokernel of this map by ${\mathrm{cl}}_{L,S}^{T}$.
Moreover, we denote the $S(L)$-units of $L$ by $E_{L,S}$ and define
$E_{L,S}^T := \left\{x \in E_{L,S}: x \equiv 1 \bmod \mathfrak{M}_{L}^{T} \right\}$.
All these modules are equipped with a natural $G$-action and we have the following exact sequences of ${\mathbb{Z}}[G]$-modules.
If $\Sigma$ is a subset of $S$ containing $S_{\infty}$, then we have
\begin{equation}\label{eqn:ray_class_sequence_ZS}
0 \longrightarrow E_{L, \Sigma}^T \longrightarrow E_{L,S}^T \stackrel{v_{L}}{\longrightarrow}
{\mathbb{Z}} [S(L) - \Sigma(L)] \longrightarrow {\mathrm{cl}}_{L, \Sigma}^{T} \longrightarrow {\mathrm{cl}}_{L,S}^{T} \longrightarrow 0,
\end{equation}
where $v_{L}(x) := \sum_{w \in S(L) - \Sigma(L)} {\mathrm{ord}}_w(x) w$ for every $x \in E_{L,S}^T$, and
\begin{equation}\label{eqn:ray_class_sequence}
0 \longrightarrow E_{L,S}^T \longrightarrow E_{L,S} \longrightarrow (\mathcal{O}_{L,S} / \mathfrak{M}_{L}^{T}){^{\times}}
\stackrel{\nu}{\longrightarrow} {\mathrm{cl}}_{L,S}^{T} \longrightarrow {\mathrm{cl}}_{L,S} \longrightarrow 0,
\end{equation}
where the map $\nu$ lifts an element $\overline x \in (\mathcal{O}_{L,S} / \mathfrak{M}_{L}^{T})^{\times}$ to
$x \in \mathcal{O}_{L,S}$ and
sends it to the ideal class $[(x)] \in {\mathrm{cl}}_{L,S}^{T}$ of the principal ideal $(x)$.

\subsection{\'{E}tale cohomology}
Let $L/K$ be a finite Galois extension of number fields with Galois group $G$.
We fix two finite disjoint nonempty sets $S$ and $T$ of places of $K$ such that $S$ contains $S_{\infty}$. 
We put $U_{S} := {\mathrm{Spec}}(\mathcal{O}_{L,S})$ and $Z_{T} := {\mathrm{Spec}}(\mathcal{O}_{L,S} / \mathfrak{M}_{L}^{T})$. 
Let $\mathbb{G}_{m,X}$ denote the \'{e}tale sheaf defined by the group of units of a scheme $X$. 
The closed immersion $\iota: Z_{T} \rightarrow U_{S}$ induces a canonical morphism 
$\mathbb{G}_{m,U_{S}} \rightarrow \iota_{\ast} \mathbb{G}_{m,Z_{T}}$, which can be shown to be surjective 
by considering stalks. 
Let $\mathbb{G}_{m,U_{S}}^{T}$ denote the kernel of this morphism; then 
we have an exact sequence of \'{e}tale sheaves
\begin{equation} \label{eqn:ses-etale-sheaves}
0 \longrightarrow \mathbb{G}_{m,U_{S}}^{T} \longrightarrow \mathbb{G}_{m,U_{S}}
\longrightarrow \iota_{\ast} \mathbb{G}_{m,Z_{T}} \longrightarrow 0.
\end{equation}
If $w$ is a finite place of $L$, we let $\mathcal{O}_{L,w}$ be the localisation of $\mathcal{O}_{L}$ at $w$.
We denote the field of fractions of the Henselisation $\mathcal{O}_{L,w}^{h}$ of $\mathcal{O}_{L,w}$ by $L_{w}$.
If $w$ is archimedean, we let $L_{w}$ be the completion of $L$ at $w$. In both cases we let ${\mathrm{Br}}(L_{w})$ be the Brauer group of $L_{w}$.

The main purpose of this subsection is to generalise the following result.

\begin{prop}\label{prop:cohomology_of_Gm}
Let $S$ be a finite set of places of $K$ containing $S_{\infty}.$ Then
\begin{eqnarray*}
H^{0}_{\mathrm{\acute{e}t}}(U_{S}, \mathbb{G}_{m,U_{S}}) & \simeq & E_{L,S},\\
H^{1}_{\mathrm{\acute{e}t}}(U_{S}, \mathbb{G}_{m,U_{S}}) & \simeq & {\mathrm{cl}}_{L,S},
\end{eqnarray*}
there is an exact sequence
\[
0 \longrightarrow H^{2}_{\mathrm{\acute{e}t}}(U_{S}, \mathbb{G}_{m,U_{S}}) \longrightarrow \bigoplus_{w \in S(L)} {\mathrm{Br}}(L_{w}) \longrightarrow {\mathbb{Q}} / {\mathbb{Z}}
\longrightarrow H^{3}_{\mathrm{\acute{e}t}}(U_{S}, \mathbb{G}_{m,U_{S}}) \longrightarrow 0,
\]
and
\[
H^{i}_{\mathrm{\acute{e}t}}(U_{S}, \mathbb{G}_{m,U_{S}}) \simeq \bigoplus_{w \in S_{\infty}(L) \atop w \mbox{ \tiny{real}}}
H^{i}_{\mathrm{\acute{e}t}}({\mathrm{Spec}}(L_{w}), \mathbb{G}_{m,{\mathrm{Spec}}(L_{w})}), \quad i \geq 4.
\]
\end{prop}

\begin{proof}
This is \cite[Chapter II, Proposition 2.1]{MR2261462}.
\end{proof}

\begin{lemma}\label{lem:qis-i-zt}
Let $S$ and $T$ be as above. 
Then $R\Gamma_{\mathrm{\acute{e}t}}(U_{S}, \iota_{\ast}\mathbb{G}_{m,Z_{T}}) \simeq R\Gamma_{\mathrm{\acute{e}t}}(Z_{T}, \mathbb{G}_{m,Z_{T}})$.
Moreover, $H^{0}_{\mathrm{\acute{e}t}}(Z_{T},\mathbb{G}_{m,Z_{T}}) \simeq (\mathcal{O}_{L,S} / \mathfrak{M}_{L}^{T})^{\times}$ and 
$H^{i}_{\mathrm{\acute{e}t}}(Z_{T},\mathbb{G}_{m,Z_{T}})=0$ for $i \geq 1$.
\end{lemma}

\begin{proof}
For a finite field $\mathbb{F}$, the cohomology of $R\Gamma_{\mathrm{\acute{e}t}}({\mathrm{Spec}}(\mathbb{F}), \mathbb{G}_{m,{\mathrm{Spec}}(\mathbb{F})})$
vanishes outside degree $0$ and $H_{\mathrm{\acute{e}t}}^{0}({\mathrm{Spec}}(\mathbb{F}), \mathbb{G}_{m,{\mathrm{Spec}}(\mathbb{F})}) \simeq {\mathbb{F}}^{\times}$.
Moreover, we have an isomorphism 
\[
R\Gamma_{\mathrm{\acute{e}t}}(Z_{T}, \mathbb{G}_{m,Z_{T}})
\simeq \bigoplus_{w \in T(L)} R\Gamma_{\mathrm{\acute{e}t}}({\mathrm{Spec}}(L(w)), \mathbb{G}_{m,{\mathrm{Spec}}(L(w))}),
\]
where $L(w)$ denotes the finite field $\mathcal{O}_{L} / \mathfrak{P}_{w}$, and so the second claim follows.
Note that the natural map 
$H^{0}_{\mathrm{\acute{e}t}}(Z_{T},\mathbb{G}_{m,Z_{T}}) \rightarrow H^{0}_{\mathrm{\acute{e}t}}(U_{S},\iota_{*}\mathbb{G}_{m,Z_{T}})$ is in fact an isomorphism.
Furthermore, the functor $\iota_{\ast}$ is exact for the \'{e}tale topology by \cite[Chapter II, Corollary 3.6]{MR559531}.
Thus the universal property of derived functors gives the first claim.
\end{proof}

Let $\Sigma$ be a subset of $S$ containing $S_{\infty}$.
We shall consider the natural map 
\begin{equation}\label{eqn:define-psi}
\psi_{\Sigma,S}^{T} = \psi_{\Sigma,S}^{T}(L):
R\Gamma_{\mathrm{\acute{e}t}}(U_{\Sigma}, \mathbb{G}_{m, U_{\Sigma}}^{T}) \longrightarrow R\Gamma_{\mathrm{\acute{e}t}}(U_{S}, \mathbb{G}_{m, U_{S}}^{T}).
\end{equation}
Here, the set $T$ may be empty, in which case we put $\mathbb{G}_{m, U_{S}}^{\emptyset} := \mathbb{G}_{m, U_{S}}$ 
and similarly with $\mathbb{G}_{m, U_{\Sigma}}$.
Sequence \eqref{eqn:ses-etale-sheaves} and Lemma \ref{lem:qis-i-zt} for $S$ and $\Sigma$ 
induce a commutative diagram
\begin{equation}\label{eqn:change-sigma-to-S}
\xymatrix{
{R\Gamma_{\mathrm{\acute{e}t}}(U_{\Sigma}, \mathbb{G}_{m, U_{\Sigma}}^{T})} \ar[r] \ar[d]^{\psi_{\Sigma,S}^{T}} &
R\Gamma_{\mathrm{\acute{e}t}}(U_{\Sigma}, \mathbb{G}_{m, U_{\Sigma}}) \ar[r] \ar[d]^{\psi_{\Sigma,S}} & R\Gamma_{\mathrm{\acute{e}t}}(Z_{T}, \mathbb{G}_{m,Z_{T}}) \ar@{=}[d]\\
{R\Gamma_{\mathrm{\acute{e}t}}(U_{S}, \mathbb{G}_{m, U_{S}}^{T})} \ar[r] &
R\Gamma_{\mathrm{\acute{e}t}}(U_{S}, \mathbb{G}_{m, U_{S}}) \ar[r] & R\Gamma_{\mathrm{\acute{e}t}}(Z_{T}, \mathbb{G}_{m,Z_{T}})
} 
\end{equation}
where the rows are exact triangles.
Let $D_{\Sigma,S}^{T} = D_{\Sigma,S}^{T}(L)$ be the cone of $\psi_{\Sigma,S}^{T}$. The diagram shows that
$D_{\Sigma,S}^{T}$ does not in fact depend on $T$ and thus we denote it by $D_{\Sigma,S}$.

\begin{prop}\label{prop:cohomology-of-cone}
Let $S,T$ and $\Sigma$ be as above. We have
\[
H^{i}(D_{\Sigma,S}^{T}(L)) \simeq H^{i}(D_{\Sigma,S}(L)) \simeq \left\{
\begin{array}{lll}
{\mathbb{Z}}[S(L)-\Sigma(L)] & \mbox{ if } & i=0\\
{\mathbb{Q}} / {\mathbb{Z}}[S(L)-\Sigma(L)] & \mbox{ if } & i=2\\
0 & \mbox{ if } & i\not=0,2.\\
\end{array}
\right.
\]
\end{prop}

\begin{proof}
As $D_{\Sigma,S}^{T}=D_{\Sigma,S}$ does not depend on $T$, we can and do assume that $T$ is empty.
Let $Z_{\Sigma,S} := U_{\Sigma} - U_{S}$; then $Z_{\Sigma,S}$ is a closed subscheme of $U_{\Sigma}$ and
by \cite[Chapter III, Proposition 1.25]{MR559531} we have an isomorphism
\[
D_{\Sigma,S} \simeq R\Gamma_{Z_{\Sigma,S}}(U_{\Sigma}, \mathbb{G}_{m, U_{\Sigma}})[-1],
\]
where the righthand side denotes cohomology with support on $Z_{\Sigma,S}$.
We now apply \cite[Chapter III, Corollary 1.28]{MR559531} and \cite[Chapter II, Proposition 1.5]{MR2261462} to
obtain the desired result.
\end{proof}

\begin{prop}\label{prop:cohomology_of_GmT}
Let $S$ and $T$ be as above. Then
\[
H^{i}_{\mathrm{\acute{e}t}}(U_{S}, \mathbb{G}_{m,U_{S}}^{T}) \simeq \left\{
\begin{array}{lll}
E_{L,S}^{T} & \mbox{ if } & i=0\\
{\mathrm{cl}}_{L,S}^{T} & \mbox{ if } & i=1\\
H^{i}_{\mathrm{\acute{e}t}}(U_{S}, \mathbb{G}_{m,U_{S}}) & \mbox{ if } & i \geq 2.\\
\end{array}
\right.
\]
\end{prop}

\begin{proof}
By Proposition \ref{prop:cohomology_of_Gm} and Lemma \ref{lem:qis-i-zt}, the 
long exact sequence of cohomology groups induced by \eqref{eqn:ses-etale-sheaves} yields an exact sequence
\[
0 \longrightarrow H^{0}_{\mathrm{\acute{e}t}}(U_{S}, \mathbb{G}_{m,U_{S}}^{T}) \longrightarrow E_{L,S} \longrightarrow (\mathcal{O}_{L,S} / \mathfrak{M}_{L}^{T}){^{\times}}
\longrightarrow H^{1}_{\mathrm{\acute{e}t}}(U_{S}, \mathbb{G}_{m,U_{S}}^{T}) \longrightarrow {\mathrm{cl}}_{L,S} \longrightarrow 0
\]
and isomorphisms $ H^{i}_{\mathrm{\acute{e}t}}(U_{S}, \mathbb{G}_{m,U_{S}}^{T}) \simeq H^{i}_{\mathrm{\acute{e}t}}(U_{S}, \mathbb{G}_{m,U_{S}})$ for all $i \geq 2$.
It follows from this and \eqref{eqn:ray_class_sequence} that 
$H^{0}_{\mathrm{\acute{e}t}}(U_{S}, \mathbb{G}_{m,U_{S}}^{T})  \simeq E_{L,S}^{T}$, and that
$H^{1}_{\mathrm{\acute{e}t}}(U_{S}, \mathbb{G}_{m,U_{S}}^{T})$ and ${\mathrm{cl}}_{L,S}^{T}$ have the same cardinality;
it remains to show that they are in fact isomorphic. 

We now change notation as follows: let $\Sigma=S$ for the choice of $S$ as in the statement of the proposition; and
enlarge $S$ in such a way that $S$ is finite and disjoint from $T$ and that ${\mathrm{cl}}_{L,S}^{T}$ vanishes.
The same reasoning as above shows that $H^{1}_{\mathrm{\acute{e}t}}(U_{S}, \mathbb{G}_{m,U_{S}}^{T})$ also vanishes.
Therefore the long exact cohomology sequence induced 
by \eqref{eqn:define-psi} yields an exact sequence
\begin{equation}\label{eqn:seq-induced-by-sigma-S-cone}
0 \longrightarrow E_{L, \Sigma}^T \longrightarrow E_{L,S}^T \longrightarrow
{\mathbb{Z}} [S(L) - \Sigma(L)] \longrightarrow H_{\mathrm{\acute{e}t}}^{1}(U_{\Sigma},\mathbb{G}_{m,U_{\Sigma}}^{T}) \longrightarrow 0,
\end{equation}
where $H^{0}(D_{\Sigma,S}^{T}(L)) \simeq {\mathbb{Z}} [S(L) - \Sigma(L)]$ by Proposition \ref{prop:cohomology-of-cone}.
Comparing \eqref{eqn:seq-induced-by-sigma-S-cone} to \eqref{eqn:ray_class_sequence_ZS} yields 
$H_{\mathrm{\acute{e}t}}^{1}(U_{\Sigma},\mathbb{G}_{m,U_{\Sigma}}^{T}) \simeq {\mathrm{cl}}_{L,\Sigma}^{T}$, as desired.
\end{proof}

\begin{remark}
The proof of Proposition \ref{prop:cohomology_of_GmT} shows that the long exact sequence in cohomology induced by \eqref{eqn:define-psi} 
yields an exact sequence whose first terms coincide with \eqref{eqn:ray_class_sequence_ZS}. 
Similarly, taking global sections in \eqref{eqn:ses-etale-sheaves} gives a long exact sequence
in cohomology whose first terms coincide with \eqref{eqn:ray_class_sequence}.
\end{remark}

\subsection{The non-abelian Brumer conjecture}\label{subsec:non-abelian-Brumer}
Let $L/K$ be a finite Galois CM-extension of number fields with Galois group $G$.
Let $S_{\mathrm{ram}}=S_{\mathrm{ram}}(L/K)$ be the set of all places of $K$ that ramify in $L/K$
and recall that $S_{\infty}$ denotes the archimedean places of $K$.
\begin{hypothesis*}
Let $S$ and $T$ be finite sets of places of $K$. 
We say that ${\mathrm{Hyp}}(S,T)$ is satisfied if
(i)
$S_{\mathrm{ram}} \cup S_{\infty} \subseteq S$,
(ii)
$S \cap T = \emptyset$, and
(iii)
$E_{L,S}^T$ is torsionfree.
\end{hypothesis*}

\begin{remark}\label{rmk:conditions-on-T}
Condition (iii) means that there are no roots of unity of $L$ congruent to $1$ modulo all primes in $T(L)$. 
In particular, this will be satisfied if $T$ contains primes of two different residue characteristics or one prime of sufficiently large norm. 
\end{remark}

We choose a maximal order $\mathfrak{M}(G)$ such that ${\mathbb{Z}}[G] \subseteq \mathfrak{M}(G) \subseteq {\mathbb{Q}}[G]$.
For a fixed set $S$ we define $\mathfrak{A}_{S}$ to be the $\zeta({\mathbb{Z}}[G])$-submodule of $\zeta(\mathfrak{M}(G))$ generated
by the  elements $\delta_T(0)$, where $T$ runs through the finite sets of places of $K$
such that ${\mathrm{Hyp}}(S \cup S_{\mathrm{ram}} \cup S_{\infty},T)$ is satisfied.
The following conjecture was formulated in \cite{MR2976321} and is a non-abelian generalisation of Brumer's conjecture.

\begin{conj}[$B(L/K,S)$] \label{conj:Brumer}
Let $S$ be a finite set of places of $K$ containing $S_{\mathrm{ram}} \cup S_{\infty}$.
Then $\mathfrak{A}_S \theta_S \subseteq \mathcal{I}(G)$ and for each $x \in \mathcal H(G)$ we have
\[
x \cdot \mathfrak{A}_S \theta_S \subseteq {\mathrm{Ann}}_{{\mathbb{Z}}[G]} ({\mathrm{cl}}_L).
\]
\end{conj}

\begin{remark}
If $G$ is abelian, \cite[Lemma 1.1, p.~82]{MR782485} implies that $\mathfrak{A}_{S} = {\mathrm{Ann}}_{{\mathbb{Z}}[G]}(\mu_{L})$.
In this case the results in \cite{MR525346, MR524276, MR579702} each imply that
$\mathfrak{A}_{S} \theta_{S} \subseteq \mathcal{I}(G) = {\mathbb{Z}}[G]$ and,
since $\mathcal{H}(G) = {\mathbb{Z}}[G]$ in this case, Conjecture \ref{conj:Brumer} recovers Brumer's conjecture.
\end{remark}

\begin{remark}\label{rmk:p-part-of-Brumer}
Replacing the class group ${\mathrm{cl}}_{L}$ by its $p$-part ${\mathrm{cl}}_{L}(p):={\mathbb{Z}}_{p} \otimes_{\mathbb{Z}} {\mathrm{cl}}_{L}$ for each rational prime $p$,
Conjecture $B(L/K,S)$ naturally decomposes into local conjectures $B(L/K,S,p)$.
It is then possible to replace $\mathcal{H}(G)$ by $\mathcal{H}_{p}(G)$ by \cite[Lemma 1.4]{MR2976321}.
\end{remark}

\begin{remark}
Burns \cite{MR2845620} has also formulated a conjecture which generalises many refined Stark conjectures to the
non-abelian situation. In particular, it implies Conjecture \ref{conj:Brumer} (see \cite[Proposition 3.5.1]{MR2845620}).
\end{remark}

\subsection{The non-abelian Brumer-Stark conjecture}
For $\alpha \in L{^{\times}}$ we define
\[
S_{\alpha} :=  \{ \mathfrak{p} \mbox{ finite place of } K \mid v_{\mathfrak{p}}(N_{L/K}(\alpha))>0 \}, 
\]
where $v_{\mathfrak{p}}$ is the $\mathfrak{p}$-adic valuation on $K$.
Let $j \in G$ denote complex conjugation. We call $\alpha$ an {\it anti-unit} if $\alpha^{1+j} = 1$.
Let $\omega_L := {\mathrm{nr}} (|\mu_L|)$. The following is a non-abelian generalisation of the Brumer-Stark conjecture (\cite[Conjecture 2.7]{MR2976321}).

\begin{conj}[$BS(L/K,S)$] \label{conj:Brumer-Stark}
Let $S$ be a finite set of places of $K$ containing $S_{\mathrm{ram}} \cup S_{\infty}$.
Then $\omega_L \cdot \theta_S \in \mathcal I(G)$ and for each $x \in \mathcal H(G)$ and each fractional ideal $\mathfrak{a}$ of $L$,
there is an anti-unit $\alpha = \alpha(x,\mathfrak{a},S) \in L{^{\times}}$ such that
\[
\mathfrak{a}^{x \cdot \omega_L \cdot \theta_S} = (\alpha)
\]
and for each finite set $T$ of primes of $K$ such that ${\mathrm{Hyp}}(S \cup S_{\alpha},T)$ is satisfied there is an $\alpha_{T} \in E_{L,S_{\alpha}}^{T}$
such that
\begin{equation} \label{eqn:abelian-ersatz}
\alpha^{z \cdot \delta_T(0)} = \alpha_T^{z \cdot \omega_L}
\end{equation}
for each $z \in \mathcal H(G)$.
\end{conj}

\begin{remark}
If $G$ is abelian, we have $\mathcal{I}(G) = \mathcal{H}(G) = {\mathbb{Z}}[G]$ and $\omega_{L} = |\mu_{L}|$.
Hence it suffices to treat the case $x=z=1$ in this situation.
Then \cite[Proposition 1.2, p.~83]{MR782485} states that condition \eqref{eqn:abelian-ersatz}
on the anti-unit $\alpha$ is equivalent to the assertion that the extension $L(\alpha^{1/\omega_{L}}) / K$ is abelian.
\end{remark}

\begin{remark}
As in Remark \ref{rmk:p-part-of-Brumer}, we obtain local conjectures $BS(L/K,S,p)$ for each prime $p$.
\end{remark}

If $M$ is a ${\mathbb{Z}}$-module and $p$ is a prime, we put $M(p) := {\mathbb{Z}}_{p} \otimes_{\mathbb{Z}} M$.
For any ${\mathbb{Z}}_{p}[G]$-module $M$ we denote the Pontryagin dual ${\mathrm{Hom}} (M, {\mathbb{Q}}_{p} / {\mathbb{Z}}_{p})$ of $M$ by $M^{\vee}$ which
is endowed with the contragredient $G$-action $(gf)(m) = f (g^{-1} m)$ for $f \in M^{\vee}$, $g \in G$ and $m \in M$.
For every $G$-module $M$
we write $M^{+}$ and $M^{-}$ for the submodules of $M$ upon which $j$ acts as $1$ and $-1$, respectively.
In particular, we will be interested in $({\mathrm{cl}}_{L,S}^{T}(p))^-$ for odd primes $p$; we will abbreviate this module
to $A_{L,S}^{T}$ when $p$ is clear from the context. Note that $A_{L,S}^{T}$ is a finite module
over the ring ${\mathbb{Z}}_{p}[G]_{-} := {\mathbb{Z}}_{p}[G]/(1+j)$.
We shall need the following variant of \cite[Proposition 3.9]{MR2976321}.

\begin{prop} \label{prop:SBS_implies_BS}
Let $S$ be a finite set of places of $K$ containing $S_{\mathrm{ram}} \cup S_{\infty}$ and let $p$ be an odd prime.
Suppose that for every finite set $T$ of places of $K$ such that ${\mathrm{Hyp}}(S,T)$ is satisfied we have
\[
(\theta_S^T)^{\sharp} \in {\mathrm{Fitt}}^{\max}_{{\mathbb{Z}}_{p}[G]_{-}}((A_{L}^{T})^{\vee}).
\]
Then both $BS(L/K,S,p)$ and $B(L/K,S,p)$ are true.
\end{prop}

\begin{proof}
This has already been shown within the proof of \cite[Corollary 4.6]{MR3072281}, but we repeat the argument for convenience.
Since $BS(L/K,S,p)$ implies $B(L/K,S,p)$ by \cite[Lemma 2.12]{MR2976321}, we need only treat the case of the Brumer-Stark conjecture.
Moreover, \cite[Proposition 3.9]{MR2976321} says that $BS(L/K,S,p)$ is implied by the so-called strong Brumer-Stark property, 
which is fulfilled if $\theta_{S}^{T} \in {\mathrm{Fitt}}_{{\mathbb{Z}}_{p}[G]_{-}}^{\max}(A_{L}^{T})$ (see \cite[Definition 3.6]{MR2976321}).
However, the proof of \cite[Proposition 3.9]{MR2976321} carries over unchanged once we observe that
\[
{\mathrm{Ann}}_{{\mathbb{Z}}_{p}[G]_{-}}(M) = {\mathrm{Ann}}_{{\mathbb{Z}}_{p}[G]_{-}}(M^{\vee})^{\sharp}
\]
for every finite ${\mathbb{Z}}_{p}[G]_{-}$-module $M$.
\end{proof}

\subsection{The Brumer-Stark conjecture and Iwasawa theory}\label{subsec:brumer-stark-and-Iwasawa-theory}
The purpose of this subsection is to prove the following theorem.
\begin{theorem}\label{thm:EIMC-implies-BS}
Let $L/K$ be a finite Galois CM-extension of number fields with Galois group $G$.
Let $S$ and $T$ be two finite sets of places of $K$ satisfying ${\mathrm{Hyp}}(S,T)$.
Let $p$ be an odd prime and
let $L(\zeta_p)^{+}_{\infty}$ be the cyclotomic ${\mathbb{Z}}_{p}$-extension of $L(\zeta_p)^{+}$.
Suppose that $S_{p} \subseteq S$ and that the EIMC holds for $L(\zeta_p)^{+}_{\infty} / K$. Then
\begin{equation}\label{eqn:strongBS}
    (\theta_{p,S}^{T})^{\sharp} \in {\mathrm{Fitt}}^{\max}_{{\mathbb{Z}}_{p}[G]_{-}}((A_{L}^{T})^{\vee}).
\end{equation}
\end{theorem}

\begin{remark}
When the Iwasawa $\mu$-invariant attached to $L(\zeta_{p})_{\infty}$ vanishes, 
Theorem \ref{subsec:brumer-stark-and-Iwasawa-theory} recovers \cite[Theorem 4.5]{MR3072281},
which in turn is the non-abelian analogue of \cite[Theorem 6.5]{GrP-EIMC}. 
The main reason why some version of the $\mu=0$ hypothesis is assumed in these results
is of course to ensure that the EIMC holds (see Remark \ref{rmk:mu=0}).
However, both results use a version of the EIMC that requires a $\mu=0$ hypothesis even for its formulation.
For this reason our proof is very different from (though partly inspired by) those in
\cite{GrP-EIMC,MR3072281}. 
Note that in \cite[\S 4]{MR3072281} the identity \eqref{eqn:values-padic-complex} is implicitly assumed to hold.
\end{remark}

\begin{corollary} \label{cor:BS-holds}
Let $L/K$ be a finite Galois CM-extension of number fields.
Let $p$ be an odd prime and 
let $S$ be a finite set of places of $K$ such that $S_{p} \cup S_{\mathrm{ram}}(L/K) \cup S_{\infty} \subseteq S$.
If ${\mathrm{Gal}}(L^{+}/K)$ is monomial and the EIMC holds for $L(\zeta_p)^{+}_{\infty} / K$ then both $BS(L/K,S,p)$ and $B(L/K,S,p)$ are true.
In particular, this is the case when  $L^{+}/K$ is any of the extensions considered in Corollary \ref{cor:EIMC-Frobenius}
or Examples \ref{ex:EIMC-dicyclic}, \ref{ex:EIMC-modified-affine} or \ref{ex:EIMC-S4}.
\end{corollary}

\begin{proof}
The first claim follows from Theorem \ref{thm:EIMC-implies-BS}, Proposition \ref{prop:SBS_implies_BS} 
and Lemma \ref{lem:p-adic-vs-complex-Stickelberger}. 
Applying Corollary \ref{cor:pth-root-base-change} in the situation described in Remark \ref{rmk:base-change-in-particular}
shows that the EIMC holds for both $L^{+}_{\infty}/K$ and $L(\zeta_{p})^{+}_{\infty}/K$ in all the cases of the second claim.
Thus by Theorem \ref{thm:EIMC-implies-BS} it remains to verify that the occurring Galois groups are monomial. 
It is straightforward to check that $S_{4}$ is a monomial group, and the
group in Example \ref{ex:EIMC-modified-affine} is monomial by an application
of \cite[Chapter 2, Theorem 3.10]{MR655785}.
The remaining cases follow from Lemma \ref{lem:monomial-Frobenius}
as the Frobenius complements in question are supersoluble and thus monomial by \cite[Theorem 4.8 (1)]{MR1984740}. 
 \end{proof}

\begin{proof}[Proof of Theorem \ref{thm:EIMC-implies-BS}]
This proof will occupy the rest of this subsection and we shall prove several propositions and lemmas along the way.
We first prove a reduction step. 

Put $L' := L(\zeta_{p})$ and $C := {\mathrm{Gal}}(L'/L)$. 
Then $C$ is a cyclic group whose order divides $p-1$.
Let $N:A_{L'}^{T} \rightarrow A_{L}^{T}$ and $i:A_{L}^{T} \rightarrow A_{L'}^{T}$ 
denote the homomorphisms induced by the norm and inclusion maps on ideals, respectively.
Then $N \circ i : A_{L}^{T} \rightarrow A_{L}^{T}$ is multiplication by $|C|$ and thus is an isomorphism of $p$-groups.
In particular, $i$ is injective and $N$ is surjective. 
Let $\Delta(C)$ denote the kernel of the augmentation map ${\mathbb{Z}}[C] \twoheadrightarrow {\mathbb{Z}}$ which maps each $c \in C$ to $1$. 
The composite map $i \circ N: A^{T}_{L'} \rightarrow A^{T}_{L'}$
(also referred to as a norm map) is given by multiplication by $\sum_{c \in C} c$.
Since the orders of $C$ and $A^{T}_{L'}$ are coprime, $A^{T}_{L'}$ is cohomologically trivial as a $C$-module and thus
$\ker(i \circ N) =  \Delta(C) A_{L'}^{T}$. 
As $i$ is injective we thus have  $\ker(N) =  \Delta(C) A_{L'}^{T}$ and since $N$ is surjective we conclude that it induces an isomorphism
 $(A_{L'}^{T})_{C} \simeq A_{L}^{T}$.
Using standard properties of Pontryagin duals and (co)-invariants, we conclude that 
\[
((A_{L'}^{T})^{\vee})_{C} \simeq ((A_{L'}^{T})^{C})^{\vee} \simeq ((A_{L'}^{T})_{C})^{\vee} \simeq (A_{L}^{T})^{\vee}.
\]
The idempotent $e_{C} := |C|^{-1} \sum_{c \in C} c$ belongs to the group ring ${\mathbb{Z}}_{p}[{\mathrm{Gal}}(L'/K)]$ and so
\[
{\mathrm{Fitt}}^{\max}_{{\mathbb{Z}}_{p}[G]_{-}}((A_{L}^{T})^{\vee}) = e_{C} {\mathrm{Fitt}}^{\max}_{{\mathbb{Z}}_{p}[{\mathrm{Gal}}(L'/K)]_{-}}((A_{L'}^{T})^{\vee})
\]
by Lemma \ref{lem:Fitting-properties} (iii).
As Stickelberger elements also behave well under base change, i.e., $e_{C} \theta_{p,S}^{T}(L'/K) = \theta_{p,S}^{T}(L/K)$, we may assume without loss of generality that $\zeta_{p} \in L$.
Note that as we are considering the $p$-parts we only need that $E^{T}_{L,S}(p)$ is torsionfree,
as opposed to the stronger requirement that $E^{T}_{L,S}$ is torsionfree.
Since $S_{p} \subseteq S$ and ${\mathrm{Hyp}}(S,T)$ holds, this hypothesis is unaffected by replacing $L$ with $L(\zeta_{p})$.

For clarity, we now list the assumptions that we shall use for the rest of this proof 
(including the lemmas and propositions proved along the way).
We can make these assumptions either for the reasons just explained or because they are direct consequences 
of our hypotheses. Note that $S_{\mathrm{ram}}(L/K) \cup S_{p} = S_{\mathrm{ram}}(L_{\infty}/K)$.

\begin{assumptions*}
We henceforth assume that $S,T$ are finite sets of places of $K$ and that
(i) $\zeta_{p} \in L$, (ii) $S \cap T = \emptyset \neq T$, (iii)  $S_{p} \cup S_{\mathrm{ram}}(L/K) \cup S_{\infty} \subseteq S$,
and (iv) $E^{T}_{L,S}(p)$ is torsionfree.
\end{assumptions*}

Now taking $p$-minus parts of sequence \eqref{eqn:ray_class_sequence} for $S = S_{\infty}$ 
yields an exact sequence of ${\mathbb{Z}}_{p}[G]_{-}$-modules
\begin{equation} \label{eqn:minus-rayclass-sequence}
0 \longrightarrow \mu_{p}(L) \longrightarrow (\mathcal{O}_{L} / \mathfrak{M}_{L}^{T})^{\times} (p)^{-} \longrightarrow A_{L}^{T}
\longrightarrow A_{L} \longrightarrow 0,
\end{equation}
where $\mu_{p}(L)$ denotes the group of $p$-power roots of unity in $L$.
The middle arrow $(\mathcal{O}_{L} / \mathfrak{M}_{L}^{T}){^{\times}} (p)^{-} \rightarrow A_{L}^{T}$ defines a complex
$C^{T \bullet}(L/K)$ in $\mathcal{D}({\mathbb{Z}}_{p}[G]_{-})$, where we place the first module in degree zero.
For complexes $C^{\bullet}$ in $\mathcal{D}({\mathbb{Z}}_{p}[G])$ and $D^{\bullet}$ in $\mathcal{D}({\mathbb{Z}}[G])$
we put $C^{\bullet -} := {\mathbb{Z}}_{p}[G]_{-} \otimes^{\mathbb{L}}_{{\mathbb{Z}}_{p}[G]} C^{\bullet} \in \mathcal{D}({\mathbb{Z}}_{p}[G]_{-})$
and $D^{\bullet}(p) := {\mathbb{Z}}_{p}\otimes^{\mathbb{L}}_{\mathbb{Z}} D^{\bullet} \in \mathcal{D}({\mathbb{Z}}_{p}[G])$, respectively.
Note that taking $p$-minus parts is an exact functor as $p$ is odd.
We also put $U_{L} := U_{S_{\infty}} = {\mathrm{Spec}} (\mathcal{O}_{L})$.

\begin{prop}\label{prop:coh-interpretation-L/K}
There is an isomorphism
\[
C^{T \bullet}(L/K) \simeq R\Gamma_{\mathrm{\acute{e}t}}(U_{L}, \mathbb{G}_{m,U_{L}})(p)^{-}
\]
in $\mathcal{D}({\mathbb{Z}}_{p}[G]_{-})$. In particular, $C^{T \bullet}(L/K)$ does not depend on $T$.
\end{prop}

\begin{proof}
Proposition \ref{prop:cohomology_of_Gm} describes the cohomology of $R\Gamma_{\mathrm{\acute{e}t}}(U_{L}, \mathbb{G}_{m,U_{L}})(p)^{-}$ as follows.
First, we have isomorphisms
\[
H^{0}_{\mathrm{\acute{e}t}}(U_{L}, \mathbb{G}_{m,U_{L}})(p)^{-} \simeq \mu_{p}(L), \quad
H^{1}_{\mathrm{\acute{e}t}}(U_{L}, \mathbb{G}_{m,U_{L}})(p)^{-} \simeq A_{L}.
\]
As $L$ is totally complex,
we have $H^{i}_{\mathrm{\acute{e}t}}(U_{L}, \mathbb{G}_{m,U_{L}}) = 0$ for every $i \geq 4$. Finally, we have an exact sequence
\[
0 \longrightarrow H^{2}_{\mathrm{\acute{e}t}}(U_{L}, \mathbb{G}_{m,U_{L}}) \longrightarrow \bigoplus_{w \in S_{\infty}(L)} {\mathrm{Br}}(L_{w}) \longrightarrow {\mathbb{Q}} / {\mathbb{Z}}
\longrightarrow H^{3}_{\mathrm{\acute{e}t}}(U_{L}, \mathbb{G}_{m,U_{L}}) \longrightarrow 0.
\]
However, the Brauer groups ${\mathrm{Br}}(L_{w}) = {\mathrm{Br}}({\mathbb{C}})$ vanish for all archimedean places $w$ of $L$ and $({\mathbb{Q}} / {\mathbb{Z}})^{-} = 0$.
We therefore have $H^{i}_{\mathrm{\acute{e}t}}(U_{L}, \mathbb{G}_{m,U_{L}})(p)^{-} = 0$ for every $i \geq 2$. 
Using Proposition \ref{prop:cohomology_of_GmT} and the assumption that $E_{L,S}^{T}(p)$ is torsionfree,
we likewise find that
\[
H^{i}_{\mathrm{\acute{e}t}}(U_{L}, \mathbb{G}_{m,U_{L}}^{T})(p)^{-} \simeq \left\{ \begin{array}{lll}
A_{L}^{T} & \mbox{ if } & i = 1\\
0 & \mbox{ if } & i \not=1.
\end{array} \right.
\]
Now sequence \eqref{eqn:ses-etale-sheaves} and Lemma \ref{lem:qis-i-zt} together give an exact triangle
\[ \xymatrix{
{R\Gamma_{\mathrm{\acute{e}t}}(U_{L}, \mathbb{G}_{m,U_{L}}^{T})(p)^{-}} \ar[r] \ar[d]^{\simeq} &
R\Gamma_{\mathrm{\acute{e}t}}(U_{L}, \mathbb{G}_{m,U_{L}})(p)^{-} \ar[r] & R\Gamma_{\mathrm{\acute{e}t}}(U_{L}, \iota_{\ast} \mathbb{G}_{m,Z_{T}})(p)^{-} \ar[d]^{\simeq}\\
{A_{L}^{T} [-1]} & & (\mathcal{O}_{L} / \mathfrak{M}_{L}^{T})^{\times}(p)^{-}
}\]
and thus we obtain the required isomorphism in $\mathcal{D}({\mathbb{Z}}_{p}[G]_{-})$.
\end{proof}

We now work at the `infinite level' and use the techniques of Iwasawa theory.
Let $\mathcal{G} := {\mathrm{Gal}}(L_{\infty}/K)$, which we may write as $\mathcal{G} = H \rtimes \Gamma$ where $\Gamma \simeq {\mathbb{Z}}_{p}$ 
and $H$ is finite (see \S \ref{subsec:algebras-in-Iwasawa-thy}). 
Let $\Gamma_{0}$ be an open subgroup of $\Gamma$ that is central in $\mathcal{G}$ and recall from 
 \eqref{eq:Lambda-R-decomp} that $\Lambda(\mathcal{G}):={\mathbb{Z}}_{p}[[\mathcal{G}]]$ is a free
$R := {\mathbb{Z}}_{p}[[\Gamma_{0}]]$-order in $\mathcal{Q}(\mathcal{G})$.
Let $j \in \mathcal{G}$ denote complex conjugation (this an abuse of notation because its image in the quotient group
$G:={\mathrm{Gal}}(L/K)$ is also denoted by $j$) and let $\mathcal{G}^{+} := \mathcal{G} / \langle j \rangle = {\mathrm{Gal}}(L_{\infty}^{+}/K)$.
Then $j \in H$ and so again $\Lambda(\mathcal{G}^{+})$ is a free $R$-order in $\mathcal{Q}(\mathcal{G}^{+})$. 
Moreover, $\Lambda(\mathcal{G})_{-} := \Lambda(\mathcal{G}) / (1+j)$ is also a free $R$-module of finite rank.
For any $\Lambda(\mathcal{G})$-module $M$ we write $M^{+}$ and $M^{-}$ for the submodules of $M$ upon 
which $j$ acts as $1$ and $-1$, respectively, and consider these as modules over $\Lambda(\mathcal{G}^{+})$ and
$\Lambda(\mathcal{G})_{-}$, respectively. 
We note that $M$ is $R$-torsion if and only if both $M^{+}$ and $M^{-}$ are $R$-torsion.
Furthermore, any $R$-module that is finitely generated as a ${\mathbb{Z}}_{p}$-module is necessarily $R$-torsion.

Let $\chi_{\mathrm{cyc}}:\mathcal{G} \rightarrow {\mathbb{Z}}_{p}^{\times}$ denote the $p$-adic cyclotomic character (recall the assumption that $\zeta_{p} \in L$).
Let $\mu_{p^{n}}=\mu_{p^{n}}(L_{\infty})$ denote the group of $p^{n}$th roots of unity in $L_{\infty}^{\times}$
and let $\mu_{p^{\infty}}$ be the nested union (or direct limit) of these groups.
Let ${\mathbb{Z}}_{p}(1):= \varprojlim_{n} \mu_{p^{n}}$ be endowed with the action of $\mathcal{G}$ given by 
$\chi_{\mathrm{cyc}}$.
For any $r \geq 0$ define ${\mathbb{Z}}_{p}(r) := {\mathbb{Z}}_{p}(1)^{\otimes r}$ and ${\mathbb{Z}}_{p}(-r) := {\mathrm{Hom}}_{{\mathbb{Z}}_{p}}({\mathbb{Z}}_{p}(r),{\mathbb{Z}}_{p})$
endowed with the naturally associated actions. 
For any $\Lambda(\mathcal{G})$-module $M$, we define the $r$th Tate twist to be $M(r):= {\mathbb{Z}}_{p}(r) \otimes_{{\mathbb{Z}}_{p}} M$
with the natural $\mathcal{G}$ action; hence $M(r)$ is simply $M$ with the modified $\mathcal{G}$-action 
$g \cdot m = \chi_{\mathrm{cyc}}(g)^{r} g(m)$ for $g \in \mathcal{G}$ and $m \in M$.
In particular, we have ${\mathbb{Q}}_{p} / {\mathbb{Z}}_{p} (1) \simeq \mu_{p^{\infty}}$ and $\Lambda(\mathcal{G}^{+})(-1) \simeq \Lambda(\mathcal{G})_{-}$.
Recall that for a ${\mathbb{Z}}$-module $M$ we previously defined $M(p):={\mathbb{Z}}_{p} \otimes_{\mathbb{Z}} M$; 
we shall use both notations $M(p)$ and $M(r)$ in the sequel, believing the meaning to be clear from context. 
We note that the property of being $R$-torsion is preserved under taking Tate twists. 

For every place $v$ of $K$ we denote the decomposition subgroup of $\mathcal{G}$ at a chosen prime $w_{\infty}$
above $v$ by $\mathcal{G}_{w_{\infty}}$ (everything will only depend on $v$ and not on $w_{\infty}$ in the following).
We note that the index $[\mathcal{G}:\mathcal{G}_{w_{\infty}}]$ is finite when $v$ is a finite place of $K$.

Let $L_{n}$ be the $n$th layer in the cyclotomic ${\mathbb{Z}}_{p}$-extension of $L$.
Then $\varinjlim_{n} C^{T \bullet}(L_{n}/K)$ defines a complex $C^{T \bullet}(L_{\infty}/K)$ in $\mathcal{D}(\Lambda(\mathcal{G})_{-})$.
We define $A_{L_{\infty}} := \varinjlim_{n} A_{L_{n}}$.

\begin{lemma}\label{lem:cohomology_of_C^T}
We have isomorphisms
\[
H^{i}(C^{T \bullet}(L_{\infty}/K)) \simeq \left\{ \begin{array}{lll}
\mu_{p^{\infty}} \simeq {\mathbb{Q}}_{p} / {\mathbb{Z}}_{p} (1) & \mbox{ if } & i=0 \\
A_{L_{\infty}} & \mbox{ if } & i=1 \\
0 & \mbox{ if } & i\not= 0,1.
\end{array} \right.
\]
\end{lemma}

\begin{proof}
Recall that $C^{T \bullet}(L_{n}/K)$ is defined by the middle arrow of the sequence \eqref{eqn:minus-rayclass-sequence} for the layer $L_{n}$.
Taking the direct limit over all $n$ gives an exact sequence of $\Lambda(\mathcal{G})_{-}$-modules
\begin{equation}\label{eqn:limit-sequence}
0 \longrightarrow \mu_{p^{\infty}} \longrightarrow \left( \bigoplus_{v \in T} {\mathrm{ind}}_{\mathcal{G}_{w_{\infty}}}^{\mathcal{G}} \mu_{p^{\infty}} \right)^{-}
 \longrightarrow A_{L_{\infty}}^{T} \longrightarrow A_{L_{\infty}} \longrightarrow 0, 
\end{equation}
where $A_{L_{\infty}}^{T} := \varinjlim_{n} A_{L_{n}}^{T}$.
Thus $C^{T \bullet}(L_{\infty}/K)$ is the complex in degrees $0$ and $1$ given by the middle arrow of \eqref{eqn:limit-sequence}, 
giving the desired result. 
\end{proof}

For every complex $C^{\bullet}$ in $\mathcal{D}(\Lambda(\mathcal{G}))$
we put $C^{\bullet -} := \Lambda(\mathcal{G})_{-} \otimes^{\mathbb{L}}_{\Lambda(\mathcal{G})} C^{\bullet} \in \mathcal{D}(\Lambda(\mathcal{G})_{-})$.
For a finite set $S$ of places of $K$ we let $U_{\infty, S} := {\mathrm{Spec}}(\mathcal{O}_{L_{\infty}, S \cup S_{\infty}})$
and $A_{L_{\infty}, S} := \varinjlim_{n} A_{L_{n}, S \cup S_{\infty}}$.

\begin{lemma}\label{lem:H2-vanishes}
We have isomorphisms
\[
H_{\mathrm{\acute{e}t}}^{i}(U_{\infty, S_{p}}, \mathbb{G}_{m, U_{\infty, S_{p}}})(p)^{-} \simeq \left\{
\begin{array}{lll}
{\mathbb{Z}}_{p} \otimes \mathcal{O}_{L_{\infty}, S_{p} \cup S_{\infty}}^{\times,-} & \mbox{ if } & i=0\\
A_{L_{\infty}, S_{p}} & \mbox{ if } & i=1\\
0 & \mbox{ if } & i \not= 0,1.
\end{array}
\right.
\]
\end{lemma}

\begin{proof}
By \cite[Chapter III, Lemma 1.16]{MR559531} we have isomorphisms
\[
H_{\mathrm{\acute{e}t}}^{i}(U_{\infty, S_{p}}, \mathbb{G}_{m, U_{\infty, S_{p}}}) \simeq
\varinjlim_{n} H_{\mathrm{\acute{e}t}}^{i}(U_{L_{n}, S_{p}}, \mathbb{G}_{m, U_{L_{n}, S_{p}}})
\]
for all $i \geq 0$. 
Thus Proposition \ref{prop:cohomology_of_Gm} immediately implies the result for all $i \neq 2$.
Now let $n_{0}>0$ be sufficiently large such that for $n \geq n_{0}$
and for each $w_{n} \in S_{p}(L_{n})$ there is exactly one $w_{n+1} \in S_{p}(L_{n+1})$ above $w_{n}$.
Then we have $\varinjlim_{n \geq n_{0}} {\mathrm{Br}}(L_{n, w_{n}})(p) \simeq \varinjlim_{n \geq n_{0}} {\mathbb{Q}}_{p} / {\mathbb{Z}}_{p} = 0$,
where the transition maps in the second direct limit
are multiplication by $p$. Hence by Proposition \ref{prop:cohomology_of_Gm} we have
\begin{eqnarray*}
H_{\mathrm{\acute{e}t}}^{2}(U_{\infty, S_{p}}, \mathbb{G}_{m, U_{\infty, S_{p}}})(p)^{-} & \simeq &
\varinjlim_{n} H_{\mathrm{\acute{e}t}}^{2}(U_{L_{n}, S_{p}}, \mathbb{G}_{m, U_{L_{n}, S_{p}}})(p)^{-} \\
& \simeq & (\varinjlim_{n} \bigoplus_{v \in S_{p}} {\mathrm{Br}}(L_{n, w_{n}})(p))^{-} \\
& = & 0
\end{eqnarray*}
as desired.
\end{proof}

The following proposition can be viewed as a `derived version' of results that are well-known at the level of cohomology.
It seems possible that this result is known to experts, but the authors were unable to locate a proof in the literature.

\begin{prop} \label{prop:coh-interpretation-Linfty/K}
There is an isomorphism
\[
C^{T \bullet}(L_{\infty}/K)  \simeq R\Gamma_{\mathrm{\acute{e}t}}(U_{\infty, S_{p}}, \mu_{p^{\infty}})^{-}
\]
in $\mathcal{D}(\Lambda(\mathcal{G})_{-})$. In particular, $C^{T \bullet}(L_{\infty}/K)$ does not depend on $T$.
\end{prop}

\begin{proof}
The assumption that $\zeta_{p} \in L$ is crucial in this proof.
Even though it is not strictly necessary, we first check that the two complexes compute the same cohomology.
Let $M_{S_{p}}$ be the maximal extension of $L_{\infty}$ which is unramified outside $S_{p}$ and let $M_{S_{p}}^{\mathrm{ab}}(p)$ be the maximal abelian
pro-$p$-extension of $L_{\infty}$ inside $M_{S_{p}}$.
We put $H_{S_{p}} := {\mathrm{Gal}}(M_{S_{p}} / L_{\infty})$ and $X_{S_{p}} := {\mathrm{Gal}}(M_{S_{p}}^{\mathrm{ab}}(p) / L_{\infty})$.
There is a canonical isomorphism with Galois cohomology
\begin{equation} \label{eqn:etale-to-Galois}
R\Gamma_{\mathrm{\acute{e}t}}(U_{L_{\infty}, S_{p}}, \mu_{p^{\infty}}) \simeq R\Gamma(H_{S_{p}}, \mu_{p^{\infty}}).
\end{equation}
The strict cohomological $p$-dimension of $H_{S_{p}}$ equals $2$ by \cite[Corollary 10.3.26]{MR2392026} and thus
$H^{i}(H_{S_{p}}, \mu_{p^{\infty}}) = 0$ for all $i \not= 0,1,2$. As the weak Leopoldt conjecture holds for the cyclotomic
${\mathbb{Z}}_{p}$-extension, we also have $H^{2}(H_{S_{p}}, \mu_{p^{\infty}}) = H^{2}(H_{S_{p}}, {\mathbb{Q}}_{p} / {\mathbb{Z}}_{p})(1) = 0$
by \cite[Theorem 11.3.2]{MR2392026}. We clearly have $H^{0}(H_{S_{p}}, \mu_{p^{\infty}}) = \mu_{p^{\infty}}$. Finally
\[
H^{1}(H_{S_{p}}, \mu_{p^{\infty}})^{-} =  {\mathrm{Hom}}(H_{S_{p}}, \mu_{p^{\infty}})^{-} =  {\mathrm{Hom}}(X_{S_{p}}, \mu_{p^{\infty}})^{-}
= {\mathrm{Hom}}(X_{S_{p}}^{+}, \mu_{p^{\infty}}) \simeq  A_{L_{\infty}},
\]
where the last isomorphism is Kummer duality \cite[Theorem 11.4.3]{MR2392026}. 
If we compare this with Lemma \ref{lem:cohomology_of_C^T},
we see that $R\Gamma_{\mathrm{\acute{e}t}}(U_{\infty, S_{p}}, \mu_{p^{\infty}})^{-}$ and $C^{T \bullet}(L_{\infty}/K)$
compute the same cohomology.

In the following we simply write $\mathbb{G}_{m}$ for $\mathbb{G}_{m, U_{\infty, S_{p}}}$.
Recall from \eqref{eqn:change-sigma-to-S} that we have an exact triangle
\[
R\Gamma_{\mathrm{\acute{e}t}}(U_{L}, \mathbb{G}_{m, U_{L}}) \longrightarrow
R\Gamma_{\mathrm{\acute{e}t}}(U_{S_{p}}, \mathbb{G}_{m, U_{S_{p} \cup S_{\infty} }}) \longrightarrow D_{S_{\infty}, S_{p} \cup S_{\infty}}(L)
\]
and similarly for each layer $L_{n}$ in the cyclotomic ${\mathbb{Z}}_{p}$-extension.
We consider (the $p$-minus parts of) these triangles for all $n \geq 0$, and put
 \[
D_{\infty} := \varinjlim_{n} D_{S_{\infty},S_{\infty} \cup S_{p}}(L_{n})(p)^{-}.
\]
Taking the direct limit over all $n$ we obtain an exact triangle
\begin{equation} \label{eqn:first-exact-triangle}
C^{T \bullet}(L_{\infty}/K) \longrightarrow R\Gamma_{\mathrm{\acute{e}t}}(U_{\infty, S_{p}}, \mathbb{G}_{m})(p)^{-}
\stackrel{d_{\infty}}{\longrightarrow} D_{\infty}
\end{equation}
by Proposition \ref{prop:coh-interpretation-L/K} and \cite[Chapter III, Lemma 1.16]{MR559531}.
By Lemma \ref{lem:H2-vanishes} the associated long exact cohomology sequence is
\begin{equation} \label{eqn:long-coh-sequence}
0 \longrightarrow \mu_{p^{\infty}} \longrightarrow {\mathbb{Z}}_{p} \otimes \mathcal{O}_{L_{\infty}, S_{p} \cup S_{\infty}}^{\times,-} 
\longrightarrow H^{0}(D_{\infty}) \longrightarrow A_{L_{\infty}} \longrightarrow A_{L_{\infty}, S_{p}} \longrightarrow 0.
\end{equation}
We see that $H^{i}(D_{\infty})$ vanishes unless $i=0$. 
By Proposition \ref{prop:cohomology-of-cone} we have
\[
H^{0}(D_{S_{\infty},S_{p} \cup S_{\infty}}(L_{n}))(p)^{-} \simeq {\mathbb{Z}}_{p} S_{p}(L_{n})^{-}.
\]
Choose a sufficiently large integer $n_{0}$ such that every place in $S_{p}(L_{n_{0}})$ is totally ramified in $L_{\infty} / L_{n_{0}}$.
We then have isomorphisms
\begin{equation} \label{eqn:D_infty-cohomology}
H^{0}(D_{\infty}) \simeq \varinjlim_{n \geq n_{0}} {\mathbb{Z}}_{p} S_{p}(L_{n})^{-} \simeq {\mathbb{Q}}_{p} S_{p}(L_{n_{0}})^{-},
\end{equation}
where the transition maps in the direct limit are given by multiplication by $p$.
Moreover, the limit of the natural maps $v_{n}: {\mathbb{Z}}_{p} \otimes \mathcal{O}_{L_{n}, S_{p} \cup S_{\infty}}^{\times,-} \rightarrow {\mathbb{Z}}_{p} S_{p}(L_{n})^{-}$
(that are induced from the maps $v_{L_{n}}$ that occur in sequence \eqref{eqn:ray_class_sequence_ZS}) gives a map
\[
v_{\infty}: {\mathbb{Z}}_{p} \otimes \mathcal{O}_{L_{\infty}, S_{p} \cup S_{\infty}}^{\times,-} \longrightarrow {\mathbb{Q}}_{p} S_{p}(L_{n_{0}})^{-} \simeq H^{0}(D_{\infty}).
\]
This map then coincides with the map $H^0(d_{\infty})$ which occurs in the long exact cohomology sequence \eqref{eqn:long-coh-sequence}.

For all integers $a>0$ we have the following commutative diagram of \'{e}tale sheaves on $U_{\infty,S_{p}}$ with exact rows
\[ 
\xymatrix{
{0} \ar[r] & {\mu_{p^{a}}} \ar[r] \ar@{^{(}->}[d] & \mathbb{G}_{m} \ar@{=}[d] \ar[r]^{p^{a}} & \mathbb{G}_{m} \ar[d]^{p} \ar[r] & 0\\
{0} \ar[r] & {\mu_{p^{a+1}}} \ar[r]  & \mathbb{G}_{m}  \ar[r]^{p^{a+1}} & \mathbb{G}_{m} \ar[r] & 0
}
\]
As the cohomology of $D_{\infty}$ is concentrated in degree zero and $H^0(D_{\infty})$ is uniquely $p$-divisible by \eqref{eqn:D_infty-cohomology},
we may construct a commutative diagram
\[
\xymatrixcolsep{.8pc}
\xymatrix{
{R\Gamma_{\mathrm{\acute{e}t}}(U_{\infty, S_{p}}, \mu_{p^{a}})^{-}} \ar[r] \ar[dd] &
R\Gamma_{\mathrm{\acute{e}t}}(U_{\infty, S_{p}}, \mathbb{G}_{m})(p)^{-} \ar[rr]^{p^{a}} \ar@{=}[dd] \ar[dr]^{d_{\infty}} & &
R\Gamma_{\mathrm{\acute{e}t}}(U_{\infty, S_{p}}, \mathbb{G}_{m})(p)^{-} \ar@{->}'[d]^(.75){p}[dd] \ar[dr] & \\
& & D_{\infty} \ar@{=}[rr] \ar@{=}[dd] & & D_{\infty} \ar@{=}[dd] \\
{R\Gamma_{\mathrm{\acute{e}t}}(U_{\infty, S_{p}}, \mu_{p^{a+1}})^{-}} \ar[r] &
R\Gamma_{\mathrm{\acute{e}t}}(U_{\infty, S_{p}}, \mathbb{G}_{m})(p)^{-} \ar@{->}'[r][rr]^(.25){p^{a+1}} \ar[dr]^{d_{\infty}} & &
R\Gamma_{\mathrm{\acute{e}t}}(U_{\infty, S_{p}}, \mathbb{G}_{m})(p)^{-} \ar[dr] & \\
& & D_{\infty} \ar@{=}[rr] & & D_{\infty}
}\]
where the rows are exact triangles and $d_{\infty}$ was defined in \eqref{eqn:first-exact-triangle}. 
Taking direct limits we obtain a commutative diagram
\[ \xymatrix{
{R\Gamma_{\mathrm{\acute{e}t}}(U_{\infty, S_{p}}, \mu_{p^{\infty}})^{-}} \ar[r] &
R\Gamma_{\mathrm{\acute{e}t}}(U_{\infty, S_{p}}, \mathbb{G}_{m})(p)^{-} \ar[r] \ar[d]^{d_{\infty}} &
{\mathbb{Q}}_{p} \otimes_{{\mathbb{Z}}_{p}}^{\mathbb{L}} R\Gamma_{\mathrm{\acute{e}t}}(U_{\infty, S_{p}}, \mathbb{G}_{m})(p)^{-} \ar[d]^{\hat{d}_{\infty}} \\
& D_{\infty} \ar@{=}[r] & D_{\infty}
}\]
where the upper row is an exact triangle. 
We claim that $\hat{d}_{\infty}$ is an isomorphism in $\mathcal{D}(\Lambda(\mathcal{G})_{-})$.
Indeed, the cohomology of ${\mathbb{Q}}_{p} \otimes_{{\mathbb{Z}}_{p}}^{\mathbb{L}} R\Gamma_{\mathrm{\acute{e}t}}(U_{\infty, S_{p}}, \mathbb{G}_{m})(p)^{-}$
is concentrated in degree zero by Lemma \ref{lem:H2-vanishes}, and
\[
H^0(\hat{d}_{\infty}): {\mathbb{Q}}_{p} \otimes \mathcal{O}_{L_{\infty}, S_{p} \cup S_{\infty}}^{\times,-}
\stackrel{\simeq}{\longrightarrow} {\mathbb{Q}}_{p} S_{p}(L_{n_{0}})^{-}
\]
is induced by $v_{\infty}$.
In summary, we have a diagram
\[
\xymatrix{
{R\Gamma_{\mathrm{\acute{e}t}}(U_{\infty, S_{p}}, \mu_{p^{\infty}})^{-}} \ar[r] &
R\Gamma_{\mathrm{\acute{e}t}}(U_{\infty, S_{p}}, \mathbb{G}_{m})(p)^{-} \ar[r] \ar@{=}[d] &
{\mathbb{Q}}_{p} \otimes_{{\mathbb{Z}}_{p}}^{\mathbb{L}} R\Gamma_{\mathrm{\acute{e}t}}(U_{\infty, S_{p}}, \mathbb{G}_{m})(p)^{-} \ar[d]^{\hat{d}_{\infty}} \\
{C^{T \bullet}(L_{\infty}/K)} \ar[r] &
R\Gamma_{\mathrm{\acute{e}t}}(U_{\infty, S_{p}}, \mathbb{G}_{m})(p)^{-} \ar[r]^{d_{\infty}} & D_{\infty}
}\]
where the rows are exact triangles and $\hat{d}_{\infty}$ is an isomorphism in $\mathcal{D}(\Lambda(\mathcal{G})_{-})$.
Thus also
\[
R\Gamma_{\mathrm{\acute{e}t}}(U_{\infty, S_{p}}, \mu_{p^{\infty}})^{-} \simeq C^{T \bullet}(L_{\infty}/K)
\]
in $\mathcal{D}(\Lambda(\mathcal{G})_{-})$ as desired.
 \end{proof}

For a finite set $S$ of places of $K$ containing $S_{p} \cup S_{\infty}$ recall the definition of the complex
\[
C_{S}^{\bullet}(L_{\infty}^{+}/K) := R\Gamma_{\mathrm{\acute{e}t}}({\mathrm{Spec}}(\mathcal{O}_{L_{\infty}^{+},S}), {\mathbb{Q}}_{p} / {\mathbb{Z}}_{p})^{\vee}
\in \mathcal{D}(\Lambda({\mathrm{Gal}}(L_{\infty}^{+}/K)))
\]
which occurs in the EIMC. 
For an integer $m$ we let $C_{S}^{\bullet}(L_{\infty}^{+}/K)(m) := {\mathbb{Z}}_{p}(m) \otimes^{\mathbb{L}}_{\mathbb{Z}} C_{S}^{\bullet}(L_{\infty}^{+}/K)$
be the $m$-fold Tate twist.

\begin{corollary}\label{cor:isom-cT-complex}
We have $C_{S_{p}}^{\bullet}(L_{\infty}^{+}/K)(-1) \simeq C^{T \bullet}(L_{\infty}/K)^{\vee}$ in $\mathcal{D}(\Lambda(\mathcal{G})_{-})$. 
\end{corollary}

\begin{lemma}\label{lem:IT-proj-dim-at-most-one}
Let $T'$ be a non-empty finite set of places of $K$ disjoint from $S_{p} \cup S_{ram} \cup S_{\infty}$
and let 
$
I_{T'}:=\left( \bigoplus_{v\in T'} {\mathrm{ind}}_{\mathcal{G}_{w_{\infty}}}^{\mathcal{G}} {\mathbb{Z}}_{p}(-1)\right)^{-}.
$
Then 
\begin{enumerate}
\item $I_{T'}$ is an $\Lambda(\mathcal{G})_{-}$-module of projective dimension at most one,
\item for any $\Lambda(\mathcal{G})_{-}$-module $M$ we have ${\mathrm{Ext}}^{i}_{\Lambda(\mathcal{G})_{-}} (I_{T'}, M)=0$ for $i \geq 2$, and
\item $I_{T'}$ is $R$-torsion.
\end{enumerate}
\end{lemma}

\begin{proof}
For each $v \in T'$ we have an exact sequence of $\Lambda(\mathcal{G}_{w_{\infty}})$-modules
\begin{equation}\label{eqn:resolution-zp(-1)}
0 \longrightarrow \Lambda(\mathcal{G}_{w_{\infty}}) \longrightarrow \Lambda(\mathcal{G}_{w_{\infty}}) \longrightarrow {\mathbb{Z}}_{p}(-1) \longrightarrow 0,
\end{equation}
where the injection is right multiplication by $1 - \chi_{\mathrm{cyc}}(\phi_{w_{\infty}}) \phi_{w_{\infty}}$
and $\phi_{w_{\infty}}$ denotes the Frobenius automorphism at $w_{\infty}$. 
Claim (i) now follow easily, and this implies claim (ii).
For each $v \in T'$ the index $[\mathcal{G}:\mathcal{G}_{w_{\infty}}]$ is finite; 
thus $I_{T'}$ is finitely generated as a ${\mathbb{Z}}_{p}$-module, giving claim (iii).
\end{proof}

\begin{prop}\label{prop:construction-of-YST(-1)}
There exists a $\Lambda(\mathcal{G})_{-}$-module $Y_{S}^{T} (-1)$ and a commutative diagram
\begin{equation}\label{eqn:complex-diagram} 
\xymatrix{
{0} \ar[r] & X_{S}^{+} (-1) \ar[r] \ar[d] & Y_{S}^{T} (-1) \ar[r] \ar[d] &
I_{T} \ar[r] \ar@{=}[d] & {\mathbb{Z}}_{p}(-1) \ar[r] \ar@{=}[d] & 0\\
{0} \ar[r] & X_{S_{p}}^{+} (-1) \ar[r] \ar[d] &  {\mathrm{Hom}}(A_{L_{\infty}}^{T}, {\mathbb{Q}}_{p} / {\mathbb{Z}}_{p}) \ar[r] \ar[d] &
I_{T} \ar[r] &  {\mathbb{Z}}_{p}(-1) \ar[r] & 0,\\
& 0 & 0 & & &
}\end{equation}
with exact rows and columns
where the middle two terms of upper and lower rows (concentrated in degrees $-1$ and $0$) represent $C_{S}^{\bullet}(L_{\infty}^{+}/K)(-1)$ and $C_{S_{p}}^{\bullet}(L_{\infty}^{+}/K)(-1)$,
respectively.
\end{prop}

\begin{proof}
Using \eqref{eqn:limit-sequence} and Corollary \ref{cor:isom-cT-complex}
we see that for every finite set $T$ of places of $K$ such that ${\mathrm{Hyp}}(S_{p} \cup S_{\mathrm{ram}} \cup S_{\infty},T)$ is satisfied,
the extension class of the complex $C_{S_{p}}^{\bullet}(L_{\infty}^{+}/K)(-1)$ in
${\mathrm{Ext}}^{2}_{\Lambda(\mathcal{G})_{-}} ({\mathbb{Z}}_{p}(-1), X_{S_{p}}^{+} (-1))$ may be represented by the exact sequence
\begin{equation}\label{eqn:ext-rep-I_T}
0 \longrightarrow X_{S_{p}}^{+} (-1) \longrightarrow {\mathrm{Hom}}(A_{L_{\infty}}^{T}, {\mathbb{Q}}_{p} / {\mathbb{Z}}_{p}) \longrightarrow
I_{T} \longrightarrow {\mathbb{Z}}_{p}(-1) \longrightarrow 0.
\end{equation}

Let $Z_{S}$ be the kernel of the natural surjection $X_{S}^{+}(-1) \twoheadrightarrow X_{S_{p}}^{+}(-1)$
and let $W_{T}$ be the kernel of the right-most surjection in \eqref{eqn:ext-rep-I_T}.
Then the short exact sequences
\[
0 \longrightarrow W_{T} \longrightarrow I_{T} \longrightarrow {\mathbb{Z}}_{p}(-1) \longrightarrow 0, \qquad 
0 \longrightarrow Z_{S} \longrightarrow X_{S}^{+}(-1) \longrightarrow X_{S_{p}}^{+}(-1) \longrightarrow 0,
\]
induce long exact sequences in cohomology, which by Lemma \ref{lem:IT-proj-dim-at-most-one} (ii) give the following diagram 
\[
\xymatrix{
{\mathrm{Ext}}^{1}(W_{T},Z_{S}) \ar[r] \ar[d] &
{\mathrm{Ext}}^{2}({\mathbb{Z}}_{p}(-1),Z_{S}) \ar[r] \ar[d] & 0\\
{\mathrm{Ext}}^{1}(W_{T},X_{S}^{+}(-1)) \ar[r]^{\alpha_{S}} \ar[d]^{\gamma} & {\mathrm{Ext}}^{2}({\mathbb{Z}}_{p}(-1),X_{S}^{+}(-1)) \ar[r] \ar[d]^{\beta} & 0\\
{\mathrm{Ext}}^{1}(W_{T},X_{S_{p}}^{+}(-1)) \ar[r]^{\alpha_{p}} \ar[d] & {\mathrm{Ext}}^{2}({\mathbb{Z}}_{p}(-1),X_{S_{p}}^{+}(-1))  \ar[r] \ar[d] & 0\\
{\mathrm{Ext}}^{2}(W_{T},Z_{S}) \ar[r]^{\sim}  & {\mathrm{Ext}}^{3}({\mathbb{Z}}_{p}(-1),Z_{S}), &
}
\]
where we have omitted the subscript $\Lambda(\mathcal{G})_{-}$ from all ${\mathrm{Ext}}$-groups, 
and all rows and columns are exact. 
Note that the top two squares are commutative (see \cite[(7.3), p.\ 140]{MR1438546}) and that the bottom square is anticommutative
(see  \cite[Exercise 9.9, p.\ 156]{MR1438546}). 
By \cite[Lemma 2.4]{MR2371375} $\beta$ maps (the class of) $C_{S}^{\bullet}(L_{\infty}^{+}/K)(-1)$ to 
$C_{S_{p}}^{\bullet}(L_{\infty}^{+}/K)(-1)$.
Let 
\[
\varepsilon := [ 0 \longrightarrow X_{S_{p}}^{+} (-1) \longrightarrow {\mathrm{Hom}}(A_{L_{\infty}}^{T}, {\mathbb{Q}}_{p} / {\mathbb{Z}}_{p}) \longrightarrow
W_{T} \longrightarrow 0] \in {\mathrm{Ext}}_{\Lambda(\mathcal{G})_{-}}^{1}(W_{T},X_{S_{p}}^{+}(-1)).
\]
Then the fact that $C_{S_{p}}^{\bullet}(L_{\infty}^{+}/K)(-1)$ is represented by 
\eqref{eqn:ext-rep-I_T} shows that $\alpha_{p}$ maps $\varepsilon$ to $C_{S_{p}}^{\bullet}(L_{\infty}^{+}/K)(-1)$.
Now a diagram chase shows that there exists a preimage of $\varepsilon$ under $\gamma$ that is mapped to 
$C_{S}^{\bullet}(L_{\infty}^{+}/K)(-1)$ by $\alpha_{S}$.
\end{proof}

The proof of the following lemma explains why we write $Y_{S}^{T} (-1)$ rather than $Y_{S}^{T}$.
Note that every $\Lambda(\mathcal{G})_{-}$-module $M$ may be written as an $n$-fold Tate twist for every $n \in {\mathbb{Z}}$;
simply write $M = M(-n)(n)$.

\begin{lemma}\label{lem:YST(-1)-has-proj-dim-at-most-one}
The projective dimension of the $\Lambda(\mathcal{G})_{-}$-module $Y_{S}^{T} (-1)$ is at most one.
Moreover, $Y_{S}^{T} (-1)$ is $R$-torsion.
\end{lemma}

\begin{proof}
Let $T \subseteq T''$ be a second finite set of places of $K$ such that ${\mathrm{Hyp}}(S,T'')$ is satisfied and let $T':=T''-T$.
The short exact sequence
\[
0 \longrightarrow W_{T} \longrightarrow W_{T''} \longrightarrow I_{T'} \longrightarrow 0
\]
induces a long exact sequence in cohomology which by Lemma \ref{lem:IT-proj-dim-at-most-one} (ii) becomes
\[
{\mathrm{Ext}}^{1}_{\Lambda(\mathcal{G})_{-}}(W_{T''},X_{S}^{+}(-1)) 
\longrightarrow {\mathrm{Ext}}^{1}_{\Lambda(\mathcal{G})_{-}}(W_{T},X_{S}^{+}(-1)) \longrightarrow 0.
\]
Thus using top row of \eqref{eqn:complex-diagram} for $T$ and $T''$
we have a commutative diagram 
\begin{equation*}\label{eqn:change-of-Ts} 
\xymatrix{
{0} \ar[r] & X_{S}^{+} (-1) \ar[r] \ar@{=}[d] & Y_{S}^{T}(-1) \ar[r] \ar[d] &
I_{T} \ar[r] \ar[d] & {\mathbb{Z}}_{p}(-1) \ar[r] \ar@{=}[d] & 0\\
{0} \ar[r] & X_{S}^{+} (-1) \ar[r] &  Y_{S}^{T''}(-1)  \ar[r] &
I_{T''} \ar[r] &  {\mathbb{Z}}_{p}(-1) \ar[r] & 0.
}
\end{equation*}
Applying the snake lemma now gives a short exact sequence
\[
0 \longrightarrow Y_{S}^{T} (-1) \longrightarrow Y_{S}^{T''} (-1) \longrightarrow I_{T'} \longrightarrow 0.
\]
Therefore Lemma \ref{lem:IT-proj-dim-at-most-one} shows that the claim does not depend on the particular choice of $T$ and so by Remark \ref{rmk:conditions-on-T} we can and do assume that $T$ consists of a single place $v$.

Recall that $\mathcal{G}^{+} := \mathcal{G} / \langle j \rangle= {\mathrm{Gal}}(L_{\infty}^{+}/K)$ and let $\mathcal{G}^{+}_{w_{\infty}^{+}}$
denote the decomposition subgroup at $w_{\infty}^{+}$,
where $w_{\infty}^{+}$ denotes the place of $L_{\infty}^{+}$ below $w_{\infty}$. 
Then we have an isomorphism of $\Lambda(\mathcal{G}^{+})$-modules
\begin{equation} \label{eqn:ind-Zp-iso}
({\mathrm{ind}}_{\mathcal{G}_{w_{\infty}}}^{\mathcal{G}} {\mathbb{Z}}_{p}(-1))^{-}(1) \simeq {\mathrm{ind}}_{\mathcal{G}^{+}_{w^{+}_{\infty}}}^{\mathcal{G}^{+}} {\mathbb{Z}}_{p}.
\end{equation}

Let $\Delta(\mathcal{G}^{+})$ denote the kernel of the augmentation map $\Lambda(\mathcal{G}^{+}) \twoheadrightarrow {\mathbb{Z}}_{p}$.
Given a $\Lambda(\mathcal{G}^{+})$-monomorphism $\psi:\Lambda(\mathcal{G}^{+}) \rightarrow \Delta(\mathcal{G}^{+})$, 
Ritter and Weiss \cite[\S 4]{MR2114937} construct a four term exact sequence whose class in 
${\mathrm{Ext}}^{2}_{\Lambda(\mathcal{G}^{+})}({\mathbb{Z}}_{p},X_{S}^{+})$ is in fact independent of the choice of $\psi$.
To make this construction explicit, we now choose $\psi$ to be given by right multiplication with $(1 - \phi_{w_{\infty}^{+}})$, 
where $\phi_{w_{\infty}^{+}}$ denotes the Frobenius automorphism at $w_{\infty}^{+}$.
Let $\tilde{\psi}$ be $\psi$ followed by the inclusion $\Delta(\mathcal{G}^{+}) \subset \Lambda(\mathcal{G}^{+})$.
Then the cokernel of $\tilde{\psi}$ identifies with ${\mathrm{ind}}_{\mathcal{G}^{+}_{w^{+}_{\infty}}}^{\mathcal{G}^{+}} {\mathbb{Z}}_{p}$, 
and the Ritter and Weiss construction gives a four term exact sequence
\[
0 \longrightarrow X_{S}^{+} \longrightarrow \tilde{Y}^{\{v\}}_{S} \longrightarrow {\mathrm{ind}}_{\mathcal{G}^{+}_{w^{+}_{\infty}}}^{\mathcal{G}^{+}} {\mathbb{Z}}_{p}
\longrightarrow {\mathbb{Z}}_{p} \longrightarrow 0,
\]
where the $\Lambda(\mathcal{G}^{+})$-module $\tilde{Y}^{\{v\}}_{S}$ has projective dimension at most one
and is $R$-torsion.

It follows from \cite[Theorem 2.4]{MR3072281}, Corollary \ref{cor:isom-cT-complex} and \eqref{eqn:ind-Zp-iso} 
that we have a commutative diagram
\[ 
\xymatrix{
{0} \ar[r] & X_{S}^{+} \ar[r] \ar@{=}[d] & Y_{S}^{\{v\}}  \ar[r] \ar[d] &
{\mathrm{ind}}_{\mathcal{G}^{+}_{w^{+}_{\infty}}}^{\mathcal{G}^{+}} {\mathbb{Z}}_{p} \ar[r] \ar@{=}[d] & {\mathbb{Z}}_{p} \ar[r] \ar@{=}[d] & 0\\
{0} \ar[r] & X_{S}^{+} \ar[r] &  \tilde{Y}^{\{v\}}_{S} \ar[r] &
{\mathrm{ind}}_{\mathcal{G}^{+}_{w^{+}_{\infty}}}^{\mathcal{G}^{+}} {\mathbb{Z}}_{p} \ar[r] &  {\mathbb{Z}}_{p} \ar[r] & 0.
}
\]
Hence the vertical arrow must be an isomorphism, and thus the projective dimension of the $\Lambda(\mathcal{G}^{+})$-module
$Y_{S}^{\{v\}}$ is at most one and $Y_{S}^{\{v\}}$ is $R$-torsion.
Therefore the same claims are true of the $\Lambda(\mathcal{G})_{-}$-module $Y_{S}^{\{v\}} (-1)$.
\end{proof}

Recall the definitions of $\chi_{\mathrm{cyc}}, \omega, \kappa$ and $j_{\chi}^{r}$
from \S \ref{subsec:EIMC-reformulation}.
Let $F$ be a sufficiently large finite extension of ${\mathbb{Q}}_{p}$ (in the sense of \S \ref{subsec:sufficiently-large}).
For $s \in {\mathbb{Z}}$ let $x \mapsto t_{\mathrm{cyc}}^{s}(x)$ and $x \mapsto t_{\omega}^{s}(x)$
be the automorphisms on $\mathcal{Q}^{F}(\mathcal{G})$ induced by 
$g \mapsto \chi_{\mathrm{cyc}}^{s}(g)g$ 
and $g \mapsto \omega^{s}(g)g$ for $g \in \mathcal{G}$, respectively. 

\begin{lemma}  \label{lem:jchi-tcyc-composition}
Let $r,s \in {\mathbb{Z}}$ and $\chi \in {\mathrm{Irr}}_{{\mathbb{Q}}_p^{c}}(\mathcal{G})$. Then for every $x \in \zeta(\mathcal{Q}^{F}(\mathcal{G}))$
we have $j_{\chi}^{r}( t_{\mathrm{cyc}}^{s}(x)) = j_{\chi \omega^{s}}^{r+s}(x)$. 
\end{lemma}

\begin{proof}
 Recall the definitions of $w_{\chi}$ and $\gamma_{\chi}$ from \S \ref{subsec:idempotents} and 
\S \ref{subsec:sufficiently-large}, respectively. It follows easily from the definitions that $w_{\chi} = w_{\chi \omega}$.
We claim that we have an equality
\begin{equation} \label{eqn:gammachi}
t_{\omega}^{1}(\gamma_{\chi \omega}) = \gamma_{\chi}.
\end{equation}
Write $\gamma_{\chi \omega} = g_{\chi \omega} c_{\chi \omega}$ with
$g_{\chi \omega} \in \mathcal{G}$ and $c_{\chi \omega} \in ({\mathbb{Q}}_p^{c}[H] e_{\chi \omega})^{\times}$
as in \S \ref{subsec:sufficiently-large}. Put $g_{\chi} := g_{\chi \omega}$ and
$c_{\chi} := \omega(g_{\chi \omega}) t_{\omega}^{1}(c_{\chi \omega})$.
It is then easily checked that $g_{\chi} c_{\chi}$ has the defining properties of $\gamma_{\chi}$, and thus
$t_{\omega}^{1}(\gamma_{\chi \omega}) = g_{\chi} c_{\chi} = \gamma_{\chi}$. This shows \eqref{eqn:gammachi}.
Recalling that $u = \kappa(\gamma_{K})$ we compute
\[
t_{\mathrm{cyc}}^{1}(\gamma_{\chi \omega}) 
= \chi_{\mathrm{cyc}}(g_{\chi \omega}) g_{\chi \omega} t_{\omega}^{1} (c_{\chi \omega})
= u^{w_{\chi \omega}} t_{\omega}^{1}(\gamma_{\chi \omega})
=  u^{w_{\chi}} \gamma_{\chi},
\]
where we have used \eqref{eqn:gammachi} for the last equality. Finally, we have that
\[
j_{\chi}^{r} (t_{\mathrm{cyc}}^{s}(\gamma_{\chi \omega^{s}}))
= j_{\chi}^{r} (u^{w_{\chi} \cdot s} \gamma_{\chi})
=  u^{w_{\chi} \cdot s}(u^{r} \gamma_{K})^{w_{\chi}}
= (u^{r+s} \gamma_{K})^{w_{\chi}}
= j_{\chi \omega^{s}}^{r+s}(\gamma_{\chi \omega^{s}})
\]
for every $r,s \in {\mathbb{Z}}$ as desired.
\end{proof}

We will henceforth assume that all ramified places belong to $S$. 
For $v \in T$ we put
\[
\xi_{v} := {\mathrm{nr}}(1 - \chi_{\mathrm{cyc}}(\phi_{w_{\infty}}) \phi_{w_{\infty}}).
\]
We put
\[
\Psi_{S,T} = \Psi_{S,T}(L_{\infty} / K) :=  t_{\mathrm{cyc}}^{1}(\Phi_{S}) \cdot \prod_{v \in T} \xi_{v}.
\]
Note that this slightly differs from the corresponding element $\Psi_{S,T}$ in \cite{MR3072281}.

\begin{prop}\label{prop:EIMC-gives-Fitting}
Suppose that the EIMC holds for $L_{\infty}^{+}/K$. 
Then $\Psi_{S,T}$ is a generator of ${\mathrm{Fitt}}_{\Lambda(\mathcal{G})_{-}}(Y_{S}^{T} (-1))$.
\end{prop}

\begin{proof}
Since the EIMC holds, by definition \eqref{eqn:fitt-of-complex} we have that 
$\Phi_{S}^{-1}$ generates the Fitting invariant of $C_{S}^{\bullet}(L_{\infty}^{+}/K) \in \mathcal{D}^{\mathrm{perf}}{_{\mathrm{tor}}}(\Lambda(\mathcal{G}^{+}))$.
However, $t_{\mathrm{cyc}}^{1}$ induces an isomorphism $\Lambda(\mathcal{G}^{+})(-1) \simeq \Lambda(\mathcal{G})_{-}$ and so 
$t_{\mathrm{cyc}}^{1}(\Phi_{S})^{-1}$ generates the Fitting invariant of $C_{S}^{\bullet}(L_{\infty}^{+}/K)(-1) \in \mathcal{D}^{\mathrm{perf}}{_{\mathrm{tor}}}(\Lambda(\mathcal{G})_{-})$.
Lemmas \ref{lem:IT-proj-dim-at-most-one} and \ref{lem:YST(-1)-has-proj-dim-at-most-one} show that 
$I_{T}$ and $Y_{S}^{T} (-1)$ are both $R$-torsion $\Lambda(\mathcal{G})_{-}$-modules of projective dimension at most one; 
hence they both have quadratic presentations by \cite[Lemma 6.2]{MR2609173}.
Therefore combining Proposition \ref{prop:construction-of-YST(-1)}
and Lemma \ref{lem:fitt-eq-complex-quad} gives
\[
{\mathrm{Fitt}}_{\Lambda(\mathcal{G})_{-}}(Y_{S}^{T} (-1))  =  {\mathrm{Fitt}}_{\Lambda(\mathcal{G})_{-}}\left(C_{S}^{\bullet}(L_{\infty}^{+}/K)(-1)\right)^{-1}
\cdot {\mathrm{Fitt}}_{\Lambda(\mathcal{G})_{-}} (I_{T}).
\]
The exact sequence \eqref{eqn:resolution-zp(-1)} shows that each $({\mathrm{ind}}_{\mathcal{G}_{w_{\infty}}}^{\mathcal{G}} {\mathbb{Z}}_{p}(-1))^{-}$
has a quadratic presentation and that its Fitting invariant is generated by $\xi_{v}$.
Hence Lemma \ref{lem:Fitting-properties} (ii) gives
\[
{\mathrm{Fitt}}_{\Lambda(\mathcal{G})_{-}}(I_{T}) = 
\prod_{v\in T} {\mathrm{Fitt}}_{\Lambda(\mathcal{G})_{-}}\left(({\mathrm{ind}}_{\mathcal{G}_{w_{\infty}}}^{\mathcal{G}} {\mathbb{Z}}_{p}(-1))^{-}\right) = 
\prod_{v \in T} \left[ \langle  \xi_{v} \rangle_{\zeta(\Lambda(\mathcal{G})_{-})}\right]_{{\mathrm{nr}}(\zeta(\Lambda(\mathcal{G})_{-})},
\]
and we therefore obtain the desired result.
\end{proof}

We now suppose that the EIMC holds. The surjection
\[
Y_{S}^{T} (-1) \longrightarrow {\mathrm{Hom}}(A_{L_{\infty}}^{T}, {\mathbb{Q}}_{p} / {\mathbb{Z}}_{p}) \longrightarrow 0
\]
in diagram \eqref{eqn:complex-diagram}, Lemma \ref{lem:Fitting-properties} (i) and Proposition \ref{prop:EIMC-gives-Fitting} then imply that
\[
\Psi_{S,T} \in {\mathrm{Fitt}}_{\Lambda(\mathcal{G})_{-}}^{\max}({\mathrm{Hom}}(A_{L_{\infty}}^{T}, {\mathbb{Q}}_{p} / {\mathbb{Z}}_{p})).
\]
As the transition maps in the direct limit $A_{L_{\infty}}^{T} = \varinjlim_{n} A_{L_{n}}^{T}$ are injective
by \cite[Lemma 2.9]{GrP-EIMC}, the transition maps in the projective limit
${\mathrm{Hom}}(A_{L_{\infty}}^{T}, {\mathbb{Q}}_{p} / {\mathbb{Z}}_{p}) = \varprojlim_{n} (A_{L_{n}}^{T})^{\vee}$ are surjective.
As $\Gamma_{L}:={\mathrm{Gal}}(L_{\infty}/L)$ clearly acts trivially on $(A_{L}^{T})^{\vee}$, we have a surjection
\begin{equation} \label{eqn:epi-descent}
{\mathrm{Hom}}(A_{L_{\infty}}^{T}, {\mathbb{Q}}_{p} / {\mathbb{Z}}_{p})_{\Gamma_{L}} \longrightarrow (A_{L}^{T})^{\vee} \longrightarrow 0.
\end{equation}
Fix an odd character $\chi  \in {\mathrm{Irr}}_{{\mathbb{Q}}_{p}^{c}}(G)$ and view $\chi$ as an irreducible character of $\mathcal{G}$ with open kernel.
We have 
\[
\phi(j_{\chi}(t_{\mathrm{cyc}}^{1}(\Phi_{S}))) = \phi(j_{\chi \omega}^{1}(\Phi_{S})) = L_{p,S}(0, \chi \omega),
\]
where the first and second equalities follow from Lemma \ref{lem:jchi-tcyc-composition} and  \eqref{eq:PhiS-jr-p-adic}, respectively.
Moreover, $\phi(j_{\chi}(\prod_{v \in T} \xi_{v})) = \delta_{T}(0,\check \chi)$,
where $\check \chi$ denotes the character contragredient to $\chi$.
As Fitting invariants behave well under base change by Proposition \ref{prop:Fitting-descent}, we have
\[
(\theta_{p,S}^{T})^{\sharp} = \sum_{\chi \in {\mathrm{Irr}}_{{\mathbb{Q}}_{p}^{c}}(G)} \phi(j_{\chi}(\Psi_{S,T})) e(\chi) \in
{\mathrm{Fitt}}_{{\mathbb{Z}}_{p}[G]_{-}}^{\max}({\mathrm{Hom}}(A_{L_{\infty}}^{T}, {\mathbb{Q}}_{p} / {\mathbb{Z}}_{p})_{\Gamma_{L}}) \subseteq
{\mathrm{Fitt}}_{{\mathbb{Z}}_{p}[G]_{-}}^{\max}((A_{L}^{T})^{\vee}),
\]
where we have again used Lemma \ref{lem:Fitting-properties} (i).
This completes the proof of Theorem \ref{thm:EIMC-implies-BS}.
\end{proof}

\bibliography{hybrid-iwasawa-Bib}{}
\bibliographystyle{amsalpha}

\end{document} 

\documentclass[11pt,english,a4paper]{smfart}
\usepackage[english]{babel}
\selectlanguage{english}
\marginparwidth=10 true mm
\oddsidemargin=0 true mm
\evensidemargin=0 true mm
\marginparsep=5 true mm
\topmargin=0 true mm
\headheight=8 true mm
\headsep=4 true mm
\topskip=0 true mm
\footskip=15 true mm

\numberwithin{equation}{section}

 \raggedbottom
\setlength{\textwidth}{150 true mm}
\setlength{\textheight}{220 true mm}
\setlength{\hoffset}{8 true mm}
\setlength{\voffset}{2 true mm}

\usepackage{enumerate}
\usepackage{hyperref,cite}
\usepackage{amsmath,amssymb, bm,amsfonts}
\usepackage{bbm}

\usepackage[all]{xy}
\entrymodifiers={+!!<0pt,\fontdimen22\textfont2>}

\usepackage{ mathrsfs }

\newtheorem{theorem}{Theorem}[section]
\newtheorem{lemma}[theorem]{Lemma}
\newtheorem{observation}[theorem]{Observation}
\newtheorem{corollary}[theorem]{Corollary}
\newtheorem{proposition}[theorem]{Proposition}
\newtheorem*{theo1}{Theorem 1}
\newtheorem*{theo2}{Theorem 2}
\newtheorem*{theo3}{Theorem 3}
\newtheorem*{theo4}{Theorem 4}
\newtheorem*{theo5}{Theorem 5}
\newtheorem*{theo6}{Theorem 6}
\newtheorem*{theo7}{Theorem 7}
\newtheorem*{theo8}{Theorem 8}
\newtheorem*{coro*}{Corollary}

{\theoremstyle{definition}\newtheorem{notation}[theorem]{Notation}}

{\theoremstyle{definition}\newtheorem{example}[theorem]{Example}}

\theoremstyle{definition}
\newtheorem{definition}[theorem]{Definition}
\newtheorem{question}[theorem]{Question}
\newtheorem{fact}[theorem]{Fact}
\newtheorem{claim}[theorem]{Claim}

{\theoremstyle{definition}\newtheorem{remark}[theorem]{Remark}}

\author{Catalin Badea}
\address{Universit\'{e} de Lille, CNRS, UMR 8524 - Laboratoire Paul Painlev\'{e}, 59000 Lille, France}
\email{catalin.badea@univ-lille1.fr}
\author{Sophie Grivaux}
\address{CNRS,
Laboratoire Ami\'enois de Math\'{e}matique Fondamentale et Appliqu\'{e}e, UMR 7352,
Universit\'{e} de Picardie Jules Verne,
33 rue Saint-Leu,
80039 Amiens Cedex 1,
France}
\email{sophie.grivaux@u-picardie.fr}
\date{\today}
\title[Kazhdan sets and equidistribution properties]{Kazhdan sets in groups and equidistribution properties}

\begin{document}
\begin{abstract}
 We study {Kazhdan}\ sets in topological groups which do not necessarily have 
Property (T). We provide a new criterion for a subset ${Q}$ of 
a group $G$ which generates $G$ to be a {Kazhdan}\ set; it relies on the 
existence of a positive number $\varepsilon$ such that every unitary representation of $G$ with a 
$({Q},\varepsilon )$-invariant vector has a finite dimensional 
subrepresentation. Using this result, we give an equidistribution 
criterion for a subset of $G$ which generates $G$ to be a {Kazhdan}\ set. In the 
case 
where $G={\ensuremath{\mathbb Z}}$, this shows that if $(n_{k})_{k\ge 1}$ is a sequence of integers such that $(e^{2i\pi \theta 
n_{k}})_{k\ge 1}$ is uniformly distributed in the unit circle for all 
real numbers $\theta $ except at most countably many, 
$\{n_{k}\,;\,k\ge 1\}$ is a {Kazhdan}\ set in ${\ensuremath{\mathbb Z}}$ as soon as it generates ${\ensuremath{\mathbb Z}}$. 
This answers a question of Y.\ Shalom. 
We also obtain a complete characterization of {Kazhdan}\ sets in second countable locally compact abelian groups, in the Heisenberg groups, the $ax+b$ group, and $SL_2({\ensuremath{\mathbb R}})$.
This answers in particular a question from 
[B.~Bekka, P.~de la~Harpe, A.~Valette, Kazhdan's property (T), 
Cambridge Univ. Press, 2008].
Lastly, we provide some further applications of {Kazhdan}\ sets: we generalize a local rigidity result of Rapinchuk, and give a characterization of {Kazhdan}\ sets in terms of strong ergodicity of groups actions; this last result is related to a characterization of Property (T) due to Connes and Weiss.
\end{abstract}
\subjclass{22D10, 22D40, 37A15, 11K069, 43A07, 46M05}
\keywords{Kazhdan sets, topological groups, Property (T), 
equidistributed sequences in groups, tensor products of unitary 
representations, weakly mixing representations, compact unitary representations, 
abstract Wiener theorem, Heisenberg groups, $ax+b$ group, Howe-Moore property, locally rigid representations, strongly ergodic group actions.}
\thanks{This work was supported in part by the Labex CEMPI (ANR-11-LABX-0007-01)}
\maketitle

\section{Introduction}\label{Section 0}
Property (T)  is a rigidity property of topological groups which has been
introduced by {Kazhdan}\ in \cite{K} for locally compact groups, and which has 
spectacular applications to many fields, such as  the study of discrete 
subgroups of Lie groups, ergodic theory, operators algebras, random walks, 
explicit constructions of expander graphs 
and generation of (nearly) uniformly distributed random elements in groups. 
We refer the reader to the monograph \cite{BdHV} by Bekka, de la Harpe, and Valette 
for a comprehensive presentation of Kazhdan's Property (T) and its applications (see also \cite{HarpeVal}). 
Among amenable locally compact groups, the only ones which have Property 
(T) are the compact ones. If $K$ is a local field and $n\ge 2$ is an integer, the groups $SL_{n}(K)$ 
have Property (T) if and only if $n\ge 3$. A 
lattice $\Gamma $ in a locally compact group $G$ has Property (T) if and 
only if $G$ has it, so that for instance the groups $SL_{n}({\ensuremath{\mathbb Z}})$ have 
Property (T) for $n\ge 3$. The group $SL_{2}({\ensuremath{\mathbb Z}})$ does not have Property (T).
\par\smallskip 
Property (T) can be defined as the existence of a {Kazhdan}\ pair consisting 
of a \emph{compact} {Kazhdan}\ set and its corresponding {Kazhdan}\ constant 
(the precise definitions will be given in Section \ref{Sec2} below). Locally compact groups with Property (T) 
are compactly generated. In particular, discrete groups with Property (T) are finitely generated 
and it is known (see \cite[Prop.~1.3.2]{BdHV}) that the {Kazhdan}\ 
 subsets of a discrete group with Property (T) are exactly the  generating subsets of the group. 
More generally \cite[Prop.~1.3.2]{BdHV}, a generating set of a locally compact group which has Property (T) is 
a Kazhdan set and, conversely, a Kazhdan set which has non-empty interior is necessarily a generating set.
However, the determination of explicit {Kazhdan}\ pairs in a group with Property (T) is usually a difficult 
task. This natural problem of identifying {Kazhdan}\ pairs has been raised by Serre and by de la~Harpe and Valette 
(see \cite{Burg} and \cite[p.~133]{HarpeVal}). Besides these references, 
we refer the reader to \cite[Ch.~4]{BdHV}, 
which presents a method of 
Shalom \cite{Sha2} for obtaining {Kazhdan}\ pairs for such groups as 
$SL_{n}({\ensuremath{\mathbb Z}})$, $n\ge 3$, \cite{Kass1} which gives the exact asymptotic 
behaviour, as $n$ goes to infinity, $n\ge 3$, of the Kazhdan constant of the 
Kazhdan subset of $SL_{n}({\ensuremath{\mathbb Z}})$ consisting of 
the elementary matrices, \cite{Zuk} where {Kazhdan}\ constants are obtained for 
certain discrete groups given by presentations, and \cite{Be} where it is 
proved that the unitary group ${\mathscr{U}}(H)$ of a separable infinite dimensional 
Hilbert space $H$, equipped with the strong operator topology, has 
Property (T), and where explicit finite {Kazhdan}\ sets in ${\mathscr{U}}(H)$
are given (in this last example, the group is not locally compact). Notice also that while is not difficult to show that compact groups have Property (T), the 
computation of Kazhdan constants is nontrivial even for this class of groups; see for instance 
\cite{Neuh} and the references therein. 
\par\smallskip 
Even if a topological group does not have Property (T) it is still interesting to 
investigate some natural weaker rigidity properties of the group. See
for example \cite{LZ}, or \cite{Lubo94} which introduces the weaker Property ($\tau$). Property ($\tau$) is for instance sufficient for applications to expander graphs. 
It is also an interesting and difficult problem, mentioned in \cite[Sec.~7.12]{BdHV} to exhibit ``small''
{Kazhdan}\ subsets of groups which do not have Property (T) and, more generally, to determine the class of all {Kazhdan}\ subsets of such groups.

\par\smallskip 
The main motivations for the present paper are the following two questions of \cite[Sec.~7.12]{BdHV}. The first one is due to Y.~Shalom: 

\begin{question}\label{Question 0}\cite[Sec.~7.12]{BdHV}
 ``The question of knowing if a subset ${Q}$ of ${\ensuremath{\mathbb Z}}$ is a {Kazhdan}\ set is 
possibly related to the equidistribution of the sequence $(e^{2i\pi 
n\theta 
})_{n\,\in Q}$ for $\theta $ irrational, in the sense of Weyl.''
\end{question}

It is not difficult to see that if $p\ge 2$, $p{\ensuremath{\mathbb Z}}$ is not a {Kazhdan}\ set in 
${\ensuremath{\mathbb Z}}$, while the sequence $(e^{2i\pi pn\theta })_{n\,\in{\ensuremath{\mathbb Z}}}$ is uniformly 
distributed  in the circle ${\ensuremath{\mathbb T}}$ for every irrational number $\theta $. 
Also, the set $\{2^{k}+k\,;\,k\ge 1\}$ is a {Kazhdan}\ set in ${\ensuremath{\mathbb Z}}$, while 
there are irrational numbers $\theta $ such that $(e^{2i\pi (2^{k}+k)\theta 
})_{k\ge 1}$ is not equidistributed in ${\ensuremath{\mathbb T}}$ (see Example \ref{Example B} and Remark \ref{rem+} below). 
So Question 
\ref{Question 0} may be rephrased as follows:
\begin{question}\label{Question 1}
 Let ${Q}=\{n_{k}\,;\,k\ge 1\}$  be a subset of ${\ensuremath{\mathbb Z}}$ such that the subgroup of ${\ensuremath{\mathbb Z}}$ 
generated by ${Q}$ is equal to {\ensuremath{\mathbb Z}}. Suppose that the 
sequence $(e^{2i\pi n_{k}\theta })_{k\,\ge 1}$ is uniformly distributed in 
${\ensuremath{\mathbb T}}$ for every irrational number $\theta $. Is ${Q}$ a {Kazhdan}\ set in ${\ensuremath{\mathbb Z}}$?
\end{question}

The second question of \cite[Sec.~7.12]{BdHV} runs as follows:

\begin{question}\label{Question 00}\cite[Sec.~7.12]{BdHV}
``More generally, what are the Kazhdan subsets 
of  ${\ensuremath{\mathbb Z}}^k$, ${\ensuremath{\mathbb R}}^k$, the Heisenberg group, or other infinite amenable groups?''
\end{question}

There are several possible ways of defining uniformly distributed sequences in ${\ensuremath{\mathbb Z}}$ and, more generally, in locally compact groups.
We refer the reader to the books \cite{KuiNied} and \cite{DT}, and to the papers by Veech \cite{V1}, \cite{V2} and Losert and Rindler \cite{LR}, \cite{GLR} for a presentation of various notions of uniform distribution in general locally compact groups. If $(g_k)_{k\ge 1}$ is a sequence of elements in a locally compact group $G$, uniform distribution of $(g_k)_{k\ge 1}$ in any of these senses requires a certain form of convergence, as $N$ tends to infinity, of the means 
\begin{equation}\label{means}
 \dfrac{1}{N}{\displaystyle}\sum_{k=1}^{N}\pi(g_{k})
\end{equation}
to the orthogonal projection $P_{\pi}$ on the subspace of invariant vectors for $\pi$, for a certain class of unitary representations $\pi$ of $G$.
Veech \cite{V1}, \cite{V2} calls $(g_k)_{k\ge 1}$ uniformly distributed in $G$ if the convergence of the means (\ref{means}) holds in the weak operator topology for all unitary representations of $G$ (or, equivalently, for all irreducible unitary representations of $G$, provided $G$ is supposed to be second countable). Unitary uniform distribution in the sense of Losert and Rindler \cite{LR}, \cite{GLR} requires the convergence in the strong operator topology of the means (\ref{means}) for all irreducible unitary representations of the group, while Hartman uniform distribution only requires convergence in the strong operator topology for all finite dimensional unitary representations. All these notions coincide when $G$ is a Moore group, i.e. when all irreducible unitary representations of the group are finite dimensional. It is essentially obvious that 
if $(g_{k})_{k\ge 1}$ is a sequence of elements of a locally compact group $G$ which is uniformly distributed in $G$ in any of these senses, the set $\{g_k\;;\; k\ge 1\}$ is a {Kazhdan}\ set in $G$. 
\par\smallskip
In the case of the group ${\ensuremath{\mathbb Z}}$, the classic Weyl criterion \cite{KuiNied} implies that a sequence $(n_{k})_{k\ge 1}$ of 
elements of ${\ensuremath{\mathbb Z}}$ is unitarily or Hartman uniformly distributed in ${\ensuremath{\mathbb Z}}$ if and only if the sequence
 $(e^{2i\pi n_{k}\theta })_{k\,\ge 1}$ is uniformly distributed in 
${\ensuremath{\mathbb T}}$ for every  $\theta \in{\ensuremath{\mathbb R}}\setminus{\ensuremath{\mathbb Z}}$,
  \newsavebox{\abbaa}
 \savebox{\abbaa}{\smash[b]{\xymatrix@C=13pt{\scriptstyle 
 N\ar[r]&\scriptstyle+\infty}}}
where ${\ensuremath{\mathbb T}}$ denotes as usual the unit circle in ${\ensuremath{\mathbb C}}$. 
As observed just above, it is in this case straightforward to show that $\{n_{k}\,;\,k\ge 1\}$ is a {Kazhdan}\ set in ${\ensuremath{\mathbb Z}}$. 
Now, it is in many settings much more convenient and natural to consider the weaker equidistribution condition that $(e^{2i\pi\theta n_k})_{k\ge 1}$ is uniformly distributed in ${\ensuremath{\mathbb T}}$ for every $\theta\in{\ensuremath{\mathbb R}}\setminus{\ensuremath{\mathbb Q}}$: this is the condition which appears in 
Shalom's Question \ref{Question 0}. 
It is of course a weaker requirement on $(n_k)_{k\ge 1}$ than that of being Hartman uniformly distributed, and
the sequence $(n_k)_{k\ge 1}$ given by  $n_k= k^2$ verifies the equidistribution 
condition  of Question~\ref{Question 0} although it is not Hartman uniformly distributed in ${\ensuremath{\mathbb Z}}$. 
\par\smallskip
We will consider in this paper a natural extension to general locally compact groups $G$ of
the equidistribution condition of
 Question \ref{Question 0}:
if $(g_{k})_{k\ge 1}$ is a sequence of elements of $G$, we require the means (\ref{means}) to converge for all finite dimensional unitary representations of $G$ except at most countably many.
In the case of the group ${\ensuremath{\mathbb Z}}$, sequences $(n_{k})_{k\ge 1}$ of integers such that $(e^{2i\pi\theta n_k})_{k\ge 1}$ is uniformly distributed in ${\ensuremath{\mathbb T}}$ for all $\theta\in{\ensuremath{\mathbb R}}$ except countably many are
said to be of first kind (see for instance \cite{Ha2}).
The generalization of Question \ref{Question 1}
to this setting  runs as follows:
\begin{question}\label{Question 1.3}
 Let $(g_{k})_{k\ge 1}$ be a sequence of elements of a 
locally compact group $G$ which
satisfies the following equidistribution property:

\begin{equation}\label{equi1}
 \begin{minipage}{130 mm}
{for all finite dimensional irreducible unitary
representations $\pi $ of $G$ on a Hilbert space $H$, except at most countably many,}
\[
\xymatrix@C=50pt{\dfrac{1}{N}{\displaystyle}\sum_{k=1}^{N}{\ensuremath{{\langle {\pi (g_{k})x },{y }\rangle}}}\ar[r]_-{\usebox{\abbaa}}&0}
\quad\textrm{for every}\ x,y\in H.
\]
 \end{minipage}
\end{equation}
 \noindent
When is it true that $Q=\{g_{k}\,;\, k\ge 1\}$ is a {Kazhdan}\ set  in $G$?
\end{question}

We provide in this work an answer to Question \ref{Question 1.3} 
when $G$ is a second countable {Moore group}, and thus fully answer Shalom's Question \ref{Question 0} as well. We also answer
Question \ref{Question 00} by
giving a complete description of Kazhdan sets in many classic groups which do not have Property (T), including the groups ${\ensuremath{\mathbb Z}}^k$ and ${\ensuremath{\mathbb R}}^k$, $k\ge 1$, and the Heisenberg groups of all dimensions.
Let us now describe our main results in more detail.

\section{Main results}\label{Sec2}
We first briefly set down some notation and recall some important definitions.
\subsection{Definitions and notation}
The Hilbert spaces will always be supposed in this paper to be complex, 
and endowed with an inner product ${\ensuremath{{\langle {\,\cdot\,},{\cdot\,}\rangle}}}$ 
which is linear in the first variable and antilinear in the second 
variable. A \emph{unitary representation} of $G$ on $H$ is a group 
morphism 
from 
$G$ into the group ${\mathscr{U}}(H)$ of all unitary operators on $H$ which 
is strongly continuous, i.\,e.\ such that the map
$\smash{\xymatrix{g\ar@{|->}[r]&\pi (g)x }}$ is continuous from $G$ into 
$H$ for all vectors $x\in H$. As all the representations  we consider 
in this paper are unitary, we will often drop the word ``unitary'' and 
speak simply of representations of a group $G$ on a Hilbert space $H$. 
\begin{definition}\label{Definition 1}
 Let ${Q}$ be a subset of a topological group $G$, $\varepsilon $ a 
positive 
real number, and $\pi$ a unitary 
representation of $G$ on a Hilbert space $H$. A vector $x\in H$ is said to be 
\emph{$(Q,\varepsilon )$-invariant for $\pi $} if 
\[
\sup_{g\,\in {Q}} ||\pi (g)x -x||<\varepsilon ||x ||.
\]
A $(Q,\varepsilon )$-invariant vector for $\pi $ is in particular non-zero.
A \emph{$G$-invariant vector} for $\pi$ is a vector $x\in H$ such that $\pi (g)x=x
$ for all $g\in G$.
\end{definition}
The notions of {Kazhdan}\ sets and {Kazhdan}\ pairs will be fundamental in our work.
\begin{definition}\label{Definition 2}
 A subset ${Q}$ of a topological group $G$ is a \emph{{Kazhdan}\ set in $G$} if 
there exists $\varepsilon >0$ such that the following property holds true:
any unitary representation $\pi $ of $G$ on a complex Hilbert space $H$ 
with a 
$({Q},\varepsilon )$-invariant vector has a non-zero $G$-invariant vector.
\noindent
In this case, the pair $({Q},\varepsilon )$ is \emph{{Kazhdan}\ pair}, and 
$\varepsilon $ is a \emph{{Kazhdan}\ constant for ${Q}$}. A group $G$ has \emph{Property} (T), or is a \emph{{Kazhdan}\ group}, if it 
admits a compact {Kazhdan}\ set. 
\end{definition}

Lastly, we introduce a generating property of a subset of a group, which is weaker that the notion of generating set as used for instance in \cite{BdHV}.

\begin{definition}\label{Definition 3}
If ${Q}$ is any subset of a group $G$, we denote by 
$\langle{Q}\rangle$ the smallest subgroup of $G$ containing ${Q}$, i.\,e.\ 
the set of all elements of the form $g_{1}^{\,\pm 1}\dots g_{n}^{\,\pm 
1}$, where ${n\ge 1}$ and $g_{1},\dots,g_{n}$ belong to ${Q}$. We say that 
\emph{${Q}$ generates $G$} if $\langle{Q}\rangle=G$. It is equivalent to requiring that $\widetilde{Q}={Q}\cup{Q}^{-1}$ be a generating set in the sense of \cite[Sec.~7.1]{BdHV}.
\end{definition}

\subsection{Equidistributed sets in Moore groups}
Our first main result provides a complete answer to Question
\ref{Question 1.3} when the group $G$ is  a second countable Moore group, i.\,e.\ when all irreducible representations of $G$ are finite dimensional. Locally compact Moore groups are completely described in \cite{Mo}: a Lie group is a Moore group if and only 
if it has a closed subgroup $H$ such that $H$ modulo its center is compact, 
and a locally compact group is a Moore group if and only if it is a projective 
limit of Lie groups which are Moore groups. See also the survey \cite{Pal} 
for more information concerning the links between various properties of topological 
groups, among them the property of being a Moore group. Of course 
all locally compact abelian groups are Moore groups.

\begin{theorem}\label{Theorem 3}
 Let $G$ be a second countable locally compact Moore group. Let ${Q}$ be a  
subset of $G$ of the form ${Q}=\{g_{k}\,;\,k\ge 1 \}$ 
where $(g_{k})_{k\ge 1}$ is a sequence of elements of $G$. Suppose that ${Q}$ generates 
$G$, and that $(g_{k})_{k\ge 1 } $ satisfies the following 
assumption:
\begin{equation}\label{Pro3}
 \begin{minipage}{130 mm}
{for all (finite dimensional) irreducible unitary
representations $\pi $ of $G$ on a Hilbert space $H$, except at most countably many,}
\[
\xymatrix@C=50pt{\dfrac{1}{N}{\displaystyle}\sum_{k=1}^{N}{\ensuremath{{\langle {\pi (g_{k})x },{y }\rangle}}}\ar[r]_-{\usebox{\abbaa}}&0}
\quad\textrm{for every}\ x,y\in H.
\]
 \end{minipage}
\end{equation}
Then ${Q}$ is a {Kazhdan}\ set in $G$.
\end{theorem}

As explained in the introduction, property (\ref{Pro3}) above is an equidistribution property of the sequence
$(g_{k})_{k\ge 1 }$. 
It takes a more 
familiar form when the group $G$ is supposed to be abelian: it is equivalent to requiring that condition \eqref{(o)} below holds true
for all characters $\chi $ of the group except possibly countably many.

\begin{theorem}\label{Theorem 1}
 Let $G$ be a locally compact abelian group, and let $(g_{k
})_{k\ge 1}$ be  a sequence of elements of $G$. Suppose that $Q=\{g_{k}\,;\,k\ge 1\}$ generates $G$, and that
 \newsavebox{\abba}
\savebox{\abba}{\smash[b]{\xymatrix@C=13pt{\scriptstyle 
N\ar[r]&\scriptstyle+\infty}}}
 \begin{equation}\label{(o)}
 \xymatrix@C=50pt{\dfrac{1}{N}{\displaystyle}\sum_{k=1}^{N}\chi (g_{k})\ar[r]_-{\usebox{\abba}}&0}
 \end{equation}
for all characters $\chi $ on $G$, except at most countably many. 
Then $Q$ is a {Kazhdan}\ set in $G$.
\end{theorem}
Theorem \ref{Theorem 1} can thus be seen as a particular case of Theorem \ref{Theorem 
3}, except for the fact that there is no need to suppose that the group is 
second countable when it is known to be abelian.
Specializing Theorem \ref{Theorem 1} to the case where $G={\ensuremath{\mathbb Z}}$, we obtain 
the 
following complete answer to Question \ref{Question 1} above:
\begin{theorem}\label{Theorem 4}
 Let $(n_{k})_{k\ge 1}$ be a sequence of elements of 
${\ensuremath{\mathbb Z}}$, and set ${Q}=\{n_{k}\,;\,k\ge 1\}$. Suppose 
that the sequence $(e^{2i\pi n_{k}\theta })_{k\ge 1}$ is equidistributed 
in the unit circle ${\ensuremath{\mathbb T}}$ for all real numbers $\theta $ except at 
most countably 
many. The following assertions are equivalent:
\begin{enumerate}
 \item [{(a)}] ${Q}$ is a {Kazhdan}\ set in ${\ensuremath{\mathbb Z}}$;
 \item[{(b)}] ${Q}$ generates ${\ensuremath{\mathbb Z}}$.
\end{enumerate}
\end{theorem}

\subsection{Kazhdan sets and finite dimensional subrepresentations}
The proof of Theorem \ref{Theorem 1} relies on Theorem \ref{Theorem 0} below, which gives a new  
condition for a ``small perturbation'' of a subset ${Q}$ of a 
group $G$ to be a {Kazhdan}\ set in $G$. Theorem \ref{Theorem 0} constitutes the core of the paper, and  has, besides the proofs of Theorems \ref{Theorem 3}, \ref{Theorem 1}, and \ref{Theorem 4}, several interesting applications
which we will present in Sections \ref{Section 7} and \ref{Section 8}.

\begin{theorem}\label{Theorem 0}
 Let $G$ be a topological group, and let $(W_{n})_{n\ge 1}$ be an increasing sequence of 
subsets of $G$ such that $W_{1}$ is a 
neighborhood of the unit element $e$ of $G$ and $\bigcup_\gnW_{n}=G$.
Let ${Q}$ be a subset of $G$ satisfying the following assumption:
\begin{equation}\label{Pro1}
 \tag{*}\quad \begin{minipage}{120 mm}
there exists a positive constant $\varepsilon $ such that every unitary 
representation $\pi $ of $G$ on a Hilbert space $H$ admitting a 
$({Q},\varepsilon )$-invariant vector has a finite dimensional 
subrepresentation.
\end{minipage}
\end{equation}
\par\smallskip 
\noindent Then there exists an integer ${n\ge 1}$ such that ${Q}_{n}=
W_{n}\cup{Q}$ is a {Kazhdan}\ set in $G$.
\par\smallskip 
If the group $G$ is locally compact, the same statement holds true for any increasing
sequence $(W_{n})_{n\ge 1}$ of subsets of $G$ such that 
$\bigcup_\gnW_{n}=G$ (we no longer need to request that $W_{1}$ be a 
neighborhood of $e$).
\end{theorem}
If the group $G$ is locally compact and $\sigma $-compact, we can choose 
for $(W_{n})_{n\ge 1}$ an increasing sequence of compact sets whose union is equal to 
$G$. If ${Q}$ is a subset of $G$ satisfying (\ref{Pro1}), there exists then by 
Theorem \ref{Theorem 0} a compact subset $L$ of $G$ such that $L\cup {Q}$ 
is a {Kazhdan}\ set in $G$. We thus retrieve a characterization of Property (T) 
for $\sigma $-compact locally compact groups due to Bekka and Valette 
\cite{BV}, see also \cite[Th.~2.12.9]{BdHV}. The original proof of this result relies on 
the Delorme-Guichardet theorem that such a group has Property (T) if and 
only if it has property (FH), that is every affine isometric action of $G$ on a real Hilbert space has a 
fixed point. See Section \ref{Section 6} for more details.
\par\smallskip 
Theorem \ref{Theorem 0} admits a simpler formulation if we build the  
sequence $(W_{n})_{n\ge 1}$ starting from a set which generates the group in the sense of Definition \ref{Definition 3}:

\begin{corollary}\label{Corollary 1}
Let $G$ be a topological group.
 Let ${Q}_{0}$ be a subset of $G$ which generates $G$ and has non-empty 
interior, 
and 
 ${Q}$ a subset of $G$. If $Q$ satisfies assumption \emph{(\ref{Pro1})} of Theorem 
\ref{Theorem 0}, ${Q}_{0}\cup{Q}$ is a {Kazhdan}\ set in $G$.
\par
If $G$ is a locally compact group, it suffices to assume that 
${Q}_{0}$ generates $G$ in order to obtain the same conclusion.
\end{corollary}

One of the main consequences of Corollary \ref{Corollary 1} is Theorem 
\ref{Theorem 2} below, which shows in particular that property (\ref{Pro1}) of Theorem 
\ref{Theorem 0} characterizes {Kazhdan}\ sets among sets which generate the 
group (and have non-empty interior -- this assumption has to be added if the group is not supposed to be locally compact).

\begin{theorem}\label{Theorem 2}
Let $G$ be a topological group and let ${Q}$ be a subset of $G$ which 
generates $G$ and has non-empty interior. Then the following assertions are equivalent:
\begin{enumerate}[(a)]
\item ${Q}$ is a {Kazhdan}\ set in $G$;\label{aa}

\item there exists a constant $\delta\in (0,1)$ such that every unitary representation $\pi$ of $G$ on a Hilbert space $H$ admitting a vector $x\in H$ such that $\inf_{g\in{Q}}\left|{\ensuremath{{\langle {\pi(g)x},{x}\rangle}}}\right| > \delta\|x\|^2$ has a finite dimensional subrepresentation;\label{bb}

\item there exists a positive constant $\varepsilon $ such that every unitary 
representation $\pi $ of $G$ on a Hilbert space $H$ admitting a 
$({Q},\varepsilon )$-invariant vector has a finite dimensional 
subrepresentation.\label{cc}
\end{enumerate}
If the group $G$ is locally compact and ${Q}$ generates $G$, assertions (a), (b) and (c) above are equivalent.
\end{theorem}

 Condition (b) in Theorem \ref{Theorem 2} is easily seen to be equivalent to condition (c), which is nothing else than assumption (\ref{Pro1}) of Theorem \ref{Theorem 0}. Its interest will become clearer in Section \ref{Section 7} below, where it will be used to obtain a characterization of Kazhdan sets in second countable locally compact abelian groups.
\par\smallskip
The assumption that ${Q}$ generates $G$ cannot be dispensed with in 
Theorem \ref{Theorem 2}: ${Q}=2{\ensuremath{\mathbb Z}}$ is a subset of ${\ensuremath{\mathbb Z}}$ which 
satisfies property (\ref{cc}), but ${Q}$ is clearly not a {Kazhdan}\ 
set in ${\ensuremath{\mathbb Z}}$, while $\{1\}\cup {Q}$ is one. Indeed, if $G$ is a discrete 
group, a {Kazhdan}\ subset ${Q}$ of $G$ must necessarily generate $G$ (see \cite[Prop.~1.3.2]{BdHV}). Hence
Theorem \ref{Theorem 2} can be reformulated in the  case of discrete groups so as to give a complete
characterization of {Kazhdan}\ sets in $G$ in terms of assumption (\ref{Pro1}) 
of Theorem \ref{Theorem 0}:
\begin{theorem}\label{Corollary 100}
 Let $G$ be a discrete group, and let ${Q}$ be a subset of $G$. The 
following assertions are equivalent:
\begin{enumerate}[(a)]
\item ${Q}$ is a {Kazhdan}\ set in $G$;\label{aaa}

\item ${Q}$ generates $G$, and there exists a constant $\delta \in (0,1)$ such that every unitary representation $\pi$ of $G$ on a Hilbert space $H$ for which there exists a vector $x\in H$ such that $\inf_{g\in{Q}}\left|{\ensuremath{{\langle {\pi(g)x},{x}\rangle}}}\right| > \delta\|x\|^2$ has a finite dimensional subrepresentation;\label{bbb}

\item ${Q}$ generates $G$, and there exists a positive constant $\varepsilon $ such that every unitary 
representation $\pi $ of $G$ on a Hilbert space $H$ admitting a 
$({Q},\varepsilon )$-invariant vector has a finite dimensional 
subrepresentation.\label{ccc}
\end{enumerate}
\end{theorem}

\subsection{Examples and applications}
We characterize {Kazhdan}\ sets in various groups in which irreducible unitary representations are completely described. Using the new criterion provided by Theorem \ref{Theorem 2}, we first provide such a characterization of {Kazhdan}\ sets in second countable locally compact abelian groups (Theorem \ref{nimporte}). In the case of the group ${\ensuremath{\mathbb Z}}$, the characterization we obtain (Theorem \ref{nimportebis}) involves a classic class of sets in hamonic analysis, called Kaufman sets. We also present some examples of {Kazhdan}\ sets in this setting. We then describe {Kazhdan}\ sets in the Heisenberg groups $H_{n}$, $n\ge  1$ (Theorem \ref{Proposition F}) and in the $ax+b$ group (Theorem \ref{Proposition N}). We finally study {Kazhdan}\ sets in groups with the Howe-Moore property, and obtain a complete characterization of {Kazhdan}\ sets in groups with the Howe-Moore property but without Property (T) (Theorem \ref{Proposition I}), which yields a 
description of {Kazhdan}\ sets in $SL_{2}({\ensuremath{\mathbb R}})$. See Section \ref{Section 7} for detailed statements of these results, which provide an answer to Question \ref{Question 00}.
\par\smallskip
We also obtain two further applications of {Kazhdan}\ sets. The first one (Theorem \ref{thm:rapinchuk}) is a generalization of a result of Rapinchuk \cite{rapin} concerning the local rigidity of finite dimensional unitary representations in discrete groups with Property (T).
The second application (Theorem \ref{Theorem AA}) consists in a version for {Kazhdan}\ sets of a characterization of Property (T) due to Connes and Weiss \cite{CW}. Namely, we prove that if $Q$
is a subset of a topological group $G$ which generates $G$, it is a {Kazhdan}\ set in $G$ if and only if every weakly mixing action of $G$ on a probability space is $Q$-strongly ergodic (see Section \ref{Section 8.2} 
for definitions). These applications highlight one more time the importance of determining {Kazhdan}\ sets in general groups.

\subsection{Organization of the paper} The first half of the paper is devoted to the proof of Theorem \ref{Theorem 2}. Section  \ref{Section 3} is mostly expository, and presents some essential results and  tools concerning ergodic theory for unitary
 representations of arbitrary topological groups. Section \ref{Section 4} 
is devoted to the proof of an abstract version of the Wiener Theorem 
(Theorem \ref{Theorem 5.6}): if 
$\pi $ is any unitary representation of a topological group $G$ on a 
Hilbert space 
$H$, Theorem \ref{Theorem 5.6} provides an explicit expression for the 
quantities $m_{G}(\,|{\ensuremath{{\langle {\pi (\,\centerdot\,)x},{y}\rangle}}}|^{2})$, $x,y\in H$,
where $m_{G}$ is the unique invariant mean on $\textrm{WAP(G)}$.
This theorem is crucial for the proof of Theorem \ref{Theorem 0}, which 
proceeds by contradiction: we use Theorem \ref{Theorem 5.6} to show that 
certain infinite tensor products 
of finite dimensional representations are weakly mixing. 
The proof of Theorem \ref{Theorem 0} is given in Section \ref{Section 5}. The equidistribution criteria for {Kazhdan}\ sets (Theorems \ref{Theorem 3}, \ref{Theorem 1} and \ref{Theorem 4}) are proved in Section \ref{Section 6}, as well as  
Corollary 
\ref{Corollary 1}, Theorem \ref{Theorem 2} and  Corollary \ref{Corollary 100}. Section \ref{Section 7} is devoted to examples; the characterizations of {Kazhdan}\ sets in second countable locally compact abelian groups, and in ${\ensuremath{\mathbb Z}}$ in particular, in the Heisenberg groups, the 
$ax+b$ group, and groups with the Howe-Moore property are proved in Theorems  \ref{nimporte}, \ref{nimportebis}, \ref{Proposition F}, \ref{Proposition N} and \ref{Proposition I} respectively. The extensions of the Rapinchuk rigidity result \cite{rapin} and of the Connes-Weiss characterization of Property (T) \cite{CW} are proved in Section \ref{Section 8}. The paper also contains an appendix which reviews some constructions of infinite tensor products of Hilbert spaces and of unitary representations, which are used in the proof of Theorem \ref{Theorem 0}.

\section{Ergodic theory and mixing properties for unitary representations 
of topological groups}\label{Section 3}
The proof of our main theorem will require a rather precise knowledge of 
the structure of unitary representations of a topological group $G$. We 
first recall in this section some definitions and results that we will need. They can be found for instance in the forthcoming book \cite{Ke}, the notes 
\cite{Pet}, and the paper \cite{BR} by Bergelson and
Rosenblatt.
\par\smallskip 
Let $\pi $ be a unitary representation 
of $G$ on a complex Hilbert space $H$. In the case where $G={\ensuremath{\mathbb Z}}$, the 
well-known (weak) ergodic theorem of von Neumann states that if $U$ is a 
unitary operator on $H$,
\[
\xymatrix{\dfrac{1}{N}{\displaystyle}\sum_{n=1}^{N}{\ensuremath{{\langle {U^{n}x},{y }\rangle}}}\ar[r]&{\ensuremath{{\langle {
Px },{y }\rangle}}}}
\]
as $N$ tends to infinity for every vectors $x, y \in H$, where $P$ 
denotes the orthogonal projection in $H$ onto the subspace $\ker(U-I)$ of 
fixed points of $U$. This result admits an extension to amenable groups 
(see \cite{Dix} or \cite{Dye}) and, by using the existence of an invariant 
mean on weakly almost periodic functions on $G$, to all topological 
groups. Recall that the class \mbox{WAP$(G)$} of \emph{weakly almost 
periodic 
functions} on $G$ is defined as follows: if 
$\ell^{\,\infty }(G)$ denotes the space of bounded functions on $G$,
$f\in\ell^{\,\infty }(G)$ belongs 
to \mbox{WAP$(G)$} if the weak closure in $\ell^{\,\infty }(G)$ of the 
set $\{f(s^{-1}\centerdot)\,;\,s\in G\}$ is weakly compact. For each $s\in G$,
$f(s^{-1}\centerdot)$ denotes the function 
$\smash{\xymatrix{t\ar@{|->}[r]&f(s^{-1}t)}}$ on $G$. By comparison, recall that $f\in\ell^{\,\infty }(G)$ is an \emph{almost periodic function} on $G$, written $f\in\mbox{AP$(G)$}$, if the norm closure in $\ell^{\,\infty }(G)$ of $\{f(s^{-1}\centerdot)\,;\,s\in G\}$ is compact.
If $\pi $ is a 
unitary representation of $G$ on a Hilbert space $H$, the functions
\[
{\ensuremath{{\langle {\pi (\centerdot)x},{y}\rangle}}},\quad \bigl|{\ensuremath{{\langle {\pi (\centerdot)x
},{y }\rangle}}} \bigr|,\quad  \textrm{and}\quad \bigl|{\ensuremath{{\langle {\pi (\centerdot)x
},{y }\rangle}}} \bigr|^{2}\!,
\]
where $x $ and $y $ are any vectors of $H$, belong to 
\mbox{WAP$(G)$}. 
For more on weakly almost periodic functions on a group, see for instance \cite{Bur} or \cite[Ch. 1, Sec. 9]{Gl}. 
The interest of the class of weakly almost periodic functions on $G$ in 
our context is that there exists on \mbox{WAP$(G)$} a unique $G$-invariant 
mean $m$. It satisfies
\[
m(f(s^{-1}\centerdot))=m(f(\,\centerdot\,\, s^{-1}))=m(f)
\]
for every $f\in\mbox{WAP$(G)$}$ and every $s\in G$.
The abstract ergodic theorem then states that if 
$\pi $ is a unitary representation 
of $G$ on $H$\!, 
\[
m({\ensuremath{{\langle {\pi (\centerdot)x },{y }\rangle}}})={\ensuremath{{\langle {P_{\pi }x },{y }\rangle}}}
\]
for every vectors $x,y\in H$, where $P_{\pi }$ denotes the 
projection of $H$ onto the space
\[
E_{\pi }=\{x  \in H\,;\,\pi (g)x =x \ \ \textrm{for every}\ 
g\in G\}
\]
of $G$-invariant vectors for $\pi $. The representation $\pi $ is \emph{ergodic} 
(i.\,e.\ 
admits no non-zero $G$-invariant vector) if and only if $m({\ensuremath{{\langle {\pi 
(\centerdot)x },{y}\rangle}}})=0$ for every $x,y \in H$.
\par\smallskip 
Following \cite{BR}, let us now recall that the representation 
$\pi $ is said to be \emph{weakly mixing}
if $m(|{\ensuremath{{\langle {\pi (\centerdot)x },{x }\rangle}}}|)=0$ for every $x\in H$, or, equivalently, 
$m(|{\ensuremath{{\langle {\pi (\centerdot)x},{x }\rangle}}}|^{2})=0$ for every $x\in H$. Then 
$m(|{\ensuremath{{\langle {\pi (\centerdot)x },{y }\rangle}}}|)=m(|{\ensuremath{{\langle {\pi (\centerdot)x},{y
}\rangle}}}|^{2})=0$ for every $x,y\in H$.
\par\smallskip 
There are several important characterizations of weak mixing for unitary 
representations, one of which involves the adjoint representation 
${\overline{{\pi}}}$ of a unitary representation 
$\pi $ of $G$ on $H$. It is a representation of $G$ on 
the adjoint Hilbert space ${\overline{{H}}}$: ${\overline{{H}}}$ 
coincides with $H$ as a set, and its elements are denoted by 
${\overline{{x }}}$, where $x\in H$. The multiplication by a scalar 
$\lambda $ on ${\overline{{H}}}$ is defined by $\lambda 
{\overline{{x }}}={\overline{{{\overline{{\lambda }}}x }}}$ for every 
${\overline{{x }}}\in {\overline{{H}}}$, and the scalar product of 
two elements ${\overline{{x}}}$ and ${\overline{{y }}}$ of 
${\overline{{H}}}$ is given by ${\ensuremath{{\langle {{\overline{{x}}}},{{\overline{{y 
}}}}\rangle}}}={\overline{{{\ensuremath{{\langle {x},{y }\rangle}}}}}}$. Then the representation 
${\overline{{\pi }}}$ of $G$ on ${\overline{{H}}}$ 
is defined by setting ${\overline{{\pi}}}(g){\overline{{x}}}={\overline{{\pi (g)x}}}$ for every $g\in G$ and every ${\overline{{x }}}\in 
{\overline{{H}}}$.
\par\smallskip 
Weak mixing for unitary representations is characterized by the fact that 
the tensor product representation $\pi {\otimes} {\overline{{\pi }}}$ of 
$G$ on $H{\otimes} {\overline{{H}}}$ is ergodic, i.\,e.\ admits 
no non-zero \mbox{$G$-invariant} vector. This relies on the observation 
that for any vectors $x,y \in H$,
\[
m({\ensuremath{{\langle {\pi {\otimes}{\overline{{\pi }}}(\,\centerdot\,)x {\otimes}{\overline{{x }}}},{y 
{\otimes}{\overline{{y }}}}\rangle}}})=m(|{\ensuremath{{\langle {\pi (\,\centerdot\,)x },{y }\rangle}}}|^{2}).
\]
\par
The ergodic theorem applied to the representation $\pi {\otimes}{\overline{{\pi }}}$ and 
to the 
vectors $x {\otimes}{\overline{{x}}}$ and $y {\otimes}{\overline{{y }}}$ thus yields that
\begin{equation}\label{Eq3}
 m\bigl(|{\ensuremath{{\langle {\pi (\,\centerdot\,)x },{y }\rangle}}}|^{2} \bigr)={\ensuremath{{\langle {P_{\pi 
{\otimes}{\overline{{\pi }}}}\,x{\otimes}{\overline{{x }}}},{y {\otimes}{\overline{{y }}}}\rangle}}}.
\end{equation}
The representation $\pi {\otimes}{\overline{{\pi }}}$ is equivalent to a 
representation of $G$ on the space $HS(H)$ of Hilbert-Schmidt 
operators on $H$, which is often more convenient to work with. Recall that 
$HS(H)$ is a Hilbert space when endowed with the scalar product defined 
by the formula ${\ensuremath{{\langle {A},{B}\rangle}}}=\textrm{tr}(B^{*}\!A)$ for every $A,B\in 
HS(H)$. The space $H{\otimes} {\overline{{H}}}$ is identified to $HS(H)$ by 
associating to each elementary tensor product $x{\otimes}{\overline{{y }}}$ of 
$H{\otimes}{\overline{{H}}}$ the rank-one operator ${\ensuremath{{\langle {\,\centerdot\,},{y}\rangle}}}\,{x
}$ on $H$. 
This map $\smash{{{\Theta }:
\xymatrix{{H{\otimes}{\overline{{H}}}}\ar[r]&{HS(H)}}
}}$ extends into a unitary 
isomorphism, and we have for every $g\in G$ and every $T\in HS(H)$
\[
\Theta\, \pi {\otimes}{\overline{{\pi }}} (g)\,\Theta ^{-1}(T)=\pi (g)\,T\,\pi (g^{-1}).
\]
We will, when needed, identify $\pi {\otimes}{\overline{{\pi }}}$ with this equivalent 
representation, and use it in particular in Section \ref{Section 
4} to obtain a 
concrete description of the space $E_{\pi {\otimes}{\overline{{\pi }}}}$ of $G$-invariant 
vectors for $\pi {\otimes}{\overline{{\pi }}}$, which is identified to the subspace of 
$HS(H)$
\[
{\textbf{\textsf{E}}}_{\pi}=\{T\in HS(H)\,;\,\pi (g)\,T=T\,\pi (g)\  
\textrm{for every}\ g\in G\}\cdot 
\]
Seeing $\pi {\otimes}{\overline{{\pi }}}$ as a representation of $G$ on $HS(H)$ also 
allows one to prove that $\pi {\otimes}{\overline{{\pi }}}$ has a non-zero $G$-invariant 
vector if and only if $\pi $ admits a finite dimensional 
subrepresentation. We summarize this in the following proposition:
\begin{proposition}\label{Proposition 4.1}
 Let $\pi $ be a unitary representation of $G$ on a Hilbert space $H$. 
The following assertions are equivalent:
\begin{enumerate}
 \item [(1)]$\pi $ is weakly mixing;
 \item[(2)]$\pi $ admits no finite dimensional subrepresentation;
 \item[(3)]$\pi {\otimes}{\overline{{\pi }}} $ has no non-zero $G$-invariant vector.
\end{enumerate}
\end{proposition}
\par
A companion to the property of weak mixing for unitary representation is 
that of compactness: given a unitary 
representation $\pi $ of $G$ on $H$ a vector $x\in 
H$ is \emph{compact} for $\pi $ if the norm closure of the set $\{\pi 
(g)x\,;\,g\in G\}$ is compact in $H$. The representation $\pi $ itself 
is said to be \emph{compact} if every vector of $H$ is compact for $\pi 
$. Compact representations decompose as direct sums of irreducible 
finite dimensional representations. A representation $\pi $ is weakly 
mixing if and only if the only compact vector for $\pi $ is the zero 
vector. The general structural result for unitary representations is given 
by Proposition \ref{Proposition 4.2}. It extends a well-known result
of Jacobs, de Leeuw and Glicksberg (see for instance \cite{Farkas} for an account of this result) which associates to any contraction $T$ on a Hilbert space $H$ a decomposition of $H$ into two $T$- and $T^*$-invariant subspaces of $T$. This two spaces correspond respectively to the weakly mixing and compact parts in the statement below.

\begin{proposition}\label{Proposition 4.2}
 A unitary representation $\pi $ of $G$ on a Hilbert space $H$ 
decomposes as a direct sum of a weakly mixing representation and a compact 
representation: 
\[
 \smash[b]{H=H_{w}{\mathop{\oplus}}\limits^{\perp}H_{c}\,,}
\]
 where 
$H_{w}$ and $H_{c}$ are both $G$-invariant closed subspaces of $H$, 
$\pi_{w}=\pi \vert_{H_{w}}$ is weakly mixing and $\pi_{c}=\pi 
\vert_{H_{c}}$ is compact. Hence $\pi $ decomposes as a direct sum of a 
weakly mixing representation and finite dimensional 
irreducible subrepresentations.
\end{proposition}
See \cite[Ch.~1]{Pet}, \cite[Appendix~M]{BdHV}, \cite{BR} or \cite{Dye} 
(in the amenable case) for detailed proofs of these results.

\section{An abstract version of the Wiener Theorem}\label{Section 4}
\subsection{Notation}\label{Section 4.1}
Let $\pi $ be a compact representation of $G$ on a Hilbert 
space $H$. We have seen in Section \ref{Section 3} that $\pi$ can be 
decomposed as a direct sum of irreducible finite dimensional 
representations of $G$. We sort out these representations by equivalence 
classes, and index the distinct equivalence classes by an index $j$ 
belonging to a set $J$, which may be finite or infinite (and which is 
countable if $H$ is separable). For every ${j\,\in J}$, we index 
by ${i\,\in I_{j}}$ all the 
representations appearing in the decomposition of $\pi $ which are in the 
$j$-th equivalence class. More precisely, we can decompose $H$ and $\pi 
$ as 
\[
H={\mathop{\oplus}}\limits_{j\,\in J}\bigl({\mathop{\oplus}}\limits_\iijH{_{i,\,j}} 
\bigr)\quad 
\textrm{and}\quad  \pi 
={\mathop{\oplus}}\limits_{j\,\in J}\bigl({\mathop{\oplus}}\limits_{i\,\in I_{j}}\pi {_{i,\,j}} 
\bigr)
\]
respectively, where the following holds true: 
\begin{enumerate}
 \item [--] for every ${j\,\in J}$ the spaces $H{_{i,\,j}}$, $i\in I_{j}$, are equal. 
We denote by 
$K_{j}$ this common space, and by $d_{j}$ its dimension (which is finite). 
We also write 
\[
{\widetilde{{H}}}_{j}={\mathop{\oplus}}\limits_\iijH{_{i,\,j}},\quad\textrm{so that}\quad 
H={\mathop{\oplus}}_{j\,\in J}{\widetilde{{H}}}_{j};
\]
\item[--] for every ${j\,\in J}$, there exists an irreducible representation 
$\pi 
_{j} $ of $G$ on $K_{j}$ such that $\pi {_{i,\,j}}$ is equivalent to $\pi _{j}$ 
for every ${i\,\in I_{j}}$. We also denote by $U{_{i,\,j}}$ a unitary operator from 
$H{_{i,\,j}}$ into
$K_{j}$ 
such that $\pi {_{i,\,j}}=U{_{i,\,j}}^{*}\,\pi _{j}\,U{_{i,\,j}}^{}$;
\item[--] if $j$, $j'$ belong to $J$ and $j\neq j'$, $\pi _{j}$ and 
$\pi_{j'}$ are not equivalent.
\end{enumerate}
\par 
If we denote by $U$ the unitary operator 
${\mathop{\oplus}}\limits_{j\,\in J}\bigl({\mathop{\oplus}}\limits_\iijU{_{i,\,j}} 
\bigr)$ from $H={\mathop{\oplus}}\limits_{j\,\in J}\bigl({\mathop{\oplus}}\limits_\iijH{_{i,\,j}} 
\bigr)$ into ${\mathop{\oplus}}\limits_{j\,\in J}\bigl({\mathop{\oplus}}\limits_\iijK_{j} 
\bigr)$, 
we have $\pi =U^{*}\pi '\, U,\quad \textrm{where}\quad \pi 
'={\mathop{\oplus}}\limits_{j\,\in J}\bigl({\mathop{\oplus}}\limits_{i\,\in I_{j}}\pi _{j} 
\bigr)$.
\par\smallskip 
For every ${j\,\in J}$, we denote by ${\widetilde{{P}}}_{j}$ the orthogonal projection of 
$H$ on ${\widetilde{{H}}}_{j}$, and, for every ${i\,\in I_{j}}$, by $P{_{i,\,j}}$ the orthogonal 
projection of $H$ on $H{_{i,\,j}}$ and by $\tau {_{i,\,j}}$ the canonical injection 
of 
$H{_{i,\,j}}$ into $H$.
\par\smallskip 
Let $A\in \mathscr{B}(H)$. For every $k,l\in J$, every $u\in 
I_{k}$, 
and every $v\in I_{l}$, we denote by $A_{{u,\,v}} $ the operator from 
$H_{v,\,l}$ into $H_{u,\,k}$ defined by 
$A_{{u,\,v}} =P_{u,\,k}\,A\vert_{H_{v,\,l}}$. Thus we can write $A$ in  
block-matrix form with respect to the decomposition 
\[
 H={\mathop{\oplus}}\limits_{j\,\in J}\bigl({\mathop{\oplus}}\limits_\iijH{_{i,\,j}} 
\bigr)\quad 
\textrm{as}\quad A=\bigl(A_{{u,\,v}}  \bigr)_{k,\,l\,\in\, J,\, u\,\in\, 
I_{k},\, v\,\in\, I_{l}}\cdot 
\]
If we define, for $k,l\in J$, 
${\widetilde{{A}}}_{k,\,l}={\widetilde{{P}}}_{k}\,A\vert_{{\widetilde{{H}}}_{l}}$, we can also write $A$ in 
block-matrix form with respect to the decomposition
\[
 H={\mathop{\oplus}}\limits_{j\,\in J}{\widetilde{{H}}}_{j}\quad 
\textrm{as}\quad A=\bigl(
{\widetilde{{A}}}_{k,\,l}\bigr)_{k,\,l\,\in  J},
\]
and of course ${\widetilde{{A}}}_{k,\,l}=\bigl(A_{{u,\,v}}  \bigr)_{u\,\in\, I_{k},\,
v\,\in\,I_{l}}$ for every $k,\,l\in J$.
\subsection{A formula for the projection ${\textbf{\textsf{P}}}_{\pi }$ of $HS(H)$ on 
${\textbf{\textsf{E}}}_{\pi }$.}
Our first aim in this section is to give an explicit formula for the 
projection ${\textbf{\textsf{P}}}_{\pi }A$ of a Hilbert-Schmidt operator $A\in HS(H)$ on the closed 
subspace of $HS(H)$
\[
{\textbf{\textsf{E}}}_{\pi }=\{T\in HS(H)\,;\,\pi (g)\,T=T\,\pi (g)\ \textrm{for 
every}\ g
\in G\}
\]
and to compute the norm of ${\textbf{\textsf{P}}}_{\pi }A$.
\begin{proposition}\label{Proposition 5.1}
 Let $\pi $ be a compact representation of $G$ on $H$, written in the form $\pi 
={\mathop{\oplus}}\limits_{j\,\in J}\bigl({\mathop{\oplus}}\limits_{i\,\in I_{j}}\pi {_{i,\,j}} 
\bigr)$ as discussed in Section \ref{Section 4.1} above. For every 
Hilbert-Schmidt operator $A$ on $H$, 
we have
\begin{align*}
 {\textbf{\textsf{P}}}_{\pi }A&=\sum_{j\,\in J}\,\dfrac{1}{d_{j}}\sum_{u,\,v\,\in\, 
I_{j}}\,\emph{tr}\bigl(U_{v,\,j}^{*}\,U_{u,\,j}^{}\,A_{u,\,v}^{} 
\bigr)\,\tau_{u,\,j}^{}\,U_{u,\,j}^{*}\,U_{v,\,j}^{}\,P_{v,\,j}^{}.
\intertext{Hence}
||{\textbf{\textsf{P}}}_{\pi }A||^{2}&=\sum_{j\,\in J}\,\dfrac{1}{d_{j}}\sum_{u,\,v\,\in 
I_{j}}\bigl|\emph{tr}
\bigl(U_{v,\,j}^{*}\,U_{u,\,j}^{}\,A_{u,\,v}^{} 
\bigr)\bigr|^{2}.
\end{align*}
\end{proposition}
\begin{remark}\label{Remark 5.2} In the particular case where $\pi $ has 
the form $\pi =\pi '={\mathop{\oplus}}\limits_{j\,\in J}\bigl({\mathop{\oplus}}\limits_{i\,\in I_{j}}\pi _{j} 
\bigr)$, i.\,e.\ when for every ${j\,\in J}$ the representations $\pi{_{i,\,j}}$, 
${i\,\in I_{j}}$, are equal (and not only equivalent), the formula becomes
\[
{\textbf{\textsf{P}}}_{\pi }A=\sum_{j\,\in J}\,\dfrac{1}{d_{j}}\sum_{u,\,v\,\in 
I_{j}}\textrm{tr}
(A_{{u,\,v}} )\,\tau_{u,\,j}\,i_{{u,\,v}} \,P_{v,\,j}
\]
 where $i_{{u,\,v}} $ is the identity operator from $H_{v,\,j}$ into 
$H_{u,\,j}$.
\end{remark}
\begin{remark}\label{Remark 5.2.1}
If $k,l\in\,J$, $u\in I_{k}$, $v\in I_{l}$, and $A_{{u,\,v}} $ is any 
bounded operator from $H_{v,\,l}$ into $H_{u,\,k} $, the 
operator $\tau _{u,\,k}\,A_{{u,\,v}} \,P_{v,\,l}$ is simply the canonical 
extension $A'$ of $A_{{u,\,v}} $ to the whole space $H$. Since $H_{v,\,l}$ 
and 
 $H_{u,\,k}$ are finite dimensional, $A'$ is a Hilbert-Schmidt 
operator on 
$H$ and $||A'||_{{HS}}=||A_{{u,\,v}} ||_{{HS}}$. For any $B
\in{HS}(H)$, we have
\[
{\ensuremath{{\langle {A'},{B}\rangle}}}=\textrm{tr}\bigl(B_{{u,\,v}} ^{*}\,A_{{u,\,v}}^{}  \bigr)=
{\ensuremath{{\langle {A_{{u,\,v}}},{B_{{u,\,v}}}\rangle}}},
\]
the last notation being slightly abusive.
\end{remark}
The proof of Proposition \ref{Proposition 5.1} relies on the following 
lemma, which gives a complete description of ${\textbf{\textsf{E}}}_{\pi }$ in terms of 
block-matrices.
\begin{lemma}\label{Lemma 5.3}
 The space ${\textbf{\textsf{E}}}_{\pi }$ consists of the operators $T\in\emph{HS}(H)$ such 
that 
\begin{enumerate}
 \item [--] for every $k,l\in J$ with $k\neq l$, 
${\widetilde{{T}}}_{k,\,l}=0$;
\item[--] for every $k\in J$ and every $u,v\in I_{k}$, there exists a 
complex number $\lambda _{{u,\,v}} $ such that $T_{{u,\,v}} =\lambda _{{u,\,v}} 
\,U_{u,\,k}^{*}\,U_{v,\,k}^{}$. Thus ${\widetilde{{T}}}_{k,\,k}=\bigl(\lambda _{{u,\,v}} 
\,
U_{u,\,k}^{*}\,U_{v,\,
k}^{}
\bigr)_{u,\,v\,\in I_{k}}$.
\end{enumerate}
\end{lemma}
\begin{proof}[Proof of Lemma \ref{Lemma 5.3}]
Let $T\in {HS}(H)$. The equalities $\pi (g)\,T=T\,\pi (g)$ for every 
  $g\in G$ mean that for every $k,l\in J$, $u\in I_{k}$ and 
$v\in I
_{l}$, $\pi _{u,\,k}(g)\,T_{{u,\,v}} =T_{{u,\,v}} \,\pi _{v,\,l}(g)$ for every $ g\in G$, i.\,e.\ 
$\pi 
_{k}(g)\,U_{u,\,k}\,T_{{u,\,v}} \,U_{v,\,l}^{*}=U_{u,\,k}
\,T_{{u,\,v}} \,U_{v,\,l}^{*}\,\pi _{l}(g)$ for every $g\in G$.
Thus the operator 
$\smash{{{S_{{u,\,v}} =U_{u,\,k}\,T_{{u,\,v}} \,U_{v,\,l}^{*}}:
\xymatrix{{K_{l}}\ar[r]&{
K_{k}}}
}}$ intertwines the two representations $\pi _{k}$ and $\pi _{l}$. 
If $S_{{u,\,v}} $ is non-zero, it follows from Schur's Lemma 
that $S_{{u,\,v}} $ is an isomorphism. The representations $\pi _{k}$ and
$\pi _{l}$ are thus isomorphically (and hence unitarily) equivalent. Since 
$\pi _{k}$ and $\pi _{l}$ are not equivalent for $k\neq l$, it follows 
that $S_{{u,\,v}} =0$, and hence $T_{{u,\,v}} =0$ for every $u\in I_{k}$ and $v\in 
I_{l}$ as soon as $k\neq l$.
\par\smallskip 
If now $k=l$, the equalities $\pi _{k}(g)\,S_{{u,\,v}}=S_{{u,\,v}}\,\pi _{k}(g)$ 
for every $g\in G$ and  Schur's Lemma again imply that $S_{{u,\,v}}=\lambda
_{{u,\,v}}\,i_{k}$ for some scalar $\lambda _{{u,\,v}}$, where $i_{k}$ is the identity operator from $K_{k}$ into itself. So $T_{{u,\,v}}=
\lambda _{{u,\,v}}\,U_{u,\,k}^{*}\,U_{v,\,k}^{}$. Thus any operator $T\in
{\textbf{\textsf{E}}}_{\pi }$ satisfies the two conditions of the lemma. The converse is 
obvious.
\end{proof}
The proof of Proposition \ref{Proposition 5.1} is now straightforward.
\begin{proof}[Proof of Proposition \ref{Proposition 5.1}]
Consider, for every ${j\,\in J}$ and $u,v\in I_{j}$, the 
one-dimensional 
subspace ${\textbf{\textsf{E}}}_{{u,\,v}}$ of $HS(H)$ spanned by the operator
$V_{{u,\,v}}^{}=\tau _{u,\,j}^{}\,U_{u,\,j}^{*}\,U_{v,\,j}^{}\,P_{v,\,j}^{}$. 
These subspaces are pairwise orthogonal in \mbox{HS$(H)$}, and by Lemma 
\ref{Lemma 5.3} we have
\[
{\textbf{\textsf{E}}}_{\pi }={\mathop{\oplus}}\limits_{j\,\in J}\,\bigl({\mathop{\oplus}}\limits_{{{u,\,v}}\,\in I_{j}}{\textbf{\textsf{E}}}_{{u,\,v}} 
\bigr).
\]
Hence, for every $A\in HS(H)$, ${\textbf{\textsf{P}}}_{\pi }A$ is the sum of the 
projections of $A$ on all the subspaces ${\textbf{\textsf{E}}}_{{u,\,v}}$, i.e.\
\[
 {\textbf{\textsf{P}}}_{\pi }A=\sum_{j\,\in J}\,\sum_{{{u,\,v}}\,\in I_{j}}\Bigl\langle 
A,\dfrac{V_{{u,\,v}}}{||V_{{u,\,v}}||}\Bigr\rangle 
\,\dfrac{V_{{u,\,v}}}{||V_{{u,\,v}}||}=\sum_{j\,\in J}\,\,\dfrac{1}{d_{j}}\sum_{{{u,\,v}}\,
\in I_{j}}\emph{tr}
\bigl(U_{v,\,j}^{*}\,U_{u,\,j}^{}\,A_{u,\,v}^{} 
\bigr)\,V_{{u,\,v}},
\]
which gives the formula we were looking for. We also have
\[
||{\textbf{\textsf{P}}}_{\pi }A||^{2}=\sum_{j\,\in J}\,\sum_{{{u,\,v}}\,\in I_{j}}\Bigl|\Bigl\langle 
A,\dfrac{V_{{u,\,v}}}{||V_{{u,\,v}}||}\Bigr\rangle \Bigr|^{2}
=\sum_{j\,\in J}\,\,\dfrac{1}{d_{j}}\sum_{{{u,\,v}}\,
\in I_{j}}\bigl|\emph{tr}
\bigl(U_{v,\,j}^{*}\,U_{u,\,j}^{}\,A_{u,\,v}^{} 
\bigr)\bigr|^{2}.
\]
\end{proof}

\begin{corollary}\label{Corollary 5.4}
 Let $\pi $ be a compact representation of $G$ on $H$, written 
 $\pi 
={\mathop{\oplus}}\limits_{j\,\in J}\bigl({\mathop{\oplus}}\limits_{i\,\in I_{j}}\pi {_{i,\,j}} 
\bigr)\!\!\!$ as in Proposition \ref{Proposition 5.1}. Let 
$x 
={\mathop{\oplus}}\limits_{j\,\in J}\bigl({\mathop{\oplus}}\limits_\iijx {_{i,\,j}} 
\bigr)$ and $y 
={\mathop{\oplus}}\limits_{j\,\in J}\bigl({\mathop{\oplus}}\limits_\iijy {_{i,\,j}} 
\bigr)$ be two vectors of $H$, and let $A\in\mbox{HS$(H)$}$ be the 
rank-one 
operator ${\ensuremath{{\langle {\,\centerdot\,},{y}\rangle}}}\,x$. Then
\[
{\textbf{\textsf{P}}}_{\pi }A=\sum_{j\,\in J}\,\,\dfrac{1}{d_{j}}\sum_{{{u,\,v}}\,\in I_{j}}
{\ensuremath{{\langle {U_{u,\,j}\,x_{u,\,j}},{U_{v,\,j}\,y_{v,\,j}}\rangle}}}\,\tau 
_{u,\,j}\,U^{*}_{u,\,j}\, 
U_{v,\,j}^{}\,P_{v,\,j}^{}.
\]
\end{corollary}
\begin{proof}
 For every ${j\,\in J}$ and $u,v\in I_{j}$, 
$A_{{u,\,v}}={\ensuremath{{\langle {\,\centerdot\,},{y_{v,\,j}}\rangle}}}\,x_{u,\,j}$, so that 
\[
\textrm{tr}\bigl(U^{*}_{v,\,j}\,U_{u,\,j}\,A_{{u,\,v}} \bigr)=
{\ensuremath{{\langle {U_{u,\,j}\,x_{u,\,j}},{U_{v,\,j}\,y_{v,\,j}}\rangle}}}.
\]
The result then follows from Proposition \ref{Proposition 5.1}.
\end{proof}
\subsection{An abstract version of the Wiener Theorem.}\label{Section 4.3}
As recalled in Section \ref{Section 3}, ${\textbf{\textsf{E}}}_{\pi }$ is the space 
of $G$-invariant vectors for the representation $\pi {\otimes}{\overline{{\pi }}}$ on 
$HS(H)$, where for every $x,y\in H$, $x{\otimes}{\overline{{y}}}$ is identified with the rank-one 
operator ${\ensuremath{{\langle {\,\centerdot\,},{y}\rangle}}}\,x$. For every pair 
$(x,y)$ of vectors of $H$, denote by $\pmb{b}_{x,\,y}$ the element of 
$K{\otimes} K$, with
$K={\mathop{\oplus}}\limits_\jnjK_{j}$, defined by 
\[
\pmb{b}_{x,\,y}=\sum_{j\,\in J}\,\dfrac{1}{\sqrt{d_{j}}}\,\Bigl(\,\sum_{i\,\in I_{j}} 
x{_{i,\,j}}{\otimes}{\overline{{y}}}{_{i,\,j}}\Bigr).
\]
It should be pointed out that for a fixed index $j\in J$ the vectors 
$x{_{i,\,j}}$ and $y{_{i,\,j}}$ are 
understood in the formula above as belonging to the same space $K_{j}$ 
(and not to the various orthogonal spaces $H{_{i,\,j}}$). So $\pmb{b}_{x,\,y}$ 
is a 
vector of $K{\otimes} K$, not of $H{\otimes} H$. Thus
\[
||\pmb{b}_{x,\,y}||^{2}=\sum_{j\,\in J}\,\dfrac{1}{{d_{j}}} \sum_{{{u,\,v}}\,\in I_{j}}
{\ensuremath{{\langle {x_{u,\,j}},{x_{v,\,j}}\rangle}}}\,{\overline{{{\ensuremath{{\langle {y_{u,\,j}},{y_{v,\,j}}\rangle}}}}}}.
\]
Combining Corollary \ref{Corollary 5.4} with the formula
\[
m\bigl(|{\ensuremath{{\langle {\pi (\,\centerdot\,)x},{y}\rangle}}}|^{2} \bigr)={\ensuremath{{\langle {P_{\pi
{\otimes}\,{\overline{{\vphantom{t}\pi }}}}\,\,x{\otimes}{\overline{{x}}}},{y{\otimes}{\overline{{y}}}\,}\rangle}}}={\ensuremath{{\langle {{\textbf{\textsf{P}}}_{\pi 
}{\ensuremath{{\langle {\,\centerdot\,},{x}\rangle}}}x},{{\ensuremath{{\langle {\,\centerdot\,},{y}\rangle}}}{y}}\rangle}}}
\]
yields
\begin{corollary}\label{Corollary 5.5}
 Let $\pi $ be a compact representation of $G$ on $H$, written in the form
 $\pi 
={\mathop{\oplus}}\limits_{j\,\in J}\bigl({\mathop{\oplus}}\limits_{i\,\in I_{j}}\pi {_{i,\,j}} 
\bigr)$ as in Proposition \ref{Proposition 5.1}. For every vectors 
$x 
={\mathop{\oplus}}\limits_{j\,\in J}\bigl({\mathop{\oplus}}\limits_\iijx {_{i,\,j}} 
\bigr)$ and $y 
={\mathop{\oplus}}\limits_{j\,\in J}\bigl({\mathop{\oplus}}\limits_\iijy {_{i,\,j}} 
\bigr)$ of $H$, 
\begin{align}
  m\bigl(|{\ensuremath{{\langle {\pi (\,\centerdot\,)x},{y}\rangle}}}|^{2} 
\bigr)&=\sum_{j\,\in J} \dfrac{1}{{d_{j}}} \sum_{{{u,\,v}}\,
 \in I_{j}}{\ensuremath{{\langle {U_{u,\,j}\,x_{u,\,j}},{U_{v,\,j}\,x_{v,\,j}}\rangle}}}\,.\,
 {\overline{{{\ensuremath{{\langle {U_{u,\,j}\,y_{u,\,j}},{U_{v,\,j}\,y_{v,\,j}}\rangle}}}}}}\label{Eq3bis}\\
 &=||\pmb{b}_{\,Ux,\,Uy}||^{2}.\notag
\end{align}
\end{corollary}
We thus obtain the following abstract version of the Wiener Theorem for 
unitary representations of a group $G$:
\begin{theorem}\label{Theorem 5.6}
 Let $\pi $ be a unitary representation of $G$ on a Hilbert 
space $H$. 
We decompose $H$ as $H=H_{w}{\mathop{\oplus}} H_{c}$ and $\pi$ as $\pi =\pi _{w}{\mathop{\oplus}} \pi _{c}$, where 
$\pi_{w}$ is the weakly mixing part of $\pi $ and $\pi 
_{c}$ its compact part. Writing $\pi 
_{c}={\mathop{\oplus}}\limits_{j\,\in J}\bigl({\mathop{\oplus}}\limits_{i\,\in I_{j}}\pi {_{i,\,j}} 
\bigr)$ as in Corollary \ref{Corollary 5.5} above, we have for every 
vectors $x,y$ of $H$, written respectively $x=x_{w}{\mathop{\oplus}} x_{c}$ and $y=y_{w}{\mathop{\oplus}} y_{c}$ with respect to the decomposition $H=H_{w}{\mathop{\oplus}} H_{c}$,
\begin{equation}\label{Eq4}
 m\bigl(|{\ensuremath{{\langle {\pi (\,\centerdot\,)x},{y}\rangle}}}|^{2} 
\bigr)=||\,\pmb{b}_{\,Ux_{c},\,Uy_{c}}||^{2}.
\end{equation}
\end{theorem}
\begin{proof}
 This follows directly from the facts that
\[
m(|{\ensuremath{{\langle {\pi (\,\centerdot\,)x},{y}\rangle}}}|^{2})=
m\bigl(|{\ensuremath{{\langle {\pi_{w} (\,\centerdot\,)x_{w}},{y_{w}}\rangle}}}|^{2} 
\bigr)+m\bigl(|{\ensuremath{{\langle {\pi_{c} (\,\centerdot\,)x_{c}},{y_{c}}\rangle}}}|^{2} 
\bigr)
\] and that
$m(|{\ensuremath{{\langle {\pi_{w} (\,\centerdot\,)x_{w}},{y_{w}}\rangle}}}|^{2})=0,$
and from Corollary \ref{Corollary 5.5}.
\end{proof}
\par 
Theorem \ref{Theorem 5.6} will be a crucial tool for the proof of our main 
result, to be given in Section \ref{Section 5}. We will also need an 
inequality on the quantities $m\bigl(|{\ensuremath{{\langle {\pi (\,\centerdot\,)x},{y}\rangle}}}|^{2} 
\bigr)$ for a compact representation $\pi$, which is a direct consequence of Corollary \ref{Corollary 5.5}. Using the same notation as in the statement of Corollary \ref{Corollary 5.5}, we
denote by $x={\mathop{\oplus}}_{j\,\in J}{\widetilde{{x}}}_{j}$ and $y={\mathop{\oplus}}_{j\,\in J}{\widetilde{{y}}}_{j}$ the 
respective decompositions of the vectors $x$ and $y$ of $H$ with respect 
to the decomposition $H={\mathop{\oplus}}_{j\,\in J}{\widetilde{{H}}}_{j}$ of $H$. Then we have
\begin{corollary}\label{Corollary 5.7}
Let $\pi$ be a compact representation of $G$ on $H$.
 Under the assumptions of Theorem \ref{Theorem 5.6}, we have for every 
vectors $x$ and $y$ of $H$
\[
m\bigl(|{\ensuremath{{\langle {\pi (\,\centerdot\,)x},{y}\rangle}}}|^{2} 
\bigr)\le\sum_{j\,\in J}\,\dfrac{1}{{d_{j}}}\, ||{\widetilde{{x}}}_{j}||^{2}.\,||{\widetilde{{y}}}_{j}||^{2}\le\sum_{j\,\in J}\, ||{\widetilde{{x}}}_{j}||^{2}.\,||{\widetilde{{y}}}_{j}||^{2}.
\]
\end{corollary}
\begin{proof}
 Applying the Cauchy-Schwarz inequality twice to (\ref{Eq3bis}) yields that
 \begin{align*}
m\bigl(|{\ensuremath{{\langle {\pi (\,\centerdot\,)x},{y}\rangle}}}|^{2} 
\bigr)&\le  \sum_{j\,\in J}\,\dfrac{1}{{d_{j}}}\,
\sum_{{{u,\,v}}\,\in 
I_{j}} ||x_{u,\,j}||\,.\,||x_{v,\,j}||\,.\,||y_{u,\,j}||\,.\,||y_{v,\,j}||\\
&=\sum_{j\,\in J}\,\dfrac{1}{{d_{j}}}\,\Bigl(\,\sum_{u\,\in I_{j}}||x_{u,\,j}||\,.\, 
||y_{u,\,j}||\Bigr) ^{2}\\
&\le \sum_{j\,\in J}\,\dfrac{1}{{d_{j}}}\,\Bigl(\,\sum_{u\,\in I_{j}} ||x_{u,\,j}||^{2}\Bigr)\,.\,
\Bigl(\,\sum_{u\,\in I_{j}} ||y_{u,\,j}||^{2}\Bigr)\\
&=\sum_{j\,\in J}\,\dfrac{1}{{d_{j}}}\,||{\widetilde{{x}}}_{j}||^{2}.\,||{\widetilde{{y}}}_{j}||^{2}
\le\sum_{j\,\in J}\,||{\widetilde{{x}}}_{j}||^{2}.\,||{\widetilde{{y}}}_{j}||^{2}
 \end{align*}
 since $d_{j}\ge 1$ for every $j\in J$.
\end{proof}
\subsection{Why is (\ref{Eq4}) an abstract version of the Wiener 
Theorem?}\label{Section 4.4}
 Recall that 
the classic Wiener Theorem states that for any probability measure 
$\sigma $ on the unit 
circle ${\ensuremath{\mathbb T}}$,
\newsavebox{\abb}
\savebox{\abb}{\smash[b]{\xymatrix@C=13pt{\scriptstyle 
N\ar[r]&\scriptstyle+\infty}}}
\begin{equation}\label{Eq5}
 \xymatrix@C=60pt{\dfrac{1}{2N+1}{\displaystyle}\sum_{n=-N}^{N}|\,\widehat{\sigma 
}(n)|^{2}\ar[r]\ar@{}@<.6ex>[r]_-{\usebox{\abb}}&{\displaystyle}\sum_{
\lambda\,\in\,{\ensuremath{\mathbb T}}}\,\sigma (\{\lambda \})^{2}}.
\end{equation}
The operator $M_{\sigma }$ of multiplication by $e^{i\theta }$ on $L^{2}
({\ensuremath{\mathbb T}},\sigma )$ is unitary, and thus induces a unitary representation $\pi 
_{\sigma }$ of ${\ensuremath{\mathbb Z}}$ on $L^{2}({\ensuremath{\mathbb T}},\sigma )$. Its compact part $\pi _{\sigma 
 _{d} }$ is induced by
the multiplication operator by $e^{i\theta }$ on $L^{2}({\ensuremath{\mathbb T}},\sigma_{d} )$, 
where $\sigma  _{d}$ is the discrete part of $\sigma  $: $\sigma _{d}=\sum_{\lambda \,\in\,D}\sigma (\{\lambda \})\delta 
_{\lambda }$, with $D=\{\lambda \in{\ensuremath{\mathbb T}}\,;\,\sigma (\{\lambda \})>0\}$. We 
have $L^{2}({\ensuremath{\mathbb T}},\sigma_{d})={\mathop{\oplus}}_{\lambda \,\in\,D}L^{2}({\ensuremath{\mathbb T}},
\sigma (\{\lambda \})\delta _{\lambda })$, and $\pi _{\sigma _{d}
}={\mathop{\oplus}}_{\lambda 
\,\in\,D}\pi _{\delta _{\lambda 
}}$, where each $\pi _{\delta _{\lambda 
}}$ is the one-dimensional irreducible representation on 
$L^{2}({\ensuremath{\mathbb T}},\sigma (\{\lambda \})\delta _{\lambda })$ given by 
multiplication by $\lambda $. The representations $\pi _{\delta _{\lambda 
}}$ are pairwise non-equivalent. Hence the operator $U$ (defined in 
Section \ref{Section 4.1}) involved in the 
decomposition of $\pi _{\sigma _{d}
}$ as a direct sum of irreducible representations is equal to the identity 
operator on 
$L^{2}({\ensuremath{\mathbb T}},\sigma )$ and we have for every functions $f$ and $g$ of 
$L^{2}({\ensuremath{\mathbb T}},\sigma )$
\[
||\,\pmb{b}_{f,\,g}||^{2}=\sum_{\lambda \,\in\,D}|\,f(\lambda )\,|
^{2}.\,|g(\lambda )\,|^{2}.\,\sigma (\{\lambda \})^{2}.\]
Applying Theorem \ref{Theorem 5.6} to the functions $f=g=1$ and 
remembering that 
\[
m(F)=\lim_{N\to+\infty }\dfrac{1}{2N+1}\sum_{n=-N}^{N} F(n)
\]
for every $F\in \mbox{WAP({\ensuremath{\mathbb Z}})}$ yields (\ref{Eq5}).
\par\smallskip 
As seen above in the case where $G={\ensuremath{\mathbb Z}}$, Theorem \ref{Theorem 5.6} admits a 
much simpler formulation in the case where $G$ is an abelian group: all 
irreducible representations are one-dimensional and either equal or 
non-equivalent, so that $U$ is the identity operator on $H$. If $\pi$ is a compact representation of $G$, the formula 
(\ref{Eq3bis}) thus becomes 
in this case
\[
m\bigl(|{\ensuremath{{\langle {\pi (\,\centerdot\,)x},{y}\rangle}}}|^{2} 
\bigr)=\sum_{j\,\in J}\
\sum_{{{u,\,v}}\,\in\,I_{j}}x_{u,\,j}\,{\overline{{x}}}_{v,\,j}\,{\overline{{y}}}_{u,\,j}
y_{v,\,j}
\]
since $x_{i,\,j}$ and $y_{i,\,j}$, $i\in I_{j}$, ${j\,\in J}$, are simply 
scalars.
Using the notation of Corollary \ref{Corollary 5.7}, we have
\begin{equation}\label{Eq6}
m\bigl(|{\ensuremath{{\langle {\pi (\,\centerdot\,)x},{y}\rangle}}}|^{2} 
\bigr)=\sum_{j\,\in J}\,\Bigl|\sum_{u\,\in\,I_{j}}x_{u,\,j}\,{\overline{{y}}}_{u,\,j}
\Bigr|^{2}= 
\sum_{j\,\in J}\,\bigl|{\ensuremath{{\langle {{\widetilde{{x}}}_{j}},{{\widetilde{{y}}}_{j}}\rangle}}}\bigr|^{2}\!.
\end{equation}
For every character $\chi \in\Gamma $ (where $\Gamma $ denotes the 
dual 
group of $G$), we denote by $E_{\chi }$ the subspace of $H$
\[
E_{\chi }=\{x\in H\,;\,\pi (g)x=\chi (g)x\ \textrm{for every}\ g\in G\}
\]
and by $P_{\chi }$ 
the orthogonal projection of $H$ on $E_{\chi}$.
Each representation $\pi _{j}$, $j\in J$, being in fact a character $\chi 
_{j}$ on the group $G$, we can identify the space ${\widetilde{{H}}}_{j}$ with 
$E_{\chi_{j} }$. Equation (\ref{Eq6}) then yields the following corollary:

\begin{corollary}\label{cor+}
 Let $G$ be an abelian group, and let $\pi$ be a representation of $G$ on a Hilbert space $H$. Then we have for every $x,y\in H$
 \[
m\bigl(|{\ensuremath{{\langle {\pi (\,\centerdot\,)x},{y}\rangle}}}|^{2} 
\bigr)=\sum_{j\,\in J}\,\bigl|{\ensuremath{{\langle {P_{E_{\chi _{j}}}x},{P_{E_{\chi _{j}}}y}\rangle}}}
\bigr|^{2}= 
\sum_{\chi\,\in\,\Gamma }\,\bigl|{\ensuremath{{\langle {P_{E_{\chi}}x},{P_{E_{\chi 
}}y}\rangle}}}\bigr|^{2} \!.
\]
In particular, if $x=y$,
\[
m\bigl(|{\ensuremath{{\langle {\pi (\,\centerdot\,)x},{x}\rangle}}}|^{2} 
\bigr)=\sum_{\chi\,\in\,\Gamma }\,\bigl|\bigl|P_{E_{\chi 
}}x\bigr|\bigr|^{4}\!.
\]
\end{corollary}

Specializing Corollary \ref{cor+} to the case where $G={\ensuremath{\mathbb Z}}$ yields the well-known result (see for instance \cite{BalloGold} or \cite{Farkas}) that for any 
unitary operator $U$ on $H$, and any vectors $x,y\in H$,
\newsavebox{\zzz}
\savebox{\zzz}{\smash[b]{\xymatrix@C=13pt{\scriptstyle 
N\ar[r]&\scriptstyle+\infty}}}
\[
\xymatrix@C=60pt{
\dfrac{1}{2N+1}{\displaystyle}\sum_{n=-N}^{N}\,\bigl|{\ensuremath{{\langle {U^{n}x},{y}\rangle}}}\bigr|
^{2}\ar[r]\ar@{}@<.6ex>[r]_-{\usebox{\zzz}}&{\displaystyle}\sum_{\lambda 
\,\in\,{\ensuremath{\mathbb T}}}\,\bigl|{\ensuremath{{\langle {P_{\ker(U-\lambda 
\textrm{Id}_{H})}x},{P_{\ker(U-\lambda 
\textrm{Id}_{H})}y}\rangle}}}\bigr|^{2}\!.
}
\]
In particular, we have
\newsavebox{\ccc}
\savebox{\ccc}{\smash[b]{\xymatrix@C=13pt{\scriptstyle 
N\ar[r]&\scriptstyle+\infty}}}
\[
\xymatrix@C=60pt{
\dfrac{1}{2N+1}{\displaystyle}\sum_{n=-N}^{N}\,\bigl|{\ensuremath{{\langle {U^{n}x},{x}\rangle}}}\bigr|
^{2}\ar[r]\ar@{}@<.6ex>[r]_-{\usebox{\ccc}}&{\displaystyle}\sum_{\lambda 
\,\in\,{\ensuremath{\mathbb T}}}\,\bigl|\bigl|P_{\ker(U-\lambda 
\textrm{Id}_{H})}x\bigr|\bigr|^{4}\!.
}
\]

We refer the reader to \cite{AnBi,Ballo,BalloGold,BjFi,Farkas} and the references therein for related 
aspects and generalizations of the Wiener  Theorem. 
\par\smallskip
We now have all the necessary tools for the proof of Theorem \ref{Theorem 
0}, which we present in the next section.

\section{Proof of Theorem \ref{Theorem 0}}\label{Section 5}
\subsection{Notation}\label{Section 5.1}
Let $(W_{n})_{n\ge 1}$ be an increasing sequence of subsets of $G$ satisfying the 
assumptions of Theorem \ref{Theorem 0}, and let ${Q}$ be a subset of $G$. 
For each ${n\ge 1}$, we denote by ${Q_{n}}$ the set ${Q_{n}}=
W_{n}\cup Q$. Remark that $G$ is the increasing union of the sets 
${Q_{n}}$, ${n\ge 1}$.
\par\smallskip 
 In order to prove Theorem \ref{Theorem 0}, we argue by contradiction, and 
suppose that property (\ref{Pro1}) of Theorem \ref{Theorem 0} holds true, 
while $Q_{n}$ is a non-{Kazhdan}\ set in $G$ for every ${n\ge 1}$. We will 
then 
construct for every $\varepsilon >0$ a representation $\pi $ of $G$ which 
admits a $(Q,\varepsilon )$-invariant vector, but is weakly mixing (which, 
by Proposition \ref{Proposition 4.1}, is equivalent to the fact that $\pi 
$ has no finite dimensional subrepresentation), and this will contradict 
(\ref{Pro1}). We denote by $\varepsilon _{0}$ a positive constant such 
that assumption (\ref{Pro1}) holds true: any representation of $G$ 
admitting a $({Q},\varepsilon _{0})$-invariant vector has a 
finite dimensional subrepresentation. 

\subsection{Construction of a sequence $(\pi _{n})_{n\ge 1}$ of 
finite dimensional representations of $G$}\label{Section 5.2}
The first step of the proof is to show that assumption (\ref{Pro1}) 
combined with the 
hypothesis that ${Q_{n}}$ is a non-{Kazhdan}\ set for every ${n\ge 1}$ implies the 
existence of sequences of finite dimensional representations of $G$ with 
certain properties.
\begin{lemma}\label{Lemma 6.1}
For every sequence $(\varepsilon _{n})_{n\ge 1}$ of positive real numbers 
decreasing to zero with  $\varepsilon _{1}\in(0,\varepsilon _{0}]$, 
there exist a sequence $(H_{n})_{n\ge 1}$ of finite dimensional Hilbert 
spaces and a sequence $(\pi_{n})_{n\ge 1}$ of unitary representations of $G$ such that for every $n\ge 1$, $\pi _{n}$ is a representation of $G$ on ${H_{n}} $ with the following two properties:
\begin{enumerate}
 \item [--] $\pi _{n}$ has no non-zero $G$-invariant vector;
 \item[--] $\pi _{n}$ has  $({Q_{n}},{\varepsilon _{n}})$-invariant unit vector 
${a_{n}}\in H_{n}$: $||{a_{n}}||=1$ and 
\[
\sup_{g\,\in\,{Q_{n}}}
||\,\pi _{n}(g){a_{n}}-{a_{n}}||<{\varepsilon _{n}}.
\]
\end{enumerate}
\end{lemma}
\begin{proof}[Proof of Lemma \ref{Lemma 6.1}] Let ${n\ge 1}$. Since ${Q_{n}}$ is a not a {Kazhdan}\ set in 
$G$, there exists a representation $\rho _{n}$ of $G$ on a Hilbert space 
${K_{n}}$ which has no non-zero $G$-invariant vector, but is such that there 
exists a unit vector ${x_{n}}\in{K_{n}}$ with
\[
\sup_{g\,\in\,{Q_{n}}} ||\,\rho _{n}(g){x_{n}}-{x_{n}}||<{\varepsilon _{n}}.
\]
 Since ${\varepsilon _{n}}\le\varepsilon _{0}$, assumption (\ref{Pro1}) 
implies that $\rho _{n}$ has a finite dimensional subrepresentation. 
By Proposition \ref{Proposition 4.1}, $\rho_{n}$ is not weakly mixing. This 
means that if we decompose $K_{n}$ as $K_{n}=K_{n,\,w}{\mathop{\oplus}} K_{n,\,c}$ and 
$\rho _{n}$ as $\rho _{n}=\rho  _{n,\,w}{\mathop{\oplus}} \rho _{n,\,c}$, where 
$\rho _{n,w}$ and 
$\rho _{n,c}$ are respectively 
the weakly mixing and compact parts of $G$, $\rho _{n,\,c}$ is a non-zero 
representation. Since $\rho _{n}$ has no non-zero $G$-invariant vector, 
neither have $\rho _{n,\,w}$ nor $\rho _{n,\,c}$.
\par\smallskip 
Decomposing ${x_{n}}$ as ${x_{n}}=x_{n,\,w}{\mathop{\oplus}} x_{n,\,c}$, we have
$1=||x_{n,\,w}||^{2}+||x_{n,\,c}||^{2}$. We claim that
$\underline{\lim}_{\,n\to+\infty }\,||\,x_{n,\,c}||>0$. Indeed, suppose 
that it is not the case. Then $\overline{\lim}_{\,n\to+\infty 
}||x_{n,\,w}||=1$. Since $||\rho _{n}(g){x_{n}}-{x_{n}}||^{2}=||\rho 
_{n,\,w}(g)x_{n,\,w}-x_{n,\,w}||^{2}+||\rho 
_{n,\,c}(g)x_{n,\,c}-x_{n,\,c}||^{2}$ for every $g\in G$, we have
\begin{align*}
 &\sup_{g\,\in\,Q_{n}}||\rho 
_{n,\,w}(g)x_{n,\,w}-x_{n,\,w}||<{\varepsilon _{n}},
\intertext{so that}
&\sup_{g\,\in\,Q_{n}}\Bigl|\Bigl|\rho 
_{n,\,w}(g)\dfrac{x_{n,\,w}}{||x_{n,\,w}||}-\dfrac{x_{n,\,w}}{||x_{n,\,w}||
}\Bigr|\Bigr|<\dfrac{\varepsilon _{n}}{||x_{n,\,w}||} 
\end{align*}
as soon as $x_{n,\,w}$ is non-zero. Since 
$\overline{\lim}_{\,n\to+\infty }||x_{n,\,w}||=1$ and since $(\varepsilon _{n})_{n\ge 1}$ is decreasing, this implies that for 
any $\delta >0$ there exists an integer $n$ such that $\rho _{n,\,w}$ has 
a $({Q_{n}},\delta )$-invariant vector of norm $1$. 
Applying this to $\delta =\varepsilon_{0} $, there exists $n_{0}\ge 1$ such that $\rho _{n_{0},\,w}$ has a $(Q_{n_{0}}, \varepsilon _{0})$-invariant vector, hence a $(Q,\varepsilon _{0})$-invariant vector.
But
$\rho _{n_{0},\,w}$ is weakly mixing, so has no 
finite dimensional subrepresentation. This 
contradicts assumption (\ref{Pro1}). So we deduce that
$\underline{\lim}_{\,n\to+\infty }\,||\,x_{n,\,c}||=\gamma >0$.
The same observation as above, applied to the representation $\rho 
_{n,\,c}$, shows that
\begin{align*}
 \sup_{g\,\in\,Q_{n}}\Bigl|\Bigl|\rho 
_{n,\,c}(g)\dfrac{x_{n,\,c}}{||x_{n,\,c}||}-\dfrac{x_{n,\,c}}{||x_{n,\,c}||
}\Bigr|\Bigr|&<\dfrac{\varepsilon _{n}}{||x_{n,\,c}||} 
\intertext{for every $n$ such that $x_{n,\,c}$ is non-zero, and thus that }
\sup_{g\,\in\,Q_{n}}\Bigl|\Bigl|\rho 
_{n,\,c}(g)\dfrac{x_{n,\,c}}{||x_{n,\,c}||}-\dfrac{x_{n,\,c}}{||x_{n,\,c}||
}\Bigr|\Bigr|&<\dfrac{2{\varepsilon _{n}}}{\gamma } 
\end{align*}
for infinitely many integers $n$. For these integers, $\rho _{n,\,c}$ is a 
compact representation for  which ${y_{n}}=x_{n,\,c}/||x_{n,\,c}||$ is a $({Q_{n}},2\,{\varepsilon _{n}}/\gamma 
)$-invariant vector of norm $1$. It has no non-zero 
$G$-invariant vector. Since $\rho _{n,\,c}$ is compact, it can be 
decomposed as a direct sum of finite dimensional representations: 
$\rho _{n,\,c}={\mathop{\oplus}}\limits_{l\,\in{L_{n}}}\rho _{n,\,c}^{(l)}$, where 
$K_{n,\,c}={\mathop{\oplus}}\limits_{l\,\in{L_{n}}}K_{n,\,c}^{(l)}$, each space 
$K_{n,\,c}^{(l)}$ is finite dimensional, $\rho _{n,\,c}^{(l)}$ is a 
representation of $G$ on $K_{n,\,c}^{(l)}$, and ${L_{n}}$ is a certain set of 
indices. If ${L_{n}}$ is finite, $\rho _{n,\,c}$ is already a 
finite dimensional 
representation of $G$ with a $({Q}_{n},2{\varepsilon _{n}}/\gamma )$-invariant vector 
and no non-zero $G$-invariant vector. So suppose that ${L_{n}}$ is infinite: 
writing ${y_{n}}$ as
${y_{n}}={\mathop{\oplus}}\limits_{l\,\in{L_{n}}}{y_{n}}^{(l)}$ with respect to the decomposition 
$K_{n,\,c}={\mathop{\oplus}}\limits_{l\,\in{L_{n}}}K_{n,\,c}^{(l)}$, we have for every finite 
subset $F$ of $L_{n}$ and every $g\in Q_{n}$
\begin{align*}
\Bigl|\Bigl|\,{\mathop{\oplus}}_{l\,\in L_{n}\,\cap\,F }\rho 
_{n,\,c}^{(l)}(g)\,\,\Bigl(\,{\mathop{\oplus}}_{l\,\in L_{n}\,\cap\,F }y_{n}^{(l)} 
\Bigr)-{\mathop{\oplus}}_{l\,\in L_{n}\,\cap\,F }
{y_{n}}^{(l)}\Bigr|\Bigr|^{2}\!\!&<\Bigl(\dfrac{2\,{\varepsilon _{n}}}{\gamma } 
\Bigr)^{2}+\sum_{l\,\in\, L_{n}\,\setminus\,F }
\bigl|\bigl|\rho _{n,\,c}^{(l)}(g)\,{y_{n}}^{(l)}-{y_{n}}^{(l)}\Bigr|\Bigr|^{2}\\
&<\Bigl(\dfrac{2\,{\varepsilon _{n}}}{\gamma } 
\Bigr)^{2}+\sum_{l\,\in\,L_{n}\,\setminus\,F }
4\,\bigl|\bigl|{y_{n}}^{(l)} \bigr| \bigr|^{2}\!.
\end{align*}
Since $\bigl|\bigl|{y_{n}} \bigr| 
\bigr|^{2}={\displaystyle}\sum_{l\,\in{L_{n}}}\bigl|\bigl|{y_{n}}^{(l)} \bigr| \bigr|^{2}=1$, 
there exists a finite subset $F$ of $L$ such that the remainder term 
${\displaystyle}\sum_{l\,\in\,L_{n}\,\setminus\,F }
4\,\bigl|\bigl|{y_{n}}^{(l)} \bigr| \bigr|^{2}$ is less than $(\varepsilon 
_{n}/\gamma )^{2}$.
It follows that 
\[
\sup_{g\,\in\,Q_{n}}\Bigl|\Bigl|\,{\mathop{\oplus}}_{l\,\in L_{n}\,\cap\,F }\rho 
_{n,\,c}^{(l)}(g)\,\,\Bigl(\,{\mathop{\oplus}}_{l\,\in L_{n}\,\cap\,F }y_{n}^{(l)} 
\Bigr)-{\mathop{\oplus}}_{l\,\in L_{n}\,\cap\,F }
{y_{n}}^{(l)}\Bigr|\Bigr|^{2}<{5}\,\,\Bigl(\dfrac{\varepsilon _{n}}{\gamma } 
\Bigr)^{2}\cdot
\]
Recalling that
\[
\Bigl|\Bigl|{\mathop{\oplus}}_{l\,\in L_{n}\,\cap\,F }
{y_{n}}^{(l)}\Bigr|\Bigr|^{2}=\sum_{l\,\in L_{n}\,\cap\,F }
\bigl|\bigl|{y_{n}}^{(l)} \bigr| \bigr|^{2}>1-\Bigl(\dfrac{\varepsilon 
_{n}}{2\gamma } \Bigr)^{2},
\]
we thus obtain that (for $n$ large enough) ${\displaystyle}{\mathop{\oplus}}_{l\,\in L_{n}\,\cap\,F }
\rho _{n,\,c}^{(l)}$ is a finite dimensional representation of $G$ having 
no non-zero 
$G$-invariant vector but admitting a $({Q_{n}},4\,{\varepsilon _{n}}/\gamma 
)$-invariant vector. 
Summing things up, we obtain that there exists for infinitely many 
integers ${n\ge 1}$ a finite dimensional representation $\sigma _{n}$ of $G$ with a 
$({Q}_{n},4{\varepsilon _{n}}/\gamma )$-invariant vector but no non-zero $G$-invariant 
vector. Since $(\varepsilon _{n})_{n\ge 1}$ decreases to zero,
Lemma \ref{Lemma 6.1} follows immediately.
\end{proof}
\subsection{{Kazhdan}\ property for ampliations of finite dimensional 
irreducible representations}\label{Section 5.3}
In order to complete the proof of Theorem \ref{Theorem 0}, we need to 
consider 
separately two cases, depending on whether or not infinitely many of the 
non-{Kazhdan}\ sets ${Q_{n}}$ satisfy a weak form of the {Kazhdan}\ property, which we 
call the \emph{{Kazhdan}\ property for ampliations of 
finite dimensional irreducible 
representations.} Let us first introduce the following definition.
\begin{definition}\label{Definition 6.1.1}
 Let $\mathscr{C}$ be a class of representations of $G$ on a Hilbert 
space. 
We say that a subset ${Q}$ of $G$ is a \emph{{Kazhdan}\ set for the class 
$\mathscr{C}$} if there exists a positive constant $\varepsilon 
'$ such that the following holds true: if $\pi $ is a 
representation of $G$ belonging to $\mathscr{C}$ which admits a 
$({Q},\varepsilon')$-invariant vector, then $\pi $ has a 
non-zero $G$-invariant vector.
\end{definition}
Kazhdan's Property (T) for particular classes of representations is a much studied topic. 
For instance Property ($\tau$) for discrete groups 
(see, among other references \cite{Kass2} or \cite{Lub}) is Property (T) 
for the class of representations which factor through a finite index subgroup.
\par\smallskip
We will consider here the class $\mathscr{C}_{0}$ consisting of all 
ampliations of finite dimensional irreducible representations of $G$. 
These are the 
representations of $G$ of the form \[\pi _{I}={\mathop{\oplus}}\limits_{i\,\in\,I}\pi 
_{i},\] 
where $I$ is a finite or infinite set of indices, and $\pi _{i}=\pi $ for 
every $i\in I$, 
with $\pi $ an irreducible representation of $G$ on a 
finite dimensional space $H$. The representation $\pi _{I}$ is thus a 
representation of $G$ on $H_{I}={\mathop{\oplus}}\limits_{i\,\in\,I}H_{i}$, with 
$H_{i}=H$ 
for every $i\in I$. When no confusion arises, we simply write 
$\pi _{I}={\mathop{\oplus}}\limits_{i\,\in\,I}\pi$ and $H_{I}={\mathop{\oplus}}\limits_{i\,\in\,I}H$.

\begin{fact}\label{Fact 6.1.2} Let ${Q}$ be a subset of a topological group $G$. The following assertions are equivalent:
\par\smallskip
\begin{itemize}
 \item [$(\alpha )$]\label{a} ${Q}$ has the {Kazhdan}\ property for ampliations of 
finite dimensional irreducible 
representations, with associated constant $\varepsilon '$;
\item [$(\beta )$]\label{b} let $\pi$ be a finite dimensional irreducible 
representation of $G$ on $H$, and let
$(x_{i})_{i\,\in\,I}$ be a finite or infinite family of vectors of $H$. If
\[
\sup_{g\,\in\,{Q}}\Bigl({\displaystyle}\sum_{i\,\in\,I}\,||\pi (g)x_{i}-x_{i}||^{2} 
\Bigr)^{1/2}<\varepsilon '\Bigl({\displaystyle}\sum_{i\,\in\,I}||x_{i}||^{2}\Bigr)^{1/2},
\]
 then $\pi $ is the unit representation $\mathbbm{1}_{G}$;
\item [$(\gamma )$]\label{c} let $\pi$ be a finite dimensional irreducible 
representation of $G$ on $H$, and let
$(x_{i})_{i\,\in\,I}$ be a finite family of vectors of $H$. If
\[
\sup_{g\,\in\,{Q}}\Bigl({\displaystyle}\sum_{i\,\in\,I}\,||\pi (g)x_{i}-x_{i}||^{2} 
\Bigr)^{1/2}<{\varepsilon '}\, \Bigl({\displaystyle}\sum_{i\,\in\,I}||x_{i}||^{2}\Bigr)^{1/2},
\]
 then $\pi =\mathbbm{1}_{G}$.
\end{itemize}
\end{fact}

\begin{proof}
 Let $\pi $ be a finite dimensional irreducible  representation of $G$ on 
a Hilbert space $H$, and let $I$ be a finite or infinite set of indices. Then a vector 
$x={\mathop{\oplus}}\limits_{i\,\in\,I}x_{i}$ of ${\mathop{\oplus}}\limits_{i\,\in\,I}H$ is a 
$({Q},\varepsilon')$-invariant vector for $\pi _{I}$ if and only if 
\[
\sup_{g\,\in\,{Q}}||\pi 
_{I}(g)x-x||=\sup_{g\,\in\,{Q}}\Bigl(\,\sum_{i\,\in\,I}||\pi 
(g)x_{i}-x_{i}||^{2} 
\Bigr)^{1/2}\!\!\!<\varepsilon'\Bigl(\sum_{i\,\in\,I}||x_{i}||^{2} \Bigr)^{1/2} .
\]
So property 
($\beta $) is equivalent to the fact that if 
$\pi $ is a finite dimensional irreducible representation having an 
ampliation with a 
$({Q},\varepsilon')$-invariant vector, $\pi =\mathbbm{1}_{G}$. 
Since $\pi $ is assumed to be irreducible, $\pi $ has a non-zero $G$-invariant vector 
if and only if $\pi=\mathbbm{1}_{G}$. Thus pro\-perties
($\alpha $) and ($\beta $) are equivalent. 
That ($\beta $) implies ($\gamma  $) is obvious. Suppose now that ($\gamma $) holds true. 
Let $\pi $ be a finite dimensional irreducible representation of $G$ on $H$, 
and let $(x_{i})_{i\,\in\,I}$ be an infinite family of vectors of $H$ such that
\[
\sup_{g\,\in\,{Q}}\Bigl({\displaystyle}\sum_{i\,\in\,I}\,||\pi (g)x_{i}-x_{i}||^{2} 
\Bigr)^{1/2}<\varepsilon '\Bigl({\displaystyle}\sum_{i\,\in\,I}||x_{i}||^{2}\Bigr)^{1/2}\!\!.
\]
There exists a $\delta >0$ such that 
\[
\sup_{g\in{Q}}\sum_{i\in I}||\pi (g)x_{i}-x_{i}||^{2}<(\varepsilon '^{\,2} -\delta )\sum_{i\in I}||x_{i}||^{2}.
\]
For any finite subset $F$ of $I$, we have 
\begin{align*}
\sup_{g\,\in\,{Q}}{\displaystyle}\sum_{i\,\in\,F}\,||\pi (g)x_{i}-x_{i}||^{2} 
&<\varepsilon '^{\,2} {\displaystyle}\sum_{i\,\in\,F}||x_{i}||^{2}+\varepsilon '^{\,2} \!\! {\displaystyle}\sum_{i\,\in\,I\setminus F}||x_{i}||^{2}.
\end{align*}
It is not difficult to see that if the finite set $F$ is sufficiently large, 
\[
(\varepsilon '^{\,2} -\delta )\sum_{i\,\in\,I\setminus F}||x_{i}||^{2}<\delta \sum_{i\,\in\,F}||x_{i}||^{2}.
\]
We then have 
\[
\sup_{g\,\in\,{Q}}\Bigl({\displaystyle}\sum_{i\,\in\,F}\,||\pi (g)x_{i}-x_{i}||^{2} 
\Bigr)^{1/2}<{\varepsilon '}\, \Bigl({\displaystyle}\sum_{i\,\in\,F}||x_{i}||^{2}\Bigr)^{1/2}.
\]
By property ($\gamma $), $\pi=\mathbbm{1}_{G}$. This finishes the proof of Fact \ref{Fact 6.1.2}.
\end{proof}

Let us now go back to the different cases in the proof of Theorem 
\ref{Theorem 0}.
\subsection{Case 1}\label{Section 6.5} 
Suppose that there exists an integer $n_{0}$ such that ${Q}_{n_{0}}$ is 
a {Kazhdan}\ set for ampliations of finite dimensional irreducible 
representations, with associated 
constant $\varepsilon' _{0}$. 
Then, since $({Q_{n}})_{n\ge 1}$ is an increasing sequence of sets, ${Q_{n}}$ is for 
every $n\ge n_{0}$ a {Kazhdan}\ set for ampliations of finite dimensional 
irreducible representations with a common associated constant $\varepsilon'
 _{0} $.
\par
Let now $\varepsilon >0$ be an arbitrary positive number. We fix a 
sequence $({\varepsilon _{n}})_{n\ge 1}$ of positive numbers decreasing to zero so fast that
the following properties hold:
\begin{enumerate}
 \item [(i)] $0<{\varepsilon _{n}}<\varepsilon _{0}$ for every ${n\ge 1}$, and 
$\sum_{n\ge 1}{\varepsilon _{n}}<\varepsilon ^{2}/2$;
\item[(ii)] for every $n\ge n_{0}$, ${\varepsilon _{n}}<2^{-n}\varepsilon' _{0}$.
\end{enumerate}
This sequence $(\varepsilon _{n})_{n\ge 1}$ being fixed, we consider the representation
$\pmb{\pi} ={\otimes}_{n\ge 1}\,\pi _{n}$ of $G$ on the infinite tensor product  space 
$\pmb{H}={\bigotimes_{n\ge 1}^{\mkern 1.5 mu\pmb{a}}{H_{n}}}$, 
where the spaces $H_{n}$, the 
representations $\pi _{n}$ and the vectors ${a_{n}}$ are associated to 
${\varepsilon _{n}}$ for each ${n\ge 1}$ by Lemma \ref{Lemma 6.1}. Our aim is to prove the 
following:
\begin{proposition}\label{Proposition 6.2}
 Under the assumptions above, $\pmb{\pi} $ is well-defined and enjoys the following properties:
 \begin{enumerate}
  \item [{(a)}]$\pmb{\pi} $ is a strongly continuous representation 
of 
$G$ 
on $\pmb{H}=
  {\bigotimes_{n\ge 1}^{\mkern 1.5 mu\pmb{a}}{H_{n}}}$;
  \item[{(b)}] $\pmb{\pi}$ has a $({Q},\varepsilon )$-invariant 
vector;
  \item[{(c)}]$\pmb{\pi}$ is weakly mixing.
 \end{enumerate}
\end{proposition}
Proposition \ref{Proposition 6.2} 
proves the existence, for every $\varepsilon >0$, of a representation of $G$ with a $({Q},\varepsilon 
)$-invariant vector but no finite dimensional subrepresentation. This
clearly violates assumption (\ref{Pro1}) 
of Theorem \ref{Theorem 0}. It thus suffices to prove Proposition 
\ref{Proposition 6.2} to conclude the proof of Theorem \ref{Theorem 0} 
under the additional assumption of Case 1.
\begin{proof}[Proof of Proposition \ref{Proposition 6.2}] We will prove 
successively properties (a) to (c), the hardest one being property (c).
\par\smallskip 
\noindent\textsf{Proof of property} \textrm{(a)}.
 In order to prove that $\pmb{\pi} $ is well-defined and strongly continuous, it suffices 
to 
check that the assumptions of Proposition \ref{Proposition 3.2} in the appendix hold true. 
For every $g\in G$ and ${n\ge 1}$, we have
$|1-{\ensuremath{{\langle {\pi _{n}(g){a_{n}}},{a_{n}}\rangle}}}|\le||\pi _{n}(g){a_{n}}-{a_{n}}||$ so that 
$$\sup_{g\,\in\,Q_{n}}|1-{\ensuremath{{\langle {\pi _{n}(g){a_{n}}},{a_{n}}\rangle}}}|<\varepsilon _{n}.$$
By assumption (i), the series $\sum_{n\ge 1}{\varepsilon _{n}}$ is convergent. Since every 
element $g\in G$ belongs to all the sets ${Q_{n}}$ except finitely many, the 
series $\sum_{n\ge 1}|1-{\ensuremath{{\langle {\pi _{n}(g){a_{n}}},{a_{n}}\rangle}}}|$ is convergent for every 
$g\in G$. Moreover, it is uniformly convergent on $Q_{1}$, and hence on 
$W_{1}$. The function 
\[
\smash{\xymatrix{g\ar@{|->}[r]&{\displaystyle}\sum_{n\ge 1}|1-{\ensuremath{{\langle {\pi _{n}(g){a_{n}}},{a_{n}}\rangle}}}|}}
\]
\par\smallskip 
\noindent
is thus continuous on $W_{1}$, which is a neighborhood of $e$. It follows 
then from Proposition \ref{Proposition 3.2} that $\pmb{\pi } $ is strongly 
continuous on $\pmb{H}$. If $G$ is locally compact,  
Proposition \ref{Proposition 3.2.0} and the first part of the argument 
above suffice to show that $\pmb{\pi }$ is strongly continuous, even when 
 $W_{1}$ is not a neighborhood of $e$.
\par\smallskip  
\noindent\textsf{Proof of property} (b).
It is easy to check that the 
elementary vector ${\mkern 1.5 mu\pmb{a}}={\otimes}_{n\ge 1}{a_{n}}$ of ${\bigotimes_{n\ge 1}^{\mkern 1.5 mu\pmb{a}}{H_{n}}}$ satisfies
$||{\mkern 1.5 mu\pmb{a}}||=1$ and $\sup_{g\in{Q}}||\pmb{\pi } (g){\mkern 1.5 mu\pmb{a}}-{\mkern 1.5 mu\pmb{a}}||<\varepsilon 
$.
 Indeed $||{\mkern 1.5 mu\pmb{a}}||=\prod_{n\ge 1}||{a_{n}}||=1$, and for every $g\in {Q}$ 
we have (using the fact that ${Q}\subseteq {Q_{n}}$ for every ${n\ge 1}$)
\begin{align*}
 ||\pmb{\pi } (g){\mkern 1.5 mu\pmb{a}}-{\mkern 1.5 mu\pmb{a}}||^{2}&=2\,(1-\Re e{\ensuremath{{\langle {\pmb{\pi} 
(g){\mkern 1.5 mu\pmb{a}}},{\mkern 1.5 mu\pmb{a}}\rangle}}})\le 
2\,
 \Bigl|1-\prod_{n\ge 1}{\ensuremath{{\langle {\pi _{n}(g){a_{n}}},{a_{n}}\rangle}}}\Bigr|\\
 &\le 2\sum_{n\ge 1}|1-{\ensuremath{{\langle {\pi _{n}(g){a_{n}}},{a_{n}}\rangle}}}|<2\sum_{n\ge 1}{\varepsilon _{n}} .
\end{align*}
Assumption (i) on the sequence $({\varepsilon _{n}})_{n\ge 1}$ implies that 
$\sup_{g\,\in\,{Q}} ||\pmb{\pi } (g){\mkern 1.5 mu\pmb{a}}-{\mkern 1.5 mu\pmb{a}}||^{2}<\varepsilon ^{2}$,
and ${\mkern 1.5 mu\pmb{a}}$ is thus a $({Q},\varepsilon )$-invariant vector for $\pmb{\pi }$.
\par\smallskip 
\noindent\textsf{Proof of property} (c).
 It remains to show that $\pmb{\pi }$ is weakly mixing. By Proposition 
\ref{Proposition 4.3}, it suffices to prove that 
$\underline{\lim}_{\,\,n\to+\infty\, }m(\,|{\ensuremath{{\langle {\pi 
_{n}(\,\centerdot\,){a_{n}}},{a_{n}}\rangle}}}|^{2})=0$, where $m$ denotes as in the appendix and in Section
\ref{Section 3} the invariant mean on 
\mbox{WAP$(G)$}. Let $n \ge n_{0}$. Using the notation of Section \ref{Section 4.1}, we 
write $\pi _{n}$ and $H_{n}$ as
\[
\pi _{n}={\mathop{\oplus}}_{j\,\in\,J_{n}}\Bigl(\ {\mathop{\oplus}}_{i\,\in\, I_{j,\,n}}\pi 
_{i,\,{j,\,n} } \Bigr)\quad\textrm{and}\quad 
H _{n}={\mathop{\oplus}}_{j\,\in\,J_{n}}\Bigl(\ {\mathop{\oplus}}_{i\,\in\, I_{j,\,n}}H
_{i,\,{j,\,n} } \Bigr)
\]
respectively. Since $H_{n}$ is finite dimensional, all the sets $J_{n}$ 
and $I_{j,\,n}$, 
$j\in J_{n}$, are finite. For every $j\in J_{n}$, $H_{i,\,j,\,n}=K_{j,\,n}$, 
which is of dimension $d_{j,\,n}$. We have, using again the notation of 
Section \ref{Section 4.1},
\[
\pi _{i,\,j,\,n}=U^{*}_{i,\,j,\,n}\,\pi_{j,\,n}^{}\,U_{i,\,j,\,n}^{},\quad j\in J_{n},\ 
i\in  I_{j,\,n}.
\]
Recall that $\pi _{j,\,n}$ is an irreducible representation of $G$ on $K_{j,\,n}$, that the representations $\pi _{j,\,n}$ for distinct $j$ are mutually inequivalent, and that
$U_{i,\,j,\,n}$ is for each $j\in J_{n}$ and $i\in I_{j,n}$ a unitary operator from $H_{i,\,j,\,n}$ into $K_{j,\,n}$. We  also decompose ${a_{n}}\in 
H_{n}$ as ${a_{n}}={\mathop{\oplus}}\limits_{j\,\in J_{n}}\,\bigl(\,\,{\mathop{\oplus}}\limits_{i\,\in 
I_{j,\,n}}a_{i,\,j,\,n}\, \bigr)$. We have
\[
||a_{n}||= \sum_{j\,\in J_{n}}\,\,\sum_{i\,\in I_{j,\,n}}||a_{i,\,j,\,n}||^{2}=1
\]
and
\begin{align}
 ||\pi _{n}(g){a_{n}}-{a_{n}}||^{2}&=\sum_{j\,\in J_{n}}\,\,\sum_{i\,\in I_{j,\,n}}||
 \pi _\ijna_{i,\,j,\,n}-a_{i,\,j,\,n}||^{2} \notag\\
 &=\sum_{j\,\in J_{n}}\,\,\sum_{i\,\in I_{j,\,n}}
 ||\pi _{j,\,n}(g)\,U_\ijna_{i,\,j,\,n}-U_\ijna_{i,\,j,\,n}||^{2}\label{Eq7}
 \end{align}
 for every $g\in G$.
In order to simplify the notation, we set $b_{i,\,j,\,n}=U_\ijna_{i,\,j,\,n}$ for 
every $j\in J_{n}$ and every $i\in I_{j,\,n}$. It follows from (\ref{Eq7}) 
that we 
have for every $j\in J_{n}$
\begin{equation}\label{Eq8}
 \sup_{g\,\in\,{Q_{n}}}\Bigl(\sum_{i\,\in I_{j,\,n}}||\pi 
_{j,\,n}(g)\,b_{i,\,j,\,n}-b_{i,\,j,\,n}||^{2}\Bigr)^{1/2}<{\varepsilon _{n}}
.
\end{equation}
Let us stress here that each vector $b_{i,\,j,\,n}$, $j\in J_{n}$, $i\in I_{j,\,n}$, belongs to $K_{j,\,n}$. It is at this point that we use the fact that ${Q_{n}}$
is a {Kazhdan}\ set for ampliations of finite dimensional irreducible 
representations for every $n\ge n_{0}$, with associated constant $\varepsilon '_{0}$: since $\pi 
_{j,\,n}$ is irreducible, and different from $\mathbbm{1}_{G}$ (as $\pi _{n}$ 
has no non-zero $G$-invariant vector), it follows from Fact \ref{Fact 
6.1.2} and (\ref{Eq8}) that
\[
{\varepsilon _{n}}>\varepsilon'_{0}\,  \Bigl({{\displaystyle}\sum_{i\,\in 
I_{j,\,n}}\!\!||\,b_{i,\,j,\,n}||^{2}}\Bigr)^{1/2}.
\]
Remembering that (with the notation of Section \ref{Section 4.3}) $\smash{{\displaystyle}\sum_{i\,\in 
I_{j,\,n}}\!\!||\,b_{i,\,j,\,n}||^{2}=||{\widetilde{{b}}}_{j,\,n}||^{2}=||{\widetilde{{a}}}_{j,\,n}||^{2}}$,
we thus get that
\[
||{\widetilde{{a}}}_{j,\,n}||<\dfrac{\varepsilon _{n}}{\varepsilon'
_{0}}
\]
 for every $j\in J_{n}$.
Using Corollary \ref{Corollary 5.7}, the fact that $\sum_{j\,\in 
J_{n}}||\,{\widetilde{{a}}}_{j,\,n}||^{2}=||{a_{n}}||^{2}=1$, and 
 assumption (ii) on the sequence $({\varepsilon _{n}})_{n\ge 1}$, we obtain that
\[
 m(\,|{\ensuremath{{\langle {\pi _{n}(\,\centerdot\,){a_{n}}},{a_{n}}\rangle}}}|^{2})\le\sum_{j\,\in J_{n}}
 ||\,{\widetilde{{a}}}_{j,\,n}||^{4}\le \max_{j\,\in 
J_{n}}||\,{\widetilde{{a}}}_{j,\,n}||^{2}\,.\,\sum
_{j\,\in J_{n}}||\,{\widetilde{{a}}}_{j,\,n}||^{2}
<\Bigl(\dfrac{\varepsilon _{n}}{\varepsilon' 
_{0}} \Bigr)^{2}<4^{-n}.
\]
 It follows that 
$m(\,|{\ensuremath{{\langle {\pi _{n}
(\,\centerdot\,){a_{n}}},{a_{n}}\rangle}}}|^{2})$ tends to zero as $n$ tends to infinity.
So $\pmb{\pi} $ is weakly mixing by 
Proposition \ref{Proposition 4.3}, and 
Proposition \ref{Proposition 6.2} is proved.
\end{proof} 
\subsection{Case 2}\label{Section 6.6} Suppose that ${Q_{n}}$ is 
never a {Kazhdan}\ set for ampliations of finite dimensional irreducible 
representations. Let $\varepsilon >0$ 
be an arbitrary positive number. We fix a sequence $({\varepsilon _{n}})_{n\ge 1}$ of 
positive numbers decreasing to zero so fast that the following properties 
hold true:
\begin{enumerate}
 \item [(i)] $\sum_{n\ge 1}{\varepsilon _{n}}<\varepsilon ^{2}/2$;
 \item[(ii)] for every ${n\ge 1}$, $\frac{1}{n+1}\sum_{j=n}^{2n}\varepsilon 
_{j}^{2}<{\varepsilon _{n}}^{2}$.
\end{enumerate}
Our additional assumption on the sets ${Q_{n}}$ implies  by Fact \ref{Fact 6.1.2} that there exists, for every 
${n\ge 1}$,  an irreducible representation $\pi _{n}$ of $G$ on a 
finite dimensional Hilbert space $H_{n}$, without non-zero $G$-invariant 
vectors, and a finite family $(a_{i,\,n})_{i\,\in I_{n}}$ of vectors 
of 
$H_{n}$ (with $I_{n}\subseteq {\ensuremath{\mathbb N}}$) such that 
\begin{equation}\label{eq++}
\Bigl(\sum_{i\,\in I_{n}}||\,a_{i,\,n}||^{2} 
\Bigr)^{1/2}\!\!\!\!=1\quad\textrm{and}\quad\sup_{g\,\in\,{Q_{n}}}
\Bigl(\sum_{i\,\in I_{n}}||\pi _{n}(g)\,a_{i,\,n}-a_{i,\,n} 
||^{2}\Bigr)^{1/2}\!\!\!<{\varepsilon _{n}}.
\end{equation}
 If we write 
\[
{\widetilde{{H}}}_{n}={\mathop{\oplus}}_{i\,\in I_{n}}H_{n},\quad {\widetilde{{a}}}_{n}={\mathop{\oplus}}_{i\,\in 
I_{n}}a_{i,\,n},\quad \textrm{and}\quad {\widetilde{{\pi }}}_{n}={\mathop{\oplus}}_{i\,\in 
I_{n}}\pi 
_{n},
\]
this means that $||{\widetilde{{a}}}_{n}||=1$ and $\sup_{g\,\in\,{Q_{n}}}||{\widetilde{{\pi 
}}}_{n}(g){\widetilde{{a}}}_{n}-
{\widetilde{{a}}}_{n}||<{\varepsilon _{n}}$.
\par\smallskip 
 Now we again have to consider separately two cases.
 \subsubsection{Case 2.a}\label{Section 6.6.1} There exists an infinite 
subset $D$ of ${\ensuremath{\mathbb N}}$ such that 
whenever $k$ and $l$ are two distinct elements of $D$, $\pi _{k}$ and $\pi 
 _{l}$ are not equivalent. Replacing the sequence $(\pi 
_{n})_{n\ge 1}$ by $(\pi 
_{n})_{n\in D}$, we can 
 suppose without loss of generality 
that for every distinct integers $m$ and $n$, with $m,n\ge 1$, $\pi _{m}$ and $\pi _{n}$ 
are not equivalent.
\par\smallskip 
Consider for every ${n\ge 1}$ the representation
\[
\rho _{n}={\widetilde{{\pi }}}_{n}{\mathop{\oplus}}\cdots{\mathop{\oplus}}{\widetilde{{\pi}}} _{2n}\quad 
\textrm{of}\ G\ \textrm{on}\quad
{\mathscr{H}} _{n}={\widetilde{{H}}}_{n}{\mathop{\oplus}}\cdots{\mathop{\oplus}}{\widetilde{{H}}}_{2n},
\]
and the vector ${a_{n}}=\dfrac{1}{\sqrt{n+1}}\bigl(\,{\widetilde{{a}}}_{n}{\mathop{\oplus}}\cdots{\mathop{\oplus}} 
{\widetilde{{a}}}_{2n} \bigr)$
of ${\mathscr{H}} _{n}$, which satisfies $||{a_{n}}||=1$. For every $g\in {Q_{n}}$ we 
have, since ${Q_{n}}$ is contained in $Q_{j}$ for every $j\ge n$,
\[
||\rho_{n}(g){a_{n}}-{a_{n}}||^{2}=\dfrac{1}{n+1}\sum_{j=n}^{2n}\ 
\bigl|\bigl|{\widetilde{{\pi 
}}}_{j}(g){\widetilde{{a}}}_{j}-{\widetilde{{a}}}_{j}\bigr|\bigr|^{2}<\dfrac{1}{n+1}\sum_{j=n}^{2n}
\varepsilon _{j}^{2}.
\]
By assumption
(ii) on the sequence $({\varepsilon _{n}})_{n\ge 1}$, we obtain that 
$\sup_{g\,\in\,{Q_{n}}}||\rho _{n}(g){a_{n}}-{a_{n}}||<{\varepsilon _{n}}$. Let 
now $\pmb{\rho}
={\otimes}_{n\ge 1}\rho _{n}$ be the infinite tensor product of the representations $\rho 
_{n}$ on the space 
$\pmb{\mathscr{H}} =\bigotimes_{n\ge 1}^{\mkern 1.5 mu\pmb{a}}{\mathscr{H}} _{n}$. Using similar tools 
to those employed in the proof of Proposition \ref{Proposition 6.2}, we 
are going to show:
\begin{proposition}\label{Proposition 6.3}
Under the assumptions above on the sequence $(\varepsilon _{n})_{n\ge 1}$, 
$\pmb{\rho} $ enjoys the 
following properties:
\begin{enumerate}
 \item [{(a)}] $\pmb{\rho}  $ is a strongly continuous 
representation of $G$ 
on 
 $\pmb{\mathscr{H}} =\bigotimes_{n\ge 1}^{\mkern 1.5 mu\pmb{a}}{\mathscr{H}} _{n}$;
 \item[{(b)}] $\pmb{\rho} $ has a $({Q},\varepsilon )$-invariant 
vector;
 \item[{(c)}] $\pmb{\rho} $ is weakly mixing. 
\end{enumerate}
\end{proposition}
\begin{proof}[Proof of Proposition \ref{Proposition 6.3}]
The proofs of properties (a) and (b) are exactly the same as the ones 
given in Proposition \ref{Proposition 6.2}, using the fact that 
$\sup_{g\,\in\,{Q_{n}}}||\rho _{n}(g){a_{n}}-{a_{n}}||<{\varepsilon _{n}}$ for every ${n\ge 1}$. In 
order to 
prove that $\pmb{\rho} $ is weakly mixing, we will show as in 
Proposition \ref{Proposition 6.2} that
$m(|{\ensuremath{{\langle {\rho
_{n}(\,\centerdot\,){a_{n}}},{a_{n}}\rangle}}}|^{2})$ tends to zero as $n$ tends to 
infinity.
Recall that for every ${n\ge 1}$, the representations $\pi _{n},\dots,\pi 
_{2n}$ are mutually non-equivalent, so that, with the notation of Section \ref{Section 4.1}, $J_{n}=\{n,\ldots, 2n\}$ and $I_{j,n}=I_{n}$ for every $j\in J_{n}$. By Corollary \ref{Corollary 5.7}, we 
have for every ${n\ge 1}$
\[
 m(|{\ensuremath{{\langle {\rho_{n}(\,\centerdot\,){a_{n}}},{a_{n}}\rangle}}}|^{2})\le
 \sum_{j=n}^{2n}\Bigl|\Bigl|\dfrac{1}{\sqrt{n+1}}{\widetilde{{a}}}_{j} \Bigr| 
\Bigr|^{4}\le \max_{n\le j\le 2n}
\Bigl(\,\dfrac{1}{n+1}||{\widetilde{{a}}}_{j}||^{2}\Bigr)
\sum_{j=n}^{2n}\Bigl|\Bigl|\dfrac{{\widetilde{{a}}}_{j}}{\sqrt{n+1}} \Bigr|\Bigr|^{2}
\le\dfrac{1}{n+1}\cdot 
\]
So $m(|{\ensuremath{{\langle {\rho_{n}(\,\centerdot\,){a_{n}}},{a_{n}}\rangle}}}|^{2})$ tends to zero as $n$ tends 
to infinity. By Proposition \ref{Proposition 4.3},
$\pmb{\rho }$ is weakly mixing, and Proposition \ref{Proposition 
6.3} is 
proved.
\end{proof}

The other case we have to consider is when there exists an integer $n_{0}\ge 1$ such that for every $n\ge n_{0}$, $\pi_{n}$ is equivalent to one of the representations $\pi _{1},\dots,\pi _{n_{0}}$. Indeed, if there is no such integer, we can construct a strictly increasing sequence $(n_{k})_{k\ge 1}$ of integers such that, for every $k\ge 1$, $\pi_{n_{k}}$ is not equivalent to one of the representations $\pi _{1},\dots,\pi _{n_{k-1}}$. The set $D=\{n_{k}\, ;\, k\ge 1\}$ then has the property that whenever $m$ and $n$ are two distinct elements of $D$, $\pi_{m}$ and $\pi_{n}$ are not equivalent, and we are back to the setting of Case $1$.

\subsubsection{Case 2.b}\label{Section 6.6.2} There exists an integer 
$n_{0}\ge 1$ such that for every 
 $n\ge n_{0}$, $\pi _{n}$ is equivalent to one of the representations
 $\pi _{1},\dots,\pi _{n_{0}}$. There exists hence in this case an index  $j_{0}\in\{1,\ldots, n_{0}\}$ such that $\pi_{n}$ is equivalent to $\pi_{j_{0}}$ for infinitely many integers $n$. Extracting a subsequence from $(\pi_{n})_{n\ge 1}$, we 
can suppose without loss of generality 
that $\pi _{n}$ is equivalent to $\pi _{1}$ for every $n\ge 1$, and so 
that $H_{n}=H_{1}$ for every ${n\ge 1}$. For each ${n\ge 1}$, let  
$U_{n}$ be a unitary operator  on $H_{1}$ such that $\pi _{n}
=U_{n}^{*}\,\pi _{1}^{}\,U_{n}^{}$. By (\ref{eq++}), there exists for every 
${n\ge 1}$ a finite family $(a_{i,\,n})_{i\,\in I_{n}}$ of vectors of 
$H_{1}$ 
such that 
\[
\Bigl(\sum_{i\,\in I_{n}}||\,a_{i,\,n}||^{2} 
\Bigr)^{1/2}=1\quad\textrm{and}\quad\sup_{g\,\in\,{Q_{n}}}
\Bigl(\sum_{i\,\in I_{n}}||\pi _{1}(g)U_{n}\,a_{i,\,n}-U_{n}a_{i,\,n} 
||^{2}\Bigr)^{1/2}\!\!<{\varepsilon _{n}}.
\]
For each ${n\ge 1}$, set $b_{n}={\mathop{\oplus}}\limits
_{i\,\in I_{n}} U_{n}a_{i,\,n}$, seen as a vector of the infinite direct 
sum
$H={\mathop{\oplus}}\limits_{j\ge 1}H_{1}$ by defining its $j^{th}$ coordinate to be zero when $j$ does not belong to $I_{n}$. Let also $\pi $ be the infinite ampliation 
$\pi ={\mathop{\oplus}}\limits_{j\ge 1}\pi _{1}$ of $\pi _{1}$ on $H$. Then we have, 
for every ${n\ge 1}$, 
\[
||b_{n}||=1\quad\textrm{and}\quad\sup_{g\,\in\,{Q_{n}}} ||\pi 
(g)b_{n}-b_{n}||<{\varepsilon _{n}}.
\]
\par\smallskip 
Let now $S$ be a finite subset of $G$. There exists an integer $n_{S}\ge 
1$ such that $S\subseteq {Q_{n}}$ for every $n\ge n_{S}$, and hence
\[
\sup_{g\,\in S}\,||\pi (g)b_{n}-b_{n}||<{\varepsilon _{n}}\quad\textrm{for every $n\ge 
n_{S}$.}
\]
It follows that $\pi $ has almost-invariant vectors for finite sets in the 
sense of \cite[Sec.~1.5]{Pet}: for every $\delta >0$ and every finite 
subset $S$ of $G$, $\pi $ has an $(S,\delta )$-invariant vector. This 
implies that $\pi _{1}$ itself has almost-invariant vectors for finite 
sets (see \cite[Sec.~1.5]{Pet} or \cite{Ke}). 
Since $\pi 
_{1}$ is a finite dimensional representation, it follows that $\pi _{1}$ 
has almost-invariant vectors for any subset $C$ of $G$, and in particular 
for compact subsets of $G$. Indeed, let $C$ be any subset of $G$, and 
suppose that there exists a $\delta >0$ such that $\pi _{1}$ has no 
$(C,\delta )$-invariant vector. This means that for every $x$ belonging 
to the unit sphere 
$S_{H_{1} } $ of $H_{1}$, there exists an element $g_{x}$ of $C$ such that 
$||\pi _{1}(g_{x})x-x||>\delta $. The sets 
$V_{x}=\{y\in S_{H_{1}}\,;\,||\pi _{1}(g_{x})y-y||>\delta \}$ form an open 
covering of $S_{H_{1}}$. Since $S_{H_{1}}$ is compact, there exists a 
finite 
subset $\{x_{1},\dots,x_{r}\}$ of $S_{H_{1}}$ such that for every $x\in
S_{H_{1}}$, $\sup_{1\le i\le r}||\pi _{1}(g_{x_{i}})x-x||>\delta $. If we 
set $S=\{g_{x_{1}},\dots,g_{x_{r}}\}$, this means that $\pi 
_{1}$ has no $(S,\delta )$-invariant vector, which contradicts what we 
just proved above.
\par\smallskip 
So $\pi _{1}$ has almost-invariant vectors in the sense of
\cite[Def.~1.1.1]{BdHV}, and the unit representation $\mathbbm{1}_{G}$ 
is weakly contained in $\pi _{1}$ \cite[Cor.~F.1.5]{BdHV}. As $\pi 
_{1}$ is finite dimensional, it follows from \cite[Cor.~F.2.9]{BdHV} that 
$\mathbbm{1}_{G}$ is contained in $\pi _{1}$, i.\,e.\ that $\pi _{1}$ has a 
non-zero $G$-invariant vector. This contradicts our initial assumption on 
$\pi _{1}$, and shows that the hypothesis of Case 2.b cannot be fulfilled.
\par\smallskip
So under the assumption of Case 2, there exists for every $\varepsilon 
>0$ a representation of $G$ with a $(Q,\varepsilon )$-invariant vector but 
no finite dimensional subrepresentation (this is Proposition \ref{Proposition 
6.3}). This contradicts again assumption (\ref{Pro1}) of Theorem 
\ref{Theorem 0}, and concludes the proof in this second case.

 \section{Some consequences of Theorem \ref{Theorem 0}}\label{Section 6}
 We begin this section by proving the two characterizations of {Kazhdan}\ sets obtained as consequences of Theorem \ref{Theorem 0}.
 
 \subsection{Proofs 
of Corollary \ref{Corollary 1} and Theorem \ref{Theorem 2}} \label{Section 
7.2}
Let us first prove Corollary 
\ref{Corollary 1}.
\begin{proof}[Proof of Corollary \ref{Corollary 1}]
Let ${Q}_{0}$ 
be a subset of $G$ which has non-empty interior and which generates $G$.
Denote for each ${n\ge 1}$ by ${Q}_{0}^{\,\pm n}$ the set $\{g_{1}^{\,\pm 1}
\dots g_{n}^{\,\pm 1}\,;\, g_{1},\dots,g_{n}\in{Q}_{0}\}$. Then 
$G=\bigcup_{n\ge 1}{Q}_{0}^{\,\pm n}$. Let $g_{0}$ be an element of the 
interior of 
 ${Q}_{0}$. Then $g_{0}^{-1}{Q}_{0}$ is a neighborhood of $e$. There 
exists $n_{0}\ge 1$ such that $g_{0}^{-1}$ belongs to 
${Q}_{0}^{\,\pm n_{0}}$, 
and thus ${Q}_{0}^{\,\pm(n_{0}+1)}$ is a neighborhood of $e$. If we set
$W_{n}={Q}_{0}^{\,\pm(n_{0}+n)}$ for ${n\ge 1}$, the sequence of sets 
$(W_{n})_{n\ge 1}$ is increasing, $W_{1}$ is a neighborhood of $e$, and $(W_{n})_{n\ge 1}$
satisfies the assumptions of Theorem \ref{Theorem 0}. So if ${Q}$ is a 
subset of $G$ for which assumption (\ref{Pro1}) of Theorem \ref{Theorem 0} 
holds true, there exists 
$n\ge 1$ such that ${Q}_{0}^{\,\pm(n+n_{0})}\cup Q$ is a {Kazhdan}\ set in $G$. 
Let $\varepsilon >0$ be a {Kazhdan}\ constant for this set. Then $\varepsilon 
/(n+n_{0})$ is a {Kazhdan}\ constant for ${Q}_{0}\cup{Q}$,
and ${Q}_{0}\cup {Q}$ is a {Kazhdan}\ set in 
$G$. If $G$ is locally compact, the same proof 
holds true without the assumption that ${Q}_{0}$ has non-empty interior.
\end{proof}

\begin{proof}[Proof of Theorem \ref{Theorem 2}]
Let us first show that (a) implies (b). If (a) holds true, then, according to the definition of a {Kazhdan}\ set, there exists $\varepsilon >0$ such that the following property holds true:
any unitary representation $\pi $ of $G$ on a complex Hilbert space $H$ 
with a 
$({Q},\varepsilon )$-invariant vector has a non-zero $G$-invariant vector. Without any loss of generality one can assume that $\varepsilon < 2$. Let 
$\delta = \sqrt{1-{\varepsilon}/{2}}$ 
and consider a unitary representation $\pi$ of $G$ on a Hilbert space $H$ for 
which there is a vector $x\in H$ of norm one such that 
$\inf_{g\in{Q}}\left|{\ensuremath{{\langle {\pi(g)x},{x}\rangle}}}\right|  > \delta.$
Then the representation $\pi {\otimes} {\overline{{\pi }}}$ of 
$G$ on $H{\otimes} {\overline{{H}}}$ verifies
\[
{\ensuremath{{\langle {\pi {\otimes}{\overline{{\pi }}}(g)x {\otimes}{\overline{{x }}}},{x 
{\otimes}{\overline{{x }}}}\rangle}}}=|{\ensuremath{{\langle {\pi (g)x },{x }\rangle}}}|^{2} > \delta^2
\]
for every $g\in{Q}$.
Therefore we have 
\[
2\Re e{\ensuremath{{\langle {\pi {\otimes}{\overline{{\pi }}}(g)x {\otimes}{\overline{{x }}}},{x 
{\otimes}{\overline{{x }}}}\rangle}}}=2|{\ensuremath{{\langle {\pi (g)x },{x }\rangle}}}|^{2} > 2 - \varepsilon
\]
for every $g\in{Q}$. Hence
$
\left\|\pi {\otimes}{\overline{{\pi }}}(g)x {\otimes}{\overline{{x }}} - x {\otimes}{\overline{{x }}}\right\| < \varepsilon
$
for every $g\in{Q}$. Using (a) we obtain that the representation $\pi {\otimes} {\overline{{\pi }}}$ of 
$G$ on $H{\otimes} {\overline{{H}}}$ has a non-zero \mbox{$G$-invariant} vector. It follows from Proposition \ref{Proposition 4.1}  that $\pi$ has a finite dimensional subrepresentation. Thus $(b)$ is true.
Suppose now that (b) holds true for some $\delta \in (0,1)$. Let 
$\varepsilon = \sqrt{2(1-\delta)}$ 
and consider a unitary representation $\pi $ of $G$ on a complex Hilbert space $H$ 
with a $({Q},\varepsilon )$-invariant vector $x$ of norm one. Noticing again that 
$\varepsilon^2 > \|\pi(g)x-x\|^2 = 2-2\Re e{\ensuremath{{\langle {\pi (g)x },{x }\rangle}}}$
we get that
$ \left|{\ensuremath{{\langle {\pi (g)x },{x }\rangle}}}\right| \ge \Re e{\ensuremath{{\langle {\pi (g)x },{x }\rangle}}} > \delta$
for every $g\in{Q}$. It follows from (b)  that $\pi$ has a finite dimensional subrepresentation. Thus (b) implies (c).
The fact that (c) implies (a) is a consequence of Theorem \ref{Theorem 0}.
\end{proof}

\subsection{Property (T) in $\sigma $-compact 
locally compact groups}\label{Section7.1}
As a  consequence of Theorem \ref{Theorem 0}, we retrieve a 
characterization of Property (T) due to Bekka and Valette \cite{BV}, 
\cite[Th.~2.12.9]{BdHV}, valid for $\sigma $-compact locally compact  
groups. Recall that a unitary representation $\pi $ of $G$ on a Hilbert 
space $H$ has \emph{almost-invariant vectors} if it has $({Q},\varepsilon 
)$-invariant vectors for every compact subset ${Q}$ of $H$ and every 
$\varepsilon >0$. Then Property (T) can be characterized by the fact that 
every unitary representation of $G$ with almost-invariant vectors has a 
non-zero $G$-invariant vector \cite[Prop.~1.2.1]{BdHV}. A result due to Bekka and Valette \cite{BV} (see also 
\cite[Th. 2.12.9]{BdHV}) states the following:
\begin{theorem}[\!\!\cite{BV}]\label{Theorem 7.1}
Let $G$ be a $\sigma $-compact locally compact group. 
Then $G$ has 
Property (T) if and only if every unitary representation of $G$ with 
almost-invariant vectors has a non-trivial finite dimensional 
subrepresentation.
\end{theorem}
The proof of \cite{BV} relies on the equivalence between Property (T) and 
Property (FH) for such groups \cite[Th.~2.12.4]{BdHV}: every affine 
isometric action of $G$ on a real Hilbert space has a fixed point. As a 
direct consequence of Theorem \ref{Theorem 0}, we will derive a new proof of 
Theorem \ref{Theorem 7.1} which does not involve property (FH).
\par\smallskip
If ${Q}$ is a subset of a topological group $G$, and if $\pi $ is a unitary representation of $G$ on a Hilbert space $H$, we say that $\pi $ has \emph{${Q}$-almost-invariant vectors} if it has $({Q},\varepsilon )$-invariant vectors for every $\varepsilon >0$. The same argument as in  \cite[Prop.~1.2.1]{BdHV} shows that ${Q}$ is a {Kazhdan}\ set in $G$ if and only if every representation of $G$ with ${Q}$-almost-invariant vectors has a non-zero $G$-invariant vector. As a direct corollary of Theorem \ref{Theorem 2}, we obtain the following characterization of {Kazhdan}\ sets which generate the group:
\begin{corollary}\label{CorollaryA}
Let ${Q}$ be a subset of a locally compact group $G$ which generates $G$. Then ${Q}$ is a {Kazhdan}\ set in $G$ if and only if every representation $\pi $ of $G$ with ${Q}$-almost-invariant vectors has a non-trivial finite dimensional subrepresentation.
\end{corollary}
\begin{proof}[Proof of Corollary \ref{CorollaryA}]
 The only thing to prove is that if every representation $\pi $ of $G$ with ${Q}$-almost-invariant vectors has a non-trivial finite dimensional representation, ${Q}$ is a {Kazhdan}\ set. For this it suffices to show the existence of an $\varepsilon >0$ such that assumption (\ref{Pro1}) of Theorem \ref{Theorem 0} holds true. 
 The argument is again exactly the same as the one given in \cite[Prop.~1.2.1]{BdHV}: suppose that there is no such $\varepsilon $, and let, for every $\varepsilon >0$, $\pi _{\varepsilon }$ be a representation of $G$ with a $({Q},\varepsilon )$-invariant vector but no finite dimensional subrepresentation. Then $\pi =\bigoplus_{\varepsilon >0}\pi _{\varepsilon }$ has ${Q}$-almost-invariant vectors but no finite dimensional subrepresentation (this follows immediately from \cite[Prop.~A.1.8]{BdHV}), contradicting our initial assumption. 
 \end{proof}
 
\begin{proof}[Proof of Theorem \ref{Theorem 7.1}]
It is clear that Property (T) implies that every representation of $G$ with almost-invariant vectors 
has a non-trivial finite dimensional subrepresentation.
 Conversely, suppose that every representation of $G$ with almost-invariant vectors 
has a non-trivial finite dimensional subrepresentation. Using the same argument as in the proof of Corollary \ref{CorollaryA}, we see that there exists 
a compact subset ${Q}$ of $G$ such that assumption (\ref{Pro1}) of 
Theorem \ref{Theorem 0} holds true. 
Choosing for $(W_{n})_{n\ge 1}$ an increasing sequence of compact subsets of $G$ 
such that $\bigcup_\gnW_{n}=G$, Theorem \ref{Theorem 
0} implies that there exists an ${n\ge 1}$ such that 
$W_{n}\cup{Q}$ is a {Kazhdan}\ set in $G$. Since $W_{n}\cup{Q}$ is compact, 
$G$ has Property (T).
\end{proof}

\subsection{Equidistribution assumptions: proofs of Theorems 
\ref{Theorem 1}, \ref{Theorem 
3} and \ref{Theorem 4}} Let $G$ be a second countable locally compact 
group, and let $\pi $ be a unitary 
representation of $G$ on a separable Hilbert space $H$. Such a 
representation 
can be decomposed as a direct integral of irreducible unitary 
representations over a Borel space (see for instance 
\cite[Sec.~F.5]{BdHV} or \cite{Fo}). More precisely, there exists a finite positive 
measure $\mu $ on a standard Borel space $Z$, a measurable field
$\smash{\xymatrix{z\ar@{|->}[r]&H_{z}}}$ of Hilbert spaces over 
$Z$, and a measurable field of irreducible representations 
$\smash{\xymatrix{z\ar@{|->}[r]&\pi_{z}}}$, where each $\pi _{z}$ is a representation of $G$ on $H_{z}$,
such that $\pi $ is unitarily equivalent to the direct integral
$\pi _{\mu }={\displaystyle}\int_{Z}^{\mathop{\oplus}}\pi_{z}\,d\mu (z)$ on ${\mathscr{H}}={\displaystyle}\int_{Z}^\oplH_{z}\,d\mu 
(z)$. The Hilbert space
${\mathscr{H}}$ is the set of equivalence classes of square integrable vector 
fields $\smash{\xymatrix{z\ar@{|->}[r]&x_z}}$, with $x_z\in H_{z}$, 
with respect to the measure $\mu $;
$\pi _{\mu }$ is the 
representation of $G$ on ${\mathscr{H}}$ defined by 
$
\pi _{\mu }(g)x=[\smash{\xymatrix{z\ar@{|->}[r]&\pi_{z}(g)x_z}}]
$
for every $g\in G$ and $x\in{\mathscr{H}}$. If $\mu _{c}$ and $\mu _{d}$ denote respectively the 
continuous and discrete parts of the measure $\mu $, ${\mathscr{H}}$ decomposes as 
an 
orthogonal direct sum ${\mathscr{H}}={\mathscr{H}}_{c}{\mathop{\oplus}}{\mathscr{H}}_{d}$, and $\pi _{\mu } $ as a direct 
sum $\pi _{\mu }  =\pi _{\mu ,c} {\mathop{\oplus}}\pi _{\mu,d } $, where
\begin{align*}
 &{\mathscr{H}}_{c}=\int_{Z}^\oplH_{z}\,d\mu _{c}(z),\quad\pi _{\mu ,c} =\int_{Z}^{\mathop{\oplus}}\pi_{z}\,d\mu_{c}(z)\\
&{\mathscr{H}}_{d}=\int_{Z}^\oplH_{z}\,d\mu _{d}(z),\quad\pi _{\mu ,d} =\int_{Z}^{\mathop{\oplus}}\pi_{z}\,d\mu_{d}(z).
\end{align*}
If we denote by $Z_{d}$ the support of $\mu _{d}$, which is a countable 
subset of $Z$, 
\[
{\mathscr{H}}_{d}={\mathop{\oplus}}_{z\,\in\, Z_{d}}H_{z}\quad\textrm{and}\quad
\pi _{\mu ,d} ={\mathop{\oplus}}_{z\,\in\, Z_{d}}\pi_{z}.
\]
It follows that if $\mu $ has a non-trivial discrete part, one of the 
irreducible representations $\pi_{z}$ is a subrepresentation of $\pi_{\mu } $. 
This observation lies at the core of the proof of Theorem \ref{Theorem 3}.
\begin{proof}[Proof of Theorem \ref{Theorem 3}] Our aim is to show that, 
under the hypothesis of Theorem \ref{Theorem 3}, assumption (\ref{Pro1}) 
of Theorem \ref{Theorem 0} is satisfied. Fix $\varepsilon \in(0,1)$ and 
let $\pi $ be a representation of $G$ on 
a Hilbert space $H$. Since $G$ is second countable, we can suppose that 
$H$ is separable. Suppose that $\pi $ admits a $({Q},\varepsilon 
)$-invariant vector $x \in H$. By the result recalled above, we can 
suppose 
without loss of generality that 
\[
\pi =\int_{Z}^{\mathop{\oplus}}\pi_{z}\,d\mu (z),\quad x 
=[\xymatrix{\!z\ar@{|->}[r]&x_{z}\!}],\quad \textrm{and}\quad
H=\int_{Z}^\oplH_{z}\,d\mu (z),
\]
where $\mu $ is a finite positive Borel measure on a standard Borel space 
$Z$,
$\smash{\xymatrix{z\ar@{|->}[r]& H_{z}}}$ is a measurable field of Hilbert 
spaces over $Z$, $\smash{\xymatrix{z\ar@{|->}[r]&\pi_{z}}}$ is an 
associated measurable field of irreducible representations, and 
$\smash{\xymatrix{z\ar@{|->}[r]&x_{z}}}$ is a square integrable vector 
field with respect to $\mu $. We have for 
every $k\ge 1$
\begin{align}
 ||\pi (g_{k})x -x ||^{2}&=2\,(1-\Re e\,{\ensuremath{{\langle {\pi (g_{k})
 x },{x }\rangle}}})\notag=2\,\Bigl(1-\Re e\int_{Z}{\ensuremath{{\langle {\pi_{z}(g _{k})
 x_{z} },{x_{z}}\rangle}}}\,d\mu (z)\Bigr)<\varepsilon, \notag
 \intertext{so that}
 &\Re e\int_{Z}\dfrac{1}{N}\sum_{k=1}^{N}{\ensuremath{{\langle {\pi_{z}(g_{k})x_{z}},{x_{z}}\rangle}}}\,d\mu 
(z)> 1-\dfrac{\varepsilon }{2}\quad \textrm{for every}\, N\ge 1. \label{Eq10}
\end{align}
Now, by assumption (\ref{Pro3}) of Theorem \ref{Theorem 3}, 
\[
\xymatrix@C=50pt{\dfrac{1}{N}{\displaystyle}\sum_{k=1}^{N}{\ensuremath{{\langle {\pi (g_{k})x_{z}},{x_{z} }\rangle}}}\ar[r]_-{\usebox{\abbaa}}&0}
\quad\textrm{for}\; \mu\textrm{-}\textrm{almost every}\; z\in Z\setminus Z_{0},
\]
 where $Z_{0}$ 
is a subset of $Z$ which is countable. 
We infer from this that $\mu (Z_{0})>0$, and thus $\mu $ has a non-trivial 
discrete part. Hence one of the representations $\pi_{z}$, $z\in Z_{0}$, 
is 
a subrepresentation of $\pi $. Since all irreducible representations of 
$G$ are supposed to be finite dimensional, $\pi $ has a finite dimensional 
subrepresentation. So assumption (\ref{Pro1}) of Theorem \ref{Theorem 0} 
is satisfied. As ${Q}$ generates $G$, it now follows from 
Theorem \ref{Theorem 2} that ${Q}$ is a {Kazhdan}\ set in $G$.
\end{proof}
\begin{proof}[Proof of Theorem \ref{Theorem 1}]
The proof of Theorem \ref{Theorem 1} is exactly the same as that of 
Theorem \ref{Theorem 3}, using the fact that if $G$ is a locally compact 
abelian group (not necessarily second countable), any unitary 
representation of $G$ is equivalent to a direct integral of irreducible 
representations (see for instance \cite[Th.~7.36]{Fo}).
\end{proof}
\noindent Theorem \ref{Theorem 4} is 
a direct consequence of Theorem \ref{Theorem 1} and 
Corollary \ref{Corollary 100}.
We finish this section with the following remark.

\begin{remark}\label{Remarque Fin}
 When the equidistribution assumption (\ref{Pro3}) of Theorem \ref{Theorem 3} is satisfied, but ${Q}$ does not generate $G$, the same proof shows that ${Q}$ becomes a {Kazhdan}\ set when one adds to it a suitable ``small'' perturbation. More precisely:
\end{remark}
\begin{theorem}\label{Theorem +}
 Let $G$ be a second countable locally compact Moore group, and let $(W_{n})_{n\ge 1}$ be an increasing sequence of subsets of $G$ such that $\bigcup_\gnW_{n}=G$. Let $(g_{k})_{k\ge 1}$ be a  
sequence of elements of $G$ which satisfies the following assumption:
\begin{equation}\label{Equation3}
\begin{minipage}{120 mm}
for all irreducible representations 
${\pi }$ of $G$ on a Hilbert space $H$, except at most countably 
many,
\[
\xymatrix@C=50pt{\dfrac{1}{N}{\displaystyle}\sum_{k=1}^{N}{\ensuremath{{\langle {\pi (g_{k})x},{y }\rangle}}}\ar[r]_-{\usebox{\abbaa}}&0}\quad
\textrm{for every}\; x,y \in H. 
\]
\end{minipage}
\end{equation}
Set $Q=\{g_{k}\; ; \; k\ge 1\}$.
There exists $n\ge 1$ such that $W_{n}\cup{Q}$ is a {Kazhdan}\ set in $G$. 
\end{theorem}

\section{Examples and applications}\label{Section 7}
We present in this section some more examples of {Kazhdan}\ sets in different kinds of groups, some statements being obtained as consequences of Theorems \ref{Theorem 0}, \ref{Theorem 2} or \ref{Corollary 100}. We do not try to be exhaustive, and our aim here is rather to highlight some interesting phenomena which appear when looking for {Kazhdan}\ sets, as well as the connections of these phenomena with some remarkable properties of the group. We begin with the simplest case, that of locally compact abelian (LCA) groups.
\subsection{{Kazhdan}\ sets in locally compact abelian groups}\label{Subsection 8.1}
Let $G$ be a second countable LCA group, the dual group of which we denote by $\Gamma$.  If $\sigma $ is a finite Borel measure on $\Gamma $, recall that its Fourier-Stieljes transform is defined by 
\[
\widehat{\sigma }(g)=\int_{\Gamma }\gamma (g)\,d\sigma (\gamma )\quad\textrm{for every}\ g\in G.
\] 
The spectral theorem for unitary representations of $G$ acting on a (separable) Hilbert space easily yields the following characterization of {Kazhdan}\ subsets of $G$ in terms of Fourier coefficients of probability measures on $\Gamma $.
\begin{proposition}\label{Proposition A}
 Let $Q$ be a subset of  a second countable LCA group $G$. The following assertions are equivalent:
\begin{enumerate}
 \item [(1)] ${Q}$ is a {Kazhdan}\ set in $G$;
\item[(2)] there exists $\varepsilon >0$ such that any probability measure $\sigma $ on $\Gamma $ with $\sup_{g\in{Q}}|\widehat{\sigma }(g)-1|<\varepsilon$
satisfies $\sigma (\{{1}\})>0$, where ${1}$ denotes the trivial character on $G$.
\end{enumerate}
\end{proposition}
Proposition \ref{Proposition A} is extremely useful to obtain various examples of {Kazhdan}\ and non-{Kazhdan}\ sets in classic LCA groups such as ${\ensuremath{\mathbb Z}}^{d}$ or ${\ensuremath{\mathbb R}}^{d}$. We give here two such examples.
\begin{example}\label{Example B}
 The set ${Q}=\{2^{k}+k\,;\,k\ge 0\}$ is a {Kazhdan}\ set in ${\ensuremath{\mathbb Z}}$, while the set ${Q}'=\{2^{k}\,;\,k\ge 0\}$ is not a {Kazhdan}\ set in ${\ensuremath{\mathbb Z}}$.
\end{example}
\begin{proof}{}
 The sequence $(n_{k})_{k\ge 0}$ defined by $n_{k}=2^{k}+k$ for every $k\ge 0$ satisfies the relation $2n_{k}=n_{k+1}+k-1$ for every 
 $k\ge 0$. If $\sigma $ is a probability measure on ${\ensuremath{\mathbb T}}$ such that $\sup_{k\ge 0}|\widehat{\sigma }(n_{k})-1|<1/3$, it follows from the fact that 
\[
|\widehat{\sigma }(k-1)-1|\le 2|{\widehat{{\sigma }}}(n_{k})-1|+|{\widehat{{\sigma }}}(n_{k+1})-1| \quad\textrm{for all}\ k\ge 1
\]
that $\sup_{k\ge 0}|\widehat{\sigma }(k)-1|<1$. Hence $\sigma (\{1\})>0$, and
${Q}=\{n_{k}\,;\,k\ge 0\}$ is a {Kazhdan}\ set in ${\ensuremath{\mathbb Z}}$. The fact that ${Q}'$ is not a {Kazhdan}\ set in ${\ensuremath{\mathbb Z}}$ relies on the observation that $2^{k}$ divides $2^{k+1}$ for every $k\ge 0$. Using the same construction as the one of \cite[Prop.~3.9]{EG}, one can construct for every $\varepsilon >0$ a continuous probability measure $\sigma $ on ${\ensuremath{\mathbb T}}$, given as an infinite convolution of two-points Dirac measures, such that 
\[
\sup_{k\ge 0}|\widehat{\sigma }(2^{k})-1|<\varepsilon,
\]
and this proves that ${Q}'$ is not a {Kazhdan}\ set in ${\ensuremath{\mathbb Z}}$.
\end{proof}

\begin{remark}\label{rem+}
The set $\{2^{k}+k\,;\, k\ge 0\}$ being lacunary, it follows from a result proved independently 
by Pollington \cite{Pol} and De Mathan \cite{DM} that there exists a subset $A$ of $[0,1]$ of Hausdorff measure $1$ such that for every $\theta $ in $ A$, the set
of all fractional parts of $(2^{k}+k)\theta $, $k\ge 0$, is not dense in $[0,1]$. One of these numbers $\theta $ is irrational, and $((2^{k}+k)\theta )_{k\ge 0}$ is of course not uniformly distributed modulo $1$. 
\end{remark}

As a consequence of
Theorem \ref{Theorem 2} and of the spectral theorem for unitary representations of the group, we obtain the following characterization of {Kazhdan}\ sets which generate the group
in any second countable LCA group. 

\begin{theorem}\label{nimporte}
 Let $G$ be a second countable LCA group, and let ${Q}$ a subset of $G$ which generates $G$. The following assertions are equivalent:
 \begin{enumerate}
 \item [(1)] ${Q}$ is a {Kazhdan}\ set in $G$;
\item[(2)] there exists $\delta\in (0,1)$ such that any probability measure $\sigma $ on $\Gamma $ with $\inf_{g\in{Q}}|\widehat{\sigma }(g)|>\delta$
has a discrete part;
\item[(3)] there exists $\varepsilon >0$ such that any probability measure $\sigma $ on $\Gamma $ with $\sup_{g\in{Q}}|\widehat{\sigma }(g)-1|<\varepsilon$
has a discrete part.
\end{enumerate}
\end{theorem}

It is worth stating explicitly the corresponding 
characterization of non-{Kazhdan}\ subsets of $G$: ${Q}$ is not a {Kazhdan}\ subset of $G$ if and only if
there exists for every $\delta\in (0,1)$  a continuous probability measure $\sigma $ on $\Gamma $ such that $\inf_{g\in{Q}}|\widehat{\sigma }(g)|>\delta$, or, equivalently, if and only
if there exists for every $\varepsilon >0$ a continuous probability measure $\sigma $ on $\Gamma $ such that $\sup_{g\in{Q}}|\widehat{\sigma }(g)-1|<\varepsilon$. If the group $G$ is discrete, Theorem \ref{Corollary 100} combined with the spectral theorem for unitary representations again 
implies that a subset ${Q}$ of $G$ is a {Kazhdan}\ set in $G$ if and only if ${Q}$ generates $G$ and either assertion $(2)$ or $(3)$ of Theorem \ref{nimporte} holds true.

\par\smallskip
Theorem \ref{nimporte} becomes particularly meaningful in the case of the group ${\ensuremath{\mathbb Z}}$, as it yields a characterization of {Kazhdan}\ subsets of ${\ensuremath{\mathbb Z}}$ involving some classic sets in harmonic analysis, introduced by Kaufman in \cite{Ka1}. They are called \emph{w-sets} by Kaufman \cite{Ka2}, and \emph{Kaufman sets} (\textbf{Ka} sets) by other authors, such as Hartman \cite{Ha1}, \cite{Ha2}.

\begin{definition}
Let ${Q}$ be a subset of ${\ensuremath{\mathbb Z}}$, and let $\delta\in (0,1)$.

$\bullet$ We say that ${Q}$ belongs to the class \textbf{Ka} if there exists a finite complex-valued  continuous Borel measure $\mu$ on ${\ensuremath{\mathbb T}}$ such that $\inf_{n\in{Q}}|\hat{\mu }(n)|>0$, and to the class  \textbf{$\delta$-Ka} if there exists a finite complex-valued  continuous Borel measure $\mu$ on ${\ensuremath{\mathbb T}}$ with $\mu({\ensuremath{\mathbb T}})=1$ such that $\inf_{n\in{Q}}|\hat{\mu }(n)|>\delta$.

$\bullet$ We say that ${Q}$ belongs to the class \textbf{Ka}$^{+}$ if there exists a   continuous probability measure $\sigma$ on ${\ensuremath{\mathbb T}}$ such that $\inf_{n\in{Q}}|\hat{\sigma}(n)|>0$, and to the class  \textbf{$\delta$-Ka$^{+}$} if there exists a  continuous probability measure $\sigma$ on ${\ensuremath{\mathbb T}}$  such that $\inf_{n\in{Q}}|\hat{\sigma }(n)|>\delta$.
\end{definition}

Our final characterization of {Kazhdan}\ subsets of ${\ensuremath{\mathbb Z}}$ is given by Theorem \ref{nimportebis} below:

\begin{theorem}\label{nimportebis}
 Let ${Q}$ a subset of ${\ensuremath{\mathbb Z}}$. The following assertions are equivalent:
 \begin{enumerate}
  \item [(1)] ${Q}$ is a {Kazhdan}\ set in ${\ensuremath{\mathbb Z}}$;
  \item [(2)] ${Q}$ generates ${\ensuremath{\mathbb Z}}$, and there exists a $\delta\in(0,1)$ such that ${Q}$ does not belong to \textbf{$\delta$-Ka$^{+}$}.
 \end{enumerate}
 
 In particular, if ${Q}$ generates ${\ensuremath{\mathbb Z}}$, ${Q}$ is not a {Kazhdan}\ subset of ${\ensuremath{\mathbb Z}}$ if and only if ${Q}$ belongs to  \textbf{$\delta$-Ka$^{+}$} for every $\delta\in (0,1)$.
\end{theorem}

It is interesting to remark \cite{Ha2} that a set ${Q}$ belongs to \textbf{Ka} if and only if it belongs to \textbf{$\delta$-Ka} for every $\delta\in (0,1)$. There is no similar statement for the class \textbf{Ka}$^{+}$: any sufficiently lacunary subset of ${\ensuremath{\mathbb Z}}$, such as ${Q}=\{3^k+k \; ;\; k\ge 1\}$, is easily seen to belong to \textbf{Ka}$^{+}$;
but the same reasoning as in Example \ref{Example B} above shows that this set ${Q}$ is a {Kazhdan}\ subset of ${\ensuremath{\mathbb Z}}$. Thus there exists by Theorem \ref{nimportebis} a $\delta\in (0,1)$ such that ${Q}$
does not belong to \textbf{$\delta$-Ka$^{+}$}.
\par\smallskip
Theorem \ref{Theorem 4} is another powerful  tool for exhibiting {Kazhdan}\ sets in ${\ensuremath{\mathbb Z}}^{d}$ or ${\ensuremath{\mathbb R}}^{d}$. Here are typical examples of {Kazhdan}\ sets in ${\ensuremath{\mathbb Z}}$ or ${\ensuremath{\mathbb R}}$ obtained in this way.
\begin{example}\label{Example C} 
 If $p$ is a non-constant polynomial with integer coefficients such that $p({\ensuremath{\mathbb Z}})$ is included in $a{\ensuremath{\mathbb Z}}$ for no integer $a\in{\ensuremath{\mathbb Z}}$, then ${Q}=\{p(k)\,;\,k\ge 0\}$ is a {Kazhdan}\ set in ${\ensuremath{\mathbb Z}}$.
\end{example}
 \begin{proof}
The sequence $(\lambda ^{p(k)})_{k\ge 0}$ is uniformly distributed in ${\ensuremath{\mathbb T}}$ for every  $\lambda =e^{2i\pi \theta }$ with $\theta \in{\ensuremath{\mathbb R}}\setminus{\ensuremath{\mathbb Q}}$ (see for instance \cite{KuiNied}). By Theorem \ref{Theorem 3}, $\{p(k)\,;\,k\ge 0\}$ is a Kazhdan set in ${\ensuremath{\mathbb Z}}$ as soon as it generates ${\ensuremath{\mathbb Z}}$, which is the case by our assumption that $p({\ensuremath{\mathbb Z}})$ is included in $a{\ensuremath{\mathbb Z}}$ for no integer $a\in{\ensuremath{\mathbb Z}}$.
\end{proof}

\begin{example}\label{Example 8.5}
 Let $p$ be a non-constant real polynomial, and let ${Q}=\{p(k)\,;\,k\ge 0\}$. Then $(-\delta ,\delta )\cup {Q}$ is a {Kazhdan}\ subset of ${\ensuremath{\mathbb R}}$ for any $\delta >0$.
\end{example}
\begin{proof}
 Write $p$ as $p(x)=\sum_{j=0}^{d}a_{j}x^{j}$, $d\ge 1$, and let $r\in\{1,\dots,d\}$ be such that $a_{r}\neq 0$. It is well-known that the sequence $(e^{2i\pi tp(k)})_{k\in{\ensuremath{\mathbb Z}}}$ is uniformly distributed in ${\ensuremath{\mathbb T}}$ as soon as  $ta_{r}$ is irrational. This condition excludes only countably many values of $t$. Set now $W_{n}=(-n,n)$ for every integer ${n\ge 1}$. Applying Theorem \ref{Theorem +}, we obtain that there exists ${n\ge 1}$ such that $(-n,n)\cup{Q}$ is a {Kazhdan}\ set in ${\ensuremath{\mathbb R}}$. Let $\varepsilon >0$ be a {Kazhdan}\ constant for this set. Fix $\delta >0$. In order to prove that $(-\delta ,\delta )\cup{Q}$ is a {Kazhdan}\ set in ${\ensuremath{\mathbb R}}$, we use Proposition \ref{Proposition A}: let $\gamma $ be a positive number which will be fixed  later on, and let $\sigma $ be a probability measure on ${\ensuremath{\mathbb R}}$ such that $\sup_{t\in(-\delta ,\delta )\cup{Q}}\,\bigl|\widehat{\sigma }(t)-1\bigr|<\gamma $.
For any $a\in{\ensuremath{\mathbb N}}$ and any $t\in(-\delta ,\delta )$,
\[
 2(1-\Re e\,\widehat{\sigma }(at))=\int_{\ensuremath{\mathbb R}}\bigl|e^{iatx}-1\bigr|^{2}d\sigma (x)\le a^{2}\int_{\ensuremath{\mathbb R}}\bigl|e^{itx}-1\bigr|^{2}d\sigma (x)\le 2a^{2} \Re e\,(1-\widehat{\sigma }(t))
\]
so that ${\displaystyle}\sup_{t\in(\delta ,\delta )}\,(1-\Re e\,\widehat{\sigma }(at))<a^{2}\gamma $. If we choose $a>n/\delta $, we obtain that
\[
\sup_{t\in(-n,n)}\,\bigl|1-\widehat{\sigma }(t)\bigr|^{2}\le 2\sup_{t\in(-n,n)}\,\bigl(1-\Re e\,\widehat{\sigma }(t)\bigr)<2a^{2}\gamma .
\]
So if $\gamma $ is chosen so small that $\gamma <\min(\varepsilon ,\varepsilon ^{2}/(2a^{2}))$, we obtain that 
\[
\sup_{t\in(-n,n)\cup{Q}}\,\bigl|1-\widehat{\sigma }(t)\bigr|<\varepsilon,
\] 
and since $\varepsilon $ is a {Kazhdan}\ constant for $(-n,n)\cup{Q}$, $\sigma (\{0\})>0$. Hence $\gamma $ is a {Kazhdan}\ constant for $(-\delta ,\delta )\cup{Q}$.
\end{proof}

\begin{remark}\label{Remark 8.6}
 It is necessary to add a small interval to the set ${Q}$ in order to turn it into a {Kazhdan}\ subset of ${\ensuremath{\mathbb R}}$, even when ${Q}$ generates a dense subgroup of ${\ensuremath{\mathbb R}}$. Indeed, consider the polynomial $p(x)=x+\sqrt{2}$. The set ${Q}=\{k+\sqrt{2}\,;\,k\ge 0\}$ is not a {Kazhdan}\ set in ${\ensuremath{\mathbb R}}$: for any $\varepsilon >0$, let $b\in{\ensuremath{\mathbb N}}$ be such that $|e^{2i\pi b\sqrt{2}}-1|<\varepsilon $. The measure 
$\sigma$ defined as the Dirac mass at the point ${2\pi b}$ satisfies $\sup_{k\ge 0}\,|\widehat{\sigma }(k+\sqrt{2})-1|<\varepsilon $, so that, by Proposition \ref{Proposition A}, ${Q}$ is not a {Kazhdan}\ set in ${\ensuremath{\mathbb R}}$.\end{remark}

 We finish this section by exhibiting a link between {Kazhdan}\ subsets of ${\ensuremath{\mathbb Z}}^{d}$ and 
 {Kazhdan}\ subsets of ${\ensuremath{\mathbb R}}^{d}$, $d\ge 1$. Let $Q$ be a subset of ${\ensuremath{\mathbb Z}}^{d}$. Seen as a subset of ${\ensuremath{\mathbb R}}^{d}$, $Q$ is never a {Kazhdan}\ set in ${\ensuremath{\mathbb R}}^{d}$. But as a consequence of Theorem \ref{Theorem 0}, we see that $Q$ becomes a {Kazhdan}\ set in ${\ensuremath{\mathbb R}}^{d}$ if we add a small perturbation to it.
 
\begin{proposition}\label{Proposition D}
 Fix an integer $d\ge 1$, and let $(W_{n})_{n\ge 1}$ be an increasing sequence of subsets of ${\ensuremath{\mathbb R}}^{d}$ such that $\bigcup_\gnW_{n}={\ensuremath{\mathbb R}}^{d}$. Let ${Q}$ be a {Kazhdan}\ subset of ${\ensuremath{\mathbb Z}}^{d}$. There exists an ${n\ge 1}$ such that  $W_{n}\cup {Q}$ is a {Kazhdan}\ set in ${\ensuremath{\mathbb R}}^{d}$. Also, $B(0,\delta )\cup Q$ is a {Kazhdan}\ subset of ${\ensuremath{\mathbb R}}^{d}$ for any $\delta >0$, where $B(0,\delta )$ denotes the open unit ball of radius $\delta $ for the Euclidean norm on ${\ensuremath{\mathbb R}}^{d}$.
\end{proposition}
\begin{proof}
 Let $\varepsilon >0$ be a {Kazhdan}\ constant for ${Q}$, seen as a subset of ${\ensuremath{\mathbb Z}}^{d}$. Consider a unitary representation $\pi $ of ${\ensuremath{\mathbb R}}^{d}$ on a separable Hilbert space $H$ which admits a $({Q},\varepsilon ^{2}/2)$-invariant vector $x\in H$. We are going to show that $\pi $ admits a finite dimensional subrepresentation, which by Theorem \ref{Theorem 2} will imply that $W_{n}\cup{Q}$ is a {Kazhdan}\ subset of ${\ensuremath{\mathbb R}}^{d}$ for some ${n\ge 1}$. 
 \par\smallskip
Without loss of generality we can suppose that $\pi $ is a direct integral on a Borel space $Z$, with respect to a finite measure $\mu $ on $Z$, of a family $(\pi _{z})_{z\in{\ensuremath{\mathbb Z}}}$ of irreducible representations of ${\ensuremath{\mathbb R}}^{d}$. So $\pi $ is a representation of ${\ensuremath{\mathbb R}}^{d}$ on $L^{2}(Z,\mu )$. We write elements $f$ of $L^{2}(Z,\mu )$ as $f=(f_{z})_{z\in Z}$.
We suppose that $||x||=1$; our hypothesis implies that
\[
\sup_{{\mkern 1.5 mu\pmb{t}}\in{Q}}\,\bigl|1-{\ensuremath{{\langle {\pi ({\mkern 1.5 mu\pmb{t}})x},{x}\rangle}}}\bigr|<\dfrac{\varepsilon ^{2}}{2}\cdot 
\]
Each representation $\pi _{z}$ acts  on vectors ${\mkern 1.5 mu\pmb{t}}=(t_{1},\dots,t_{d})$ of ${\ensuremath{\mathbb R}}^{d}$ as 
$\pi _{z}({\mkern 1.5 mu\pmb{t}})=\exp(2i\pi {\ensuremath{{\langle {\mkern 1.5 mu\pmb{t}},{{\mkern 1.5 mu\pmb{\theta }} _{z}}\rangle}}})$ for some vector ${\mkern 1.5 mu\pmb{\theta }} _{z}=(\theta _{1,z},\dots,\theta _{d,z})$ of ${\ensuremath{\mathbb R}}^{d}$. Hence
\[
\sup_{{\mkern 1.5 mu\pmb{t}}\in{Q}}\,\Bigl|1-\int_{Z}e^{2i\pi {\ensuremath{{\langle {\mkern 1.5 mu\pmb{t}},{{\mkern 1.5 mu\pmb{\theta }} _{z}}\rangle}}}}|x_z|^{2}d\mu (z)\Bigr|<\dfrac{\varepsilon ^{2}}{2}\cdot 
\]
Consider now the representation $\rho $ of ${\ensuremath{\mathbb Z}}^{d}$ on $L^{2}(Z,\mu )$ defined by 
$\smash{\xymatrix{\rho ({\mkern 1.5 mu\pmb{n}})\,f:z\ar@{|->}[r]&e^{2i\pi {\ensuremath{{\langle {\mkern 1.5 mu\pmb{n}},{{\mkern 1.5 mu\pmb{\theta }} _{z}}\rangle}}}}f_{z}}\!}$ for every 
${\mkern 1.5 mu\pmb{n}}=(n_{1},\dots,n_{d})\in{\ensuremath{\mathbb Z}}^{d}$ and every $f\in L^{2}(Z,\mu )$. We have
\[
\sup_{{\mkern 1.5 mu\pmb{n}}\in{Q}}||\rho ({\mkern 1.5 mu\pmb{n}})x-x||^{2}\le 2\sup_{{\mkern 1.5 mu\pmb{n}}\in{Q}}\bigl|1-{\ensuremath{{\langle {\rho ({\mkern 1.5 mu\pmb{n}})x},{x}\rangle}}}\bigr|<\varepsilon ^{2},
\]
and since $\varepsilon $ is a {Kazhdan}\ constant for ${Q}$ as a subset of ${\ensuremath{\mathbb Z}}^{d}$, $\rho $ has a non-zero ${\ensuremath{\mathbb Z}}^{d}$-invariant vector. There exists hence $f\in L^{2}(Z,\mu )$ with $||f||=1$ such that 
$\rho ({\mkern 1.5 mu\pmb{n}})f=f$ for every ${\mkern 1.5 mu\pmb{n}}\in{\ensuremath{\mathbb Z}}^{d}$. Fix a representative of $f\in L^{2}(Z,\mu )$, and set 
$Z_{0}=\{z\in Z\,;\, f_{z}\neq 0\}$. Then $\mu (Z_{0})>0$. For every $z\in Z_{0}$ we have 
$e^{2i\pi {\ensuremath{{\langle {\mkern 1.5 mu\pmb{n}},{{\mkern 1.5 mu\pmb{\theta }} _{z}}\rangle}}}}=1$ for every ${\mkern 1.5 mu\pmb{n}}\in {\ensuremath{\mathbb Z}}^{d}$, which implies that 
${\mkern 1.5 mu\pmb{\theta }} _{z}\in{\ensuremath{\mathbb Z}}^{d}$. For each ${\mkern 1.5 mu\pmb{n}}=(n_{1},\dots,n_{d})\in {\ensuremath{\mathbb Z}}^{d}$, let 
$Z_{\mkern 1.5 mu\pmb{n}}=\{z\in Z_{0}\,;\,\theta _{i,z}=n_{i}\ \textrm{for each}\ i\in\{1,\dots,n\}\}$ and
$H_{\mkern 1.5 mu\pmb{n}}=\{f\in L^{2}(Z,\mu )\,;\,f=0\ \mu\textrm{-a.\,e. on}\ Z\setminus Z_{\mkern 1.5 mu\pmb{n}}\}$. We have
$\bigcup_{{\mkern 1.5 mu\pmb{n}}\in{\ensuremath{\mathbb Z}}^{d}}Z_{\mkern 1.5 mu\pmb{n}}=Z_{0}$, so there exists ${\mkern 1.5 mu\pmb{n}}_{0}\in{\ensuremath{\mathbb Z}}^{d}$ such that $\mu (Z_{{\mkern 1.5 mu\pmb{n}}_{0}})>0$. Each subspace $H_{\mkern 1.5 mu\pmb{n}}$ is easily seen to be invariant for $\pi $, and the representation $\pi _{\mkern 1.5 mu\pmb{n}}$ induced by $\pi $ on $H_{\mkern 1.5 mu\pmb{n}}$ is given by $\smash{\xymatrix{\pi _{\mkern 1.5 mu\pmb{n}}({\mkern 1.5 mu\pmb{t}})\,f\, :\,z\ar@{|->}[r]&e^{2i\pi {\ensuremath{{\langle {\mkern 1.5 mu\pmb{t}},{\mkern 1.5 mu\pmb{n}}\rangle}}}}f_{z}}}$ for every ${\mkern 1.5 mu\pmb{t}}\in{\ensuremath{\mathbb R}}^{d}$ and every $f\in H_{\mkern 1.5 mu\pmb{n}}$. So $\pi $ admits a subrepresentation of dimension $1$ as soon as $H_{\mkern 1.5 mu\pmb{n}}$ is non-zero, i.\,e.\ as soon as $\mu (Z_{\mkern 1.5 mu\pmb{n}})>0$. Since $\mu (Z_{{\mkern 1.5 mu\pmb{n}}_{0}})>0$, $\pi$  admit a subrepresentation  of dimension $1$. This finishes the proof of our claim, and an application of Theorem \ref{Theorem 2} shows that $W_{n}\cup {Q}$ is a {Kazhdan}\ set in ${\ensuremath{\mathbb R}}^{d}$ for some ${n\ge 1}$. If we choose $W_{n}=B(0,n)$ for every $n\ge 1$, and proceed as in the proof of Example \ref{Example 8.5}, we obtain that $B(0,\delta )\cup Q$ is a {Kazhdan}\ set in ${\ensuremath{\mathbb R}}^{d}$ for every $\delta >0$.
\end{proof}
We now move out of the commutative setting, and present a characterization of {Kazhdan}\ sets in the Heisenberg groups $H_{n}$.

\subsection{{Kazhdan}\ sets in the Heisenberg groups $H_{n}$}\label{Subsection 8.2}
The Heisenberg group  of dimension $n\ge 1$, denoted by $H_{n}$, is formed of triples $(t,{\mkern 1.5 mu\pmb{q}},{\mkern 1.5 mu\pmb{p}})$ of ${\ensuremath{\mathbb R}}\times {\ensuremath{\mathbb R}}^{n}\times{\ensuremath{\mathbb R}}^{n}={\ensuremath{\mathbb R}}^{2n+1}$. The group operation is given by
\[
(t_{1},{\mkern 1.5 mu\pmb{q}}_{1},{\mkern 1.5 mu\pmb{p}}_{1})\cdot (t_{2},{\mkern 1.5 mu\pmb{q}}_{2},{\mkern 1.5 mu\pmb{p}}_{2})=(t_{1}+t_{2}+\frac{1}{2}(
{\mkern 1.5 mu\pmb{p}}_{1}\cdot {\mkern 1.5 mu\pmb{q}}_{2}-{\mkern 1.5 mu\pmb{p}}_{2}\cdot{\mkern 1.5 mu\pmb{q}}_{1},{\mkern 1.5 mu\pmb{q}}_{1}+{\mkern 1.5 mu\pmb{q}}_{2},{\mkern 1.5 mu\pmb{p}}_{1}+{\mkern 1.5 mu\pmb{p}}_{2}),
\]
where ${\mkern 1.5 mu\pmb{p}}\cdot{\mkern 1.5 mu\pmb{q}}$ denotes the scalar product of two vectors ${\mkern 1.5 mu\pmb{p}}$ and ${\mkern 1.5 mu\pmb{q}}$ of ${\ensuremath{\mathbb R}}^{n}$.
Unitary representations of $H_{n}$ are completely classified (see for instance \cite[Ch.~2]{T}, or \cite[Sec.~6.6]{Fo}): there are two distinct families of such representations, which we denote respectively by $(\mathcal{F}_{1})$ and $(\mathcal{F}_{2})$:
\par\smallskip 
-- the representations belonging to the family $(\mathcal{F}_{1})$ are representations of $H_{n}$ on $L^{2}({\ensuremath{\mathbb R}}^{n})$, the space of complex-valued square-summable functions on ${\ensuremath{\mathbb R}}^{n}$. They are parametrized by an element of ${\ensuremath{\mathbb R}}$, which we write as $\pm\lambda $ with $\lambda >0$. Then 
$\pi _{\pm\lambda }(t,{\mkern 1.5 mu\pmb{q}},{\mkern 1.5 mu\pmb{p}})$, $(t,{\mkern 1.5 mu\pmb{q}},{\mkern 1.5 mu\pmb{p}})\in{\ensuremath{\mathbb R}}^{2n+1}$, acts on $L^{2}({\ensuremath{\mathbb R}}^{n})$ as
\[
\xymatrix{
\pi _{\pm\lambda }(t,{\mkern 1.5 mu\pmb{q}},{\mkern 1.5 mu\pmb{p}})\,u\,:\,{\mkern 1.5 mu\pmb{x}}\ar@{|->}[r]&e^{i(\pm\lambda t\pm\sqrt{\lambda }{\mkern 1.5 mu\pmb{q}}\cdot {\mkern 1.5 mu\pmb{x}}+\frac{\lambda }{2}{\mkern 1.5 mu\pmb{q}}\cdot{\mkern 1.5 mu\pmb{p}})}u({\mkern 1.5 mu\pmb{x}}+\sqrt{\lambda }\,{\mkern 1.5 mu\pmb{p}})
}
\]
where $u$ belongs to $L^{2}({\ensuremath{\mathbb R}}^{n})$. 
These representations have the following important property, which will appear again in the next subsection:
\begin{fact}\label{Fact E}
 For every $\pm\lambda \in{\ensuremath{\mathbb R}}$ and every $u\in L^{2}({\ensuremath{\mathbb R}}^{n})$, 
\[
\xymatrix{{\ensuremath{{\langle {\pi _{\pm\lambda }(t,{\mkern 1.5 mu\pmb{q}},{\mkern 1.5 mu\pmb{p}})\,u},{u}\rangle}}}\ar[r]&0}\quad\textrm{as}\quad
\xymatrix{|{\mkern 1.5 mu\pmb{p}}|\ar[r]&+\infty .}
\]
\end{fact}
\begin{proof}
 This follows directly from the dominated convergence theorem if $u$ has compact support in ${\ensuremath{\mathbb R}}^{n}$. It then suffices to approximate $u\in L^{2}({\ensuremath{\mathbb R}}^{n})$ by functions with compact support to get the result.
\end{proof}
\par\smallskip 
-- The representations belonging to the family $(\mathcal{F}_{2})$ are one-dimensional. They are parametrized by elements $({\mkern 1.5 mu\pmb{y}},{\mkern 1.5 mu\pmb{\eta}} )$ of ${\ensuremath{\mathbb R}}^{2n}$: for every $(t,{\mkern 1.5 mu\pmb{q}},{\mkern 1.5 mu\pmb{p}})\in H_{n}$,
\[
\pi _{{\mkern 1.5 mu\pmb{y}},{\mkern 1.5 mu\pmb{\eta}} }(t,{\mkern 1.5 mu\pmb{p}},{\mkern 1.5 mu\pmb{q}})=e^{i({\mkern 1.5 mu\pmb{y}}\cdot{\mkern 1.5 mu\pmb{q}}+{\mkern 1.5 mu\pmb{\eta}}\cdot {\mkern 1.5 mu\pmb{p}})}.
\]
\par\smallskip 
Our main result concerning {Kazhdan}\ sets in $H_{n}$ is the following:

\begin{theorem}\label{Proposition F}
 Let ${Q}$ be a subset of the Heisenberg group $H_{n}$, $n\geq 1$. The following assertions are equivalent:
\begin{enumerate}
 \item [(1)] ${Q}$ is a {Kazhdan}\ set in $H_{n}$;
\item[(2)] the set ${Q}_{0}=\{({\mkern 1.5 mu\pmb{q}},{\mkern 1.5 mu\pmb{p}})\in{\ensuremath{\mathbb R}}^{2n}\,;\,\exists\,t\in{\ensuremath{\mathbb R}},\ (t,{\mkern 1.5 mu\pmb{q}},{\mkern 1.5 mu\pmb{p}})\in{Q}\}$ is a {Kazhdan}\ set in ${\ensuremath{\mathbb R}}^{2n}$.
\end{enumerate}
\end{theorem}

\begin{proof}
 The proof of Theorem \ref{Proposition F} relies on the same kind of ideas as those employed in the proof of Proposition \ref{Proposition D}. We start with the easy implication, which is that $(1)$ implies $(2)$. Suppose that ${Q}$  is a {Kazhdan}\ set in $H_{n}$, and let $\varepsilon >0$ be a {Kazhdan}\ constant for ${Q}$. In order to show that ${Q}_{0}$ is a {Kazhdan}\ set in ${\ensuremath{\mathbb R}}^{2n}$, we apply Proposition \ref{Proposition A}: let $\sigma $ be a probability measure on ${\ensuremath{\mathbb R}}^{2n}$ such that
\begin{equation}\label{Equation 20}
 \sup_{({\mkern 1.5 mu\pmb{q}},{\mkern 1.5 mu\pmb{p}})\in{Q}_{0}}\,\bigl|\widehat{\sigma }({\mkern 1.5 mu\pmb{q}},{\mkern 1.5 mu\pmb{p}})-1\bigr|=\sup_{({\mkern 1.5 mu\pmb{q}},{\mkern 1.5 mu\pmb{p}})\in{Q}_{0}}
\Bigl|\int_{{\ensuremath{\mathbb R}}^{2n}}e^{i({\mkern 1.5 mu\pmb{y}}\cdot {\mkern 1.5 mu\pmb{q}}+{\mkern 1.5 mu\pmb{\eta}} \cdot{\mkern 1.5 mu\pmb{p}})}d\sigma ({\mkern 1.5 mu\pmb{y}},{\mkern 1.5 mu\pmb{\eta}} )-1\Bigr|<\dfrac{\varepsilon^{2}}{2} 
\end{equation}
and consider the representation $\rho $ of $H_{n}$ on $L^{2}({\ensuremath{\mathbb R}}^{2n},\sigma )$ defined by 
\[
\xymatrix{
\rho (t,{\mkern 1.5 mu\pmb{q}},{\mkern 1.5 mu\pmb{p}})f\,:\,({\mkern 1.5 mu\pmb{y}},{\mkern 1.5 mu\pmb{\eta}} )\ar@{|->}[r]&e^{i({\mkern 1.5 mu\pmb{y}}\cdot{\mkern 1.5 mu\pmb{q}}+{\mkern 1.5 mu\pmb{\eta}} \cdot{\mkern 1.5 mu\pmb{p}})}f({\mkern 1.5 mu\pmb{y}},{\mkern 1.5 mu\pmb{\eta}} )
}
\]
for every $(t,{\mkern 1.5 mu\pmb{q}},{\mkern 1.5 mu\pmb{p}})\in H_{n}$ and every $f\in L^{2}({\ensuremath{\mathbb R}}^{2n},\sigma )$. Then (\ref{Equation 20}) implies that the constant function $\mathbbm{1}$ is a $({Q},\varepsilon )$-invariant vector for $\rho $. Since $({Q},\varepsilon )$ is a {Kazhdan}\ pair in $H_{n}$, it follows that $\rho $ admits a non-zero $H_{n}$-invariant function $f\in L^{2}({\ensuremath{\mathbb R}}^{2n},\sigma )$. Fix a representative of $f$, and consider the subset $A$ of ${\ensuremath{\mathbb R}}^{2n}$ consisting of pairs $({\mkern 1.5 mu\pmb{y}},{\mkern 1.5 mu\pmb{\eta}})$ such that $f({\mkern 1.5 mu\pmb{y}},{\mkern 1.5 mu\pmb{\eta}} )\neq 0$. Then $\sigma (A)>0$, and $\sigma $-almost every element $({\mkern 1.5 mu\pmb{y}},{\mkern 1.5 mu\pmb{\eta}} )$ of $A$ satisfies ${\mkern 1.5 mu\pmb{y}}\cdot{\mkern 1.5 mu\pmb{q}}+{\mkern 1.5 mu\pmb{\eta}} \cdot{\mkern 1.5 mu\pmb{p}}\in 2\pi {\ensuremath{\mathbb Z}}$ for every $({\mkern 1.5 mu\pmb{q}},{\mkern 1.5 mu\pmb{p}})\in{\ensuremath{\mathbb Q}}^{2n}$. By continuity, ${\mkern 1.5 mu\pmb{y}}\cdot{\mkern 1.5 mu\pmb{q}}+{\mkern 1.5 mu\pmb{\eta}} \cdot{\mkern 1.5 mu\pmb{p}}$ belongs to $2\pi {\ensuremath{\mathbb Z}}$ for every $({\mkern 1.5 mu\pmb{q}},{\mkern 1.5 mu\pmb{p}})\in{\ensuremath{\mathbb R}}^{2n}$, so that $({\mkern 1.5 mu\pmb{y}},{\mkern 1.5 mu\pmb{\eta}} )=({\mkern 1.5 mu\pmb{0}},{\mkern 1.5 mu\pmb{0}})$. We have thus proved that 
$\sigma (\{{\mkern 1.5 mu\pmb{0}},{\mkern 1.5 mu\pmb{0}}\})>0$, and it follows from Proposition \ref{Proposition A} that ${Q}_{0}$ is a {Kazhdan}\ set in ${\ensuremath{\mathbb R}}^{2n}$. 
\par\smallskip
Let us now prove the converse implication. Suppose that ${Q}_{0}$ is a {Kazhdan}\ set in ${\ensuremath{\mathbb R}}^{2n}$, and let $0<\varepsilon <3$ be a {Kazhdan}\ constant for  ${Q}_{0}$.  
 Let $\pi $ be a unitary representation of $H_{n}$ on a separable Hilbert space $H$, which admits a  $({Q}, \frac{\varepsilon}{8} )$-invariant vector $x\in H$ of norm $1$. We write as usual $\pi $ as a direct integral $\pi =\int_{Z}^{\oplus}\pi _{z}\,d\mu (z)$, where $\mu $ is a finite Borel measure on a standard Borel space $Z$, and $x$ as $(x_{z})_{z\in Z}$, with 
$\int_{Z}||x_{z}||^{2} d\mu (z)=1$. We have 
\begin{equation}\label{Eq17}
\sup_{(t,{\mkern 1.5 mu\pmb{q}},{\mkern 1.5 mu\pmb{p}})\in{Q}}\,\Bigl|1-\int_{Z}{\ensuremath{{\langle {\pi _{z}(t,{\mkern 1.5 mu\pmb{q}},{\mkern 1.5 mu\pmb{p}})x_{z}},{x_{z}}\rangle}}}d\mu (z)\Bigr|<\dfrac{\varepsilon}{8}\cdot 
\end{equation}
For every $z\in Z$, the irreducible representation $\pi_{z}$ belongs to one of the two families $(\mathcal{F}_{1})$ and $(\mathcal{F}_{2})$. If $\pi _{z}$ belongs to $(\mathcal{F}_{1})$, we write it as $\pi _{\pm\lambda_{z} }$ for some $\pm\lambda _{z}\in{\ensuremath{\mathbb R}}$, and if $\pi $ belongs to $(\mathcal{F}_{2})$, as $\pi _{{\mkern 1.5 mu\pmb{y}}_{z},{\mkern 1.5 mu\pmb{\eta}} _{z}}$ for some $({\mkern 1.5 mu\pmb{y}}_{z},{\mkern 1.5 mu\pmb{\eta}} _{z})\in{\ensuremath{\mathbb R}}^{2n}$. Let, for $i=1,2$, $Z_{i}$ be the subset of $Z$ consisting of the elements $z\in Z$ such that $\pi _{z}$ belongs to $(\mathcal{F}_{i})$. We have $x_{z}\in L^{2}({\ensuremath{\mathbb R}}^{n})$ for every $z\in Z_{1}$, and $x_{z}\in{\ensuremath{\mathbb C}}$ for every $x\in Z_{2}$.
We now observe the following:
\begin{lemma}\label{Lemma G}
 A {Kazhdan}\ subset of ${\ensuremath{\mathbb R}}^{2n}$ contains elements $({\mkern 1.5 mu\pmb{q}},{\mkern 1.5 mu\pmb{p}})$ such that the Euclidean norm $|{\mkern 1.5 mu\pmb{p}}|$ of ${\mkern 1.5 mu\pmb{p}}$ is arbitrarily large.
\end{lemma}
\begin{proof}
 Let ${Q}_{0}$ be a {Kazhdan}\ subset of ${\ensuremath{\mathbb R}}^{2n}$, with {Kazhdan}\ constant $\varepsilon >0$, and suppose that there exists a constant $M>0$ such that $|{\mkern 1.5 mu\pmb{p}}|\le M$ for every $({\mkern 1.5 mu\pmb{q}},{\mkern 1.5 mu\pmb{p}})\in{Q}_{0}$. Let $\delta >0$ be such that $2M\delta <\varepsilon $ and consider the probability measure on ${\ensuremath{\mathbb R}}^{2n}$ defined by 
\[
\sigma =\delta _{\mkern 1.5 mu\pmb{0}}\times\mathbbm{1}_{B({\mkern 1.5 mu\pmb{0}},\delta )}\,\dfrac{d{\mkern 1.5 mu\pmb{p}}}{|B({\mkern 1.5 mu\pmb{0}},\delta)| }\cdot 
\]
For every $({\mkern 1.5 mu\pmb{q}},{\mkern 1.5 mu\pmb{p}})\in{Q}$,
\[
|\widehat{\sigma }({\mkern 1.5 mu\pmb{q}},{\mkern 1.5 mu\pmb{p}})-1|=\Bigl|\int_{B({\mkern 1.5 mu\pmb{0}},\delta )}e^{i{\mkern 1.5 mu\pmb{s}}\cdot{\mkern 1.5 mu\pmb{p}}}\,\dfrac{d{\mkern 1.5 mu\pmb{s}}}{|B({\mkern 1.5 mu\pmb{0}},\delta )| }-1\Bigr|\le 2\delta |{\mkern 1.5 mu\pmb{p}}|\le 2M\delta <\varepsilon. 
\]
But $\sigma (\{({\mkern 1.5 mu\pmb{0}},{\mkern 1.5 mu\pmb{0}})\})=0$, and it follows from Proposition \ref{Proposition A} that ${Q}_{0}$ is not a {Kazhdan}\ set in ${\ensuremath{\mathbb R}}^{2n}$, which is a contradiction.
\end{proof}

By Fact \ref{Fact E}, we have for every $x\in Z_{1}$
\[
\xymatrix{
{\ensuremath{{\langle {\pi _{\pm\lambda_{z} }(t,{\mkern 1.5 mu\pmb{q}},{\mkern 1.5 mu\pmb{p}})x_{z}},{x_{z}}\rangle}}}\ar[r]&0
}\quad\textrm{as}\ \xymatrix@C=10pt{|{\mkern 1.5 mu\pmb{p}}|\ar[r]&+\infty },\ (t,{\mkern 1.5 mu\pmb{q}},{\mkern 1.5 mu\pmb{p}})\in H_{n}.
\]
Since $|{\ensuremath{{\langle {\pi _{\pm\lambda }(t,{\mkern 1.5 mu\pmb{q}},{\mkern 1.5 mu\pmb{p}})x_{z}},{x_{z}}\rangle}}}|\le||x_{z}||^{2}$ for every $z\in Z_{1}$, and 
$\int_{Z}||x_{z}||^{2}d\mu (z)=1$, the dominated convergence theorem implies that
\[
\xymatrix{{\displaystyle}\int_{Z_{1}}
{\ensuremath{{\langle {\pi _{\pm\lambda_{z} }(t,{\mkern 1.5 mu\pmb{q}},{\mkern 1.5 mu\pmb{p}})x_{z}},{x_{z}}\rangle}}}d\mu (z)\ar[r]&0
}\quad\textrm{as}\ \xymatrix@C=10pt{|{\mkern 1.5 mu\pmb{p}}|\ar[r]&+\infty },\ (t,{\mkern 1.5 mu\pmb{q}},{\mkern 1.5 mu\pmb{p}})\in H_{n}.
\]
By Lemma \ref{Lemma G}, there exists an element $(t_{0},{\mkern 1.5 mu\pmb{q}}_{0},{\mkern 1.5 mu\pmb{p}}_{0})$ of ${Q}$ with $|{\mkern 1.5 mu\pmb{p}}_{0}|$ so large that 
\[
\Bigl|\int_{Z_{1}}
{\ensuremath{{\langle {\pi _{\pm\lambda_{z} }(t_{0},{\mkern 1.5 mu\pmb{q}}_{0},{\mkern 1.5 mu\pmb{p}}_{0})x_{z}},{x_{z}}\rangle}}}d\mu (z)\Bigr|<\dfrac{\varepsilon }{8}\cdot 
\]
Property (\ref{Eq17}) implies then that
\[
\Bigl|1-\int_{Z_{2}}\pi _{{\mkern 1.5 mu\pmb{y}}_{z},{\mkern 1.5 mu\pmb{\eta}} _{z}}(t_{0},{\mkern 1.5 mu\pmb{q}}_{0},{\mkern 1.5 mu\pmb{p}}_{0})|x_{z}|^{2}d\mu (z)\Bigr|<\dfrac{\varepsilon }{4}
\] 
from which it follows that ${\displaystyle}\int_{Z_{2}}|x_{z}|^{2}d\mu (z)>1-\dfrac{\varepsilon }{4}$, so that
${\displaystyle}\int_{Z_{1}}||x_{z}||^{2}d\mu (z)<\dfrac{\varepsilon }{4}$.  Plugging this into (\ref{Eq17}) yields that
\[
\sup_{(t,{\mkern 1.5 mu\pmb{q}},{\mkern 1.5 mu\pmb{p}})\in{Q}}\,\Bigl|1-\int_{Z_{2}}\pi _{{\mkern 1.5 mu\pmb{y}}_{z},{\mkern 1.5 mu\pmb{\eta}} _{z}}(t,{\mkern 1.5 mu\pmb{q}},{\mkern 1.5 mu\pmb{p}})|x_{z}|^{2}d\mu (z)\Bigr|<\dfrac{3\varepsilon }{8}\cdot 
\]
Since $\int_{Z_{2}}|x_{z}|^{2}d\mu (z)>1-\varepsilon /4$ and $0<\varepsilon <3$, we can, by normalizing the family $(x_{z})_{z\in Z_{2}}$, suppose without loss of generality that $Z=Z_{2}$,  $\int_{Z}|x_{z}|^{2}d\mu (z)=1$ and that
\begin{equation}\label{Eq18}
 \sup_{(t,{\mkern 1.5 mu\pmb{q}},{\mkern 1.5 mu\pmb{p}})\in{Q}}\,\Bigl|1-\int_{Z}e^{i({\mkern 1.5 mu\pmb{y}}_{z}\cdot{\mkern 1.5 mu\pmb{q}}+{\mkern 1.5 mu\pmb{\eta}} _{z}\cdot{\mkern 1.5 mu\pmb{p}})}\,|x_{z}|^{2\,}d\mu (z)\Bigr|<\varepsilon . 
\end{equation}
 Consider now the unitary representation $\rho $ of ${\ensuremath{\mathbb R}}^{2n}$ on $L^{2}(Z,\mu )$ defined by 
\[
\xymatrix{
\rho ({\mkern 1.5 mu\pmb{q}},{\mkern 1.5 mu\pmb{p}})\,f\,:\,z\ar@{|->}[r]&e^{i({\mkern 1.5 mu\pmb{y}}_{z}\cdot{\mkern 1.5 mu\pmb{q}}+{\mkern 1.5 mu\pmb{\eta}} _{z}\cdot{\mkern 1.5 mu\pmb{p}})}f_z
}
\]
for every $({\mkern 1.5 mu\pmb{q}},{\mkern 1.5 mu\pmb{p}})\in{\ensuremath{\mathbb R}}^{2n}$ and every $f=(f_z)_{z\in Z}\in L^{2}(Z,\mu )$. Then (\ref{Eq18}) can be 
rewritten as 
\begin{align*}
 \sup_{(t,{\mkern 1.5 mu\pmb{q}},{\mkern 1.5 mu\pmb{p}})\in{Q}}\,\bigl|1-{\ensuremath{{\langle {\rho ({\mkern 1.5 mu\pmb{q}},{\mkern 1.5 mu\pmb{p}})x},{x}\rangle}}}\bigr|<\varepsilon,\quad\textrm{i.e.}\quad
\sup_{({\mkern 1.5 mu\pmb{q}},{\mkern 1.5 mu\pmb{p}})\in{Q}_{0}}\,\bigl|1-{\ensuremath{{\langle {\rho ({\mkern 1.5 mu\pmb{q}},{\mkern 1.5 mu\pmb{p}})x},{x}\rangle}}}\bigr|<\varepsilon.
\end{align*}
Since $\varepsilon $ is a {Kazhdan}\ constant for ${Q}_{0}$, the representation $\rho $ admits a non-zero ${\ensuremath{\mathbb R}}^{2n}$-invariant vector $f\in L^{2}(Z,\mu )$. Proceeding as in the proof of $(1)\Longrightarrow (2)$, we see that for every $({\mkern 1.5 mu\pmb{q}},{\mkern 1.5 mu\pmb{p}})\in{\ensuremath{\mathbb R}}^{2n}$, $e^{i({\mkern 1.5 mu\pmb{y}}_{z}\cdot{\mkern 1.5 mu\pmb{q}}+{\mkern 1.5 mu\pmb{\eta}} _{z}\cdot{\mkern 1.5 mu\pmb{p}})}f(z)=f(z)$ $\mu $-almost everywhere on $Z$, so that there exists a subset $Z_{0}$ of $Z$ with $\mu (Z_{0})>0$
such that $f(z)\neq 0$ for every $z\in Z_{0}$, and ${\mkern 1.5 mu\pmb{y}}_{z}\cdot{\mkern 1.5 mu\pmb{q}}+{\mkern 1.5 mu\pmb{\eta}} _{z}\cdot{\mkern 1.5 mu\pmb{p}}$ belongs to $2\pi {\ensuremath{\mathbb Z}}$ for every $z\in Z_{0}$ and every $({\mkern 1.5 mu\pmb{q}},{\mkern 1.5 mu\pmb{p}})\in {\ensuremath{\mathbb Q}}^{2n}$. By continuity, ${\mkern 1.5 mu\pmb{y}}_{z}\cdot{\mkern 1.5 mu\pmb{q}}+{\mkern 1.5 mu\pmb{\eta}} _{z}\cdot{\mkern 1.5 mu\pmb{p}}$ belongs to $2\pi {\ensuremath{\mathbb Z}}$ for every $z\in Z_{0}$ and every $({\mkern 1.5 mu\pmb{q}},{\mkern 1.5 mu\pmb{p}})\in{\ensuremath{\mathbb R}}^{2n}$, so that $({\mkern 1.5 mu\pmb{y}}_{z},{\mkern 1.5 mu\pmb{\eta}} _{z})=({\mkern 1.5 mu\pmb{0}},{\mkern 1.5 mu\pmb{0}})$ for every $z\in Z_{0}$. So if we set $Z_{0}'=\{z\in Z\,;\,({\mkern 1.5 mu\pmb{y}}_{z},{\mkern 1.5 mu\pmb{\eta}} _{z})=({\mkern 1.5 mu\pmb{0}},{\mkern 1.5 mu\pmb{0}})\}$, we have $\mu (Z_{0}')>0$. The function 
$f=\mathbbm{1}_{Z'_{0}}$ is hence a non-zero element of $L^{2}(Z,\mu )$, which is clearly an $H_{n}$-invariant vector for the representation $\pi $. So $({Q},\frac{\varepsilon }{8})$ is a {Kazhdan}\ pair in $H_{n}$, and Theorem \ref{Proposition F} is proved.
\end{proof}

\subsection{{Kazhdan}\ sets in the $\bm{ax+b}$ group}
The underlying space of the $ax+b$ group is $(0,+\infty )\times {\ensuremath{\mathbb R}}$, and the group law is given by 
$(a,b)(a',b')=(aa',b+ab')$, where $(a,b)$ and $(a',b')$ belong to $ (0,+\infty )\times {\ensuremath{\mathbb R}}$. We denote this group by $G$. As in the case of the Heisenberg groups, the irreducible unitary representations of $G$ are completely classified (see \cite[Sec.~6.7]{Fo}) and fall within two classes:
\par\smallskip 
-- the class $(\mathcal{F}_{1})$ consists of two infinite dimensional representations $\pi _{+}$ and $\pi _{-}$ of $G$, which act respectively on the Hilbert spaces $L^{2}((0,+\infty ),ds)$ and $L^{2}((-\infty ,0),ds)$. They are both defined by the formula
\[
\xymatrix{
\pi _{\pm}(a,b)f\,:\,s\ar@{|->}[r]&\sqrt{a}\,e^{2i\pi bs}f(as)
}
\] where $(a,b)\in G$, and $f\in L^{2}((0,+\infty ),ds)$ in the case of $\pi _{+}$, and  $f\in L^{2}((-\infty ,0),ds)$ in the case of $\pi _{-}$. It is a direct consequence of the Riemann-Lebesgue lemma that the analogue of Fact \ref{Fact E} holds true for the two representations $\pi _{+}$ and $\pi _{-}$ of $G$:
\begin{fact}\label{Fact M}
 For every $f\in L^{2}((0,+\infty ),ds)$ and every $g\in L^{2}((-\infty ,0),ds)$, we have
\[
\xymatrix{{\ensuremath{{\langle {\pi _{+}(a,b)f},{f}\rangle}}}\ar[r]&0}\quad \textrm{and}\quad \xymatrix{{\ensuremath{{\langle {\pi _{-}(a,b)g},{g}\rangle}}}\ar[r]&0}\quad\textrm{as}\quad
\xymatrix{|b|\ar[r]&+\infty .}
\]
\end{fact}
\par\smallskip 
-- the representations of $G$ belonging to the family $(\mathcal{F}_{2})$ are one-dimensional. They are parametrized by ${\ensuremath{\mathbb R}}$, and $\pi _{\lambda }$ is defined for every $\lambda \in{\ensuremath{\mathbb R}}$ by the formula
\[
\pi _{\lambda }(a,b)=a^{i\lambda }\quad\textrm{for every}\ (a,b)\in G.
\]
Proceeding as in the proof of Theorem \ref{Proposition F}, we characterize the {Kazhdan}\ subsets of the $ax+b$ group in the following way:
\begin{theorem}\label{Proposition N}
 Let ${Q}$ be a subset of the $ax+b$ group G. The following assertions are equivalent:
\begin{enumerate}
 \item [{(1)}] ${Q}$ is a {Kazhdan}\ set in $G$;
\item[{(2)}] the set ${Q}_{0}=\{t\in{\ensuremath{\mathbb R}}\,;\,\exists\,b\in{\ensuremath{\mathbb R}}\ (e^{t},b)\in{Q}\}$ is a {Kazhdan}\ set in ${\ensuremath{\mathbb R}}$.
\end{enumerate}
\end{theorem}

\begin{proof}
 The proof is similar to that of Theorem \ref{Proposition F}, and we will not give it in full detail here. Let us first sketch briefly a proof of the implication $(1)\Longrightarrow(2)$. Suppose that ${Q}$ is a {Kazhdan}\ set in $G$, and let $\varepsilon >0$ be a {Kazhdan}\ constant for ${Q}$. Consider a probability measure $\sigma $ on ${\ensuremath{\mathbb R}}$ such that 
\begin{equation}\label{Equation 21}
 \sup_{t\in{Q}_{0}}\,\bigl|\widehat{\sigma }(t)-1\bigr|<\dfrac{\varepsilon ^{2}}{2}\cdot 
\end{equation}
We associate to $\sigma $ a representation $\rho $ of $G$ on $L^{2}({\ensuremath{\mathbb R}},\sigma )$ by setting, for every $(a,b)\in G$ and every $f\in L^{2}({\ensuremath{\mathbb R}},\sigma )$,
$\xymatrix{
\rho (a,b)f\,:\,s\ar@{|->}[r]&e^{i s (ln a)}f(s).
}$ Since
\[
||\rho (a,b)\mathbbm{1}-\mathbbm{1}||^{2}\le2\Bigl|\int_{\ensuremath{\mathbb R}}\bigl(e^{is(\ln a)}-1)d\sigma (s)\Bigr|\quad \textrm{for every}\ (a,b)\in G,
\]
(\ref{Equation 21}) implies that ${\displaystyle}\sup_{\{(a,b)\,;\,\ln a\in{Q}_{0}\}}\,||\rho (a,b)\mathbbm{1}-\mathbbm{1}||<\varepsilon $, i.\,e.\ ${\displaystyle}\sup_{(a,b)\in{Q}}\,||\rho (a,b)\mathbbm{1}-\mathbbm{1}||<\varepsilon $. Hence $\rho $ admits a non-zero $G$-invariant function $f\in L^{2}({\ensuremath{\mathbb R}},\sigma )$, and the same argument as in the proof of Theorem \ref{Proposition F} shows then that $\sigma (\{0\})>0$. The converse implication 
$(2)\Longrightarrow(1)$ is proved in exactly the same way as in Theorem \ref{Proposition F}, using the same modifications as those outlined above.
The group ${\ensuremath{\mathbb R}}^{2n}$ has to be replaced by the multiplicative group $((0,+\infty ),\times)$ and the analogue of Lemma \ref{Lemma G} is that {Kazhdan}\ subsets of this group contain elements of arbitrarily large absolute value. If ${Q}_{0}$ is a {Kazhdan}\ set in ${\ensuremath{\mathbb R}}$, with {Kazhdan}\ constant $\varepsilon $ small enough, the same argument as in the proof of Theorem \ref{Proposition F} (involving the same notation) shows that it suffices to prove the following statement: let $\mu $ be a finite Borel measure on a Borel space $Z$, $x=(x_{z})_{z\in Z}$ a scalar-valued function of $L^{2}(Z,\mu )$ with $\int_{Z}|x_{z}|^{2}d\mu (z)=1$, and $\pi $ a representation of $G$ of the form $\pi =\int_{Z}^{\oplus}\pi _{\lambda _{z}}d\mu (z)$ with
\[
\sup_{(a,b)\in{Q}}\bigl|1-{\ensuremath{{\langle {\pi(a,b)x},{x}\rangle}}}\bigr|=\sup_{\{a\,;\,\ln  a\,\in\,{Q}_{0}\}}\Bigl|1-\int_{Z}e^{i(\ln a)\lambda _{z}}|x_{z}|^{2}d\mu (z)\Bigr|<\varepsilon.
\]
Then the set $Z_{0}=\{z\in Z\,;\,\lambda _{z}=0\}$ satisfies $\mu (Z_{0})>0$. The proof of this statement uses the same argument as the one employed in the proof of Theorem \ref{Proposition F}. It involves the representation $\rho $ of the group $((0,+\infty ),\times)$ on $L^{2}(Z,\mu )$ defined by 
$\smash{\xymatrix{\rho (a)f:z\ar@{|->}[r]&e^{i(\ln a)\lambda _{z}}f_{z}}}$ for every $a>0$ and every $(f_{z})_{z\in Z}\in L^{2}(Z,\mu )$, and uses the obvious fact that since ${Q}_{0}$ is a {Kazhdan}\ set in ${\ensuremath{\mathbb R}}$, $\{a\,;\,\ln a\,\in\,{Q}_{0}\}$ is a {Kazhdan}\ set in $((0,+\infty ),\times)$.
\end{proof}

Facts \ref{Fact E} and \ref{Fact M} have played a crucial role in the proofs of Theorems \ref{Proposition F} and \ref{Proposition N} respectively, as they allowed us to discard all irreducible representations except the one-dimensional  inequalities of the form (\ref{Eq17}). There are locally compact groups in which all non-trivial irreducible representations have the vanishing property of the matrix coefficients stated in Facts \ref{Fact E} or \ref{Fact M}: these groups are said to have the Howe-Moore property. The study of {Kazhdan}\ sets becomes easier in such groups, and it is even possible to characterize {Kazhdan}\ sets in groups which do not have Property (T), but have the Howe-Moore property, such as $SL_{2}({\ensuremath{\mathbb R}})$. This is the object of the next section.

\subsection{{Kazhdan}\ sets in groups with the Howe-Moore property}\label{Subsection 8.3}
The Howe-Moore property was introduced by Howe and Moore in \cite{HM}: a locally compact group $G$ has the \emph{Howe-Moore property} if whenever $\pi $ is a unitary representation of $G$ on a Hilbert space $H$, the following alternative holds true:
\begin{enumerate}
 \item [--] either $\pi $ has a non-zero $G$-invariant vector;
\item[--] or whenever $x$ and $y$ are two vectors of $H$, the matrix coefficient ${\ensuremath{{\langle {\pi (g)x},{y}\rangle}}}$ vanishes at infinity in the sense that for any $\delta >0$, the set 
$\{g\in G\,;\,
\bigl|{\ensuremath{{\langle {\pi (g)x},{y}\rangle}}}\bigr|\ge \delta \}$ is compact in $G$.
\end{enumerate}
It was proved in \cite{HM} and \cite{Z} that connected, non-compact, simple real Lie groups with finite center have the Howe-Moore property. We refer the reader to the references \cite{BM}, \cite{C}, and \cite{CdHC} for more on this topic. If the group $G$ is locally compact and second countable, it follows from the decomposition of unitary representations as direct integrals of irreducible representations that $G$ has the Howe-Moore property if and only if all matrix coefficients of non-trivial irreducible representations vanish at infinity.
\par\smallskip
The following proposition states that all subsets with non-compact closure are {Kazhdan}\ sets in groups with the Howe-Moore property. This testifies of the rigidity of the structure of such groups, and stands in sharp contrast with all the examples we have presented until now.
\begin{proposition}\label{Proposition H}
 Let $G$ be a locally compact group with the Howe-Moore property, and let ${Q}$ be a subset of $G$. If the closure ${\overline{Q}}$ of ${Q}$ in $G$ is non-compact, ${Q}$ is a {Kazhdan}\ set in $G$.
\end{proposition}
\begin{proof}
 Let $\varepsilon \in(0,2)$, and let $\pi $ be a representation of $G$ on a Hilbert space $H$ which admits a $({Q},\varepsilon )$-invariant vector $x\in H$, which we suppose to be of norm $1$. Then 
\[
\sup_{g\in{Q}}\,\bigl(1-\Re e {\ensuremath{{\langle {\pi (g)x},{x}\rangle}}}\bigr)<\dfrac{\varepsilon }{2}, \quad\textrm{so that}\
\bigl|{\ensuremath{{\langle {\pi (g)x},{x}\rangle}}}\bigr|>1-\dfrac{\varepsilon }{2}\quad \textrm{for every}\ g\in{Q}.
\]
Suppose that $\pi $ has no non-zero $G$-invariant vector. Since $G$ has the Howe-Moore property, and $1-\varepsilon /2>0$, the set $\{g\in\ G\,;\,\bigl|{\ensuremath{{\langle {\pi (g)x},{x}\rangle}}}\bigr|\ge 1-\varepsilon /2\}$ is compact. It contains ${Q}$, hence ${\overline{Q}}$, so ${\overline{Q}}$ is compact. It follows that if ${\overline{Q}}$ is not compact, $({Q},\varepsilon )$ is a {Kazhdan}\ pair in $G$ for any $\varepsilon \in(0,2)$.
\end{proof}
As a direct consequence of Proposition \ref{Proposition H}, we obtain the following characterization of {Kazhdan}\ sets in groups without Property (T), but which have the Howe-Moore property.
\begin{theorem}\label{Proposition I}
 Let $G$ be a locally compact group with the Howe-Moore property, but without Property (T). A subset ${Q}$ of $G$ is a {Kazhdan}\ set in $G$ if and only if ${\overline{Q}}$ is not compact.
\end{theorem}

\begin{proof}
 We only have to prove that if ${\overline{Q}}$ is compact, ${Q}$ is not a {Kazhdan}\ set in $G$. This follows directly from the observation that since unitary representations are strongly continuous,  ${Q}$ is a {Kazhdan}\ set in $G$ as soon as  ${\overline{Q}}$ is a {Kazhdan}\ set in $G$. Since $G$ does not have Property (T), it admits no compact {Kazhdan}\ set, and hence a set with compact closure is not a {Kazhdan}\ set.
\end{proof}
As $SL_{2}({\ensuremath{\mathbb R}})$ is a non-compact connected simple real Lie group with finite center, it has the Howe-Moore property. But it does not have Property (T), and so we have:
\begin{example}\label{Example J}
 The {Kazhdan}\ sets in $SL_{2}({\ensuremath{\mathbb R}})$ are exactly the subsets of $SL_{2}({\ensuremath{\mathbb R}})$ with non-compact closure.
 \end{example}

\section{Further applications of {Kazhdan}\ sets}\label{Section 8}
\subsection{{Kazhdan}\ sets and local rigidity}\label{sect:rapinchuk}  
It was proved by Rapinchuk \cite{rapin} that finite dimensional 
unitary representations of a discrete group $G$ with Property (T) are \emph{locally rigid} in the following
sense: let $S$ be a finite set of generators of $G$ (recall that discrete groups with Property (T) are finitely generated, and that the {Kazhdan}\ subsets of such groups are the generating sets), and let $d\ge 1$ be an integer. If two $d$-dimensional unitary representations of $G$ are sufficiently close 
on $S$, they are unitarily equivalent. 
A generalization of this result is given in the next theorem, following an idea from \cite{BOTulam}.
\begin{theorem}\label{thm:rapinchuk}
Let $G$ be a topological group, and let $({Q},\varepsilon)$ be a Kazhdan pair for $G$. Let $d\ge 1$ be an integer. 
For every $\alpha > 0$, there exists $\beta > 0$ such that
the following property holds true: for 
all unitary representations $\pi_1$ and $\pi_2$ of $G$ into a $d$-dimensional Hilbert space $H$ with
$$ \sup_{g\in Q} \|\pi_1(g) - \pi_2(g)\| < \beta ,$$
there exists a unitary operator $U$ on $H$ such that $\|U-I\| < \alpha$ 
and $\pi_2(g) = U^*\pi_1(g)U$ for every $g\in G$. 
If $0<\alpha \le 2$, we can choose $\beta = \alpha \varepsilon/(2\sqrt{d})$.
\end{theorem}
\begin{proof}
Without loss of generality we can assume that $0<\alpha \le 2$. 
Set $\beta = \alpha \varepsilon/(2\sqrt{d})$ and let $\pi_1$ and $\pi_2$ be two representations of $G$ on a $d$-dimensional Hilbert space $H$ such 
that $\sup_{g\in Q}  \|\pi_1(g) - \pi_2(g)\| < \beta$. Consider 
the unitary representation $\rho$ of $G$ into the space $HS(H)$ of Hilbert-Schmidt operators on $H$ given by
$ \rho(g)(A) = \pi_1(g)A\pi_2(g)^{-1}$ for every $g\in G$ and every $ A \in HS(H)$.
Denote by $|\cdot|_{HS}$ the Hilbert-Schmidt norm on $HS(H)$
and by $I$ the identity operator on $H$.
Then
\[
\sup_{g\in Q}\left|\rho(g)(I)-I\right|_{HS} = \sup_{g\in Q}\left|\pi_1(g)-\pi_2(g)\right|_{HS} \le  \sqrt{d}\,\sup_{g\in Q}\|\pi_1(g)-\pi_2(g)\| < \sqrt{d}\beta .
\]
Therefore
$$ \sup_{g\in Q} |\rho(g)(d^{-1/2}I) - d^{-1/2}I|_{HS} < \beta 
= \left(\frac{\alpha}{2\sqrt{d}}\right)\varepsilon .$$
As $({Q},\varepsilon)$ is a Kazhdan pair for $G$, it follows 
from \cite[Prop.~1.1.9]{BdHV} that there exists 
 a $G$-invariant vector $L' \in HS(H)$ for $\rho$ which 
is $\left(\frac{\alpha}{2\sqrt{d}}\right)$-close to $d^{-1/2}I$. 
Then $L = \sqrt{d}L'$ is a $G$-invariant vector for $\rho$ satisfying
\begin{equation}\label{eq:LminusI}
\|L-I\| \le \left|L-I\right|_{HS} < \frac{\alpha}{2}\cdot
\end{equation}
Since $\alpha \le 2$, the operator $L$ is invertible. The equation $\rho(g)(L) = L$ for every $g\in G$ implies that
\begin{equation}\label{eq:qqch}
 \pi_2(g) = L^{-1}\pi_1(g)L\quad  \textrm{ for every } g\in G .
\end{equation}
Consider now  the polar decomposition $L = U(L^*L)^{1/2}$ of $L$. 
Using Equation \eqref{eq:LminusI} and \cite[Lem.~4.5]{BOTulam} we obtain that
$\|U-I\| < \alpha$.
The fact that $\pi_2(g) = U^{*}\pi_1(g)U$ for every $g\in G$ is a consequence of Equation (\ref{eq:qqch}) and of the fact that $(L^*L)^{1/2}$ is a norm limit of polynomials in $L^*L$.
\end{proof} 

\subsection{{Kazhdan}\ sets and asymptotically invariant sequences}\label{Section 8.2}
Connes and Weiss provide (\!\!\!\cite{CW}, see also \cite{Sch}) a characterization of Property (T) for  countable groups in terms of ergodicity properties of actions of the group on probability spaces. 
This result is extended in \cite[Th.~6.3.4]{BdHV} to second countable locally compact groups, and a simplified proof is given there. 
If $G$ is such a group, and $(\Omega ,\mathcal{F},\nu  )$  is a probability space, a measure-preserving action 
$(T_{g})_{g\in G}$ of $G$ on $(\Omega ,\mathcal{F},\nu  )$ gives rise to a unitary representation $\pi _{\nu }$ of $G$ on $L^{2}(\Omega ,\mathcal{F},\nu  )$ defined by setting 
$\pi _{\nu }(g)f(\omega )=f(T_{g^{-1}}\omega )$ for every $f\in L^{2}(\Omega ,\mathcal{F},\nu  )$, $g\in G$ and $\omega \in \Omega $. The restriction of $\pi _{\nu }$ to the space $L^{2}_{0}(\Omega ,\mathcal{F},\nu  )$ of functions of $L^{2}(\Omega ,\mathcal{F},\nu  )$ of mean zero is denoted by $\pi _{\nu }^{0}$.
\par\smallskip 
The Connes-Weiss characterization of Property (T) is given in terms of \emph{asymptotically invariant sequences} $(A_{n})_{n\ge 1}$ of sets for measure preserving actions of $G$. Given such an action $(T_g)_{g\in G}$ of $G$  on a probability space $(\Omega ,\mathcal{F},\nu  )$, these are the sequences $(A_{n})_{n\ge 1}$ of elements of $\mathcal{F}$ such that for every compact subset ${Q}$ of $G$,
\begin{equation}\label{AS}
 \lim_{n\to+\infty }\sup_{g\in{Q}}\nu (T_g A_{n}\bigtriangleup  A_{n})=0.
\end{equation}
Such a sequence $(A_{n})_{n\ge 1}$ of sets is non-trivial if $\inf_{n\ge 1}\nu (A_{n})(1-\nu (A_{n}))>0$. A measure-preserving action of $G$ is said to be \emph{strongly ergodic} if it admits no non-trivial asymptotically invariant sequence.
\begin{theorem}[\!\!\cite{CW}]\label{OOO} 
 Let $G$ be a second countable locally compact group. The following assertions are equivalent:
\begin{enumerate}
\item[\emph{(a)}] $G$ has Property (T);
\item[\emph{(b)}] every measure-preserving ergodic action of $G$ is strongly ergodic;
\item[\emph{(c)}] every measure-preserving weakly mixing action of $G$ is strongly ergodic.
\end{enumerate}
\end{theorem}
We refer the reader to \cite[Sec.~6.3]{BdHV} or \cite[Ch. 13]{Gl} for more on the consequences of Property (T) on the measure-preserving actions of the group. Our aim here is to present a last consequence of Theorem \ref{Theorem 0}, which is a Connes-Weiss type characterization of  {Kazhdan}\ subsets of the group.
\par\smallskip  
Consider a measure-preserving action $(T_g)_{g\in G}$ of $G$ on a probability space $(\Omega ,\mathcal{F},\nu  )$, and let ${Q}$ be a subset of $G$. A sequence $(A_{n})_{n\ge 1}$ of elements of $\mathcal{F}$ is said to be \emph{${Q}$-asymptotically invariant} for this action if (\ref{AS}) holds true for this particular set ${Q}$.
We will call the action $(T_g)_{g\in G}$ \emph{${Q}$-strongly-ergodic} if it has no non-trivial ${Q}$-asymptotically invariant sequence.
\begin{theorem}\label{Theorem AA}
 Let $G$ be a second countable locally compact group, and let ${Q}$ be a subset of $G$ which generates $G$. The following assertions are equivalent:
\begin{enumerate}
 \item [\emph{(a)}]${Q}$ is a {Kazhdan}\ set in $G$;
\item[\emph{(b)}] every measure-preserving ergodic action of $G$ is ${Q}$-strongly ergodic;
\item[\emph{(c)}] every measure-preserving weakly mixing action of $G$ is ${Q}$-strongly ergodic.
\end{enumerate}
\end{theorem}
\begin{proof}
 The proof of Theorem \ref{Theorem AA} follows in spirit that of the Connes-Weiss characterization of Property (T) given in \cite[Th.~6.3.4]{BdHV}, and so we will content ourselves with highlighting the main differences between the two proofs.
The proof of $\textrm{(a)}\Longrightarrow\textrm{(b)}$ is the consequence of the following generalization of \cite[Prop.~6.3.2]{BdHV} and the fact that if ${Q}$ is a {Kazhdan}\ set in $G$, representations of $G$ with ${Q}$-almost-invariant vectors have non-zero  $G$-invariant vectors:

\begin{fact}\label{Fact BB}
 Let $G$ be a second countable locally compact group, and let ${Q}$ be a subset of $G$. If a measure-preserving action of $G$ on a probability space $(\Omega ,\mathcal{F},\nu )$ admits a non-trivial ${Q}$-asymptotically invariant sequence, then $\pi _{\nu }^{0}$ has ${Q}$-almost-invariant vectors.
\end{fact}

The implication $\textrm{(b)}\Longrightarrow\textrm{(c)}$ is obvious, and we concentrate on the proof of the implication $\textrm{(c)}\Longrightarrow\textrm{(a)}$. Suppose that assumption (c) holds true, but that ${Q}$ is not a {Kazhdan}\ set in $G$. By Corollary \ref{CorollaryA}, there exists a unitary representation $\pi $ of $G$ on a Hilbert space $H$ (which, as $G$ is second countable, can be assumed to be separable) which has ${Q}$-almost-invariant vectors but no finite dimensional subrepresentation. Let $H_{\ensuremath{\mathbb R}}$ denote the space $H$ viewed as a real Hilbert space with the scalar product defined by 
${\ensuremath{{\langle {x},{y}\rangle}}}_{\ensuremath{\mathbb R}}=\Re e{\ensuremath{{\langle {x},{y}\rangle}}}$ for every $x,y\in H$. The representation $\pi $ acting on $H_{\ensuremath{\mathbb R}}$, denoted by $\pi _{\ensuremath{\mathbb R}}$, is an orthogonal representation of $G$. It also has ${Q}$-invariant vectors but no finite dimensional subrepresentation (if $E$ is a finite dimensional subspace of $H_{\ensuremath{\mathbb R}}$ which is invariant by the representation $\pi _{\ensuremath{\mathbb R}}$, 
$\widetilde{E}=\{\lambda x\, ;\,x\in E,\lambda \in{\ensuremath{\mathbb C}}\}$ is a finite dimensional subspace of $H$ which is invariant by $\pi $).
\par\smallskip 
We proceed now as in the proof of \cite[Th.~6.3.4]{BdHV}. Let $K$ be an infinite dimensional Gaussian subspace of the space $L^{2}_{\ensuremath{\mathbb R}}(\Omega ,\mathcal{F},\nu )$ of real-valued square integrable functions on the probability space $(\Omega ,\mathcal{F},\nu )$. Let $\smash[t]{{{\phi }:
\xymatrix{{H_{\ensuremath{\mathbb R}}}\ar[r]&{K}}
}}$ be an isometric isomorphism, and let 
$\smash{{{\widetilde{\phi }}:
\xymatrix{{S(H_{\ensuremath{\mathbb R}})}\ar[r]&{L^{2}_{\ensuremath{\mathbb R}}(\Omega ,\mathcal{F},\nu )}}
}}$ be its extension on the symmetric Fock space of $H_{\ensuremath{\mathbb R}}$. It is an isometric isomorphism, and there exists a measure-preserving action $(T_{g})_{g\in G}$ of $G$ on $(\Omega ,\mathcal{F},\nu )$ such that 
\[
\pi _{\nu ,{\ensuremath{\mathbb R}}}(g)\widetilde{\phi }=\widetilde{\phi }\ \bigoplus_{n\ge 0}S^{n}(\pi _{\ensuremath{\mathbb R}})(g)\ \textrm{for every}\ g\in G,
\]
where $\pi _{\nu ,{\ensuremath{\mathbb R}}}$ denotes the orthogonal representation of $G$ on $L^{2}_{\ensuremath{\mathbb R}}(\Omega ,\mathcal{F},\nu )$ associated to the action $(T_{g})_{g\in G}$. Since $\pi _{\ensuremath{\mathbb R}}$ has no finite dimensional subrepresentation, the same argument as in \cite[Th.~6.3.4]{BdHV} shows that $(T_{g})_{g\in G}$  is a weakly mixing  action of $G$ on $(\Omega ,\mathcal{F},\nu )$.
\par\smallskip 
In order to get a contradiction, it remains to show that the action $(T_{g})_{g\in G}$ admits a non-trivial ${Q}$-asymptotically-invariant sequence. Exactly as in \cite[Th.~6.3.4]{BdHV}, consider a sequence $(\xi _{n})_{n\ge 1}$ of vectors of norm one in $H_{\ensuremath{\mathbb R}}$ such that 
\[
\xymatrix{\sup_{g\in {Q}}||\pi _{\ensuremath{\mathbb R}}(g)\xi _{n}-\xi _{n}||\ar[r]&0,}\quad \textrm{i.e.}\quad 
\xymatrix{\sup_{g\in {Q}}{\ensuremath{{\langle {\pi _{\ensuremath{\mathbb R}}(g)\xi _{n}},{\xi_{n}}\rangle}}}\ar[r]&1}\ \textrm{as}\ 
\xymatrix{n\ar[r]&+\infty.}
\]
Such a sequence of vectors exists, as $\pi _{\ensuremath{\mathbb R}}$ has ${Q}$-almost invariant vectors. For every ${n\ge 1}$, let $X_{n}=\phi (\xi _{n})$, and $A_{n}=\{X_{n}\ge 0\}$. Then $\nu (A_{n})=1/2$ for every ${n\ge 1}$. Define also, for ${n\ge 1}$ and $g\in G$, $\alpha _{n}(g)\in[0,\pi ]$ by the relation
$\cos \alpha _{n}(g)={\ensuremath{{\langle {\pi _{\ensuremath{\mathbb R}}(g)\xi _{n}},{\xi_{n}}\rangle}}}$. Then $\smash{\xymatrix{\sup_{g\in{Q}}\alpha _{n}(g)\ar[r]&0}}$ as 
$\smash{\xymatrix{n\ar[r]&+\infty .}}$
As $\nu (T_{g}A_{n}\bigtriangleup A_{n})=\alpha _{n}(g)/\pi $ for every ${n\ge 1}$ and every $g\in G$ (see \cite[Th.~6.3.4]{BdHV} for details), $(A_{n})_{n\ge 1}$ is a ${Q}$-asymptotically invariant sequence for $(T_{g})_{g\in G}$, which is non-trivial since $\nu (A_{n})=1/2$ for every ${n\ge 1}$. We have thus constructed a weakly mixing action of $G$ which is not ${Q}$-strongly ergodic, which contradicts assumption (c).
\end{proof}
\begin{remark}
 Exactly the same proof shows that if $({Q}_{k})_{k\ge 1}$ is an increasing sequence of subsets of $G$ such that, for every $k\ge 1$, ${Q}_{k}$ generates $G$ and ${Q}_{k}$ is not a {Kazhdan}\ set in $G$, there exists a weakly mixing action of $G$ on a probability space $(\Omega ,\mathcal{F},\nu )$ and a non-trivial sequence $(A_{n})_{n\ge 1}$ of elements of $\mathcal{F}$ which is, for each $k\ge 1$, ${Q}_{k}$-asymptotically invariant for this action. 
\end{remark}

 \appendix
\section{Infinite tensor products of Hilbert spaces}
We  briefly describe in this appendix some constructions of tensor 
products of infinite families of Hilbert spaces, and of tensor products of infinite families of unitary representations. These last objects play an important role in the proof of Theorem \ref{Theorem 0}. We review here the properties and results which we need, following the original works of von Neumann \cite{VN} and Guichardet \cite{Gui}.
\subsection{The complete and incomplete tensor products of Hilbert 
spaces}
The original construction of the complete and incomplete tensor products 
of a family $(H_{\alpha })_{\alpha \in I}$ of Hilbert spaces is due to von 
Neumann \cite{VN}. It was later on taken up by Guichardet in \cite{Gui} 
under a somewhat different point of view, and the incomplete tensor 
products of von Neumann are rather known today as the Guichardet tensor 
products of Hilbert spaces. Although these constructions can be carried 
out starting from an arbitrary  family $(H_{\alpha })_{\alpha \in I}$ of 
Hilbert spaces, we will present them here only in the case of a countable 
family $({H_{n}})_{n\ge 1}$ of (complex) Hilbert spaces.
\par\smallskip 
The \emph{complete infinite tensor product} ${\bigotimes_{n\ge 1}^{{}}{H_{n}}}$ 
of the Hilbert spaces ${H_{n}} $ is defined in \cite[Part~II,~Ch.~3]{VN} in 
the following way: the elementary infinite tensor products are the elements
${\mkern 1.5 mu\pmb{x}}={{\otimes}_{n\ge 1}{x_{n}}}$, where ${x_{n}}$ belongs to ${H_{n}} $ for each 
${n\ge 1}$  and the infinite product $\prod_{n\ge 1}||{x_{n}}||$ is convergent 
in the sense of \cite[Def.~2.2.1]{VN}, which by \cite[Lem.~2.4.1]{VN} is equivalent to the fact that either $x_{n}=0$ for some $n\ge 1$ or the series $\sum_{n\ge 1}\max(||x_{n}||-1,0)$ is convergent. Sequences 
${({x_{n}})_{n\ge 1}}$ with this property are called by von Neumann in 
\cite{VN} \emph{$C$-sequences}. A scalar product is then defined on the 
set of finite linear combinations of elementary tensor products by setting 
\[
{\ensuremath{{\langle {\mkern 1.5 mu\pmb{x}},{\mkern 1.5 mu\pmb{y}}\rangle}}}=\prod_{n\ge 1}{\ensuremath{{\langle {x_{n}},{y_{n}}\rangle}}}
\]
for any elementary tensor products ${\mkern 1.5 mu\pmb{x}}={{\otimes}_{n\ge 1}{x_{n}}}$ and 
${\mkern 1.5 mu\pmb{y}}={{\otimes}_{n\ge 1}{y_{n}}}$, and extending the definition by linearity to 
finite linear combinations of such elements.
The product defining ${\ensuremath{{\langle {\mkern 1.5 mu\pmb{x}},{\mkern 1.5 mu\pmb{y}}\rangle}}}$ for two elementary vectors ${\mkern 1.5 mu\pmb{x}}$ and ${\mkern 1.5 mu\pmb{y}}$ is quasi-convergent in the sense of \cite[Def.~2.5.1]{VN}, i.e. $\prod_{n\ge 1}|{\ensuremath{{\langle {x_{n}},{y_{n}}\rangle}}}|$ is convergent. The value of this quasi-convergent product is $\prod_{n\ge 1}{\ensuremath{{\langle {x_{n}},{y_{n}}\rangle}}}$ if the product is convergent in the usual sense, and $0$ if it is not.
\par\smallskip 
That this is indeed a scalar product which turns the set of finite linear 
combinations of elementary tensor products into a complex prehilbertian 
space is proved in \cite[Lem.~3.21 and Theorem II]{VN}. 
For any elementary tensor product ${\mkern 1.5 mu\pmb{x}}={{\otimes}_{n\ge 1}{x_{n}}}$, 
$||{\mkern 1.5 mu\pmb{x}}||=\prod_{n\ge 1}||{x_{n}}||$. The space ${\bigotimes_{n\ge 1}^{{}}{H_{n}}}$ is 
the completion of this space for the topology induced  by the scalar 
product. It is always non-separable.
\par\smallskip 
The \emph{incomplete tensor products} are closed subspaces of the complete 
tensor product. They are defined by von Neumann using an equivalence 
relation between sequences ${({x_{n}})_{n\ge 1}}$ of vectors with ${x_{n}}\in
{H_{n}} $ for each ${n\ge 1}$ and such that the series 
$\sum_{n\ge 1}\bigl|1-||{x_{n}}||\bigr|$ is convergent. Such sequences are called 
\emph{$C_{0}$-sequences}. They are $C$-sequences, and if 
${({x_{n}})_{n\ge 1}}$ is a $C$-sequence such that $\prod_{n\ge 1}||{x_{n}}||>0$ (i.\,e.\ 
${({x_{n}})_{n\ge 1}}$ is non-zero in ${\bigotimes_{n\ge 1}^{{}}{H_{n}}}$) then 
${({x_{n}})_{n\ge 1}}$ is a $C_{0}$-sequence. If ${({x_{n}})_{n\ge 1}}$ is a $C_{0}$-sequence, ${({x_{n}})_{n\ge 1}}$ is 
bounded, and the series $\sum_{n\ge 1}\bigl|1-||{x_{n}}||^{2}\bigr|$ is 
convergent.
\par\smallskip 
Two $C_{0}$-sequences ${({x_{n}})_{n\ge 1}}$ and ${({y_{n}})_{n\ge 1}}$ are \emph{equivalent} if the 
series $\sum_{n\ge 1}\bigl|1-{\ensuremath{{\langle {x_{n}},{y_{n}}\rangle}}}\bigr|$ is convergent. If 
${\mathcal{A}}$ denotes an equivalence class of $C_{0}$-sequences for this 
equivalence relation, the \emph{incomplete tensor product} 
${\bigotimes_{n\ge 1}^{{\mathcal{A}}}{H_{n}}}$ \emph{associated to} ${\mathcal{A}}$ is the 
closed linear span in ${\bigotimes_{n\ge 1}^{{}}{H_{n}}}$ of the vectors ${\mkern 1.5 mu\pmb{x}}=
{{\otimes}_{n\ge 1}{x_{n}}}$, where ${({x_{n}})_{n\ge 1}}$ belongs to ${\mathcal{A}}$ \cite[Def.~4.1.1]{VN}. 
If ${\mathcal{A}}$ and ${\mathcal{A}}'$ are two different equivalence classes, the spaces 
$\smash[t]{{\bigotimes_{n\ge 1}^{{\mathcal{A}}}{H_{n}}}}$ and $\smash[t]{{\bigotimes_{n\ge 1}^{{{\mathcal{A}}'}}{H_{n}}}}$ are orthogonal, and 
the linear span of the 
incomplete tensor products ${\bigotimes_{n\ge 1}^{{\mathcal{A}}}{H_{n}}}$, where ${\mathcal{A}}$ runs over all 
equivalence classes of $C_{0}$-sequences, is dense in the complete tensor 
product ${\bigotimes_{n\ge 1}^{{}}{H_{n}}}$.
\par\smallskip 
If ${\mathcal{A}}$ is an equivalence class of $C_{0}$-sequences, ${\bigotimes_{n\ge 1}^{{\mathcal{A}}}{H_{n}}}$ 
admits another, more transparent description, which runs as follows 
\cite[Lem.~4.1.2]{VN}, see also \cite[Rem.~1.1]{Gui}: let ${({a_{n}})_{n\ge 1}}$ be a 
sequence with ${a_{n}}\in {H_{n}} $ and $||{a_{n}}||=1$ for every ${n\ge 1}$, such that the equivalence class of 
${({a_{n}})_{n\ge 1}}$ is ${\mathcal{A}}$ (such a sequence ${({a_{n}})_{n\ge 1}} $ does exist: if ${({x_{n}})_{n\ge 1}} $ is any non-zero $C_{0}$-sequence belonging to ${\mathcal{A}}$, $x_{n}$ is non-zero for every $n\ge 1$, and we can define a $C_{0}$-sequence ${({a_{n}})_{n\ge 1}} $ by setting $a_{n}=x_{n}/||x_{n}||$ for every $n\ge 1$. It is not difficult to check that ${({a_{n}})_{n\ge 1}} $ is equivalent to ${({x_{n}})_{n\ge 1}} $, and so belongs to ${\mathcal{A}}$). Then ${\bigotimes_{n\ge 1}^{{\mathcal{A}}}{H_{n}}}$ coincides with the closed linear span 
in ${\bigotimes_{n\ge 1}^{{}}{H_{n}}}$ of vectors ${\mkern 1.5 mu\pmb{x}}={{\otimes}_{n\ge 1}{x_{n}}}$, where ${x_{n}}={a_{n}}$ for all 
but finitely many  integers ${n\ge 1}$. Denoting the vector ${{\otimes}_{n\ge 1}{a_{n}}}$ by ${\mkern 1.5 mu\pmb{a}}$, 
we write this closed linear span as ${\bigotimes_{n\ge 1}^{\mkern 1.5 mu\pmb{a}}{H_{n}}}$ (see \cite{Gui}), and 
thus ${\bigotimes_{n\ge 1}^{\mkern 1.5 mu\pmb{a}}{H_{n}}}={\bigotimes_{n\ge 1}^{{\mathcal{A}}}{H_{n}}}$, where ${\mathcal{A}}$ is the equivalence class of 
${\mkern 1.5 mu\pmb{a}}$. The space ${\bigotimes_{n\ge 1}^{\mkern 1.5 mu\pmb{a}}{H_{n}}}$ is usually called \emph{the Guichardet 
tensor product} of the spaces ${H_{n}} $ \emph{associated to the sequence} 
${({a_{n}})_{n\ge 1}}$. Proposition 1.1 of \cite{Gui} states the following, which is a direct consequence of
the discussion above: if ${\mkern 1.5 mu\pmb{x}}={({x_{n}})_{n\ge 1}}$ is a $C_{0}$-sequence 
which is equivalent to ${\mkern 1.5 mu\pmb{a}}$, ${\mkern 1.5 mu\pmb{x}}$ belongs to ${\bigotimes_{n\ge 1}^{\mkern 1.5 mu\pmb{a}}{H_{n}}}$. Vectors 
${\mkern 1.5 mu\pmb{x}}$ of this form are also said to be \emph{decomposable with respect 
to} ${\mkern 1.5 mu\pmb{a}}$, while vectors ${\mkern 1.5 mu\pmb{x}}={({x_{n}})_{n\ge 1}}$ with ${x_{n}}={a_{n}}$ for all but finitely 
many indices $n$ are called \emph{elementary vectors} of ${\bigotimes_{n\ge 1}^{\mkern 1.5 mu\pmb{a}}{H_{n}}}$.
\par\smallskip
Suppose that all the spaces ${H_{n}}$, ${n\ge 1}$, are separable.
For each ${n\ge 1}$, let $(e_{p,n})_{1\le p\le p_{n}}$ be a Hilbertian basis of 
${H_{n}} $, with $1\le p_{n}\le+\infty $ and $e_{1,n}=a_{n}$. The family of all
elementary vectors ${\mkern 1.5 mu\pmb{e}}_{\beta }={\otimes}_{n\ge 1}e_{\beta (n),n}$ of ${\bigotimes_{n\ge 1}^{\mkern 1.5 mu\pmb{a}}{H_{n}}}$, where 
$\beta $ is a map from ${\ensuremath{\mathbb N}}$ into itself such that $1\le\beta (n)\le p_{n}$ 
for every ${n\ge 1}$ and $\beta (n)=1$ for all but finitely many integers $n\ge 
1$, forms a Hilbertian basis of ${\bigotimes_{n\ge 1}^{\mkern 1.5 mu\pmb{a}}{H_{n}}}$ \cite[Lem.~4.1.4]{VN}. In 
particular, ${\bigotimes_{n\ge 1}^{\mkern 1.5 mu\pmb{a}}{H_{n}}}$ is a separable complex Hilbert space.

\subsection{Tensor products of unitary representations} 
Let $G$ be a topological group, and let $({H_{n}})_{n\ge 1}$ be a sequence of 
complex separable Hilbert spaces. Let ${({a_{n}})_{n\ge 1}}$ be a sequence of 
vectors with ${a_{n}}\in {H_{n}} $ and $||{a_{n}}||=1$ for every ${n\ge 1}$. We are 
looking 
for conditions under which one can define a unitary representation 
$\pmb{\pi} $ 
of $G$ on ${\bigotimes_{n\ge 1}^{\mkern 1.5 mu\pmb{a}}{H_{n}}}$ which satisfies 
\begin{equation}\label{Eq1}
 \pmb{\pi} (g){{\otimes}_{n\ge 1}{x_{n}}}={\otimes}_{n\ge 1}\pi_{n} (g){x_{n}}
\end{equation}
for every $g\in G$ and every decomposable vector ${\mkern 1.5 mu\pmb{x}}={{\otimes}_{n\ge 1}{x_{n}}}$ with respect 
to ${\mkern 1.5 mu\pmb{a}}$. 
Observe that without any assumption, the equality $\pmb{\pi} 
(g){{\otimes}_{n\ge 1}{x_{n}}}={\otimes}_{n\ge 1}\pi_{n} (g){x_{n}}$ does not make any sense, since 
$(\pi_{n} (g){x_{n}})_{n\ge 1}$, which is a $C_{0}$-sequence, may not be 
equivalent 
to ${\mkern 1.5 mu\pmb{a}}$, and thus may not belong to ${\bigotimes_{n\ge 1}^{\mkern 1.5 mu\pmb{a}}{H_{n}}}$.
\par\smallskip 
Infinite tensor products of unitary representations have already been 
studied in various contexts (see for instance \cite{BC} and the 
references therein). In   
\cite[Prop.~2.3]{BC}, the following observation is made: suppose that, 
for each ${n\ge 1}$, $U_{n}$ is a unitary 
operator on ${H_{n}}$. Then there exists a unitary operator 
$\pmb{U}={\otimes}_\gnU_{n}$ on ${\bigotimes_{n\ge 1}^{\mkern 1.5 mu\pmb{a}}{H_{n}}}$ satisfying 
\[\pmb{U}\bigl({{\otimes}_{n\ge 1}{x_{n}}} \bigr)={\otimes}_\gnU_{n}{x_{n}}\]
for every decomposable vector ${\mkern 1.5 mu\pmb{x}}={{\otimes}_{n\ge 1}{x_{n}}}$ with respect to ${\mkern 1.5 mu\pmb{a}}$ if and only if the series 
$\sum_{n\ge 1}\bigl|1-{\ensuremath{{\langle {U_{n}{a_{n}}},{a_{n}}\rangle}}} \bigr|$ is convergent (which is 
equivalent to requiring that the $C_{0}$-sequence $(U_{n}{a_{n}})_{n\ge 1}$ be 
equivalent to ${({a_{n}})_{n\ge 1}}$, i.\,e.\ to the fact that ${\otimes}_\gnU_{n}a_{n}$  be a 
decomposable vector with respect to ${\mkern 1.5 mu\pmb{a}}$). It follows from  this 
result that the formula (\ref{Eq1}) makes sense in ${\bigotimes_{n\ge 1}^{\mkern 1.5 mu\pmb{a}}{H_{n}}}$ if and 
only if the series 
\begin{equation}\label{Eq2}
 \sum_{n\ge 1}\,\bigl|1-{\ensuremath{{\langle {\pi _{n}(g)a_{n}},{a_{n}}\rangle}}}\bigr|
\end{equation}
is convergent for every $g\in G$.
Under this condition $\pmb{\pi} (g)={\otimes}_{n\ge 1}\pi _{n}(g)$ is a unitary 
operator on ${\bigotimes_{n\ge 1}^{\mkern 1.5 mu\pmb{a}}{H_{n}}}$ for every 
$g\in 
G$, and $\pmb{\pi} (gh)=\pmb{\pi} 
(g)\,\pmb{\pi} (h)$ for 
every $g,h\in G$.
\par\smallskip 
If the group $G$ is discrete, this tensor product representation is of 
course automatically strongly continuous. It is also the case if $G$ is 
supposed to be locally compact. 
\begin{proposition}\label{Proposition 3.2.0}
 Suppose that $G$ is locally compact. If the series 
\[
 \sum_{n\ge 1}|1-{\ensuremath{{\langle {\pi 
 _{n}(g){a_{n}} },{a_{n}}\rangle}}}|
\] 
 is convergent for every $g\in G$, then $\pmb{\pi} 
={\otimes}_{n\ge 1}\pi _{n}$ is strongly continuous, and is hence a unitary 
representation of $G$ on ${\bigotimes_{n\ge 1}^{\mkern 1.5 mu\pmb{a}}{H_{n}}}$.
\end{proposition}
\begin{proof}
 Since all the spaces ${H_{n}} $, ${n\ge 1}$, are separable, ${\bigotimes_{n\ge 1}^{\mkern 1.5 mu\pmb{a}}{H_{n}}}$ is 
separable too, and by \cite[Lem.~A.6.2]{BdHV} it suffices to show that 
$\smash{\xymatrix{g\ar@{|->}[r]&{\ensuremath{{\langle {\pmb{\pi} (g)\,\pmb{\xi} },{\pmb{\xi} 
}\rangle}}}}}$ 
is a 
measurable 
map from $G$ into ${\ensuremath{\mathbb C}}$ for every vector $\pmb{\xi} \in{\bigotimes_{n\ge 1}^{\mkern 1.5 mu\pmb{a}}{H_{n}}}$ 
(measurability 
refers to the Haar measure on $G$). Since the linear span of the elementary vectors is dense in
${\bigotimes_{n\ge 1}^{\mkern 1.5 mu\pmb{a}}{H_{n}}}$, standard arguments show that it 
suffices to prove this for elementary vectors ${\mkern 1.5 mu\pmb{x}}={\otimes}_{n\ge 1}{x_{n}}$ of 
${\bigotimes_{n\ge 1}^{\mkern 1.5 mu\pmb{a}}{H_{n}}}$. 
Since 
each  map
$\smash{\xymatrix{g\ar@{|->}[r]&{\ensuremath{{\langle {\pi_{n} (g){x_{n}} },{x_{n}}\rangle}}}}}$ is 
continuous 
on 
$G$, it is clear that $\smash{\xymatrix{\!\!g\ar@{|->}[r]&{\ensuremath{{\langle {\pmb{\pi} 
(g){\mkern 1.5 mu\pmb{x}} 
},{\mkern 1.5 mu\pmb{x}}\rangle}}}}
=\prod_{n\ge 1}{\ensuremath{{\langle {\pi _{n}(g){x_{n}}},{x_{n}}\rangle}}}}$ is measurable on $G$.
\end{proof}
In the general case one 
needs 
to impose  an additional condition on the representations $\pi _{n}$ and on the vectors $a_{n}$ in order that $\pmb{\pi} $ be a strongly 
continuous representation of $G$ on ${\bigotimes_{n\ge 1}^{\mkern 1.5 mu\pmb{a}}{H_{n}}}$.
\begin{proposition}\label{Proposition 3.2} 
 Suppose that the series $\sum_{n\ge 1}\,\bigl|1-{\ensuremath{{\langle {\pi 
_{n}(g)a_{n}},{a_{n}}\rangle}}}\bigr|$ is convergent for every $g\in G$ and that the 
function 
\[\xymatrix{g\ar@{|->}[r]&{\displaystyle}\sum_{n\ge 1}\,\bigl|1-{\ensuremath{{\langle {\pi 
_{n}(g)a_{n}},{a_{n}}\rangle}}}\bigr|}\] 
is continuous on a neighborhood of the 
identity element $e$ of $G$. Then $\pmb{\pi} ={\otimes}_{n\ge 1}\pi _{n}$ is 
strongly 
continuous, and is hence a unitary representation of $G$ on ${\bigotimes_{n\ge 1}^{\mkern 1.5 mu\pmb{a}}{H_{n}}}$.
\end{proposition}

\begin{proof}[Proof of Proposition \ref{Proposition 3.2}]
Since the linear span of the elementary vectors is dense in ${\bigotimes_{n\ge 1}^{\mkern 1.5 mu\pmb{a}}{H_{n}}}$, 
and the operators $\pmb{\pi }(g)$, $g\in G$, are unitary, it suffices to 
prove that the map 
${\smash{\xymatrix{g\ar@{|->}[r]&\pmb{\pi}(g){\mkern 1.5 mu\pmb{x}}}}}$ is continuous at $e$ 
for every elementary vector ${\mkern 1.5 mu\pmb{x}}={\otimes}_{n\ge 1}{x_{n}}$ of norm $1$ of 
${\bigotimes_{n\ge 1}^{\mkern 1.5 mu\pmb{a}}{H_{n}}}$. Let $N\ge 1$ be such that ${x_{n}}={a_{n}}$ for every 
$n>N$. We have for every $g\in G$:
\begin{align*}
 ||\pmb{\pi }(g){\mkern 1.5 mu\pmb{x}}-{\mkern 1.5 mu\pmb{x}}||^{2}&=2(1-\Re e{\ensuremath{{\langle {\pmb{\pi }(g){\mkern 1.5 mu\pmb{x}}},{\mkern 1.5 mu\pmb{x}}\rangle}}})=
 2\biggl(1-\prod_{n\ge 1}\Re e{\ensuremath{{\langle {\pi 
_{n}(g)\dfrac{x_{n}}{||{x_{n}}||}},{\dfrac{x_{n}}{||{x_{n}}||}}\rangle}}}\biggr)
\intertext{since $||{\mkern 1.5 mu\pmb{x}}||={\displaystyle}\prod_{n\ge 1}{||{x_{n}}||}$=1. Thus}
||\pmb{\pi }(g){\mkern 1.5 mu\pmb{x}}-{\mkern 1.5 mu\pmb{x}}||^{2}&\le 2\sum_{n\ge 1}\,\Bigl| 1-{\ensuremath{{\langle {\pi 
_{n}(g)\dfrac{x_{n}}{||{x_{n}}||}},{\dfrac{x_{n}}{||{x_{n}}||}}\rangle}}}\Bigr|\\
&\le 2\sum_{n=1}^{N}\,\Bigl| 1-{\ensuremath{{\langle {\pi 
_{n}(g)\dfrac{x_{n}}{||{x_{n}}||}},{\dfrac{x_{n}}{||{x_{n}}||}}\rangle}}}\Bigr|
+2\sum_{n\ge 1}|1-{\ensuremath{{\langle {\pi _{n}(g){a_{n}}},{a_{n}}\rangle}}}|.
\end{align*}
Let $\varepsilon $ be any positive number. Since the representations $\pi 
_{1},\dots,\pi _{N} $ are strongly conti\-nuous, and the map
 $\smash{\xymatrix{g\ar@{|->}[r]&\sum_{n\ge 1}|1-{\ensuremath{{\langle {\pi 
_{n}(g){a_{n}}},{a_{n}}\rangle}}}|}}$ is continuous at the point $e$, the 
quantity 
$||\pmb{\pi }(g){\mkern 1.5 mu\pmb{x}}-{\mkern 1.5 mu\pmb{x}}||^{2}$ is less than $\varepsilon ^{2}$ if $g$ 
lies in a suitable neighborhood of $e$. This proves the continuity of the 
map 
$\smash{\xymatrix{g\ar@{|->}[r]&\pmb{\pi }(g){\mkern 1.5 mu\pmb{x}}.}}$
\end{proof}

\par\smallskip 
We finish this appendix by giving a sufficient condition for an infinite 
tensor product representation on a space ${\bigotimes_{n\ge 1}^{\mkern 1.5 mu\pmb{a}}{H_{n}}}$ to be weakly 
mixing: let, for each ${n\ge 1}$, ${H_{n}}$ be a separable Hilbert space, ${a_{n}}$ a 
vector of ${H_{n}}$ with $||{a_{n}}||=1$, and 
$\pi _{n}$ a unitary representation of $G$ on $H_{n}$. 
We suppose that the assumptions of either Proposition \ref{Proposition 3.2.0} (when $G$ is locally compact) or Proposition \ref{Proposition 3.2} (in the general case) are 
satisfied, so that $\pmb{\pi} ={\otimes}_{n\ge 1}\pi _{n}$ is a unitary 
representation of $G$
on ${\bigotimes_{n\ge 1}^{\mkern 1.5 mu\pmb{a}}{H_{n}}}$. Then
\begin{proposition}\label{Proposition 4.3}
 In the case where $$\underline{\lim}_{\,n\to+\infty }m(|{\ensuremath{{\langle {\pi 
_{n}(\,\centerdot\,){a_{n}}},{a_{n}}\rangle}}}|^{2})=0,$$ the representation $\pmb{\pi} 
={\otimes}_{n\ge 1}\pi 
_{n} $ is weakly mixing.
\end{proposition}
\begin{proof}
 The proof of Proposition \ref{Proposition 4.3} relies on the same idea as 
that of Proposition \ref{Proposition 3.2}: let ${\mkern 1.5 mu\pmb{x}}={\otimes}_{n\ge 1}{x_{n}}$ and 
${\mkern 1.5 mu\pmb{y}}={\otimes}_{n\ge 1}{y_{n}}$ be two elementary vectors in ${\bigotimes_{n\ge 1}^{\mkern 1.5 mu\pmb{a}}{H_{n}}}$ with
$||{\mkern 1.5 mu\pmb{x}}||=||{\mkern 1.5 mu\pmb{y}}||=1$. We have 
\begin{align*}
 |{\ensuremath{{\langle {\pmb{\pi} (g)\,{\mkern 1.5 mu\pmb{x}}},{\mkern 1.5 mu\pmb{y}}\rangle}}}|^{2}&=\prod_{k\ge 1}\,\Bigl|\Bigl\langle 
\pi 
_{k}(g)\,\dfrac{x_{k}}{||x_{k}||},\dfrac{y_{k}}{||y_{k}||}\Bigr\rangle 
\Bigr|^{2}\le \Bigl|\Bigl\langle \pi 
_{n}(g)\,\dfrac{x_{n}}{||{x_{n}}||},\dfrac{y_{n}}{||y_{n}||}\Bigr\rangle 
\Bigr|^{2}
\intertext{for every $n\ge 1$ and every $g\in G$. But}
\Bigl|\Bigl\langle \pi 
_{n}(g)\,\dfrac{x_{n}}{||{x_{n}}||},\dfrac{y_{n}}{||{y_{n}}||}\Bigr\rangle 
\Bigr|
&\le|{\ensuremath{{\langle {\pi _{n}(g){a_{n}}},{a_{n}}\rangle}}}|+\biggl| \biggl| 
\dfrac{x_{n}}{||{x_{n}}||}-{a_{n}}\biggr| \biggr|+\biggl| \biggl|
\dfrac{y_{n}}{||{y_{n}}||}-{a_{n}}\biggr| \biggr|.\\
\intertext{Squaring and taking the mean on both sides we obtain that}
m(|{\ensuremath{{\langle {\pmb{\pi} (\,\centerdot\,){\mkern 1.5 mu\pmb{x}}},{\mkern 1.5 mu\pmb{y}}\rangle}}}|^{2})&\le 4\,m(|{\ensuremath{{\langle {\pi_{n} 
(\,\centerdot\,)\,{a_{n}}},{a_{n}}\rangle}}}|^{2})+4\,\biggl| \biggl| 
\dfrac{x_{n}}{||{x_{n}}||}-{a_{n}}\biggr| \biggr|^{2}+4\,\biggl| \biggl|
\dfrac{y_{n}}{||{y_{n}}||}-{a_{n}}\biggr| \biggr|^{2}
\end{align*}
for every $n\ge 1$. Since 
$\underline{\lim}_{\,n\to+\infty 
}m(|{\ensuremath{{\langle {\pi 
_{n}(\,\centerdot\,){a_{n}}},{a_{n}}\rangle}}}|^{2})=0$ and the two other terms are equal to 
zero for $n$ sufficiently large, $m(|{\ensuremath{{\langle {\pmb{\pi} 
(\,\centerdot\,){\mkern 1.5 mu\pmb{x}}},{\mkern 1.5 mu\pmb{y}}\rangle}}}|^{2})=0$. 
Weak mixing of $\pmb{\pi} $ now follows from standard density arguments.
\end{proof}

 
\begin{thebibliography}{99999}
{

\bibitem{AnBi} \textsc{M.~Anoussis, A.~Bisbas,}
\newblock
Continuous measures on compact Lie groups,
\newblock \emph{Ann. Inst. Fourier} {\bf 50} (2000), p.\,1277--1296. 

\bibitem{Ballo} \textsc{M.~Ballotti,}
\newblock
 Convergence rates for Wiener's theorem for contraction semigroups,
\newblock \emph{Houston J. Math.} {\bf 11} (1985), p.\,435--445. 

\bibitem{BalloGold} \textsc{M.~Ballotti, J.~Goldstein,}
\newblock
 Wiener's theorem and semigroups of operators, in:
\newblock Infinite dimensional systems (Retzhof, 1983), 
\emph{Lecture Notes in Math.} {\bf 1076} (1984), Springer, p.\,16--22.

\bibitem{BC} \textsc{E.~Bedos, R.~Conti,}
\newblock On infinite tensor products of projective unitary 
representations,
\newblock \emph{Rocky Mountain J. Math.} {\bf 34} (2004), p.\,467--493.

\bibitem{Be} \textsc{B.~Bekka,}
\newblock {Kazhdan}'s Property (T) for the unitary group of a separable Hilbert 
space, 
\newblock \emph{Geom. Func. Anal.} {\bf 13} (2003), p.\,509--520.

\bibitem{BV} \textsc{B.~Bekka, A.~Valette,}
\newblock {Kazhdan}'s Property (T) and amenable representations, 
\newblock \emph{Math. Zeit.} {\bf 212} (1993), p.\,293--299.

\bibitem{BdHV}  \textsc{B.~Bekka, P.~de~la~Harpe, A.~Valette,}
\newblock  {Kazhdan}'s Property (T),
\newblock \emph{New Mathematical Monographs} {\bf 11} (2008), Cambridge 
University Press.

\bibitem{BM}
\textsc{B.~Bekka, M.~Mayer,}
\newblock Ergodic theory and topological dynamics of group actions on homogeneous spaces,
\newblock \emph{London Mathematical Society Lecture Note Series} \textbf{269} (2000), Cambridge University Press.

\bibitem{BR} \textsc{V.~Bergelson, J.~Rosenblatt,} 
\newblock Mixing actions of groups,
\newblock \emph{Illinois J. Math.} {\bf 32} (1998), p.\,65--80.

\bibitem{BjFi} \textsc{M.~Bj\"orklund, A.~Fish,}
\newblock Continuous measures on homogenous spaces,
\newblock \emph{Ann. Inst. Fourier} {\bf 59} (2009), p.\,2169--2174.  

\bibitem{Bur} \textsc{R.~Burckel,}
\newblock Weakly almost periodic functions on semigroups, Gordon and 
Breach (1970).

\bibitem{Burg} \textsc{M.~Burger,} 
\newblock Kazhdan constants for $SL(3,{\ensuremath{\mathbb Z}})$, 
\newblock \emph{J. Reine Angew. Math.} {\bf 413} (1991), p.\,36--67.

\bibitem{BOTulam}
\textsc{M.~Burger, N.~Ozawa, A.~Thom,} 
\newblock On Ulam stability,
\newblock \emph{Israel J. Math.} {\bf 193} (2013), p.\,109--129. 

\bibitem{CdHC}
\textsc{R.~Cluckers, Y.~ de Cornulier, N.~Louvet, R.~Tessera, A.~Valette,}
\newblock The Howe-Moore property for real and p-adic groups,
\newblock \emph{Math. Scand.} \textbf{109} (2011), p.\,201--224.

\bibitem{C}
\textsc{C.~Ciobotaru,} 
\newblock A unified proof of the Howe-Moore property,
\newblock \emph{J. Lie Theory} \textbf{25} (2015), p.\,65--89.

\bibitem{CW}
\textsc{A.~Connes, B.~Weiss,}
\newblock Property (T) and asymptotically invariant sequences,
\newblock \emph{Israel J. Math.} \textbf{37} (1980), p.\,209--210.

\bibitem{Dix} \textsc{J.~Dixmier,}
\newblock Les moyennes invariantes dans les semi-groupes et leurs 
applications, 
\newblock \emph{Acta Sci. Math. (Szeged)} {\bf 12} (1950), p.\,213--227.

\bibitem{DT} 
\textsc{M.~Drmota, R.~Tichy},
\newblock Sequences, discrepancies and applications,
\newblock \emph{Lecture Notes in Mathematics} \textbf{1651} (1997), Springer-Verlag. 

\bibitem{Dye} \textsc{H.~Dye,} 
\newblock On the ergodic mixing theorem,
\newblock \emph{Trans. Amer. Math. Soc.} {\bf 118} (1965), p.\,123--130.

\bibitem{Eckm} \textsc{B.~Eckmann,} 
\newblock \"Uber monothetische Gruppen, 
\newblock \emph{Comment. Math. Helv.,} {\bf 16} (1943-44), p.\,249--263. 

\bibitem{EG}
\textsc{T.~Eisner, S.~Grivaux,}
\newblock Hilbertian Jamison sequences and rigid dynamical systems,
\newblock \emph{J. Funct. Anal.} \textbf{261} (2011), p.\,2013--2052. 

\bibitem{Farkas} \textsc{B.~Farkas,}
\newblock Wiener's lemma and the Jacobs-de Leeuw-Glicksberg decomposition,
\newblock \emph{Annales Univ. Sci. Budapest, Sec. Math.}, to appear, 
available at\\ \verb=http://www.fan.uni-wuppertal.de/en/staff/farkas/publikationen.html=

\bibitem{Fo} \textsc{G. Folland,}
\newblock A course in abstract harmonic analysis, \emph{Studies in 
Advanced Mathematics} (1995), CRC Press.

\bibitem{Gl} \textsc{E. Glasner,}
\newblock Ergodic theory via joinings, \emph{Mathematical Surveys and Monographs} \textbf{101} (2003),
American Mathematical Society.

\bibitem{GLR}
\textsc{H.~Gr\"ochenig, V.~Losert, H.~Rindler,}
\newblock Uniform distribution in solvable groups,
\newblock Probability measures on groups VIII (Oberwolfach, 1985), 
\emph{Lecture Notes in Math.} \textbf{1210} (1986),  Springer,
p.\,97--107. 

\bibitem{Gui} \textsc{A.~Guichardet,} 
\newblock Produits tensoriels infinis et repr\'esentations des relations 
d'anti\-com\-mutation,
\newblock \emph{Ann. Sci. \'E.N.S.} {\bf 83} (1966), p.\,1--52.

\bibitem{HarpeVal} \textsc{P.~de la Harpe, A.~Valette,} 
\newblock La propri\'et\'e (T) de Kazhdan pour les groupes localement compacts 
(avec un appendice de Marc Burger),
\newblock \emph{Ast\'erisque} {\bf 175} (1989), Societ\'e Math\'ematique de France. 

\bibitem{Ha1}
\textsc{S.~Hartman,} 
\newblock The method of Grothendieck-Ramirez and weak topologies in C(T), 
\newblock \emph{Studia Math.} \textbf{44} (1972), p.\,181--197. 

\bibitem{Ha2}
\textsc{S.~Hartman,} 
\newblock On harmonic separation,
\newblock \emph{Colloq. Math.} \textbf{42} (1979), p.\,209--222. 

\bibitem{HM}
\textsc{R.~Howe, C.~Moore,}
\newblock Asymptotic properties of unitary representations,
\newblock \emph{J. Funct. Anal.} \textbf{32} (1979), p.\,72--96. 

\bibitem{Kass1}
\textsc{M.~Kassabov,}
\newblock Kazhdan constants for $SL_n({\ensuremath{\mathbb Z}})$,
\newblock \emph{Int. J. Algebra Comput.} \textbf{15} (2005), p.\,971--995.

\bibitem{Kass2}
\textsc{M.~Kassabov,}
\newblock Universal lattices and property $(\tau )$,
\newblock \emph{Inv. Math.} \textbf{170} (2007), p.\,297--326.

\bibitem{Ka1}
\textsc{R.~Kaufman,}
\newblock Remark on Fourier-Stieltjes transforms of continuous measures, 
\newblock \emph{Colloq. Math.} \textbf{22} (1971) p.\,279--280.

\bibitem{Ka2}
\textsc{R.~Kaufman,} 
\newblock Continuous measures and analytic sets, 
\newblock \emph{Colloq. Math.} \textbf{58} (1989), p.\,17--21. 

\bibitem{K} 
\textsc{D.~Kazhdan,}
\newblock Connection of the dual space of a group with the structure of 
its closed subgroups,
\newblock \emph{Func. Anal. Appl.}  {\bf 1} (1967), p.\,63--65.

\bibitem{Ke} 
\textsc{D.~Kerr, H.~Li,}
\newblock Ergodic Theory: Independence and dichotomies, book in progress,
available at \verb=http://www.math.tamu.edu/~kerr/book=

\bibitem{KuiNied}
\textsc{L.~Kuipers, H.~Niederreiter,}
\newblock Uniform distribution of sequences,
\emph{Pure and Applied Mathematics}, Wiley-Interscience (1974). 

\bibitem{LR}
\textsc{V.~Losert, H.~Rindler,}
\newblock Uniform distribution and the mean ergodic theorem,
\newblock \emph{Inv. Math.} \textbf{50} (1978), p.\,65--74.

\bibitem{Lubo94} \textsc{A.~Lubotzky,}
\newblock Discrete Groups, Expanding Graphs, and Invariant Measures, Birkh\"auser (1994).

\bibitem{LZ}
\textsc{A.~Lubotzky, R.~Zimmer,}
\newblock Variants of Kazhdan's property for subgroups of semisimple groups,
\newblock \emph{Israel. J. Math.} \textbf{66} (1989), p.\,289--299.

\bibitem{Lub}
\textsc{A.~Lubotzky, A.~\.Zuk,}
\newblock On Property $(\tau)$, book in progress,
available at\\ \verb=http://www.ma.huji.ac.il/~alexlub/=

\bibitem{DM}
\textsc{B.~De~Mathan,}
\newblock Sur un probl\`eme de densit\'e modulo $1$,
\newblock \emph{C. R. Acad. Sci. Paris}  \textbf{287} (1978), p.\,277--279.

\bibitem{Mo} \textsc{C.~Moore,}
\newblock Groups with finite dimensional irreducible representations,
\newblock \emph{Trans. Amer. Soc.} {\bf 166} (1972), p.\,401--410.

\bibitem{Neuh} \textsc{M.~Neuhauser,}
\newblock Kazhdan constants for conjugacy classes of compact groups,
\newblock \emph{J. Algebra} {\bf 270} (2003), p.\,564--582. 

\bibitem{VN} 
\textsc{J.~von~Neumann,} 
\newblock On infinite direct products, 
\newblock \emph{Compositio Math.} {\bf 6} (1938), p.\,1--77.

\bibitem{Pal}
\textsc{T.~Palmer,} 
\newblock Classes on nonabelian, noncompact, locally compact groups, 
\newblock \emph{Rocky Mountain J. of Math.} {\bf 8} (1978), p.\,683--741.

\bibitem{Pet} \textsc{J.~Peterson,}
\newblock Lecture notes on ergodic theory, available at 
\verb=http://www.math.=
\par\noindent
\verb=vanderbilt.edu/peterson/teaching/Spring2011/ErgodicTheoryNotes.pdf=

\bibitem{Pol}
\textsc{A.~D.~Pollington,}
\newblock On the density of sequences $(n_k\xi)$,
\newblock \emph{Illinois J. Math}  \textbf{23} (1979), p.\,511--515.

\bibitem{rapin}
\textsc{A.~Rapinchuk,}
\newblock On the finite dimensional unitary representations of Kazhdan groups,
\newblock \emph{Proc. Amer. Math. Soc.}  \textbf{127} (1999), p.\,1557--1562.

\bibitem{Sch}
\textsc{K.~Schmidt,}
\newblock Asymptotically invariant sequences and an action of $SL_{2}({\ensuremath{\mathbb Z}})$ on the $2$-sphere,
\newblock \emph{Israel J.~Math.}  \textbf{37} (1980), p.\,193--208.

\bibitem{Sha2} \textsc{Y.~Shalom,}
\newblock Bounded generation and {Kazhdan}'s property (T),
\newblock \emph{Publ. Math. I.H.E.S.}  {\bf 90} (1999), p.\,145--168.

\bibitem{T}
\textsc{M.~Taylor,}
\newblock Noncommutative harmonic analysis,
\newblock \emph{Mathematical Surveys and Monographs} \textbf{22} (1986), American Mathematical Society.

\bibitem{V1}
\textsc{W.~Veech,}
\newblock Some questions of uniform distribution,
\newblock \emph{Ann. of Math.}  \textbf{94} (1971), p.\,125--138. 

\bibitem{V2}
\textsc{W.~Veech,} 
\newblock Topological dynamics, 
\newblock \emph{Bull. Amer. Math. Soc.} \textbf{83} (1977), p.\,775--830. 

\bibitem{Z}
\textsc{R.~Zimmer,}
\newblock Ergodic theory and semisimple groups, 
\newblock \emph{Monographs in Mathematics} \textbf{81} (1984), Birkh\"auser Verlag.

\bibitem{Zuk} \textsc{A.~\.Zuk,}
\newblock Property (T) and {Kazhdan}'s constants for discrete groups,
\newblock \emph{Geom. Func. Anal.}  {\bf 13} (2008), p.\,643--670.

}
 \end{thebibliography}

\end{document}


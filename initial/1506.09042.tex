\pdfoutput=1
\newif\ifpersonal
\RequirePackage[l2tabu, orthodox]{nag} 
\documentclass[12pt,a4paper,reqno]{amsart} 
\linespread{1.1}
\usepackage{amsmath,amsthm,amssymb,mathrsfs,mathtools,bm,eucal,tensor} 
\usepackage{microtype,fixltx2e,lmodern} 
\usepackage{enumerate,comment,braket,xspace,tikz-cd,csquotes} 
\usepackage[utf8]{inputenc} 
\usepackage[centering,vscale=0.7,hscale=0.7]{geometry}
\usepackage[hidelinks]{hyperref}
\usepackage[capitalize]{cleveref}

\makeatletter

\makeatother

\ifpersonal

\else

\fi

\theoremstyle{plain}
\newtheorem{thm-intro}{Theorem}
\newtheorem{thm}{Theorem}[section]
\newtheorem{lem}[thm]{Lemma}
\newtheorem{prop}[thm]{Proposition}
\newtheorem{conj}[thm]{Conjecture}
\newtheorem{cor-intro}[thm-intro]{Corollary}
\newtheorem{cor}[thm]{Corollary}
\newtheorem{assumption}[thm]{Assumption}
\theoremstyle{definition}
\newtheorem{defin}[thm]{Definition}
\newtheorem{notation}[thm]{Notation}
\newtheorem{eg-intro}[thm-intro]{Example}
\newtheorem{eg}[thm]{Example}
\newtheorem{rem-intro}[thm-intro]{Remark}
\newtheorem{rem}[thm]{Remark}
\theoremstyle{remark}
\numberwithin{equation}{section}
\newtheorem{construction}[thm]{Construction}

\DeclareMathAlphabet{\mathpzc}{OT1}{pzc}{m}{it}

 

 
 
 

\begin{document}

\title{Derived $\mathbb C$-analytic geometry I: GAGA theorems}

\author{Mauro PORTA}
\address{Mauro PORTA, Institut de Math\'ematiques de Jussieu, CNRS-UMR 7586, Case 7012, Universit\'e Paris Diderot - Paris 7, B\^atiment Sophie Germain 75205 Paris Cedex 13 France}
\email{mauro.porta@imj-prg.fr}

\subjclass[2010]{Primary 14A20; Secondary 32C35 14F05}
\keywords{derived analytic stack, derived complex geometry, infinity category, analytic stack, Grauert's theorem, analytification, GAGA, derived GAGA, derived algebraic geometry}

\begin{abstract}
	We further develop the foundations of derived {$\mathbb C$-analytic\xspace} geometry introduced in \cite{DAG-IX} by J. Lurie.
	We introduce the notion of coherent sheaf on a derived {$\mathbb C$-analytic\xspace} space.
	Moreover, building on the previous joint work with T.\ Y.\ Yu \cite{Porta_Yu_Higher_analytic_stacks_2014}, we introduce the notion of proper morphism of derived {$\mathbb C$-analytic\xspace} spaces.
	We show that these definitions are solid by proving a derived version of Grauert's proper direct image theorem and of the GAGA theorems.
	The proofs rely on two main ingredients: on one side, we prove a comparison result between the $\infty$-category of higher {Deligne-Mumford\xspace} analytic stacks introduced in \cite{Porta_Yu_Higher_analytic_stacks_2014} and the $\infty$-category of derived {$\mathbb C$-analytic\xspace} spaces.
	On the other side, we carry out a careful analysis of the analytification functor introduced in \cite{DAG-IX} and prove that the canonical map $X{^\mathrm{an}} \to X$ is flat in the derived sense.
	This is part of a series of articles \cite{Porta_Comparison_2015,Porta_Analytic_deformation_2015} that will soon appear.
\end{abstract}

\maketitle

{\ignorespaces}

\tableofcontents

\section*{Introduction}

Since its appearance, derived algebraic geometry has been proved an useful tool to deal with situations where the techniques of classical algebraic geometry couldn't yield a full understanding of the involved mathematical phenomena.
Striking examples are given by the work of B.\ To\"en on derived Azumaya algebras \cite{Toen_Derived_Azumaya_2012} and by the approach to the geometric Langlands program of D.\ Gaitsgory \cite{Gaitsgory_Outline}. Other applications can be found in Gromov-Witten theory, see for example the papers \cite{Schurg_Toen_Determinants_2015, Mann_Robalo_GW_invariants_2015}.
On the other side, derived algebraic geometry has been an active research field on its own, as the recent works on shifted symplectic and Poisson structures \cite{Pantev_Toen_Shifted_symplectic_2013, Toen_Vezzosi_Shifted_Poisson_2015} demonstrate.

However, before reaching the foundations nowadays accepted \cite{HAG-II, DAG-V, DAG-VII, DAG-VIII}, there has been several different approaches and attempts, see e.g.\ \cite{Kapranov_Derived_Quot_2001}. There has been recent activity toward the possibility of having a derived version of {$\mathbb C$-analytic\xspace} geometry, but the state of the art still resembles very much the one of derived algebraic geometry a decade ago.
In \cite{Ben-Bassat_Non-archimedean_2013}, the authors propose an approach to analytic geometry from the point of view of relative algebraic geometry \cite{Toen_Vaquie_Under_SpecZ}. As far as we understand, this would amount to take the affine objects to be simplicial ind-Banach rings. This is indeed a reasonable proposal, which is, however, quite different from the perspective we are adopting in the present work.

\subsection*{Pregeometries and derived geometry}

The approach to derived {$\mathbb C$-analytic\xspace} geometry used in this article is due to J.\ Lurie and relies on his general theory of pregeometries developed in \cite{DAG-V}.
He proposed to use it to lay foundations of derived {$\mathbb C$-analytic\xspace} geometry in \cite[§11, §12]{DAG-IX}.
As this way of constructing a category of derived objects differs significantly from the perhaps more familiar one of \cite{HAG-II}, it seems worth to describe the general ideas.

The procedure of \cite{HAG-II} consists in taking as input an ambient ``linear'' $\infty$-category ${\mathcal C}$ endowed with a symmetric monoidal structure and satisfying certain basic assumptions. Out of this ambient $\infty$-category ${\mathcal C}$ one obtains a category of affine objects $\mathrm{Aff}$, which precisely coincides with the $\infty$-category of commutative monoids in ${\mathcal C}$ (with respect to the tensor structure). Then, one endows this category of affine objects with some geometrical extra structure, such as a Grothendieck topology $\tau$ and a collection of ``smooth'' morphisms $\mathbf P$. These data are required to satisfy a certain amount of compatibility conditions (we refer to \cite[§ 1.3.2]{HAG-II} for the precise formulation of these ideas).
At this point, one can form the $\infty$-category of hypercomplete sheaves ${\mathrm{Sh}}(\mathrm{Aff}, \tau)^\wedge$, and inside this category it is possible to isolate objects that behave much like schemes, algebraic spaces, {Deligne-Mumford\xspace} stacks etc.\ known as \emph{geometric stacks}.
All of this is applied to derived algebraic geometry by taking ${\mathcal C}$ to be the $\infty$-category $\mathrm{sMod}_A$ of simplicial $A$-modules, where $A$ is any (discrete) commutative ring. Commutative monoids in $\mathrm{sMod}_A$ can be identified with simplicial commutative $A$-algebras, whose underlying $\infty$-category precisely coincides with the (opposite of the) category of affine derived schemes.

However, in developing derived analytic geometry it is not so clear what the ambient $\infty$-category ${\mathcal C}$ should be. Possible attempts are the category of simplicial ind-Banach spaces, or of simplicial bornological spaces (we refer to \cite{Bambozzi_Ben_Bassat_Dagger_2015} where it is shown that it is indeed possible to endow such categories with the structure of HAG contexts).
Still, the framework of \cite{HAG-II} can be successfully applied as soon as the category of affine objects is known. Together with T.\ Y.\ Yu, in \cite{Porta_Yu_Higher_analytic_stacks_2014} we showed that the Grauert proper direct image theorem and the GAGA theorems hold for (underived) higher Artin analytic stacks (see also \cite{Toen_Algebrisation_2008} for a different kind of applications in a very similar framework).
But, again, it is not at all clear what should the affine objects of derived {$\mathbb C$-analytic\xspace} geometry look like, though some version of analytic rings \cite{Dubuc_Taubin}, could yield a workable setting.

The approach taken by J.\ Lurie in \cite{DAG-V} is quite different and it can be used to deal with derived {$\mathbb C$-analytic\xspace} geometry, and we hope that this paper will contribute to show the solidity of the foundations laid in \cite{DAG-IX}.
The essential idea of \cite{DAG-V} is that we do not need an ambient linear category to develop derived algebraic geometry, nor we need to start with a well-known ($\infty$-)category of affine objects: we only need a category ${\mathcal C}$ of classical geometrical objects that we declare to be smooth, together with a collection of smooth morphisms between such objects. With only this, it is possible to construct the ``free category of derived ${\mathcal C}$-objects'' simply by requiring that pullbacks along smooth morphisms have to be classical (i.e.\ computed in ${\mathcal C}$).
The correct way of formalizing this striking idea is given with the language of pregeometries. We refer the reader to \cref{subsec:review} (and obviously to \cite{DAG-V}) for a more detailed technical account on this notion. For the time being, let us say that a pregeometry ${\mathcal T}$ is an $\infty$-category with finite products, equipped with a Grothendieck topology $\tau$ and a collection of admissible morphisms (these data are required to satisfy some mutual compatibility conditions, see \cref{def:admissibility_structure}).
The reader could think of this as a (multisorted) Lawvere theory equipped with some geometrical extra structure, and indeed the case of multisorted Lawvere theories is covered by the theory of \cite{DAG-V} when the Grothendieck topology is discrete and the admissible morphisms are precisely the equivalences.
As in the case of a (multisorted) Lawvere theory, there is a notion of a ``model'' for ${\mathcal T}$. This concept is referred to in \cite{DAG-V} as ${\mathcal T}$-structures. More specifically, if ${\mathcal X}$ is an $\infty$-topos (the reader might want to think to the $\infty$-category of spaces ${\mathcal S}$ or to sheaves of spaces on some topological space), a ${\mathcal T}$-structure on ${\mathcal X}$ is a product preserving functor $F \colon {\mathcal T} \to {\mathcal X}$ satisfying the following additional conditions:
\begin{enumerate}
	\item the functor $F$ preserves the pullbacks in ${\mathcal C}$ of the form
	\[ \begin{tikzcd}
		Y' \arrow{d} \arrow{r} & Y \arrow{d}{f} \\
		X' \arrow{r} & X
	\end{tikzcd} \]
	where $f$ is an admissible morphism.
	\item Whenever a family of morphisms $\{U_i \to X\}$ generates a covering sieve, the induced morphism
	\[ \coprod F(U_i) \to F(X) \]
	is an (effective) epimorphism in the $\infty$-topos ${\mathcal X}$.
\end{enumerate}
The full subcategory of $\operatorname{Fun}({\mathcal T}, {\mathcal X})$ spanned by ${\mathcal T}$-structures will be denoted ${\mathrm{Str}}_{\mathcal T}({\mathcal X})$.
However, as in the case of locally ringed spaces, we do not really wish to work with all the morphisms of ${\mathcal T}$-structures, but only with those that have a sufficiently regular behavior on the stalks of the topos ${\mathcal X}$. This can be achieved with the notion of local transformation: a morphism of ${\mathcal T}$-structures $\alpha \colon {\mathcal O} \to {\mathcal O}'$ is said to be \emph{local} if for every admissible morphism $f \colon U \to V$ in ${\mathcal T}$ the induced square
\[ \begin{tikzcd}
	{\mathcal O}(U) \arrow{r}{{\mathcal O}(f)} \arrow{d}{\alpha_U} & {\mathcal O}(V) \arrow{d}{\alpha_V} \\
	{\mathcal O}'(U) \arrow{r}{{\mathcal O}'(f)} & {\mathcal O}'(V)
\end{tikzcd} \]
is a pullback in ${\mathcal X}$. The subcategory of ${\mathrm{Str}}_{\mathcal T}({\mathcal X})$ spanned by local morphisms will be denoted by ${\mathrm{Str}^\mathrm{loc}}_{\mathcal T}({\mathcal X})$.

Before going any further, it is important to discuss a couple of examples, on which the intuition is based.
The first important situation is the one of a \emph{discrete} pregeometry (that is, of a multisorted Lawvere theory).

\begin{eg-intro} \label{eg:Tdisc}
	Let $k$ be a (discrete) commutative ring. We let ${{\mathcal T}_{\mathrm{disc}}(k)}$ be the opposite category of finitely presented free $k$-algebras. Admissible morphisms are precisely isomorphisms and the Grothendieck topology is the discrete one.
	In this case, ${\mathrm{Str}}_{{\mathcal T}_{\mathrm{disc}}(k)}({\mathcal S}) = {\mathrm{Str}^\mathrm{loc}}_{{\mathcal T}_{\mathrm{disc}}(k)}({\mathcal S})$ can be identified with the underlying $\infty$-category of simplicial $k$-algebras. If $\mathrm{char}(k) = 0$ we can further interpret this as the category of $\mathbb E_\infty$ connective $\mathrm Hk$-algebras, but it is important to remark that in positive (or mixed) characteristic we recover the simplicial formalism rather than the spectral one.
	Similarly, if ${\mathcal X}$ is an $\infty$-topos, ${\mathrm{Str}}_{{\mathcal T}_{\mathrm{disc}}(k)}({\mathcal X}) = {\mathrm{Str}^\mathrm{loc}}_{{\mathcal T}_{\mathrm{disc}}(k)}({\mathcal X})$ can be identified with the $\infty$-category of sheaves of simplicial commutative algebras over ${\mathcal X}$.
\end{eg-intro}

Thus, interpreting in spaces gave us back the $\infty$-category considered in \cite{HAG-II} to start the construction of derived algebraic geometry. However, this procedure does not provide us with the geometrical extra structure needed in \cite{HAG-II}.
Let us try to modify the pregeometry ${{\mathcal T}_{\mathrm{disc}}(k)}$ in order to encode the Zariski or the \'etale topology:

\begin{eg-intro}
	Let $k$ be a (discrete) commutative ring. We let ${{\mathcal T}_{\mathrm{Zar}}}(k)$ (resp.\ ${{\mathcal T}_{\mathrm{\acute{e}t}}}(k)$) be the category of standard Zariski open (resp.\ \'etale) maps to $\mathbb A^n_k$. We say that a morphism is admissible if it is an open Zariski immersion (resp.\ an \'etale map), and we consider the Zariski (resp.\ \'etale) topology.
	In this case, the forgetful functors
	\[ {\mathrm{Str}}_{{{\mathcal T}_{\mathrm{Zar}}}(k)}({\mathcal S}) \to {\mathrm{Str}}_{{\mathcal T}_{\mathrm{disc}}}({\mathcal S}), \qquad {\mathrm{Str}}_{{{\mathcal T}_{\mathrm{\acute{e}t}}}(k)}({\mathcal S}) \to {\mathrm{Str}}_{{\mathcal T}_{\mathrm{disc}}}({\mathcal S}) \]
	are fully faithful and the essential image is the collection of simplicial $k$-algebras $A$ such that $\pi_0(A)$ is local (resp.\ strictly henselian). Moreover,
	\[ {\mathrm{Str}^\mathrm{loc}}_{{{\mathcal T}_{\mathrm{\acute{e}t}}}(k)}({\mathcal S}) \to {\mathrm{Str}^\mathrm{loc}}_{{{\mathcal T}_{\mathrm{Zar}}}(k)}({\mathcal S}) \]
	is fully faithful, and a morphism $f \colon A \to B$ in ${\mathrm{Str}}_{{{\mathcal T}_{\mathrm{Zar}}}(k)}({\mathcal S})$ lies in ${\mathrm{Str}^\mathrm{loc}}_{{{\mathcal T}_{\mathrm{Zar}}}(k)}({\mathcal S})$ if and only if the induced $\pi_0(f) \colon \pi_0(A) \to \pi_0(B)$ is a (local) morphism of local rings. The same analysis can be carried over a generic $\infty$-topos.
\end{eg-intro}

Perhaps surprisingly, this example shows that when we interpret ${{\mathcal T}_{\mathrm{Zar}}}$ (or ${{\mathcal T}_{\mathrm{\acute{e}t}}}$) in spaces ${\mathcal S}$ we do \emph{not} recover the full $\infty$-category of simplicial commutative rings. We only get back the ones with good locality behavior with respect to the Grothendieck topology we took into account.
This suggests that the way of constructing the category of free ${\mathcal T}$-objects out of a pregeometry ${\mathcal T}$ has to be slightly more complicated.
As we said at the beginning, the category we are looking for should be determined by the requirement that pullbacks along admissible morphisms can be computed in ${\mathcal T}$, and it is freely generated by ${\mathcal T}$ otherwise.
This leads to the key of \emph{geometry}, which is an $\infty$-category ${\mathcal G}$ having all finite limits and equipped with the very same geometrical data of a pregeometry (that is, a Grothendieck topology and a collection of admissible morphism).
The notion of a structure for a geometry ${\mathcal G}$ is modified accordingly: ${\mathrm{Str}}_{\mathcal G}({\mathcal X})$ is now the full subcategory of $\operatorname{Fun}({\mathcal G}, {\mathcal X})$ spanned by finite limit preserving functors which takes $\tau$-covering to effective epimorphisms.

It is easy to imagine what a universal geometry generated by a pregeometry ${\mathcal T}$ should be; namely, a continuous functor of Grothendieck sites $\varphi \colon {\mathcal T} \to {\mathcal G}$ which moreover preserves products and admissible pullbacks, and satisfies the following universal property: for every $\infty$-topos ${\mathcal X}$, composition with $\varphi$ should induce an equivalence
\[ {\mathrm{Str}}_{\mathcal G}({\mathcal X}) \to {\mathrm{Str}}_{\mathcal T}({\mathcal X}) \]
In this situation, we say that $\varphi$ exhibits ${\mathcal G}$ as a \emph{geometric envelope of ${\mathcal T}$}. It can be shown that it always exists (see \cite[Lemma 3.4.3]{DAG-V}).
It is considerably harder than \cref{eg:Tdisc} to show that when ${\mathcal T} = {{\mathcal T}_{\mathrm{Zar}}}$, the geometric envelope is precisely the $\infty$-category of finitely presented simplicial commutative rings, equipped with the (derived) Zariski topology. Nevertheless, it is true (see \cite[Proposition 4.2.3]{DAG-V}). Even more remarkably, it can be shown \cite[Proposition 4.3.15]{DAG-V} that the geometric envelope of ${{\mathcal T}_{\mathrm{\acute{e}t}}}$ is the $\infty$-category of finitely presented simplicial commutative rings equipped with the (derived) \'etale topology.

So far we described only the procedure that allows to recover the affine objects, which, as we discussed at the beginning, is the only thing we need as the subsequent globalization step can be handled with the techniques of \cite{HAG-II}. However, it is important to remark here that the framework of \cite{DAG-V} comes with a gluing procedure that allows the author to introduce a structured space point of view on derived geometry.
Roughly speaking, this consists in defining an $\infty$-category of ${\mathcal T}$-structured topoi and then in isolating a full subcategory of ${\mathcal T}$-schemes inside.
These categories are respectively denoted ${\mathcal T\mathrm{op}}({\mathcal T})$ and ${\mathrm{Sch}}({\mathcal T})$, but we refer to the introduction of \cite{Porta_Comparison_2015} for an expository account of these ideas. The reader who wish to see the details can consult \cite[Definitions 1.4.8, 2.3.9]{DAG-V}.
In this work we will take the latter point of view, but some comparison result with the approach stemming from \cite{HAG-II} is provided (see \cref{sec:analytic_functor_of_points}).

\subsection*{The analytic pregeometry and the analytification functor}

We are left to explain which pregeometry should lead to a meaningful notion of derived {$\mathbb C$-analytic\xspace} space.
Following J.\ Lurie \cite[11.2, 11.3]{DAG-IX}, we define ${{\mathcal T}_{\mathrm{an}}}$ to be the category of open subsets of $\mathbb C^n$, and declare a morphism to be admissible if it is an open immersion. Let us explain what ${{\mathcal T}_{\mathrm{an}}}$-structures roughly look like. We hope this description will help the intuition of the reader in moving through this paper. For simplicity, we will think of ${{\mathcal T}_{\mathrm{an}}}$-structures in ${\mathrm{Set}}$ rather than in the $\infty$-category of spaces ${\mathcal S}$.
First of all, it is important to observe that ${{\mathcal T}_{\mathrm{an}}}$-structures are very similar to (local) ${\mathcal C}^\infty$-rings as they are introduced for example in \cite{Spivak_Derived_smooth_manifolds}. A similar approach to {$\mathbb C$-analytic\xspace} geometry has been taken by \cite{Dubuc_Taubin}, with the difference that they were taking a larger class of admissible pullbacks.
Roughly speaking, the idea behind this Lawvere-style approach to {$\mathbb C$-analytic\xspace} geometry is that the affine objects ought to be related to (commutative) Banach algebras, but that the very topological nature of such objects prevents their category from having good categorical properties.
However, we don't necessarily need the full structure of Banach algebra to give a working definition of {$\mathbb C$-analytic\xspace} spaces. What is really important is that Banach algebras admit a so-called holomorphic functional calculus, which is a formal way of encoding the action of the algebra of holomorphic functions on some open subset $U \subset \mathbb  C^n$ on a given (commutative) Banach algebra $A$.
Unraveling the definitions, it emerges that a ${{\mathcal T}_{\mathrm{an}}}$-structure in ${\mathrm{Set}}$ is precisely the given of a commutative ring $A$ together with the choice of a subset $A(U)$ of $A^n$ for every open subset $U \subset \mathbb C^n$ and a choice of a map $A(U) \to A$ for every holomorphic function $U \to \mathbb C$, satisfying being compatible with the compositions.

As we anticipated, following J.\ Lurie we will define the category of derived {$\mathbb C$-analytic\xspace} spaces ${\mathrm{dAn}_{\mathbb C}}$ as a full subcategory of ${\mathcal T\mathrm{op}}({{\mathcal T}_{\mathrm{an}}})$ (see \cref{def:derived_canal_space} for the precise formulation). We warn the reader that the category ${\mathrm{dAn}_{\mathbb C}}$ is \emph{different} from the category ${\mathrm{Sch}}({{\mathcal T}_{\mathrm{an}}})$ we introduced before. A careful comparison of the two has been carried out by J.\ Lurie in \cite[Corollary 12.22, Proposition 12.23]{DAG-IX}.

The last important idea of \cite{DAG-V} that plays a major role in this paper is the notion of \emph{relative spectrum}, which is used in \cite[Remark 12.26]{DAG-IX} in order to define the analytification functor.
The starting point is the universal property of the classical analytification functor described by Grothendieck in \cite[Expos\'e XII]{SGA1}. Recall that if $X$ is a scheme locally of finite presentation over $\mathbb C$, then an analytification of $X$ is the given of a {$\mathbb C$-analytic\xspace} space $X{^\mathrm{an}}$ together with a morphism of \emph{locally ringed spaces} $X{^\mathrm{an}} \to X$ inducing isomorphisms
\[ \operatorname{Hom}_{{\mathrm{An}_{{\mathbb C}}}}(Y, X{^\mathrm{an}}) \simeq \operatorname{Hom}_{\mathrm{LRingSpaces}}(Y,X)  \]
for every {$\mathbb C$-analytic\xspace} space $Y$.
This idea generalizes directly to the derived setting.
J.\ Lurie constructed in \cite[Theorem 2.1.1]{DAG-V} a far reaching generalization of the analytification functor, which is known as \emph{relative spectrum} associated to a morphism of pregeometries $\varphi \colon {\mathcal T} \to {\mathcal T}'$. To put it simply, he proved that the natural forgetful functor
\[ {\mathcal T\mathrm{op}}({\mathcal T}') \to {\mathcal T\mathrm{op}}({\mathcal T}) \]
(which sends a ${\mathcal T}'$-structured $\infty$-topos $({\mathcal X}, {\mathcal O})$ in the ${\mathcal T}$-structured $\infty$-topos $({\mathcal X}, {\mathcal O} \circ \varphi)$) admits a right adjoint, denoted $\operatorname{Spec}^{{\mathcal T}'}_{\mathcal T}$.
Furthermore he showed that this functor respects the subcategories of schemes.
Then, \cite[Corollary 12.22]{DAG-IX} allows to see that ${\mathrm{Spec}^{{\mathcal T}_{\mathrm{an}}}_{{\mathcal T}_{\mathrm{\acute{e}t}}}}$ takes the category of derived {Deligne-Mumford\xspace} stacks to ${\mathrm{dAn}_{\mathbb C}}$. This will therefore be the analytification functor we will be using through this paper.
As its definition is abstract (and the proof of the existence rather indirect), some work is required to show that ${\mathrm{Spec}^{{\mathcal T}_{\mathrm{an}}}_{{\mathcal T}_{\mathrm{\acute{e}t}}}}$ enjoys the good properties everyone would expect. This is one of the main parts of this article.

\subsection*{The main results}

We now explain the content of the current paper.
In \cref{sec:setup} we begin by reviewing in a less expository way the notion of pregeometry and of derived {$\mathbb C$-analytic\xspace} space.
We take the opportunity to summarizing the main results of \cite{DAG-IX}.
Next, in \cref{subsec:weak_Morita}, where we introduce the notion of weak Morita equivalence of pregeometries. This is a rather technical notion that nevertheless plays an important role in the study of the properties of germs of derived {$\mathbb C$-analytic\xspace} spaces. The main result of this section asserts that the $\infty$-categories ${\mathrm{Str}^\mathrm{loc}}_{\mathcal T}({\mathcal X})_{/{\mathcal O}}$ are always presentable for a pregeometry ${\mathcal T}$ and a ${\mathcal T}$-structure ${\mathcal O}$ on ${\mathcal X}$ (while ${\mathrm{Str}^\mathrm{loc}}_{\mathcal T}({\mathcal X})$ is not, in general, presentable).

In \cref{sec:local_analytic_rings} we show an important structure theorem for ${{\mathcal T}_{\mathrm{an}}}$-structures with values in spaces: namely, we show that except for the null ${{\mathcal T}_{\mathrm{an}}}$-structure, they are always canonically augmented over $\mathbb C$. This is a reminiscence of the well-known fact in functional analysis asserting that a commutative $\mathbb C$-Banach algebra which is a field is in fact isomorphic to $\mathbb C$. We use this fact to give a different description of the $\infty$-category of ${{\mathcal T}_{\mathrm{an}}}$-structures in ${\mathcal S}$. It is easy to provide a model categorical presentation for this alternative description, and we exploit this fact to carry out a few basic computations that will prove useful in dealing with the analytification functor.

With \cref{sec:analytic_functor_of_points} we enter in the main body of the article. The goal is to provide an functor of points description of derived {$\mathbb C$-analytic\xspace} spaces and use this to (partially) compare them with the higher analytic stacks introduced in \cite{Porta_Yu_Higher_analytic_stacks_2014}. To be more precise, we introduce a category of derived Stein spaces ${\mathrm{Stn}^{\mathrm{der}}_{\mathbb C}}$ and we endow it with a Grothendieck topology $\tau$ and a collection of morphisms ${\mathbf P}_{\mathrm{\acute{e}t}}$. We then prove:

\begin{thm-intro}[{\cref{prop:little_phi_fully_faithful} and \cref{thm:analytic_functor_of_points}}] \label{thm_intro_comparison_DM_stacks}
	There exists a fully faithful functor
	\[ \phi \colon {\mathrm{dAn}_{\mathbb C}} \to {\mathrm{Sh}}({\mathrm{Stn}^{\mathrm{der}}_{\mathbb C}}, \tau) \]
	Let now ${\mathrm{dAn}_{\mathbb C}}^{\mathrm{loc}}$ be the full subcategory of ${\mathrm{dAn}_{\mathbb C}}$ spanned by those derived {$\mathbb C$-analytic\xspace} spaces $({\mathcal X}, {\mathcal O}_{\mathcal X})$ for which ${\mathcal X}$ is an $n$-localic $\infty$-topos for some integer $n$.
	If $X \in {\mathrm{dAn}_{\mathbb C}}^{\mathrm{loc}}$, then $\phi(X)$ is a geometric stack with respect to the context $({\mathrm{Stn}^{\mathrm{der}}_{\mathbb C}}, \tau, {\mathbf P}_{\mathrm{\acute{e}t}})$ (see \cite[Definitions 2.11 and 2.15]{Porta_Yu_Higher_analytic_stacks_2014}). Vice-versa, geometric stacks for the context $({\mathrm{Stn}^{\mathrm{der}}_{\mathbb C}}, \tau, {\mathbf P}_{\mathrm{\acute{e}t}})$ constitute the essential image of the restriction $\phi \colon {\mathrm{dAn}_{\mathbb C}}^{\mathrm{loc}} \to {\mathrm{Sh}}({\mathrm{Stn}^{\mathrm{der}}_{\mathbb C}}, \tau)$.
\end{thm-intro}

\begin{rem-intro}
	An $\infty$-topos is $n$-localic if it is a category of sheaves on some $n$-category.
	We redirect the reader to \cite[§ 6.4.5]{HTT} for the definition and the main properties of $n$-localic $\infty$-topoi.
\end{rem-intro}

\begin{rem-intro}
	An analogous statement holds in the case of algebraic stacks, in which case the result can be thought as a precise comparison between the notion of derived {Deligne-Mumford\xspace} stack in the sense of \cite{HAG-II} and the one introduced in \cite{DAG-V}. This precise formulation seems to be a folklore result and it can somehow be found scattered through the DAG series of J.\ Lurie. Nevertheless, a concentrated proof will soon be available in \cite{Porta_Comparison_2015}.
\end{rem-intro}

We end \cref{sec:analytic_functor_of_points} by discussing in some details the notion of truncations of derived {$\mathbb C$-analytic\xspace} spaces and the basic properties enjoyed by this operation. We obtain the following comparison result:

\begin{cor-intro}[{\cref{cor:comparison_analytic_DM_stacks}}]
	Suppose $X \in {\mathrm{dAn}_{\mathbb C}}$ is a truncated derived {$\mathbb C$-analytic\xspace} space belonging to ${\mathrm{dAn}_{\mathbb C}}^{\mathrm{loc}}$.
	Then $\phi(X)$ is a higher analytic {Deligne-Mumford\xspace} stack in the sense of \cite{Porta_Yu_Higher_analytic_stacks_2014}.
\end{cor-intro}

In \cref{sec:coherent_sheaves} we simply introduce the notion of coherent sheaf on a derived {$\mathbb C$-analytic\xspace} space, and we continue the comparison with \cite{Porta_Yu_Higher_analytic_stacks_2014} by proving:

\begin{thm-intro}[{\cref{prop:comparison_coherent_sheaves}}] \label{thm_intro_comparison_coherent_sheaves}
	Suppose $X \in {\mathrm{dAn}_{\mathbb C}}$ is a truncated {$\mathbb C$-analytic\xspace} space satisfying the same finiteness conditions of \cref{thm_intro_comparison_DM_stacks}.
	Then the $\infty$-category of coherent sheaves on $X$ is equivalent to the $\infty$-category of coherent sheaves on $\phi(X)$ introduced in \cite{Porta_Yu_Higher_analytic_stacks_2014}.
\end{thm-intro}

In \cref{sec:Grauert_theorem}, we show that we inherit from \cite{Porta_Yu_Higher_analytic_stacks_2014} a notion of proper morphism for derived {Deligne-Mumford\xspace} stacks. Using the comparison results \cref{thm_intro_comparison_DM_stacks} and \cref{thm_intro_comparison_coherent_sheaves}, we obtain the first main result of this article: a derived version of Grauert's proper direct image theorem.

\begin{thm-intro}[{\cref{prop:proper_direct_image_derived_DM_stacks}}]
	Let $f \colon X \to Y$ be a proper morphism of derived {$\mathbb C$-analytic\xspace} spaces, both satisfying the same finiteness conditions of \cref{thm_intro_comparison_DM_stacks}. Then the derived direct image functor ${\mathrm R} f_* \colon {\mathcal O}_X \textrm{-} {\mathrm{Mod}} \to {\mathcal O}_Y \textrm{-} {\mathrm{Mod}}$ takes ${\mathrm{Coh}}^+(X)$ to ${\mathrm{Coh}}^+(Y)$.
\end{thm-intro}

\Cref{sec:analytification_functor} is the veritable heart of the article. Here, we carry over a detailed analysis of the analytification functor introduced in \cite[Remark 12.26]{DAG-IX}. We can summarize the main results of this section as follows:

\begin{thm-intro}[{\cref{prop:comparison_analytification_DM_stacks} and \cref{cor:analytification_flat}}]
	Let $X$ be a derived {Deligne-Mumford\xspace} stack which is locally of finite presentation over $\mathbb C$. Then:
	\begin{enumerate}
		\item (flatness) There is a natural map $X{^\mathrm{an}} \to X$ in the category of $\mathbb E_\infty$-ringed topoi which is flat in the derived sense.
		\item (comparison) If $X$ is truncated, then $\phi(X{^\mathrm{an}})$ coincides with the analytification of the functor of points associated to $X$, this analytification being understood in the sense of \cite{Porta_Yu_Higher_analytic_stacks_2014}. In particular, when $X$ is a scheme of finite presentation over $\mathbb C$, $X{^\mathrm{an}}$ can be identified with the usual analytification in the sense of Serre, cf.\ \cite[Expos\'e XII]{SGA1}.
	\end{enumerate}
\end{thm-intro}

The flatness part is perhaps the most technical part of this article. The proof relies on the computations made in \cref{sec:local_analytic_rings}.
As immediate consequence of this result, we obtain a rather explicit description of the analytification of a derived {Deligne-Mumford\xspace} stack in terms of the analytification of its truncation (see \cref{cor:description_analytification} for a precise statement).

Flatness unlocks moreover the two GAGA theorems, which we can now prove by reducing to the analogous results proven in \cite{Porta_Yu_Higher_analytic_stacks_2014}:

\begin{thm-intro}[{Derived GAGA-1, \cref{thm:derived_GAGA_1}}]
	Let $f \colon X \to Y$ be a proper morphism of derived {Deligne-Mumford\xspace} stacks locally of finite presentation over $\mathbb C$.
	Then for every ${\mathcal F} \in {\mathrm{Coh}}^+(X)$ the canonical analytification map
	\[ ({\mathrm R} f_*({\mathcal F})){^\mathrm{an}} \to {\mathrm R} f{^\mathrm{an}}_*({\mathcal F}{^\mathrm{an}}) \]
	is an equivalence.
\end{thm-intro}

\begin{thm-intro}[{Derived GAGA-2, \cref{thm:derived_GAGA_2}}]
	Let $X$ be a proper {Deligne-Mumford\xspace} stack locally of finite presentation over $\mathbb C$.
	Then the analytification functor induces an equivalence of $\infty$-categories
	\[ {\mathrm{Coh}}(X) \to {\mathrm{Coh}}(X{^\mathrm{an}}) \]
\end{thm-intro}

\begin{rem-intro}
	In \cite{Porta_Yu_Higher_analytic_stacks_2014} the second GAGA theorem is stated only for the categories ${\mathrm{Coh}^{\mathrm{b}}}$.
	However, the hard part of the proof is to deal with the case of sheaves in the heart ${\mathrm{Coh}}^\heartsuit(X)$.
	When \cite{Porta_Yu_Higher_analytic_stacks_2014} appeared, T.\ Y.\ Yu and I weren't aware that the same proof could also yield this stronger result.
\end{rem-intro}

We conclude the article with the short \cref{sec:extension_Artin}, where we explain how the main results of this article can be extended to derived Artin stacks.

\subsection*{Future work}

This paper is part of my ongoing Ph.D. thesis at the university of Paris Diderot.
It is part of a larger program of exploration of derived analytic geometry.
In some sense, it is also a natural continuation of \cite{Porta_Yu_Higher_analytic_stacks_2014}.

It will soon be followed by two related papers, \cite{Porta_Comparison_2015,Porta_Analytic_deformation_2015}.
In \cite{Porta_Comparison_2015} we carry out a similar analysis to the one of \cref{sec:analytic_functor_of_points} in the setting of derived algebraic geometry.
In other words, we set a precise comparison between the notion of derived {Deligne-Mumford\xspace} stack of \cite{HAG-II} and the one of \cite{DAG-V}. This is certainly a folklore result which is nevertheless difficult to find in the literature. As the proof is subtler than the one we carry over in \cref{sec:analytic_functor_of_points}, we feel like it deserves to be discussed at length and separately.

On the other side, in \cite{Porta_Analytic_deformation_2015} we will deal more extensively with the notion of (coherent) sheaf of modules over an analytic space. 
There are in fact at least two reasonable definitions for this category, and both are important to have a reasonably complete theory. A large part of \cite{Porta_Analytic_deformation_2015} will be devoted to the discussion of a comparison between these two notions. We plan to apply this theory to provide a workable theory of square-zero extensions and Postnikov truncations in the setting of derived {$\mathbb C$-analytic\xspace} geometry.

About longer term projects, we plan to concentrate on a couple of very natural questions that this article leaves somehow open.
Namely, the proof of the first GAGA theorem \cref{thm:derived_GAGA_1} can be easily extended (both in the derived and in the underived setting) to the \emph{unbounded} category of coherent sheaves under the additional requirement for the maps $f$ and $f{^\mathrm{an}}$ to be of finite cohomological dimension.
On the algebraic side one can reason by noetherian induction using the generic flatness lemma \cite[Lemma 8.5]{Porta_Yu_Higher_analytic_stacks_2014} to show that properness implies bounded cohomological dimension. On the analytic side the picture seems slightly more complicated. As it seems an interesting question on its own, we will investigate it further.

Another question left aside both from this paper and from \cite{Porta_Yu_Higher_analytic_stacks_2014} is a version of both GAGA theorems relative to a {$\mathbb C$-analytic\xspace} base (note that instead the underived rigid analytic version has been dealt with in \cite{Porta_Yu_Higher_analytic_stacks_2014}). This problem is interesting as it is often needed in the practical problems of computing the analytifications of given geometric stacks. We will therefore continue to pursue the matter. 

On a even longer term program, and for sake of an ongoing joint project with T.\ Y.\ Yu we are currently investigating the properties of the analytic cotangent complex and its relation with a version of Artin representability theorem for derived {$\mathbb C$-analytic\xspace} spaces.

Finally, we also plan to apply the derived GAGA theorems to the study of some examples of derived non abelian mixed hodge structures, following a recent proposal of Simpson, To\"en, Vaqui\'e and Vezzosi and their kind suggestion.

Last but not least, Jacob Lurie has very recently informed us that some of the results in this paper will also appear in a forthcoming draft of his new book. The two authors worked on this topics independently, so the two final versions will probably differ. As an example, our GAGA Theorems \ref{thm:derived_GAGA_1} and \ref{thm:derived_GAGA_2} are planned to appear in such a first draft under the additional assumption that the involved stacks are actually spectral algebraic spaces. As J.\ Lurie informed us, he planned to include our version for derived {Deligne-Mumford\xspace} stacks in a later version of the aforementioned book.

\subsection*{Conventions}

Throughout this paper we will work freely with the notion of $(\infty,1)$-category.
We will refer to such objects simply as $\infty$-categories.
Sometimes, it will be necessary to consider $(n,1)$-categories. We refer to \cite[§2.3.4]{HTT} for the basic theory of such objects.
In the article, we will denote them simply by $n$-categories (no confusion will arise because there will be no need of considering objects such as $(\infty,n)$-categories throughout this paper).

For reasons of practical convenience, we chose to work within the framework of \cite{HTT} and of \cite{Lurie_Higher_algebra}.
When citing from these sources, we will suppress the words Definition, Lemma, Proposition etc.
The notation ${\mathcal S}$ will be reserved to the $\infty$-category of spaces.
In \cite[6.3.1.5]{HTT} two categories of $\infty$-topoi are introduced, ${\mathrm{L} \mathcal{T} \mathrm{op}}$ and ${\mathcal{RT} \mathrm{op}}$.
If not specified otherwise, we will denote by ${\mathcal T\mathrm{op}}$ the $\infty$-category ${\mathcal{RT} \mathrm{op}}$.

We will denote by $\mathrm{CAlg}_{\mathbb C}$ the $\infty$-category of connective ${\mathrm H} \mathbb C$-algebras.
Equivalently, this can be identified with the underlying $\infty$-category of simplicial $\mathbb C$-algebras.
Since we work within the derived framework most of the time, we prefer to reserve the notation $B \otimes_A C$ for the derived tensor product. Whenever the underived one is needed, we will denote it by ${\mathrm{Tor}}^A_0(B, C)$.

\subsection*{Acknowledgments}

I tried to make my intellectual debt to J.\ Lurie and his monumental work as evident as possible since the very introduction, and I would like to emphasize it once more here.
I am deeply thankful to my advisor Gabriele Vezzosi for suggesting me this very interesting research topic, for his kindness and generosity in suggesting many possible further developments.
I would like to express my gratitude toward Carlos Simpson, who, directly and indirectly, encouraged me in moving on during this project.
Finally, I am very grateful to  V.\ Melani, M.\ Robalo, G.\ Vezzosi and T.\ Y.\ Yu for countless many stimulating discussions and for very helpful comments that helped me while writing this paper. I would also like to thank V.\ F.\ Zenobi for introducing me to the idea of the holomorphic functional calculus.

\section{The setup} \label{sec:setup}

\subsection{The {$\mathbb C$-analytic\xspace} pregeometry} \label{subsec:review}

This section is mainly for the convenience of the reader.
We review the basic definitions and results of \cite{DAG-IX}.

\begin{defin}[{\cite[Def.\ 1.2.1]{DAG-V}}] \label{def:admissibility_structure}
	Let ${\mathcal C}$ be an $\infty$-category.
	An \emph{admissibility structure} on ${\mathcal C}$ consists of the following data:
	\begin{enumerate}
		\item a subcategory ${\mathcal C}^{\mathrm{ad}} \subset {\mathcal C}$ containing every object of ${\mathcal C}$. Morphisms in ${\mathcal C}^{\mathrm{ad}}$ will be referred to as \emph{admissible} morphisms.
		\item a Grothendieck topology on ${\mathcal C}$ which is generated by admissible morphisms in the following sense: every covering sieve ${\mathcal C}^{(0)}_{/X} \subset {\mathcal C}_{/X}$ contains a covering sieve which is generated by admissible morphisms $\{U_\alpha \to X\}$.
	\end{enumerate}
	These data are required to satisfy the following compatibilities:
	\begin{enumerate}
		\item whenever $f \colon U \to X$ is an admissible morphism and $g \colon X' \to X$ is any morphism, the pullback
		\[ \begin{tikzcd}
			U' \arrow{r}{f'} \arrow{d} & X' \arrow{d}{g} \\
			U \arrow{r}{f} & X
		\end{tikzcd} \]
		exists and $f'$ is again an admissible morphism.
		\item Given a commutative triangle in ${\mathcal C}$
		\[ \begin{tikzcd}
			X \arrow{dr}{h} \arrow{rr}{f} & & Y \arrow{dl}{g} \\
			& Z
		\end{tikzcd} \]
		such that $g$ and $h$ are admissible, the same goes for $f$.
		\item Admissible morphisms are stable under retracts.
	\end{enumerate}
\end{defin}

\begin{defin}[{\cite[Def.\ 3.1.1]{DAG-V}}]
	A pregeometry ${\mathcal T}$ consists of an $\infty$-category with finite products and an admissibility structure on it.
\end{defin}

We let ${{\mathcal T}_{\mathrm{an}}}$ be the pregeometry defined as follows: the underlying $\infty$-category is the (nerve of) the category of open subsets of $\mathbb C^n$. We say that a morphism is admissible if it is an open immersion and we endow ${{\mathcal T}_{\mathrm{an}}}$ with the analytic (Grothendieck) topology.
Following \cite{DAG-IX} we introduce the notion of derived {$\mathbb C$-analytic\xspace} space:

\begin{defin}[{\cite[Def.\ 12.3]{DAG-IX}}] \label{def:derived_canal_space}
	A derived {$\mathbb C$-analytic\xspace} space is a ${{\mathcal T}_{\mathrm{an}}}$-structured $\infty$-topos $({\mathcal X}, {\mathcal O}_{\mathcal X})$ such that
	\begin{enumerate}
		\item there exists a covering $U_i$ of ${\mathcal X}$ such that ${\mathcal X}_{/U_i}$ is the $\infty$-topos of a topological space $X_i$.
		\item The pair $(X_i, \pi_0( {\mathcal O}_{\mathcal X}{^\mathrm{alg}}|_{U_i}))$ is a classical {$\mathbb C$-analytic\xspace} space.
		\item For every $i$, the sheaves $\pi_i( {\mathcal O}_{\mathcal X}{^\mathrm{alg}} |_{U_i})$ are coherent as $\pi_0({\mathcal O}_{\mathcal X}{^\mathrm{alg}} |_{U_i})$-modules.
	\end{enumerate}
	We will denote by ${\mathrm{dAn}_{\mathbb C}}$ the $\infty$-category of derived {$\mathbb C$-analytic\xspace} spaces.
\end{defin}

We summarize the basic results about ${\mathrm{dAn}_{\mathbb C}}$ in the following proposition:

\begin{prop} \label{prop:basic_properties_of_DAn}
	\begin{enumerate}
		\item The category ${\mathrm{dAn}_{\mathbb C}}$ has pullbacks, and the forgetful functor ${\mathrm{dAn}_{\mathbb C}} \to {\mathcal T\mathrm{op}}$ commutes with them.
		\item There is a canonical functor $\Phi \colon {\mathrm{An}_{{\mathbb C}}} \to {\mathrm{dAn}_{\mathbb C}}$ going from classical {$\mathbb C$-analytic\xspace} spaces to derived {$\mathbb C$-analytic\xspace} spaces which is fully faithful. Moreover, it commutes with pullbacks along local biholomorphisms.
		\item The natural forgetful functor ${\mathrm{dAn}_{\mathbb C}} \to {\mathcal T\mathrm{op}}({{\mathcal T}_{\mathrm{\acute{e}t}}})$ commutes with pullbacks along closed immersions.
	\end{enumerate}
\end{prop}

\begin{proof}
	These results are obtained in \cite[§12]{DAG-IX}. The functor $\Phi$ is defined as follows: if $X$ is a {$\mathbb C$-analytic\xspace} space, $\Phi(X) = ({\mathrm{Sh}}(X), {\mathcal O}_{\mathcal X})$, where ${\mathrm{Sh}}(X)$ is the $\infty$-topos of (a priori non-hypercomplete) sheaves on $X$, and ${\mathcal O}_{\mathcal X}$ is the functor
	\[ {\mathcal O}_{\mathcal X} \colon {{\mathcal T}_{\mathrm{an}}} \to {\mathrm{Sh}}(X) \]
	defined by
	\[ {\mathcal O}_{\mathcal X}(U) \coloneqq \operatorname{Hom}(U,-) \colon \mathrm{Opens}(X) \to {\mathcal S} \]
	where $\operatorname{Hom}(U,V)$ for $V$ an open in $X$ denotes the \emph{set} of holomorphic maps from $U$ to $V$. In particular, we see that ${\mathcal O}_{\mathcal X}$ is $0$-truncated.
\end{proof}

\begin{rem}
	The functor $\Phi \colon {\mathrm{An}_{{\mathbb C}}} \to {\mathrm{dAn}_{\mathbb C}}$ factorizes by construction through the full subcategory of derived {$\mathbb C$-analytic\xspace} spaces $({\mathcal X}, {\mathcal O}_{\mathcal X})$ such that ${\mathcal O}_{\mathcal X}$ is $0$-truncated and ${\mathcal X}$ is $0$-localic. It can be shown that every such derived {$\mathbb C$-analytic\xspace} space lies in the essential image of $\Phi$.
\end{rem}

To ease future reference, we collect here also the definitions of \'etale morphism and closed immersion of derived {$\mathbb C$-analytic\xspace} spaces.

\begin{defin}\label{def:etale_and_closed_morphisms}
	Let $f \colon ({\mathcal X}, {\mathcal O}_{\mathcal X}) \to ({\mathcal Y}, {\mathcal O}_{\mathcal Y})$ be a morphism of derived {$\mathbb C$-analytic\xspace} spaces.
	We will say that:
	\begin{enumerate}
		\item (\cite[Definition 2.3.1]{DAG-V}) $f$ is \emph{\'etale} if the underlying geometric morphism of $\infty$-topoi $f{^{-1}} \colon {\mathcal Y} \rightleftarrows {\mathcal X} \colon f_*$ is \'etale in the sense of \cite[§ 6.3.5]{HTT} and the induced morphism $f{^{-1}} {\mathcal O}_{\mathcal Y} \to {\mathcal O}_{\mathcal X}$ is an equivalence.
		\item (\cite[Definition 1.1]{DAG-IX}) $f$ is a \emph{closed immersion} if the underlying geometric morphism of $\infty$-topoi $f{^{-1}} \colon {\mathcal Y} \rightleftarrows {\mathcal X} \colon f_*$ is a closed immersion in the sense of \cite[7.3.2.7]{HTT} and the induced morphism $f{^{-1}} {\mathcal O}_{\mathcal Y} \to {\mathcal O}_{\mathcal X}$ is an effective epimorphism.
	\end{enumerate}
\end{defin}

Let ${{\mathcal T}_{\mathrm{disc}}} = {{\mathcal T}_{\mathrm{disc}}}(\mathbb C)$ be the pregeometry discussed in the Introduction, \cref{eg:Tdisc}.
There is a natural morphism of pregeometries ${{\mathcal T}_{\mathrm{disc}}} \to {{\mathcal T}_{\mathrm{an}}}$ which induces a forgetful functor
\[ {\mathcal T\mathrm{op}}({{\mathcal T}_{\mathrm{an}}}) \to {\mathcal T\mathrm{op}}({{\mathcal T}_{\mathrm{disc}}}) \]
Accordingly to the notations introduced in \cite{DAG-IX}, if $({\mathcal X}, {\mathcal O}_{\mathcal X}) \in {\mathcal T\mathrm{op}}({{\mathcal T}_{\mathrm{an}}})$, we will denote by $({\mathcal X}, {\mathcal O}_{\mathcal X}{^\mathrm{alg}})$ its image under this functor.
Observe that ${\mathcal O}_{\mathcal X}{^\mathrm{alg}}$ can be identified with a sheaf of connective ${\mathrm H} \mathbb C$-algebras on ${\mathcal X}$.

\subsection{Weak Morita equivalences of pregeometries} \label{subsec:weak_Morita}

Let ${\mathcal T}$ be a pregeometry and let ${\mathcal X}$ be an $\infty$-topos.
In general, the $\infty$-category of ${\mathcal T}$-structures on ${\mathcal X}$ is not a presentable $\infty$-category.
The problem is that ${\mathrm{Str}}_{\mathcal T}({\mathcal X})$ and ${\mathrm{Str}^\mathrm{loc}}_{\mathcal T}({\mathcal X})$ are not cocomplete (see \cite[Prop.\ 1.5.1]{DAG-V} for a more detailed discussion), and the ultimate reason for this is to be found in the compatibilities that the objects of ${\mathrm{Str}}_{\mathcal T}({\mathcal X})$ are required to have with the Grothendieck topology of ${\mathcal T}$.
This could be an issue in developing derived {$\mathbb C$-analytic\xspace} geometry.
However, it is still true in general that, whenever ${\mathcal O} \in {\mathrm{Str}}_{\mathcal T}({\mathcal X})$ is a ${\mathcal T}$-structure, the overcategory ${\mathrm{Str}^\mathrm{loc}}_{\mathcal T}({\mathcal X})_{/{\mathcal O}}$ is a presentable $\infty$-category. The goal of this subsection is to provide a proof of this statement.
In the economy of the present work, the relevance of this section is mainly for its application to the structure of local analytic rings, that will be discussed in the next section.

We begin with a definition:

\begin{defin}
	Let $\varphi \colon {\mathcal T}' \to {\mathcal T}$ be a morphism of pregeometries.
	We will say that $\varphi$ is a \emph{weak Morita equivalence} if for every $\infty$-topos ${\mathcal X}$ and every ${\mathcal T}$-structure ${\mathcal O} \in {\mathrm{Str}}_{\mathcal T}({\mathcal X})$ the restriction functor
	\[ {\mathrm{Str}^\mathrm{loc}}_{\mathcal T}({\mathcal X})_{/ {\mathcal O}} \to {\mathrm{Str}^\mathrm{loc}}_{{\mathcal T}'}({\mathcal X})_{/ {\mathcal O} \circ \varphi} \]
	is an equivalence of $\infty$-categories.
\end{defin}

\begin{rem}
	In \cite[§3.2]{DAG-V} J. Lurie introduced the notion of Morita equivalence of pregeometries in order to understand which modifications of the admissibility structure on a pregeometry ${\mathcal T}$ gives rise to the same $\infty$-category of ${\mathcal T}$-structured topoi.
	The notion we introduced above is quite different in the spirit.
	It is in fact meant to be an intermediate step between Morita equivalences and the other notion, not yet appeared, of \emph{stable Morita equivalences}.
	The latter is of great importance for the theory of modules for analytic structures, and we will come back on the subject in \cite{Porta_Analytic_deformation_2015}.
	Some of the results collected here which do not find an immediate application are stated for sake of future reference.
\end{rem}

\begin{lem} \label{lem:strloc_fully_faitful_str}
	Let ${\mathcal T}$ be a pregeometry and let ${\mathcal X}$ be an $\infty$-topos.
	For every ${\mathcal O} \in {\mathrm{Str}}_{\mathcal T}({\mathcal X})$ the canonical functor
	\[ j \colon {\mathrm{Str}^\mathrm{loc}}_{\mathcal T}({\mathcal X})_{/{\mathcal O}} \to {\mathrm{Str}}_{\mathcal T}({\mathcal X})_{/{\mathcal O}} \]
	is fully faithful.
\end{lem}

\begin{proof}
	The functor ${\mathrm{Str}^\mathrm{loc}}_{\mathcal T}({\mathcal X})_{/{\mathcal O}} \to {\mathrm{Str}}_{\mathcal T}({\mathcal X})$ is faithful, and therefore $j$ is faithful as well. Let ${\mathcal A}, {\mathcal B} \in {\mathrm{Str}^\mathrm{loc}}_{\mathcal T}({\mathcal X})_{/{\mathcal O}}$ and denote by $p \colon {\mathcal A} \to {\mathcal O}$, $q \colon {\mathcal B} \to {\mathcal O}$ the structural morphisms.
	If we show that every morphism $\alpha \colon {\mathcal A} \to {\mathcal B}$ is a local morphism, the fullness of $j$ will follow at once.
	Let $f \colon U \to V$ be an admissible morphism in ${\mathcal T}$ and consider the commutative diagram
	\[ \begin{tikzcd}
	{\mathcal A}(U) \arrow{r}{\alpha_U} \arrow{d} & {\mathcal B}(U) \arrow{r}{q_U} \arrow{d} & {\mathcal O}(U) \arrow{d} \\
	{\mathcal A}(V) \arrow{r}{\alpha_V} & {\mathcal B}(V) \arrow{r}{q_V} & {\mathcal O}(V)
	\end{tikzcd} \]
	The horizontal composites are equivalent to $p_U$ and $p_V$ respectively.
	Since $p$ and $q$ are local morphisms, we see that the right square as well as the outer one are pullback.
	It follows that the left square is a pullback too, completing the proof.
\end{proof}

\begin{lem} \label{lem:slice_category_faithful}
	Let ${\mathcal C}$ be an $\infty$-category and let $X$ be an object in ${\mathcal C}$.
	There is a faithful functor
	\[ {\mathcal C}_{/X} \to \operatorname{Fun}(\Delta^1, {\mathcal C}) \]
	whose essential image is given by those arrows with target equivalent to $X$.
	Moreover, a morphism in $\operatorname{Fun}(\Delta^1 , {\mathcal C})$ belongs to ${\mathcal C}_{/X}$ if and only its restriction to the target is equivalent to the identity of $X$.
\end{lem}

\begin{proof}
	The simplicial model for ${\mathcal C}_{/X}$ of \cite[1.2.9.2]{HTT} provides us with the desired functor.
	Moreover, ${\mathcal C}_{/X}$ can be exhibited as the (homotopy) pullback diagram
	\[ \begin{tikzcd}
		{\mathcal C}_{/X} \arrow{r} \arrow{d} & \operatorname{Fun}(\Delta^1, {\mathcal C}) \arrow{d}{\mathrm{ev}_1} \\
		\Delta^0 \arrow{r}{X} & {\mathcal C}
	\end{tikzcd} \]
	from which the lemma follows.
\end{proof}

\begin{prop} \label{prop:simplifying_strloc}
	Let ${\mathcal T}$ be a pregeometry and let ${\mathcal X}$ be an $\infty$-topos.
	Let ${\mathcal O} \in {\mathrm{Str}}_{\mathcal T}({\mathcal X})$ be a ${\mathcal T}$-structure.
	There exists a faithful functor
	\begin{equation} \label{eq:strloc_arrows}
	{\mathrm{Str}^\mathrm{loc}}_{\mathcal T}({\mathcal X})_{/ {\mathcal O}} \to \operatorname{Fun}({\mathcal T} \times \Delta^1, {\mathcal X})
	\end{equation}
	whose essential image consists of those functors $F \colon {\mathcal T} \times \Delta^1 \to {\mathcal X}$ satisfying the following conditions:
	\begin{enumerate}
		\item the restriction $F_1 \coloneqq F |_{{\mathcal T} \times \{1\}}$ is equivalent to ${\mathcal O}$;
		\item the restriction $F_0 \coloneqq F |_{{\mathcal T} \times \{0\}}$ commutes with products;
		\item for every admissible morphism $f \colon U \to V$ in ${\mathcal T}$ the induced square
		\[ \begin{tikzcd}
		F_0(U) \arrow{r} \arrow{d} & F_0(V) \arrow{d} \\
		F_1(U) \arrow{r} & F_1(V)
		\end{tikzcd} \]
		is a pullback square.
	\end{enumerate}
	Moreover, let $F, G \in {\mathrm{Str}^\mathrm{loc}}_{\mathcal T}({\mathcal X})_{/{\mathcal O}}$ and let $\alpha \colon F \to G$ be a morphism between them in $\operatorname{Fun}({\mathcal T} \times \Delta^1, {\mathcal X})$.
	Then $\alpha$ belongs to ${\mathrm{Str}^\mathrm{loc}}_{\mathcal T}({\mathcal X})_{/{\mathcal O}}$ if and only if the restriction $\alpha |_{{\mathcal T} \times {0}}$ is equivalent to the identity of ${\mathcal O}$.
\end{prop}

\begin{proof}
	We can factor the functor \eqref{eq:strloc_arrows} as 
	\[ {\mathrm{Str}^\mathrm{loc}}_{\mathcal T}({\mathcal X})_{/{\mathcal O}} \xrightarrow{u} {\mathrm{Str}}_{\mathcal T}({\mathcal X})_{/{\mathcal O}} \xrightarrow{v} \operatorname{Fun}({\mathcal T} \times \Delta^1, {\mathcal X}) . \]
	\Cref{lem:strloc_fully_faitful_str} shows that $u$ is fully faithful. The first statement and the last one follow therefore from \cref{lem:slice_category_faithful}.
	It moreover clear that all the objects in the essential image of \eqref{eq:strloc_arrows} satisfy conditions (1) to (3).
	Let $F \colon {\mathcal T} \times \Delta^1 \to {\mathcal X}$ be a functor satisfying these conditions. Using (1) and (3) we see that this functor determines a local morphism $F_0 \to {\mathcal O}$ in $\operatorname{Fun}({\mathcal T}, {\mathcal X})$.
	To conclude, we only need to show that $F_0 \in {\mathrm{Str}}_{\mathcal T}({\mathcal X})$. Since $F_0$ commutes with products in virtue of condition (2), we only need to check that $F_0$ commutes with admissible pullbacks and takes coverings in ${\mathcal T}$ to effective epimorphisms.
	Let
	\[ \begin{tikzcd}
	U' \arrow{r} \arrow{d}{f'} & U \arrow{d}{f} \\
	V' \arrow{r} & V
	\end{tikzcd} \]
	be a pullback square in ${\mathcal T}$, where $f \colon U \to V$ is an admissible morphism.
	Consider the commutative diagram
	\[ \begin{tikzcd}
	F_0(U') \arrow{r} \arrow{d} & F_0(U) \arrow{d} \arrow{r} & {\mathcal O}(U) \arrow{d} \\
	F_0(V') \arrow{r} & F_0(V) \arrow{r} & {\mathcal O}(V)
	\end{tikzcd} \]
	and observe that the right square is a pullback because $f$ is admissible.
	Moreover, the outer square can be also factored as
	\[ \begin{tikzcd}
	F_0(U') \arrow{r} \arrow{d} & {\mathcal O}(U') \arrow{r} \arrow{d} & {\mathcal O}(U) \arrow{d} \\
	F_0(V') \arrow{r} & {\mathcal O}(V') \arrow{r} & {\mathcal O}(V)
	\end{tikzcd} \]
	The left square is a pullback because ${\mathcal O} \in {\mathrm{Str}}_{\mathcal T}({\mathcal X})$, and the left square is a pullback because $f' \colon U' \to V'$ is an admissible pullback.
	It follows that $F_0$ preserves admissible pullbacks.
	
	Now let $\{U_i \to U\}$ be an admissible covering in ${\mathcal T}$. Since ${\mathcal X}$ is an $\infty$-topos, we see that there is a pullback square
	\[ \begin{tikzcd}
	\coprod F_0(U_i) \arrow{r} \arrow{d} & F_0(U) \arrow{d} \\
	\coprod {\mathcal O}(U_i) \arrow{r} & {\mathcal O}(U)
	\end{tikzcd} \]
	Since the bottom line is an effective epimorphism, the result now follows from the stability of effective epimorphisms under pullbacks.
\end{proof}

\Cref{prop:simplifying_strloc} allows to produce several examples of weak Morita equivalences.

\begin{defin}
	Let ${\mathcal T}$ be a pregeometry. The \emph{associated discrete pregeometry} ${\mathcal T}_{\mathrm{d}}$ is the discrete pregeometry having the same underlying $\infty$-category of ${\mathcal T}$.
	The \emph{associated semi-discrete pregeometry} ${\mathcal T}_{\mathrm{sd}}$ is the pregeometry having the same underlying $\infty$-category and the same admissible morphisms of ${\mathcal T}$, but discrete Grothendieck topology.
\end{defin}

Observe that we have morphisms of pregeometries ${\mathcal T}_{\mathrm{d}} \to {\mathcal T}_{\mathrm{sd}} \to {\mathcal T}$.

\begin{cor} \label{cor:semi_discrete_weak_Morita}
	Let ${\mathcal T}$ be a pregeometry. The morphism ${\mathcal T}_{\mathrm{sd}} \to {\mathcal T}$ is a weak Morita equivalence.
\end{cor}

\begin{proof}
	Let ${\mathcal X}$ be an $\infty$-topos and let ${\mathcal O} \in {\mathrm{Str}}_{\mathcal T}({\mathcal X})$. Denote by ${\mathcal O}'$ the restriction of ${\mathcal O}$ along ${\mathcal T}_{\mathrm{sd}} \to {\mathcal T}$.
	\Cref{prop:simplifying_strloc} allows to identify both ${\mathrm{Str}^\mathrm{loc}}_{\mathcal T}({\mathcal X})_{/{\mathcal O}}$ and ${\mathrm{Str}^\mathrm{loc}}_{{\mathcal T}_{\mathrm{sd}}}({\mathcal X})_{/{\mathcal O}'}$ with the same subcategory of $\operatorname{Fun}({\mathcal T} \times \Delta^1, {\mathcal X})$. Therefore, they are equivalent.
\end{proof}

\begin{lem} \label{lem:Str_presentable}
	Let ${\mathcal T}$ be a pregeometry in which the Grothendieck topology is discrete. Then for every $\infty$-topos ${\mathcal X}$, the $\infty$-category ${\mathrm{Str}}_{\mathcal T}({\mathcal X})$ is presentable.
\end{lem}

\begin{proof}
	This is a direct consequence of \cite[Lemmas 5.5.4.18, 5.5.4.19]{HTT}.
\end{proof}

\begin{prop} \label{prop:O_premodules_presentable}
	Let ${\mathcal T}$ be a pregeometry and let ${\mathcal X}$ be an $\infty$-topos.
	For every ${\mathcal O} \in {\mathrm{Str}}_{\mathcal T}({\mathcal X})$ the $\infty$-category ${\mathrm{Str}^\mathrm{loc}}_{\mathcal T}({\mathcal X})_{/ {\mathcal O}}$ is presentable.
\end{prop}

\begin{proof}
	In virtue of \cref{cor:semi_discrete_weak_Morita} we can assume that the Grothendieck topology on ${\mathcal T}$ is discrete. In this case, \cref{lem:Str_presentable} shows that ${\mathrm{Str}}_{\mathcal T}({\mathcal X})$ is a presentable $\infty$-category.
	\cite[Theorem 1.3.1]{DAG-V} shows that there exists a factorization system $(S^{\mathcal X}_L, S^{\mathcal X}_R)$ on ${\mathrm{Str}}_{\mathcal T}({\mathcal X})$ such that $S^{\mathcal X}_R$ is the collection of morphisms in ${\mathrm{Str}^\mathrm{loc}}_{\mathcal T}({\mathcal X})$.
	
	Let ${\mathcal D}$ be the full subcategory of $\operatorname{Fun}(\Delta^1, {\mathrm{Str}}_{\mathcal T}({\mathcal X}))$ spanned by the elements of $S^{\mathcal X}_R$.
	\cite[Lemma 5.2.8.19]{HTT} shows that ${\mathcal D}$ is a localization of $\operatorname{Fun}(\Delta^1, {\mathrm{Str}}_{\mathcal T}({\mathcal X}))$. Since filtered colimits commutes with pullbacks, we see that ${\mathcal D}$ is closed under filtered colimits in $\operatorname{Fun}(\Delta^1, {\mathrm{Str}}_{\mathcal T}({\mathcal X}))$. In other words, ${\mathcal D}$ is an accessible localization of the latter category. Since ${\mathrm{Str}}_{\mathcal T}({\mathcal X})$ is a presentable $\infty$-category, \cite[Proposition 5.5.1.2]{HTT} shows that ${\mathcal D}$ is presentable as well.
	Observe now that ${\mathrm{Str}^\mathrm{loc}}_{\mathcal T}({\mathcal X})_{/{\mathcal O}}$ fits into a pullback square
	\[ \begin{tikzcd}
	{\mathrm{Str}^\mathrm{loc}}_{\mathcal T}({\mathcal X})_{/{\mathcal O}} \arrow{r} \arrow{d} & {\mathcal D} \arrow{d}{\mathrm{ev}_1} \\
	\{{\mathcal O}\} \arrow{r} & {\mathrm{Str}}_{\mathcal T}({\mathcal X})
	\end{tikzcd} \]
	\cite[Theorem 5.5.3.18]{HTT} shows that ${\mathrm{Str}^\mathrm{loc}}_{\mathcal T}({\mathcal X})_{/{\mathcal O}}$ is presentable, thus completing the proof.
\end{proof}

For future convenience, we record the following easy consequence:

\begin{cor}
	Let ${\mathcal T}$ be a pregeometry and let ${\mathcal X}$ be an $\infty$-topos. For every ${\mathcal O} \in {\mathrm{Str}}_{\mathcal T}({\mathcal X})$ the category of ${\mathcal O}$-modules $\operatorname{Sp}({\mathrm{Str}^\mathrm{loc}}_{\mathcal T}({\mathcal X})_{/{\mathcal O}})$ is a presentable $\infty$-category.
\end{cor}

\begin{proof}
	This follows from \cref{prop:O_premodules_presentable} and \cite[Proposition 1.4.4.4.(1)]{Lurie_Higher_algebra}.
\end{proof}

\begin{prop} \label{prop:creating_connected_limits_sifted_colimits}
	Let ${\mathcal T}$ be a pregeometry and let ${\mathcal X}$ be an $\infty$-topos.
	Let ${\mathcal O} \in {\mathrm{Str}}_{\mathcal T}({\mathcal X})$ be a ${\mathcal T}$-structure.
	The functor ${\mathrm{Str}^\mathrm{loc}}_{\mathcal T}({\mathcal X})_{/{\mathcal O}} \to \operatorname{Fun}({\mathcal T} \times \Delta^1, {\mathcal X})$ creates connected limits and sifted colimits.
\end{prop}

\begin{proof}
	Let $K$ be a simplicial set and let $F \colon K \to \operatorname{Fun}({\mathcal T} \times \Delta, {\mathcal X})$ be a functor factorizing through ${\mathrm{Str}^\mathrm{loc}}_{\mathcal T}({\mathcal X})_{/ {\mathcal O}}$.
	Let $\widetilde{F} \colon K^\triangleright \to \operatorname{Fun}({\mathcal T} \times \Delta^1, {\mathcal X})$ be a limit diagram and set $G \coloneqq F(v_0)$, where $v_0$ denotes the initial vertex of $K^\triangleleft$.
	$G_0$ commutes with product and $G_1$ is the limit of the constant diagram associated to ${\mathcal O}$. Since sifted colimits are connected, we see that $G_1$ is equivalent to ${\mathcal O}$.
	Moreover, if $f \colon U \to V$ is an admissible morphism in ${\mathcal T}$, the diagram
	\[ \begin{tikzcd}
	G_0(U) \arrow{r} \arrow{d} & G_0(V) \arrow{d} \\
	G_1(U) \arrow{r} & G_1(V)
	\end{tikzcd} \]
	is the limit of pullback diagrams and it is therefore a pullback diagram by itself.
	
	Let now $K$ be a sifted simplicial set and let $F \colon K \to \operatorname{Fun}({\mathcal T} \times \Delta, {\mathcal X})$ be a functor factorizing through ${\mathrm{Str}^\mathrm{loc}}_{\mathcal T}({\mathcal X})_{/ {\mathcal O}}$.
	Let $\widetilde{F} \colon K^\triangleleft \to \operatorname{Fun}({\mathcal T} \times \Delta^1, {\mathcal X})$ be a colimit diagram and set $G \coloneqq F(v_\infty)$, where $v_\infty$ denotes the final object of $K^\triangleleft$.
	Since $K$ is sifted, we see that $G_0$ commutes with products. As before, we see that $G_1$ is equivalent to ${\mathcal O}$.
	Let $f \colon U \to V$ be an admissible morphism in ${\mathcal T}$. Since ${\mathcal X}$ is an $\infty$-topos, we see that
	\begin{align*}
	G_1(U) \times_{G_1(V)} G_0(V) & \simeq {\mathcal O}(U) \times_{{\mathcal O}(V)} G_0(V) \\
	& \simeq {\mathcal O}(U) \times_{{\mathcal O}(V)} \operatorname*{colim}_{K} (F_0(V)) \\
	& \simeq \operatorname*{colim}_K ( {\mathcal O}(U) \times_{{\mathcal O}(V)} F_0(V) ) \simeq \operatorname*{colim}_K F_0(U) \simeq G_0(U)
	\end{align*}
	We therefore conclude that $G$ belongs to the ${\mathrm{Str}^\mathrm{loc}}_{\mathcal T}({\mathcal X})_{/{\mathcal O}}$.
\end{proof}

\begin{cor} \label{cor:Phi_limits_sifted_colimits}
	Let $\varphi \colon {\mathcal T}' \to {\mathcal T}$ be a morphism of pregeometries.
	Let ${\mathcal O} \in {\mathrm{Str}}_{\mathcal T}({\mathcal X})$ be a ${\mathcal T}$-structure.
	The restriction functor
	\[ \overline{\Phi} \colon {\mathrm{Str}^\mathrm{loc}}_{\mathcal T}({\mathcal X})_{/{\mathcal O}} \to {\mathrm{Str}^\mathrm{loc}}_{\mathcal T}({\mathcal X})_{/{\mathcal O} \circ \varphi} \]
	commutes with limits and sifted colimits.
\end{cor}

\begin{proof}
	Observe that the relevant functor takes the final object to the final object by the very construction. It will be therefore sufficient to show that it commutes with connected limits and sifted colimits.
	Consider the commutative diagram
	\[ \begin{tikzcd}
	{\mathrm{Str}^\mathrm{loc}}_{\mathcal T}({\mathcal X})_{/{\mathcal O}} \arrow{r} \arrow{d} & {\mathrm{Str}^\mathrm{loc}}_{{\mathcal T}'}({\mathcal X})_{/{\mathcal O} \circ \varphi} \arrow{d} \\
	\operatorname{Fun}({\mathcal T} \times \Delta^1, {\mathcal X}) \arrow{r} & \operatorname{Fun}({\mathcal T}' \times \Delta^1, {\mathcal X})
	\end{tikzcd} \]
	\cref{prop:creating_connected_limits_sifted_colimits} shows that the vertical morphisms creates both connected limits and sifted colimits, while $\operatorname{Fun}({\mathcal T} \times \Delta^1, {\mathcal X}) \to \operatorname{Fun}({\mathcal T}' \times \Delta^1 , {\mathcal X})$ preserves all (small) limits and (small) colimits.
	The proof is therefore complete.
\end{proof}

\begin{cor} \label{cor:Phi_right_adjoint}
	Let $\varphi \colon {\mathcal T}' \to {\mathcal T}$ be a morphism of pregeometries.
	Let ${\mathcal O} \in {\mathrm{Str}}_{\mathcal T}({\mathcal X})$ be a ${\mathcal T}$-structure.
	The restriction functor
	\[ {\mathrm{Str}^\mathrm{loc}}_{\mathcal T}({\mathcal X})_{/{\mathcal O}} \to {\mathrm{Str}^\mathrm{loc}}_{\mathcal T}({\mathcal X})_{/{\mathcal O} \circ \varphi} \]
	is a right adjoint.
\end{cor}

\section{Local analytic rings} \label{sec:local_analytic_rings}

The goal of this section is to analyze in detail the category ${\mathrm{Str}^\mathrm{loc}}_{{\mathcal T}_{\mathrm{an}}}({\mathcal S})$. We introduce a special notation:

\begin{defin}
	The $\infty$-category of \emph{local analytic rings} is the $\infty$-category ${\mathrm{Str}^\mathrm{loc}}_{{\mathcal T}_{\mathrm{an}}}({\mathcal S})$.
	We will denote it by $\mathrm{AnRing}_{\mathbb C}$.
\end{defin}

\subsection{Functional spectrum} \label{subsec:functional_spectrum}

Let us fix some notations. We will denote by ${\mathcal H}_0$ the ${{\mathcal T}_{\mathrm{an}}}$-structure defined by
\[ {\mathcal H}_0(U) = \operatorname{Hom}_{{\mathcal T}_{\mathrm{an}}}(\{*\}, U) \simeq U \in {\mathcal S} \]
where $U$ is thought as a simple (discrete) set.
Observe that ${\mathcal H}_0$ is indeed a ${{\mathcal T}_{\mathrm{an}}}$-structure and that it is moreover discrete.
If ${\mathcal O} \in {\mathrm{AnRing}_{\mathbb C}}$, then we see that, by the Yoneda lemma, one has
\[ \operatorname{Map}_{\mathrm{AnRing}_{\mathbb C}}({\mathcal H}_0, F) \simeq F(\{*\}) \simeq \{*\} \]
In particular, ${\mathcal H}_0$ is an initial object in ${\mathrm{AnRing}_{\mathbb C}}$.
On the other side the category ${\mathrm{AnRing}_{\mathbb C}}$ has a final object, which we will denote by $0$.
We will refer to it as the \emph{null analytic ring}.
It is the constant functor associated to $\{*\} \in {\mathcal S}$.

The main goal of this section is to prove the following result:

\begin{thm} \label{thm:local_analytic_rings_augmented}
	Let ${\mathrm{Str}}_{{\mathcal T}_{\mathrm{an}}}({\mathcal S})'$ be the full subcategory of ${\mathrm{Str}}_{{\mathcal T}_{\mathrm{an}}}({\mathcal S})$ spanned by every object but the null analytic ring.
	Then ${\mathcal H}_0$ is a final object in ${\mathrm{Str}}_{{\mathcal T}_{\mathrm{an}}}({\mathcal S})'$. Moreover, for every ${\mathcal O} \in {\mathrm{Str}}_{{\mathcal T}_{\mathrm{an}}}({\mathcal S})'$ the canonical map ${\mathcal O} \to {\mathcal H}_0$ is a local transformation.
\end{thm}

\begin{cor} \label{cor:Strloc_Tan_presentable}
	One has ${\mathrm{Str}}_{{\mathcal T}_{\mathrm{an}}}({\mathcal S})' \simeq {\mathrm{Str}}_{{\mathcal T}_{\mathrm{an}}}({\mathcal S})_{/{\mathcal H}_0} \simeq \mathrm{AnRing}_{\mathbb C/{\mathcal H}_0}$.
	In particular, all these categories are presentable.
\end{cor}

\begin{proof}
	The two equivalences follow directly from the theorem. The last assertion is a consequence of \cref{prop:O_premodules_presentable}.
\end{proof}

\begin{rem}
	We also see that the above corollary, together with \cref{lem:strloc_fully_faitful_str}, implies that ${\mathrm{Str}^\mathrm{loc}}_{{\mathcal T}_{\mathrm{an}}}({\mathcal S}) \simeq {\mathrm{Str}}_{{\mathcal T}_{\mathrm{an}}}({\mathcal S})$.
\end{rem}

\begin{defin} \label{def:local_analytic_ring}
	We will refer to the $\infty$-category ${\mathrm{Str}^\mathrm{loc}}_{{\mathcal T}_{\mathrm{an}}}({\mathcal S})_{/{\mathcal H}_0}$ as the $\infty$-category of \emph{local analytic rings}.
	We will denote it ${\mathrm{AnRing}^{\mathrm{loc}}_{\mathbb C}}$.
\end{defin}

Observe that since ${\mathcal H}_0$ is $0$-truncated, any map ${\mathcal O} \to {\mathcal H}_0$ factors uniquely as ${\mathcal O} \to \tau_{\le 0} {\mathcal O} \to {\mathcal H}_0$.
Observe also that since ${{\mathcal T}_{\mathrm{an}}}$ is compatible with $n$-truncations (see \cite[Definition 3.3.2]{DAG-V} for the definition and \cite[Proposition 11.4]{DAG-IX} for a proof), $\tau_{\le 0} {\mathcal O}$ is a ${{\mathcal T}_{\mathrm{an}}}$-structure in ${\mathcal S}$, and the morphism ${\mathcal O} \to \tau_{\le 0} {\mathcal O}$ is a local morphism.
Therefore, to prove \cref{thm:local_analytic_rings_augmented} it is sufficient to deal with \emph{discrete} local analytic rings.
We will denote by ${\mathrm{AnRing}_{\mathbb C}}^0$ the full subcategory of ${\mathrm{AnRing}_{\mathbb C}}$ spanned by discrete objects.
The above considerations show that the first statement of \cref{thm:local_analytic_rings_augmented} is in fact equivalent to prove the following key result:

\begin{prop} \label{prop:discrete_analytic_rings_augmented}
	Let ${\mathcal O} \in {\mathrm{AnRing}_{\mathbb C}}^0$ and suppose ${\mathcal O} \ne 0$. Then ${\mathcal O}{^\mathrm{alg}}$ is a local ring with residue field $\mathbb C$.
\end{prop}

\begin{rem}
	Observe that this result would be false if we replaced ${{\mathcal T}_{\mathrm{an}}}$ with ${\mathcal T}_{\mathrm{Zar}}$.
\end{rem}

We will need several lemmas.

\begin{lem}
	Let ${\mathcal O} \in {\mathrm{AnRing}^{\mathrm{loc}}_{\mathbb C}}$ and suppose ${\mathcal O} \ne 0$. Then ${\mathcal O}(\emptyset) = \emptyset$.
\end{lem}

\begin{proof}
	Since we have an admissible morphism $\emptyset \to \{*\}$, we see that ${\mathcal O}(\emptyset) \to {\mathcal O}(\{*\})$ is a monomorphism.
	Therefore ${\mathcal O}(\emptyset)$ has cardinality at most $1$. Suppose by contradiction that ${\mathcal O}(\emptyset)$ has one element.
	Then, consider the commutative triangle
	\[ \begin{tikzcd}
	\emptyset \arrow{d} \arrow{dr} \\
	\mathbb C \arrow{r}{t_a} & \mathbb C ,
	\end{tikzcd} \]
	where $t_a$ denotes the translation by the element $a \in \mathbb C$. Applying ${\mathcal O}$ we would get another commutative triangle, and the morphism ${\mathcal O}(\emptyset) \to {\mathcal O}(\mathbb C)$ would select an element $f \in {\mathcal O}(\mathbb C)$ such that $f + a = f$. This is impossible as soon as $a \ne 0$. Therefore ${\mathcal O}(\emptyset) = \emptyset$.
\end{proof}

Let ${\mathcal O} \in {\mathrm{AnRing}_{\mathbb C}}^0$ be a non-null discrete analytic ring and let $f \in {\mathcal O}{^\mathrm{alg}}$. Define the spectrum of $f$ as follows:
\[ \sigma_{\mathcal O}(f) = \sigma(f) \coloneqq \{a \in \mathbb C \mid f - a \not \in ({\mathcal O}{^\mathrm{alg}})^\times\} \]

\begin{lem} \label{lem:discrete_analytic_rings_invertible_elements}
	Let $f \in {\mathcal O}{^\mathrm{alg}}$. Then $a \not \in \sigma(f)$ if and only if $f \in {\mathcal O}(\mathbb C^*_a)$, where $\mathbb C^*_a \coloneqq \mathbb C \setminus \{a\}$.
\end{lem}

\begin{proof}
	When $a = 0$, we will simply write $\mathbb C^*$ instead of $\mathbb C_0$.
	If $f \in {\mathcal O}(\mathbb C^*_a)$ then $f - a \in {\mathcal O}(\mathbb C^*)$ and therefore $f-a \in ({\mathcal O}{^\mathrm{alg}})^\times$.
	Conversely, suppose that $f - a$ is invertible. Consider the pullback square
	\[ \begin{tikzcd}
	\mathbb C^* \times \mathbb C^* \arrow{r} \arrow{d} & \mathbb C \times \mathbb C \arrow{d} \\
	\mathbb C^* \arrow{r} & \mathbb C
	\end{tikzcd} \]
	The horizontal morphisms are open immersions, and therefore the induced square
	\[ \begin{tikzcd}
	{\mathcal O}(\mathbb C^*) \times {\mathcal O}(\mathbb C^*) \arrow{r} \arrow{d} & {\mathcal O}(\mathbb C) \times {\mathcal O}(\mathbb C) \arrow{d} \\
	{\mathcal O}(\mathbb C^*) \arrow{r} & {\mathcal O}(\mathbb C)
	\end{tikzcd} \]
	is a pullback square as well. Let $g$ be an inverse for $f - a$. Then the pair $((f-a, g), 1)$ determines an element of the above pullback. This means that both $f-a$ and $g$ factors through ${\mathcal O}(\mathbb C^*)$, completing the proof.
\end{proof}

\begin{lem}
	If $f \in {\mathcal O}{^\mathrm{alg}}$ then $\sigma(f) \ne \emptyset$.
\end{lem}

\begin{proof}
	Suppose by contradiction that $\sigma(f) = \emptyset$ for some $f \in {\mathcal O}{^\mathrm{alg}}$.
	Then for every $a \in \mathbb C$, $f - a$ is invertible, that is, $f \in {\mathcal O}(\mathbb C^*_a)$ for every $a \in \mathbb C$.
	Fix $a \in \mathbb C$ and write
	\[ \mathbb C^*_a = \bigcup_{n > 0} \left( \mathbb C \setminus \overline{\mathrm D}\left(a, \frac{1}{n}\right) \right) \]
	It follows that there exists $\varepsilon > 0$ such that $f \in {\mathcal O}(\mathbb C \setminus \overline{\mathrm D}(a, \varepsilon))$.
	Since ${\mathcal O}(\emptyset) = \emptyset$, we conclude that $f \not \in {\mathcal O}(\mathrm D(a, \varepsilon))$. But then we can cover $\mathbb C$ with open subsets $U_i$ such that $f \not \in {\mathcal O}(U_i)$ for every $i$, which is a contradiction.
\end{proof}

\begin{lem}
	Suppose that ${\mathcal O} \in {\mathrm{AnRing}_{\mathbb C}}^0$ and let $f \in {\mathcal O}{^\mathrm{alg}}$.
	Then $\sigma(f)$ has at most one point.
\end{lem}

\begin{proof}
	Suppose that $a,b \in \sigma(f)$, with $a \ne b$.
	Then $f \not \in {\mathcal O}(\mathbb C^*_a)$ and $f \not \in {\mathcal O}(\mathbb C_b)$, which implies that $f \in {\mathcal O}(\mathrm D(a, \varepsilon)) \cap {\mathcal O}(\mathrm D(b, \eta))$ for every $\varepsilon, \eta > 0$. This is a contradiction since ${\mathcal O}(\emptyset) = \emptyset$.
\end{proof}

\begin{cor}
	Let ${\mathcal O} \in {\mathrm{AnRing}_{\mathbb C}}^0$ be a non-null discrete analytic ring. For every $f \in {\mathcal O}{^\mathrm{alg}}$, $\sigma(f)$ consists precisely of one point.
\end{cor}

At this point we obtain a well defined function of sets
\[ \sigma \colon {\mathcal O}{^\mathrm{alg}} \to \mathbb C \]

\begin{lem} \label{lem:sigma_morphism_of_rings}
	One has $\sigma(f + g) = \sigma(f) + \sigma(g)$ and $\sigma(fg) = \sigma(f) \sigma(g)$.
\end{lem}

\begin{proof}
	Set $\sigma(f) = a$ and $\sigma(g) = b$. It will be sufficient to check that $a + b \in \sigma(f + g)$.
	Suppose that there exists $h \in {\mathcal O}{^\mathrm{alg}}$ such that $(f + g - a - b)h = 1$. Then the ideal generated by $f - a$ and $g -b$ is the unit ideal.
	However, ${\mathcal O}$ is local and both $f - a$ and $g - b$ belongs to the maximal ideal. This is therefore impossible.
	Similarly, suppose there exists $h$ such that $(fg - ab) h = 1$. Then $(f-a)gh + ah(g-b) = 1$, and the proof proceeds as for the addition.
\end{proof}

We can now prove \cref{prop:discrete_analytic_rings_augmented}:

\begin{proof}[Proof of \cref{prop:discrete_analytic_rings_augmented}.]
	Let ${\mathcal O} \in {\mathrm{AnRing}_{\mathbb C}}$ be a non-null analytic structure.
	We deduce from \cref{lem:sigma_morphism_of_rings} that $\sigma \colon {\mathcal O}{^\mathrm{alg}} \to \mathbb C$ is a morphism of rings.
	\Cref{lem:discrete_analytic_rings_invertible_elements} shows that the kernel of $\sigma$ is precisely the set of non invertible elements of ${\mathcal O}{^\mathrm{alg}}$.
	It follows that $\ker(\sigma)$ has to be the unique maximal ideal of ${\mathcal O}{^\mathrm{alg}}$. Since $\sigma$ is evidently surjective, the proof is complete.
\end{proof}

The next result achieves the proof of \cref{thm:local_analytic_rings_augmented}.

\begin{prop}
	Let ${\mathcal O}_1 \in {\mathrm{AnRing}_{\mathbb C}}^0$ be a non-null discrete local analytic ring. Then ${\mathcal O}{^\mathrm{alg}}_1$ is canonically augmented toward $\mathbb C$ and the morphism $\sigma_1 \colon {\mathcal O}_1 \to \mathbb C$ comes from a \emph{local} morphism $s_1 \colon {\mathcal O}_1 \to {\mathcal H}_0$.
	Moreover, if ${\mathcal O}_2$ is any other non-null discrete local analytic ring and $f \colon {\mathcal O}_1 \to {\mathcal O}_2$ is any morphism between them, one has $s_2 \circ f = s_1$.
\end{prop}

\begin{proof}
	Let us prove that ${\mathcal O}_1 \to {\mathcal H}_0$ is a natural transformation. This is equivalent to prove that if $f \in {\mathcal O}(U)$ then $\sigma(f) \in U$. Let $a \not \in U$.
	Then there exists $\varepsilon > 0$ such that $f \not \in {\mathcal O}(\mathrm D(a, \varepsilon))$, which means that $f \in {\mathcal O}_1(\mathbb C^*_a)$. Therefore $f - a$ is invertible, i.e.\ $a \not \in \sigma(f)$. Therefore $\sigma(f) \in U$.
	
	Let us now prove that ${\mathcal O}_1 \to {\mathcal H}_0$ is a local morphism. We claim that if $f \in {\mathcal O}_1(\mathbb C)$ and $\sigma(f) \in U$, then $f \in {\mathcal O}_1(U)$.
	Let $a = \sigma(f)$, so that $f \not \in {\mathcal O}_1(\mathbb C^*_a)$. Since $\mathbb C = \mathbb C^*_a \cup U$, we conclude that $f \in {\mathcal O}_1(U)$. This completes the proof.
	
	Finally, we prove the last statement. Let $x \in {\mathcal O}_1(\mathbb C)$, $a = \sigma_1(x)$. Then $x - a$ is non-invertible. This means that $x-a$ belongs to ${\mathcal O}_1(\mathrm D(0, \varepsilon))$ for every $\varepsilon > 0$. Then $f(x - a) = f(x) - a$ belongs to ${\mathcal O}_2(\mathrm D(0, \varepsilon))$ for every $\varepsilon > 0$.
		It follows that $\sigma_2(f(x) - a) \in \mathrm D(0, \varepsilon)$ for every $\varepsilon > 0$.
		Therefore $\sigma_2(f(x) - a) = 0$, i.e.\ $\sigma_2(f(x)) = a = \sigma_1(x)$.
\end{proof}

\begin{cor}
	For any non-null local analytic ring ${\mathcal O} \in {\mathrm{AnRing}_{\mathbb C}}$ (not necessarily discrete), the canonical morphism ${\mathcal H}_0 \to {\mathcal O}$ is a local morphism.
\end{cor}

\subsection{Quotients of local analytic rings}

This subsection and the next one are not, strictly speaking, needed for this work.
Nevertheless, we chose to include them for two reasons: on one side, they help in building a $1$-categorical intuition on what kind of objects (local) analytic rings are, and on the other side the results collected here will be used in \cite{Porta_Analytic_deformation_2015}.

Let $A \in {\mathrm{AnRing}_{\mathbb C}}^0$ and let $I \subset A{^\mathrm{alg}}$ be a (proper) ideal.
Observe that $I$ is contained in the maximal ideal of $A$ and therefore $\sigma(I) = 0$.

\begin{construction}
	We will construct an analytic ring $A / I \colon {{{\mathcal T}_{\mathrm{an}}}^{\; \! 0}} \to {\mathrm{Set}}$ (we are using the results of \cref{sec:pregeometry_germs}).
	Define
	\[ (A/I)(p,\mathbb C^n) \coloneqq \{\overline{a} \in (A{^\mathrm{alg}} / I)^n \mid \sigma(a) = p\} \]
	Observe that this is well defined. Indeed, if $\overline{a} = \overline{b}$, then $a - b \in I^n$ and therefore $\sigma(a - b) = 0$.
	Let now $\varphi \colon (p, \mathbb C^n) \to (q, \mathbb C^m)$ be a germ of holomorphic map.
	Set
	\[ (A/I)(\varphi)(\overline{a}) \coloneqq \overline{A(\varphi)(a)} \]
	This is well defined. Indeed, if $\overline{a} = \overline{b}$, then we can write $a = b + x$, where $x \in I^n$.
	Therefore
	\[ A(\varphi)(a) = A(\varphi)(b) + \sum A \left( \frac{\partial f}{\partial z_i} \right)(b) x_i + \sum A(g_{ij})(b+x) x_i x_j \]
	and therefore
	\[ \overline{A(\varphi)(a)} = \overline{A(\varphi)(b)} \]
	It follows from the construction that $A/I$ commutes with products.
	In particular, we see that $A / I$ defines a local analytic ring.
	Moreover, the natural projection map $p \colon A \to A/I$ is a morphism of local analytic rings.
	Observe finally that $(A/I){^\mathrm{alg}} = A{^\mathrm{alg}} / I$.
\end{construction}

\begin{lem}
	Let $f \colon A \to B$ be a morphism of discrete local analytic rings.
	The following are equivalent:
	\begin{enumerate}
		\item the induced map $A(0, \mathbb C) \to B(0, \mathbb C)$ is surjective;
		\item the induced map $A(0, \mathbb C^n) \to B(0, \mathbb C^n)$ is surjective for every $n \ge 0$;
		\item the induced map $A(p, \mathbb C^n) \to B(p, \mathbb C^n)$ is surjective for every $n$ and every $p$.
	\end{enumerate}
\end{lem}

\begin{proof}
	The implications $3. \Rightarrow 2.$ and $2. \Rightarrow 1.$ are trivial.
	Suppose that $A(0, \mathbb C) \to B(0, \mathbb C)$ is surjective.
	Since both $A$ and $B$ commute with products, 2.\ follows at once.
	Since $(p, \mathbb C^n)$ and $(0, \mathbb C^n)$ are isomorphic, 3.\ follows as well.
\end{proof}

\begin{defin}
	We will say that a morphism of discrete local analytic rings $f \colon A \to B$ is \emph{surjective} if it satisfies the equivalent conditions of the above lemma.
\end{defin}

\begin{lem}
	Let $f \colon A \to B$ be a surjective map of discrete local analytic rings and set $I \coloneqq \ker(A{^\mathrm{alg}} \to B{^\mathrm{alg}})$.
	Then for every other discrete local analytic ring $C$, we have
	\[ \operatorname{Hom}(B,C) = \{g \colon A \to C \mid g{^\mathrm{alg}}(I) = 0\} \]
\end{lem}

\begin{proof}
	Let $g \colon A \to C$ be such that $g{^\mathrm{alg}}(I) = 0$.
	Define $\widetilde{g} \colon B \to C$ as follows.
	If $b \in B(p, \mathbb C^n)$, we can choose an element $a \in A(p, \mathbb C^n)$ such that $f(a) = b$.
	Define
	\[ \widetilde{g}(b) \coloneqq g(a) \]
	If $a' \in A(p, \mathbb C^n)$ is another element satisfying $f(a') = b$, we see that $a - a' \in A(0, \mathbb C^n)$ and $a - a' \in I$.
	Therefore $f(a - a') = 0$, i.e.\ $f(a) = f(a')$.
	This shows that $\widetilde{g}$ is well defined.
	Let us show that $\widetilde{g}$ is a natural transformation.
	If $\varphi \colon (p, \mathbb C^n) \to (q, \mathbb C^m)$ is a germ of a holomorphic map, we can consider the diagram
	\[ \begin{tikzcd}
	A(p, \mathbb C^n) \arrow{d} \arrow{r} & B(p, \mathbb C^n) \arrow{d} \arrow{r} & C(p, \mathbb C^n) \arrow{d} \\
	A(q, \mathbb C^m) \arrow{r} & B(q, \mathbb C^m) \arrow{r} & C(q, \mathbb C^m)
	\end{tikzcd} \]
	Since the maps $A(p, \mathbb C^n) \to B(p, \mathbb C^n)$ and $A(q, \mathbb C^m) \to B(q, \mathbb C^m)$ are surjective, in order to check that the square on the right commutes, it is sufficient to check that the outer rectangle does.
	This follows from the hypothesis that $g$ is a natural transformation.
\end{proof}

Let $A \in {\mathrm{AnRing}_{\mathbb C}}^0$ and let $I \subset A{^\mathrm{alg}}$ be an ideal.
Then the canonical map $A \to (A / I)$ is surjective and therefore it is characterized by the universal property of the previous lemma.
Vice-versa, if $A \to B$ is a surjective morphism of local analytic rings and $I \coloneqq \ker(A{^\mathrm{alg}} \to B{^\mathrm{alg}})$, then $B \simeq A / I$.

For the next result, we will need a bit of notation.
As we discussed in the introduction, analytic rings are a way of axiomatizing the holomorphic functional calculus enjoyed by commutative Banach $\mathbb C$-algebras.
For this reason, we choose to denote the pushouts $B \coprod_A C$ in ${\mathrm{AnRing}_{\mathbb C}}^0$ by $B {\widehat{\otimes}}_A C$.

\begin{cor} \label{cor:discrete_alg_surjective}
	Suppose that $f \colon A \to B$ is a surjective morphism of local analytic rings.
	For every other morphism $g \colon A \to C$, we have
	\[ (B {\widehat{\otimes}}_A C){^\mathrm{alg}} \simeq B{^\mathrm{alg}} \otimes_{A{^\mathrm{alg}}} C{^\mathrm{alg}} \]
	In other words, the functor $(-){^\mathrm{alg}}$ commutes with pushouts in which at least one of the two maps is surjective.
\end{cor}

\begin{proof}
	Write $B = A / I$.
	Then
	\[ \operatorname{Hom}(B {\widehat{\otimes}}_A C, R) = \{g \colon C \to R \mid g{^\mathrm{alg}}(I C{^\mathrm{alg}}) = 0\} \]
	In other words, $B {\widehat{\otimes}}_A C = C / IC{^\mathrm{alg}}$.
	It follows that $(B {\widehat{\otimes}}_A C){^\mathrm{alg}} = C{^\mathrm{alg}} / IC{^\mathrm{alg}} = B{^\mathrm{alg}} \otimes_{A{^\mathrm{alg}}} C{^\mathrm{alg}}$.
\end{proof}

\begin{rem}
	This result, whose counterpart for Banach algebras is very well-known, has been generalized to a great extent by J.\ Lurie in \cite{DAG-IX}.
	We can without any doubt say that this generalization is the most important idea contained there.
	The precise way of formulating this uses once again the language of pregeometries, and the relevant notion is that of \emph{unramified transformation of pregeometries} (see loc.\ cit., Definition 10.1) The condition of being unramified is a rather simple test for a morphism of pregeometries (see loc.\ cit., Remark 10.2 for a simplified formulation), and yet it has the powerful consequence expressed in loc.\ cit., Proposition 10.3.
	The proof of this striking result passes through the highly technical Proposition 2.2, which is perhaps an interesting result on its own.
\end{rem}

\subsection{Analytic algebras}

We will denote by $\mathrm{LRingSpaces}$ the ($1$-)category of locally ringed topological spaces.

Let $A_n = \mathbb C\{z_1, \ldots, z_n\}$ be the algebra of germs of holomorphic functions around $0 \in \mathbb C^n$.
This is an analytic algebra in the sense of Malgrange.
We define an enhancement of $A_n$ to the setting of analytic rings as follows.
Define ${\mathcal H}_n \colon {{\mathcal T}_{\mathrm{an}}} \to {\mathrm{Set}}$ by setting:
\begin{align*}
{\mathcal H}_n(U) & \coloneqq \{\textrm{germs of holomorphic functions } (0, \mathbb C^n) \to U\} \\
& = \mathrm{Map}_{\mathrm{LRingSpaces}}(({\mathcal S}, A_n), U)
\end{align*}
It is clear that ${\mathcal H}_n$ defines a functor ${{\mathcal T}_{\mathrm{an}}} \to {\mathrm{Set}}$ and that moreover ${\mathcal H}_n(\mathbb C) = A_n$.

\begin{lem}
	The functor ${\mathcal H}_n$ is a ${{\mathcal T}_{\mathrm{an}}}$-structure on ${\mathrm{Set}}$.
\end{lem}

\begin{proof}
	It follows directly from the definition that ${\mathcal H}_n$ commutes with all the limits that exist in ${{\mathcal T}_{\mathrm{an}}}$.
	In particular, it commutes with products and with admissible pullbacks.
	Let now $\{U_i \subset U\}$ be an open cover of $U$.
	The morphisms $U_i \subset U$ are open immersions in the category of locally ringed spaces, and therefore they enjoy the following universal property: a morphism $Z \to U$ from a locally ringed space $Z$ factors through $U_i$ if and only if it factors topologically.
	Therefore, we see that for every morphism $({\mathcal S}, A_n) \to U$, there exists an index $i$ such that this morphism factorizes through $U_i$.
	It follows that the map $\coprod {\mathcal H}_n(U_i) \to {\mathcal H}_n(U)$ is surjective.
\end{proof}

Our next goal is to characterize ${\mathcal H}_n$ with a universal property.

\begin{prop} \label{prop:UMP_of_H_n}
	Let $B \in {\mathrm{AnRing}_{\mathbb C}}^0$. Then
	\[ \operatorname{Hom}_{{\mathrm{AnRing}_{\mathbb C}}^0}({\mathcal H}_n, B) = \{(b_1, \ldots, b_n) \in B(\mathbb C) \mid \sigma(b_1) = \cdots = \sigma(b_n) = 0 \} \]
\end{prop}

\begin{proof}
	Suppose given a natural transformation $\varphi \colon {\mathcal H}_n \to B$.
	Let us denote by $z_1, \ldots, z_n$ the germs at $0$ of the coordinate functions on $\mathbb C^n$. That is, we have $z_1, \ldots, z_n \in {\mathcal H}_n(\mathbb C) = {\mathcal H}_n$.
	Their image via $\varphi$ define elements $b_1, \ldots, b_n \in B$.
	Moreover, \cref{prop:morphisms_respect_sigma} shows that $\sigma(b_i) = \sigma(\varphi(z_i)) = \varphi(z_i) = 0$.
	
	Conversely, suppose given elements $b_1, \ldots, b_n \in B(\mathbb C)$ such that $\sigma(b_1) = \cdots = \sigma(b_n) = 0$, we can define a morphism $\varphi \colon {\mathcal H}_n \to B$ as follows.
	Let $U \in {{\mathcal T}_{\mathrm{an}}}$. An element $f \in {\mathcal H}_n(U)$ can be represented by a holomorphic function $\widetilde{f} \colon V \to U$ for some open neighborhood $V$ of $0 \in \mathbb C^n$.
	Since the functional spectrum of $b_1, \ldots, b_n$ is zero, we see that
	\[ (b_1, \ldots, b_n) \in B(V) \]
	We therefore define
	\[ \varphi_U(f) \coloneqq B(\widetilde{f})(b_1, \ldots, b_n) \in B(U) \]
	If $\widetilde{g} \colon V' \to U$ is another representation of $f$, then we can suppose that $V' \subset V$ and that $\widetilde{g} = \widetilde{f} |_{V'}$.
	In this case, we have a commutative triangle
	\[ \begin{tikzcd}
	B(V') \arrow{r} \arrow{dr}[swap]{B(\widetilde{g})} & B(V) \arrow{d}{B(\widetilde{f})} \\
	& B(U)
	\end{tikzcd} \]
	and $(b_1, \ldots, b_n) \in B(V')$. This shows that the definition of $\varphi_U(f)$ does not depend on the choice of the representation of $f$.
	It is straightforward to check that $\varphi_U$ defines a natural transformation.
	Finally, the two constructions we defined are clearly one the inverse of each other.
\end{proof}

\begin{cor}
	The coproduct of ${\mathcal H}_n$ and ${\mathcal H}_m$ in ${\mathrm{AnRing}_{\mathbb C}}^0$ is ${\mathcal H}_{n + m}$.
\end{cor}

{\ignorespaces}

Let now $I$ be an ideal of ${\mathcal H}_n$.
This corresponds to a germ of {$\mathbb C$-analytic\xspace} space $(0, X)$, together with a map of germs $j \colon (0, X) \to (0, \mathbb C^n)$.
Define a functor ${\mathcal H}_n / I \colon {{\mathcal T}_{\mathrm{an}}} \to {\mathrm{Set}}$ as follows:
\[ ({\mathcal H}_n / I)(U) \coloneqq \{\text{germs of holomorphic maps } (0, X) \to U\} \]
Clearly, ${\mathcal H}_n / I \in {\mathrm{AnRing}_{\mathbb C}}^0$. Moreover, $({\mathcal H}_n /I)(\mathbb C) = {\mathcal H}_n / I$.
Composition with $j$ produces a natural transformation $\pi \colon {\mathcal H}_n \to {\mathcal H}_n / I$.

\begin{prop}
	Let $B \in {\mathrm{AnRing}_{\mathbb C}}^0$.
	We have
	\[ \operatorname{Hom}({\mathcal H}_n / I, B) = \{\varphi \colon {\mathcal H}_n \to B \mid \varphi{^\mathrm{alg}}(I) = 0\} \]
\end{prop}

\begin{proof}
	Given $\psi \colon {\mathcal H}_n / I \to B$ we obtain $\varphi \coloneqq \psi \circ \pi \colon {\mathcal H}_n \to B$, and it is clear that $\varphi{^\mathrm{alg}}(I) = 0$.
	Conversely, if $\varphi \colon {\mathcal H}_n \to B$ is such that $\varphi{^\mathrm{alg}}$ factors through ${\mathcal H}_n / I$, then define $\psi \colon {\mathcal H}_n / I \to B$ as follows.
	Fix $U \in {{\mathcal T}_{\mathrm{an}}}$ and let $f \in ({\mathcal H}_n / I)(U)$ be a germ of a holomorphic function $(0, X) \to U$.
	Using Oka principle, {\ignorespaces} we can represent $f$ as a holomorphic map $V \to U$, where $V$ is some open neighborhood of $0$ in $\mathbb C^n$.
	In other words, $f$ is the restriction of some $\widetilde{f} \in {\mathcal H}_n(U)$.
	Set
	\[ \psi_U(f) \coloneqq \varphi_U(\widetilde{f}) \]
	It is clear that this definition doesn't change if we shrink $V$.
	On the other hand, suppose that $\widetilde{g} \colon V \to U$ is another extension of the germ $f \colon (0, X) \to U$.
	Then $\widetilde{f} - \widetilde{g} \in {\mathcal H}_n(U) \subset {\mathcal H}_n(\mathbb C)^m$ belong to the ideal $I^m$.
	Therefore $\varphi_U(\widetilde{f}) = \varphi_U(\widetilde{g})$ by hypothesis.
\end{proof}

\begin{rem}
	This proposition allows to identify ${\mathcal H}_n / I$ with the categorical quotient of ${\mathcal H}_n$ by the ideal $I$, as defined in the previous subsection.
\end{rem}

\begin{cor} \label{cor:Malgrange_algebras}
	The category of analytic algebras in the sense of Malgrange embeds fully faithfully in ${\mathrm{AnRing}_{\mathbb C}}^0$.
\end{cor}

\begin{rem}
	This is a low-tech version of the fully faithful embedding of \cite[Theorem 12.8]{DAG-IX}.
\end{rem}

\subsection{Pregeometries of germs} \label{sec:pregeometry_germs}

\cref{cor:Strloc_Tan_presentable} is to some extent a very surprising result, at least for two reasons: it doesn't hold when the topos is different from ${\mathcal S}$, and it doesn't hold for a general pregeometry, even for ${\mathcal X} = {\mathcal S}$.
We will show in this subsection that there is a deeper reason for this result. Namely, we will describe ${\mathrm{Str}^\mathrm{loc}}_{{\mathcal T}_{\mathrm{an}}}({\mathcal S})$ as algebras for a suitable (multi-sorted) Lawvere theory.
We could therefore deduce \cref{cor:Strloc_Tan_presentable} directly from \cref{lem:Str_presentable}.

Let us define the category ${{{\mathcal T}_{\mathrm{an}}}^{\; \! 0}}$ as follows:
\begin{enumerate}
	\item the objects of ${{{\mathcal T}_{\mathrm{an}}}^{\; \! 0}}$ are pairs $(n, p)$ where $n$ is a natural number and $p \in \mathbb C^n$.
	\item A morphism from $(n,p)$ to $(m,q)$ is a germ of holomorphic function from $(p, \mathbb C^n)$ to $(q, \mathbb C^m)$.
\end{enumerate}

Observe that the category ${{{\mathcal T}_{\mathrm{an}}}^{\; \! 0}}$ has products. Indeed, $(n,p) \times (m,q) = (n+m, (p,q))$.
Let us say that a morphism $f \colon (n,p) \to (m,q)$ in ${{{\mathcal T}_{\mathrm{an}}}^{\; \! 0}}$ if it can be represented by an open immersion. In particular we have $n = m$.
Let $f \colon (n,p_0) \to (n,p_1)$ be an admissible morphism and let $g \colon (m,q) \to (n,p_1)$ be any other morphism.
Then
\[ \begin{tikzcd}
(m,q) \arrow{r}{\mathrm{id}} \arrow{d}[swap]{g + p_0 - p_1} & (m,q) \arrow{d}{g} \\
(n,p_0) \arrow{r}{f} & (n, p_1)
\end{tikzcd} \]
is a pullback diagram in ${{{\mathcal T}_{\mathrm{an}}}^{\; \! 0}}$.
In particular, we see that ${{{\mathcal T}_{\mathrm{an}}}^{\; \! 0}}$ can be equipped with a structure of semi-discrete pregeometry.
However, since all admissible morphisms are isomorphisms, we see that ${{{\mathcal T}_{\mathrm{an}}}^{\; \! 0}}$ is in fact a discrete pregeometry.
In other words, ${{{\mathcal T}_{\mathrm{an}}}^{\; \! 0}}$ is a (multi-sorted) Lawvere theory.

\begin{thm} \label{thm:Strloc_Tan_S_Lawvere}
	There exists an equivalence of $\infty$-categories ${\mathrm{Str}}_{{\mathcal T}_{\mathrm{an}}}({\mathcal S}) \simeq {\mathrm{Str}}_{{{\mathcal T}_{\mathrm{an}}}^{\; \! 0}}({\mathcal S})$.
\end{thm}

\begin{proof}[Sketch of a proof.]
	Define a functor
	\[ {\mathrm{Str}}_{{{\mathcal T}_{\mathrm{an}}}^{\; \! 0}}({\mathcal S}) \to {\mathrm{Str}}_{{\mathcal T}_{\mathrm{an}}}({\mathcal S}) \]
	by sending ${\mathcal O}^0 \in {\mathrm{Str}}_{{{\mathcal T}_{\mathrm{an}}}^{\; \! 0}}({\mathcal S})$ to the functor ${\mathcal O} \colon {{\mathcal T}_{\mathrm{an}}} \to {\mathcal S}$ defined by
	\[ {\mathcal O}(U) \coloneqq \coprod_{p \in U} {\mathcal O}^0(n,p) \]
	It is clear that ${\mathcal O}$ is a functor and that it depends functorially on ${\mathcal O}^0$.
	Finally, it commutes with admissible pullbacks and it takes coverings to effective epimorphisms.
	
	On the other direction, if ${\mathcal O} \in {\mathrm{Str}}_{{\mathcal T}_{\mathrm{an}}}({\mathcal S})$ is a ${{\mathcal T}_{\mathrm{an}}}$-structure on ${\mathcal S}$, we define ${\mathcal O}^0 \colon {{{\mathcal T}_{\mathrm{an}}}^{\; \! 0}} \to {\mathcal S}$ as follows:
	\[ {\mathcal O}(n,p) \coloneqq {\mathcal O}(\mathbb C^n) \times_{\mathbb C^n} \{p\} \]
	where the map defining this \emph{homotopy} fiber product is ${\mathcal O}(\mathbb C^n) \to \pi_0({\mathcal O})(\mathbb C^n) \xrightarrow{\sigma} \mathbb C^n$.
	It is straightforward to extend this definition on morphisms.
	Finally, it is clear that these two functors are mutually inverse.
\end{proof}

\subsection{Strict models} \label{subsec:strict_models}

It will be of some importance to develop in this section an explicit presentation for the $\infty$-category ${\mathrm{Str}^\mathrm{loc}}_{{\mathcal T}_{\mathrm{an}}}({\mathcal S})$.
In virtue of \cref{thm:Strloc_Tan_S_Lawvere}, we can replace ${{\mathcal T}_{\mathrm{an}}}$ with the discrete pregeometry ${{{\mathcal T}_{\mathrm{an}}}^{\; \! 0}}$.
In particular, we have
\[ {\mathrm{Str}^\mathrm{loc}}_{{\mathcal T}_{\mathrm{an}}}({\mathcal S}) = \operatorname{Fun}^\times({{{\mathcal T}_{\mathrm{an}}}^{\; \! 0}}, {\mathcal S}) \]
where the right hand side is the category of functors that preserve \emph{finite} products.
Consider now the $1$-category $\mathrm{sAnRing} \coloneqq \mathrm{Funct}^\times({{{\mathcal T}_{\mathrm{an}}}^{\; \! 0}}, {\mathrm{sSet}})$ of (strict) functors preserving \emph{finite} products.
Invoking \cite[5.5.9.1, 5.5.9.2]{HTT}, we see that defining a morphism $f \colon F \to G$ in $\mathrm{sAnRing}$ to be:
\begin{enumerate}
	\item a fibration if it is an objectwise fibration;
	\item a weak equivalence if it is an objectwise weak equivalence.
\end{enumerate}
we obtain a model structure on $\mathrm{sAnRing}$ whose underlying $\infty$-category is precisely $\operatorname{Fun}^\times({{{\mathcal T}_{\mathrm{an}}}^{\; \! 0}}, {\mathcal S})$.

\begin{rem}
	Observe that \cref{thm:Strloc_Tan_S_Lawvere} implies in particular that ${\mathrm{Str}}_{{\mathcal T}_{\mathrm{an}}}({\mathrm{Set}}) \simeq {\mathrm{Str}}_{{{\mathcal T}_{\mathrm{an}}}^{\; \! 0}}({\mathrm{Set}})$.
	We therefore deduce an equivalence of $1$-categories
	\[ {\mathrm{Str}}_{{\mathcal T}_{\mathrm{an}}}({\mathrm{sSet}}) \simeq \mathrm{Funct}^\times({{{\mathcal T}_{\mathrm{an}}}^{\; \! 0}}, {\mathrm{sSet}}) \]
	Under this equivalence, the object ${\mathcal H}_n$ can be identified with the functor ${{{\mathcal T}_{\mathrm{an}}}^{\; \! 0}} \to {\mathrm{sSet}}$ corepresented by the germ $(0, \mathbb C^n)$.
	The universal property explained in \cref{prop:UMP_of_H_n} becomes therefore a direct consequence of the Yoneda lemma.
\end{rem}

We will need a better understanding of this model structure.

\begin{lem}
	Let $f_0 \colon K_0 \to L_0$ and $f_1 \colon K_1 \to L_1$ be Kan fibrations (resp.\ weak equivalences) of simplicial sets.
	Then $f_0 \times f_1 \colon K_0 \times K_1 \to L_0 \times L_1$ is a Kan fibration (resp.\ a weak equivalence).
\end{lem}

\begin{proof}
	The statement on weak equivalences follows at once from the fact that the geometric realization functor commutes with products and so do the homotopy groups.
	The statement on Kan fibrations follows immediately from the characterization with the right lifting property against anodyne maps.
\end{proof}

\begin{prop} \label{prop:strict_model_fibration_weak_equiv}
	Let $f \colon R \to S$ be a morphism in $\mathrm{sAnRing}$.
	The following conditions are equivalent:
	\begin{enumerate}
		\item the induced morphism $R(0, \mathbb C) \to S(0, \mathbb C)$ is a fibration (resp.\ a weak equivalence);
		\item the induced morphism $R(0, \mathbb C^n) \to S(0, \mathbb C^n)$ is a fibration (resp.\ a weak equivalence) for every $n$;
		\item the induced morphism $R(p, \mathbb C^n) \to S(p, \mathbb C^n)$ is a fibration (resp.\ a weak equivalence) for every $n$ and every $p \in \mathbb C^n$.
	\end{enumerate}
\end{prop}

\begin{proof}
	The implications $3. \Rightarrow 2.$ and $2. \Rightarrow 1.$ are obvious.
	Suppose now that $R(0, \mathbb C) \to S(0, \mathbb C)$ is a fibration (resp.\ a weak equivalence).
	Since both $R$ and $S$ strictly preserve products, $2.$ follows from the previous lemma.
	Finally, $3.$ follows because we have isomorphisms $(0, \mathbb C^n) \simeq (p, \mathbb C^n)$.
\end{proof}

\begin{cor} \label{cor:strict_model_every_object_is_fibrant}
	Every object in $\mathrm{sAnRing}$ is fibrant.
\end{cor}

\begin{proof}
	Observe that the germ $(0, \mathbb C)$ has an abelian group structure (given by pointwise sum of germs of holomorphic functions).
	Since every $F \in \mathrm{sAnRing}$ commutes with products, we see that $F(0, \mathbb C)$ is a simplicial group and therefore it is a Kan complex.
	At this point, the conclusion follows directly from \cref{prop:strict_model_fibration_weak_equiv}.
\end{proof}

Consider the forgetful functor
\[ U \colon \mathrm{sAnRing} \to {\mathrm{sSet}} \]
given by $U(R) \coloneqq R(0, \mathbb C)$.

\begin{rem}
	Observe that this functor is quite different from the underlying algebra functor $\overline{\Phi}$.
	Indeed, $\overline{\Phi}(R) = \coprod_{p \in \mathbb C} R(p, \mathbb C)$.
\end{rem}

$U$ commutes with limits and sifted colimits. In particular, since both $\mathrm{sAnRing}$ and ${\mathrm{sSet}}$ are presentable, it follows that it has a left adjoint, which we denote by
\[ {\mathcal H}\{-\} \colon {\mathrm{sSet}} \to \mathrm{sAnRing} \]
It follows from \cref{prop:strict_model_fibration_weak_equiv} that $U$ is a right Quillen functor, so that ${\mathcal H}\{-\} \dashv U$ is a Quillen adjunction.

\begin{prop}
	The collection of morphisms $I \coloneqq \{{\mathcal H}\{\partial \Delta^n\} \to {\mathcal H}\{\Delta^n\}\}_{n \in \mathbb N}$ is a set of generating cofibrations for $\mathrm{sAnRing}$.
	The family of morphisms $J \coloneqq \{{\mathcal H}\{\Lambda^n_i\} \to {\mathcal H}\{\Delta^n\}\}_{n \in \mathbb N, 0 \le i \le n}$ is a set of generating trivial cofibrations for $\mathrm{sAnRing}$.
\end{prop}

\begin{proof}
	We already remarked that ${\mathcal H}\{-\}$ is a left Quillen functor.
	It follows that all the morphisms in $J$ are trivial cofibrations and all the morphisms in $I$ are cofibrations.
	Let now $p \colon R \to S$ a morphism in $\mathrm{sAnRing}$.
	Observe that the lifting problems
	\[ \begin{tikzcd}
		{{\mathcal H}\{\Lambda^n_i\}} \arrow{r} \arrow{d} & R \arrow{d}{p} \\
		{{\mathcal H}\{\Delta^n\}} \arrow{r} \arrow[dotted]{ur} & S
	\end{tikzcd} \qquad \begin{tikzcd}
		{\mathcal H}\{\partial \Delta^n\} \arrow{r} \arrow{d} & R \arrow{d}{p} \\
		{\mathcal H}\{\Delta^n\} \arrow{r} \arrow[dotted]{ur} & S
	\end{tikzcd} \]
	are respectively equivalent to the lifting problems
	\[ \begin{tikzcd}
		\Lambda^n_i \arrow{d} \arrow{r} & U(R) \arrow{d}{U(f)} \\
		\Delta^n \arrow{r} \arrow[dotted]{ur} & U(S)
	\end{tikzcd} \qquad \begin{tikzcd}
		\partial \Delta^n \arrow{r} \arrow{d} & U(R) \arrow{d}{U(f)} \\
		\Delta^n \arrow{r} \arrow[dotted]{ur} & U(S)
	\end{tikzcd} \]
	In particular, we see that $p$ has the right lifting property with respect to maps in $J$ (resp.\ $I$) if and only if $U(p)$ is a fibration (resp.\ a trivial fibration).
	This completes the proof.
\end{proof}

\begin{lem}
	Let $I_n$ be the set with $n$ elements.
	Then $\pi_0({\mathcal H}\{I_n\}) = {\mathcal H}_n$.
\end{lem}

\begin{proof}
	The two universal properties match.
\end{proof}

\begin{prop} \label{prop:pi_*_of_the_point}
	For every $n$, we have
	\[ \pi_i( \overline{\Phi}({\mathcal H}\{\Delta^n\}) ) = \begin{cases} \mathbb C\{z\} & \text{if } $n = 0$ \\ 0 & \text{otherwise.} \end{cases} \]
	where $\mathbb C\{z\}$ denotes the algebra of germs of holomorphic functions around $0 \in \mathbb C$.
\end{prop}

\begin{proof}
	The morphism $\Delta^0 \to \partial \Delta^n$ selecting the $0$-th vertex is an acyclic cofibration.
	It follows that $\overline{\Phi}({\mathcal H}\{\partial \Delta^n\})$ is weakly equivalent to $\overline{\Phi}({\mathcal H}\{\Delta^0\})$.
	It is therefore sufficient to prove the proposition for $n = 0$.
	The case $i = 0$ follows directly from the previous lemma.
	If $i \ge 1$, we first observe that it is sufficient to show that $\pi_i({\mathcal H}\{\Delta^0\}(0, \mathbb C)) = 0$.
	For a simplicial set $K$ with finitely many simplexes in every degree, the $n$-simplexes of ${\mathcal H}\{\Delta^0\}(0, \mathbb C)$ are in bijection with $\mathbb C\{z_1, \ldots, z_m\}_0$, where the variables $z_j$ correspond to the $n$-simplexes of $K$.
	In particular, we see that ${\mathcal H}\{\Delta^0\}$ is the constant simplicial set associated to $\mathbb C\{z\}_0$.
	The proof is therefore complete.
\end{proof}

\begin{prop} \label{prop:pi_0_of_the_boundary}
	We have $\pi_0( \overline{\Phi}({\mathcal H}\{\partial \Delta^n\} )) = \mathbb C\{z\}$.
\end{prop}

\begin{proof}
	It will be enough to prove that $\pi_0( {\mathcal H}\{\partial \Delta^n\}(0, \mathbb C))$ can be identified with the set of germs of holomorphic maps $(0, \mathbb C) \to (0, \mathbb C)$.
	We can explicitly represent ${\mathcal H}\{\partial \Delta^n \}(0, \mathbb C)$ as a simplicial algebra whose $1$-skeleton is
	\[ \mathbb C\{z_{0,0}, z_{0,1}, \ldots, z_{i,j}, \ldots, z_{n-1,n} \}_0 \to \mathbb C\{z_0, \ldots, z_n\}_0 \]
	where the variable $z_{ij}$ corresponds to the edge of $\partial \Delta^n$ connecting the vertex $i$ to the vertex $j$ (hence $i \le j$) and
	\[ d_0(z_{i,j}) = z_i, \qquad d_1(z_{i,j}) = - z_j, \qquad s_0(z_i) = z_{i,i} \]
	Applying Dold-Kan and computing the $0$-th cohomology of the resulting complex, we deduce that
	\[ \pi_0( {\mathcal H}\{ \partial \Delta^n \}(0, \mathbb C) ) = \mathbb C\{z_0, \ldots, z_n\}_0 / (z_i + z_j)_{0 \le i < j \le n} \simeq \mathbb C\{z\}_0 \]
\end{proof}

\section{The functor of points} \label{sec:analytic_functor_of_points}

With \cref{def:derived_canal_space} we are giving a ``structured space'' perspective on derived {$\mathbb C$-analytic\xspace} spaces.
However, in the practice, it is often important to rely on the dual perspective of the functor of points.
The goal of this section is precisely to introduce this alternative point of view and to prove a comparison result with \cref{def:derived_canal_space}.
Concretely, this means that we will have to discuss the following points:
\begin{enumerate}
	\item we have to exhibit a geometric context $({\mathcal C}, \tau, \mathbf P)$ (in the precise sense of \cite[Definition 2.11]{Porta_Yu_Higher_analytic_stacks_2014}) in such a way that ${\mathrm{dAn}_{\mathbb C}}$ can be exhibited as a full subcategory of ${\mathrm{Sh}}({\mathcal C}, \tau)$.
	\item We will have to identify the essential image of (a large subcategory of) ${\mathrm{dAn}_{\mathbb C}}$ with the geometric stacks in the sense of \cite[Definition 2.15]{Porta_Yu_Higher_analytic_stacks_2014}.
\end{enumerate}
In dealing with the first point of the list, one could try to invoke \cite[Theorem 2.4.1]{DAG-V}.
Unfortunately, this result is not sufficient for our purposes: indeed, there is discrepancy between the ${{\mathcal T}_{\mathrm{an}}}$-schemes and the derived {$\mathbb C$-analytic\xspace} spaces (see \cite[Corollary 12.22 and Proposition 12.23]{DAG-IX} for a detailed discussion of this difference).
Nevertheless, the same idea of the proof in loc.\ cit.\ applies to our context and therefore allows to deal successfully with the first point.

\subsection{The geometric context} \label{subsec:analytic_functor_of_points_I}

\begin{defin}
	We let ${\mathrm{Stn}^{\mathrm{der}}_{\mathbb C}}$ be the full subcategory of ${\mathrm{dAn}_{\mathbb C}}$ spanned by those derived {$\mathbb C$-analytic\xspace} spaces $({\mathcal X}, {\mathcal O}_{\mathcal X})$ such that $({\mathcal X}, \pi_0 {\mathcal O}_{\mathcal X}{^\mathrm{alg}})$ is of the form $\Phi(S)$, with $S$ a Stein space. We will refer to ${\mathrm{Stn}^{\mathrm{der}}_{\mathbb C}}$ as the category of derived ({$\mathbb C$-analytic\xspace}) Stein spaces.
\end{defin}

The notion of \'etale morphism of ${{\mathcal T}_{\mathrm{an}}}$-structured topoi (see \cref{def:etale_and_closed_morphisms}) induces a Grothendieck topology $\tau$ on the $\infty$-category ${\mathrm{Stn}^{\mathrm{der}}_{\mathbb C}}$.
We define a functor $\widetilde{\phi} \colon {\mathrm{dAn}_{\mathbb C}} \to {\mathrm{PSh}}({\mathrm{Stn}^{\mathrm{der}}_{\mathbb C}})$ as follows:
\[ {\mathrm{dAn}_{\mathbb C}} \xrightarrow{j} \operatorname{Fun}({\mathrm{dAn}_{\mathbb C}}^{\mathrm{op}}, {\mathcal S}) \to \operatorname{Fun}(({\mathrm{Stn}^{\mathrm{der}}_{\mathbb C}})^{\mathrm{op}}, {\mathcal S}) \]
Our first task is to show that $\widetilde{\phi}$ is fully faithful.
To achieve this, we will show that it can be factored as
\[ \phi \colon {\mathrm{dAn}_{\mathbb C}} \to {\mathrm{Sh}}({\mathrm{Stn}^{\mathrm{der}}_{\mathbb C}}, \tau) \]
and that moreover $\phi$ is a fully faithful functor.
We will need a couple of preliminary facts.

\begin{lem} \label{lem:derived_canal_spaces_hypercomplete}
	Let $({\mathcal X}, {\mathcal O}_{\mathcal X})$ be a derived {$\mathbb C$-analytic\xspace} space.
	Then ${\mathcal X}$ is hypercomplete.
\end{lem}

\begin{proof}
	Let ${\mathcal Y}$ be an $\infty$-topos. The hypercompletion of ${\mathcal Y}$ is defined to be the full subcategory ${\mathcal Y}^{\wedge}$ of ${\mathcal Y}$ spanned by hypercomplete objects. Moreover, ${\mathcal Y}$ is said to be hypercomplete if the inclusion ${\mathcal Y}^\wedge \subset {\mathcal Y}$ is a categorical equivalence.
	Therefore, it follows from \cite[6.5.2.21 6.5.2.22]{HTT} that being hypercomplete is local on ${\mathcal Y}$, in the sense that if there are objects $U_i \in {\mathcal Y}$ such that the map $\coprod U_i \to \mathbf 1_{\mathcal Y}$ is an effective epimorphism and each $\infty$-topos ${\mathcal Y}_{/U_i}$ is hypercomplete, then the same goes for ${\mathcal Y}$.
	
	In the case of our interest, the definition of derived {$\mathbb C$-analytic\xspace} space allows us to choose objects $\coprod U_i \in {\mathcal X}$ such that each $({\mathcal X}_{/U_i}, \pi_0 {\mathcal O}_{\mathcal X}{^\mathrm{alg}}|_{U_i})$ is a local model for {$\mathbb C$-analytic\xspace} spaces.
	In particular, we can identify ${\mathcal X}_{/U_i} \simeq {\mathrm{Sh}}(X_i)$, where $X_i$ is a locally compact Hausdorff space of finite covering dimension.
	It follows that the homotopy dimension of $X_i$ is finite and therefore that ${\mathcal X}_{/U_i}$ is hypercomplete.
\end{proof}

Let $X = ({\mathcal X}, {\mathcal O}_{\mathcal X})$ be a derived {$\mathbb C$-analytic\xspace} space and suppose that ${\mathcal X}$ is $0$-localic.
We will denote by $X{^\mathrm{top}}$ the underlying topological space of ${\mathcal X}$.
As every point in a {$\mathbb C$-analytic\xspace} space has a Stein open neighbourhood, we can find an hypercover $U^\bullet$ of $X{^\mathrm{top}}$ made by open Stein subsets of $X{^\mathrm{top}}$.
Each open subset of $X{^\mathrm{top}}$ defines a (discrete) object of the $\infty$-topos ${\mathcal X} = {\mathrm{Sh}}(X{^\mathrm{top}})$.
Therefore, the simplicial object $U^\bullet$ determines by composition with the Yoneda embedding a simplicial object in $X{^\mathrm{top}}$, which we will still denote $U^\bullet$.
It follows directly from the definitions and from the criterion \cite[7.2.1.14]{HTT} that $U^\bullet$ is an hypercover of ${\mathcal X}$.
Let us denote by $X^n$ the derived {$\mathbb C$-analytic\xspace} space defined as
\[ X^n \coloneqq ({\mathcal X}_{/\mathcal U^n}, {\mathcal O}_{\mathcal X}|_{\mathcal U^n}) \]
The universal property of \'etale morphisms of structured ${{\mathcal T}_{\mathrm{an}}}$-topoi (see \cite[Remark 2.3.4]{DAG-V}) shows that we can arrange the $X^n$ into a simplicial object $X^\bullet$ in the $\infty$-category ${\mathcal T\mathrm{op}}({{\mathcal T}_{\mathrm{an}}})$.
We claim that the geometric realization of the diagram $X^\bullet$ in ${\mathcal T\mathrm{op}}({{\mathcal T}_{\mathrm{an}}})$ (and hence in ${\mathrm{dAn}_{\mathbb C}}$) coincides precisely with the original derived {$\mathbb C$-analytic\xspace} space $X$.
To prove this it is sufficient to show the following two assertions:
\begin{enumerate}
	\item one has $|{\mathcal X}_{/U^\bullet}| \simeq {\mathcal X}$;
	\item Let $j^n_* \colon {\mathcal X}_{/U^\bullet} \to {\mathcal X}$ be the given geometric morphism. Then in ${\mathrm{Str}^\mathrm{loc}}_{{\mathcal T}_{\mathrm{an}}}({\mathcal X})$ one has
	\[ {\mathcal O}_{\mathcal X} \simeq \varprojlim_{\mathbf \Delta} j^n_* {\mathcal O}_{\mathcal X} |_{U^n} \]
\end{enumerate}

The first assertion is a consequence of the general descent theory for $\infty$-topoi (see \cite[Theorem 6.1.3.9]{HTT}) and the fact that in ${\mathcal X}$ one has
\[ |U^\bullet| \simeq U \]
as it follows from \cref{lem:derived_canal_spaces_hypercomplete} and from \cite[6.5.3.12]{HTT}.
{\ignorespaces}
As for the second statement, since ${\mathrm{Str}^\mathrm{loc}}_{{\mathcal T}_{\mathrm{an}}}({\mathcal X})$ is closed under limits in $\operatorname{Fun}({{\mathcal T}_{\mathrm{an}}}, {\mathcal X})$, it will be enough to show that for every $V \in {{\mathcal T}_{\mathrm{an}}}$, in ${\mathcal X}$ one has
\[ {\mathcal O}_{\mathcal X}(V) \simeq \varprojlim_{\mathbf \Delta} j^n_* {\mathcal O}_{\mathcal X}(V) |_{U^n} \]
Since ${\mathcal X} = {\mathrm{Sh}}(X{^\mathrm{top}})$ is closed under limits in ${\mathrm{PSh}}(X{^\mathrm{top}})$, we see that it is enough to prove that whenever $W$ is an open of $X{^\mathrm{top}}$ one has
\[ {\mathcal O}_{\mathcal X}(V)(W) \simeq \varprojlim_{\mathbf \Delta} {\mathcal O}_{\mathcal X}(V)(U^n \times W) \]
Since $U^\bullet \times W$ is an hypercover of $W$, this statement is equivalent to say that each ${\mathcal O}_{\mathcal X}(V) \in {\mathcal X}$ is hypercomplete, which is obvious since ${\mathcal X}$ is itself hypercomplete in virtue of \cref{lem:derived_canal_spaces_hypercomplete}.

Summarizing, we proved the following:

\begin{lem} \label{lem:derived_Stein_hypercover}
	Let $X = ({\mathcal X}, {\mathcal O}_{\mathcal X})$ be a derived {$\mathbb C$-analytic\xspace} space such that ${\mathcal X}$ is $0$-localic.
	Then there exists an hypercover $\mathcal U^\bullet$ in ${\mathcal X}$ such that each derived {$\mathbb C$-analytic\xspace} space $X^n \coloneqq ({\mathcal X}_{/\mathcal U^n}, {\mathcal O}_{\mathcal X}|_{\mathcal U^n})$ is a derived Stein space.
	Moreover, for each such hypercover, the geometric realization of $X^\bullet$ in ${\mathrm{dAn}_{\mathbb C}}$ is $X$.
\end{lem}

\begin{cor} \label{cor:little_phi_hypersheaves}
	The Grothendieck topology $\tau$ on ${\mathrm{Stn}^{\mathrm{der}}_{\mathbb C}}$ is subcanonical.
	Moreover, if $X = ({\mathcal X}, {\mathcal O}_{\mathcal X})$ is a derived {$\mathbb C$-analytic\xspace} space, then $\widetilde{\phi}(X)$ belongs to the hypercompletion of ${\mathrm{Sh}}({\mathrm{Stn}^{\mathrm{der}}_{\mathbb C}}, \tau)$.
\end{cor}

One can prove something more general. Indeed, the notion of \'etale morphism of ${{\mathcal T}_{\mathrm{an}}}$-structures defines also a Grothendieck topology $\tau'$ on ${\mathrm{dAn}_{\mathbb C}}$ and the same arguments used above, one can prove that the Yoneda embedding
\[ j \colon {\mathrm{dAn}_{\mathbb C}} \to {\mathrm{PSh}}({\mathrm{dAn}_{\mathbb C}}) \]
factors through ${\mathrm{Sh}}({\mathrm{dAn}_{\mathbb C}}, \tau')$.
This has the following useful consequence:

\begin{cor} \label{cor:little_phi_commutes_with_etale_colimits}
	Let $X = ({\mathcal X}, {\mathcal O}_{\mathcal X})$ be a derived {$\mathbb C$-analytic\xspace} space and let $p \colon U \to \mathbf 1_{\mathcal X}$ be an effective epimorphism.
	Let $U^\bullet$ be the \v{C}ech nerve of $p$ and set $X^n \coloneqq ({\mathcal X}_{/U^n}, {\mathcal O}_{\mathcal X}|_{U^n})$.
	Then in ${\mathrm{Sh}}({\mathrm{Stn}^{\mathrm{der}}_{\mathbb C}}, \tau)$ one has
	\[ \phi(X) \simeq \operatorname*{colim}_{\mathbf \Delta} \phi(X^\bullet) \]
\end{cor}

\begin{proof}
	Let us temporarily denote by $\psi \colon {\mathrm{dAn}_{\mathbb C}} \to {\mathrm{Sh}}({\mathrm{dAn}_{\mathbb C}}, \tau')$ the functor obtained by factorizing the Yoneda embedding $j \colon {\mathrm{dAn}_{\mathbb C}} \to {\mathrm{PSh}}({\mathrm{dAn}_{\mathbb C}})$.
	The above discussion makes clear that the relation
	\[ \psi(X) \simeq \operatorname*{colim}_{\mathbf \Delta} \psi(X^\bullet) \]
	holds in ${\mathrm{Sh}}({\mathrm{dAn}_{\mathbb C}}, \tau')$.
	The morphism of sites $({\mathrm{Stn}^{\mathrm{der}}_{\mathbb C}}, \tau) \to ({\mathrm{dAn}_{\mathbb C}}, \tau')$ is both continuous and cocontinuous.
	It follows from \cite[Lemma 2.30]{Porta_Yu_Higher_analytic_stacks_2014} that the restriction along this functor is a left adjoint.
	{\ignorespaces}
	In particular, it commutes with colimits, so that the proof is complete.
\end{proof}

\begin{prop} \label{prop:little_phi_fully_faithful}
	The functor $\phi$ is fully faithful.
\end{prop}

\begin{proof}
	Let $X,Y \in {\mathrm{dAn}_{\mathbb C}}$ and consider the natural map
	\[ \psi_{X,Y} \colon \operatorname{Map}_{\mathrm{dAn}_{\mathbb C}}(X,Y) \to \operatorname{Map}_{{\mathrm{Sh}}({\mathrm{Stn}^{\mathrm{der}}_{\mathbb C}}, \tau)}(\phi(X), \phi(Y)) \]
	Keeping $Y$ fixed, consider the full subcategory ${\mathcal C}$ of ${\mathrm{dAn}_{\mathbb C}}$ spanned by those $X$ for which $\psi_{X,Y}$ is an equivalence.
	
	Choose objects $U_i$ of $X = ({\mathcal X}, {\mathcal O}_{\mathcal X})$ in such a way that $p \colon \coprod U_i \to \mathbf 1_{\mathcal X}$ is an effective epimorphism and that each ${\mathcal X}_{/U_i}$ is $0$-localic. Let $U \coloneqq \coprod U_i$ and let $U^\bullet$ be the \v{C}ech nerve of $p$. Finally, set $X^n \coloneqq ({\mathcal X}_{/U^n}, {\mathcal O}_{\mathcal X}|_{U^n})$.
	It follows from \cref{cor:little_phi_commutes_with_etale_colimits} that
	\[ \phi(X) \simeq \operatorname*{colim}_{\mathbf \Delta} \phi(X^n) \]
	We can therefore reduce ourselves to the case where ${\mathcal X}$ is $0$-localic.
	Invoking \cref{lem:derived_Stein_hypercover} and reasoning in the same way, we can further reduce to the case where $X \in {\mathrm{Stn}^{\mathrm{der}}_{\mathbb C}}$, and the lemma is now a restatement of the Yoneda lemma.
\end{proof}

\subsection{Geometric stacks}

We know turn to the second main goal of this section, that is, the characterization of the essential image of $\phi \colon {\mathrm{dAn}_{\mathbb C}} \to {\mathrm{Sh}}({\mathrm{Stn}^{\mathrm{der}}_{\mathbb C}}, \tau)$.
First of all, let us observe that letting ${\mathbf P}_{\mathrm{\acute{e}t}}$ be the collection of \'etale morphisms in ${\mathrm{Stn}^{\mathrm{der}}_{\mathbb C}}$, the triple $({\mathrm{Stn}^{\mathrm{der}}_{\mathbb C}}, \tau, {\mathbf P}_{\mathrm{\acute{e}t}})$ becomes a geometric context in the sense of \cite[Definition 2.11]{Porta_Yu_Higher_analytic_stacks_2014}.
We will refer to geometric stacks relative to this context as \emph{derived {Deligne-Mumford\xspace} analytic stacks}.
We will denote by $\mathrm{DM}$ the full subcategory of ${\mathrm{Sh}}({\mathrm{Stn}^{\mathrm{der}}_{\mathbb C}}, \tau)^\wedge$ spanned by derived {Deligne-Mumford\xspace} analytic stacks.
We will further denote by $\mathrm{DM}_n$ the full subcategory of $\mathrm{DM}$ spanned by $n$-geometric stacks.
On the other side, we will denote by ${\mathrm{dAn}_{\mathbb C}}^{\le n}$ the full subcategory of ${\mathrm{dAn}_{\mathbb C}}$ spanned by those derived {$\mathbb C$-analytic\xspace} spaces $({\mathcal X}, {\mathcal O}_{\mathcal X})$ for which ${\mathcal X}$ is an $n$-localic topos. Finally, we will denote by ${\mathrm{dAn}_{\mathbb C}}^{\mathrm{loc}}$ the reunion of all the subcategories ${\mathrm{dAn}_{\mathbb C}}^{\le n}$ as $n$ ranges through the integers.
With these notations, we can formulate the main result of this section:

\begin{thm} \label{thm:analytic_functor_of_points}
	For every $n \ge 0$, the functor $\phi \colon {\mathrm{dAn}_{\mathbb C}} \to {\mathrm{Sh}}({\mathrm{Stn}^{\mathrm{der}}_{\mathbb C}}, \tau)$ restricts to an equivalence ${\mathrm{dAn}_{\mathbb C}}^{\le n} \simeq \mathrm{DM}_n$.
\end{thm}

\begin{rem}
	The previous theorem (when combined with \cref{cor:comparison_analytic_DM_stacks}) provides a large generalization of the comparison result \cite[Theorem 12.8]{DAG-IX}.
	See however Remark 12.25 loc.\ cit.\ for something going in this direction.
\end{rem}

The following lemma will be used repeatedly in the proof of \cref{thm:analytic_functor_of_points}.

\begin{lem} \label{lem:little_phi_reflects_etale}
	Let $f \colon U \to V$ be a morphism in ${{\mathcal T}_{\mathrm{an}}}$.
	Let us write $\operatorname{Spec}^{{\mathcal T}_{\mathrm{an}}}(U) = ({\mathcal X}_U, {\mathcal O}_U)$ and $\operatorname{Spec}^{{\mathcal T}_{\mathrm{an}}}(V) = ({\mathcal X}_V, {\mathcal O}_V)$.
	The following conditions are equivalent:
	\begin{enumerate}
		\item $f$ is \'etale (resp.\ an \'etale monomorphism);
		\item the induced morphism $(f_*, \varphi) \colon \operatorname{Spec}^{{\mathcal T}_{\mathrm{an}}}(U) \to \operatorname{Spec}^{{\mathcal T}_{\mathrm{an}}}(V)$ is an \'etale morphism of derived {$\mathbb C$-analytic\xspace} spaces (resp.\ is \'etale and ${\mathcal X}_U \simeq ({\mathcal X}_V)_{/W}$ where $W$ is a discrete object of ${\mathcal X}_V$).
	\end{enumerate}
\end{lem}

\begin{proof}
	Suppose first that $f$ is \'etale. Then the assertion follows immediately from \cite[Example 2.3.8]{DAG-V}.
	If moreover $f$ is a monomorphism, then ${\mathcal X}_U$ can be identified with the topos associated to an open subset $W$ of $V{^\mathrm{top}}$ and therefore ${\mathcal X}_U \simeq ({\mathcal X}_V)_{/W}$ and $W$ is clearly a discrete object of ${\mathcal X}_V$.
	Vice-versa, suppose that $(f_*, \varphi) \colon \operatorname{Spec}^{{\mathcal T}_{\mathrm{an}}}(U) \to \operatorname{Spec}^{{\mathcal T}_{\mathrm{an}}}(V)$ is an \'etale morphism of derived {$\mathbb C$-analytic\xspace} spaces.
	Then $f$ can be recovered entirely from the geometric morphism $f_* \colon {\mathcal X}_U \to {\mathcal X}_V$ and the very definition of \'etale morphisms of $\infty$-topoi makes clear that $f$ has to be a local homeomorphism, which is injective if ${\mathcal X}_U$ is the \'etale subtopos of ${\mathcal X}_V$ associated to a discrete object. Since it was a holomorphic map to begin with, it also follows that $f$ is a local biholomorphism, thus completing the proof.
\end{proof}

The proof of \cref{thm:analytic_functor_of_points} naturally splits into two parts:
\begin{enumerate}
	\item to show that whenever $X \in {\mathrm{dAn}_{\mathbb C}}^{\le n+1}$ the sheaf $\phi(X)$ is an $n$-geometric stack;
	\item to show that every $n$-geometric stack arises in this way.
\end{enumerate}
The rest of this subsection will be devoted to the proof of the first step.
We will deal with the second one in \cref{subsec:essential_surjectivity}, after having discussed the notion of truncation at length in \cref{subsec:truncation_of_derived_spaces}.

\Cref{cor:little_phi_hypersheaves} shows that $\phi$ factors in fact through ${\mathrm{Sh}}({\mathrm{Stn}^{\mathrm{der}}_{\mathbb C}}, \tau)^\wedge$, and therefore we only have to prove that $\phi(X)$ admits an $n$-atlas and that the diagonal of $\phi(X)$ is $(n-1)$-representable.
In order to simplify the proof, it will be convenient for the rest of this section to introduce the category ${\mathrm{dAn}_{\mathbb C}}^0$ of $0$-localic derived {$\mathbb C$-analytic\xspace} spaces.
We will endow ${\mathrm{dAn}_{\mathbb C}}^0$ with the \'etale topology $\sigma$ and we will let $\mathbf Q_{\mathrm{\acute{e}t}}$ be the collection of \'etale morphisms.
The morphism
\[ ({\mathrm{Stn}^{\mathrm{der}}_{\mathbb C}}, \tau, {\mathbf P}_{\mathrm{\acute{e}t}}) \to ({\mathrm{dAn}_{\mathbb C}}^0, \sigma, \mathbf Q_{\mathrm{\acute{e}t}}) \]
is a morphism of geometric contexts, and every object in ${\mathrm{dAn}_{\mathbb C}}^0$ defines a $1$-geometric stack for the Stein context.
Therefore we can apply \cite[Lemma 2.36]{Porta_Yu_Higher_analytic_stacks_2014} and temporarily work with the geometric stacks for the second context.
This has an important advantage that we are going to describe.
The geometric context $({\mathrm{dAn}_{\mathbb C}}^0, \sigma, \mathbf Q_{\mathrm{\acute{e}t}})$ is closed under $\sigma$-descent in the sense that whenever we are given a morphism $F \to G$ in ${\mathrm{Sh}}({\mathrm{dAn}_{\mathbb C}}^0, \tau)^\wedge$ with $G$ being representable, if there exists a $\tau$-covering $G_i \to G$ (with $G_i$ being representable) such that each base change $G_i \times_G F$ is representable, then the same goes for $F$.
This is easy to see: indeed, we have $G_i = \phi(U_i)$ for certain $U_i \in {\mathrm{dAn}_{\mathbb C}}^0$ and $G = \phi(Y)$. Since $\phi$ is fully faithful in virtue of \cref{prop:little_phi_fully_faithful}, the morphisms $G_i \to G$ are represented by morphisms $U_i \to Y$ which are \'etale by \cref{lem:little_phi_reflects_etale}.
Let $U \coloneqq \coprod U_i$, $p \colon U \to Y$ the total morphism and $U^\bullet$ the \v{C}ech nerve of $p$.
Since $\phi$ commutes with limits and with disjoint unions, we see that $\phi(U^\bullet)$ is the \v{C}ech nerve of $\coprod G_i \to G$.
By hypothesis, each level of the simplicial object $\phi(U^\bullet) \times_G F$ is representable. Since $\phi$ is fully faithful, we can form a simplicial object $V^\bullet$ in ${\mathrm{dAn}_{\mathbb C}}^0$ in such a way that $\phi(V^\bullet) \simeq \phi(U^\bullet) \times_G F$.
{\ignorespaces}
\Cref{lem:little_phi_reflects_etale} shows that all the face maps in $V^\bullet$ \'etale.
As consequence, the geometric realization of $V^\bullet$ exists in ${\mathrm{dAn}_{\mathbb C}}^0$.
Let us denote by $X$ this colimit.
\Cref{cor:little_phi_commutes_with_etale_colimits} shows that
\[ \phi(X) \simeq |\phi(V^\bullet)| \simeq |\phi(U^\bullet) \times_G F| \simeq F . \]
Thus the proof of the claim is completed.

The importance of this fact is the following: since the geometric context $({\mathrm{dAn}_{\mathbb C}}^0, \sigma, \mathbf Q_{\mathrm{\acute{e}t}})$ is closed under $\sigma$-descent, the tool of groupoid presentations becomes available for geometric stacks on this context.
In particular, the requirement on the representability of the diagonal in the definition of geometric stack is now superfluous (cf.\ \cite[Remark 1.3.3.2]{HAG-II}).
Therefore, in proving that $\phi(X)$ is geometric with respect to this context, we will only need to show that it admits an $n$-atlas. 
Taking into account the shift of the geometric level coming from \cite[Lemma 2.36]{Porta_Yu_Higher_analytic_stacks_2014}, we have now to show that whenever $n \ge -1$ and $X \in {\mathrm{dAn}_{\mathbb C}}^{\le n + 1}$, then $\phi(X)$ is $n$-geometric.

\begin{proof}[Proof of \cref{thm:analytic_functor_of_points}, Step 1.]
	Let $X$ be an $n$-localic derived {$\mathbb C$-analytic\xspace} space. We will prove the statement by induction on $n$.
	If $n = 0$, then $\phi(X)$ coincides with the representable sheaf associated to $X$ itself. Therefore, $\phi(X)$ is $(-1)$-geometric and hence the base of the induction holds.

	Let now $n \ge 0$ and suppose that the statement has already been proved for $n$-localic derived {$\mathbb C$-analytic\xspace} spaces.
	Fix $X = ({\mathcal X}, {\mathcal O}_{\mathcal X}) \in {\mathrm{dAn}_{\mathbb C}}^{\le n + 1}$.
	Choose an effective epimorphism $\coprod U_i \to \mathbf 1_{\mathcal X}$ in such a way that $\infty$-topos ${\mathcal X}_{/U_i}$ is $0$-localic.
	We claim that each $U_i$ is $n$-truncated.
	Assuming for the moment this fact, we see that the morphisms $\phi(U_i) \to \phi(X)$ are \'etale and the total morphism $\coprod \phi(U_i) \to \phi(X)$ is an effective epimorphism in ${\mathrm{Sh}}({\mathrm{Stn}^{\mathrm{der}}_{\mathbb C}}, \tau)^\wedge$. Since each $U_i$ is $n$-truncated, we conclude that the morphisms $\phi(U_i) \to \phi(X)$ are representable by sheaves of the form $\phi(Y)$ with $Y \in {\mathrm{dAn}_{\mathbb C}}^{\le n}$. The inductive hypothesis implies that each $\phi(Y)$ is $(n-1)$-geometric and therefore we obtain that $\phi(X)$ is $n$-geometric.
	
	We are therefore left to prove the claim. To do so, we replace $X$ with ${\mathrm{t}_0}(X) \coloneqq ({\mathcal X}, \pi_0 {\mathcal O}_{\mathcal X})$. We can review the latter as a ${\mathcal G}^0_{\mathrm{an}}$-structured topos, where ${\mathcal G}^0_{\mathrm{an}}$ is a $0$-truncated geometric envelope for the pregeometry ${{\mathcal T}_{\mathrm{an}}}$.
	\cite[Lemma 2.6.19]{DAG-V} shows that $({\mathcal X}, \pi_0 {\mathcal O}_{\mathcal X})$ is an $(n+1)$-truncated object of ${\mathcal T\mathrm{op}}({\mathcal G}^0_{\mathrm{an}})$.
	Let $F$ be the functor on $\mathrm{Stn}_{\mathbb C}$ represented by ${\mathrm{t}_0}(X)$ and let $F_i$ the one represented by ${\mathrm{t}_0}(U_i)$.
	It is enough to show that the fibers of $F_i(S) \to F(S)$ are $n$-truncated for every Stein space $S$. Since ${\mathrm{t}_0}(X)$ is $(n+1)$-truncated, we see that $F(S)$ is $(n+1)$-truncated. On the other side, $F_i(S)$ is $0$-truncated. The assertion now follows from the long exact sequence of homotopy groups.	
\end{proof}

Before discussing the essential surjectivity, we will need a digression on the truncation functor for derived {$\mathbb C$-analytic\xspace} spaces.

\subsection{Truncations of derived {$\mathbb C$-analytic\xspace} spaces} \label{subsec:truncation_of_derived_spaces}

Let $(\mathrm{Stn}, \tau_0, \mathbf P_0)$ be the geometric context introduced in \cite[§3.2]{Porta_Yu_Higher_analytic_stacks_2014}.
Observe that there is a continuous morphism of geometric contexts $u \colon (\mathrm{Stn}, \tau_0, \mathbf P_0) \to ({\mathrm{Stn}^{\mathrm{der}}_{\mathbb C}}, \tau, \mathbf P_{\mathrm{\acute{e}t}})$.
This functor is fully faithful in virtue of \cite[Theorem 12.8]{DAG-IX}.
It follows that there exists a fully faithful functor
\[ u_s \colon {\mathrm{Sh}}(\mathrm{Stn}, {\tau_\mathrm{q\acute{e}t}}) \to {\mathrm{Sh}}({\mathrm{Stn}^{\mathrm{der}}_{\mathbb C}}, {\tau_\mathrm{q\acute{e}t}}) \]
which moreover preserves geometric stacks (we refer to \cite[§2.4]{Porta_Yu_Higher_analytic_stacks_2014} for a discussion of the notation employed).
Conversely, using \cref{thm:analytic_functor_of_points} we have:

\begin{cor} \label{cor:comparison_analytic_DM_stacks}
	Let $X = ({\mathcal X}, {\mathcal O}_{\mathcal X})$ be an $n$-localic $0$-truncated derived {$\mathbb C$-analytic\xspace} space.
	Then $\phi(X) = ({\mathcal X}, {\mathcal O}_{\mathcal X}) \in {\mathrm{Sh}}({\mathrm{Stn}^{\mathrm{der}}_{\mathbb C}}, \tau_0)$ belongs to the essential image of $u_s$ and is a higher analytic {Deligne-Mumford\xspace} stack in the sense of \cite{Porta_Yu_Higher_analytic_stacks_2014}.
\end{cor}

\begin{proof}
	We already know that $\phi(X)$ is geometric.
	To prove that $\phi(X)$ belongs to the essential image of $u_s$, we can proceed by induction on $n$.
	When $n = 0$, the statement is clear, and the induction step follows from the construction of the atlas of $\phi(X)$ given in the proof of \cref{thm:analytic_functor_of_points} and the fact that $u_s$ commutes with geometric realizations of \'etale groupoids (being a left adjoint).
\end{proof}

One of the most basic and yet useful constructions in derived geometry is the truncation of a derived object.
This often allows to reduce the proofs to the classical setting, where they can be handled with different techniques.
As we will see, this is exactly the case for many of the main results of this article.
For this reason, we introduce now the truncation functor ${\mathrm{t}_0}$.
Roughly speaking this has simply to be the functor sending a derived {$\mathbb C$-analytic\xspace} space $({\mathcal X}, {\mathcal O}_{\mathcal X})$ into $({\mathcal X}, \tau_{\le 0} {\mathcal O}_{\mathcal X})$ (that the latter is still a derived {$\mathbb C$-analytic\xspace} space is a consequence of \cite[Proposition 3.3.3]{DAG-V} and of \cite[Proposition 11.4]{DAG-IX}).
However, in order to construct this as an $\infty$-functor we will need to describe it in a rather different fashion.

As it is consequence of a much more general fact concerning geometries, let us switch for a short while to this setting.
Let ${\mathcal G}$ be a geometry (e.g.\ any geometric envelope for ${{\mathcal T}_{\mathrm{an}}}$) and let ${\mathcal X}$ be an $\infty$-topos.
Inside ${\mathrm{Str}}_{\mathcal G}({\mathcal X})$ we can look for the full subcategory spanned by $n$-truncated objects. Let us denote it by ${\mathrm{Str}}_{\mathcal G}({\mathcal X})_{\le n}$.
A very natural question is whether the latter category can be obtained as the category of structures for a suitable modification of ${\mathcal G}$.
This is indeed the case; the relevant object is referred to as the \emph{$n$-stub} of ${\mathcal G}$ and its existence is guaranteed by \cite[Proposition 1.5.11]{DAG-V}.
Let us denote it by ${\mathcal G}_{\le n}$. By construction, it comes equipped with a morphism of geometries ${\mathcal G} \to {\mathcal G}_{\le n}$.
When $n = 0$ we can ask whether the relative spectrum functor associated to such a morphism coincides with the functor ${\mathrm{t}_0}$ we are trying to define.
This is probably not true without additional hypotheses on ${\mathcal G}$. The point is that, in general, it is not true that if ${\mathcal O}$ is a ${\mathcal G}$-structure on an $\infty$-topos ${\mathcal X}$ then $\tau_{\le n} \circ {\mathcal O}$ is again a ${\mathcal G}$-structure.
There is a sufficient condition for this to be true, though: it happens when the geometry is compatible with $n$-truncations (see \cite[Definition 3.3.2]{DAG-V}).
Under this condition we are able to prove:

\begin{prop} \label{prop:relative_spectrum_truncated_structured_topoi}
	Let ${\mathcal G}$ be a geometry compatible with $n$-truncations and let ${\mathcal G} \to {\mathcal G}_{\le n}$ be an $n$-stub for ${\mathcal G}$.
	Then
	\[ \mathrm{Spec}^{{\mathcal G}_{\le n}}_{\mathcal G} \colon {\mathcal T\mathrm{op}}({\mathcal G}) \to {\mathcal T\mathrm{op}}({\mathcal G}_{\le n}) \]
	coincides on objects with the assignment
	\[ ({\mathcal X}, {\mathcal O}_{\mathcal X}) \mapsto ({\mathcal X}, \tau_{\le n} {\mathcal O}_{\mathcal X}) \]
\end{prop}

\begin{proof}
	Let $({\mathcal X}, {\mathcal O}_{\mathcal X})$ be a ${\mathcal G}$-structured topos.
	It follows from \cite[Proposition 3.3.3]{DAG-V} that $\tau_{\le n} {\mathcal O}_{\mathcal X}$ is a ${\mathcal G}$-structure on ${\mathcal X}$.
	Since $\tau_{\le n} {\mathcal O}_{\mathcal X}$ is $n$-truncated, \cite[Proposition 1.5.14]{DAG-V} shows that it defines a ${\mathcal G}_{\le n}$-structure on ${\mathcal X}$.
	The morphism ${\mathcal O}_{\mathcal X} \to \tau_{\le n} {\mathcal O}_{\mathcal X}$ is a local morphism because ${\mathcal G}$ is compatible with $n$-truncations.
	Therefore, it defines a well defined morphism
	\[ p_n \colon ({\mathcal X}, \tau_{\le n} {\mathcal O}_{\mathcal X}) \to ({\mathcal X}, {\mathcal O}_{\mathcal X}) \]
	in ${\mathcal T\mathrm{op}}({\mathcal G})$.
	We claim that for every $({\mathcal Y}, {\mathcal O}_{\mathcal Y}) \in {\mathcal T\mathrm{op}}({\mathcal G}_{\le n})$, the canonical morphism
	\[ \operatorname{Map}_{{\mathcal T\mathrm{op}}({\mathcal G}_{\le n})}(({\mathcal Y}, {\mathcal O}_{\mathcal Y}), ({\mathcal X}, \tau_{\le n} {\mathcal O}_{\mathcal X})) \to \operatorname{Map}_{{\mathcal T\mathrm{op}}({\mathcal G})}(({\mathcal Y}, {\mathcal O}_{\mathcal Y}), ({\mathcal X}, {\mathcal O}_{\mathcal X})) \]
	is a homotopy equivalence.
	Indeed, we have a commutative diagram of fiber sequences
	\[ \begin{tikzcd}
	\operatorname{Map}_{{\mathrm{Str}^\mathrm{loc}}_{{\mathcal G}_{\le n}}({\mathcal Y})}(f{^{-1}} \tau_{\le n} {\mathcal O}_{\mathcal X}, {\mathcal O}_{\mathcal Y}) \arrow{r} \arrow{d} & \operatorname{Map}_{{\mathrm{Str}^\mathrm{loc}}_{\mathcal G}({\mathcal Y})}(f{^{-1}} {\mathcal O}_{\mathcal X}, {\mathcal O}_{\mathcal Y} ) \arrow{d} \\
	\operatorname{Map}_{{\mathcal T\mathrm{op}}({\mathcal G}_{\le n})}(({\mathcal Y}, {\mathcal O}_{\mathcal Y}), ({\mathcal X}, \tau_{\le n} {\mathcal O}_{\mathcal X})) \arrow{r} \arrow{d} & \operatorname{Map}_{{\mathcal T\mathrm{op}}({\mathcal G})}(({\mathcal Y}, {\mathcal O}_{\mathcal Y}), ({\mathcal X}, {\mathcal O}_{\mathcal X})) \arrow{d} \\
	\operatorname{Map}_{\mathcal T\mathrm{op}}({\mathcal Y}, {\mathcal X}) \arrow{r}{\mathrm{id}} & \operatorname{Map}_{\mathcal T\mathrm{op}}({\mathcal Y}, {\mathcal X})
	\end{tikzcd} \]
	where both the fibers are computed over the geometric morphism $f{^{-1}} \colon {\mathcal X} \rightleftarrows {\mathcal Y} \colon f_*$.
	Since $f{^{-1}}$ is left exact, we see that
	\[ f{^{-1}} \tau_{\le n} {\mathcal O}_{\mathcal X} \simeq \tau_{\le n} f{^{-1}} {\mathcal O}_{\mathcal X} \]
	Finally, since the functor ${\mathrm{Str}^\mathrm{loc}}_{{\mathcal G}_{\le n}}({\mathcal Y}) \to {\mathrm{Str}^\mathrm{loc}}_{\mathcal G}({\mathcal Y})$ is fully faithful (in virtue of \cite[Proposition 1.5.14]{DAG-V}), we see that the top horizontal morphism is a homotopy equivalence.
	The proof is therefore complete.
\end{proof}

\begin{defin}
	Let ${\mathcal G}$ be a geometry compatible with $0$-truncations. We will refer to the functor $\mathrm{Spec}^{{\mathcal G}_{\le 0}}_{\mathcal G}$ as the \emph{truncation functor} and we will denote it ${\mathrm{t}_0}^{\mathcal G}$, or simply by ${\mathrm{t}_0}$ when the geometry ${\mathcal G}$ is clear from the context.
\end{defin}

The analytic pregeometry ${{\mathcal T}_{\mathrm{an}}}$ is compatible with $n$-truncations for every $n \ge 0$ in virtue of \cite[Proposition 11.4]{DAG-IX}.
Therefore this allows to introduce the truncation functor for derived {$\mathbb C$-analytic\xspace} spaces.
Let us denote by ${\mathrm{dAn}_{\mathbb C}}^0$ the full subcategory of ${\mathrm{dAn}_{\mathbb C}}$ spanned by the $0$-truncated derived {$\mathbb C$-analytic\xspace} spaces.
Similarly, let us denote by ${\mathcal T\mathrm{op}}^0({{\mathcal T}_{\mathrm{an}}})$ the full subcategory of ${\mathcal T\mathrm{op}}({{\mathcal T}_{\mathrm{an}}})$ spanned by $0$-truncated ${{\mathcal T}_{\mathrm{an}}}$-structured topoi.
Then we have (cf.\ \cite[Proposition 2.2.4.4]{HAG-II}):

\begin{prop} \label{prop:truncation_and_finite_limits}
	Let $i \colon {\mathrm{dAn}_{\mathbb C}}^0 \to {\mathrm{dAn}_{\mathbb C}}$ be the natural inclusion functor. Then:
	\begin{enumerate}
		\item the functor ${\mathrm{t}_0} \colon {\mathcal T\mathrm{op}}({{\mathcal T}_{\mathrm{an}}}) \to {\mathcal T\mathrm{op}}^0({{\mathcal T}_{\mathrm{an}}})$ restricts to a functor ${\mathrm{t}_0} \colon {\mathrm{dAn}_{\mathbb C}} \to {\mathrm{dAn}_{\mathbb C}}^0$;
		\item the functor $i$ is left adjoint to the functor ${\mathrm{t}_0}$;
		\item the functor $i$ is fully faithful.
	\end{enumerate}
\end{prop}

\begin{proof}
	It follows from inspection that ${\mathrm{t}_0}$ respects the category of derived {$\mathbb C$-analytic\xspace} spaces.
	Therefore the points (1) and (2) follow immediately.
	As for (3), the result follows from the description of the unit of this adjunction given in \cref{prop:relative_spectrum_truncated_structured_topoi} and the fact that the truncation functor $\tau_{\le 0}$ of any $\infty$-topos is idempotent.
\end{proof}

We close this section with a couple of important remark.
Let $X = ({\mathcal X}, {\mathcal O}_{\mathcal X})$ be an $n$-localic $0$-truncated derived {$\mathbb C$-analytic\xspace} space.
If $Y \to X$ is an \'etale morphism of derived {$\mathbb C$-analytic\xspace} spaces, we see that $Y$ has to be $0$-truncated.
This observation together with the fully faithfulness of the functor $\phi$ proved in \cref{prop:little_phi_fully_faithful} yields the following result (cf.\ \cite[Proposition 2.2.4.4.(4)]{HAG-II}):

\begin{cor} \label{cor:equivalence_of_small_etale_sites}
	Let $X = ({\mathcal X}, {\mathcal O}_{\mathcal X})$ be an $n$-localic $0$-truncated derived {$\mathbb C$-analytic\xspace} space.
	Then the small \'etale site $(\mathrm{Stn}^{\mathrm{der}}_{\mathbb C / X})_{\mathrm{\acute{e}t}}$ is canonically equivalent to the small \'etale site $(\mathrm{Stn}_{\mathbb C / \phi(X)})_{\mathrm{\acute{e}t}}$.
\end{cor}

We also have the following useful equivalence, very familiar to the panorama of derived algebraic geometry (cf.\ \cite[Corollary 2.2.2.9]{HAG-II}):

\begin{prop} \label{prop:equivalence_etale_sites_truncation}
	Let $X = ({\mathcal X}, {\mathcal O}_{\mathcal X})$ be a derived {$\mathbb C$-analytic\xspace} space and let ${\mathrm{t}_0}(X) = ({\mathcal X}, \pi_0 {\mathcal O}_{\mathcal X})$ be its truncation.
	The base change along the morphism ${\mathrm{t}_0}(X) \to X$ induces an equivalence
	\[ (\mathrm{Stn}^{\mathrm{der}}_{\mathbb C / X})_{\mathrm{\acute{e}t}} \simeq (\mathrm{Stn}^{\mathrm{der}}_{\mathbb C / {\mathrm{t}_0}(X)})_{\mathrm{\acute{e}t}} \]
\end{prop}

\begin{proof}
	The universal property of \'etale morphisms of ${{\mathcal T}_{\mathrm{an}}}$-structured topoi described in \cite[Example 2.3.4]{DAG-V} shows that both sides can be identified with the full subcategory of ${\mathcal X}$ spanned by those objects $U \in {\mathcal X}$ such that ${\mathcal X}_{/U} \simeq {\mathrm{Sh}}(S)$ for some topological space $S$ which is the underlying space of a {$\mathbb C$-analytic\xspace} Stein space.
\end{proof}

\subsection{Essential surjectivity} \label{subsec:essential_surjectivity}

We are now ready to complete the proof of \cref{thm:analytic_functor_of_points}.
We will switch again to the context $({\mathrm{dAn}_{\mathbb C}}^0, \tau, \mathbf P_{\mathrm{\acute{e}t}})$.
We will need to discuss a variation of \cref{prop:equivalence_etale_sites_truncation}.
Let $X$ be a geometric stack for the context $({\mathrm{dAn}_{\mathbb C}}^0, \tau, \mathbf P_{\mathrm{\acute{e}t}})$.
We will denote by $X_{\mathrm{\acute{e}t}}$ its small \'etale site, that is the full subcategory of $({\mathrm{dAn}_{\mathbb C}}^0)_{/X}$ spanned by \'etale morphisms (we refer to \cite[Remark 2.17]{Porta_Yu_Higher_analytic_stacks_2014} for a definition).
Furthermore, we have a fully faithful inclusion of sites
\[ j \colon {\mathrm{An}_{{\mathbb C}}} \to {\mathrm{dAn}_{\mathbb C}}^0 \]
which is cocontinuous in virtue of the same observation used to prove \cref{cor:equivalence_of_small_etale_sites}.
In particular, it induces a restriction functor on the level of geometric stacks.
We will denote such functor again by ${\mathrm{t}_0}$.

\begin{prop}
	Let $X$ be a geometric stack for the context $({\mathrm{dAn}_{\mathbb C}}^0, \tau, \mathbf P_{\mathrm{\acute{e}t}})$. Then the functor ${\mathrm{t}_0} \colon X_{\mathrm{\acute{e}t}} \to ({\mathrm{t}_0}(X))_{\mathrm{\acute{e}t}}$ is an equivalence of sites.
\end{prop}

\begin{proof}
	We prove this by induction on the geometric level of $X$.
	If $X$ is $(-1)$-representable, this follows from \cref{prop:equivalence_etale_sites_truncation}.
	Suppose now that $X$ is $n$-geometric and that the statement holds true for $(n-1)$-geometric stacks.
	Choose an \'etale $n$-groupoid presentation $U^\bullet$ for $X$.
	Recall that this means that $U^\bullet$ is a groupoid object in the $\infty$-category $\mathrm{dSt}$, that each $U^m$ is $(n-1)$-geometric and that the map $U^0 \to X$ is $(n-1)$-\'etale.
	Since ${\mathrm{t}_0}$ commutes with products in virtue of \cref{prop:truncation_and_finite_limits} and takes effective epimorphisms to effective epimorphisms by \cite[7.2.1.14]{HTT}, we see that $V^\bullet \coloneqq {\mathrm{t}_0}(U^\bullet)$ is a groupoid presentation for ${\mathrm{t}_0}(X)$.
	
	Now, let $Y \to {\mathrm{t}_0}(X)$ be an \'etale map. We see that $Y \times_{{\mathrm{t}_0}(X)} V^\bullet \to V^\bullet$ is an \'etale map (i.e.\ it is a map of groupoids which is \'etale in each degree). By the inductive hypothesis, we obtain a map of simplicial objects $Z^\bullet \to U^\bullet$, which is such that
	\[ {\mathrm{t}_0}(Z^\bullet) = Y \times_{{\mathrm{t}_0}(X)} V^\bullet \]
	Since $Y \times_{{\mathrm{t}_0}(X)} V^\bullet$ was a groupoid, the same goes for $Z^\bullet$ (here we use again the equivalence guaranteed by the inductive hypothesis).
	The geometric realization of $Z^\bullet$ provides us with an \'etale map $Z \to X$. Since ${\mathrm{t}_0}$ preserves effective epimorphisms, we conclude that ${\mathrm{t}_0}(Z) = Y$.
	This construction is functorial in $Y$, and it provides the inverse to the functor ${\mathrm{t}_0}$.
\end{proof}

If $X$ is a $n$-geometric stack with respect to the context $({\mathrm{dAn}_{\mathbb C}}^0, \tau, \mathbf P_{\mathrm{\acute{e}t}})$, its truncation is a $(n+1)$-truncated, as it follows from the same argument given in \cite[Lemma 2.1.1.2]{HAG-II}.
It follows that the mapping spaces in $X_{\mathrm{\acute{e}t}}$ are $(n+1)$-truncated and therefore $X_{\mathrm{\acute{e}t}}$ itself is equivalent to an $(n+1)$-category.
It follows that the category of (non hypercomplete) $\infty$-sheaves ${\mathcal X} \coloneqq {\mathrm{Sh}}(X_{\mathrm{\acute{e}t}}, \tau)$ is an $(n+1)$-localic topos.
Consider the composition
\[ {{\mathcal T}_{\mathrm{an}}} \times (X_{\mathrm{\acute{e}t}})^{\mathrm{op}} \to {\mathrm{dAn}_{\mathbb C}}^0 \times ({\mathrm{dAn}_{\mathbb C}}^0)^{\mathrm{op}} \xrightarrow{y} {\mathcal S} \]
where the last arrow is the functor classifying the Yoneda embedding (see \cite[§ 5.2.1]{Lurie_Higher_algebra}).
This induces a well defined functor
\[ {\mathcal O}_X \colon {{\mathcal T}_{\mathrm{an}}} \to {\mathrm{PSh}}(X_{\mathrm{\acute{e}t}}) \]
and \cref{cor:little_phi_hypersheaves} shows that is hypercomplete.

\begin{lem}
	Keeping the above notations, ${\mathcal O}_X$ commutes with products, admissible pullbacks and takes $\tau$-covers to effective epimorphisms.
	In other words, ${\mathcal O}_X$ defines a ${{\mathcal T}_{\mathrm{an}}}$-structure on ${\mathcal X}$.
\end{lem}

\begin{proof}
	The functor ${{\mathcal T}_{\mathrm{an}}} \to {\mathrm{dAn}_{\mathbb C}}^0$ commutes with products and admissible pullbacks by \cref{prop:basic_properties_of_DAn}. Moreover, it takes $\tau$-covers to effective epimorphisms in virtue of \cref{cor:little_phi_commutes_with_etale_colimits}.
	At this point, the conclusion is straightforward.
\end{proof}

If $\{U_i \to X\}$ is an \'etale atlas of $X$, each $U_i$ defines an object $V_i$ in ${\mathcal X}$. 
Unraveling the definitions, we see that the ${{\mathcal T}_{\mathrm{an}}}$-structured topos $({\mathcal X}_{/V_i}, {\mathcal O}_X |_{V_i})$ is canonically isomorphic to $U_i \in {\mathrm{dAn}_{\mathbb C}}^0$ itself.
Therefore $X' \coloneqq ({\mathcal X}, {\mathcal O}_X)$ is a derived {$\mathbb C$-analytic\xspace} space.

We are left to prove that $\phi(X') \simeq X$. We can proceed by induction on the geometric level $n$ of $X$.
If $n = -1$, the statement is clear.
If $n \ge 0$, the first part of the proof of \cref{thm:analytic_functor_of_points} shows that the \v{C}ech nerve of $\coprod U_i \to X$ is a groupoid presentation for $\phi(X')$. Since $\phi$ commutes with \v{C}ech nerves of \'etale maps and their realizations (in virtue of \cref{cor:little_phi_commutes_with_etale_colimits}), we conclude that $\phi(X')$ is equivalent to $X$ itself.
The proof of \cref{thm:analytic_functor_of_points} is now achieved.

\section{Coherent sheaves on a derived {$\mathbb C$-analytic\xspace} space} \label{sec:coherent_sheaves}

In this section we introduce the notion of coherent sheaf on a derived {$\mathbb C$-analytic\xspace} space.
Let us start by recalling that given an $\infty$-topos ${\mathcal X}$ it is possible to define the category of ${{{\mathcal D}({\mathrm{Ab}})}}$-valued sheaves as in \cite[§1.1]{DAG-V}.
This $\infty$-category inherits a symmetric monoidal structure from the one of ${{{\mathcal D}({\mathrm{Ab}})}}$.
If ${\mathcal O}_{\mathcal X}$ is $\mathbb E_\infty$-ring on ${\mathcal X}$, we can therefore form the $\infty$-category ${\mathcal O}_{\mathcal X} \textrm{-} {\mathrm{Mod}}$ of ${\mathcal O}_{\mathcal X}$-modules.
We refer to \cite[Proposition 2.1.3]{DAG-VIII} for the main properties of this $\infty$-category.

\begin{defin} \label{def:coherent_sheaf_derived_canal_space}
	Let $X$ be a derived {$\mathbb C$-analytic\xspace} space.
	The $\infty$-category ${\mathrm{Coh}}(X)$ is the full subcategory of ${\mathcal O}_{\mathcal X}{^\mathrm{alg}} \textrm{-} {\mathrm{Mod}}$ spanned by those ${\mathcal O}_{\mathcal X}{^\mathrm{alg}}$-modules ${\mathcal F}$ such that the cohomology sheaves ${\mathcal H}^i({\mathcal F})$ are locally on ${\mathcal X}$ coherent sheaves of $\pi_0({\mathcal O}_{\mathcal X}{^\mathrm{alg}})$-modules.
	We let ${\mathrm{Coh}}^+(X)$ be the full subcategory of ${\mathrm{Coh}}(X)$ spanned by those coherent ${\mathcal O}_{\mathcal X}{^\mathrm{alg}}$-modules ${\mathcal F}$ such that ${\mathcal H}^i({\mathcal F}) = 0$ for all $i \ll 0$ (in cohomological notation).
	We let ${\mathrm{Coh}}^+(X)$ be the full subcategory of ${\mathrm{Coh}}(X)$ spanned by those coherent ${\mathcal O}_{\mathcal X}{^\mathrm{alg}}$-modules ${\mathcal F}$ such that ${\mathcal H}^i({\mathcal F}) = 0$ for all $|i| \gg 0$.
\end{defin}

Clearly, both ${\mathrm{Coh}}^+(X)$ and ${\mathrm{Coh}}^b(X)$ are stable subcategories of ${\mathcal O}_X \textrm{-} {\mathrm{Mod}}$.
Moreover, the cohomology sheaves ${\mathcal H}^i$ allow to define a $t$-structure on ${\mathrm{Coh}}(X)$, and one has
\[ {\mathrm{Coh}}^\heartsuit(X) \simeq {\mathrm{Coh}}^\heartsuit({\mathrm{t}_0}(X)) \]
as it follows combining \cref{prop:equivalence_etale_sites_truncation} and \cite[Remark 2.1.5]{DAG-VIII}.

\begin{rem}
	It could seem that \cref{def:coherent_sheaf_derived_canal_space} is to some extent arbitrary because it doesn't take at all into account the additional analytic structure of ${\mathcal O}_{\mathcal X}$. This is not quite true, but a justification of this fact is beyond the scope of the present article. We will come back to this subject in \cite{Porta_Analytic_deformation_2015}.
\end{rem}

In order to freely use the results proved in \cite{Porta_Yu_Higher_analytic_stacks_2014} for underived higher analytic stacks, we will need to compare the two categories of sheaves on them. We already moved a first step in this direction with \cref{cor:equivalence_of_small_etale_sites}.
We will now complete the task as follows.
Let $X = ({\mathcal X}, {\mathcal O}_{\mathcal X})$ be a derived {$\mathbb C$-analytic\xspace} space.
It follows from the very definition of \'etale morphism of derived {$\mathbb C$-analytic\xspace} space that there exists a fully faithful functor
\[ (\mathrm{Stn}^{\mathrm{der}}_{\mathbb C / X})_{\mathrm{\acute{e}t}} \to {\mathcal X} \]
defined by sending an \'etale map $Y \to X$ to the object $U \in {\mathcal X}$ such that ${\mathcal Y} \simeq {\mathcal X}_{/U}$.
Since $X$ admits an atlas made of derived Stein spaces, we conclude that ${\mathcal X}$ is equivalent to the $\infty$-category of sheaves on $((\mathrm{Stn}^{\mathrm{der}}_{\mathbb C / X})_{\mathrm{\acute{e}t}}, \tau)$.
Therefore we obtain:

\begin{prop} \label{prop:comparison_coherent_sheaves}
	Let $X = ({\mathcal X}, {\mathcal O}_{\mathcal X})$ be a localic $0$-truncated derived {$\mathbb C$-analytic\xspace} space.
	Then the $\infty$-category of ${\mathcal O}_{\mathcal X}$-modules (resp.\ of coherent ${\mathcal O}_{\mathcal X}$-modules) on $X$ is canonically equivalent to the one on $\phi(X)$ in the sense of \cite[§5.1]{Porta_Yu_Higher_analytic_stacks_2014}.
\end{prop}

\section{The Grauert theorem for derived stacks} \label{sec:Grauert_theorem}

In this short section we explain how the Grauert theorem proved in \cite{Porta_Yu_Higher_analytic_stacks_2014} for underived higher Artin analytic stacks induces an analogous theorem for derived {$\mathbb C$-analytic\xspace} spaces.

\begin{defin} \label{def:proper_morphism_derived_canal_spaces}
	Let $f \colon X \to Y$ be a morphism of derived {$\mathbb C$-analytic\xspace} spaces and suppose that $X$ and $Y$ are $n$-localic for some $n$.
	We will say that $f$ is \emph{proper} if ${\mathrm{t}_0}(f) \colon {\mathrm{t}_0}(X) \to {\mathrm{t}_0}(Y)$ is proper as morphism of higher {Deligne-Mumford\xspace} analytic stacks (in the sense of \cite[Definition 4.8]{Porta_Yu_Higher_analytic_stacks_2014}).
\end{defin}

\begin{rem}
	As the formulation of a reasonable definition of proper map between higher analytic stacks has perhaps been the newest concept introduced in \cite{Porta_Yu_Higher_analytic_stacks_2014}, it is worth of recalling it. In this remark, we will limit ourselves to the {$\mathbb C$-analytic\xspace} case. The definition is given by induction on the geometric level of the map $f \colon X \to Y$ and it relies on the feebler notion of \emph{weakly proper map}. In the {$\mathbb C$-analytic\xspace} setting, one can say that a morphism of analytic stacks $f \colon X \to Y$ with $Y$ representable is weakly proper if for every Stein open subset $W \Subset Y$ and every atlas $\{U_i\}_{i \in I}$ of $X$ there exists a finite subset $I' \subset I$ such that $\{W \times_Y U_i\}_{i \in I'}$ is an atlas for $W \times_Y X$.
	This definition is vaguely reminiscent of the topological notion of compact space. It is not quite the definition adopted in \cite{Porta_Yu_Higher_analytic_stacks_2014}, but it is equivalent: see Definition 4.5 loc.\ cit.\ for the original one and Lemma 6.2 for the equivalence with the one we reported here.
	
	Once this notion is established, we say that a morphism $f \colon X \to Y$ of higher {$\mathbb C$-analytic\xspace} stacks is proper if it is weakly proper and separated (i.e.\ the diagonal, whose geometric level is strictly less than the one of $f$, is proper). For example, the stack $\mathrm BG$ with $G$ a {$\mathbb C$-analytic\xspace} Lie group is proper if only if the group $G$ was compact to begin with.
\end{rem}

\begin{rem}
	It follows from \cite[Lemma 4.12]{Porta_Yu_Higher_analytic_stacks_2014} and \cref{prop:truncation_and_finite_limits} that proper morphisms are stable under base change. It can be further proved that they are stable under composition.
\end{rem}

\begin{lem} \label{lem:proper_direct_image_DM_truncation}
	Let $X$ be a derived {$\mathbb C$-analytic\xspace} space and let $i \colon {\mathrm{t}_0}(X) \to X$ be the inclusion of its truncation.
	Then ${\mathrm R} i_*$ takes ${\mathrm{Coh}}^+({\mathrm{t}_0}(X))$ to ${\mathrm{Coh}}^+(X)$ and it is of cohomological dimension $0$.
\end{lem}

\begin{proof}
	Since $X$ is a derived {Deligne-Mumford\xspace} stack we can represent it as $({\mathcal X}, {\mathcal O}_{\mathcal X})$, where ${\mathcal X}$ is the small \'etale topos of $X$.
	With this notation, we can identify ${\mathrm{t}_0}(X)$ with $({\mathcal X}, \pi_0({\mathcal O}_{\mathcal X}))$.
	The functor ${\mathrm R} i_*$ is therefore identified with the forgetful functor along ${\mathcal O}_{\mathcal X} \to \pi_0({\mathcal O}_{\mathcal X})$.
	It follows immediately that ${\mathrm R} i_*$ is of cohomological dimension $0$ (in fact, it is $t$-exact).
	If ${\mathcal F} \in {\mathrm{Coh}}^+({\mathrm{t}_0}({\mathcal X}))$, then each ${\mathcal H}^i({\mathcal F})$ is a coherent $\pi_0({\mathcal O}_{\mathcal X})$-modules.
	Hence, it follows from the definitions that ${\mathcal F} \in {\mathrm{Coh}}^+({\mathcal X})$.
\end{proof}

With these definitions it is immediate to prove the following:

\begin{prop} \label{prop:proper_direct_image_derived_DM_stacks}
	Let $f \colon {\mathcal X} \to {\mathcal Y}$ be a proper morphism of derived {Deligne-Mumford\xspace} stacks.
	Suppose moreover that ${\mathrm{t}_0}({\mathcal Y})$ is locally noetherian.
	Then the derived pushforward
	\[ {\mathrm R} f_* \colon {\mathcal O}_{\mathcal X} \textrm{-} {\mathrm{Mod}} \to {\mathcal O}_{\mathcal Y} \textrm{-} {\mathrm{Mod}} \]
	takes the full subcategory ${\mathrm{Coh}}^+(X)$ to ${\mathrm{Coh}}^+(Y)$.
\end{prop}

\begin{proof}
	Let ${\mathcal C}$ be the full subcategory of ${\mathcal O}_{\mathcal X} \textrm{-} {\mathrm{Mod}}$ spanned by those ${\mathcal O}_{\mathcal X}$-modules ${\mathcal F}$ such that ${\mathrm R} f_*({\mathcal F}) \in {\mathrm{Coh}}^+(Y)$.
	We make the following remarks:
	\begin{enumerate}
		\item ${\mathcal C}$ is closed under loops and suspensions in ${\mathcal O}_{\mathcal X} \textrm{-} {\mathrm{Mod}}$: this is obvious, since ${\mathrm R} f_*$ is an exact functor of stable $\infty$-categories;
		\item ${\mathcal C}$ is closed under extensions in ${\mathcal O}_{\mathcal X} \textrm{-} {\mathrm{Mod}}$: again, this follows from the fact that ${\mathrm R} f_*$ takes fiber sequences to fiber sequences (being an exact functor of stable $\infty$-categories);
		\item ${\mathcal C}$ contains ${\mathrm{Coh}}^\heartsuit(X)$. Indeed, we have the following commutative square:
		\[ \begin{tikzcd}
		{\mathrm{t}_0}({\mathcal X}) \arrow{r}{f_0} \arrow{d}{i} & {\mathrm{t}_0}({\mathcal Y}) \arrow{d}{j} \\
		{\mathcal X} \arrow{r}{f} & {\mathcal Y}
		\end{tikzcd} \]
		which induces a commutative square
		\[ \begin{tikzcd}
		\pi_0({\mathcal O}_{\mathcal X}) \textrm{-} {\mathrm{Mod}} \arrow{r}{{\mathrm R} f_{0*}} \arrow{d}{{\mathrm R} i_*} & \pi_0({\mathcal O}_{\mathcal Y}) \textrm{-} {\mathrm{Mod}} \arrow{d}{{\mathrm R} j_*} \\
		{\mathcal O}_{\mathcal X} \textrm{-} {\mathrm{Mod}} \arrow{r}{{\mathrm R} f_*} & {\mathcal O}_{\mathcal Y} \textrm{-} {\mathrm{Mod}}
		\end{tikzcd} \]
		If ${\mathcal F} \in {\mathrm{Coh}}^\heartsuit({\mathcal X})$, we can write ${\mathcal F} = {\mathrm R} i_*({\mathcal F}')$ with ${\mathcal F}' \in {\mathrm{Coh}}^\heartsuit({\mathrm{t}_0}({\mathcal X}))$, as it follows from \cite[Remark 2.1.5]{DAG-VIII}.
		Therefore \cite[Theorem 5.11]{Porta_Yu_Higher_analytic_stacks_2014} shows that ${\mathrm R} f_{0*}({\mathcal F}') \in {\mathrm{Coh}}^+({\mathrm{t}_0}({\mathcal Y}))$, and \cref{lem:proper_direct_image_DM_truncation} shows that ${\mathrm R} j_*({\mathrm R} f_{0*}({\mathcal F}')) \in {\mathrm{Coh}}^+({\mathcal Y})$.
	\end{enumerate}
	
	To conclude the proof, we only need to observe that if $\tau_{\le n} {\mathcal F} \in {\mathrm{Coh}}^+({\mathcal X}) \cap {\mathcal C}$ for every $n$, then ${\mathcal F} \in {\mathcal C}$.
	For a fiber sequence
	\[ \tau_{\le n} {\mathcal F} \to {\mathcal F} \to \tau_{> n} {\mathcal F} \]
	As ${\mathrm R} f_*$ is an exact functor of stable $\infty$-categories, we have a fiber sequence
	\[ {\mathrm R} f_* (\tau_{\le n} {\mathcal F}) \to {\mathrm R} f_* {\mathcal F} \to {\mathrm R} f_*(\tau_{> n} {\mathcal F}) \]
	As ${\mathrm R} f_*$ is left $t$-exact, we see that ${\mathcal H}^i({\mathrm R} f_*(\tau_{> n} {\mathcal F}))$ vanishes whenever $i \le n$.
	Therefore the long exact sequence of cohomology groups shows that
	\[ {\mathcal H}^i( {\mathrm R} f_*(\tau_{\le n} {\mathcal F}) ) \to {\mathcal H}^i( {\mathrm R} f_* {\mathcal F} ) \]
	is an equivalence for every $i \le n$. Since $\tau_{\le n} {\mathcal F} \in {\mathrm{Coh}}^+({\mathcal F})$, letting $n$ vary, we obtain that the cohomology sheaves of ${\mathrm R} f_* {\mathcal F}$ are coherent. In other words, ${\mathcal F} \in {\mathcal C}$, and the proof is now complete.
\end{proof}

\section{The analytification functor} \label{sec:analytification_functor}

As we explained in the introduction, this section contain the most important result of this article, from a technical point of view.
Following \cite[Remark 12.26]{DAG-IX}, we consider the morphism of pregeometries ${{\mathcal T}_{\mathrm{\acute{e}t}}} \to {{\mathcal T}_{\mathrm{an}}}$ induced by the classical analytification functor of \cite[Expos\'e XII]{SGA1}.
This morphism induces a forgetful functor
\[ (-){^\mathrm{alg}} \colon {\mathcal T\mathrm{op}}({{\mathcal T}_{\mathrm{an}}}) \to {\mathcal T\mathrm{op}}({{\mathcal T}_{\mathrm{\acute{e}t}}}) \]
defined informally by the rule $({\mathcal X}, {\mathcal O}_{\mathcal X}) \mapsto ({\mathcal X}, {\mathcal O}_{\mathcal X}{^\mathrm{alg}})$.
It follows from the general theory of \cite[§2.1]{DAG-V} that this functor admits a right adjoint, denoted by $\mathrm{Spec}^{{\mathcal T}_{\mathrm{an}}}_{{\mathcal T}_{\mathrm{\acute{e}t}}}$.
Moreover, \cite[Proposition 2.3.18]{DAG-V} shows that $\mathrm{Spec}^{{\mathcal T}_{\mathrm{an}}}_{{\mathcal T}_{\mathrm{\acute{e}t}}}$ takes ${{\mathcal T}_{\mathrm{\acute{e}t}}}$-schemes locally of finite presentation to ${{\mathcal T}_{\mathrm{an}}}$-schemes locally of finite presentation. We can therefore invoke \cite[Corollary 12.22]{DAG-IX} to conclude that $\mathrm{Spec}^{{\mathcal T}_{\mathrm{an}}}_{{\mathcal T}_{\mathrm{\acute{e}t}}}$ takes derived Deligne-Mumford stacks locally of finite presentation to derived analytic spaces.
In this section, we show that this analytification functor satisfies all the good properties one would expect.
In particular, we will show that if $(X, {\mathcal O}_X)$ is a classical scheme locally of finite type over $\mathbb C$, then $\mathrm{Spec}^{{\mathcal T}_{\mathrm{an}}}_{{\mathcal T}_{\mathrm{\acute{e}t}}}(X, {\mathcal O}_X)$ can be canonically identified with the classical analytification of $(X, {\mathcal O}_X)$.
Moreover, we will show that if $({\mathcal X}, {\mathcal O}_{\mathcal X})$ is a derived {Deligne-Mumford\xspace} stack locally of finite presentation over $\mathbb C$, then the canonical map
\[ \mathrm{Spec}^{{\mathcal T}_{\mathrm{an}}}_{{\mathcal T}_{\mathrm{\acute{e}t}}}({\mathcal X}, {\mathcal O}_{\mathcal X}){^\mathrm{alg}} \to ({\mathcal X}, {\mathcal O}_{\mathcal X}) \]
is flat.
We will use these results to deduce derived versions of GAGA theorems in the next section.

\subsection{Relative spectrum functor}

Let us begin with a couple of general results concerning the relative spectrum functor associated to a morphism of geometries.

\begin{prop} \label{prop:relative_spectrum_and_truncations}
	Let $\varphi \colon {\mathcal G}' \to {\mathcal G}$ be a morphism of geometries and suppose that both ${\mathcal G}'$ and ${\mathcal G}$ are compatible with $n$-truncations.
	Let $({\mathcal X}, {\mathcal O}_{\mathcal X}) \in {\mathcal T\mathrm{op}}({\mathcal G}')$.
	Then the canonical morphism
	\[ \mathrm{Spec}^{\mathcal G}_{{\mathcal G}'}({\mathcal X}, \tau_{\le n} {\mathcal O}_{\mathcal X}) \to \mathrm{Spec}^{\mathcal G}_{{\mathcal G}'}({\mathcal X}, {\mathcal O}_{\mathcal X}) \]
	Exhibits $\mathrm{Spec}^{\mathcal G}_{{\mathcal G}'}({\mathcal X}, \tau_{\le n})$ as $n$-truncation of $\mathrm{Spec}^{\mathcal G}_{{\mathcal G}'}({\mathcal X}, {\mathcal O}_{\mathcal X})$.
	In particular, it induces an equivalence on the underlying $\infty$-topos.
\end{prop}

\begin{proof}
	Let ${\mathcal G}' \to {\mathcal G}'_{\le n}$ and ${\mathcal G} \to {\mathcal G}_{\le n}$ be $n$-stubs for ${\mathcal G}'$ and ${\mathcal G}$ respectively.
	The universal property defining $n$-stubs, implies the existence of a commutative square of morphism of geometries
	\[ \begin{tikzcd}
	{\mathcal G}' \arrow{r}{\varphi} \arrow{d} & {\mathcal G} \arrow{d} \\
	{\mathcal G}'_{\le n} \arrow{r}{\varphi_n} & {\mathcal G}_{\le n}
	\end{tikzcd} \]
	Therefore we have
	\[ \mathrm{Spec}^{{\mathcal G}_{\le n}}_{\mathcal G} \circ \mathrm{Spec}^{\mathcal G}_{{\mathcal G}'} \simeq \mathrm{Spec}^{{\mathcal G}_{\le n}}_{{\mathcal G}'_{\le n}} \circ \mathrm{Spec}^{{\mathcal G}'_{\le n}}_{{\mathcal G}'} \]
	Combining this with \cref{prop:relative_spectrum_truncated_structured_topoi}, we obtain the desired result.
\end{proof}

Furthermore we can prove the following result:

\begin{prop} \label{prop:relative_spectrum_truncated_objects}
	Let ${\mathcal G}, {\mathcal G}'$ be geometries compatible with $n$-truncations.
	Let $\varphi \colon {\mathcal G}' \to {\mathcal G}$ be a morphism of geometries and let ${\mathcal G}' \to {\mathcal G}'_{\le n}$, ${\mathcal G} \to {\mathcal G}_{\le n}$ be $n$-stubs for ${\mathcal G}'$ and ${\mathcal G}$, respectively.
	The diagram
	\[ \begin{tikzcd}
	{\mathrm{Sch}}({\mathcal G}'_{\le n}) \arrow{rr}{\mathrm{Spec}^{{\mathcal G}_{\le n}}_{{\mathcal G}'_{\le n}}} \arrow{d} & & {\mathrm{Sch}}({\mathcal G}_{\le n}) \arrow{d} \\
	{\mathrm{Sch}}({\mathcal G}') \arrow{rr}{\mathrm{Spec}^{\mathcal G}_{{\mathcal G}'}} & & {\mathrm{Sch}}({\mathcal G})
	\end{tikzcd} \]
	commutes.
\end{prop}

\begin{proof}
	Since $\varphi$ commutes with finite limits, the induced morphism
	\[ \widetilde{\varphi} \colon \mathrm{Ind}(({\mathcal G}')^{\mathrm{op}}) \to \mathrm{Ind}({\mathcal G}^{\mathrm{op}}) \]
	commutes with finite limits as well.
	In particular, we see that it takes $k$-truncated objects to $k$-truncated objects.
	The statement now follows from the following pair of observations:
	\begin{enumerate}
		\item if $A \in \mathrm{Ind}({\mathcal G}^{\mathrm{op}})$ is $k$-truncated, then, writing $\operatorname{Spec}^{\mathcal G}(A) = ({\mathcal X}_A, {\mathcal O}_A)$, ${\mathcal O}_A$ is $k$-truncated;
		\item if $A \in \mathrm{Ind}(({\mathcal G}')^{\mathrm{op}})$, then $\mathrm{Spec}^{\mathcal G}_{{\mathcal G}'}(\operatorname{Spec}^{\mathcal G}(A)) \simeq \mathrm{Spec}^{{\mathcal G}'}(\widetilde{\varphi}(A))$.
	\end{enumerate}
	As these statements follow directly from the definitions, the proof is complete.
\end{proof}

\begin{rem}
	We don't know whether it is possible to extend the above result to the category of ${\mathcal G}'$-structured topoi.
	In what follows, we won't need but the result we proved.
\end{rem}

\subsection{Comparison with the classical analytification}

Consider the morphism of pregeometries ${{\mathcal T}_{\mathrm{\acute{e}t}}} \to {{\mathcal T}_{\mathrm{an}}}$.
The associated relative spectrum functor is by definition the analytification functor.
Since it preserves the class of schemes locally of finite presentations, it defines a functor
\[ \mathrm{Sch}^{\mathrm{f.p.}}({{\mathcal T}_{\mathrm{\acute{e}t}}}) \to \mathrm{Sch}^{\mathrm{f.p.}}({{\mathcal T}_{\mathrm{an}}}) \subset \mathrm{dAn}_{\mathbb C} \]
The goal of this section is to show that it coincides with the classical analytification of Grothendieck.

\begin{lem} \label{lem:analytification_open}
	Let $X = ({\mathcal X}, {\mathcal O}_{\mathcal X})$ be a derived {Deligne-Mumford\xspace} stack and let $X{^\mathrm{an}} = ({\mathcal X}{^\mathrm{an}}, {\mathcal O}_{{\mathcal X}{^\mathrm{an}}})$ be its analytification.
	Let $Y = ({\mathcal Y}, {\mathcal O}_{\mathcal Y}) \to X$ be an \'etale morphism of {Deligne-Mumford\xspace} stacks.
	Then the analytification $Y{^\mathrm{an}}$ can be explicitly described as the pair $({\mathcal Z}, {\mathcal O}_{\mathcal Z})$, where the $\infty$-topos ${\mathcal Z}$ is defined to be the pullback of
	\[ \begin{tikzcd}
	{\mathcal Z} \arrow{r}{j_*} \arrow{d} & {\mathcal X}{^\mathrm{an}} \arrow{d}{p_*} \\
	{\mathcal Y} \arrow{r}{i_*} & {\mathcal X}
	\end{tikzcd} \]
	while ${\mathcal O}_{\mathcal Z}$ is defined to be $j{^{-1}} {\mathcal O}_{{\mathcal X}{^\mathrm{an}}}$.
	Suppose furthermore that ${\mathcal Y} \simeq {\mathcal X}_{/U}$ and that $U \to \mathbf 1_{\mathcal X}$ is an effective epimorphism.
	Then ${\mathcal Z} \simeq {\mathcal X}{^\mathrm{an}}_{/p{^{-1}} U}$ and $p{^{-1}} U \to \mathbf 1_{\mathcal Z}$ is an effective epimorphism.
\end{lem}

\begin{proof}
	The first part is just a reformulation of \cite[Lemma 2.1.3]{DAG-V}.
	The second part follows from the universal property of \'etale subtopoi (cf.\ \cite[6.3.5.8]{HTT}) and from the fact that $p{^{-1}}$ commutes with truncations and (therefore) with effective epimorphisms (see \cite[5.5.6.28]{HTT}).
\end{proof}

\begin{prop} \label{prop:analytification_smooth}
	Let $X \in {{\mathcal T}_{\mathrm{\acute{e}t}}}$ be a smooth (derived) scheme.
	Then the analytification of $X$ is $0$-localic and $0$-truncated, and it coincides with the classical analytification defined in \cite{SGA1}.
\end{prop}

\begin{proof}
	In virtue of \cref{lem:analytification_open}, we only need to show that this results holds true for $X = \mathbb A^n_{\mathbb C}$, the algebraic $n$-dimensional affine space.
	If $({\mathcal Y}, {\mathcal O}_{\mathcal Y})$ is any ${{\mathcal T}_{\mathrm{\acute{e}t}}}$-structured topos, we have
	\[ \operatorname{Hom}_{{\mathrm{L} \mathcal{T} \mathrm{op}}({{\mathcal T}_{\mathrm{\acute{e}t}}})^{\mathrm{op}}} (({\mathcal Y}, {\mathcal O}_{\mathcal Y}), \operatorname{Spec}^{{\mathcal T}_{\mathrm{\acute{e}t}}}(\mathbb A^n_{\mathbb C})) \simeq {\mathcal O}_{\mathcal Y}(\mathbb A^1)^n \]
	Suppose now that $({\mathcal Z}, {\mathcal O}_{\mathcal Z})$ is a derived {$\mathbb C$-analytic\xspace} space. Then, if we denote by $\mathcal E^n_{\mathbb C}$ the analytic $n$-dimensional affine space, we have
	\[ \operatorname{Hom}_{{\mathrm{L} \mathcal{T} \mathrm{op}}({{\mathcal T}_{\mathrm{an}}})^{\mathrm{op}}} ( ({\mathcal Z}, {\mathcal O}_{\mathcal Z}), \operatorname{Spec}^{{\mathcal T}_{\mathrm{an}}}(\mathcal E^n_{\mathbb C}) ) \simeq {\mathcal O}_{\mathcal Z}(\mathcal E^1_{\mathbb C})^n \simeq {\mathcal O}_{\mathcal Z}{^\mathrm{alg}}(\mathbb A^1_{\mathbb C})^n \]
	We conclude that $\operatorname{Spec}^{{\mathcal T}_{\mathrm{an}}}(\mathcal E^n_{\mathbb C})$ is the analytification of $\mathbb A^n_{\mathbb C}$.
\end{proof}

\begin{prop} \label{prop:comparison_classical_analytification}
	Let $X = ({\mathcal X}, {\mathcal O}_X)$ be a scheme locally of finite type over $\mathbb C$, seen as a derived scheme.
	Let $({\mathcal X}{^\mathrm{an}}, {\mathcal O}_{X{^\mathrm{an}}})$ be the analytification in the sense of \cite{DAG-IX}.
	Then ${\mathcal O}_{X{^\mathrm{an}}}$ is $0$-truncated and moreover the canonical morphism
	\[ ({\mathcal X}{^\mathrm{an}}, {\mathcal O}_{X{^\mathrm{an}}}{^\mathrm{alg}}) \to ({\mathcal X}, {\mathcal O}_X) \]
	exhibits $({\mathcal X}{^\mathrm{an}}, {\mathcal O}_{X{^\mathrm{an}}}{^\mathrm{alg}})$ as analytification of $X$ in the sense of \cite{SGA1}.
\end{prop}

\begin{proof}
	Choose a geometric envelope ${{\mathcal T}_{\mathrm{an}}} \to {\mathcal G}_{\mathrm{an}}$.
	The universal property defining ${\mathcal G}_{\mathrm{an}}$ shows that the morphism ${{\mathcal T}_{\mathrm{\acute{e}t}}} \to {{\mathcal T}_{\mathrm{an}}}$ induces a (essentially) unique limit-preserving functor ${\mathcal G}_{\mathrm{\acute{e}t}} \to {\mathcal G}_{\mathrm{an}}$.
	Since the analytification functor of \cite{DAG-IX} is defined to be the relative spectrum $\mathrm{Spec}^{{\mathcal G}_{\mathrm{an}}}_{{\mathcal G}_{\mathrm{\acute{e}t}}}$, the first statement follows directly from \cref{prop:relative_spectrum_truncated_objects}.
	
	As for the second statement, we can assume $X$ to be affine and therefore choose a closed immersion $X \to \mathbb A^n_{\mathbb C}$.
	Let $J \subset \mathbb C[X_1, \ldots, X_n]$ be the ideal defining $X$.
	Choose generators $f_1, \ldots, f_m \in J$ and consider the morphism $f \colon \mathbb A^n \to \mathbb A^m$ classified by these elements.
	We can therefore describe $X$ as the truncation of the pullback
	\[ \begin{tikzcd}
	X \arrow{d} \arrow{r} & \mathbb A^n_{\mathbb C} \arrow{d}{f} \\
	\operatorname{Spec}(\mathbb C) \arrow{r}{0} & \mathbb A^m_{\mathbb C}
	\end{tikzcd} \]
	computed in $\mathrm{dSch}_{\mathbb C}$.
	The analytification functor of \cite{DAG-IX} is a right adjoint. In particular, it commutes with limits.
	We can therefore identify $({\mathcal X}{^\mathrm{an}}, {\mathcal O}_{X{^\mathrm{an}}})$ with the pullback in ${\mathcal T\mathrm{op}}({{\mathcal T}_{\mathrm{an}}})$
	\[ \begin{tikzcd}
	({\mathcal X}{^\mathrm{an}}, {\mathcal O}_{X{^\mathrm{an}}}) \arrow{r} \arrow{d} & \operatorname{Spec}^{{\mathcal T}_{\mathrm{an}}}(\mathcal E^n_{\mathbb C}) \arrow{d} \\
	\operatorname{Spec}^{{\mathcal T}_{\mathrm{an}}}(*) \arrow{r} & \operatorname{Spec}^{{\mathcal T}_{\mathrm{an}}}(\mathcal E^m_{\mathbb C})
	\end{tikzcd} \]
	Observe that the bottom horizontal morphism is a closed immersion.
	Therefore, this is a pullback in $\mathrm{dAn}_{\mathbb C}$.
	In particular, we see that ${\mathcal X}{^\mathrm{an}}$ is a closed subtopos of ${\mathcal X}_{\mathcal E^n_{\mathbb C}}$, and it is therefore $0$-localic, and inspection shows that it is the topos associated to the topological space $X{^\mathrm{an}}$, the analytification of \cite{SGA1}.
	The result now follows from \cite[Lemma 12.19]{DAG-IX}.
\end{proof}

\begin{cor}
	Let $f \colon X \to Y$ be a closed immersion of derived schemes locally of finite presentation over $\mathbb C$.
	Then $f{^\mathrm{an}} \colon X{^\mathrm{an}} \to Y{^\mathrm{an}}$ is a closed immersion of derived {$\mathbb C$-analytic\xspace} spaces.
\end{cor}

\begin{proof}
	In virtue of \cref{prop:relative_spectrum_and_truncations}, it is enough to prove the statement when both $X$ and $Y$ are $0$-truncated.
	In this case, the result is an immediate consequence of \cref{prop:comparison_classical_analytification} and of \cite[Expos\'e XII, Proposition 3.2]{SGA1}.
\end{proof}

Let now $X = ({\mathcal X}, {\mathcal O}_{\mathcal X})$ be a $0$-truncated higher {Deligne-Mumford\xspace} stack, locally of finite presentation over $\operatorname{Spec}(\mathbb C)$.
The functor $\psi \colon {\mathrm{Sch}}({{\mathcal G}_{\mathrm{\acute{e}t}}^\mathrm{der}(k)}) \to {\mathrm{Sh}}(\mathrm{dAff}, {\tau_\mathrm{q\acute{e}t}})$ of \cite[Theorem 2.4.1]{DAG-V} is fully faithful.
It can be shown that if ${\mathcal X}$ is $n$-localic, then $\psi(X)$ is a geometric {Deligne-Mumford\xspace} stack in the sense of \cite{HAG-II} (see \cite{Porta_Comparison_2015} for a proof and a more precise comparison statement).
On the other side, $X{^\mathrm{an}} = ({\mathcal X}{^\mathrm{an}}, {\mathcal O}_{{\mathcal X}{^\mathrm{an}}})$ defines via the functor $\phi \colon {\mathrm{dAn}_{\mathbb C}} \to {\mathrm{Sh}}({\mathrm{Stn}^{\mathrm{der}}_{\mathbb C}}, \tau)$ of \cref{subsec:analytic_functor_of_points_I} an analytic geometric stack.
Since $X$ was $0$-truncated, \cref{prop:relative_spectrum_truncated_objects} shows that $X{^\mathrm{an}}$ is $0$-truncated and therefore \cref{cor:comparison_analytic_DM_stacks} shows that we can identify $\phi(X{^\mathrm{an}})$ with a higher {Deligne-Mumford\xspace} analytic stack in the sense of \cite{Porta_Yu_Higher_analytic_stacks_2014}.
We can therefore compare $\phi(X{^\mathrm{an}})$ with $\psi(X){^\mathrm{an}}$, where the latter is the analytification of $\psi(X)$ in the sense of \cite[§6.1]{Porta_Yu_Higher_analytic_stacks_2014}.
It is a rather easy task to show that the two notions coincide:

\begin{prop} \label{prop:comparison_analytification_DM_stacks}
	Let $({\mathcal X}, {\mathcal O}_{\mathcal X})$ be an $n$-localic $0$-truncated higher {Deligne-Mumford\xspace} stack locally of finite presentation over $\operatorname{Spec}(\mathbb C)$.
	Keeping the above notations, there exists a natural isomorphism $\phi(X{^\mathrm{an}}) \to \psi(X){^\mathrm{an}}$.
\end{prop}

\begin{proof}
	To be clearer in this proof, let us explicitly write ${\mathrm{Spec}^{{\mathcal T}_{\mathrm{an}}}_{{\mathcal T}_{\mathrm{\acute{e}t}}}}(X)$ instead of $X{^\mathrm{an}}$, and let us reserve the notation $(-){^\mathrm{an}}$ for the analytification functor of \cite{Porta_Yu_Higher_analytic_stacks_2014}.
	It follows from \cref{lem:analytification_open} that ${\mathrm{Spec}^{{\mathcal T}_{\mathrm{an}}}_{{\mathcal T}_{\mathrm{\acute{e}t}}}}(X)$ commutes with geometric realization of \'etale groupoids.
	The same is true for $\phi$ (see \cref{cor:little_phi_commutes_with_etale_colimits}) and for $\psi$ (see \cite{Porta_Comparison_2015}).
	Finally, $(-){^\mathrm{an}}$ is defined to be a left Kan extension, and therefore it has this property by construction.
	Therefore, it is sufficient to prove the statement when $X$ is an affine scheme of finite presentation over $\operatorname{Spec}(\mathbb C)$, and in this case the statement follows directly from \cref{prop:comparison_classical_analytification}.
\end{proof}

\subsection{Flatness I}

Let us begin with the following definition.

\begin{defin} \label{def:flat_morphism_of_structured_topoi}
	Let $({\mathcal X}, {\mathcal O}_{\mathcal X})$ and $({\mathcal Y}, {\mathcal O}_{\mathcal Y})$ be (connective) $\mathbb E_\infty$-structured topoi (i.e.\ elements of ${\mathcal T\mathrm{op}}({{\mathcal T}_{\mathrm{disc}}})$).
	We will say that a morphism $(f, \varphi) \colon ({\mathcal X}, {\mathcal O}_{\mathcal X}) \to ({\mathcal Y}, {\mathcal O}_{\mathcal Y})$ is \emph{flat} if the $\varphi \colon f{^{-1}} {\mathcal O}_{\mathcal Y} \to {\mathcal O}_{\mathcal X}$ is a flat morphism of sheaves of $\mathbb E_\infty$-rings.
\end{defin}

\begin{rem}
	Recall that a morphism $\varphi \colon {\mathcal A} \to {\mathcal B}$ of (connective) $\mathbb E_\infty$-sheaves on an $\infty$-topos ${\mathcal X}$ is said to be flat if the induced base change functor
	\[ - \otimes_{\mathcal A} {\mathcal B} \colon {\mathcal A} \textrm{-} {\mathrm{Mod}} \to {\mathcal B} \textrm{-} {\mathrm{Mod}} \]
	is $t$-exact (with respect to the canonical $t$-structures of ${\mathcal A} \textrm{-} {\mathrm{Mod}}$ and ${\mathcal B} \textrm{-} {\mathrm{Mod}}$).
	If ${\mathcal X}$ has enough points, this is equivalent to ask that for every geometric point $\eta{^{-1}} \colon {\mathcal X} \rightleftarrows {\mathcal S} \colon \eta_*$, the induced morphism $\eta{^{-1}}(\varphi) \colon \eta{^{-1}} {\mathcal A} \to \eta{^{-1}} {\mathcal B}$ is a flat morphism of (connective) $\mathbb E_\infty$-rings.
	In other words, the $\pi_0(\eta{^{-1}}(\varphi))$ is flat and one has equivalences
	\[ \pi_i(\eta{^{-1}} {\mathcal A}) \otimes_{\pi_0 (\eta{^{-1}} {\mathcal A})} \pi_0 (\eta{^{-1}} {\mathcal B}) \to \pi_i (\eta{^{-1}} {\mathcal B}) \]
	(that is, the morphism $\eta{^{-1}}(f)$ is strong).
\end{rem}

Let $({\mathcal X}, {\mathcal O}_{\mathcal X})$ be a derived {Deligne-Mumford\xspace} stack locally of finite presentation over $\mathbb C$ and let us write
\[ \mathrm{Spec}^{{\mathcal T}_{\mathrm{an}}}_{{\mathcal T}_{\mathrm{\acute{e}t}}}({\mathcal X}, {\mathcal O}_{\mathcal X}) = ({\mathcal X}{^\mathrm{an}}, {\mathcal O}_{{\mathcal X}{^\mathrm{an}}}) \]
Our goal is to show that the canonical morphism
\[ ({\mathcal X}{^\mathrm{an}}, {\mathcal O}_{{\mathcal X}{^\mathrm{an}}}{^\mathrm{alg}}) \to ({\mathcal X}, {\mathcal O}_{\mathcal X}) \]
is flat in the sense of \cref{def:flat_morphism_of_structured_topoi}.
This statement is local on the {Deligne-Mumford\xspace} stack $({\mathcal X}, {\mathcal O}_{\mathcal X})$, and therefore it will be sufficient to prove it for $\operatorname{Spec}^{{\mathcal T}_{\mathrm{\acute{e}t}}}(A)$, where $A$ is a connective $\mathbb E_\infty$-ring of finite presentation over $\mathbb C$.
The task will be greatly simplified if we could replace the ${{\mathcal T}_{\mathrm{\acute{e}t}}}$-scheme $\operatorname{Spec}^{{\mathcal T}_{\mathrm{\acute{e}t}}}(A)$ (whose underlying $\infty$-topos is $1$-localic) with the ${{\mathcal T}_{\mathrm{Zar}}}$-scheme $\operatorname{Spec}^{{\mathcal T}_{\mathrm{Zar}}}(A)$ (whose underlying $\infty$-topos is $0$-localic).
For this, we will need a digression on the relative spectrum associated to the morphism of pregeometries ${{\mathcal T}_{\mathrm{Zar}}} \to {{\mathcal T}_{\mathrm{\acute{e}t}}}$.

Let ${\mathcal X}$ be an $\infty$-topos.
Recall that the forgetful functor
\[ {\mathrm{Str}^\mathrm{loc}}_{{\mathcal T}_{\mathrm{\acute{e}t}}}({\mathcal X}) \to {\mathrm{Str}^\mathrm{loc}}_{{\mathcal T}_{\mathrm{Zar}}}({\mathcal X}) \]
is fully faithful.
If moreover we suppose that the hypercompletion ${\mathcal X}^{\wedge}$ has enough points, then an hypercomplete object ${\mathcal O} \in {\mathrm{Str}^\mathrm{loc}}_{{\mathcal T}_{\mathrm{Zar}}}({\mathcal X}^\wedge)$ belongs to the essential image of this functor if and only if all the stalks $\eta{^{-1}} {\mathcal O}$ are strictly henselian $\mathbb E_\infty$-rings.
Therefore, if $({\mathcal X}, {\mathcal O}_{\mathcal X})$ is a ${{\mathcal T}_{\mathrm{\acute{e}t}}}$-structure topos, we will denote by again by $({\mathcal X}, {\mathcal O}_{\mathcal X})$ the associated ${{\mathcal T}_{\mathrm{Zar}}}$-structured topos.

\begin{prop} \label{prop:relative_zariski_etale_spectrum}
	Let $A \in \mathrm{CAlg}_{\mathbb C}$ be a connective $\mathbb C$-algebra.
	Then:
	\begin{enumerate}
		\item $\mathrm{Spec}^{{\mathcal T}_{\mathrm{an}}}_{{\mathcal T}_{\mathrm{\acute{e}t}}}( \operatorname{Spec}^{{\mathcal T}_{\mathrm{Zar}}}(A) ) \simeq \operatorname{Spec}^{{\mathcal T}_{\mathrm{\acute{e}t}}}(A)$.
		\item $\mathrm{Spec}^{{\mathcal T}_{\mathrm{an}}}_{{\mathcal T}_{\mathrm{Zar}}} (\operatorname{Spec}^{{\mathcal T}_{\mathrm{Zar}}}(A)) \simeq \mathrm{Spec}^{{\mathcal T}_{\mathrm{an}}}_{{\mathcal T}_{\mathrm{\acute{e}t}}} (\operatorname{Spec}^{{\mathcal T}_{\mathrm{\acute{e}t}}}(A))$.
		\item the canonical map $q \colon \operatorname{Spec}^{{\mathcal T}_{\mathrm{\acute{e}t}}}(A) \to \operatorname{Spec}^{{\mathcal T}_{\mathrm{Zar}}}(A)$ is flat in the sense of \cref{def:flat_morphism_of_structured_topoi}.
	\end{enumerate}
\end{prop}

\begin{proof}
	Statements (1) and (2) follow directly from the universal properties of the relative and absolute spectrum functors.
	We will prove statement (3).
	Let us denote by ${\mathcal O}_A^{\mathrm{\acute{e}t}}$ the ${{\mathcal T}_{\mathrm{\acute{e}t}}}$-structure sheaf of $\operatorname{Spec}^{{\mathcal T}_{\mathrm{\acute{e}t}}}(A)$ and by ${\mathcal O}_A$ the ${{\mathcal T}_{\mathrm{Zar}}}$-structure sheaf of $\operatorname{Spec}^{{\mathcal T}_{\mathrm{Zar}}}(A)$.
	We know that ${\mathcal O}_A^{\mathrm{\acute{e}t}}$ is an hypercomplete object of ${\mathrm{Sh}}(A_{\mathrm{\acute{e}t}}, {\tau_\mathrm{q\acute{e}t}})$ (see \cite[Theorem 8.4.2.(3)]{DAG-VII}).
	Let $A \to B$ be an \'etale morphism. We can then factor it as $A \to A' \to B$, where $A \to A'$ is a Zariski open immersion, and one has $q{^{-1}}({\mathcal O}_A)(B) = {\mathcal O}_A(A') = A'$.
	From this, we conclude that $q{^{-1}} {\mathcal O}_A$ is hypercomplete as well.
	
	It follows from J.\ Lurie's version of Deligne's completeness theorem \cite[Theorem 4.1]{DAG-VII} that the hypercompletion ${\mathrm{Sh}}(A_{\mathrm{\acute{e}t}}, {\tau_\mathrm{q\acute{e}t}})^\wedge$ has enough points, we can check the flatness of ${\mathcal O}_A^{\mathrm{\acute{e}t}} \to q{^{-1}} {\mathcal O}_A$ on stalks.
	If $\eta{^{-1}} \colon {\mathrm{Sh}}(A_{\mathrm{\acute{e}t}}, {\tau_\mathrm{q\acute{e}t}}) \leftrightarrows {\mathcal S} \colon \eta_*$ is a geometric point, \cite[Proposition 1.1.6]{DAG-VIII} shows that we can find a filtered diagram $\{A_\alpha\}$ of \'etale $A$-algebras such that for every $F \in {\mathrm{Sh}}(A_{\mathrm{acute{e}t}}, {\tau_\mathrm{q\acute{e}t}})$ one has
	\[ \eta{^{-1}}(F) = \lim F(A_\alpha)  \]
	Moreover, $\eta{^{-1}}$ is canonically determined by a morphism $x \colon \operatorname{Spec}(k) \to \operatorname{Spec}(A)$ for some algebraically closed field $k$.
	It follows that
	\[ \eta{^{-1}} q{^{-1}} {\mathcal O}_A = A_x \]
	where $A_x$ denotes the Zariski stalk at the point $x$.
	On the other side, $A_x^{\mathrm{s.h.}} = \eta{^{-1}} {\mathcal O}_A^{\mathrm{\acute{e}t}}$ is a strict henselianization of $A_x$.
	It follows that the canonical morphism $A_x \to A_x^{\mathrm{s.h.}}$ is flat.
	{\ignorespaces}
	This completes the proof.
\end{proof}

\begin{cor} \label{cor:replacing_etale_with_zariski}
	Let $A$ be a connective $\mathbb E_\infty$-algebra of finite presentation over $\mathbb C$.
	Let us write $\mathrm{Spec}^{{\mathcal T}_{\mathrm{an}}}_{{\mathcal T}_{\mathrm{Zar}}}(\operatorname{Spec}^{{\mathcal T}_{\mathrm{Zar}}}(A)) = ({\mathcal X}, {\mathcal O}_{\mathcal X})$.
	Then the following are equivalent:
	\begin{enumerate}
		\item the morphism $({\mathcal X}, {\mathcal O}_{\mathcal X}{^\mathrm{alg}}) \to \operatorname{Spec}^{{\mathcal T}_{\mathrm{Zar}}}(A)$ is flat;
		\item the morphism $({\mathcal X}, {\mathcal O}_{\mathcal X}{^\mathrm{alg}}) \to \operatorname{Spec}^{{\mathcal T}_{\mathrm{\acute{e}t}}}(A)$ is flat.
	\end{enumerate}
\end{cor}

\subsection{Analytification of germs}

From this point on, we will focus on the Zariski analytification functor $\mathrm{Spec}^{{\mathcal T}_{\mathrm{an}}}_{{\mathcal T}_{\mathrm{Zar}}}$.

Let us recall that the analytification functor of \cite[Expos\'e XII]{SGA1} is defined only for schemes locally of finite presentation over $\mathbb C$.
One of the advantages of the relative spectrum functor introduced in \cite{DAG-V} is that it can be applied also to germs of schemes, that is to ${{\mathcal T}_{\mathrm{Zar}}}$-topoi of the form $({\mathcal S}, A)$, where $A$ is a local $\mathbb E_\infty$-algebra.
We will turn our attention to the study of the ${{\mathcal T}_{\mathrm{an}}}$-structured topos $\mathrm{Spec}^{{\mathcal T}_{\mathrm{an}}}_{{\mathcal T}_{\mathrm{Zar}}}({\mathcal S}, A)$ in the special case where $A$ arises as the ring of Zariski germs at one point of a derived (Zariski) scheme locally of finite presentation over $\mathbb C$.
{\ignorespaces}

\begin{lem} \label{lem:point_pullback_analytification}
	Let $f \colon X \to Y$ be a morphism of topological spaces.
	Let $y \in Y$ be a closed point and suppose that $f{^{-1}}(y) = \{x\}$.
	Then the diagram
	\[ \begin{tikzcd}
	{\mathcal S} \arrow{d}{x_*} \arrow{r}{\mathrm{id}} & {\mathcal S} \arrow{d}{y_*} \\
	{\mathrm{Sh}}(X) \arrow{r}{f_*} & {\mathrm{Sh}}(Y)
	\end{tikzcd} \]
	is a pullback square in ${\mathcal T\mathrm{op}}$.
\end{lem}

\begin{proof}
	The morphism $y_*$ is a closed immersion of $\infty$-topoi.
	Set $V \coloneqq Y \setminus \{y\}$.
	Seeing $V$ as a $(-1)$-truncated object in ${\mathrm{Sh}}(Y)$, we can identify $y_* \colon {\mathcal S} \to {\mathrm{Sh}}(Y)$ with the inclusion ${\mathrm{Sh}}(Y)/V \to {\mathrm{Sh}}(Y)$.
	The pullback along $f_*$ can therefore be identified with ${\mathrm{Sh}}(X) / f{^{-1}}(V)$.
	Observe that $f{^{-1}}(V)$ can be identified with the $(-1)$-truncated object in ${\mathrm{Sh}}(Y)$ represented by the inverse image $U$ of $V$ along $f$.
	By hypothesis $X \setminus U = \{x\}$.
	It follows that ${\mathrm{Sh}}(X) / f{^{-1}}(V) \to {\mathrm{Sh}}(X)$ can be identified with $x_* \colon {\mathcal S} \to {\mathrm{Sh}}(X)$.
	The proof is now complete.
\end{proof}

Let $(X, {\mathcal O}_X)$ be a (Zariski) derived scheme locally of finite presentation over $\mathbb C$ and let $({\mathcal X}{^\mathrm{an}}, {\mathcal O}_{X{^\mathrm{an}}})$ be its analytification in the sense of \cite{DAG-IX}.
It follows from \cref{prop:comparison_classical_analytification} that ${\mathcal X}{^\mathrm{an}} = {\mathrm{Sh}}(X{^\mathrm{an}})$, where $X{^\mathrm{an}}$ is the underlying topological space of the analytification of $(X, {\mathcal O}_X)$ in the sense of \cite{SGA1}.
Let $x_* \colon {\mathcal S} \to {\mathrm{Sh}}(X{^\mathrm{an}})$ be a geometric point.
The induced map
\[ x{^{-1}} p{^{-1}} {\mathcal O}_X \to x{^{-1}} {\mathcal O}_{X{^\mathrm{an}}}{^\mathrm{alg}} \]
can be seen as a morphism
\[ ({\mathcal S}, y{^{-1}} {\mathcal O}_{X{^\mathrm{an}}}{^\mathrm{alg}}) \to ({\mathcal S}, x{^{-1}} {\mathcal O}_{X}) \]
It follows from \cref{lem:point_pullback_analytification} and \cite[Lemma 2.1.3]{DAG-V} that we can identify, via the above morphism, $({\mathcal S}, y{^{-1}} {\mathcal O}_{X{^\mathrm{an}}})$ with the analytification of $({\mathcal S}, x{^{-1}} {\mathcal O}_{X})$.

We will now give a more explicit characterization of the ${{\mathcal T}_{\mathrm{an}}}$-structure ${\mathcal O}_{X{^\mathrm{an}}, y} \coloneqq y{^{-1}} {\mathcal O}_{X{^\mathrm{an}}}$ in term of ${\mathcal O}_{X,x} \coloneqq x{^{-1}} {\mathcal O}_X$.
Introduce the functor
\[ \overline{\Psi} \colon {\mathrm{Str}^\mathrm{loc}}_{{\mathcal T}_{\mathrm{Zar}}}({\mathcal S})_{/\mathbb C} \to {\mathrm{Str}^\mathrm{loc}}_{{\mathcal T}_{\mathrm{an}}}({\mathcal S})_{/{\mathcal H}_0} \]
which is by definition the left adjoint to
\[ \overline{\Phi} \colon {\mathrm{Str}^\mathrm{loc}}_{{\mathcal T}_{\mathrm{an}}}({\mathcal S})_{/{\mathcal H}_0} \to {\mathrm{Str}^\mathrm{loc}}_{{\mathcal T}_{\mathrm{Zar}}}({\mathcal S})_{/\mathbb C} \]
We have:

\begin{lem} \label{lem:Psi_computes_absolute_analytification}
	Keeping the above notations, ${\mathcal O}_{X{^\mathrm{an}}, y} = \overline{\Psi}({\mathcal O}_{X,x})$.
\end{lem}

\begin{proof}
	We already argued that $({\mathcal S}, {\mathcal O}_{X{^\mathrm{an}}, y})$ is the analytification in the sense of \cite{DAG-IX} of $({\mathcal S}, {\mathcal O}_{X,x})$.
	The universal property of the analytification shows therefore that for every ${\mathcal O} \in {\mathrm{Str}^\mathrm{loc}}_{{\mathcal T}_{\mathrm{an}}}({\mathcal S})_{/{\mathcal H}_0}$ we have
	\begin{align*}
	\operatorname{Map}_{{\mathrm{Str}^\mathrm{loc}}_{{\mathcal T}_{\mathrm{an}}}({\mathcal S})_{/{\mathcal H}_0}}({\mathcal O}_{X{^\mathrm{an}},y}, {\mathcal O}) & \simeq \operatorname{Map}_{{\mathcal T\mathrm{op}}({{\mathcal T}_{\mathrm{an}}})}(({\mathcal S}, {\mathcal O}), ({\mathcal S}, {\mathcal O}_{X{^\mathrm{an}},y}) ) \\
	& \simeq \operatorname{Map}_{{\mathcal T\mathrm{op}}({{\mathcal T}_{\mathrm{Zar}}})}(({\mathcal S}, {\mathcal O}{^\mathrm{alg}}), ({\mathcal S}, {\mathcal O}_{X,x})) \\
	& \simeq \operatorname{Map}_{{\mathrm{Str}^\mathrm{loc}}_{{\mathcal T}_{\mathrm{Zar}}}({\mathcal S})_{/\mathbb C}}({\mathcal O}_{X,x}, {\mathcal O}{^\mathrm{alg}} )
	\end{align*}
	Therefore, we conclude that ${\mathcal O}' = \overline{\Psi}({\mathcal O})$.
\end{proof}

We formulate the following conjecture generalizing \cref{lem:Psi_computes_absolute_analytification}:

\begin{conj}
	Let ${\mathcal O} \in {\mathrm{Str}^\mathrm{loc}}_{{\mathcal T}_{\mathrm{Zar}}}({\mathcal S})_{/\mathbb C}$.
	Introduce the functor
	\[ \overline{\Psi} \colon {\mathrm{Str}^\mathrm{loc}}_{{\mathcal T}_{\mathrm{Zar}}}({\mathcal S})_{/\mathbb C} \to {\mathrm{Str}^\mathrm{loc}}_{{\mathcal T}_{\mathrm{an}}}({\mathcal S})_{/{\mathcal H}_0} \]
	left adjoint to the underlying algebra functor.
	Then the unit ${\mathcal O} \to \overline{\Psi}({\mathcal O}){^\mathrm{alg}}$ exhibits $({\mathcal S}, \overline{\Psi}({\mathcal O}))$ as analytification of $({\mathcal S}, {\mathcal O})$.
\end{conj}

{\ignorespaces}
	
\subsection{Flatness II}
	
Let us denote by $\mathrm{CAlg}_{\mathbb C}^{\mathrm{loc}}$ the $\infty$-category of local $\mathbb C$-algebras (with residue field $\mathbb C$).
We have a canonical identification $\mathrm{CAlg}_{\mathbb C}^{\mathrm{loc}} \simeq {\mathrm{Str}^\mathrm{loc}}_{{\mathcal T}_{\mathrm{Zar}}}({\mathcal S})_{/ \mathbb C}$.
Recall also from \cref{def:local_analytic_ring} that we denote by ${\mathrm{AnRing}_{\mathbb C}}^{\mathrm{loc}}$ the $\infty$-category ${\mathrm{Str}^\mathrm{loc}}_{{\mathcal T}_{\mathrm{an}}}({\mathcal S})_{/{\mathcal H}_0}$.
The adjunction considered at the end of the previous section can be rewritten as
\[ \overline{\Psi} \colon \mathrm{CAlg}_{\mathbb C}^{\mathrm{loc}} \rightleftarrows {\mathrm{AnRing}_{\mathbb C}}^{\mathrm{loc}} \colon \overline{\Phi} \]
Accordingly to the notations of \cite{DAG-IX}, we will write $(-){^\mathrm{alg}}$ to denote the functor $\overline{\Phi}$.
There is a forgetful functor
\[ \mathrm{CAlg}^{\mathrm{loc}} \to {\mathcal S} \]
which admits a left adjoint. We will denote it by $\mathbb C[-]$.
Observe that $\pi_0 \mathbb C[\{*\}]$ can be identified with the ring of germs $\mathbb C[T]_{(0)}$.

The universal properties involved show that there is a natural equivalence ${\mathcal H}\{-\} \simeq \overline{\Psi} \circ \mathbb C[-]$, where ${\mathcal H}\{-\}$ is the $\infty$-functor associated to the functor introduced in \cref{subsec:strict_models}.
To further simplify notations, we will write $A{^\mathrm{an}}$ instead of $\overline{\Psi}(A){^\mathrm{alg}}$.
If $K \in {\mathcal S}$ is a space, we will further write $\mathbb C\{K\}$ instead of $\mathbb C[K]{^\mathrm{an}}$.
	
\begin{lem} \label{lem:surjective_on_pi_0_argument}
	Suppose that $f \colon A \to B$ is a morphism in the $\infty$-category ${\mathrm{AnRing}_{\mathbb C}}^{\mathrm{loc}}$ which is surjective on $\pi_0$.
	Then for every other morphism $g \colon A \to C$, the forgetful functor $(-){^\mathrm{alg}} \colon {\mathrm{AnRing}_{\mathbb C}}^{\mathrm{loc}} \to \mathrm{CAlg}^{\mathrm{loc}}_{\mathbb C}$ preserves the homotopy pushout of this diagram.
\end{lem}
	
\begin{proof}
	We can see $A$, $B$ and $C$ as ${{\mathcal T}_{\mathrm{an}}}$-structures on ${\mathcal S}$.
	The maps $A \to B$ and $A \to C$ define morphisms $({\mathcal S}, B) \to ({\mathcal S}, A)$ and $({\mathcal S}, C) \to ({\mathcal S}, A)$ in ${\mathcal T\mathrm{op}}({{\mathcal T}_{\mathrm{an}}})$.
	The result follows now from combining \cite[Proposition 10.3, Lemma 11.10]{DAG-IX}.
\end{proof}
	
\begin{lem} \label{lem:flatness_of_the_boundary}
	For every $n \in \mathbb N$, the maps $\mathbb C[\Delta^n] \to \mathbb C\{\Delta^n\}$ and $\mathbb C[\partial \Delta^n] \to \mathbb C\{\partial \Delta^n\}$ are flat.
\end{lem}
	
\begin{proof}
	The morphism $\mathbb C[\Delta^0] \to \mathbb C[\Delta^n]$ is an acyclic cofibration, and the same goes for $\overline{\Psi}(\mathbb C[\Delta^0]) \to \overline{\Psi}(\mathbb C[\Delta^n])$.
	In particular, $\mathbb C\{\Delta^0\} \to \mathbb C\{\Delta^n\}$ is a weak equivalence.
	Now we have $\pi_0(\mathbb C[\Delta^0]) = (\mathbb C[\Delta^0])_0 = \mathbb C[T]$, while $\pi_0(\mathbb C\{\Delta^0\}) = (\mathbb C\{\Delta^0\})_0 = \mathbb C\{T\}$. Since $\mathbb C[T] \to \mathbb C\{T\}$ is flat, the first statement follows at once.

	Let us turn to the case of $\partial \Delta^n$.
	When $n = 0, 1$, the result is trivial.
	If $n \ge 2$, we can present $\partial \Delta^n$ as the (homotopy) pushout of a diagram of the form
	\[ \begin{tikzcd}
		\coprod \Delta^{n-2} \sqcup \Delta^{n-2} \arrow{r}{f} \arrow{d} & \coprod \Delta^{n-1} \arrow{d} \\
		\coprod \Delta^{n-2} \arrow{r} & \partial \Delta^n
	\end{tikzcd} \]
	Both the functors $\mathbb C[-]$ and $\overline{\Psi}(\mathbb C[-])$ preserve this pushout.
	Moreover, the morphism
	\[ \coprod \Delta^{n-2} \sqcup \Delta^{n-2} \to \coprod \Delta^{n-2} \]
	is a (degreewise) epimorphism, and the functors $\mathbb C[-]$ and $\overline{\Psi}$ commute with degreewise epimorphisms (being left adjoints).
	It follows that the induced morphism
	\[ \widehat{\bigotimes} \overline{\Psi}(\mathbb C[\Delta^{n-2}]) {\widehat{\otimes}} \overline{\Psi}(\mathbb C[\Delta^{n-2}]) \to \widehat{\bigotimes} \overline{\Psi}(\mathbb C[\Delta^{n-2}]) \]
	is a degreewise epimorphism.
	It is moreover an epimorphism on $\pi_0$, and therefore the image via $\mathbb C\{-\}$ of the above homotopy pushout is still a homotopy pushout.
	In other words, the diagram
	\[ \begin{tikzcd}
		\mathbb C \left\{ \coprod \Delta^{n-2} \sqcup \Delta^{n-2} \right\} \arrow{r} \arrow{d} & \mathbb C \left\{ \coprod \Delta^{n-1} \right\} \arrow{d} \\
		\mathbb C \left\{ \coprod \Delta^{n-2} \right\} \arrow{r} & \mathbb C\{\partial \Delta^n\}
	\end{tikzcd} \]
	is both a homotopy pushout and a strict pushout.
	{\ignorespaces}
	Observe moreover that
	\[ \begin{tikzcd}
		\mathbb C \left[ \coprod \Delta^{n-2} \sqcup \Delta^{n-2} \right] \arrow{r} \arrow{d} & \mathbb C \left[ \coprod \Delta^{n-2} \right] \arrow{d} \\
		\mathbb C \left\{ \coprod \Delta^{n-2} \sqcup \Delta^{n-2} \right\} \arrow{r} & \mathbb C \left\{ \coprod \Delta^{n-2} \right\}
	\end{tikzcd} \]
	is a strict pushout, that this diagram is homotopy equivalent to a (strict) pushout diagram of discrete objects and that the left vertical map is flat.
	Therefore it is homotopy equivalent to a homotopy pushout and therefore it is a homotopy pushout itself.
	{\ignorespaces}
	As consequence, we see that the map $\mathbb C[\partial \Delta^n] \to \mathbb C\{\partial \Delta^n\}$ can be identified with the homotopy pushout of the map $\mathbb C [\coprod \Delta^{n-1}] \to \mathbb C\{\coprod \Delta^{n-1}\}$, which is flat.
	It follows that $\mathbb C[\partial \Delta^n] \to \mathbb C\{\partial \Delta^n\}$ is flat as well, thus completing the proof.
\end{proof}
	
\begin{lem} \label{lem:analytification_boundary_pushout}
	The diagram
	\[ \begin{tikzcd}
		\mathbb C[\partial \Delta^n] \arrow{r} \arrow{d} & \mathbb C[t] \arrow{d} \\
		\mathbb C\{\partial \Delta^n\} \arrow{r} & \mathbb C \{z\}
	\end{tikzcd} \]
	is a pushout in the category of connective $\mathbb E_\infty$-algebras over $\mathbb C$.
\end{lem}
	
\begin{proof}
	Let $R = \mathbb C\{\partial \Delta^n\} \otimes_{\mathbb C[\Delta^n]} \mathbb C[t]$. We have a canonical map $R \to \mathbb C\{z\}$.
	It will be sufficient to show that this map induces an equivalence on homotopy groups.
	We know that
	\[ \pi_0(R) = {\mathrm{Tor}}^{\pi_0(\mathbb C[\partial \Delta^n])}_0( \pi_0(\mathbb C\{\partial \Delta^n\}, \pi_0(\mathbb C[t])) \]
	Using the computations of \cref{prop:pi_*_of_the_point} and \cref{prop:pi_0_of_the_boundary}, we see that this (underived) tensor product is simply $\mathbb C\{z\}$.
	We are therefore reduced to prove that $\pi_i(R) = 0$ for $i > 0$.
	Consider the spectral sequence
	\[ \begin{tikzcd}
		E^2_{pq} = {\mathrm{Tor}}^{\pi_*(\mathbb C[\partial \Delta^n])}_p( \pi_* (\mathbb C \{\partial \Delta^n\}), \pi_*(\mathbb C[t]))_q \Rightarrow \pi_{p+q}R
	\end{tikzcd} \]
	\Cref{lem:flatness_of_the_boundary} shows that $\pi_*(\mathbb C[\partial \Delta^n]) \to \pi_*(\mathbb C\{\partial \Delta^n\})$ is flat, so that this spectral sequence degenerates at the second page.
	Moreover, $\pi_*(\mathbb C[t])$ is concentrated in degree $0$, and therefore we conclude that the only non-vanishing homotopy group of $R$ is $\pi_0$.
	This completes the proof.
	{\ignorespaces}
\end{proof}
	
\begin{thm} \label{thm:analytification_flat}
	Let $A$ be a local (connective) $\mathbb E_\infty$-algebra over $\mathbb C$.
	Then the morphism $A \to A{^\mathrm{an}}$ is flat.
\end{thm}
	
\begin{proof}
	We can obtain $A$ as colimit of a (possibly transfinite) diagram $\{A_\alpha\}_{\alpha < \lambda}$, depicted as
	\[ A_0 \to A_1 \to \cdots \to A_n \to \cdots \]
	where $A_0 = \mathbb C$ each map $A_{\alpha} \to A_{\alpha+1}$ is the pushout of
	\[ \begin{tikzcd}
		\mathbb C[\partial \Delta^m] \arrow{r} \arrow{d} & \mathbb C[\Delta^0] \arrow{d} \\
		A_\alpha \arrow{r} & A_{\alpha+1} 
	\end{tikzcd} \]
	We will prove by transfinite induction that $A_\alpha \to A_\alpha{^\mathrm{an}}$ is flat. When $\alpha = 0$ there is nothing to prove.
	Since the functor $(-){^\mathrm{alg}}$ commutes with sifted colimits (thus in particular with sequential ones) and since flat morphisms are stable under filtered colimits, we only need to prove that $A_{\alpha+1} \to A_{\alpha +1}{^\mathrm{an}}$ is flat given that $A_\alpha \to A_\alpha{^\mathrm{an}}$ is.

	Applying $\overline{\Psi}$ to the above diagram, we obtain a pushout
	\[ \begin{tikzcd}
		{\mathcal H}\{\partial \Delta^m\} \arrow{r} \arrow{d} & {\mathcal H}\{\Delta^0\} \arrow{d} \\
		\overline{\Psi}(A_n) \arrow{r} & \overline{\Psi}(A_{n+1})
	\end{tikzcd} \]
	\Cref{prop:pi_0_of_the_boundary} shows that ${\mathcal H}\{\partial \Delta^m\} \to {\mathcal H}\{\Delta^0\}$ is a surjection on $\pi_0$ and therefore \cref{lem:surjective_on_pi_0_argument} guarantees that $(-){^\mathrm{alg}}$ preserves this pushout.
	{\ignorespaces}
	In particular, we obtain a commutative diagram
	\[ \begin{tikzcd}[column sep = small, row sep = small]
		{} & \mathbb C[\partial \Delta^m] \arrow{rr} \arrow{dd} \arrow{dl} & & \mathbb C[\Delta^0] \arrow{dd} \arrow{dl} \\
		\mathbb C\{\partial \Delta^m\} \arrow{dd} \arrow[crossing over]{rr} & & \mathbb C\{\Delta^0\} \\
		{} & A_n \arrow{rr} \arrow{dl} & & A_{n+1} \arrow{dl} \\
		A_n{^\mathrm{an}} \arrow{rr} & & A_{n+1}{^\mathrm{an}} \arrow[crossing over, leftarrow]{uu}
	\end{tikzcd} \]
	\Cref{lem:analytification_boundary_pushout} shows that the top square is a pushout.
	Therefore, in the rectangle
	\[ \begin{tikzcd}
		\mathbb C[\partial \Delta^m] \arrow{r} \arrow{d} & A_n \arrow{r} \arrow{d} & A_n{^\mathrm{an}} \arrow{d} \\
		\mathbb C[\Delta^0] \arrow{r} & A_{n+1} \arrow{r} & A_{n+1}{^\mathrm{an}}
	\end{tikzcd} \]
	both the left square and the outer one are pushouts. Therefore the same goes for the one on the right.
	Since $A_n \to A_n{^\mathrm{an}}$ was flat by hypothesis, we conclude that the same goes for $A_{n+1} \to A_{n+1}{^\mathrm{an}}$.
\end{proof}
	
\begin{cor} \label{cor:analytification_flat}
	Let $({\mathcal X}, {\mathcal O}_{\mathcal X})$ be a derived {Deligne-Mumford\xspace} stack locally of finite presentation over $\mathbb C$ and write $\mathrm{Spec}^{{\mathcal T}_{\mathrm{an}}}_{{\mathcal T}_{\mathrm{\acute{e}t}}}({\mathcal X}, {\mathcal O}_{\mathcal X}) = ({\mathcal X}{^\mathrm{an}}, {\mathcal O}_{{\mathcal X}{^\mathrm{an}}})$.
	Then the canonical map $p \colon ({\mathcal X}{^\mathrm{an}}, {\mathcal O}_{{\mathcal X}{^\mathrm{an}}}{^\mathrm{alg}}) \to ({\mathcal X}, {\mathcal O}_{\mathcal X})$ is flat (in the sense of \cref{def:flat_morphism_of_structured_topoi}).
\end{cor}
	
\begin{proof}
	The question being local on $({\mathcal X}, {\mathcal O}_{\mathcal X})$, we can assume $({\mathcal X}, {\mathcal O}_{\mathcal X}) = \operatorname{Spec}^{{\mathcal T}_{\mathrm{\acute{e}t}}}(A)$ for a connective $\mathbb E_\infty$-algebra $A$ of finite presentation over $\mathbb C$.
	It follows from \cref{cor:replacing_etale_with_zariski} that it is enough to prove that the canonical map
	\[ ({\mathcal X}{^\mathrm{an}}, {\mathcal O}_{{\mathcal X}{^\mathrm{an}}}{^\mathrm{alg}}) \to \operatorname{Spec}^{{\mathcal T}_{\mathrm{Zar}}}(A) \]
	is flat. In this case, \cref{prop:relative_spectrum_truncated_objects} and \cref{prop:comparison_classical_analytification} show that ${\mathcal X}{^\mathrm{an}}$ is a $0$-localic $\infty$-topos and \cref{lem:derived_canal_spaces_hypercomplete} shows that ${\mathcal X}{^\mathrm{an}}$ is hypercomplete and has enough points.
	We are therefore reduced to show that for every geometric point $x_* \colon {\mathcal S} \to {\mathcal X}{^\mathrm{an}}$, the induced map
	\[ x{^{-1}} p{^{-1}} {\mathcal O}_{\mathcal X} \to x{^{-1}} {\mathcal O}_{{\mathcal X}{^\mathrm{an}}}{^\mathrm{alg}} \]
	is flat.
	Therefore, we can invoke \cref{lem:Psi_computes_absolute_analytification} to reduce to the case of \cref{thm:analytification_flat}.
	This completes the proof.
\end{proof}
	
\subsection{Computing the analytification} \label{subsec:computing_analytification}
	
A first important application of this flatness result is that it allows to give a more concrete description of the analytification of a derived {Deligne-Mumford\xspace} stack in terms of the analytification of its truncation.
	
Before giving the details of this, though, we will need to recall some terminology.
If $A$ is a simplicial ring and $M$ is a connective $A$-module one can form the split square-zero extension $A \oplus M$.
This is rather an easy task if we are working in the model category $\mathrm{sCRing}$.
However, if we need to generalize it to a less elementary $\infty$-category, things become substantially more complicated.
We refer to \cite[§7.3.4]{Lurie_Higher_algebra} for a detailed account of this construction.
To stress the non-triviality of this construction, we prefer to suppress the notation $A \oplus M$ in virtue of $\Omega^\infty_A(M)$, which is reminiscent of the way the construction goes.
The framework developed in loc.\ cit.\ applies as well to sheaves of connective $\mathbb E_\infty$-rings on any $\infty$-topos and to their category of modules, and we will be using it precisely in this setting.
	
Let $({\mathcal X}, {\mathcal O}_{\mathcal X})$ be a {Deligne-Mumford\xspace} stack locally of finite presentation over $\mathbb C$.
\Cref{prop:relative_spectrum_and_truncations} shows that for every $n \ge 0$ the natural morphism
\[ {\mathrm{Spec}^{{\mathcal T}_{\mathrm{an}}}_{{\mathcal T}_{\mathrm{\acute{e}t}}}}({\mathcal X}, \tau_{\le n} {\mathcal O}_{\mathcal X}) \to {\mathrm{Spec}^{{\mathcal T}_{\mathrm{an}}}_{{\mathcal T}_{\mathrm{\acute{e}t}}}}({\mathcal X}, {\mathcal O}_{\mathcal X}) \]
exhibits ${\mathrm{Spec}^{{\mathcal T}_{\mathrm{an}}}_{{\mathcal T}_{\mathrm{\acute{e}t}}}}({\mathcal X}, \tau_{\le n} {\mathcal O}_{\mathcal X})$ as the $n$-truncation of ${\mathrm{Spec}^{{\mathcal T}_{\mathrm{an}}}_{{\mathcal T}_{\mathrm{\acute{e}t}}}}({\mathcal X}, {\mathcal O}_{\mathcal X})$.
We can therefore write
\[ {\mathrm{Spec}^{{\mathcal T}_{\mathrm{an}}}_{{\mathcal T}_{\mathrm{\acute{e}t}}}}({\mathcal X}, \tau_{\le n} {\mathcal O}_{\mathcal X}) = ({\mathcal X}{^\mathrm{an}}, \tau_{\le n} {\mathcal O}_{{\mathcal X}{^\mathrm{an}}}) \]
Let us further denote by $p_* \colon {\mathcal X}{^\mathrm{an}} \rightleftarrows {\mathcal X} \colon p{^{-1}}$ the induced geometric morphism.
	
We know that $\tau_{\le n} {\mathcal O}_{\mathcal X} \to \tau_{\le n - 1} {\mathcal O}_{\mathcal X}$ is a square-zero extension.
There exists therefore a pullback diagram
\[ \begin{tikzcd}
	\tau_{\le n} {\mathcal O}_{\mathcal X} \arrow{r} \arrow{d} & \tau_{\le n - 1} {\mathcal O}_{\mathcal X} \arrow{d}{d} \\
	\tau_{\le n - 1} {\mathcal O}_{\mathcal X} \arrow{r} & \Omega^\infty(\pi_n({\mathcal O}_{\mathcal X})[n+1])
\end{tikzcd} \]
Applying the functor $p{^{-1}}$ we obtain the pullback diagram
\[ \begin{tikzcd}
	\tau_{\le n} p{^{-1}} {\mathcal O}_{\mathcal X} \arrow{r} \arrow{d} & \tau_{\le n - 1} p{^{-1}} {\mathcal O}_{\mathcal X} \arrow{d}{p{^{-1}}(d)} \\
	\tau_{\le n - 1} p{^{-1}} {\mathcal O}_{\mathcal X} \arrow{r} & \Omega^\infty(\pi_n (p{^{-1}} {\mathcal O}_{\mathcal X} )[n+1])
\end{tikzcd} \]
Since the morphism $p{^{-1}} {\mathcal O}_{\mathcal X} \to {\mathcal O}_{{\mathcal X}{^\mathrm{an}}}{^\mathrm{alg}}$ is flat, we can invoke \cref{lem:flat_base_change_Postnikov_invariant} to conclude that
\[ \begin{tikzcd}
	\tau_{\le n} {\mathcal O}_{{\mathcal X}{^\mathrm{an}}}{^\mathrm{alg}} \arrow{r} \arrow{d} & \tau_{\le n - 1} {\mathcal O}_{{\mathcal X}{^\mathrm{an}}}{^\mathrm{alg}} \arrow{d}{p{^{-1}}(d) \otimes_{p{^{-1}} {\mathcal O}_{\mathcal X}} {\mathcal O}_{{\mathcal X}{^\mathrm{an}}}{^\mathrm{alg}}} \\
	\tau_{\le n - 1} {\mathcal O}_{{\mathcal X}{^\mathrm{an}}}{^\mathrm{alg}} \arrow{r} & \Omega^\infty(\pi_n({\mathcal O}_{{\mathcal X}{^\mathrm{an}}}{^\mathrm{alg}})[n+1])
\end{tikzcd} \]
is a pullback square.
In other words, the analytification of the $n$-th Postnikov invariant of ${\mathcal O}_{\mathcal X}$ is the $n$-th Postnikov invariant of ${\mathcal O}_{{\mathcal X}{^\mathrm{an}}}{^\mathrm{alg}}$.
	
\begin{cor} \label{cor:description_analytification}
	Let $({\mathcal X}, {\mathcal O}_{\mathcal X})$ be a {Deligne-Mumford\xspace} stack locally of finite presentation over $\mathbb C$.
	Let $({\mathcal X}{^\mathrm{an}}, {\mathcal O})$ be the analytification of $({\mathcal X}, \Omega^\infty_{\tau_{\le n - 1} {\mathcal O}_{\mathcal X}}(\pi_n({\mathcal O}_{\mathcal X}[n+1])))$.
	Then ${\mathcal O}{^\mathrm{alg}} = \Omega^\infty_{\tau_{\le n - 1} {\mathcal O}_{{\mathcal X}{^\mathrm{an}}}{^\mathrm{alg}}}(\pi_n({\mathcal O}_{{\mathcal X}{^\mathrm{an}}}{^\mathrm{alg}}[n+1]))$.
\end{cor}
	
\begin{proof}
	Indeed, we know that ${\mathcal O}{^\mathrm{alg}}$ is $n$-truncated and
	\[ \tau_{\le n - 1} {\mathcal O} \simeq \tau_{\le n - 1} {\mathcal O}_{{\mathcal X}{^\mathrm{an}}}{^\mathrm{alg}} \simeq \tau_{\le n - 1} \Omega^\infty_{\tau_{\le n - 1} {\mathcal O}_{\mathcal X}}(\pi_n({\mathcal O}_{\mathcal X})[n+1]) \]
	Therefore ${\mathcal O}{^\mathrm{alg}}$ is determined by its $n$-th Postnikov invariant.
	The above discussion allows to identify it with the analytification of the $n$-th Postnikov invariant of $\Omega^\infty_{\tau_{\le n - 1} {\mathcal O}_{\mathcal X}}(\pi_n({\mathcal O}_{\mathcal X})[n+1])$, which is the null derivation.
	Therefore the $n$-th Postnikov invariant of ${\mathcal O}{^\mathrm{alg}}$ is the null derivation as well, and therefore we see that ${\mathcal O}{^\mathrm{alg}}$ can be identified with the split square-zero extension of $\tau_{\le n - 1} {\mathcal O}$ by $\pi_n({\mathcal O})[n+1] \simeq \pi_n({\mathcal O}_{{\mathcal X}{^\mathrm{an}}}{^\mathrm{alg}})$.
	The conclusion follows.
\end{proof}
	
\begin{cor}
	Let $({\mathcal X}, {\mathcal O}_{\mathcal X})$ be a {Deligne-Mumford\xspace} stack locally of finite presentation over $\mathbb C$.
	Then the analytification functor ${\mathrm{Spec}^{{\mathcal T}_{\mathrm{an}}}_{{\mathcal T}_{\mathrm{\acute{e}t}}}}$ preserves the pushout:
	\[ \begin{tikzcd}
		({\mathcal X}, \Omega^\infty(\pi_n({\mathcal O}_{\mathcal X})[n+1])) \arrow{r} \arrow{d} & ({\mathcal X}, \tau_{\le n - 1} {\mathcal O}_{\mathcal X}) \arrow{d} \\
		({\mathcal X}, \tau_{\le n - 1} {\mathcal O}_{\mathcal X}) \arrow{r} & ({\mathcal X}, \tau_{\le n} {\mathcal O}_{\mathcal X})
	\end{tikzcd} \]
\end{cor}
	
\begin{proof}
	Let us write $({\mathcal X}{^\mathrm{an}}, {\mathcal O}) = {\mathrm{Spec}^{{\mathcal T}_{\mathrm{an}}}_{{\mathcal T}_{\mathrm{\acute{e}t}}}}({\mathcal X}, \Omega^\infty(\pi_n({\mathcal O}_{\mathcal X})[n+1])$.
	We only need to check that the commutative diagram
	\[ \begin{tikzcd}
		\tau_{\le n} {\mathcal O}_{{\mathcal X}{^\mathrm{an}}} \arrow{r} \arrow{d} & \tau_{\le n - 1} {\mathcal O}_{{\mathcal X}{^\mathrm{an}}} \arrow{d} \\
		\tau_{\le n - 1} {\mathcal O}_{{\mathcal X}{^\mathrm{an}}} \arrow{r} & {\mathcal O}
	\end{tikzcd} \]
	is a pullback.
	Since $(-){^\mathrm{alg}}$ preserves pullbacks and it is conservative, it is enough to check that
	\[ \begin{tikzcd}
		\tau_{\le n} {\mathcal O}_{{\mathcal X}{^\mathrm{an}}}{^\mathrm{alg}} \arrow{r} \arrow{d} & \tau_{\le n - 1} {\mathcal O}_{{\mathcal X}{^\mathrm{an}}}{^\mathrm{alg}} \arrow{d} \\
		\tau_{\le n - 1} {\mathcal O}_{{\mathcal X}{^\mathrm{an}}}{^\mathrm{alg}} \arrow{r} & {\mathcal O}{^\mathrm{alg}}
	\end{tikzcd} \]
	is a pullback.
	This follows from the previous corollary and the discussion before it.
\end{proof}

\begin{rem}
	One could ask whether a similar description holds \emph{before} passing to the underlying ${{\mathcal T}_{\mathrm{\acute{e}t}}}$-structured topos.
	The answer is affirmative, but the proof requires quite a bit of machinery that is beyond the scope of the present article.
	We will return on this point in \cite{Porta_Analytic_deformation_2015}.
\end{rem}

\section{GAGA for derived {Deligne-Mumford\xspace} stacks} \label{sec:derived_GAGA}

This section is devoted to the two GAGA theorems.
Let us first introduce the notion of proper morphism of derived {Deligne-Mumford\xspace} stacks.
As for the analogous notion for derived {$\mathbb C$-analytic\xspace} spaces, we will reduce via the truncation to the definition given in \cite{Porta_Yu_Higher_analytic_stacks_2014}.
Let us briefly recall the definitions given there:

\begin{defin}[{\cite[Definition 4.7]{Porta_Yu_Higher_analytic_stacks_2014}}]
	A morphism $f\colon X\to Y$ of algebraic {Deligne-Mumford\xspace} stacks is said to be \emph{weakly proper} if there exists an atlas $\{Y_i\}_{i\in I}$ of $Y$ such that for every $i\in I$,
	there exists a scheme $P_i$ proper over $Y_i$ and a proper surjective $Y_i$-morphism from $P_i$ to $X\times_Y Y_i$.
\end{defin}

\begin{defin}[{\cite[Definition 4.8]{Porta_Yu_Higher_analytic_stacks_2014}}]
	We define by induction on $n\geq 0$.
	\begin{enumerate}[(i)]
		\item An $n$-representable morphism of algebraic {Deligne-Mumford\xspace} stacks is said to be \emph{separated} if its diagonal being an $(n-1)$-representable morphism is proper.
		\item An $n$-representable morphism of algebraic stacks is said to be \emph{proper} if it is separated and weakly proper.
	\end{enumerate}
\end{defin}

We now introduce the following definition:

\begin{defin}
	Let $f \colon X \to Y$ be a morphism of derived {Deligne-Mumford\xspace} stacks.
	We will say that $f$ is \emph{separated} (resp.\ \emph{proper}) if ${\mathrm{t}_0}(f)$ is separated (resp.\ proper) in the sense introduced above.
\end{defin}

\begin{rem}[Algebraic proper direct image theorem]
	Using \cite[Theorem 5.11]{Porta_Yu_Higher_analytic_stacks_2014} as basis of the induction, the same proof given in \cref{prop:proper_direct_image_derived_DM_stacks} yields a version of the proper direct image theorem statement for proper morphisms of higher algebraic {Deligne-Mumford\xspace} stacks whose truncation is locally noetherian. This is probably a folklore result (as it was the aforementioned theorem), but we couldn't locate it in the literature.
\end{rem}

\subsection{GAGA 1}

Let $f \colon (X, {\mathcal O}_X) \to (Y, {\mathcal O}_Y)$ be a morphism of derived {Deligne-Mumford\xspace} stacks locally of finite presentations over $\mathbb C$.
We have the following commutative diagram in ${\mathcal T\mathrm{op}}({{\mathcal T}_{\mathrm{\acute{e}t}}})$:

\[ \begin{tikzcd}
(X{^\mathrm{an}}, {\mathcal O}_{X{^\mathrm{an}}}{^\mathrm{alg}}) \arrow{r}{h_X} \arrow{d}{f{^\mathrm{an}}} & (X, {\mathcal O}_X) \arrow{d}{f} \\
(Y{^\mathrm{an}}, {\mathcal O}_{Y{^\mathrm{an}}}{^\mathrm{alg}}) \arrow{r}{h_Y} & (Y, {\mathcal O}_Y)
\end{tikzcd} \]
which in turn induces a commutative diagram of stable $\infty$-categories
\[ \begin{tikzcd}
{\mathcal O}_X \textrm{-} {\mathrm{Mod}} \arrow{d}{{\mathrm R} f_*} & {\mathcal O}_{X{^\mathrm{an}}}{^\mathrm{alg}} \textrm{-} {\mathrm{Mod}} \arrow{d}{{\mathrm R} f{^\mathrm{an}}_*} \arrow{l}[swap]{{\mathrm R} h_{X*}} \\
{\mathcal O}_Y \textrm{-} {\mathrm{Mod}} & {\mathcal O}_{Y{^\mathrm{an}}}{^\mathrm{alg}} \textrm{-} {\mathrm{Mod}} \arrow{l}[swap]{{\mathrm R} h_{Y*}}
\end{tikzcd} \]
Given ${\mathcal F} \in {\mathcal O}_X \textrm{-} {\mathrm{Mod}}$, adjoint nonsense produces a canonical map
\[ \varphi_{\mathcal F} \colon ({\mathrm R} f_* {\mathcal F}){^\mathrm{an}} \to {\mathrm R} f{^\mathrm{an}}_*({\mathcal F}{^\mathrm{an}}) \]
where ${\mathcal F}{^\mathrm{an}} \coloneqq h_X^*({\mathcal F})$ (and we don't write ${\mathrm L} h_X^*({\mathcal F})$ thanks to the flatness result \cref{cor:analytification_flat}).
Similarly, we wrote $({\mathrm R} f_* {\mathcal F}){^\mathrm{an}}$ to denote $h_Y^*({\mathrm R} f_* {\mathcal F})$.

\begin{thm} \label{thm:derived_GAGA_1}
	Let $f \colon X \to Y$ be a proper morphism of derived {Deligne-Mumford\xspace} stacks locally of finite presentation over $\mathbb C$.
	For every ${\mathcal F} \in {\mathrm{Coh}}^+({\mathcal X})$, the canonical map
	\[ \varphi_{\mathcal F} \colon ({\mathrm R} f_* {\mathcal F}){^\mathrm{an}} \to {\mathrm R} f{^\mathrm{an}}_*({\mathcal F}{^\mathrm{an}}) \]
	is an equivalence.
\end{thm}

\begin{proof}
	Let ${\mathcal C}$ be the full subcategory of ${\mathcal O}_X \textrm{-} {\mathrm{Mod}}$ spanned by those ${\mathcal F}$ for which $\varphi_{\mathcal F}$ is an equivalence.
	We observe that:
	\begin{enumerate}
		\item ${\mathcal C}$ is stable under loops, suspensions and extensions: indeed, this follows immediately from the fact that all the functors ${\mathrm R} f_*$, ${\mathrm R} f{^\mathrm{an}}_*$, $h_X^*$ and $h_Y^*$ are exact functors between stable $\infty$-categories;
		\item ${\mathcal C}$ contains ${\mathrm{Coh}}^\heartsuit(X) \simeq {\mathrm{Coh}}^\heartsuit({\mathrm{t}_0}(X))$. Indeed, we have the commutative diagram
		\[ \begin{tikzcd}
		{\mathrm{t}_0}(X) \arrow{r}{f_0} \arrow{d}{i} & {\mathrm{t}_0}(Y) \arrow{d}{j} \\
		X \arrow{r}{f} & Y
		\end{tikzcd} \]
		If ${\mathcal F} \in {\mathrm{Coh}}^\heartsuit(X)$, we can write ${\mathcal F} \simeq {\mathrm R} i_*({\mathcal F}')$ with ${\mathcal F}' \in {\mathrm{Coh}}^\heartsuit({\mathrm{t}_0}(X))$ and therefore ${\mathrm R} f_*({\mathcal F}) \simeq {\mathrm R} j_*({\mathrm R} f_{0*} {\mathcal F})$.
		The GAGA theorem of \cite[Theorem 7.3]{Porta_Yu_Higher_analytic_stacks_2014} combined with \cref{prop:comparison_analytification_DM_stacks} and \cref{prop:comparison_coherent_sheaves} shows that the canonical map
		\[ ({\mathrm R} f_{0*} ({\mathcal F}')){^\mathrm{an}} \to {\mathrm R} f{^\mathrm{an}}_{0*} (({\mathcal F}'){^\mathrm{an}}) \]
		is an equivalence.
		We are therefore left to prove that the theorem holds for the inclusion $j \colon {\mathrm{t}_0}(Y) \to Y$. This follows combining \cref{cor:analytification_flat} and \cref{prop:relative_spectrum_truncated_objects}.
	\end{enumerate}
	At this point, the d\'evissage lemma \cite[Lemma 5.10]{Porta_Yu_Higher_analytic_stacks_2014} shows that ${\mathrm{Coh}}^b({\mathcal X}) \subset {\mathcal C}$.
	To complete the proof, we only need to show that whenever $\tau_{\le n} {\mathcal F} \in {\mathrm{Coh}}^+(X) \cap {\mathcal C}$ for every $n$, then ${\mathcal F} \in {\mathcal C}$.
	
	Form a fiber sequence
	\[ \tau_{\le n} {\mathcal F} \to {\mathcal F} \to \tau_{> n} {\mathcal F} \]
	Since ${\mathrm R} f_*$, ${\mathrm R} f{^\mathrm{an}}_*$ and $(-){^\mathrm{an}}$ are exact functors between stable $\infty$-categories, we obtain a morphism of fiber sequences
	\[ \begin{tikzcd}
		({\mathrm R} f_*(\tau_{\le n} {\mathcal F})){^\mathrm{an}} \arrow{r} \arrow{d} & ({\mathrm R} f_* {\mathcal F}){^\mathrm{an}} \arrow{r} \arrow{d} & ({\mathrm R} f_* (\tau_{> n} {\mathcal F})){^\mathrm{an}} \arrow{d} \\
		{\mathrm R} f{^\mathrm{an}}_* (\tau_{\le n} {\mathcal F}){^\mathrm{an}} \arrow{r} & {\mathrm R} f{^\mathrm{an}}_* {\mathcal F}{^\mathrm{an}} \arrow{r} & {\mathrm R} f{^\mathrm{an}}_* (\tau_{> n} {\mathcal F}){^\mathrm{an}}
	\end{tikzcd} \]
	\Cref{cor:analytification_flat} shows that $(\tau_{>n} {\mathcal F}){^\mathrm{an}} \simeq \tau_{> n}({\mathcal F} {^\mathrm{an}})$ and therefore both ${\mathrm H}^i({\mathrm R} f_* (\tau_{>n} {\mathcal F}){^\mathrm{an}})$ and ${\mathrm H}^i({\mathrm R} f{^\mathrm{an}}_*(\tau_{>n} {\mathcal F}){^\mathrm{an}})$ vanish for $i \le n$.
	Since $\tau_{\le n} {\mathcal F} \in {\mathrm{Coh}^{\mathrm{b}}}(X)$ by hypothesis, we see the morphism $({\mathrm R} f_* {\mathcal F}){^\mathrm{an}} \to {\mathrm R} f{^\mathrm{an}}_* {\mathcal F}{^\mathrm{an}}$ induces an isomorphism on the cohomology groups ${\mathrm H}^i$ for every $i \le n$.
	Letting $n$ vary, we conclude that $\varphi_{\mathcal F}$ is an equivalence, thus completing the proof.
\end{proof}

\subsection{GAGA 2}

\begin{lem} \label{lem:analytification_inverse_limits}
	Let $X$ be a derived {Deligne-Mumford\xspace} stack locally of finite presentation over $\mathbb C$.
	Let ${\mathcal F} \in {\mathcal O}_{\mathcal X} \textrm{-} {\mathrm{Mod}}$ and suppose it can be written as
	\[ {\mathcal F} \simeq \lim \tau_{\ge -n} {\mathcal F} \]
	Then the analytification $(-){^\mathrm{an}}$ commutes with this limit.
\end{lem}

\begin{proof}
	It follows from \cref{cor:analytification_flat} that $(\tau_{\ge n} {\mathcal F}){^\mathrm{an}} \simeq \tau_{\ge n} {\mathcal F}{^\mathrm{an}}$.
	Therefore we have
	\[ {\mathcal F}{^\mathrm{an}} \simeq \lim \tau_{\ge -n} {\mathcal F}{^\mathrm{an}} \simeq \lim (\tau_{\ge -n} {\mathcal F}){^\mathrm{an}} \]
	completing the proof.
\end{proof}

\begin{thm} \label{thm:derived_GAGA_2}
	Let $X = ({\mathcal X}, {\mathcal O}_X)$ be a derived {Deligne-Mumford\xspace} stack proper over $\mathbb C$.
	The analytification functor
	\[ (-){^\mathrm{an}} \colon {\mathcal O}_{\mathcal X} \textrm{-} {\mathrm{Mod}} \longrightarrow {\mathcal O}_{{\mathcal X}{^\mathrm{an}}}{^\mathrm{alg}} \textrm{-} {\mathrm{Mod}} \]
	restricts to an equivalence
	\[ (-){^\mathrm{an}} \colon {\mathrm{Coh}}^-(X) \longrightarrow {\mathrm{Coh}}^-(X{^\mathrm{an}}) \]
\end{thm}

\begin{proof}
	Let us start by proving fully faithfulness.
	Since ${\mathcal O}_{\mathcal X} \textrm{-} {\mathrm{Mod}}$ and ${\mathcal O}_{{\mathcal X}{^\mathrm{an}}}{^\mathrm{alg}} \textrm{-} {\mathrm{Mod}}$ are stable and $\mathbb C$-linear, they are canonically enriched over ${{{\mathcal D}({\mathrm{Ab}})}}$, see \cite[Examples 7.4.14, 7.4.15]{Gepner_Enriched_2013}.
	We will denote by $\operatorname{Map}^{{{\mathcal D}({\mathrm{Ab}})}}_{{\mathcal O}_{\mathcal X}}$ and $\operatorname{Map}{{{\mathcal D}({\mathrm{Ab}})}}_{{\mathcal O}_{{\mathcal X}{^\mathrm{an}}}{^\mathrm{alg}}}$ the enriched mapping spaces of these two categories, respectively.
	For every ${\mathcal F}, {\mathcal G} \in {\mathcal O}_{\mathcal X} \textrm{-} {\mathrm{Mod}}$ there is a natural map
	\[ \psi_{{\mathcal F}, {\mathcal G}} \colon \operatorname{Map}^{{{\mathcal D}({\mathrm{Ab}})}}_{{\mathcal O}_{\mathcal X}}({\mathcal F}, {\mathcal G}) \to \operatorname{Map}^{{{\mathcal D}({\mathrm{Ab}})}}_{{\mathcal O}_{{\mathcal X}{^\mathrm{an}}}{^\mathrm{alg}}}({\mathcal F}{^\mathrm{an}}, {\mathcal G}{^\mathrm{an}}) \]
	and we want to prove that $\psi_{{\mathcal F}, {\mathcal G}}$ is an equivalence.
	Observe that when ${\mathcal F}, {\mathcal G} \in {\mathrm{Coh}}^\heartsuit(X) \simeq {\mathrm{Coh}}^\heartsuit({\mathrm{t}_0}(X))$ the statement follows from the analogous result for higher {Deligne-Mumford\xspace} stacks proved in \cite[Proposition 7.2]{Porta_Yu_Higher_analytic_stacks_2014}. The same extension argument given in loc.\ cit.\ shows that $\psi_{{\mathcal F}, {\mathcal G}}$ is an equivalence whenever ${\mathcal F}, {\mathcal G} \in {\mathrm{Coh}^{\mathrm{b}}}(X)$.
	
	Let us now turn to the general case. Since ${{{\mathcal D}({\mathrm{Ab}})}}$ is both left and right $t$-complete, it is enough to prove that for every integer $n \in \mathbb Z$, $\pi_{-n} \psi_{{\mathcal F}, {\mathcal G}}$ is an isomorphism of abelian groups.
	Recalling that
	\begin{gather*}
		\pi_{-n} \operatorname{Map}^{{{\mathcal D}({\mathrm{Ab}})}}_{{\mathcal O}_{\mathcal X}}({\mathcal F}, {\mathcal G}) = \pi_0 \operatorname{Map}^{{{\mathcal D}({\mathrm{Ab}})}}_{{\mathcal O}_{\mathcal X}}({\mathcal F}, {\mathcal G}[n]) \\
		\pi_{-n} \operatorname{Map}^{{{\mathcal D}({\mathrm{Ab}})}}_{{\mathcal O}_{{\mathcal X}{^\mathrm{an}}}{^\mathrm{alg}}}({\mathcal F}{^\mathrm{an}}, {\mathcal G}{^\mathrm{an}}) = \pi_0 \operatorname{Map}^{{{\mathcal D}({\mathrm{Ab}})}}_{{\mathcal O}_{{\mathcal X}{^\mathrm{an}}}{^\mathrm{alg}}}({\mathcal F}{^\mathrm{an}}, {\mathcal G}{^\mathrm{an}}[n])
	\end{gather*}
	we see that it is enough to treat the case $n = 0$.
	Observe that $\pi_0 \operatorname{Map}^{{{\mathcal D}({\mathrm{Ab}})}}_{{\mathcal O}_{\mathcal X}}({\mathcal F}, {\mathcal G})$ can be identified with the global sections of the cohomology sheaf ${\mathcal H}^0( {{\mathrm R}\!\mathcal H\!\mathit{om}}_{{\mathcal O}_{\mathcal X}}({\mathcal F}, {\mathcal G}))$.
	Since ${\mathcal F}$ and ${\mathcal G}$ are hypercomplete objects (see \cite[Proposition 2.3.21]{DAG-VIII}), we can write
	\[ {\mathcal F} \simeq \operatorname*{colim}_n \tau_{\le n} {\mathcal F}, \qquad {\mathcal G} \simeq \lim_m \tau_{\ge m} {\mathcal G} \]
	where the limits and colimits are computed in ${\mathcal O}_X \textrm{-} {\mathrm{Mod}}$.
	Using the fact that $(-){^\mathrm{an}}$ commutes with all colimits and invoking \cref{lem:analytification_inverse_limits}, we conclude that
	\[ {\mathcal F}{^\mathrm{an}} \simeq \operatorname*{colim}_n (\tau_{\le n} {\mathcal F}){^\mathrm{an}}, \qquad {\mathcal G}{^\mathrm{an}} \simeq \lim_m (\tau_{\ge m} {\mathcal G}){^\mathrm{an}} \]
	Moreover, \cref{cor:analytification_flat} shows that $(\tau_{\le n} {\mathcal F}){^\mathrm{an}} \simeq \tau_{\le n} {\mathcal F}{^\mathrm{an}}$ and $(\tau_{\ge m} {\mathcal G}){^\mathrm{an}} \simeq \tau_{\ge m} {\mathcal G}{^\mathrm{an}}$.
	We are therefore reduced to the case where ${\mathcal F} \in {\mathrm{Coh}}^-(X)$ and ${\mathcal G} \in {\mathrm{Coh}}^+(X)$.
	Suppose more precisely that ${\mathcal H}^i({\mathcal F}) = 0$ for $i \ge n_0$ and $cH^j({\mathcal G}) = 0$ for $j \le m_0$.
	Then the same bounds hold for ${\mathcal F}{^\mathrm{an}}$ and ${\mathcal G}{^\mathrm{an}}$, so that we obtain:
	\begin{gather*}
		\pi_0 \operatorname{Map}^{{{\mathcal D}({\mathrm{Ab}})}}_{{\mathcal O}_{\mathcal X}}({\mathcal F}, {\mathcal G}) \simeq \pi_0 \operatorname{Map}^{{{\mathcal D}({\mathrm{Ab}})}}_{{\mathcal O}_{\mathcal X}}(\tau_{\ge m_0 - 1} {\mathcal F}, \tau_{\le n_0 + 1} {\mathcal G}) \\
		\pi_0 \operatorname{Map}^{{{\mathcal D}({\mathrm{Ab}})}}_{{\mathcal O}_{{\mathcal X}{^\mathrm{an}}}{^\mathrm{alg}}}({\mathcal F}{^\mathrm{an}}, {\mathcal G}{^\mathrm{an}}) \simeq \pi_0 \operatorname{Map}^{{{\mathcal D}({\mathrm{Ab}})}}_{{\mathcal O}_{\mathcal X}}(\tau_{\ge m_0 - 1} {\mathcal F}{^\mathrm{an}}, \tau_{\le n_0 + 1} {\mathcal G}{^\mathrm{an}})
	\end{gather*}
	Since both $\tau_{\ge m_0 - 1} {\mathcal F}$ and $\tau_{\le n_0 + 1} {\mathcal G}$ belong to ${\mathrm{Coh}^{\mathrm{b}}}(X)$, we already know that the canonical map
	\[ \operatorname{Map}^{{{\mathcal D}({\mathrm{Ab}})}}_{{\mathcal O}_{\mathcal X}}(\tau_{\ge m_0 - 1} {\mathcal F}, \tau_{\le n_0 + 1} {\mathcal G}) \to \operatorname{Map}^{{{\mathcal D}({\mathrm{Ab}})}}_{{\mathcal O}_{\mathcal X}}(\tau_{\ge m_0 - 1} {\mathcal F}{^\mathrm{an}}, \tau_{\le n_0 + 1} {\mathcal G}{^\mathrm{an}}) \]
	is an equivalence.
	In conclusion, $\psi_{{\mathcal F}, {\mathcal G}}$ is an equivalence for every ${\mathcal F}, {\mathcal G} \in {\mathrm{Coh}}(X)$, completing the first part of the proof.

	Let us now turn to the essential surjectivity part. Let ${\mathcal C}$ be the full subcategory of ${\mathrm{Coh}^{\mathrm{b}}}(X{^\mathrm{an}})$ spanned by the essential image of the analytification functor $(-){^\mathrm{an}}$. Since ${\mathrm{Coh}}^\heartsuit(X{^\mathrm{an}}) \simeq {\mathrm{Coh}}^\heartsuit({\mathrm{t}_0}(X{^\mathrm{an}}))$, \cref{prop:comparison_analytification_DM_stacks} and \cref{prop:comparison_coherent_sheaves} can be combined together with the GAGA theorem of \cite[Theorem 7.3]{Porta_Yu_Higher_analytic_stacks_2014} to conclude that ${\mathcal C}$ contains ${\mathrm{Coh}}^\heartsuit(X{^\mathrm{an}})$.
	Observe that ${\mathcal C}$ is clearly stable under loop and suspensions in ${\mathcal O}_{\mathcal X} \textrm{-} {\mathrm{Mod}}$.
	We claim that ${\mathcal C}$ is a thick subcategory of ${\mathcal O}_{\mathcal X} \textrm{-} {\mathrm{Mod}}$.
	Indeed, if we are given a fiber sequence
	\[ {\mathcal F}' \to {\mathcal F} \to {\mathcal F}'' \]
	with ${\mathcal F}', {\mathcal F}'' \in {\mathcal C}$, we can rotate it and express ${\mathcal F}$ as the fiber of ${\mathcal F}'' \to {\mathcal F}'[1]$.
	Let ${\mathcal G}', {\mathcal G}'' \in {\mathrm{Coh}^{\mathrm{b}}}(X)$ be such that $({\mathcal G}'){^\mathrm{an}} \simeq {\mathcal F}'$ and $({\mathcal G}''){^\mathrm{an}} \simeq {\mathcal F}''$.
	By the fully faithfulness we already proved, the map ${\mathcal F}'' \to {\mathcal F}'[1]$ is the analytification of a map ${\mathcal G}'' \to {\mathcal G}'[1]$.
	Let ${\mathcal G}$ be the fiber of this map.
	Since $(-){^\mathrm{an}}$ is an exact functor between stable $\infty$-categories, we see that ${\mathcal G}{^\mathrm{an}} \simeq {\mathcal F}$, thus completing the proof of the claim.
	Therefore, the hypotheses of the d\'evissage lemma \cite[Lemma 5.10]{Porta_Yu_Higher_analytic_stacks_2014} are satisfied and therefore we conclude that ${\mathcal C}$ contains the whole ${\mathrm{Coh}^{\mathrm{b}}}(X{^\mathrm{an}})$.
	If now ${\mathcal F} \in {\mathrm{Coh}}^-(X{^\mathrm{an}})$, we can write
	\[ {\mathcal F} \simeq \lim \tau_{\ge n} {\mathcal F} \]
	Since the analytification functor $(-){^\mathrm{an}}$ is fully faithful and essentially surjective on ${\mathrm{Coh}^{\mathrm{b}}}(X{^\mathrm{an}})$, we see that the diagram $\{\tau_{\ge n} {\mathcal F}\}$ is the analytification of a tower $\{{\mathcal G}_n\}$ in ${\mathrm{Coh}^{\mathrm{b}}}(X)$.
	Consider the morphism ${\mathcal G}_n \to \tau_{\ge n} {\mathcal G}_n$.
	\Cref{cor:analytification_flat} shows that it becomes an equivalence after passing to the analytification.
	Since $(-){^\mathrm{an}}$ is conservative, we conclude that ${\mathcal G}_n \in {\mathrm{Coh}}^{\ge n}(X) \cap {\mathrm{Coh}^{\mathrm{b}}}(X)$.
	Therefore there are canonical maps $\tau_{\ge n} {\mathcal G}_{n-1} \to {\mathcal G}_n$, which become equivalences after applying $(-){^\mathrm{an}}$.
	These remarks show that ${\mathcal G} \coloneqq \lim {\mathcal G}_n \in {\mathrm{Coh}}^-(X)$.
	At this point, we can apply \cref{lem:analytification_inverse_limits} to get
	\[ {\mathcal G}{^\mathrm{an}} \simeq \lim {\mathcal G}_n{^\mathrm{an}} \simeq \lim \tau_{\ge n} {\mathcal F} \simeq {\mathcal F} . \]
	At last, this proves that $(-){^\mathrm{an}}$ is essentially surjective also on ${\mathrm{Coh}}^-(X{^\mathrm{an}})$, completing the proof.
	Since we already know that $(-){^\mathrm{an}} \colon {\mathrm{Coh}}(X) \to {\mathrm{Coh}}(X{^\mathrm{an}})$ is fully faithful, we conclude now that $(-){^\mathrm{an}} \colon {\mathrm{Coh}}^-(X) \to {\mathrm{Coh}}^-(X{^\mathrm{an}})$ is an equivalence.

	Finally, let ${\mathcal F} \in {\mathrm{Coh}}(X{^\mathrm{an}})$. Repeating the same reasoning as before but using the truncations $\tau_{\le n} {\mathcal F}$ and using the fact that $(-){^\mathrm{an}}$ commutes with limits, we finally conclude that there exists ${\mathcal G} \in {\mathrm{Coh}}(X)$ such that ${\mathcal G}{^\mathrm{an}} \simeq {\mathcal F}$.
	The proof is therefore complete.
\end{proof}

\section{Extension to Artin stacks} \label{sec:extension_Artin}

In this section we outline how it is possible to extend all the results we obtained so far to the setting of derived Artin stacks.

We begin with a definition of derived Artin analytic stacks.

\begin{defin}
	A morphism $f \colon X \to Y$ in ${\mathrm{Stn}^{\mathrm{der}}_{\mathbb C}}$ is said to be \emph{smooth} if it is strong and its truncation ${\mathrm{t}_0}(f)$ is smooth.
\end{defin}

Let $\mathbf P_{\mathrm{sm}}$ be the collection of smooth morphisms in ${\mathrm{Stn}^{\mathrm{der}}_{\mathbb C}}$.
Then the triple $({\mathrm{Stn}^{\mathrm{der}}_{\mathbb C}}, \tau, \mathbf P_{\mathrm{sm}})$ is a geometric context in the sense of \cite{Porta_Yu_Higher_analytic_stacks_2014}.
We therefore give the following definition:

\begin{defin}
	A \emph{derived Artin analytic stack} is a geometric stack for the context $({\mathrm{Stn}^{\mathrm{der}}_{\mathbb C}}, \tau, \mathbf P_{\mathrm{sm}})$.
\end{defin}

With this definition, the proof of the comparison results \cref{cor:comparison_analytic_DM_stacks} and \cref{prop:comparison_coherent_sheaves} remain unchanged. The analytification functor for derived Artin analytic stacks locally of finite presentation over $\mathbb C$ is obtained as left Kan extension of the analytification functor
\[ {\mathrm{Spec}^{{\mathcal T}_{\mathrm{an}}}_{{\mathcal T}_{\mathrm{\acute{e}t}}}} \colon \mathrm{dAff}^{\mathrm{f.p.}}_{\mathbb C} \to {\mathrm{Stn}^{\mathrm{der}}_{\mathbb C}} \]
exactly as it is done in \cite{Porta_Yu_Higher_analytic_stacks_2014, Toen_Algebrisation_2008}.
Therefore the comparison result \cref{prop:comparison_analytification_DM_stacks} also extends to the setting of derived Artin stacks.
We now observe that the proofs of our main theorems \ref{prop:proper_direct_image_derived_DM_stacks}, \ref{thm:derived_GAGA_1} and \ref{thm:derived_GAGA_2} all rely on the analogous results of \cite{Porta_Yu_Higher_analytic_stacks_2014} by d\'evissage to the heart using the comparison results we cited above.
Therefore the same technique applies to the setting of derived Artin stacks.

\appendix

\section{Flat morphisms}

We collect in this section a couple of basic facts about flat morphisms of $\mathbb E_\infty$-rings we couldn't find a reference for.

\begin{lem}
	Let $f \colon A \to B$ be a flat morphism of connective $\mathbb E_\infty$-rings.
	Then $\tau_{\le n}(A) \otimes_A B \simeq \tau_{\le n}(B)$.
\end{lem}

\begin{proof}
	It follows from \cite[7.2.2.13]{Lurie_Higher_algebra} that for every $A$-module $M$ one has
	\[ \pi_i(M) \otimes_{\pi_0 A} \pi_0 B \simeq \pi_i(M \otimes_A B) \]
	On the other side
	\[ \pi_i(M) \otimes_A B = \pi_i(M) \otimes_{\pi_0 A} \pi_0 A \otimes_A B \]
	Since $A \to B$ is flat, we see that $\pi_0 A \otimes_A B$ is discrete and therefore there exists a map
	\[ \pi_0 B \to \pi_0 A \otimes_A B \]
	and the spectral sequence of \cite[7.2.1.19]{Lurie_Higher_algebra} shows that this is indeed an isomorphism.
	In conclusion, we see that
	\[ \pi_i(M) \otimes_A B \simeq \pi_i(M \otimes_A B) \]
	Now, since $A \to B$ is flat, we see that $\tau_{\le n} A \otimes_A B$ is $n$-truncated, and therefore there exists a morphism
	\[ \tau_{\le n} B \to \tau_{\le n} A \otimes_A B \]
	The above argument shows that this morphism induces isomorphisms on the $\pi_i$ for every $i$ and therefore it is an equivalence.
\end{proof}

For the terminology used in the following lemma we refer back to the discussion at the beginning of \cref{subsec:computing_analytification}.

\begin{lem}
	Let $f \colon A \to B$ be a morphism of $\mathbb E_\infty$-rings.
	Let $M$ be an $A$-module.
	Then $\Omega^\infty_A(M) \otimes_A B \simeq \Omega^\infty_B(f^*(M))$.
\end{lem}

\begin{proof}
	The functor $- \otimes_A B \colon \mathrm{CAlg}_{/A} \to \mathrm{CAlg}_{/B}$ commutes with finite limits (because finite limits are computed in the category $\mathrm{Sp}_{/A}$ and $\mathrm{Sp}_{/B}$ respectively and $- \otimes_A B$ is a functor between stable $\infty$-categories).
	Therefore, the conclusion follows.
\end{proof}

\begin{lem} \label{lem:flat_base_change_Postnikov_invariant}
	Let $f \colon A \to B$ be a flat morphism of connective $\mathbb E_\infty$-rings.
	Let $d \colon \tau_{\le n - 1} A \to \Omega^\infty_A(\pi_n(A)[n+1])$ be the $n$-th Postnikov invariant of $A$.
	Then $d \otimes_{\tau_{\le n - 1} A} \tau_{\le n - 1} B \colon \tau_{\le n - 1} B \to \Omega^\infty_B(\pi_n(B)[n+1])$ is the $n$-th Postnikov invariant of $B$.
\end{lem}

\begin{proof}
	The $n$-th Postnikov invariant of $A$ is characterized by the fact that it makes the diagram
	\[ \begin{tikzcd}
	\tau_{\le n} A \arrow{r} \arrow{d} & \tau_{\le n-1} A \arrow{d} \\
	\tau_{\le n - 1} A \arrow{r} & \Omega^\infty_A(\pi_n(A)[n+1])
	\end{tikzcd} \]
	into a pullback diagram in the category of connective $\mathbb E_\infty$-rings.
	In particular, it is a pullback diagram in the category of $\tau_{\le n} A$-modules.
	Therefore the functor $- \otimes_{\tau_{\le n} A} \tau_{\le n} B$ preserves this pullback.
	Since $\tau_{\le n - 1} A \otimes_{\tau_{\le n} A} \tau_{\le n} B \simeq \tau_{\le n - 1} B$ and $\Omega^\infty_A(\pi_n(A)[n+1]) \otimes_{\tau_{\le n} A} \tau_{\le n} B \simeq \Omega^\infty_B(\pi_n(B)[n+1])$, we conclude that the diagram
	\[ \begin{tikzcd}
	\tau_{\le n} B \arrow{r} \arrow{d} & \tau_{\le n - 1} B \arrow{d}{d \otimes_{\tau_{\le n} A} \tau_{\le n} B} \\
	\tau_{\le n - 1} B \arrow{r} & \Omega^\infty_B(\pi_n B[n+1])
	\end{tikzcd} \]
	is a pullback square in the category of $B$-modules and, a posteriori, in the category of $B$-algebras.
	We conclude that $d \otimes_{\tau_{\le n} A} \tau_{\le n} B$ is the $n$-th Postnikov invariant of $B$.
\end{proof}

\ifpersonal

\section{Comparison between HAG-II and DAG-V}

We want to prove the following theorem:

\begin{thm}
	There exists an equivalence of $\infty$-categories
	\[ \Psi \colon \mathbf{DM} \simeq \mathrm{Sch}^{\mathrm{loc}}({{\mathcal G}_{\mathrm{\acute{e}t}}^\mathrm{der}(k)}) \colon \Phi \]
	where $\mathrm{Sch}^{\mathrm{loc}}({{\mathcal G}_{\mathrm{\acute{e}t}}^\mathrm{der}(k)})$ denotes the full $\infty$-subcategory of $\mathrm{Sch}({{\mathcal G}_{\mathrm{\acute{e}t}}^\mathrm{der}(k)})$ spanned by those schemes $({\mathcal X}, {\mathcal O}_{\mathcal X})$ such that ${\mathcal X}$ is $n$-localic for some $n$.
	Moreover, if $n \ge 1$, an object $X \in \mathbf{DM}$ is $n$-geometric if and only if the underlying $\infty$-topos of $\Phi(X)$ is $(n+1)$-localic.
\end{thm}

We begin with the construction of the two functors.
First of all, let us remark that the category ${\mathcal C} = \mathrm{sCRing}$ can be identified with $\mathrm{Ind}(({{\mathcal G}_{\mathrm{\acute{e}t}}^\mathrm{der}(k)})^{\mathrm{op}}) \simeq \mathrm{Pro}({{\mathcal G}_{\mathrm{\acute{e}t}}^\mathrm{der}(k)})^{\mathrm{op}}$.
\cite[Theorem 2.4.1]{DAG-V} provides us with a fully faithful embedding
\[ \phi \colon \mathrm{Sch}({{\mathcal G}_{\mathrm{\acute{e}t}}^\mathrm{der}(k)}) \to \operatorname{Fun}(\mathrm{Ind}(({{\mathcal G}_{\mathrm{\acute{e}t}}^\mathrm{der}(k)})^{\mathrm{op}}), {\mathcal S}) \]
It follows from \cite[Lemma 2.4.13]{DAG-V} that this functor factors through ${\mathrm{Sh}}({\mathcal C}^{\mathrm{op}}, {\tau_\mathrm{q\acute{e}t}})$.

Therefore, we are left to show that the restriction of $\phi$ to $\mathrm{Sch}^{\mathrm{loc}}({{\mathcal G}_{\mathrm{\acute{e}t}}^\mathrm{der}(k)})$ factors through $\mathbf{DM}$, and that it is essentially surjective.
Let $X = ({\mathcal X}, {\mathcal O}_{\mathcal X}) \in \mathrm{Sch}({{\mathcal G}_{\mathrm{\acute{e}t}}^\mathrm{der}(k)})$.
Unraveling the definition of $\phi$ given in \cite[Theorem 2.4.1]{DAG-V}, we see that $\phi(X)$ is the functor of points
\[ \phi(X) \colon {\mathcal C} \to {\mathcal S} \]
defined by
\[ \phi(X)(A) \coloneqq \operatorname{Map}_{\mathrm{Sch}({{\mathcal T}_{\mathrm{\acute{e}t}}(k)})}(\operatorname{Spec}^{\mathrm{\acute{e}t}}(A), X) \]
We will have to prove the following facts:
\begin{enumerate}
	\item if the underlying $\infty$-topos of $X$ is $n$-localic for some $n$, then $\phi(X)$ is hypercomplete;
	\item if the underlying $\infty$-topos of $X$ is $n$-localic for some $n$, then $\phi(X)$ is geometric;
	\item every object in $\mathbf{DM}$ arises is of the form $\phi(X)$ for $X \in \mathrm{Sch}^{\mathrm{loc}}({{\mathcal G}_{\mathrm{\acute{e}t}}^\mathrm{der}(k)})$.
\end{enumerate}

\subsection{$\phi(X)$ is geometric}

Let us first prove that this functor $\phi$ factors through $\mathbf{DM}$.
Let $X = ({\mathcal X}, {\mathcal O}_X)$ be an object in $\mathrm{Sch}({{\mathcal G}_{\mathrm{\acute{e}t}}^\mathrm{der}(k)})$.

\begin{lem}
	Let $X = ({\mathcal X}, {\mathcal O}_{\mathcal X})$ be a ${{\mathcal G}_{\mathrm{\acute{e}t}}^\mathrm{der}(k)}$-scheme and suppose that ${\mathcal X}$ is $n$-localic, with $n \ge 1$.
	Then the functor $\phi(X) \colon {\mathcal C} \to {\mathcal S}$ is an hypercomplete sheaf.
\end{lem}

\begin{proof}
	Let $U^\bullet \to U$ be an \'etale hypercover in the category $\mathrm{dAff}$.
	Let ${\mathcal T\mathrm{op}}^{\le n}({{\mathcal G}_{\mathrm{\acute{e}t}}^\mathrm{der}(k)})$ be the $\infty$-category of ${{\mathcal G}_{\mathrm{\acute{e}t}}^\mathrm{der}(k)}$-structured $\infty$-topoi which are $m$-localic for some $m \le n$.
	We claim that the geometric realization of the simplicial object $\operatorname{Spec}^{\mathrm{\acute{e}t}}(U^\bullet)$ is ${\mathcal T\mathrm{op}}^{\le n}({{\mathcal G}_{\mathrm{\acute{e}t}}^\mathrm{der}(k)})$ is precisely $\operatorname{Spec}^{\mathrm{\acute{e}t}}(U)$.
	The claim implies directly the lemma, since
	\begin{align*}
	\phi(X)(\operatorname{Spec}^{\mathrm{\acute{e}t}}(U)) & = \operatorname{Map}_{\mathrm{Sch}({{\mathcal T}_{\mathrm{\acute{e}t}}(k)})}(\operatorname{Spec}^{\mathrm{\acute{e}t}}(U), X) \\
	& = \operatorname{Map}_{{\mathcal T\mathrm{op}}^{\le n}({{\mathcal T}_{\mathrm{\acute{e}t}}(k)})}(\operatorname{Spec}^{\mathrm{\acute{e}t}}(U), X) \\
	& = \lim \operatorname{Map}_{{\mathcal T\mathrm{op}}^{\le n}({{\mathcal T}_{\mathrm{\acute{e}t}}(k)})}(\operatorname{Spec}^{\mathrm{\acute{e}t}}(U^\bullet), X) \\
	& = \lim \phi(X)(\operatorname{Spec}^{\mathrm{\acute{e}t}}(U^\bullet))
	\end{align*}
	
	We are therefore reduced to prove the claim. Let us denote by ${\mathcal X}_U$ the topos of (non hypercomplete) sheaves on the small \'etale site of $U$.
	It follows from \cite[Example 2.3.8]{DAG-V} that each face map
	\[ \operatorname{Spec}^{\mathrm{\acute{e}t}}(U^n) \to \operatorname{Spec}^{\mathrm{\acute{e}t}}(U^{n-1}) \]
	is \'etale.
	Therefore, we can find objects $V^n \in {\mathcal X}_U$ and identifications ${\mathcal X}_{U^n} \simeq ({\mathcal X}_U)_{/V^n}$.
	The universal property of \'etale subtopoi (see \cite[6.3.5.6]{HTT}), shows that we can arrange the $V^n$ into a simplicial object in ${\mathcal X}_U$.
	Now, we know from the previous discussion that the sheaf ${\mathcal O}_U$ is hypercomplete. Therefore, the canonical coaugmentation ${\mathcal O}_U \to {\mathcal O}_U |_{V^\bullet}$ induces an equivalence
	\[ {\mathcal O}_U \simeq \varprojlim {\mathcal O}_U |_{V^\bullet} \]
	In order to complete the proof of our claim, we only have to prove that
	\[ {\mathcal X}_U \simeq \operatorname*{colim} {\mathcal X}_{U^\bullet} \]
	where the colimit is computed in the category ${\mathcal T\mathrm{op}}^{\le n}$ of $m$-localic $\infty$-topoi for $m \le n$.
	Observe that ${\mathcal T\mathrm{op}}^{\le n}$ is (equivalent to) an $(n+1,1)$-category: indeed,
	\[ \operatorname{Map}_{{\mathcal T\mathrm{op}}^{\le n}}({\mathcal X}, {\mathcal Y}) \simeq \operatorname{Map}_{{\mathcal T\mathrm{op}}_n}(\tau_{\le n-1} {\mathcal X}, \tau_{\le n - 1} {\mathcal Y}) \to \operatorname{Fun}(\tau_{\le n - 1} {\mathcal X}, \tau_{\le n - 1} {\mathcal Y}) \]
	Now, $\tau_{\le n - 1} {\mathcal Y}$ is an $n$-category, and therefore $\operatorname{Fun}(\tau_{\le n - 1} {\mathcal X}, \tau_{\le n - 1} {\mathcal Y})$ is an $n$-category as well (see \cite[2.3.4.8]{HTT}). We conclude that the maximal Kan complex contained in $\operatorname{Fun}(\tau_{\le n - 1} {\mathcal X}, \tau_{\le n - 1} {\mathcal Y})$ is $n$-truncated.
	Since the map
	\[ \operatorname{Map}_{{\mathcal T\mathrm{op}}_n}(\tau_{\le n-1} {\mathcal X}, \tau_{\le n - 1} {\mathcal Y}) \to \operatorname{Fun}(\tau_{\le n - 1} {\mathcal X}, \tau_{\le n - 1} {\mathcal Y}) \]
	is a monomorphism of simplicial sets, we see that the Kan complex
	\[ \operatorname{Map}_{{\mathcal T\mathrm{op}}_n}(\tau_{\le n-1} {\mathcal X}, \tau_{\le n - 1} {\mathcal Y}) \]
	is in fact an $n$-category.
	It follows from \cite[2.3.4.19]{HTT} that it is $n$-truncated as well. In other words, ${\mathcal T\mathrm{op}}^{\le n}$ is an $(n+1, 1)$-category.
	{\ignorespaces}
	
	Therefore functors with values in ${\mathcal T\mathrm{op}}^{\le n}$ satisfy hyperdescent if and only if they satisfy descent (see \cref{prop:descent_vs_hyperdescent}).
	The result now follows.
\end{proof}

\begin{lem} \label{lem:topos_theoretic_etale}
	Let $f \colon B \to A$ be a morphism in $\mathrm{sCRing}$ between finitely presented objects.
	The following conditions are equivalent:
	\begin{enumerate}
		\item $f$ is \'etale;
		\item the morphism $\operatorname{Spec}^{\mathrm{\acute{e}t}}(A) \to \operatorname{Spec}^{\mathrm{\acute{e}t}}(B)$ is \'etale.
	\end{enumerate}
\end{lem}

\begin{proof}
	{\ignorespaces}
	The implication $1. \Rightarrow 2.$ is \cite[Example 2.3.8]{DAG-V}.
	Let us prove $2. \Rightarrow 1.$
	Since both $A$ and $B$ are finitely presented, we see that $\pi_0(A) \to \pi_0(B)$ is finitely presented.
	If we show that $\mathbb L_{A / B} \simeq 0$, we will obtain that $B \to A$ is finitely presented and \'etale.
	
	Let
	\[ f{^{-1}} \colon {\mathrm{Sh}}(A_{\mathrm{\acute{e}t}}, {\tau_\mathrm{q\acute{e}t}}) \to {\mathrm{Sh}}(B_{\mathrm{\acute{e}t}}) \]
	be the inverse image functor.
	Consider the sheaf $\mathbb L_{{\mathcal O}_A / f{^{-1}} {\mathcal O}_B}$ on $A_{\mathrm{\acute{e}t}}$ defined by
	\[ C \mapsto \mathbb L_{{\mathcal O}_A(C) / f{^{-1}} {\mathcal O}_B(C)} = \mathbb L_{C / f{^{-1}} {\mathcal O}_B(C)} \]
	Since the morphism $\operatorname{Spec}^{\mathrm{\acute{e}t}}(A) \to \operatorname{Spec}^{\mathrm{\acute{e}t}}(B)$ is \'etale, we see that $f{^{-1}} {\mathcal O}_B \simeq {\mathcal O}_A$.
	Therefore this sheaf is identically zero.
	
	On the other side, if $\eta{^{-1}} \colon {\mathrm{Sh}}(A_{\mathrm{\acute{e}t}}, {\tau_\mathrm{q\acute{e}t}}) \to {\mathcal S}$ is a geometric point, then
	\[ \eta{^{-1}}(\mathbb L_{{\mathcal O}_A / f{^{-1}} {\mathcal O}_B}) \simeq \mathbb L_{\eta{^{-1}} {\mathcal O}_A / \eta{^{-1}} f{^{-1}} {\mathcal O}_B} \]
	We can identify $\eta{^{-1}} f{^{-1}} {\mathcal O}_B$ with a strictly henselian $B$-algebra $B'$.
	Since the map $B \to B'$ is formally \'etale, we conclude that
	\[ \mathbb L_{\eta{^{-1}} {\mathcal O}_A / \eta{^{-1}} f{^{-1}} {\mathcal O}_B} \simeq \mathbb L_{\eta{^{-1}} {\mathcal O}_A / B} \]
	This is also the stalk of the sheaf on $A_{\mathrm{\acute{e}t}}$ defined by
	\[ C \mapsto \mathbb L_{C / B} \]
	Therefore, this sheaf vanishes as well. In particular, $\mathbb L_{A / B} \simeq 0$, completing the proof.
\end{proof}

\begin{prop} \label{prop:phi_X_geometric}
	Let $X = ({\mathcal X}, {\mathcal O}_X)$ and suppose that ${\mathcal X}$ is $n$-localic for $n \ge 1$.
	Then the stack $\phi(X)$ is $n$-geometric.
\end{prop}

\begin{proof}
	First suppose that $X = \operatorname{Spec}^{\mathrm{\acute{e}t}}(A)$.
	In this case, the universal property of $\operatorname{Spec}^{\mathrm{\acute{e}t}}$ proved in \cite[§2.2]{DAG-V} shows that $\phi(X)$ is just the representable functor associated to $A$.
	In particular, $\phi(X)$ is $(-1)$-geometric.
	
	Suppose now that $X = ({\mathcal X}, {\mathcal O}_X)$ is a ${{\mathcal G}_{\mathrm{\acute{e}t}}^\mathrm{der}(k)}$-scheme with ${\mathcal X}$ being $n$-localic, $n \ge 1$.
	By definition, we can find a collection of objects $V_i \in {\mathcal X}$ such that:
	\begin{enumerate}
		\item the morphism $\coprod V_i \to \mathbf 1_{\mathcal X}$ is an effective epimorphism;
		\item the ${{\mathcal G}_{\mathrm{\acute{e}t}}^\mathrm{der}(k)}$-schemes $({\mathcal X}_{/V_i}, {\mathcal O}_X |_{V_i})$ are equivalent to $\operatorname{Spec}^{\mathrm{\acute{e}t}}(U_i)$ for $U_i \in \mathrm{dAff}$, and each $U_i$ is of finite presentation.
	\end{enumerate}
	Set $V \coloneqq \coprod V_i$.
	By functoriality, we obtain a map
	\[ \coprod \phi(V_i) \to \phi(X) \]
	We only need to show that this map is representable by \'etale morphisms and that it is an effective epimorphism.
	Using the fact that $\operatorname{Spec}^{\mathrm{\acute{e}t}}$ preserves \'etale morphisms (see \cref{lem:topos_theoretic_etale}), we immediately see that this morphism is an effective epimorphism.
	
	We claim that if ${\mathcal X}$ is $n$-localic, then each $V_i$ is $(n-1)$-truncated.
	In order to prove this, we replace $X$ with $({\mathcal X}, \pi_0 {\mathcal O}_X)$, which is a ${{\mathcal G}_{\mathrm{\acute{e}t}}(k)}$-scheme.
	Let $F$ be the functor of points associated to it.
	Let $F_i$ be the functor of points associated to the \'etale subtopos associated to $V_i$.
	Reasoning as in the proof of \cite[Theorem 2.6.18]{DAG-V}, we see that to prove that each $V_i$ is $(n-1)$-truncated is equivalent to prove that for every (discrete) $k$-algebra $B$ the fibers of $F_i(B) \to F(B)$ are $(n-1)$-truncated.
	\cite[Lemma 2.6.19]{DAG-V} shows that $F(B)$ is $n$-truncated for every $k$-algebra $B$.
	On the other side, $F_i(B)$ is discrete.
	It follows that the fibers of $F_i(B) \to F(B)$ are $(n-1)$-truncated, thus completing the proof.
	
	We will now prove by induction on $n$ that each morphism $\phi({\mathcal X}_{/V_i}, {\mathcal O}_X |_{V_i}) \to \phi(X)$ is $(n-1)$-representable by \'etale maps.
	Let us denote by $\operatorname{Spec}(A)$ the representable functor associated to $A \in \mathrm{sCRing}$.
	We already observed that $\operatorname{Spec}(A) = \phi(\operatorname{Spec}^{\mathrm{\acute{e}t}}(A))$.
	Since $\phi$ commutes with fiber products and is fully faithful, we see that
	\[ \operatorname{Spec}(A) \times_{\phi(X)} \phi(V_i) \simeq \phi(\operatorname{Spec}^{\mathrm{\acute{e}t}}(A) \times_{({\mathcal X}, {\mathcal O}_X)} ({\mathcal X}_{/V_i}, {\mathcal O}_X |_{V_i} )) \]
	Let $(f_*, \varphi) \colon \operatorname{Spec}^{\mathrm{\acute{e}t}}(A) \to ({\mathcal X}, {\mathcal O}_X)$ be the given map.
	Then the fiber product $\operatorname{Spec}^{\mathrm{\acute{e}t}}(A) \times_{({\mathcal X}, {\mathcal O}_X)} ({\mathcal X}_{/V_i}, {\mathcal O}_X |_{V_i} )$ is the \'etale map to $\operatorname{Spec}^{\mathrm{\acute{e}t}}(A)$ classified by the object $f{^{-1}}(V_i) \in {\mathcal X}_A$.
	
	If $n = 1$, each object $V_i$ is $0$-truncated.
	It follows from \cref{prop:algebraic_spaces} that the fiber product $\operatorname{Spec}(A) \times_{\phi(X)} \phi(V_i)$ is $0$-geometric.
	Therefore, $\phi(X)$ is $1$-geometric.	
	Now suppose that ${\mathcal X}$ is $n$-localic for $n > 1$.
	Since each $V_i$ is $(n-1)$-truncated, \cite[Lemma 2.3.16]{DAG-V} shows that the underlying $\infty$-topos of
	\[ \operatorname{Spec}^{\mathrm{\acute{e}t}}(A) \times_{({\mathcal X}, {\mathcal O}_X)} ({\mathcal X}_{/V_i}, {\mathcal O}_X |_{V_i} ) \]
	is $(n-1)$-localic.
	The inductive hypothesis shows therefore that its image via the functor $\phi$ is $(n-1)$-geometric, and that the map to $\operatorname{Spec}(A)$ is \'etale.
	The proof is therefore complete.
\end{proof}

\subsection{The case of algebraic spaces}

Let $A \in \mathrm{sCRing}_k$.
We denote by $A_{\mathrm{\acute{e}t}}$ the small \'etale site of $A$: that is, its underlying $\infty$-category is the opposite of the full subcategory of $(\mathrm{sCRing}_k)_{A/}$ spanned by those maps $A \to B$ which are \'etale. The Grothendieck topology is the usual \'etale one.
We denote by $A_{\text{big, \'et}}$ the big \'etale site of $A$: that is, its underlying $\infty$-category is the opposite of $(\mathrm{sCRing}_k)_{A/}$, and the Grothendieck topology is the \'etale one.
Let us further denote by $(\mathrm{dAff},{\tau_\mathrm{q\acute{e}t}})$ the Grothendieck site of derived affine schemes.
There are continuous and cocontinuous morphisms of $\infty$-sites
\[ \begin{tikzcd}
(A_{\mathrm{\acute{e}t}}, {\tau_\mathrm{q\acute{e}t}}) \arrow{r}{u} & (A_{\text{big, \'et}}, {\tau_\mathrm{q\acute{e}t}}) \arrow{r}{v} & (\mathrm{dAff}, {\tau_\mathrm{q\acute{e}t}})
\end{tikzcd} \]
Observe that $u$ commutes with finite limits.
It follows that the induced adjunction
\[ u^s \colon {\mathrm{Sh}}(A_{\text{\'et}}, {\tau_\mathrm{q\acute{e}t}}) \rightleftarrows {\mathrm{Sh}}(A_{\text{big, \'et}}, {\tau_\mathrm{q\acute{e}t}}) \colon u_s \]
is a geometric morphism of $\infty$-topoi (in other words, $u^s$ commutes with finite limits).
In particular, we see that $u^s$ takes $n$-truncated objects to $n$-truncated objects (in fact, it commutes with homotopy groups).

This is not true for $v$, because it commutes only with connected limits.
However, we still have an adjunction
\[ v^s \colon {\mathrm{Sh}}(A_{\text{big, \'et}}, {\tau_\mathrm{q\acute{e}t}}) \rightleftarrows {\mathrm{Sh}}(\mathrm{dAff}, {\tau_\mathrm{q\acute{e}t}}) \colon v_s \]
Let us denote by $F_0$ the final object of ${\mathrm{Sh}}(A_{\text{big, \'et}})$.
We have that $v^s(F_0) \simeq \operatorname{Spec}(A)$.
In particular, for every $F \in {\mathrm{Sh}}(A_{\text{big, \'et}}, {\tau_\mathrm{q\acute{e}t}})$, we have a canonical natural transformation
\[ \alpha \colon v^s(F) \to \operatorname{Spec}(A) \]

\begin{lem}
	If $F$ is $n$-truncated, then $\alpha$ is $n$-truncated.
\end{lem}

\begin{proof}
	We have a canonical identification ${\mathrm{Sh}}(A_{\text{big, \'et}}, {\tau_\mathrm{q\acute{e}t}}) \simeq {\mathrm{Sh}}(\mathrm{dAff}, {\tau_\mathrm{q\acute{e}t}})_{/\operatorname{Spec}(A)}$, therefore the statement is tautological.
\end{proof}

\begin{defin}
	Let $k$ be a commutative ring, $A$ a commutative $k$-algebra and $X \in {\mathrm{Sh}}(\mathrm{dAff}_k, {\tau_\mathrm{q\acute{e}t}})$ any sheaf equipped with a natural transformation $\alpha \colon X \to \operatorname{Spec}(A)$. We will say that $\alpha$ exhibits $X$ as an \'etale algebraic space over $\operatorname{Spec}(A)$ if there exists a $0$-truncated sheaf $F \in {\mathrm{Sh}}(A_{\mathrm{\acute{e}t}}, {\tau_\mathrm{q\acute{e}t}})$ and an equivalence $X \simeq v^s(u^s(F))$ in ${\mathrm{Sh}}(\mathrm{dAff}_k, {\tau_\mathrm{q\acute{e}t}})_{/\operatorname{Spec}(A)}$.
\end{defin}

{\ignorespaces}

\begin{prop} \label{prop:algebraic_spaces}
	Let $\alpha \colon Y \to \operatorname{Spec}(A)$ be a natural transformation of stacks.
	Write $\operatorname{Spec}^{\mathrm{\acute{e}t}}(A) = ({\mathcal X}, {\mathcal O}_{\mathcal X})$.
	The following conditions are equivalent:
	\begin{enumerate}
		\item $\alpha$ exhibits $Y$ as a derived algebraic space over $\operatorname{Spec}(A)$;
		\item $Y$ is representable by a ${{\mathcal G}_{\mathrm{\acute{e}t}}^\mathrm{der}(k)}$-scheme $({\mathcal Y}, {\mathcal O}_{\mathcal Y})$ and $\alpha$ induces an equivalence $({\mathcal Y}, {\mathcal O}_{\mathcal Y}) \simeq ({\mathcal X}_{/U}, {\mathcal O}_{\mathcal X} |_U)$ for some discrete object $U \in {\mathcal X}$.
		\item the morphism $\alpha$ is $0$-truncated and $0$-representable by \'etale maps.
	\end{enumerate}
\end{prop}

\begin{proof}
	We first prove the equivalence of 1. and 2.
	If $\alpha$ exhibits $Y$ as a derived algebraic space over $\operatorname{Spec}(A)$, we can find a $0$-truncated sheaf $U \in {\mathrm{Sh}}(A_{\mathrm{\acute{e}t}}, {\tau_\mathrm{q\acute{e}t}})$ and an equivalence $Y \simeq v^s(u^s(U))$ in ${\mathrm{Sh}}(\mathrm{dAff}, {\tau_\mathrm{q\acute{e}t}})_{/\operatorname{Spec}(A)}$.
	Now, \cite[Remark 2.3.4]{DAG-V} shows that the functor represented by $({\mathcal X}_{/U}, {\mathcal O}_{\mathcal X} |_U)$ coincides with $Y$.
	Viceversa, if 2. is satisfied, then $U$ defines a derived algebraic space $v^s(u^s(U))$ over $\operatorname{Spec}(A)$, and \cite[Remark 2.3.4]{DAG-V} again allows to identify it with $Y$.
	
	Let us now prove the equivalence of 1. and 3. First, assume that 3. is satisfied.
	In this case, we can define a sheaf $U \colon A_{\mathrm{\acute{e}t}} \to {\mathcal S}$ by sending an \'etale map $f \colon A \to B$ to the fiber product
	\[ \begin{tikzcd}
	U(B) \arrow{r} \arrow{d} & Y(B) \arrow{d}{\alpha_B} \\
	\{*\} \arrow{r}{f} & \operatorname{Map}(A,B)
	\end{tikzcd} \]
	Since $\alpha$ is $0$-truncated, we see that $U$ takes value in ${\mathrm{Set}}$.
	Since it is obviously a sheaf, it defines a $0$-truncated object in ${\mathrm{Sh}}(A_{\mathrm{\acute{e}t}}, {\tau_\mathrm{q\acute{e}t}})$.
	\cite[Remark 2.3.4]{DAG-V} shows that $v^s(u^s(U))$ can be canonically identified with $Y$.
	
	Finally, let us prove $1. \Rightarrow 3.$
	We already know that, in this situation, $\alpha$ is $0$-truncated.
	Choosing sections $\eta_\alpha \in Y(A_\alpha)$ which generate $Y$, we obtain an effective epimorphism
	\[ \coprod \operatorname{Spec}(A_\alpha) \to v^s(u^s(Y)) \]
	in ${\mathrm{Sh}}(\mathrm{dAff}_k, {\tau_\mathrm{q\acute{e}t}})$.
	Suppose that there exists a $(-1)$-truncated morphism $v^s(u^s(Y)) \to \operatorname{Spec}(B)$ for some $B \in \mathrm{sCRing}$.
	In this case, we see that
	\[ \operatorname{Spec}(A_\alpha) \times_{v^s(u^s(Y))} \operatorname{Spec}(A_\beta) \simeq \operatorname{Spec}(A_\alpha) \times_{\operatorname{Spec}(B)} \operatorname{Spec}(A_\beta) \simeq \operatorname{Spec}(A_\alpha \otimes_B A_\beta) \]
	In the general case, each fiber product $Y_{\alpha, \beta} \coloneqq \operatorname{Spec}(A_\alpha) \times_{v^s(u^s(Y))} \operatorname{Spec}(A_\beta)$ is again a derived algebraic space \'etale over $A$. We claim moreover that the canonical morphism $Y_{\alpha, \beta} \to \operatorname{Spec}(A_\alpha \otimes_A A_\beta)$ is $(-1)$-truncated.
	Assuming the claim, it follows that $Y_{\alpha, \beta} \to \operatorname{Spec}(A)$ is $(-1)$-representable by \'etale maps, hence it would follow that the morphism $\operatorname{Spec}(A_\alpha) \to v^s(u^s(Y))$ is $0$-representable.
	Finally, we see that it is representable by \'etale maps combining the equivalence between 1. and 2. with \cref{lem:topos_theoretic_etale}.
	
	We are left to prove the claim. Fix $f_\alpha \colon A_\alpha \to B$, $f_\beta \colon A_\beta \to B$ together with a homotopy making the diagram
	\[ \begin{tikzcd}
	A \arrow{r} \arrow{d} & A_\alpha \arrow{d}{f_\alpha} \\
	A_\beta \arrow{r}{f_\beta} & B
	\end{tikzcd} \]
	commutative.
	We have pullback squares
	\[ \begin{tikzcd}
	Y_{\alpha, \beta} \arrow{r} \arrow{d} & v^s(u^s(Y)) \arrow{d} \\
	\operatorname{Spec}(A_\alpha) \times \operatorname{Spec}(A_\beta) \arrow{r} & v^s(u^s(Y)) \times_{\operatorname{Spec}(A)} v^s(u^s(Y))
	\end{tikzcd} \]
	and since $\alpha \colon v^s(u^s(Y)) \to \operatorname{Spec}(A)$ is $0$-truncated, the statement follows.
\end{proof}

\subsection{Essential surjectivity}

We now prove that $\phi$ is essentially surjective.
Let $X \in \mathbf{DM}$ be $n$-geometric.
In particular, ${\mathrm{t}_0}(X)$ is $(n+1)$-truncated.
It follows that the small \'etale site $({\mathrm{t}_0}(X))_{\mathrm{\acute{e}t}}$ is equivalent to an $(n+1,1)$-category.
Recall that there is an equivalence of $\infty$-categories
\[ X_{\mathrm{\acute{e}t}} \leftrightarrows ({\mathrm{t}_0}(X))_{\mathrm{\acute{e}t}} \]
We conclude that $X_{\mathrm{\acute{e}t}}$ is an $(n+1,1)$-category.
In particular, the $\infty$-topos ${\mathcal X} \coloneqq {\mathrm{Sh}}(X_{\mathrm{\acute{e}t}}, {\tau_\mathrm{q\acute{e}t}})$ is $(n+1)$-localic.
Define a ${{\mathcal T}_{\mathrm{\acute{e}t}}}$-structure on ${\mathcal X}$ as follows:
\[ {{\mathcal T}_{\mathrm{\acute{e}t}}} \times (X_{\mathrm{\acute{e}t}})^{\mathrm{op}} \to {\mathcal S} \]
defined as
\[ (U, V) \mapsto \operatorname{Map}_{\mathrm{dAff}}(V, U) \]
{\ignorespaces}
Fix $U \in {{\mathcal T}_{\mathrm{\acute{e}t}}}$. Since the Grothendieck topology on $\mathrm{dAff}$ is (hyper-)subcanonical, we see that the resulting object of $\operatorname{Fun}((X_{\mathrm{\acute{e}t}})^{\mathrm{op}}, {\mathcal S})$ is a (hyper-)sheaf.
In particular, we obtain a well defined functor
\[ {\mathcal O}_X \colon {{\mathcal T}_{\mathrm{\acute{e}t}}} \to {\mathrm{Sh}}(X_{\mathrm{\acute{e}t}}, {\tau_\mathrm{q\acute{e}t}}) \]
that in fact factors through hypercompletion of this category.
In order to show that it is a ${{\mathcal T}_{\mathrm{\acute{e}t}}}$-structure, we only need to check the following statements:
\begin{enumerate}
	\item ${\mathcal O}_X$ commutes with products;
	\item ${\mathcal O}_X$ commutes with admissible pullbacks;
	\item ${\mathcal O}_X$ takes ${\tau_\mathrm{q\acute{e}t}}$-coverings to effective epimorphisms.
\end{enumerate}
Since limits in ${\mathrm{Sh}}(X_{\mathrm{\acute{e}t}}, {\tau_\mathrm{q\acute{e}t}})$ are computed objectwise, the first two statements follow from the following elementary result:

\begin{lem}
	The inclusion functor ${{\mathcal T}_{\mathrm{\acute{e}t}}(k)} \to {{\mathcal G}_{\mathrm{\acute{e}t}}^\mathrm{der}(k)}$ commutes with products and pullback along \'etale morphisms.
\end{lem}

\begin{proof}
	Both those statements are consequence of the fact that this inclusion functor commutes with pullbacks along flat morphisms (and the proof is obvious).
	Indeed, for every object $U \in {{\mathcal T}_{\mathrm{\acute{e}t}}(k)}$, the canonical map $U \to \operatorname{Spec}(k)$ is smooth, hence flat.
\end{proof}

We are left to show that ${\mathcal O}_X$ takes ${\tau_\mathrm{q\acute{e}t}}$-coverings to effective epimorphisms.
Let $\{U_i \to U\}$ be a ${\tau_\mathrm{q\acute{e}t}}$-cover in ${{\mathcal T}_{\mathrm{\acute{e}t}}(k)}$.
We have to show that the morphism
\[ \coprod {\mathcal O}_X(U_i) \to {\mathcal O}_X(U) \]
is an effective epimorphism. In other words, we have to show that
\[ \coprod \pi_0 {\mathcal O}_X(U_i) \to \pi_0 {\mathcal O}_X(U) \]
is an epimorphism of sheaves of sets.
Such sheaves can be identified with ordinary sheaves over the Grothendieck site $(\mathrm h(X_{\mathrm{\acute{e}t}}), {\tau_\mathrm{q\acute{e}t}})$.
{\ignorespaces}
If $V \in \mathrm h(X_{\mathrm{\acute{e}t}})$, an element in $(\pi_0 {\mathcal O}_X(U))(V)$ is an \'etale covering $V_j \to V$ plus morphisms $V_j \to U$.
For each index $j$, we can find an \'etale covering $W_{jl} \to V_j$ such that the morphism $W_{jl} \to U$ factors through the cover $U_i \to U$.
Therefore, up to refining the cover $V_j \to V$, we see that the element in $(\pi_0 {\mathcal O}_X(U))(V)$ comes from the coproduct.

We therefore conclude that ${\mathcal O}_X$ is a hypercomplete ${{\mathcal T}_{\mathrm{\acute{e}t}}(k)}$-structure on ${\mathcal X}$.
Since ${{\mathcal G}_{\mathrm{\acute{e}t}}^\mathrm{der}(k)}$ is a geometric envelope for ${{\mathcal T}_{\mathrm{\acute{e}t}}(k)}$, we can identify ${\mathcal O}_X$ with a ${{\mathcal G}_{\mathrm{\acute{e}t}}^\mathrm{der}(k)}$-structure on ${\mathcal X}$.

\begin{prop} \label{prop:Get_scheme}
	The pair $({\mathcal X}, {\mathcal O}_X)$ is a ${{\mathcal G}_{\mathrm{\acute{e}t}}^\mathrm{der}(k)}$-scheme.
\end{prop}

\begin{proof}
	Choose an \'etale atlas $p \colon \coprod U_i \to X$ in the category $\mathbf{DM}$.
	Since each morphism $p_i \colon U_i \to X$ is \'etale, we see each of them defines an element in the small \'etale site $(X_{\mathrm{\acute{e}t}}, {\tau_\mathrm{q\acute{e}t}})$.
	Since this site is subcanonical, we can identify each $U_i$ with objects $V_i \in {\mathcal X}$.
	Moreover, the \'etale subtopos $({\mathcal X}_{/V_i}, {\mathcal O}_X |_{V_i})$ is canonically identified with $({\mathrm{Sh}}((U_i)_{\mathrm{\acute{e}t}}, {\tau_\mathrm{q\acute{e}t}}), {\mathcal O}_{U_i})$.
	The construction of the (absolute) spectrum functor of \cite[§2.2]{DAG-V}, shows that
	\[ \operatorname{Spec}^{\mathrm{\acute{e}t}}(U_i) \simeq ({\mathrm{Sh}}((U_i)_{\mathrm{\acute{e}t}}, {\tau_\mathrm{q\acute{e}t}}), {\mathcal O}_{U_i}) \]
	It will therefore be sufficient to show that the morphism $\coprod V_i \to \mathbf 1_{\mathcal X}$ is an effective epimorphism.
	In order to do this it will be convenient to replace the small \'etale site $X_{\mathrm{\acute{e}t}}$ with the site $((\mathrm{Geom}^{\le n}_{/X})_{\mathrm{\acute{e}t}}, {\tau_\mathrm{q\acute{e}t}})$ of \'etale maps $Y \to X$ where $Y$ is an $n$-geometric stack.
	The natural inclusion
	\[ (X_{\mathrm{\acute{e}t}}, {\tau_\mathrm{q\acute{e}t}}) \to ((\mathrm{Geom}^{\le n}_{/X})_{\mathrm{\acute{e}t}}, {\tau_\mathrm{q\acute{e}t}}) \]
	is an equivalence of sites in virtue of \cite[Lemma 2.34]{Porta_Yu_Higher_analytic_stacks_2014}.
	{\ignorespaces}
	In this way, we see that $\mathbf 1_{\mathcal X}$ is the representable sheaf associated to the identity map $\mathrm{id}_X \colon X \to X$.
	We are therefore left to show that
	\[ \coprod \pi_0 \operatorname{Map}(-, U_i) \to \pi_0 \operatorname{Map}(-, X) \]
	is an epimorphism of sheaves on $((\mathrm{Geom}^{\le n}_{/X})_{\mathrm{\acute{e}t}}, {\tau_\mathrm{q\acute{e}t}})$.
	This follows immediately from the fact that the maps $U_i \to X$ were an atlas for $X$.
\end{proof}

We are left to prove that $\phi({\mathcal X}, {\mathcal O}_X) \simeq X$.
We can proceed by induction on the geometric level $n$ of $X$.
If $n = -1$, the statement is obvious.
Otherwise, let $U_i \to X$ be an \'etale atlas for $X$.
Let $U \coloneqq \coprod U_i$ and let $U^\bullet$ be the \v{C}ech nerve of $U \to X$.
Combining the proof of \cref{prop:Get_scheme}, \cref{prop:phi_X_geometric} and the induction hypothesis, we see that $U^\bullet$ is a groupoid presentation for both $X$ and $\phi({\mathcal X}, {\mathcal O}_X)$.
We therefore proved that the essential image of the functor
\[ \phi \colon \mathrm{Sch}({{\mathcal G}_{\mathrm{\acute{e}t}}^\mathrm{der}(k)}) \to {\mathrm{Sh}}({\mathcal C}, {\tau_\mathrm{q\acute{e}t}}) \]
contains all the {Deligne-Mumford\xspace} stacks in the sense of \cite{HAG-II}.

\section{Descent vs hyperdescent}

The goal of this section is to prove the following folklore result:

\begin{prop} \label{prop:descent_vs_hyperdescent}
	Let $({\mathcal C}, \tau)$ be an $\infty$-Grothendieck site and let ${\mathcal D}$ be an $(n+1,1)$-category.
	Then A functor $F \colon {\mathcal C}^{\mathrm{op}} \to {\mathcal D}$ satisfies descent if and only if it satisfies hyperdescent.
\end{prop}

\begin{proof}
	Let $D \in {\mathcal D}$ be any object and let $c_D \colon {\mathcal D} \to {\mathcal S}$ be the functor \emph{corepresented} by $D$.
	Then $F$ satisfies descent (resp.\ hyperdescent) if and only if $c_D \circ F$ does.
	Since ${\mathcal D}$ is an $(n+1,1)$-category, we see that $c_D \circ F$ takes values in $\tau_{\le n} {\mathcal S}$. 
	Therefore, we may replace ${\mathcal D}$ with ${\mathcal S}$ and suppose that $F$ takes values in the full subcategory of $n$-truncated objects.
	For every $U \in {\mathcal C}$, let us denote by $h_U$ the sheafification of the presheaf associated to $U$.
	Since $F$ is an $n$-truncated object, we see that
	\[ \operatorname{Map}_{{\mathrm{Sh}}_{\le n}({\mathcal C}, \tau)}(\tau_{\le n} h_U, F) \simeq \operatorname{Map}_{{\mathrm{Sh}}({\mathcal C}, \tau)}(h_U, F) \simeq F(U) \]
	where the last equivalence is obtained combining the universal property of the sheafification with the Yoneda lemma.
	Therefore, it will be sufficient to show that for every hypercover $U^\bullet \to U$ in ${\mathcal C}$, the augmented simplicial diagram
	\[ \tau_{\le n} h_{U^\bullet} \to \tau_{\le n} h_U \]
	is a colimit diagram in ${\mathrm{Sh}}_{\le n}({\mathcal C}, \tau)$.
	Since $\tau_{\le n}$ is a left adjoint, we see that in ${\mathrm{Sh}}_{\le n}({\mathcal C}, \tau)$ the relation
	\[ |\tau_{\le n} h_{U^\bullet}| \simeq \tau_{\le n} |h_{U^\bullet}| \]
	holds.
	Moreover, since $U^\bullet \to U$ is an hypercover, the morphism $|h_{U^\bullet}| \to h_U$ is $\infty$-connected in virtue of \cite[6.5.3.11]{HTT}.
	Since $\tau_{\le n}$ commutes with $\infty$-connected morphisms, {\ignorespaces} we conclude that
	\[ \tau_{\le n} |h_{U^\bullet}| \to \tau_{\le n} h_U \]
	is an $\infty$-connected morphism between $n$-truncated objects.
	Therefore it is an equivalence in ${\mathrm{Sh}}({\mathcal C}, \tau)$.
	In conclusion, the morphism $|\tau_{\le n} h_{U^\bullet}| \to \tau_{\le n} h_U$ is an equivalence in ${\mathrm{Sh}}_{\le n}({\mathcal C}, \tau)$.
	The proof is now complete.
\end{proof}

\section{A sharper version of GAGA 1}

In the previous subsection we were able to prove GAGA 2 for the whole unbounded category ${\mathrm{Coh}}(X)$.
It is a rather natural question to ask whether it is possible to have also an unbounded version of GAGA 1.
In this section we provide a sufficient criterion for this to be true.

Let us begin by making a simple consideration. Suppose we are in the setting of \cref{thm:derived_GAGA_1} and let ${\mathcal F} \in {\mathrm{Coh}}(X)$. Assume moreover that ${\mathrm R} f{^\mathrm{an}}_*$ takes ${\mathrm{Coh}}(X{^\mathrm{an}})$ in ${\mathrm{Coh}}(X{^\mathrm{an}})$. In order to prove that the natural map
\[ \varphi_{\mathcal F} \colon ({\mathrm R} f_* {\mathcal F}){^\mathrm{an}} \to {\mathrm R} f{^\mathrm{an}}_* {\mathcal F}{^\mathrm{an}} \]
is an equivalence, we only need to show that it induces an equivalence on every cohomology sheaf.
If we knew that both the maps $f$ and $f{^\mathrm{an}}$ have (coherent) cohomological dimensions bounded by some integer $d$, then using \cref{cor:analytification_flat} we would obtain:
\begin{gather*}
{\mathcal H}^i(({\mathrm R} f_* {\mathcal F}){^\mathrm{an}}) \simeq ({\mathcal H}^i({\mathrm R} f_* {\mathcal F})){^\mathrm{an}} \simeq ({\mathcal H}^i( {\mathrm R} f_* ( \tau_{\ge i - d - 1} {\mathcal F} ) )){^\mathrm{an}} \\
{\mathcal H}^i( ({\mathrm R} f{^\mathrm{an}}_* {\mathcal F}{^\mathrm{an}}) ) \simeq {\mathcal H}^i( ( {\mathrm R} f{^\mathrm{an}}_* ( \tau_{\ge i - d - 1} {\mathcal F}{^\mathrm{an}} ) ))
\end{gather*}
Repeating this for every integer $i \in \mathbb Z$, it would follow from \cref{thm:derived_GAGA_1} that $\varphi_{\mathcal F}$ is an equivalence for every ${\mathcal F} \in {\mathrm{Coh}}(X)$.
In order to summarize this discussion for future reference, let us give the following definition:

\begin{defin}
	We say that a morphism $f \colon X \to Y$ of derived {$\mathbb C$-analytic\xspace} spaces is of \emph{bounded coherent cohomological dimension} if the functor ${\mathrm R} f_* \colon {\mathcal O}_{\mathcal X} \textrm{-} {\mathrm{Mod}} \to {\mathcal O}_{\mathcal Y} \textrm{-} {\mathrm{Mod}}$ satisfies the following two conditions:
	\begin{enumerate}
		\item ${\mathrm R} f_*({\mathcal F}) \in {\mathrm{Coh}}(Y)$ whenever ${\mathcal F} \in {\mathrm{Coh}}(X)$;
		\item there exists an integer $d$ such that ${\mathrm R} f_*({\mathcal F}) \in {\mathrm{Coh}}^{\le d}(Y)$ whenever ${\mathcal F} \in {\mathrm{Coh}}^{\le 0}(X)$.
	\end{enumerate}
\end{defin}

\begin{lem} \label{lem:GAGA_1_unbounded}
	Let $f \colon X \to Y$ be a proper morphism of derived {Deligne-Mumford\xspace} stacks locally of finite presentation over $\mathbb C$.
	If both $f$ is of bounded cohomological dimension and $f{^\mathrm{an}}$ is of bounded coherent cohomological dimension, then the natural map
	\[ \varphi_{\mathcal F} \colon ({\mathrm R} f_* {\mathcal F}){^\mathrm{an}} \to {\mathrm R} f{^\mathrm{an}}_* {\mathcal F}{^\mathrm{an}} \]
	is an equivalence for every ${\mathcal F} \in {\mathrm{Coh}}(X)$.
\end{lem}

It becomes therefore interesting to look for sufficient conditions for the hypotheses of the previous lemma to be satisfied.
We begin by analyzing the algebraic side:

\begin{prop}
	Let $f \colon X \to Y$ be a proper morphism of derived Artin stacks which are locally noetherian over some base.
	Then $f$ is of bounded cohomological dimension.
\end{prop}

\begin{rem}
	When both $X$ and $Y$ are {Deligne-Mumford\xspace} stacks, the previous proposition is a consequence of \cite[Theorem 1.4.2]{Drinfeld_Gaitsgory_Finiteness_questions_2011}. However, for $X$ and $Y$ Artin the notion of properness (which implies that the diagonal $X \to X \times_Y X$ is also proper) is incompatible with the condition of having affine stabilizers. Therefore the above proposition is not a trivial consequence of the cited result.
\end{rem}

\begin{proof}
	Reasoning as in \cite{Drinfeld_Gaitsgory_Finiteness_questions_2011} we see that it is enough to consider the case when $Y$ is affine and both $X$ and $Y$ are classical and reduced.
	In this case...
\end{proof}

We now turn to the analytic side:

\begin{lem}
	Let $f \colon X \to Y$ be a proper morphism of derived {$\mathbb C$-analytic\xspace} spaces.
	Suppose that there exists an integer $d$ such that for every ${\mathcal F} \in {\mathrm{Coh}}^\heartsuit(X)$ one has ${\mathcal H}^i({\mathrm R} f_* {\mathcal F}) = 0$ for every $i > n$.
	Then $f$ is of bounded coherent cohomological dimension.
\end{lem}

\begin{proof}
	It follows from the hypothesis and from the d\'evissage lemma \cite[Lemma 5.10]{Porta_Yu_Higher_analytic_stacks_2014} that ${\mathrm R} f_* {\mathcal G}$ is cohomologically bounded for every ${\mathcal G} \in {\mathrm{Coh}^{\mathrm{b}}}(X)$.
	Let ${\mathcal F} \in {\mathrm{Coh}}^{\le 0}(X)$.
	We can write
	\[ {\mathcal F} \simeq \lim_n \tau_{\ge n} {\mathcal F} . \]
	Since ${\mathrm R} f_*$ commutes with limits, we deduce that
	\[ {\mathrm R} f_* {\mathcal F} \simeq \lim_n {\mathrm R} f_*( \tau_{\ge n} {\mathcal F} ) \]
	Now, each ${\mathrm R} f_*(\tau_{\ge n} {\mathcal F})$ is cohomologically bounded.
\end{proof}

\fi

\bibliographystyle{plain}
\bibliography{dahema}

\end{document}


\documentclass[11pt]{amsart}
\usepackage{amsfonts,amssymb,amsmath,euscript}
\usepackage{graphicx}
\usepackage[matrix,arrow,curve]{xy}
\usepackage{wrapfig}

    \newtheorem{thm}{Theorem}                     [section]
    \newtheorem{thm*}{Theorem}
    \newtheorem{prop}[thm]{Proposition}
    \newtheorem{lemma}[thm]{Lemma}
    \newtheorem{cor}[thm]{Corollary}
    \newtheorem{step}{Step}
    \newtheorem{lemma*}{Lemma}    
    \newtheorem{probl}{Problem}                      [section]
    \newtheorem{assump}[thm]{Assumption}
    \newtheorem{axiom}{Axiom}

    
    \newtheorem{rthm}{Òåîðåìà}
    \newtheorem{rprop}[rthm]{Ïðåäëîæåíèå}
    \newtheorem{rlemma}[rthm]{Ëåììà}
    \newtheorem{rcor}[rthm]{Ñëåäñòâèå}

    \newtheorem{defn}[thm]{Definition}                 
    \newtheorem{example}[thm]{Example}                 
    \newtheorem{exercise}{Exercise}                 
    \newtheorem{obs}{Observation}                 
    \newtheorem{zadacha}{Zadacha}                 
    \newtheorem{hypo}{Hypothesis}                 
    \newtheorem{conj}{Conjecture}                 
    \newtheorem{oprobl}{Open problem}                 
    \newtheorem{rems}[thm]{Remark}                     
    \newtheorem{ques}{Question}            
    \newtheorem{solvedques}[ques]{Solved Question}
    \newtheorem{rems*}{Remark}   

    

    \newtheorem{rdefn}[rthm]{Îïðåäåëåíèå}
    \newtheorem{rexample}[rthm]{Ïðèìåð}
    \newtheorem{robs}[rthm]{Íàáëþäåíèå}
    \newtheorem{rques}[rthm]{Âîïðîñ}
    \newtheorem{rrems}[rthm]{Çàìå÷àíèå}
    
    \newtheorem{rzadacha}[rthm]{Çàäà÷à}
    \newtheorem{rprobl}[rthm]{Ïðîáëåìà}
    \newtheorem{rhypo}[rthm]{Ãèïîòåçà}

 

\def\rndef[1]{\begin{center} \it\small #1 \end{center}}

{\myaddress}{\clA}{{\mathcal A}} {\myaddress}{\rmA}{{\mathrm A}} {\myaddress}{\mbA}{{\mathbb A}} {\myaddress}{\bfA}{{\mathbf A}} {\myaddress}{\euA}{{\EuScript A}} {\myaddress}{\frA}{{\mathfrak A}}
{\myaddress}{\clB}{{\mathcal B}} {\myaddress}{\rmB}{{\mathrm B}} {\myaddress}{\mbB}{{\mathbb B}} {\myaddress}{\bfB}{{\mathbf B}} {\myaddress}{\euB}{{\EuScript B}} {\myaddress}{\frB}{{\mathfrak B}}
{\myaddress}{\clC}{{\mathcal C}} {\myaddress}{\rmC}{{\mathrm C}} {\myaddress}{\mbC}{{\mathbb C}} {\myaddress}{\bfC}{{\mathbf C}} {\myaddress}{\euC}{{\EuScript C}} {\myaddress}{\frC}{{\mathfrak C}}
{\myaddress}{\clD}{{\mathcal D}} {\myaddress}{\rmD}{{\mathrm D}} {\myaddress}{\mbD}{{\mathbb D}} {\myaddress}{\bfD}{{\mathbf D}} {\myaddress}{\euD}{{\EuScript D}} {\myaddress}{\frD}{{\mathfrak D}}
{\myaddress}{\clE}{{\mathcal E}} {\myaddress}{\rmE}{{\mathrm E}} {\myaddress}{\mbE}{{\mathbb E}} {\myaddress}{\bfE}{{\mathbf E}} {\myaddress}{\euE}{{\EuScript E}} {\myaddress}{\frE}{{\mathfrak E}}
{\myaddress}{\clF}{{\mathcal F}} {\myaddress}{\rmF}{{\mathrm F}} {\myaddress}{\mbF}{{\mathbb F}} {\myaddress}{\bfF}{{\mathbf F}} {\myaddress}{\euF}{{\EuScript F}} {\myaddress}{\frF}{{\mathfrak F}}
{\myaddress}{\clG}{{\mathcal G}} {\myaddress}{\rmG}{{\mathrm G}} {\myaddress}{\mbG}{{\mathbb G}} {\myaddress}{\bfG}{{\mathbf G}} {\myaddress}{\euG}{{\EuScript G}} {\myaddress}{\frG}{{\mathfrak G}}
{\myaddress}{\clH}{{\mathcal H}} {\myaddress}{\rmH}{{\mathrm H}} {\myaddress}{\mbH}{{\mathbb H}} {\myaddress}{\bfH}{{\mathbf H}} {\myaddress}{\euH}{{\EuScript H}} {\myaddress}{\frH}{{\mathfrak H}}
{\myaddress}{\clI}{{\mathcal I}} {\myaddress}{\rmI}{{\mathrm I}} {\myaddress}{\mbI}{{\mathbb I}} {\myaddress}{\bfI}{{\mathbf I}} {\myaddress}{\euI}{{\EuScript I}} {\myaddress}{\frI}{{\mathfrak I}}
{\myaddress}{\clJ}{{\mathcal J}} {\myaddress}{\rmJ}{{\mathrm J}} {\myaddress}{\mbJ}{{\mathbb J}} {\myaddress}{\bfJ}{{\mathbf J}} {\myaddress}{\euJ}{{\EuScript J}} {\myaddress}{\frJ}{{\mathfrak J}}
{\myaddress}{\clK}{{\mathcal K}} {\myaddress}{\rmK}{{\mathrm K}} {\myaddress}{\mbK}{{\mathbb K}} {\myaddress}{\bfK}{{\mathbf K}} {\myaddress}{\euK}{{\EuScript K}} {\myaddress}{\frK}{{\mathfrak K}}
{\myaddress}{\clL}{{\mathcal L}} {\myaddress}{\rmL}{{\mathrm L}} {\myaddress}{\mbL}{{\mathbb L}} {\myaddress}{\bfL}{{\mathbf L}} {\myaddress}{\euL}{{\EuScript L}} {\myaddress}{\frL}{{\mathfrak L}}
{\myaddress}{\clM}{{\mathcal M}} {\myaddress}{\rmM}{{\mathrm M}} {\myaddress}{\mbM}{{\mathbb M}} {\myaddress}{\bfM}{{\mathbf M}} {\myaddress}{\euM}{{\EuScript M}} {\myaddress}{\frM}{{\mathfrak M}}
{\myaddress}{\clN}{{\mathcal N}} {\myaddress}{\rmN}{{\mathrm N}} {\myaddress}{\mbN}{{\mathbb N}} {\myaddress}{\bfN}{{\mathbf N}} {\myaddress}{\euN}{{\EuScript N}} {\myaddress}{\frN}{{\mathfrak N}}
{\myaddress}{\clO}{{\mathcal O}} {\myaddress}{\rmO}{{\mathrm O}} {\myaddress}{\mbO}{{\mathbb O}} {\myaddress}{\bfO}{{\mathbf O}} {\myaddress}{\euO}{{\EuScript O}} {\myaddress}{\frO}{{\mathfrak O}}
{\myaddress}{\clP}{{\mathcal P}} {\myaddress}{\rmP}{{\mathrm P}} {\myaddress}{\mbP}{{\mathbb P}} {\myaddress}{\bfP}{{\mathbf P}} {\myaddress}{\euP}{{\EuScript P}} {\myaddress}{\frP}{{\mathfrak P}}
{\myaddress}{\clQ}{{\mathcal Q}} {\myaddress}{\rmQ}{{\mathrm Q}} {\myaddress}{\mbQ}{{\mathbb Q}} {\myaddress}{\bfQ}{{\mathbf Q}} {\myaddress}{\euQ}{{\EuScript Q}} {\myaddress}{\frQ}{{\mathfrak Q}}
{\myaddress}{\clR}{{\mathcal R}} {\myaddress}{\rmR}{{\mathrm R}} {\myaddress}{\mbR}{{\mathbb R}} {\myaddress}{\bfR}{{\mathbf R}} {\myaddress}{\euR}{{\EuScript R}} {\myaddress}{\frR}{{\mathfrak R}}
{\myaddress}{\clS}{{\mathcal S}} {\myaddress}{\rmS}{{\mathrm S}} {\myaddress}{\mbS}{{\mathbb S}} {\myaddress}{\bfS}{{\mathbf S}} {\myaddress}{\euS}{{\EuScript S}} {\myaddress}{\frS}{{\mathfrak S}}
{\myaddress}{\clT}{{\mathcal T}} {\myaddress}{\rmT}{{\mathrm T}} {\myaddress}{\mbT}{{\mathbb T}} {\myaddress}{\bfT}{{\mathbf T}} {\myaddress}{\euT}{{\EuScript T}} {\myaddress}{\frT}{{\mathfrak T}}
{\myaddress}{\clU}{{\mathcal U}} {\myaddress}{\rmU}{{\mathrm U}} {\myaddress}{\mbU}{{\mathbb U}} {\myaddress}{\bfU}{{\mathbf U}} {\myaddress}{\euU}{{\EuScript U}} {\myaddress}{\frU}{{\mathfrak U}}
{\myaddress}{\clV}{{\mathcal V}} {\myaddress}{\rmV}{{\mathrm V}} {\myaddress}{\mbV}{{\mathbb V}} {\myaddress}{\bfV}{{\mathbf V}} {\myaddress}{\euV}{{\EuScript V}} {\myaddress}{\frV}{{\mathfrak V}}
{\myaddress}{\clW}{{\mathcal W}} {\myaddress}{\rmW}{{\mathrm W}} {\myaddress}{\mbW}{{\mathbb W}} {\myaddress}{\bfW}{{\mathbf W}} {\myaddress}{\euW}{{\EuScript W}} {\myaddress}{\frW}{{\mathfrak W}}
{\myaddress}{\clX}{{\mathcal X}} {\myaddress}{\rmX}{{\mathrm X}} {\myaddress}{\mbX}{{\mathbb X}} {\myaddress}{\bfX}{{\mathbf X}} {\myaddress}{\euX}{{\EuScript X}} {\myaddress}{\frX}{{\mathfrak X}}
{\myaddress}{\clY}{{\mathcal Y}} {\myaddress}{\rmY}{{\mathrm Y}} {\myaddress}{\mbY}{{\mathbb Y}} {\myaddress}{\bfY}{{\mathbf Y}} {\myaddress}{\euY}{{\EuScript Y}} {\myaddress}{\frY}{{\mathfrak Y}}
{\myaddress}{\clZ}{{\mathcal Z}} {\myaddress}{\rmZ}{{\mathrm Z}} {\myaddress}{\mbZ}{{\mathbb Z}} {\myaddress}{\bfZ}{{\mathbf Z}} {\myaddress}{\euZ}{{\EuScript Z}} {\myaddress}{\frZ}{{\mathfrak Z}}

{\myaddress}{\tA}{{\widetilde A}} {\myaddress}{\tcA}{{\widetilde\clA}} {\myaddress}{\ttcA}{\widetilde{\tcA}} {\myaddress}{\sfA}{{\textsf A}} {\myaddress}{\ttA}{\widetilde{\tA}} {\myaddress}{\dzA}{{A^\sharp}}
{\myaddress}{\tB}{{\widetilde B}} {\myaddress}{\tcB}{{\widetilde\clB}} {\myaddress}{\ttcB}{\widetilde{\tcB}} {\myaddress}{\sfB}{{\textsf B}} {\myaddress}{\ttB}{\widetilde{\tB}} {\myaddress}{\dzB}{{B^\sharp}}
{\myaddress}{\tC}{{\widetilde C}} {\myaddress}{\tcC}{{\widetilde\clC}} {\myaddress}{\ttcC}{\widetilde{\tcC}} {\myaddress}{\sfC}{{\textsf C}} {\myaddress}{\ttC}{\widetilde{\tC}} {\myaddress}{\dzC}{{C^\sharp}}
{\myaddress}{\tD}{{\widetilde D}} {\myaddress}{\tcD}{{\widetilde\clD}} {\myaddress}{\ttcD}{\widetilde{\tcD}} {\myaddress}{\sfD}{{\textsf D}} {\myaddress}{\ttD}{\widetilde{\tD}} {\myaddress}{\dzD}{{D^\sharp}}
{\myaddress}{\tE}{{\widetilde E}} {\myaddress}{\tcE}{{\widetilde\clE}} {\myaddress}{\ttcE}{\widetilde{\tcE}} {\myaddress}{\sfE}{{\textsf E}} {\myaddress}{\ttE}{\widetilde{\tE}} {\myaddress}{\dzE}{{E^\sharp}}
{\myaddress}{\tF}{{\widetilde F}} {\myaddress}{\tcF}{{\widetilde\clF}} {\myaddress}{\ttcF}{\widetilde{\tcF}} {\myaddress}{\sfF}{{\textsf F}} {\myaddress}{\ttF}{\widetilde{\tF}} {\myaddress}{\dzF}{{F^\sharp}}
{\myaddress}{\tG}{{\widetilde G}} {\myaddress}{\tcG}{{\widetilde\clG}} {\myaddress}{\ttcG}{\widetilde{\tcG}} {\myaddress}{\sfG}{{\textsf G}} {\myaddress}{\ttG}{\widetilde{\tG}} {\myaddress}{\dzG}{{G^\sharp}}
{\myaddress}{\tH}{{\widetilde H}} {\myaddress}{\tcH}{{\widetilde\clH}} {\myaddress}{\ttcH}{\widetilde{\tcH}} {\myaddress}{\sfH}{{\textsf H}} {\myaddress}{\ttH}{\widetilde{\tH}} {\myaddress}{\dzH}{{H^\sharp}}
{\myaddress}{\tI}{{\widetilde I}} {\myaddress}{\tcI}{{\widetilde\clI}} {\myaddress}{\ttcI}{\widetilde{\tcI}} {\myaddress}{\sfI}{{\textsf I}} {\myaddress}{\ttI}{\widetilde{\tI}} {\myaddress}{\dzI}{{I^\sharp}}
{\myaddress}{\tJ}{{\widetilde J}} {\myaddress}{\tcJ}{{\widetilde\clJ}} {\myaddress}{\ttcJ}{\widetilde{\tcJ}} {\myaddress}{\sfJ}{{\textsf J}} {\myaddress}{\ttJ}{\widetilde{\tJ}} {\myaddress}{\dzJ}{{J^\sharp}}
{\myaddress}{\tK}{{\widetilde K}} {\myaddress}{\tcK}{{\widetilde\clK}} {\myaddress}{\ttcK}{\widetilde{\tcK}} {\myaddress}{\sfK}{{\textsf K}} {\myaddress}{\ttK}{\widetilde{\tK}} {\myaddress}{\dzK}{{K^\sharp}}
{\myaddress}{\tL}{{\widetilde L}} {\myaddress}{\tcL}{{\widetilde\clL}} {\myaddress}{\ttcL}{\widetilde{\tcL}} {\myaddress}{\sfL}{{\textsf L}} {\myaddress}{\ttL}{\widetilde{\tL}} {\myaddress}{\dzL}{{L^\sharp}}
{\myaddress}{\tM}{{\widetilde M}} {\myaddress}{\tcM}{{\widetilde\clM}} {\myaddress}{\ttcM}{\widetilde{\tcM}} {\myaddress}{\sfM}{{\textsf M}} {\myaddress}{\ttM}{\widetilde{\tM}} {\myaddress}{\dzM}{{M^\sharp}}
{\myaddress}{\tN}{{\widetilde N}} {\myaddress}{\tcN}{{\widetilde\clN}} {\myaddress}{\ttcN}{\widetilde{\tcN}} {\myaddress}{\sfN}{{\textsf N}} {\myaddress}{\ttN}{\widetilde{\tN}} {\myaddress}{\dzN}{{N^\sharp}}
{\myaddress}{\tO}{{\widetilde O}} {\myaddress}{\tcO}{{\widetilde\clO}} {\myaddress}{\ttcO}{\widetilde{\tcO}} {\myaddress}{\sfO}{{\textsf O}} {\myaddress}{\ttO}{\widetilde{\tO}} {\myaddress}{\dzO}{{O^\sharp}}
{\myaddress}{\tP}{{\widetilde P}} {\myaddress}{\tcP}{{\widetilde\clP}} {\myaddress}{\ttcP}{\widetilde{\tcP}} {\myaddress}{\sfP}{{\textsf P}} {\myaddress}{\ttP}{\widetilde{\tP}} {\myaddress}{\dzP}{{P^\sharp}}
{\myaddress}{\tQ}{{\widetilde Q}} {\myaddress}{\tcQ}{{\widetilde\clQ}} {\myaddress}{\ttcQ}{\widetilde{\tcQ}} {\myaddress}{\sfQ}{{\textsf Q}} {\myaddress}{\ttQ}{\widetilde{\tQ}} {\myaddress}{\dzQ}{{Q^\sharp}}
{\myaddress}{\tR}{{\widetilde R}} {\myaddress}{\tcR}{{\widetilde\clR}} {\myaddress}{\ttcR}{\widetilde{\tcR}} {\myaddress}{\sfR}{{\textsf R}} {\myaddress}{\ttR}{\widetilde{\tR}} {\myaddress}{\dzR}{{R^\sharp}}
{\myaddress}{\tS}{{\widetilde S}} {\myaddress}{\tcS}{{\widetilde\clS}} {\myaddress}{\ttcS}{\widetilde{\tcS}} {\myaddress}{\sfS}{{\textsf S}} {\myaddress}{\ttS}{\widetilde{\tS}} {\myaddress}{\dzS}{{S^\sharp}}
{\myaddress}{\tT}{{\widetilde T}} {\myaddress}{\tcT}{{\widetilde\clT}} {\myaddress}{\ttcT}{\widetilde{\tcT}} {\myaddress}{\sfT}{{\textsf T}} {\myaddress}{\ttT}{\widetilde{\tT}} {\myaddress}{\dzT}{{T^\sharp}}
{\myaddress}{\tU}{{\widetilde U}} {\myaddress}{\tcU}{{\widetilde\clU}} {\myaddress}{\ttcU}{\widetilde{\tcU}} {\myaddress}{\sfU}{{\textsf U}} {\myaddress}{\ttU}{\widetilde{\tU}} {\myaddress}{\dzU}{{U^\sharp}}
{\myaddress}{\tV}{{\widetilde V}} {\myaddress}{\tcV}{{\widetilde\clV}} {\myaddress}{\ttcV}{\widetilde{\tcV}} {\myaddress}{\sfV}{{\textsf V}} {\myaddress}{\ttV}{\widetilde{\tV}} {\myaddress}{\dzV}{{V^\sharp}}
{\myaddress}{\tW}{{\widetilde W}} {\myaddress}{\tcW}{{\widetilde\clW}} {\myaddress}{\ttcW}{\widetilde{\tcW}} {\myaddress}{\sfW}{{\textsf W}} {\myaddress}{\ttW}{\widetilde{\tW}} {\myaddress}{\dzW}{{W^\sharp}}
{\myaddress}{\tX}{{\widetilde X}} {\myaddress}{\tcX}{{\widetilde\clX}} {\myaddress}{\ttcX}{\widetilde{\tcX}} {\myaddress}{\sfX}{{\textsf X}} {\myaddress}{\ttX}{\widetilde{\tX}} {\myaddress}{\dzX}{{X^\sharp}}
{\myaddress}{\tY}{{\widetilde Y}} {\myaddress}{\tcY}{{\widetilde\clY}} {\myaddress}{\ttcY}{\widetilde{\tcY}} {\myaddress}{\sfY}{{\textsf Y}} {\myaddress}{\ttY}{\widetilde{\tY}} {\myaddress}{\dzY}{{Y^\sharp}}
{\myaddress}{\tZ}{{\widetilde Z}} {\myaddress}{\tcZ}{{\widetilde\clZ}} {\myaddress}{\ttcZ}{\widetilde{\tcZ}} {\myaddress}{\sfZ}{{\textsf Z}} {\myaddress}{\ttZ}{\widetilde{\tZ}} {\myaddress}{\dzZ}{{Z^\sharp}}

{\myaddress}{\bfa}{{\mathbf a}}
{\myaddress}{\bfb}{{\mathbf b}}
{\myaddress}{\bfc}{{\mathbf c}}
{\myaddress}{\bfd}{{\mathbf d}}

{\myaddress}{\euu}{{\EuScript u}}

  {\myaddress}{\eps}{\varepsilon}

\let\geq\geqslant
\let\leq\leqslant

{\myaddress}{\lims}[1]{\lim\limits_{#1}}
{\myaddress}{\sums}[1]{\sum\limits_{#1}}
{\myaddress}{\ints}[1]{\int_{#1}}
{\myaddress}{\sups}[1]{\sup\limits_{#1}}
{\myaddress}{\liminfty}[1]{\lims{#1\to\infty}}
{\myaddress}{\suminf}[1]{\sums{#1=1}^\infty}

{\myaddress}{\limo}[1]{\omega\mbox{-}\!\!\!\lims{#1\to\infty}}          
{\myaddress}{\limL}[1]{\rmL\mbox{-}\!\!\!\lims{#1\to\infty}}            
{\myaddress}{\limLOne}[1]{\clL_1\mbox{-}\!\lims{#1}}
{\myaddress}{\tildelimo}[1]{\tilde\omega\mbox{-}\!\!\!\lims{#1\to\infty}}
{\myaddress}{\slim}{\mathrm{s}\mbox{-}\!\!\lim}          
{\myaddress}{\wlim}{\mathrm{w}\mbox{-}\!\lim}          

{\myaddress}{\Aut}{\operatorname{Aut}}      
{\myaddress}{\Ch}{\operatorname{ch}}        
{\myaddress}{\End}{\operatorname{End}}      
{\myaddress}{\Hom}{\operatorname{Hom}}      
\rndef{\ker}{\operatorname{ker}}      
{\myaddress}{\coker}{\operatorname{coker}}      
{\myaddress}{\im}{\operatorname{im}}        
{\myaddress}{\Log}{\operatorname{Log}}      
{\myaddress}{\OP}{\operatorname{OP}}        
{\myaddress}{\Op}{\operatorname{Op}}        
{\myaddress}{\Symb}{\operatorname{Symb}}    
{\myaddress}{\Tr}{\operatorname{Tr}}        
{\myaddress}{\Wres}{\operatorname{Wres}}    
{\myaddress}{\cl}{\operatorname{cl}}        
{\myaddress}{\com}{\operatorname{com}}
{\myaddress}{\const}{\operatorname{const}}  
{\myaddress}{\conv}{\operatorname{conv}}    
\rndef{\det}{\operatorname{det}}     
{\myaddress}{\Var}{\operatorname{Var}}
{\myaddress}{\Cov}{\operatorname{Cov}}

{\myaddress}{\detFK}[1]{\Delta\brs{#1}} 
{\myaddress}{\detFKrel}[2]{\Delta_{#2}\brs{#1}} 

{\myaddress}{\adj}{\operatorname{adj}}    
{\myaddress}{\diag}{\operatorname{diag}}    
{\myaddress}{\dist}{\operatorname{dist}}    
{\myaddress}{\dom}{\operatorname{dom}}      
{\myaddress}{\ec}{\operatorname{ec}}        
{\myaddress}{\id}{\mathrm{Id}}                        
{\myaddress}{\ind}{\operatorname{ind}}      
{\myaddress}{\mydeg}{\operatorname{deg}}    
{\myaddress}{\op}{\operatorname{op}}
{\myaddress}{\rank}{\operatorname{rank}}
{\myaddress}{\res}{\operatorname{res}}      
{\myaddress}{\rng}{\operatorname{ran}}      
{\myaddress}{\sflow}{\operatorname{sf}}     
{\myaddress}{\isf}{\operatorname{isf}}      
{\myaddress}{\sign}{\operatorname{sign}}    
{\myaddress}{\sgn}{\operatorname{sgn}}      
{\myaddress}{\sing}{\operatorname{sing}}    
{\myaddress}{\supp}{\operatorname{supp}}    
{\myaddress}{\tr}{\operatorname{tr}}        
{\myaddress}{\var}{\operatorname{var}}      
{\myaddress}{\vol}{\operatorname{vol}}      
{\myaddress}{\wn}{\operatorname{wn}}        
{\myaddress}{\wres}{\operatorname{wres}}    
\rndef{\Im}{\operatorname{Im}}       
\rndef{\Re}{\operatorname{Re}}       

{\myaddress}{\prng}[1]{\mathrm R_{#1}} 
{\myaddress}{\pker}[1]{\mathrm N_{#1}} 
{\myaddress}{\rprng}[2]{\mathrm R_{#1}^{#2}}           
{\myaddress}{\rpker}[2]{\mathrm N_{#1}^{#2}}           
{\myaddress}{\rsupp}[1]{\supp_r(#1)}
{\myaddress}{\lsupp}[1]{\supp_l(#1)}
{\myaddress}{\rslv}[1]{R_z(#1)}      
{\myaddress}{\HH}{H}                 
{\myaddress}{\tHH}{\tilde \HH}       
{\myaddress}{\VV}{V}                 
{\myaddress}{\Rz}{R_z}               
{\myaddress}{\tRz}{\tR_z}            
{\myaddress}{\psif}[1]{#1^{[1]}} 
{\myaddress}{\WPlus}[1]{W_{#1}(\mbR)} 

{\myaddress}{\bndl}{\xi}                         
{\myaddress}{\bndlA}{\eta}                       
{\myaddress}{\GlueMap}{\varphi}                  
{\myaddress}{\ChartMap}{h}                       
{\myaddress}{\chern}{\ensuremath{\mathrm{ch}}}
{\myaddress}{\hilb}{\clH}                     
{\myaddress}{\hilba}{\clH^{(a)}}                    
{\myaddress}{\hilbs}{\clH^{(s)}}                    
   {\myaddress}{\hilbasargument}{(\hilb)} 
{\myaddress}{\LpH}[1]{\clL_{#1}\hilbasargument}       
{\myaddress}{\saLpH}[1]{\clL_{sa}^{#1}\hilbasargument}       
{\myaddress}{\clBH}{\clB\hilbasargument}              
{\myaddress}{\ubBH}{\clB_1\hilbasargument}            
{\myaddress}{\clCH}{\clC\hilbasargument}              
{\myaddress}{\clKH}{\clK\hilbasargument}              
{\myaddress}{\clFH}{\clF\hilbasargument}              
{\myaddress}{\clUH}{\clU\hilbasargument}              
{\myaddress}{\clCFH}{{\clC\clF}\hilbasargument}       
{\myaddress}{\saBH}{\clB_{sa}\hilbasargument}         
{\myaddress}{\saCH}{\clC_{sa}\hilbasargument}         
{\myaddress}{\saFH}{\clF_{sa}\hilbasargument}         
{\myaddress}{\saKH}{\clK_{sa}\hilbasargument}         
{\myaddress}{\saCFH}{\clC\clF_{sa}\hilbasargument}    
{\myaddress}{\clUFH}{\clU\clF\hilbasargument}         
{\myaddress}{\Uinj}{\clU_{inj}\hilbasargument}        
{\myaddress}{\UFinj}{\clU\clF_{inj}\hilbasargument}   

{\myaddress}{\spproj}[2]{E^{#1}_{#2}}                      
{\myaddress}{\spprojb}[2]{E^{#2}_{#1}}                     

{\myaddress}{\LpN}[1]{\clL^{#1}(\clN,\tau)}     
{\myaddress}{\saLpN}[1]{\clL^{#1}_{sa}(\clN,\tau)} 
{\myaddress}{\rLpN}[1]{L^{#1}(\clN,\tau)}       
{\myaddress}{\clAND}{(\clA,\clN,D)}             
{\myaddress}{\clBA}{{\clB(\clA)}}
{\myaddress}{\saKN}{{\clK_{sa}(\clN,\tau)}}          
{\myaddress}{\clKN}{{\clK(\clN,\tau)}}          
{\myaddress}{\clKtN}{{\clK(\tilde\clN,\tau)}}   
{\myaddress}{\clFN}{{\clF(\clN,\tau)}}          
{\myaddress}{\saFN}{{\clF_{sa}(\clN,\tau)}}     
{\myaddress}{\clPN}{\clP(\clN)}                 
{\myaddress}{\clQN}{\clQ(\clN,\tau)}            
{\myaddress}{\infPN}{{\clP_\tau^\infty(\clN)}}  
{\myaddress}{\clOF}[2]{\clF_{#1\mbox{-}#2}(\clN,\tau)}         
{\myaddress}{\oind}[2]{{\rm \tau\mbox{-}ind}_{#1\mbox{-}#2}}   
{\myaddress}{\tind}{\tau\mbox{-}\ind}                  
{\myaddress}{\DInd}{\ind_{\clD,\tau}}           
{\myaddress}{\BF}{Breuer-Fredholm}              
{\myaddress}{\skewfred}[2]{$(#1\cdot #2)$ $\tau$\tire Fredholm}   
{\myaddress}{\affl}{\eta}                       
{\myaddress}{\vNa}{von Neumann algebra}         
{\myaddress}{\nsf}{faithful normal semifinite } 
{\myaddress}{\taubrs}[1]{\tau\brackets{#1}}     
{\myaddress}{\sqbrs}[1]{[#1]}        
{\myaddress}{\Sqbrs}[1]{\big[#1\big]}        
{\myaddress}{\SqBrs}[1]{\Big[#1\Big]}        

{\myaddress}{\domd}{\bigcap\limits_{n\ge 0} \dom\;\delta^n}         
{\myaddress}{\DiffOP}{{\rm \clD}}
{\myaddress}{\ADA}{\clA \cup [\clD,\clA]}
{\myaddress}{\DixIdeal}[1]{\LpH{#1,\infty}}               
{\myaddress}{\dixideal}{\ell^{1,\infty}}                  
{\myaddress}{\WDixIdeal}{\LpH{1,\mathrm w}}               
{\myaddress}{\DixIdealPos}[1]{\DixIdeal{#1}_+}            
{\myaddress}{\DixIdealN}[1]{\LpN{#1,\infty}}              
{\myaddress}{\DixIdealNPar}[2]{\clL^{#1,\infty}_{#2}(\clN,\tau)}    
{\myaddress}{\DixIdealNPos}[1]{\LpN{#1,\infty}_+}                   
{\myaddress}{\TrD}{\Tr_\omega}                                      
{\myaddress}{\tauD}{{\tau_\omega}}                                  
{\myaddress}{\ILogN}{\frac 1{\log(1+N)}}
{\myaddress}{\DixNorm}[1]{\norm{#1}_{(1,\infty)}}                   
{\myaddress}{\DixInt}[1]{\ints 0^t \mu_s(#1)\,ds}
{\myaddress}{\DixIntL}[1]{\ints 0^{\lambda_{1/t}(#1)}\mu_s(#1)\,ds}
    {\myaddress}{\SmallIdeal}{{\clL^{1, \mathrm w}}}
    {\myaddress}{\SmallIdealMeas}{{\clL^{1, \mathrm w}_m}}
    {\myaddress}{\DixIntII}[1]{\int_0^t \mu_s(#1)\,ds}
    {\myaddress}{\DixIntf}[1]{\Phi_t(#1)}
    {\myaddress}{\DixIntg}[1]{\Psi_t(#1)}

{\myaddress}{\lpi}{\clL^{1,\pi}(\clN,\tau)}

{\myaddress}{\strl}[1]{\stackrel \longrightarrow {#1}}
{\myaddress}{\IIinfty}{$\mathrm{II}_\infty$\ }

{\myaddress}{\fourier}[1]{\clF(#1)}          
{\myaddress}{\HaarMeasBohrs}{\nu}            
{\myaddress}{\BrownsMeas}{\mu}               
{\myaddress}{\BohrCont}[1]{\tilde{#1}}       
{\myaddress}{\APMean}{{M}}                   
{\myaddress}{\CDSS}{{\clA_B}}                
{\myaddress}{\matr}{{\rm Mat}}               
{\myaddress}{\seque}[1]{\ensuremath{\{#1_n\}_{n=1}^\infty}}    
{\myaddress}{\sequen}[2]{\ensuremath{\{#1_#2\}_{#2=1}^\infty}}    
{\myaddress}{\Seque}[1]{\ensuremath{\left(#1_0,#1_1,#1_2,\dots\right)}}    
{\myaddress}{\Cesaro}{H}                           
{\myaddress}{\CesaroRPlus}{M}                      
{\myaddress}{\Dilation}{D}                         
{\myaddress}{\Shift}{T}                            

{\myaddress}{\norm}[1]{\left\Vert#1\right\Vert}    
{\myaddress}{\TrNorm}[1]{\norm{#1}_1}              
{\myaddress}{\HSNorm}[1]{\norm{#1}_2}              
{\myaddress}{\InftyNorm}[1]{\norm{#1}_\infty}      
{\myaddress}{\normQN}[1]{\norm{#1}_{\clQN}}        
{\myaddress}{\clLpnorm}[2]{\norm{#2}_{\clL^{#1}}}    
{\myaddress}{\clLnorm}[1]{\clLpnorm{1}{#1}}    

{\myaddress}{\ccurve}{\gamma}                      

{\myaddress}{\abs}[1]{\left\lvert#1\right\rvert}   
{\myaddress}{\set}[1]{\left\{#1\right\}}           
{\myaddress}{\brackets}[1]{\left(#1\right)}        
{\myaddress}{\brs}[1]{\brackets{#1}}               
{\myaddress}{\Brs}[1]{\big(#1\big)}                
{\myaddress}{\BRS}[1]{\Big(#1\Big)}                
{\myaddress}{\scal}[2]{\left\langle #1,#2\right\rangle}               
{\myaddress}{\precprec}{\prec\!\!\!\prec}
{\myaddress}{\qeq}{\stackrel?=}
{\myaddress}{\spectrum}[1]{\sigma_{#1}} 
{\myaddress}{\spectruma}[1]{\sigma^{(a)}_{#1}} 
{\myaddress}{\numrange}[1]{\mathrm{W}(#1)}                         
\rndef{\emptyset}{\varnothing}                              
{\myaddress}{\csupp}{c}                           
{\myaddress}{\closure}[1]{\overline{#1}}
{\myaddress}{\linspan}[1]{\mathrm{span}\ {#1}}
{\myaddress}{\bddborel}[1]{B(#1)}                 
{\myaddress}{\charfunc}{\chi}
{\myaddress}{\FrDer}{\euD}                        
{\myaddress}{\LieDer}[1]{\pounds_{#1}\,}          
{\myaddress}{\dds}{\left.\frac d{ds} \right|_{s = 0}}
{\myaddress}{\ortcmp}[1]{#1^{\scriptscriptstyle \perp}}            
{\myaddress}{\Laplace}{\Delta}                    

{\myaddress}{\matrPQ}[3]
{
    \left(
      \begin{array}{cc}
        #1_{11} & #1_{12} \\
        #1_{21} & #1_{22}
      \end{array}
    \right)_{[#2,#3]}
}

{\myaddress}{\margOK}{\marginpar{\bf \small OK}}

\newcounter{margcomcount}
\setcounter{margcomcount}{0}
{\myaddress}{\margcom}[1]{\marginpar{\bf \small #1} \addtocounter{margcomcount}{1}
   \index{\indexcom{{\bf COMMENT: #1}}}}

\newcounter{margproof}
\setcounter{margproof}{0}
{\myaddress}{\margproof}{\marginpar{\bf \small PROOF} \addtocounter{margproof}{1}
  \index{**** \indexcom{{\bf PROOF}}}}

\newcounter{margdetails}
\setcounter{margdetails}{0}
{\myaddress}{\margdetails}{\marginpar{\bf Details} \addtocounter{margdetails}{1}
  \index{**** \indexcom{{\bf DETAILS}}}}

\newcounter{margproofb}
\setcounter{margproofb}{0}
{\myaddress}{\margproofb}[1]{\marginpar{\bf \small Proof(B) #1} \addtocounter{margproofb}{1}
  \index{**** \indexcom{{\bf PROOF(B): #1}}}}

\newcounter{margdetailsb}
\setcounter{margdetailsb}{0}
{\myaddress}{\margdetailsb}[1]{\marginpar{\bf \small Details(B)} \addtocounter{margdetailsb}{1}
  \index{**** \indexcom{{\bf DETAILS(B): \\ #1}}}}

\newcounter{margdetailsc}
\setcounter{margdetailsc}{0}
{\myaddress}{\margdetailsc}[1]{\marginpar{\bf \small Details(C)} \addtocounter{margdetailsc}{1}
  \index{**** \indexcom{{\bf DETAILS(C): \\ #1}}}}

\newcounter{margcomcountb}
\setcounter{margcomcountb}{0}
{\myaddress}{\margcomb}[1]{\marginpar{\bf \small #1} \addtocounter{margcomcountb}{1}
   \index{\indexcom{{\bf COMMENT(B): \\ #1}}}}

{\myaddress}{\mytimes}{\!\times\!}
{\myaddress}{\sss}[1]{\subsubsection{}\label{#1}}
\let\myphi\phi
\rndef{\phi}{\varphi} {\myaddress}{\OpenUnitDisk}{D}
{\myaddress}{\RHS}{RHS}                            
{\myaddress}{\LHS}{LHS} 
{\myaddress}{\ttt}{\Leftrightarrow}
{\myaddress}{\then}{\Rightarrow}
{\myaddress}{\tto}{\longrightarrow}
{\myaddress}{\nno}{\nonumber\\}
{\myaddress}{\newn}[1]{\index{#1} {\bfseries #1}}       
{\myaddress}{\la}{\langle}
{\myaddress}{\ra}{\rangle}
{\myaddress}{\dbar}{{\;\bar{\phantom{o}} \!\!\!\! d}}
{\myaddress}{\stl}[1]{\stackrel{\vbox to 0pt{\vss\hbox{$\scriptstyle #1$}}}}
{\myaddress}{\mathcomment}[1]{{\hfill \qquad\qquad\qquad\text{by (#1)}}}        
{\myaddress}{\mathcomm}[1]{{\hfill \qquad\qquad\qquad\qquad\qquad\text{#1}}}        
{\myaddress}{\details}[1]{\smallskip\begin{center} {\bf Here:}
#1\end{center}\medskip} {\myaddress}{\indexcom}[1]{ --- #1}
{\myaddress}{\longsim}{\ \sim \ }              
{\myaddress}{\tire}{-}              
{\myaddress}{\intinfinf}{\int_{-\infty}^\infty}
     {\myaddress}{\npartial}{\slash\!\!\!\partial}
     {\myaddress}{\Heis}{\operatorname{Heis}}
     {\myaddress}{\Solv}{\operatorname{Solv}}
     {\myaddress}{\Spin}{\operatorname{Spin}}
     {\myaddress}{\SO}{\operatorname{SO}}
     
     {\myaddress}{\Index}{\operatorname{index}}
     
     
     

             
             {\myaddress}{\p}{\partial}
             {\myaddress}{\dd}{|\clD|}
             {\myaddress}{\n}{\parallel}

     
     

\let\LatexCite={\futurelet\NChar\CleverCite}  

\let\ifnumref\iffalse 

{\myaddress}{\ifuncited}[4]{\expandafter\ifx\csname used#4\endcsname\relax}

{\myaddress}{\ifcited}[4]{\expandafter\ifx\csname used#4\endcsname\relax\else}

  {\myaddress}{\papertitle}[1]{ \emph{#1}, }
  {\myaddress}{\paperauthor}[2]{#2}  
  {\myaddress}{\pbbi}[9]{      \ifcited{#1}{#2}{#3}{#5}        \ifnumref          \bibitem{#5}\paperauthor{#1}{#6},\papertitle{#7}#8.        \else          \advance #9 by 1          \ifnum#9<1            \bibitem[#4]{#5}\paperauthor{#1}{#6}, \papertitle{#7}#8.          \else            \bibitem[#4$_{\the#9}$]{#5}\paperauthor{#1}{#6},\papertitle{#7}#8.          \fi        \fi      \fi  }
  {\myaddress}{\mbbi}[8]{     \ifcited{#1}{#2}{#3}{#5}        \ifnumref          \bibitem{#5}\paperauthor{#1}{#6},\papertitle{#7}#8.        \else          \bibitem[#4]{#5}\paperauthor{#1}{#6},\papertitle{#7}#8.        \fi     \fi  }

{\myaddress}{\AddCite}[1]{   \ifuncited{0}{0}{0}{#1}     \expandafter   \fi}

{\myaddress}{\AddCites}[1]{\relax,}

{\myaddress}{\CiteWithoutExtension}[1]{   \AddCites{#1}   \LatexCite{#1}}

{\myaddress}{\CleverCite}{    \ifx\NChar[        \let\MyCite=\CiteWithExtension     \else        \let\MyCite=\CiteWithoutExtension     \fi     \MyCite}

      {\myaddress}{\volume}[1]{{\bf #1}}
      {\myaddress}{\VolYearPP}[3]{\ifnum#2=0 (to appear)\else\volume{#1} (#2), #3\fi}
      {\myaddress}{\VolNoYearPP}[4]{\ifnum#3=0 (to appear)\else\volume{#1} #2 (#3), #4\fi}
      {\myaddress}{\libcode}[1]{}

{\myaddress}{\jnActaMath}[3]{Acta Math. \VolYearPP{#1}{#2}{#3}}                       
{\myaddress}{\jnAdvMath}[3]{Adv. in~Math. \VolYearPP{#1}{#2}{#3}}                     
{\myaddress}{\jnAlgAnal}[3]{Algebra i~Analiz \VolYearPP{#1}{#2}{#3}}
{\myaddress}{\jnAmerJMath}[3]{Amer. J. Math. \VolYearPP{#1}{#2}{#3}}                  
{\myaddress}{\jnAmerMathMonth}[3]{Amer. Math. Monthly \VolYearPP{#1}{#2}{#3}}         
{\myaddress}{\jnAnnMath}[4]{Ann. of~Math. \VolNoYearPP{#1}{#2}{#3}{#4}}               
{\myaddress}{\jnAnalMath}[3]{J. Anal. Math. \VolYearPP{#1}{#2}{#3}}                   
{\myaddress}{\jnArchRatMechAnal}[3]{Arch. Rational Mech. Anal. \VolYearPP{#1}{#2}{#3}}                   
{\myaddress}{\jnBullLondMathSoc}[3]{Bull. London Math. Soc. \VolYearPP{#1}{#2}{#3}}   
{\myaddress}{\jnBullAMS}[3]{Bull. Amer. Math. Soc. \VolYearPP{#1}{#2}{#3}}   
{\myaddress}{\jnCanMathBull}[3]{Canad. Math. Bull. \VolYearPP{#1}{#2}{#3}}            
{\myaddress}{\jnCanMath}[3]{Canad. J.~Math. \VolYearPP{#1}{#2}{#3}}             
{\myaddress}{\jnCommMathPhys}[3]{Comm. Math. Phys. \VolYearPP{#1}{#2}{#3}}             
{\myaddress}{\jnCommPDE}[3]{Comm. Partial Differential Equations \VolYearPP{#1}{#2}{#3}}             
{\myaddress}{\jnComptRendue}[3]{C.\,R.~Acad. Sci. Paris S\'er. A-B \VolYearPP{#1}{#2}{#3}}      
{\myaddress}{\jnContMath}[3]{Contemporary Math. \VolYearPP{#1}{#2}{#3}}               {\myaddress}{\jnDukeMJ}[3]{Duke Math. J. \VolYearPP{#1}{#2}{#3}}
{\myaddress}{\jnDiffGeom}[3]{J.~Diff. Geom. \VolYearPP{#1}{#2}{#3}}                   
{\myaddress}{\jnErgodicTheory}[3]{Ergodic Theory and Dynamical Systems \VolYearPP{#1}{#2}{#3}} 
{\myaddress}{\jnFuncAnal}[3]{J.~Functional Analysis \VolYearPP{#1}{#2}{#3}}           
{\myaddress}{\jnFunkAnalPril}[4]{Funct. Anal. Appl. \VolNoYearPP{#1}{#2}{#3}{#4}}  
{\myaddress}{\jnGAFA}[3]{GAFA \VolYearPP{#1}{#2}{#3}}                                 
{\myaddress}{\jnIHES}[3]{IHES Publ. Math. (Paris) \VolYearPP{#1}{#2}{#3}}             
{\myaddress}{\jnIEOT}[3]{Integral Equations Operator Theory   \VolYearPP{#1}{#2}{#3}} 
{\myaddress}{\jnIsrMath}[3]{Israel J.~Math. \VolYearPP{#1}{#2}{#3}}                   
{\myaddress}{\jnKTheory}[3]{K-Theory \VolYearPP{#1}{#2}{#3}}                          
{\myaddress}{\jnLetMathPhys}[3]{Lett. Math. Phys. \VolYearPP{#1}{#2}{#3}}             
{\myaddress}{\jnMathAnn}[3]{Math. Ann. \VolYearPP{#1}{#2}{#3}}                        
{\myaddress}{\jnMathAnalAppl}[3]{J.~Math. Anal. and Appl. \VolYearPP{#1}{#2}{#3}}     
{\myaddress}{\jnMathNachr}[3]{Math. Nachr. \VolYearPP{#1}{#2}{#3}}
{\myaddress}{\jnMathPhys}[3]{J. Math. Phys. \VolYearPP{#1}{#2}{#3}}
{\myaddress}{\jnMathSocJap}[3]{J. Math. Soc. Japan \VolYearPP{#1}{#2}{#3}}
{\myaddress}{\jnOperTheory}[3]{J.~Operator Theory \VolYearPP{#1}{#2}{#3}}             
{\myaddress}{\jnPacJMath}[3]{Pacific J.~Math. \VolYearPP{#1}{#2}{#3}}                  
{\myaddress}{\jnPositivity}[3]{Positivity \VolYearPP{#1}{#2}{#3}}
{\myaddress}{\jnProcAmerMS}[3]{Proc. Amer. Math. Soc. \VolYearPP{#1}{#2}{#3}}         
{\myaddress}{\jnProcCambPhilSoc}[3]{Math. Proc. Camb. Phil. Soc. \VolYearPP{#1}{#2}{#3}}
{\myaddress}{\jnReineAngew}[3]{J.~Reine Angew. Math. \VolYearPP{#1}{#2}{#3}}          
{\myaddress}{\jnTokyoMath}[3]{Tokyo J.~Math. \VolYearPP{#1}{#2}{#3}}
{\myaddress}{\jnTopology}[3]{Topology \VolYearPP{#1}{#2}{#3}}
{\myaddress}{\jnTransAmerMathSoc}[3]{Trans. Amer. Math. Soc. \VolYearPP{#1}{#2}{#3}}
{\myaddress}{\jnIzvANSSSR}[3]{Izv. Akad. Nauk SSSR, Ser. Mat. \VolYearPP{#1}{#2}{#3}}
{\myaddress}{\jnIzvVyshUchZav}[3]{Izv. Vyssh. Uch. Zav., Mat. \VolYearPP{#1}{#2}{#3} (Russian)}
{\myaddress}{\jnIzdatLenUniv}[2]{Izdat. Leningrad. Univ., Leningrad, (#1), #2 (Russian)}
{\myaddress}{\jnFieldsInsComm}[3]{Fields Inst. Comm. \VolYearPP{#1}{#2}{#3}}
{\myaddress}{\jnDoklANSSSR}[3]{Dokl. Akad. Nauk SSSR \VolYearPP{#1}{#2}{#3}}
{\myaddress}{\jnMatZametki}[3]{Matem. zametki \VolYearPP{#1}{#2}{#3}}
{\myaddress}{\jnRussMathSurvey}[3]{Russian Math. Surveys \VolYearPP{#1}{#2}{#3}}
{\myaddress}{\jnSibMathJ}[3]{Sib. Math.~J. \VolYearPP{#1}{#2}{#3}}
{\myaddress}{\jnSovMath}[3]{J.~Soviet math. \VolYearPP{#1}{#2}{#3}}
{\myaddress}{\jnTransMoscMathSoc}[3]{Trans. Moscow Math. Soc. \VolYearPP{#1}{#2}{#3}}
{\myaddress}{\jnUMN}[3]{Uspekhi Mat. Nauk \VolYearPP{#1}{#2}{#3}}

{\myaddress}{\bkTransMathMon}[2]{Trans. Math. Monographs, AMS, \volume{#1}, #2}

{\myaddress}{\pbBirkhauser}[1]{Birkh\"auser, Boston, #1}
{\myaddress}{\pbFactorial}[1]{Moscow, Factorial, #1}
{\myaddress}{\pbGauthier}[1]{Gauthier-Villars, Paris, #1}
{\myaddress}{\pbNauka}[1]{Moscow, Nauka, #1 (Russian)}
{\myaddress}{\pbNaukaR}[1]{Ìîñêâà, Íàóêà, #1}
{\myaddress}{\pbPrinceton}[1]{Princeton University Press, Princeton, New Jersey, #1}
{\myaddress}{\pbPublPerish}[1]{Publish or Perish Inc., Berkeley, #1}
{\myaddress}{\pbSpringer}[1]{Springer-Verlag, #1}

{\myaddress}{\myauthor}[1]{\mbox{#1}}

{\myaddress}{\Agmon}{\myauthor{Sh.\,Agmon}}
{\myaddress}{\Ahiezer}{\myauthor{N.\,I.\,Ahiezer}}
{\myaddress}{\Arazy}{\myauthor{J.\,Arazy}}
{\myaddress}{\Aronszajn}{\myauthor{N.\,Aronszajn}}
{\myaddress}{\Astashkin}{\myauthor{S.\,V.\,Astashkin}}
{\myaddress}{\Atiyah}{\myauthor{M.\,Atiyah}}
{\myaddress}{\Avron}{\myauthor{J.\,E.\,Avron}}
{\myaddress}{\Azamov}{\myauthor{N.\,A.\,Azamov}}
{\myaddress}{\Banach}{\myauthor{S.\,Banach}}
{\myaddress}{\Benameur}{\myauthor{M-T.\,Benameur}}
{\myaddress}{\Bennett}{\myauthor{C.\,Bennett}}
{\myaddress}{\Berezin}{\myauthor{F.\,A.\,Berezin}}
{\myaddress}{\Berline}{\myauthor{N.\,Berline}}
{\myaddress}{\Birman}{\myauthor{M.\,Sh.\,Birman}}
{\myaddress}{\Blackadar}{\myauthor{B.\,Blackadar}}
{\myaddress}{\Bogolyubov}{\myauthor{N.\,N.\,Bogolyubov}}
{\myaddress}{\Bonsall}{\myauthor{F.\,F.\,Bonsall}}
{\myaddress}{\Bony}{\myauthor{J.\,F.\,Bony}}
{\myaddress}{\BoosBavnbek}{\myauthor{B.\,Boo$\beta$-Bavnbek}}
{\myaddress}{\Bott}{\myauthor{R.\,Bott}}
{\myaddress}{\Branges}{\myauthor{L.\,de Branges}}
{\myaddress}{\Bratteli}{\myauthor{O.\,Bratteli}}
{\myaddress}{\Bredon}{\myauthor{G.\,E.\,Bredon}}
{\myaddress}{\Breuer}{\myauthor{M.\,Breuer}}
{\myaddress}{\Brown}{\myauthor{L.\,G.\,Brown}}
{\myaddress}{\Bruneau}{\myauthor{V.\,Bruneau}}
{\myaddress}{\Buslaev}{\myauthor{V.\,S.\,Buslaev}}
{\myaddress}{\Carey}{\myauthor{A.\,L.\,Carey}}
{\myaddress}{\CareyRW}{\myauthor{R.\,W.\,Carey}} 
{\myaddress}{\Cartan}{\myauthor{H.\,Cartan}}
{\myaddress}{\Chilin}{\myauthor{V.\,I.\,Chilin}}
{\myaddress}{\Coburn}{\myauthor{L.\,A.\,Coburn}}
{\myaddress}{\Connes}{\myauthor{A.\,Connes}}
{\myaddress}{\Cornfeld}{\myauthor{I.\,P.\,Cornfeld}}
{\myaddress}{\Daletskii}{\myauthor{Yu.\,L.\,Daletski\u\i}}   
{\myaddress}{\Dixmier}{\myauthor{J.\,Dixmier}}
{\myaddress}{\DoddsPG}{\myauthor{P.\,G.\,Dodds}}
{\myaddress}{\DoddsTK}{\myauthor{T.\,K.\,Dodds}}
{\myaddress}{\Douglas}{\myauthor{R.\,G.\,Douglas}}
{\myaddress}{\Dubrovin}{\myauthor{B.\,A.\,Dubrovin}}
{\myaddress}{\Dugundji}{\myauthor{J.\,Dugundji}}
{\myaddress}{\Duncan}{\myauthor{J.\,Duncan}}
{\myaddress}{\Dunford}{\myauthor{N.\,Dunford}}
{\myaddress}{\Dykema}{\myauthor{K.\,J.\,Dykema}}
{\myaddress}{\Edwards}{\myauthor{R.\,E.\,Edwards}}
{\myaddress}{\Eilenberg}{\myauthor{S.\,Eilenberg}}
{\myaddress}{\Entina}{\myauthor{S.\,B.\,\`Entina}}
{\myaddress}{\Fack}{\myauthor{T.\,Fack}} 
{\myaddress}{\Faddeev}{\myauthor{L.\,D.\,Faddeev}}
{\myaddress}{\Farber}{\myauthor{M.\,Farber}}
{\myaddress}{\Farforovskaya}{\myauthor{Yu.\,B.\,Farforovskaya}}
{\myaddress}{\Federer}{\myauthor{H.\,Federer}}
{\myaddress}{\Fedosov}{\myauthor{B.\,V.\,Fedosov}}
{\myaddress}{\Figiel}{\myauthor{T.\,Figiel}} 
{\myaddress}{\Figueroa}{\myauthor{H.\,Figueroa}}
{\myaddress}{\Fillmore}{\myauthor{P.\,A.\,Fillmore}}
{\myaddress}{\Fomenko}{\myauthor{A.\,T.\,Fomenko}} 
{\myaddress}{\Fomin}{\myauthor{S.\,V.\,Fomin}}
{\myaddress}{\Frohlich}{\myauthor{J.\,Fr\"ohlich}}
{\myaddress}{\Fuglede}{\myauthor{B.\,Fuglede}}
{\myaddress}{\Furutani}{\myauthor{K.\,Furutani}}
{\myaddress}{\Gelfand}{\myauthor{I.\,M.\,Gelfand}}
{\myaddress}{\Gesztesy}{\myauthor{F.\,Gesztesy}}     
{\myaddress}{\Getzler}{\myauthor{E.\,Getzler}} 
{\myaddress}{\Gilkey}{\myauthor{P.\,B.\,Gilkey}}
{\myaddress}{\Gitler}{\myauthor{S.\,Gitler}}
{\myaddress}{\Glazman}{\myauthor{I.\,M.\,Glazman}}
{\myaddress}{\Glimm}{\myauthor{J.\,Glimm}}
{\myaddress}{\Gohberg}{\myauthor{I.\,C.\,Gohberg}}
{\myaddress}{\Goldshtein}{\myauthor{Ya.\,Goldshtein}}
{\myaddress}{\Golze}{\myauthor{F.\,Golze}}
{\myaddress}{\GraciaBondia}{\myauthor{J.\,M.\,Gracia-Bond\'{i}a}}
{\myaddress}{\Greenleaf}{\myauthor{F.\,P.\,Greenleaf}}
{\myaddress}{\Gromov}{\myauthor{M.\,Gromov}}
{\myaddress}{\Gunning}{\myauthor{R.\,C.\,Gunning}}
{\myaddress}{\Haagerup}{\myauthor{U.\,Haagerup}}
{\myaddress}{\Haag}{\myauthor{R.\,Haag}}
{\myaddress}{\Halmos}{\myauthor{P.\,R.\,Halmos}}
{\myaddress}{\Hardy}{\myauthor{G.\,H.\,Hardy}}
{\myaddress}{\Herbst}{\myauthor{I.\,W.\,Herbst}}
{\myaddress}{\Higson}{\myauthor{N.\,Higson}}  
{\myaddress}{\Hoermander}{\myauthor{L.\,Hoermander}} 
{\myaddress}{\Hoffman}{\myauthor{K.\,Hoffman}} 
{\myaddress}{\Ito}{\myauthor{K.\,Ito}}
{\myaddress}{\Ikebe}{\myauthor{T.\,Ikebe}}
{\myaddress}{\Jaffe}{\myauthor{A.\,Jaffe}}
{\myaddress}{\James}{\myauthor{I.\,M.\,James}}
{\myaddress}{\Javrjan}{\myauthor{V.\,A.\,Javrjan}}
{\myaddress}{\Jitomirskaya}{\myauthor{S.\,Jitomirskaya}}
{\myaddress}{\Kadison}{\myauthor{R.\,V.\,Kadison}}
{\myaddress}{\Kalton}{\myauthor{N.\,J.\,Kalton}} 
{\myaddress}{\Kato}{\myauthor{T.\,Kato}} 
{\myaddress}{\Kobayashi}{\myauthor{S.\,Kobayashi}}
{\myaddress}{\Koplienko}{\myauthor{L.\,S.\,Koplienko}}
{\myaddress}{\Korotyaev}{\myauthor{E.\,Korotyaev}}
{\myaddress}{\Kosaki}{\myauthor{H.\,Kosaki}}
{\myaddress}{\Kostrykin}{\myauthor{V.\,Kostrykin}}
{\myaddress}{\Kotani}{\myauthor{S.\,Kotani}}
{\myaddress}{\Krein}{\myauthor{Kre\u\i n}}
{\myaddress}{\KreinMG}{\myauthor{M.\,G.\,Kre\u\i n}}
{\myaddress}{\KreinSG}{\myauthor{S.\,G.\,Kre\u\i n}}
{\myaddress}{\Kuroda}{\myauthor{S.\,T.\,Kuroda}}
{\myaddress}{\Leichtnam}{\myauthor{E.\,Leichtnam}}
{\myaddress}{\Lesch}{\myauthor{M.\,Lesch}}
{\myaddress}{\Lesniewski}{\myauthor{A.\,Lesniewski}}
{\myaddress}{\Levitan}{\myauthor{B.\,M.\,Levitan}}
{\myaddress}{\Lidskii}{\myauthor{V.\,B.\,Lidskii}}
{\myaddress}{\Lifshitz}{\myauthor{I.\,M.\,Lifshitz}}
{\myaddress}{\Lindenstrauss}{\myauthor{J.\,Lindenstrauss}}
{\myaddress}{\Loday}{\myauthor{J.-L.\,Loday}}
{\myaddress}{\Lord}{\myauthor{S.\,Lord}}      
{\myaddress}{\Lorentz}{\myauthor{G.\,Lorentz}}
{\myaddress}{\Magnus}{\myauthor{W.\,Magnus}}
{\myaddress}{\Makarov}{\myauthor{K.\,A.\,Makarov}}
{\myaddress}{\MakarovN}{\myauthor{N.\,Makarov}}
{\myaddress}{\Mathai}{\myauthor{V.\,Mathai}}         
{\myaddress}{\McKean}{\myauthor{H.\,P.\,McKean}}
{\myaddress}{\Mishchenko}{\myauthor{A.\,S.\,Mishchenko}}
{\myaddress}{\Molchanov}{\myauthor{S.\,A.\,Molchanov}}
{\myaddress}{\Moore}{\myauthor{C.\,C.\,Moore}}
{\myaddress}{\Moscovici}{\myauthor{H.\,Moscovici}}  
{\myaddress}{\Motovilov}{\myauthor{A.\,K.\,Motovilov}}
{\myaddress}{\Moyer}{\myauthor{R.\,D.\,Moyer}}
{\myaddress}{\Naboko}{\myauthor{S.\,N.\,Naboko}}
{\myaddress}{\Narasimhan}{\myauthor{R.\,Narasimhan}}
{\myaddress}{\Nomizu}{\myauthor{K.\,Nomizu}}
{\myaddress}{\Novikov}{\myauthor{S.\,P.\,Novikov}}
{\myaddress}{\Osterwalder}{\myauthor{K.\,Osterwalder}}
{\myaddress}{\Patodi}{\myauthor{V.\,Patodi}}
{\myaddress}{\Pagter}{\myauthor{B.\,de~Pagter}}  
{\myaddress}{\Pastur}{\myauthor{L.\,A.\,Pastur}}  
{\myaddress}{\Pavlov}{\myauthor{B.\,S.\,Pavlov}}
{\myaddress}{\Pedersen}{\myauthor{G.\,K.\,Pedersen}}
{\myaddress}{\Peller}{\myauthor{V.\,V.\,Peller}}
{\myaddress}{\Perera}{\myauthor{V.\,S.\,Perera}}
{\myaddress}{\Petunin}{\myauthor{Ju.\,I.\,Petunin}}
{\myaddress}{\Phillips}{\myauthor{J.\,Phillips}}  
{\myaddress}{\Piazza}{\myauthor{P.\,Piazza}}   
{\myaddress}{\Pincus}{\myauthor{J.\,D.\,Pincus}}   
{\myaddress}{\Poincare}{Poincar\'e}
{\myaddress}{\Postnikov}{\myauthor{M.\,M.\,Postnikov}} 
{\myaddress}{\Povzner}{\myauthor{A.\,Ya.\,Povzner}}
{\myaddress}{\Prinzis}{\myauthor{R.\,Prinzis}}
{\myaddress}{\Privalov}{\myauthor{I.\,I.\,Privalov}}
{\myaddress}{\Pushnitski}{\myauthor{A.\,B.\,Pushnitski}} 
{\myaddress}{\Raeburn}{\myauthor{I.\,Raeburn}}
{\myaddress}{\Raikov}{\myauthor{G.\,Raikov}}
{\myaddress}{\Reed}{\myauthor{M.\,Reed}}
{\myaddress}{\Rennie}{\myauthor{A.\,Rennie}}
{\myaddress}{\Rickart}{\myauthor{C.\,E.\,Rickart}}
{\myaddress}{\Riesz}{\myauthor{F.\,Riesz}}
{\myaddress}{\Ringrose}{\myauthor{J.\,Ringrose}}
{\myaddress}{\Rio}{\myauthor{R.\,del Rio}}
{\myaddress}{\Robinson}{\myauthor{D.\,Robinson}}
{\myaddress}{\Rossi}{\myauthor{H.\,Rossi}}
{\myaddress}{\Rudin}{\myauthor{W.\,Rudin}}
{\myaddress}{\Ruelle}{\myauthor{D.\,Ruelle}}
{\myaddress}{\Ruzhansky}{\myauthor{M.\,Ruzhansky}}
{\myaddress}{\Sakai}{\myauthor{Sh.\,Sakai}}
{\myaddress}{\Sargsjan}{\myauthor{I.\,S.\,Sargsjan}}
{\myaddress}{\Sato}{\myauthor{H.\,Sato}}
{\myaddress}{\Schaeffer}{\myauthor{D.\,G.\,Schaeffer}}
{\myaddress}{\Schluchtermann}{\myauthor{G.\,Schluchtermann}}
{\myaddress}{\Schochet}{\myauthor{C.\,Schochet}}
{\myaddress}{\SchroedingerE}{\myauthor{E.\,Schr\"odinger}}
{\myaddress}{\Schroedinger}{\myauthor{Schr\"odinger}}
{\myaddress}{\Schrohe}{\myauthor{E.\,Schrohe}}
{\myaddress}{\Schwartz}{\myauthor{J.\,T.\,Schwartz}}
{\myaddress}{\Sedaev}{\myauthor{A.\,A.\,Sedaev}}
{\myaddress}{\Seiler}{\myauthor{R.\,Seiler}}
{\myaddress}{\Semenov}{\myauthor{E.\,M.\,Semenov}}
{\myaddress}{\Shabat}{\myauthor{B.\,V.\,Shabat}}
{\myaddress}{\Shafarevich}{\myauthor{I.\,R.\,Shafarevich}}
{\myaddress}{\Sharpley}{\myauthor{R.\,Sharpley}}
{\myaddress}{\Shilov}{\myauthor{G.\,E.\,Shilov}}
{\myaddress}{\Shirkov}{\myauthor{D.\,V.\,Shirkov}}
{\myaddress}{\Shubin}{\myauthor{M.\,A.\,Shubin}}
{\myaddress}{\Silverman}{\myauthor{H.\,Silverman}}
{\myaddress}{\Simon}{\myauthor{B.\,Simon}}
{\myaddress}{\Sinai}{\myauthor{Ya.\,G.\,Sinai}}
{\myaddress}{\Singer}{\myauthor{I.\,M.\,Singer}}
{\myaddress}{\Solomyak}{\myauthor{M.\,Z.\,Solomyak}}
{\myaddress}{\Soloviev}{\myauthor{Yu.\,P.\,Soloviev}}
{\myaddress}{\Spivak}{\myauthor{M.\,Spivak}}
{\myaddress}{\Stein}{\myauthor{E.\,M.\,Stein}}
{\myaddress}{\Stenkin}{\myauthor{V.\,V.\,Sten'kin}}
{\myaddress}{\Stratila}{\myauthor{S.\,Stratila}}
{\myaddress}{\Sucheston}{\myauthor{L.\,Sucheston}}
{\myaddress}{\Sukochev}{\myauthor{F.\,A.\,Sukochev}}
{\myaddress}{\Switzer}{\myauthor{R.\,M.\,Switzer}}
{\myaddress}{\SzNagy}{\myauthor{B.\,Sz.-Nagy}}
{\myaddress}{\Takesaki}{\myauthor{M.\,Takesaki}}
{\myaddress}{\Taylor}{\myauthor{M.\,E.\,Taylor}}
{\myaddress}{\Treves}{\myauthor{F.\,Treves}}
{\myaddress}{\Troitsky}{\myauthor{E.\,V.\,Troitsky}}
{\myaddress}{\Tzafriri}{\myauthor{L.\,Tzafriri}}
{\myaddress}{\Varilly}{\myauthor{J.\,C.\,V\'{a}rilly}}
{\myaddress}{\Vergne}{\myauthor{M.\,Vergne}}
{\myaddress}{\Vladimirov}{\myauthor{V.\,S.\,Vladimirov}}
{\myaddress}{\Voiculescu}{\myauthor{D.\,Voiculescu}}
{\myaddress}{\Weiss}{\myauthor{G.\,Weiss}}
{\myaddress}{\Wells}{\myauthor{R.\,O.\,Wells}}
{\myaddress}{\Williams}{\myauthor{J.\,P.\,Williams}}
{\myaddress}{\Winkler}{\myauthor{S.\,Winkler}}
{\myaddress}{\Witten}{\myauthor{E.\,Witten}}
{\myaddress}{\Wodzicki}{\myauthor{M.\,Wodzicki}}
{\myaddress}{\Wojciechowski}{\myauthor{K.\,P.\,Wojciechowski}}
{\myaddress}{\Yafaev}{\myauthor{D.\,R.\,Yafaev}}
{\myaddress}{\Yosida}{\myauthor{K.\,Yosida}}
{\myaddress}{\Zsido}{\myauthor{L.\,Zsido}}

\numberwithin{equation}{section}

 

{\myaddress}{\hlambda}{{\mathfrak h_\lambda}}
\rndef{\iff}{\Leftrightarrow} {\myaddress}{\Rindex}{\euR} \sloppy

\textwidth 16cm \textheight 22cm \oddsidemargin 0cm
\evensidemargin 0cm
\begin{document}
\title{Resonance index and singular $\mu$-invariant}
\author{Nurulla Azamov}
\address{School of Computer Science, Engineering and Mathematics
   \\ Flinders University
   \\ Bedford Park, 5042, SA Australia.}
\email{nurulla.azamov@flinders.edu.au}
 \keywords{Resonance index, $\mu$-invariant, singular spectral shift
 function, scattering matrix}

 \subjclass[2010]{ 
     Primary 47A55, 47A10, 47A70, 47A40;
     Secondary 35P99.
     
     
 }

\begin{abstract} In this paper we give a direct proof of equality of the total resonance index
and of singular part of the $\mu$-invariant under mild
conditions which include $n$-dimensional Schr\"odinger operators. Previously it was proved for trace class perturbations
that each of these two integer-valued functions were equal to the
singular spectral shift function.

The proof is self-contained and is based on application of the Argument Principle from complex analysis
to poles and zeros of analytic continuation of scattering matrix considered as a function of coupling constant.
\end{abstract}

\maketitle

\section*{Introduction}
Let $\lambda$ be a real number. Let~$H_0$ be a self-adjoint
operator on a separable Hilbert space $\hilb$ and let $V$ be a
$H_0$-compact self-adjoint operator on $\hilb$ such that
\begin{enumerate}
  \item $V$ admits a factorization $V = F^*JF,$ where $F \colon \hilb \to \clK$ is a closed operator from $\hilb$
  to another Hilbert space $\clK$ and $J$ is a self-adjoint bounded operator on $\clK;$
  \item The operator $FR_z(H_0)F^*,$ where $R_z(H) = (H-z)^{-1}$ is the resolvent of~$H,$ is compact for some and thus for any complex number $z$ which does not belong to the spectrum of~$H_0;$
  \item The uniform limit
  $$
     T_{\lambda+i0}(H_0) := FR_{\lambda+i0}(H_0)F^*
  $$
  of the sandwiched resolvent $$T_{\lambda+iy}(H_0):=FR_{\lambda+iy}(H_0)F^*$$
  as $y$ approaches $0$ exists, $y \in \mbR.$
\end{enumerate}
Under these conditions one can consider the following operator, which can be interpreted as the scattering matrix for the pair of operators $(H_0,H_r),$ $r \in \mbR,$
(see e.g. {\futurelet\NChar\CleverCite}[Chapter 5, \S 5]{Ya})
\begin{equation} \label{F: stationary formula}
  S(\lambda+i0; H_r,H_0) = 1 - 2ir \sqrt{\Im T_{\lambda+i0}(H_0)} J (1+rT_{\lambda+i0}(H_0)J)^{-1}\sqrt{\Im T_{\lambda+i0}(H_0)},
\end{equation}
where~$H_r = H_0 + rV,$ as long as the operator
$1+rT_{\lambda+i0}(H_0)J$ has bounded inverse. By the analytic
Fredholm alternative, the set $R(\lambda; H_0,V; F)$ of real
numbers~$r$ for which the operator $1+rT_{\lambda+i0}(H_0)J$ is
not invertible is discrete; elements of this set were called in
{\futurelet\NChar\CleverCite}{Az3v6,Az9} \emph{resonance points}.
It is a well-known fact which can also be verified by a simple calculation that for all non-resonant values of the coupling constant $r$ the operator $S(\lambda+i0; H_0,H_r)$ 
is unitary. Moreover, for all $y \in (0,\infty)$ and for all $r \in \mbR$ the operator
$$
  S(\lambda+iy; H_r,H_0) = 1 - 2i \sqrt{\Im T_{\lambda+iy}(H_0)} rJ (1+rT_{\lambda+iy}(H_0)J)^{-1}\sqrt{\Im T_{\lambda+iy}(H_0)}
$$
is also unitary and depends analytically on $y.$ Further, as $y
\to +\infty,$ the operator $S(\lambda+iy; H_0,H_r)$ converges in
uniform topology to the identity operator $1$ and it can be shown that this convergence is
locally uniform with respect to $r$ in~$\mbR.$ 
If the value $r=1$ of the coupling constant is non-resonant then for a fixed number $e^{i\theta}$ on the unit circle, one can count the number of eigenvalues of the scattering matrix
$S(\lambda+iy; H_1,H_0)$ which cross the point $e^{i\theta}$ in clockwise direction as $y$ goes from $0^+$ to $+\infty.$
This number was denoted $\mu(\theta,\lambda; H_1,H_0)$ in {\futurelet\NChar\CleverCite}{Pu01FA} and was called $\mu$-invariant.
The $\mu$-invariant measures spectral flow of eigenvalues of the scattering matrix.
Pushnitski showed in {\futurelet\NChar\CleverCite}{Pu01FA} that for relatively trace-class perturbations $V$
one has the formula
$$
  \xi(\lambda; H_1,H_0) = - \frac {1}{2\pi} \int_0^{2\pi} \mu(\theta, \lambda; H_1,H_0)\,d\lambda, \ \text{a.e.}\ \lambda,
$$
where $\xi(\lambda; H_1,H_0)$ is the Lifshitz-Krein spectral shift function, see e.g. {\futurelet\NChar\CleverCite}[Chapter 8]{Ya}.
There are several formulas for the spectral shift function $\xi(\lambda; H_1,H_0)$ and one of them is the Birman-Solomyak formula
which defines the spectral shift function as distribution by formula
$$
  \xi(\phi) = \int_0^1 \Tr(V\phi(H_r))\,dr
$$
for a test function $\phi.$

In {\futurelet\NChar\CleverCite}{Az2,Az3v6} it was observed that there is another natural way to deform the scattering matrix
$S(\lambda+i0; H_1,H_0)$ continuously to the identity operator: by sending the coupling constant $r$ from~$1$ to~$0.$
Indeed, by analytic Fredholm alternative the operator $S(\lambda+i0; H_r,H_0)$ is a meromorphic function of the coupling constant~$r$ considered as a complex variable.
Since the operator-function $S(\lambda+i0; H_r,H_0)$ is unitary for real values of~$r$ it cannot have poles on the real axis. Thus, it can be continued analytically to the resonant values of~$r$
for which the factor $(1+rT_{\lambda+i0}(H_0)J)^{-1}$ in the stationary formula (\ref{F: stationary formula}) for the scattering matrix is undefined.
This circumstance allows one to measure spectral flow of eigenvalues of the scattering matrix corresponding to the path $\set{S(\lambda+i0; H_r,H_0) \colon r \in [0,1]}.$
This spectral flow was denoted in {\futurelet\NChar\CleverCite}{Az2,Az3v6} by ${\mu^{(a)}}(\theta, \lambda; H_1,H_0)$ and was called the absolutely continuous part of Pushnitski $\mu$-invariant.
This terminology is justified as follows.
For trace class perturbations $V$ it was shown in {\futurelet\NChar\CleverCite}{Az3v6} that
$$
  {\xi^{(a)}}(\lambda; H_1,H_0) = - \frac {1}{2\pi} \int_0^{2\pi} {\mu^{(a)}}(\theta, \lambda; H_1,H_0)\,d\theta, \ \text{a.e.} \ \lambda,
$$
where ${\xi^{(a)}}(\lambda; H_1,H_0)$ is the absolutely continuous part of the spectral shift function which was introduced in {\futurelet\NChar\CleverCite}{Az3v6}
(see also {\futurelet\NChar\CleverCite}{Az,Az2}) as a distribution by formula
$$
  {\xi^{(a)}}(\phi) = \int_0^1 \Tr(V\phi(H^{(a)}_r))\,dr, \ \ \phi \in C^\infty_c(\mbR).
$$
Here~$H_r^{(a)}$ is the absolutely continuous part of the
self-adjoint operator~$H_r,$ and we shall soon use~$H_r^{(s)}$ the
singular part of the self-adjoint operator~$H_r.$ The two formulas
connecting $\xi$ and ${\xi^{(a)}}$ with $\mu$ and ${\mu^{(a)}}$ respectively,
imply that
\begin{equation*}
  \begin{split}
    {\xi^{(s)}}(\lambda; H_1,H_0) & = - \frac {1}{2\pi} \int_0^{2\pi} {\mu^{(s)}}(\lambda; H_1,H_0)\,d\theta
    \\ & = - {\mu^{(s)}}(\lambda; H_1,H_0),
  \end{split}
\end{equation*}
where ${\xi^{(s)}}$ is the singular part of the spectral shift function defined as distribution by formula
$$
  {\xi^{(s)}}(\phi) := \int_0^1 \Tr(V\phi(H^{(s)}_r))\,dr
$$
and where
$$
  {\mu^{(s)}}(\lambda; H_1,H_0) := \mu(\theta, \lambda; H_1,H_0) - {\mu^{(a)}}(\theta, \lambda; H_1,H_0)
$$
is the singular part of $\mu$-invariant. Topological considerations imply that the function ${\mu^{(s)}}(\lambda; H_1,H_0)$
does not depend on the angle $\theta$ and so this variable is omitted in the list of arguments of ${\mu^{(s)}},$ see {\futurelet\NChar\CleverCite}[Section 9]{Az3v6}.
This implies in particular that ${\xi^{(s)}}(\lambda;H_1,H_0)$ takes integer values for a.e. $\lambda \in \mbR.$

One can easily check that a real number $r_\lambda$ is a resonance point if and only if the real number $(s-r_\lambda)^{-1}$ is an eigenvalue of positive algebraic multiplicity $N$ for the compact operator
$T_{\lambda+i0}(H_s)J.$ This definition is independent from $s \in \mbR$ as long as it is non-resonant.
If we shift $\lambda+i0$ to $\lambda+iy,$ the eigenvalue $(s-r_\lambda)^{-1}$ also changes and in general splits to eigenvalues
$(s-r^1_{\lambda+iy})^{-1},\ldots,(s-r^N_{\lambda+iy})^{-1}$ of $T_{\lambda+iy}(H_s)J,$ where the eigenvalues are listed according to their multiplicities.
It is well-known and is not difficult to show that these shifted eigenvalues are
all non-real. Let $N_+$ and $N_-$ be the numbers of the shifted eigenvalues in the upper $\mbC_+$ and the lower $\mbC_-$ complex half-planes respectively.
The so-called \emph{resonance index} of the triple $(\lambda; H_{r_\lambda},V)$ is defined by formula
$$
  \ind_{res}(\lambda; H_{r_\lambda},V) = N_+-N_-.
$$
In {\futurelet\NChar\CleverCite}{Az7} (see also {\futurelet\NChar\CleverCite}[\S 6]{Az9}) it was shown that ${\xi^{(s)}}(\lambda; H_1,H_0)$ is equal for a.e. $\lambda \in \mbR$ to the total resonance index
$$
  \sum_{r_\lambda \in [0,1]} \ind_{res}(\lambda; H_{r_\lambda},V),
$$
where the sum is taken over all resonance points from $[0,1],$ of which there is a finite number.

Hence, for trace-class perturbations $V$ and for a.e. $\lambda$ we have the equality
\begin{equation} \label{F: mus=sum ind(res)}
  - {\mu^{(s)}}(\lambda; H_1,H_0) = \sum_{r_\lambda \in [0,1]} \ind_{res}(\lambda; H_{r_\lambda},V).
\end{equation}
The minus sign here appears because the $\mu$-invariant calculates the spectral flow of eigenvalues of the scattering matrix in clockwise direction.

In this paper we give a new and direct proof of this formula
assuming only that the pair~$H_0$ and~$V$ satisfy the conditions
(1)--(3) above.

\begin{thm*}
If~$H_0$ is a self-adjoint operator on a separable Hilbert space
$\hilb$ and if~$V$ is a~$H_0$-compact self-adjoint operator on
$\hilb$ such that the conditions (1)--(3) hold, then the equality
(\ref{F: mus=sum ind(res)}) holds.
\end{thm*}
The conditions (1)--(3) are well-known in scattering theory, see
e.g. {\futurelet\NChar\CleverCite}{BE,KK71,Agm,Kur,Ya}. Three classical examples for which
the conditions (1)--(3) hold for a.e. $\lambda$ are:
\begin{enumerate}
  \item arbitrary self-adjoint operator~$H_0$ and a trace
class self-adjoint operator~$V,$
  \item a Schr\"odinger operator~$H_0 = -\Delta +
V_0(x)$ on $L_2(\mbR^\nu)$ with bounded measurable real-valued
function~$V_0(x)$ and~$V$ an operator of multiplication by a real-valued
function~$V(x)$ such that $\abs{V(x)} \leq C
(1+\abs{x})^{-\nu-\eps}$ for some $C,\eps>0,$
 \item the operator~$H_0$ is the Laplacian $-\Delta$ on $L_2(\mbR^\nu)$ and~$V$ an operator of multiplication by a real-valued
function~$V(x)$ such that $\abs{V(x)} \leq C
(1+\abs{x})^{-1-\eps}$ for some $C,\eps>0$.
\end{enumerate}

Finally, we note that the equality (\ref{F: mus=sum ind(res)})
makes a nontrivial sense even for self-adjoint matrices~$H_0$
and~$V$ on a finite dimensional Hilbert
space $\mbC^n,$ in which case the equality can be tested in numerical experiments. 

For more motivation for this work one may also look at the introduction of {\futurelet\NChar\CleverCite}{Az9}.

\bigskip
{\it Acknowledgements.} I would like to thank Prof. Peter Dodds for a useful discussion
concerning the material of subsection \ref{SS: uniformity}.

\section{Sketch and idea of proof of Theorem 1}
In this section we give a brief sketch of the idea of proof of Theorem 1.
Full details are given in the next section.

There are two ways to deform continuously the scattering matrix
$$
  S(\lambda+iy; H_r, H_0)\big|_{y=0, r=1} = S(\lambda+i0; H_1, H_0)
$$
to the identity operator 1: by sending the imaginary part $y$ of energy from $0^+$ to $+\infty$ and by sending coupling constant $r$
from $1$ to $0.$ Why these paths are continuous was explained in the introduction. As the scattering matrix is deformed to the identity operator
the eigenvalues of it are also continuously deformed to (the number) 1. The difference of spectral flows through a point $e^{i\theta}$ on the unit circle distinct from 1
does not depend on $\theta.$ This difference is the singular $\mu$-invariant. Now we are going to deform continuously one of these two paths
to another one but as we do so at one point we shall encounter an obstruction in the form of poles and zeros of the scattering matrix
as a function of the coupling constant $s.$ Overcoming this obstruction gives the resonance index $N_+-N_-$ which is the difference of the number
of poles and of the number of zeros from the upper complex half-plane, where poles and zeros are counted according to their multiplicities.

Firstly we deform the path $(y \colon 0 \rightsquigarrow + \infty, \ r = 1)$ to the path $(y \colon 0 \rightsquigarrow \eps, r = 1 \ \& \ y = \eps, \ r \colon 1 \rightsquigarrow 0),$
as shown in the figure below. The rectangle $[0,+\infty] \times [0,1]$ is the domain of the operator-function $f(y,r) = S(\lambda+iy; H_r, H_0)$ of real variables $y$ and $r.$
(Contrary to the common tradition, in the figure the first variable $y$ changes along the vertical axis).
This deformation does not meet any obstructions since the function is continuous in the rectangle $[\eps,+\infty] \times [0,1].$
But the deformation cannot be pushed all the way to the lower rim $y=0$ of the rectangle since the lower rim may have resonance points at which the function
$f(y,r)$ is not continuous.
It is to be noted however that the function $f(0,r)$ of variable $r$ is continuous and deformation along this path gives the ${\mu^{(a)}}$-invariant.
As we shall see, what happens here is cancellation of poles and zeros such as in the function $\frac{s+iy}{s-iy}$ in the limit $y \to 0.$

\bigskip
\begin{picture}(125,150)
\put(40,40){\vector(1,0){170}}
\put(40,40){\vector(0,1){100}}
\put(40,130){\line(1,0){150}}
\put(65,133){\tiny 1 \quad 1 \quad 1 \quad 1 \quad 1 \quad 1 \quad 1 \quad 1 }
\put(190,40){\line(0,1){90}}
\put(35,50){\tiny 1}
\put(35,65){\tiny 1}
\put(35,80){\tiny 1}
\put(35,95){\tiny 1}
\put(35,110){\tiny 1}

\put(185,30){{\small~$1$}}
\put(35,30){{\small~$0$}}
\put(25,38){{\small~$0$}}
\put(40,144){{\small~$y$}}
\put(15,127){{\small~$+\infty$}}
\put(210,40){{\small~$r$}}
\put(190,40){\circle*{3}}
\put(101,40){\circle*{3}}
\put(101,30){\tiny $r_\lambda$}
\put(38,10){\small The three points $r_\lambda, r'_\lambda,r''_\lambda$ represent resonance points from $[0,1]$.}
\put(123,40){\circle*{3}}
\put(123,30){\tiny $r'_\lambda$}
\put(156,40){\circle*{3}}
\put(156,30){\tiny $r''_\lambda$}
\put(208,57){\vector(-1,-1){15}}
\put(202,61){\tiny $S(0,1) = S(\lambda+i0; H_1, H_0)$}
\thicklines
\put(40,130){\line(1,0){150}}
\put(40,40){\line(0,1){90}}

\put(188,42){\vector(0,1){86}}
\put(186,42){\line(0,1){6}}
\put(186,48){\vector(-1,0){143}}
\put(110,52){\tiny $y=\eps$}

\thinlines
\qbezier(186,75)(160,75)(150,52)
\put(152,56){\vector(-1,-2){3}}
\end{picture}

So far we were interested in real values of the coupling constant $s.$
To stress on the fact that the coupling constant is considered as a complex variable we now denote it by~$s.$
Now we fix a small value $\eps$ of $y$ and consider the function $f(\eps,s)$ of the complex variable~$s.$

While for $y=0$ the factor $(1+sT_{\lambda+iy}J)^{-1}$ may have real singularities (which are by definition real resonance points),
for non-real values of $y$ the factor $(1+sT_{\lambda+iy}J)^{-1}$ as a function of $s$ is holomorphic on the real axis.
A pole~$r_\lambda$ of the factor $(1+sT_{\lambda+iy}J)^{-1}\big|_{y=0}$ for small non-real values of $y$
gets shifted out of the real axis. In general it may split into a group of several poles $r_{\lambda+iy}^1,\ldots,r_{\lambda+iy}^N,$
which are listed here according to their multiplicities. We note that while for real values of $s$ the scattering matrix $f(y,s)$ is a unitary operator for any $y,$
for complex values of $s$ this is not the case anymore. In particular, for some values of $s$ the scattering matrix $f(y,s)$ may have a non-trivial kernel.
These values we shall call zeros of $f(y,s).$ It turns out that conjugates of the poles $r_{\lambda+iy}^1,\ldots,r_{\lambda+iy}^N$ are zeros of $f(y,s)$
and vice versa.
Formally, this can be seen from the following formula for the scattering matrix proof of which is a simple algebraic transformation:
$$
  \sqrt{\Im T_{z}(H_0)}S(z,s) = (1+s T_{\bar z}(H_0)J)(1+s T_{z}(H_0)J)^{-1} \sqrt{\Im T_{z}(H_0)}.
$$

\begin{picture}(200,100)
\put(10,90){\small $s$-plane for $y=\eps.$ Black dots are resonance points, white dots are zeros (anti-resonance points).}
\put(0,50){\vector(1,0){330}}

\put(20,20){\vector(0,1){60}}
\put(290,50){\line(0,1){4}}
\put(289,40){$1$}

\put(70,50){\circle*{2}}
\put(67,56){\circle*{3}} \put(67,44){\circle{3}}

\put(160,50){\circle*{2}}
\put(154,58){\circle*{3}} \put(154,42){\circle{3}}
\put(157,67){\circle*{3}} \put(157,33){\circle{3}}
\put(172,55){\circle*{3}} \put(172,45){\circle{3}}
\put(166,60){\circle{3}}  \put(166,40){\circle*{3}}

\put(220,50){\circle*{2}}
\put(227,56){\circle{3}} \put(227,44){\circle*{3}}
\put(218,58){\circle{3}} \put(218,42){\circle*{3}}
\end{picture}

As it was discussed above, the deformation of the $S$-matrix $f(y,s)\big|_{y=0, s=1}$ to the identity operator which corresponds to $\mu$-invariant
is shown in the following figure.

\begin{picture}(200,100)
\put(280,65){\small $s$-plane for $y=\eps \ll 1.$} 
\put(0,50){\vector(1,0){330}}

\put(20,20){\vector(0,1){60}}
\put(290,50){\line(0,1){4}}
\put(289,40){$1$}

\put(70,50){\circle*{2}}
\put(67,56){\circle*{3}} \put(67,44){\circle{3}}

\put(160,50){\circle*{2}}
\put(154,58){\circle*{3}} \put(154,42){\circle{3}}
\put(157,67){\circle*{3}} \put(157,33){\circle{3}}
\put(172,55){\circle*{3}} \put(172,45){\circle{3}}
\put(166,60){\circle{3}}  \put(166,40){\circle*{3}}

\put(220,50){\circle*{2}}
\put(227,56){\circle{3}} \put(227,44){\circle*{3}}
\put(218,58){\circle{3}} \put(218,42){\circle*{3}}

\thicklines
\put(290,50){\vector(-1,0){270}}
\end{picture}

The deformation of the $S$-matrix $f(y,s)$ to the identity operator which corresponds to ${\mu^{(a)}}$-invariant
is shown in the following figure. That is, to obtain ${\mu^{(a)}}$-invariant, we should circumvent resonant and anti-resonant points in the $s$-plane.
Indeed, as we take the limit $y \to 0$ the $\mu$-invariant of the path shown below does not change for topological reasons since there are no obstructions,
but at the same time all resonance and anti-resonance points
start to converge to the corresponding real resonance point where they eventually get cancelled by each other (as it happens to the pole and the zero of the
holomorphic function $\frac{z-i\eps}{z+i\eps}$ in the limit $\eps \to 0$). Once we get to $y = 0$ the path below can be deformed to the straight path from $s=1$
to $s=0$ since the resulting function $f(0,s)$ has no singularities in a neighbourhood of the interval $[0,1].$

\begin{picture}(200,100)
\put(280,65){\small $s$-plane for $y=\eps \ll 1.$}
\put(0,50){\vector(1,0){330}}

\put(20,20){\vector(0,1){60}}
\put(290,50){\line(0,1){4}}
\put(289,40){$1$}

\put(70,50){\circle*{2}}
\put(67,55){\circle*{3}} \put(67,45){\circle{3}}

\put(160,50){\circle*{2}}
\put(154,58){\circle*{3}} \put(154,42){\circle{3}}
\put(157,67){\circle*{3}} \put(157,33){\circle{3}}
\put(172,55){\circle*{3}} \put(172,45){\circle{3}}
\put(166,60){\circle{3}}  \put(166,40){\circle*{3}}

\put(220,50){\circle*{2}}
\put(227,56){\circle{3}} \put(227,44){\circle*{3}}
\put(218,58){\circle{3}} \put(218,42){\circle*{3}}

\thicklines
\put(290,50){\vector(-1,0){55}}
\put(220,50){\oval(30,30)[t]}
\put(220,65){\vector(-1,0){4}}
\put(205,50){\vector(-1,0){20}}
\put(160,50){\oval(50,50)[t]}
\put(160,75){\vector(-1,0){4}}
\put(135,50){\vector(-1,0){55}}
\put(70,50){\oval(20,20)[t]}
\put(70,60){\vector(-1,0){4}}
\put(60,50){\vector(-1,0){40}}
\end{picture}

Hence, the singular $\mu$-invariant, that is the difference in winding numbers of eigenvalues of $f(y,s)$
corresponding to the two deformations, is equal to the total number of windings of eigenvalues of the $S$-matrix along the clockwise oriented closed contours
enclosing those and only those poles and zeros of the groups of real resonance points which belong to the upper half-plane, see the next figure.

\begin{picture}(200,100)
\put(280,65){\small $s$-plane for $y=\eps \ll 1.$}
\put(0,50){\vector(1,0){330}}

\put(20,20){\vector(0,1){60}}
\put(290,50){\line(0,1){4}}
\put(289,40){$1$}

\put(70,50){\circle*{2}}
\put(67,55){\circle*{3}} \put(67,45){\circle{3}}

\put(160,50){\circle*{2}}
\put(154,58){\circle*{3}} \put(154,42){\circle{3}}
\put(157,67){\circle*{3}} \put(157,33){\circle{3}}
\put(172,55){\circle*{3}} \put(172,45){\circle{3}}
\put(166,60){\circle{3}}  \put(166,40){\circle*{3}}

\put(220,50){\circle*{2}}
\put(227,56){\circle{3}} \put(227,44){\circle*{3}}
\put(218,58){\circle{3}} \put(218,42){\circle*{3}}

\thicklines
\put(235,50){\vector(-1,0){30}}
\put(220,50){\oval(30,30)[t]}
\put(220,65){\vector(1,0){4}}

\put(185,50){\vector(-1,0){50}}
\put(160,50){\oval(50,50)[t]}
\put(160,75){\vector(1,0){4}}
\put(70,50){\oval(20,20)[t]}
\put(70,60){\vector(1,0){4}}
\put(80,50){\vector(-1,0){20}}
\end{picture}

It remains to apply the Argument Principle from complex analysis to eigenvalues of the $S$-matrix to infer that
the singular $\mu$-invariant is equal to the number of poles minus the number of zeros inside of these contours in the upper complex half-plane.

We note that it is not essential to circumvent the poles and zeros in the upper half-plane. Of course, in this case the difference of the number
of poles and zeros changes sign but this is compensated by the change of direction of circumvention of the contours from clockwise to anti-clockwise.

Finally, some care should be taken in the case there are degenerate resonance or anti-resonance points. This is done in the next section.

\section{Proof of Theorem 1}
\subsection{The Argument Principle}
The following theorem from complex analysis known as the Argument Principle can be found in e.g. {\futurelet\NChar\CleverCite}{Shab1}.
\begin{thm} \label{T: Argument Principle} Let $f(z)$ be a meromorphic function in a simply connected domain $\Omega$ of the complex plane.
Let $\gamma$ be a closed contour in $\Omega$ which does not contain poles and zeros of $f(z).$ As $z$ makes one round along the curve $\gamma$
the argument of $f(z)$ makes $N-P$ rounds around zero where $N$ is the number of zeros, counting multiplicities, of $f(z)$ inside $\gamma$ and $P$ is the number
of poles, counting multiplicities, of $f(z)$ inside $\gamma.$
\end{thm}
\subsection{Zeros and poles of the scattering matrix}
Let $H_0,$ $F$ and $V$ be as in the introduction.
Without loss of any generality we shall assume that the range of $F$ is dense in $\clK.$
Let $z$ be an arbitrary complex number which does not belong to the resolvent set of $H_0,$
so in particular the operator $T_z(H_0)$ exists.
We denote by $S(z,s)$ the operator function
\begin{equation} \label{F: S(s,z)=I formula}
  S(z; H_s,H_0) = 1 - 2i s \sqrt{\Im T_{z}(H_0)} J (1+sT_{z}(H_0)J)^{-1}\sqrt{\Im T_{z}(H_0)},
\end{equation}
where $s$ is treated as a complex variable and $\Im z > 0.$
The right hand side can also be written as
\begin{equation} \label{F: S(s,z)=II formula}
  S(z; H_s,H_0) = 1 - 2i s \sqrt{\Im T_{z}(H_0)} (1+sJT_{z}(H_0))^{-1} J \sqrt{\Im T_{z}(H_0)}.
\end{equation}
By definition, a point $s$ is a \emph{resonance point}
corresponding to $z$ if the operator
$$
  1+sJT_z(H_0)
$$
has a non-zero kernel. Dimension of this kernel will be called \emph{algebraic multiplicity} of the resonance point.
Non-zero vectors from the kernel will be called \emph{resonance vectors} for given values of $z$ and $s.$
A point $s$ is an \emph{anti-resonance point} corresponding to $z$ if the operator
$$
  1+sJT_{\bar z}(H_0)
$$
has a non-zero kernel. Dimension of this kernel will be called \emph{algebraic multiplicity} of the anti-resonance point.
Non-zero vectors from this kernel will be called \emph{anti-resonance vectors} for given values of $z$ and $s.$
It can be easily shown that anti-resonance points are complex conjugates of resonance points and algebraic multiplicities of a pair of resonance and anti-resonance points are equal,
see e.g. {\futurelet\NChar\CleverCite}[Section 3]{Az9}.
Resonance points will usually be denoted by $r_z$ and anti-resonance points by $\bar r_z.$

\begin{lemma} For any complex number $z$ with positive imaginary part the equalities
\begin{equation} \label{F: sqrt(Im T) S = M sqrt(Im T)}
    \sqrt{\Im T_{z}(H_0)} S(z,s) = (1+s T_{\bar z}(H_0)J)(1+s T_{z}(H_0)J)^{-1}\sqrt{\Im T_{z}(H_0)}
\end{equation}
and
\begin{equation} \label{F: S sqrt(Im T) = sqrt(Im T) M}
    S(z,s) \sqrt{\Im T_{z}(H_0)} = \sqrt{\Im T_{z}(H_0)} (1+s J T_z(H_0))^{-1}(1+s J T_{\bar z}(H_0)),
\end{equation}
hold as equalities between meromorphic functions of~$s.$
\end{lemma}
\begin{proof} Using (\ref{F: S(s,z)=I formula}), we have
\begin{equation*}
  \begin{split}
    \sqrt{\Im T_{z}(H_0)} S(z,s) & = \sqrt{\Im T_{z}(H_0)} - 2i s\Im T_{z}(H_0) J (1+sT_{z}(H_0)J)^{-1}\sqrt{\Im T_{z}(H_0)}
    \\ & = \SqBrs{1 - s(T_{z}(H_0) -T_{\bar z}(H_0))J (1+sT_{z}(H_0)J)^{-1}}\sqrt{\Im T_{z}(H_0)}
    \\ & = \SqBrs{1 - sT_{z}(H_0)J(1+sT_{z}(H_0)J)^{-1} + sT_{\bar z}(H_0)J (1+sT_{z}(H_0)J)^{-1}}\sqrt{\Im T_{z}(H_0)}
    \\ & = \SqBrs{(1+sT_{z}(H_0)J)^{-1} + sT_{\bar z}(H_0)J (1+sT_{z}(H_0)J)^{-1}}\sqrt{\Im T_{z}(H_0)}
    \\ & =  (1 + sT_{\bar z}(H_0)J) (1+sT_{z}(H_0)J)^{-1}\sqrt{\Im T_{z}(H_0)}.
  \end{split}
\end{equation*}
The equality (\ref{F: S sqrt(Im T) = sqrt(Im T) M}) is proved similarly but instead of (\ref{F: S(s,z)=I formula})
one uses (\ref{F: S(s,z)=II formula}).
\end{proof}

Given a non-real complex number $z,$ we say that $s$ is \emph{non-critical} if $s$ is neither resonant nor anti-resonant for $z.$

\begin{lemma} \label{L: S(z,s) is bdd for non-critical s}
Let $\Im z > 0.$ For any non-critical $s \in \mbC$ the operator $S(z,s)$ has a bounded inverse.
\end{lemma}
\begin{proof} By analytic Fredholm alternative, the function $S(z,s)$ is a meromorphic function of $s,$
which may have poles only at resonance points. Hence, for non-resonant $s$ the operator $S(z,s)$ is well-defined and bounded.

Let $s$ be a point which is not resonant.
If $\phi_j$ is an eigenvector of $S(z,s)$ corresponding to eigenvalue zero, then
(\ref{F: sqrt(Im T) S = M sqrt(Im T)}) implies that the vector
$$
  (1+s T_{z}(H_0)J)^{-1}\sqrt{\Im T_{z}(H_0)} \phi_j
$$
is a non-zero anti-resonance vector for~$z$ and~$s.$ Therefore, in this case $s$ is an anti-resonance point corresponding to~$z.$

It follows that if $s$ is neither resonant nor anti-resonant then $S(z,s)$ is a bounded operator with zero kernel.
Since $S(z,s)-1$ is compact, by Fredholm alternative, this implies that in this case  $S(z,s)$ is a bounded invertible operator.
\end{proof}

Let ${\mathfrak h_z}$ be the range of the operator $\sqrt{\Im T_{z}(H_0)}:$
$$
  {\mathfrak h_z} := \rng \sqrt{\Im T_{z}(H_0)}.
$$
The set ${\mathfrak h_z}$ is a dense subspace of the Hilbert space~$\clK.$
If $s$ is neither resonant nor anti-resonant point, then by Fredholm alternative
the operator $(1+s J T_{z}(H_0))^{-1} (1+s J T_{\bar z}(H_0))$ is invertible.
Hence, the equality (\ref{F: S sqrt(Im T) = sqrt(Im T) M})
implies that for such~$s$
$$
  S(z,s) {\mathfrak h_z} = {\mathfrak h_z}.
$$

Let
\begin{equation} \label{F: M-function}
  \begin{split}
     M(z,s) & := (1+s T_{\bar z}(H_0)J)(1+s T_{z}(H_0)J)^{-1} \\
       & = 1 - 2is \Im T_z(H_0)(1+s T_{z}(H_0)J)^{-1}.
  \end{split}
\end{equation}

The equality (\ref{F: sqrt(Im T) S = M sqrt(Im T)}) can be rewritten as
\begin{equation} \label{F: sqrt S = M sqrt}
  \sqrt{\Im T_{z}(H_0)} S(z,s) = M(z,s) \sqrt{\Im T_{z}(H_0)}.
\end{equation}
Since by Lemma \ref{L: S(z,s) is bdd for non-critical s} the operator $S(z,s)$ is invertible for non-critical~$s,$ for such $s$ we have $S(z,s)\clK = \clK.$
Hence, (\ref{F: sqrt S = M sqrt}) implies that for all non-critical~$s$
\begin{equation*} \label{F: M(z,s)euE(z)=euE(z)}
  M(z,s) {\mathfrak h_z} = {\mathfrak h_z}.
\end{equation*}

We have proved the following lemma.
\begin{lemma} \label{L: S Ez=Ez} Given a non-real complex number~$z,$ for any non-critical $s$
$$
  S(z,s) {\mathfrak h_z} = {\mathfrak h_z} \quad \text{and} \quad M(z,s) {\mathfrak h_z} = {\mathfrak h_z}.
$$
\end{lemma}
\bigskip

For a given number~$z$ we say that a point~$s_0$ is a \emph{zero} of the meromorphic function~$S(z,s)$ if and only if~$0$ is an eigenvalue of~$S(z,s).$
Multiplicity of zero~$s_0$ is the algebraic multiplicity of the eigenvalue~$0.$ Geometrically, this means that if multiplicity of a zero~$s_0$
is~$m$ then~$m$ eigenvalues (counting multiplicities) of~$S(z,s)$ approach zero as~$s$ approaches~$s_0.$

\begin{lemma} Let $z$ be a non-real complex number and let $s$ be a non-critical value.
All eigenvectors of the operator $S(z,s)$ corresponding to eigenvalues not equal to $1$ belong to ${\mathfrak h_z}.$
All eigenvectors of the operator $M(z,s)$ corresponding to eigenvalues not equal to $1$ belong to $\rng \Im_z(H_0).$
Further, a vector $\phi_j$ is an eigenvector (a generalized eigenvector) of $S(z,s)$ corresponding to a non-identity
eigenvalue if and only if $\sqrt{\Im_z(H_0)} \phi_j$ is an eigenvector
(respectively, a generalized eigenvector of the same order) of $M(z,s)$ corresponding to the same non-identity eigenvalue.
\end{lemma}
\begin{proof} The first assertion follows directly from (\ref{F: S(s,z)=I formula}).
The second assertion follows from the second of the two equalities (\ref{F: M-function}).
The third assertion follows from (\ref{F: sqrt S = M sqrt}), the previous two assertions and the fact that $\sqrt{\Im_z(H_0)}$ has trivial kernel for non-real~$z.$
\end{proof}
\begin{cor} The operators $S(z,s)$ and $M(z,s)$ have identical spectra, including multiplicities of eigenvalues.
\end{cor}
This corollary implies the following
\begin{thm} \label{T: poles and zeros of S and M} For any non-real complex number $z$ the meromorphic functions $S(z,s)$
and $M(z,s)$ have the same sets of poles and zeros; moreover, multiplicities of these poles and zeros are also the same.
\end{thm}
Indeed, $s$ is a zero of multiplicity $N$ for $S(z,s)$ or $M(z,s)$ iff zero is an eigenvalue
of $S(z,s)$ or $M(z,s)$ of multiplicity $N.$

This theorem allows us to work with either $S(z,s)$ or $M(z,s)$ as far as we are concerned only with eigenvalues of these operators.

\begin{cor} For any non-real $z$ the meromorphic function $S(z,s)$ of $s$ has poles at resonance points $r_z$
and zeros at antiresonance points $\bar r_z.$ There are no other poles and zeros of $S(z,s).$ Further, multiplicities of
a pole $r_z$ and of a zero $\bar r_z$ coincide.
\end{cor}

We are interested in the following question. Assume that $r_z$ is a resonance point (respectively, anti-resonance point)
of algebraic multiplicity~$N.$ In this case $s=r_z$ is a pole (respectively, zero) of $S(z,r_z)$ of algebraic multiplicity~$N.$
For values of $s$ close to $r_z$ the operator $S(z,s)$ will have $N$ eigenvalues close to infinity (respectively, to zero);
these eigenvalues are called eigenvalues of the group infinity (respectively, zero).
As the variable $s$ makes one round around $r_z$
the eigenvalues of the operator $S(z,s)$ which belong to the group of infinity (respectively, zero) will undergo a permutation. We are interested in the total number of windings
which these eigenvalues make. If we knew that all the eigenvalues of the group of infinity (respectively, zero) admit single-valued analytic continuation to a neighbourhood of $r_z$
then the Argument Principle would imply that the total number of windings would be $N.$ But we don't know whether the eigenvalues admit such single-valued
analytic continuations. We shall call the total number of windings of the eigenvalues of $S(z,r_z)$ about zero \emph{$S$-index} of the resonance point $r_z.$
Our aim is to prove that $S$-index of a resonance point (respectively, anti-resonance point) of
algebraic multiplicity~$N$ is equal to~$-N$ (respectively,~$N$).

The following assertion is trivial.
\begin{lemma} \label{L: S-index of a group}
$S$ index of a resonance point $r_z$ is equal to the sum of $S$-indices of resonance points into which $r_z$
splits when $z$ is slightly perturbed.
\end{lemma}

It is possible that a resonance point $r_z$ corresponding to $z$ is also an anti-resonance point.
This creates some technical difficulties.
The following lemma allows to avoid them.
\begin{lemma} \label{L: no res and anti-res}
A resonance point $r^i_z$ as a function of $z$ cannot coincide with an anti-resonance point $\bar r^j_z$ as a function of $z.$
As a result, if $r_z$ is both a resonance and anti-resonance point then if $z$ is perturbed slightly $r_z$ will split into resonance points none of which is an anti-resonance point.
\end{lemma}
\begin{proof} Indeed, a resonance point $r^i_z$ is a holomorphic function of $z$ while an anti-resonance point $\bar r^j_z$ is an anti-holomorphic function of $z.$
So, they can coincide only if both of them are constants. But a resonance point $r_z$ cannot be a constant function of $z.$
\end{proof}
This lemma implies that slightly perturbing $z$ we can always achieve a situation where neither of a finite number of resonance points is an anti-resonance point.

\begin{lemma} \label{L: S-index = alg mult} $S$-index of a resonance point of algebraic multiplicity $N$ is equal to~$-N.$
\end{lemma}
\begin{proof} According to Theorem \ref{T: poles and zeros of S and M}, as far as the total number of windings
of eigenvalues is concerned, instead of the operator $S(z,s)$ we can consider the operator $M(z,s).$
The operator $1+sT_z(H_0)J$ for $s=r_z$ has zero as an eigenvalue of multiplicity~$N.$
Let $\eps_j(s)\big|_{s=r_z}$ be this eigenvalue. When $s$ is perturbed to a value close to $r_z$ the zero eigenvalue $\eps_j(r_z)$ shifts
to $\eps_j(s)$ but does not split.
By the Argument Principle (Theorem \ref{T: Argument Principle}),
when $s$ makes one winding around $r_z$ this eigenvalue $\eps_j(s)$ makes one winding around zero.
But since this eigenvalue has multiplicity~$N,$ it results in $S$-index being equal to~$N.$
Hence, $S$-index of the operator
$$
  (1+sT_z(H_0)J)^{-1}
$$
is equal to~$-N.$

By Lemmas \ref{L: S-index of a group} and \ref{L: no res and anti-res} we can assume that $r_z$ is a stable (non-splitting) resonance point
and that this resonance point is not an anti-resonance point.
Thus, the operator $1+sT_{\bar z}(H_0)J$ is invertible.
Since $S$-index is plainly stable under continuous deformations and since
the invertible operator $1+sT_{\bar z}(H_0)J$ can be continuously deformed into the identity operator in the space of invertible operators,
it follows that the $S$-index of the operator
$$
  (1+sT_{\bar z}(H_0)J)(1+sT_z(H_0)J)^{-1}
$$
is equal to~$-N.$
\end{proof}

A similar argument proves the following
\begin{lemma} \label{L: S-index = alg mult (2)} $S$-index of an anti-resonance point of algebraic multiplicity $N$ is equal to~$N.$
\end{lemma}

\begin{rems*} \rm If zeros and poles of eigenvalues of the function $S(z,s)$ of $s$ for a fixed $z$ were known to have single-valued analytic continuations
in some neighbourhoods of resonance $r_z$ and anti-resonance points $\bar r_z$ then,
combined with the Argument Principle, this would directly imply Lemmas \ref{L: S-index = alg mult} and \ref{L: S-index = alg mult (2)}.
But Lemmas \ref{L: S-index = alg mult} and \ref{L: S-index = alg mult (2)} alone do not imply that
the eigenvalues have the single-valued analytic continuations. For example, if a zero $\bar r_z$ of an eigenvalue of $S(z,s)$ has algebraic multiplicity four,
then it is possible that as $s$ makes one winding around $\bar r_z$ two eigenvalues swap making half windings around zero and the other two eigenvalues swap making one and a half windings around zero,
resulting in $S$-index four in agreement with Lemmas \ref{L: S-index = alg mult} and \ref{L: S-index = alg mult (2)}.
\end{rems*}

\subsection{Uniformity of spectrum of $S(z,s)$ as $\Im z \to +\infty$}
\label{SS: uniformity}
As usual, we are assuming conditions (1)--(3) from the introduction.

The aim of this subsection is to prove the following theorem.
\begin{thm} \label{T: eigenvalues of S(z,s) converge uniformly}
For any $\lambda \in \mbR$ the operator $S(\lambda+iy,s)$
converges to the identity operator as $y \to +\infty$ locally uniformly with respect to $s \in \mbR.$
Moreover, eigenvalues $\eps_j(\lambda+iy,s)$ of the operator $S(\lambda+iy,s)$
also converge to~1 as $y \to +\infty$ locally uniformly with respect to $s \in \mbR.$
\end{thm}

We shall use the following lemma, see e.g. {\futurelet\NChar\CleverCite}[Lemma 6.1.3]{Ya}.
\begin{lemma} \label{L: Ya L 6.1.3} If a sequence of bounded operators $A_n$ converges to zero in $*$-strong operator topology
and if $T$ is a compact operator then $TA_n$ and $A_nT$ converge to zero uniformly.
\end{lemma}

\begin{lemma} \label{L: Im part of Tz}
Let $z$ be any non-real complex number.
The operators
$$
  FE_{(-\infty,-M)}(H_0) \Im R_z(H_0) F^*
$$
and
$$
  FE_{(M,+\infty)}(H_0) \Im R_z(H_0) F^*
$$
converge to zero in norm as $M \to +\infty.$
\end{lemma}
If $F$ is bounded then this assertion is an easy consequence of the spectral theorem.
\begin{proof} We prove the assertion for the first operator.
By Lemma \ref{L: Ya L 6.1.3}, the operator
$$
  F E_{(-\infty,-M)}(H_0)R_z(H_0)
$$
converges in norm to zero as $M \to \infty$ since it is product of a compact operator
$F R_z(H_0)$ (for a proof that this operator is compact see e.g. {\futurelet\NChar\CleverCite}[Lemma 2.8]{Az9})
and of an operator $E_{(-\infty,-M)}(H_0)$ which converges $*$-strongly to~$0$ as $M \to +\infty.$

It follows that the operator
$$
  F E_{(-\infty,-M)}(H_0)R_{z}(H_0) R_{\bar z}(H_0) E_{(-\infty,-M)}(H_0) F^*
$$
also converges to zero as $\Delta \to \mbR.$
By the first resolvent identity, this implies that
$$
  F E_{(-\infty,-M)}(H_0)(R_{z}(H_0) - R_{\bar z}(H_0)) F^* \to 0.
$$
\end{proof}
\begin{lemma} \label{L: Re part of Tz}
Let $z$ be any non-real complex number.
The operators
$$
  FE_{(-\infty,-M)}(H) \Re R_z(H) F^*
$$
and
$$
  FE_{(M,+\infty)}(H) \Re R_z(H) F^*
$$
converge to zero in norm as $M \to +\infty.$
\end{lemma}
\begin{proof} We prove the assertion for the second operator.
Without loss of generality we assume that $z = i.$ Then
\begin{equation} \label{F: op-r}
  FE_{(M,+\infty)}(H) \Re R_z(H) F^* = FE_{(M,+\infty)}(H) \frac{H}{H^2+1}F^*.
\end{equation}
By Lemma \ref{L: Ya L 6.1.3}, the compact operator $FE_{(M,+\infty)}(H) \frac{H}{H^2+1}$ converges to zero in norm as $M \to +\infty.$
Using this one can show that the operator (\ref{F: op-r}) converges to zero strongly as $M \to +\infty.$
Indeed, for $M \in \mbR$ let
$$
  T_M:= FE_{(M,+\infty)}(H) \frac{H}{H^2+1}F^*.
$$
If $f \in \dom F^*$ then clearly $T_M f \to 0$ as $M \to +\infty.$ If $f \notin \dom F^*,$ then for any $\eps>0$ we find $g \in \dom F^*$
such that $\norm{f - g} < \frac{\eps}{4\norm{T_0}}.$ Let $h = f-g.$ Then we have for any $M>0$
$$
  \norm{T_M h} \leq \norm{(T_0-T_M) h} + \norm{T_0 h} < \norm{T_0-T_M} \frac \eps {4\norm{T_0}} + \frac \eps 4.
$$
Since for any $M>0$ we have $0 \leq T_0-T_M \leq T_{0},$ it follows that $\norm{T_0-T_M} \leq \norm{T_0},$ see e.g. {\futurelet\NChar\CleverCite}{GK}.
Hence, for any $M>0$
$$
  \norm{T_M (f-g)} = \norm{T_M h} < \frac \eps 2.
$$
Since $T_M g \to 0$ as $M \to +\infty$ and since $\eps>0$ is arbitrary, it follows from this that $T_M f \to 0$ as $M \to +\infty.$

Thus, we have a decreasing directed family of positive compact operators which converge to zero operator in strong operator topology.
It is a well-known result then that this family converges to zero in norm which can be deduced e.g. from {\futurelet\NChar\CleverCite}[Lemma 3.5]{DDP93}
and the fact that the norm $\norm{A}$ of a compact operator coincides with its largest singular value $s_1(A).$
\end{proof}

Lemmas \ref{L: Im part of Tz} and \ref{L: Re part of Tz} imply the following corollary.
\begin{cor} \label{C: full Tz}
Let $z$ be any non-real complex number.
The operators
$$
  FE_{(-\infty,-M)}(H) R_z(H) F^*
$$
and
$$
  FE_{(M,+\infty)}(H) R_z(H) F^*
$$
converge to zero in norm as $M \to +\infty.$
\end{cor}

In order to omit the condition ``$F$ is bounded'' we need the following lemma.
\begin{lemma} \label{L: On limit of T(z) as y to infty} Let $F$ be an arbitrary possibly unbounded rigging. Then
$$
  \norm{T_{\lambda+iy}(H_0)} \to 0
$$
as $y \to +\infty.$
\end{lemma}
Proof of this lemma is obvious, if the rigging operator $F$ is
bounded, since in this case we have an estimate
$$
  \norm{T_{\lambda+iy}(H_0)} = \norm{FR_{\lambda+iy}(H_0)F^*} \leq \norm{F} \norm{R_{\lambda+iy}(H_0)} \norm{F^*} \leq \frac 1y \norm{F}^2.
$$
Moreover, in this case the convergence is uniform with respect to~$H_0.$

\begin{proof} We write
$$
  T_{\lambda+iy}(H_0) = T^\Delta_{\lambda+iy}(H_0) + T^{\Delta^c}_{\lambda+iy}(H_0),
$$
where $\Delta$ is a finite open interval which contains~$\lambda,$ $\Delta^c$ is the complement of $\Delta$ and
$$
  T^\Delta_{\lambda+iy}(H_0) =: F R^\Delta_{\lambda+iy}(H_0)F^* =: F E_\Delta(H_0)R_{\lambda+iy}(H_0)F^*.
$$
Since for a compact $\Delta$ the operator $F E_\Delta$ is bounded,
the norm of the term $T^\Delta_{\lambda+iy}(H_0)$ can be estimated by
$$
  \norm{F E_\Delta(H_0)R_{\lambda+iy}(H_0)E_\Delta(H_0) F^*} \leq  \norm{R_{\lambda+iy}(H_0)} \norm{F E_\Delta}^2 \leq \frac 1y \norm{F E_\Delta}^2.
$$
Hence, this term converges to~$0$ as $y \to +\infty.$

Now Corollary \ref{C: full Tz} implies that for any fixed $y>0$ the term $T^{\Delta^c}_{\lambda+iy}(H_0)$ converges to zero as $\Delta \to \mbR.$
This completes the proof.
\end{proof}

{\it Proof of Theorem \ref{T: eigenvalues of S(z,s) converge uniformly}.} \
As can be seen from the formula (\ref{F: S(s,z)=I formula}), Lemma \ref{L: On limit of T(z) as y to infty} implies that $S(z,s)$ converges
locally uniformly with respect to $s \in \mbR$ to the identity operator.
This fact combined with unitarity of $S(z,s)$ also proves the second part of Theorem \ref{T: eigenvalues of S(z,s) converge uniformly}.
$\Box$

\subsection{Proof of Theorem 1}
Given a continuous path $\gamma$ of unitary operators of the class ``identity + compact operator'' which ends at the identity operator,
we denote by $\mu(\theta, \gamma)$ the spectral flow of eigenvalues of the path through the point $e^{i\theta}.$ A continuous deformation of such a path does not change
the $\mu$-invariant of the path provided that ends stay fixed, see in this regard e.g. {\futurelet\NChar\CleverCite}{AzDaTa}.
Since for $y>0$ and real $s$ the operator $S(\lambda+iy,s)$ is well-defined
(this is because for non-real $z$ the operator $S(z,s)$ has no real resonance points), it is continuous in the rectangle $\set{(y,s) \colon (y,s) \in [y_0,Y_0] \times [0,1]}.$
Theorem \ref{T: eigenvalues of S(z,s) converge uniformly}  in fact shows that $S(z,s)$ is continuous in the rectangle $\set{(y,s) \colon (y,s) \in [y_0,+\infty] \times [0,1]}.$
This implies that the $\mu$-invariant $\mu(\theta,\lambda; H_1,H_0)$ is equal to the $\mu$-invariant of the path $(0,1) \to (y_0,1) \to (y_0,0)$ with very small $y_0.$
The difference between the $\mu$-invariant of this path and of the one which circumvents resonance and antiresonance points $r_{\lambda+iy}^1,\ldots,r_{\lambda+iy}^N$
and $\bar r_{\lambda+iy}^1,\ldots,\bar r_{\lambda+iy}^N$ of the group of~$r_\lambda$ from above is equal to the sum of $S$-indices of the points from the upper half-plane.
By Lemmas \ref{L: S-index = alg mult} and \ref{L: S-index = alg mult (2)}, this sum of $S$-indices is equal to the resonance index of~$r_\lambda.$

Further, the $\mu$-invariant of the path which circumvents resonance and antiresonance points $r_{\lambda+iy}^1,\ldots,r_{\lambda+iy}^N$ from above does not change as $y \to 0^+$
(the resonance and anti-resonance points of the group of~$r_\lambda$ converge to~$r_\lambda$ but this point is circumvented).
The path thus obtained can further be continuously deformed to the straight line path from $s=1$ to $s=0$ in the complex plane, since when $y=0$
the function $f(y,r)$ is holomorphic in a neighbourhood of the real axis.
Hence, the $\mu$-invariant of this path is equal to the absolutely continuous part of the $\mu$-invariant.

\rndef{\emph}[1]{{\it #1}}

\mathsurround 0pt
{\myaddress}{\AndSoOn}{$\dots$}
\begin{thebibliography}{XXXXX}

{\myaddress}{\sbOthers}{0} 
{\myaddress}{\sbDiffGm}{1} 
{\myaddress}{\sbFuncAn}{2} 
{\myaddress}{\sbPseuDO}{3} 
{\myaddress}{\sbAlgTop}{4} 
{\myaddress}{\sbNCGeom}{5} 
{\myaddress}{\sbQuanFT}{6} 
{\myaddress}{\sbMOI}{8}    
{\myaddress}{\sbSymmSp}{9} 
{\myaddress}{\sbOpeAlg}{A} 
{\myaddress}{\sbCompAn}{B} 
{\myaddress}{\sbAlgGeo}{C} 
{\myaddress}{\sbHomAlg}{D} 
{\myaddress}{\sbOperTh}{E} 
{\myaddress}{\sbSSF}{F}    
{\myaddress}{\sbScatTh}{H} 
{\myaddress}{\sbSpecTh}{I} 
{\myaddress}{\sbGenTop}{J} 
{\myaddress}{\sbHarmAn}{K} 
{\myaddress}{\sbKTheor}{L} 
{\myaddress}{\sbStrThe}{M} 

{\myaddress}{\zzBk}{1}
{\myaddress}{\zzPr}{0}

\mbbi1\sbScatTh\zzPr{Ag}{Agm}{\Agmon}{Spectral properties of \Schroedinger\ operators and scattering theory}
                    {Ann. Scuola Norm. Sup. Pisa Cl. Sci. \VolYearPP{2}{1975}{151--218}}
\mbbi0\sbOperTh\zzBk{AG}{AG}{\Ahiezer, \Glazman}{Theory of linear operators in Hilbert space}{New York, F.\,Ungar Pub. Co., 1963\libcode{515.73 A315t.N}}
\mbbi0\sbFuncAn\zzBk{AK}{AK}{\Ahiezer, \KreinMG}{Some questions in the theory of moments}{Providence, AMS, 1962\libcode{511 T772 2}}
\mbbi0\sbSSF\zzPr{Al}{Al87}{A.\,B.\,Alexandrov}{The multiplicity of the boundary values of inner functions}
                    {Sov. J. Comtemp. Math. Anal. \VolNoYearPP{22}{5}{1987}{74--87}}
\mbbi0\scScatTh\zzPr{ASch}{ASch71}{P.\,Alsholm, G.\,Schmidt}{Spectral and scattering theory for Schr\"odinger operators}{\jnArchRatMechAnal{40}{1971}{281--311}}
\mbbi0\sbOperTh\zzPr{AS}{AmSin}{W.\,O.\,Amrein, K.\,B.\,Sinha}{On pairs of projections in a Hilbert space}{Linear Algebra Appl. \VolYearPP{208/209}{1994}{425--435}}
\mbbi0\sbOthers\zzPr{Ar}{Ar}{\Arazy}{\AndSoOn}{\AndSoOn}
\mbbi0\sbCompAn\zzPr{Ar}{Ar57}{\Aronszajn}{On a problem of Weyl in the theory of singular Sturm-Liouville equations}{\jnAmerJMath{79}{1957}{597--610}}
\mbbi0\sbCompAn\zzPr{AD}{AD56}{\Aronszajn, W.\,F.\,Donoghue}{On exponential representation of analytic
        functions in the upper half-plane with positive imaginary part}{J.\,d'Anal. Math. \VolYearPP{5}{1956}{321--388}}
\mbbi0\sbFuncAn\zzBk{AB}{AB}{E.\,Asplund, L.\,Bungart}{A First Course in Integration}{New York, Holt, Rinehart and Winston, 1966\libcode{515.73 A315t.N}}

\mbbi0\sbSpecTh\zzPr{ABBFR}{ABBFR}{M.\,A.\,Astaburuaga, Ph.\,Briet, \Bruneau, C.\,Fernandez, \Raikov}
{Dynamical resonances and SSF singularities for a magnetic Schroedinger operator}{arXiv:0710.0502}

\mbbi0\sbOthers\zzPr{As}{Ast}{\Astashkin}{\AndSoOn}{\AndSoOn}
\mbbi1\sbAlgTop\zzBk{At}{At}{\Atiyah}{Lectures on $K$- theory}{Harvard Univ. Cambridge Mass. 1965\libcode{no lib code}}
   \newcount\bbiAPS
\pbbi0\sbDiffGm\zzPr{APS}{APS75}{\Atiyah, \Patodi, \Singer}{Spectral Asymmetry and Riemannian Geometry. I}{\jnProcCambPhilSoc{77}{1975}{43--69}}\bbiAPS
\pbbi0\sbDiffGm\zzPr{APS}{APS76}{\Atiyah, \Patodi, \Singer}{Spectral Asymmetry and Riemannian Geometry. III}{\jnProcCambPhilSoc{79}{1976}{71--99}}\bbiAPS

\mbbi0\sbSpecTh\zzPr{AHS}{AHS}{\Avron, \Herbst,
\Simon}{\Schroedinger\ operators with magnetic fields. I. General
interactions}{\jnDukeMJ{45}{1978}{847--883}}
\mbbi0\sbSSF\zzPr{ASS}{ASS}{\Avron, \Seiler, \Simon}{The index of
a pair of projections}{\jnFuncAnal{120}{1994}{220--237}}
   \newcount\bbiAz
\pbbi1\sbScatTh\zzPr{Az}{Az}{\Azamov}{Infinitesimal spectral flow and scattering matrix}{arXiv:0705.3282v4}\bbiAz
\pbbi1\sbScatTh\zzPr{Az}{Az2}{\Azamov}{Pushnitski's $\mu$\tire invariant and \Schroedinger\ operators with embedded eigenvalues}{arXiv:0711.1190v1}\bbiAz
\pbbi1\sbScatTh\zzPr{Az}{Az3v6}{\Azamov}{Absolutely continuous and singular spectral shift functions}{Dissertationes Math. {\bf 480} (2011), 1--102}\bbiAz
\pbbi1\sbScatTh\zzPr{Az}{Az4}{\Azamov}{Singular spectral shift and Pushnitski $\mu$\tire invariant}{work in progress}\bbiAz
\pbbi1\sbScatTh\zzPr{Az}{Az5}{\Azamov}{Singular spectral shift is additive}{arXiv:1008.2847v1}\bbiAz
\pbbi1\sbScatTh\zzPr{Az}{Az6}{\Azamov}{Non-trivial singular spectral shift functions exist}{arXiv:1008.4231v1}\bbiAz
\pbbi1\sbScatTh\zzPr{Az}{Az7}{\Azamov}{Resonance index and singular spectral shift function}{arXiv:1104.1903}\bbiAz
\pbbi1\sbScatTh\zzPr{Az}{Az8}{\Azamov}{Absolutely continuous and singular spectral shift functions II}{work in progress}\bbiAz
\pbbi1\sbScatTh\zzPr{Az}{Az9}{\Azamov}{Spectral flow inside essential spectrum}{arXiv 1404.3551v1}\bbiAz
\pbbi1\sbScatTh\zzPr{Az}{Az11}{\Azamov}{Resonance index and singular $\mu$-invariant}{in preparation}\bbiAz
\pbbi1\sbOperTh\zzBk{Az}{Azbook}{\Azamov}{Spectral shift function in von Neumann algebras}{Thesis, 2008 Flinders University, 199pp}\bbiAz
\pbbi1\sbScatTh\zzPr{Az}{Az10}{\Azamov}{A constructive approach to stationary scattering theory}{arXiv math-ph 1302.4142}\bbiAz
\mbbi1\sbSSF\zzPr{ACDS}{ACDS}{\Azamov, \Carey, \DoddsPG, \Sukochev}{Operator integrals, spectral shift and spectral flow}{\jnCanMath{61}{2009}{241--263}}
\mbbi1\sbSSF\zzPr{ACS}{ACS}{\Azamov, \Carey, \Sukochev}{The spectral shift function and spectral flow}{\jnCommMathPhys{276}{2007}{51--91}}
\mbbi1\sbScatTh\zzPr{AzD}{AzDa}{\Azamov, Th.\,Daniels}{Singular spectral shift function for 1D Schr\"odinger operators}{in preparation}
\mbbi1\sbScatTh\zzPr{ADT}{AzDaTa}{\Azamov, Th.\,Daniels, Y.\,Tanaka}{Infinite-dimensional analogues of T.\,Kato's continuous enumeration and spectral flow}{in preparation}
\mbbi1\sbSSF\zzPr{ADS}{ADS}{\Azamov, \DoddsPG, \Sukochev}{The Krein spectral shift function
         in semifinite von Neumann algebras}{\jnIEOT{55}{2006}{347--362}}
   \newcount\bbiAS
\pbbi1\sbNCGeom\zzPr{AS}{AS}{\Azamov, \Sukochev}{A Lidskii type formula for Dixmier traces}{\jnComptRendue{340}{2005}{107--112}}\bbiAS 
\pbbi1\sbSSF\zzPr{AzS}{AS2}{\Azamov, \Sukochev}{Spectral averaging for trace compatible operators}{\jnProcAmerMS{136}{2008}{1769--1778}}\bbiAS

\mbbi1\sbSSF\zzPr{AW}{AW11JTA}{S.\,Azzali, Ch.\,Wahl}{Spectral flow, index and the signature operator}{J. Topol. Anal. 3 (2011), no. 1, 37-–67}

\mbbi1\sbOthers\zzBk{Ba}{Bakh}{N.\,S.\,Bakhvalov}{Numerical Methods}{Moscow, Nauka, 1973 (Russian)}
\mbbi0\sbOperTh\zzBk{Ba}{Ba}{\Banach}{Theorie des operations lineaires}{Chelsea, New York, 1963\libcode{515.72 B212}} 
\mbbi0\sbScatTh\zzBk{BW}{BW}{H.\,Baumg\"artel, M.\,Wollenberg}{Mathematical scattering theory}{Basel; Boston, Birkhauser, 1983\libcode{515.724 B348m}} 
\mbbi1\sbNCGeom\zzPr{BF}{BeF}{\Benameur, \Fack}{Type II noncommutative geometry. I. Dixmier trace in von Neumann algebras}{preprint}
\mbbi1\sbSSF\zzPr{BCPRSW}{BCPRSW}{\Benameur, \Carey, \Phillips, \Rennie, \Sukochev, \Wojciechowski}{An analytic approach to spectral flow
              in von Neumann algebras}{Analysis, geometry and topology of elliptic operators, 297--352, World Sci. Publ., Hackensack, NJ, 2006}

\mbbi1\sbFuncAn\zzBk{BS}{BS}{\Bennett, \Sharpley}{Interpolation of Operators}{Academy Press, 1988} 
\mbbi0\sbPseuDO\zzBk{B}{Ber}{Yu.\,M.\,Berezanski\u\i}{Expansions in eigenfunctions of selfadjoint operators}{AMS, Providence, RI, 1968}
\mbbi0\sbFuncAn\zzBk{BeSh}{BeShu}{\Berezin, \Shubin}{\Schroedinger\ equation}{Dordrecht; Boston: Kluwer Academic Publishers, 1991}
\mbbi1\sbDiffGm\zzBk{BGV}{BGV}{\Berline, \Getzler, \Vergne}{Heat Kernels and Dirac Operators}{\pbSpringer{2004}\libcode{515.353 B515h}} 
\mbbi1\sbOpeAlg\zzPr{Bi}{Bik}{A.\,M.\,Bikchentaev}{On a property of $L_p$- spaces on semifinite \vNa s}{\jnMatZametki{64}{1998}{185--190}}
\mbbi1\sbSSF\zzPr{BE}{BE}{\Birman, \Entina}{The stationary approach in abstract scattering theory}{\jnIzvANSSSR{31}{1967}{401--430};
               English translation in Math. USSR Izv. \volume{1} (1967)}
\mbbi0\sbOperTh\zzPr{BKS}{BKS}{\Birman, \Koplienko, \Solomyak}{Estimates for the spectrum of the difference between fractional
              powers of two self-adjoint operators}{Soviet Mathematics, (3) \VolYearPP{19}{1975}{1--6}}
\mbbi0\sbScatTh\zzPr{BK}{BK62DAN}{\Birman, \KreinMG}{On the theory of wave operators and scattering operators}{\jnDoklANSSSR{144}{1962}{475--478}}
\mbbi0\sbSSF\zzPr{BP}{BP98IEOT}{\Birman, \Pushnitski}{Spectral shift function, amazing and multifaceted. Dedicated to the memory
                       of Mark Grigorievich Krein (1907--1989)}{\jnIEOT{30}{1998}{191--199}}
    \newcount\bbiBiSo
\pbbi0\sbOperTh\zzBk{BS}{BSbook}{\Birman, \Solomyak}{Spectral theory of self-adjoint operators in Hilbert space}
         {D.\,Reidel Publishing Co., Dordrecht, 1987}\bbiBiSo
\pbbi1\sbSSF\zzPr{BS}{BS75SM}{\Birman, \Solomyak}{Remarks on the spectral shift function}{\jnSovMath{3}{1975}{408--419}}\bbiBiSo
\pbbi1\sbMOI\zzPr{BS}{BS66DOI}{\Birman, \Solomyak}{Double Stieltjes operator integrals. I}{in \lq Problems of Mathematical Physics,\rq
       No.\,1, Spectral Theory and Wave Processes, \jnIzdatLenUniv{1966}{33--67}}\bbiBiSo
\pbbi1\sbMOI\zzPr{BS}{BS67DOI}{\Birman, \Solomyak}{Double Stieltjes operator integrals. II}{in \lq Problems of Mathematical Physics,\rq
            No.\,2, Spectral Theory, Diffraction Problems, \jnIzdatLenUniv{1967}{26--60}}\bbiBiSo
\pbbi1\sbMOI\zzPr{BS}{BS73DOI}{\Birman, \Solomyak}{Double Stieltjes operator integrals. III}{\jnIzdatLenUniv{1973}{27--53}}\bbiBiSo
\pbbi1\sbMOI\zzPr{BS}{BS03DOI}{\Birman, \Solomyak}{Double operator integrals in a Hilbert space}{\jnIEOT{47}{2003}{131--168}}\bbiBiSo
\pbbi1\sbFuncAn\zzPr{BS}{BS77RMS}{\Birman, \Solomyak}{Estimates of singular numbers of integral operators}{\jnRussMathSurvey{32:1}{1977}{15--89}}\bbiBiSo

    \newcount\bbiBiYa
\pbbi1\sbSSF\zzPr{BY}{BYa92AA}{\Birman, \Yafaev}{The spectral shift function, the work of M.\,G.\,Krein and its further development (in Russian)}
            {\jnAlgAnal{4}{1992}{no. 5, 1--44}}\bbiBiYa
\pbbi1\sbScatTh\zzPr{BY}{BYa92AA2}{\Birman, \Yafaev}{Spectral properties of the scattering matrix (in Russian)}
            {\jnAlgAnal{4}{1992}{no. 5, 1--27}; English translation in {St.\,Petersburg Math. J.} \VolYearPP{4}{1993}{no. 6}}\bbiBiYa

\mbbi1\sbNCGeom\zzBk{Bl}{Bl}{\Blackadar}{$K$-theory for Operator Algebras}{Math. Sci. Res. Inst. Publ., 5, Springer, New York, 1986\libcode{512.55 B627K}} 
\mbbi0\sbQuanFT\zzBk{BoSh}{BSh}{\Bogolyubov, \Shirkov}{Introduction to the theory of quantized fields}{John Wiley, New York, 1980\libcode{530.14 B675i.3}} 
\mbbi0\sbQuanFT\zzBk{Bo}{Bohm}{A.\,B\" ohm}{Quantum mechanics}{Springer-Verlag, New York, Heidelberg, Berlin}
\mbbi1\sbFuncAn\zzBk{BD}{BD}{\Bonsall, \Duncan}{Numerical Ranges of Operators on Normed Spaces and of Elements of Normed Algebras}{Cambridge University Press, 1971}

\mbbi0\sbSpecTh\zzPr{BBR}{BBR}{\Bony, \Bruneau, \Raikov}{Resonances and spectral shift function near the Landau levels}{arXiv:math/0603731}

\mbbi0\sbNCGeom\zzPr{BF}{BFTokyo98}{\BoosBavnbek, \Furutani}{The Maslov index: a functional analytic definition and the spectral flow formula}{\jnTokyoMath{21}{1998}{No.\,1, 1--34}}
\mbbi0\sbNCGeom\zzPr{BLP}{BLP}{\BoosBavnbek, \Lesch \ and \Phillips}{Unbounded Fredholm operators and spectral flow}{\jnCanMath{57}{2005}{225--250}}
\mbbi1\sbAlgTop\zzBk{BGJ}{BGJ}{\Bott, \Gitler, \James}{Lectures on Algebraic and Differential Topology}{LNM 279, Springer-Verlag, Berlin, 1972}
\mbbi1\sbSpecTh\zzPr{Br}{deBranges}{\Branges}{Perturbation of self-adjoint transformations}{\jnAmerJMath{84}{1962}{543--560}}
    \newcount\bbiBR
\pbbi1\sbOpeAlg\zzBk{BR}{BR1}{\Bratteli, \Robinson}{Operator Algebras and Quantum Statistical Mechanics I}{\pbSpringer{1979}\libcode{512.55 B824 1}}\bbiBR 
\pbbi1\sbOpeAlg\zzBk{BR}{BR2}{\Bratteli, \Robinson}{Operator Algebras and Quantum Statistical Mechanics II}{\pbSpringer{1979}\libcode{512.55 B824 2}}\bbiBR 

\mbbi0\sbAlgTop\zzBk{B}{Bred67}{\Bredon}{Sheaf theory}{McGraw-Hill Book Company, New York, 1967\libcode{514.22 B831}\libcode{S972}\libcode{T311}}
    \newcount\bbiBr
\pbbi1\sbOpeAlg\zzPr{Br}{Br68MAnn}{\Breuer}{Fredholm theories in von Neumann algebras. I}{\jnMathAnn{178}{1968}{243--254}}\bbiBr
\pbbi1\sbOpeAlg\zzPr{Br}{Br69MAnn}{\Breuer}{Fredholm theories in von Neumann algebras. II}{\jnMathAnn{180}{1969}{313--325}}\bbiBr
\mbbi0\sbOperTh\zzBk{Brod}{BrTMM32}{M.\,S.\,Brodskii}{Triangular and Jordan representations of linear operators}{\libcode{511 T772 32}}
\mbbi1\sbOperTh\zzPr{Brn}{Br86}{\Brown}{Lidskii's theorem in the type II case}{in \emph{Proc. U.S.-Japan Seminar, Kyoto 1983}, Pitman Research Notes, Math. Ser., \VolYearPP{123}{1986}{1--35}}
\mbbi1\sbAlgTop\zzPr{BDF}{BDF77AnnM}{\Brown, \Douglas, \Fillmore}{Extensions of $C^*$- algebras and $K$- homology}{\jnAnnMath{105}{2}{1977}{265--324}; MR {\bf 56} 16399}
\mbbi1\sbOperTh\zzPr{BK}{BK90JOT}{\Brown, \Kosaki}{Jensen's inequality in semi-finite \vNa s}{\jnOperTheory{23}{1990}{3--19}}
\mbbi0\sbSSF\zzPr{BPR}{BPR04AA}{\Bruneau, \Pushnitski, \Raikov}{Spectral shift function in strong magnetic fields}{\jnAlgAnal{16}{2004}{207--238}} 
\mbbi0\sbOthers\zzBk{BD}{BurFaires}{R.\,L.\,Burden, J.\,D.\,Faires}{Numerical Analysis, 8th ed.}{Thomson Brooks/Cole}
\mbbi1\sbSSF\zzPr{BF}{BF60DAN}{\Buslaev, \Faddeev}{On trace formulas for a singular differential Sturm-Liouville operator}{\jnDoklANSSSR{132}{1960}{13--16}; English transl. in Soviet Math. Dokl. \volume{3} (1962)}
    \newcount\bbiCP
\mbbi1\mbOthers\zzPr{CFM}{CFM}{\Carey, \Farber, \Mathai}{Anomalies, complex determinants and von Neumann algebras}{unpublished notes}
\mbbi0\mbOthers\zzPr{CHMM}{CHMM}{\Carey, K.\,C.\,Hannabuss, \Mathai, P.\,McCann}{The quantum hall effect on the hyperbolic plane}
  {\jnCommMathPhys{190}{1998}{629--673}}
\pbbi1\sbOthers\zzPr{CP}{CP91KT}{\Carey, \Phillips}{Algebras almost commuting with Clifford algebras in a \IIinfty factor}{\jnKTheory{4}{1991}{445--478}}\bbiCP
\pbbi1\sbNCGeom\zzPr{CP}{CP98CJM}{\Carey, \Phillips}{Unbounded Fredholm modules and spectral flow}{\jnCanMath{50}{1998}{673--718}}\bbiCP
\pbbi1\sbNCGeom\zzPr{CP}{CP2}{\Carey, \Phillips}{Spectral flow in Fredholm modules, eta invariants and the JLO cocycle}{\jnKTheory{31}{2004}{135--194}}\bbiCP
    \newcount\bbiCPRS
\pbbi1\sbNCGeom\zzPr{CPRS}{CPRS1}{\Carey, \Phillips, \Rennie, \Sukochev}{The Hochschild class of the Chern character of semifinite spectral triples}{\jnFuncAnal{213}{2004}{111--153}}\bbiCPRS
\pbbi1\sbNCGeom\zzPr{CPRS}{CPRS2}{\Carey, \Phillips, \Rennie, \Sukochev}{The local index formula in semifinite von Neumann algebras I: spectral flow}
       {\jnAdvMath{202}{2006}{451--516}}\bbiCPRS
\pbbi1\sbNCGeom\zzPr{CPRS}{CPRS3}{\Carey, \Phillips, \Rennie, \Sukochev}{The local index formula in semifinite von Neumann algebras II: even case}
       {\jnAdvMath{202}{2006}{517--554}}\bbiCPRS
\pbbi1\sbNCGeom\zzPr{CPRS}{CPRS4}{\Carey, \Phillips, \Rennie, \Sukochev}{Review article}{in progress}\bbiCPRS
\pbbi1\sbNCGeom\zzPr{CPRS}{CPRS5}{\Carey, \Phillips, \Rennie, \Sukochev}{The resolvent cocycle is entire}{in progress}\bbiCPRS
    \newcount\bbiCPS
\pbbi1\sbOthers\zzPr{CPS}{CPS1}{\Carey, \Phillips, \Sukochev}{On unbounded p-summable Fredholm modules}{\jnAdvMath{151}{2000}{140-163}}\bbiCPS
\pbbi1\sbNCGeom\zzPr{CPS}{CPS03AdvM}{\Carey, \Phillips, \Sukochev}{Spectral flow and Dixmier traces}{\jnAdvMath{173}{2003}{68--113}}\bbiCPS

\mbbi1\sbOperTh\zzPr{CPi}{CP77ActM}{\CareyRW, \Pincus}{Mosaics, principal functions, and mean motion in von Neumann algebras}{\jnActaMath{138}{1977}{153--218}}
\mbbi1\sbOperTh\zzPr{CPS}{CPS}{\Carey, D.\,Potapov, \Sukochev}{Spectral flow is the integral of one forms on the Banach manifold of self adjoint Fredholm operators}{Adv. Math. 222 (2009), no. 5, 1809–1849}
\mbbi0\sbCompAn\zzBk{Ca}{Cartan}{\Cartan}{Elementary theory of analytic functions of one or several complex variables}{Addison-Wesley Publ. Company, Inc. 1963}
\mbbi0\sbHomAlg\zzBk{CE}{CE}{\Cartan, \Eilenberg}{Homological algebra}{Princeton, N.\,J., Princeton University Press, 1956\libcode{512.55 C322}}
\mbbi0\sbSymmSp\zzPr{CDS}{CDS}{\Chilin, \DoddsPG, \Sukochev}{The Kadec-Klee property in symmetric spaces of measurable operators}{\jnIsrMath{97}{1997}{203--219}}
\mbbi0\sbSymmSp\zzPr{ChS}{ChS}{\Chilin, \Sukochev}{Weak convergence in non-commutative symmetric spaces}{\jnOperTheory{31}{1994}{No.\,1, 35--65}}
\mbbi1\sbOthers\zzPr{CDSS}{CDSS}{\Coburn, \Douglas, \Schaeffer, \Singer}{$C^*$- algebras of operators on a half space II Index Theory}{\jnIHES{40}{1971}{69--80}}
\mbbi1\sbOthers\zzPr{CMS}{CMS}{\Coburn, \Moyer, \Singer}{$C^*$-algebras of almost periodic pseudo-differential operators}{\jnActaMath{130:3-4}{1973}{279--307}}
    \newcount\bbiCo
\pbbi1\sbNCGeom\zzPr{C}{Co88CMP}{\Connes}{The action functional in noncommutative geometry}{\jnCommMathPhys{117}{1988}{673--683}}\bbiCo
\pbbi1\sbNCGeom\zzPr{C}{Co85IHES}{\Connes}{Noncommutative Differential Geometry}{\jnIHES{62}{1985}{41--144}}\bbiCo
\pbbi0\sbNCGeom\zzPr{C}{Co88KT}{\Connes}{Cyclic cohomology of Banach algebras and characters of $\theta$- summable Fredholm modules}{\jnKTheory{1}{1988}{519--548}}\bbiCo
\pbbi0\sbNCGeom\zzPr{C}{Co89ET}{\Connes}{Compact metric spaces, Fredholm modules and hyperfiniteness}{\jnErgodicTheory{9}{1989}{207--220}}\bbiCo
\pbbi1\sbNCGeom\zzBk{C}{CoNG}{\Connes}{Noncommutative Geometry}{Academic Press, San Diego, 1994}\bbiCo 
\pbbi1\sbNCGeom\zzPr{C}{Co95LMPh}{\Connes}{Geometry from the spectral point of view}{\jnLetMathPhys{34}{1995}{203--238}}\bbiCo
\mbbi1\sbNCGeom\zzPr{CM}{CM95GAFA}{\Connes, \Moscovici}{The local index formula in noncommutative geometry}{\jnGAFA{5}{1995}{174--243}}
\mbbi0\sbFuncAn\zzBk{CSF}{CSF}{\Cornfeld, \Sinai, \Fomin}{Ergodic theory}{Springer-Verlag, New York, 1982\libcode{515.42 K84e}} 
\mbbi0\sbMOI\zzPr{DK}{DalKr}{\Daletskii, \KreinSG}{Integration and differentiation of functions of Hermitian operators
   and applications to the theory of perturbations}{Vorone\v z. Gos. Univ., Trudy Sem.
   Funkcional. Anal. \VolYearPP{1}{1956}{81--105} (Russian)}
    \newcount\bbiDix
\pbbi0\sbOpeAlg\zzBk{Di}{Dix}{\Dixmier}{Les alg\`ebres d'op\'erateurs dans l'espace Hilbertien (Alg\`ebres de von Neumann)}{\pbGauthier{1969}}\bbiDix 
\pbbi1\sbOpeAlg\zzBk{Di}{DixCStAlg}{\Dixmier}{$C^*$-algebras}{North Holland, New York, 1977\libcode{512.55 D619c.Z}}\bbiDix 
\pbbi1\sbOpeAlg\zzBk{Di}{DixvNa}{\Dixmier}{von Neumann Algebras}{North-Holland, Amsterdam, New York, 1981\libcode{512.55 D619a}}\bbiDix 
\pbbi0\sbNCGeom\zzPr{Di}{Di66CR}{\Dixmier}{Existence de traces non-normales}{\jnComptRendue{262}{1966}{A1107--A1108}}\bbiDix

\mbbi0\sbSymmSp\zzPr{DD}{DD}{\DoddsPG, \DoddsTK}{On a submajorization inequality of T.\,Ando}{Oper. Theory Adv. Appl. \VolYearPP{15}{1992}{942--972}}
    \newcount\bbiDDP
\pbbi0\sbSymmSp\zzPr{DDP}{DDP92}{\DoddsPG, \DoddsTK, \Pagter}{Fully symmetric operator spaces}{\jnIEOT{15}{1992}{942--972}}\bbiDDP
\pbbi0\sbSymmSp\zzPr{DDP}{DDP93}{\DoddsPG, \DoddsTK, \Pagter}{Noncommutative K\"othe duality}{\jnTransAmerMathSoc{339}{1993}{717--750}}\bbiDDP

\mbbi1\sbSymmSp\zzPr{DDPS}{DDPS}{\DoddsPG, \DoddsTK, \Pagter, \Sukochev}{Lipschitz continuity of the absolute value and Riesz projections in symmetric operator spaces}
                   {\jnFuncAnal{148}{1997}{28--69}}
\mbbi1\sbSymmSp\zzPr{DPSS}{DPSS}{\DoddsPG, \Pagter, \Semenov, \Sukochev}{Symmetric functionals and singular traces}{\jnPositivity{2}{1998}{47--75}}
    \newcount\bbiDPSSS
\pbbi1\sbSymmSp\zzPr{DPSSS}{DPSSS1}{\DoddsPG, \Pagter, \Sedaev, \Semenov, \Sukochev}{Singular symmetric functionals}{Zap. Nauchn. Sem.
              S.-Peterburg. Otdel. Mat. Inst. Steklov. (POMI), Issled. po Linein. Oper. i Teor. Funkts., \VolNoYearPP{290}{30}{2002}{42--71} (Russian)}\bbiDPSSS
\pbbi1\sbSymmSp\zzPr{DPSSS}{DPSSS2}{\DoddsPG, \Pagter, \Sedaev, \Semenov, \Sukochev}{Singular symmetric functionals and Banach limits with additional
              invariance properties}{Izv. Ross. Akad. Nauk, Ser. Mat., \VolNoYearPP{67}{6}{2003}{111--136} (Russian)}\bbiDPSSS
    \newcount\bbiDPSSSR
\pbbi1\sbSymmSp\zzPr{ÄÏÑÑÑ}{DPSSS1R}{Ï.\,Ã.\,Äîääñ, Á.\,äå~Ïàãòåð, À.\,À.\,Ñåäàåâ, Å.\,Ì.\,Ñåìåíîâ, Ô.\,À.\,Ñóêî÷åâ}{Ñèíãóëÿðíûå ñèììåòðè÷íûå
              ôóíêöèîíàëû}{Çàï. Íàó÷. Ñåì. Ñ.-Ïåòåðáóðã, Îòäåë Ìàò. Èíñò. Ñòåêëîâ. (ÏÎÌÈ), Èññëåä. ïî Ëèíåéí. Îïåð. è Òåîð. Ôóíêö. \VolNoYearPP{290}{30}{2002}{42--71}}\bbiDPSSSR
\pbbi1\sbSymmSp\zzPr{ÄÏÑÑÑ}{DPSSS2R}{Ï.\,Ã.\,Äîääñ, Á.\,äå~Ïàãòåð, À.\,À.\,Ñåäàåâ, Å.\,Ì.\,Ñåìåíîâ, Ô.\,À.\,Ñóêî÷åâ}{Ñèíãóëÿðíûå ñèììåòðè÷íûå
              ôóíêöèîíàëû è áàíàõîâû ïðåäåëû ñ äîïîëíèòåëüíûìè ñâîéñòâàìè èíâàðèàíòíîñòè}{Èçâ. Ðîññ. Àêàä. Íàóê, Ñåð. Ìàò., \VolNoYearPP{67}{6}{2003}{111--136}}\bbiDPSSSR
\mbbi0\sbSymmSp\zzPr{DSS}{DSS}{\DoddsPG, \Sukochev, \Schluchtermann}{Weak compactness criteria in symmetric spaces of measurable operators}{\jnProcCambPhilSoc{131}{2001}{363--384}}
\mbbi1\sbHarmAn\zzPr{D}{D}{W.\,F.\,Donoghue,Jr}{A theorem of the Fatou type}{Monatshefte f\"ur Mathematik, 67/3}
\mbbi0\sbDiffGm\zzBk{DFN}{DFN1}{\Dubrovin, \Fomenko, \Novikov}{Modern geometry --- methods and applications, Part I}{Springer-Verlag, New York, 1984\libcode{510 G733 93}}
\mbbi0\sbDiffGm\zzBk{DFN}{DFN2}{\Dubrovin, \Fomenko, \Novikov}{Modern geometry --- methods and applications, Part II}{Springer-Verlag, New York, 1984\libcode{510 G733 104}}
\mbbi0\sbGenTop\zzBk{Du}{Du}{\Dugundji}{Topology}{Allyn and Bacon, Inc., Boston, 1966\libcode{514 D868}} 
\mbbi1\sbFuncAn\zzBk{DS}{DS}{\Dunford, \Schwartz}{Linear Operators, Part I: General theory}{Interscience publishers, Inc., New York, 1964}
\mbbi1\sbOperTh\zzPr{DFWW}{DFWW}{\Dykema, \Figiel, \Weiss, \Wodzicki}{Commutator structure of operator ideals}{\jnAdvMath{185}{2004}{1--79}}
    \newcount\bbiDK
\pbbi1\sbOperTh\zzPr{DK}{DK98JRAM}{\Dykema, \Kalton}{Spectral characterization of sums of commutators II}{\jnReineAngew{504}{1998}{127--137}}\bbiDK
\pbbi1\sbOperTh\zzPr{DK}{DK2}{\Dykema, \Kalton}{Sums of commutators in ideals and modules of type II factors}{Ann. Inst. Fourier (Grenoble) \VolYearPP{55}{2005}{931--971}}\bbiDK
\mbbi0\sbSpecTh\zzBk{EK}{EK}{M.\,S.\,P.\,Eastham, H.\,Kalf}{\Schroedinger-type operators with continuous spectra}{Pitman Advanced Publishing Program, Boston, London, Melbourne}
\mbbi0\sbFuncAn\zzBk{E}{Edv}{\Edwards}{Functional analysis: theory and applications}{Holt, Rinehart and Winston, New York, 1965\libcode{no lib code}}
\mbbi0\sbOthers\zzBk{Engl}{Engl}{Unknown}{English-Russian dictionary}{Unknown\libcode{491.732 M947}}
    \newcount\bbiFack
\pbbi0\sbOpeAlg\zzPr{Fac}{Fa82JOT}{\Fack}{Sur la notion de valeur caract\'eristique}{\jnOperTheory{7}{1982}{307--333}}\bbiFack
\pbbi1\sbSymmSp\zzPr{Fac}{Fa04FA}{\Fack}{Sums of commutators in non-commutative Banach function spaces}{\jnFuncAnal{207}{2004}{358--398}}\bbiFack
\mbbi1\sbOpeAlg\zzPr{FK}{FK86PJM}{\Fack, \Kosaki}{Generalised $s$- numbers of $\tau$- measurable operators}{\jnPacJMath{123}{1986}{269--300}}
\mbbi0\sbScatTh\zzPr{Fa}{Fa64}{\Faddeev}{On a model of Friedrichs in the theory of perturbations of the continuous spectrum}{Trudy Mat. Inst. Steklov \VolYearPP{73}{1964}{292--313} (Russian)}
    \newcount\bbiFar
\pbbi0\sbSSF\zzPr{Far}{Fa75JSM}{\Farforovskaya}{Example of a Lipschitz function of self-adjoint operators that
                      gives a nonnuclear increment under a nuclear perturbation}{\jnSovMath{4}{1975}{426--433}}\bbiFar
\pbbi0\sbSSF\zzPr{Far}{Fa80JSM}{\Farforovskaya}{An estimate of the norm $|f(B)-f(A)|$ for self-adjoint
                      operators $A$ and $B$}{\jnSovMath{14}{1980}{1133--1149}}\bbiFar
\mbbi0\sbDiffGm\zzBk{Fe}{Fe}{\Federer}{Geometric measure theory}{Springer, Berlin, Heidelberg, New York, 1969\libcode{515.42 F293}}
\mbbi1\sbNCGeom\zzPr{FGLSch}{FGLSch}{\Fedosov, \Golze, \Leichtnam, \Schrohe}{The noncommutative residue for manifolds with boundary}{\jnFuncAnal{142}{1996}{1--31}}
\mbbi0\sbOthers\zzBk{FN}{FN}{R.\,P.\,Feinerman, D.\,J.\,Newman}{Polynomial approximation}{The Williams \& Wilkins Company, Baltimore}
\mbbi0\sbSymmSp\zzPr{FiKa}{FiKa}{\Figiel and \Kalton}{Symmetric linear functionals on function spaces}{to appear}

    \newcount\bbiFrie
\pbbi0\sbSpecTh\zzPr{Fr}{Fr38}{K.\,O.\,Friedrichs}{\"Uber die Spektralzerlegung eines Integral-operators}{\jnMathAnn{115}{1938}{249--272}}\bbiFrie
\pbbi0\sbSpecTh\zzPr{Fr}{Fr48}{K.\,O.\,Friedrichs}{On the perturbation of continuous spectra}{Comm. Pure Appl. Math. \VolYearPP{1}{1948}{361--406}}\bbiFrie
\pbbi0\sbSpecTh\zzBk{Fr}{FrBook}{K.\,O.\,Friedrichs}{Perturbation of spectra in Hilbert space}{Amer. Math. Soc., Providence, R.\,I., 1965}\bbiFrie

\mbbi0\sbNCGeom\zzPr{Fetc}{Fetc}{\Frohlich, \AndSoOn}{Survey}{in "Cargies Summer School series", 2000--}
\mbbi1\sbOpeAlg\zzPr{FKa}{FK52AnnM}{\Fuglede, \Kadison}{Determinant theory in finite factors}{\jnAnnMath{55}{3}{1952}{520--530}}

\mbbi1\sbOthers\zzPr{G}{Gamk}{R.\,Gamkrelidze}{Exponential representation of solutions of ordinary differential equations}{
                  Equadiff IV (Proc. Czechoslovak Conf. Differential Equations and their Applications, Prague, 1977),
                  pp. 118--129, Lecture Notes in Math., \volume{703}, Springer, Berlin, 1979}

\mbbi0\sbCompAn\zzBk{Ga}{Ga}{J.\,B.\,Garnett}{Bounded analytic functions}{Academic Press, 1981}
\mbbi0\sbFuncAn\zzBk{GSh}{GSh}{\Gelfand, \Shilov}{Generalized functions, vol. 1}{New York, Academic Press, 1964}
\mbbi0\scScatTh\zzPr{GG}{GellGold53}{M.\,Gell'Mann, M.\,L.\,Goldberger}{The formal theory of scattering}{Phys. Rev. \VolYearPP{91}{1953}{398--408}}
\mbbi1\sbSSF\zzPr{GMM}{GMM99}{\Gesztesy, \Makarov, \Motovilov}{Monotonicity and concavity properties of the spectral shift function}
              {Stochastic processes, physics and geometry: new interplays, II (Leipzig, 1999), CMS Conf. Proc.,
                  \volume{29}, Amer. Math. Soc., Providence, RI, 2000, 207--222}
                  
    \newcount\bbiGeMa
\pbbi1\sbSSF\zzPr{GM}{GM00JAnalM}{\Gesztesy, \Makarov}{The $\Xi$ operator and its relation to Krein's spectral shift function}
     {\jnAnalMath{81}{2000}{139--183}}\bbiGeMa
\pbbi1\sbSSF\zzPr{GM}{GM03AA}{\Gesztesy, \Makarov}{$SL_2(\mathbb R),$ exponential Herglotz representations,
        and spectral averaging}{\jnAlgAnal{15}{2003}{393--418}}\bbiGeMa
\mbbi1\sbSSF\zzPr{GMN}{GMN}{\Gesztesy, \Makarov, \Naboko}{The spectral shift operator}{}

\mbbi1\sbSSF\zzPr{GS}{GS98ActaM}{\Gesztesy, \Simon}{The xi function}{\jnActaMath{176}{1996}{49--71}}

     \newcount\bbiGetz
\pbbi0\sbNCGeom\zzPr{Ge}{Ge93Top}{\Getzler}{The odd Chern character in cyclic homology and spectral flow}{\jnTopology{32}{1993}{489--507}}\bbiGetz
\pbbi0\sbNCGeom\zzPr{Ge}{Ge}{\Getzler}{Cyclic homology and the Atiyah-Patodi-Singer index theorem}{\AndSoOn}\bbiGetz
\mbbi1\sbNCGeom\zzPr{GeSz}{GeSz}{\Getzler, A.\,Szenes}{On the Chern character of a theta-summable Fredholm module}{\jnFuncAnal{84}{1989}{343--357}}
\mbbi0\sbNCGeom\zzBk{Gi}{Gi}{\Gilkey}{Invariance theory, the heat equation and the Atiyah-Singer index theorem}        {Math. Lecture Ser., 11, Publish or Perish, Wilmington, Del., 1984\libcode{514.7 G474i}} 
\mbbi0\sbQuanFT\zzBk{GJ}{GJ}{\Glimm, \Jaffe}{Quantum Physics: A Functional Integral Point of View}{Springer-Verlag, New-York Heidelberg Berlin, 1981\libcode{515.7 G612f}}
\mbbi0\sbFuncAn\zzBk{Gof}{Gof}{C.\,Goffman, G.\,Pedrick}{First Course in Functional Analysis}{Englewood Cliffs, N.J., Prentice-Hall, 1965\libcode{515.7 G612f}} 
\mbbi1\sbOperTh\zzBk{GK}{GK}{\Gohberg, \KreinMG}{Introduction to the theory of non-selfadjoint operators}{Providence, R.\,I., AMS, \bkTransMathMon{18}{1969}\libcode{511 T772 18}}
\mbbi0\sbSpecTh\zzPr{GMP}{GMP}{\Goldshtein, \Molchanov, \Pastur}{A random one-dimensional \Schroedinger\ operator has a pure point spectrum}{\jnFunkAnalPril{11}{}{1977}{1--10}}
\mbbi0\sbSpecTh\zzPr{Gon}{Gont}{A.\,Gontar}{On the spectrum of the product $R_z(H)V$}{Honours thesis, Flinders University, 2011}
\mbbi0\sbSpecTh\zzPr{Gor}{Go94}{A.\,Ya.\,Gordon}{Pure point spectrum under 1-parameter perturbations and instability of Anderson model}{\jnCommMathPhys{164}{1994}{489--505}}
\mbbi1\sbNCGeom\zzBk{GVF}{GVF}{\GraciaBondia, \Varilly, \Figueroa}{Elements of Noncommutative Geometry}{\pbBirkhauser{2001}\libcode{}}
\mbbi0\sbQuanFT\zzBk{GSV}{GSV}{Green, Schwartz, \Witten}{String theory}{\AndSoOn\libcode{no lib code}} 
\mbbi0\sbFuncAn\zzBk{Gr}{Grn}{\Greenleaf}{Invariant Means on Topological Groups}{Van Nostrand Reinhold, New York, 1969}
\mbbi1\sbDiffGm\zzPr{Grm}{Gr}{\Gromov}{K\"ahler-hyperbolicity and $L^2$ Hodge theory}{\jnDiffGeom{33}{1991}{263--292}}
\mbbi1\sbCompAn\zzBk{GR}{GuRo}{\Gunning, \Rossi}{Analytic functions of several complex variables}{Englewood Cliffs, N.\,J., Prentice-Hall, 1965\libcode{515.94 G976}}
\mbbi0\sbQuanFT\zzBk{Haa}{Haag}{\Haag}{Local Quantum Physics: Fields, Particles, Algebras}{\pbSpringer{1993}\libcode{530.12 H111I}} 
\mbbi0\sbOpeAlg\zzPr{Haa}{Haa}{\Haagerup}{Spectral decomposition of all operators in a ${\rm II}_1$ factor, which is embeddable in $R^\omega$}{(Preliminary version), MSRI 2001}
    \newcount\bbiHal
\pbbi0\sbFuncAn\zzBk{Hal}{HalMT}{\Halmos}{Measure theory}{Princeton, N.J., Van Nostrand, 1950\libcode{515.42 H194m}}\bbiHal 
\pbbi0\sbFuncAn\zzBk{Hal}{HalET}{\Halmos}{Lectures on ergodic theory}{Chelsea, New York, 1956\libcode{515.42 H194l}}\bbiHal 
\pbbi0\sbOthers\zzBk{Hal}{HalFDVS}{\Halmos}{Finite-dimensional Vector Spaces}{Princeton, N.J., Van Nostrand, 1958\libcode{512.52 H194.2}}\bbiHal 
\pbbi0\sbOperTh\zzBk{Hal}{HalHSPB}{\Halmos}{A Hilbert space problem book, Second edition}{New York, Springer-Verlag, 1982\libcode{510 G733 19.2}}\bbiHal 
\mbbi0\sbFuncAn\zzBk{Ha}{Hardy}{\Hardy}{Divergent Series}{Clarendon, Oxford, 1949\libcode{no lib code}}
\mbbi0\sbOpeAlg\zzPr{HS}{HS84}{P.\,de la Harpe, G.\,Skandalis}{D\'eterminant associ\'e \`a une trace sur une alg\`ebre de Banach}{Ann. Inst. Fourier (Grenoble) \VolYearPP{34}{1984}{no.\,1, 241--260}}
\mbbi0\sbCompAn\zzBk{HK}{HK}{W.\,K.\,Hayman, P.\,B.\,Kennedy}{Subharmonic functions}{London, Academic Press, 1976\libcode{510 L84m 9}}
\mbbi1\sbNCGeom\zzPr{Hig}{Hig1}{\Higson}{On the Connes-Moscovici Residue Cocycle}{\AndSoOn}
\mbbi1\sbNCGeom\zzPr{Hig}{Hig2}{\Higson}{The Local Index Formula in Non-commutative Geometry}{Lectures given at the School and Conference on Algebraic $K$- theory and its Applications, Trieste, 8-26 July, 2002}
\mbbi0\sbFuncAn\zzBk{HPh}{HPh}{E.\,Hille, R.\,S.\,Phillips}{Functional Analysis and Semigroups}{AMS, Providence, R.\,I., 1957\libcode{515.7 H651f}}
    \newcount\bbiHoer
\pbbi0\sbPseuDO\zzBk{Hoer}{Hoer1}{\Hoermander}{The Analysis of Linear Partial Differential Operators I}{\pbSpringer{1985}\libcode{515.724 H694a 1}}\bbiHoer 
\pbbi0\sbPseuDO\zzBk{Hoer}{Hoer2}{\Hoermander}{The Analysis of Linear Partial Differential Operators II}{\pbSpringer{1985}\libcode{515.724 H694a 2}}\bbiHoer 
\pbbi0\sbPseuDO\zzBk{Hoer}{Hoer3}{\Hoermander}{The Analysis of Linear Partial Differential Operators III}{\pbSpringer{1985}\libcode{515.724 H694a 3}}\bbiHoer 
\pbbi0\sbPseuDO\zzBk{Hoer}{Hoer4}{\Hoermander}{The Analysis of Linear Partial Differential Operators IV}{\pbSpringer{1985}\libcode{515.724 H694a 4}}\bbiHoer 
\mbbi1\sbCompAn\zzBk{Ho}{Ho}{\Hoffman}{Banach Spaces of Analityc Functions}{Prentice Hall, 1962} 

\mbbi0\sbScatTh\zzPr{I}{Ikebe60}{\Ikebe}{Eigenfunction expansions associated with the Schr\"odinger operators and their application to scattering theory}{\jnArchRatMechAnal{5}{1960}{1--34};
Erratum, Remarks on the orthogonality of eigenfunctions for the Schr\"odinger operator on $\mbR^n,$ J. Fac. Sci. Univ. Tokyo, \VolYearPP{17}{1970}{355-361}}
\mbbi0\sbFuncAn\zzBk{IMcK}{IMcK}{\Ito, \McKean}{Diffusion processes and their sample paths}{Berlin, New York, Springer-Verlag, 1965\libcode{531.16 I89}}

\mbbi0\sbFuncAn\zzBk{Ja}{Ja}{K.\,Jacobs}{Measure and integral}{Academic Press, New-York, San Francisco, London, 1978}
\mbbi1\sbNCGeom\zzPr{JLO}{JLO}{\Jaffe, \Lesniewski, \Osterwalder}{Quantum K-theory. I. The Chern chatacter}{\jnCommMathPhys{118}{1988}{1--14}}
\mbbi0\sbSSF\zzPr{J}{Jav}{\Javrjan}{A certain inverse problem for Sturm-Liouville operators}{Izv. Akad. Nauk Armjan. SSR Ser. Mat. \VolYearPP{6}{1971}{246--251}}
\mbbi0\sbOpeAlg\zzBk{KR}{KR}{\Kadison, \Ringrose}{Fundamentals of the theory of operator algebras}{\pbBirkhauser{1983}\libcode{no lib code}}
\mbbi0\sbStrThe\zzBk{K}{Kaku}{M.\,Kaku}{Strings, Conformal Fields, and Topology: an Introduction}{\libcode{539.72 K13s}}
\mbbi1\sbNCGeom\zzPr{Kal}{Ka98JRAM}{\Kalton}{Spectral characterization of sums of commutators I}{\jnReineAngew{504}{1998}{115--125}}
\mbbi1\sbOperTh\zzPr{K}{Kalt97}{\Kalton}{A note on pairs of projections}{Bol. Soc. Mat. Mexicana (3)\VolNoYearPP{3}{75}{1997}{309--311}}
    \newcount\bbiKato
\pbbi0\sbScatTh\zzPr{Ka}{KaPJA57}{\Kato}{Perturbation of continuous spectra by trace class operators}
   {Proc. Japan. Acad. \VolYearPP{33}{1957}{260--264}}\bbiKato
\pbbi1\sbOperTh\zzBk{Ka}{Kato}{\Kato}{Perturbation theory for linear operators}{Springer-Verlag, Berlin, Heidelberg, New York, 1980\libcode{511.4 K19.2}}\bbiKato 
\pbbi0\sbOperTh\zzPr{Ka}{Kato55}{\Kato}{Notes on projections and perturbation theory}{Technical Report No. 9, University of California at Berkeley, 1955}\bbiKato
\pbbi0\scScatTh\zzPr{Ka}{Kato66}{\Kato}{Wave operators and similarity for some non-selfadjoint operators}{\jnMathAnn{162}{1966}{258--279}}\bbiKato
\pbbi0\scScatTh\zzPr{Ka}{Kato68}{\Kato}{Smooth operators and commutators}{Studia Math. \VolYearPP{31}{1968}{535--546}}\bbiKato
\pbbi0\sbScatTh\zzPr{Ka}{Kato69}{\Kato}{Some results on potential scattering}{Proc. Intrn. Conf. on Functional Anal. and Related Topics, Tokyo, 1969, Tokyo Univ. Press 1970, 206--215}\bbiKato

\mbbi1\sbScatTh\zzPr{KK}{KK71}{\Kato, \Kuroda}{The abstract theory of scattering}{Rocky Mountain J. Math. \VolYearPP{1}{1971}{127--171}}

\mbbi0\sbOthers\zzBk{Kettr}{Kettr}{Kettridge}{Engl-French dictionary}{\libcode{443.2 K43K}} 
\mbbi1\sbHarmAn\zzBk{Ki}{Ki}{A.\,A.\,Kirillov}{Elements of the theory of represnetations}{Berlin, New York: Springer-Verlag, 1976} 
\mbbi1\sbDiffGm\zzBk{KN}{KN}{\Kobayashi, \Nomizu}{Foundations of Differential Geometry}{Interscience Publishers, New York, 1969\libcode{516.36 K75f}} 
\mbbi1\sbFuncAn\zzBk{KF}{KF}{A.\,N.\,Kolmogorov, S.\,V.\,Fomin}{Elements of theory of functions and functional analysis}{} 

    \newcount\bbiKopl
\pbbi1\sbSSF\zzPr{Ko}{Ko84SMJ}{\Koplienko}{On trace formula for non trace class perturbations}{\jnSibMathJ{25}{1984}{62--71}}\bbiKopl
\pbbi1\sbSSF\zzPr{Ko}{Ko85SMJ}{\Koplienko}{Regularized spectral shift function for one-dimensional \Schroedinger\ operator
                             with slowly decreasing potential}{\jnSibMathJ{26}{1985}{72--77}}\bbiKopl
\mbbi0\sbSSF\zzPr{KP}{KP03BLMS}{\Korotyaev, \Pushnitski}{On the high-energy asymptotics of the integrated density of states}{\jnBullLondMathSoc{35}{2003}{770--776}}
\mbbi0\sbSSF\zzPr{KP}{KP03PDE}{\Korotyaev, \Pushnitski}{Trace formulae and high energy asymptotics for the Stark operator}{\jnCommPDE{28}{2003}{817--842}}
\mbbi0\sbSSF\zzPr{KP}{KP04FA}{\Korotyaev, \Pushnitski}{A trace formula and high-energy spectral asymptotics for the perturbed Landau Hamiltonian}{\jnFuncAnal{217}{2004}{221--248}} 

\mbbi0\sbSSF\zzPr{KM}{KM}{\Kostrykin, \Makarov}{On Krein's example}{arXiv: math.SP/0606249v1} 
\mbbi0\sbSpecTh\zzPr{Ko}{Ko86CM}{\Kotani}{Lyapunov exponents and spectra for one-dimensional random \Schroedinger\ operators}{\jnContMath{50}{1986}{277--286}}

    \newcount\bbiKr
\pbbi1\sbSSF\zzPr{Kr}{Kr53MS}{\KreinMG}{On the trace formula in perturbation theory}{Mat. Sb., \VolNoYearPP{33}{75}{1953}{597--626}}\bbiKr
\pbbi1\sbSSF\zzPr{Kr}{Kr62DAN}{\KreinMG}{On perturbation determinants and the trace formula for unitary and selfadjoint operators}{(Russian) \jnDoklANSSSR{144}{1962}{268--271}}\bbiKr
\pbbi0\sbSSF\zzPr{Kr}{Kr63}{\KreinMG}{Some new studies in the theory of perturbations of self-adjoint operators}{(Russian) First Math. Summer
        School (Kanev, 1963), Part I, \lq Naukova Dumka\rq, Kiev, (1964), 103---187;   English transl. in \KreinMG, Topics in differential
        and integral equations and operator theory, Birkh\"auser, Basel, 1983, pp. 107--172}\bbiKr
\pbbi1\sbSSF\zzPr{Kr}{Kr87JOT}{\KreinMG}{On perturbation determinants and trace formula for certain classes of pairs of operators}{(Russian) \jnOperTheory{17}{1987}{129--187}}\bbiKr
\mbbi1\sbFuncAn\zzBk{KPS}{KPS}{\KreinSG, \Petunin, \Semenov}{Interpolation of Linear Operators}{\bkTransMathMon{54}{1982}\libcode{511.T772 54}} 

    \newcount\bbiKur
\pbbi1\sbScatTh\zzBk{Ku}{Kur}{\Kuroda}{An introduction to scattering theory}{Aarhus Universitet, Lecture Notes Series {\bf 51}, 1976\libcode{515.9 K97.A}}\bbiKur
\pbbi1\sbScatTh\zzPr{Ku}{KuJMSJ73I}{\Kuroda}{Scattering theory for differential operators, I, operator theory}{\jnMathSocJap{25}{1973}{75--104}}\bbiKur
\pbbi1\sbScatTh\zzPr{Ku}{KuJMSJ73II}{\Kuroda}{Scattering theory for differential operators, II, self-adjoint elliptic operators}{\jnMathSocJap{25}{1973}{222--234}}\bbiKur

\mbbi1\sbFuncAn\zzBk{ÊÏÑ}{KPSRus}{Ñ.\,Ã.\,Êðåéí, Þ.\,È.\,Ïåòóíèí, Å.\,Ì.\,Ñåìåíîâ}{Èíòåðïîëÿöèÿ ëèíåéíûõ îïåðàòîðîâ}{\pbNaukaR{1978}}

\mbbi0\sbScatTh\zzPr{LF}{LF58}{O.\,A.\,Ladyzhenskaya, \Faddeev}{On the theory of perturbation of continuous spectrum}{\jnDoklANSSSR{120}{1958}{1187--1190}}

\mbbi0\sbQuanFT\zzBk{LL}{LL3}{L.\,D.\,Landau, E.\,M.\,Lifshitz}{Quantum mechanics, 3rd edition}{Pergamon Press}
\mbbi1\sbOpeAlg\zzPr{Le}{Lebed}{A.\,V.\,Lebedev}{Teorema Sege-Kolmogorova-Kreina i determinant Fuglida-Kadisona}{sbornik kakikh-to tezisov, p.\,144}
\mbbi0\sbDiffGm\zzPr{LP}{LP}{\Leichtnam, \Piazza}{Dirac index classes and the noncommutative spectral flow}{\AndSoOn}
\mbbi0\sbNCGeom\zzPr{L}{Le91JOT}{\Lesch}{On the index of the infinitesimal generator of a flow}{\jnOperTheory{26}{1991}{73--92}}
\mbbi0\sbSpecTh\zzBk{LeSa}{LeSa}{\Levitan, \Sargsjan}{Introduction to spectral theory: selfadjoint ordinary differential operators}{Providence, AMS, 1975\libcode{511 T772 39}} 
\mbbi0\sbSpecTh\zzBk{LeZh}{LeZh}{\Levitan, V.\,V.\,Zhikov}{Almost Periodic Functions and Differential Equations}{Cambridge University Press, Cambridge, New York, 1982\libcode{515.53L666p}} 
\mbbi0\sbSpecTh\zzPr{Lid}{Lid}{\Lidskii}{Non-selfadjoint operators with a trace}{\jnDoklANSSSR{125}{1959}{485--587} (Russian)}
\mbbi1\sbSSF\zzPr{L}{Li52UMN}{\Lifshitz}{On a problem in perturbation theory connected with quantum statistics}{\jnUMN{7}{1952}{171--180} (Russian)}
\mbbi1\sbSymmSp\zzBk{LiTz1}{LT1}{\Lindenstrauss, \Tzafriri}{Classical Banach Space I. Sequence Spaces}{\pbSpringer{Berlin-Heidelberg-New York, (1977)}}
\mbbi1\sbSymmSp\zzBk{LiTz2}{LT2}{\Lindenstrauss, \Tzafriri}{Classical Banach Space II. Function Spaces}{\pbSpringer{Berlin-Heidelberg-New York, (1979)}}
\mbbi0\sbScatTh\zzPr{LSch}{LippSch50}{B.\,A.\,Lippmann, J.\,Schwinger}{Variational principles for scattering processes, I}{Phys. Rev. \VolYearPP{79}{1950}{469--480}}
\mbbi1\sbNCGeom\zzBk{Lo}{Lo}{\Loday}{Cyclic Homology}{\pbSpringer{1992}}
\mbbi1\sbNCGeom\zzPr{LSS}{LSS}{\Lord, \Sedaev, \Sukochev}{Dixmier traces as singular symmetric functionals and applications to measurable operators}
     {\jnFuncAnal{224}{2005}{72--106}}
\mbbi1\sbNCGeom\zzPr{LS}{LS}{\Lord, \Sukochev}{Dominated convergence theorem for the noncommutative integral}{not ready yet}
\mbbi1\sbSymmSp\zzPr{Lo}{Lo48ActaM}{\Lorentz}{A contribution to the theory of divergent sequences}{ \jnActaMath{80}{1948}{167--190}}
\mbbi0\sbSpecTh\zzBk{MW}{MW}{\Magnus, \Winkler}{Hill's equation}{New York, Interscience Publishers, 1966\libcode{515.35 M199}}
\mbbi0\sbCompAn\zzBk{Ma}{MarkI}{A.\,I.\,Markushevich}{Theory of functions of a complex variable, volume I}{N.J., Prentice-Hall}
\mbbi0\sbCompAn\zzBk{Ma}{MarkII}{A.\,I.\,Markushevich}{Theory of functions of a complex variable, volume II}{N.J., Prentice-Hall}
\mbbi0\sbCompAn\zzBk{Ma}{MarkIII}{A.\,I.\,Markushevich}{Theory of functions of a complex variable, volume III}{N.J., Prentice-Hall}
\mbbi0\sbOperTh\zzBk{MP}{MP}{M.\,Martin, M.\,Putinar}{Lectures on hyponormal operators}{Basel, Boston, Birkhauser Verlag, 1989\libcode{515.724 M382l}}
\mbbi1\sbNCGeom\zzPr{M}{M}{\Mathai}{Spectral flow, eta invariants and von Neumann algebras}{\jnFuncAnal{109}{1992}{442--456}}
\mbbi0\sbOthers\zzBk{Matlab}{Matlab}{[Matlab]}{Matlab}{Different books\libcode{510.28 V259i}\libcode{519.4 H249m}\libcode{f515.35078 P769m}\libcode{512.5 S657m}}
\mbbi1\sbDiffGm\zzBk{Mi}{Mi84}{\Mishchenko}{Vector Bundles and their Applications}{\pbNauka{1984}}
\mbbi1\sbKTheor\zzPr{MF}{MF}{\Mishchenko, \Fomenko}{The index of Elliptic Operators over $C^*$- algebras}{\jnIzvANSSSR{43}{1979}{831--}}
\mbbi1\sbNCGeom\zzBk{MS}{MSch}{\Moore, \Schochet}{Global Analysis on Foliated Spaces}{Math. Sci. Res. Inst. Publ., 9, Springer, New York, 1988\libcode{514.7 M821g}} 
\mbbi1\sbAlgGeo\zzBk{M}{Mum}{D.\,Mumford}{Algebraic Geometry I: Complex Projective Varieties}{\libcode{516.35 M962}} 
\mbbi1\sbOpeAlg\zzBk{Mu}{Murphy}{G.\,J.\,Murphy}{$C^*$-algebras and Operator Theory}{Academic Press, Inc. 1990}

\mbbi1\sbCompAn\zzBk{Mu}{Mu}{N.\,I.\,Muskhelishivili}{Singular integral operators}{Moscow 1946 \libcode{515.45 M987}}

    \newcount\bbiNab
\pbbi0\sbSSF\zzPr{N}{Nab87}{\Naboko}{Uniqueness theorems for operator-valued functions with positive imaginary part,
       and the singular spectrum in the Friedrichs model}{Ark. Mat. \VolYearPP{25}{1987}{115--140}}\bbiNab
\pbbi0\sbSSF\zzPr{N}{Nab89}{\Naboko}{Boundary values of analytic operator functions
                                 with a positive imaginary part}{\jnSovMath{44}{1989}{786--795}}\bbiNab
\pbbi0\sbSSF\zzPr{N}{Nab90}{\Naboko}{Non-tangential boundary values of operator-valued $R$-functions in a half-plane}
               {Leningrad Math. J. \VolYearPP{1}{1990}{1255--1278}}\bbiNab

\mbbi0\sbSSF\zzPr{NP}{NP}{\Naboko, \Pushnitski}{On the embedded eigenvalues and dense point spectrum of the Stark-like Hamiltonians}{\jnMathNachr{183}{1997}{185--200}}
\mbbi0\sbFuncAn\zzBk{Nar}{Nar}{\Narasimhan}{Analysis on Real and Complex Manifolds}{Paris, Masson; Amsterdam, North-Holland, 1968\libcode{516.36 N218}} 
\mbbi0\sbFuncAn\zzBk{Nat}{Nat}{I.\,P.\,Natanson}{Theory of functions of a real variable}{Frederick Ungar Publishing Co., 1955} 
\mbbi1\sbOpeAlg\zzPr{Ne}{Ne74FA}{E.\,Nelson}{Notes on non-commutative integration}{\jnFuncAnal{15}{1974}{103--116}}

\mbbi1\sbSpecTh\zzPr{vNW}{vNW}{J.\,von Neumann, E.\,Wigner}{Uber merkw\"urdige diskrete Eigenwerte}{Phys.\,Z. \VolYearPP{30}{1929}{465--467}}

    \newcount\bbidPS
\pbbi1\sbMOI\zzPr{dPS}{dPS04FA}{\Pagter, \Sukochev}{Differentiation of operator functions in non-commutative \mbox{$L_p$-spaces}}
                          {\jnFuncAnal{212}{2004}{28--75}}\bbidPS
\pbbi1\sbOperTh\zzPr{dPS}{dPS}{\Pagter, \Sukochev}{Commutator estimates and $\mbR$- flows in non-commutative operator spaces}{preprint}\bbidPS
\mbbi1\sbMOI\zzPr{dPSW}{dPSW02FA}{\Pagter, \Sukochev, H.\,Witvliet}{Double operator integrals}{\jnFuncAnal{192}{2002}{52--111}}
\mbbi0\sbMOI\zzBk{Pa}{Pa}{\Pavlov}{On multiple operator integrals}{"Problems of mathematical analysis", Leningrad University Publ., \VolYearPP{2}{1969}{99--122} (Russian)}
\mbbi0\sbOpeAlg\zzBk{Pe}{Ped}{\Pedersen}{$C^*$- algebras and their automorphisms groups}{London Math. Soc. Monographs, \volume{14}, Academic, New York, 1979\libcode{510.L 84m14}} 
    \newcount\bbiPel
\pbbi1\sbSSF\zzPr{Pel}{PelFA85}{\Peller}{Hankel operators in perturbation theory of unitary and self-adjoint operators}{\jnFuncAnal{19}{1985}{37-51}}\bbiPel
\pbbi1\sbSSF\zzPr{Pel}{PelFA93}{\Peller}{Functional calculus for a pair of almost commuting selfadjoint operators}{\jnFunkAnalPril{112}{2}{1993}{325-345}}\bbiPel
\pbbi1\sbSSF\zzPr{Pel}{Pel1}{\Peller}{An extension of the Koplienko-Neidhardt trace formulae}{preprint}\bbiPel
\pbbi1\sbMOI\zzPr{Pel}{Pel2}{\Peller}{Multiple operator integrals and higher operator derivatives}{\jnFuncAnal{233}{2006}{no. 2, 515--544}}\bbiPel
\mbbi0\sbNCGeom\zzPr{P1}{P1}{\Perera}{Real valued spectral flow in a type \IIinfty factor}{Ph.\,D. Thesis, IUPUI, 1993}
\mbbi0\sbNCGeom\zzPr{P2}{P2}{\Perera}{Real valued spectral flow in a type \IIinfty factor}{preprint, IUPUI, 1993}
\mbbi0\sbPseuDO\zzBk{PDO}{PDO}{}{Pseudodifferential Operators}{sbornik statey, 1967}
\mbbi1\sbNCGeom\zzPr{PR}{PR94JFA}{\Phillips, \Raeburn}{An index theorem for Toeplitz operators with noncommutative symbol space}{\jnFuncAnal{120}{1994}{239--263}}
    \newcount\bbiPhi
\pbbi1\sbSSF\zzPr{Ph}{Ph96CMB}{\Phillips}{Self-adjoint Fredholm operators and spectral flow}{\jnCanMathBull{39}{1996}{460--467}}{\bbiPhi}
\pbbi1\sbSSF\zzPr{Ph}{Ph97FIC}{\Phillips}{Spectral flow in type I and type II factors --- a new approach}{\jnFieldsInsComm{17}{1997}{137--153}}{\bbiPhi}
\mbbi1\sbSpecTh\zzBk{PT}{PoTr}{J.\,P\"oschel, E.\,Trubowitz}{Inverse spectral theory}{Academic Press, Boston}
\mbbi1\sbDiffGm\zzBk{Po}{Post1}{\Postnikov}{Smooth manifolds}{\pbNauka{1988}}
\mbbi1\sbDiffGm\zzBk{Po}{Post2}{\Postnikov}{Differential Geometry}{\pbNauka{1988}}
    \newcount\bbiPov
\pbbi1\sbScatTh\zzPr{Po}{Povz53}{\Povzner}{On the expansion of arbitrary functions in the eigenfunctions of the operator $-\Delta u = cu$}{Mat. Sb. \VolYearPP{32}{1953}{109--156} (Russian),
Amer. Math. Soc. Trans., 2nd Series, \volume{60}, 1967}\bbiPov
\pbbi1\sbScatTh\zzPr{Po}{Povz55}{\Povzner}{On the expansion of arbitrary functions in the eigenfunctions of the Schr\"odinger operator}{\jnDoklANSSSR{104}{1955}{360--363} (Russian)}\bbiPov

\mbbi0\sbOperTh\zzBk{Pow}{Pow}{S.\,C.\,Power}{Hankel operators on Hilbert space}{Research Notes in Mathematics 64, Pitman Advanced Publishing Program\libcode{515.73 P887}}

\mbbi1\sbNCGeom\zzPr{Pr}{Prin}{\Prinzis}{Traces Residuelles et Asymptotique du Spectre d'Operateurs Pseudo-Differentiels}{Th\`ese, Universit\'e de Lyon, unpublished}
\mbbi0\sbCompAn\zzBk{Pr}{Priv}{\Privalov}{Boundary properties of analytic functions}{GITTL, Moscow, 1950 (Russian)\libcode{515.25 P961r}}
    \newcount\bbiPu
\pbbi0\sbSSF\zzPr{Pu}{Pu97JMP}{\Pushnitski}{The spectrum of Liouville operators and multiparticle Hamiltonians associated to one-particle Hamiltonians with singular continuous spectrum}{\jnMathPhys{38}{1997}{2266--2273}}\bbiPu
\pbbi0\sbSSF\zzPr{Pu}{Pu99JMP}{\Pushnitski}{Estimates for the spectral shift function of the polyharmonic operator}{\jnMathPhys{40}{1999}{5578--5592}}\bbiPu
\pbbi0\sbSSF\zzPr{Pu}{Pu00PDE}{\Pushnitski}{Spectral shift function of the \Schroedinger\ operator in the large coupling constant limit}{\jnCommPDE{25}{2000}{703--736}}\bbiPu
\pbbi0\sbSSF\zzPr{Pu}{Pu01FA}{\Pushnitski}{The spectral shift function and the invariance principle}{\jnFuncAnal{183}{2001}{269--320}}\bbiPu
\pbbi0\sbSSF\zzPr{Pu}{Pu97AA}{\Pushnitski}{Representation for the spectral shift function for perturbations of a definite sign (in Russian)}{\jnAlgAnal{9}{1997}{197--213}}\bbiPu
\pbbi0\sbSSF\zzPr{Pu}{Pu07arx}{\Pushnitski}{Differences of spectral projections and scattering matrix}{arXiv: math/0702253v1}\bbiPu
\pbbi0\sbSSF\zzPr{Pu}{Pu08AMST}{\Pushnitski}{The spectral flow, the Fredholm index and the spectral shift function}{Amer. Math. Soc. Transl. Ser. 2, 225, 141-155}\bbiPu
\mbbi0\sbSSF\zzPr{PR}{PR02PDE}{\Pushnitski, \Ruzhansky}{Spectral shift function of the \Schroedinger\ operator in the large coupling constant limit. II. Positive perturbations}{\jnCommPDE{27}{2002}{1373--1405}}
   \newcount\bbiRS
\pbbi1\sbFuncAn\zzBk{RS}{RS1}{\Reed, \Simon}{Methods of modern mathematical physics: 1. Functional analysis}{Academic Press, New York, 1972\libcode{530.15 R325}}\bbiRS
\pbbi1\sbHarmAn\zzBk{RS}{RS2}{\Reed, \Simon}{Methods of modern mathematical physics: 2. Fourier analysis}{Academic Press, New York, 1975\libcode{530.15 R325}}\bbiRS
\pbbi1\sbScatTh\zzBk{RS}{RS3}{\Reed, \Simon}{Methods of modern mathematical physics: 3. Scattering theory}{Academic Press, New York, 1979\libcode{530.15 R325}}\bbiRS
\pbbi1\sbOperTh\zzBk{RS}{RS4}{\Reed, \Simon}{Methods of modern mathematical physics: 4. Analysis of operators}{Academic Press, New York\libcode{530.15 R325}}\bbiRS
\mbbi1\sbHarmAn\zzPr{Re}{Rei}{H.\,Reiter}{Classical harmonic analysis and locally compact groups}{Oxford Math. Monographs, 1968\libcode{515.53 R379}}
\mbbi1\sbOpeAlg\zzBk{Ri}{Rickart}{\Rickart}{General theory of Banach algebras}{D. van Nostrand Company, Inc., Princeton, 1965}
\mbbi0\sbFuncAn\zzBk{Ri}{RieFA}{\Riesz}{Functional Analysis}{1965\libcode{515.73 R548a}}
\mbbi0\sbFuncAn\zzBk{RiNa}{RiNa}{\Riesz, \SzNagy}{Functional Analysis}{New York, F.\,Unger, 1955\libcode{515.73 R548}}
\mbbi0\sbSpecTh\zzPr{RJMS}{RJMS}{\Rio, \Jitomirskaya, \MakarovN, \Simon}{Singular continuous spectrum is generic}{\jnBullAMS{31}{1994}{208--212}}
\mbbi0\sbSpecTh\zzPr{RMS}{RMS}{\Rio, \MakarovN, \Simon}{Operators with singular continuous spectrum}{\jnCommMathPhys{165}{1994}{59--67}}
\mbbi0\sbSpecTh\zzPr{RS}{RoSa}{J.\,Robbin, D.\,Salamon}{The spectral flow and the Maslov index}{\jnBullLondMathSoc{27}{1995}{1--33}}
\mbbi0\sbSpecTh\zzBk{Ro}{Ro}{C.\,A.\,Rogers}{Hausdorff measures}{Cambridge University Press, 1970}
\mbbi0\sbScatTh\zzPr{R}{RoPJM57}{M.\,Rosenblum}{Perturbation of the continuous spectrum and unitary equivalence}{\jnPacJMath{7}{1957}{997--1010}}

   \newcount\bbiRud
\pbbi1\sbFuncAn\zzBk{Ru}{RudFA}{\Rudin}{Functional Analysis}{McGraw-Hill, New York, 1973\libcode{515.7 R916}}\bbiRud 
\pbbi0\sbFuncAn\zzBk{Ru}{RudPMA}{\Rudin}{Principles of Mathematical Analysis}{McGraw-Hill, New York, 1976\libcode{515 R916.3}}\bbiRud 
\pbbi0\sbCompAn\zzBk{Ru}{RudRCA}{\Rudin}{Real and Complex Analysis}{McGraw-Hill, New York, 1987\libcode{515 R916r.3}}\bbiRud 
\pbbi0\sbHarmAn\zzBk{Ru}{RudFAG}{\Rudin}{Fourier Analysis on Groups}{Interscience Publishers, New York, 1962\libcode{515.24 R916}}\bbiRud 
\mbbi0\sbQuanFT\zzBk{Ru}{Rue}{\Ruelle}{Statistical Mechanics: Rigorous Results}{New York, W.\,A.\,Benjamin, 1969\libcode{530.13 R921}} 
\mbbi1\sbNCGeom\zzPr{R}{R03KT}{\Rennie}{Smoothness and locality for non-unital spectral triples}{\jnKTheory{28}{2003}{127--165}}
\mbbi1\sbNCGeom\zzPr{R}{R04KT}{\Rennie}{Summability for nonunital spectral triples}{\jnKTheory{31}{2004}{71--100}}
\mbbi1\sbOperTh\zzBk{Sad}{Sad}{V.\,A.\,Sadovnichiy}{Operator theory}{Moscow, MGU Publ., 1986}
\mbbi1\sbOpeAlg\zzBk{Sak}{Sak}{\Sakai}{$C^*$- algebras and $W^*$- algebras}{\pbSpringer{1971}, pp.\, 256\libcode{512.5 S158}}
\mbbi1\sbOthers\zzBk{Saks}{Saks}{Saks}{Theory of integral}{\libcode{512.5 S158}}
\mbbi0\sbAlgTop\zzBk{Sato}{Sato}{\Sato}{Algebraic topology: an intuitive approach}{Providence, R.I., AMS, 1999\libcode{511 T772 183}}
\mbbi0\sbFuncAn\zzBk{Sch}{Sch}{\Schwartz}{Nonlinear Functional Analysis}{Gordon and Breach Science Publishers, New-York, London, Paris\libcode{515.7 S398}}
\mbbi0\sbPseuDO\zzBk{Sch}{Schech}{M.\,Schechter}{Spectra of partial differential operators}{North Holland, 1971}
\mbbi0\sbSSF\zzPr{Sem}{Sem}{G.\,Semenov}{Some paper on SSF and SF}{early 80's ???}
\mbbi1\sbOthers\zzBk{SB}{SB}{S\'eminar Bourbaki}{vol. 1971/72, Expos\'es 400--417}{LNM 317, Springer-Verlag, Berlin, 1973}
\mbbi0\sbCompAn\zzBk{Sh}{Shab1}{\Shabat}{Introduction to complex analysis}{Providence, R.\,I., AMS, 1992\libcode{511 T772 110}}
\mbbi0\sbCompAn\zzBk{Sh}{Shab2}{\Shabat}{Distribution of values of holomorphic mappings}{Providence, R.\,I., AMS, 1985\libcode{511 T772 61}}
\mbbi0\sbCompAn\zzBk{Sh}{Shab3}{\Shabat}{Introduction to complex analysis, part II}{Providence, R.\,I., AMS, 1992} 
\mbbi0\sbAlgGeo\zzBk{Sh}{Shaf}{\Shafarevich}{Basic algebraic geometry}{Springer-Verlag, Berlin, New York, 1974\libcode{516.35 S525}}
    \newcount\bbiShu
\pbbi1\sbOthers\zzPr{Sh}{Shu78UMN}{\Shubin}{Almost periodic functions and partial differential operators}{\jnRussMathSurvey{33:2}{1978}{1--52}}\bbiShu
\pbbi1\sbOthers\zzPr{Sh}{Shu79UMN}{\Shubin}{Spectral theory and index of elliptic operators with almost periodic coefficients}{\jnUMN{34:2}{1979}{95--135}}\bbiShu
\pbbi1\sbPseuDO\zzBk{Sh}{ShuPDO}{\Shubin}{PDO and spectral theory}{\pbSpringer{1978}\libcode{515.724 S562p}}\bbiShu
\pbbi1\sbOthers\zzPr{Sh}{Shu79TrMMO}{\Shubin}{Pseudodifferential almost periodic operators and von Neumann algebras}{\jnTransMoscMathSoc{1}{1979}{103--166}}\bbiShu
\mbbi0\sbCompAn\zzBk{Sil}{Silv1}{\Silverman}{Complex variables}{Houghton Mifflin, Boston, 1975\libcode{no lib code}}
\mbbi0\sbCompAn\zzBk{Sil}{Silv2}{\Silverman}{Complex analysis with applications}{Englewood Cliffs, N.\,J., Prentice-Hall, 1973\libcode{515.9 S587}}
    \newcount\bbiSim
\pbbi1\sbOperTh\zzBk{S}{SimTrId}{\Simon}{Trace ideals and their applications}{London Math. Society Lecture Note Series, \volume{35},
    Cambridge University Press, Cambridge, London, 1979\libcode{510.5 L841}}\bbiSim
\pbbi1\sbQuanFT\zzBk{S}{SimFIQPh}{\Simon}{Functional integration and quantum physics}{New York, Academic Press, 1979\libcode{530.15 S594}}\bbiSim
\pbbi1\sb\zzPr{S}{Si82BAMS}{\Simon}{\Schroedinger\ semigroups}{\jnBullAMS{7}{1982}{447--526}}\bbiSim
\pbbi1\sbSSF\zzPr{S}{Si98PAMS}{\Simon}{Spectral averaging and the Krein spectral shift}{\jnProcAmerMS{126}{1998}{1409--1413}}\bbiSim
\pbbi1\sbOperTh\zzBk{S}{SimTrId2}{\Simon}{Trace ideals and their applications: Second Edition}{Providence, AMS, 2005, Mathematical Surveys and Monographs, \volume{120}}\bbiSim
\mbbi1\sbSpecTh\zzPr{SW}{SW}{\Simon, T.\,Wolff}{Singular continuous spectrum under rank one perturbations and localization
        for random Hamiltonians}{Comm. Pure Appl. Math. \VolYearPP{39}{1986}{75--90}}
    \newcount\bbiSinai
\pbbi0\sbOthers\zzBk{Si}{Sinai1}{\Sinai}{Probability theory: an introductory course}{Springer-Verlag, Berlin, New York, 1982\libcode{519.2 S615k}}\bbiSinai
\pbbi0\sbOthers\zzBk{Si}{Sinai2}{\Sinai}{Theory of dynamical systems}{Aarhus Universitet, Matematisk Institut, 1970\libcode{f514 S615}}\bbiSinai
\mbbi0\sbNCGeom\zzPr{Si}{Si}{\Singer}{Eigenvalues of the Laplacian and invariants of manifolds}{Proceedings of the International Congress, Vancouver 1974, \volume{I}{187--200}}
\mbbi0\sbScatTh\zzBk{Si}{Si}{A.\,G.\,Sitenko}{Scattering theory}{Sprigner-Verlag Berlin\libcode{539.758 S623s}}
\mbbi1\sbMOI\zzPr{SS}{SolSt}{\Solomyak, \Stenkin}{On a class of multiple operator Stieltjes integrals}{"Problems of mathematical analysis", Leningrad University Publ., \VolYearPP{2}{1969}{122--134} (Russian)}
\mbbi1\sbNCGeom\zzBk{ST}{ST}{\Soloviev, \Troitsky}{$C^*$- algebras and Elliptic Operators in Differential Topology}{\pbFactorial{1996}\libcode{no lib code}}
\mbbi0\sbDiffGm\zzBk{Sp}{Sp}{\Spivak}{A Comprehensive Introduction to Differential Geometry}{\volume{1}$2^{nd}$ ed., \pbPublPerish{1979}\libcode{no lib code}}
\mbbi1\sbMOI\zzPr{St}{St}{\Stenkin}{On multiple operator integrals}{\jnIzvVyshUchZav{4}{1977}{102--115}}
\mbbi0\sbOpeAlg\zzBk{SZ}{SZ}{\Stratila, \Zsido}{Lectures on von Neumann algebras}{Abacus Press, Bucharest, 1979\libcode{512.55 S898l}}
\mbbi1\sbOthers\zzPr{Su}{Su67AMM}{\Sucheston}{Banach limits}{\jnAmerMathMonth{74}{1967}{308--311}}
    \newcount\bbiS
\pbbi1\sbOperTh\zzPr{S}{Su00CJM}{\Sukochev}{Operator estimates for Fredholm modules}{\jnCanMath{52}{2000}{849--896}}{\bbiS}
\pbbi1\sbOperTh\zzPr{S}{SuPrep}{\Sukochev}{Unbounded Fredholm modules and submajorization}{preprint}{\bbiS}
\mbbi1\sbSymmSp\zzPr{SCh}{SCh90DAN}{\Sukochev, \Chilin}{Symmetric spaces over semifinite von Neumann algebras}{(Russian) \jnDoklANSSSR{313}{1990}{No.\,4, 811--815}}
\mbbi0\sbAlgTop\zzBk{Sw}{Sw}{\Switzer}{Algebraic Topology --- Homotopy and Homology}{Springer-Verlag, Berlin, New York, 1975\libcode{514.23 S979}}
    \newcount\bbiTak
\pbbi1\sbOpeAlg\zzBk{Ta}{TakI}{\Takesaki}{Theory of Operator Algebras, vol. I}{\pbSpringer{2002}\libcode{no lib code}}{\bbiTak}
\pbbi1\sbOpeAlg\zzBk{Ta}{TakII}{\Takesaki}{Theory of Operator Algebras, vol. II}{\pbSpringer{2002}\libcode{no lib code}}{\bbiTak}
\pbbi1\sbOpeAlg\zzBk{Ta}{TakIII}{\Takesaki}{Theory of Operator Algebras, vol. III}{\pbSpringer{2002}\libcode{no lib code}}{\bbiTak}
\mbbi1\sbPseuDO\zzBk{Tay}{Tay}{\Taylor}{Pseudodifferential Operators}{\pbPrinceton{1981}\libcode{515.724 T244}}
\mbbi0\sbScatTh\zzBk{T}{TayST}{J.\,R.\,Taylor}{Scattering theory}{John Wiley \& Sons, Inc. New York\libcode{539.754 T243}}
\mbbi0\scScatTh\zzPr{Th}{Thoe67}{D.\,Thoe}{Eigenfunction expansions associated with Schr\"odinger operators in $\mbR_n, n\geq 4$}{\jnArchRatMechAnal{26}{1967}{335--356}}
\mbbi0\sbOthers\zzPr{Ti}{Ti}{O.\,E.\,Tikhonov}{Continuity of operator functions in topologies connected with a trace on a von Neumann algebra}
   {\jnIzvVyshUchZav{1}{1987}{77--79}; translation in Soviet Math. (Iz. VUZ) \VolYearPP{31}{1987}{110--114}}
\mbbi1\sbPseuDO\zzBk{Tr}{Tr}{\Treves}{Introduction to Pseudodifferential and Fourier Integral Operators}{New York, Plenum Press, 1980\libcode{515.724 T812}}
\mbbi1\sbOthers\zzBk{VTCh}{VTCh}{N.\,N.\,Vakhaniya, V.\,I.\,Tarieladze, S.\,A.\,Chobanyan}{Probability distributions in Banach spaces}{\pbNauka{1985}}
\mbbi0\sbCompAn\zzBk{Vl}{Vlad}{\Vladimirov}{Methods of the Theory of Functions of Many Complex Variables}{\pbNauka{1964}\libcode{515.94 V865}}
\mbbi0\sbCompAn\zzBk{Vl}{Vlad2}{\Vladimirov}{Equations of mathematical physics}{\pbNauka{1981}\libcode{no lib code}}

\mbbi0\sbCompAn\zzBk{Ve}{Veech}{W.\,A.\,Veech}{A second course in complex analysis}{W.\,A.\,Benjamin, Inc., New York, Amsterdam, 1967}

\mbbi1\sbSSF\zzPr{Voi}{Voi}{\Voiculescu}{On a trace formula of M.\,G.\,Krein}{\jnOperTheory{24}{0000}{00-00}}
    \newcount\bbiWahl
\pbbi0\sbSSF\zzPr{Wa}{Wa}{C.\,Wahl}{Spectral flow as winding number and integral formulas}{Proc. Amer. Math. Soc. 135 (2007), no. 12, 4063–4073} \bbiWahl
\pbbi0\sbSSF\zzPr{Wa}{Wa2}{C.\,Wahl}{A new topology on the space of unbounded
         selfadjoint operators and the spectral flow}{$K$K-theory and spectral flow. $C^\ast$-algebras and elliptic theory II, 297–309, Trends Math., Birkhäuser, Basel, 2008}\bbiWahl
\mbbi0\sbOperTh\zzBk{We}{We}{J.\,Weidmann}{Linear operators in Hilbert space}{\libcode{510 G733 68}}
\mbbi0\sbDiffGm\zzBk{We}{Wells}{\Wells}{Differential Analysis on Complex Manifolds}{Springer-Verlag, New York, 1980\libcode{510 G733 65}}
\mbbi1\sbOperTh\zzPr{Wi}{Wi}{H.\,Widom}{When are differentiable functions are differentiable?}    {"Linear and Complex Analysis Problem Book {\bf 3}, Part 1", Editors V.\,P.\,Havin et al., Lecture Notes in Mathematics {\bf 1573}, Springer-Verlag, Berlin-Heidelberg-New York, 1994, pp. 266--271}
\mbbi1\sbSSF\zzPr{Wi}{Wi67JMAA}{\Williams}{Spectra of products and numerical ranges}{\jnMathAnalAppl{17}{1967}{214--220}}
\mbbi0\sbNCGeom\zzPr{W}{Wo84KT}{\Wodzicki}{Noncommutative residue, Part I, Fundamentals}{\jnKTheory{0}{0}{0} arithmetic and geometry (Moscow, 1984-86), pp. 320-399, Lecture Notes Math., \volume{1289}, \pbSpringer{1987}}

    \newcount\bbiYa
\pbbi1\sbSSF\zzPr{Ya}{YaPrep}{\Yafaev}{A trace formula for the Dirac operator}{preprint}\bbiYa
\pbbi0\sbScatTh\zzBk{Y}{Ya}{\Yafaev}{Mathematical scattering theory: general theory}{Providence, R.\,I., AMS, 1992\libcode{511 T772 105}}\bbiYa
\pbbi0\sbScatTh\zzPr{Y}{Ya2001}{\Yafaev}{Lectures on scattering theory}{Lecture notes prepared by Andrew Hassell, Aust. Nat. Univ. 2001}\bbiYa

\mbbi1\sbFuncAn\zzBk{Y}{Yo}{\Yosida}{Functional Analysis}{Springer-Verlag, Berlin, New York, 1980\libcode{515.7 Y65.6}}

\mbbi0\sbOthers\zzPr{???}{XXX}{Someone unknown}{Something unknown}{Unkown publisher, 19??}
\end{thebibliography}

\end{document}


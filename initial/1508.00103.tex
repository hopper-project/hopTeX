\documentclass[12pt]{article}
\usepackage{amsfonts}
\usepackage{amsthm}
\usepackage{amsmath}
\usepackage[all]{xy}
\usepackage{titlesec}
\usepackage{dsfont}
\usepackage{amssymb}
\usepackage{extarrows}
\usepackage{mathrsfs}
\usepackage{arydshln}
\usepackage{multirow}

\theoremstyle{theorem}
\newtheorem{theorem}{Theorem}[section]

\theoremstyle{definition}
\newtheorem{definition}[theorem]{Definition}

\theoremstyle{proposition}
\newtheorem{proposition}[theorem]{Proposition}

\theoremstyle{corollary}
\newtheorem{corollary}[theorem]{Corollary}

\theoremstyle{lemma}
\newtheorem{lemma}[theorem]{Lemma}

\theoremstyle{remark}
\newtheorem{remark}[theorem]{Remark}

\theoremstyle{example}
\newtheorem{example}[theorem]{Example}

\titleformat{\section}[hang]{\normalfont\normalsize\filright\bf}{\hspace{0.01pt}\arabic{section}}{0.5em}{}
\titlespacing{\section}{0em}{*0.5}{\wordsep}

\arraycolsep=2pt
\textwidth 150mm \textheight 220mm \topmargin=-0.5mm
\oddsidemargin=0.5mm

\title{\textbf{The order of the group of self-homotopy equivalence of wedge spaces}}
\author{\Large Zhongjian, ZHU
\footnote{Email: \emph{zhuzhongjian@amss.ac.cn}}  \\
\normalsize \em{Institute of Mathematics, Chinese Academy of Science, Beijing, China}}
\date{}

\begin{document}

\maketitle
{\noindent\small{\bf{Abstract}}\quad
In this paper $Aut(\Sigma X\vee \Sigma Y)^\#$ the order of the group of self-homotopy equivalence of wedge spaces is studied. Under the condition of reducibility,  we decompose $ Aut(\bigvee\limits_{t=1}^{k}X_{t})$ to the product of subgroups which generalizes the known results for $k=2$. Then we also give the formula for $ Aut(\bigvee\limits_{t=1}^{k}\Sigma X_{t})^{\#}$.

\

{\noindent\small{\bf{keywords}}\quad
  homotopy equivalence, reducible, wedges.}

\
\

\section{Introduction}
\label{intro}
 In this paper, all spaces are connected pointed CW-complexes, all maps and homotopies are based point preserving. $[X,Y]$ denotes the set of basepoint preserving homotopy classes of based maps from $X$ to $Y$. $Aut(X)$ is a subset of $[X,X]$ formed by self-homotopy equivalences. The operation induced by the composition of homotopy classes
makes $Aut(X)$  into a group, which is normally called the group of self-homotopy equivalences of $X$.

 We are given a map $f: X\vee Y\rightarrow X\vee Y$, for $I,J\in \{X,Y\}$, denote $f{\ensuremath{{\scriptstyle\circ}}} i_{I}$ and $p_{J}{\ensuremath{{\scriptstyle\circ}}} f{\ensuremath{{\scriptstyle\circ}}} i_{I}$ by $f_{I}$ and $f_{JI}$ respectively, where $i_{I}: I\rightarrow X\vee Y$ is a coordinate inclusion and $p_{J}:  X\vee Y\rightarrow J $  is a coordinate project. Thus,
there is $f=(f_{X},f_{Y})$ by the universal property of wedge spaces. The group $Aut(X\vee Y)$ is
called reducible if for any$f\in Aut(X\vee Y)$ there are $f_{XX}\in Aut(X)$ and $f_{YY}\in Aut(Y)$.

Suppose that $M$ and $N$ are two subgroups of a group $G$. The set
  $$M\centerdot N=\{mn | m\in M, n\in N\}$$ is called the product of subgroups $M$ and $N$. Note that it is not assumed that any of $M$, $N$ is normal subgroup of $G$, hence $M\centerdot N$ is not generally a subgroup of $G$. Now let
  $$Aut_{X}(X\vee Y)=\{f\in Aut(X\vee Y)| f{\ensuremath{{\scriptstyle\circ}}} i_{X}=i_{X}\},$$
  $$Aut_{Y}(X\vee Y)=\{f\in Aut(X\vee Y)| f{\ensuremath{{\scriptstyle\circ}}} i_{Y}=i_{Y}\}.$$
 In \cite{YS1}, Yu, H.B. and Shen, W.H. showed that  $Aut_{X}(X\vee Y)$ and $Aut_{Y}(X\vee Y)$ are subgroups of $Aut(X\vee Y)$ and they get
 the following Theorem
 \begin{theorem} \label{theorem 1.1}
  If $X$ and $Y$ is simply connected and $Aut(X\vee Y)$ is reducible, then $Aut(X\vee Y)= Aut_{X}(X\vee Y)\centerdot  Aut_{Y}(X\vee Y) $.
 \end{theorem}

 Let $G^{\#}$ denote the order of a group $G$. In Section \ref{sec:2}, we will compute  $Aut(\Sigma X\vee \Sigma Y)^\#$, where $Aut(\Sigma X\vee \Sigma Y)$ can be reducible (Theorem~\ref{theorem 2.6}). The key point is to use Hilton-Minor Theorem.  In Section \ref{sec:3}, we firstly generalize the Theorem \ref{theorem 1.1} to Theorem \ref{theorem 3.2} for  $ Aut(\bigvee\limits_{t=1}^{k}X_{t})$ . Then this generalization enables us to compute  $ Aut(\bigvee\limits_{t=1}^{k}\Sigma X_{t})^{\#}$ (Theorem~\ref{theorem 3.11}).

\section{ Calculation of $Aut(\Sigma X\vee \Sigma Y)^{\#}$  }
\label{sec:2}
\begin{lemma}\label{lemma 2.1}
 The map
 $$  Aut_{X}(X\vee Y)\times  Aut_{Y}(X\vee Y) \xlongrightarrow{\phi}   Aut_{X}(X\vee Y)\centerdot Aut_{Y}(X\vee Y)$$
   given by $\phi(f,g)=f{\ensuremath{{\scriptstyle\circ}}} g$ is a bijection.
\end{lemma}
\begin{proof}
 It is enough to show that $Aut_{X}(X\vee Y)\cap Aut_{Y}(X\vee Y)=(i_{X},i_{Y})$, where $(i_{X},i_{Y})$ is the unit in the group $Aut(X\vee Y)$. It is clear by the definition of $Aut_{X}(X\vee Y)$ and $Aut_{Y}(X\vee Y)$.
\end{proof}
 Let $$[X,X\vee Y]_{\simeq X}=\{g\in [X,X\vee Y]~~|~ p_{X}{\ensuremath{{\scriptstyle\circ}}} g\in Aut(X) \},$$
   $$[Y,X\vee Y]_{\simeq Y}=\{f\in [Y,X\vee Y]~~|~ p_{Y}{\ensuremath{{\scriptstyle\circ}}} f\in Aut(Y) \}.$$
 By \cite{YS1}, we know that if $Aut(X\vee Y)$ is reducible, then
  \begin{align}
  Aut_{X}(X\vee Y)=\{(g, i_{Y})~|~ p_{X}{\ensuremath{{\scriptstyle\circ}}} g\in Aut(X)\}, \label{eq:1}\\
  Aut_{Y}(X\vee Y)=\{(i_{X},f)~|~ p_{Y}{\ensuremath{{\scriptstyle\circ}}} f\in Aut(Y)\}.   \label{eq:2}
\end{align}
    Thus it is easy to get the following Lemma
 \begin{lemma}\label{lemma 2.2}
  If $Aut(X\vee Y)$ is reducible and $X$, $Y$ are simply connected, then the following maps
  $$[X,X\vee Y]_{\simeq X} \xlongrightarrow{\phi_{Y}} Aut_{Y}(X\vee Y), ~~\phi_{Y}(g)=(g, i_{Y})$$
  $$[Y,X\vee Y]_{\simeq Y} \xlongrightarrow{\phi_{X}} Aut_{X}(X\vee Y), ~~\phi_{X}(f)=(i_{X},f)$$
  are isomorphisms of sets.
  \end{lemma}

\begin{remark}\label{remark 2.3}
From Proposition 2.1 of \cite{YS1}, for $(i_{X},f)\in Aut(X\vee Y)$, there is a map $\overline{f}: Y\rightarrow X\vee Y$ such that $(i_{X},\overline{f})$ is the homotopy inverse of  $(i_{X},f)$. Thus, $(i_{X},f){\ensuremath{{\scriptstyle\circ}}} \overline{f}=(i_{X},f){\ensuremath{{\scriptstyle\circ}}}(i_{X},\overline{f}){\ensuremath{{\scriptstyle\circ}}} i_{Y}\simeq i_{Y}$. Now define $f\cdot g=(i_{X},f){\ensuremath{{\scriptstyle\circ}}} g$,  $f^{-1}=\overline{f}$,  then $([Y,X\vee Y]_{\simeq Y}, \cdot )$ becomes a group with unit $i_{Y}$ and  $\phi_{X}$ is an isomorphism of groups.
\end{remark}

Let $B=B_{1}\vee B_{2}\vee\cdots\vee B_{k}$. Take abstract symbols $z_{1}, z_{2},\cdots,z_{k}$. Let $(F, [,])$ be the free non-associative algebraic object (over $\mathbb{Z}$) generated by $z_{1}, z_{2},\cdots,z_{k}$ with one binary operation $[,]$. $M$ is the set of monomials in $F$. The weight $wt(a)$ is the number of factors in $a\in M$.  We define and order a ``set of basic commutators" (page 438 of \cite{BaHH}) $Q\subset M$ inductively as follows:  For weight 1, $z_{1}<z_{2}<\cdots<z_{k}$; Now suppose that all elements of weight $<w$ are defined and ordered, then an element of weight $w>1$ is a bracket $[a,b]$ where $wt(a)+wt(b)=w, a<b$  and if $b=[c,d]$ then $c\leq a$. For $k=2$,
$Q=\{z_{1}<z_{2}<[z_2,[z_1,z_2]]<[z_1,[z_1,z_2]]<\cdots \}$.

Let $A\cong_{S}B$  denote the isomorphic of sets $A$ and $B$. By the Hilton-Milnor Theorem, we get that
\begin{theorem}\label{theorem 2.4}
  Let $B=X\vee Y$, then there is an isomorphism of sets
  $$\prod\limits_{c\in Q}[\Sigma Y, \Sigma \wedge^{c}B]\cong_{S}[\Sigma Y, \Sigma X\vee \Sigma Y] $$
  which is given by $(f_{c})_{c\in Q}\mapsto \sum\limits_{c\in Q}i_{c}{\ensuremath{{\scriptstyle\circ}}} f_{c}$, where $\wedge^{c}B$ and the iterated Whitehead product $i_{c}\in [\Sigma \wedge^{c}B, \Sigma B]$ are defined on page 438 of \cite{BaHH}, and the sum of $\sum\limits_{c\in Q}i_{c}{\ensuremath{{\scriptstyle\circ}}} f_{c}$ is in the order indicated by the order of $Q$.
 \end{theorem}

\begin{corollary}\label{corollary2.5}
If $Aut(\Sigma X\vee \Sigma Y)$ is reducible,  then
$$Aut_{X}(\Sigma X\vee \Sigma Y)\cong_{S}[\Sigma Y, \Sigma X\vee \Sigma Y]_{\simeq Y}\cong_{S}[\Sigma Y, \Sigma X]\times Aut(\Sigma Y)\times\prod\limits_{c\in Q,wt(c)>1}[\Sigma Y, \Sigma \wedge^{c}B]$$
\end{corollary}
\begin{proof}
Note that $i_{z_{1}}=i_{\Sigma X}, i_{z_{2}}=i_{\Sigma Y}$. For any $i_{c}=[a,b]$, $p_{\Sigma Y}{\ensuremath{{\scriptstyle\circ}}} i_{c}=p_{\Sigma Y}{\ensuremath{{\scriptstyle\circ}}} [a,b]=[p_{\Sigma Y}{\ensuremath{{\scriptstyle\circ}}} a, p_{\Sigma Y}{\ensuremath{{\scriptstyle\circ}}} b]$. By the definition of $i_{c}, c\in Q$, if $wt(c)>1$, then there is a factor $i_{\Sigma X}$ in $i_{c}$. Hence $p_{\Sigma Y}{\ensuremath{{\scriptstyle\circ}}} i_{c}=\left\{
                                                                                  \begin{array}{ll}
                                                                                    0, & \hbox{if $c\neq z_{2}$;} \\
                                                                                    i_{\Sigma Y}, & \hbox{if $c= z_{2}$.}
                                                                                  \end{array}
                                                                                \right.$
 So $p_{Y}{\ensuremath{{\scriptstyle\circ}}}(\sum\limits_{c\in Q}i_{c}{\ensuremath{{\scriptstyle\circ}}} f_{c})\in Aut(\Sigma Y)$ if and only if $f_{z_{2}}\in Aut(\Sigma Y)$. Now by Lemma \ref{lemma 2.2} and Theorem \ref{theorem 2.4}, we get the isomorphisms above.
\end{proof}

\begin{theorem}\label{theorem 2.6}
 If $Aut(\Sigma X\vee \Sigma Y)$ is reducible, then there is an isomorphism of sets
  $$
     \begin{array}{rl}
        Aut(\Sigma X\vee \Sigma Y) & \cong_{S}Aut(\Sigma X)\times Aut(\Sigma Y)\times [\Sigma X, \Sigma Y]\times [\Sigma Y, \Sigma X]\times \\
        &  \prod\limits_{c\in Q, wt(c)>1}([\Sigma X, \Sigma \wedge^{c}B]\times [\Sigma Y, \Sigma \wedge^{c}B]) \\
      \end{array}
     $$
   Hence  if $Aut(\Sigma X\vee \Sigma Y)^{\#}<\infty$, then
 $$   \begin{array}{rl}
        Aut(\Sigma X\vee \Sigma Y)^{\#} &=Aut(\Sigma X)^{\#}\cdot Aut(\Sigma Y)^{\#}\cdot[\Sigma X, \Sigma Y]^{\#}\cdot [\Sigma Y, \Sigma X]^{\#}\cdot \\
        &  \prod\limits_{c\in Q, wt(c)>1}([\Sigma X, \Sigma \wedge^{c}B]^{\#}\cdot [\Sigma Y, \Sigma \wedge^{c}B]^{\#}). \\
      \end{array}  $$
\end{theorem}
\begin{proof}
The theorem is easily obtained form Theorem \ref{theorem 1.1}, Lemma \ref{lemma 2.1} and Corollary \ref{corollary2.5}.
\end{proof}

 From Theorem \ref{theorem 2.6}, the following Corollary is immediately obtained.
 \begin{corollary}\label{corollary 2.7}
Let $CW^{k}_{n}$ be the full subcategory of homotopy category formed by $(n-1)$-connected and at
most $(n+k)$-dimensional CW-complexes. Suppose $ Aut(\Sigma X\vee \Sigma Y)$ is reducible where $\Sigma X$, $\Sigma Y$ are two objects of $CW^{k}_{n}$.
\begin{itemize}
  \item [(i)] If $k\leq n-2$, then $ Aut(\Sigma X\vee \Sigma Y)\cong_{S}Aut(\Sigma X)\times Aut(\Sigma Y)\times [\Sigma X, \Sigma Y]\times[\Sigma Y, \Sigma X]$.
  \item [(ii)]If  $k\leq 2n-3$, then $ Aut(\Sigma X\vee \Sigma Y)\cong_{S}Aut(\Sigma X)\times Aut(\Sigma Y)\times [\Sigma X, \Sigma Y]\times[\Sigma Y, \Sigma X]\times[\Sigma X, \Sigma X\wedge Y]\times[\Sigma Y, \Sigma X\wedge Y]$.
 \end{itemize}
\end{corollary}

\begin{example}\label{example2.8}
  $ Aut(S^{n}\vee S^{m})$ for $n>m>1$.

  Since $Hom(H_{k}(S^{n}), H_{k}(S^{m}))=0$ for any $k>0$, $Aut(S^{n}\vee S^{m})$ is reducible \cite{YS2}.
  By Theorem \ref{theorem 2.6}, $Aut(S^{n}\vee S^{m})\cong_{S} Aut(S^{n})\times Aut(S^{m})\times [S^{n}, S^{m}]$. Since $Aut(S^{k})=\mathbb{Z}/2$ for any $k>0$, we get
  \begin{itemize}
    \item [(i)] $ Aut(S^{n}\vee S^{m})^{\#}=\infty$ if and only if $m$ is even and $n=2m-1$;
    \item [(ii)] If $m$ is odd  or $n\neq 2m-1$, then $Aut(S^{n}\vee S^{m})^{\#}=4\pi_{n}(S^{m})^{\#}$.
  \end{itemize}
\end{example}

\begin{example}\label{example2.9}
$ Aut(S^{n}\vee \Sigma \mathbb{R}P^2)$ for $n>1$.

 Since $Hom(H_{k}( \Sigma \mathbb{R}P^2), H_{k}(S^n))=0$ for $k>0$, $Aut(S^{n}\vee \Sigma \mathbb{R}P^2)$ is reducible.

 (i) For $n=2$, $S^2$ and $\Sigma\mathbb{R}P^2$ are two objects of $CW^1_{2}$. Note that $[S^{n}, S^{n}\wedge\Sigma\mathbb{R}P^2]=0$. From Corollary \ref{corollary 2.7} we have
  $$Aut(S^{2}\vee \Sigma \mathbb{R}P^2)\cong_{S}Aut(S^{2})\times Aut(\Sigma\mathbb{R}P^2)\times \pi_{2}(\Sigma\mathbb{R}P^2)\times [\Sigma\mathbb{R}P^2,S^{2}]\times [\Sigma\mathbb{R}P^2,\Sigma^2\mathbb{R}P^2].$$
 It is easy to know $Aut(S^2)=\mathbb{Z}/2$ and $\pi_{2}(\Sigma\mathbb{R}P^2)\cong H_{2}(\Sigma\mathbb{R}P^2)\cong \mathbb{Z}/2.$
 Using the Barratt-Puppe sequence for cofibration sequence $S^{2}\rightarrow S^{2}\rightarrow \Sigma\mathbb{R}P^2$, we get
$$[\Sigma\mathbb{R}P^2,S^{2}]=\mathbb{Z}/2,~~ [\Sigma\mathbb{R}P^2,\Sigma^2\mathbb{R}P^2]=\mathbb{Z}/2.$$
 By Theorem $4$ of $\cite{Sieradski}$, there is a short sequence of groups:
  $$0\rightarrow Ext(\mathbb{Z}/2, \pi_{3}(\Sigma\mathbb{R}P^2))\rightarrow Aut(\Sigma\mathbb{R}P^2)\rightarrow Aut(\mathbb{Z}/2)\rightarrow 1.$$
 From \cite{BaHH}, $\pi_{3}(\Sigma\mathbb{R}P^2)=\mathbb{Z}/4$, hence $Aut(\Sigma\mathbb{R}P^2)=\mathbb{Z}/2.$  Thus
$$ Aut(S^{2}\vee \Sigma \mathbb{R}P^2)^{\#}=32.$$

(ii) For $n=3$, $S^3$ and $\Sigma\mathbb{R}P^2$ are also two objects of $CW^1_{2}$ and since $[\Sigma\mathbb{R}P^2,\Sigma^3\mathbb{R}P^2]=0$,
 $$Aut(S^{3}\vee \Sigma \mathbb{R}P^2)\cong_{S}Aut(S^{3})\times Aut(\Sigma\mathbb{R}P^2)\times \pi_{3}(\Sigma\mathbb{R}P^2)\times [\Sigma\mathbb{R}P^2,S^{3}].$$
  By $ [\Sigma\mathbb{R}P^2,S^{3}]=\mathbb{Z}/2$,  $$ Aut(S^{3}\vee \Sigma \mathbb{R}P^2)^{\#}=32.$$

(iii) For $n>3$, then by Theorem \ref{theorem 2.6},
$$Aut(S^{n}\vee \Sigma \mathbb{R}P^2)\cong_{S}Aut(S^{n})\times Aut(\Sigma\mathbb{R}P^2)\times \pi_{n}(\Sigma\mathbb{R}P^2)$$
 Thus, $$ Aut(S^{n}\vee \Sigma \mathbb{R}P^2)^{\#}=4\cdot\pi_{n}(\Sigma\mathbb{R}P^2)^{\#}.$$
\end{example}

\section{ generalization to $Aut(\bigvee\limits_{t=1}^{k}X_{t})$ and  $Aut(\bigvee\limits_{t=1}^{k}\Sigma X_{t})^{\#}$  }
\label{sec:3}

In this section, we will generalize Theorem \ref{theorem 1.1} to $Aut(\bigvee\limits_{t=1}^{k}X_{t}), k\geq3$, where $X_{t}$ is a simply connected CW-complexes for any $1\leq t\leq k$, then we compute  $Aut(\bigvee\limits_{t=1}^{k}\Sigma X_{t})^{\#}$.

We are given a map $\bigvee\limits_{t=1}^{k}X_{t} \xlongrightarrow{f} \bigvee\limits_{t=1}^{k}X_{t}$. Denote $f_{i}:=f{\ensuremath{{\scriptstyle\circ}}} i_{X_{i}}$ and  $f_{ji}:=p_{X_{j}}{\ensuremath{{\scriptstyle\circ}}} f{\ensuremath{{\scriptstyle\circ}}} i_{X_{i}}$, where $i_{X_{i}}: X_{i}\rightarrow \bigvee\limits_{t=1}^{k}X_{t}$ is a coordinate inclusion and $p_{X_{j}}: \bigvee\limits_{t=1}^{k}X_{t}\rightarrow X_{j}$  is a coordinate project. We have $f=(f_{t})_{t=1}^{k}$  by the universal property of wedge spaces.

 \begin{definition}\label{definition3.1}
The group $Aut(\bigvee\limits_{t=1}^{k}X_{t})$  is
called reducible if for any $f\in Aut(\bigvee\limits_{t=1}^{k}X_{t})$ and any $i\in\{1,2,\cdots,k\}$, $f_{ii}\in Aut(X_{i})$.
  \end{definition}

Let ${\ensuremath{Aut_{/X_{{i}}}(\bigvee\limits_{t=1}^{k}X_{t})}}=\{f\in Aut(\bigvee\limits_{t=1}^{k}X_{t})~|~f_{t}=i_{X_{t}}$ for any $t\neq i\}$.

\begin{theorem}\label{theorem 3.2}
If $ Aut(\bigvee\limits_{t=1}^{k}X_{t})$ is reducible and every $X_{t}$ is simply connected,  then
$$Aut(\bigvee\limits_{t=1}^{k}X_{t})={\ensuremath{Aut_{/X_{{1}}}(\bigvee\limits_{t=1}^{k}X_{t})}}\centerdot{\ensuremath{Aut_{/X_{{2}}}(\bigvee\limits_{t=1}^{k}X_{t})}}\centerdot \cdots\centerdot{\ensuremath{Aut_{/X_{{k}}}(\bigvee\limits_{t=1}^{k}X_{t})}}.$$
\end{theorem}

The following Proposition \ref{proposition 3.3}, Proposition \ref{proposition 3.4}, Proposition \ref{proposition 3.5} and Proposition \ref{proposition 3.6} can be obtained from \cite{YS1} by considering
$$f=(i_{X_1},\cdots,i_{X_{j-1}},f_{j},i_{X_{j+1}},\cdots,i_{X_{k}})\in [\bigvee\limits_{t=1}^{k}X_{t}, \bigvee\limits_{t=1}^{k}X_{t}],  ~~ 1\leq j\leq k$$
as $(f_{j}, i_{Y})\in [X_{j}\vee Y, X_{j}\vee Y ]$, where $Y=\bigvee\limits_{t=1,t\neq j}^{k}X_{t}$.

\begin{proposition}\label{proposition 3.3}
For $f=(i_{X_1},\cdots,i_{X_{j-1}},f_{j},i_{X_{j+1}},\cdots,i_{X_{k}})\in Aut(\bigvee\limits_{t=1}^{k}X_{t})$~($1\leq j\leq k$), then there is a map $\overline{f_{j}}:X_{j}\rightarrow \bigvee\limits_{t=1}^{k}X_{t}$ such that $\overline{f}=(i_{X_1},\cdots,i_{X_{j-1}},\overline{f_{j}},i_{X_{j+1}},\cdots,i_{X_{k}})$ is the homotopy inverse of $f$.
\end{proposition}

\begin{proposition}\label{proposition 3.4}
For any $j\in\{1,\cdots,k\}$, ${\ensuremath{Aut_{/X_{{j}}}(\bigvee\limits_{t=1}^{k}X_{t})}}$ is a subgroup of  $ Aut(\bigvee\limits_{t=1}^{k}X_{t})$.
\end{proposition}

\begin{proposition}\label{proposition 3.5}
  $X_{t}$ is simply connected for any $t\in\{1,\cdots,k\}$, then $f=(i_{X_1},\cdots,i_{X_{j-1}}$, $f_{j},$ $i_{X_{j+1}}$, $\cdots,i_{X_{k}})$
  is in ${\ensuremath{Aut_{/X_{{j}}}(\bigvee\limits_{t=1}^{k}X_{t})}}$ if and only if $p_{X_{j}}{\ensuremath{{\scriptstyle\circ}}} f_{j}\in Aut(X_j)$.
\end{proposition}

\begin{proposition}\label{proposition 3.6}
If $ Aut(\bigvee\limits_{t=1}^{k}X_{t})$ is reducible, every $X_{t}$ is simply connected and $f=(f_{t})_{t=1}^{k}\in Aut(\bigvee\limits_{t=1}^{k}X_{t})$, then for any $j\in\{1,\cdots,k\}$,
$$(i_{X_1},\cdots,i_{X_{j-1}},f_{j},i_{X_{j+1}},\cdots,i_{X_{k}})\in{\ensuremath{Aut_{/X_{{j}}}(\bigvee\limits_{t=1}^{k}X_{t})}}.$$

\end{proposition}

\begin{proof}[Proof of Theorem \ref{theorem 3.2}]

Suppose $f=(f_{t})_{t=1}^{k}\in Aut(\bigvee\limits_{t=1}^{k}X_{t})$. Since $Aut(\bigvee\limits_{t=1}^{k}X_{t})$ can be reducible, by Proposition \ref{proposition 3.6}, for any $j\in\{1,\cdots,k\}$, $$^{j}f:=(i_{X_1},\cdots,i_{X_{j-1}},f_{j},i_{X_{j+1}},\cdots,i_{X_{k}})\in {\ensuremath{Aut_{/X_{{j}}}(\bigvee\limits_{t=1}^{k}X_{t})}}.$$
By Proposition \ref{proposition 3.3}, there is a homotopy inverse of $^{1}f$ with the form
$$\overline{^{1}f}:=(\overline{f_{1}},i_{X_{2}},\cdots,i_{X_{k}})\in {\ensuremath{Aut_{/X_{{1}}}(\bigvee\limits_{t=1}^{k}X_{t})}}$$
where $\overline{f_{1}}: X_{1}\rightarrow \bigvee\limits_{t=1}^{k}X_{t}$. Hence $p_{X_{1}}{\ensuremath{{\scriptstyle\circ}}}\overline{f_{1}}\in Aut(X_{1})$.
For any $l$ with $k\geq l\geq 2$,
$$\overline{^{1}f}{\ensuremath{{\scriptstyle\circ}}}~ ^{l}f=(\overline{f_{j}},i_{X_2},\cdots,i_{X_{l-1}},\overline{^{1}f}{\ensuremath{{\scriptstyle\circ}}} f_{l},i_{X_{l+1}},\cdots,i_{X_{k}})\in Aut(\bigvee\limits_{t=1}^{k}X_{t}).$$
By Proposition \ref{proposition 3.6}, $^{l}h:=(i_{X_1},i_{X_2},\cdots,i_{X_{l-1}},\overline{^{1}f}{\ensuremath{{\scriptstyle\circ}}} f_{l},i_{X_{l+1}},\cdots,i_{X_{k}})\in {\ensuremath{Aut_{/X_{{l}}}(\bigvee\limits_{t=1}^{k}X_{t})}}.$
Now we can easily see that $f= ~^{1}f{\ensuremath{{\scriptstyle\circ}}} ^{2}h{\ensuremath{{\scriptstyle\circ}}} ^{3}h{\ensuremath{{\scriptstyle\circ}}} \cdots{\ensuremath{{\scriptstyle\circ}}} ^{k}h$. The proof is finished.
\end{proof}

Note that if  $\{i_1,i_2,\cdots,i_r\}\bigcap\{j_1,j_2,\cdots,j_s\}=\emptyset$, then
$${\ensuremath{Aut_{/X_{{i_1}}}(\bigvee\limits_{t=1}^{k}X_{t})}}\centerdot\cdots\centerdot{\ensuremath{Aut_{/X_{{i_r}}}(\bigvee\limits_{t=1}^{k}X_{t})}}\bigcap {\ensuremath{Aut_{/X_{{j_1}}}(\bigvee\limits_{t=1}^{k}X_{t})}}\centerdot\cdots\centerdot{\ensuremath{Aut_{/X_{{j_s}}}(\bigvee\limits_{t=1}^{k}X_{t})}}=\{ id\}.$$
Hence  the following Lemma \ref{lemma3.7} which is similar to Lemma \ref{lemma 2.1} is obtained.

\begin{lemma}\label{lemma3.7}
 The map
  $$\prod\limits_{r=1}^{k}{\ensuremath{Aut_{/X_{{r}}}(\bigvee\limits_{t=1}^{k}X_{t})}} \xlongrightarrow{\phi}{\ensuremath{Aut_{/X_{{1}}}(\bigvee\limits_{t=1}^{k}X_{t})}}\centerdot{\ensuremath{Aut_{/X_{{2}}}(\bigvee\limits_{t=1}^{k}X_{t})}}\centerdot \cdots\centerdot{\ensuremath{Aut_{/X_{{k}}}(\bigvee\limits_{t=1}^{k}X_{t})}} $$
given by $\phi((f_{r})_{r=1}^{k})=f_{1}{\ensuremath{{\scriptstyle\circ}}} f_{2}{\ensuremath{{\scriptstyle\circ}}}\cdots{\ensuremath{{\scriptstyle\circ}}} f_{k}$ is a bijection.
 \end{lemma}

Now, the following  Lemmas, Propositions, and Theorems in the rest of the paper are easily generalized from the case $k=2$ in Section \ref{sec:2}. We will omit the proof of them.

 \begin{lemma}\label{lemma 3.8}
  If $Aut(\bigvee\limits_{t=1}^{k}X_{t})$ is reducible and every $X_{t}$ is simply connected, then the following map
  $$[X_{j},\bigvee\limits_{t=1}^{k}X_{t}]_{\simeq X_{j}} \xlongrightarrow{\phi_{j}}{\ensuremath{Aut_{/X_{{j}}}(\bigvee\limits_{t=1}^{k}X_{t})}}$$
  $$\phi_{j}(f)=(i_{X_{1}},\cdots, i_{X_{j-1}},f, i_{X_{j+1}},\cdots,i_{X_{k}} )$$
  is an isomorphism of sets for any $j\in\{1,2,\cdots,k\}$, where $$[X_{j},\bigvee\limits_{t=1}^{k}X_{t}]_{\simeq X_{j}}=\{f\in [X_{j},\bigvee\limits_{t=1}^{k}X_{t}]~~|~p_{X_{j}}{\ensuremath{{\scriptstyle\circ}}} f\in Aut(X_{j})\}.$$
  \end{lemma}

\begin{theorem}\label{theorem 3.9}
 Let $B=X_{1}\vee X_2\vee\cdots \vee X_{k}$, for any $j\in\{1,\cdots,k\}$, there is an isomorphism of sets
  $$\prod\limits_{c\in Q}[\Sigma X_{j}, \Sigma \wedge^{c}B]\cong_{S}[\Sigma X_{j}, \Sigma B] $$
  which is given by $(f_{c})_{c\in Q}\mapsto \sum\limits_{c\in Q}i_{c}{\ensuremath{{\scriptstyle\circ}}} f_{c}$, where the ``set of basic commutators" $Q$,  $\wedge^{c}B$ and the iterated Whitehead product $i_{c}\in [\Sigma \wedge^{c}B, \Sigma B]$ are defined on page 438 of \cite{BaHH}, and the sum of $\sum\limits_{c\in Q}i_{c}{\ensuremath{{\scriptstyle\circ}}} f_{c}$ is in the order indicated by the order of $Q$.
 \end{theorem}

\begin{corollary}\label{corollary3.10}
If $Aut(\bigvee\limits_{t=1}^{k}\Sigma X_{t})$ is reducible, then for any $j\in\{1,\cdots,k\}$,
$$Aut_{/\Sigma X_{j}}(\bigvee\limits_{t=1}^{k}\Sigma X_{t})\cong_{S}[\Sigma X_{j},\bigvee\limits_{t=1}^{k}\Sigma X_{t}]_{\simeq \Sigma X_{j}}\cong_{S}Aut(\Sigma X_{j})\times\prod\limits_{c\in Q, c\neq z_{j}}[\Sigma X_j, \Sigma \wedge^{c}B]$$
\end{corollary}

\begin{theorem}\label{theorem 3.11}
 If $Aut(\bigvee\limits_{t=1}^{k}\Sigma X_{t})$ is reducible, then there is an isomorphism of sets
 $$
     \begin{array}{rl}
        Aut(\bigvee\limits_{t=1}^{k}\Sigma X_{t}) &\cong_{S}\prod\limits_{t=1}^{k}Aut(\Sigma X_{t})\times \prod\limits_{1\leq r<s\leq k}[\Sigma X_{r},\Sigma X_{s}]\times \\
        & \prod\limits_{c\in Q, wt(c)\geq 2, j\in\{1\cdots k\}}[\Sigma X_{j},\Sigma \wedge^{c}B]. \\
      \end{array}
     $$
   Hence  if $Aut(\bigvee\limits_{t=1}^{k}\Sigma X_{t})^{\#}<\infty$, then
$$  Aut(\bigvee\limits_{t=1}^{k}\Sigma X_{t})^{\#}=\prod\limits_{t=1}^{k}Aut(\Sigma X_{t})^{\#}\cdot \prod\limits_{1\leq r<s\leq k}[\Sigma X_{r},\Sigma X_{s}]^{\#}\cdot\prod\limits_{c\in Q, wt(c)\geq 2, j\in\{1\cdots k\}}[\Sigma X_{j},\Sigma \wedge^{c}B]^{\#}  .$$
  \end{theorem}

\begin{example}\label{example3.12}
$Aut(S^n\vee S^m\vee S^l),  n>m>l>1$

By considering the homology groups, it is easy to get the reducibility of $Aut(S^n\vee S^m\vee S^l)$ for  $n>m>l>1$.
\begin{itemize}
  \item [(i)] If $l+m>n+1$, then $[S^n, S^m\wedge S^{l-1}]=0$. By Theorem \ref{theorem 3.11},
   $$Aut(S^n\vee S^m\vee S^l)\cong_{S}Aut(S^n)\times Aut(S^m)\times Aut(S^l)\times \pi_{n}(S^m)\times \pi_{n}(S^l).$$
    $$Aut(S^n\vee S^m\vee S^l)^{\#}=8\cdot \pi_{n}(S^m)^{\#}\cdot \pi_{n}(S^l)^{\#} $$

\item [(ii)]  If $min\{2m+l-1,2l+m-1\}>n+1$, then by Theorem \ref{theorem 3.11}, we have
   $$Aut(S^n\vee S^m\vee S^l)\cong_{S}Aut(S^n)\times Aut(S^m)\times Aut(S^l)\times \pi_{n}(S^m)\times \pi_{n}(S^l)\times \pi_{n}(S^{l+m-1}).$$
    $$Aut(S^n\vee S^m\vee S^l)^{\#}=8\cdot \pi_{n}(S^m)^{\#}\cdot \pi_{n}(S^l)^{\#}\cdot \pi_{n}(S^{l+m-1})^{\#}. $$

\end{itemize}

\end{example}

\begin{thebibliography}{}

\bibitem{BaHH}
Baues, H.J.: Homotopy type and homology. Clarendon Press, (1996).

\bibitem{Sieradski}
Sieradski, A.J.: Stabilization of self-equivalences of the pseudoprojective spaces. The Michigan Mathematical Journal 19.2 (1972), 109-119.

\bibitem{YS1}
Yu, H.B., Shen, W.H.: The Group of Self-homotopy Equivalences of Wedge Spaces. Acta Mathematica Sinica-Chinese edition, Vol.48, No.5 (2005), 895-900.

\bibitem{YS2}
Yu, H.B., Shen, W.H.: On reductibility of the self-homotopy equivalences of wedge spaces. Acta Mathematica Scientia, 32(2), (2012), 813-817.

\end{thebibliography}

\end{document}


\documentclass[12pt]{amsart}
\usepackage{amssymb,amsmath,amsfonts,latexsym}
\usepackage{bm}
\setlength{\textheight}{600pt} \setlength{\textwidth}{475pt}
\oddsidemargin -0mm \evensidemargin -0mm \topmargin -0pt

\setcounter{footnote}{1}

\newtheorem{Theorem}{\sc Theorem}[section]
\newtheorem{Lemma}[Theorem]{\sc Lemma}
\newtheorem{Proposition}[Theorem]{\sc Proposition}
\newtheorem{Corollary}[Theorem]{\sc Corollary}
\newtheorem{Definition}[Theorem]{\sc Definition}
\newtheorem{Example}[Theorem]{\sc Example}
\newtheorem{Remark}[Theorem]{\sc Remark}
\newtheorem{Note}[Theorem]{\sc Note}
\newtheorem{Question}{\sc Question}
\newtheorem{ass}[Theorem]{\sc Assumption}

\begin{document}

\title[Operator Positivity and Analytic Models]{Operator Positivity and Analytic Models of Commuting Tuples of Operators}

\author[Monojit Bhattacharjee]{Monojit Bhattacharjee}
\address{Indian Institute of Science, Department of Mathematics, Bangalore, 560012, India}

\author[Jaydeb Sarkar]{Jaydeb Sarkar}
\address{Indian Statistical Institute, Statistics and Mathematics Unit, 8th Mile, Mysore Road, Bangalore, 560059, India}
\email{jay@isibang.ac.in, jaydeb@gmail.com}

\subjclass[2000]{47A13, 47A15, 47A20, 47A45, 47A80, 47B32, 47B38}

\keywords{Weighted Bergman spaces, hypercontractions, multipliers, reproducing kernel Hilbert spaces, invariant subspaces}

\begin{abstract}
We study analytic models of operators of class $C_{\cdot 0}$ with
natural positivity assumptions. In particular, we prove that for an
$m$-hypercontraction $T \in C_{\cdot 0}$ on a Hilbert space ${\mathcal{H}}$,
there exists a Hilbert space ${\mathcal{E}}$ and a partially isometric
multiplier $\theta \in {\mathcal{M}}(H^2({\mathcal{E}}), A^2_m({\mathcal{H}}))$ such that
\[{\mathcal{H}} \cong {\mathcal{Q}}_{\theta} = A^2_m({\mathcal{H}}) \ominus \theta H^2({\mathcal{E}}), \quad \quad \mbox{and} \quad \quad T \cong
P_{{\mathcal{Q}}_{\theta}} M_z|_{{\mathcal{Q}}_{\theta}},\]where $A^2_m$ is the
weighted Bergman space and $H^2$ is the Hardy space over the unit
disc ${\mathbb{D}}$. We then proceed to study and develop analytic models for
doubly commuting $n$-tuples of operators and investigate their
applications to joint shift co-invariant subspaces of reproducing
kernel Hilbert spaces over polydisc. In particular, we completely
analyze doubly commuting quotient modules of a large class of
reproducing kernel Hilbert modules, in the sense of Arazy and
Englis, over the unit polydisc ${\mathbb{D}}^n$.
\end{abstract}

\maketitle

\section*{Notation}

\begin{list}{\quad}{}
\item $\mathbb{N}$ \quad \quad \; Set of all natural numbers
including 0.

\item $n$ \quad \quad \; Natural number $n \geq 2$, unless
specifically stated otherwise.
\item $\mathbb{N}^n$ \quad \quad $\{\bm{k} = (k_1, \ldots, k_n) : k_i \in \mathbb{N}, i = 1,
\ldots, n\}$.
\item $\mathbb{C}^n$  \quad \quad Complex $n$-space.
\item $\bm{z}$ \quad \quad \; $(z_1, \ldots, z_n) \in \mathbb{C}^n$.
\item $\bm{z}^{\bm{k}}$ \quad \quad \,$z_1^{k_1}\cdots
z_n^{k_n}$.
\item $T$ \quad \quad \; $n$-tuple of commuting operators $(T_1, \ldots, T_n)$.
\item $T^{\bm{k}}$ \quad \quad $T_1^{k_1} \cdots
T_n^{k_n}$.
\item $\mathbb{C}[{\bm{z}}]$  \quad \, $\mathbb{C}[z_1, \ldots, z_n]$,
the polynomial ring over $\mathbb{C}$ in $n$-commuting variables.
\item $\mathbb{D}^n$  \quad \quad Open unit polydisc $\{{\bm{z}} : |z_i|
<1\}$.
\end{list}

Throughout this note all Hilbert spaces are over the complex field
and separable. Also for a closed subspace ${\mathcal{S}}$ of a Hilbert space
${\mathcal{H}}$, we denote by $P_{\mathcal{S}}$ the orthogonal projection of ${\mathcal{H}}$
onto ${\mathcal{S}}$.

{\setcounter{equation}{0} \section{{Introduction}}}

The Sz.-Nagy and Foias analytic model theory for contractions on
Hilbert spaces is a powerful tool for studying operators on Hilbert
spaces and holomorphic function spaces on the open unit disc ${\mathbb{D}}$ in
${\mathbb{C}}$. It says that if $T$ is a contraction (that is, $I - T T^* \geq
0$) on a Hilbert space and in $C_{\cdot 0}$ class (see the
definition below) then $T^*$ is unitarily equivalent to the
restriction of the backward shift $M_z^*$ on a vector-valued Hardy
space to a $M_z^*$-invariant subspace. More precisely, there exists
a coefficient Hilbert space ${\mathcal{E}}_*$ and a $M_z^*$-invariant closed
subspace ${\mathcal{Q}}$ of ${\mathcal{E}}_*$-valued Hardy space $H^2({\mathcal{E}}_*)$ such
that \[T \cong P_{\mathcal{Q}} M_z|_{\mathcal{Q}}.\] Moreover, there exists a
Hilbert space ${\mathcal{E}}$ and a ${\mathcal{B}}({\mathcal{E}}, {\mathcal{E}}_*)$-valued inner
multiplier $\theta \in H^\infty_{{\mathcal{B}}({\mathcal{E}}, {\mathcal{E}}_*)}({\mathbb{D}})$, also
known as the characteristic function of $T$ (see \cite{NF}), such
that
\[{\mathcal{Q}} = H^2({\mathcal{E}}_*)/\theta H^2({\mathcal{E}}).\] Also, it is important to
note that (i) $\theta$ is a complete unitary invariant, and (ii)
$\theta$ and the Hilbert spaces ${\mathcal{E}}$ and ${\mathcal{E}}_*$ are canonically
associated with $T$ \cite{NF}.

Recall that a bounded linear operator $T$ on a Hilbert space ${\mathcal{H}}$
is said to be in $C_{\cdot 0}$ class if $\|T^{*l}h\| {\rightarrow} 0$ as $l
{\rightarrow} \infty$ for all $h \in {\mathcal{H}}$, that is, if $T^{*l} {\rightarrow} 0$ as
$l {\rightarrow} \infty$ in the strong operator topology.

In \cite{Ag}, Agler introduced and studied the theory
hypercontraction operators from operator positivity point of view.
He showed that the vector-valued Hardy space in the dilation space
of a contraction can be replaced by a vector-valued weighted Bergman
space if the contractivity assumption on the operator is replaced by
a weighted Bergman type positivity. Later, Muller and Vasilescu
\cite{MV}, Curto and Vasilescu \cite{CV}, Ambrozie and Timotin
\cite{AT1, AT2}, Pott \cite{Pott}, Arazy, Englis and Muller
\cite{AEM} and Arazy and Englis \cite{AE} extended these ideas to a
more general class of operators. This viewpoint has proved to be
extremely fruitful in studying commuting tuples of operators.

The purpose of this paper is to explore how one might do dilation
theory and analytic model theory for a general class of operators
and doubly commuting tuples of operators. In particular, we
associate a partially isometric multiplier with every operator
satisfying weighted Bergman-type positivity condition. Another basic
result in this direction is the following: Let $T = (T_1, \ldots,
T_n)$ be a doubly commuting tuple of pure operators on a Hilbert
space ${\mathcal{H}}$ (that is, $T_i \in C_{\cdot 0}$, $T_i T_j = T_j T_i$
and $T_p T_q^* = T_q^* T_p$ for all $i, j = 1, \ldots, n$, and $1
\leq p < q \leq n$). Then $(T_1^*, \ldots, T_n^*)$ is joint
unitarily equivalent to the restriction of $(M_{z_1}^*, \ldots,
M_{z_n}^*)$ to a joint invariant subspace of a vector-valued
weighted Bergman space if and only if $T$ satisfies (joint) weighted
Bergman type positivity. Moreover, in this case, the orthocomplement
of the co-invariant subspace of the weighted Bergman space is of
``Beurling-lax-Halmos'' type (see Theorems \ref{MT} and
\ref{min-dil}).

The key idea of our approach is to construct a dilation map for
doubly commuting tuples of operators using the theory of single
operators and to develop a holomorphic structure in the dilation
space, in both one and several variables set up. Although our method
works for more general cases (see Section 7), for simplicity we will
restrict our discussion to hypercontractions (see Section 2).

Here is a brief description of the paper. In Section 2 we set up
notations, recall some basic notions from the theory of
hypercontractions and construct an analytic structure on the model
space. Our main tool here is the Agler dilation theorem for
hypercontractions \cite{Ag} combined with a Beurling-Lax-Halmos type
representations of shift invariant subspaces of analytic reproducing
kernel Hilbert spaces (\cite{BV1}, \cite{JS}). In Section 3 we
discuss a dilation theory for a class of doubly commuting operator
tuples satisfying weighted Bergman type positivity condition.  In
Section 4 we formulate a version of Sz.-Nagy and Foias analytic
model for doubly commuting tuple of hypercontractions. In Section 5,
we continue the discussion of Section 4 and relate the defect space
in the dilation space of doubly commuting tuples of operators. This
is relevant to the scalar valued weighted Bergman space case. In
particular, the defect space of a doubly commuting quotient module
of a weighted Bergman space is at most one dimensional. This will be
discussed in Section 6. Finally in section 7 we study
$K$-contractive tuples of operators in the spirit of Arazy and
Englis \cite{AE}.

\section{Functional Models for Hypercontractions}

The main purpose of this section is to study and to develop an
analytic functional model for the class of hypercontractions on
Hilbert spaces.

We first recall the definition of weighted Bergman spaces and review
the construction of the dilation maps for hypercontractions. We
refer the reader to Agler's work \cite{AIEOT} and \cite{Ag} for more
details.

The weighted Bergman kernel on the open unit disc ${\mathbb{D}}$ with weight
$\alpha >0$ is, by definition, the kernel function:
\[B_{\alpha}(z, w) = (1 - z \bar{w})^{-\alpha}. \quad \quad (z, w
\in {\mathbb{D}})\] For each $\alpha
>0$, we let $A^2_{\alpha}$, denote the weighted Bergman space corresponding to the kernel
$B_{\alpha}$. The shift operator or the multiplication operator by
the coordinate function $M_z$ on $A^2_{\alpha}$ is defined by
\[(M_z f)(w) = w f(w). \quad \quad \quad (f \in A^2_{\alpha}, w \in
{\mathbb{D}})\]It is easy to see that the shift operator $M_z$ on
$A^2_{\alpha}$, $\alpha \geq 1$, is a $C_{\cdot 0}$-contraction.

In the following discussion, we shall mostly use weighted Bergman
spaces with only integer weights. Let us point out an important
special case: $A^2_1 = H^2$, the Hardy space over ${\mathbb{D}}$.

For a multi-index ${\bm{m}} = (m_1, \ldots, m_n) \in {\mathbb{N}}^n$ we denote the
corresponding weighted Bergman space on ${\mathbb{D}}^n$ by $A^2_{\bm{m}}$. The
weighted Bergman kernel on ${\mathbb{D}}^n$ with weight ${\bm{m}}$ is, by
definition, the reproducing kernel function \[B_{\bm{m}}({\bm{z}}, {\bm{w}}) =
\prod_{i=1}^n B_{m_i}(z_i, w_i) = \prod_{i=1}^n (1 - z_i
\bar{w_i})^{-m_i}. \quad \quad ({\bm{z}}, {\bm{w}} \in {\mathbb{D}}^n)\] For each ${\bm{w}} \in
{\mathbb{D}}^n$, we denote by $B_{\bm{m}}(\cdot, {\bm{w}})$ the kernel function at ${\bm{w}}$,
where
\[(B_{\bm{m}}(\cdot, {\bm{w}}))({\bm{z}}) = B_{\bm{m}}({\bm{z}}, {\bm{w}}). \quad \quad \quad ({\bm{z}} \in
{\mathbb{D}}^n)\]

{\noindent}\textsf{Convention:} Let $p({\bm{z}}, {\bm{w}}) = \sum_{\bm{p}, \bm{q} \in
{\mathbb{N}}^n} a_{\bm{p} \bm{q}} {\bm{z}}^{\bm{p}} \bar{\bm{w}}^{\bm{q}}$ be a
polynomial in $\{z_1, \ldots, z_n\}$ and $\{\bar{w}_1, \ldots,
\bar{w}_n\}$. For a commuting tuple of bounded linear operators $T =
(T_1, \ldots, T_n)$ on a Hilbert space ${\mathcal{H}}$ (that is, $T_i T_j =
T_j T_i$ for all $i, j = 1, \ldots, n$) we denote by $p({\bm{z}}, {\bm{w}})(T,
T^*)$ the corresponding hereditary functional calculus in the sense
of Agler \cite{Ag}, that is,
\begin{equation}\label{hered}p(z, w)(T, T^*) = \sum_{\bm{p}, \bm{q} \in {\mathbb{N}}^n} a_{\bm{p}
\bm{q}} T^{\bm{p}} T^{* \bm{q}},
\end{equation}
where $T^{\bm{k}} = T^{k_1} \cdots T^{k_n}_n$ and $T^{*\bm{k}} =
T_1^{*k-1} \cdots T_n^{*k_n}$ for all $\bm{k} =(k_1, \ldots, k_n)
\in \mathbb{N}^n$.

\begin{Definition}
A bounded linear operator $T$ on ${\mathcal{H}}$ is said to be
$B_m$-contractive (or $T$ is a $B_m$-contraction) if $T$ is in
$C_{\cdot 0}$ class and \[\begin{split}B_m^{-1}(z, w)(T, T^*) & =
\big(\sum_{k=0}^m (-1)^k {m\choose k} {z}^k \bar{w}^{*k}\big)(T, T^*) \\
&= \sum_{k=0}^m (-1)^k {m\choose k} T^k T^{*k} \geq 0.\end{split}\]
\end{Definition}

We also recall that $T \in {\mathcal{B}}({\mathcal{H}})$ is a
\textit{hypercontraction} of order $m$ if
\[B_p^{-1}(z, w)(T, T^*) \geq 0,\]holds for all $1 \leq p \leq m$.

Now let $T$ be a $B_m$-contraction on ${\mathcal{H}}$. Since $T \in C_{\cdot
0}$, it follows from Lemma 2.11 in \cite{Ag} that
\[B_p^{-1}(z, w)(T, T^*) \geq 0, \quad \quad (1 \leq p \leq m)\]that
is, $T$ is a hypercontraction of order $m$. In other wards, these
two notions coincide for $C_{\cdot 0}$ class of operators and hence,
we will restrict our considerations for $B_m$-contractions.

Let $T$ be a $B_m$-contraction on ${\mathcal{H}}$. The \textit{defect
operator} of $T$, denoted by $D_{m,T}$, is the positive operator

\begin{equation}\label{defect}D_{m, T} = {\Big(B^{-1}_m(z, w)(T,
T^*)\Big)}^{\frac{1}{2}}.\end{equation} Moreover, since $T$ is a
contraction, we have \[\|z T^*\| = |z| \|T\| < 1. \quad \quad (z \in
{\mathbb{D}})\] Hence we see that $z \mapsto (I_{\mathcal{H}} - z T^*)^{-1}$ is a
holomorphic function from ${\mathbb{D}}$ to ${\mathcal{B}}({\mathcal{H}})$. Let us set
\begin{equation}\label{BzT}B_m(z, T) = (I_{\mathcal{H}} - z T^*)^{-m},
\quad \quad (z \in {\mathbb{D}})\end{equation} and
\begin{equation}\label{v}(\bm{v}_{m, T} f)(z) =
D_{m,T} B_m(z, T) f = D_{m,T} (I_{\mathcal{H}} - z T^*)^{-m} f. \quad \quad
(f \in {\mathcal{H}}, z \in {\mathbb{D}})\end{equation} Then $\bm{v}_{m, T} : {\mathcal{H}}
{\rightarrow} A^2_m ({\mathcal{H}})$ is a bounded linear operator. Moreover, for each
$w \in {\mathbb{D}}$ and $f, h \in {\mathcal{H}}$, we have
\[\begin{split} \langle {\bm{v}}_{m,T}^* B_m(\cdot, w) f, h\rangle &=
\langle B_m(\cdot, w) f, D_{m,T} B_m(\cdot, T) h\rangle = \langle f,
D_{m,T} B_m(w, T) h\rangle,
\end{split}\]where the last equality is obtained by the power series expansions of $z
\mapsto B_m(z, w)$ and $z \mapsto B_m(z, T)$, and by the fact that
$\{z^k\}_{k \in {\mathbb{N}}}$ is a orthogonal basis of $A^2_m$. This implies
\begin{equation}\label{v*} {\bm{v}}_{m, T}^*\Big(B_m(\cdot, w) \eta \Big) =
B_m(w, T)^* D_{m,T} \eta = (I_{\mathcal{H}} - \bar{w} T)^{-m} D_{m,T}
\eta,\end{equation}for all $w \in {\mathbb{D}}$ and $\eta \in {\mathcal{H}}$. This and
the definition of ${\bm{v}}_{m,T}$ together imply
\begin{equation}\label{vv*} \Big({\bm{v}}_{m, T} {\bm{v}}_{m, T}^* (B_m(\cdot,
w) \eta )\Big)(z) = D_{m, T} B_m(z, T) B_m(w, T)^* D_{m,T}\eta,
\quad \quad (z \in {\mathbb{D}})\end{equation}for all $w \in {\mathbb{D}}$ and $\eta \in
{\mathcal{H}}$.

We are now in a position to state the dilation theorem for
hypercontractions. The result is due to J. Agler (see \cite{Ag}).
The proof is an adaptation, as pointed out by Agler, of standard techniques
developed by de Branges and Rovnyak (cf. \cite{JS-S}), and so,
we will be rather sketchy.

\begin{Theorem}\label{Agler}\textsf{(Agler)}
Let $T \in {\mathcal{B}}({\mathcal{H}})$ be a $B_m$-contraction. Then $T \cong
P_{\mathcal{Q}} M_z|_{\mathcal{Q}}$, for some $M_z^*$-invariant closed subspace
${\mathcal{Q}}$ of $A^2_m({\mathcal{H}})$.
\end{Theorem}
{\noindent}\textsf{Proof.} It is easy to verify that the above dilation map
$\bm{v}_{m,T}$ is an isometry and \[\bm{v}_{m, T} T^* = M_z^*
\bm{v}_{m, T}.\]That is, ${\bm{v}}_{m, T}$ an isometric embedding of
${\mathcal{H}}$ into $A^2_m({\mathcal{H}})$ which intertwines $T^*$ and $M^*_z$. Then
$\bm{v}_{m, T} \bm{v}_{m, T}^*$ is an orthogonal projection and
hence \[P_{\mathcal{Q}} = \bm{v}_{m, T} \bm{v}_{m, T}^*,\]where \[{\mathcal{Q}} =
\mbox{ran} \bm{v}_{m, T},\]is an $M_z^*$-invariant subspace of
$A^2_m({\mathcal{H}})$. Then the unitary map $U := {\bm{v}}_{m,T} : {\mathcal{H}} {\rightarrow}
\mbox{ran} {\bm{v}}_{m,T}$ satisfies \[U T^* = M_z^*|_{\mathcal{Q}} U.\]
Therefore $T\cong M_z^*|_{\mathcal{Q}}$, or equivalently, $T \cong P_{\mathcal{Q}}
M_z|_{\mathcal{Q}}$. This completes the proof. {\hfill \vrule height6pt width 6pt depth 0pt}

We shall now introduce the notion of multipliers on reproducing
Kernel Hilbert spaces. Let $m_1, m_2$ be two natural numbers and
${\mathcal{E}}_1, {\mathcal{E}}_2$ be two Hilbert spaces. An operator valued
holomorphic map $\Theta : \mathbb{D} {\rightarrow} {\mathcal{B}}({\mathcal{E}}_1, {\mathcal{E}}_2)$ is
said to be a \textit{multiplier} from $A^2_{m_1}({\mathcal{E}}_1)$ to
$A^2_{m_2}({\mathcal{E}}_2)$ if
\[\Theta f \in A^2_{m_2}({\mathcal{E}}_2),\] for all $f \in
A^2_{m_1}({\mathcal{E}}_1)$. We denote the set of all multipliers from
$A^2_{m_1}({\mathcal{E}}_1)$ to $A^2_{m_2}({\mathcal{E}}_2)$ by
${\mathcal{M}}(A^2_{m_1}({\mathcal{E}}_1), A^2_{m_2}({\mathcal{E}}_2))$. We also use the
notation $M_{\Theta}$, for each $\Theta \in {\mathcal{M}}(A^2_{m_1}({\mathcal{E}}_1),
A^2_{m_2}({\mathcal{E}}_2))$, to denote the multiplication operator
\[M_{\Theta} f = \Theta f. \quad \quad \quad (f \in
A^2_{m_1}({\mathcal{E}}_1))\]

{\noindent} The multiplier space ${\mathcal{M}}(A^2_{m_1}({\mathcal{E}}_1),
A^2_{m_2}({\mathcal{E}}_2))$ admits the following useful characterization
(cf. Lemma 3.2 in \cite{S-JOT}): Let $X \in {\mathcal{B}}(A^2_{m_1}({\mathcal{E}}_1),
A^2_{m_2}({\mathcal{E}}_2))$. Then
\[X(M_z \otimes I_{{\mathcal{E}}_1}) = (M_z \otimes I_{{\mathcal{E}}_2}) X,\]if and
only if $X = M_{\Theta}$ for some $\Theta \in {\mathcal{M}}(A^2_{m_1}({\mathcal{E}}_1), A^2_{m_2}({\mathcal{E}}_2))$.

{\noindent} A multiplier $\Theta \in {\mathcal{M}}(A^2_{m_1}({\mathcal{E}}_1),
A^2_{m_2}({\mathcal{E}}_2))$ is said to be a partially isometric multiplier
if the bounded linear operator $M_{\Theta}$ is a partially isometric
operator from $A^2_{m_1}({\mathcal{E}}_1)$ to $A^2_{m_2}({\mathcal{E}}_2)$.

Before proceeding, let us for the sake of completeness recall a
Beurling-Lax-Halmos type theorem for weighted Bergman shifts (see
\cite{BV1} and Theorem 2.3 in \cite{JS}) upon which much of our
discussion in this paper will rest.

\begin{Theorem}\label{MT-blh}
Let ${\mathcal{S}}$ be a non-trivial closed subspace of $A^2_m({\mathcal{E}}_*)$. Then
${\mathcal{S}}$ is $M_z$-invariant subspace of $A^2_m({\mathcal{E}}_*)$ if and only if
there exists a Hilbert space ${\mathcal{E}}$ and a partially isometric
multiplier $\theta \in {\mathcal{M}}(A^2_1({\mathcal{E}}), A^2_m({\mathcal{E}}_*))$ such that
\[{\mathcal{S}} = \theta A^2_1({\mathcal{E}}).\]
\end{Theorem}

Once again, here $A^2_1({\mathcal{E}})$ denotes the ${\mathcal{E}}$-valued Hardy space
over ${\mathbb{D}}$, that is, \[A^2_1({\mathcal{E}}) = H^2({\mathcal{E}}).\]

We are now ready to present a functional model for the class of
$B_m$-contractions.

\begin{Theorem}\label{model-1}
Let $T$ be a $B_m$-contraction. Then there exists a Hilbert space
${\mathcal{E}}$ and a partially isometric multiplier $\theta \in
{\mathcal{M}}(A^2_1({\mathcal{E}}), A^2_m({\mathcal{H}}))$ such that \[T \cong
P_{{\mathcal{Q}}_{\theta}} M_z|_{{\mathcal{Q}}_{\theta}},\]where ${\mathcal{Q}}_{\theta} =
A^2_m({\mathcal{H}}) \ominus \theta A^2_1({\mathcal{E}})$.
\end{Theorem}
{\noindent}\textsf{Proof.} At first, by virtue of of Theorem \ref{Agler}, we
realize $T$ as $T \cong P_{\mathcal{Q}} M_z|_{\mathcal{Q}}$. Therefore, it only
remains to prove the existence of a partially isometric multiplier
$\theta$ such that ${\mathcal{Q}} = A^2_m({\mathcal{H}}) \ominus \theta A^2_1({\mathcal{E}})$.

{\noindent} Note that since ${\mathcal{Q}} = \mbox{ran~}{\bm{v}}_{m,T}$ is
$M_z^*$-invariant, $(\mbox{ran~}{\bm{v}}_{m,T})^{\perp}$ is a
$M_z$-invariant closed subspace of $A^2_m({\mathcal{H}})$. Then, applying
Theorem \ref{MT-blh} to $(\mbox{ran~}{\bm{v}}_{m,T})^{\perp}$, we obtain
a coefficient Hilbert space ${\mathcal{E}}$ and a partially isometric
multiplier $\theta \in {\mathcal{M}}(A^2_1({\mathcal{E}}), A^2_m({\mathcal{H}}))$ such that
\[(\mbox{ran~}{\bm{v}}_{m,T})^{\perp} = \theta A^2_1({\mathcal{E}}),\]that is,
\[{\mathcal{Q}} = A^2_m({\mathcal{H}}) \ominus \theta A^2_1({\mathcal{E}}).\] This completes
the proof. {\hfill \vrule height6pt width 6pt depth 0pt}

The next equality summarizes a few simple projection formulae which
will be useful later.

\begin{equation}\label{qtheta} P_{\mathcal{Q}} = P_{\mbox{ran} \bm{v}_{m, T}} = I_{A^2_m({\mathcal{H}})} -
P_{(\mbox{ran} \bm{v}_{m, T})^\perp} = I_{A^2_m({\mathcal{H}})} - M_{\theta}
M^*_{\theta}.\end{equation} We conclude this section with the
following observation: Let $T$ on ${\mathcal{H}}$ be a $B_m$-contraction.
Then we have a natural chain complex of Hilbert spaces:\[\cdots
\longrightarrow A^2_1({\mathcal{E}}) \stackrel{M_{\Theta}} \longrightarrow
A^2_m({\mathcal{H}}) \stackrel{\pi} \longrightarrow {\mathcal{H}} \longrightarrow
0,\]where $\pi = \bm{v}_{m,T}^*$.

{\setcounter{equation}{0} \section{{Dilations of commuting hypercontractions}}}

In this section, we give a proof of the fact that a doubly commuting
tuple of hypercontractions can be dilated to the tuple of
multiplication operators on a suitable weighted Bergman space over
${\mathbb{D}}^n$. We begin with a definition.

\begin{Definition}\label{Def1}
A commuting tuple of operators $T = (T_1, \ldots, T_n)$ on ${\mathcal{H}}$ is
said to be $B_{\bm{m}}$-contractive if $T_i$ is a $B_{m_i}$-contraction
for all $i = 1, \ldots, n$, and \[B_{\bm{m}}^{-1}({\bm{z}}, {\bm{w}}) (T, T^*) =
\Big( \prod_{i=1}^n B_{m_i}^{-1}(z_i, w_i) \Big)(T, T^*) \geq 0.\]
\end{Definition}

Let $T = (T_1, \ldots, T_n)$ be a doubly commuting
$B_{\bm{m}}$-contractive tuple on ${\mathcal{H}}$. In particular, since $T_i
T_j^* = T_j^* T_i$ for all $1 \leq i < j \leq n$, we have
\[T_i B_{m_j}(T_j, w) = T_i (I_{\mathcal{H}} - w T_j^*)^{-m_j} = (I_{\mathcal{H}} - w
T_j^*)^{-m_j} T_i = B_{m_j}(T_j, w) T_i, \] (see the equalities in
(\ref{hered}) and (\ref{BzT})) that is,
\begin{equation}\label{T1}T_i B_{m_j}(T_j, w) = B_{m_j}(T_j, w)
T_i,\end{equation} for all $w \in {\mathbb{D}}$. Moreover, since for each $i
\neq j$,
\[T_i (B^{-1}_{m_j}(z, w)(T_j, T_j^*)) = (B^{-1}_{m_j}(z, w)(T_j, T_j^*)) T_i,\]we
conclude that \begin{equation}\label{T2}T_i D_{m_j, T_j} = D_{m_j,
T_j} T_i,\end{equation}where $D_{m_j, T_j} = (B_{m_j}^{-1}(z,
w)(T_j, T_j^*))^{\frac{1}{2}}$ (see (\ref{defect})).

\textsf{For the rest of the paper we shall be dealing with a fixed
natural number $n \geq 2$, and a multi-index ${\bm{m}} = (m_1, \ldots,
m_n) \in \mathbb{N}^n$, $m_j \geq 1$ for all $j = 1, \ldots, n$.}

\begin{Lemma}\label{lemma1}
Let ${\bm{m}} = (m_1, \ldots, m_n) \in {\mathbb{N}}^n$. Let $T = (T_1, \ldots, T_n)$
be an $n$-tuple of doubly commuting operators on a Hilbert space
${\mathcal{H}}$ and each $T_i$ is a $B_{m_i}$-contraction, $i = 1 , \ldots,
n$. Then $T$ is a $B_{\bm{m}}$-contraction.
\end{Lemma}
{\noindent}\textsf{Proof.} By virtue of (\ref{T1}) and (\ref{T2}) we have
\[\Big(B_{m_i}^{-1}(z_i, w_i) (T_i,
T_i^*) \Big) \Big(B_{m_j}^{-1}(z_j, w_j) (T_j, T_j^*) \Big) =
\Big(B_{m_j}^{-1}(z_j, w_j) (T_j, T_j^*) \Big)
\Big(B_{m_i}^{-1}(z_i, w_i) (T_i, T_i^*) \Big),\]for all $i, j \in
\{1, \ldots, n\}$. Moreover, since \[B_{m_i}^{-1}(z_i, w_i) (T_i,
T_i^*) \geq 0, \quad \quad (i=1, \ldots, n)\]we obtain \[\Big(
\prod_{i=1}^n B_{m_i}^{-1}(z_i, w_i) \Big)(T, T^*) = \prod_{i=1}^n
\Big(B_{m_i}^{-1}(z_i, w_i) (T_i, T_i^*) \Big) \geq 0.\]This
completes the proof. {\hfill \vrule height6pt width 6pt depth 0pt}

In other wards, for a doubly commuting tuple $T$ on ${\mathcal{H}}$, $T$ is
$B_{\bm{m}}$-contractive if and only if $T_i$ is $B_{m_i}$-contractive
for all $i = 1, \ldots, n$.

Let $T = (T_1, \ldots, T_n)$ be a doubly commuting
$B_{\bm{m}}$-contractive tuple on ${\mathcal{H}}$, and let ${\bm{v}}_{m_j, T_j} : {\mathcal{H}}
\longrightarrow A^2_{m_j, T_j}({\mathcal{H}})$ be the isometric dilations of
$T_j$ (as defined in (\ref{v})), $j = 1, \ldots, n$. Then for each
$i \neq j$, and for all $w \in {\mathbb{D}}$ and $h \in {\mathcal{H}}$ we have
\[\begin{split}\Big({\bm{v}}_{m_j, T_j} T_i h\Big)(w) & = B_{m_j}(w, T_j)
D_{m_j,T_j} (T_i h)\\ & = T_i B_{m_j}(w, T_j) D_{m_j,T_j} h
\quad \quad \quad \quad (\mbox{by~} (\ref{T1}))\\
& = \Big((I_{A^2_{m_j}} \otimes T_i) {\bm{v}}_{m_j,
T_j}h\Big)(w).\end{split}\]Consequently
\begin{equation}\label{TiTj} {\bm{v}}_{m_j, T_j} T_i = T_i {\bm{v}}_{m_j,
T_j}. \quad \quad (i, j = 1, \ldots, n, i \neq j)\end{equation}

Now, we shall construct, using induction, a dilation map for a
doubly commuting $B_{\bm{m}}$-contractive tuple.

{\noindent} Let $T = (T_1, \ldots, T_n)$ be a $B_{\bm{m}}$-contractive $n$-tuple
on ${\mathcal{H}}$. We set $V_1 := {\bm{v}}_{m_1, T_1}: {\mathcal{H}} {\rightarrow}
A^2_{m_1}({\mathcal{H}})$ and define
\[V_2 := I_{A^2_{m_1}} \otimes {\bm{v}}_{m_2, T_2} : A^2_{m_1} \otimes
{\mathcal{H}} \cong A^2_{m_1}({\mathcal{H}}) \longrightarrow A^2_{m_1} \otimes
A^2_{m_2} \otimes {\mathcal{H}} \cong A^2_{(m_1, m_2)} ({\mathcal{H}}).\]Hence
\[\Big(V_2(z^l h)\Big)(z_1, z_2) = z_1^l ({\bm{v}}_{m_2, T_2} h)(z_2).
\quad \quad (l \in {\mathbb{N}}, h \in {\mathcal{H}})\] Then we define again $V_3$ as
above. Continuing this way, one can construct $n$ bounded linear
operators $V_1, \ldots, V_n$, defined by \[V_j := I_{A^2_{m_1}}
\otimes \cdots \otimes I_{A^2_{m_{j-1}}} \otimes {\bm{v}}_{m_j, T_j} :
A^2_{(m_1, \ldots, m_{j-1})} ({\mathcal{H}}) \longrightarrow A^2_{(m_1,
\ldots, m_{j})} ({\mathcal{H}}),\]where
\begin{equation}\label{V_j}\Big(V_j(z_1^{k_1} \cdots z_{j-1}^{k_{j-1}} h)\Big)(z_1, \ldots,
z_j) = z_1^{k_1} \cdots z_{j-1}^{k_{j-1}} ({\bm{v}}_{m_j, T_j}
h)(z_j),\end{equation}for all $h \in {\mathcal{H}}$ and $j = 1, \ldots, n$.
Consequently, we have the following sequence of maps:\[0
\longrightarrow {\mathcal{H}} \stackrel{V_1} \longrightarrow A^2_{m_1}({\mathcal{H}})
\stackrel{V_2} \longrightarrow A^2_{(m_1, m_2)}({\mathcal{H}})
\stackrel{V_3}\longrightarrow \cdots \stackrel{V_n}\longrightarrow
A^2_{\bm{m}}({\mathcal{H}}).\]Let us denote by $V_T$ the compositions of
$\{V_j\}_{j=1}^n$: \[V_T := V_n \circ \cdots \circ V_2 \circ V_1 :
{\mathcal{H}} {\rightarrow} A^2_{\bm{m}}({\mathcal{H}}).\]The bounded linear map $V_T \in
{\mathcal{B}}({\mathcal{H}}, A^2_{\bm{m}}({\mathcal{H}}))$ is called the \textit{dilation map} of
the tuple $T$.

To justify the term ``dilation'' let us now show that $V_T$ is
indeed an isometric dilation of the $B_{\bm{m}}$-contractive tuple $T$.

\begin{Theorem}\label{dil1}
Let $T = (T_1, \ldots, T_n)$ be a $B_{\bm{m}}$-contractive tuple on
${\mathcal{H}}$. Then $V_T$ is an isometry and \[V_T T_i^* = M_{z_i}^* V_T.
\quad \quad (i = 1, \ldots, n)\]
\end{Theorem}

{\noindent}\textsf{Proof.} For each $i = 1, \ldots, n$, we have ${\bm{v}}_{m_i,
T_i}^* {\bm{v}}_{m_i, T_i} = I_{\mathcal{H}}$, from which we immediately deduce
$V_i^* V_i = I_{\mathcal{H}}$, and finally
\[V_T^* V_T = I_{\mathcal{H}}.\] The second part of the
theorem now follows from the definition of $V_i$, the equality in
(\ref{TiTj}), and the fact
\[{\bm{v}}_{m_i, T_i} T_i^* = M_{z}^* {\bm{v}}_{m_i, T_i},\]for all $i = 1,
\ldots, n$. This completes the proof. {\hfill \vrule height6pt width 6pt depth 0pt}

As a consequence of Theorem \ref{dil1} we have the following
dilation theorem for the class of doubly commuting
$B_{\bm{m}}$-contractive tuples of operators. Recall that a pair of
commuting tuples $(T_1, \ldots, T_n)$ on ${\mathcal{H}}$ and $(S_1, \ldots,
S_n)$ on ${\mathcal{K}}$ is said to be jointly unitarily equivalent, also
denoted by $(T_1, \ldots, T_n) \cong (S_1, \ldots, S_n)$, if there
exists a unitary map $U : {\mathcal{H}} {\rightarrow} {\mathcal{K}}$ such that $U T_i = S_i U$
for all $i = 1, \ldots, n$.

\begin{Theorem}\label{dil-H}
Let $T = (T_1, \ldots, T_n)$ be a doubly commuting
$B_{\bm{m}}$-contractive tuple on a Hilbert space ${\mathcal{H}}$. Then there
exists a joint $(M_{z_1}^*, \ldots, M_{z_n}^*)$-invariant closed
subspace ${\mathcal{Q}} \subseteq A^2_{\bm{m}}({\mathcal{H}})$ such that \[(T_1, \ldots,
T_n) \cong (P_{\mathcal{Q}} M_{z_1}|_{\mathcal{Q}}, \ldots, P_{\mathcal{Q}}
M_{z_n}|_{\mathcal{Q}}).\]
\end{Theorem}
{\noindent}\textsf{Proof.} Let ${\mathcal{Q}} = \mbox{ran~} V_T$, where $V_T$ is the
dilation map of $T$ as in Theorem \ref{dil1}. Then ${\mathcal{Q}}$ is a joint
$(M_{z_1}^*, \ldots, M_{z_n}^*)$-invariant subspace of
$A^2_{\bm{m}}({\mathcal{H}})$ and $\bm{U} := V_T : {\mathcal{H}} {\rightarrow} {\mathcal{Q}}$ is a unitary
map. Moreover, \[\bm{U} T_j^* = V_T T_j^* = M_{z_j}^* V_T =
M_{z_i}^* V_T V_T^* V_T = (M_{z_j}^*|_{\mathcal{Q}}) V_T =
(M_{z_j}^*|_{\mathcal{Q}}) \bm{U},\]for all $j = 1, \ldots, n$. Hence
\[\bm{U} T_j = P_{\mathcal{Q}} M_{z_j}|_{\mathcal{Q}} \bm{U}. \quad \quad \quad (j = 1, \ldots,
n)\]This completes the proof of the theorem. {\hfill \vrule height6pt width 6pt depth 0pt}

We now proceed to a more concrete description of the dilation map
$V_T$.  Let ${\bm{w}} \in {\mathbb{D}}^n$ and $\eta \in {\mathcal{H}}$. Then \[V_T^*
(B_{\bm{m}}(\cdot, {\bm{w}}) \eta) = (V_1^* \cdots V_n^*)B_{\bm{m}}(\cdot, {\bm{w}})
\eta.\]On account of the representations of $\{V_i\}_{i=1}^n$, as in
(\ref{V_j}), this implies
\[\begin{split}V_T^* B_{\bm{m}}(\cdot, {\bm{w}}) \eta & = (V_1^* \cdots V_{n-1}^*)
(\prod_{i=1}^{n-1} B_{m_i}(\cdot, w_i) (
{\bm{v}}_{m_n, T_n}^* (B_{m_n}(\cdot, w_n) \eta)))\\
& = (V_1^* \cdots V_{n-1}^*)\prod_{i=1}^{n-1} B_{m_i}(\cdot, w_i)
B_{m_n}(w_n, T_n)^* D_{m_n, T_n} \eta
\\ & = (V_1^* \cdots V_{n-2}^*) \Big(\prod_{i=1}^{n-2} B_{m_i}(\cdot, w_i)
(\prod_{j=n-1}^n B_{m_j}(w_j, T_j)^* D_{m_j, T_j})
\eta\Big).\end{split}\]Continuing this way we have
\begin{equation}\label{V_T^*} V_T^* B_{\bm{m}}(\cdot, {\bm{w}}) \eta =
\prod_{i=1}^{n} B_{m_i}(w_i, T_i)^* D_{m_i, T_i}
\eta,\end{equation}for all ${\bm{w}} \in {\mathbb{D}}^n$ and $\eta \in {\mathcal{H}}$. Again,
by (\ref{V_j}), we have
\[\begin{split}V_T h & = V_n \cdots V_2 (V_1 h)= V_n \cdots
V_3(V_2 D_{m_1, T_1} B_{m_1}(z_1, T_1) h)\\ & =  V_n \cdots
V_3(D_{m_2, T_2} B_{m_2}(z_2, T_2) D_{m_1, T_1} B_{m_1}(z_1, T_1)
h)\\ & = V_n \cdots V_3(D_{m_1, T_1} D_{m_2, T_2} B_{m_1}(z_1, T_1)
B_{m_2}(z_2, T_2) h),
\end{split}\]for all $h \in {\mathcal{H}}$. Continuing this way we have \begin{equation}\label{VT} V_T
h = \prod_{i=1}^n D_{m_i, T_i} B_{m_i}(z_i, T_i)h.\quad \quad \quad
(h \in {\mathcal{H}}) \end{equation} Combining this with (\ref{V_T^*}) and
Theorem \ref{dil1} we have the following:

\begin{Theorem}\label{VTVT*}
Let $T = (T_1, \ldots, T_n)$ be a doubly commuting
$B_{\bm{m}}$-contractive tuple on ${\mathcal{H}}$. Then the map $V_T : {\mathcal{H}} {\rightarrow}
A^2_{\bm{m}}({\mathcal{H}})$ defined by
\[(V_Th)({\bm{z}}) = \Big(\prod_{i=1}^n D_{m_i, T_i} B_{m_i}(z_i, T_i)
\Big)h, \quad \quad (h \in {\mathcal{H}}, {\bm{z}} \in {\mathbb{D}}^n)\]is an isometry.
Moreover, \[V_T T_i^* = M_{z_i}^* V_T, \quad \quad (i = 1, \ldots,
n)\]and for each ${\bm{w}} \in {\mathbb{D}}^n$ and $\eta \in {\mathcal{H}}$, \[\Big((V_T
V_T^*) (B_{\bm{m}}(\cdot, {\bm{w}}) \eta)\Big)({\bm{z}}) = \prod_{i=1}^n D_{m_i,
T_i} B_{m_i}(z_i, T_i) B_{m_i}(w_i, T_i)^* D_{m_i, T_i} \eta. \quad
\quad ({\bm{z}} \in {\mathbb{D}}^n)\]
\end{Theorem}

The above theorem is a doubly commuting version and a particular
case of Theorem 3.16 in \cite{CV} by Curto and Vasilescu and
Corollary 16 in \cite{AEM} by Ambrozie, Englis and Muller (see also
\cite{BNS}). However, the present proof is simple, direct and brief.
The present approach is based on the idea of ``simple tensor
products of one variable dilation maps''.

The proof of the above theorem is also different from that given in
\cite{BNS} for commuting tuples of contractions.  Moreover, our
construction of explicit dilation map is especially useful in
analytic model theory.

{\setcounter{equation}{0} \section{{Analytic model}}}

We begin with the following definition, the relevance of which to
our purpose will become apparent in connection with the main results
of this paper.

\begin{Definition}\label{R1}
Let $T = (T_1, \ldots, T_n)$ be a doubly commuting
$B_{\bm{m}}$-contractive tuple on ${\mathcal{H}}$. For each $j = 1, \ldots, n$,
define $R_j : A^2_{\bm{m}}({\mathcal{H}}) {\rightarrow} A^2_{\bm{m}}({\mathcal{H}})$ by \[R_j
(B_{\bm{m}}(\cdot, {\bm{w}}) \eta) = R_j(\prod_{i=1}^n B_{m_i}(\cdot, w_i)
\eta) : = \Big(\prod_{\substack{{i=1}\\i \neq j}}^n B_{m_i}(\cdot,
w_i)\Big) \Big( {\bm{v}}_{m_j, T_j} {\bm{v}}_{m_j, T_j}^*(B_{m_j}(\cdot, w_j)
\eta)\Big),\]for all ${\bm{w}} \in {\mathbb{D}}^n$ and $\eta \in {\mathcal{H}}$. In other
wards, \[R_j = \Big(\bigotimes_{\substack{{i=1}\\i \neq j}}^n
I_{A^2_{m_i}}\Big) \otimes {\bm{v}}_{m_j, T_j} {\bm{v}}_{m_j, T_j}^*.\]
\end{Definition}

By virtue of (\ref{vv*}) we have in particular for each $j = 1,
\ldots, n$:
\begin{equation}\label{R-def}(R_j (B_{\bm{m}}(\cdot, {\bm{w}}) \eta))({\bm{z}}) =
\Big(\prod_{\substack{{i=1}\\i \neq j}}^n B_{m_i}(z_i, w_i)\Big)
\Big( D_{m_j, T_j} B_{m_j}(z_j, T_j) B_{m_j}(w_j, T_j)^* D_{m_j,
T_j} \eta \Big),\end{equation}for all ${\bm{z}}, {\bm{w}} \in {\mathbb{D}}^n$ and $\eta
\in {\mathcal{H}}$.

{\noindent}\textsf{Claim:} $\{R_1, \ldots, R_n\}$ is a family of commuting
orthogonal projections.

{\noindent}\textsf{Proof of the claim:} Since ${\bm{v}}_{m_j, T_j}$ is an
isometry, we deduce from the definition of $R_j$ that \[R_j = R_j^*
= R_j^2, \quad \quad (j = 1, \ldots, n)\]that is, $\{R_j\}_{j=1}^n$
is a family of orthogonal projections. Now let $p, q \in {\mathbb{N}}$ and $p
\neq q$. Using (\ref{R-def}), we obtain
\[\begin{split} R_p R_q
(B_{\bm{m}}(\cdot, {\bm{w}}) \eta) = & R_p \Big(\prod_{\substack{{i=1}\\i \neq
q}}^n B_{m_i}(z_i, w_i) ( D_{m_q, T_q} B_{m_q}(z_q, T_q)
B_{m_q}(w_q, T_q)^* D_{m_q, T_q} \eta) \Big) \\& =
\prod_{\substack{{i=1}\\i \neq p, q}}^n B_{m_i}(z_i, w_i) (D_{m_p,
T_p} B_{m_p}(z_p, T_p) B_{m_p}(w_p, T_p)^* D_{m_p, T_p}) \\ & \quad
\quad(D_{m_q, T_q} B_{m_q}(z_q, T_q) B_{m_q}(w_q, T_q)^* D_{m_q,
T_q} \eta) \\& = \prod_{\substack{{i=1}\\i \neq p, q}}^n
B_{m_i}(z_i, w_i) (D_{m_p, T_p} D_{m_q, T_q} B_{m_p}(z_p, T_p)
B_{m_q}(z_q, T_q)
\\ & \quad \quad  B_{m_p}(w_p, T_p)^* B_{m_q}(w_q, T_q)^* D_{m_p,
T_p}D_{m_q, T_q})\eta \\& = R_q R_p (B_{\bm{m}}(\cdot, {\bm{w}}) \eta),
\end{split}\]
for all ${\bm{w}} \in {\mathbb{D}}^n$ and $\eta \in {\mathcal{H}}$. From this we infer that
\[R_p R_q = R_q R_p. \quad \quad (p, q = 1, \ldots, n)\]The proof of
the claim is complete.

We turn now to investigate the product $\prod_{j=1}^n R_j$. For sake
of computational simplicity, let us assume, for each $i = 1, \ldots,
n$,
\[f_i(z, w) := B_{m_i}(z, T_i) B_{m_i}(w, T_i)^*. \quad \quad \quad  (z, w \in {\mathbb{D}})\]
We now compute \[\begin{split}(\prod_{j=1}^n R_j)(B_{\bm{m}}(\cdot, {\bm{w}})
\eta) & = \prod_{\substack{{i=1}\\i \neq 1}}^n R_i \Big(R_1
B_{\bm{m}}(\cdot, {\bm{w}}) \eta\Big) \\ & = \prod_{\substack{{i=1}\\i \neq
1}}^n R_i \Big( \prod_{\substack{{i=1}\\i \neq 1}}^n B_{m_i}(\cdot,
w_i) D_{m_1, T_1} f_1(\cdot, w_1) D_{m_1, T_1} \eta\Big) \\ & =
\prod_{\substack{{i=1}\\i \neq 1, 2}}^n R_i \Big(R_2
\prod_{\substack{{i=1}\\i \neq 1}}^n B_{m_i}(\cdot, w_i)  D_{m_1,
T_1} f_1(\cdot, w_1) D_{m_1, T_1} \eta\Big) \\ & =
\prod_{\substack{{i=1}\\i \neq 1, 2}}^n R_i
\Big(\prod_{\substack{{i=1}\\i \neq 1, 2}}^n B_{m_i}(\cdot, w_i)
D_{m_1, T_1} D_{m_2, T_2} f_1(\cdot, w_1) f_2(\cdot, w_2) D_{m_1, T_1}
D_{m_2, T_2}\eta\Big).\end{split}\]Continuing this way we have
\[\begin{split}(\prod_{j=1}^n R_j)(B_{\bm{m}}(\cdot, {\bm{w}}) \eta) & =
\prod_{i=1}^n D_{m_i, T_i} f(\cdot, w_i) D_{m_i, T_i}\\ & =
\prod_{i=1}^n D_{m_i, T_i} B_{m_1}(\cdot, T_i) B_{m_i}(w_i, T_i)^*
D_{m_i, T_i}.\end{split}\] Consequently, by Theorem \ref{VTVT*}, we
have \[V_T V_T^* = \prod_{i=1}^n R_i.\]Summing up, we have proved
the following theorem:

\begin{Theorem}\label{VR}
Let $T = (T_1, \ldots, T_n)$ be a doubly commuting
$B_{\bm{m}}$-contractive tuple of operators on ${\mathcal{H}}$. Then
$\{R_i\}_{i=1}^n$ is a family of commuting orthogonal projections.
Moreover, \[V_T V_T^* = \prod_{i=1}^n R_i.\]
\end{Theorem}

We need to introduce one more notion. For ${\bm{m}} \in \mathbb{N}^n$ and
for each $j = 1, \ldots, n$, set \[{\bm{m}}_j : = (m_1, \ldots, m_{j-1},
\underbrace{1}\limits_{j-\textup{th}}, m_{j+1}, \ldots, m_n).\]In particular, \[A^2_{{\bm{m}}_j} = 
A^2_{m_1} \otimes \cdots \otimes A^2_{m_{j-1}} \otimes H^2 \otimes A^2_{m_{j+1}} \otimes \cdots \otimes 
A^2_{m_n}.\]

Now let $T = (T_1, \ldots, T_n)$ be a doubly commuting
$B_{\bm{m}}$-contractive tuple on ${\mathcal{H}}$ and let ${\bm{v}}_{m_j, T_j} : {\mathcal{H}}
{\rightarrow} A^2_{m_j}({\mathcal{H}})$ be the dilation map of $T_j$, $j = 1, \ldots,
n$ (see (\ref{v})). By the same argument used in Theorem
\ref{model-1}, we have
\[{\bm{v}}_{m_j, T_j} {\bm{v}}_{m_j, T_j}^* = P_{\mbox{ran} {\bm{v}}_{m_j, T_j}} =
I_{A^2_{m_j}({\mathcal{H}})} - (I_{A^2_{m_j}({\mathcal{H}})} - P_{{(\mbox{ran}
{\bm{v}}_{m_j, T_j})}^{\perp}}) = I_{A^2_{m_j}({\mathcal{H}})} - M_{\theta_j}
M_{\theta_j}^*,\]for some partially isometric multiplier $\theta_j
\in {\mathcal{M}}(A^2_1({\mathcal{E}}_j), A^2_{m_j}({\mathcal{H}}))$ and coefficient Hilbert
space ${\mathcal{E}}_j$, $j = 1, \ldots, n$. Set
\begin{equation}\label{theta}\Theta_j({\bm{z}}) = \theta_j(z_j), \quad
\quad ({\bm{z}} \in {\mathbb{D}}^n)\end{equation}for all $j = 1, \ldots, n$. Then
$\Theta_j : {\mathbb{D}}^n {\rightarrow} {\mathcal{B}}({\mathcal{E}}_j, {\mathcal{H}})$ is an one-variable
operator-valued analytic function and the multiplication operator
$M_{\Theta_j} : A^2_{{\bm{m}}_j}({\mathcal{E}}_j) {\rightarrow} A^2_{\bm{m}}({\mathcal{H}})$ defined by
\[(M_{\Theta_j} f)({\bm{w}}) := \Theta_j({\bm{w}}) f({\bm{w}}) = \theta_j(w_j) f({\bm{w}}),
\quad \quad (f \in A^2_{{\bm{m}}_j}({\mathcal{E}}_j), {\bm{w}} \in {\mathbb{D}}^n)\]is partially
isometric. Moreover,
\[M_{\Theta_j} M_{z_i} = M_{z_i} M_{\Theta_j}. \quad \quad (i, j = 1,
\ldots, n)\]In other wards, $\Theta_j \in {\mathcal{M}}(A^2_{{\bm{m}}_j}({\mathcal{E}}_j),
A^2_{\bm{m}}({\mathcal{H}}))$ is a partially isometric multiplier for all $j = 1,
\ldots, n$. We can also realize the multiplier $M_{\Theta_j}$ as:
\[M_{\Theta_j} = \Big(\bigotimes_{\substack{{i=1}\\i \neq j}}^n
I_{A^2_{m_i}}\Big) \otimes M_{\theta_j}.\quad \quad \quad (j = 1,
\ldots, n)\]On the other hand, by definition of $R_j$ we have
\[\begin{split}R_j & = \Big(\bigotimes_{\substack{{i=1}\\i \neq j}}^n
I_{A^2_{m_i}}\Big) \otimes {\bm{v}}_{m_j, T_j} {\bm{v}}_{m_j, T_j}^*\\
& = \Big(\bigotimes_{\substack{{i=1}\\i \neq j}}^n
I_{A^2_{m_i}}\Big) \otimes (I_{A^2_{m_j}({\mathcal{H}})} - M_{\theta_j}
M_{\theta_j}^*)\\& = I_{A^2_{\bm{m}}({\mathcal{H}})} -
\Big(\bigotimes_{\substack{{i=1}\\i \neq j}}^n I_{A^2_{m_i}}\Big)
\otimes M_{\theta_j} M_{\theta_j}^* ,\end{split}\]that is, \[R_j =
I_{A^2_{\bm{m}}({\mathcal{H}})} - M_{\Theta_j} M_{\Theta_j}^*. \quad \quad (j =
1, \ldots, n)\] This along with Theorem \ref{VR}, we have the
following theorem:

\begin{Theorem}\label{VTheta}
Let $T = (T_1, \ldots, T_n)$ be a doubly commuting
$B_{\bm{m}}$-contractive tuple of operators on ${\mathcal{H}}$. Then there exists
partially isometric multipliers $\theta_k \in {\mathcal{M}}(A^2_k({\mathcal{E}}_k),
A^2_{m_k}({\mathcal{H}}))$, $k = 1, \ldots, n$, such that \[(M_{\Theta_i}
M_{\Theta_i}^*) (M_{\Theta_j} M_{\Theta_j}^*) = (M_{\Theta_j}
M_{\Theta_j}^*) (M_{\Theta_i} M_{\Theta_i}^*),\]for all $i, j = 1,
\ldots, n$, and\[V_T V_T^* = \prod_{i=1}^n (I_{A^2_{\bm{m}}({\mathcal{H}})} -
M_{\Theta_i} M_{\Theta_i}^*),\]where $\Theta_i$ is the one variable
multiplier corresponding to $\theta_i$,  $i = 1, \ldots, n$, as
defined in (\ref{theta}).
\end{Theorem}

In order to formulate our functional model for $B_{\bm{m}}$-contractive
tuples, we need to recall the following result concerning commuting
orthogonal projections (cf. Lemma 1.5 in \cite{S-JOT}):

\begin{Lemma}\label{P-F} Let $\{P_i\}_{i=1}^n$ be a collection of commuting orthogonal
projections on a Hilbert space ${\mathcal{H}}$. Then \[{\mathcal{L}} :=
\mathop{\sum}_{i=1}^n \mbox{ran} P_i,\] is closed and the orthogonal
projection of ${\mathcal{H}}$ onto ${\mathcal{L}}$ is given by
\[\begin{split}P_{\mathcal{L}} & = P_1 (I - P_2) \cdots (I - P_n) +
P_2 (I - P_3) \cdots (I - P_n) + \cdots + P_{n-1} (I - P_n) + P_n\\
& = P_n (I - P_{n-1}) \cdots (I - P_1) + P_{n-1} (I - P_{n-2})
\cdots (I - P_1) + \cdots + P_2 (I - P_1) +
P_1.\end{split}\]Moreover,
\[P_{\mathcal{L}} = I - \mathop{\prod}_{i=1}^n (I - P_i).\]
\end{Lemma}

We are now ready to present the main theorem of this section.

\begin{Theorem}\label{MT}
Let $T = (T_1, \ldots, T_n)$ be a doubly commuting
$B_{\bm{m}}$-contractive tuple of operators on ${\mathcal{H}}$. Then there exists
coefficient Hilbert spaces $\{{\mathcal{E}}_i\}_{i=1}^n$ and partial
isometric multipliers $\theta_j \in {\mathcal{M}}(A^2_j({\mathcal{E}}_j),
A^2_{m_j}({\mathcal{H}}))$ such that \[{\mathcal{H}} \cong {\mathcal{Q}}_{\Theta} :=
A^2_{\bm{m}}({\mathcal{H}})/ \sum_{i=1}^n \Theta_i A^2_{{\bm{m}}_i}({\mathcal{E}}_i),\] and
\[(T_1, \ldots, T_n) \cong (P_{{\mathcal{Q}}_{\Theta}}
M_{z_1}|_{{\mathcal{Q}}_{\Theta}}, \ldots, P_{{\mathcal{Q}}_{\Theta}}
M_{z_n}|_{{\mathcal{Q}}_{\Theta}}),\]where $\Theta_i$ is the one variable
multiplier corresponding to $\theta_i$,  $i = 1, \ldots, n$, as
defined in (\ref{theta}).
\end{Theorem}

{\noindent}\textsf{Proof.} We continue with the notation of Theorem
\ref{VTheta}. Set $P_i := M_{\Theta_i} M_{\Theta_i}^*$, $i = 1,
\ldots, n$. By virtue of Theorem \ref{VTheta} we have \[V_T V_T^* =
\prod_{i=1}^n (I_{A^2_{\bm{m}}({\mathcal{H}})} - P_i),\]and so
\[I_{A^2_{\bm{m}}({\mathcal{H}})} - V_T V_T^* = I_{A^2_{\bm{m}}({\mathcal{H}})} -
\prod_{i=1}^n (I_{A^2_{\bm{m}}({\mathcal{H}})} - P_i).\]Now by Lemma \ref{P-F},
it follows that
\[(\mbox{ran} V_T)^\perp = \sum_{i=1}^n \mbox{ran}
M_{\Theta_i} = \sum_{i=1}^n \Theta_i A^2_{{\bm{m}}_i}({\mathcal{E}}_i).\]Therefore,
\[{\mathcal{Q}}_{\Theta} := \mbox{ran} V_T = \Big(\sum_{i=1}^n \Theta_i
A^2_{{\bm{m}}_i}({\mathcal{E}}_i)\Big)^{\perp} \cong A^2_{\bm{m}}({\mathcal{H}})/ \sum_{i=1}^n
\Theta_i A^2_{{\bm{m}}_i}({\mathcal{E}}_i).\]Now using the line of argument from
the proof of Theorem \ref{dil-H} one can prove that $(T_1, \ldots,
T_n) \cong (P_{{\mathcal{Q}}_{\Theta}} M_{z_1}|_{{\mathcal{Q}}_{\Theta}}, \ldots,
P_{{\mathcal{Q}}_{\Theta}} M_{z_n}|_{{\mathcal{Q}}_{\Theta}})$. This concludes the
proof. {\hfill \vrule height6pt width 6pt depth 0pt}

{\setcounter{equation}{0} \section{{Minimal representations}}}

The purpose of this section is to stress the role of the (joint-)defect 
space in the dilation space and the functional model of
doubly commuting $B_{\bm{m}}$-contractive tuples of operators. This is
particularly more useful for the study of joint co-invariant
subspaces of tuples of shift operators on holomorphic function
spaces (see Section \ref{qm}).

To set up the stage, first we introduce a pair of new notations. Let
$T$ be a doubly commuting $B_{\bm{m}}$-contractive tuple on ${\mathcal{H}}$. We
denote the (joint-)defect operator and defect space of $T$ by (see
Lemma \ref{lemma1}):
\[D_{{\bm{m}},T} := (\prod_{i=1}^n B_{m_i}^{-1}(z_i, w_i)(T,
T^*))^\frac{1}{2}, \quad \quad  \mbox{and} \quad\quad {\mathcal{D}}_{{\bm{m}},T} =
\overline{\mbox{ran}} D_{{\bm{m}},T},\]respectively.

We begin with the following proposition.

\begin{Proposition}\label{RD} Under the hypothesis of Theorem
\ref{VR}, $A^2_{\bm{m}}({\mathcal{D}_{\bm{m},T}})$ is a $R_j$-reducing subspace of
$A^2_{\bm{m}}({\mathcal{H}})$, $j = 1, \ldots, n$.
\end{Proposition}

{\noindent}\textsf{Proof.} It is enough to prove that
\[R_j (B_{\bm{m}}(\cdot, {\bm{w}}) \eta ) \in
A^2_{\bm{m}}({\mathcal{D}_{\bm{m},T}}),\]for all ${\bm{w}} \in {\mathbb{D}}^n$ and $\eta \in
{\mathcal{D}_{\bm{m},T}}$. To this end, for each $z_j, w_j \in {\mathbb{D}}$, we compute \[\begin{split} D_{m_j, T_j} B_{m_j} & (z_j,
T_j) B_{m_j}(w_j, T_j)^* D_{m_j, T_j} \Big(\prod_{i=1}^n D_{m_i,
T_i}\Big) \\ & =  D_{m_j, T_j} B_{m_j}(z_j, T_j) B_{m_j}(w_j, T_j)^*
\Big(\prod_{\substack{{i=1}\\i \neq j}}^n D_{m_i, T_i}\Big)
D^2_{m_j, T_j} \\ & =  D_{m_j, T_j} \Big(\prod_{\substack{{i=1}\\i
\neq j}}^n D_{m_i, T_i}\Big) B_{m_j}(z_j, T_j) B_{m_j}(w_j, T_j)^*
D^2_{m_j, T_j} \\ & = \Big(\prod_{\substack{{i=1}}}^n D_{m_i,
T_i}\Big) B_{m_j}(z_j, T_j) B_{m_j}(w_j, T_j)^* D^2_{m_j, T_j}.
\end{split}\]In particular, we have \[\Big(D_{m_j, T_j} B_{m_j} (z_j, T_j)
B_{m_j}(w_j, T_j)^* D_{m_j, T_j}\Big) {\mathcal{D}_{\bm{m},T}} \subseteq {\mathcal{D}_{\bm{m},T}}.\quad \quad
(z_j, w_j \in {\mathbb{D}})\]Hence by (\ref{R-def}) and (\ref{T1}) we
have
\[\begin{split}R_j \Big( B_{\bm{m}}(\cdot, {\bm{w}}) \eta \Big) & =
\Big(\prod_{\substack{{k=1}\\k \neq j}}^n B_{m_k}(z_k, w_k)\Big)
\Big(D_{m_j, T_j} B_{m_j}(z_j, T_j) B_{m_j}(w_j, T_j)^* D_{m_j, T_j}
\eta\Big) \in A^2_{\bm{m}}({\mathcal{D}_{\bm{m},T}}), \end{split}\]where ${\bm{w}} \in {\mathbb{D}}^n$ and
$\eta = \prod_{i=1}^n D_{m_i, T_i} h \in {\mathcal{D}_{\bm{m},T}}$ for some $h \in {\mathcal{H}}$.
This proves the desired claim, and the result follows. {\hfill \vrule height6pt width 6pt depth 0pt}

In view of Proposition \ref{RD},
\begin{equation}\label{Rtilde}\tilde{R}_i := R_i|_{A^2_{\bm{m}}({\mathcal{D}_{\bm{m},T}})},
\quad \quad (i = 1, \ldots, n)\end{equation}is an orthogonal
projection on $A^2_{\bm{m}}({\mathcal{D}_{\bm{m},T}})$.

\begin{Corollary}
$(\tilde{R}_1, \ldots, \tilde{R}_n)$ is an $n$-tuple of commuting orthogonal
projections on $A^2_{\bm{m}}({\mathcal{D}_{\bm{m},T}})$. Moreover, \[\mbox{ran} \tilde{R}_i =
\mbox{ran} R_i \bigcap A^2_{\bm{m}}({\mathcal{D}_{\bm{m},T}}). \quad \quad (i = 1, \ldots,
n)\]
\end{Corollary}

{\noindent}\textsf{Proof.} The result follows immediately from the fact
$(R_1, \ldots, R_n)$ is an $n$-tuple of commuting orthogonal
projections. {\hfill \vrule height6pt width 6pt depth 0pt}

Now we consider the role of defect operator and defect space in the
dilation space of a $B_{\bm{m}}$-contractive tuple. To this end let us
first observe that, by virtue of (\ref{VT}) (or Theorem
\ref{VTVT*}), for a doubly commuting $B_{\bm{m}}$-contractive tuple $T=
(T_1, \ldots, T_n)$ on ${\mathcal{H}}$, \[\mbox{ran} V_T \subseteq
A^2_{\bm{m}}({\mathcal{D}_{\bm{m},T}}).\]In other wards, $V_T : {\mathcal{H}} {\rightarrow} A^2_{\bm{m}}({\mathcal{D}_{\bm{m},T}})$ is
an isometry. Moreover,  the dilation map $V_T$ intertwines $T_i$ on
${\mathcal{H}}$ and $M_{z_i}$ on $A^2_{\bm{m}}({\mathcal{D}_{\bm{m},T}})$ for all $i = 1, \ldots, n$.
This also follows directly from the fact that $A^2_{\bm{m}}({\mathcal{D}_{\bm{m},T}})$ is a
joint $(M_{z_1}, \ldots, M_{z_n})$-reducing subspace of
$A^2_{\bm{m}}({\mathcal{H}})$.

{\noindent} Set $\tilde{V}_T : {\mathcal{H}} {\rightarrow} A^2_{\bm{m}}({\mathcal{D}_{\bm{m},T}})$ by \[\tilde{V}_T h
= V_T h. \quad \quad (h \in {\mathcal{H}})\]By Theorems \ref{VR} and
\ref{VTheta}, we have \[\begin{split} \tilde{V}_T \tilde{V}_T^* & =
(V_T V_T^*)|_{A^2_{\bm{m}}({\mathcal{D}_{\bm{m},T}})} \\ & = \prod_{i=1}^n
(I_{A^2_{\bm{m}}({\mathcal{H}})} - M_{\Theta_i} M_{\Theta_i}^*)|_{A^2_{\bm{m}}({\mathcal{D}_{\bm{m},T}})}
\\& = \prod_{i=1}^n (I_{A^2_{\bm{m}}({\mathcal{D}}_{{\bm{m}}, T})} - \tilde{R}_i). \quad \quad
(by (\ref{Rtilde}))\end{split}\]Let us observe, moreover, that
\[\begin{split}\mbox{ran} \tilde{R}_j  & = \Big(\Big( \bigotimes_{\substack{{i=1}\\i \neq
j}}^n A^2_{m_i} \Big) \otimes \mbox{ran} ({\bm{v}}_{m_j, T_j} {\bm{v}}_{m_j,
T_j})\Big) \bigcap \Big(\Big( \bigotimes_{\substack{{i=1}\\i \neq
j}}^n A^2_{m_i} \Big) \otimes A^2_{m_j}({\mathcal{D}}_{{\bm{m}}, T})\Big)\\ & =
\Big( \bigotimes_{\substack{{i=1}\\i \neq j}}^n A^2_{m_i} \Big)
\otimes \tilde{\mathcal{S}}_j,
\end{split}\]for some $M_z^*$-invariant closed subspace $\tilde{\mathcal{S}}_j$ of
$A^2_{m_j}({\mathcal{D}}_{{\bm{m}}, T})$, $j = 1, \ldots, n$. Invoking Theorem
\ref{MT-blh} once more, we have \[\tilde{R}_j = I_{A^2_{\bm{m}}({\mathcal{D}}_{{\bm{m}},
T})} - M_{\tilde{\Theta}_j} M_{\tilde{\Theta}_j}^*,\]for some
partially isometric multiplier $\tilde{\theta}_j \in
{\mathcal{M}}(A^2_1(\tilde{\mathcal{E}}_j), A^2_{m_j}({\mathcal{D}}_{{\bm{m}}, T}))$ and auxiliary
Hilbert space $\tilde{\mathcal{E}}_j$, $j = 1, \ldots, n$, and
\[\tilde{V}_T \tilde{V}_T^* = \prod_{i=1}^n \tilde{R}_i =
\prod_{i=1}^n (I_{A^2_{\bm{m}}({\mathcal{D}}_{{\bm{m}}, T})} - M_{\tilde{\Theta}_i}
M_{\tilde{\Theta}_i}^*).\] Here $\tilde{\Theta}_j \in
{\mathcal{M}}(A^2_{{\bm{m}}_j}(\tilde{\mathcal{E}}_j), A^2_{\bm{m}}({\mathcal{D}}_{{\bm{m}}, T}))$, $j = 1,
\ldots, n$, is the one variable partially isometric multiplier
corresponding to the partially isometric multiplier
$\tilde{\theta}_j \in {\mathcal{M}}(A^2_1(\tilde{\mathcal{E}}_j),
A^2_{m_j}({\mathcal{D}}_{{\bm{m}}, T}))$ as defined in (\ref{theta}).

We can now reach our goal. Using the line of argument from the
proofs of Theorems \ref{VTheta} and \ref{MT} we can state the
following theorem.

\begin{Theorem}\label{min-dil}
Let $T = (T_1, \ldots, T_n)$ be a doubly commuting
$B_{\bm{m}}$-contractive tuple on a Hilbert space ${\mathcal{H}}$. Then there
exists coefficient Hilbert spaces $\{{\mathcal{E}}_i\}_{i=1}^n$ and one
variable partially isometric multipliers $\Theta_i \in
{\mathcal{M}}(A^2_{{\bm{m}}_j}(\tilde{\mathcal{E}}_j), A^2_{\bm{m}}({\mathcal{D}}_{{\bm{m}}, T}))$, as
defined in (\ref{theta}), such that \[{\mathcal{H}} \cong {\mathcal{Q}}_{\Theta} :=
A^2_{\bm{m}}({\mathcal{D}}_{{\bm{m}},T})/ \sum_{i=1}^n \Theta_i
A^2_{{\bm{m}}_i}({\mathcal{E}}_i),\]and
\[(T_1, \ldots, T_n) \cong (P_{{\mathcal{Q}}_{\Theta}}
M_{z_1}|_{{\mathcal{Q}}_{\Theta}}, \ldots, P_{{\mathcal{Q}}_{\Theta}}
M_{z_n}|_{{\mathcal{Q}}_{\Theta}}).\]
\end{Theorem}

In the special case that ${\bm{m}} = (1, \ldots, 1)$ we recover the functional model for doubly commuting tuples of pure contractions \cite{BNS}. Moreover, the methods
used here are different from those used in \cite{BNS}.

{\setcounter{equation}{0} \section{{Quotient modules of $A^2_{\bm{m}}$}}}\label{qm}

We have a particular interest in tuples of operators $(M_{z_1}, \ldots,
M_{z_n})$ restricted to joint $(M_{z_1}^*, \ldots,
M_{z_n}^*)$-invariant subspaces of reproducing kernel Hilbert spaces over ${\mathbb{D}}^n$. Let ${\mathcal{Q}}$ be a joint
$(M_{z_1}^*, \ldots, M_{z_n}^*)$-invariant closed subspace of
$A^2_{\bm{m}}$. Then ${\mathcal{Q}}$ is called a doubly commuting quotient module
of $A^2_{\bm{m}}$ if
\[[C_{z_i}^*, C_{z_j}] = C_{z_i}^* C_{z_j} - C_{z_j} C_{z_i}^* = 0,
\quad \quad (1 \leq i < j \leq n)\]where $C_{z_i}$ is the
compression of $M_{z_i}$ on ${\mathcal{Q}}$, that is,
\[C_{z_i} = P_{\mathcal{Q}} M_{z_i}|_{\mathcal{Q}}. \quad \quad \quad (i= 1, \ldots,
n)\]First, we compute the defect operator of a given doubly
commuting quotient module ${\mathcal{Q}}$ of $A^2_{\bm{m}}$:
\[D^2_{{\bm{m}}, C_z} = \prod_{i=1}^n B_{m_i}^{-1}(z_i, w_i) (C_z, C_z^*)
= P_{\mathcal{Q}} \Big(\prod_{i=1}^n B_{m_i}^{-1}(M_{z_i},
M_{z_i}^*)\Big)|_{\mathcal{Q}}.\] In the above we used the fact that
\[C_{z_1}^{k_1} \cdots C_{z_n}^{k_n} = P_{\mathcal{Q}} M_{z_1}^{k_1} \cdots
M_{z_n}^{k_n}|_{\mathcal{Q}}, \quad \quad\quad ((k_1, \ldots, k_n) \in
\mathbb{N}^n)\]and\[C_{z_1}^{* k_1} \cdots C_{z_n}^{* k_n} =
M_{z_1}^{*k_1} \cdots M_{z_n}^{*k_n}|_{\mathcal{Q}}, \quad \quad\quad
((k_1, \ldots, k_n) \in \mathbb{N}^n)\] and \[C_{z_i} C_{z_j}^* =
C_{z_j}^* C_{z_i} = P_{\mathcal{Q}} M_{z_i} M_{z_j}^*|_{\mathcal{Q}}. \quad \quad
\quad (i, j = 1, \ldots, n, i \neq j)\]On the other hand, it is easy to see that (cf. Theorem 3.3 in \cite{CDS})
\[D^2_{{\bm{m}}, M_z} = \prod_{i=1}^n B_{m_i}^{-1}(z_i, w_i) (M_z,
M_z^*) = \prod_{i=1}^n B_{m_i}^{-1}(M_{z_i}, M_{z_i}^*) =
P_{\mathbb{C}},\]where $P_{\mathbb{C}}$ is the orthogonal projection
of $A^2_{\bm{m}}$ onto the one dimensional subspace of all constant
functions. Consequently, \[ D^2_{{\bm{m}}, C_z} = P_{\mathcal{Q}}
P_{\mathbb{C}}|_{\mathcal{Q}},\]and hence
\begin{equation}\label{C}\mbox{rank} D_{{\bm{m}}, C_z} \leq
1.\end{equation}

\begin{Theorem}
Let ${\mathcal{Q}}$ be a quotient module of $A^2_{\bm{m}}$. Then the following
conditions are equivalent:

(i) ${\mathcal{Q}}$ is doubly commuting.

(ii) There exists $M_z^*$-invariant closed subspace ${\mathcal{Q}}_i$ of
$A^2_{m_i}$, $i = 1, \ldots, n$, such that \[{\mathcal{Q}} = {\mathcal{Q}}_1 \otimes
\cdots \otimes {\mathcal{Q}}_n.\]

(iii) There exists coefficient Hilbert spaces $\{{\mathcal{E}}_i\}_{i=1}^n$
and partially isometric multipliers $\theta_i \in {\mathcal{M}}(H^2({\mathcal{E}}_i),
A^2_{m_i})$, $i = 1, \ldots, n$, such that \[{\mathcal{Q}} = {\mathcal{Q}}_{\theta_1}
\otimes \cdots \otimes {\mathcal{Q}}_{\theta_n},\]where ${\mathcal{Q}}_{\theta_i} =
A^2_{m_i}/ \theta_j H^2({\mathcal{E}}_j)$, $j = 1, \ldots, n$.
\end{Theorem}
{\noindent}\textsf{Proof.} Let us begin by observing that the representation
of $C_{z_i}$, $i = 1, \ldots, n$, on ${\mathcal{Q}} = {\mathcal{Q}}_1 \otimes \cdots
\otimes {\mathcal{Q}}_n$ is given by
\[\begin{split} C_{z_i} & = P_{\mathcal{Q}} (I_{A^2_{m_1}} \otimes \cdots \otimes I_{A^2_{m_{i-1}}}
\otimes M_z \otimes I_{A^2_{m_{i+1}}} \otimes \cdots \otimes
I_{{\mathcal{Q}}_n})|_{\mathcal{Q}} \\ & = I_{{\mathcal{Q}}_1} \otimes \cdots \otimes
I_{{\mathcal{Q}}_{i-1}} \otimes P_{{\mathcal{Q}}_i} M_z|_{{\mathcal{Q}}_i} \otimes
I_{{\mathcal{Q}}_{i+1}} \otimes \cdots \otimes I_{{\mathcal{Q}}_n}.\end{split}\]This
yields $(ii) {\Rightarrow} (i)$ and $(iii) {\Rightarrow} (i)$. The implication $(ii)
{\Rightarrow} (iii)$ follows from Theorem \ref{MT-blh} and $(iii) {\Rightarrow} (ii)$ is
trivial. Hence it suffices to show $(i) \Rightarrow (iii)$. Assume
$(i)$. Then by Theorem \ref{min-dil} \[(C_{z_1}, \ldots, C_{z_n})
\mbox{~on~} {\mathcal{Q}} \cong (P_{{\mathcal{Q}}_{\Theta}} M_{z_1}|_{{\mathcal{Q}}_{\Theta}},
\ldots, P_{{\mathcal{Q}}_{\Theta}} M_{z_n}|_{{\mathcal{Q}}_{\Theta}}) \mbox{~on~}
{\mathcal{Q}}_{\Theta},\]where \[{\mathcal{Q}}_{\Theta} =
A^2_{\bm{m}}({\mathcal{D}}_{{\bm{m}},C_z})/\sum_{i=1}^n \Theta_i
A^2_{{\bm{m}}_i}({\mathcal{E}}_i),\]for some coefficient Hilbert spaces
$\{{\mathcal{E}}_i\}_{i=1}^n$ and one variable partially isometric
multipliers \[\Theta_i \in {\mathcal{M}}(A^2_{{\bm{m}}_i}({\mathcal{E}}_i),
A^2_{\bm{m}}({\mathcal{D}}_{{\bm{m}},T})). \quad \quad \quad(i = 1,\ldots, n)\]Now by
virtue of (\ref{C}) we have \[{\mathcal{D}}_{{\bm{m}}, C_z} \cong \{0\},
\mbox{~or~} \mathbb{C}.\] In order to avoid trivial considerations
we assume that ${\mathcal{D}}_{{\bm{m}},C_z} \cong \mathbb{C}$. Then
\[{\mathcal{Q}} \cong {\mathcal{Q}}_{\Theta} = A^2_{\bm{m}}/ \sum_{i=1}^n \Theta_i
A^2_{{\bm{m}}_i}({\mathcal{E}}_i).\]In particular, \[P_{{\mathcal{Q}}_{\Theta}} =
\prod_{i=1}^n (I_{A^2_{\bm{m}}} - M_{\Theta_i} M_{\Theta_i}^*) =
\bigotimes_{i=1}^n (I_{A^2_{m_i}} - M_{\theta_i}
M_{\theta_i}^*),\]which implies
\[{\mathcal{Q}}_{\Theta} = {\mathcal{Q}}_{\theta_1} \otimes \cdots \otimes
{\mathcal{Q}}_{\theta_n},\]and concludes the proof. {\hfill \vrule height6pt width 6pt depth 0pt}

The implication $(i) {\Rightarrow} (ii)$ in previous theorem was recently
obtained by Chattopadhyay, Das and the third author in \cite{CDS}.
For the Hardy space case $H^2(\mathbb{D}^n)$, that is, for the case
${\bm{m}} = (1, \ldots, 1)$, this was observed in \cite{BNS}, \cite{INS} and 
\cite{S-JOT}. Moreover, as we shall see in the next section, the
same result holds for more general reproducing kernel Hilbert spaces
over ${\mathbb{D}}^n$.

{\setcounter{equation}{0} \section{{$\frac{1}{K}$-calculus and $K$-contractivity}}}\label{K}

The key concept in our approach is the natural connections between
(i) operator positivity, implemented by the inverse of a positive
definite kernel function on ${\mathbb{D}}$, and a dilation map, again in terms
of the kernel function, (ii) tensor product structure of reproducing
kernel Hilbert spaces on ${\mathbb{D}}^n$, and (iii) operator positivity,
implemented by the product of $n$ positive definite kernel functions
on ${\mathbb{D}}$, of doubly commuting $n$ tuple of operators. Consequently,
our considerations can be applied even for a more general framework (in the sense of Arazy and Englis \cite{AE}).

Let $k$ be a positive definite kernel function on $\mathbb{D}$ and
that $k(z, w)$ is holomorphic in $z$ and anti-holomorphic in $w$,
and $k(z, w) \neq 0$ for all $z, w \in {\mathbb{D}}$. Let ${\mathcal{R}}_k \subseteq
{\mathcal{O}}({\mathbb{D}}, {\mathbb{C}})$ be the corresponding reproducing kernel Hilbert space.
Moreover, let

(i) ${\mathbb{C}}[z]$ is dense in ${\mathcal{R}}_k$,

(ii) the multiplication operator $M_z$ on ${\mathcal{R}}_k$ is a contraction,

(iii) there exists a sequence of polynomials $\{p_k\}_{k=0}^\infty
\in {\mathbb{C}}[z, \bar{w}]$ such that
\[p_k(z, \bar{w}) \longrightarrow \frac{1}{k(z, {w})}, \quad \quad
\quad (z, w \in {\mathbb{D}})\]and\[\sup_{k} \|p_k(M_z, M_z^*)\| < \infty.\]
We will call such a reproducing kernel Hilbert space a
\textit{standard reproducing kernel Hilbert space}, or, just SRKH
for short.

Let ${\mathcal{R}}_k$ be a SRKH and, by virtue of condition (i) in the above
definition, let $\{\psi_k\}_{k=0}^\infty \subseteq {\mathbb{C}}[z]$ be an
orthonormal basis of ${\mathcal{R}}_k$. For any nonnegative operator $C$ and
a bounded linear operator $T$ on a Hilbert space ${\mathcal{H}}$, set
\[f_{k,C}(T) = I_{\mathcal{H}} - \sum_{0 \leq m < k} \psi_m(T) C
\psi_m(T)^*.\]

\begin{Definition} Let ${\mathcal{R}}_k$ be a SRKH and $T$ be a bounded linear operator on
a Hilbert space ${\mathcal{H}}$. Then $T$ is said to be $k$-contractive if
\[\sup_{k} \|p_k(T, T^*)\| < \infty,\]and
\[C: = WOT-\lim_{k {\rightarrow} \infty} p_k(T, T^*),\]exists and
nonnegative, and
\[SOT-\lim_{k {\rightarrow} \infty} f_{k, C}(T) = 0.\]
\end{Definition}

We are now ready to state the Arazy-Englis dilation result (see
Corollary 3.2 in \cite{AE}).

\begin{Theorem}
Let ${\mathcal{R}}_k$ be a SRKH and $T \in {\mathcal{B}}({\mathcal{H}})$ be a $k$-contraction.
Then \[T \cong P_{\mathcal{Q}} M_z|_{\mathcal{Q}},\]for some $M_z^*$-invariant
closed subspace ${\mathcal{Q}}$ of ${\mathcal{R}}_k \otimes {\mathcal{H}}$.
\end{Theorem}

In this case, the dilation map $V_T$ is given by (see the equality
(1.5) in \cite{AE}): \[(V_{T} h)(z) = \sum_k \psi_k(z) \otimes
C^{\frac{1}{2}} \psi_k(T)^* h. \quad \quad \quad (h \in {\mathcal{H}})\]

Finally, note that the statement in Theorem \ref{MT} can be
generalized in this framework as follows (see Theorem 2.3 in
\cite{JS}): Let ${\mathcal{H}}$ be a Hilbert space and ${\mathcal{S}}$ be a closed
subspace of ${\mathcal{R}}_k \otimes {\mathcal{H}}$. Then ${\mathcal{S}}$ is $M_z$-invariant if
and only if ${\mathcal{S}} = \Theta H^2({\mathcal{E}})$ for some Hilbert space ${\mathcal{E}}$
and partially isometric multiplier $\Theta \in {\mathcal{M}}(H^2({\mathcal{E}}),
{\mathcal{R}}_k \otimes {\mathcal{H}})$.

{\noindent} Consequently, Theorem \ref{model-1} holds for the class of
$k$-contractions.

\begin{Theorem}
Let ${\mathcal{R}}_k$ be a SRKH and $T$ be a $k$-contraction on a Hilbert
space ${\mathcal{H}}$. Then there exists a Hilbert space ${\mathcal{E}}$ and a
partially isometric multiplier $\theta \in {\mathcal{M}}(H^2({\mathcal{E}}), {\mathcal{R}}_k
\otimes {\mathcal{H}})$ such that
\[T \cong P_{{\mathcal{Q}}_{\theta}} M_z|_{{\mathcal{Q}}_{\theta}},\]where
\[{\mathcal{Q}}_{\theta} = ({\mathcal{R}}_k \otimes {\mathcal{H}}) \ominus \theta H^2({\mathcal{E}}).\]
\end{Theorem}

Now let ${\mathcal{R}}_{k_i}$, $i = 1, \ldots, n$, be $n$ standard
reproducing kernel Hilbert spaces over ${\mathbb{D}}$ and let
\[{\mathcal{R}}_K := {\mathcal{R}}_{k_1} \otimes \cdots \otimes {\mathcal{R}}_{k_n}.\]Then
${\mathcal{R}}_K$ is a reproducing kernel Hilbert space (see Tomerlin
\cite{T}) corresponding to the kernel function \[K({\bm{z}}, {\bm{w}}) =
\prod_{i=1}^n k_i(z_i, w_i). \quad \quad ({\bm{z}}, {\bm{w}} \in {\mathbb{D}}^n)\]

Let $T = (T_, \ldots, T_n)$ be a doubly commuting tuple of operators
on a Hilbert space ${\mathcal{H}}$ and let $T_i$ be a $k_i$-contraction, $i =
1, \ldots, n$. Set \[C_i = WOT-\lim_{k {\rightarrow} \infty} p_{i,k}(T_i,
T_i^*),\]where $p_{i,k}(z, \bar{w}) {\rightarrow} \frac{1}{k_i(z, w)}$, $i =
1, \ldots, n$. In a similar way, as in Lemma \ref{Def1}, one can
prove that $C_i C_j = C_j C_i$ for all $i, j = 1, \ldots, n$, and
\[C_T:= \prod_{i=1}^n C_i \geq 0.\] By virtue of this
observation, a doubly commuting tuple $T$ is called $K$-contractive
if $T_i$ is $k_i$-contractive for all $i = 1, \ldots, n$ (see the
remark at the end of Lemma \ref{lemma1}). Consequently, all the
results and proofs in this paper hold verbatim for this notion of a
doubly commuting $K$-contractive tuples as well.

\begin{thebibliography}{99}
\bibitem{AIEOT}
J. Agler, {\em The Arveson extension theorem and coanalytic models},
Integral Equations Operator Theory 5 (1982), 608–-631.

\bibitem{Ag}
J. Agler, {\em Hypercontractions and subnormality}, J. Operator
Theory 13 (1985), 203–-217.

\bibitem{AEM}
C. Ambrozie, M. Englis and V. Muller, {\em Operator tuples and
analytic models over general domains in ${\mathbb{C}}^n$}, J. Operator Theory
47 (2002), 287–-302.

\bibitem{AT1}
C. Ambrozie and D. Timotin, {\em On an intertwining lifting theorem
for certain reproducing kernel Hilbert spaces}, Integral Equations
Operator Theory 42 (2002), 373–-384.

\bibitem{AT2}
C. Ambrozie and D. Timotin, {\em A von Neumann type inequality for
certain domains in $\mathbb{C}^n$}, Proc. Amer. Math. Soc. 131
(2003), 859–-869.

\bibitem{AE}
J. Arazy and M. Englis, {\em Analytic models for commuting operator
tuples on bounded symmetric domains}, Trans. Amer. Math. Soc. 355
(2003), 837–-864.

\bibitem{A} N. Aronszajn, {\em Theory of reproducing
kernels}, Trans. Amer. Math. Soc. 68 (1950), 337-404.

\bibitem{B}
A. Beurling, {\em On two problems concerning linear transformations
in Hilbert space}, Acta Math. 81 (1949), 239–-255.

\bibitem{BV1}
J. Ball and V. Bolotnikov, {\em A Beurling type theorem in weighted
Bergman spaces}, C. R. Math. Acad. Sci. Paris 351 (2013), 433–-436.

\bibitem{BNS}
T. Bhattacharyya, E. K. Narayanan and J. Sarkar {\em Analytic Model
of Doubly Commuting Contractions}, arXiv:1309.2384

\bibitem{CDS}
A. Chattopadhyay, B. K. Das and J. sarkar, {\em Tensor product of
quotient Hilbert modules},  J. Math. Anal. Appl. 424 (2015),
727–-747.

\bibitem{CV}
R. Curto and F.-H. Vasilescu, {\em Standard operator models in the
polydisc}, Indiana Univ. Math. J. 42 (1993), 791–-810.

\bibitem{INS}
K. Izuchi, T. Nakazi and M. Seto, {\em Backward shift invariant
subspaces in the bidisc II}, J. Oper. Theory 51 (2004), 361–-376.

\bibitem{MV}
V. Muller and F.-H. Vasilescu, {\em Standard models for some
commuting multioperators}, Proc. Amer. Math. Soc. 117 (1993),
979–-989.

\bibitem{NF}
B. Sz.-Nagy and C. Foias, {\em Harmonic Analysis of Operators on
Hilbert Space}, North Holland, Amsterdam, 1970.

\bibitem{O}
A. Olofsson, {\em A characteristic operator function for the class
of $n$-hypercontractions}, J. Funct. Anal. 236 (2006), 517–-545.

\bibitem{Pott}
S. Pott, {\em Standard models under polynomial positivity
conditions}, J. Operator Theory 41 (1999), 365–-389.

\bibitem{S-JOT}
J. Sarkar, {\em Jordan Blocks of $H^2({\mathbb{D}}^n)$}, J. Operator Theory 72
(2014) 371--385.

\bibitem{JS}
J. Sarkar, {\em An invariant subspace theorem and invariant
subspaces of analytic reproducing kernel Hilbert spaces - I},
arXiv:1309.2384. To appear in J. Operator Theory.

\bibitem{JS-S}
J. Sarkar, {\em  An introduction to Hilbert module approach to
multivariable operator theory}, To appear in Handbook of Operator
Theory, Springer. DOI 10.1007/978-3-0348-0692-3\_59-1

\bibitem{T}
A. Tomerlin, {\em Products of Nevanlinna-Pick kernels and operator
colligations}, Integral Equations Operator Theory 38 (2000),
350--356.

\end{thebibliography}

\end{document}


\documentclass[10pt,twoside,leqno]{siamltex}
\usepackage{amsmath,amssymb,amsfonts}							
\usepackage{epsfig}
\usepackage{float}
\usepackage{enumitem}
\usepackage[table]{xcolor}									
\usepackage{hyperref}

								
									
				

						
									
			
							
				
						
					
					

							
						
							
						

								
					

				
						
								
						
					

				

\setlength{\textheight}{190mm}
\setlength{\textwidth}{130mm}
\topmargin = 20mm

\setlength{\parskip}{.1in}

\newtheorem{rem}[theorem]{Remark}
\newtheorem{ex}[theorem]{Example}

\begin{document}

\title{Matrix functions that Preserve the Strong Perron-Frobenius Property\thanks{}}

\author{
Pietro Paparella\thanks{Division of Engineering and Mathematics, University of Washington Bothell, Bothell, Washington 98011, USA (pietrop@uw.edu)}
}

\pagestyle{myheadings}
\markboth{Pietro Paparella}{Matrix functions that Preserve the Strong Perron-Frobenius Property}
\maketitle

\begin{abstract}
In this note, we characterize matrix functions that preserve the strong Perron-Frobenius property using the \textit{real Jordan canonical form} of a real matrix.  
\end{abstract}

\begin{keywords}
Matrix function, Real Jordan Canonical form, Perron-Frobenius Theorem, Eventually Positive Matrix
\end{keywords}

\begin{AMS}
15A16, 15B48, 15A21.
\end{AMS}

\section{Introduction}

A real matrix has the \textit{Perron-Frobenius property} if its spectral radius is a positive eigenvalue corresponding to an entrywise nonnegative eigenvector. The \textit{strong Perron-Frobenius property} further requires that the spectral radius is simple; that it dominates in modulus every other eigenvalue; and that it has an entrywise positive eigenvector.

In \cite{mw1979}, Micchelli and Willoughby characterized matrix functions that preserve \textit{doubly nonnegative matrices}. In \cite{gkr2013}, Guillot et al.~used these results to solve the \textit{critical exponent conjecture} established in \cite{jlw2011}. In \cite{bh2008}, Bharali and Holtz characterized entire functions that preserve nonnegative matrices of a fixed order and, in addition, they characterized matrix functions that preserve nonnegative block triangular, circulant, and symmetric matrices. In \cite{es2010}, Elhashash and Szyld characterized entire functions that preserve sets of generalized nonnegative matrices. 

In this work, using the characterization of a matrix function via the \textit{real Jordan canonical form} established in \cite{mpt2014}, we characterize matrix functions that preserve the strong Perron-Frobenius property. Although our results are similar to those presented in \cite{es2010}, the assumption of entirety of a function is dropped in favor of analyticity in some domain containing the spectrum of a matrix.

\section{Notation}

Denote by ${M_{{n}}({{\mathbb{{C}}}})}$ (respectively, ${M_{{n}}({{\mathbb{{R}}}})}$) the algebra of complex (respectively, real) $n \times n$ matrices. Given $A \in {M_{{n}}({{\mathbb{{C}}}})}$, the \textit{spectrum} of $A$ is denoted by ${\sigma \left( {A} \right)}$, and the \textit{spectral radius} of $A$ is denoted by ${\rho\left({A}\right)}$. 

 The \textit{direct sum} of the matrices $A_1, \dots, A_k$, where $A_i \in {M_{{n_i}}({{\mathbb{{C}}}})}$, denoted by $A_1 \oplus \dots \oplus A_k$, $\bigoplus_{i=1}^k A_i$, or $\diag{(A_1,\dots,A_k)}$, is the $n \times n$ matrix 
\[ \begin{bmatrix} A_1 \\ & \ddots \\ & & A_k \end{bmatrix},~n = \sum_{i=1}^k n_i. \]

For $\lambda \in {\mathbb{{C}}}$, ${J_{{n}}{\left( {\lambda} \right)}}$ denotes the $n \times n$ \textit{Jordan block} with eigenvalue $\lambda$. For $A \in {M_{{n}}({{\mathbb{{C}}}})}$, denote by $J = {{Z}^{-1}} A Z = \bigoplus_{i=1}^t {J_{{n_i}}{\left( {\lambda_i} \right)}} = \bigoplus_{i=1}^t J_{n_i}$, where $\sum n_i = n$, a Jordan canonical form of $A$. Denote by $\lambda_1,\dots,\lambda_s$ the \textit{distinct} eigenvalues of $A$, and, for $i=1,\dots,s$, let $m_i$ denote the \textit{index} of $\lambda_i$, i.e., the size of the largest Jordan block associated with $\lambda_i$. Denote by ${\textup{i}}$ the imaginary unit, i.e., ${\textup{i}} := \sqrt{-1}$.
 
A domain $\mathcal{D}$ is any open and connected subset of ${\mathbb{{C}}}$. We call a domain \textit{self-conjugate} if $\bar{\lambda} \in \mathcal{D}$ whenever $\lambda \in \mathcal{D}$ (i.e., $\mathcal{D}$ is symmetric with respect to the real-axis). Given that an open and connected set is also path-connected, it follows that if $\mathcal{D}$ is self-conjugate, then ${\mathbb{{R}}} \cap \mathcal{D} \neq \emptyset$.  

\section{Background}

Although there are multiple ways to define a matrix function (see, e.g., \cite{h2008}), our preference is via the Jordan Canonical Form.

\begin{definition} \label{def:functionvalues}
{\rm Let $f : {\mathbb{{C}}} \longrightarrow {\mathbb{{C}}}$ be a function and denote by $f^{(j)}$ the $j$th derivative of $f$. The function $f$ is said to be \textit{defined on the spectrum of $A$} if the values
\begin{align*}
\begin{array}{c c c}
f^{(j)}(\lambda_i), & j=0,\dots,m_i-1, & i=1,\dots,s,
\end{array}
\end{align*}
called \textit{the values of the function $f$ on the spectrum of $A$}, exist.}
\end{definition}

\begin{definition}[Matrix function via Jordan canonical form] 
{\rm If $f$ is defined on the spectrum of $A \in {M_{{n}}({{\mathbb{{C}}}})}$, then
\begin{align*} 
f(A) := Z f(J) {{Z}^{-1}} = Z \left( \bigoplus_{i=1}^t f(J_{n_i}) \right) {{Z}^{-1}}, 
\end{align*}
where
\begin{align}
f(J_{n_i}) := 
\begin{bmatrix} 
f(\lambda_i) & f'(\lambda_i) & \dots   & \frac{f^{(n_i-1)}(\lambda_i)}{(n_i - 1)!} 	\\
	          & f(\lambda_i)  & \ddots & \vdots 						\\
	          &                       & \ddots & f'(\lambda_i)						\\
	          &  	  	   &             & f(\lambda_i)
\end{bmatrix}. \label{fjb}
\end{align}}
\end{definition}

The following theorem is well-known (for details see, e.g., \cite{hj1990}, \cite{lt1985}; for a complete proof, see, e.g., \cite{glr1986}). 

\begin{theorem}[Real Jordan canonical form] \label{thm:rjcf} 
If $A \in {M_{{n}}({{\mathbb{{R}}}})}$ has $r$ real eigenvalues (including multiplicities) and $c$ complex conjugate pairs of eigenvalues (including multiplicities), then there exists an invertible matrix $R \in {M_{{n}}({{\mathbb{{R}}}})}$ such that 
\begin{align}
{{R}^{-1}} A R =
\begin{bmatrix} 
\bigoplus_{k=1}^{r} J_{n_k} (\lambda_k) & \\ & \bigoplus_{k = r + 1}^{r + c} C_{n_k} (\lambda_k)  
\end{bmatrix},															\label{realjordform}	
\end{align}
where:
\begin{enumerate}
\item
\begin{align}
C_{j} (\lambda) := 
\begin{bmatrix} 
C(\lambda) 	& I_2                        			\\
            	& C(\lambda) & \ddots 			\\
            	&                    & \ddots & I_2 		\\ 
            	&                    &            & C(\lambda) 
\end{bmatrix} \in {M_{{2j}}({{\mathbb{{R}}}})}; 												\label{Ck_lambda} 	
\end{align}
\item
\begin{align} 
C(\lambda) := 
\begin{bmatrix} 
\Re{(\lambda)} & \Im{(\lambda)} \\ 
-\Im{(\lambda)} & \Re{(\lambda)}  
\end{bmatrix} \in {M_{{2}}({{\mathbb{{R}}}})}; 												\label{C_lambda}	
\end{align}
\item $\Im{(\lambda_k)} = 0$, $k=1, \dots, r$; and 
\item $\Im{(\lambda_k)} \neq 0$, $k=r+1, \dots, r+c$.
\end{enumerate}
\end{theorem}

\begin{proposition}[{\cite[Corollary 2.11]{mpt2014}}] \label{prop:rjcf_cor} 
Let $\lambda \in {\mathbb{{C}}}$, $\lambda \neq 0$, and let $f$ be a function defined on the spectrum of ${J_{{k}}{\left( {\lambda} \right)}} \oplus {J_{{k}}{\left( {\bar{\lambda}} \right)}}$.  For $j$ a nonnegative integer, let $f^{(j)}_\lambda$ denote $f^{(j)}(\lambda)$. If $C_k (\lambda)$ and $C(\lambda)$ are defined as in \eqref{Ck_lambda} and \eqref{C_lambda}, respectively, then
\begin{align*} 
f(C_{k} (\lambda)) = 
\begin{bmatrix} 
f(C_\lambda) & f'(C_\lambda) & \dots   & \frac{f^{(k-1)}(C_\lambda)}{(k-1)!}  	\\
		 & f(C_\lambda) & \ddots & \vdots 					\\
		 & 		     & \ddots & f'(C_\lambda)				\\
		 & 		     & 	         & f(C_\lambda)
\end{bmatrix} \in {M_{{2k}}({{\mathbb{{C}}}})},
\end{align*}
and, moreover, 
\begin{align*} 
f ( C_{k} (\lambda) ) = 
\begin{bmatrix} 
C(f_\lambda) & C(f'_\lambda) & \dots & C \left( \frac{f^{(k-1)}_\lambda}{(k-1)!} \right)	\\
& C(f_\lambda) & \ddots & \vdots 									\\
& & \ddots & C(f'_\lambda) 										\\
& & & C(f_\lambda)
\end{bmatrix} \in {M_{{2k}}({{\mathbb{{R}}}})}
\end{align*}
if and only if $\overline{f_\lambda^{(j)}} = f_{\bar{\lambda}}^{(j)}$.
\end{proposition}

We recall the Perron-Frobenius theorem for positive matrices (see \cite[Theorem 8.2.11]{hj1990}).

\begin{theorem} \label{thm:pf} 
If $A \in {M_{{n}}({{\mathbb{{R}}}})}$ is positive, then 
\begin{enumerate}[label=(\roman*)]
\item $\rho := {\rho\left({A}\right)} > 0$;
\item $\rho \in {\sigma \left( {A} \right)}$;
\item there exists a positive vector $x$ such that $Ax = \rho x$;
\item $\rho$ is a simple eigenvalue of $A$; and 
\item $|\lambda| < \rho$ for every $\lambda \in {\sigma \left( {A} \right)}$ such that $\lambda \neq \rho$. 
\end{enumerate}
\end{theorem}

One can verify that the matrix 
\begin{align} 
B = \begin{bmatrix} 2 & 1 \\ 2 & -1 \end{bmatrix} \label{matB}
\end{align} 
satisfies properties (i) through (v) of {\hyperref[{thm:pf}]{{\rm {Theorem} \ref*{{thm:pf}}}}}, but obviously contains a negative entry. This motivates the following concept.

\begin{definition} 
{\rm A matrix $A \in {M_{{n}}({{\mathbb{{R}}}})}$ is said to \textit{possess the strong Perron-Frobenius property} if $A$ satisfies properties (i) through (v) of {\hyperref[{thm:pf}]{{\rm {Theorem} \ref*{{thm:pf}}}}}.}
\end{definition}

It can also be shown that the matrix $B$ given in \eqref{matB} satisfies $B^k > 0$ for $k \geq 4$, which leads to the following generalization of positive matrices. 

\begin{definition}
{\rm A matrix $A \in {M_{{n}}({{\mathbb{{R}}}})}$ is \textit{eventually positive} if there exists a nonnegative integer $p$ such that $A^k > 0$ for all $k \geq p$.} 
\end{definition}

The following theorem relates the strong Perron-Frobenius property with eventually positive matrices (see \cite[Lemma 2.1]{h1981}, \cite[Theorem 1]{jt2004}, or \cite[Theorem 2.2]{n2006}).

\begin{theorem} \label{evpos_thm}
A real matrix $A$ is eventually positive if and only if $A$ and $A^\top$ possess the strong Perron-Frobenius property. 
\end{theorem}

\section{Main Results}

Before we state our main results, we begin with the following definition.

\begin{definition}
{\rm A function $f: {\mathbb{{C}}} \longrightarrow {\mathbb{{C}}}$ defined on a self-conjugate domain $\mathcal{D}$, $\mathcal{D} \cap {\mathbb{{R}}}^+ \neq \emptyset$, is called \textit{Frobenius}\footnote{We use the term `Frobenius' given that such a function preserves \textit{Frobenius multi-sets}, introduced by Friedland in \cite{f1978}.} if 
\begin{enumerate}[label=(\roman*)]
\item \label{item:selfconjugacy} $\overline{f(\lambda)} = f(\bar{\lambda})$, $\lambda \in \mathcal{D}$; 
\item \label{item:mod} $|f(\lambda)| < f(\rho)$, whenever $|\lambda|<\rho$, and $\lambda$, $\rho \in \mathcal{D}$.
\end{enumerate}}
\end{definition}

\begin{rem}
{\rm Condition \ref*{item:selfconjugacy} implies $f(r) \in {\mathbb{{R}}}$, whenever $r \in \mathcal{D} \cap {\mathbb{{R}}}$; and condition \ref*{item:mod} implies $f(r) \in {\mathbb{{R}}}^+$, whenever $r \in \mathcal{D} \cap {\mathbb{{R}}}^+$.}
\end{rem}

The following theorem is our first main result. 

\begin{theorem} \label{mainresultdiag}
Let $A \in {M_{{n}}({{\mathbb{{R}}}})}$ and suppose that $A$ is diagonalizable and possesses the strong Perron-Frobenius property. If $f: {\mathbb{{C}}} \longrightarrow {\mathbb{{C}}}$ is a function defined on the spectrum of $A$, then $f(A)$ possesses the strong Perron-Frobenius property if and only if $f$ is Frobenius.
\end{theorem}

\begin{proof}
Suppose that $f$ is Frobenius. For convenience, denote by $f_\lambda$ the scalar $f(\lambda)$. Following {\hyperref[{thm:rjcf}]{{\rm {Theorem} \ref*{{thm:rjcf}}}}} and {\hyperref[{prop:rjcf_cor}]{{\rm {Proposition} \ref*{{prop:rjcf_cor}}}}}, the matrix
\begin{align}  
f(A)  = 
R \begin{bmatrix} 
f_{{\rho\left({A}\right)}} & & 					\\
& \bigoplus_{k=2}^{r} f_{\lambda_k} & 	\\ 
& & \bigoplus_{k = r + 1}^{r + c} C (f_{\lambda_k})  
\end{bmatrix} {{R}^{-1}}, 													\label{fofAdiag} 
\end{align}
where $R= 
\begin{bmatrix}
x & R'
\end{bmatrix},~x>0$,
is real. If ${\sigma \left( {A} \right)} = \{ {\rho\left({A}\right)}, \lambda_2,\dots, \lambda_n \}$, then ${\sigma \left( {f(A)} \right)} = \{ f_{{\rho\left({A}\right)}},f_{\lambda_2},\dots, f_{\lambda_n} \}$ (see, e.g., \cite{h2008}[Theorem 1.13(d)]) and because $f$ is Frobenius, it follows that $|f_{\lambda_k}| < f_{{\rho\left({A}\right)}}$ for $k = 2,\dots, n$. Moreover, from \eqref{fofAdiag} it follows that $f(A)x = f_{{\rho\left({A}\right)}} x$. Thus, $f(A)$ possesses the strong Perron-Frobenius property.  

Conversely, if $f$ is not Frobenius, then the matrix $f(A)$, given by \eqref{fofAdiag}, is not real (e.g, $\exists \lambda \in {\sigma \left( {A} \right)}$, $\lambda \in {\mathbb{{R}}}$ such that $f(\lambda) \not \in {\mathbb{{R}}}$), or $f(A)$ does not retain the strong Perron-Frobenius property (e.g., $\exists \lambda \in {\sigma \left( {A} \right)}$ such that $|f(\lambda)| \geq f({\rho\left({A}\right)})$). 
\end{proof}

\begin{ex}
{\rm {\hyperref[{table:frobfunctions}]{{\rm {Table} \ref*{{table:frobfunctions}}}}} lists examples of Frobenius functions for diagonalizable matrices that possess the strong Perron-Frobenius property. 
\begin{table}[H]
\begin{center}
\begin{tabular}{c|c}
$f$ & $\mathcal{D}$										\\
\hline
$f(z) = z^p$, $p \in {\mathbb{{N}}}$ & ${\mathbb{{C}}}$ 							\\
$f(z) = |z|$ & ${\mathbb{{C}}}$										\\
$f(z) = z^{1/p}$, $p \in {\mathbb{{N}}}$, $p$ even & $\{z \in {\mathbb{{C}}}: z \not \in {\mathbb{{R}}}^- \}$  \\
$f(z) = z^{1/p}$, $p \in {\mathbb{{N}}}$, $p$ odd, $p>1$ & ${\mathbb{{C}}}$  				\\
$f(z) = \sum_{k=0}^n a_k z^k$, $a_k >0$ & ${\mathbb{{C}}}$					\\
$f(z) = \exp{(z)}$ & ${\mathbb{{C}}}$
\end{tabular}
\caption{Examples of Frobenius functions.} \label{table:frobfunctions}
\end{center}
\end{table}}
\end{ex}

For matrices that are not diagonalizable, i.e., possessing Jordan blocks of size two or greater, given \eqref{fjb} it is reasonable to assume that $f$ is complex-differentiable, i.e., \textit{analytic}. We note the following result, which is well known (see, e.g., \cite{bc2013}, \cite{c1995}, \cite[Theorem 3.2]{hmmt2005}, \cite{l1999}, or \cite{r1987}).

\begin{theorem}[Reflection Principle] \label{thm:refprinc}
Let $f$ be analytic in a self-conjugate domain $\mathcal{D}$ and suppose that $I := \mathcal{D} \cap {\mathbb{{R}}} \neq \emptyset$. Then $\overline{f(\lambda)} = f( \bar{\lambda})$ for every $\lambda \in \mathcal{D}$ if and only if $f(r) \in {\mathbb{{R}}}$ for all $r \in I$.
\end{theorem}

The Reflection Principle leads immediately to the following result.

\begin{corollary}
An analytic function $f: {\mathbb{{C}}} \longrightarrow {\mathbb{{C}}}$ defined on a self-conjugate domain $\mathcal{D}$, $\mathcal{D} \cap {\mathbb{{R}}}^+ \neq \emptyset$, is \textit{Frobenius} if and only if
\begin{enumerate}[label=(\roman*)]   
\item \label{item:real2} $f(r) \in {\mathbb{{R}}}$, whenever $r \in \mathcal{D} \cap {\mathbb{{R}}}$; and
\item \label{item:modulus2} $|f(\lambda)| < f(\rho)$, whenever $|\lambda|<\rho$ and $\lambda$, $\rho \in \mathcal{D}$.
\end{enumerate}
\end{corollary}

\begin{lemma} \label{lem:analytic}
Let $f$ be analytic in a domain $\mathcal{D}$ and suppose that $I := \mathcal{D} \cap {\mathbb{{R}}} \neq \emptyset$. If $f(r) \in {\mathbb{{R}}}$ for all $r \in I$, then $f^{(j)}(r) \in {\mathbb{{R}}}$ for all $r \in I$ and $j \in {\mathbb{{N}}}$..
\end{lemma}

\begin{proof}
Proceed by induction on $j$: when $j=1$, note that, since $f$ is analytic on $\mathcal{D}$, it is \textit{holomorphic} (i.e., complex-differentiable) on $\mathcal{D}$. Thus,   
\begin{align*}
f'(r) := \lim_{z \rightarrow r} \frac{f(z) - f(r)}{z-r}
\end{align*}
exists for all $z \in \mathcal{D}$; in particular, 
\begin{align*}
f'(r) = \lim_{x \rightarrow r} \frac{f(x) - f(r)}{x - r},~x \in I, 
\end{align*}
and the conclusion that $f'(r) \in {\mathbb{{R}}}$ follows by the hypothesis that $f(x) \in {\mathbb{{R}}}$ for all $x \in I$.

Next, assume that the result holds when $j=k-1>1$. As above, note that $f^{(k)}(r)$ exists and 
 \begin{align*}
f^{(k)}(r) = \lim_{x \rightarrow r} \frac{f^{(k-1)}(x) - f^{(k-1)}(r)}{x - r},~x \in I 
\end{align*}
so that $f^{(k)}(r) \in {\mathbb{{R}}}$.
\end{proof}

\begin{theorem} \label{thm:mainresult}
Let $A \in {M_{{n}}({{\mathbb{{R}}}})}$ and suppose that $A$ possesses the strong Perron-Frobenius property. If $f: {\mathbb{{C}}} \longrightarrow {\mathbb{{C}}}$ is an analytic function defined in a self-conjugate domain $\mathcal{D}$ containing ${\sigma \left( {A} \right)}$, then $f(A)$ possesses the strong Perron-Frobenius property if and only if $f$ is Frobenius.
\end{theorem}

\begin{proof}
Suppose that $f$ is Frobenius. Following {\hyperref[{thm:rjcf}]{{\rm {Theorem} \ref*{{thm:rjcf}}}}}, there exists an invertible matrix $R$ such that 
\begin{align*}
{{R}^{-1}} A R =
\begin{bmatrix}
{\rho\left({A}\right)} & 															\\ 
	& \bigoplus_{k=2}^{r} J_{n_k} (\lambda_k) 	& 								\\ 
	&							& \bigoplus_{k = r + 1}^{r + c} C_{n_k} (\lambda_k)  
\end{bmatrix},															
\end{align*}
where 
\begin{align*}
R= 
\begin{bmatrix}
x & R'
\end{bmatrix},~x>0.
\end{align*}
Because $f$ is Frobenius, following {\hyperref[{thm:refprinc}]{{\rm {Theorem} \ref*{{thm:refprinc}}}}}, $\overline{f(\lambda)} = f(\bar{\lambda})$ for all $\lambda \in \mathcal{D}$. Since $f$ is analytic, $f^{(j)}$ is analytic for all $j \in {\mathbb{{N}}}$ and, following {\hyperref[{lem:analytic}]{{\rm {Lemma} \ref*{{lem:analytic}}}}} $f^{(j)}(r) \in {\mathbb{{R}}}$ for all $r \in I$. Another application of {\hyperref[{thm:refprinc}]{{\rm {Theorem} \ref*{{thm:refprinc}}}}} yields that $\overline{f^{(j)}(\lambda)} = f^{(j)}(\bar{\lambda})$ for all $\lambda \in \mathcal{D}$. Hence, following {\hyperref[{prop:rjcf_cor}]{{\rm {Proposition} \ref*{{prop:rjcf_cor}}}}}, the matrix 
\begin{align*}
f(A) =
R
\begin{bmatrix}
f({\rho\left({A}\right)}) 	& 																	\\ 
		& \bigoplus_{k=2}^{r} f(J_{n_k} (\lambda_k)) 	& 									\\ 
		&								& \bigoplus_{k = r + 1}^{r + c} f(C_{n_k}(\lambda_k))  
\end{bmatrix} {{R}^{-1}}															
\end{align*}
is real and possesses the strong Perron-Frobenius property.

The proof of the converse is identical to the proof of the converse of {\hyperref[{mainresultdiag}]{{\rm {Theorem} \ref*{{mainresultdiag}}}}}. 
\end{proof}

\begin{corollary}
Let $A \in {M_{{n}}({{\mathbb{{R}}}})}$ and suppose that $A$ is eventually positive. If $f: {\mathbb{{C}}} \longrightarrow {\mathbb{{C}}}$ is an analytic function defined in a self-conjugate domain $\mathcal{D}$ containing ${\sigma \left( {A} \right)}$, then $f(A)$ is eventually positive if and only if $f$ is Frobenius.
\end{corollary}

\begin{proof}
Follows from {\hyperref[{thm:mainresult}]{{\rm {Theorem} \ref*{{thm:mainresult}}}}} and the fact that $f(A^\top) = (f(A))^\top$ (\cite[Theorem 1.13(b)]{h2008}).
\end{proof}

\begin{rem}
{\rm Aside from the function $f(z) = |z|$, which is nowhere differentiable, every function listed in {\hyperref[{table:frobfunctions}]{{\rm {Table} \ref*{{table:frobfunctions}}}}} is analytic and Frobenius.}
\end{rem}

\section{Acknowledgements}

I gratefully acknowledge Hyunchul Park for discussions arising from this research and the anonymous referees for their helpful suggestions.

\bibliographystyle{abbrv}
\bibliography{laabib,crs}

\end{document}


\documentclass[12pt]{amsart}
 \usepackage{amsfonts,amssymb,eucal}
\usepackage{amsthm} 
 \usepackage{amsmath} 
\usepackage{amscd}
 \usepackage{latexsym} 

 
\font\tencyr=wncyr10

\numberwithin{equation}{section}
\usepackage{color}
 

 

  

\newtheorem{deff}{Definition}

 

\newtheorem{Remark}{Remark}

\newtheorem{Theorem}{Theorem}

\newtheorem{Proposition}{Proposition}
\newtheorem{lemma}{Lemma}
\newtheorem{corollary}{Corollary}
\newtheorem{Comment}{Comment}
\theoremstyle{definition} 

\newtheorem{theorem}{Theorem}[section]
\textwidth15.6cm
\textheight22cm
\hoffset-1cm
\voffset-1cm
 

\begin{document}
\title{Wei-Norman and Berezin's equations of motion on the Siegel-Jacobi disk} 
\author{Stefan  Berceanu}
\address[Stefan  Berceanu]{National
 Institute for Physics and Nuclear Engineering\\
         Department of Theoretical Physics\\
         PO BOX MG-6, Bucharest-Magurele, Romania}
\email{Berceanu@theory.nipne.ro}

\begin{abstract}
We show that the Wei-Norman method applied to describe the evolution  on the
Siegel-Jacobi disk  $\mathcal{D}^J_1=\mathcal{D}_1\times\mathbb{C}^1$, 
where $\mathcal{D}_1$  denotes the Siegel disk,  determined by a hermitian 
Hamiltonian   linear in the generators of the Jacobi group $G^J_1$  and Berezin's
scheme using coherent states give the same equations of  quantum and
classical motion when are expressed in  the coordinates in which the K\"ahler  
two-form $\omega_{\mathcal{D}^J_1} $ can be written as
$\omega_{\mathcal{D}^J_1}=\omega_{\mathcal{D}_1}+\omega_{\mathbb{C}^1}$. 
The Wei-Norman equations on $\mathcal{D}^J_1$  are a particular case
of equations  of motion on the Siegel-Jacobi ball $\mathcal{D}^J_n$
generated by a hermitian Hamiltonian  linear in the generators of the
Jacobi group $G^J_n$ obtained in Berezin's approach based on  coherent 
states on $\mathcal{D}^J_n$.
\end{abstract}
\subjclass{81S10, 34A05, 32Q15, 81Q70}
\keywords{Jacobi group, Siegel-Jacobi disk,  Wei-Norman method,
  quantization  Berezin, coherent states, differential-geometric methods}

\maketitle
\noindent
\tableofcontents
\newpage
\section{Introduction}\label{intro}

The Jacobi groups -
$G^J_n=H_n\rtimes\text{Sp}(n,{\ensuremath{\mathbb{R}}})_{\ensuremath{\mathbb{C}}}$ -  where  $H_n$ denotes the 
$(2n+1)$-dimensional Heisenberg group, are unimodular,
nonreductive,  algebraic groups  \cite{ez,bs,TA99, LEE03,gem}  of Harish-Chandra type \cite{satake}. 
The Siegel-Jacobi
domains are nonsymmetric domains associated to the Jacobi groups
by the generalized Harish-Chandra embedding \cite{satake,LEE03, Y02,Y10,gem}.  The Jacobi group  is also an important object in
physics, where  sometimes it is    known  under 
 other names, as 
 {\it Hagen} \cite{hagen},  {\it Schr\"odinger} \cite{ni},    or {\it Weyl-symplectic} group
 \cite{kbw1}. The Jacobi
group describes the   squeezed states  \cite{stol,lu,ho} in Quantum Optics
\cite{mandel,ali,dr}. More references on this subject can be find in \cite{jac1,sbj,sbcg,nou,FC}.

The  Jacobi group  $G^J_n$ has been   studied  in  connection with the
 group-theoretic approach to coherent states 
\cite{perG} in 
\cite{jac1,FC} in the case $n=1$, while the case $n\in{\ensuremath{\mathbb{N}}}$ has been
treated  in \cite{mlad,sbj,nou}.  We have attached  to the Jacobi group
$G^J_n$ coherent states  based on
Siegel-Jacobi ball  ${{\mathcal{{D}}}}^J_n$ \cite{sbj}, which, as set, consists
of the points of 
${\ensuremath{\mathbb{C}}}^n\times{{\mathcal{{D}}}}_n$.  The non-compact hermitian
symmetric space $ \operatorname{Sp}(n, {\ensuremath{\mathbb{R}}}
)_{\ensuremath{\mathbb{C}}}/\operatorname{U}(n)$ admits a matrix realization  as a bounded
homogeneous domain, the Siegel ball ${{\mathcal{{D}}}}_n$,
${{\mathcal{{D}}}}_n:=\{W\in  M (n, {\ensuremath{\mathbb{C}}} ): W=W^t, {{{\mathbb{I}}_n}}-W\bar{W} > 0\}.$ 
 We have determined the $G^J_n$-invariant
K\"ahler two-form $\omega_{{{\mathcal{{D}}}}^J_n}$ on ${{\mathcal{{D}}}}^J_n$  \cite{sbj}, also 
  investigated  by Yang 
\cite{Y10}. In \cite{mlad,sbj,nou}  the $G^J_n$-invariant K\"ahler
two-form $\omega_{{{\mathcal{{D}}}}_n}(z,W)$,  where $z\in{\ensuremath{\mathbb{C}}}^n, W\in {{\mathcal{{D}}}}_n$, is  written compactly as the sum of two terms, one describing  the homogeneous K\"ahler
two-form   $\omega_{{{\mathcal{{D}}}}_n}(W)$ on 
${{\mathcal{{D}}}}_n$,  the other one is 
${\ensuremath{{\mbox{\rm{Tr}}}}}(A^t({{{\mathbb{I}}_n}}-\bar{W}W)^{-1}\!\wedge\bar{A})$,  where $ A={\operatorname{d}} z +{\operatorname{d}} W\bar{\eta}$, and 
$\eta=({{{\mathbb{I}}_n}}-W\bar{W})^{-1}(z+W \bar{z})$,  \cite{mlad,sbj}. We have 
denoted by $FC$ the change of variables $FC: ~
{\ensuremath{\mathbb{C}}}^n\times{{\mathcal{{D}}}}_n$$\ni 
(\eta,W)\rightarrow$$ (z,W)\in{{\mathcal{{D}}}}^J_n$, $z= \eta-W\bar{\eta}$. We
have shown \cite{nou} that  {\it the $FC$-transform
  is a K\"ahler homogeneous diffeomorphism}, and,  
 when expressed in the variables
$(\eta,W)\in{\ensuremath{\mathbb{C}}}^n\times{{\mathcal{{D}}}}_n$ -- let us call them $FC$-{\it variables} --
the K\"ahler two-form 
$\omega_{{{\mathcal{{D}}}}_n}(\eta,W)=\omega_{{{\mathcal{{D}}}}_n}(W)+\omega_{{\ensuremath{\mathbb{C}}}^n}({\eta})$. 
We have put in \cite{nou}
this change of variables in connection with the celebrated {\it fundamental
conjecture}  on homogeneous K\"ahler manifolds  of Gindikin and
Vinberg \cite{GV,DN} on the Siegel-Jacobi ball ${{\mathcal{{D}}}}^J_n$, 
as we did in \cite{FC} for the Siegel-Jacobi disk ${{\mathcal{{D}}}}^J_1$. Later,
we have underlined in \cite{ber14} that the $FC$-transform has a deep meaning in the
context of Perelomov coherent states  \cite{perG}: it
gives the change of variables from the representation of the normalized
to the un-normalized coherent state vector, as is recalled  in the
Appendix  in \S\ref{app2}.

The equations of motion on the Siegel-Jacobi ball ${{\mathcal{{D}}}}^J_n$
determined by a hermitian  Hamiltonian   linear in the generators of
the Jacobi group $G^J_n$ were studied in \cite{nou}, generalizing the
results presented  in \cite{jac1,FC},  obtained from the holomorphic differential
representation of the Lie algebra ${{\mathfrak{{g}}}}^J_1$  of the Jacobi group.
 
We recall that linear Hamiltonians in generators of the Jacobi group appear in
  quantum mechanics, as in the case of the  quantum oscillator
acted on by a variable external force \cite{fey,sw,hs} and  in the case of  quantum
dynamics of trapped ions \cite{viorica,ma}.

The aim of this paper is to compare some of the results obtained in
our papers \cite{jac1,FC,nou} concerning the equations of motion  on
the Siegel-Jacobi ${{\mathcal{{D}}}}^J_n$ determined by a hermitian 
Hamiltonian  linear in the generators of the Jacobi group $G^J_n$ with the
results of the paper \cite{cezar} referring to the equations
of motion on ${{\mathcal{{D}}}}^J_1$ obtained with the Wei-Norman method
\cite{wei}. In \cite{cezar} the  dynamics on ${{\mathcal{{D}}}}^J_1$ determined
by a   hermitian  Hamiltonian linear in the generators of $G^J_1$ is considered
in the general framework of Lie systems \cite{lie,lies},  as was  developed
further  in a  geometric approach in \cite{ca1,ca2}. The Wei-Norman
equations for  the  Jacobi group $G^J_1$ in real coordinates have been studied  in
\cite{ca3,ca4}.

In order to establish a correspondence between the formulae of
\cite{cezar} and our notation, we give the following dictionary  (firstly are introduced the symbols used in \cite{cezar}): affine symplectic group $G^{AS}=\text{SL}(2,{\ensuremath{\mathbb{R}}})\rtimes{\ensuremath{\mathbb{R}}}^2$ $\leftrightarrow $ Jacobi group
$G^J_1:=H_1\rtimes \text{SU}(1,1)$; extended Poincar\'e disk ${{\mathfrak{{M}}}}={{\mathfrak{{D}}}}\times{\ensuremath{\mathbb{R}}}^2$ $\leftrightarrow$
Siegel-Jacobi disk ${{\mathcal{{D}}}}^J_1={\ensuremath{\mathbb{C}}}\times{{\mathcal{{D}}}}_1$. 

Essentially, the  Wei-Norman method (see \cite{wei} and the Appendix
in \S\ref{app1}) consists in representing  the solution
of the equation 
$$\frac{{\operatorname{d}} U(t)}{{\operatorname{d}} t}= A(t)U(t), \quad U(0)=I, $$
 in the form of product of exponentials
\begin{equation}\label{tsiL}U(t)=\prod_{i=1}^n\exp(\xi_i(t)X_i),
\end{equation}
 where $A$
and $U$ are linear operators, 
\begin{equation}\label{linA}
A(t)=\sum_{i=1}^n\epsilon_i(t)X_i,\end{equation}  
$\epsilon_i(t)$ are scalar functions of
$t$ and $\{X_i\}_{i=1,\dots,n}$ are the generators of a Lie algebra ${{\mathfrak{{g}}}}$. 

On the other side, we have developed in \cite{sbcag,sbl}  an algebraic  method to obtain a
representation of a Lie algebra  ${{\mathfrak{{g}}}}$ of a Lie group $G$ as first
order holomorphic differential operator  on $M=G/H$  when
$M$ is a hermitian symmetric manifold. Later we have applied    the method
to a larger class of Lie groups, advancing the hypothesis that for the
{\it coherent type groups} \cite{lis,neeb}, i.e. Lie groups for which
the $n$-dimensional homogeneous manifold $M$ admits an holomorphic embedding in a
projective Hilbert space $M\hookrightarrow{\ensuremath{{\ensuremath{\mathbb{P}}} ({\ensuremath{\mathcal{H}^{\infty}}} )}}$,  {\it the  generators of
the Lie algebra ${{\mathfrak{{g}}}}$ of the Lie group $G$ admit a  
holomorphic differential representation} 
\begin{equation}\label{genPQ}
{{\mathfrak{{g}}}}\ni X \mapsto {{\mathbb{{X}}}}(z)=P_X(z) +\sum_iQ^i_X(z)\frac{\partial}{{\partial}
  z_i}, \end{equation} 
where  $P_X(z)$, and $Q^i_X(z)$ are {\it polynomials} defined on
$M=G/H$. We have verified \cite{jac1,sbj,nou} this hypothesis in the
case of {\it the Jacobi
group} $G^J_n$,  which {\it  is a  coherent type group}
\cite{neeb,ber14}. Following a method advanced in \cite{sbcag,sbl}, 
which uses  Perelomov coherent states \cite{perG} and a {\it
  dequantization} method developed by Berezin \cite{berezin2,berezin1}, we have determined the
equations of motion on $M={{\mathcal{{D}}}}^J_n$ when the Hamiltonian $\bf{H}$ is linear
in the generators of the Jacobi group in the case $n=1$ in  
\cite{jac1,FC} and in \cite{nou} for $n\in{\ensuremath{\mathbb{N}}}$. In general, for groups $G$  for which the representation \eqref{genPQ}  of the Lie algebra
  ${{\mathfrak{{g}}}}$  of the Lie group $G$ is true, the equations of motion on $M=G/H$
depend on the
coefficients $\epsilon_i$ in front of the generators $X_i$ of the group  $G$  which
appear  in  ${{\mbox{\boldmath{${H}$}}}}$ of the form \eqref{linA}  and the
polynomials $Q^i$ which appear in \eqref{genPQ},  as it  is recalled in
Proposition \ref{propvechi}.  To shorten the expression, we call the
equations of motion obtained with this method,  {\it Berezin's
  equations of motion}. We have shown that  for
a Hamiltonian  linear in the generators of $G^J_n$, the motion on
${{\mathcal{{D}}}}_n$ is described by a matrix Riccati equation, while the motion
in $z\in{\ensuremath{\mathbb{C}}}^n$ is a first order differential equation, with coefficients
depending also on $W\in{{\mathcal{{D}}}}_n$. It was proved in \cite{FC}  for
$G^J_1$ and in 
\cite{nou} for $G^J_n$, $n\in{\ensuremath{\mathbb{N}}}$ that, when the $FC$-transform is applied, the
first order differential equation in the variable $\eta$ becomes
decoupled from the motion on the Siegel ball.   These are {\it  exactly} the
equations of motion obtained in  \cite{cezar}  in the case of the
Siegel-Jacobi disk ${{\mathcal{{D}}}}^J_1$ using the Wei-Norman method and {\it we
want in the present paper to draw attention to    the fact that
apparently such different methods lead to  the same result}. 

The paper is laid out as follows. \S\ref{sec1} recalls  the
definition of the Jacobi algebra ${{\mathfrak{{g}}}}^J_1$ adopted  in \cite{jac1}. In \S\ref{repr} the unitary
operators associated with the Jacobi group $G^J_1$ are recalled.  To a 
linear operator  $A$
it is associated the operator  $\hat{{{\mbox{\boldmath{${A}$}}}}}(\xi):=T^{-1}(\xi)
{{\mbox{\boldmath{${A}$}}}}T(\xi)$ \cite{sbcg,sb12,sb13}, where $T(\xi)=D(\alpha)S(w)$, $D(\alpha)$ is the
unitary displacement operator associated to the Heisenberg group
$H_1$,  $S(w)$ is the positive discrete series representation
associated to the group $\text{SU}(1,1)$, and
$\xi=(\alpha,w)\in{\ensuremath{\mathbb{C}}}\times{{\mathcal{{D}}}}_1$.  We take the expressions of 
$\hat{{{\mbox{\boldmath{${a}$}}}}}(\alpha,w) $,  $ \hat{{{\mbox{\boldmath{${K}$}}}}}_{0}(\alpha,w)$ and
$\hat{{{\mbox{\boldmath{${K}$}}}}}_{-}(\alpha,w)$ from  our papers
\cite{sbcg,sb12,sb13}. Then we apply the Wei-Norman method for the
Jacobi group $G^J_1$ in complex, calculating $T^{-1}\frac{{\operatorname{d}} T}{ {\operatorname{d}}
t}$. In \S\ref{ecMSJ} we determine  the
equations of motion associated to a Hamiltonian  ${{\mbox{\boldmath{${H}$}}}}_0$   linear in the
generators of the Jacobi group $G^J_1$. Following \cite{how}, we
introduce the quasienergy operator ${{\mbox{\boldmath{${E}$}}}}$ associated to the
Hamiltonian ${{\mbox{\boldmath{${H}$}}}}_0$. In \S\ref{Reac} we change the coordinates from
complex to real. The main results of our paper are contained in
Propositions \ref{main} and \ref{conc}. In brief, {\it the Berezin's quantum and classical  equations  of motion on the Siegel-Jacobi disk
  determined by a  hermitian   Hamiltonian linear in the generators of
  Jacobi group  $G^J_1$, expressed in the $FC$-coordinates,  are the same as the equations obtained via the 
  Wei-Norman method. The Wei-Norman equations on ${{\mathcal{{D}}}}^J_1$  are a
particular case of Berezin's equations of motion on the Siegel-Jacobi ball ${{\mathcal{{D}}}}^J_n$
generated by a hermitian Hamiltonian  linear in the generators of the Jacobi group
$G^J_n$}. In a  short remark in \S \ref{PHP} are compared the phases
which appears in the method of Wei-Norman \cite{cezar}  and  in Berezin's equations
of motion on the Siegel-Jacobi disk \cite{nou}. For self-containment, in an Appendix in \S\ref{app1}  we
briefly recall the Wei-Norman method. In another Appendix  in \S\ref{app2} are
mentioned the main definitions of coherent states \cite{perG}. In
\S\ref{CLSQ} it is recalled our method for obtaining Berezin's equations of
motion.The construction of coherent states on the Siegel-Jacobi disk
${{\mathcal{{D}}}}^J_1$ is summarized in \S \ref{app33} and the equations of
motion on ${{\mathcal{{D}}}}^J_n$ obtained in \cite{FC,nou} are reproduced in
\S\ref{lasst}, in 
order to make the comparison with the results of \cite{cezar}.
\enlargethispage{1cm}

{\bf Notation}. In this paper the Hilbert space ${\ensuremath{{{\mathfrak{{H}}}}}}$ is endowed with
a scalar product $<\cdot,\cdot>$ 
antilinear in the first argument, i.e. $<\lambda x,y>=\bar{\lambda}<x,y>$,
$x,y\in{\ensuremath{{{\mathfrak{{H}}}}}},\lambda\in{\ensuremath{\mathbb{C}}}\setminus 0$. ${\ensuremath{\mathbb{R}}}$, ${\ensuremath{\mathbb{C}}}$  and ${\ensuremath{\mathbb{N}}}$ denotes the field of real,
complex numbers, respectively the ring of the integers.  We denote the imaginary unit
$\sqrt{-1}$ by ${\operatorname{i}}$, and the Real and Imaginary part of a complex
number by $\Re$ and respectively $\Im$, i.e. we have for $z\in{\ensuremath{\mathbb{C}}}$,
$z=\Re z+{\operatorname{i}} \Im z$, and $\bar{z}=\Re z-{\operatorname{i}} \Im z$, but also we use
the notation $cc(z):=\bar{z}$ for  $z\in{\ensuremath{\mathbb{C}}}$ or $cc(A)=A^{\dagger}$ for an
operator $A$.  We denote by $M_n({{\mathbb{{F}}}})$ the set of
$n\times n$ matrices with entries in the field ${{\mathbb{{F}}}}$. If $A\in
M_n({{\mathbb{{F}}}})$, then 
$A^t$ ($A^{\dagger}$) denotes the transpose (respectively, the
hermitian conjugate) of $A$. $I$  denotes the unit operator,  while
${{{\mathbb{I}}_n}}$ denotes the unit matrix  of $M_n({{\mathbb{{F}}}})$.  If $A\in
M_n({{\mathbb{{F}}}})$, we denote by $A^s:=\frac{1}{2}(A+A^t)$. If $A$ is a
matrix, then ${\ensuremath{{\mbox{\rm{Tr}}}}}(A)$ denotes the  trace of the matrix $A$.
We use Einstein convention that repeated indices are implicitly summed.
We
denote the differential by ${\operatorname{d}} $. If $\pi$ is an unitary irreducible
representation of a Lie group $G$ with Lie algebra ${{\mathfrak{{g}}}}$ on a
complex separable Hilbert space ${\ensuremath{{{\mathfrak{{H}}}}}}$, then we denote for the derived
representation ${{\mbox{\boldmath{${X}$}}}}:={\operatorname{d}} \pi(X)$, $X\in{{\mathfrak{{g}}}}$. 
\section{The Lie algebra ${{\mathfrak{{g}}}}^J_1$ of the Jacobi group}\label{sec1}

The Heisenberg   group  is the group with the
3-dimensional real  Lie algebra 
 \begin{equation}\label{nr0}{{\mathfrak{{h}}}}_1\equiv
<{\operatorname{i}} s I+\alpha a^{\dagger}-\bar{\alpha}a>_{s\in{\ensuremath{\mathbb{R}}} ,\alpha\in{\ensuremath{\mathbb{C}}}} ,\end{equation}
 where   the boson creation  (respectively, annihilation)
operators $a^{\dagger}$ ($a$)  
 verify the canonical commutation relation 
(\ref{baza1}).

We  consider the Lie algebra of the group $\text{SU}(1,1)$:
\begin{equation}\label{nr1}
{{\mathfrak{{su}}}}(1,1)=
<2{\operatorname{i}}\theta K_0+yK_+-\bar{y}K_->_{\theta\in{\ensuremath{\mathbb{R}}} ,y\in{\ensuremath{\mathbb{C}}}} , \end{equation} 
where the generators $K_{0,+,-}$ verify the standard commutation relations
(\ref{baza2}).

The Jacobi algebra is defined as the  the semi-direct sum \cite{jac1}
\begin{equation}\label{baza}
{{\mathfrak{{g}}}}^J_1:= {{\mathfrak{{h}}}}_1\rtimes {{\mathfrak{{su}}}}(1,1),
\end{equation}
where ${{\mathfrak{{h}}}}_1$ is an  ideal in ${{\mathfrak{{g}}}}^J_1$,
determined by the commutation relations \eqref{baza3}, \eqref{baza5}:
\begin{subequations}\label{baza11}
\begin{eqnarray}
& & [a,{{a}^\dagger}]=I\label{baza1}, \\
\label{baza2}
~& & \left[ K_0, K_{\pm}\right]=\pm K_{\pm}~,~ 
\left[ K_-,K_+ \right]=2K_0 , \\
\label{baza3}
& & \left[a,K_+\right]=a^{\dagger}~,~\left[ K_-,a^{\dagger}\right]=a, ~
\left[ K_+,a^{\dagger}\right]=\left[ K_-, a\right]= 0 ,\\
\label{baza5}
& & \left[ K_0  ,~a^{\dagger}\right]=\frac{1}{2}a^{\dagger}, \left[ K_0,a\right]
=-\frac{1}{2}a .
\end{eqnarray}
\end{subequations}
In the conventions of \cite{sbcg},  see equation (3), we have:
\begin{equation}\label{conva}
a=\frac{1}{2\sqrt{\mu}}(P-{\operatorname{i}} Q);~ a^{\dagger}=-\frac{1}{2\sqrt{\mu}}(P+{\operatorname{i}} Q),  ~[P,Q]=2R.
\end{equation}
which are different of the conventions used in equations (4.14)-(4.16)
in \cite{sbcg}. 
In the convention of  \cite{sbcg}, equation (8), ${{\mbox{\boldmath{${P}$}}}}=\frac{\operatorname{d}}{{\operatorname{d}} x}$, ${{\mbox{\boldmath{${Q}$}}}}=2{\operatorname{i}} \mu x$, corresponding to the derived representation of the Heisenberg group, ${{\mbox{\boldmath{${R}$}}}}={\operatorname{i}} \mu {{\mbox{\boldmath{${I}$}}}}$,  $m\in{\ensuremath{\mathbb{R}}}$.  The differential realization of \eqref{conva} corresponds to
\begin{equation}\label{difreal}
{{\mbox{\boldmath{${a}$}}}}=\frac{1}{2\sqrt{\mu}}\frac{\operatorname{d}}{{\operatorname{d}} x} + \sqrt{\mu}x:~{{\mbox{\boldmath{${a}$}}}}^{\dagger}=-\frac{1}{2\sqrt{\mu}}\frac{\operatorname{d}}{{\operatorname{d}} x}+ \sqrt{\mu}x.
\end{equation}

\section{Unitary representations associated to the Jacobi group $G^J_1$}\label{repr}
The unitary displacement operator
\begin{subequations}\label{dalpha}
\begin{align}
D(\alpha ) & :=\exp (\alpha {{{\mbox{\boldmath{${a}$}}}}}^{\dagger}-\bar{\alpha}{{{\mbox{\boldmath{${a}$}}}}})\label{da1}\\
~~~& =\exp(-\frac{1}{2}|\alpha
|^2) \exp (\alpha {{{\mbox{\boldmath{${a}$}}}}}^{\dagger})\exp(-\bar{\alpha}{{{\mbox{\boldmath{${a}$}}}}})\label{da2}\\  
~~~& =\exp(\frac{1}{2}|\alpha|^2)  
\exp (-\bar{\alpha} {{\mbox{\boldmath{${a}$}}}})\exp(\alpha{{\mbox{\boldmath{${a}$}}}}^{\dagger})\label{da3}
\end{align}
\end{subequations}
has the composition property
\begin{equation}\label{thetah}
D(\alpha_2)D(\alpha_1)=e^{{\operatorname{i}}\theta(\alpha_2,\alpha_1)}
D(\alpha_2+\alpha_1) , 
~\theta(\alpha_2,\alpha_1):=\Im (\alpha_2\bar{\alpha_1}) .
\end{equation}
Note also that
$$D(\alpha)^{\dagger}=D(\alpha)^{-1}= D(-\alpha).$$
Let us denote   by $S$ the unitary irreducible positive discrete
series representation  $D^k_+$  of the group
$\text{SU}(1,1)$ with Casimir operator $C=K^2_0-K^2_1-K^2_2=k(k-1)$,
where $k$ is the Bargmann index for $D^+_k$ \cite{bar47}. 
We introduce the notation $\underline{S}(z)=S(w)$, where 
 $w\in{\ensuremath{\mathbb{C}}},~
|w|<1$ and  $z\in{\ensuremath{\mathbb{C}}}\setminus 0$, 
are related by  \eqref{u5}. We have the relations:
\begin{subequations}
\begin{eqnarray}
\underline{S}(z) & := &\exp (z{{{\mbox{\boldmath{${K}$}}}}}_+-\bar{z}{{{\mbox{\boldmath{${K}$}}}}}_-) 
;\label{u1} \\
S(w) & = &  \exp (w{{{\mbox{\boldmath{${K}$}}}}}_+)\exp (\rho
{{{\mbox{\boldmath{${K}$}}}}}_0)\exp(-\bar{w}{{{\mbox{\boldmath{${K}$}}}}}_-)\label{u2}\\
~~~ & =& \exp (-\bar{w}{{{\mbox{\boldmath{${K}$}}}}}_-)\exp (-\rho
{{{\mbox{\boldmath{${K}$}}}}}_0)\exp(w{{{\mbox{\boldmath{${K}$}}}}}_+); \label{u6} \\
w & = &  \frac{z}{|z|}\tanh \,(|z|), ~~\rho =\ln (1-w\bar{w}), z\not=
0,   \label{u5}
\end{eqnarray}
\end{subequations}
and  $w=0$ for $z=0$ in \eqref{u5}.  
Also, it is easy to observe that
$$S(w)^{\dagger}=S(w)^{-1}= S(-w).$$
We  introduce the unitary operator $T(\xi)$:
\begin{equation}\label{txi}
T(\xi)=D(\alpha)S(w), \quad{{\mathcal{{D}}}}^J_1\ni\xi=(\alpha,w)\in{\ensuremath{\mathbb{C}}}\times{{\mathcal{{D}}}}_1.
\end{equation}
Following \cite{sbcg,sb12,sb13}, for any linear operator ${{\mbox{\boldmath{${A}$}}}}$, 
 we  define the operator
\begin{equation}\label{OpAhat}
\hat{{{\mbox{\boldmath{${A}$}}}}}(\xi):=T^{-1}(\xi) {{\mbox{\boldmath{${A}$}}}}T(\xi), \quad{{\mathcal{{D}}}}^J_1 \ni\xi=(\alpha,w)\in{\ensuremath{\mathbb{C}}}\times{{\mathcal{{D}}}}_1.
\end{equation}
where $T(\xi)$ was defined in \eqref{txi}. 

In  \cite{sb12,sb13}  we have proved that:
\begin{align}\label{kl1}
\hat{{{\mbox{\boldmath{${a}$}}}}}(\alpha,w) & =r (  {{\mbox{\boldmath{${a}$}}}}+w{{\mbox{\boldmath{${a}$}}}}^{\dagger})
+\alpha , \quad r=(1-w\bar{w})^{-\frac{1}{2}}, \\
 \hat{{{\mbox{\boldmath{${K}$}}}}}_{0}(\alpha,w)\! & = r^2[ 
 \bar{w}{{\mbox{\boldmath{${K}$}}}}_{-}\!+\left(  1+|w|^2\right)  {{\mbox{\boldmath{${K}$}}}}_{0}+w{{\mbox{\boldmath{${K}$}}}}_{+}] + \\
 \nonumber & + r \Re[\alpha(
  {{\mbox{\boldmath{${a}$}}}}^{\dagger}+\bar{w}{{\mbox{\boldmath{${a}$}}}})] +\frac{1}{2}|\alpha|^2,\\ 
\label{kl2}
\hat{{{\mbox{\boldmath{${K}$}}}}}_{-}(\alpha,w) & =  r^2[  {{\mbox{\boldmath{${K}$}}}}_{-}+2w{{\mbox{\boldmath{${K}$}}}}_{0}+w^{2}{{\mbox{\boldmath{${K}$}}}}_{+}] + \alpha r ({{\mbox{\boldmath{${a}$}}}}+w{{\mbox{\boldmath{${a}$}}}}^{\dagger}) +\frac{1}{2}\alpha^2.
\end{align}

With formulae \eqref{da2} and \eqref{u2}, we can express \eqref{txi}
as product of exponentials of the generators
as in \eqref{tsiL}, 
where for the Jacobi group $G^J_1$, $n=6$, and the generators
are numbered as 
\begin{equation}\label{x1xn}
{{\mbox{\boldmath{${X}$}}}}_1={{\mbox{\boldmath{${I}$}}}}; ~{{\mbox{\boldmath{${X}$}}}}_2={{\mbox{\boldmath{${a}$}}}}^{\dagger};~{{\mbox{\boldmath{${X}$}}}}_3={{\mbox{\boldmath{${a}$}}}};~ {{\mbox{\boldmath{${X}$}}}}_4={{\mbox{\boldmath{${K}$}}}}_+;~
{{\mbox{\boldmath{${X}$}}}}_5={{\mbox{\boldmath{${K}$}}}}_0;~{{\mbox{\boldmath{${X}$}}}}_6={{\mbox{\boldmath{${K}$}}}}_-,
\end{equation}
while the parameters $\xi_i$  in \eqref{tsiL} are respectively
\begin{equation}\label{xi1n}
\xi_1=-\frac{1}{2}|z|^2;  ~\xi_2=  z;~\xi_3= -\bar{z};  ~\xi_4= w;  ~\xi_5=\ln(1-w\bar{w}) ;  ~\xi_6= -\bar{w} .
\end{equation}
Note that the operator  $T(\xi), ~ \xi=(z,w)\in{{\mathcal{{D}}}}^j_1$ \eqref{txi}, written as the product \eqref{tsiL} with the generators \eqref{x1xn} and the parameters  \eqref{xi1n},
has the properties expressed in \eqref{tximin}. 

Now we apply \eqref{timndot} to the operator  $T(\xi)$ \eqref{txi}  in the
variables $\xi=(z,w)\in{{\mathcal{{D}}}}^J_1$ expressed with
\eqref{x1xn}, \eqref{xi1n},  taking into account  the commutation
relations \eqref{baza11} of the Lie algebra ${{\mathfrak{{g}}}}^J_1$.

For
$${{\mbox{\boldmath{${Y}$}}}}_2={{\mbox{\rm e}}}^{-\xi_6{{\mbox{\rm ad}}}{{\mbox{\boldmath{${X}$}}}}_6}e^{-\xi_5{{\mbox{\rm ad}}}{{\mbox{\boldmath{${X}$}}}}_5}e^{-\xi_4{{\mbox{\rm ad}}}{{\mbox{\boldmath{${X}$}}}}_4}e^{-\xi_3{{\mbox{\rm ad}}}{{\mbox{\boldmath{${X}$}}}}_3}{{\mbox{\boldmath{${X}$}}}}_2,$$
we get successively:
\begin{equation}
\begin{split}\label{fY2}
I_1 & = {{\mbox{\rm e}}}^{-\xi_3{{\mbox{\boldmath{${X}$}}}}_3}{{\mbox{\boldmath{${X}$}}}}_2={{\mbox{\boldmath{${a}$}}}}^{\dagger}-\xi_3;\\
I_2 & = {{\mbox{\rm e}}}^{-\xi_4{{\mbox{\boldmath{${X}$}}}}_4}I_1= I_1; \\
I_3 & = {{\mbox{\rm e}}}^{-\xi_5{{\mbox{\boldmath{${X}$}}}}_5}I_2=
-\xi_3+{{\mbox{\rm e}}}^{-\frac{\xi_5}{2}}{{\mbox{\boldmath{${a}$}}}}^{\dagger};\\
I_4& = {{\mbox{\rm e}}}^{-\xi_6{{\mbox{\boldmath{${X}$}}}}_6}I_3= -\xi_3+{{\mbox{\rm e}}}^{-\frac{\xi_5}{2}}({{\mbox{\boldmath{${a}$}}}}^{\dagger}-\xi_6{{\mbox{\boldmath{${a}$}}}}).
\end{split}
\end{equation}
For $${{\mbox{\boldmath{${Y}$}}}}_3={{\mbox{\rm e}}}^{-\xi_6{{\mbox{\rm ad}}}{{\mbox{\boldmath{${X}$}}}}_6}e^{-\xi_5{{\mbox{\rm ad}}}{{\mbox{\boldmath{${X}$}}}}_5}e^{-\xi_4{{\mbox{\rm ad}}}{{\mbox{\boldmath{${X}$}}}}_4}{{\mbox{\boldmath{${X}$}}}}_3,$$
we get successively:
\begin{equation}
\begin{split}\label{fY3}
e^{-\xi_4{{\mbox{\rm ad}}}{{\mbox{\boldmath{${X}$}}}}_4}{{\mbox{\boldmath{${X}$}}}}_3 & = {{\mbox{\boldmath{${a}$}}}}+\xi_4{{\mbox{\boldmath{${a}$}}}}^
{\dagger};\\
~~e^{-\xi_5{{\mbox{\rm ad}}}{{\mbox{\boldmath{${X}$}}}}_5}{{\mbox{\boldmath{${a}$}}}} & ={{\mbox{\rm e}}}^{\frac{\xi_5}{2}} {{\mbox{\boldmath{${a}$}}}}; \\
~e^{-\xi_5{{\mbox{\rm ad}}}{{\mbox{\boldmath{${X}$}}}}_5}{{\mbox{\boldmath{${a}$}}}}^{\dagger} & = {{\mbox{\rm e}}}^{-\frac{\xi_5}{2}} {{\mbox{\boldmath{${a}$}}}}^{\dagger}.
\end{split}
\end{equation}
For $${{\mbox{\boldmath{${Y}$}}}}_4={{\mbox{\rm e}}}^{-\xi_6{{\mbox{\rm ad}}}{{\mbox{\boldmath{${X}$}}}}_6}e^{-\xi_5{{\mbox{\rm ad}}}{{\mbox{\boldmath{${X}$}}}}_5}{{\mbox{\boldmath{${X}$}}}}_4,$$
we get successively:
\begin{equation}
\begin{split}\label{fY4}
J_1 &= e^{-\xi_5{{\mbox{\rm ad}}}{{\mbox{\boldmath{${X}$}}}}_5}{{\mbox{\boldmath{${X}$}}}}_4 = {{\mbox{\rm e}}}^{-\xi_5}
{{\mbox{\boldmath{${K}$}}}}_+;\\
J_2 & = {{\mbox{\rm e}}}^{-\xi_6{{\mbox{\rm ad}}}{{\mbox{\boldmath{${X}$}}}}_6}J_1= {{\mbox{\rm e}}}^{-\xi_5}(
{{\mbox{\boldmath{${K}$}}}}_+-2\xi_6{{\mbox{\boldmath{${K}$}}}}_0+\xi_6^2{{\mbox{\boldmath{${K}$}}}}_-).
\\
\end{split}
\end{equation}
We also have the relations:
\begin{equation}\label{fY5}
{{\mbox{\boldmath{${Y}$}}}}_5={{\mbox{\rm e}}}^{-\xi_6{{\mbox{\rm ad}}}{{\mbox{\boldmath{${X}$}}}}_6}{{\mbox{\boldmath{${X}$}}}}_5= {{\mbox{\boldmath{${K}$}}}}_0-\xi_6{{\mbox{\boldmath{${K}$}}}}_-.
\end{equation}
Summarizing \eqref{fY2} - \eqref{fY5}, we  obtained the following  expressions
of ${{\mbox{\boldmath{${Y}$}}}}_1 -\ mb{Y}_6$: 
\begin{equation}\label{THEY}
\begin{split}
{{\mbox{\boldmath{${Y}$}}}}_1& ={{\mbox{\boldmath{${I}$}}}};~{{\mbox{\boldmath{${Y}$}}}}_2= -\xi_3+{{\mbox{\rm e}}}^{-\frac{\xi_5}{2}}({{\mbox{\boldmath{${a}$}}}}^{\dagger}-\xi_6{{\mbox{\boldmath{${a}$}}}});~ {{\mbox{\boldmath{${Y}$}}}}_3=
({{\mbox{\rm e}}}^{\frac{\xi_5}{2}}-\xi_4\xi_6 {{\mbox{\rm e}}}^{-\frac{\xi_5}{2}}){{\mbox{\boldmath{${a}$}}}}+\xi_4 {{\mbox{\rm e}}}^{-\frac{\xi_5}{2}}{{\mbox{\boldmath{${a}$}}}}^{\dagger}; \\
&  {{\mbox{\boldmath{${Y}$}}}}_4= {{\mbox{\rm e}}}^{-\xi_5}({{\mbox{\boldmath{${K}$}}}}_++2\bar{\xi}_4{{\mbox{\boldmath{${K}$}}}}_0+\bar{\xi}^2_4{{\mbox{\boldmath{${K}$}}}}_-); ~ {{\mbox{\boldmath{${Y}$}}}}_5= {{\mbox{\boldmath{${K}$}}}}_0+\bar{\xi}_4{{\mbox{\boldmath{${K}$}}}}_-; ~ {{\mbox{\boldmath{${Y}$}}}}_6={{\mbox{\boldmath{${K}$}}}}_ -.
\end{split}
\end{equation}
Now we introduce the expressions \eqref{THEY} into \eqref{timndot} and we
get  for $T^{-1}\dot{T}$  in the variables $(z,w)\in{{\mathcal{{D}}}}^J_1$ the expression 
\begin{equation}\label{minTT}
T^{-1}\dot{T}={\operatorname{i}} \Im(z\dot{\bar{z}}) +r[(\dot{z}-\dot{\bar{z}}w){{\mbox{\boldmath{${a}$}}}}^{\dagger}-cc ] +[\dot{w}r^2{{\mbox{\boldmath{${K}$}}}}_+-cc]
+2{\operatorname{i}} \Im (\dot{w}\bar{w})r^2{{\mbox{\boldmath{${K}$}}}}_0 .
\end{equation}

\section{Equations of motion on the Siegel-Jacobi disk
${{\mathcal{{D}}}}^J_1$}\label{ecMSJ}
The time-dependent Schr\"odinger equation is expressed as
\begin{equation}\label{SCH}
{{\mbox{\boldmath{${H}$}}}}(t)\psi(t)={\operatorname{i}}\hbar\frac{{\operatorname{d}} \psi(t)}{{\operatorname{d}} t}.
\end{equation}

As in  \cite{cezar},  we  consider the following family of unitary operators
\begin{equation}\label{uoper}
U(\xi,\varphi):=\exp(-{\operatorname{i}}\varphi)T(\xi)
\end{equation}
where $\xi\in{{\mathcal{{D}}}}^J_1$ and $\varphi$ is a real phase. Let
$\tau=\frac{t}{\hbar}$. In accord with \cite{how},  in \cite{cezar}  it 
was introduced  the quasienergy operator ${{\mbox{\boldmath{${E}$}}}}:={\operatorname{i}}\frac{\operatorname{d}}{{\operatorname{d}} \tau}
-{{\mbox{\boldmath{${H}$}}}}$. With \eqref{OpAhat},  we get:
\begin{equation}\label{Ehat}
\hat{{{\mbox{\boldmath{${E}$}}}}}(\xi,\varphi)=\frac{{\operatorname{d}}\varphi }{{\operatorname{d}} \tau}{{\mbox{\boldmath{${I}$}}}}+{\operatorname{i}} T(\xi)^{-1}\dot{T}(\xi)-\hat{{{\mbox{\boldmath{${H}$}}}}}(\xi).
\end{equation}

In the notation of \cite{jac1,FC}, we consider a hermitian Hamiltonian linear in the generators of the Jacobi
group $G^J_1$:
\begin{equation}\label{guru}
{{\mbox{\boldmath{${H}$}}}}_0 = \epsilon_a{{\mbox{\boldmath{${a}$}}}} +\bar{\epsilon}_a{{\mbox{\boldmath{${a}$}}}}^{\dagger}
 +\epsilon_0 {{{\mbox{\boldmath{${K}$}}}}}_0 +\epsilon_+{{{\mbox{\boldmath{${K}$}}}}}_++\epsilon_-{{{\mbox{\boldmath{${K}$}}}}}_-  ,~~
\bar{\epsilon}_+=\epsilon_-, ~ \epsilon_0=\bar{\epsilon_0}.
\end{equation}
With equations \eqref{kl1}-\eqref{kl2}, we calculate
$\hat{{{\mbox{\boldmath{${H}$}}}}}_0(\xi)$, where  $\xi=(\alpha,w)\in{\ensuremath{\mathbb{C}}}\times{{\mathcal{{D}}}}_1$: 
\begin{equation}\label{guruhat}
\hat{{{\mbox{\boldmath{${H}$}}}}}_0 (\alpha,w) =I_0+C_1{{\mbox{\boldmath{${a}$}}}}^{\dagger}+\bar{C}_1{{\mbox{\boldmath{${a}$}}}}+C_0{{\mbox{\boldmath{${K}$}}}}_0+C_+{{\mbox{\boldmath{${K}$}}}}_++ \bar{C}_+{{\mbox{\boldmath{${K}$}}}}_-,
\end{equation}
where the coefficients in  \eqref{guruhat} have the values:
\begin{equation}
\begin{split}\label{ioc}
I_0 & =\epsilon_a\alpha+\bar{\epsilon_a}\bar{\alpha}+\frac{1}{2}(\epsilon_0|\alpha|^2+\epsilon_-\alpha^2+
\epsilon_+\bar{\alpha}^2),\\
\frac{C_1}{r}  & = \bar{\epsilon}_a+\epsilon_{a}w+\frac{\epsilon_0}{2}(\alpha+\bar{\alpha}w)
+\epsilon_-\alpha w+\epsilon_+\bar{\alpha},\\
\frac{C_0}{r^2} & =\epsilon_0(1+|w|^2)+2(\epsilon_-w+\epsilon_+\bar{w}),\\
\frac{C_+}{r^2} & =\epsilon_0w+\epsilon_-w^2+\epsilon_+. \\
\end{split}
\end{equation}
\section{The Jacobi group $G^J_1$ in real coordinates}\label{Reac}

In \cite{cezar} it is used as basis of the Lie algebra ${{\mathfrak{{g}}}}^J_1$ 
the real basis from \cite{bs,sbcg}.

We now list the relations between the operators used in the paper
\cite{cezar}  and the generators \eqref{baza11} of the Lie algebra ${{\mathfrak{{g}}}}^J_1$: 
\begin{equation}\label{n1n2}{{\mbox{\boldmath{${N}$}}}}_1={{\mbox{\boldmath{${a}$}}}}+{{\mbox{\boldmath{${a}$}}}}^{\dagger};\quad{{\mbox{\boldmath{${N}$}}}}_2={\operatorname{i}} ({{\mbox{\boldmath{${a}$}}}}-{{\mbox{\boldmath{${a}$}}}}^{\dagger}),
\end{equation}
with  the inverse
\begin{equation}
{{\mbox{\boldmath{${a}$}}}}=\frac{1}{2}({{\mbox{\boldmath{${N}$}}}}_1-{\operatorname{i}} {{\mbox{\boldmath{${N}$}}}}_2); \quad {{\mbox{\boldmath{${a}$}}}}^{\dagger}=\frac{1}{2}({{\mbox{\boldmath{${N}$}}}}_1+{\operatorname{i}} {{\mbox{\boldmath{${N}$}}}}_2);
\end{equation}
\begin{equation}\label{k1k2}
{{\mbox{\boldmath{${K}$}}}}_1=\frac{1}{2}({{\mbox{\boldmath{${K}$}}}}_++{{\mbox{\boldmath{${K}$}}}}_-); \quad {{\mbox{\boldmath{${K}$}}}}_2=\frac{1}{2{\operatorname{i}}}({{\mbox{\boldmath{${K}$}}}}_+-{{\mbox{\boldmath{${K}$}}}}_-),
\end{equation}
and the inverse
\begin{equation}
{{\mbox{\boldmath{${K}$}}}}_+={{\mbox{\boldmath{${K}$}}}}_1+{\operatorname{i}}{{\mbox{\boldmath{${K}$}}}}_2;\quad {{\mbox{\boldmath{${K}$}}}}_-={{\mbox{\boldmath{${K}$}}}}_1-{\operatorname{i}}{{\mbox{\boldmath{${K}$}}}}_2.
\end{equation}
In \cite{cezar} it was considered the unitary operator
\begin{equation}\label{TCEZ}
T(\xi)=D(x,y)S(u,v),\quad \xi=(u,v,x,y)\in{{\mathfrak{{M}}}},
\end{equation}
where
\begin{equation}\label{DScezar}
D(x,y)=\exp({\operatorname{i}} y{{\mbox{\boldmath{${N}$}}}}_1+{\operatorname{i}} x{{\mbox{\boldmath{${N}$}}}}_2), \quad S(u,v)=\exp({\operatorname{i}} k_1{{\mbox{\boldmath{${K}$}}}}_1 +{\operatorname{i}} k_2{{\mbox{\boldmath{${K}$}}}}_2),
\end{equation}
\begin{equation}\label{KK12}
k_1=\frac{v}{2s}\ln\frac{1+s}{1-s}, \quad k_2=\frac{u}{2s}\ln\frac{1+s}{1-s},\quad s=(u^2+v^2)^{\frac{1}{2}} .
\end{equation}
The correspondence between the real parameters in the representation \eqref{DScezar} and the complex parametrization \eqref{txi} is given by the relations
\begin{equation}\label{corT}
\alpha= x+{\operatorname{i}} y; \quad w=u+{\operatorname{i}} v. 
\end{equation}
In \cite{cezar}  the Hamiltonian \eqref{guru} was written down as:
\begin{equation}\label{hcezar}
{{\mbox{\boldmath{${H}$}}}}_0=2\varepsilon_0{{\mbox{\boldmath{${K}$}}}}_0+2\varepsilon_1{{\mbox{\boldmath{${K}$}}}}_1+ 2\varepsilon_2{{\mbox{\boldmath{${K}$}}}}_2 + 2\nu_1{{\mbox{\boldmath{${N}$}}}}_1  + 2\nu_2{{\mbox{\boldmath{${N}$}}}}_2. 
\end{equation}
 The correspondence of the real and complex coefficients of the
Hamiltonians \eqref{hcezar}  and \eqref{guru} is (see also Proposition \ref{EQLIN1})
\begin{equation}\label{corc}
\epsilon_a=\nu_1+{\operatorname{i}} \nu_2; ~\epsilon_0=2 \varepsilon_0; ~ \epsilon_+=\varepsilon_1-{\operatorname{i}} \varepsilon_2, 
\end{equation}
\begin{equation}\label{corcr}
\nu_1=a; ~\nu_2=b; ~\varepsilon_0=p;~ \varepsilon_1=m;~\varepsilon_2=n. 
\end{equation}

We express  $T^{-1}\dot{T}$ \eqref{minTT}  given in the complex
coordinates $(z,w)\in{{\mathcal{{D}}}}^J_1$ in the real coordinates $(x,y;u,w)$,
where 
$z=x+{\operatorname{i}} y$, $w=u+{\operatorname{i}} v$, and in the real operators \eqref{n1n2}, \eqref{k1k2}, and we get:
\begin{equation}\label{minTT1}
\begin{split}
-{\operatorname{i}} T^{-1}\frac{{\operatorname{d}} T}{{\operatorname{d}} \tau } & =(x\dot{y}-y\dot{x}){{\mbox{\boldmath{${I}$}}}}+r[(1+u)\dot{y}-\dot{x}v]{{\mbox{\boldmath{${N}$}}}}_1+r[(1-u)\dot{x}-\dot{y}v]{{\mbox{\boldmath{${N}$}}}}_2\\
~~~&  +2r^2[(\dot{v}u-\dot{u}v){{\mbox{\boldmath{${K}$}}}}_0+\dot{v}{{\mbox{\boldmath{${K}$}}}}_1+\dot{u}{{\mbox{\boldmath{${K}$}}}}_2].
\end{split}
\end{equation}
Now we express  the operator $\hat{{{\mbox{\boldmath{${H}$}}}}}_0(\alpha,w)$ \eqref{guruhat} in the
operators \eqref{n1n2}, \eqref{k1k2} in real coordinates $(x,y;u,v)$, where
$\alpha=x+{\operatorname{i}} y$, $w=u+{\operatorname{i}} v$, and we get:
\begin{equation}\label{hatR}
\hat{{{\mbox{\boldmath{${H}$}}}}}_0(x,y;u,v)=D_0+D_1{{\mbox{\boldmath{${N}$}}}}_1+ D_2{{\mbox{\boldmath{${N}$}}}}_2+F_0{{\mbox{\boldmath{${K}$}}}}_0+ F_1{{\mbox{\boldmath{${K}$}}}}_1+F_2{{\mbox{\boldmath{${K}$}}}}_2.
\end{equation}
We find the following values of the coefficients appearing in  \eqref{hatR}:
\begin{equation}
\begin{split}\label{h2r}
D_0 &= 2(\nu_1x-\nu_2y)+\frac{\epsilon_0}{2}(x^2+y^2)+\varepsilon_1(x^2-y^2)-2\varepsilon_2 xy,\\
\frac{D_1}{r} &=\nu_1(1+u)-\nu_2v+\frac{\epsilon_0}{2}[x(1+u)+yv]\\~~~&+\varepsilon_1[x(1+u)-yv] -\varepsilon_2[xv+(1+u)y],
\\-\frac{D_2}{r} &= \nu_1v+\nu_2(u-1)+\frac{\epsilon_0}{2}[xv+y(1-u)] \\~~~&\varepsilon_1[xv+y(u-1)]
+\varepsilon_2[x(u-1)-yv],\\
\frac{F_0}{r^2} & =\epsilon_0(1+u^2+v^2)+4(\varepsilon_1u-\varepsilon_2v),\\
\frac{F_1}{2r^2} &
=\epsilon_0u+\varepsilon_1(u^2-v^2+1)-2\varepsilon_2uv,\\
-\frac{F_2}{2r^2} & =\epsilon_0v+2\varepsilon_1uv+\varepsilon_2(-1+u^2-v^2).
\end{split}
\end{equation}
We introduce \eqref{hatR} and \eqref{minTT1} into \eqref{Ehat} and we
get the expression
\begin{equation}\label{Ehat1}
\hat{{{\mbox{\boldmath{${E}$}}}}} (x,y;u,v)= G_0{{\mbox{\boldmath{${I}$}}}}+G_1{{\mbox{\boldmath{${N}$}}}}_1+G_2{{\mbox{\boldmath{${N}$}}}}_2+H_0{{\mbox{\boldmath{${K}$}}}}_0+H_1{{\mbox{\boldmath{${K}$}}}}_1+H_2{{\mbox{\boldmath{${K}$}}}}_2,
\end{equation}
where:
\begin{equation}
\begin{split}\label{G0H}
G_0 & = \dot{\varphi} +y\dot{x}-x\dot{y}-2(\nu_1x-\nu_2y)-\varepsilon_0(x^2+y^2) -\varepsilon_1(x^2-y^2)+2\varepsilon_2xy,\\
-\frac{G_1}{r} &= (1+u)\dot{y}-\dot{x}v+\nu_1(1+u)-\nu_2v+\varepsilon_0[x(1+u)+yv]\\ ~~~&+\varepsilon_1[x(1+u)-yv]
-\epsilon_2[y(1+u)+xv],\\
\frac{G_2}{r} &= -(1-u)\dot{x}+\dot{y}v+\nu_1v+\nu_2(u-1)+\varepsilon_0[y(1-u)+xv] \\ ~~~&+\varepsilon_1[xv+y(u-1)] +\varepsilon_2[x(u-1)-yv],\\
-\frac{H_0}{2r^2} & = \dot{v}u-\dot{u}v+\varepsilon_0(1+u^2+v^2)+2(\varepsilon_1u-\varepsilon_2v),\\
-\frac{H_1}{2r^2} & =  \dot{v}+\varepsilon_0u+\varepsilon_1(u^2-v^2+1)-2\varepsilon_2uv,\\
\frac{H_2}{2r^2} & = -\dot{u}+\varepsilon_0v+2\varepsilon_1uv+\varepsilon_2(u^2-v^2-1).
\end{split}
\end{equation}
Identifying the coefficients of ${{\mbox{\boldmath{${N}$}}}}_1$, ${{\mbox{\boldmath{${N}$}}}}_2$, and
respectively  ${{\mbox{\boldmath{${K}$}}}}_1$, ${{\mbox{\boldmath{${K}$}}}}_2$, we get the equations of motion
in real coordinates
\begin{equation}\label{xyec}
\begin{split}
\dot{x} & = -\varepsilon_2x+(\varepsilon_0-\varepsilon_1)y-\nu_2,\\
\dot{y} & = -(\varepsilon_0+\varepsilon_1)x+\varepsilon_2y-\nu_1; 
\end{split}
\end{equation}
\begin{equation}\label{uvec}
\begin{split}
\dot{u} & = 2v(\varepsilon_1u+\varepsilon_0)-\varepsilon_2(1-u^2+v^2),\\
\dot{v}&  =  2u(\varepsilon_2v-\varepsilon_0)-\varepsilon_1(1+u^2-v^2). 
\end{split}
\end{equation}
Introducing \eqref{uvec} into the equation of $H_0$ in \eqref{G0H}, we have
\begin{equation}\label{H00}
-\frac{H_0}{2}=\varepsilon_0+\varepsilon_1u-\varepsilon_2v.
\end{equation}
Introducing \eqref{xyec} into expression of $G_0$ in  \eqref{G0H}, we have
\begin{equation}\label{GOOH}
G_0=\dot{\varphi}-(\nu_1x-\nu_2y).
\end{equation}
Starting from the complex representation, we have regained  the
results of Section 3 in \cite{cezar} and also our results from
\cite{nou}, reproduced in Proposition \ref{EQLIN1}:
\begin{Proposition}\label{main}
If we consider $\Phi\in{\ensuremath{{{\mathfrak{{H}}}}}}$ such that ${{\mbox{\boldmath{${K}$}}}}_0\Phi=k\Phi$, then
\begin{equation}\label{SCHCEZ}
\Psi(\xi,\varphi)=U(\xi,\varphi)\Phi= {{\mbox{\rm e}}}^{-{\operatorname{i}}\phi}T(\xi)\Phi\end{equation} is a solution of the time
dependent Schr\"odinger equation \eqref{SCH} corresponding to the
hermitian Hamiltonian \eqref{guru} (or \eqref{hcezar}) on the
Siegel-Jacobi disk ${{\mathcal{{D}}}}^J_1$, where $x,y\in{\ensuremath{\mathbb{R}}}$ verify \eqref{xyec},
$(u,v)\in{{\mathcal{{D}}}}_1$ verify \eqref{uvec}, while the phase $\varphi$ in \eqref{uoper}
verifies 
\begin{equation}\label{phicez}
\dot{\varphi}=\nu_1x-\nu_2y+2k(\varepsilon_0+\varepsilon_1u-\varepsilon_2v).
\end{equation}

The motion \eqref{uvec} on the Siegel disc ${{\mathcal{{D}}}}_1$
in the 
  complex variable $w=u+{\operatorname{i}} v$ is described by the Riccati equation 
\begin{equation}\label{wcom}
{\operatorname{i}} \dot{w}= \epsilon_++\epsilon_0 w+\epsilon_-w^2.
\end{equation}
 

The equations of motion \eqref{xyec} in 
 the complex variable $\eta= x+{\operatorname{i}} y$ reads
\begin{equation}\label{etaec}
{\operatorname{i}} \dot{\eta}=\bar{\epsilon}_a+\epsilon_+\bar{\eta}+\frac{\epsilon_0}{2}\eta,
\end{equation}
\end{Proposition}

We remark that 
 the Riccati equation \eqref{wcom} on ${{\mathcal{{D}}}}_1$ obtained with the Wei-Norman method
 coincides with the Berezin's equation of motion {\rm{(4.8b)}}  or
 {\rm{(4.10b)}}   in \cite{FC}, with the
difference of notation $\epsilon_+\leftrightarrow
\epsilon_-=\bar{\epsilon}_+$.  
The equation \eqref{etaec} for $\eta\in{\ensuremath{\mathbb{C}}}$ obtained with the Wei-Norman method is the
Berezin's equation of motion {\rm{(4.10a)}}    in
\cite{FC},  with the
correspondence $\epsilon_a\leftrightarrow
\bar{\epsilon}_a$,  $\epsilon_+\leftrightarrow
\epsilon_-=\bar{\epsilon}_+$, see Propositions \ref{EQLIN1} and  \ref{POYT}.

We have proved 
\begin{Proposition}\label{conc}The quantum and classical    Berezin's equations
  of motion on the Siegel-Jacobi disk
  determined by a  hermitian   Hamiltonian,  linear in the generators of
  Jacobi group  $G^J_1$,  expressed in the $FC$-coordinates,  are the same
  as the equations obtained applying the Wei-Norman method.

The equations of motion \eqref{wcom} on ${{\mathcal{{D}}}}_1$,
  \eqref{etaec} on ${\ensuremath{\mathbb{C}}}$  and \eqref{xyec} on ${\ensuremath{\mathbb{R}}}^2$,  determined by
 the  hermitian  Hamiltonian \eqref{guru} or \eqref{hcezar}  linear in  the generators of the Jacobi group $G^J_1$ obtained with the
  Wei-Norman method are a particular case of the Berezin's equations of motion
  \eqref{Mhip2} on ${{\mathcal{{D}}}}_n$, \eqref{hipPRT1} on ${\ensuremath{\mathbb{C}}}^n$ and
  respectively \eqref{LINe} on ${\ensuremath{\mathbb{R}}}^{2n}$,  determined by the 
  hermitian Hamiltonian \eqref{HACA} linear in the generators of the
  Jacobi group $G^J_n$. 
  

\end{Proposition}

\section{Phases}\label{PHP}
In the paper \cite{cezar} it was calculated the phase $\varphi$ which
appears  in the solution \eqref{uoper} for which we have find the
equation \eqref{phicez}. In the Berezin's approach to the equations of
motion, the solution of the time-dependent Schr\"odinger equation
\eqref{SCH}
differs from the solution parametrized by Perelomov coherent states by
a phase $\phi$ as it is recalled in  Proposition \ref{propvechi}
proven in \cite{sbcag,FC,nou}. 
Comparing the Wei-Norman solution of the Schr\"odinger equation
\eqref{SCH} for the Hamiltonian \eqref{guru}  (or \eqref{hcezar}) with
the solution \eqref{slSCH} under the form \eqref{legPH}, we see that

\begin{Remark}\label{REMPS}
The phases $\varphi$ used in the Wei-Norman method  associated to the
quasi-energy  operator \eqref{Ehat} and the phase $\phi$ which
appears in Berezin's approach are different: 
\begin{equation}\label{IMPDIF}
-\varphi(\xi)=\phi(\xi)-\frac{1}{2}\Im(\omega\bar{\alpha}^2),\quad \xi=(\alpha,w)\in{{\mathcal{{D}}}}^J_1.
\end{equation}
If we introduce the equations of motions  on ${{\mathcal{{D}}}}^J_1$
into the-$\tau$ derivative of \eqref{IMPDIF},  we get
\begin{equation}\label{fiPHI}
-\dot{\phi}=\dot{\varphi}
-\nu_1(ux+vy)+\nu_2(uy-vx)+\frac{1}{4}(\epsilon_-\bar{z}^2+\epsilon_+z^2), \end{equation} 
where $z=\alpha-w\bar{\alpha}$ and $(\alpha,w)$ are the coordinates \eqref{txi} on
${{\mathcal{{D}}}}^J_1$. This is exactly the formula obtained for $\dot{\phi}$
summing up  the explicit
expressions of the dynamical and Berry phases  \eqref{realHH} and \eqref{FFV}.
\end{Remark}
We verify the last part of the Remark. 
If we add the  expression of the Berry phases given by
\eqref{FFV} in which we introduce the equations of motion
\eqref{qqqN1} determined by the Hamiltonian  \eqref{guru} and dynamic
phase \eqref{realHH}, we get  the value of  the
phase $\phi$  in the complex  variables
$(z,w)\in{\ensuremath{\mathbb{C}}}\times{{\mathcal{{D}}}}_1$, $z=\alpha-w\bar{\alpha}$:
\begin{subequations}\label{pp}
\begin{align}-{\phi} & =k{\phi}_1 +{\phi}_0,\\
\dot{\phi}_1 & = \epsilon_0+\epsilon_-\bar{w}+\epsilon_+w,\label{pp1}\\
\dot{\phi}_0 & = \frac{1}{4}(\epsilon_-\bar{z}^2+\epsilon_+z^2)+
\frac{1}{2} (\epsilon_a\bar{z}+\bar{\epsilon}_az) \label{pp0}. 
\end{align}
\end{subequations}
When we express $z$ in the real and imaginary part in the coordinates \eqref{corT}
$(x,y,u,v)$, we have for $z$ appearing in \eqref{pp0} the value
$$z=(1-u)x-yv+{\operatorname{i}} [(1+u)y-vx],$$
while $$\dot{\phi}_1=2(\varepsilon_0+\varepsilon_1u +\varepsilon_2 v),$$
i.e. the expression  multiplying   $k$ in    \eqref{phicez},
with the correspondence $\epsilon_+\leftrightarrow
\epsilon_-=\bar{\epsilon}_+$, $\epsilon_a\leftrightarrow
\bar{\epsilon}_a$.  \\[3ex]

{\bf In conclusion}, in \cite{nou,FC,ber14} we have underlined the utility 
of the $FC$-coordinates,  the geometric  significance of the 
$FC$-transform in the context of the fundamental conjecture and also
its relevance for coherent states on the Siegel-Jacobi ball. In the
present  paper {\it we have shown that Berezin's equations of motion
on the Siegel-Jacobi disk   expressed  in the $FC$-coordinates, 
generated by a linear Hamiltonian in the generators of th Jacobi group 
are identical with the equations of
motion furnished by the Wei-Norman method}. All the calculation in the
present paper refers to the motion on the Siegel-Jacobi disk,  generated
by  a linear Hamiltonian in the generators of the Jacobi group $G^J_1$, but we
believe  that  Proposition  \ref{conc} is  true in more general
situations,  for some Lie groups which are semidirect product. We have also underlined that the phases which appears
in the two  methods are different.  Remark
\ref{REMPS}  is also  a direct  check of the correctness of our calculation
in \cite{FC} and compatibility with the  calculation in \cite{cezar}. 

\section{Appendix: The Wei-Norman method}\label{app1}

We use the following convention of notation for noncommuting operators:
\begin{equation}\label{conv}
\prod_{i=1}^nA_i:=A_1\dots A_n; \quad\prod_{i=n}^1A_i:=A_n\dots A_1.
\end{equation}

Let $\xi=(\xi_1,\dots,\xi_n)$ be  some parameters. We consider an unitary
operator which can be expressed in the basis $\{X_i\}_{i=1,\dots,n}$
of the Lie algebra ${{\mathfrak{{g}}}}$ as in \eqref{tsiL}. 
Then we have
\begin{equation}\label{tximin}
U^{-1}(\xi)=U^{\dagger}(\xi)=U(-\xi)=\prod_{j=n}^1\exp(-\xi_j{{\mbox{\boldmath{${X}$}}}}_j).
\end{equation}
Let $X,Y$ be the free generators of a ring $R$. The Baker-Hausdorff
formula (see  \cite{mag} for a proof)  reads :
\begin{equation}\label{BCH}\begin{split}
e^XYe^{-X}&=e^{{{\mbox{\rm ad}}} X}Y=\sum_{n=0}^{\infty}\frac{({{\mbox{\rm ad}}} X)^n}{n!}Y\\
~~~&=Y+[X,Y]+\frac{1}{2!}[X,[X,Y]]+\dots
+\frac{1}{n!}[\underbrace{[X,[X,\dots,[}_{n\text{~brackets}}X,Y\underbrace{]\dots]}_{n}+\dots  .
\end{split}
\end{equation}
Now let us consider that the parameters $\xi$ depend on a variable,
let call it $t$. For any $t$-dependent operator $A$, we  denote
$\dot{A}=\frac{{\operatorname{d}} A}{{\operatorname{d}} t}$. Then we can calculate the derivative of \eqref{tsiL} as
\begin{equation}\label{dotT}
\dot{U}(\xi)=\sum_{j=1}^n \dot{\xi}_j\left(\prod_{k=1}^{j-1}[\exp(\xi_k{{\mbox{\boldmath{${X}$}}}}_k)]{{\mbox{\boldmath{${X}$}}}}_j\prod_{l=j}^n[\exp(\xi_l{{\mbox{\boldmath{${X}$}}}}_l)]\right).
\end{equation}
Let us consider a linear operator $A(t)$ of the form \eqref{linA}, 
where $a_i(t)$ are scalar functions of $t$.
It can be   proved  \cite{wei} that:
\begin{lemma}If $\{{{\mbox{\boldmath{${X}$}}}}_1,\dots,{{\mbox{\boldmath{${X}$}}}}_n \}$ is  a basis of a Lie algebra
  ${{\mathfrak{{g}}}}$, then
\begin{equation}\label{lemW}
\left(\prod_{j=1}^r\exp(\xi_j{{\mbox{\boldmath{${X}$}}}}_j)\right){{\mbox{\boldmath{${X}$}}}}_i\left(\prod_{j=r}^1\exp(-\xi_j{{\mbox{\boldmath{${X}$}}}}_j)\right)=\sum_{k=1}^n\eta_{ki}{{\mbox{\boldmath{${X}$}}}}_k,~i,
r=1,2,\dots,n,
\end{equation}
where $\eta_{ki}=\eta_{ki}(\xi_1,\dots,\xi_r)$ is an analytic
function of $\xi_1,\dots,\xi_r$. 
\end{lemma}
The Wei-Norman method  \cite{wei} is expressed in the theorem:
\begin{Theorem}
If $A(t)$ is  given by \eqref{linA},  then there exists a neighborhood
of $t=0$ in which the solution of the equation
\begin{equation}\label{dotU}
\dot{U}(t)=A(t)U(t), \quad U(0)=I
\end{equation}
may be expressed in the form \eqref{tsiL}, where the $\xi_i(t)$ are
scalar functions of $t$. Moreover, $\xi^t:=(\xi_1,\dots \xi_n)$ satisfy the first order 
differential equation
\begin{equation}\label{solwei}
\eta\dot{\xi}=\epsilon,
\end{equation}
 which depend only on the Lie algebra ${{\mathfrak{{g}}}}$
and the $\epsilon(t)$'s. $\eta=(\eta_{ki})_{k,i=1,\dots,n}$ is the matrix of coefficients of \eqref{linA}, while $\epsilon$
denotes the vector with coefficients which appear in \eqref{linA},  $ \epsilon^t=(\epsilon_1,\dots,\epsilon_n)$. 
\end{Theorem}
In \cite{wei} it was proved that the representation \eqref{tsiL} is
global for any solvable Lie algebra ${{\mathfrak{{g}}}}$ and for any $2\times 2$
system of equations. 

In our calculation  in \eqref{Ehat}, instead of $\dot{T}T^{-1}$, we need
$T^{-1}\dot{T}$. 
With the convention \eqref{conv} and the Baker-Hausdorff formula \eqref{BCH}, we obtain
for $U$ defined in \eqref{tsiL}:
\begin{equation}\label{timndot}
U^{-1}(\xi)\dot{U}(\xi)=\sum_{i=1}^n\dot{\xi}_i{{\mbox{\boldmath{${Y}$}}}}_i, \quad {{\mbox{\boldmath{${Y}$}}}}_i= \left(\prod_{k=n}^{i+1}[\exp(-\xi_k{{\mbox{\rm ad}}} {{\mbox{\boldmath{${X}$}}}}_k)]\right){{\mbox{\boldmath{${X}$}}}}_i.
\end{equation}

\section{Appendix: Berezin's equations on motion}\label{app2}
We  consider the triplet $(G, \pi , {{\mathfrak{{H}}}} )$, where $\pi$ is
 a continuous, unitary, irreducible 
representation
 of the  Lie group $G$
 on the   separable  complex  Hilbert space {\ensuremath{{{\mathfrak{{H}}}}}}~   \cite{perG}. 

We
 introduce the normalized (unnormalized) vectors  $\underline{e}_x$
(respectively, $e_z$) defined on $G/H$
\begin{equation}\label{bigch}
\underline{e}_x=\exp(\sum_{\phi\in\Delta_+}x_{\phi}{{\mbox{\boldmath{${X}$}}}}_{\phi}^+-\bar{x}_{\phi}{{\mbox{\boldmath{${X}$}}}}_{\phi}^-)
e_0, ~
e_z=\exp(\sum_{\phi\in\Delta_+}z_{\phi}{{\mbox{\boldmath{${X}$}}}}_{\phi}^+)e_0, 
\end{equation}
where
$e_0$ is the extremal weight vector of the representation $\pi$, $\Delta_+$ are the positive roots of the Lie algebra ${{\mathfrak{{g}}}}$ of $G$,
and $X_\phi,\phi\in\Delta$,  are the  generators. 
${{\mbox{\boldmath{${X}$}}}}^+_{\phi}$ (${{\mbox{\boldmath{${X}$}}}}^-_{\phi}$)
corresponds to the positive (respectively, negative) generators. See
details in  \cite{perG,sb6}.

We denote by $FC$ the change of variables
$x\rightarrow z$ in formula \eqref{bigch} such that
\begin{equation}\label{etild}
\underline{e}_{x}=\tilde{e}_z, ~    \tilde{e}_z  :=
  (e_z,e_z)^{-\frac{1}{2}}e_z, ~z=FC(x). 
\end{equation}
The reason for calling the transform \eqref{etild}  a $FC$-transform  {\it
  (fundamental conjecture)}  \cite{GV,DN} is explained in Proposition
3 in \cite{ber14}. For a concrete example of $FC$-transform, see \eqref{csv}.

 \subsection{Berezin's approach to classical motion and quantum evolution}\label{CLSQ}

Let $M=G/H$ be a homogeneous  manifold with a $G$-invariant K\"ahler two-form
$\omega$
\begin{equation}\label{kall}
\omega(z)={\operatorname{i}}\sum_{\alpha\in\Delta_+} g_{\alpha,\beta}  d z_{\alpha}\wedge
d\bar{z}_{\beta}, ~g_{\alpha,\beta}=\frac{{\partial}^2}{{\partial}
  z_{\alpha} {\partial}\bar{z}_{\beta}} \ln  <e_z,e_z>. 
\end{equation}

Passing on from the dynamical system problem
 in the Hilbert space ${\ensuremath{{{\mathfrak{{H}}}}}}$ to the corresponding one on $M$ is called
sometimes {\it dequantization}, and the dynamical system on $M$ is a classical
one \cite{sbcag,sbl}. Following Berezin \cite{berezin2,berezin1}, the
motion on the classical phase space can be described by the local
equations of motion
$\dot{z}_{\alpha}={\operatorname{i}} \left\{{{\mathcal{{H}}}},z_{\alpha}\right\},
  ~\alpha \in \Delta_+ $, where ${{\mathcal{{H}}}}$ is
  the classical Hamiltonian  ({\it the covariant 
  symbol})
\begin{equation}\label{clH}{{\mathcal{{H}}}}=<e_z,e_z>^{-1}<e_z|{{\mbox{\boldmath{${H}$}}}}|{{\mbox{\rm e}}}_z>\end{equation}
  attached to
  the quantum Hamiltonian ${{\mbox{\boldmath{${H}$}}}}$, and the Poisson bracket is
  introduced using the matrix $g^{-1}$. 

We consider an algebraic Hamiltonian linear in the generators
${{{\mbox{\boldmath{${X}$}}}}}_{\lambda} $ of the
group of symmetry $G$
\begin{equation}\label{lllu}
{{\mbox{\boldmath{${H}$}}}}=\sum_{\lambda\in\Delta}\epsilon_{\lambda}{{{\mbox{\boldmath{${X}$}}}}}_{\lambda} .
\end{equation}
We look for the solution of the Schr\"odinger equation of motion
\eqref{SCH} generated by the Hamiltonian \eqref{lllu} 
as \begin{equation}\label{slSCH}
\psi(t)={{\mbox{\rm e}}}^{{\operatorname{i}} \phi}\tilde{e}_z,
\end{equation}
where $\tilde{e}_z$ is the normalized Perelomov coherent state vector
defined in \eqref{bigch}, \eqref{etild}.

We extract from  \cite{sbcag,sbl,nou}  the following Proposition: 
\begin{Proposition}\label{propvechi}
The  classical motion and quantum evolution  generated
by the linear hermitian  Hamiltonian  \eqref{lllu}  are described
by Berezin's   equations
of motion  on $M=G/H$
\begin{equation}\label{moveM}
{{\operatorname{i}}\dot{z}_{\alpha}=\sum_{\lambda\in\Delta}\epsilon_{\lambda}Q_{\lambda
,\alpha}},~\alpha\in\Delta_+ , 
\end{equation}
where the differential action corresponding to the operator
${{\mbox{\boldmath{${X}$}}}}_{\lambda}$ in \eqref{lllu} can be expressed in a local
system of coordinates as a holomorphic  first order differential
operator with polynomial coefficients  (${\partial}_{\beta}=\frac{\partial}{{\partial}
  z_{\beta}}$),
\begin{equation}\label{VBC}{{\mathbb{{X}}}}_{\lambda}=P_{\lambda}+\sum_{\beta\in\Delta_+}Q_{\lambda,
    \beta}\partial_{\beta}, \lambda\in\Delta.
\end{equation}
The phase $\phi$ in \eqref{slSCH} is given by the sum
$\phi=\phi_D+\phi_B$
of the dynamical and Berry phases, where
\begin{subequations}
\begin{align}
\dot{\phi}_D & = -{{\mathcal{{H}}}}(t); \\
\dot{\phi}_B & = \frac{\operatorname{i}}{2}\sum_{\alpha\in\Delta_+}(\dot{z}_{\alpha}{\partial}_{\alpha}-
 \dot{\bar{z}}_{\alpha}\bar{\partial}_{\alpha})\ln <e_z,e_z>. 
\end{align}
\end{subequations}
\end{Proposition}

\subsection{Coherent states on the  Siegel-Jacobi disk ${{\mathcal{{D}}}}^J_1$}\label{app33}

 We impose to the cyclic vector $e_0$ to verify simultaneously
 the conditions \cite{jac1}
\begin{equation}\label{cond}
{{\mbox{\boldmath{${a}$}}}}e_0  =  0, ~
 {{{\mbox{\boldmath{${K}$}}}}}_-e_0  =  0,~
{{{\mbox{\boldmath{${K}$}}}}}_0e_0  =  k e_0;~ k>0, 2k=2,3,... , 
\end{equation}
and we have considered in  the last relation in (\ref{cond}) the positive  discrete series
representations $D^+_k$ of $\text{SU}(1,1)$  \cite{bar47}.

 Perelomov's coherent state   vectors   associated to the group $G^J_{1}$ with 
Lie algebra the Jacobi algebra ${{\mathfrak{{g}}}}^J_{1}$, based on Siegel-Jacobi
disk $ \mathfrak{D}^J_{1}  =  H_1/{\ensuremath{\mathbb{R}}}\times \text{SU}(1,1)/\text{U}(1)$
$ ={\ensuremath{\mathbb{C}}}\times{{\mathcal{{D}}}}_1$,  
are defined as 
\begin{equation}\label{csu}
e_{z,w}:={{\mbox{\rm e}}}^{z{{\mbox{\boldmath{${a}$}}}}^{\dagger}+w{{{\mbox{\boldmath{${K}$}}}}}_+}e_0, ~z,w\in{\ensuremath{\mathbb{C}}},~ |w|<1 .
\end{equation}
We introduce also the normalized    ({\it squeezed})     CS-vector
(see also \cite{stol}) 
\begin{equation}\label{sqz}
\underline{e}_{\xi}:= T(\xi)  e_0=D(\alpha )S(w) e_0, \quad \xi=(\alpha, w)\in{\ensuremath{\mathbb{C}}}\times{{\mathcal{{D}}}}_1.\end{equation}
The normalized squeezed state vector   and the
un-normalized 
 Perelomov's coherent state vector
are related by the relation (see \cite{jac1})
\begin{equation}\label{csv}
\underline{e}_{\eta, w} = (1-w\bar{w})^k
\exp (-\frac{\bar{\eta}}{2}z)e_{z,w},~ z=\eta-w\bar{\eta}.
\end{equation}
We recall \cite{jac1, nou}  that 
\begin{subequations}\label{KKK}
\begin{align}
<e_{z,w},e_{z,w}> & = (1-w\bar{w})^{-2k}\exp({{\mathcal{{F}}}}),\\
2{{\mathcal{{F}}}} &= \frac{2z{\bar{z}}+z^2\bar{w}+\bar{z}^2w}{1-w\bar{w}}
=2|\eta|^2-\bar{w}\eta^2-w\bar{\eta}^2.
\end{align}
\end{subequations}

From \eqref{csv} and \eqref{KKK}, we get for \eqref{etild} on
${{\mathcal{{D}}}}^J_1$ the relation
\begin{equation}\label{legPH}
\underline{e}_{\eta,w}=\exp[\frac{1}{4}(w\overline{\eta}^2-cc)]\tilde{e}_{z,w},\quad z=FC(\eta)=\eta-w\bar{\eta}.
\end{equation}

The general scheme \cite{sbcag,sbl} associates to elements of the
Lie algebra ${{\mathfrak{{g}}}}$  first order holomorphic differential operators
with polynomial coefficients $X\in{{\mathfrak{{g}}}}\rightarrow{{\mathbb{{X}}}}$ as in \eqref{VBC}. The
calculation in \cite{jac1},  based on 
\eqref{BCH}, gives:  
\begin{lemma}\label{mixt}The differential action of the generators
 of
the Jacobi algebra {\em (\ref{baza})} is given by the formulas:
\begin{subequations}\label{summa}
\begin{eqnarray}
& & {{\mbox{\boldmath{${a}$}}}}=\frac{\partial}{{\partial} z};~{{\mbox{\boldmath{${a}$}}}}^{\dagger}=z+w\frac{\partial}{{\partial} z} ,
~z,w\in{\ensuremath{\mathbb{C}}}, ~|w|<1; \\
 & & {{\mathbb{{K}}}}_-=\frac{\partial}{{\partial} w};~{{\mathbb{{K}}}}_0=k+\frac{1}{2}z\frac{\partial}{{\partial} z}+
w\frac{\partial}{{\partial} w};\\
& & {{\mathbb{{K}}}}_+=\frac{1}{2}z^2+2kw +zw\frac{\partial}{{\partial} z}+w^2\frac{\partial}{{\partial}
w} . 
\end{eqnarray}
\end{subequations}
\end{lemma}
Applying Proposition \ref{propvechi} to the representation given by
Lemma \ref{mixt}, we have obtained  in \cite{FC} Berezin's equations
of motion: 
\begin{Proposition}\label{EQLIN1}
The equations of motion on the Siegel-Jacobi disk $\mathcal{D}^J_1$
 generated by
the linear Hamiltonian {\em(\ref{guru})}
 are:
\begin{subequations}\label{qqqN}
\begin{eqnarray}
{\operatorname{i}}\dot{z} & = & \epsilon_a+{\bar{\epsilon}}_a w+(\frac{\epsilon_0}{2}
+\epsilon_+  w )z,~z,w\in{\ensuremath{\mathbb{C}}}, ~|w|<1, \label{guru2}\\
{\operatorname{i}}\dot{w} & = & \epsilon_- + \epsilon_0w+
\epsilon_+w^2 .\label{guru1}
\end{eqnarray}
\end{subequations}
 For the $\eta$ defined in the $FC^{-1}$ transform \emph{(\ref{csv})}, the
 system of first order differential  equations \emph{(\ref{qqqN})} becomes  the
system of separate equations 
\begin{subequations}\label{qqqN1}
\begin{eqnarray}
i\dot{\eta} & = & \epsilon_a
+\epsilon_-\bar{\eta}+\frac{\epsilon_0}{2}\eta,~\eta \in{\ensuremath{\mathbb{C}}},\label{guru22}\\
i\dot{w} & = & \epsilon_- + \epsilon_0w+
\epsilon_+w^2,~ w\in{\ensuremath{\mathbb{C}}},|w|<1, .\label{guru12}
\end{eqnarray}
\end{subequations}
With the change of function $w=XY^{-1}$, the Riccati equation
\eqref{guru1} became the linear hamiltonian system
\begin{equation}\label{GURRU11}
\left(\begin{array}{c}\dot{X} \\ \dot{Y}\end{array}\right)=h_c
\left(\begin{array}{c}X \\ Y\end{array}\right),\quad h_c={\operatorname{i}} 
\left(\begin{array}{cc} -\frac{\epsilon_0}{2} & -\epsilon_-\\
    \epsilon_+  & 
    \frac{\epsilon_0}{2} \end{array}\right) \in {{\mathfrak{{sp}}}}(1,{\ensuremath{\mathbb{R}}})_{\ensuremath{\mathbb{C}}}. 
\end{equation}
If in \eqref{guru22} we make the change of variables $\eta=\xi-{\operatorname{i}}
\zeta$, then we  get the system of linear differential equations in real
($\epsilon_a=b+{\operatorname{i}} a$)
\begin{equation}\label{GURRU12}
\dot{Z}=h_rZ+F, \quad Z=
\left(\begin{array}{c}\zeta \\ \eta\end{array}\right), ~ F=\left(\begin{array}{c} a \\
    b \end{array}\right),   ~ h_r= 
\left(\begin{array}{cc} n & m-p\\
    m+p  & 
    -n  \end{array}\right) \in {{\mathfrak{{sp}}}}(1,{\ensuremath{\mathbb{R}}}). 
\end{equation}
\end{Proposition}
We also reproduce the results concerning the phases obtained in \cite{FC}:
\begin{Proposition}\label{prFAZ}
The Berry phase on the Siegel-Jacobi disk ${{\mathcal{{D}}}}^J_1$ expressed in
the variables $(\eta,w)$, $z=\eta-w\bar{\eta}$, reads
\begin{equation}\label{FFV}
\frac{2}\iid\phi_B=
(\frac{2k\bar{w}}{1-w\bar{w}}-\frac{\bar{\eta}^2}{2}) dw
+(\bar{\eta} + \bar{w}\eta)d \eta - cc . 
\end{equation}
The energy  function \eqref{clH} attached to the Hamiltonian
\eqref{guru}  in  the
coherent state representation \eqref{csu} can be
written as  ${{\mathcal{{H}}}}=
{{\mathcal{{H}}}}_{\eta}+{{\mathcal{{H}}}}_{w}$, where 
\begin{subequations}\label{realHH}
\begin{align}
~{{\mathcal{{H}}}}_{\eta}  ~ & =  ~  \bar{\epsilon}_a\eta+\epsilon_a\bar{\eta} + 
\frac{1}{2}(\epsilon_+\eta^2+\epsilon_-\bar{\eta}^2+\epsilon_0\eta\bar{\eta}),\label{realHH1} \\
~ {{\mathcal{{H}}}}_{w} ~ & =  ~  k\epsilon_0+ \frac{2k}{1-w\bar{w}}(\epsilon_+w+\epsilon_-\bar{w}+\epsilon_0w\bar{w}). \label{realHH2}
\end{align}
\end{subequations}
\end{Proposition}
\subsection{Equations of motion on the Siegel-Jacobi ball
  ${{\mathcal{{D}}}}^J_n$}\label{lasst} 
Following \cite{nou}, we  consider  a  Hamiltonian linear in the generators of
the group $G^J_n$ 
\begin{equation}\label{HACA}
{{\mbox{\boldmath{${H}$}}}}= \epsilon_i{{\mbox{\boldmath{${a}$}}}}_i+\overline{\epsilon}_i{{\mbox{\boldmath{${a}$}}}}_i^{\dagger} +  
\epsilon^0_{ij}{{\mbox{\boldmath{${K}$}}}}^0_{ij}+
\epsilon^-_{ij}{{\mbox{\boldmath{${K}$}}}}^-_{ij}+\epsilon^+_{ij}{{\mbox{\boldmath{${K}$}}}}^+_{ij}. 
\end{equation}
 The hermiticity condition imposes to the matrices of  coefficients $\epsilon_{0,\pm}=(\epsilon^{0,\pm})_{i,j=1,\dots,n} $  the restrictions
\begin{equation}\label{CONDI}
\epsilon_0^{\dagger}=\epsilon_0; ~\epsilon_-=\epsilon_-^t;~
\epsilon_+=\epsilon_+^t; ~ \epsilon_+^{\dagger}=\epsilon_-.
\end{equation}
It is useful to introduce   the matrices $m,n,p,q\in\text{M}(n,{\ensuremath{\mathbb{R}}})$
such that 
\begin{equation}\label{epsmn}
\epsilon_-=m+{\operatorname{i}} n, ~
  \epsilon_0^t/2=p+{\operatorname{i}} q; p^t=p; m^t= m; n^t =n; q^t=-q.
\end{equation} 
We consider 
 a matrix Riccati equation \eqref{RICC} on the manifold $M$ and a linear differential
equation \eqref{LINZ}  in $z\in{\ensuremath{\mathbb{C}}}^n$
  \begin{subequations}\label{TOTAL}
\begin{align}
\dot{W} & =AW+WD+B+WCW, ~A,B,C,D\in M(n,{\ensuremath{\mathbb{C}}}); \label{RICC}\\
\dot{z} & = M+Nz; ~M= E+WF; ~ N= A+WC, ~E,F\in C^n. \label{LINZ}
\end{align}
\end{subequations}
Firstly, we recall  how {\bf to solve the matrix Riccati
equation} \eqref{RICC}  {\bf by linearization}.
{\it If we proceed to the homogenous coordinates} $W=XY^{-1}$, $X,Y\in
M(n,{\ensuremath{\mathbb{C}}})$,  {\it a linear
system of ordinary differential equations is attached to the matrix
Riccati equation} \eqref{RICC} (cf. \cite{levin}, see also \cite{sbl})
\begin{equation}\label{RICClin}
\left( \begin{array}{c}\dot{X}\\\dot{Y}\end{array}\right)=h
\left( \begin{array}{c}{X}\\{Y}\end{array}\right), ~
h=\left(\begin{array}{cc} A & B\\ -C & -D \end{array}\right) . 
\end{equation}
{\it Every solution of} (\ref{RICClin}) {\it is a solution of} (\ref{RICC}),
{\it whenever} $\det (Y)\not=0$. 

\begin{Proposition} \label{POYT}
The classical motion and quantum evolution generated by  the linear
hermitian  Hamiltonian \eqref{HACA}, \eqref{CONDI}  are described by
first order differential equations:\\
a) On ${{\mathcal{{D}}}}^J_n$,  $(z,W)\in {\ensuremath{\mathbb{C}}}^n\times{{\mathcal{{D}}}}_n$ verifies
\eqref{TOTAL}, 
with coefficients
 \begin{subequations}\label{hip}
\begin{align}
A_c & =  -\frac{\operatorname{i}}{2}\epsilon_0^t, ~ B_c=-{\operatorname{i}}\epsilon_-, ~C_c=-{\operatorname{i}}\epsilon_+, ~
D_c= A_c^t ; \label{hip2}\\
E_c & =-{\operatorname{i}}\epsilon, ~F_c=-{\operatorname{i}}\bar{\epsilon}\label{hip1}.
\end{align}
\end{subequations}
b) Explicitly,  the differential equations  for $(W,z)\in{{\mathcal{{D}}}}^J_n$ are
\begin{subequations}\label{Mhip}
\begin{eqnarray}
{\operatorname{i}} \dot{W}  & = &\epsilon_-  +
(W\epsilon_0)^s +W\epsilon_+W, ~ W\in{{\mathcal{{D}}}}_n,\label{Mhip2}\\
{\operatorname{i}} \dot{z} & = & \epsilon +
W\overline{\epsilon}+  \frac{1}{2}\epsilon_0^tz+W\epsilon_+z,
~z\in{\ensuremath{\mathbb{C}}}^n , \label{Mhip1}
\end{eqnarray}
\end{subequations}
c) Under the $FC$ transform, $z=\eta-W\bar{\eta}$,   
the differential equations 
 in the variables $\eta\in{\ensuremath{\mathbb{C}}}^n$, $W\in{{\mathcal{{D}}}}_n$ become independent:  $W$
verifies \eqref{RICC}  with coefficients \eqref{hip2} and $\eta $ verifies  
\begin{equation}\label{hipPRT1}
{\operatorname{i}} \dot{\eta} =  \epsilon +
\epsilon_-\bar{\eta} + \frac{1}{2}\epsilon_0^t\eta,~\eta\in{\ensuremath{\mathbb{C}}}^n.
\end{equation}
d) The linear system of differential equations  \eqref{RICClin} attached to the matrix
Riccati equation \eqref{Mhip2} is
 \begin{equation}\label{Rlin}
\left( \begin{array}{c}\dot{X}\\\dot{Y}\end{array}\!\right)=h_c
\left(\! \begin{array}{c}{X}\\{Y}\end{array}\right), ~
h_c=\left(\begin{array}{cc} -{\operatorname{i}}(\frac{\epsilon_0}{2})^t &
    -{\operatorname{i}}\epsilon_-\\ {\operatorname{i}}\epsilon_+& 
{\operatorname{i}}\frac{\epsilon_0}{2}\end{array}\right)\in {{\mathfrak{{sp}}}}(n,{\ensuremath{\mathbb{R}}})_{\ensuremath{\mathbb{C}}} , ~W=X/Y\in{{\mathcal{{D}}}}_n. 
\end{equation}
e) In \eqref{hipPRT1} we
introduce $\eta=\xi - {\operatorname{i}} \zeta$, $\xi,\zeta\in{\ensuremath{\mathbb{R}}}^n$ and  we put 
$\epsilon= b+{\operatorname{i}} a$, where $a,b\in{\ensuremath{\mathbb{R}}}^n$.  The first order complex differential equation equation
\eqref{hipPRT1} is equivalent with a system of first order real 
differential equations with real coefficients, which we write as  
\begin{equation}\label{LINe} \dot{Z}=h_rZ + F, ~ Z = 
\left( \begin{array}{c}{\xi} \\
    {\zeta}\end{array}\right), ~ 
 F = \left(\begin{array}{c}a \\ b \end{array}\right),
h_r\!=\! \left(\begin{array}{cc} \!n+q
    & m-p\\  m+ p  & -n+q \!\end{array}\!\right)\! \in{{\mathfrak{{sp}}}}(n,{\ensuremath{\mathbb{R}}}).
\end{equation}
\end{Proposition}
\vspace{3ex}

\subsection*{Acknowledgments}
This research  was conducted in  the  framework of the 
ANCS project  program PN 09 37 01 02/2009  and     the UEFISCDI - Romania 
 program PN-II Contract No. 55/05.10.2011. 

  
\begin{thebibliography}{99}

\bibitem{ali}S. T. Ali, J.-P. Antoine and J.-P. Gazeau, {\it Coherent states,
wavelets, and their generalizations}, Springer-Verlag, New York, 2000.

\bibitem{bar47}V. Bargmann, {\it Irreducible unitary representations of the
Lorentz group},  Ann. of Math.  {\bf 48} (1947) 568--640.

\bibitem{sbcag}S. Berceanu  and A. Gheorghe,  {\it On equations of motion on Hermitian
 symmetric  spaces},
  J. Math. Phys.   {\bf 33}  (1992)  998--1007.

\bibitem{sbl}S.  Berceanu   and  L. Boutet de Monvel,  
{\it  Linear dynamical systems, coherent
state manifolds, flows and matrix Riccati equation},
   J. Math. Phys.  {\bf 34}  (1993) 2353--2371.

\bibitem{sb6}S. Berceanu,    Realization of coherent state algebras
by differential operators,
in  {\it {Advances in Operator Algebras and Mathematical Physics}},
eds.   F. Boca, O. Bratteli, R. Longo and  H. Siedentop, 
The Theta Foundation, Bucharest, 1--24,  2005.

\bibitem{jac1} S. Berceanu,  {\it A holomorphic representation of the Jacobi algebra},
Rev. Math. Phys.  {\bf 18}  (2006) 163-199; {\it  Errata},   Rev.  Math. Phys.
{\bf 24}  (2012) 1292001 (2 pages);  arXiv: 0408219   [math.DG].

\bibitem{mlad} S. Berceanu, {\it Coherent states associated to the Jacobi group -} 
{\it a variation on a theme by Erich K\"ahler},
 J. Geom.  Symmetry  Physics {\bf 9} (2007) 1--8.

\bibitem{sbj}
S. Berceanu,   A holomorphic 
representation of Jacobi algebra in several 
dimensions, in {\it { Perspectives in Operator Algebra and Mathematical
Physics}},  eds.  F.-P. Boca, R. Purice and  S. Stratila, The Theta Foundation,
 Bucharest 1-25, 2008;  arXiv: 060404381 [math.DG].

\bibitem{sbcg}S. Berceanu  and A. Gheorghe, {\it Applications of the Jacobi group to
Quantum Mechanics},  Romanian J.  Phys. {\bf 53}  (2008)
 1013--1021; arXiv: 0812.0448  [math.DG].

\bibitem{sb12} S. Berceanu, {\it Generalized Coherent States Based on Siegel-Jacobi disk},
   Romanian J. Phys. {\bf 56}  (2011) 856--867.  

 \bibitem {sb13}S. Berceanu  and  A. Gheorghe, {\it Matrix elements of  the Jacobi
   group}, Romanian J. Phys. {\bf 56}   (2011)  1056--1068.   

\bibitem{gem}S. Berceanu and  A. Gheorghe, {\it On the geometry of Siegel-Jacobi domains},
 Int. J. Geom. Methods Mod. Phys. {\bf 8}  (2011) 1783--1798;  arXiv:
 1011.3317v1 [math.DG]. 

\bibitem{nou}S. Berceanu,  {\it A convenient coordinatization of Siegel-Jacobi
    domains},   Rev.  Math. Phys.  {\bf 24}  (2012) 1250024 (38 pages);
  arXiv: 1204.5610 [math.DG]. 

\bibitem{FC} S. Berceanu,  {\it Consequences of the fundamental conjecture for the motion
  on the  Siegel-Jacobi disk},   Int.  J.  Geom. Methods Mod. Phys. {\bf
  10} (2013) 1250076 (18 pages);   arXiv: 1110.5469v2  [math.DG]. 

\bibitem{ber14}  S. Berceanu, {\it Coherent states and geometry on the Siegel-Jacobi
disk},  Int.  J. Geom. Methods Mod. Phys. (2014) 1450035 (25 pages);
arXiv: 1307.4219v2 [math.DG].

\bibitem{berezin2}F. A. 
 Berezin    {\it Quantization in complex symmetric spaces}, (Russian)
 Izv. Akad. Nauk SSSR Ser. Mat.     {\bf 39}  (1975) 363--402, 472. 
  

\bibitem{berezin1} F. A. Berezin,  {\it Models of Gross-Neveu type are
  quantization of a Classical Mechanics with a nonlinear phase space},
 Commun. Math. Phys.  {\bf 63}  (1978) 131--153.

\bibitem{bs}R. Berndt and  R.  Schmidt, {\it  Elements of the representation
theory of the Jacobi group}, Progress in Mathematics {\bf 163}, 
Birkh\"auser, Basel, 1998. 

\bibitem{ca1} J. Cari\~{n}ena,  J. Grabowski and  G. Marmo, {\it  Lie-Scheffers
  systems: a geometric approach}, Naples, Bibliopolis, 2000. 

\bibitem{ca2} J. F.  Cari\~{n}ena,  J. Grabowski and  A. Ramos, {\it Reduction of time-dependent systems
admitting a superposition principle}, Acta Appl. Math. {\bf 66}
(2001) 67--87.

\bibitem{ca3} J. F. Cari\~{n}ena and  A. Ramos, Applications of Lie systems in quantum mechanics and control theory, in
“{\it Classical and quantum integrability}, Banach Center Publ. {\bf
  59} (2003)  Warsaw,
 143--162. 

\bibitem{ca4} J. F. Cari\~{n}ena and J. de Lucas, {\it Lie systems: theory, generalisations, and applications}, Dissertationes
Math. (Rozprawy Mat.) {\bf 479}  (2011) 162 pp. 

\bibitem{DN}J. Dorfmeiser and  K. Nakajima, {\it The fundamental
    conjecture for homogeneous K\"ahler manifolds}, Acta Mathematica 
  {\bf 161} (1988) 23--70.

\bibitem{dr} P. D.  Drummond and  Z. Ficek, Editors, \emph{Quantum squeezing}, Springer, Berlin, 2004.

\bibitem{ez} M. Eichler and D. Zagier, \emph{The theory of Jacobi forms},
Progress in Mathematics  {\bf 55},
Birkh\"auser, Boston, MA, 1985.

\bibitem{fey} R. Feynman, {\it An operator calculus having applications in
  Quantum Electrodynamics},  Phys. Rev. \textbf{84}  (1951) 108--128.

\bibitem{cezar}A. Gheorghe, {\it Quantum and classical Lie systems for extended symplectic groups}, Romanian J. Phys. {\bf 58} (2013) 1436--1445.

\bibitem{viorica} V.  N. Gheorghe and   F. Vedel, {\it Quantum dynamics of trapped  ions}, 
 Phys. Rev. A  {\bf 45}   (1992) 4828--4831. 

\bibitem{hagen}C. R.  Hagen, {\it Scale and conformal transformations in
  Galilean-covariant field theory},   Phys. Rev. D 
{\bf 5}   (1972) 377--388.

\bibitem{ho}J. N. Hollenhors, {\it Quantum limits on resonant-mass
gravitational-wave detectors}, {Phys. Rev. D} {\bf 19} (1979) 1669--1679.

\bibitem{how}J. S. Howland,   {\it Stationary scattering theory for time-dependent
Hamiltonians}, Math. Ann.    {\bf 207}  (1974) 315--335.

\bibitem{hs} K. Husimi, {\it Miscelllanea in elementary Quantum
    Mechanics, II},  Prog. Theor. Phys. {\bf 9}  (1953) 381--402.

\bibitem{LEE03} M. H. Lee, \textit{Theta functions on hermitian symmetric
domains and Fock representations},  J. Aust. Math. Soc. 74 (2003) 201--234.

  
\bibitem{levin}J. J. Levin, {\it On the matrix Riccati equation},
  Proc. Am. Math. Soc. {\bf 10} (1959) 519--524.  

\bibitem{lie} S. Lie, {\it Allgemeine Untersuchungen \"uber Differentialgleichungen, die eine continuirliche
endliche Gruppe gestatten} (German), Math. Ann. {\bf 6}   (1880) 441--528.

\bibitem{lies} S. Lie, G. Scheffers, {\it Vorlesungen ¨uber continuierliche Gruppen mit geometrischen und anderen
Anwendungen}  (German), Leipzig, Teubner, 1893.

\bibitem{lis}W. Lisiecki, {\it Coherent state representations. A survey},
  Rep. Math. Phys. {\bf 35} (1995) 327--358.

\bibitem{lu}E. Y. C. Lu,  {\it New coherent states of the
electromagnetic field}, Lett. Nuovo. Cimento  {\bf 2}  (1971)
1241--1244. 

\bibitem{mag}W. Magnus, {\it On the exponential solution of differential
equations for a linear operator}, Comm. Pure Appl. Math. {\bf 7}
(1954) 649--673.

\bibitem{ma}F. Major, V. N. Gheorghe and  G. Werth, {\it Charged Particle Traps, Physics and Techniques of charged
Particle Field Confinement},  Springer, Berlin, 2005.

\bibitem{mandel} L. Mandel and   E. Wolf, {\it Optical coherence and Quantum Optics}, Cambridge University Press, 1995.

\bibitem{neeb}K.-H. Neeb, {\it Holomorphy and Convexity in Lie Theory}, de
 Gruyter Expositions in Mathematics {\bf 28}, Walter de Gruyter,
 Berlin, New York, 2000.

\bibitem{ni}U. Niederer, {\it  The maximal kinematical invariance
group of the harmonic oscillator}, {Helv. Phys. Acta} {\bf 46} (1973)
191--200.

\bibitem{perG}A. M.  Perelomov, {\it Generalized Coherent States and their
Applications}, Springer, Berlin, 1986.

\bibitem{satake}I. Satake, {\it Algebraic structures of symmetric domains},
Publ. Math. Soc. Japan {\bf 14},  Princeton Univ. Press,  1980.

\bibitem{sw}J. Schwinger, {\it The Theory of Quantized Fields. III.},
  Phys. Rev. \textbf{91}  (1953) 728--740.  

\bibitem{swA}A.   Shapere  and  F.  Wilczek, eds.,  {\it Geometric Phases in
    Physics}, Singapore,   World Scientific, 1989. 

\bibitem{stol}P.  Stoler, {\it Equivalence classes of minimum uncertainty packets},
  Phys. Rev. D   {\bf  1}  (1970)  3217--3219.

\bibitem{TA99} K. Takase, \textit{On Siegel modular forms of half-integral
weights and Jacobi forms}, Trans.  Amer. Math. Soc. {\bf 351}  (1999)
735--780. 

\bibitem{GV}E. B. Vinberg and  S. G. Gindikin, {\it K\"ahlerian
    manifolds admitting a   transitive solvable automorphism group}, 
Math. Sb. 74 (116) (1967) 333--351.

\bibitem{wei}J. Wei and E. Norman, {\it On global representations of
    the solutions of linear differential equations as a product of
    exponentials}, Proc. AMS  {\bf 15} (1964) 327--334.

\bibitem{kbw1} K. B.  Wolf, The Heisenberg-Weyl ring in quantum
mechanics, in
 {\it Group theory and its applications\/}, {\bf 3},  E. M. Loebl,
 Editor, 
Academic Press, New York, 189--247, 1975.

\bibitem{Y02} J.-H. Yang, \textit{The method of orbits for real Lie groups},
Kyungpook Math. J. {\bf 42}  (2002) 199--272.

\bibitem{Y10} J.-H. Yang, \textit{Invariant metrics and Laplacians on the
Siegel-Jacobi disk}, Chin. Ann. Math.  {\bf 31B}   (2010) 85--100.

\end{thebibliography}

\today
\end{document}


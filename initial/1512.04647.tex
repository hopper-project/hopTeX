\documentclass{amsproc}

\usepackage{amsthm}

\usepackage{graphicx}
\usepackage{epsfig}
\usepackage{epstopdf}

\usepackage{latexsym}
\usepackage{amsmath}
\usepackage{amssymb}
\usepackage{amsfonts}

\theoremstyle{definition}

\newtheorem{theorem}{Theorem}
\newtheorem{lemma}[theorem]{Lemma}

\theoremstyle{definition}
\newtheorem{definition}[theorem]{Definition}
\newtheorem{proposition}[theorem]{Proposition}
\newtheorem{corollary}[theorem]{Corollary}
\newtheorem{prop}[theorem]{Proposition}

\theoremstyle{remark}
\newtheorem{remark}[theorem]{Remark}

\begin{document}

\title{Krull dimensions of rings of holomorphic functions}
\author{Michael Kapovich}
\address{Department of Mathematics, 
University of California, Davis, 
CA 95616}
\email{kapovich@math.ucdavis.edu}
\thanks{The author was supported in part by the NSF Grant DMS-12-05312 and by the Korea Institute for Advanced Study (KIAS)}

\subjclass{Primary 32A10, 16P70}

\keywords{Krull dimension; ring of holomorphic functions}

\date{December 27, 2015}

\begin{abstract}
We prove that the Krull dimension of the ring of holomorphic functions of a connected complex manifold is at least the cardinality of continuum iff it is $>0$. 
\end{abstract}

\maketitle

Let $R$ be a commutative ring. Recall that the {\em Krull dimension} $\dim(R)$ of $R$ is the supremum of cardinalities lengths of chains of  distinct proper prime ideals in $R$. Our main result is: 

\begin{theorem}\label{main}
Let $M$ be a connected complex manifold and $H(M)$ be the ring of holomorphic functions on $M$. Then the Krull dimension of $H(M)$ either equals $0$ (iff $H(M)= {{\mathbb C}}$) or is infinite, iff $M$ admits a nonconstant holomorphic function $M\to {{\mathbb C}}$. More precisely, unless $H(M)={{\mathbb C}}$, $\dim H(M)\ge {\mathfrak c}$, i.e., the ring $H(M)$ 
contains a chain of distinct prime ideals whose length has cardinality of continuum. 
\end{theorem}

Our proof of this theorem mostly follows the lines of the proof by Sasane \cite{Sasane}, who proved that for each nonempty domain $M\subset {{\mathbb C}}$ the Krull dimension of $H(M)$ is infinite (he did not prove that $\dim H(M)\ge {\mathfrak c}$). 

\begin{remark}
We note that Henricksen \cite{Henricksen} was the first to prove that the Krull dimension of the ring of entire functions on ${{\mathbb C}}$ has cardinality at least continuum.
\end{remark}

In our proof we will use the Axiom of Choice  in two ways: (a) to establish existence of certain maximal ideals and  
(b) to get existence of a nonprincipal ultrafilter ${\omega}$ on ${{\mathbb N}}$ and, hence of the ordered field ${{}^{*}{{\mathbb R}}}$ of 
{\em nonstandard real} (or, {\em surreal}) numbers. The field ${{}^{*}{{\mathbb R}}}$ contains ${{}^{*}{{\mathbb N}}}$, the {\em nonstandard natural} (or {\em surnatural}) numbers.

The field ${{}^{*}{{\mathbb R}}}$ is a certain quotient of the countable direct product $\prod_{k\in {{\mathbb N}}} {{\mathbb R}}$; we will denote the equivalence class (in ${{}^{*}{{\mathbb R}}}$) of a sequence $(x_k)$ in ${{\mathbb R}}$  by $[x_k]$. Accordingly, ${{}^{*}{{\mathbb N}}}$ consists of equivalence classes $[n_k]$ of sequences of natural numbers. Roughly speaking, we will use ${{}^{*}{{\mathbb N}}}$ and certain order relation on it to compare rates of growth of sequences of  natural numbers. 

\begin{definition}\label{defn:ample}
A commutative unital ring $R$ is {\em ample} if there exists a sequence of valuations $\nu_k$ on $R$ such that for each ${\beta}\in {{}^{*}{{\mathbb N}}}$, there $a=a_{\beta}\in R$ with the property
\begin{equation}\label{1}
[\nu_{k}(a)]= {\beta}  . 
\end{equation}
\end{definition}

\medskip
The main technical result of this paper is: 

\begin{theorem}\label{thm:T}
For each ample ring  $R$, $\dim(R)\ge {\mathfrak c}$.  In particular, $R$ has infinite Krull dimension. 
\end{theorem}

This theorem and its proof are inspired by Theorem 2.2 of \cite{Sasane}, although some parts of the proof resemble the ones of \cite{Henricksen}. 

We will verify, furthermore, that whenever $M$ is a connected complex manifold which has a nonconstant holomorphic function, the ring $H(M)$ is ample.  This, combined with Theorem \ref{thm:T}, will immediately imply Theorem \ref{main}. 

\begin{remark}
1. We refer the reader to Section 5.3 of \cite{Clark} for further discussion of algebraic properties of rings of holomorphic functions.

2. Theorem \ref{main} shows that for every Stein manifold $M$ (of positive dimension), the ring $H(M)$ has infinite Krull dimension. In particular, this applies to any noncompact connected Riemann surfaces (since every such surface is Stein, \cite{BS}). 

3. Noncompact connected complex manifolds $M$ of dimension $>1$ can have  $H(M)={{\mathbb C}}$; for instance, take $M$ to be the complement to a finite subset in a compact connected complex manifold (of dimension $>1$). 
\end{remark}

\medskip
\noindent {\bf Acknowledgements.} This note grew out of the mathoverflow question, 

 http://mathoverflow.net/questions/94537, 

\noindent and I am grateful to Georges Elencwajg for asking the question. I am also grateful to Pete Clark for pointing at several errors in earlier versions of the paper (most importantly, pointing out that Lemma \ref{lem:max} is needed for the proof, which forced me to use ultrafilters) and providing references. 

\section{Surreal numbers}
 
We refer the reader to \cite{Goldblatt} for a detailed treatment of surreal numbers, below is a brief introduction. 
A nonprincipal ultrafilter on ${{\mathbb N}}$ can be regarded as a finitely-additive probability measure on ${{\mathbb N}}$ which vanishes on each finite subset and takes  the value $0$ or $1$ on each subset of ${{\mathbb N}}$. The existence of nonprincipal ultrafilters (the {\em ultrafilter lemma}) follows from the Axiom of Choice. Subsets of full measure are called {\em ${\omega}$-large}. Using ${\omega}$ one defines the following equivalence relation on the product 
$$
\prod_{k\in {{\mathbb R}}} {{\mathbb R}}. 
$$
Two sequences $(x_k)$ and $(y_k)$ are equivalent if $x_k=y_k$ for an ${\omega}$-all $k$, i.e. the set 
$$
\{k: x_k=y_k\}
$$ 
is ${\omega}$-large. The quotient by this equivalence relation, denoted  
$$
{{}^{*}{{\mathbb R}}}=  \prod_{k\in {{\mathbb N}}} {{\mathbb R}}/{\omega}, 
$$
is the set of surreal numbers. Let $[x_k]$ be the equivalence class of the sequence $(x_k)$. 

The binary operations on sequences of real numbers project to binary operations on ${{}^{*}{{\mathbb R}}}$ making ${{}^{*}{{\mathbb R}}}$ a field. The total order  $\le$ on ${{}^{*}{{\mathbb R}}}$ is defined by $[x_k]\le [y_k]$ iff $x_k\le y_k$ for an ${\omega}$-all $k\in {{\mathbb N}}$. With this order, ${{}^{*}{{\mathbb R}}}$ becomes an ordered field.  

The set of real numbers embeds into ${{}^{*}{{\mathbb R}}}$ as the set of equivalence classes of constant sequences; the image of a real number $x$ under this embedding is still denoted $x$. We set ${{}^{*}{{\mathbb R}}}_+:=\{{\alpha}\in {{}^{*}{{\mathbb R}}}: {\alpha}>0\}$. 

The projection of
$$
\prod_{k\in {{\mathbb N}}} {{\mathbb N}}  \subset \prod_{k\in {{\mathbb N}}} {{\mathbb R}}
$$
to ${{}^{*}{{\mathbb R}}}$ is denoted ${{}^{*}{{\mathbb N}}}$, this is the set of {\em surnatural numbers}. We define a further equivalence relation $\sim_u$ on 
${{}^{*}{{\mathbb R}}}$ by:
$$
{\alpha}\sim_u {\beta}
$$
if there exist positive real numbers $a, b$ such that
$$
a{\alpha} \le {\beta} \le b {\alpha}. 
$$
The equivalence class $({\alpha})$ of ${\alpha}\in {{}^{*}{{\mathbb R}}}$ (for this equivalence relation) is a multiplicative analogue of the {\em galaxy} $gal({\alpha})$ of ${\alpha}$, see   \cite{Goldblatt}:

\begin{definition}
The {\em galaxy} $gal({\alpha})$ of a surreal number ${\alpha}\in {{}^{*}{{\mathbb R}}}$ is the union
$$
\bigcup_{n\in {{\mathbb N}}} [{\alpha}-n, {\alpha}+n] \subset {{}^{*}{{\mathbb R}}}. 
$$ 
In other words, ${\beta}\in gal({\alpha})$ iff there exist a real number $a$ such that ${\alpha}-a\le {\beta} \le {\alpha}+ a$. 
\end{definition}

The next lemma is immediate:

\begin{lemma}
For ${\alpha}\in {{}^{*}{{\mathbb R}}}_+$, the equivalence class $({\alpha})$ of ${\alpha}$ equals $\exp( gal ( \log({\alpha})))$. 
\end{lemma}

We let ${{}^{u}{{\mathbb R}}}$ denote the quotient ${{}^{*}{{\mathbb R}}}/\sim_u$ and ${{}^{u}{{\mathbb N}}}$ the projection of ${{}^{*}{{\mathbb N}}}$ to ${{}^{u}{{\mathbb R}}}$. Define the total order 
$\gg$ on ${{}^{u}{{\mathbb R}}}$ by
$$
({\beta}) \gg ({\alpha})  
$$  
if for every real number $c$, $c{\alpha} < {\beta}$. By abusing the notation, we will simply say that ${\beta} \gg {\alpha}$, with ${\alpha}, {\beta}\in {{}^{*}{{\mathbb R}}}$. 

For the reader who prefers to think in terms of sequences of (positive) real numbers, 
the relation $({\beta}) \gg ({\alpha})$ is an analogue of the relation  
$$
(a_n)= o((b_n)), \quad n\to \infty. 
$$

\begin{remark}
The equivalence relation $\sim_u$ and the order $\gg$ are similar to the ones used by Henricksen in \cite{Henricksen}. \end{remark}

\begin{prop}\label{prop:un}
The set ${{}^{u}{{\mathbb N}}}$ has the cardinality of continuum. 
\end{prop}
{\par\medskip\noindent{\it Proof. }} Note first, that ${{}^{*}{{\mathbb R}}}$ has cardinality of continuum, hence, the cardinality of ${{}^{u}{{\mathbb N}}}$ is at most ${\mathfrak c}$. The proof of the proposition then reduces to two lemmata. 

\begin{lemma}\label{lem:cont}
The set $gal({{}^{*}{{\mathbb R}}}_+)$ of galaxies $\{gal({\alpha}): {\alpha}\in {{}^{*}{{\mathbb R}}}_+\}$ has the cardinality of continuum.
\end{lemma}
{\par\medskip\noindent{\it Proof. }} For each ${\alpha}=[a_k]\in {{}^{*}{{\mathbb R}}}_+$, the  galaxy $gal({\alpha})$ contains the surnatural number $\lceil {\alpha} \rceil= [ b_k]$, 
where $b_k=   \lceil a_k \rceil$. For each surnatural number ${\beta}\in {{}^{*}{{\mathbb N}}}$, and natural number $n\in {{\mathbb N}}$, the intersection
$$
[{\beta} -n, {\beta}+ n]\cap {{}^{*}{{\mathbb N}}}
$$
is finite, equal $\{{\beta} -n,..., {\beta} + n\}$. Therefore, $gal({\beta}) \cap {{}^{*}{{\mathbb N}}}= \{{\beta}\} + {{\mathbb Z}}$. It follows that the map 
$$
{{}^{*}{{\mathbb N}}} \to gal({{}^{*}{{\mathbb R}}}_+), \quad {\beta} \mapsto gal({\beta}) 
$$
is a bijection modulo ${{\mathbb Z}}$. Lastly, the set of surnatural numbers ${{}^{*}{{\mathbb N}}}$ has the cardinality of continuum. \qed 

\begin{lemma}\label{lem:surj}
The map $\lambda: {{}^{*}{{\mathbb N}}}\to gal({{}^{*}{{\mathbb R}}}_+)$, $\lambda: {\beta}\mapsto gal(\log(n))$,  is surjective. 
\end{lemma}
{\par\medskip\noindent{\it Proof. }} For each ${\alpha}\in {{}^{*}{{\mathbb R}}}_+$ let ${\beta} = \lceil \exp({\alpha}) \rceil \in {{}^{*}{{\mathbb N}}}$. Since $\log(x+1) -\log(x) \le 1$ for $x\ge 1$, we have that
$$
\log ({\beta}) \in  gal({\alpha}). \qed 
$$ 

Now, we can finish the proof of the proposition. The map ${\lambda}: {{}^{*}{{\mathbb N}}}\to gal({{}^{*}{{\mathbb R}}}_+)$ descends to a map $\mu: {{}^{u}{{\mathbb N}}}\to gal({{}^{*}{{\mathbb R}}}_+)$. According to Lemma \ref{lem:surj}, the map $\mu$ is surjective. By Lemma \ref{lem:cont}  the set  $gal({{}^{*}{{\mathbb R}}}_+)$ has the cardinality of continuum.  \qed 

\medskip 
We will prove Theorem \ref{thm:T} in the next section by showing that for each ample ring $R$, the ordered set 
$({{}^{u}{{\mathbb N}}}, \gg)$ embeds into the poset of prime ideals in $R$ reversing the order:
$$
({\beta}) \gg ({\alpha}) \Rightarrow P_\beta\subsetneq P_{\alpha}
$$
for certain prime ideals $P_{\gamma}\subset R$ determines by $({\gamma})\in {{}^{u}{{\mathbb N}}}$.  
Proposition \ref{prop:un} will then imply that the Krull dimension of $R$ is at least ${\mathfrak c}$. 

\section{Krull dimension of ample rings}

 
\noindent Recall  that a {\em valuation} on a unital ring $R$ is a map $\nu: R\to {{\mathbb R}}_+\cup \{\infty\}$ such that:

1. $\nu(a+b)\ge \min(a, b)$,

2. $\nu(ab)=\nu(a)+\nu(b)$. 

3. $\nu(a)=\infty \iff a=0$. 

4. $\nu(1)=0$.  

\noindent For the following lemma, see  Theorem 10.2.6 in \cite{Cohn} (see also Proposition 4.8 of \cite{Clark} or Theorem 1 in \cite{Kaplansky}). 

\begin{lemma}\label{L0}
Let $I$ be an ideal in a commutative ring $A$ and $M\subset A\setminus I$ be a subset closed under multiplication. Then there exists an ideal $J\subset A$ containing  $I$ and disjoint from $M$, so that $J$ is maximal with respect to this property. Furthermore, $J$ is a prime ideal in $A$. 
\end{lemma}

Let $R$ be an ample ring and $\nu_k$ the corresponding sequence of valuations on $R$. 
For each ${\beta}\in {{}^{*}{{\mathbb N}}}$ we define 
$$
I_{\beta}:=\{a\in R | ~~ [\nu_k(a)] \gg [{\beta}] \}\subset R. 
$$  

\begin{lemma}\label{lem:add}
Each $I_{\alpha}$ is an ideal in $R$. 
\end{lemma}
{\par\medskip\noindent{\it Proof. }} We will check that $I_{\alpha}$ is additive since it is clearly closed under multiplication by elements of $R$. 
Take $p', p''\in I_{\alpha}$,  
$$
[\nu_k(p')]\gg {\alpha}, [\nu_k(p'')] \gg {\alpha}. 
$$
By the definition of a valuation,  
$$
n_k:=\nu_k(p'+ p'') \ge \min( \nu_k(p'), \nu_k(p'')),
$$
for each $k\in {{\mathbb N}}$. For $m\in {{\mathbb N}}$, define the ${\omega}$-large sets 
$$
A'=\{k: \nu_k(p')\ge m {\alpha}\}, \quad A''=\{k: \nu_k(p'')\ge m {\alpha}\}.$$ 
Therefore, their intersection $A=A'\cap A''$ is ${\omega}$-large as well, which implies that 
$$
\forall m\in {{\mathbb N}}, [n_k] \ge m {\alpha} \Rightarrow [n_k] \gg {\alpha}.  \qed 
$$

Then for each ${\gamma} \gg {\beta}$, the element $a_{\gamma}$ as in Definition \ref{defn:ample}, belongs to $I_\beta$.
It follows that $I_{\beta}\ne 0$ for every ${\beta}$. Define the subsets    
$$
M_{\beta}:= \{ a\in R  | \exists n\in {{\mathbb N}}, [\nu_k(a)] \le n {\beta} \}\subset R; 
$$
each $M_{\beta}$ is  closed under the multiplication. It is immediate that whenever ${\alpha}\le {\beta}$, we have the inclusions 
$$
I_{\beta} \subset I_{\alpha} , \quad M_{\alpha}\subset M_{\beta}. 
$$
It is also clear that $I_{\beta} \cap M_{\beta}=\emptyset$. At the same time,  for each 
${\beta} \gg {\alpha}$,
$$
a_\beta\in I_{\alpha} \cap M_\beta. 
$$

 For each ${\alpha}$ we let 
${\mathcal J}_{\alpha}$ denote the set of ideals $P\subset R$ such that 
$$
I_{\alpha}\subset P, P\cap  M_{\alpha}=\emptyset.$$ 
By Lemma \ref{L0}, every maximal element $P\in {\mathcal J}_{\alpha}$ is a prime ideal. 

\begin{lemma}\label{lem:max}
Every ${\mathcal J}_{\alpha}$ contains unique maximal element, which we will denote $P_{\alpha}$ in what follows. 
\end{lemma}
{\par\medskip\noindent{\it Proof. }} Suppose that $P', P''$ are two  maximal elements of  ${\mathcal J}_{\alpha}$. We define the ideal 
$P=P' + P''$. Clearly, $P$ contains $I_{\alpha}$. To prove that $P$ is disjoint from $M_{\alpha}$, take  
 $p'\in P', p''\in P''$, since $p'\notin M_{\alpha}, p''\notin M_{\alpha}$. Then the same proof as in Lemma \ref{lem:add} shows that 
 $[\nu_k(p'+ p'') ]\gg {\alpha}$ which means that $p'+p''\notin M_{\alpha}$. 
 Thus, $P\in {\mathcal J}_{\alpha}$ and, in view of maximality of $P', P''$, we obtain
$$
P'= P= P''. \qed 
$$

For each ${\beta} \gg {\alpha}$ we define the ideal $Q_{{\alpha}{\beta}}:=I_{\alpha}+ P_{\beta}$. 

\begin{lemma}\label{L1}
$Q_{{\alpha}{\beta}} \cap M_{\alpha}=\emptyset$. 
\end{lemma}
{\par\medskip\noindent{\it Proof. }} The proof is similar to the one of the previous lemma. 
Let $q=c+p$, $c\in I_{\alpha}, p\in P_{\beta}$. Since $p\notin M_{\beta}$, $p\notin M_{\alpha}$ as well. Therefore, 
$$
[\nu_{k}(p)] \gg {\alpha}.   
$$ 
Since $c\in I_{\alpha}$, 
$$
[\nu_{k}(c)] \gg {\alpha}. 
$$
Hence, 
$$
[\nu_{k}(c+p)] \gg {\alpha} 
$$
as well. Thus, $q\notin M_{\alpha}$. \qed 

\begin{corollary}\label{cor:C}
$Q_{{\alpha}{\beta}}\in {\mathcal J}_{\alpha}$. In particular, $Q_{\alpha}\subset P_{\alpha}$. 
\end{corollary}
{\par\medskip\noindent{\it Proof. }} It suffices to note that $I_{\alpha}\subset Q_{{\alpha}{\beta}}$ according to the definition of $Q_{{\alpha}{\beta}}$. \qed 

\begin{lemma}\label{L2}
The inequality ${\beta} \gg {\alpha}$ implies $P_{\beta}\subset P_{\alpha}$ and this inclusion is proper. 
\end{lemma}
{\par\medskip\noindent{\it Proof. }} By the definition of $Q_{{\alpha}{\beta}}$ and Corollary \ref{cor:C}, we have the inclusions
$$
P_{\beta}\subset Q_{\alpha}\subset P_{\alpha}. 
$$
We now claim that $P_{\beta}\ne Q_{{\alpha}{\beta}}=I_{\alpha}+ P_{\beta}$. Recall that $a_{\alpha}\in I_{\alpha}\subset Q_{{\alpha}{\beta}}$ and 
$a_{\alpha}\in M_{\beta}$, while $M_{\beta}\cap P_{\beta}=\emptyset$. Thus, $a_{\alpha}\in Q_{{\alpha}{\beta}} \setminus P_{\beta}$. \qed  

\medskip
According to Proposition \ref{prop:un}, the set  ${{}^{*}{{\mathbb N}}}$ of surnatural numbers contains a subset $S$ of cardinality continuum such that for all ${\alpha}< {\beta}$ in $S$, we have
${\beta} \gg {\alpha}$. The map
$$
{\alpha}\mapsto P_{\alpha}
$$
sends each ${\alpha}\in S$ to a prime ideal in $R$; ${\alpha}< {\beta}$ implies that $P_{\beta} \subsetneq P_{\alpha}$. 

We conclude that the ring $R$ contains the (descending) chain of distinct prime ideals $P_{\alpha}, {\alpha}\in S$; the length of this chain has the cardinality of continuum.
In particular, $\dim(R)\ge {\mathfrak c}$. Theorem \ref{thm:T} follows. \qed 

\section{Ampleness of rings of holomorphic functions}

\noindent We will need the following classical result, see e.g. \cite[Ch. VII, Theorem 5.15]{Conway}: 

\begin{theorem}\label{thm:W}
Let $D\subset {{\mathbb C}}$ be a domain, and let $c_k\in D$ be a sequence which does not accumulate anywhere in $D$ and let $m_k$ be a sequence of natural numbers. Then there exists a holomorphic function $g$ in $D$ which has zeroes only at the points $c_k$ and such that $m_k$ is the order of zero of $g$ at $c_k$, $k\in {{\mathbb N}}$. 
\end{theorem}

\begin{corollary}\label{cor:main}
If $M$ is a connected complex manifold which admits a nonconstant holomorphic function $h: M\to {{\mathbb C}}$, 
then the ring $H(M)$ is ample. 
\end{corollary}
{\par\medskip\noindent{\it Proof. }} We let $D$ denote the image of $h$. Pick a sequence $c_k\in D$ which converges to a point in 
$\hat{{\mathbb C}} \setminus  D$ and which consists of regular values of $h$.  
(Here $\hat{{\mathbb C}}$ is the Riemann sphere.) For each $c_k$ the preimage $C_k:= h^{-1}(c_k)$ 
is a complex submanifold in $M$; in each $C_k$ pick a point $b_k$. Define valuations 
$$\nu_k: H(M)\to {{\mathbb Z}}_+ \cup \{\infty\}$$ 
by $\nu_k(f):= ord_{b_k}(f)$, the total order of $f$ at $b_k$, cf.  \cite[Chapter C, Definition 1]{Gunning}.  

Now, given ${\beta}\in {{}^{*}{{\mathbb N}}}$, ${\beta}=[m_k]$, we let $g=g_{\beta}$ denote a holomorphic function on $D$ as in Theorem 
\ref{thm:W}.   Define $a=a_{\beta}:= g\circ h\in H(M)$. Then 
$\nu_k(a)= m_k$, which implies that the ring $H(M)$ is ample. \qed 

\medskip 
Ampleness of $H(M)$ together with Theorem \ref{thm:T} imply Theorem \ref{main}. 

\begin{thebibliography}{BLP05}

\bibitem[BS]{BS}
H. Behnke, K. Stein, {\em 
Entwicklung analytischer Funktionen auf Riemannschen Fl\"{a}chen}, 
Math. Ann. Vol. {\bf 120} (1949) p. 430--461. 

\bibitem[Cla]{Clark}
P. Clark, ``Commutative Algebra''. Preprint. 

\bibitem[Coh]{Cohn}
P. M. Cohn, ``Basic Algebra: Groups, Rings and Fields'', Springer Verlag,  2004.  

\bibitem[Con]{Conway}
J. B. Conway, ``Functions of One Complex Variable'' I, 2nd edition. Springer Verlag, New York (1978). 

\bibitem[Go]{Goldblatt}
 R. Goldblatt, ``Lectures on the hyperreals,''
Graduate Texts in Mathematics, Vol.  {\bf 188}, Springer-Verlag, 1998. 

\bibitem[Gu]{Gunning}
R. Gunning, ``Introduction to Holomorphic Functions of Several Variables,'' Volume {\bf 1}, Wadsworth \& Brooks/Cole, 1990. 

\bibitem[H]{Henricksen} 
M. Henricksen, {\em On the prime ideals of the ring of entire functions}, Pacific J. Math. Vol. {\bf 3} (1953) p. 711--720.

\bibitem[K]{Kaplansky}
I. Kaplansky, ``Commutative Rings'', Allyn and Bacon, Inc., Boston, Mass. 1970. 

\bibitem[S]{Sasane}
A.\ Sasane, 
{\em On the Krull dimension of rings of transfer functions}, 
Acta Appl. Math. Vol. {\bf 103} (2008) p. 161--168. 

\end{thebibliography}

\end{document}
\end


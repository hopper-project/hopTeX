\documentclass{compositio}

\usepackage{amsmath}
\usepackage{amsthm,amsfonts,amssymb,mathrsfs}
\usepackage{epic,eepic}
\usepackage{yfonts}
\usepackage{paralist,enumerate}
\usepackage[all]{xy}

\newtheorem{theorem}{Theorem}
\newtheorem{proposition}[theorem]{Proposition}
\newtheorem{lemma}[theorem]{Lemma}
\newtheorem{corollary}[theorem]{Corollary}
\newtheorem{conjecture}[theorem]{Conjecture}
\newtheorem{problem}[theorem]{Problem}
\newtheorem{question}[theorem]{Question}



\newtheorem*{Brunn-Minkowski}{Brunn-Minkowski Inequality}
\newtheorem*{proposition*}{Proposition}
\newtheorem*{conjecture*}{Conjecture}
\newtheorem*{question*}{Question}
\newtheorem*{observation}{Observation}
\newtheorem*{problem*}{Problem}
\newtheorem*{theorem*}{Theorem}

\theoremstyle{definition}
\newtheorem{definition}[theorem]{Definition}
\newtheorem*{definition*}{Definition}
\newtheorem*{Var}{Varchenko's conjecture}

\theoremstyle{remark}
\newtheorem{remark}[theorem]{Remark}
\newtheorem{example}[theorem]{Example}
\newtheorem*{example*}{Example}
\newtheorem*{remark*}{Remark}

\usepackage{ifpdf}
\ifpdf
  \usepackage{graphicx}
  \usepackage{xcolor}
\else
  \usepackage[dvipdfmx]{graphicx}
  \usepackage[dvipdfmx]{xcolor}
\fi


\begin{document}

\title[$h$-vectors of matroids and logarithmic concavity]{$h$-vectors of matroids and logarithmic concavity}
\author{June Huh}
\email{junehuh@umich.edu}
\address{Department of Mathematics, University of Michigan\\
Ann Arbor, MI 48109 \\ USA}
\classification{05B35, 52C35}
\keywords{matroid, hyperplane arrangement, $f$-vecot, $h$-vector, log-concavity, characteristic polynomial.}
\begin{abstract}
Let $M$ be a matroid on $E$, representable over a field of characteristic zero. We show that $h$-vectors of the following simplicial complexes are log-concave:
\begin{enumerate}[1.]
\item The matroid complex of independent subsets of $E$.
\item The broken circuit complex of $M$ relative to an ordering of $E$.
\end{enumerate}
The first implies a conjecture of Colbourn on the reliability polynomial of a graph, and the second implies a conjecture of Hoggar on the chromatic polynomial of a graph.
The proof is based on the geometric formula for the characteristic polynomial of Denham, Garrousian, and Schulze.
\end{abstract}

\maketitle 

\section{Introduction and results}

A sequence $e_0,e_1,\ldots,e_n$ of integers is said to be 
\emph{log-concave} if for all $0< i< n$,
\[
e_{i-1}e_{i+1}\le e_i^2,
\]
and is said to have \emph{no internal zeros} if there do not exist $i < j <  k$ satisfying 
\[
e_i \neq 0, \quad e_j=0, \quad e_k\neq 0.
\] 
Empirical evidence has suggested that many important enumerative sequences are log-concave, but proving the log-concavity can sometimes be a non-trivial task. See \cite{Brenti,Stanley,Stanley00} for a wealth of examples arising from algebra, geometry, and combinatorics. The purpose of this paper is to demonstrate the use of an algebro-geometric tool to the log-concavity problems.

Let $X$ be a complex algebraic variety. A subvariety of $X$ is an irreducible closed algebraic subset of $X$. 
If $V$ is a subvariety of $X$, then the top dimensional homology group $H_{2\dim(V)}(V;\mathbb{Z}) \simeq \mathbb{Z}$ has a canonical generator, and the closed embedding of $V$ in $X$ determines a homomorphism
\[
H_{2\dim(V)}(V;\mathbb{Z}) \longrightarrow H_{2\dim(V)}(X;\mathbb{Z}).
\]
The image of the generator is called the \emph{fundamental class} of $V$ in $X$, denoted $[V]$. A homology class in $H_*(X;\mathbb{Z})$ is said to be \emph{representable} if it is the fundamental class of a subvariety. 

Hartshorne asks in \cite[Question 1.3]{HartshorneSurvey} 
which even dimensional homology classes of $X$ are representable by a smooth subvariety. Although the question is exceedingly difficult in general, it has a simple partial answer when $X$ is the product of complex projective spaces  $\mathbb{P}^m \times \mathbb{P}^n$. Note in this case that the $2k$-dimensional homology group of $X$ is freely generated by the classes of subvarieties of the form $\mathbb{P}^{k-i} \times \mathbb{P}^{i}$. 


Representable homology classes of $\mathbb{P}^m \times \mathbb{P}^n$ can be characterized numerically as follows \cite[Theorem 20]{Huh}.


\begin{theorem}\label{LC}
Write $\xi \in H_{2k}(\mathbb{P}^m \times \mathbb{P}^n;\mathbb{Z})$ as the integral linear combination
\[
\xi =\sum_{i} e_i \big[\mathbb{P}^{k-i} \times \mathbb{P}^{i}\big].
\]
\begin{enumerate}
\item If $\xi$ is an integer multiple of either
\[
\big[\mathbb{P}^m \times \mathbb{P}^n\big], \big[\mathbb{P}^m \times \mathbb{P}^0\big], \big[\mathbb{P}^0 \times \mathbb{P}^n\big], \big[\mathbb{P}^0 \times \mathbb{P}^0\big],
\]
then $\xi$ is representable if and only if the integer is $1$.
\item If otherwise, some positive integer multiple of $\xi$ is representable if and only if the $e_i$ form a nonzero log-concave sequence of
nonnegative integers with no internal zeros.
\end{enumerate}
\end{theorem}

In short, subvarieties of $\mathbb{P}^m \times \mathbb{P}^n$ correspond to log-concave sequences of nonnegative integers with no internal zeros. Therefore, when trying to prove the log-concavity of a sequence, it is reasonable to look for a subvariety of $\mathbb{P}^m \times \mathbb{P}^n$ which witnesses this property. We demonstrate this method by proving the log-concavity of $h$-vectors of two simplicial complexes associated to a matroid, when the matroid is representable over a field of characteristic zero. Other illustrations can be found in \cite{Huh,Huh-Katz,Lenz}.

In order to fix notations, we recall from \cite{Bjorner3} some basic definitions on simplicial complexes associated to a matroid. We use Oxley's book as our basic reference on matroid theory \cite{Oxley}. 

Let $\Delta$ be an abstract simplicial complex of dimension $r$. The \emph{$f$-vector} of $\Delta$ is a sequence of integers $f_0,f_1,\ldots,f_{r+1}$, where
\[
f_i = \big(\text{the number of $(i-1)$-dimensional faces of $\Delta$}\big).
\]
For example, $f_0$ is one, $f_1$ is the number of vertices of $\Delta$, and $f_{r+1}$ is the number of facets of $\Delta$. The \emph{$h$-vector} of $\Delta$ is defined from the $f$-vector by the polynomial identity
\[
\sum_{i=0}^{r+1} f_i (q-1)^{r+1-i}= \sum_{i=0}^{r+1} h_i q^{r+1-i}.
\]
When there is a need for clarification, we write the coefficients by $f_i(\Delta)$ and $h_i(\Delta)$ respectively.

Let $M$ be a matroid of rank $r+1$ on an ordered set $E$ of cardinality $n+1$. We are interested in the $h$-vectors of the following simplicial complexes associated to $M$:
\begin{enumerate}[1.]
\item The matroid complex $\text{IN}(M)$, the collection of subsets of $E$ which are independent in $M$.
\item The broken circuit complex $\text{BC}(M)$, the collection of subsets of $E$ which do not contain any broken circuit of $M$. 
\end{enumerate}
Recall that a \emph{broken circuit} is a subset of $E$ obtained from a circuit of $M$ by deleting the least element relative to the ordering of $E$.
We note that the isomorphism type of the broken circuit complex does depend on the ordering of $E$. However, the results of this paper will be independent of the ordering of $E$.

\begin{remark}
A pure $r$-dimensional simplicial complex is said to be \emph{shellable} if there is an ordering of its facets
such that each facet intersects the complex generated by its predecessors in a pure $(r-1)$-dimensional complex.
$\text{IN}(M)$ and $\text{BC}(M)$ are pure of dimension $r$, and are shellable. As a consequence, the $h$-vectors of both complexes consist of nonnegative integers \cite{Bjorner3}. This nonnegativity is recovered in Theorem \ref{main} below.
\end{remark}


Dawson conjectured that the $h$-vector of a matroid complex is a log-concave sequence \cite[Conjecture 2.5]{Dawson}. Our main result verifies this conjecture for matroids representable over a field of characteristic zero.

\begin{theorem}\label{main}
Let $M$ be a matroid representable over a field of characteristic zero.
\begin{enumerate}
\item  The $h$-vector of the matroid complex of $M$ is a log-concave sequence of nonnegative integers with no internal zeros.
\item The $h$-vector of the broken circuit complex of $M$ is a log-concave sequence of nonnegative integers with no internal zeros.
\end{enumerate}
\end{theorem}

Indeed, as we explain in the following section, there is a subvariety of a product of projective spaces which witnesses the validity of Theorem \ref{main}. 





It can be shown that the log-concavity of the $h$-vector implies the strict log-concavity of the $f$-vector:
\[
f_{i-1}f_{i+1}<f_i^2, \qquad i=1,2,\ldots,r.
\]
See  \cite[Lemma 5.1]{Lenz}. Therefore Theorem \ref{main} implies that the two $f$-vectors associated to $M$ are strictly log-concave.
The first statement of the following corollary recovers \cite[Theorem 1.1]{Lenz}.

\begin{corollary}\label{f-vector}
Let $M$ be a matroid representable over a field of characteristic zero.
\begin{enumerate}
\item  The $f$-vector of the matroid complex of $M$ is a strictly log-concave sequence of nonnegative integers with no internal zeros.
\item The $f$-vector of the broken circuit complex of $M$ is a strictly log-concave sequence of nonnegative integers with no internal zeros.
\end{enumerate}
\end{corollary}

The main special cases of Theorem \ref{main} and Corollary \ref{f-vector} are treated in the following subsections.

\begin{remark}
A pure simplicial complex is a matroid complex if and only if every ordering of the vertices induces a shelling \cite[Theorem 7.3.4]{Bjorner3}. 
In view of this characterization of matroids, one should contrast Theorem \ref{main} with examples of other `nice' shellable simplicial complexes whose $f$-vector and $h$-vector fail to be log-concave. In fact, the unimodality of the $f$-vector already fails for simplicial polytopes in dimension $\ge 20$ \cite{Billera-Lee,Bjorner1}. 

These shellable simplicial complexes led to suspect that various log-concavity conjectures on matroids might not be true in general \cite{Stanley00,Wagner}.
Theorem \ref{main} shows that there is a qualitative difference between the $h$-vectors of
\begin{enumerate}[1.]
\item matroid complexes and other shellable simplicial complexes, and/or 
\item matroids representable over a field and matroids in general.
\end{enumerate}
See \cite{Stan} and \cite{StanCM} for characterizations of $h$-vectors of simplicial polytopes and, respectively, shellable simplicial complexes in general. We note that the method of the present paper to prove the log-concavity crucially depends on the assumption that the matroid is representable over a field.
\end{remark}

\subsection{The reliability polynomial of a graph}

The \emph{reliability} of a connected graph $G$ is the probability that the graph remains connected when each edge is independently removed with the same probability $1-p$. If the graph has $e$ edges and $v$ vertices, then the reliability of $G$ is the polynomial
\[
\text{Rel}_G(p) = \sum_{i=0}^{e-v+1} f_i \hspace{0.5mm} p^{e-i}(1-p)^i,
\]
where $f_i$ is the number of cardinality $i$ sets of edges whose removal does not disconnect $G$. For example, $f_0$ is one, $f_1$ is the number of edges of $G$ that are not isthmuses, and $f_{e-v+1}$ is the number of spanning trees of $G$. The \emph{$h$-sequence} of the reliability polynomial is the sequence $h_i$ defined by the expression
\[
\text{Rel}_G(p) = p^{v-1} \Bigg(\sum_{i=0}^{e-v+1} h_i \hspace{0.5mm} (1-p)^i\Bigg).
\] 
In other words, the $h$-sequence is the $h$-vector of the matroid complex of the cocycle matroid of $G$.
Since the cocycle matroid of a graph is representable over every field, Theorem \ref{main} confirms a conjecture of Colbourn
that the $h$-sequence of the reliability polynomial of a graph is log-concave \cite{Colbourn}.

\begin{corollary}\label{Colbourn's conjecture}
The $h$-sequence of the reliability polynomial of a connected graph is a log-concave sequence of nonnegative integers with no internal zeros.
\end{corollary}

It has been suggested that Corollary \ref{Colbourn's conjecture} has practical applications in combinatorial reliability theory \cite{Brown-Colbourn}.

\subsection{The chromatic polynomial of a graph}

The \emph{chromatic polynomial} of a graph $G$ is the polynomial defined by
\[
\chi_G(q) = (\text{the number of proper colorings of $G$ using $q$ colors}). 
\]
The chromatic polynomial depends only on the cycle matroid of the graph, up to a factor of the form $q^c$. More precisely, the absolute value of the $i$-th coefficient of the chromatic polynomial is the number of cardinality $i$ sets of edges which contain no broken circuit \cite{Whitney}.
Since the cycle matroid of a graph is representable over every field, Corollary \ref{f-vector} confirms a conjecture of Hoggar that the coefficients of the chromatic polynomial of a graph form a strictly log-concave sequence \cite{Hoggar}. 
 
\begin{corollary}\label{sub}
The coefficients of the chromatic polynomial of a graph form a sign-alternating strictly log-concave
sequence of integers with no internal zeros.
\end{corollary}

Corollary \ref{sub} has been previously verified for all graphs with $\le 11$ vertices \cite{Lundow-Markstrom}.

 



\section{Proof of Theorem \ref{main}}

We shall assume familiarity with the M\"obius function $\mu(x,y)$ of the lattice of flats $\mathscr{L}_M$. For this and more, we refer to \cite{Aigner,Zaslavsky}. An important role will be played by the \emph{characteristic polynomial} $\chi_M(q)$. For a loopless matroid $M$, the characteristic polynomial is defined from $\mathscr{L}_M$ by the formula
\[
\chi_M(q) = \sum_{x \in \mathscr{L}_M} \mu(\varnothing, x) q^{r+1 - \text{rank} (x)}=\sum_{i=0}^{r+1} (-1)^i w_i q^{r+1-i}.
\]
If $M$ has a loop, then $\chi_M(q)$ is defined to be the zero polynomial. The nonnegative integers $w_i$ are called the \emph{Whitney numbers of the first kind}. The characteristic polynomial is always divisible by $q-1$, defining the \emph{reduced characteristic polynomial}
\[
\overline{\chi_M}(q)=\chi_M(q)/(q-1).
\]

\subsection{Brylawski's theorem I}

We need to quote a few results from Brylawski's analysis on the broken circuit complex \cite{Brylawski}. 
The first of these says that the Whitney number $w_i$ is the number of cardinality $i$ subsets of $E$ which contain no broken circuit relative to any fixed ordering of $E$ \cite[Theorem 3.3]{Brylawski}.
This observation goes back to Hassler Whitney, who stated it for graphs \cite{Whitney}.

Fix an ordering of $E$, and let $0$ be the smallest element of $E$. We write $\overline{\text{BC}}(M)$ for the \emph{reduced broken circuit complex} of $M$, the family of all subsets of $E \setminus \{0\}$ that do not contain any broken circuit of $M$. Since the broken circuit complex is the cone over $\overline{\text{BC}}(M)$ with apex $0$, the above quoted fact says that
\[
\overline{\chi_M}(q)= \sum_{i=0}^r (-1)^if_i\big(\overline{\text{BC}}(M)\big) q^{r-i}.
\]
In terms of the $h$-vector, we have
\[
\overline{\chi_M}(q+1)= \sum_{i=0}^r (-1)^{i} h_i\big(\overline{\text{BC}}(M)\big) q^{r-i}=\sum_{i=0}^r (-1)^{i} h_i\big(\text{BC}(M)\big) q^{r-i}, \quad h_{r+1}\big(\text{BC}(M)\big)=0.
\]
Therefore the second assertion of Theorem \ref{main} is equivalent to the statement that the coefficients of $\overline{\chi_M}(q+1)$ form a sign-alternating log-concave sequence with no internal zeros. 

\subsection{Brylawski's theorem II}

We show that the first assertion of Theorem \ref{main} is implied by the second. This follows from the fact that the matroid complex of $M$ is the reduced broken circuit complex of the free dual extension of $M$ \cite[Theorem 4.2]{Brylawski}. We note that not every reduced broken circuit complex can be realized as a matroid complex \cite[Remark 4.3]{Brylawski}. The second assertion of Theorem \ref{main} is strictly stronger than the first in this sense.


Recall that the \emph{free dual extension} of $M$ is defined by taking the dual of $M$, placing a new element $p$ in general position (taking the free extension), and again taking the dual. In symbols,
\[
M \times p := (M^* + p)^*.
\]
If $M$ is representable over a field, then $M \times p$ is representable over some finite extension of the same field. 
Choose an ordering of $E \cup \{p\}$ such that $p$ is smaller than any other element.
Then, with respect to the chosen ordering,
\[
\text{IN}(M)=\overline{\text{BC}}(M \times  p).
\]
For more details on the free dual extension, see \cite{Brylawski,BrylawskiBook,Lenz}.


\subsection{Reduction to simple matroids}

A standard argument shows that it is enough to prove the assertion on $\overline{\chi_M}(q+1)$ when $M$ is simple:

\begin{enumerate}[1.]
\item If $M$ has a loop, then the reduced characteristic polynomial of $M$ is zero, so there is nothing to show in this case.
\item If $M$ is loopless but has parallel elements, replace $M$ by its \emph{simplification} $\overline{M}$ as defined in \cite[Section 1.7]{Oxley}. Then 
the reduced characteristic polynomials of $M$ and $\overline{M}$ coincide 
because $\mathscr{L}_M \simeq \mathscr{L}_{\overline{M}}$. 
\end{enumerate}
Hereafter $M$ is assumed to be simple of rank $r+1$ with $n+1$ elements, representable over a field of characteristic zero.

\subsection{Reduction to complex hyperplane arrangements}

We reduce the main assertion to the case of essential arrangements of affine hyperplanes.
We use the book of Orlik and Terao as our basic reference in hyperplane arrangements \cite{Orlik-Terao}.

Note that the condition of representability for matroids of given rank and given number of elements can be expressed in a first-order sentence in the language of fields.
Since the theory of algebraically closed fields of characteristic zero is complete \cite[Corollary 3.2.3]{Marker},
a matroid representable over a field of characteristic zero is in fact representable over $\mathbb{C}$.

Let $\widetilde{\mathcal{A}}$ be a central arrangement of $n+1$ distinct hyperplanes in $\mathbb{C}^{r+1}$ representing $M$. 
This means that there is a bijective correspondence between $E$ and the set of hyperplanes of $\widetilde{\mathcal{A}}$ which identifies the geometric lattice $\mathscr{L}_M$ with the lattice of flats of $\widetilde{\mathcal{A}}$.
Choose any one hyperplane from the projectivization of $\widetilde{\mathcal{A}}$ in $\mathbb{P}^r$. The \emph{decone} of the central arrangement, denoted $\mathcal{A}$, is the essential arrangement of $n$ hyperplanes in $\mathbb{C}^r$ obtained by declaring the chosen hyperplane to be the hyperplane at infinity.
If $\chi_{\mathcal{A}}(q)$ is the characteristic polynomial of the decone, then 
\[
\chi_{\mathcal{A}}(q)=\overline{\chi_M}(q).
\]
Therefore it suffices to prove that the coefficients of $\chi_\mathcal{A}(q+1)$ form a sign-alternating log-concave sequence of integers with no internal zeros.

\subsection{The variety of critical points}

Finally, the geometry comes into the scene. We are given an essential arrangement $\mathcal{A}$ of $n$ affine hyperplanes in $\mathbb{C}^r$. Our goal is find a subvariety of a product of projective spaces, whose fundamental class encodes the coefficients of the translated characteristic polynomial $\chi_\mathcal{A}(q+1)$. 

The choice of the subvariety is suggested by an observation of Varchenko on the critical points of the master function of an affine hyperplane arrangement \cite{Varchenko}. 
Let $L_1,\ldots,L_n$ be the linear functions defining the hyperplanes of $\mathcal{A}$. A \emph{master function} of $\mathcal{A}$ is a nonvanishing holomorphic function defined on the complement $\mathbb{C}^r \setminus \mathcal{A}$ as the product of powers
\[
\varphi_{\bf u}:=\prod_{i=1}^n L_i^{u_i}, \qquad \mathbf{u}=(u_1,\ldots,u_n) \in \mathbb{Z}^n.
\]
 


\begin{Var}
If the exponents $u_i$ are sufficiently general, then all critical points of $\varphi_{\bf u}$ are nondegenerate, and
the number of critical points is equal to $(-1)^r \chi_\mathcal{A}(1)$.
\end{Var}

Note that $(-1)^r \chi_\mathcal{A}(1)$ is equal to the number of bounded regions in the complement $\mathbb{R}^r \setminus \mathcal{A}$ when $\mathcal{A}$ is defined over the real numbers, and to the signed topological Euler characteristic of the complement $\mathbb{C}^r \setminus \mathcal{A}$. The conjecture is proved by Varchenko in the real case \cite{Varchenko}, and by Orlik and Terao in general \cite{Orlik-Terao}.


In order to encode all the coefficients of $\chi_\mathcal{A}(q+1)$ in an algebraic variety, we consider the totality of critical points of all possible (multivalued) master functions of $\mathcal{A}$. More precisely, we define the \emph{variety of critical points} $\mathfrak{X}(\mathcal{A})$ as the closure
\[
\mathfrak{X}(\mathcal{A})=\overline{\mathfrak{X}^\circ(\mathcal{A})} \subseteq \mathbb{P}^r \times \mathbb{P}^{n-1}, \qquad \mathfrak{X}^\circ(\mathcal{A})=\Bigg\{\sum_{i=1}^n u_i \cdot \text{dlog}(L_i)(x)=0 \Bigg\} \subseteq (\mathbb{C}^r \setminus \mathcal{A}) \times \mathbb{P}^{n-1},
\]
where $\mathbb{P}^{n-1}$ is the projective space with the homogeneous coordinates $u_1,\ldots,u_n$. 
The variety of critical points first appeared implicitly in \cite{Orlik-Terao2}, and further studied in \cite{Cohen-Denham-Falk-Varchenko,Denham-Garrousian-Schulze}.
See also \cite[Section 2]{HuhML}.

The variety of critical points is irreducible because $\mathfrak{X}^\circ(\mathcal{A})$ is a projective space bundle over the complement $\mathbb{C}^r \setminus \mathcal{A}$.
The cardinality of a general fiber of the second projection 
\[
\text{pr}_2 : \mathfrak{X}(\mathcal{A}) \longrightarrow \mathbb{P}^{n-1}
\] 
is equal to $(-1)^r\chi_\mathcal{A}(1)$, as stated in Varchenko's conjecture. More generally, we have
\[
\big[\mathfrak{X}(\mathcal{A})\big] =
\sum_{i=0}^{r} v_i \big[ \mathbb{P}^{r-i} \times \mathbb{P}^{n-1-r+i}\big] \in H_{2n-2}(\mathbb{P}^r \times \mathbb{P}^{n-1};\mathbb{Z}),
\]
where $v_i$ are the coefficients of the characteristic polynomial
\[
\chi_\mathcal{A}(q+1)=\sum_{i=0}^{r} (-1)^i v_i \hspace{0.7mm} q^{r-i}.
\]
The previous statement is \cite[Corollary 3.11]{HuhML}, which is essentially the geometric formula for the characteristic polynomial of Denham, Garrousian, and Schulze \cite[Theorem 1.1]{Denham-Garrousian-Schulze}, modulo a minor technical difference pointed out in \cite[Remark 2.2]{HuhML}. 
A conceptual proof of the geometric formula can be summarized as follows \cite[Section 3]{HuhML}:

\begin{enumerate}[1.]
\item Applying a logarithmic version of the Poincar\'e-Hopf theorem to a compactification of the complement $\mathbb{C}^r \setminus \mathcal{A}$, one shows that the fundamental class of the variety of critical points captures the characteristic class of $\mathbb{C}^r \setminus \mathcal{A}$.
\item The characteristic class of $\mathbb{C}^r \setminus \mathcal{A}$ agrees with the characteristic polynomial $\chi_\mathcal{A}(q+1)$, because the two are equal at $q=0$ and satisfy the same inclusion-exclusion formula.
\end{enumerate}
See \cite[Section 3]{Denham-Garrousian-Schulze} for a more geometric approach. 

The proof of Theorem \ref{main} is completed by applying Theorem \ref{LC} to the fundamental class of the variety of critical points of $\mathcal{A}$.
\qed


Simple examples show that equalities may hold throughout in the inequalities of Theorem \ref{main}. For example, if $M$ is the uniform matroid of rank $r+1$ with $r+2$ elements, then
\[
h_i\big(\text{IN}(M)\big) = h_i\big(\text{BC}(M)\big)  =1, \qquad i=1,\ldots,r.
\]
However, a glance at the list of $h$-vectors of small matroid complexes generated in \cite{DeLoera-Kemper-Klee} suggests that there are stronger conditions on the $h$-vectors
than those that are known or conjectured. The answer to the interrogative title of \cite{Wilf} seems to be out of reach at the moment.

\begin{thebibliography}{CDFV12}

\bibitem[Aig87]{Aigner} Martin Aigner,
			\emph{Whitney numbers}, Combinatorial geometries, 139--160, 
			Encyclopedia of Mathematics and its Applications {\bf 29}, Cambridge University Press, Cambridge, 1987. 
			
\bibitem[BL81]{Billera-Lee} Louis Billera and Carl Lee,
			\emph{A proof of the sufficiency of McMullen's conditions for f-vectors of simplicial convex polytopes}, 
			Journal of Combinatorial Theory Series A {\bf 31} (1981), 237--255. 

\bibitem[Bjo81]{Bjorner1} Anders Bj\"orner,
			\emph{The unimodality conjecture for convex polytopes},
			Bulletin of the American Mathematical Society {\bf 4} (1981), 187--188.

\bibitem[Bjo92]{Bjorner3} Anders Bj\"orner,
			\emph{The homology and shellability of matroids and geometric lattices},
			Matroid Applications, 226-283, 
			Encyclopedia Mathematics and its Applications {\bf 40}, Cambridge University Press, Cambridge, 1992.
			

\bibitem[Bre94]{Brenti} Francesco Brenti, 
			\emph{Log-concave and unimodal sequences in algebra, combinatorics, and geometry: an update},
			Jerusalem Combinatorics '93, 71--89,
			Contemporary Mathematics {\bf 178}, American Mathematical Society, Providence, 1994.

\bibitem[BC94]{Brown-Colbourn} Jason Brown and Charles Colbourn, 
			\emph{On the log concavity of reliability and matroidal sequences},
			Advances in Applied Mathematics {\bf 15} (1994), 114--127. 			
			
\bibitem[Bry77]{Brylawski} Thomas Brylawski,
			\emph{The broken-circuit complex},
			Transactions of the American Mathematical Society {\bf 234} (1977),
			417--433.

\bibitem[Bry86]{BrylawskiBook} Thomas Brylawski, 
			\emph{Constructions}, Theory of Matroids, 127--223, 
			Encyclopedia Mathematics and its Applications {\bf 26}, Cambridge University Press, Cambridge, 1986. 

\bibitem[CDFV12]{Cohen-Denham-Falk-Varchenko} Daniel Cohen, Graham Denham, Michael Falk, and Alexander Varchenko,
			\emph{Critical points and resonance of hyperplane arrangements},
			Canadian Journal of Mathematics {\bf 63} (2011), 1038--1057.

\bibitem[Col87]{Colbourn} Charles Colbourn,
			\emph{The Combinatorics of Network Reliability},
			International Series of Monographs on Computer Science, 
			The Clarendon Press, Oxford University Press, New York, 1987.

\bibitem[Daw84]{Dawson} Jeremy Dawson,
			\emph{A collection of sets related to the Tutte polynomial of a matroid},
			Graph theory, Singapore 1983, 193--204, Lecture Notes in Mathematics {\bf 1073}, Springer, Berlin, 1984.

\bibitem[DKK11]{DeLoera-Kemper-Klee} Jesus De Loera, Yvonne Kemper, and Steven Klee,
			\emph{$h$-vectors of small matroid complexes},
			2011, \texttt{arXiv:1106.2576}.

\bibitem[DGS12]{Denham-Garrousian-Schulze} Graham Denham, Mehdi Garrousian, and Mathias Schulze,
			\emph{A geometric deletion-restriction formula}, 
			Advances in Mathematics {\bf 230} (2012), 1979--1994.
			



\bibitem[Har74]{HartshorneSurvey} Robin Hartshorne, 
			\emph{Varieties of small codimension in projective space}, 
			Bulletin of the American Mathematical Society {\bf 80} (1974), 1017--1032.


\bibitem[Hog74]{Hoggar} Stuart Hoggar,
			\emph{Chromatic polynomials and logarithmic concavity},
			Journal of Combinatorial Theory Series B {\bf 16} (1974), 248--254.

\bibitem[Huh12a]{Huh} June Huh,
			\emph{Milnor numbers of projective hypersurfaces and the chromatic polynomial of graphs},
			Journal of the American Mathematical Society {\bf 25} (2012), 907--927.

\bibitem[Huh12b]{HuhML} June Huh,
			\emph{The maximum likelihood degree of a very affine variety},
			2012, \texttt{arXiv:1207.0553}.

\bibitem[HK12]{Huh-Katz} June Huh and Eric Katz,
			\emph{Log-concavity of characteristic polynomials and the Bergman fan of matroids},
			Mathematische Annalen, to appear.

\bibitem[Len12]{Lenz} Matthias Lenz,
			\emph{The f-vector of a realizable matroid complex is strictly log-concave},
			Combinatorics, Probability, and Computing, to appear.

\bibitem[LM06]{Lundow-Markstrom} Per H\aa kan Lundow and Klas Markstr\"om,
			\emph{Broken-cycle-free subgraphs and the log-concavity conjecture for chromatic polynomials},
			Experimental Mathematics {\bf 15} (2006), 343--353.

\bibitem[Mar02]{Marker} David Marker,
			\emph{Model Theory. An Introduction},
			Graduate Texts in Mathematics {\bf 217},
			Springer-Verlag, New York, 2002.
			

\bibitem[OT92]{Orlik-Terao} Peter Orlik and Hiroaki Terao, 
			\emph{Arrangements of Hyperplanes}, 
			Grundlehren der Mathematischen Wissenschaften {\bf 300}, Springer-Verlag, Berlin, 1992.

\bibitem[OT95]{Orlik-Terao2} Peter Orlik and Hiroaki Terao, 
			\emph{The number of critical points of a product of powers of linear functions}, 
			Inventiones Mathematicae {\bf 120} (1995), 1--14.
						
\bibitem[Oxl11]{Oxley} James Oxley, 
			\emph{Matroid theory},
			Second edition, Oxford Graduate Texts in Mathematics {\bf 21}, Oxford University Press, Oxford, 2011.


\bibitem[Sta77]{StanCM} Richard Stanley,
			\emph{Cohen-Macaulay complexes}, Higher combinatorics, 
			51--62, NATO Advanced Science Institutes Series C: Mathematical and Physical Sciences {\bf 31}, Reidel, Dordrecht, 1977. 

\bibitem[Sta80]{Stan} Richard Stanley,
			\emph{The number of faces of a simplicial convex polytope},
			Advances in Mathematics {\bf 35} (1980), 236--238.
						
\bibitem[Sta89]{Stanley} Richard Stanley, 
			\emph{Log-concave and unimodal sequences in algebra, combinatorics, and geometry}, 
			Graph Theory and Its Applications: East and West (Jinan 1986), 500--535,
			Annals of New York Academy of Sciences {\bf 576}, 1989.
			
\bibitem[Sta00]{Stanley00} Richard Stanley,
			\emph{Positivity problems and conjectures in algebraic combinatorics}, 
			Mathematics: Frontiers and Perspectives, 295Ð-319, American Mathematical Society, Providence, 2000. 

\bibitem[Var95]{Varchenko} Alexander Varchenko, 
			\emph{Critical points of the product of powers of linear functions and families of bases of singular vectors},
			 Compositio Mathematica {\bf 97} (1995), 385--401.
			 			
\bibitem[Wag08]{Wagner} David Wagner,
			\emph{Negatively correlated random variables and Mason's conjecture for independent sets in matroids}, 
			Annals of Combinatorics {\bf 12} (2008), 211-Ð239. 

			
\bibitem[Whi32]{Whitney} Hassler Whitney,
			\emph{A logical expansion in mathematics},
			Bulletin of the American Mathematical Society {\bf 38} (1932), 572--579.
			
\bibitem[Wil76]{Wilf} Herbert Wilf,
			\emph{Which polynomials are chromatic?},
			Colloquio Internazionale sulle Teorie Combinatorie (Roma, 1973), Tomo I, 247--256, 
			Atti dei Convegni Lincei {\bf 17}, 
			Accademia Nazionale dei Lincei, Rome, 1976. 

\bibitem[Zas87]{Zaslavsky} Thomas Zaslavsky,
			\emph{The M\"obius function and the characteristic polynomial}, Combinatorial geometries, 114--138, 
			Encyclopedia of Mathematics and its Applications {\bf 29}, Cambridge University Press, Cambridge, 1987. 
			

\end{thebibliography}

\end{document}


\newpage

\begin{proof}
Note that $M$ is representable over an algebraically closed field of characteristic zero. By the first-order nature of the assumption on $M$, we may assume without loss of generality that this field is the field of complex numbers $\mathbb{C}$. See, for example, the proof of \cite[Corollary 27]{Huh}.

We first prove the second assertion on the coefficients of $\overline{\chi_M}(q+1)$. Since the characteristic polynomial depends only on the lattice of flats, we may assume furthermore that $M$ has no loops nor parallel edges. In this case, $M$ is represented by a central arrangement of $(n+1)$ hyperplanes in $\mathbb{C}^{r+1}$. Declaring the hyperplane at infinity in the corresponding projective arrangement in $\mathbb{P}^r$, we have an essential arrangement of $n$ affine hyperplanes in $\mathbb{C}^r$. Denote this affine arrangement by $\mathcal{A}$. If $\chi_\mathcal{A}(q)$ is the characteristic polynomial of the affine arrangement \cite[Definition 2.52]{Orlik-Terao}, we have
\[
\chi_\mathcal{A}(q+1)=\overline{\chi_M}(q+1).
\]

The plan is to apply Theorem 1 to the variety of critical points of $\mathcal{A}$ studied in \cite{Denham-Garrousian-Schulze}. We briefly introduce the variety and its basic properties, using the notation and convention of \cite{HuhML}. Our goal is to state a geometric formula for the characteristic polynomial of $\mathcal{A}$, which is \cite[Theorem 1.1]{Denham-Garrousian-Schulze} modulo minor technical differences.




Let $f_1,\ldots,f_d \in V^*$ be linear forms defining the hyperplanes $H_1,\ldots,H_d$ of $\mathcal{A}$, and let $V^\circ$ be the complement $V \setminus \bigcup_{i=1}^d H_i$. In the course of study of critical points of the master function
\[
\Phi_\mathbf{a} = \prod_{i=1}^d f_i^{a_i},\qquad \mathbf{a}=(a_1,\ldots,a_d) \in \mathbb{C}^d,
\]
one is led to consider the variety $\Sigma(\mathcal{A}) \subset V^\circ \times \mathbb{C}^d$ defined by the vanishing of the $1$-form
\[
\omega_\mathbf{a} = \sum_{i=1}^d a_i \frac{df_i}{f_i}.
\]
Since $\Sigma(\mathcal{A})$ is preserved under the diagonal actions of $\mathbb{C}^*$ on $V$ and $\mathbb{C}^d$, the same is true for its closure $\overline{\Sigma}(\mathcal{A})$ in $V \times \mathbb{C}^d$. The variety of critical points is obtained by taking the quotient
\[
\mathfrak{X}(\mathcal{A}) := \overline{\Sigma}(\mathcal{A})/(\mathbb{C}^* \times \mathbb{C}^*) \subset \mathbb{P}V \times \mathbb{P}^{d-1}.
\]
Then $\mathfrak{X}(\mathcal{A})$ is either empty (when $\mathcal{A}$ is the Boolean arrangement) or irreducible of codimension $r<d$ \cite[Corollary 2.10]{Cohen-Denham-Falk-Varchenko}. An amazing observation of \cite{Denham-Garrousian-Schulze} is that the construction of the vareity of critical points is compatible with the process of deletion-restriction. Numerically, the result shows that the cohomology class of $\mathfrak{X}(\mathcal{A})$ in the product of projective spaces is a Tutte-Grothendieck invariant. To be more precise, write the Chow ring of the ambient variety by
\[
A^*(\mathbb{P}V \times \mathbb{P}^{d-1}) \simeq \mathbb{Z}[h,k]/(h^r,k^d),
\]
where $h=[H]$ and $k=[K]$ denote the classes of hyperplanes $H$ and $K$ in $\mathbb{P}V$ and $\mathbb{P}^{d-1}$ respectively. From their geometric deletion-restriction formula \cite[Theorem 3.1]{Denham-Garrousian-Schulze}, Denham, Garrousian, and Schulze deduce the equality
\[
\big[\mathfrak{X}(\mathcal{A})\big] = \widetilde{\chi}_M(-h,k-h) \in A^*(\mathbb{P}V \times \mathbb{P}^{d-1})  
\]
where $\widetilde{\chi}_M(s,t)=s^r \chi_M(t/s)$ denotes the homogenized characteristic polynomial of $M$. Under the identification of the Chow group and the ring
\[
A_*(\mathbb{P}V \times \mathbb{P}^{d-1}) = A^*(\mathbb{P}V \times \mathbb{P}^{d-1}),
\]
we have
\[
\big[\mathfrak{X}(\mathcal{A})\big] =
\sum_{i=0}^{r-1} v_i \big[ \mathbb{P}^{r-1-i} \times \mathbb{P}^{d-r-1+i}\big]
\]
where
\[
\chi_M(q+1)=\sum_{i=0}^{r-1} (-1)^i v_i \hspace{0.7mm} q^{r-i}.
\]
Note that $\chi_M(1)=0$. Theorem 3 is now obtained from Theorem 1.

Next we deduce Theorem 2 from Theorem 3. The idea is that both the $h$-vector and the characteristic polynomial is a specialization of the Tutte polynomial (which behaves well under several natural constructions in matroid theory). More precisely, the link between the two specializations is provided by the free coextension $M^\dagger$ of $M$ \cite{Brylawski, BrylawskiBook, Lenz}. $M^\dagger$ is obtained from $M$ by first taking the dual, then adjoining an element in general position, and again taking the dual. In particular, if $M$ is representable over an infinite field, then $M^\dagger$ is representable over the same field. Define the $f$- and $h$-polynomial of a matroid $M$ by
\[
f_M(q)=\sum_{i=0}^r f_i q^{r-i}, \qquad h_M(q)=\sum_{i=0}^r h_i q^{r-i}.
\]
Then we have the relation \cite[Section 7.4]{BrylawskiBook}, \cite[Proposition 3.3]{Lenz}
\[
(-1)^{r+1}  \chi_{M^\dagger}(-q)= (1+q) f_M(q)
\]
which can be modified to give
\[
\hspace{1.5mm} \chi_{M^\dagger}(q+1)= (-1)^{r} q \hspace{1mm} h_M(-q).
\]
Since $M^\dagger$ is representable over $\mathbb{C}$, Theorem 3 applies to $\chi_{M^\dagger}(q+1)$. This shows that the $h$-vector of $M$ form a log-concave sequence of nonnegative integers with no internal zeros.
\end{proof}

\begin{proof}[Proof of Corollary \ref{sub}]
We may assume that $G$ has no loops and no parallel edges. In this case, cycles of $G$ defines a simple matroid $M$ on the set of vertices. If $G$ has $c$ connected components and $M$ is the cycle matroid of $G$, then
\[
\chi_G(q) = q^{c} \chi_{M}(q).
\]
See \cite[Theorem xx]{Zaslavsky}.  The cycle matroid $M$ of $G$ is representable over every field, and Theorem \ref{main} applies to $\chi_M(q)$. 

An elementary lemma of Lenz says that the log-concavity of $\chi_M(q+1)$ implies the strict log-concavity of $\chi_M(q)$ \cite[Lemma 5.1]{Lenz}. Therefore Theorem \ref{main} verifies the conjecture of Hoggar. (Remark on the log-concavity of $f$-vector.) (Cite \cite{Lundow-Markstrom}.)
\end{proof}

Simple examples show that equalities may hold throughout in Theorems \ref{main}. As an example, consider the rank $r$ uniform matroids $U_{r,r}$ and $U_{r,r+1}$ with $r$ and $r+1$ elements respectively. Then $h$-vectors of the matroid complexes are, and $h$-vectors of broken circuit complexes are
the first two rank $r$ uniform matroids have the characteristic polynomials and the h-polynomials
\begin{eqnarray*}
\chi_{U_{r,r}}(q+1)=q^r, &\quad& \chi_{U_{r,r+1}}(q+1)= q^r-q^{r-1}+\cdots+(-1)^{r-1}q, \\
\hspace{-2mm} h_{U_{r,r}}(q)  \hspace{2mm} = q^r, &\quad&  \hspace{4mm} h_{U_{r,r+1}}(q) \hspace{2.3mm} = q^r+q^{r-1}+\cdots+1.
\end{eqnarray*}



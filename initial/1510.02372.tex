
\documentclass[12pt]{amsart}
\usepackage{amsthm, amssymb}
\usepackage{amsfonts, color, amscd,mathabx}
\usepackage{epsfig,multicol}
\usepackage{xypic}
\setlength{\topmargin}{0cm} \setlength{\oddsidemargin}{0cm}
\setlength{\evensidemargin}{0cm} \setlength{\textheight}{ 22cm}
\setlength{\textwidth}{16cm}
\theoremstyle{plain} \numberwithin{equation}{section}
\newtheorem{thm}{Theorem}[section]
\newtheorem{cor}[thm]{Corollary}
\newtheorem{prop}[thm]{Proposition}
\newtheorem{lem}[thm]{Lemma}

\theoremstyle{definition}
\newtheorem{defn}{Definition}
\newtheorem{exam}[thm]{Example}
\theoremstyle{remark}
 \newtheorem{rem}{Remark}

 
 
 

\begin{document}
\title[Self-dual binary codes from small covers and simple polytopes]{\large \bf Self-dual binary codes from small covers and simple polytopes}
\author[Bo Chen, Zhi L\"u and Li Yu]{Bo Chen, Zhi L\"u and Li Yu}
\keywords{Self-dual code, polytope, small cover.}
 \subjclass[2010]{57S25, 94B05, 57M60, 57R91}
\thanks{Supported in part by grants from NSFC (No.\,11371093, No.\,11371188 and No.\,11431009).}

\address{School of Mathematics and Statistics, Huazhong University of Science and Technology, Wuhan, 430074, P. R.  China}
\email{bobchen@hust.edu.cn}
\address{School of Mathematical Sciences, Fudan University, Shanghai,
200433, P.R. China.} \email{zlu@fudan.edu.cn}
\address{Department of Mathematics and IMS, Nanjing University, Nanjing, 210093, P.R.China}
\email{yuli@nju.edu.cn}

\begin{abstract}
We explore the connection between simple polytopes and self-dual binary codes 
 via the theory of small covers. We first show that a small cover
$M^n$ over a simple $n$-polytope $P^n$ produces a self-dual code in the sense of Kreck--Puppe if and only if $P^n$ is $n$-colorable and $n$ is odd. 
Then we show how to describe such a self-dual binary code in terms of the combinatorics of $P^n$.
 Moreover, we construct a family of
 binary linear codes $\mathfrak{B}_k(P^n)$, $0\leq k\leq n$, for an arbitrary simple 
$n$-polytope $P^n$ and discuss when $\mathfrak{B}_k(P^n)$ is self-dual. 
 A spinoff of our investigation gives some
  new ways to judge whether a simple $n$-polytope 
  $P^n$ is $n$-colorable in terms of 
  the associated binary codes $\mathfrak{B}_k(P^n)$.
  In addition, we prove that the minimum distance of 
  the self-dual binary code obtained from a $3$-colorable simple $3$-polytope
  is always $4$. 

\end{abstract}

\maketitle

\section{Introduction}\label{int}

 A (linear) \emph{binary code} $C$ of length $l$ is a linear subspace of the $l$-dimensional linear space
${\Bbb F}_2^l$ over ${\Bbb F}_2$ (the binary field). 
 The \emph{Hamming weight} of an element
  $u=(u_1, ..., u_l)\in {\Bbb F}_2^l$, denoted by $wt(u)$,
   is the number of nonzero coordinates $u_i$ in $u$. 
   Any element of $C$ is called a \emph{codeword}.
   The \emph{Hamming distance} $d(u,v)$ of two codewords $u,v\in C$ is defined by:
   $$d(u,v)=wt(u-v).$$ 
  The minimum of the Hamming distances $d(u, v)$ for all $u, v\in C$ , $u\neq v$, is
called the \emph{minimum distance} of $C$ (which also equals 
 the minimum Hamming weight of nonzero elements in $C$). A 
 binary code $C\subset {\Bbb F}_2^l$ is called \emph{type} $[l,k,d]$ if $\dim_{{\Bbb F}_2} C =k$
 and the minimum distance of $C$ is $d$. We call two binary codes in ${\Bbb F}_2^l$
  \emph{equivalent} if they differ only by a permutation of coordinates.\vskip .1cm

 The standard bilinear form
    $\langle\ ,\, \rangle$ on ${\Bbb F}_2^l$ is defined by
$$\langle u, v\rangle :=\sum_{i=1}^l u_iv_i,\ 
 u=(u_1, ..., u_l), v=(v_1, ..., v_l)\in {\Bbb F}_2^l.$$ 
 Note that $\langle u , v\rangle = \frac{1}{2} \big(wt(u) + wt(v)-wt(u+v)\big) 
  \ \text{mod} \ 2$ for any $u,v\in {\Bbb F}_2^l$, and
  $$\langle u, u\rangle=\sum_{i=1}^l u_i,\ u=(u_1, ..., u_l)\in {\Bbb F}_2^l.$$
 Then any linear binary code $C$ in ${\Bbb F}_2^l$ has a \emph{dual code} $C^{\perp}$ defined by 
  $$C^\perp :=\{u\in {\Bbb F}_2^l\,|\, \langle u, c\rangle=0 \text{ for all $c\in C$}\}$$ 
  It is clear that $\dim_{{\Bbb F}_2} C + \dim_{{\Bbb F}_2}C^{\perp} =l$.
 We call $C$ {\em self-dual} if $C=C^\perp$. 
 For a self-dual binary code $C$, we can easy show the following
 \begin{itemize}
   \item The length $l=2\dim_{{\Bbb F}_2}C$ must be even; \vskip .1cm
   \item For any $u\in C$,
  the Hamming weight $wt(u)$ is an even integer since $\langle u, u \rangle=0$;\vskip .1cm
  \item The minimum distance of $C$ is an even integer.
  \end{itemize}
   
 Self-dual linear codes play an important role in coding theory and have been studied 
 extensively (see~\cite{rs} for a detailed survey). 
 \vskip .1cm

Puppe in~\cite{p} found an interesting connection between
 closed manifolds and self-dual binary codes.
  It was shown in~\cite{p}
  that an involution $\tau$ on an odd dimensional closed manifold $M$ with ``maximal number
   of isolated fixed points'' (i.e., with only isolated fixed
points and the number of fixed points 
   $|M^{\tau}|=\dim_{{\Bbb F}_2}(\bigoplus_i H^i(M;{\Bbb F}_2))$) determines a 
  self-dual binary code of length $|M^{\tau}|$. Such an involution $\tau$ is called an
    \emph{$\mathrm{m}$-involution}. Conversely,
    Kreck--Puppe~\cite{kp} proved a somewhat surprising theorem that any self-dual binary 
    code can be obtained from an $\mathrm{m}$-involution on some closed 
    $3$-manifold.   
   Hence it is an interesting problem for us to search
     $\mathrm{m}$-involutions on closed manifolds. But in practice
     it is very difficult to construct all possible $\mathrm{m}$-involutions 
    on a given manifold. \vskip .1cm
    
    
    On the other hand, Davis and Januszkiewicz in~\cite{dj} introduced a class of closed smooth
  manifolds $M^n$ with locally standard actions of elementary 2-group ${\Bbb Z}_2^n$, called  
   \emph{small covers}, whose orbit space is an
    $n$-dimensional simple convex polytope $P^n$ in ${\Bbb R}^n$. It was shown in~\cite{dj} that 
    many geometric and topological properties of $M^n$ can be explicitly 
    computed in terms of the combinatorics of $P^n$ and some characteristic function 
 on $P^n$ determined by the ${\Bbb Z}_2^n$-action. 
  For example, the mod 2 Betti numbers of $M^n$ correspond to the $h$-vector of $P^n$.
   Any nonzero element $g\in{\Bbb Z}_2^n$ determines a nontrivial involution on $M^n$, denoted by
   $\tau_g$. We call $\tau_g$ a \emph{regular involution} on the small cover. So 
   whenever $\tau_g$ is an $\mathrm{m}$-involution on $M^n$, 
    we obtain a self-dual binary code from $(M^n,\tau_g)$. \vskip .1cm
   
 
  Motivated by Kreck--Puppe and Davis--Januszkiewicz's work, our purpose in this paper is 
  to explore the connection between the theory of binary codes and 
  the combinatorics of simple polytopes via the topology of small covers.   
  We will show that a small cover $M^n$ over an $n$-dimensional simple polytope $P^n$ admits
  a regular $\mathrm{m}$-involution only when
  $P^n$ is $n$-colorable. 
  A polytope is \emph{$n$-colorable} if we can color all the facets (codimension-one faces) of the polytope
   by $n$ different colors so that any
  neighboring facets are assigned different colors.
  Moreover, we find that the self-dual binary code 
  obtained from a regular $\mathrm{m}$-involution on $M^n$ 
  depends only on the combinatorial structure of $P^n$.
 This motivates us to define a family of binary codes $\mathfrak{B}_k(P^n)$, 
 $0\leq k \leq n$ for any simple polytope $P^n$ (not necessarily $n$-colorable). 
 It turns out
 that a binary code of this type can be self-dual only when
 the corresponding simple polytope $P^n$ is $n$-colorable and $n$ is odd. 
\vskip .1cm

  
  The arrangement of this paper is as follows. In section 2, we first recall some basic facts of
  small covers and explain the procedure of obtaining self-dual binary codes in~\cite{p}
   from 
  $\mathrm{m}$-involutions on manifolds. In addition, we  
  investigate what kind of small covers can admit regular $\mathrm{m}$-involutions 
  (see Theorem~\ref{max-involution}). In section 3, we 
  spell out the self-dual binary code from a small cover with a regular 
  $\mathrm{m}$-involution (see Corollary~\ref{Cor:Main-2}).
  It turns out that the binary code we obtain depends only on the combinatorial structure of the 
  underlying simple polytope. Moreover, we find a combinatorial way to
   write a linear basis of the self-dual binary code (see Proposition~\ref{Prop:Code_Basis}).
   In section 4, we 
 study a family of binary codes $\mathfrak{B}_k(P^n)$, 
 $0\leq k \leq n$, associated to any simple $n$-polytope $P^n$. We 
 prove that $\mathfrak{B}_k(P^n)$ is self-dual if and only if $n$ is odd,
 $P^n$ is
 $n$-colorable and $k=\frac{n-1}{2}$ (see Theorem~\ref{Thm:Main-3}). This implies that
  self-dual binary codes that arise from $\mathfrak{B}_k(P^n)$ agree with those
  from small covers.
 In addition, a spinoff of our investigation gives some
  new ways to judge whether $P^n$ is $n$-colorable in terms of 
  the associated binary codes $\mathfrak{B}_k(P^n)$ (see Proposition~\ref{collection}).
  In section 5, we show that the minimum distance of the
   self-dual binary code obtained from a $3$-colorable simple $3$-polytope is always
    $4$ (see Proposition~\ref{prop:3-polytope}).
  In section 6, we study some special properties of 
   $n$-colorable simple $n$-polytopes. In section 7, we study what kind of 
  doubly-even binary codes can be obtained from $n$-colorable simple $n$-polytopes. 
  In particular, we show
  that the extended Golay code and 
  the extended quadratic residue code can not be obtained from 
   any $n$-colorable simple $n$-poltyopes.\\

  \section{Small covers with $\mathrm{m}$-involutions}

  \subsection{Small covers}\label{small}
  An $n$-dimensional simple (convex) polytope is a polytope such that each vertex of the polytope is exactly the intersection of $n$ \emph{facets} ($(n-1)$-dimensional faces) in the polytope.
   Following \cite{dj}, an $n$-dimensional {\em small cover} $\pi: M^n\rightarrow P^n$ is a 
   closed smooth $n$-manifold $M^n$ with a locally
  standard ${\Bbb Z}_2^n$-action whose orbit space is homeomorphic to an $n$-dimensional
   simple convex polytope
  $P^n$, where a locally standard ${\Bbb Z}_2^n$-action on $M^n$ means that this
${\Bbb Z}_2^n$-action on $M^n$ is locally isomorphic to a 
faithful representation of ${\Bbb Z}_2^n$ on ${\Bbb R}^n$.
 Let $\mathcal{F}(P^n)$ denote the set of all facets of $P^n$.
 For any facet $F$ of $P^n$, the isotropy subgroup of
    $\pi^{-1}(F)$ in $M^n$ with respect to the ${\Bbb Z}_2^n$-action is
    a rank one subgroup of ${\Bbb Z}_2^n$ generated by an element of ${\Bbb Z}_2^n$, 
    denoted by $\lambda(F)$.
    Then we obtain a map $\lambda: \mathcal{F}(P^n) \rightarrow
    {\Bbb Z}_2^n$ which is called
    the \emph{characteristic function} associated
     to $M^n$. Notice that the $n$ facets meeting at each vertex of $P^n$ 
     are mapped to $n$ linearly independent elements in ${\Bbb Z}_2^n$.
     It is shown in~\cite{dj} that
     up to equivariant homeomorphism, $M^n$
     can be recovered from $(P^n,\lambda)$ in a canonical way (see~\eqref{Equ:Glue-Back}).
  Moreover, many algebraic
 topological invariants of a small cover $\pi: M^n\rightarrow P^n$ can be easily
 computed from $(P^n, \lambda)$. Here is a list of facts on the cohomology rings of 
 small covers proved in~\cite{dj}. \vskip .1cm
 
\begin{itemize}

\item[(R1)] Let $b_i(M;{\Bbb F_2})$ be the $i$-th mod 2 Betti number of $M^n$. Then
$$ b_i(M;{\Bbb F_2}) = h_i(P^n), \ 0\leq i \leq n$$
where $(h_0(P^n), h_1(P^n), \ldots, h_n(P^n))$ is the $h$-vector of $P^n$. \vskip .1cm

\item[(R2)] Let $V(P^n)$ denote the set of vertices of $P^n$. Then 
 $$|M^{{\Bbb Z}_2^n}|=\sum_{i=0}^n b_i(M;{\Bbb F_2})=\sum_{i=0}^n h_i(P^n)=
   |V(P^n)|.$$ \vskip .1cm

\item[(R3)] The equivariant cohomology $H^*_{{\Bbb Z}_2^n}(M; {\Bbb F}_2)$ is isomorphic as graded rings to the Stanley--Reisner ring of $P^n$
 \begin{equation} \label{Equ:Equiv-Cohomology}
     H^*_{{\Bbb Z}_2^n}(M; {\Bbb F}_2) \cong {\Bbb F}_2(P^n)={\Bbb F}_2[a_{F_1},
        \ldots, a_{F_m}]/I_{P^n}
     \end{equation}
where $F_1, \ldots, F_m$ are all the facets of $P^n$ and $a_{F_1},\ldots,a_{F_m}$ are of degree $1$, and $I_{P^n}$ is the ideal generated by all square free monomials $a_{F_{i_1}}\cdots a_{F_{i_s}}$ with $F_{i_1}\cap\cdots\cap F_{i_s}=\emptyset$ in $P^n$.\vskip .1cm

\item[(R4)] The mod-$2$ cohomology ring $H^*(M; {\Bbb F}_2)\cong {\Bbb F}_2[a_{F_1}, 
 \ldots, a_{F_m}]/I_P+J_\lambda$, where $J_\lambda$ is an ideal determined $\lambda$. In particular, 
 $H^*(M; {\Bbb F}_2)$ is generated by degree $1$ elements.

\end{itemize}

\subsection{Spaces constructed from simple polytopes with ${\Bbb Z}_2^r$-colorings} \ \vskip .1cm
  Let $P^n$ be an $n$-dimensional simple polytope in ${\Bbb R}^n$.
    For any $r\geq 0$, a 
  ${\Bbb Z}_2^r$-coloring on $P$ is a map
  $\mu: \mathcal{F}(P^n) \rightarrow {\Bbb Z}_2^r$. For any facet $F$ of $P$,
  $\mu(F)$ is called the \emph{color of $F$}.
  Let $f=F_1\cap \cdots \cap F_k$ be a codimension-$k$ face of $P$ where
  $F_1,\cdots, F_k\in \mathcal{F}(P^n)$. Define
     \begin{equation}\label{Equ:Subgroup}
     G^{\mu}_f = \text{the subgroup of ${\Bbb Z}_2^r$ generated by}\ 
     \mu(F_1), \cdots , \mu(F_k).
   \end{equation}  
   Besides, let $G^{\mu}$ be the subgroup of ${\Bbb Z}_2^r$ generated by
    $\{ \mu(F)\,;\, F\in \mathcal{F}(P) \}$.
   The rank of $G^{\mu}$ is called the \emph{rank of $\mu$}, denoted by $\mathrm{rank}(\mu)$. 
   It is clear that 
   $\mathrm{rank}(\mu) \leq r$.\vskip .1cm
    
    For any point $p\in P^n$, let $f(p)$ denote the unique face of $P^n$ that contains $p$ in
  its relative interior.   
   Then we define a space associated to $(P^n,\mu)$ by:
    \begin{equation} \label{Equ:Glue-Back}
        M(P^n,\mu) = P^n\times {\Bbb Z}_2^r \slash \sim
     \end{equation}
   where $(p,g) \sim (p',g')$ if and only if $p=p'$ and 
   $g^{-1}g' \in G^{\mu}_{f(p)}$.
  
   \begin{itemize}
     \item $M(P^n,\mu)$ is a closed manifold if and only if 
        $\mu$ is \emph{non-degenerate}, which means that
    $\mu(F_1)\cdots, \mu(F_k)$ are linearly independent whenever 
   $F_1\cap\cdots \cap F_k\neq \varnothing$. \vskip .1cm
   \item $M(P^n,\mu)$ has $2^{r-\mathrm{rank}(\mu)}$ connected components. So $M(P^n,\mu)$
    is connected if and only if
   $\mathrm{rank}(\mu)=r$. \vskip .1cm
   \item There is a canonical ${\Bbb Z}_2^r$-action on $M(P^n,\mu)$ defined by:
     $$h\cdot [(x,g)] = [(x,g+h)],\ x\in P^n,\, g,h\in {\Bbb Z}_2^r.$$
     let 
    $\pi_{\mu}: M(P^n,\mu)\rightarrow P^n $ be the map sending any $[(x,g)]\in  M(P^n,\mu)$ to 
    $x\in P^n$.
   \end{itemize}
   \vskip .1cm
    
    For any face $f$ of $P^n$ with $\dim(f)\geq 1$, let
     $r(f) = r-\mathrm{rank}(G^{\mu}_f)$ and 
    $$\eta_f: {\Bbb Z}_2^r \rightarrow {\Bbb Z}_2^r\slash G^{\mu}_f \cong {\Bbb Z}_2^{r(f)}$$ be 
    the quotient homomorphism.
    Then $\mu$ induces a ${\Bbb Z}_2^{r(f)}$-coloring $\mu_f$ on $f$ by: 
       \begin{equation}\label{Induced-Coloring}
        \mu_f(F\cap f) := \eta_f( \mu(F)), \ \text{where} \ F\in \mathcal{F}(P), \ 
          \dim(F\cap f)=\dim(f)-1. 
       \end{equation}   
    It is easy to see that $\pi^{-1}_{\mu}(f)$ is homeomorphic to $M(f, \mu_f)$. \vskip .1cm
    
   \begin{exam} \label{Exam:Small-Cover}
   Suppose $\pi: M^n \rightarrow P^n$ is a small cover
    with characteristic function $\lambda$. Then
    $M^n$ is homeomorphic to $M(P^n,\lambda)$. Moreover, for any face $f$ of $P^n$,
     $\pi^{-1}(f)\cong M(f, \lambda_f)$ is a closed connected submanifold of $M^n$, called 
     a \emph{facial submanifold} of $M^n$. Let $F_1,\cdots, F_m$ be all the facets of $P^n$.
    Then the generators $a_{F_1},\cdots, a_{F_m} \in H^*_{{\Bbb Z}_2^n}(M; {\Bbb F}_2)$ 
    in~\eqref{Equ:Equiv-Cohomology} are the equivariant
    Euler classes of the normal bundles of the facial submanifolds
    $\pi^{-1}(F_1),\cdots, \pi^{-1}(F_m)$ in $M^n$.
    
  \end{exam}  
   
   \begin{rem}
    The construction~\eqref{Equ:Glue-Back} makes sense for 
    any ${\Bbb Z}_2^r$-coloring on a nice manifold with corners.
    \end{rem} 

  \subsection{Small covers with regular $\mathbf{m}$-involutions}\label{involution}
  Let $\pi: M^n\rightarrow P^n$ be a small cover over an $n$-dimensional simple polytope $P^n$ 
  and $\lambda: \mathcal{F}(P^n)\rightarrow {\Bbb Z}_2^n$ be its characteristic
function. Let us discuss under what condition there exists a regular $\mathrm{m}$-involution 
on $M^n$.

\begin{thm}\label{max-involution}
The following statements are equivalent.
\begin{itemize}
\item[(a)] There exists a regular $\mathrm{m}$-involution on $M^n$.
\item[(b)] There exists a regular involution on $M^n$ with only isolated fixed points;
\item[(c)] The image $\operatorname{Im} \lambda$ of $\lambda$ 
   is a basis of ${\Bbb Z}_2^n$ (which implies that $P^n$ is
 $n$-colorable).
\end{itemize}
\end{thm}
\begin{proof}
 It is trivial that (a) implies (b). \vskip .1cm
 
 (b)$\Rightarrow$(c) Suppose there exists $g\in {\Bbb Z}_2^n$ so that the fixed points of $\tau_g$ 
  on $M^n$ are all isolated.
  Let $v$ be an arbitrary vertex on $P^n$ and $F_1,\cdots,F_n$ be the $n$ facets
  meeting at $v$. By the construction of small covers, 
  $\pi^{-1}(v)=p$ is a fixed point of the whole group ${\Bbb Z}_2^n$. 
  Let $U\subset M$ be a small neighborhood of $p$. Since the action of ${\Bbb Z}_2^n$ on $M^n$ is locally standard, we observe that for 
  $h= \lambda(F_{i_1}) + \cdots + \lambda(F_{i_s}) \in {\Bbb Z}_2^n$, $1\leq i_1 < \cdots < i_s\leq n$,
  the dimension of the fixed point set of $\tau_h$ in $U$ is equal to $n-s$. 
  Then since the fixed points of $\tau_g$ are all isolated, we must have
   $g=\lambda(F_1)+\cdots +
\lambda(F_n)$.\vskip .1cm
  Next, take an edge of $P^n$ with two endpoints $v_1, v_2$.
  Since $P^n$ is simple, there are $n+1$ facets $F_1, ..., F_n, F'_n$ such that
 $v_1=F_1\cap\cdots\cap F_{n-1}\cap F_n$ and $v_2=F_1\cap\cdots\cap
F_{n-1}\cap F'_{n}$. Then
$\lambda(F_1)+\cdots+\lambda(F_{n-1})+\lambda(F_n)=g=\lambda(F_1)+\cdots+\lambda(F_{n-1})+\lambda(F'_{n})$,
which implies $\lambda(F_n)=\lambda(F'_{n})$.
Since the 1-skeleton of $P^n$ is connected, 
we can deduce the image $\operatorname{Im} \lambda$ of $\lambda$ consists of $n$ elements of ${\Bbb Z}_2^n$
 which form a basis of ${\Bbb Z}_2^n$.\vskip .1cm
 
 (c)$\Rightarrow$(a) Suppose $\operatorname{Im} \lambda =\{ g_1,\cdots, g_n\}$ 
 is a basis of ${\Bbb Z}_2^n$. Then by the construction of small covers, the fixed point set of 
  the regular involution $\tau_{g_1+\cdots+g_n}$ on $M^n$ is 
  $$\{ \pi^{-1}(v)\,|\, v\in V(P^n)\} = M^{{\Bbb Z}_2^n}.$$
 So the number of fixed points of $\tau_{g_1+\cdots+g_n}$ is equal to the number of
 vertices of $P^n$, which is known to be $h_0(P^n)+h_1(P^n)+\cdots+h_n(P^n)$. 
 Then by the result (R1) in Subsection~\ref{small}, $\tau_{g_1+\cdots+g_n}$ is 
 an $\mathrm{m}$-involution
 on $M^n$.
\end{proof}

  \subsection{Binary codes from $\mathbf{m}$-involutions on manifolds} 
 Let $\tau$ be an involution on a closed
 connected $n$-dimensional manifold $M$, which has only isolated fixed points.
Let $G_{\tau} \cong {\Bbb Z}_2$ denote the binary group generated by $\tau$. 
By Conner~\cite[p.82]{cf}, the number $|M^{G_{\tau}}|$ of the
 fixed points of $G_{\tau}$ must be even. So we assume $|M^{G_\tau}|=2r$, $r\geq 1$ in the following discussions. \vskip .1cm
By~\cite[Proposition(1.3.14)]{ap}, the following statements are equivalent.

\begin{itemize}
\item[(a)]  $|M^{G_{\tau}}|=\sum_{i=0}^n b_i(M; \Bbb F_2)$ (i.e. $\tau$ is 
an $\mathrm{m}$-involution); \vskip .1cm
\item[(b)]  $H^*_{G_\tau}(M; {\Bbb F}_2)$ is a free $H^*(BG_\tau; {\Bbb F}_2)$-module, so 
  $$H^*_{G_\tau}(M; {\Bbb F}_2)=H^*(M;{\Bbb F}_2)\otimes H^*(BG_\tau; {\Bbb F}_2);$$
\item[(c)] The inclusion of the fixed point set, $\iota: M^{G_{\tau}}\hookrightarrow M$, induces a monomorphism 
$$\iota^*: H^*_{G_\tau}(M; {\Bbb F}_2)\rightarrow H^*_{G_\tau}(M^{G_\tau}; {\Bbb F}_2) \cong
{\Bbb F}^{2r}_2 \otimes {\Bbb F}_2[t].$$
\end{itemize}

 Next we assume that $\tau$ is an $\mathrm{m}$-involution on $M$.
 So  
 the image of $H^*_{G_{\tau}}(M; {\Bbb F}_2)$ in ${\Bbb F}^{2r}_2 \otimes {\Bbb F}_2[t]$ under the
 localization map $\iota^*$ is isomorphic to $H^*_{G_{\tau}}(M; {\Bbb F}_2)$ as graded algebras.
 It is shown in~\cite{cl,p} that the image $\iota^*(H^*_{G_{\tau}}(M; {\Bbb F}_2))$ 
 can be described
 in the following way. For any vectors $x=(x_1,...,x_{2r})$ and $y=(y_1,...,y_{2r})$ in
${\Bbb F}_2^{2r}$, define
   $$x\circ y=(x_1y_1,...,x_{2r}y_{2r}).$$
It is clear that ${\Bbb F}_2^{2r}$ forms a commutative ring with respect to
two operations $+$ and $\circ$. Actually, $({\Bbb F}_2^{2r}, +, \circ)$ is a boolean ring.   Let
\begin{equation} \label{Equ:Big-V}
   \mathcal{V}_{2r}=\big\{x=(x_1,...,x_{2r})\in 
{\Bbb F}_2^{2r} \,\big| \, \langle x, x\rangle=\sum\limits_{i=1}^{2r}x_i=0\big\}.
\end{equation}
 
  Then  $\mathcal{V}_{2r}$ is a $(2r-1)$-dimensional linear 
  subspace of ${\Bbb F}_2^{2r}$. Note that for any $u\in \mathcal{V}_{2r}$, the Hamming weight
  $wt(u)$ of $u$ is an even integer. The following lemma is immediate from our definitions.

\begin{lem} \label{code} 
 Let $C$ be a binary code in ${\Bbb F}_2^{2r}$ with $\dim_{{\Bbb F}_2} C =r$.
 The following statements are equivalent.
\begin{enumerate}
\item[(C1)] $C$ is a self-dual code; \vskip .1cm
\item[(C2)] For any $x, y\in C$, $\langle x, y\rangle=0$; \vskip .1cm
\item[(C3)] For any $x, y\in C$, $x\circ y\in \mathcal{V}_{2r}$.
\end{enumerate}
\end{lem}

   Moreover, let
 \begin{equation} \label{Equ:V_i}
     V^M_k = \{ y\in {\Bbb F}_2^{2r}  \,\big|\, y\otimes t^k \in \mathrm{Im}(\iota^*)\}
     \subset {\Bbb F}_2^{2r}, \  k=0,\cdots, n.
 \end{equation}
 By the localization theorem for equivariant cohomology (see~\cite{ap}), we have isomorphisms
 \begin{equation} \label{Equ:Isom-Cohom}
   H^k(M^n;{\Bbb F}_2) \cong V_k^M\slash V_{k-1}^M, \ 0\leq k \leq n. 
 \end{equation}  
 \vskip .1cm
 
\begin{thm} [{\cite[Theorem 3.1]{cl}}] \label{ring} 
 For any $0\leq k \leq n$, the dimension of $V^M_k$ is
$$\dim_{{\Bbb F}_2} V^M_k=\sum\limits_{j=0}^k b_j(M;\Bbb F_2).$$ 
 In addition, $H^*_{G_\tau}(M^n;{\Bbb F}_2)$ is isomorphic to
the graded ring
$$\mathcal{R}_M=V^M_0+ V^M_1t+\cdots+
V^M_{n-2}t^{n-2}+ V^M_{n-1}t^{n-1}+ {\Bbb
F}_2^{2r}(t^{n}+t^{n+1}+\cdots)$$ where the ring structure of
$\mathcal{R}_M$ is given by
\begin{enumerate}
\item[(a)] ${\Bbb F}_2\cong V^M_0\subset V_1^M\subset\cdots\subset V_{n-2}^M\subset V_{n-1}^M=\mathcal{V}_{2r}\subset   V^M_n = {\Bbb F}^{2r}_2$,
where $V^M_0$ is generated by $\underline{1}=(1,...,1)\in {\Bbb F}_2^{2r}$;
\item[(b)] For $d=\sum\limits_{i=0}^{n-1}i d_i<n$ with
each $d_i\geq0$, $v_{\omega_{d_0}}\circ\cdots\circ
v_{\omega_{d_{n-1}}}\in V^M_{d}$, where
$$v_{\omega_{d_i}}=v^{(i)}_1\circ\cdots\circ v^{(i)}_{d_i},\ v^{(i)}_j\in V^M_i.$$
 The operation $\circ$ on ${\Bbb F}^{2r}_2$ corresponds to
the cup product in $H^*_{G_\tau}(M;{\Bbb F}_2)$. 
\end{enumerate}

\end{thm}

Each $V^M_k$ above can be thought of as a binary code in ${\Bbb F}_2^{2r}$. Note that
for any $x, y\in {\Bbb F}_2^{2r}$, 
 $\langle x, y\rangle=0$ if and only if $x\circ y\in \mathcal{V}_{2r}$. So Theorem~\ref{ring} implies that 
 \begin{equation} \label{Equ:Perp}
     (V^M_k)^{\perp} = V^M_{n-1-k}. 
  \end{equation}   
 This is because $V^M_{n-1-k}$ is perpendicular to $V^M_k$ with respect to $\langle\ , \, \rangle$
 and by the Poincar\'e duality of $M$, we have
 $\dim_{{\Bbb F}_2} V^M_k  + \dim_{{\Bbb F}_2} V^M_{n-1-k} = 
 \sum\limits_{j=0}^n b_j(M;\Bbb F_2)=2r$.\vskip .1cm
 
 So in particular when $n$ is odd,
  $V^M_{\frac{n-1}{2}}$ is a self-dual binary code in ${\Bbb F}_2^{2r}$.
 It is easy to see that $V^M_{\frac{n-1}{2}}$ is the largest 
  subspace of $\mathcal{V}_{2r}$ that is closed under the operation $\circ$ in this case 
  (see~\cite[Corollary 3.2]{cl}).  
\vskip .3cm

\subsection{Descriptions of $n$-colorable $n$-dimensional simple polytopes}
\ \vskip .1cm

  The following descriptions of $n$-colorable simple $n$-polytopes are due to Joswig~\cite{jos}.
  
 \begin{thm} [{\cite[Theorem 16 and Corollary 21]{jos}}]\label{j}
 Let $P^n$ be an $n$-dimensional simple polytope. The following statements are equivalent.
 \begin{itemize}
   \item[(a)]$P^n$ is $n$-colorable;\vskip .1cm
   \item[(b)] Each $2$-face of $P^n$ has an even number of vertices.\vskip .1cm
   \item[(c)] Each face of $P^n$ with dimension greater than $0$ (including $P^n$ itself) has an even number of vertices.
   \item[(d)] Any proper $k$-face of $P^n$ is $k$-colorable.\\
 \end{itemize}  
 \end{thm}

 

\section{Self-dual binary codes from small covers}
 Let $\pi: M^n\rightarrow P^n$ be an $n$-dimensional small cover which admits a regular $\mathrm{m}$-involution.
 By Theorem~\ref{max-involution}, $P^n$ is an $n$-dimensional $n$-colorable simple polytope
 with an even number of vertices. Let
  $\{ v_1,\cdots, v_{2r} \}$ be all the vertices of $P^n$. 
 The characteristic function $\lambda$ of $M^n$
 satisfies: $\mathrm{Im}(\lambda) =\{e_1,\cdots, e_n\}$ is a basis of ${\Bbb Z}_2^n$.
By Theorem~\ref{max-involution}, $\tau_{e_1+\cdots+e_n}$ is an $\mathrm{m}$-involution
on $M^n$. So by the discussion in Subsection 2.4, we obtain a filtration 
$${\Bbb F}_2\cong V^{M}_0\subset V_1^{M}\subset\cdots\subset V_{n-2}^{M}\subset V_{n-1}^{M}=\mathcal{V}_{2r}\subset V^M_n= {\Bbb F}^{2r}_2.$$
In particular, when $n$ is odd, $C_{M^n} = V^{M}_{\frac{n-1}{2}} \subset {\Bbb F}_2^{2r} $ is 
a self-dual binary code determined by $(M^n,\tau_{e_1+\cdots+e_n})$.
 In this section, we will describe each $V_k^{M}$, $0 \leq k \leq n$, explicitly in terms of the combinatorics of $P^n$, and hence determines the code $C_{M^n}$.\vskip .1cm

 First, any face $f$ of $P^n$ determines an element $\xi_f\in {\Bbb F}^{2r}_2$ where
  the $i$-th entry of $\xi_f$ is $1$ if and only if $v_i$ is a vertex of $f$.                          
 In particular, $\xi_{P^n} = \underline{1} = (1,\cdots, 1)\in {\Bbb F}^{2r}_2 $. 
 Note that for any faces $f_1,\cdots, f_s$ of $P^n$, we have
  \begin{equation} \label{Equ:Product-Face}
     \xi_{f_1\cap\cdots\cap f_s} = \xi_{f_1}\circ\cdots\circ \xi_{f_s}. 
  \end{equation}   
 In addition, we define a sequence of binary codes $\mathfrak{B}_k(P^n) \subset {\Bbb F}^{2r}_2$ 
 as follows.
 \begin{equation} \label{Equ:Def-Bk}
    \mathfrak{B}_k(P^n):= \text{Span}_{\Bbb F_2}\{\xi_f \,;\,  f  \ 
                                                 \text{is a codimension-$k$ face of
      $P$}\},\ 0\leq k \leq n.
    \end{equation}  
    \vskip .1cm
    
\begin{rem}
  Up to equivalences of binary codes, each
   $\mathfrak{B}_k(P^n)$ is uniquely determined by the simple polytope $P^n$. 
\end{rem}    
    
 \begin{lem} \label{Lem:Inclusion}
 For any $n$-colorable simple $n$-polytope $P^n$ with $2r$ vertices, we have
   $$ \mathfrak{B}_0(P^n) \subset \mathfrak{B}_1(P^n) \subset\cdots\subset 
   \mathfrak{B}_{n-1}(P^n) = \mathcal{V}_{2r} \subset \mathfrak{B}_n(P^n)\cong {\Bbb F}_2^{2r}.$$
 \end{lem}               
 \begin{proof}
  By definition, $P^n$ can be colored by $n$ colors $\{e_1,\cdots, e_n\}$. Now choose an arbitrary color say $e_j$, we observe that
    each vertex of $P^n$ is contained in exactly one facet of $P^n$ colored by $e_j$.
    This implies that
    $$ \xi_{P^n} = \xi_{F_1} +\cdots + \xi_{F_s} $$
    where $F_1,\cdots, F_s$ are all the facets of $P^n$ colored by $e_j$.  
    So $\mathfrak{B}_0(P^n) \subset \mathfrak{B}_1(P^n)$.
   Moreover, by Theorem~\ref{j}(d), the facets $F_1,\cdots, F_s$ are $(n-1)$-dimensional simple polytopes which are $(n-1)$-colorable. So by repeating the above argument, we can
   show that $\mathfrak{B}_1(P^n) \subset \mathfrak{B}_2(P^n)$ and so on. Now it remains to show
   $\mathfrak{B}_{n-1}(P^n) = \mathcal{V}_{2r}$. \vskip .1cm
   
   By definition, $\mathfrak{B}_{n-1}(P^n)$ is spanned by $\{ \xi_f \, |\, f\ \text{is an edge (or $1$-face) of}\ P^n \}$. So it is obvious that $\mathfrak{B}_{n-1}(P^n) \subset \mathcal{V}_{2r}$.
  Let $\{ v_1,\cdots, v_{2r} \}$ be all the vertices of $P^n$. It is easy to see that 
  $\mathcal{V}_{2r}$ is spanned by $\{ \xi_{v_i} + \xi_{v_j} \,|\, 1\leq i\neq j \leq 2r\}$.
  Then since there exists an edge path on $P^n$ between any two vertices $v_i$ and $v_j$ of 
  $P^n$, $\xi_{v_i} + \xi_{v_j}$ belongs to $\mathfrak{B}_{n-1}(P^n)$. So 
  $\mathcal{V}_{2r}\subset \mathfrak{B}_{n-1}(P^n)$. This finishes the proof.
    \end{proof}
 \vskip .1cm
 
 Later we will prove that 
 the condition in Lemma~\ref{Lem:Inclusion} is also sufficient for an $n$-dimensional 
 simple polytope 
 to be $n$-colorable (see Proposition~\ref{collection}).\vskip .1cm  
 
 \begin{thm} \label{Thm:Main-1}
  Let $\pi: M^n\rightarrow P^n$ be an $n$-dimensional small cover which admits a regular $\mathrm{m}$-involution.
  For any $0\leq k \leq n$, the space $V^{M}_k$ coincides with $\mathfrak{B}_k(P^n)$. 
 \end{thm}
  \vskip .1cm
 
 
 \begin{cor}  \label{Cor:Main-2}
    Let $\pi: M^n\rightarrow P^n$ be an $n$-dimensional small cover which admits a regular $\mathrm{m}$-involution where $n$ is odd.
     The self-dual binary code $C_{M^n} = V^{M}_{\frac{n-1}{2}} = 
    \mathfrak{B}_{\frac{n-1}{2}}(P^n) \subset {\Bbb F}^{2r}_2$ is spanned by
    $\{ \xi_{f}\,;\, f\  \text{is any face of $P^n$ with}\  \dim(f) = \frac{n+1}{2} \}$.
     So the minimum distance of $C_{M^n}$ is less or equal to
     $\mathrm{min}\{ \# (\text{vertices of}\ f)  \, ;\, f \ \text{is 
  a}\ \frac{n+1}{2}\text{-dimensional face of}\ P^n\}$.
 \end{cor}
 

  
 \noindent \textbf{Problem:} For any $n$-dimensional small cover $M^n$ 
 which admits a regular $\mathrm{m}$-involution where $n$ is odd, determine 
 the minimum distance of the self-dual binary code $C_{M^n}$. \vskip .2cm

 We will see in Proposition~\ref{prop:3-polytope} that 
 when $n=3$, the minimum distance of 
 $C_{M^n}$ is always equal to $4$. For higher dimensions, 
 it seems to us that the minimum distance of $C_{M^n}$ should be equal to
 $\mathrm{min}\{ \# (\text{vertices of}\ f)  \, ;\, f \ \text{is 
  a}\ \frac{n+1}{2}\text{-dimensional face of}\ P^n\}$. But the proof is not clear to us.

 \vskip .2cm
 
 In the following, we will prove Theorem~\ref{Thm:Main-1} in two different ways.
  For brevity, let 
   $$\tau=\tau_{e_1+\cdots+e_n}, \ \ \
   G_{\tau}=\langle\tau\rangle\cong {\Bbb Z}_2 \subset {\Bbb Z}_2^n.$$ 
    By the construction of $M^n$,
  all the fixed points of $\tau$ on $M^n$ are
  $\tilde{v}_1,\cdots, \tilde{v}_{2r}$ where 
   $$\tilde{v}_i=\pi^{-1}(v_i) \in M^n, \ i=1,\cdots, 2r.$$
 
 \subsection{The first proof of Theorem~\ref{Thm:Main-1}}\ \vskip .1cm
 
   Since $n=1$ case is trivial, we will assume $n\geq 2$ in the rest of the proof.
   Choose a small $G_{\tau}$-invariant disk $D^n_i$ around each $\tilde{v}_i$ in $M^n$. Let 
   $$ S^{n-1}_i =\partial D^n_i, \ \ {\Bbb R} P^{n-1}_i = S^{n-1}_i \slash G_{\tau}.$$ 
   If not particularly indicated, we assume ${\Bbb F}_2$-coefficients for all  
   cohomology and homology groups below. It is clear that
   the inclusion $\tilde{v}_i\hookrightarrow D^n_i$ induces an isomorphism 
   $$ H^*_{G_{\tau}}(D^n_i) \overset{\cong}{\longrightarrow} H^*_{G_{\tau}}(\tilde{v}_i).$$  
   So to compute the image of the localization 
    $\iota^* : H^*_{G_{\tau}}(M) \rightarrow H^*_{G_{\tau}}(M^{G_{\tau}})$, it is equivalent to
   computing the image of $\kappa^*: H^*_{G_{\tau}}(M^n)\rightarrow H^*_{G_{\tau}}(\bigcup D^n_i)$ where
   $\kappa: \bigcup D^n_i \rightarrow M^n$ denotes the inclusion. 
   Consider the long exact sequence of equivariant cohomology for
   the inclusion of
    $ \big( M^n\backslash \bigcup \overset{\circ}{D^n_i}, \bigcup S^{n-1}_i\big) 
   \hookrightarrow \big( M^n, \bigcup D^n_i\big)$, we
    obtain a commutative diagram
  \[
     \xymatrix{
        H^{j}_{G_{\tau}}\big(M^n, \bigcup D^n_i \big) \ar[d]^{\cong} \ar[r]^{} &
          H^j_{G_{\tau}}(M^{n}) \ar[d] \ar[r]^{\kappa^*} &
          H^j_{G_{\tau}}\big(\bigcup D^n_i\big) \ar[d] \ar[r] &  H^{j+1}_{G_{\tau}} \big( M^n, \bigcup D^n_i \big) 
          \ar[d]^{\cong}    \\
    H^{j}_{G_{\tau}}\big(M^n\backslash \bigcup \overset{\circ}{D^n_i}, \bigcup S^{n-1}_i \big)   
    \ar[r]  &  H^j_{G_{\tau}}(M^n\backslash \bigcup \overset{\circ}{D^n_i})  \ar[r] &
       H^j_{G_{\tau}}(\bigcup S^{n-1}_i) \ar[r]
   & H^{j+1}_{G_{\tau}}\big(M^n\backslash \bigcup \overset{\circ}{D^n_i}, \bigcup S^{n-1}_i\big)
    }
       \]
  Here the isomorphism $ H^{j}_{G_{\tau}}\big(M^n, \bigcup D^n_i \big) \overset{\cong}{\longrightarrow}
   H^{j}_{G_{\tau}}\big(M^n\backslash \bigcup \overset{\circ}{D^n_i}, \bigcup S^{n-1}_i \big)$, $j\geq 0$
    is the excision of cohomology.
   Note that the equivariant cohomology of a free $G$-space is the ordinary 
  cohomology of the orbit space, we can identify $H^*_{G_{\tau}}(\bigcup S^{n-1}_i)$ with
  $H^*({\Bbb R} P^{n-1}_i)$.\vskip .1cm
  
 \textbf{Claim:} 
   The group homomorphism $ H^j_{G_{\tau}}(\bigcup D^n_i) \longrightarrow  
   H^j_{G_{\tau}}(\bigcup S^{n-1}_i) = H^j\left(\bigcup {\Bbb R} P^{n-1}_i \right) $
    is an isomorphism for any $0\leq j\leq n-1$. 
    \vskip .1cm
    
   Indeed, for each disk $D^n_i$, we have canonical inclusions
   \[ S^{n-1}_i \overset{(id,*)}{\longrightarrow} S^{n-1}_i \times S^{\infty} 
   \longrightarrow D^n_i\times 
   S^{\infty} \overset{p}{\longrightarrow} S^{\infty}  \]
   So we have a graded ring homomorphism:
    $$H^*_{G_{\tau}}(D^n_i) 
    \cong H^*({\Bbb R} P^{\infty}) \cong {\Bbb F}_2[t] \longrightarrow {\Bbb F}_2[t]/ (t^n) \cong 
     H^*({\Bbb R} P^{n-1}) \cong H^*_{G_{\tau}}(S^{n-1}_i)$$
     which is an isomorphism in degree less than $n$. The claim is proved.\vskip .1cm
   
   Since $G$ acts freely on $M^n\backslash \bigcup_{2r} \overset{\circ}{D^n_i}$, let 
   $W = \big(M^n\backslash \bigcup_{2r} \overset{\circ}{D^n_i} \big)\slash  \tau $
    which is a manifold with boundary $\partial W = \bigcup^{2r}_{i=1} {\Bbb R} P^{n-1}_i $ and,
     there is an isomorphism
    $H^*_{G_{\tau}}(M^n\backslash \bigcup \overset{\circ}{D^n_i}) \cong H^*(W)$. So
   we have a commutative diagram for each $0\leq j \leq n-1$.
    \[ 
      \xymatrix{
        H^{j}_{G_{\tau}}\big(M^n, \bigcup D^n_i \big) \ar[d]^{\cong} \ar[r]^{} &
          H^j_{G_{\tau}}(M^{n}) \ar[d] \ar[r]^{\kappa^*} &
          H^j_{G_{\tau}}\big(\bigcup D^n_i\big) \ar[d]^{\cong} \ar[r] &  H^{j+1}_{G_{\tau}} \big( M^n, \bigcup D^n_i \big) 
          \ar[d]^{\cong}    \\
    H^{j}_{G_{\tau}}\big(M^n\backslash \bigcup \overset{\circ}{D^n_i}, \bigcup S^{n-1}_i \big)   
    \ar[r]  &  H^j(W)  \ar[r] &
       H^j(\bigcup {\Bbb R} P^{n-1}_i) \ar[r]
   & H^{j+1}_{G_{\tau}}\big(M^n\backslash \bigcup \overset{\circ}{D^n_i}, \bigcup S^{n-1}_i\big)
    }
     \]    
     
  Then by the five-lemma, for any $0\leq j\leq n-1$, there are isomorphisms\vskip .1cm
  \noindent 
  $
     \mathrm{Im}\Big(H^j_{G_{\tau}}(M)\overset{\iota^*}{\rightarrow} H^j_{G_{\tau}}(M^{G_{\tau}})\Big) \cong
     \mathrm{Im}\Big(H^j_{G_{\tau}}(M^n)\overset{\kappa^*}{\rightarrow} H^j_{G_{\tau}}(\bigcup D^n_i)\Big) \cong
   \mathrm{Im}\Big(H^j(W) \rightarrow H^j(\bigcup {\Bbb R} P^{n-1}_i)\Big)
 $.\vskip .1cm
 
    So for any $0\leq k \leq n-1$, the space $V^{M}_k$ can be identified with
     the linear subspace spanned by the image
 $$ \mathrm{Im}\Big(\bigoplus^k_{j=0} H^j(W) \longrightarrow 
  \bigoplus^k_{j=0} H^j(\bigcup {\Bbb R} P^{n-1}_i)\Big). $$
 By the Poincar\'e-Lefschetz duality, 
  we have\vskip .2cm
    \begin{center}
       $\mathrm{Im}\Big(H^j(W) \rightarrow 
   H^j(\bigcup  {\Bbb R} P^{n-1}_i)\Big) 
  \cong \mathrm{Im}\Big(H_{n-j}(W,\partial W)\rightarrow H_{n-j-1}(\partial W) \Big)$.
   \end{center}
   \vskip .1cm
   Then the space $V^{M}_k$ can be further identified with the 
   linear space spanned by the image 
    \begin{equation} \label{Equ:C_M}
       \mathrm{Im}\Big( \bigoplus_{j=n-k}^n 
          H_j(W,\partial W)\longrightarrow \bigoplus^n_{j=n-k} H_{j-1}(\partial W) \Big), \ 
     \end{equation}  
    
        
 
   Next, we construct a cell decomposition of $W$ 
      to 
     compute $\mathrm{Im}\big( H_{j}(W,\partial W)\rightarrow H_{j-1}(\partial W) \big)$ for all
     $j$.
       Let $M\slash G_{\tau}$ be the orbit space of the $G_{\tau}$-action on $M$. Observe that
   \begin{center}
     $ M\slash G_{\tau} = W \bigcup \big(\overset{2r}{\underset{i=1}{\bigcup}} Cone({\Bbb R} P^{n-1}_i) \big)$
     \end{center} \vskip .1cm
  
   In addition, it is easy to see that $M\slash G_{\tau} = M(P^n,\lambda) \slash G_{\tau}$ is 
   homeomorphic to $M(P^n, \bar{\lambda})$ 
   where $\bar{\lambda}$ is a ${\Bbb Z}_2^{n-1}$-coloring 
  on $P^n$ defined by:
    \begin{equation} \label{Equ:Induced-Color}
     \bar{\lambda} (F) =  \begin{cases}
      e_1+\cdots + e_{n-1},   &   \text{ if $\lambda(F)=e_n$; } \\
       \lambda(F),     &   \text{ otherwise. } \\
 \end{cases}
 \end{equation}
   Here we think of ${\Bbb Z}_2^{n-1}$ as the subgroup of ${\Bbb Z}_2^n$
   generated by $e_1,\cdots, e_{n-1}$. 
     \vskip .1cm
   We can obtain $W$ from $M\slash G_{\tau}$ by removing
    an open cone of ${\Bbb R} P^{n-1}_i$ from $M\slash G_{\tau}$
   for each $1\leq i \leq 2r$. In the orbit spaces, this operation
   corresponds to cutting off from $P^n$ a small neighborhood of each vertex of $P^n$.
    Let
   $P^{\Delta}$ denote the new
   simple polytope obtained after these cuttings on $P^n$ (see Figure~\ref{p:Cut_Polytope}).
   We consider $P^{\Delta}$ as a subset of $P^n$.
   
    \vskip .1cm
   \begin{figure}
         \includegraphics[width=0.65\textwidth]{Cut_Polytope.eps}\\
          \caption{Cut a neighborhood from each vertice  }\label{p:Cut_Polytope}
      \end{figure}

   The cut section of $P^n$ at $v_i$ is
      a facet of $P^{\Delta}$, denoted by $E_i$,
       which is isomorphic an $(n-1)$-simplex. 
   Let $\{ u_{i,1},\cdots, u_{i,n} \}$ be the vertices of $E_i$. Then all the vertices of 
   $P^{\Delta}$ are 
   $$V(P^{\Delta}) = \{ u_{i,j} \,|\, 1\leq i \leq 2r, 1\leq j \leq n \}.$$
  For any $1\leq k \leq n$, a $k$-face $f$ of 
  $P^n$ is cut into a face $f^{\Delta}$ of $P^{\Delta}$ where
   $$ f^{\Delta} = f\cap P^{\Delta},\ \dim(f^{\Delta}) = \dim (f).$$
 It is clear that a $k$-face of $P^{\Delta}$ is either
 $f^{\Delta}$ for some $k$-face of $P^n$ or a $k$-face of $E_i$.
 In particular, all the facets of $P^{\Delta}$ are 
  $$ \mathcal{F}(P^{\Delta}) = \{ F^{\Delta}\,;\, F\ \text{is any facet of}\ P^n  \} 
      \cup \{ E_i \,;\, 1\leq i \leq 2r \}.$$ 
    
   In addition, the ${\Bbb Z}_2^{n-1}$-coloring $\bar{\lambda}$ on $P^n$
    induces a ${\Bbb Z}_2^{n-1}$-coloring $\lambda^{\Delta}$
     on $P^{\Delta}$ by:   
    \begin{itemize}
    \item  $\lambda^{\Delta}(F^{\Delta})=\bar{\lambda}(F)\in {\Bbb Z}_2^{n-1}$
    for any facet $F$ of $P^n$.\vskip .1cm
    
     \item $\lambda^{\Delta}(E_i)=0\in {\Bbb Z}_2^{n-1}$, $1\leq i \leq 2r$.
   \end{itemize}   
     It is easy to see that $W$ is homeomorphic to 
      $M(P^{\Delta}, \lambda^{\Delta})$. For brevity,
      let 
      $$ \pi_{\Delta} =\pi_{\lambda^{\Delta}} :
        M(P^{\Delta}, \lambda^{\Delta}) \rightarrow P^{\Delta}$$
     Then ${\Bbb R} P^{n-1}_i = \pi^{-1}_{\Delta}(E_i)\cong 
       M(E_i,\lambda^{\Delta}_{E_i})$ (see~\eqref{Induced-Coloring}).
     Note that for any $k$-face $f'$ of $E_i$, $\pi^{-1}_{\Delta}(f')$ is homeomorphic to
       ${\Bbb R} P^k$, which can be considered as a generator of $H_k({\Bbb R} P^{n-1}_i)$.
       
  \vskip .1cm
  
  Let $f$ be a face of $P^n$ with $\dim(f)=j\geq 1$ and 
   $\{v_{p_0},\cdots, v_{p_j}\}$ be the vertex set of
  $f$. We can easily see that 
  $M_{f^{\Delta}}:=\pi^{-1}_{\Delta}(f^{\Delta})$ is a connected submanifold of $W$ whose boundary  
  $$\partial M_{f^{\Delta}} = \bigcup^j_{i=0} \pi^{-1}_{\Delta}(f^{\Delta}\cap E_{p_i}) 
      \subset \partial W.$$
    So $M_{f^{\Delta}}$ defines an element
  $[M_{f^{\Delta}}] \in H_j(W,\partial W)$. Since $f^{\Delta}\cap E_{p_i}$ is a $(j-1)$-face of $E_{p_i}$,
  $[\pi^{-1}_{\Delta}(f^{\Delta}\cap E_{p_i})]$ is a generator of 
  $H_{j-1}({\Bbb R} P^{n-1}_i)$. Then if we identify $H_{j-1}(\partial W) = H_{j-1}(\bigcup {\Bbb R} P^{n-1}_i)$ 
        with the subgroup ${\Bbb F}_2^{2r}\otimes t^{j-1}$ in 
        ${\Bbb F}^{2r}_2\otimes {\Bbb F}_2[t] =H^*_{G_{\tau}}(M^{G_{\tau}}) $, the
   homology class $[\partial M_{f^{\Delta}}]\in H_{j-1}(\partial W)$ is represented by
  the element $\xi_f \otimes t^{j-1}$.\vskip .1cm
  
  \textbf{Claim:} For any $j\geq 1$, the image
   $\mathrm{Im}\big( H_{j}(W,\partial W)\rightarrow H_{j-1}(\partial W) \big)$ is spanned by 
   the homology classes 
  $\{ [\partial M_{f^{\Delta}}] \in H_{j-1}(\partial W) \,;\, f\
   \text{is any face of}\ P^n\ \text{with}\ \dim(f)=j \}$.
   \vskip .1cm
   
   To prove this claim, we introduce a cell decomposition of $W$ below.
      The argument is similar to the construction of perfect cell structures on small covers
   in~\cite[Theorem 3.1]{dj}.\vskip .1cm
   
    First, we realize $P^n$ as a convex polytope
   in ${\Bbb R}^n$ and choose a vector $w$ in ${\Bbb R}^n$ which is generic in the sense that
   it is tangent to no proper
   face of $P^n$. Choose an inner product $(  \, , )$ 
   in ${\Bbb R}^n$ and let $\phi:{\Bbb R}^n\rightarrow {\Bbb R}$ defined by
     $\phi(x) = ( x,w)$. 
     We can think of $\phi$ as the height function on $P^n$.
    Using $\phi$, one makes the $1$-skeleton of $P^n$ into a directed graph by
    orienting each edge so that $\phi$ increases along it.
    Then for any face $f$ of $P^n$ with dimension $>0$, $\phi|_f$ assumes
    its maximum (or minimun) at a vertex. Since $\phi$ is generic, 
    each face $f$ of $P^n$ of a unique ``top'' and a unique ``bottom'' vertex.
    For any vertex $v$, let $m(v)$ denote the number of incident edges 
    which point toward $v$, and let $f_v$ be the smallest face of $P^n$ which contains 
    all the inward pointing edges incident to $v$.
    It is clear that $\dim(f_v) = m(v)$.
    A simple argument (see~\cite[p.115]{br}) shows that 
      \begin{equation} \label{Equ:fact}
      \text{ The number of
     vertices $v$ of $P^n$ with $m(v)=k$ is equal to $h_k(P^n)$},
     \end{equation} 
     where $(h_0(P^n),\cdots, h_n(P^n))$ is the \emph{$h$-vector} of $P^n$.
     By our previous notation, let 
     $$f^{\Delta}_v = f_v\cap P^{\Delta}.$$
     \vskip .1cm
    
     In addition, let
    $\hat{f}_v$ denote the union of the relative interiors
    of those faces whose top vertex is $v$. $\hat{f}_v$ is obtained 
    from $f_v$ by removing all the faces of $f_v$ that
     do not contain $v$. So $\hat{f}_v$ is diffeomorphic to the ``quadrant'' ${\Bbb R}^{m(v)}_+$.
    Let $\hat{f}^{\Delta}_v = \hat{f}_v\cap P^{\Delta}$.
   \vskip .1cm
     
    By cutting $P^n$ properly, we can assume that $\phi$ is a generic function
     on $P^{\Delta}$ as well. Moreover for any $1\leq i \leq 2r$, we can assume 
     $\phi(u_{i,1})> \cdots > \phi(u_{i,n})$. Similarly to $P^n$, we can define 
     $f_{u_{i,j}}$ and $\hat{f}_{u_{i,j}}$ for $P^{\Delta}$.    
     Observe that by the new notation, 
          $$E_i = f_{u_{i,1}}, \ 1\leq i \leq 2r.$$      
     Moreover, for any $f_{v_i} \subset P^n$ with $\dim(f_{v_i})\geq 1$,
      there exists a unique $2\leq l_i\leq n$ so that
     $$ f^{\Delta}_{v_i} = f_{u_{i,l_i}}, \ \ \hat{f}^{\Delta}_{v_i} = \hat{f}_{u_{i,l_i}}.$$ 
      It is easy to see that 
      $\{ f^{\Delta}_{v_i} = f_{u_{i,l_i}}\, ;\, 1\leq i \leq 2r \ \text{and}\ 
        \dim(f^{\Delta}_{v_i}) \geq 1 \}$ are exactly all the faces
       in $\{ f_{u_{i,j}} \, ;\, 1\leq i \leq 2r, 1\leq j \leq n \}$ 
        that are not contained in any of $E_1,\cdots, E_{2r}$.
     \vskip .1cm
       
   
    For each vertex $u_{i,j}$ of $P^{\Delta}$, let
    $e_{i,j} = \pi_{\Delta}^{-1}(\hat{f}_{u_{i,j}})$.
    We see from the construction~\eqref{Equ:Glue-Back} that
     $\{ e_{i,j}\,;\, 1\leq i \leq 2r, 1\leq j \leq n\}$ is a cell decomposition of $W=M(P^{\Delta}, \lambda^{\Delta})$.
      Moreover, the cellular boundary of each $e_{i,j}$ is either empty or 
     lies in $\partial W$. 
     So the relative 
      cellular chain complex $C_*(W,\partial W) = C_*(W)\slash C_*(\partial W)$ 
      has a \emph{perfect basis}
      $$ \{ e_{i,l_i} =\pi_{\Delta}^{-1}(\hat{f}^{\Delta}_{v_i})\, ;\, 
       1\leq i \leq 2r \ \text{and}\ \dim(f^{\Delta}_{v_i}) \geq 1 \}. $$ 
     Here ``perfect'' means that the boundary of each $e_{i,l_i}$ 
       is trivial (relative to $\partial W$). Note
     $$\dim(e_{i,l_i})= \dim(\hat{f}^{\Delta}_{v_i}) = \dim(f^{\Delta}_{v_i}) = \dim(f_{v_i})=
    m(v_i).$$
    So by~\eqref{Equ:fact}, for any $1\leq j \leq n$,
     $H_j(W,\partial W) = C_j(W,\partial W) \cong ({\Bbb F}_2)^{h_j}$ is generated by 
    $$ \{ e_{i,l_i};\, \dim(e_{i,l_i})= m(v_i)=j\}.$$   
  Therefore, the image $\mathrm{Im}\big(H_j(W,\partial W)\rightarrow H_{j-1}(\partial W) \big)$ is 
    spanned by the homology classes
     \begin{equation}\label{Equ:Span-basis}
        \{ [\partial(e_{i,l_i})] \in H_{j-1}(\partial W)\, ;\,  \dim(e_{i,l_i})=j \}.
      \end{equation}  
   Note that the homology class defined by 
   $e_{i,l_i}=\pi_{\Delta}^{-1}(\hat{f}^{\Delta}_{v_i})$ coincides with
   $M_{f^{\Delta}_{v_i}} = \pi_{\Delta}^{-1}(f^{\Delta}_{v_i})$. So
   the image $\mathrm{Im}\big(H_j(W,\partial W)\rightarrow H_{j-1}(\partial W) \big)$ is 
    spanned by the homology classes 
   \begin{equation}\label{Equ:Span-basis-2}
     \{ [\partial M_{f^{\Delta}_{v_i}}]\in H_{j-1}(\partial W) \,;\, f_{v_i}\ 
      \text{is any face of}\ P^n\ \text{with}\ \dim(f_{v_i})=m(v_i)=j \}.
     \end{equation}
   Then our claim follows.\vskip .1cm

   
    By the above claim and the identification
    of $H_{j-1}(\partial W) = H_{j-1}(\bigcup {\Bbb R} P^{n-1}_i)$ 
        with the subgroup ${\Bbb F}_2^{2r}\otimes t^{j-1}$ in 
        ${\Bbb F}^{2r}_2\otimes {\Bbb F}_2[t] =H^*_{G_{\tau}}(M^{G_{\tau}}) $, the image
        $\mathrm{Im}\big(H_j(W,\partial W)\rightarrow H_{j-1}(\partial W) \big)$ is 
    spanned by $\{ \xi_f \otimes t^{j-1}\, ;\, f\ \text{is any face of}\ P^n\ \text{with}\ \dim(f)=j\}$.
     So by~\eqref{Equ:C_M}, the space $V^{M}_k$ is spanned by 
   $\{ \xi_f\, ;\, f\ \text{is any face of}\ P^n\ \text{with}\ n-k\leq \dim(f) \leq n\}$.
   So we have $V^{M}_k = \mathfrak{B}_0(P) +  \mathfrak{B}_1(P) +\cdots + \mathfrak{B}_k(P)
    = \mathfrak{B}_k(P)$ by Lemma~\ref{Lem:Inclusion}. This finishes the 
    proof of Theorem~\ref{Thm:Main-1}.
  \hfill $\qed$ 
  
  \vskip .2cm
   
   
 
  According to~\eqref{Equ:Span-basis-2}, the linear space $V^{M}_k$ is actually
  spanned by a smaller set
  \begin{equation} \label{Equ:Span-Basis-3}
    \mathcal{A}_k =  \{ \xi_{f_v}\,;\,  v\ \text{is any vertex of}\ P^n  \ 
       \text{with}\ n-k \leq m(v)\leq n,  \} \subset  {\Bbb F}_2^{2r}.
   \end{equation}    
  The order of $\mathcal{A}_k$ is equal to the number of vertices $v\in V(P^n)$ with
   $n-k\leq m(v) \leq n$. So
   $$ |\mathcal{A}_k| = h_{n-k}(P^n) + \cdots + h_{n}(P^n) = h_k(P^n) +\cdots + h_0(P^n)$$
   by Dehn-Sommerville equation.
   On the other hand, Theorem~\ref{ring} tells us that
  $$\dim_{{\Bbb F}_2} V^M_k=\sum\limits_{j=0}^k b_j(M;\Bbb F_2)= \sum\limits_{j=0}^k h_j(P^n)
  =|\mathcal{A}_k|.$$ 
  This implies that $\mathcal{A}_k$ is a linear basis of $ V^M_k$.
  So we have the following proposition.
  
  \begin{prop} \label{Prop:Code_Basis}
   Let $\pi: M^n\rightarrow P^n$ be an $n$-dimensional small cover which admits a regular $\mathrm{m}$-involution.
  For any $0\leq k \leq n$,
  the set $\mathcal{A}_k$ is a linear basis of $ V^M_k$. 
  In particular when $n$ is odd, $\mathcal{A}_{\frac{n-1}{2}}$ is a linear basis of the self-dual binary code $C_{M^n}$. 
  \end{prop}
  
\vskip .4cm

 
 \subsection{The second proof of Theorem~\ref{Thm:Main-1}} \ \vskip .1cm
 
  According to Subsection 2.1(R4), 
  the cohomology ring  $H^*(M; {\Bbb F}_2)$ is generated by degree $1$ elements.
  So as an algebra over $H^*(BG_\tau; {\Bbb F}_2)={\Bbb F}_2[t]$,
   the equivariant cohomology ring 
   $H^*_{G_\tau}(M; {\Bbb F}_2)=H^*(M;{\Bbb F}_2)\otimes H^*(BG_\tau; {\Bbb F}_2)$ is 
   generated by elements of degree $1$. Since the operation $\circ$ on ${\Bbb F}^{2r}_2$ corresponds to
the cup product in $H^*_{G_\tau}(M;{\Bbb F}_2)$, 
 it follows from Theorem~\ref{ring} that for any $1\leq k \leq n$,
   $V^M_k =  \underset{k}{\underbrace{V^M_1\circ\cdots \circ V^M_1}}$.
   On the other hand, \vskip .1cm
              
 \textbf{Claim:}
   $\mathfrak{B}_k(P) =  \underset{k}{\underbrace{\mathfrak{B}_1(P)\circ\cdots\circ \mathfrak{B}_1(P)}}$.
  
   Indeed, for any facets $F_{i_1},\cdots, F_{i_k}$ of $P$, 
   their intersection $F_{i_1}\cap\cdots\cap F_{i_k}$
   is either empty or a face of codimension $k$. So by~\eqref{Equ:Product-Face}, we have
   $\xi_{F_{i_1}}\circ\cdots\circ \xi_{F_{i_k}} = \xi_{F_{i_1}\cap\cdots\cap F_{i_k}} \in \mathfrak{B}_{k}(P)$. 
   Conversely, 
    any codimension-$k$ face $f$ of $P$ can be written as 
 $f=F_{i_1}\cap\cdots\cap F_{i_k}$ where $F_{i_1}, ..., F_{i_k}$ are $k$ different facets of $P$. 
  So 
  $\xi_f=\xi_{F_{i_1}\cap\cdots\cap F_{i_k}}=\xi_{F_{i_1}}\circ\cdots\circ \xi_{F_{i_k}}$. 
 The Claim is proved.\vskip .1cm
   
  So to prove Theorem~\ref{Thm:Main-1}, it is sufficient to prove that 
   $V^M_1 = \mathfrak{B}_1(P)$, i.e., $V^M_1$ is spanned by the set
   $\{ \xi_F\,;\, F\  \text{is any facet of $P$} \}$.
   Next, we examine the localization of $H^1_{{\Bbb Z}_2^n}(M)$ to
 $H^1_{{\Bbb Z}_2^n}(M^{{\Bbb Z}_2^n})$ more carefully.\vskip .1cm
 
    
  Let $\mathcal{F}(P)=\{F_1, ..., F_m\}$ be the set of all facets of $P$.
  By our previous notations, 
   the regular involution $\tau = \tau_{e_1+\cdots+e_n}$ on $M$ only has isolated fixed points 
    where 
    $$M^{G_{\tau}}=M^{{\Bbb Z}_2^n} =\{ \tilde{v}_1,\cdots, \tilde{v}_{2r} \}.$$ 
        
 Clearly the inclusion $G_\tau\hookrightarrow {\Bbb Z}_2^n$ induces the diagonal maps
 $\Delta_E: EG_\tau\longrightarrow E{\Bbb Z}_2^n$ and $\Delta_B: BG_\tau\longrightarrow B{\Bbb Z}_2^n$ such that the following diagram commutes
 $$\CD
  EG_\tau  @>\Delta_E >> E{\Bbb Z}_2^n \\
  @V  VV @V  VV  \\
  BG_\tau  @>\Delta_B >> B{\Bbb Z}_2^n.
\endCD$$
Since $M^{G_\tau}=M^{{\Bbb Z}_2^n}$ consists of isolated points,  
 we have a commutative diagram
$$\xymatrix{
  EG_\tau\times M^{G_\tau} \ar[rrr]^{\Delta_E\times\text{id}}
  \ar[dr]^{i_1} \ar[ddd]_{} & & &
  E{\Bbb Z}_2^n\times M^{{\Bbb Z}_2^n} \ar[dl]_{i_2} \ar[ddd]^{ }\\
  & EG_\tau\times M \ar[r]^{\Delta_E\times \text{id} }
  \ar[d]_{}
      & E{\Bbb Z}_2^n\times M \ar[d]^{ } &      \\
  & EG_\tau\times_{G_\tau} M \ar[r]^{\phi }&
    E{\Bbb Z}_2^n\times_{{\Bbb Z}_2^n} M  &  \\
  EG_\tau\times_{G_\tau} M^{G_\tau}\ar[ur]^{i_3} \ar[rrr]_{\Delta_B\times\text{id}=\psi} & &  &
  E{\Bbb Z}_2^n\times_{{\Bbb Z}_2^n} M^{{\Bbb Z}_2^n}\ar[ul]_{i_4}        }$$
where  $\phi$ is the map
induced by $\Delta_E\times \text{id}$, and $i_1, i_2, i_3, i_4$ are all inclusions. Furthermore, we have the following commutative diagram
\begin{equation}\label{graph}
\xymatrix{
  H^*_{{\Bbb Z}_2^n}(M;{\Bbb F}_2) \ar[r]^{\  \phi^*}\ar[d]_{i_4^*}\ar[dr]_g &
  H^*_{G_\tau}(M; {\Bbb F}_2)\ar[d]^{i_3^*}\\
  H^*_{{\Bbb Z}_2^n}(M^{{\Bbb Z}_2^n};{\Bbb F}_2) \ar[r]_{\ \psi^*} &
  H^*_{G_\tau}(M^{G_\tau};{\Bbb F}_2)
}
\end{equation}
 where $g=i_3^*\phi^*=\psi^*i_4^*$. Note that $i_3^*$ and $i_4^*$ are injective, and  
 $$H^*_{{\Bbb Z}_2^n}(M^{{\Bbb Z}_2^n};{\Bbb F}_2)
\cong \bigoplus_{v\in V(P)}H^*_{{\Bbb Z}_2^n}(\tilde{v};{\Bbb F}_2),\ \ 
 H^*_{G_\tau}(M^{G_\tau};{\Bbb F}_2)
\cong \bigoplus_{ v\in V(P)} H^*_{G_\tau}(\tilde{v};{\Bbb F}_2),$$ 
where $\tilde{v}=\pi^{-1}(v)$ is the fixed point corresponding to a vertex $v\in V(P)$.
Then by the fact that
 $H^*_{{\Bbb Z}_2^n}(\tilde{v};{\Bbb F}_2)\cong H^*(B{\Bbb Z}_2^n;{\Bbb F}_2)$ and $H^*_{G_\tau}(\tilde{v};{\Bbb F}_2)\cong H^*(BG_\tau;{\Bbb F}_2)$, we 
 can regard $\psi^*$ as a direct sum:
   $$\psi^* = \bigoplus_{v\in V(P)}\Delta_B^*.$$
It is well known that $H^*(B{\Bbb Z}_2^n;{\Bbb F}_2)={\Bbb F}_2[t_1, ..., t_n]$ with $\deg t_i=1$, and $H^*(BG_\tau;{\Bbb F}_2)={\Bbb F}_2[t]$ with $\deg t=1$. Then we know by K\"unneth theorem that $t_i$ is the cross product $\underbrace{1\times\cdots \times 1}_{i-1}\times t\times 1\times\cdots\times 1$ where $1\in {\Bbb F}_2$. Furthermore, by \cite[Theorem 61.3]{m} we have 
  $$\Delta_B^*(t_i)=t, \ 1\leq i \leq n.$$
 For any fixed point $\tilde{v}\in M^{{\Bbb Z}_2^n}$, the inclusion $i_{\tilde{v}}:\{\tilde{v}\}\hookrightarrow M$ induces a homomorphism 
  $$i_{\tilde{v}}^*: H^*_{{\Bbb Z}_2^n}(M;{\Bbb F}_2)\cong {\Bbb F}_2[a_{F_1}, ..., a_{F_m}]/I\longrightarrow H^*(B{\Bbb Z}_2^n;{\Bbb F}_2)={\Bbb F}_2[t_1, ..., t_n].$$ 
 Then we can write 
   $$i_4^*=\bigoplus_{v\in V(P)}i^*_{\tilde{v}}.$$\vskip .1cm
  
  
\begin{lem} \label{Lem:Local}
 Let $\lambda$ be the characteristic function of the small cover $M$
 so that $\mathrm{Im}(\lambda) =\{e_1,\cdots, e_n\}$ is a basis of ${\Bbb Z}_2^n$.
Suppose $F$ is a facet of $P$ with $\lambda(F) = e_{j}$ for some $1\leq j\leq n$.
 Then for any vertex $v$ of $P$, the fixed point $\tilde{v} = \pi^{-1}(v) \in M^{{\Bbb Z}_2^n}$  satisfies:
$$i_{\tilde{v}}^*(a_F)= \left\{\begin{array}{ll}
   t_{j}, &\text{ if } v \in F  ;\\
    0,           &\text{ if } v\notin F.
                              \end{array}
\right.
$$

\end{lem}

\begin{proof}
 For convenience, let $M_F=\pi^{-1}(F)$. Set $i_{M_F}: M_F
\hookrightarrow M.$ It is well known that 
   $(i_{M_F})_!(1)=a_{F}\in H^1_{{\Bbb Z}_2^n}(M)$ (recall that $a_F$ is the equivariant Euler
   class of the normal bundle of $M_F$ in $M$). In addition 
   according to~\cite[(5.3.14)]{ap}, if $ \tilde{v} \notin M_F$, 
 we have $$i^*_{\tilde{v}}(i_{M_F})_!(1)=0.$$
So for any vertex $v \notin F$, we get
$i^*_{\tilde{v}}(a_F)=0 \in {\Bbb F}_2[t_1,\cdots,t_n]$. \vskip .1cm

 Next, we compute $i^*_{\tilde{v}}(a_{F})$ for any vertex $v\in F$.
Since $\tilde{v}$ is a fixed point of ${\Bbb Z}_2^n$, the inclusion $i_{\tilde{v}}: 
  \{\tilde{v}\}\hookrightarrow M$ induces an inclusion
   between the Borel constructions 
     $$i_{\tilde{v}}: \{\tilde{v}\}_{{\Bbb Z}_2^n}
\hookrightarrow M_{{\Bbb Z}_2^n}.$$ 
On the other hand, we have a canonical projection
on the Borel constructions
 $$\mathfrak{p}: M_{{\Bbb Z}_2^n} \rightarrow
\{\tilde{v}\}_{{\Bbb Z}_2^n}.$$
It is clear that $\mathfrak{p}\circ i_{\tilde{v}} = id_{\{\tilde{v}\}_{{\Bbb Z}_2^n}}$, and 
    $\{\tilde{v}\}_{{\Bbb Z}_2^n} \cong B{\Bbb Z}_2^n$. The induced maps $\mathfrak{p}^*$ and
    $i_{\tilde{v}}^*$ in the cohomology of degree one
    are:
$$\xymatrix{\text{id}: &
H^1(B{\Bbb Z}_2^n)\ar[r]^{\mathfrak{p}^*} \ar@{=}[d] &
H^1_{{\Bbb Z}_2^n}(M)\ar[r]^{i_{\tilde{v}}^*} \ar@{=}[d]&
H^1(B{\Bbb Z}_2^n)\ar@{=}[d]\\
\text{id}:& \text{span}\{t_1,\cdots,t_n\}\ar[r]^{\mathfrak{p}^*} &
\text{span}\{a_{F_1},\cdots, a_{F_m}\}\ar[r]^{\ \; i_{\tilde{v}}^*} &
\text{span}\{t_1,\cdots,t_n\} }
   $$

Since
$\text{Im}(\lambda)=\{e_1,...,e_n\}$ is a basis of
${\Bbb Z}_2^n$, by the argument in \cite[p.438]{dj}, we get
$$\mathfrak{p}^*(t_j)=\lambda^*(t_j)=\sum_{\lambda(F_l)=e_j}a_{F_l},$$
where the characteristic map $\lambda$ of the small cover $\pi: M \rightarrow P$, is regarded as a linear map  $$\lambda:{\Bbb Z}_2^m=\text{span}\{F_1,\cdots,F_m\} \rightarrow {\Bbb Z}_2^n=\text{span}\{e_1,\cdots,e_n\}$$
 with a representing matrix
$A_{n\times m}=(\lambda(F_1),\cdots,\lambda(F_m))$, and its dual $$\lambda^*:\text{span}\{t_1,\cdots,t_n\} \rightarrow \text{span}\{a_{F_1},\cdots,a_{F_m}\}$$
with a representing matrix $A^t$ (the transpose of $A$).
So
$$ t_j=i_{\tilde{v}}^*(\mathfrak{p}^*(t_j))=i_{\tilde{v}}^*\Big(\sum_{\lambda(F_l)=e_j} a_{F_l}\Big)=\sum_{\lambda(F_l)=e_j} i_{\tilde{v}}^*(a_{F_l}) = \sum_{v\in F_l, \, \lambda(F_l)=e_j}
     i_{\tilde{v}}^*(a_{F_l}).$$
 Observe that among all the $n$ facets containing $v$, there is only one facet (i.e. $F$) 
   colored by $e_j$. So we obtain
$$  \sum_{v\in F_l, \, \lambda(F_l)=e_j}
     i_{\tilde{v}}^*(a_{F_l}) = i_{\tilde{v}}^*(a_{F})=t_j.$$
The lemma is proved.
\end{proof}

\vskip 1cm

  Now suppose $\lambda(F) = e_j$. We get from Lemma~\ref{Lem:Local} that
\begin{equation} \label{Equ:i4}
i_4^*(a_F)=\bigoplus_{v\in V(P)} i_{\tilde{v}}^*(a_F)= \sum_{v\in F} t_j\xi_{v} = t_{j} \xi_F. 
\end{equation}
Recall that $\xi_v$ denotes the vector in ${\Bbb Z}_2^{2r}={\Bbb Z}_2^{|V(P)|}$ 
 with $1$ at the coordinate
corresponding to $v$ and zero everywhere else. 
 \vskip .1cm
 
 Combining~\eqref{Equ:i4} with $\psi^*=\bigoplus_{v\in V(P)}\Delta_B^*$, we obtain
\begin{equation}\label{f}
g(a_F)=\psi^*i_4^*(a_F) =t\xi_F.
\end{equation}
So $g(H^1_{{\Bbb Z}_2^n}(M;{\Bbb F}_2))= t\mathfrak{B}_1(P)$ since $g$ is a graded ring homomorphism.
According to Proposition 4.3 in the next section, 
 $\text{dim}_{{\Bbb F}_2}\mathfrak{B}_1(P)=m-n+1$ where $m$ is the number of facets of $P$.
 So
$\text{dim}_{{\Bbb F}_2}g(H^1_{{\Bbb Z}_2^n}(M;{\Bbb F}_2)) = m-n+1$.
 In addition, the commutativity of the diagram~\eqref{graph} implies that
$g(H^1_{{\Bbb Z}_2^n}(M;{\Bbb F}_2))\subset i_3^*(H^1_{G_{\tau}}(M;{\Bbb F}_2))$.\vskip .1cm

 On the other hand, since $i_3^*$ is a monomorphism, 
 \begin{align*}
    \text{dim}_{{\Bbb F}_2}i_3^*(H^1_{G_{\tau}}(M;{\Bbb F}_2)) 
      = \text{dim}_{{\Bbb F}_2}H^1_{G_{\tau}}(M;{\Bbb F}_2) 
     &= b_0(M;{\Bbb F}_2)+b_1(M;{\Bbb F}_2) \ \ \text{(by Theorem~\ref{ring})}\\
    &= h_0(P)+h_1(P) \ \ \text{(by the property (R1))} \\
     &= 1 + (m-n)
 \end{align*}   
So
$\text{dim}_{{\Bbb F}_2}g(H^1_{{\Bbb Z}_2^n}(M;{\Bbb F}_2))
   =\text{dim}_{{\Bbb F}_2}i_3^*(H^1_{G_{\tau}}(M;{\Bbb F}_2))$. This implies that
$$ g(H^1_{{\Bbb Z}_2^n}(M;{\Bbb F}_2)) = i_3^*(H^1_{G_{\tau}}(M;{\Bbb F}_2)) \ \Longrightarrow  \
 \mathfrak{B}_1(P)= V_1^M.$$
The theorem is proved.  \qed \\
  
  
   \section{Binary codes from general simple polytopes}
  

  The definition of $\mathfrak{B}_k(P)$ in~\eqref{Equ:Def-Bk} clearly 
  makes sense for an arbitrary $n$-dimensional simple polytope $P$. 
 We call $\mathfrak{B}_k(P) \subset {\Bbb F}_2^{|V(P)|}$
  the {\em codimension-$k$ face code of $P$}.
 It is obvious that $\mathfrak{B}_0(P)=\{\underline{0}, \underline{1}\}\cong {\Bbb F}_2$, and $\mathfrak{B}_n(P)\cong {\Bbb F}_2^{|V(P)|}$ where
  $$ \underline{0}= (0,\cdots , 0),\ \  \underline{1}=(1,\cdots ,1).$$
 In this section, we study the properties of $\mathfrak{B}_k(P)$ and investigate what 
  kind of simple polytope $P$ will the code
   $\mathfrak{B}_k(P)$ be self-dual (see Theorem~\ref{Thm:Main-3}). The arguments in this section
   are completely combinatorial and are independent from the discussion of small covers in the previous section.
   
\vskip .1cm
 
 For each $0\leq k\leq n$, $\mathfrak{B}_k(P)$ determines a matrix $M_k(P)$ with columns  
$\xi_f\in \mathfrak{B}_k(P)$ with respect to an ordering of all the vertices and
 all the codimension-$k$ faces $f$ of $P$.
 We call $M_k(P)$ the {\em code matrix of codimension-$k$ faces} of $P$.

\begin{exam}

Under the labeling of the vertices of the $6$-prism $P$ in Figure~\ref{6prism}, 
the code matrix $M_1(P)$ is a $12\times 8$ binary matrix shown in Figure~\ref{6prism}.
\begin{figure}[h]
\includegraphics[width=0.55\textwidth]{6-Prism.eps}
\caption{A $6$-prism and its code matrix of codimension-$1$ faces}
\label{6prism}
\end{figure}

\end{exam}

\vskip .1cm

   
\begin{lem}\label{dim} Let $P$ be an $n$-dimensional simple polytope with $m$ facets. Then
$$\dim_{{\Bbb F}_2} \mathfrak{B}_1(P)\geq m-n+1.$$
\end{lem}

\begin{proof} 
Let $\{ F_1, ..., F_m\}$ be all the faces of $P$. 
Without the loss of generality, assume that $F_1, ..., F_n$ intersect at a vertex $v$ of $P$.
We claim that $\xi_{F_n},\xi_{F_{n+1}}, ..., \xi_{F_m}$ are linearly independent in 
$\mathfrak{B}_1(P)$, so $\dim_{{\Bbb F}_2} \mathfrak{B}_1(P)\geq m-n+1$.

Assume $ \sum\limits_{i=n}^m \epsilon_i \xi_{F_i} =\underline{0}$, where
$\epsilon_i\in {\Bbb F}_2$. We need to show that $\epsilon_i=0$ for all $n\leq i\leq m$.
 Indeed, by multiplying both sides of $\sum\limits_{i=n}^m \epsilon_i \xi_{F_i} =\underline{0}$ by $\xi_{F_1\cap\cdots\cap F_{n-1}}$,
we obtain 
$$\sum\limits_{i=n}^m \epsilon_i \xi_{F_1\cap\cdots\cap F_{n-1}}\circ \xi_{F_i}= \sum\limits_{i=n}^m \epsilon_i \xi_{F_1\cap\cdots\cap F_{n-1}\cap F_i} =\underline{0}.$$
 For any $i>n$,  we have $\xi_{F_1\cap\cdots\cap F_{n-1}\cap F_i}=\underline{0}$
 since $F_1\cap\cdots\cap F_{n-1}\cap F_i=\emptyset$. So we obtain
  $\epsilon_n\xi_{F_1\cap\cdots\cap F_{n-1}\cap F_n}=\epsilon_n\xi_v = \underline{0}$, which implies $\epsilon_n=0$.\vskip .1cm

 Now let $e$ be an edge with $v$ as an endpoint. Let $v'$ be another endpoint of $e$. Then $e=F_{i_1}\cap\cdots\cap F_{i_{n-1}} $ for some $\{i_1,\cdots,i_{n-1}\}\subset \{1,2,\cdots,n \}$. Without loss of generality, assume that $v'=F_{i_1}\cap\cdots\cap F_{i_{n-1}}\cap F_{n+1}$.
 By multiplying both sides of $\sum\limits_{i=n}^m \epsilon_i \xi_{F_i} =\underline{0}$ by
  $\xi_{F_{i_1}\cap\cdots\cap F_{i_{n-1}}}$, we obtain $\epsilon_{n+1}=0$. Moreover, if 
  we consider any other edge $e'$ with $v'$ as an endpoint, the similar argument shows 
  that $\epsilon_i =0$ where $F_i$ intersects $e'$ transversely at the other end. 
  Since the
1-skeleton of $P$ is connected, we can show that  $\epsilon_i=0$ for all $n\leq i\leq m$
by repeating the above argument.
\end{proof}

Next let us look at what happens when $\dim_{{\Bbb F}_2} \mathfrak{B}_1(P)= m-n+1$.
\begin{prop}\label{dim1}
Let $P$ be an $n$-dimensional simple polytope with $m$ facets. Then $\dim_{{\Bbb F}_2} \mathfrak{B}_1(P)= m-n+1$ if and only if $P$ is $n$-colorable.
\end{prop}

\begin{proof}
Let $\{ F_1, ..., F_m\}$ be all the faces of $P$.
Suppose $P$ is $n$-colorable. Then $P$ admits a coloring $\lambda: \mathcal{F}(P)\rightarrow {\Bbb Z}_2^n$ such that
the image $\mathrm{Im} \lambda$  is a basis  $\{e_1,\cdots,e_n\}$ in ${\Bbb Z}_2^n$.
Set $$\mathcal{F}_i=\{F\in \mathcal{F}(P)\,|\, \lambda(F)=e_i \}, i=1,2,\cdots,n.$$
 By the definition of $\lambda$, 
  each vertex of $P$ is incident to exactly one facet in $\mathcal{F}_i$. So we have
\begin{equation}\label{eq}
\bigcup_{F\in \mathcal{F}_i} V(F)=V(P), \ \ \sum_{F\in \mathcal{F}_i}\xi_F
   =\sum_{v\in V(P)}\xi_v=\underline{1}.
\end{equation}

Without loss of generality, assume that the facets $F_1, ..., F_n$ meet at a vertex, and $\lambda(F_i)=e_i$, $i=1, ..., n$.  We know from the proof of Lemma~\ref{dim} that
$\xi_{F_n}, \xi_{F_{n+1}}, ..., \xi_{F_m}$ are linearly independent in $\mathfrak{B}_1(P)$.
We claim that 
for each $1\leq i\leq n-1$, $\xi_{F_i}, \xi_{F_n}, \xi_{F_{n+1}}, ..., \xi_{F_m}$ are linearly dependent in $\mathfrak{B}_1(P)$. Indeed, it follows from~\eqref{eq} that  
 \begin{equation} \label{Equ:F-eq}
   \sum\limits_{F\in \mathcal{F}_i} \xi_{F} + \sum\limits_{F\in \mathcal{F}_n} \xi_{F}=\underline{1}+\underline{1}=\underline{0}.
   \end{equation}
Observe that
$$ \{\xi_F| F\in\mathcal{F}_i\}\subset \{\xi_{F_i},\xi_{F_n},\xi_{F_{n+1}},\cdots
\xi_{F_m}\},\ \ \{\xi_F| F\in\mathcal{F}_n\}\subset \{\xi_{F_i},\xi_{F_n},\xi_{F_{n+1}},\cdots
\xi_{F_m}\}.$$
 So~\eqref{Equ:F-eq} implies that $\xi_{F_i}, \xi_{F_n}, \xi_{F_{n+1}}, ...,  \xi_{F_m}$ are linearly dependent. Then by Lemma~\ref{dim}, we can conclude 
 $\dim_{{\Bbb F}_2} \mathfrak{B}_1(P)= m-n+1$.

\vskip .2cm
Conversely, suppose $\text{dim}_{{\Bbb F}_2} \mathfrak{B}_1(P)=m-n+1$.
If $P$ is not $n$-colorable, by Theorem~\ref{j} there exists a $2$-face $f^2$ of $P$ which
has odd number of vertices, say $v_1,\cdots,v_{2k+1}$. Without loss of generality, assume that  $f^2=F_1\cap\cdots\cap F_{n-2}$ and $v_1=F_1\cap F_2\cap\cdots\cap F_n$. By the proof of
 Lemma~\ref{dim}, $\{ \xi_{F_n},\xi_{F_{n+1}}, ..., \xi_{F_m} \}$ is
  a basis of $\mathfrak{B}_1(P)$.
Without loss of generality, we may assume that (see Figure~\ref{p:Proof-1})
\begin{equation*}\label{2}
\begin{cases}
v_1=f^2\cap F_{n-1}\cap F_n\\
  v_2=f^2\cap F_n\cap F_{n+1}\\
  \cdots\\
  v_i=f^2\cap F_{n+i-2}\cap F_{n+i-1}\\
  \cdots\\
  v_{2k}=f^2\cap F_{n+2k-2} \cap F_{n+2k-1}\\
  v_{2k+1}=f^2\cap F_{n+2k-1} \cap F_{n-1}\\
\end{cases}
\end{equation*}

\begin{figure}[h]
\includegraphics[width=0.35\textwidth]{Proof-1.eps}
\caption{A face $f$ with odd number of vertices}
\label{p:Proof-1}
\end{figure}

Assume that there exists $\epsilon_i\in {\Bbb F}_2$, $i=1, n, ..., m$ so that
\begin{equation}\label{2}
  \epsilon_1\xi_{F_1}+\epsilon_n\xi_{F_n}+\epsilon_{n+1}\xi_{F_{n+1}}+\cdots++\epsilon_m\xi_{F_m}
   =\underline{0}.
\end{equation}
 For each $i$, by taking the inner product with $\xi_{v_i}$
  on both sides of (\ref{2}),  we get 
 \begin{equation}\label{3}
   \begin{cases}
  \epsilon_1 +\epsilon_n =0,\\
  \epsilon_1+\epsilon_n+\epsilon_{n+1} =0,\\
  \epsilon_1+\epsilon_{n+1}+\epsilon_{n+2} =0,\\
  \cdots\\
  \epsilon_1+\epsilon_{n+i-2}+\epsilon_{n+i-1} =0,\\
  \cdots\\
  \epsilon_1+\epsilon_{n+2k-2}+\epsilon_{n+2k-1} =0,\\
  \epsilon_1+\epsilon_{n+2k-1} =0.\\
   \end{cases}
    \end{equation}
The coefficient matrix of the above linear system is a $(2k+1)\times (2k+1)$ matrix
over ${\mathbb F}_2$.
 $$
\begin{pmatrix}
1&1&0&0&0&\cdots&0&0&0\\
1&1&1&0&0&\cdots&0&0&0\\
1&0&1&1&0&\cdots&0&0&0\\
1&0&0&1&1&\cdots&0&0&0\\
&&&&......\\
1&0&0&0&0&\cdots&1&1&0\\
1&0&0&0&0&\cdots&0&1&1\\
1&0&0&0&0&\cdots&0&0&1
\end{pmatrix}_{(2k+1)\times (2k+1)}.$$
 It is easy to show that the determinant of this matrix is $1$. So  
 the linear system~\eqref{3} only has zero solution, which implies that
  $\xi_{F_1},\xi_{F_n}, ..., \xi_{F_m}$ are linearly independent.
 Then we have $\dim_{{\Bbb F}_2} \mathfrak{B}_1(P)\geq m-n+2$. But this contradicts our
  assumption that $\dim_{{\Bbb F}_2} \mathfrak{B}_1(P)=m-n+1$. So the proposition is proved.
\end{proof}

 Next, we study when $\mathfrak{B}_k(P)$ can be a self-dual code. 
 Note that if for some $0\leq k\leq n$, $\mathfrak{B}_k(P)$ is a self-dual binary code, 
 it is necessary that $|V(P)|$ is even.

\begin{lem} \label{Lem:Polytope-Colorable}
 Let $P$ be an $n$-dimensional simple polytope with $2r$ vertices. 
 If for some $0\leq k \leq n$, $\mathfrak{B}_k(P)$ is a self-dual code in ${\Bbb F}^{2r}_2$,
  then $P$ is $n$-colorable.
\end{lem}

\begin{proof}
 Since the lemma is trivial for $n\leq 2$, we assume $n\geq 3$ in the rest. 
 It is clear that $\mathfrak{B}_n(P)\cong {\Bbb F}^{2r}_2$
 is never self-dual in ${\Bbb F}^{2r}_2$.
 Next, we show that $\mathfrak{B}_{n-1}(P)$ can not be a self-dual code. In fact, for any
two edges $f_1, f_2$ of $P$ with a common vertex, we have
$\langle \xi_{f_1}, \xi_{f_2} \rangle =1$, so $\mathfrak{B}_{n-1}(P)$ is not
self-dual. So if $\mathfrak{B}_k(P)$ is a self-dual code, then $0 \leq k \leq n-2$. 
For any 2-face $f$ of $P$, we can write $f$ as the intersection of
$n-k-2$ faces $H_1, ..., H_{n-k-2}$ of codimension $k$. By Lemma~\ref{code}(C3), 
 we have $\xi_f=\xi_{H_1}\circ \cdots\circ \xi_{H_{n-k-2}}\in \mathcal{V}_{2r}$. This implies 
 that $f$ has an even number of vertices. So $P$ is $n$-colorable by Theorem~\ref{j}.
\end{proof}

 The above discussion suggests several new ways to judge whether
 a simple $n$-polytope $P$ is $n$-colorable in terms of 
 $\mathfrak{B}_k(P)$.

\begin{prop}\label{collection}
Let $P^n$ be an $n$-dimensional simple polytope with $m$ facets. 
Then the following statements are equivalent.
\begin{itemize}
 \item[(1)] $P^n$ is $n$-colorable. \vskip .1cm
 \item[(2)] There exists  a partition $\mathcal{F}_1, ..., \mathcal{F}_n$ of the set $\mathcal{F}(P^n)$ of all facets, such that for each $1\leq i\leq n$, all the 
 facets in $\mathcal{F}_i$ are pairwise disjoint and $\sum_{F\in \mathcal{F}_i}\xi_F=\underline{1}$ (i.e.,
   each vertex of $P^n$ is incident to exactly one facet from every $\mathcal{F}_i$). \vskip .1cm
 \item[(3)] $\mathfrak{B}_0(P^n) \subset \mathfrak{B}_1(P^n) \subset\cdots\subset \mathfrak{B}_{n-1}(P^n)\subset \mathfrak{B}_n(P^n)\cong {\Bbb F}_2^{|V(P^n)|}.$ \vskip .1cm
 \item[(4)] $\mathfrak{B}_{n-2}(P^n)\subset \mathfrak{B}_{n-1}(P^n)$. \vskip .1cm
 \item[(5)] $\dim_{{\mathbb F}_2} \mathfrak{B}_1(P^n)= m-n+1$.
  \end{itemize}
\end{prop}

\begin{proof}
 It is easy to verify the above equivalences when $n\leq 2$. So we assume $n\geq 3$ below.
 In the proof of Proposition~\ref{dim1}, we have proved 
  $(1)\Rightarrow (2)$ and $(1)\Leftrightarrow (5)$. 

\vskip .2cm
Now we show that $(2)\Rightarrow (3)$. By the condition in (2), we clearly have
 $$\mathfrak{B}_0(P^n) \subset \mathfrak{B}_1(P^n), \ \ 
  \mathfrak{B}_{n-1}(P^n) \subset \mathfrak{B}_{n}(P^n).$$
It remains to show that $\mathfrak{B}_k(P^n) \subset \mathfrak{B}_{k+1}(P^n)$ for
each $1\leq k\leq n-2$. Let $f^{n-k}$ be a codimension-$k$ face of $P^n$. 
Without the loss of generality, we assume that 
   $$f^{n-k}=F_1\cap F_2\cap\cdots\cap F_k, \ \text{where}\ 
     F_i\in \mathcal{F}_i,\ i=1, ..., k.$$
For each $j=k+1,\cdots,n$, we have that
 $$\sum_{F\in \mathcal{F}_j}\xi_{F\cap f^{n-k}}=
   \sum_{F\in \mathcal{F}_j}\xi_{F}\circ\xi_{f^{n-k}}
  =\xi_{f^{n-k}}\circ(\sum_{F\in \mathcal{F}_j}\xi_{F} )
  =\xi_{f^{n-k}}\circ \underline{1}=\xi_{f^{n-k}}.$$
In the above equality, if $F\cap f^{n-k}=\emptyset$, then $\xi_{F\cap f^{n-k}}=\xi_\emptyset=\underline{0}$.
If $F\cap f^{n-k}\neq \emptyset$, then $F\cap f^{n-k}$ is a face of
  codimension $k+1$. 
so $\xi_{F\cap f^{n-k}}\in \mathfrak{B}_{k+1}(P^n)$.
Thus we get $\xi_{f^{n-k}}=\sum_{F\in \mathcal{F}_j}\xi_{F\cap f^{n-k}} \in \mathfrak{B}_{k+1}(P^n)$.
This completes the proof of $(2)\Rightarrow (3)$.

\vskip .2cm

 It is trivial that $(3)\Rightarrow (4)$. Next we show $(4)\Rightarrow (1)$.
Assume $\mathfrak{B}_{n-2}(P^n)\subset \mathfrak{B}_{n-1}(P^n)$.
Notice that the number of nonzero coordinates in any vector in 
 $\mathfrak{B}_{n-1}(P^n)$ must be even. 
 So for any $2$-face $f^2$ of $P^n$, we have
  $\xi_{f^{2}}\in \mathfrak{B}_{n-2}(P^n)\subset \mathfrak{B}_{n-1}(P^n)$, which
  implies that $f^2$ has an even number vertices. Hence $P^n$ is $n$-colorable by Theorem~\ref{j}.
\end{proof}

   \vskip .1cm
 
\begin{prop} \label{prop:dim}
 If $P^n$ is an $n$-dimensional $n$-colorable simple polytope, then
  $$\dim_{{\Bbb F}_2}\mathfrak{B}_k(P^n)=
  \sum_{i=0}^k h_i(P^n), \  0\leq k\leq n.$$
\end{prop}
\begin{proof}
Let $M^n$ be a small cover over $P^n$ whose characteristic function
 $\lambda: \mathcal{F}(P^n)\rightarrow {\Bbb Z}_2^n$ satisfies:
the image $\mathrm{Im} \lambda$  is a basis $\{e_1,\cdots,e_n\}$ in ${\Bbb Z}_2^n$.
By Theorem~\ref{Thm:Main-1}, the space $\mathfrak{B}_k(P^n)$ coincides with 
$V^M_k$ (see~\eqref{Equ:V_i}). 
So the proposition follows from Theorem~\ref{ring}.
\end{proof}
   
\vskip .1cm

\begin{thm} \label{Thm:Main-3}
  Let $P^n$ be an $n$-dimensional simple polytope with $2r$ vertices. Then $\mathfrak{B}_k(P^n)$ is a self-dual code in ${\Bbb F}^{2r}_2$ 
   if and only if $P^n$ is $n$-colorable, $n$ is odd and $k=\frac{n-1}{2}$.
 \end{thm}
\begin{proof}
   If $P^n$ is $n$-colorable and $n$ is odd, 
   the space $\mathfrak{B}_{\frac{n-1}{2}}(P^n)$ coincides with
   $V^M_{\frac{n-1}{2}}$ which is a self-dual 
   binary code by Corollary~\ref{Cor:Main-2}.\vskip .1cm
   
   Conversely, if $\mathfrak{B}_k(P^n)$ is a self-dual code in ${\Bbb F}^{2r}_2$ for 
   some $0< k < n$, Lemma~\ref{Lem:Polytope-Colorable} says that $P^n$ must be $n$-colorable.
    Then by Proposition~\ref{prop:dim}, 
     $$\dim_{{\Bbb F}_2} \mathfrak{B}_k(P^n) = \sum_{i=0}^k h_i(P^n)  = r = \frac{|V(P^n)|}{2}
      =  \frac{\sum_{i=0}^n h_i(P^n)}{2}.$$
   Then because $h_i(P^n) >0$ and
    the Dehn-Sommerville equations $h_i(P^n) = h_{n-i}(P^n)$ for any 
    $0\leq i \leq n$,  
   we deduce that $n$ must be odd and $k=\frac{n-1}{2}$.   
\end{proof}

 By comparing Theorem~\ref{Thm:Main-1} and Theorem~\ref{Thm:Main-3}, 
 we see that the set of self-dual binary codes 
 we can possibly obtain from small covers agree with those obtained from simple polytopes.
 \\
 
 \section{Self-dual binary codes from $3$-dimensional
 simple polytopes}

 \begin{prop} \label{prop:3-polytope}
 For any $3$-dimensional $3$-colorable simple polytope $P^3$, 
 the minimum distance of the self-dual code $\mathfrak{B}_1(P^3)$ is always equal to $4$.
 \end{prop}
\begin{proof}
 It is well known that any $3$-dimensional simple polytope must have a $2$-face
with less than $6$ vertices. Then since $P^3$ is even, there must be a $4$-gon $2$-face 
 in $P^3$. So by Corollary~\ref{Cor:Main-2}, the minimum distance of $\mathfrak{B}_1(P^3)$ is 
  less or equal to $4$. In addition, 
  we know that the Hamming weight of any element in $\mathfrak{B}_1(P^3)$
  is an even integer. So we only 
  need to prove that for any $2$-face $F_1,\cdots, F_k$ of $P^3$,  the Hamming weight of $\alpha=\xi_{F_1}+\cdots + \xi_{F_k}\in \mathfrak{B}_1(P^3)$ can not be $2$. 
  \vskip .1cm
 $\bullet$ Let $V(\alpha)$ be the union of all the vertices of $F_1,\cdots, F_k$.
 \vskip .1cm
 
 $\bullet$ Let $\Gamma(\alpha)$ be the union of all the vertices and edges of
   $F_1,\cdots, F_k$.
  So $\Gamma(\alpha)$ is a graph with vertex set $V(\alpha)$. \vskip .1cm
  
  A vertex $v$ in $V(\alpha)$ is called \emph{type-$j$} if $v$ is incident to
  exactly $j$ facets in $F_1,\cdots, F_k$. Then since $P^3$ is simple, 
 any vertex in $V(\alpha)$ is of type-$1$, type-$2$ or type-$3$ (see Figure~\ref{p:Graph}).
 Suppose there are $l_j$ vertices of type-$j$ in $V(\alpha)$, $j=1,2,3$.
 It is easy to see that
  the Hamming weight of $\alpha$ is equal to $l_1+l_3$. Assume that $wt(\alpha)=l_1+l_3 = 2$.
 Then we have three cases for $l_1$ and $l_3$:
   $$ 
    \mathrm{(a)}\ l_1=2, l_3=0;\ \ \ \mathrm{(b)} \ l_1=1, l_3=1; \ \ \ 
     \mathrm{(a)}\ l_1=0, l_3=2. $$
 
 
  Note that any vertex of type-$2$ or type-$3$ in $V(\alpha)$ meets exactly three edges 
  in $\Gamma(\alpha)$. In other words,
   $\Gamma(\alpha)$ is a graph whose vertices are all $3$-valent except the type-$1$ vertices.
  Let $\Gamma(P^3)$ denote the graph of $P^3$ (the union of all the vertices  and edges of $P^3$). and let $\overline{\Gamma}(\alpha) = \Gamma(P^3)\backslash \Gamma(\alpha)$.
Observe that 
  $\Gamma(\alpha)$ meets $\overline{\Gamma}(\alpha)$ only at the type-$1$ vertices in $V(\alpha)$.
  \vskip .1cm
 
 \begin{itemize}
    \item In the case (a), there are two type-$1$ vertices in 
   $V(\alpha)$, denoted by $v$ and $v'$. 
   Then since $\Gamma(\alpha)$ meets $\overline{\Gamma}(\alpha)$ only at $\{v, v'\}$, removing 
    $v$ and $v'$ from the graph $\Gamma(P^3)$ will disconnect $\Gamma(P^3)$
    (see Figure~\ref{p:Graph} for example). 
    But according to Balinski's theorem (see~\cite{bal}), 
  the graph of any $3$-dimensional simple polytope is a $3$-connected
  graph (i.e. removing any two vertices from the graph does not disconnect it). 
  So (a) is impossible.\vskip .1cm
  
  \item In the case (b), there is only one type-$1$ vertex in $V(\alpha)$, denoted by $v$.
  By the similar argument as above, removing 
    $v$ from the graph $\Gamma(P^3)$ will disconnect $\Gamma(P^3)$. This 
    contradicts the $3$-connectivity of $\Gamma(P^3)$. So (b) is impossible either.\vskip .1cm
    
     \item In the case (c), there are no type-$1$ vertices in $V(\alpha)$. So
  $\Gamma(\alpha)$ is a $3$-valent graph. This implies that $\Gamma(\alpha)$
  is the whole $1$-skeleton of $P^3$, and so $V(\alpha)=V(P^3)$. Then the Hamming wight
  $wt(\alpha) = wt(\xi_{F_1}+\cdots + \xi_{F_k}) = wt(\underline{1}) = |V(P^3)| \geq 4$.
   But this
  contradicts our assumption that $wt(\alpha)=2$. So (c) is impossible. 
  \end{itemize}
  
  Therefore, the Hamming weight of any element of $\mathfrak{B}_1(P^3)$ 
  can not be $2$. So we finish the proof of the theorem.
  \end{proof}
   
\begin{figure}[h]
\includegraphics[width=0.62\textwidth]{Graph-Polytope.eps}
\caption{The graph of a simple $3$-polytope}
\label{p:Graph}
\end{figure}
 

  
\section{Properties of $n$-dimensional $n$-colorable simple polytopes}
 
 For brevity, we use the words ``\emph{even polytope}'' to refer to an 
  $n$-dimensional $n$-colorable simple polytope in the rest of the paper.
  Indeed, this term has already been used by Joswig~\cite{jos}. 
  
  
\begin{defn}   (\cite{perles} and \cite{kalai} Remark 2)
Let $F$ be a facet of a simple polytope $P$ and $V(F)$ be the set of vertices of $F$.
 Define a map
 $$\Xi_F: V(F)\rightarrow V(P)\setminus V(F)$$ as follows.
 For each $v\in V(F)$, there is exactly one edge $e$ of $P$, such that $e\nsubseteq F$, $v\in e$
 (since $P$ is simple and $F$ is codimension one). Then $v$ is 
  mapped to the other endpoint $\Xi_F(v)$ of $e$.
\end{defn}
 \vskip .1cm
 
\begin{exam}
 Let $P$ be the $6$-prism in Figure~\ref{6prism} and $F$ be the facet with vertex set
  $\{3,4,9,10\}$. Then by definition, $\Xi_F: \{3,4,9,10 \}\rightarrow \{1,2,5,6,7,8,11,12\}$ where
$$ \Xi(3)= 2,\ \Xi(4)= 5,\ \Xi(9)= 8,\ \Xi(10)= 11. $$
\end{exam}

 \vskip .1cm

\begin{prop} \label{Prop:inject}
 For an even polytope $P$, the map
 $\Xi_F$ is injective for any facet $F$ of $P$.
\end{prop}
\begin{proof}
Assume $\Xi_F$ is not injective.
There must exist two vertices $p_1,p_2\in F$ and a vertex 
$v\notin F$ such that $v$ is connected to both
$p_1$ and $p_2$ by edges in $P$ (see Figure~\ref{p:proof-2}).
 Let $f_i$ be the edge
 with end points $p_i$ and $v$, $i=1,2$. 
  Suppose the dimension of $P$ is $n$. Then there exist
  $n$ facets, $F_1,F_2,\cdots,F_n$, distinct to $F$, such that
$$v=\bigcap_{i=1}^nF_i,\ \  f_1=\bigcap_{i=1}^{n-1}F_i,\ \ f_2=\bigcap_{i=2}^nF_i.$$

Then we have 
 $$p_1=F\bigcap \big(\bigcap_{i=1}^{n-1}F_i \big), \ \
    p_2=F\bigcap \big(\bigcap_{i=2}^{n}F_i\big). $$

\begin{figure}[h]
\includegraphics[width=0.38 \textwidth]{Proof-2.eps}
\caption{A facet $F$ with $\Xi_F$ non-injective}
\label{p:proof-2}
\end{figure}

Since $P$ is $n$-colorable, we can color all the facets of $P$ by 
 $n$-colors $e_1,\cdots, e_n$ such that no adjacent facets are assigned
  the same color. Suppose $F_i$ is colored by $e_i$, $i=1,\cdots, n$. Then at $p_1$,
  $F$ has to be colored by $e_n$ while at $p_2$, $F$ has to be colored by $e_1$, contradiction.  
\end{proof}
\vskip .1cm

\begin{prop}\label{lowerbound}  Let $P$ be an even polytope.
 For any facet $F$ of $P$, we have
 $$ |V(P)|\geq 2 |V(F)|.$$ 
 Moreover, $|V(P)|= 2|V(F)|$ if and only if
$P=F\times [0,1]$ where $[0,1]$ denotes a $1$-simplex.
\end{prop}

\begin{proof}
 By Proposition~\ref{Prop:inject}, the map
    $\Xi_F: V(F)\rightarrow V(P)\setminus V(F)$ is injective. So we have
$$|V(F)|\leq |V(P)\setminus V(F)|=|V(P)|-|V(F)|.$$
So $|V(P)|\geq 2|V(F)|$. If $|V(P)|= 2|V(F)|$, the injectivity of $\Xi_F$ implies
 $P=F\times [0,1]$.
\end{proof}

\vskip .1cm

\begin{cor} \label{Cor:bound}
Let $f$ be a codimension-$k$ face of an even polytope $P$. Then $|V(P)| \geq 2^k|V(f)|$.
Moreover,
$|V(P)| = 2^k|V(f)|$ if and only if $P=f\times [0,1]^k$.
\end{cor}

\vskip .1cm

\begin{cor}\label{min} 
 For any $n$-dimensional even polytope $P$, we must have
$|V(P)|\geq 2^n$.
 In particular, $|V(P)|=2^n$ if and only if $P=[0,1]^n$ (the $n$-dimensional cube).
\end{cor}

\vskip .1cm

\begin{cor}\label{Cor:3-face}
 Suppose $P$ is an $n$-dimensional even polytope, $n\geq 4$. 
 If there exists a facet
 $F$ of $P$ with $|V(P)| = 2|V(F)|$, then there there exists a $3$-face of $P$ 
 which is isomorphic to a $3$-dimensional cube.
\end{cor}
\begin{proof}
  It is well known that any $3$-dimensional simple polytope must have a $2$-face $f$ with
  less than $6$ vertices. Now since $P$ is even, any $2$-face of $P$ must have an even number of 
  vertices. So there exists a $4$-gon face $f$ in $F$. Then since $|V(P)| = 2|V(F)|$, we have
  $P=F\times [0,1]$ by Corollary~\ref{Cor:bound}. 
  So $P$ has a $3$-face $f\times [0,1]$ which is a $3$-cube.
\end{proof}
 \vskip .8cm

 \section{Doubly-even binary codes}
 
A binary code $C$ is called \emph{doubly-even}
  if the Hamming weight of any codeword
 in $C$ is divisible by $4$. Doubly-even self-dual codes are of special importance among binary
 codes and have been extensively studied. According to Gleason~\cite{gl}, the length of any doubly-even self-dual code is divisible by $8$. In addition,
 Mallows-Sloane~\cite{ms} showed that if $C$ is a double-even self-dual code
 of length $l$, it is necessary that the minimum distance $d$ of $C$ satisfies
   $ d \leq
      4 \big[ \frac{l}{24}\big] +4
      $.
And $C$ is called \emph{extremal} if the equality holds. 
 \vskip .1cm
 
 A result of Zhang~\cite{zhang} tells us that 
 an extremal doubly-even self-dual binary code must have length less or equal to $3928$. 
However, the existence of extremal doubly-even self-dual binary codes is only known for the following lengths (see~\cite{h} and~\cite[p.273]{rs})
 $$l = 8, 16, 24, 32, 40, 48, 56, 64, 80, 88, 104, 112, 136.$$
 For example, the extended Golay code $\mathcal{G}_{24}$ 
 is the only doubly-even self-dual [24,12,8] code, and
 the extended quadratic residue code $QR_{48}$ is the only doubly-even self-dual
  [48,24,12] code (see~\cite{hltp}). In addition,  
  the existence of an extremal doubly-even self-dual 
code of length $72$ is a long-standing open question 
 (see~\cite{s} and~\cite[Section 12]{rs}). \vskip .1cm
 
The following proposition is an immediate consequence of Corollary~\ref{Cor:Main-2}, which gives
us a way to construct doubly-even self-dual codes from simple polytopes.
 
\begin{prop} \label{Prop:3-face}
     For an $(2k+1)$-dimensional even polytope $P$, the self-dual binary code 
     $\mathfrak{B}_{k}(P)$ is doubly-even if and only if 
     the number of vertices of any $(k+1)$-dimensional face of $P$
     is divisible by $4$.     
\end{prop}
 
 
 
\begin{defn}
 We say that a self-dual binary code $C$ can be \emph{realized by an even polytope} if
 there exists a $(2k+1)$-dimensional 
even polytope $P$ so that $C = \mathfrak{B}_{k}(P)$. 
\end{defn}
\vskip .1cm

\begin{exam}
 An extremal doubly-even self-dual binary code of length $8$ and $16$ can be realized by
 the $3$-cube and the $8$-prism ($8$-gon$\times [0,1]$), respectively.
\end{exam}

\vskip .1cm

\begin{prop}\label{golay}
The $[24, 12,8]$ extended Golay code $\mathcal{G}_{24}$ can not be realized by any even polytope.
\end{prop}
\begin{proof}
Assume $\mathcal{G}_{24}$ can be realized by an $n$-dimensional even polytope 
 $P^n$, where $n$ is odd. Then $P^n$ has 24 vertices.
By Corollary~\ref{min}, we have $24\geq 2^n$ which implies $n=1,3$.
But $n=1$ is clearly impossible. And by Proposition~\ref{prop:3-polytope}, 
$n=3$ is impossible either. \vskip .1cm

Another way to prove this result is that since $P^3$ must have a $4$-gon face, 
the code $\mathfrak{B}_1(P^3)$ must have a codeword with Hamming
weight $4$. But it is known that
any codeword of $\mathcal{G}_{24}$ must have Hamming weight  
$0, 8, 12, 16$ or $24$.
\end{proof}

\vskip .1cm

\begin{prop}\label{quadratic}
The $[48, 24,12]$ extended quadratic residue code
 $QR_{48}$ can not be realized by any even polytope.
\end{prop}
\begin{proof}
Suppose $QR_{48}$ can be realized by an $n$-dimensional even polytope 
 $P^n$. Then by Corollary~\ref{min}, we must have $n=1,3$ or $5$.
 But by Proposition~\ref{prop:3-polytope}, $n$ can not be $1$ or $3$.
 If $n=5$, since $|V(P^5)|=48$, any $3$-face of $P^5$ has to be an even polytope with $12$ 
 vertices by Corollary~\ref{Cor:bound} and the fact that the minimum distance of $QR_{48}$ is $12$. Then $P^5$ is isomorphic to the product of a simple
 $3$-polytope with $[0,1]^2$ by Corollary~\ref{Cor:bound} again. This implies that $P^5$ has a 
  $3$-face isomorphic to a $3$-cube. But this contradicts the fact that any $3$-face of $P^5$ has
  $12$ vertices.
\end{proof}

 \vskip .1cm

\begin{prop}
 An extremal doubly-even self-dual codes of length $72$ (if exists) can not be
 realized by any even polytope.
\end{prop}
 \begin{proof}
  Assume that $C$ is an extremal doubly-even self-dual 
   binary code of length $72$ and $C$ 
  can be realized by an even polytope $P$. Then by the definition of extremity, 
  the minimum distance of $C$ is $16$ and $P$ has $72$ vertices. Moreover, we have
 \begin{itemize}
  \item[(i)] the dimension of $P$ has to be $5$ by Corollary~\ref{min} and 
     Proposition~\ref{prop:3-polytope}; \vskip .1cm
     
  \item[(ii)] any $3$-face of $P$ must be an even polytope with $16$ vertices by
Corollary~\ref{Cor:bound} and Proposition~\ref{Prop:3-face}.
\end{itemize}
 Then any $4$-face of $P$ must have $32$ or $36$ vertices by Corollary~\ref{Cor:bound}.\vskip .1cm
 
$\bullet$ If $P$ has a $4$-face $F$ with $32$ vertices, then $F=f\times [0,1]$ where $f$ is a $3$-face
 with $16$ vertices by (ii) and Corollary~\ref{Cor:bound}. This implies that
 $P$ has a $3$-face isomorphic to a $3$-cube by Corollary~\ref{Cor:3-face}. But this
  contradicts (ii).\vskip .1cm
 
$\bullet$ If $P$ has a $4$-face $F$ with $36$ vertices, then 
$P = F\times [0,1]$ by Corollary~\ref{Cor:bound}. So
 $P$ has a $3$-face isomorphic to
  a $3$-cube by Corollary~\ref{Cor:3-face}. This contradicts (ii) again. \vskip .1cm
 
 So by the above argument, such an even polytope $P$ does not exist. 
\end{proof}
\vskip .7cm

\section*{Acknowledgements}
This paper is inspired by a question on 
self-dual binary codes and small covers, which was proposed to the authors 
by professor Matthias Kreck during his visit to China in September 2012.\\

\begin{thebibliography}{99}
\bibitem{ap}
C.~Allday and V.~Puppe, \emph{Cohomological methods in transformation
groups}, Cambridge Studies in Advanced Mathematics, vol. {\bf 32}.
Cambridge University Press, London (1993).

\bibitem{bal}
 M.~L.~Balinski, \emph{On the graph structure of convex polyhedra in $n$-space}, 
   Pacific J. Math. 11 (1961), 431--434.

\bibitem{b} G.E. Bredon, \emph{Intrduction to compact transformation groups}, In: Pure and Applied Mathematics, vol.
{\bf 46}. Academic Press, New York, London (1972)

\bibitem{br} 
A.~Br{\o}ndsted, \emph{An introduction to convex polytopes}, Graduate Texts in Mathematics, 90,  
 Springer-Verlag, New York-Berlin, 1983. 

\bibitem{bp} V.~Buchstaber and T.~Panov,
  \emph{Torus
actions and their applications in topology and combinatorics},
 University Lecture Series, vol. {\bf 24}, Amer.
 Math. Soc., Providence, RI, 2002.

\bibitem{cl} B.~Chen and Z.~L\"{u},
  \emph{Equivariant cohomology and analytic descriptions of ring isomorphisms},
    Math. Z. {\bf 261} (2009), No. 4, 891--908.

\bibitem{cf} P.E. Conner, \emph{Differentiable periodic maps}, Second edition,
 Lecture Notes in Mathematics 738, Springer, Berlin (1979).
 
\bibitem{dj} M.~Davis and T.~Januszkiewicz,
 \emph{Convex polytopes, Coxeter orbifolds and torus Actions},
 Duke Math. J.  {\bf 62} (1991), 417--451.

\bibitem{gl}
 A.~M.~Gleason, \emph{Weight polynomials of self-dual codes and the MacWillams
identities}, Actes du Congr\`es International des Math\'ematiciens, volume 3,
 (Nice, 1970), Gauthier-Villars, Paris, 1971, 211--215,

\bibitem{h} 
M.~Harada, \emph{An extremal doubly even self-dual code of length $112$}, 
Electron. J. Combin. 15 (2008), no.~\text{1}, Note 33.

\bibitem{hltp}
S.~K.~Houghten, C.~W.~H.~Lam, L.~H.~Thiel and J.~A.~Parker,
\emph{The extended quadratic Residue code is the only
$(48; 24; 12)$ self-dual doubly-even code},  IEEE Trans. Inform. Theory, 49, no.~1 (2003),
53--59.

\bibitem{jos}
  M.~Joswig, \emph{Projectivities in simplicial complexes and colorings of simple polytopes},
  Math. Z. {\bf 240} (2002), no. 2, 243--259.

\bibitem{kalai} G. Kalai, \emph{A Simple Way to Tell a Simple Polytope from Its Graph}, J.
of Combinatorial Theory, Series A {\bf 49} (1988), 381--383.

\bibitem{kp} M.~Kreck and V.~Puppe,
 \emph{Involutions on 3-manifolds and self-dual, binary codes},
 Homology, Homotopy Appl.~10 (2008), no.~\textbf{2}, 139--148.

\bibitem{lint}
  J.H. van Lint,
   \emph{Introduction to coding theory}, Second edition,
    Graduate Texts in Mathematics \textbf{86}, Springer-Verlag, Berlin (1992).
  
\bibitem{ms}
  C.~L.~Mallows and N.~J.~A.~Sloane, \emph{An upper bound for self-dual
codes}, Inf. Contr., vol. \textbf{22} (1973), 188--200.
    
\bibitem{m} J. R. Munkres, {\em Elements of Algebraic Topology}, Addison-Wesley Publishing Company, Menlo Park, CA, 1984.    
    

\bibitem{perles}M.A. Perles, \emph{Results and problems on reconstruction of polytopes}, Jerusalem
1970, unpublished.

\bibitem{p} V.~Puppe,  \emph{Group actions and codes}. Can. J. Math. l53, 212--224
(2001).

\bibitem{r}

E.~M.~Rains, \emph{Shadow bounds for self-dual-codes}, 
 IEEE Trans. Inf. Theory, vol. 44 (1998), 134--139.

\bibitem{rs}
  E.~M.~Rains and N.J.~Sloane,
   \emph{Self-dual codes},
   Handbook of coding theory, Vol. I, II, 177--294, North-Holland, Amsterdam, 1998.
   
\bibitem{s}
N.~J.~A.~Sloane, \emph{Is there a $(72; 36)$ $d = 16$ self-dual code?}
  IEEE Trans. Inform. Theory 19 (1973), 251.   

\bibitem{zhang}
  S.~Zhang, \emph{On the nonexistence of extremal self-dual codes},
   Discrete Appl. Math., vol. 91 (1999), 277--286. 
   
   
\bibitem{ziegler}
 G.~Ziegler, \emph{Lectures on Polytopes}, Graduate Texts in Mathematics \textbf{152},
 Springer-Verlag (1995).

\end{thebibliography}

\end{document}


\documentclass[11pt]{article}


\usepackage[margin=1in]{geometry}

\usepackage{graphicx}

\usepackage{rotating}

\usepackage{color}

\usepackage{bm}

\usepackage{amssymb,amsmath,amsfonts}

\usepackage{amscd}

\usepackage{verbatim}

\usepackage{subfigure}

\begin{document}


\[
\|{{\mathbf v}}\|_{0,\omega} := 
\|\omega^{1/2}{{\mathbf v}}\|, \quad
|\phi|_{1,\omega}:= \|\nabla \phi\|_{0,\omega} =\|\omega^{1/2}\nabla \phi\|.
\]

\begin{equation}\label{contrast}
\kappa = \max_{x \in \Omega} (\|K(x)\|_{\ell^2}\|K^{-1}(x)\|_{\ell^2}). 
\end{equation}

\begin{equation}\label{H-div-prod}
\Lambda({{\mathbf u}}, {{\mathbf v}})= ({{\mathbf u}},{{\mathbf v}}) + ({\operatorname{div}} {{\mathbf u}}, {\operatorname{div}} {{\mathbf v}}).
\end{equation}

\begin{equation}\label{H-D}
H^1_D({{\Omega}}) = \{ \phi \in H^1({{\Omega}}): ~~\phi(x)=0 \quad \text{on} \quad \Gamma_D\}.
\end{equation}

\begin{equation}\label{H-D-2}
H^1_D({{\Omega}}) = \{ \phi \in H^1({{\Omega}}): ~~\int_{\Omega}\phi=0\}.
\end{equation}

\begin{equation}\label{H-N}
  {{\boldsymbol H}_{\hspace{-0.2mm}N}}({\operatorname{div}}) 
:= {{\boldsymbol H}_{\hspace{-0.2mm}N}}({\operatorname{div}}; {{\Omega}})= \{ {{\mathbf v}} \in {{\boldsymbol H}}({\operatorname{div}}; {{\Omega}}): ~~{{\mathbf v}}(x) \cdot {{\mathbf n}}=0 \quad \text{on} \quad \Gamma_N\}.
\end{equation}

\begin{equation}\label{WH-div-prod}
\Lambda_{{\alpha}}({{\mathbf u}}, {{\mathbf v}})= ({{\alpha}}~{{\mathbf u}},{{\mathbf v}}) + ({\operatorname{div}} {{\mathbf u}}, {\operatorname{div}} {{\mathbf v}}), 
\quad \alpha(x) = K^{-1}(x).
\end{equation}

\[
\|{{\mathbf v}}\|^2_{\Lambda_\alpha}  = 
\Lambda_{{\alpha}}({{\mathbf v}}, {{\mathbf v}})= 
\|{{\mathbf v}}\|^2_{0,\alpha} + \|{\operatorname{div}} {{\mathbf v}}\|^2.
\]

\begin{eqnarray*}
\|K^{-1}\|_{\ell^2}(K\xi\cdot\xi)& \ge &
(\xi\cdot\xi)\|K^{-1}\|_{\ell^2}
\inf_{\theta\in \mathbb{R}^d}\frac{(K\theta\cdot \theta)}{(\theta\cdot\theta)}\\
&= &
(\xi\cdot\xi) \, \|K^{-1}\|_{\ell^2}
\inf_{\theta\in
  \mathbb{R}^d}\frac{(\theta\cdot\theta)}{(K^{-1}\theta\cdot\theta)}
=
(\xi\cdot\xi)\|K^{-1}\|_{\ell^2}\frac{1}{\|K^{-1}\|_{\ell^2}}=(\xi\cdot\xi).
\end{eqnarray*}

\begin{equation}\label{bound_below_K}
(\xi\cdot \xi) \le  (K(x)\xi\cdot\xi), \quad \xi\in \mathbb{R}^d.   
\end{equation}

\[
K(x)\leftarrow 
K(x)\left\|\|K^{-1}(x)\|_{\ell^2}\right\|_{L^{\infty}(\Omega)}. 
\]

\begin{equation}\label{first}
(K^{-1}(x) {{\mathbf u}}, {{\mathbf v}}) - (p, {\operatorname{div}} {{\mathbf v}}) = 0
\end{equation}

\begin{equation}\label{div-eqn}
({\operatorname{div}} {{\mathbf u}}, q)=(f,q).
\end{equation}

\begin{equation}\label{eq:dual_mixed}
{{\mathcal A}}^{DM}({{\mathbf u}},p; {{\mathbf v}},q)= - (f, q),
\quad \mbox{for all}\quad ({{\mathbf v}}, q) \in {{\boldsymbol V}} \times W,
\end{equation}

\begin{equation}\label{A-form-mixed}
{{\mathcal A}}^{DM}({{\mathbf u}},p; {{\mathbf v}},q):=({{\alpha}} {{\mathbf u}}, {{\mathbf v}})
-(p, {\operatorname{div}} {{\mathbf v}}) - ({\operatorname{div}} {{\mathbf u}}, q). 
\end{equation}

$$
{{\mathcal A}}^{DM}_\beta ({{\mathbf u}},p; {{\mathbf v}},q)
:=({{\alpha}} {{\mathbf u}}, {{\mathbf v}}) + \beta ({\operatorname{div}} {{\mathbf u}}, {\operatorname{div}} {{\mathbf v}})-(p, {\operatorname{div}} {{\mathbf v}}) - ({\operatorname{div}} {{\mathbf u}}, q). 
$$

\begin{equation}\label{eq:LS}
{{\mathcal A}}^{LS}({{\mathbf u}},p; {{\mathbf v}},q)=(f, {\operatorname{div}} {{\mathbf v}}), 
\quad \mbox{for all}\quad ({{\mathbf v}}, q) \in {{\boldsymbol H}_{\hspace{-0.2mm}N}}({\operatorname{div}}) \times H^1_D({{\Omega}}),
\end{equation}

\begin{equation}\label{A-form-LS}
{{\mathcal A}}^{LS}({{\mathbf u}},p; {{\mathbf v}},q):=({{\alpha}} {{\mathbf u}}, {{\mathbf v}}) +({\operatorname{div}} {{\mathbf u}}, {\operatorname{div}} {{\mathbf v}}) + (p, {\operatorname{div}} {{\mathbf v}}) + ({\operatorname{div}} {{\mathbf u}}, q) + (K \nabla p, \nabla q).
\end{equation}

\begin{equation}\label{Poincare}
\text{there is  such that} ~~\mbox{for all}\quad 
q \in H^1_D(\Omega) \quad
\|q\|^2 \le C_P \| \nabla q\|^2. 
\end{equation}

$$
\|q\|^2 \le C_P \| \nabla q\|^2 \le 
C_P \|\nabla q\|^2_{0,K}.
$$

\begin{equation}\label{continuity}
{{\mathcal A}}^{DM}({{\mathbf u}},p; {{\mathbf v}},q) 
\le ( \|{{\mathbf u}}\|^2_{\Lambda_{{\alpha}}} +\|p\|^2)^\frac12( \|{{\mathbf v}}\|^2_{\Lambda_{{\alpha}}} +\|q\|^2)^\frac12;
\end{equation}

\begin{equation}\label{A-infsup}
\sup_{{{\mathbf v}} \in {{\boldsymbol V}}, \, q \in L^2} 
\frac{{{\mathcal A}}^{DM}({{\mathbf u}},p; {{\mathbf v}},q) }{(\|{{\mathbf v}}\|^2_{\Lambda_{{\alpha}}} +\|q\|^2)^\frac12} 
\ge \alpha_0 ( \|{{\mathbf u}}\|^2_{\Lambda_{{\alpha}}} +\|p\|^2 )^\frac12
\end{equation}

\begin{equation}\label{inf_sup_stab_var}
  \inf_{q\in W} \sup_{{{\mathbf v}}\in{{\boldsymbol V}}}\frac{(\nabla\cdot {{\mathbf v}},q)}{\Vert {{\mathbf v}} \Vert_{\Lambda_{{\alpha}}}
\Vert q \Vert}\ge 
\gamma >0, \qquad \mbox{for all}\quad {{\mathbf v}}\in{{\boldsymbol V}},\quad \mbox{for all}\quad q\in W. 
 \end{equation}

\begin{equation}\label{lem:stab_var_1}
\|{{\mathbf w}}\|_{0,\alpha}^2\le C_P \, \Vert q \Vert^2.
\end{equation}

\begin{equation}\label{eq:standard-variational}
( K(x) \nabla \varphi,\nabla\chi ) = (q, \chi), 
\quad \mbox{for all}\quad \chi\in H_D^1(\Omega).
\end{equation}

\begin{eqnarray*}
\|{{\mathbf w}}\|_{0,\alpha}^2& = & 
(\alpha\;K\nabla\varphi,{{\mathbf w}})= (\nabla\varphi,{{\mathbf w}})
= -(\varphi,{\operatorname{div}} {{\mathbf w}})=(\varphi, q)\\ &\le& 
\|\varphi\|\|q\|
\le \sqrt{C_P}(\nabla\varphi,\nabla\varphi)^\frac12 \|q\|\\
&\le& \sqrt{C_P} \,|\nabla\varphi|_{0,K}\|q\|
= \sqrt{C_P} \|{{\mathbf w}}\|_{0,\alpha}\|q\|.
\end{eqnarray*}

\begin{eqnarray*}
\sup_{{{\mathbf v}}\in {{\boldsymbol V}}} \frac{(q,{\operatorname{div}} {{\mathbf v}})}{\|{{\mathbf v}}\|_{\Lambda_\alpha}} &\ge &
\frac{(q,{\operatorname{div}} {{\mathbf w}})}{\|{{\mathbf w}}\|_{\Lambda_\alpha}}=  
\frac{\|q\|^2}{ (\|{{\mathbf w}}\|_{0,\alpha}^2+\|{\operatorname{div}} {{\mathbf w}}\|^2)^\frac12}\\
& = & 
\frac{\|q\|^2}{( \|{{\mathbf w}}\|_{0,\alpha}^2+\|q\|^2)^\frac12}
\ge \frac{\|q\|}{\sqrt{C_P+1}}.
\end{eqnarray*}

\begin{equation}\label{continuity_LS}
{{\mathcal A}}^{LS}({{\mathbf u}},p; {{\mathbf v}},q) 
\le 2( \|{{\mathbf u}}\|^2_{\Lambda_{{\alpha}}} +\|p\|^2_{1,K})^\frac12( \|{{\mathbf v}}\|^2_{\Lambda_{{\alpha}}} +\|q\|^2_{1,K})^\frac12;
\end{equation}

\begin{equation}\label{A-coercivity}
{{\mathcal A}}^{LS}({{\mathbf v}},q; {{\mathbf v}},q) 
\ge \frac13 ( \|{{\mathbf v}}\|^2_{\Lambda_{{\alpha}}} +\|q\|^2_{1,K})
\end{equation}

\begin{equation}\label{div_bound}
{{\mathcal A}}^{LS}({{\mathbf v}},q; {{\mathbf v}},q)  = \|{\operatorname{div}} {{\mathbf v}} \|^2 + (\alpha {{\mathbf v}} - \nabla q, {{\mathbf v}} - K \nabla q) 
\ge \|{\operatorname{div}} {{\mathbf v}} \|^2.
\end{equation}

\begin{eqnarray*}
{{\mathcal A}}^{LS}({{\mathbf v}},q; {{\mathbf v}},q)   &&  \ge  (1-\frac{1}{\epsilon})\|{\operatorname{div}} {{\mathbf v}}\|^2  +  
\|{{\mathbf v}}\|_{0,\alpha}^2 + \|q\|^2_{1,K} - \epsilon \|q \|^2 \\
 &&   \ge (1-\frac{1}{\epsilon})\|{\operatorname{div}} {{\mathbf v}} \|^2 
+ \|{{\mathbf v}}\|^2_{0,\alpha} + (1 - \epsilon C_P)\|\nabla q\|_{0,K}, \quad \epsilon >0.
\end{eqnarray*}

$$
{{\mathcal A}}^{LS}({{\mathbf v}},q; {{\mathbf v}},q) \ge (1-2 C_P)\|{\operatorname{div}} {{\mathbf v}} \|^2 
+ \|{{\mathbf v}}\|_{0,\alpha}^2 + \frac12 \|\nabla q\|_{0,K}^2.
$$

$$
(1+ \beta) {{\mathcal A}}^{LS}({{\mathbf v}},q; {{\mathbf v}},q) \ge ( 1-2 C_P + \beta) \|{\operatorname{div}} {{\mathbf v}} \|^2 
+ \|{{\mathbf v}}\|^2_{0,\alpha} + \frac12 \|\nabla q\|_{0,K}^2.
$$

\begin{equation}\label{space Vh}
{{\boldsymbol V}_{\hspace{-0.2mm}h}}=\{ {{\mathbf v}} \in {{\boldsymbol V}}: \, {{\mathbf v}} |_T \in {{\mathcal{RT}}_{{0}}} \,\,\, \mbox{for} \,\, T \in {\mathcal T}_h\} 
\end{equation}

\begin{equation}\label{space Wh}
W_h =\{ q \in L^2(\Omega): \, q|_{T} \in {\mathcal P}_0, 
\, \text{i.e.  is a piece-wise constant function on} \,\,  {\mathcal T}_h\}.
\end{equation}

\begin{equation}\label{eq:dual_mixed_FEM}
{{\mathcal A}}^{DM}({{\mathbf u}}_h,p_h; {{\mathbf v}},q)=(f,q), 
\quad \mbox{for all}\quad ({{\mathbf v}}, q) \in {{\boldsymbol V}_{\hspace{-0.2mm}h}} \times W_h,
\end{equation}

\begin{equation}\label{inf_sup_fin_set}
 \inf_{q_h\in W_h}\sup_{{{\mathbf v}}_h\in{{\boldsymbol V}_{\hspace{-0.2mm}h}}}
\frac{({\operatorname{div}} {{\mathbf v}}_h,q_h)}{\Vert {{\mathbf v}}_h\Vert_{\Lambda_{{\alpha}}} \Vert q_h\Vert}\ge 
\gamma >0 . 
\end{equation}

\begin{equation}\label{DM-Fem_stability}
\alpha_0 ( \|{{\mathbf u}}\|^2_{\Lambda_{{\alpha}}} +\|p\|^2)^\frac12 \le \sup_{{{\mathbf v}} \in{{\boldsymbol V}_{\hspace{-0.2mm}h}}, q\in W_h} 
\frac{{{\mathcal A}}^{DM}({{\mathbf u}},p; {{\mathbf v}},q) }{(\|{{\mathbf v}}\|^2_{\Lambda_{{\alpha}}} +\|q\|^2)^\frac12} \le 
( \|{{\mathbf u}}\|^2_{\Lambda_{{\alpha}}} +\|p\|^2)^\frac12.
\end{equation}

\begin{equation}\label{LS-Fem_coercivity}
\frac13 ( \|{{\mathbf v}}\|^2_{\Lambda_{{\alpha}}} +\|q\|^2_{1,K})  \le {{\mathcal A}}^{LS}({{\mathbf v}},q; {{\mathbf v}},q)   
\end{equation}

\begin{equation}\label{LS-FEM-continuity}
 {{\mathcal A}}^{LS}({{\mathbf u}},p; {{\mathbf v}},q)  \le 2
(\|{{\mathbf u}}\|^2_{\Lambda_{{\alpha}}} +\|p\|^2_{1,K})^\frac12
( \|{{\mathbf v}}\|^2_{\Lambda_{{\alpha}}} +\|q\|^2_{1,K})^\frac12.
\end{equation}

\begin{equation}\label{mixed-operator}
 \mathcal{A}_h {\bm x}_h = {\bm f}_h, \quad \mbox{for} \quad
{\bm f}_h =({\bm 0}, f_h) \in X_h,
\end{equation}

$$
( \mathcal{A}_h {\bm x}_h, {\bm y}_h) = {{\mathcal A}}^{DM}({{\mathbf u}}_h,p_h; {{\mathbf v}}_h,q_h)
\quad
\mbox{or} \quad
 ( \mathcal{A}_h {\bm x}_h, {\bm y}_h) = {{\mathcal A}}^{LS}({{\mathbf u}}_h,p_h; {{\mathbf v}}_h,q_h).
$$

\begin{equation}\label{eq:operator_norms}
 \Vert\mathcal{A}_h\Vert_{\mathcal{L}(X_h,X_h^{\star})}\;\; \mbox{and}\;\; 
\Vert\mathcal{A}_h^{-1}\Vert_{\mathcal{L}(X_h^{\star},X_h)} \;\; \mbox{are uniformly bounded}. 
\end{equation}

\begin{equation}\label{eq:precond_norms}
 \Vert\mathcal{B}_h\Vert_{\mathcal{L}(X_h,X_h^{\star})}\;\; \mbox{and}\;\; 
\Vert\mathcal{B}_h^{-1}\Vert_{\mathcal{L}(X_h^{\star},X_h)} \;\; \mbox{being uniformly bounded in} \;\; h \;\; \mbox{and} 
\;\; \kappa
\end{equation}

\begin{equation}\label{eq:AFW_preconditioner}
 \mathcal{B}_h:=\left[
\begin{array}{cc}
 A_h & 0\\[2ex]
 0 & D_h
\end{array}
\right]
\end{equation}

$$
(A_h{{\mathbf u}}_h,{{\mathbf v}}_h):=\Lambda_{{\alpha}}({{\mathbf u}}_h,{{\mathbf v}}_h)=
({{\alpha}} \, {{\mathbf u}}_h,{{\mathbf v}}_h)+(\nabla\cdot {{\mathbf u}}_h,\nabla\cdot {{\mathbf v}}_h)
$$

$$
{\bm x}=\left[
\begin{array}{c}
 {\bf u} \\
 {\bf p}
\end{array} \right ],
\quad \mbox{where} \quad
\quad {\bf u} \in \mathbb{R}^{|\mathcal{E}_h|},
\quad {\bf p} \in \mathbb{R}^{|\mathcal{T}_h|},
\quad \mbox{are vector columns} 
$$

\begin{equation}\label{eq:saddle_point_sys}
 \left[
\begin{array}{cc}
{{M}_{{\alpha}}} & -{B_{\operatorname{div}}}^T \\
-{B_{\operatorname{div}}} & 0
\end{array}
\right]
\left[
\begin{array}{c}
 {\bf u} \\
 {\bf p}
\end{array}
\right]=
\left[
\begin{array}{c}
 \bf{0} \\
 \bf{f}
\end{array}
\right], 
\quad
\left[
\begin{array}{cc}
 A & {B_{\operatorname{div}}}^T \\
{B_{\operatorname{div}}} & D
\end{array}
\right]
\left[
\begin{array}{c}
 {\bf u} \\
 {\bf p}
\end{array}
\right]=
\left[
\begin{array}{c}
 \bf{0} \\
 \bf{f}
\end{array}
\right], 
\end{equation}

\begin{equation}\label{algebraic-hdiv}
A {{\mathbf u}} = {{\mathbf b}}, \quad {{\mathbf u}}, {{\mathbf b}} \in  \mathbb{R}^N, \quad N:={|\mathcal{E}_h|}.
\end{equation}

\begin{equation*}
 \overline{\Omega}=\bigcup_{i=1}^{n} \overline{\Omega}_{i}.
\end{equation*}

$$
A=\sum_{i=1}^{n} R_{{i}}^T A_{{i}} R_{{i}},
$$

\begin{equation}\label{eq:partitioning_DOF}
\mathcal{D} = \mathcal{D}_{\rm f} \oplus \mathcal{D}_{\rm c},
\end{equation}

\begin{equation}\label{eq:A_two_by_two-x}
A=\left[
        \begin{array}{cc}
        A_{11} & A_{12} \\
        A_{21} & A_{22}
        \end{array}
\right],\qquad
 A_{i}=\left[
        \begin{array}{cc}
        A_{i:11} & A_{i:12} \\
        A_{i:21} & A_{i:22}
        \end{array}
\right], \quad i=1,\dots,n.
\end{equation}

\begin{equation}\label{eq:auxA}
\widetilde{A}=
\left[ \begin{array}{ccccc}
A_{1:11} &&&& A_{1:12} R_{1:2} \\[0.5ex]
& A_{2:11} &&& A_{2:12} R_{2:2} \\[0.5ex]
&& \ddots && \vdots \\[0.5ex]
&&& A_{{n}:11} & A_{{n}:12} R_{n:2} \\[0.5ex]
R_{1:2}^T A_{1:21} & R_{2:2}^T A_{2:21} & \hdots & R^T_{n:2} A_{{n}:21} 
& \displaystyle \sum_{i=1}^{n} R^T_{i:2} A_{i:22} R_{i:2}
                \end{array}
\right] ,
\end{equation}

\begin{equation}\label{eq:A_two_by_two}
\widetilde{A}=\left[
        \begin{array}{cc}
        \widetilde{A}_{11} & \widetilde{A}_{12} \\
        \widetilde{A}_{21} & \widetilde{A}_{22}
        \end{array}
\right].
\end{equation}

\begin{equation}
A=R \widetilde{A} R^T 
\end{equation}

\begin{equation}\label{eq:R}
R=\left[ \begin{array}{cc}
           R_1 & 0 \\ 0 & I_2
           \end{array}
\right] , \qquad
R_1^T=\left[ \begin{array}{c}
           R_{1:1} \\ R_{2:1} \\ \vdots \\ R_{n:1}
           \end{array}
\right] .
\end{equation}

$$Q:= \widetilde{A}_{22} - 
\widetilde{A}_{21} \widetilde{A}_{11}^{-1} \widetilde{A}_{12}
=\sum_{i=1}^{n} R_{{i}:2}^T (A_{{i}:22} - A_{{i}:21} 
A_{{i}:11}^{-1} A_{{i}:12}) R_{{i}:2}.
$$

\begin{equation}\label{eq:Pi}
\Pi_{\widetilde{D}}=(R \widetilde{D} R^T)^{-1} R \widetilde{D}, 
\end{equation}

\begin{equation}\label{eq:tilde_D}
\widetilde{D}=\left[
             \begin{array}{cc}
              \widetilde{D}_{11} & 0 \\
              0 & I
             \end{array}
           \right]
\end{equation}

\begin{equation}\label{eq:two-grid_1}
C^{-1}=\Pi_{\widetilde{D}} \widetilde{A}^{-1} \Pi^T_{\widetilde{D}}. 
\end{equation}

\begin{equation}\label{eq:two-grid_3}
B^{-1} := \overline{M}^{-1} 
+ (I - M^{-T} A) C^{-1} (I - A M^{-1})  
\end{equation}

\begin{equation}\label{eq:eb_1}
\underbar{}\langle{{\bf{{v}}}},{{\bf{{v}}}}\rangle \le \rho_A \langle \overline{M}^{-1}{{\bf{{v}}}},{{\bf{{v}}}}\rangle
\le \bar{c}\langle{{\bf{{v}}}},{{\bf{{v}}}}\rangle, 
\end{equation}

\begin{equation}\label{eq:eb_2}
 \Vert M^{-T} A {{\bf{{v}}}}\Vert^2\le \frac{\eta}{\rho_A}\Vert {{\bf{{v}}}} \Vert_A^2, \quad 
\end{equation}

\begin{equation}\label{eq:two-grid_4}
\Pi:=(I-M^{-T} A)\Pi_{\widetilde{D}}=
(I-M^{-T} A)(R\widetilde{D}R^T)^{-1}R\widetilde{D}
\end{equation}

\begin{equation}\label{eq:eb_3}
\Vert \Pi \tilde{{{\bf{{v}}}}}\Vert_A^2\le c_{\Pi} 
\Vert \tilde{{{\bf{{v}}}}}\Vert_{\widetilde{A}}^2, \quad 
\mbox{for all}\quad \tilde{{{\bf{{v}}}}}\in \widetilde{V}={{\mathbb{R}}^{{\widetilde{N}}}}. 
\end{equation}

\begin{equation} \label{eq:eb_7}
  \frac{\underbar{c}}{\underbar{c}+\eta} \le    \lambda_{\min}(B^{-1}A)  
\le \lambda(B^{-1}A) \le     \lambda_{\max}(B^{-1}A)\le\bar{c}+c_{\Pi},
\end{equation}

\begin{equation}\label{eq:pi}
\kappa(B^{-1}A)\le c_{\Pi}=c=\Vert\pi_{\widetilde{D}}
\Vert^2_{\widetilde{A}},
\quad\mbox{where}\quad
\pi_{\widetilde{D}} := R^T \Pi_{\widetilde{D}} .
\end{equation}

\begin{equation}\label{factorizationK}
({\widetilde{A}}^{(k)})^{-1} = 
(\widetilde{L}^{(k)})^T \widetilde{D}^{(k)} \widetilde{L}^{(k)} ,
\end{equation}

\begin{equation}\label{factorizationKl}
\widetilde{L}^{(k)} =
\left [
\begin{array}{cc}
I & \\
-\widetilde{A}^{(k)}_{21} (\widetilde{A}^{(k)}_{11})^{-1} & I
\end{array}
\right ] , \quad
\widetilde{D}^{(k)} =
\left [
\begin{array}{cc}
(\widetilde{A}^{(k)}_{11})^{-1} & \\
& {Q^{(k)}}^{-1}
\end{array}
\right ] 
\end{equation}

\begin{equation}\label{factorizationK2}
 A^{(k+1)}:=Q^{(k)}.
\end{equation}

\begin{equation}\label{multigrid_preconditioner}
{B^{(k)}}^{-1} := 
{\overline{M}^{(k)}}^{-1} + (I - {M^{(k)}}^{-T} A^{(k)})
\Pi^{(k)} 
 (\widetilde{L}^{(k)})^T  \overline{D}^{(k)}
\widetilde{L}^{(k)} {\Pi^{(k)}}^T (I - A^{(k)} {M^{(k)}}^{-1}),
 \end{equation}

\begin{equation}
\overline{D}^{(k)} :=
\left [
\begin{array}{cc}
(\widetilde{A}^{(k)}_{11})^{-1} & \\
& B_{\nu}^{(k+1)}
\end{array}
\right ]
\end{equation}

\begin{equation}
B_{\nu}^{(\ell)} := {A^{(\ell)}}^{-1}.
\end{equation}

$$
\begin{array}{ccc}
 B_{\nu}^{(k+1)} & := & (I-p^{(k)}({B^{(k+1)}}^{-1}A^{(k+1)})){A^{(k+1)}}^{-1} \\
 & =: & q^{(k)}({B^{(k+1)}}^{-1}A^{(k+1)})){B^{(k+1)}}^{-1}
\end{array}
$$

$$
p^{(k)}(0)=1,\qquad q^{(k)}(t):=\frac{1-p^{(k)}(t)}{t}\approx \frac{1}{t}.
$$

$$B_{\nu}^{(k+1)}=B_{\nu}^{(k+1)}[\cdot]$$

$${B^{(k)}}^{-1}={B^{(k)}}^{-1}[\cdot], \quad \mbox{for all } k<\ell.$$

\begin{equation}\label{eq:A_hat}
 \widehat{A}=J^T A J, \quad  
\widehat{A}, J,  A  \in {\mathbb{R}}^{ |{\mathcal E}_h| \times  |{\mathcal E}_h| },
\end{equation}

\begin{equation}\label{eq:transformation_J}
 J=P J_{\pm}, \quad P, J_{\pm} \in {\mathbb{R}}^{ |{\mathcal E}_h| \times  |{\mathcal E}_h| }.
\end{equation}

$$
J_{\pm}=\left[\begin{array}{cc}
    I& \\
    & J_{22}
    \end{array}\right],
\quad \mbox{where} \quad I \in {\mathbb{R}}^{(4 |{\mathcal{T}_H}|) \times (4 |{\mathcal{T}_H}|)},
$$

$$
J_{22}=\frac12 \left[\begin{array}{cccccccc}
    1 & -1 & & & & & & \\
    &  &  1 & -1 & & & & \\
    & & & &  \ddots & \ddots & & \\
    & & & & & & -1 & 1 \\
    1 & 1 & & & & & & \\
    &  &  1 & 1 & & & & \\
    & & & &  \ddots & \ddots & & \\
    & & & & & & 1 & 1 
    \end{array}\right],
\quad \mbox{where} \quad J_{22} \in {\mathbb{R}}^{(2|{\mathcal{E}_H}|) \times (2 |{\mathcal{E}_H}|)}.
$$

\begin{equation}\label{eq:A_i_hat}
\widehat{A}_i=J^T_i A_i J_i,
\end{equation}

\begin{equation}\label{eq:compatibility1}
 \widehat{A}=J^T A J = \sum_{i=1}^{n_{\mathcal G}} \widehat{R}_i^T \widehat{A}_i \widehat{R}_i
\end{equation}

\begin{equation}\label{eq:compatibility2}
 R_i J=J_i \widehat{R}_i.
\end{equation}

$$\widehat{A}^{(k)}={J^{(k)}}^T A^{(k)} J^{(k)}, \quad \mbox{for all } k<\ell$$

\[-1ex]
\hrule \vspace{1ex}
\begin{tabular}{lll}
& Pre-smoothing: &  \\
& Auxiliary space correction: &  \\
& Post-smoothing: & 
\end{tabular}
\\[1ex]
\hrule
\end{algorithm}

The second algorithm applies the nonlinear ASMG method directly in the original basis of
standard Raviart-Thomas basis functions and reads as follows. 
\begin{algorithm}{Nonlinear ASMG method--Variant~II: Action of~\eqref{multigrid_preconditioner}
on a vector }\label{algorithm2} \\[-1ex]
\hrule \vspace{1ex}
\begin{tabular}{lll}
& Pre-smoothing: &  \\
& Auxiliary space correction: &  \\
& Post-smoothing: & 
\end{tabular}
\\[1ex]
\hrule
\end{algorithm}
\begin{remark}
Note that the matrices , ,
 are identical in both algorithms. 
\end{remark}
\begin{remark}
In general the matrices  have slightly
more nonzero entries as compared to  and thus Algorithm~\ref{algorithm2}
increases computational memory requirements.
\end{remark}
\begin{remark}
Considering the two-level preconditioners  and  defined according
to Algorithm~\ref{algorithm1} and Algorithm~\ref{algorithm2}, and assuming that no
smoothing is performed, i.e., , the corresponding condition number
bounds for the preconditioned operators in two-level basis and standard basis read

and
.

\end{remark}

\section{Numerical Experiments}\label{sec:numerics}

\subsection{Description of the parameters and the numerical test examples}

Subject to numerical testing are three representative cases of problems characterized by a highly 
varying coefficient , namely:
\begin{enumerate}
\item[[a\hspace{-1ex}]] A binary distribution of the coefficient described by islands on which
  against a background where , see~Figure~\ref{fig:islands_binary}; 
 \item[[b\hspace{-1ex}]] Inclusions with  and a background with a coefficient 
  that is constant on each element ,
 where the random integer exponent  is uniformly distributed,
see Figure~\ref{fig:islands_random};
\item[[c\hspace{-1ex}]] Three two-dimensional slices of the SPE10 
(Society of Petroleum Engineers) 
benchmark problem, see~\cite{SPE10_project},
where the contrast  is  for slices 44 and 74 and  for slice~54,
see~Figure~\ref{fig:spe_10}.
\end{enumerate}

\begin{figure}[hb]
\begin{center}
\subfigure[ mesh]{
\includegraphics[width=0.21\textwidth]{islands_32}
\label{fig:islands_32}}
\hspace{10mm}
\subfigure[ mesh]{
\includegraphics[width=0.21\textwidth]{islands_128}
\label{fig:islands_128}}
\hspace{10mm}
\subfigure[ mesh]{
\includegraphics[width=0.21\textwidth]{islands_512}
\label{fig:islands_512}}
\caption{Binary distribution of the permeability  corresponding to test case [a]}
\label{fig:islands_binary}
 \end{center}
\end{figure}

\begin{figure}[hb]
\begin{center}
\subfigure[ mesh]{
\includegraphics[width=0.21\textwidth]{islands_32r}
\label{fig:islands_32r}}
\hspace{10mm}
\subfigure[ mesh]{
\includegraphics[width=0.21\textwidth]{islands_128r}
\label{fig:islands_128r}}
\hspace{10mm}
\subfigure[ mesh]{
\includegraphics[width=0.21\textwidth]{islands_512r}
\label{fig:islands_512r}}
\caption{Random distribution of   corresponding to test case [b]}
\label{fig:islands_random}
 \end{center}
\end{figure}

\begin{figure}[hb]
\begin{center}
\subfigure[Slice 44]{
\includegraphics[width=0.28\textwidth]{spe10_slice44}
\label{fig:slice44}}
\hspace{3mm}
\subfigure[Slice 54]{
\includegraphics[width=0.28\textwidth]{spe10_slice54}
\label{fig:slice54}}
\hspace{3mm}
\subfigure[Slice 74]{
\includegraphics[width=0.28\textwidth]{spe10_slice74}
\label{fig:slice74}}
\caption{Distributions of the permeability  along planes  form 
the benchmark SPE10 on a  mesh}
\label{fig:spe_10}
 \end{center}
\end{figure}

The numerical experiments are performed over a uniform mesh consisting of  elements (squares) 
where , i.e. up to  velocity DOF and  pressure DOF.
We have used a direct method to solve the problems on the coarsest grid.
The iterative process has been initialized with a random vector.
Its convergence has been tested for linear systems with right-hand side zero.
We have used overlapping coverings of the domain as shown in Figure~\ref{fig:covering}, 
where the subdomains are composed of  elements and overlap with half of their
width/height. 
In the presentation of results we use the following notations:
\begin{itemize}
\item  denotes the number of levels;
\item  is the logarithm of the contrast ;
\item  is the number of auxiliary space multigrid iterations;
\item  is the number of 
point Gauss-Seidel pre- and post-smoothing steps;
\item  is the average convergence factor defined by
\begin{equation}\label{eq:average_factor}
\rho=\Bigg(\frac{\Vert u_{n_{ASMG}} \Vert}{\Vert u_0 \Vert}\Bigg)^{1/n_{ASMG}},
\end{equation}
where 
 is the
first iterate (approximate solution of \eqref{algebraic-hdiv}) for which the residual
has decreased by a factor of at least .
\end{itemize}
The matrix  is as in~\eqref{eq:tilde_D} where 
. This choice of  requires an
additional preconditioner for the iterative solution of linear systems with the matrix
, which is part of the efficient application of the operator
. The systems with  are solved using the preconditioned conjugate
gradient (PCG) method. The stopping criterion for this inner iterative process is a
residual reduction by a factor , the number of PCG iterations to reach it--where
reported--is denoted by . A robust and uniform preconditioner  for 
can be constructed based on incomplete factorization using exact local factorization (ILUE).
The definition of  is as follows:
 $$
B_{ILUE}:=LU,\qquad U:=\sum_{i=1}^n R_i^T U_i R_i,\qquad L:=U^T {\rm diag}(U)^{-1}, 
$$  
where 
 $$
D_i=L_i U_i,\qquad D=\sum_{i=1}^n R_i^T D_i R_i, \qquad {\rm diag}(L_i)=I,
$$ 
for details see~\cite{Kraus2009robust}.
Note that as  are the local contributions to  related to the subdomains
, , they are all non-singular.

The next two sections are devoted to the presentation of numerical results.
The experiments fall into two categories.
The first category, presented in Section~\ref{sec:Hdiv}, serves the evaluation of the
performance of the ASMG method on linear systems arising from discretization of the
weighted  bilinear form~\eqref{WH-div-prod}. All three test cases, [a], [b],
and [c], are considered and both variants of the ASMG method, Algorithm~\ref{algorithm1}
and~\ref{algorithm2}, are compared, testing V- and W-cycle with and without smoothing.

The second category of experiments, discussed in Section~\ref{sec:saddle_point_sys},
addresses the solution of the indefinite linear system~\eqref{eq:saddle_point_sys}
arising from problem~\eqref{eq:dual_mixed} by a preconditioned MinRes method.
The main purposes are, on the one hand, to confirm the robustness of the block-diagonal
preconditioner~\eqref{eq:AFW_preconditioner} with respect to arbitrary multiscale
coefficient variations, and on the other hand, to demonstrate its numerical scalability.

\subsection{Numerical tests for solving the system~\eqref{algebraic-hdiv}}\label{sec:Hdiv}
Here we test the ASMG preconditioner for solving the system~\eqref{algebraic-hdiv} with
a matrix corresponding 
to the discretization
of the form . 
\begin{example} \label{ex:1} 
The first set of experiments is for the test cases~[a] and~[b].
In Tables~\ref{table:a_bilinear_alg1_V_m0}--\ref{table:b_bilinear_alg1_W_m1} we report the 
number of outer iterations  for the -level V-cycle and W-cycle ASMG method
defined by
Algorithm~\ref{algorithm1}.
The coarsest mesh is composed of  squares corresponding to  DOF 
on the  space. 
\end{example}

\begin{table}[ht!]
 \begin{center}
 \begin{tabular}{c| c  c | c  c | c c  | c c | c  c }
 \multicolumn {11}{c}{ASMG V-cycle: bilinear form~\eqref{WH-div-prod}, 
Algorithm~\ref{algorithm1}} \\
\multicolumn{1}{c}{~} & \multicolumn{2}{ c |}{} & \multicolumn{2}{c|}{} 
& \multicolumn{2}{c|}{} & \multicolumn{2}{c|}{}
& \multicolumn{2}{c}{} 
\\
\cline{2-11}
&  &  &  &  &  &    &    &  &  &  \\% &  &  \\
\hline 
   &  &   &  &   &   &   &  &   &  &  \\% & ~~ & ~~ \\ 
   &  &   &  &   &   &   &  &   &  &  \\% & ~~ & ~~  \\
   &  &   &  &   &  &   &  &   &  &  \\% & ~~ & ~~  \\
   &  &   &  &   &  &   &  &   &  &  \\% & ~~ & ~~  \\
   &  &   &  &   &  &   &  &   &  &  \\% & ~~ & ~~   \\
   &  &   &  &   &  &   &  &   &  &  \\% & ~~ & ~~   \\
   &  &   &  &   &  &   &  &   &  &  \\% & ~~ & ~~   \\
\end{tabular} \vspace{2ex}
\caption{Example~\ref{ex:1}: case [a] with  and no smoothing steps ()}\label{table:a_bilinear_alg1_V_m0}
 \end{center}
\end{table}

\begin{table}[h!]
 \begin{center}
 \begin{tabular}{c| c  c | c  c | c c  | c c | c  c }
 \multicolumn {11}{c}{ASMG V-cycle: bilinear form~\eqref{WH-div-prod}, 
Algorithm~\ref{algorithm1}} \\
\multicolumn{1}{c}{~} & \multicolumn{2}{ c |}{} & \multicolumn{2}{c|}{} 
& \multicolumn{2}{c|}{} & \multicolumn{2}{c|}{}
& \multicolumn{2}{c}{} 
\\
\cline{2-11}
&  &  &  &  &  &    &    &  &  &  \\ 
\hline 
   &  &   &  &   &  &   &  &   &   &  \\% & ~~ & ~~ \\ 
   &  &   &  &   &  &   &  &   &   &  \\% & ~~ & ~~ \\
   &  &   &  &   &  &   &  &   &   &  \\% & ~~ & ~~ \\
   &  &   &  &   &  &   &  &   &   &  \\% & ~~ & ~~ \\
   &  &   &  &   &  &   &  &   &  &  \\% & ~~ & ~~ \\
   &  &   &  &   &  &   &  &   &  &  \\% & ~~ & ~~ \\
   &  &   &  &   &  &   &  &   &  &  \\% & ~~ & ~~ \\
\end{tabular} \vspace{2ex}
\caption{Example~\ref{ex:1}: case [a] with  and two smoothing steps ()}\label{table:a_bilinear_alg1_V_m2}
 \end{center}
\end{table}

As we can see by comparing the results summarized in Tables~\ref{table:a_bilinear_alg1_V_m0}
and~\ref{table:a_bilinear_alg1_V_m2} the -cycle multigrid preconditioner gains robustness
with respect to the contrast of a binary distribution of a piecewise constant permeability
coefficient when increasing the number of smoothing steps from zero to two. We further observe
an increase of the number of ASMG iterations for decreasing mesh size  (in average -
times). At the same time, as seen in Table~\ref{table:a_bilinear_alg1_W_m1}, the -cycle
preconditioner with one smoothing step is robust with respect to both the contrast and
the mesh size .

\begin{table}[h!]
 \begin{center}
 \begin{tabular}{c| c  c | c  c | c c  | c c | c  c }
 \multicolumn {11}{c}{ASMG W-cycle: bilinear form~\eqref{WH-div-prod}, 
Algorithm~\ref{algorithm1}} \\
\multicolumn{1}{c}{~} & \multicolumn{2}{ c |}{} & \multicolumn{2}{c|}{} 
& \multicolumn{2}{c|}{} & \multicolumn{2}{c|}{}
& \multicolumn{2}{c}{} 
\\
\cline{2-11}
&  &  &  &  &  &    &    &  &  &  \\ 
\hline 
   &  &   &  &   &  &   &  &   &  &  \\% & ~~ & ~~ \\ 
   &  &   &  &   &  &   &  &   &  &  \\% & ~~ & ~~ \\
   &  &   &  &   &  &   &  &   &  &  \\% & ~~ & ~~ \\
   &  &   &  &   &  &   &  &   &  &  \\% & ~~ & ~~ \\
   &  &   &  &   &  &   &  &   &  &  \\% & ~~ & ~~ \\
   &  &   &  &   &  &   &  &   &  &  \\ 
   &  &   &  &   &  &   &  &   &  &  \\ 
\end{tabular} \vspace{2ex}
\caption{Example~\ref{ex:1}: case [a], one smoothing step ()}\label{table:a_bilinear_alg1_W_m1}
 \end{center}
\end{table}

In the next set of numerical experiments we consider the same
distribution of inclusions of low permeability
as before but this time against a background of a randomly distributed piecewise constant
permeability coefficient as shown on Figure~\ref{fig:islands_random}. The results, presented
in Tables~\ref{table:b_bilinear_alg1_V_m0} and~\ref{table:b_bilinear_alg1_V_m2}, are even
better than those obtained for the binary distribution in the sense that here both -
and -cycle are robust with respect to the contrast.

\begin{table}[h!]
 \begin{center}
 \begin{tabular}{c| c  c | c  c | c c  | c c | c  c }
 \multicolumn {11}{c}{ASMG V-cycle: bilinear form~\eqref{WH-div-prod},
Algorithm~\ref{algorithm1}} \\
\multicolumn{1}{c}{~} & \multicolumn{2}{ c |}{} & \multicolumn{2}{c|}{} 
& \multicolumn{2}{c|}{} & \multicolumn{2}{c|}{}
& \multicolumn{2}{c}{} 
\\
\cline{2-11}
&  &  &  &  &  &    &    &  &  &  \\% &  &  \\
\hline 
   &  &   &  &   &   &   &  &   &  &  \\% & ~~ & ~~ \\ 
   &  &   &  &   &   &   &  &   &  &  \\% & ~~ & ~~ \\
   &  &   &  &   &   &   &  &   &  &  \\% & ~~ & ~~ \\
   &  &   &  &   &   &   &  &   &  &  \\% & ~~ & ~~ \\
   &  &   &  &   &   &   &  &   &  &  \\% & ~~ & ~~ \\
   &  &   &  &   &  &   &  &   &  &  \\% & ~~ & ~~ \\
   &  &   &  &   &  &   &  &   &  &  \\% & ~~ & ~~ \\
\end{tabular} \vspace{2ex}
\caption{Example~\ref{ex:1}: case [b], no smoothing steps ()}\label{table:b_bilinear_alg1_V_m0}
 \end{center}
\end{table}
\begin{table}[ht!]
 \begin{center}
 \begin{tabular}{c| c  c | c  c | c c  | c c | c  c }
 \multicolumn {11}{c}{ASMG V-cycle: bilinear form~\eqref{WH-div-prod}, Algorithm~\ref{algorithm1}} \\
\multicolumn{1}{c}{~} & \multicolumn{2}{ c |}{} & \multicolumn{2}{c|}{} 
& \multicolumn{2}{c|}{} & \multicolumn{2}{c|}{}
& \multicolumn{2}{c}{} 
\\
\cline{2-11}
&  &  &  &  &  &    &    &  &  &  \\% &  &  \\
\hline 
   &  &   &  &   &  &   &  &   &   &  \\% & ~~ & ~~ \\ 
   &  &   &  &   &  &   &  &   &   &  \\% & ~~ & ~~ \\
   &  &   &  &   &  &   &  &   &   &  \\% & ~~ & ~~ \\
   &  &   &  &   &  &   &  &   &   &  \\% & ~~ & ~~ \\
   &  &   &  &   &  &   &  &   &   &  \\% & ~~ & ~~ \\
   &  &   &  &   &  &   &  &   &   &  \\% & ~~ & ~~ \\
   &  &   &  &   &  &   &  &   &  &  \\% & ~~ & ~~ \\
\end{tabular} \vspace{2ex}
\caption{Example~\ref{ex:1}: case [b], two smoothing steps ()}\label{table:b_bilinear_alg1_V_m2}
 \end{center}
\end{table}

\begin{table}[ht!]
 \begin{center}
 \begin{tabular}{c| c  c | c  c | c c  | c c | c  c }
 \multicolumn {11}{c}{ASMG W-cycle: bilinear form~\eqref{WH-div-prod}, Algorithm~\ref{algorithm1}} \\
\multicolumn{1}{c}{~} & \multicolumn{2}{ c |}{} & \multicolumn{2}{c|}{} 
& \multicolumn{2}{c|}{} & \multicolumn{2}{c|}{}
& \multicolumn{2}{c}{} 
\\
\cline{2-11}
&  &  &  &  &  &    &    &  &  &   \\% &  &  \\
\hline 
   &  &   &  &   &  &   &  &   &  &  \\% & ~~ & ~~ \\ 
   &  &   &  &   &  &   &  &   &  &  \\% & ~~ & ~~ \\
   &  &   &  &   &  &   &  &   &  &  \\% & ~~ & ~~ \\
   &  &   &  &   &  &   &  &   &  &  \\% & ~~ & ~~ \\
   &  &   &  &   &  &   &  &   &  &  \\% & ~~ & ~~ \\
   &  &   &  &   &  &   &  &   &  &  \\% & ~~ & ~~ \\
   &  &   &  &   &  &   &  &   &  &  \\% & ~~ & ~~ \\
\end{tabular} \vspace{2ex}
\caption{Example~\ref{ex:1}: case [b], one smoothing step ()}\label{table:b_bilinear_alg1_W_m1}
 \end{center}
\end{table}

\begin{example} \label{ex:2} 
The second set of experiments is related to the test case~[c] where, similarly to 
Example~\ref{ex:1}, we examine the performance of the preconditioner for the bilinear
form~\eqref{WH-div-prod}.  Here we compare the ASMG preconditioners defined by
Algorithm~\ref{algorithm1} and Algorithm~\ref{algorithm2}. In this example the
finest mesh is always composed of  elements meaning that changing
the number of levels  refers to a different size of the coarse-grid problem.
Tables~\ref{table:c44_bilinear_V}--\ref{table:c74_bilinear_W} report the number
of outer iterations  and the maximum number of inner iterations  needed
to reduce the residual of the linear systems with the matrix  by a
factor of~.
\end{example}

Tables~\ref{table:c44_bilinear_V} and~\ref{table:c44_bilinear_W} contain the results
for SPE10 slice 44. We see that while the number of inner iterations is about the
same, the number  of outer iterations in case of the V-cycle is on average
 times higher than those for the W-cycle. However, since the complexity of the V-cycle
is lower, the overall performance of these two methods is comparable. Comparing the
two algorithms, we see that they have approximately the same number of inner and outer
iterations. Due to its lower memory requirements we could therefore recommend
Algorithm~\ref{algorithm1} for these kinds of problems. 
On Tables \ref{table:c74_bilinear_V}--\ref{table:c74_bilinear_W}  we present the results
for SPE10 slice 74, which has slightly different permeability distribution but has
the same contrast, . Computational results are pretty much the same for
this case as well.

\begin{table}[ht!]
 \begin{center}
 \begin{tabular}{c| c  c  c  | c c c | c c c | c c c }
 \multicolumn {13}{c}{ASMG V-cycle: bilinear form~\eqref{WH-div-prod}} \\
\multicolumn{1}{c}{~} & \multicolumn{6}{c|}{Algorithm~\ref{algorithm1}} & \multicolumn{6}{c}{Algorithm~\ref{algorithm2}}
\\
\cline{2-13}
 & \multicolumn{3}{ c |}{} & \multicolumn{3}{c|}{} 
& \multicolumn{3}{c|}{} & \multicolumn{3}{c}{}\\
\cline{2-13}
&  &  &  &  &  &    &    &  &  &  &  &   \\
\hline 
   &   &  &  &   &  &     &   &  &  &   &  &    \\ 
   &  &  &  &   &  &     &  &  &  &  &  &    \\ 
   &  &  &  &  &  &     &  &  &  &  &  &    \\
   &  &  &  &  &  &     &  &  &  &  &  &    \\
   &  &  &  &  &  &     &  &  &  &  &  &    \\
\end{tabular} \vspace{2ex}
\caption{Example~\ref{ex:2}: case [c] - slice 44 of SPE10 benchmark}\label{table:c44_bilinear_V}
 \end{center}
\end{table}
\begin{table}[ht!]
 \begin{center}
 \begin{tabular}{c| c  c  c  | c c c | c c c | c c c }
 \multicolumn {13}{c}{ASMG W-cycle: bilinear form~\eqref{WH-div-prod} } \\
\multicolumn{1}{c}{~} & \multicolumn{6}{c|}{Algorithm~\ref{algorithm1}} & \multicolumn{6}{c}{Algorithm~\ref{algorithm2}}
\\
\cline{2-13}
 & \multicolumn{3}{ c |}{} & \multicolumn{3}{c|}{} 
& \multicolumn{3}{c|}{} & \multicolumn{3}{c}{}\\
\cline{2-13}
&  &  &  &  &  &    &    &  &  &  &  &   \\
\hline 
   &  &  &  &  &  &     &  &  &  &  &  &     \\ 
   &  &  &  &  &  &     &  &  &  &  &  &     \\
   &  &  &  &  &  &     &  &  &  &  &  &     \\
   &  &  &  &  &  &     &  &  &  &  &  &     \\
   &  &  &  &  &  &     &  &  &  &  &  &     \\
\end{tabular} \vspace{2ex}
\caption{Example~\ref{ex:2}: case [c] - slice 44 of SPE10 benchmark}\label{table:c44_bilinear_W}
 \end{center}
\end{table}
\begin{table}[ht!]
 \begin{center}
 \begin{tabular}{c| c  c  c  | c c c | c c c | c c c }
 \multicolumn {13}{c}{ASMG V-cycle: bilinear form~\eqref{WH-div-prod}} \\
\multicolumn{1}{c}{~} & \multicolumn{6}{c|}{Algorithm~\ref{algorithm1}} & \multicolumn{6}{c}{Algorithm~\ref{algorithm2}}
\\
\cline{2-13}
 & \multicolumn{3}{ c |}{} & \multicolumn{3}{c|}{} 
& \multicolumn{3}{c|}{} & \multicolumn{3}{c}{}\\
\cline{2-13}
&  &  &  &  &  &    &    &  &  &  &  &   \\
\hline 
   &   &  &  &   &  &     &    &  &  &   &  &     \\ 
   &  &  &  &   &  &     &   &  &  &  &  &     \\
   &  &  &  &  &  &     &   &  &  &  &  &     \\
   &  &  &  &  &  &     &   &  &  &  &  &     \\
   &  &  &  &  &  &     &   &  &  &  &  &     \\
\end{tabular} \vspace{2ex}
\caption{Example~\ref{ex:2}: case [c] - slice 54 of SPE10 benchmark}\label{table:c54_bilinear_V}
 \end{center}
\end{table}
\begin{table}[ht!]
 \begin{center}
 \begin{tabular}{c| c  c  c  | c c c | c c c | c c c }
 \multicolumn {13}{c}{ASMG W-cycle: bilinear form~\eqref{WH-div-prod} } \\
\multicolumn{1}{c}{~} & \multicolumn{6}{c|}{Algorithm~\ref{algorithm1}} & \multicolumn{6}{c}{Algorithm~\ref{algorithm2}}
\\
\cline{2-13}
 & \multicolumn{3}{ c |}{} & \multicolumn{3}{c|}{} 
& \multicolumn{3}{c|}{} & \multicolumn{3}{c}{}\\
\cline{2-13}
&  &  &  &  &  &    &    &  &  &  &  &   \\
\hline 
   &  &  &  &  &  &     &  &  &  &  &  &     \\ 
   &  &  &  &  &  &     &  &  &  &  &  &     \\
   &  &  &  &  &  &     &  &  &  &  &  &     \\
   &  &  &  &  &  &     &  &  &  &  &  &     \\
   &  &  &  &  &  &     &  &  &  &  &  &     \\
\end{tabular} \vspace{2ex}
\caption{Example~\ref{ex:2}: case [c] - slice 54 of SPE10 benchmark}\label{table:c54_bilinear_W}
 \end{center}
\end{table}
\begin{table}[ht!]
 \begin{center}
 \begin{tabular}{c| c  c  c  | c c c | c c c | c c c }
 \multicolumn {13}{c}{ASMG V-cycle: bilinear form~\eqref{WH-div-prod}} \\
\multicolumn{1}{c}{~} & \multicolumn{6}{c|}{Algorithm~\ref{algorithm1}} & \multicolumn{6}{c}{Algorithm~\ref{algorithm2}}
\\
\cline{2-13}
 & \multicolumn{3}{ c |}{} & \multicolumn{3}{c|}{} 
& \multicolumn{3}{c|}{} & \multicolumn{3}{c}{}\\
\cline{2-13}
&  &  &  &  &  &   &   &  & 
&  &  &   \\
\hline 
   &   &  &  &   &  &     &   &  &  &   &  &     \\ 
   &  &  &  &  &  &     &  &  &  &  &  &     \\
   &  &  &  &  &  &     &  &  &  &  &  &     \\
   &  &  &  &  &  &     &  &  &  &  &  &     \\
   &  &  &  &  &  &     &  &  &  &  &  &     \\
\end{tabular} \vspace{2ex}
\caption{Example~\ref{ex:2}: case [c] - slice 74 of SPE10 benchmark}\label{table:c74_bilinear_V}
 \end{center}
\end{table}
\begin{table}[ht!]
 \begin{center}
 \begin{tabular}{c| c  c  c  | c c c | c c c | c c c }
 \multicolumn {13}{c}{ASMG W-cycle: bilinear form~\eqref{WH-div-prod} } \\
\multicolumn{1}{c}{~} & \multicolumn{6}{c|}{Algorithm~\ref{algorithm1}} & \multicolumn{6}{c}{Algorithm~\ref{algorithm2}}
\\
\cline{2-13}
 & \multicolumn{3}{ c |}{} & \multicolumn{3}{c|}{} 
& \multicolumn{3}{c|}{} & \multicolumn{3}{c}{}\\
\cline{2-13}
&  &  &  &  &  &    &    &  &  &  &  &   \\
\hline 
   &  &  &  &  &  &     &  &  &  &  &  &     \\ 
   &  &  &  &  &  &     &  &  &  &  &  &     \\
   &  &  &  &  &  &     &  &  &  &  &  &     \\
   &  &  &  &  &  &     &  &  &  &  &  &     \\
   &  &  &  &  &  &     &  &  &  &  &  &     \\
\end{tabular} \vspace{2ex}
\caption{Example~\ref{ex:2}: case [c] - slice 74 of SPE10 benchmark}\label{table:c74_bilinear_W}
 \end{center}
\end{table}

\subsection{Testing of block-diagonal preconditioner for system~\eqref{eq:saddle_point_sys}
within MinRes iteration}\label{sec:saddle_point_sys}
Now we present a number of numerical experiments for solving the mixed finite element
system~\eqref{eq:saddle_point_sys} by using
a preconditioned MinRes method.
We consider two different examples, first, Example~\ref{ex:3} in which the performance
of the block-diagonal preconditioner and its dependence on the accuracy of the inner
solves with W-cycle ASMG preconditioner is evaluated, and second, 
Example~\ref{ex:4}
that tests the scalability of the MinRes iteration, again using a W-cycle ASMG preconditioner
with one smoothing step for the inner iterations.
\begin{example}\label{ex:3} 
  Here we apply the MinRes iteration to
  solve~\eqref{eq:saddle_point_sys} for test case [c]. The hierarchy 
  of meshes is the same as in Example~\ref{ex:2}. 
An ASMG W-cycle
  based on Algorithm~\ref{algorithm1} with one smoothing step has been
  used as a preconditioner on the 
  space. Table~\ref{table:c44_saddle_W_m1} shows the number of MinRes
  iterations denoted by , the
  maximum number of ASMG iterations  needed to achieve an
  ASMG residual reduction by . 
\end{example}
\begin{table}[ht!]
 \begin{center}
 \begin{tabular}{c |  c  c  | c c | c c }
 \multicolumn {7}{c}{MinRes iteration: saddle point system~\eqref{eq:saddle_point_sys}} \\
\multicolumn{1}{c}{~} & \multicolumn{2}{c|}{} & \multicolumn{2}{c|}{} 
& \multicolumn{2}{c}{}
 \\
\cline{2-7}
&   &  
&  &   
&  &   
\\
\hline 
 &  &  
&  &  
&  &  
\\ 
 &  &  
&  &  
&  &  
\\
 &  &  
&  &  
&  &  
\\
 &  &  
&  &  
&  &  
\\
 &  &  
&  &  
&  &  
\\
\end{tabular} \vspace{2ex}
\caption{Example~\ref{ex:3}: case [c] - slice 44 of SPE10 benchmark. The hierarchy 
  of meshes is the same as in Example~\ref{ex:2}.}\label{table:c44_saddle_W_m1}
 \end{center}
\end{table}

\begin{example}\label{ex:4}
In the last set of experiments the MinRes iteration has been used to 
solve~\eqref{eq:saddle_point_sys} for test case~[c] on the same hierarchy of 
meshes as in Example~\ref{ex:1}. An ASMG W-cycle based on Algorithm~\ref{algorithm1} with
one smoothing step has been used as a preconditioner on the  space for a residual
reduction by . Table~\ref{table:c44_saddle_W_m1_m} shows the number of MinRes
iterations ,
the maximum number of (inner) ASMG iterations  per (outer) MinRes iteration,
and the number of DOF. Note that as long as the product  is constant, 
the total number of arithmetic operations required to achieve any prescribed 
accuracy is proportional to the number of DOF. 
\end{example}
\begin{table}[ht!]
 \begin{center}
 \begin{tabular}{c|ccr}
 \multicolumn {4}{c}{MinRes iteration: saddle point system~\eqref{eq:saddle_point_sys}} \\
 &  &  & \multicolumn{1}{c}{DOF}
\\
\hline 
 &  &  &    
\\
 &  &  &   
\\
 &  &  &   
\\
 &  &  &  
\\
 &  &  &  
\\
\end{tabular} \vspace{2ex}
\caption{Example~\ref{ex:3}: case [c] - slice 44 of SPE10
  benchmark. The hierarchy of meshes is the same as in
  Example~\ref{ex:1}.}\label{table:c44_saddle_W_m1_m}
 \end{center}
\end{table}

\subsection{Some conclusions and general comments regarding the numerical experiments}

The presented numerical results clearly demonstrate the efficiency of the proposed algebraic
multilevel iteration (AMLI)-cycle auxiliary space multigrid (ASMG) preconditioner
for problems with highly varying
coefficients
as they typically arise in the mathematical modelling of physical processes in high-contrast
and high-frequency media. 

The first group of tests examines
the convergence behavior of the nonlinear ASMG method
for the weighted bilinear 
form~\eqref{WH-div-prod}. 
This is a key 
point in the presented study.
The cases~[a] and~[b] are designed to represent a typical multiscale geometry with
islands and channels. Although case~[b] (a background with a random coefficient)
appears to be more complicated, the impact of the multiscale heterogeneity seems to
be stronger in the binary case~[a] where the number of iterations is slightly larger. 
However, in both cases we
observe a uniformly converging
ASMG V-cycle 
with  and W-cycle () with . Some small 
fluctuations of the results
in the right-lower corner of the tables could be due to some round off effects. 
Case~[c] (SPE10) is
a popular benchmark problem in the petroleum engineering
community. Here we observe robust and uniform convergence with respect
to the number of levels , or, equivalently, mesh-size . Note
that such uniform convergence is present for the ASMG V-cycle even
without smoothing iterations (i.e.  ) and for both,
Algorithm~\ref{algorithm1} and Algorithm~\ref{algorithm2}. In
addition, Algorithm~\ref{algorithm2} is computationally more favorable
when compared to Algorithm~\ref{algorithm1} because
the matrices used in Algorithm~\ref{algorithm2}
have fewer non-zeroes (they are sparser).

The last two tables,
Table~\ref{table:c44_saddle_W_m1}--\ref{table:c44_saddle_W_m1_m},
confirm the expected optimal convergence rate
of the block-diagonally preconditioned MinRes iteration applied
to the coupled saddle point system~\eqref{eq:saddle_point_sys}.
The results in Table~\ref{table:c44_saddle_W_m1} demonstrate 
how the efficiency (in terms of the product ) is achieved 
for a relative accuracy of  of the inner ASMG solver. 
Table \ref{table:c44_saddle_W_m1_m} illustrates
the scalability of the solver
indicated by a (almost) constant number of  MinRes and ASMG
iterations since the total computational work is
in terms of fine grid matrix vector multiplications is proportional to
the product .

Although not in the scope of this study, we note that the proposed
auxiliary space multigrid method would be suitable for implementation
on distributed memory computer architectures.

\appendix
\section{Discrete inf-sup condition}\label{append}
Here we provide a proof of the discrete inf-sup condition
\eqref{inf_sup_fin_set} for the bilinear form arising in the mixed
finite element method.  We begin by introducing some details regarding
the finite element spaces involved in the approximation of
problem~\eqref{eq:dual_mixed} or \eqref{eq:LS}.

\subsection{Raviart-Thomas-{N\'ed\'elec\ } space} 

We consider the standard lowest order
Raviart-Thomas-{N\'ed\'elec\ } space . 
Recall that every element  can be written as
\begin{equation}\label{eq:RT}
{{\mathbf v}} = \sum_{e\in \mathcal{E}_h} \sigma_e({{\mathbf v}} ){{\boldsymbol \psi}}_e({{\mathbf x}}),
\quad \sigma_e({{\mathbf v}})  = \int_e {{\mathbf v}}\cdot{{\mathbf n}}_e.
\end{equation}
Here the vector  has a fixed direction (normal to ), and
this direction is set once and for all for every
face .  For an element 
let  define the unit normal vector to  which points outward with respect to . 
Now, if  is the intersection of two elements from
, , then  is the element for
which   and 
 is the element for which 
.  
If  is on  the boundary of the domain then the corresponding
element is  and  is missing. Finally, for a
piece-wise constant function  we denote
\[
q_{\pm,e} = q\big|_{T^\pm_e}, \quad e\in \mathcal{E}_h.
\]

\begin{equation}\label{eq:RT-basis}
{{\boldsymbol \psi}}_e\big|_T = \frac{c_d}{|T|}({{\mathbf x}}-{{\mathbf x}}_{P_e}).
\end{equation}

\begin{equation}\label{eq:RT-basis-cube}
{{\boldsymbol \psi}}_k^{\pm}({{\mathbf x}}) = \frac{({{\mathbf x}}-{{\mathbf x}}_{M,k}^{\mp})^T {{\mathbf e}}_k  }{|T|} {{\mathbf e}}_k .
\end{equation}

\[
\int_{F_j^{\pm}}{{\boldsymbol \psi}}_k^+\cdot {{\mathbf n}}_{F^\pm_j} = \int_{F_k^{-}}{{\boldsymbol \psi}}_k^+\cdot {{\mathbf n}}_{F^-_k} = 0,
\;\mbox{for}\;  j\neq k,\; 
\;\mbox{and}\; 
\int_{F_k^{+}}{{\boldsymbol \psi}}_k^+\cdot {{\mathbf n}}_{F^+_k} = 1.
\]

\begin{equation}\label{eq:scaling}
\|{{\boldsymbol \psi}}_e\|_{0,T}^2 \sim h_e^{2-d},\quad h_e=\operatorname{diam}(e),\quad
\quad T^+\cap T^-=e\in \mathcal{E}_h. 
\end{equation}

\[
({{\mathbf u}},{{\mathbf v}})_{*,\omega} = \sum_{e\in \mathcal{E}_h} 
\omega_{e}\sigma_{e}({{\mathbf u}}) \sigma_{e}({{\mathbf v}})\| {{\boldsymbol \psi}}_e\|^2, \quad
({{\mathbf u}},{{\mathbf v}})_{*,1} = \sum_{e\in \mathcal{E}_h} 
\sigma_{e}({{\mathbf u}}) \sigma_{e}({{\mathbf v}})\|{{\boldsymbol \psi}}_e\|^2.
\]

\begin{equation}\label{eq:operator}
\sigma_e(D_{\omega} {{\mathbf v}}) = \omega_e\sigma_e({{\mathbf v}}). 
\end{equation}

\begin{equation}\label{discrete-grad}
({{\mathbf v}},\nabla_h q)_{*,1}  = -({\operatorname{div}} {{\mathbf v}},q),
\quad\mbox{for all}\quad {{\mathbf v}}\in{{\boldsymbol V}_{\hspace{-0.2mm}h}}.
\end{equation}

$$
\sigma_e(\nabla_h q) = 
\frac{{\lbrack\!\lbrack {q} \rbrack\!\rbrack}_e}{\|{{\boldsymbol \psi}}_e\|^2}, \quad {\lbrack\!\lbrack {q} \rbrack\!\rbrack}_e = q_{-,e}-q_{+,e}.
$$

\[
(D_\omega \nabla_h \varphi,\nabla_h \chi)_{*,1} = 
\sum_{e\in\mathcal{E}_h}
\frac{\omega_e}{\| {{\boldsymbol \psi}}_e\|^2}{\lbrack\!\lbrack {\varphi} \rbrack\!\rbrack}_e{\lbrack\!\lbrack {\chi} \rbrack\!\rbrack}_e
\]

\[
(D_\omega \nabla_h \varphi,\nabla_h \varphi)_{*,1} =
(\nabla_h \varphi,\nabla_h \varphi)_{*,\omega} = 
\sum_{e\in\mathcal{E}_h}
\frac{\omega_e}{\|{{\boldsymbol \psi}}_e\|^2}({\lbrack\!\lbrack {\varphi} \rbrack\!\rbrack}_e)^2\ge 0.
\]

\begin{equation}\label{eq:alphae}
\begin{aligned}
\widehat{\alpha}\in \mathbb{R}^{|\mathcal{E}_h|}, \quad 
\widehat{\alpha}_e =  
\frac{\| {{\boldsymbol \psi}}_e\|^2_{0,\alpha,T_e^+}}{\| {{\boldsymbol \psi}}_e\|^2}+
\frac{\|{{\boldsymbol \psi}}_e\|^2_{0,\alpha,T_e^-}}{\| {{\boldsymbol \psi}}_e\|^2},
\quad \widehat{\kappa}\in \mathbb{R}^{|\mathcal{E}_h|}, 
\quad \widehat{\kappa}_e = \widehat{\alpha}_e^{-1}.
\end{aligned}
\end{equation}

\begin{eqnarray*}
\|{{\mathbf u}}\|^2_{0,\alpha} & = & 
\sum_T \int_T \alpha_T{{\mathbf u}}\cdot {{\mathbf u}}  
=  
\sum_T 
\sum_{e\in\partial T}\sum_{e'\in \partial T}
\sigma_e({{\mathbf u}})\sigma_{e'}({{\mathbf u}})\int_T \alpha_T{{\boldsymbol \psi}}_e\cdot {{\boldsymbol \psi}}_{e'}
\\[1.7ex]
&\le &
\sum_T 
\sum_{e\in\partial T}\sum_{e'\in \partial T}
|\sigma_e({{\mathbf u}})||\sigma_{e'}({{\mathbf u}})|\|{{\boldsymbol \psi}}_e\|_{0,\alpha,T}\|{{\boldsymbol \psi}}_{e'}\|_{0,\alpha,T}
\\[1.7ex]
&\le &
\frac12\sum_T 
\sum_{e\in\partial T}\sum_{e'\in \partial T}
\left([\sigma_e({{\mathbf u}})]^2\|{{\boldsymbol \psi}}_e\|^2_{0,\alpha,T}+
[\sigma_{e'}({{\mathbf u}})]^2\|{{\boldsymbol \psi}}_{e'}\|^2_{0,\alpha,T}\right)
\\[1.7ex]
& = & 
(d+1)\sum_T 
\sum_{e\in\partial T}[\sigma_e({{\mathbf u}})]^2\|{{\boldsymbol \psi}}_e\|^2_{0,\alpha,T}. 
\end{eqnarray*}

\begin{equation}\label{eq:norm-equivalence}
\begin{array}{rcl}
\|{{\mathbf u}}\|^2_{0,\alpha} & \le & 
(d+1)\sum_{e\in\mathcal{E}_h}\left(\| {{\boldsymbol \psi}}_e\|^2_{0,\alpha,T^+_e}
  +\|{{\boldsymbol \psi}}_e\|^2_{0,\alpha,T_e^-} 
\right)
[\sigma_e({{\mathbf u}})]^2 \\[1.7ex]
& = & 
(d+1)\sum_{e\in\mathcal{E}_h}\widehat{\alpha}_{e}[\sigma_e({{\mathbf u}})]^2\|  {{\boldsymbol \psi}}_e\|^2
 = (d+1)\|{{\mathbf u}}\|^2_{*,\widehat{\alpha}}.
\end{array}
\end{equation}

\begin{equation}\label{eq:mass-matrix}
\sum_{e\in\partial T}\sum_{e'\in \partial T}
 \sigma_e({{\mathbf u}} )\sigma_{e'}({{\mathbf u}})\int_T {{\boldsymbol \psi}}_e {{\boldsymbol \psi}}_{e'}
 \eqsim \sum_{e\in\partial T} [\sigma_e({{\mathbf u}})]^2\| {{\boldsymbol \psi}}_e\|^2.
\end{equation}

\begin{equation}\label{eq:Poincare}
\|\chi\|^2 \le C_P\|\nabla_h \chi\|^2_{*,1}, 
\end{equation}

\begin{equation}\label{eq:inf-sup-1}
\sup_{{{\mathbf v}}\in {{\boldsymbol V}_{\hspace{-0.2mm}h}}}\frac{(q,{\operatorname{div}} {{\mathbf v}})}{\|{{\mathbf v}}\|_{\Lambda_1}}\ge \widetilde{\gamma}\|q\|. 
\end{equation}

\begin{eqnarray*}
\|\nabla_h \chi\|_{*,1} 
&= &\sup_{{{\mathbf v}}\in {{\boldsymbol V}}_h}\frac{(\nabla_h \chi,{{\mathbf v}})_{*,1}}{\|{{\mathbf v}}\|_{*,1}}
= \sup_{{{\mathbf v}}\in {{\boldsymbol V}}_h}\frac{(\chi,{\operatorname{div}} {{\mathbf v}})}{\|{{\mathbf v}}\|_{*,1}}\\
&\gtrsim& \sup_{{{\mathbf v}}\in {{\boldsymbol V}}_h}\frac{(\chi,{\operatorname{div}} {{\mathbf v}})}{\|{{\mathbf v}}\|}
\ge \sup_{{{\mathbf v}}\in {{\boldsymbol V}}_h}\frac{(\chi,{\operatorname{div}} {{\mathbf v}})}{\|{{\mathbf v}}\|_{\Lambda_1}}
\ge \widetilde{\gamma}\|\chi\|.
\end{eqnarray*}

\[
\widehat{\kappa}_e 
=\frac{2K_{+,e}K_{-,e}}{K_{+,e}+K_{-,e}}
\]

\begin{equation}\label{eq:variational}
(D_{\widehat{\kappa}}\nabla_h \varphi,\nabla_h \chi)_{*,1} = (q,\chi),\quad \mbox{for all}\quad \chi\in 
W_h. 
\end{equation}

\begin{equation}\label{eq:def-w}
{{\mathbf w}} = D_{\widehat{\kappa}}\nabla_h \varphi
\end{equation}

\begin{equation}\label{eq:div-w}
{\operatorname{div}} D_{\widehat{\kappa}}\nabla_h\varphi=-q.
\end{equation}

\begin{equation}\label{eq:a-priori}
\|{{\mathbf w}}\|^2_{*,\widehat{\alpha}} \le 2C_P\|q\|^2.
\end{equation}

\begin{equation*}
\begin{aligned}
\|{{\mathbf w}}\|^2_{*,\widehat{\alpha}}& =  
(D_{\widehat{\alpha}}{{\mathbf w}},{{\mathbf w}})_{*,1} 
&\qquad\mbox{[by Proposition~\ref{proposition:simple}(i)]}\\
& = 
(D_{\widehat{\alpha}}\;D_{\widehat{\kappa}}\nabla_h\varphi,{{\mathbf w}})_{*,1}
& \qquad\mbox{[by the definition of ]}
\\
& =   (\nabla_h\varphi,{{\mathbf w}})_{*,1}
&\qquad \mbox{by [Lemma~\ref{lemma:mass-equivalence}(ii)]}\\
&=  -(\varphi,{\operatorname{div}} {{\mathbf w}})=(\varphi, q) \le  \|\varphi\|\|q\|. 
&\qquad\mbox{[by~\eqref{eq:div-w}]}
\end{aligned}
\end{equation*}

\begin{equation*}
\begin{aligned}
\|\varphi\|^2 &\le C_P(\nabla_h\varphi,\nabla_h\varphi)_{*,1} 
&\qquad\mbox{[by \eqref{eq:Poincare}]}\\
&\le 2C_P(D_{\widehat{\kappa}}\nabla_h\varphi,\nabla_h\varphi)_{*,1}
&\qquad\mbox{[by Lemma~\ref{lemma:mass-equivalence}(iii)]}\\
&=
2C_P(D_{\widehat{\kappa}}\nabla_h\varphi,D_{\widehat{\alpha}}D_{\widehat{\kappa}}
  \nabla_h\varphi)_{*,1},
&\qquad\mbox{[by Lemma~\ref{lemma:mass-equivalence}(ii)]}\\
& =  2C_P\|{{\mathbf w}}\|^2_{*,\widehat{\alpha}} 
& \qquad\mbox{[by the definition of ]}
\end{aligned}
\end{equation*}

\begin{equation}
\inf_{q\in W_h}\sup_{{{\mathbf v}}\in {{\boldsymbol V}_{\hspace{-0.2mm}h}}}\frac{(q,{\operatorname{div}}
  {{\mathbf v}})}{\|{{\mathbf v}}\|_{\Lambda_\alpha}\;\|q\|}\ge \gamma.
\end{equation}

\begin{equation*}
\begin{aligned}
  \sup_{{{\mathbf v}}\in {{\boldsymbol V}_{\hspace{-0.2mm}h}}} \frac{(q,{\operatorname{div}} {{\mathbf v}})}{\|{{\mathbf v}}\|_{\Lambda_\alpha}} &
  \ge \|q\|^2 /\left (\|{{\mathbf w}}\|_{0,\alpha}^2+\|{\operatorname{div}} {{\mathbf w}}\|^2
    \right )^\frac12  &\qquad \mbox{[by the definition of ]}
  \\
  &\ge  \|q\|^2 / \left (  (d+1)\|{{\mathbf w}}\|^2_{*,\widehat{\alpha}}+\|{\operatorname{div}} {{\mathbf w}}\|^2
      \right )^\frac12 &\qquad 
\mbox{[by      Lemma~\ref{lemma:mass-equivalence}(i)]} 
\\
&= \|q\|^2 / \left ( (d+1)\|{{\mathbf w}}\|^2_{*,\widehat{\alpha}}+\|q\|^2 \right )^\frac12
&\qquad \mbox{[by \eqref{eq:div-w}]}\\
&\ge \|q\|/ \sqrt{2(d+1)C_P+1},
& \qquad \mbox{[by Lemma~\ref{lemma:a-priori}]}
\end{aligned}
\end{equation*}

\begin{equation}\label{eq:cea}
  \|{{\mathbf u}}-{{\mathbf u}}_h\|_{\Lambda_\alpha} +\|p-p_h\|
\le 
C \left( \inf_{{{\mathbf v}}\in {{\boldsymbol V}_{\hspace{-0.2mm}h}}}\|{{\mathbf u}}-{{\mathbf v}}\|_{\Lambda_\alpha}+
\inf_{q\in W_h}\|p-q\|\right),
\end{equation}

\end{document}
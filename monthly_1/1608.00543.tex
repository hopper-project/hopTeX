
\documentclass{amsart}
\usepackage[english]{babel}
\usepackage[cp1250]{inputenc}
\usepackage{amsmath,amssymb}
\usepackage{microtype}
\usepackage{float}
\usepackage{enumitem}
\usepackage{mathrsfs}

\usepackage{tikz}
\usetikzlibrary{matrix,arrows}
\usetikzlibrary{decorations.markings}

\pagestyle{plain}

\newtheorem{thm}{Theorem}[section]
\newtheorem{prop}[thm]{Proposition}
\newtheorem{lem}[thm]{Lemma}
\newtheorem{cor}[thm]{Corollary}
\newtheorem{claim}{Claim}
\newtheorem*{claim*}{Claim}

\theoremstyle{definition}
\newtheorem{definition}[thm]{Definition}
\newtheorem{remark}[thm]{Remark}
\newtheorem{example}[thm]{Example}
\newtheorem{notation}[thm]{Notation}
\newtheorem{alg}[thm]{Algorithm}
\newtheorem*{remark*}{Remark}

\begin{document}

\title[Fillability of small Seifert fibered spaces]{Fillability of small Seifert fibered spaces}

\author{Irena Matkovi\v{c}}
\address{Department of Mathematics, Central European University, 1051 Budapest, Hungary}
\email{matkovic\_irena@phd.ceu.edu}

\begin{abstract}
We characterize fillable structures among zero-twisting contact structures on small Seifert fibered spaces of the form $M(-1;r_1,r_2,r_3)$.
\end{abstract}

\subjclass[2010]{57R17}
\keywords{Seifert fibered 3-manifolds, fillability, monodromy factorization} 

\maketitle

\section{Introduction}
Tight contact structures on small Seifert fibered spaces $M(e_0;r_1,r_2,r_3), e_0\in{\mathbb Z}, r_i\in{\mathbb Q}\cap(0,1),$ whenever $e_0\neq -1$ or $-2$ are completely classified \cite{Wu, GLS0}, they are all given by Legendrian surgery construction, hence Stein fillable. Same holds for Seifert manifolds with $e_0=-2$ which are $L$-spaces \cite{G}. On $M(-1;r_1,r_2,r_3)$ tightness of zero-twisting structures is conjecturally \cite{S} characterized by equality $d_3(\xi)=d(M,\mathbf{t}_\xi)$; in particular case of $L$-spaces this covers all tight structures and has been confirmed in \cite{M}.

\begin{figure}[H]
\begin{tikzpicture}
\begin{scope}[shift={(-2,0)}, decoration={markings,mark=at position 0.5 with {\arrow{>}}}]
\begin{scope}[shift={(0,0)}]
\draw [domain=-0.5*pi:1.5*pi, scale=0.5] plot (\x, {2*sin(\x r)});
\draw [scale=0.5,postaction={decorate}] (-0.5*pi,-2)--(1.5*pi,-2);
\draw [scale=0.5] (-0.5*pi,-2) circle (0.1pt) node[left,scale=0.7]{$-\frac{1}{r_3}$};
\end{scope}
\begin{scope}[shift={(.06,-.3)},scale=0.95]
\draw [domain=-0.5*pi:1.5*pi, scale=0.5] plot (\x, {2*sin(\x r)});
\draw [scale=0.5,postaction={decorate}] (-0.5*pi,-2)--(1.5*pi,-2) node[right,scale=0.7]{$-\frac{1}{r_2}$};
\end{scope}
\begin{scope}[shift={(.12,-.6)},scale=0.9]
\draw [domain=-0.5*pi:1.5*pi, scale=0.5] plot (\x, {2*sin(\x r)});
\draw [scale=0.5,postaction={decorate}] (-0.5*pi,-2)--(1.5*pi,-2);
\draw [scale=0.5] (-0.5*pi,-2) circle (0.1pt) node[left,scale=0.7]{$-\frac{1}{r_1}$};
\end{scope}
\begin{scope}[shift={(.18,-.9)},scale=0.85]
\draw [domain=-0.5*pi:1.5*pi, scale=0.5] plot (\x, {2*sin(\x r)});
\draw [scale=0.5,postaction={decorate}] (-0.5*pi,-2)--(1.5*pi,-2)node[right,scale=0.7]{$+1$};
\end{scope}
\begin{scope}[shift={(.24,-1.2)},scale=0.8]
\draw [domain=-0.5*pi:1.5*pi, scale=0.5] plot (\x, {2*sin(\x r)});
\draw [scale=0.5,postaction={decorate}] (-0.5*pi,-2)--(1.5*pi,-2)node[right,scale=0.7]{$+1$};
\end{scope}
\end{scope}

\draw[white] (1,-0.5) circle (0pt) node[right, black]{$\cong$};

\begin{scope}[shift={(-0.5,0)}]
\draw (3.5,-0.5) circle (1.3cm);
\draw [fill=white,white] (4.5,-0.4) rectangle (4.9,0);
\draw (4.8,-0.4) circle (0pt) node[above,scale=0.6]{$-\frac{r_3+1}{r_3}$};
\draw (3.35,-0.5) circle (1.1cm);
\draw [fill=white,white] (4.2,-0.9) rectangle (4.5,-0.5);
\draw (4.45,-0.5) circle (0pt) node[below,scale=0.6]{$-\frac{r_2+1}{r_2}$};
\draw (3.2,-0.5) circle (0.9cm);
\draw [fill=white,white] (3.8,-0.45) rectangle (4.2,-0.1);
\draw (4.1,-0.5) circle (0pt) node[above,scale=0.6]{$-\frac{r_1+1}{r_1}$};
\draw (3.05,-0.5) circle (0.7cm);
\draw (3.7,-0.6) circle (0pt) node[right,scale=0.7]{$0$};
\draw (2.9,-0.5) circle (0.5cm);
\draw (3.35,-0.5) circle (0pt) node[right,scale=0.7]{$0$};
\draw [fill=white] (2.1,-0.6) rectangle (2.5,-0.3) node[below left,scale=0.7]{$-1$};
\end{scope}

\draw[white] (4.9,-0.5) circle (0pt) node[right,black]{$\stackrel{\partial}{\cong}$};

\begin{scope}[scale=0.8]
\draw (7.5,-0.5) circle (0pt) node[left,scale=0.7]{$-1$};
\draw (9,0) ellipse (1.8cm and 0.6cm);
\draw [fill=white,white] (8.35,-0.45) rectangle (8.55,-0.65);
\draw [fill=white,white] (9.3,-0.6) rectangle (9.45,-0.4);
\draw [fill=white,white] (10.1,-0.35) rectangle (10.3,-0.55);
\draw (7.9,-.55) arc (-160:170:0.3);
\draw[white] (8,-0.8) circle (0pt) node[below,black,scale=0.7]{$-\frac{1}{r_1}$};
\draw (8.8,-.7) arc (-160:160:0.3);
\draw[white] (8.9,-.9) circle (0pt) node[below,black,scale=0.7]{$-\frac{1}{r_2}$};
\draw (9.7,-.65) arc (-160:170:0.3);
\draw[white] (9.85,-.9) circle (0pt) node[below,black,scale=0.7]{$-\frac{1}{r_3}$};
\end{scope}

\end{tikzpicture}
\caption{Contact structures on $M(-1;r_1,r_2,r_3)$, followed by the smoothened surgery diagram of the underlying $3$-manifold and its standard presentation.}
\label{fig:SFS}
\end{figure}

These tight structures are all described by contact surgery diagrams of Figure \ref{fig:SFS}, as shown by Lisca and Stipsicz in \cite{LS.III}, and in particular they are all supported by planar open books. But in contrast to $e_0\neq -1$ cases not all of them are Stein fillable, with addition of a theorem of Wendl \cite{W} non-Stein fillable are not fillable at all. Non-fillability was first observed by Ghiggini, Lisca and Stipsicz in \cite{GLS} for a particular structure on $M(-1;\frac{1}{2},\frac{1}{2},\frac{1}{p})$. Based on their classification \cite{GLS} of tight structures on $M(-1;r_1,r_2,r_3)$ for $r_1\geq r_2\geq\frac{1}{2}$, Plamenevskaya and Van Horn-Morris \cite{PVH-M} then recognized exactly which of those manifolds admit non-fillable tight structures using Wendl's work and obstructing existence of positive factorizations in (abelianization of) standardly associated (planar) open books. To the other end, Lecuona and Lisca \cite{LL} showed that when $r_i+r_j<1$ for all pairs $i,j$ (called, for manifolds of special type) topology (diagonalization argument) prevents existence of Stein fillings.

Here we show that all fillable zero-twisting structures on $M(-1;r_1,r_2,r_3)$ arise as Legendrian surgeries on tight $S^1\times S^2$. More specifically, we obtain the following.

\begin{thm}\label{thm}
Assume that a contact structure $\xi$ on $M(-1;r_1,r_2,r_3)$ is given by some surgery diagram of Figure \ref{fig:SFS}. For each pair $i,j$ for which $r_i+r_j\geq1$, form a sublink $L_{ij}$ of the surgery link consisting of two unknots with $+1$-coefficient and two truncated chains such that rational numbers $-\frac{1}{s_i}$ they present satisfy $s_i\leq r_i, s_j\leq r_j,$ and $s_i+s_j=1$. Then $\xi$ is fillable if and only if there exists $L_{ij}$ which describes tight $S^1\times S^2$.
\end{thm}

In words, fillability of a given surgery presentation is completely decided on specific sublinks representing $S^1\times S^2$, whose tightness is in turn met by a unique choice of rotation numbers for this sublink (Proposition \ref{prop:+factor}). In particular, we reprove the result of Lecuona and Lisca that small Seifert fibered manifolds of special type do not admit any fillable structure.

\subsection*{Overview} Proof of the theorem is split between two sections. In Section \ref{Sec2}, following the approach of Plamenevskaya and Van Horn-Morris, we obstruct positive factorization of some monodromies. In Section \ref{Sec3} we confirm for all remaining structures the existence of surgery link of $S^1\times S^2$ as a sublink of contact presentations, and show that given rotation numbers on it provide tight structure.

\subsection*{Acknowledgement} I thank my mentor, Andr\'as Stipsicz, for his perceptiveness and support.

\section{Obstructing positive factorizations}\label{Sec2}
\subsection{Planar open books and contact surgery presentation}
Recall that Legendrian surgeries of Figure \ref{fig:SFS} can be given on a planar page of the associated open book, describing its monodromy, as follows. One $+1$-surgery along an unknot with $\operatorname{tb}=-1$ is presented by an annulus with identity monodromy, the other $+1$-surgery manifests in a negative Dehn twist along its core. Any other unknot contributes a positive Dehn twist. From the starting unknot of each leg we get a twist along a push-off of the core where each positive stabilization corresponds to encircling an additional stabilization hole (hole, encircled by one positive Dehn twist), and each negative stabilization corresponds to avoiding a stabilization hole. The subsequent unknots in each leg are obtained by described modification on a push-off of the previous unknot (corresponding twist).

\begin{notation}[see Figure \ref{fig:view}]
Given expansion $$-\frac{1}{r_i}=a_1^i-\frac{1}{\ddots-\frac{1}{a_{n_i}^i}}=[a_1^i,\ldots,a_{n_i}^i],\ a_j^i\leq 2,$$ denote by $\lambda_j^i, \rho_j^i,$ each of the stabilization holes which are additionally encircled, respectively left out, by the positive Dehn twist corresponding to the $a_j^i$-unknot in comparison to the $a_{j-1}^i$-unknot (core in case $j=1$). Using $|\cdot |$ for the number of respective holes, we see $2+|\lambda_j^i|+|\rho_j^i|=-a_j^i$ ($-a_1^i+1$ when $j=1$), $-1-|\lambda_j^i|-|\rho_j^i|=\operatorname{tb}_j^i$ and $|\lambda_j^i|-|\rho_j^i|=\operatorname{rot}_j^i$. Additionally, write $\rho^{in}$ and $\lambda^{out}$ for the inner and the outer boundary of the annulus. When grouped into certain types, we use $\lambda^i$ for any of $\cup_{j}\lambda_j^i$, and $\lambda$ to denote any of $\lambda^{out}\cup\lambda^i$; and analogously for $\rho$-type holes.
\end{notation}

\begin{figure}
\begin{tikzpicture}
\begin{scope}[shift={(-1,0)}, scale=1.17,decoration={markings, mark=at position 0.73 with {\arrow{<}}}]
\begin{scope}[gray!70]
\draw (0,0) circle (2.55cm);\fill (1.85,-1.85) circle (0pt) node[right, scale=.5]{$\lambda^{out}$};\draw (0,0) circle (0.15cm) node[scale=.5]{$\rho^{in}$};
\draw (.7,.7) circle (0.15cm) node[scale=.5]{$\rho_1^1$}; \draw (-.65,.56) circle (0.15cm) node[scale=.5]{$\rho_1^2$}; \draw (.27,.35) circle (0.15cm) node[scale=.5]{$\rho_2^1$}; 
\draw (-.5,-1.43) circle (.15cm) node[scale=.5]{$\lambda_1^3$};\draw (.5,-1.43) circle (.15cm) node[scale=.5]{$\lambda_1^3$};\draw (1.63,-1) circle (.15cm) node[scale=.5]{$\lambda_2^3$};
\draw (1.91,1.34) circle (.15cm) node[scale=.5]{$\lambda_2^2$};
\end{scope}
\draw (.7,.7) circle (0.18cm); \draw (-.65,.56) circle (0.18cm); \draw (.27,.35) circle (0.18cm); 
\draw[dashed,postaction={decorate}] (0,0) circle (1.3cm); \draw[postaction={decorate}] (0,0) circle (1.71cm); \draw[postaction={decorate}] (0,0) circle (2.12cm); 
\draw[rotate around={42:(.43,.33)},postaction={decorate}] (.43,.33) ellipse (.74cm and .4cm);
\draw[rotate around={33:(.7,.7)},postaction={decorate}] (.75,.6) ellipse (1.5cm and .66cm);
\draw[rotate around={140:(-.4,.33)},postaction={decorate}] (-.4,.33) ellipse (.7cm and .3cm);\draw[rotate around={140:(-.4,.33)},postaction={decorate}] (-.4,.33) ellipse (.73cm and .33cm);
\draw[rotate around={147:(-.17,.3)},postaction={decorate}] (-.17,.3) ellipse (.92cm and .6cm);
\draw (-.5,-1.43) circle (.18cm); \draw (.5,-1.43) circle (.18cm); \draw (1.63,-1) circle (.18cm); \draw (1.91,1.34) circle (.18cm);
\end{scope}

\begin{scope}[shift={(5,0)}, scale=1.17]
\begin{scope}[gray!70]
\draw (.5,0) circle (2.59cm);\fill (-1.57,-1.95) circle (0pt) node[right, scale=.5]{$\lambda_1^3$};\draw (0,0) circle (0.15cm) node[scale=.5]{$\rho^{in}$};
\draw (.7,.7) circle (0.15cm) node[scale=.5]{$\rho_1^1$}; \draw (-.65,.56) circle (0.15cm) node[scale=.5]{$\rho_1^2$}; \draw (.27,.35) circle (0.15cm) node[scale=.5]{$\rho_2^1$}; 
\draw (.5,-1.47) circle (.15cm) node[scale=.5]{$\lambda_1^3$};\draw (1.93,-.7) circle (.15cm) node[scale=.5]{$\lambda_2^3$};
\draw (1.91,1.34) circle (.15cm) node[scale=.5]{$\lambda_2^2$};
\draw (2.5,0) circle (.17cm) node[scale=.5]{$\lambda^{out}$};
\end{scope}
\draw (.5,0) circle (2.55cm);
\draw (.7,.7) circle (0.18cm); \draw (-.65,.56) circle (0.18cm); \draw (.27,.35) circle (0.18cm); 
\draw[dashed] (0,0) circle (1.3cm); 
\draw[rotate around={115:(2.2,.7)}] (2.2,.7) ellipse (1cm and .4cm);
\draw[rotate around={90:(2.1,.33)}] (2.1,.37) ellipse (1.39cm and .73cm);
\draw[rotate around={42:(.43,.33)}] (.43,.33) ellipse (.74cm and .4cm);
\draw[rotate around={33:(.7,.7)}] (.75,.6) ellipse (1.5cm and .66cm);
\draw[rotate around={140:(-.4,.33)}] (-.4,.33) ellipse (.7cm and .3cm);\draw[rotate around={140:(-.4,.33)}] (-.4,.33) ellipse (.73cm and .33cm);
\draw[rotate around={147:(-.17,.3)}] (-.17,.3) ellipse (.92cm and .6cm);
\draw (.5,-1.47) circle (.18cm); \draw (1.93,-.7) circle (.18cm); \draw (1.91,1.34) circle (.18cm);
\end{scope}
\end{tikzpicture}
\caption{Illustration of our notation conventions on an example: $L_1=[-2,-3,-2]\ \text{and}\ \operatorname{rot}^1=(-1,-1,0),\ L_2=[-2,-3]\ \text{and}\ \operatorname{rot}^2=(-1,1),\ L_3=[-3,-3]\ \text{and}\ \operatorname{rot}^3=(2,1).$ In gray are boundary components of the punctured disk, full curves correspond to positive Dehn twists, and dashed curves to negative Dehn twists. The page is shown in two perspectives: with initial outer boundary and with outer boundary in one $\lambda_1^3$-hole.}
\label{fig:view}
\end{figure}

Let us now briefly review the characteristic features of the abelianized planar mapping classes as used by Plamenevskaya and Van Horn-Morris \cite{PVH-M}.

The mapping class group of a planar surface (in the presentation of Margalit and McCammond) is described (geometrically) on a disk, $\mathbf D_n$, with $n$ holes arranged in the roots of unity. It is generated by all convex Dehn twists (that is, the underlying curve is boundary of the convex hull of a set of holes), and quotient out by commutators of disjoint twists and all lantern relations. Then, up to conjugation -- as an element of $\operatorname{AbMap}\mathbf D_n$ -- a Dehn twist is determined by the set of holes it encircles. Furthermore, any monodromy $\phi$ factors into a product of Dehn twists, and each Dehn twist can be using the lantern relations decomposed into only pairwise and boundary Dehn twists (when it encircles $r$ holes, it provides $r-1$ pairwise twists and $r-2$ negative boundary twists, both around each of its holes). Hence, $\phi$ as an element of $\operatorname{AbMap}\mathbf D_n$ is actually uniquely determined by a collection of multiplicities $\{m_\alpha,m_{\alpha\beta}\}$, i.e. the number of twists (counted with signs) on the disks with all but one hole $\alpha$, or a pair of holes $\alpha,\beta$, capped off. Finally, being interested only in positive factorization, the number of its non-boundary twists around any hole is bounded from above by the number of all twists encircling the same hole in any given presentation \cite[Lemma 3.1]{PVH-M}.

\subsection{Necessary condition for tightness, via convex surface theory} 
(For details look at \cite[Section 2]{GLS} or \cite[Section 4]{M}.) Consider convex decomposition of the manifold $M(-1;r_1,r_2,r_3)$ into a product of the pair of pants with the circle $\Sigma\times S^1$ (the background), and neighborhoods of the three singular fibers. Trivialize the cutting tori by the section and the fiber in the background basis, and by the meridian and the longitude in the standard toric basis. The gluing map is described by $\ A_i=$ {\tiny$\begin{pmatrix}     \alpha_i & \alpha_i'   \\  -\beta_i   & \beta_i' \end{pmatrix}$}$\in\operatorname{SL}_2({\mathbb Z})$ where $\frac{\beta_i}{\alpha_i}=r_i\ (r_1-\ 1 \text{ for the first leg})$. In particular, the $\infty$-slope of the neighborhoods of singular fibers correspond to $-\frac{\alpha_i}{\alpha_i'}=[a_{k_i}^i,...,a_1^i]$ in the toric basis. The slopes of the basic slice decomposition of the solid tori (considered neighborhoods) are given -- in order (from outside in) -- by increasing the last entry of this fraction.

\begin{prop}\label{prop:opposite}
Necessarily for tightness, the presentation admits a leg starting in a fully positively stabilized unknot and a leg starting in a fully negatively stabilized unknot.
\end{prop}

\proof
(The proof is essentially the same as its special case \cite[Proposition 2.11]{GLS}.) There are always two legs (say, the first and the second) which admit the same signed basic slice in their first continued fraction block (same-signed stabilization on corresponding starting unknots). After shuffling we may assume these slices to be their outermost slices. Then peeling off these slices from the singular tori and adding them to $\Sigma\times S^1$ we get a circle bundle over the pair of pants with boundary slopes $0,-1,\infty$. Connecting the rulings of the tori of non-$\infty$ slope by an annulus and edge-rounding, we obtain a torus parallel to the cutting torus of the third fiber, and of $0$-slope. The toric annulus between this torus and the $\infty$-slope boundary of the neighborhood of the third fiber, forms a basic slice of the opposite sign as the two chosen above. Pulled-back (to the neighborhood of the third fiber) the slope is $[a^3_{k_3}, . . . , a_2^3]$, which means that the outermost continued fraction block of the third fiber together with this additional basic slice gives a glued-together basic slice (subsequent slopes $[a^3_{k_3}, . . . , a_2^3-1],[a^3_{k_3}, . . . , a_2^3]$), and by Gluing Lemma it is tight exactly when all its subslices have the same sign. In particular, all the signs in the first (outermost) continued fraction block of the third fiber are the same and opposite to the sign of the two initially chosen slices.
\endproof

\subsection{Nonfillability result}
Now, look back at the monodromies as translated from the surgery presentations.

\begin{notation}
Number the first unknot of the $i^\text{th}$ leg whose stabilizations are not all on the same side as for the $a_1^i$-unknot, by $k_i+1$; when the $a_1^i$-unknot admits positive and negative stabilizations, we choose $k_i=0$. Name the corresponding continued fraction by $-\frac{1}{q_i}:=[a_1^i,\dots,a_{k_i}^i]$, or $-\frac{1}{q_i}:=-\infty$ when $k_i=0$.
\end{notation}

\begin{prop}\label{prop:no+factor}
When there is no pair of legs $i$ and $j$ for which $a_1^i$- and $a_1^j$-unknot are stabilized oppositely, and $q_i+q_j\geq 1$, the corresponding monodromy does not admit positive factorization. 
\end{prop}

\proof
We start with the factorization $\Phi$ of the monodromy $\phi$ as being read from the surgery presentation, and we try to build a positive factorization of $\phi$, at least on the level of abelianization. Abusing the notation, we use same names in $\operatorname{AbMap}$, and in fact, throughout the proof we are interested in Dehn twists only up to conjugation.

Without loss of generality (due to Proposition \ref{prop:opposite}), we can assume there is only one leg, say $3^\text{rd}$, whose starting unknot is stabilized fully positively (only down cusps), otherwise we turn our perspective interchanging the outer and the inner boundary of the annulus. (The difference appears in the way we describe the mapping classes, namely on the disk by the holes encircled.) 

We will interchangeably use three perspectives: initial with $\lambda^{out}$ as the outer boundary of the disk, turned-over with $\rho^{in}$ as the outer boundary, and finally, call it $\mathbf D=\mathbf D_{\rho,\lambda}$, the punctured disk obtained by setting one of the $\lambda_1^3$-holes to be the outer boundary. The multiplicities with respect to each viewpoint will be denoted by $m$ for the initial disk, by $m'$ for the turned-over one, and by capital $M$ in $\mathbf D$. Let $\mathbf D$ be our preferred viewpoint if not stated otherwise.

To begin with, let us study how possible positive factorizations behave with respect to $\rho$-holes.

\begin{lem}\label{l1} 
By capping off all $\lambda$-holes, except the outer boundary of $\mathbf D$, we descend from $\operatorname{AbMap} \mathbf D_{\rho,\lambda}$ to $\operatorname{AbMap} \mathbf D_{\rho}$, $\phi\mapsto \overline{\phi}$, which maps the given factorization $\Phi\mapsto\overline\Phi$. This $\overline\Phi$ is a composition $\overline{\Phi}_1\overline{\Phi}_2\overline{\Phi}_3,$ where $\overline{\Phi}_i$ is a product of Dehn twists coming from $a^i$-unknots and boundary twists around $\rho^i$-holes. Every positive factorization $\overline\Psi$ of  $\overline\phi$ splits into subfactorizations $\overline\Psi=\overline{\Psi}_1\overline{\Psi}_2\overline{\Psi}_3$ so that $\overline{\Psi}_i$ and $\overline{\Phi}_i$ describe the same element in $\operatorname{AbMap}\mathbf D_{\rho}$.

Furthermore, for $i=1,2$ write out positive factorizations as $\overline{\Phi}_i=f_1^i\cdots f_{n_i}^i\cdots f_{N_i}^i,$ and $\overline{\Psi}_i=p_1^i\cdots p_{n_i}^i\cdots p_{M_i}^i$, where we order the Dehn twist factors with the ones containing $\rho^{in}$ first and in the decreasing order of the number of holes they include. Then for $K_i:=\min\{k;f_k^i\neq p_k^i\}$, $p_{K_i}^i$ is a strict subset of $f_{K_i}^i$, and neither of $\rho^i$-holes not in $f_{K_i}^i$ are encircled by any non-boundary twist $p_k^i,k\geq K_i$.
\end{lem}

\proof
The $\overline\Phi$ itself presents a positive factorization of restricted monodromy $\overline\phi$ (the only negative twist of $\Phi$ cancels with the boundary twist of the outer $\lambda_1^3$ after capping-off $\lambda$-holes).

Now, set $\rho^{in}$ as the outer boundary and consider the capped-off page in the turned-over perspective. Here, no $\rho^i$-hole is encircled together with any $\rho^j$-hole for $i\neq j$, in symbols $m'_{\rho^i\rho^j}=0$, and the only remaining $\lambda_1^3$ is in at most $n_1+n_2$ twists (number of twists around it in $\overline\Phi$). On the other hand, the pairwise multiplicity of $\lambda_1^3$ with $\rho_1^i$ is exactly $m'_{\lambda_1^3\rho_1^i}=n_i$. So, there are exactly $n_i$ twists encircling $\lambda_1^3$ together with only $\rho^i$-holes. Therefore, since there are no twists containing $\rho^i$ and $\rho^j$ together, all $\rho$-multiplicities are given by $\overline\Phi$, and the twists around $\lambda_1^3$ are distributed so that $\lambda_1^3$-multiplicity is $n_1,n_2$, and $0$, respectively, we can consider the whole (abelianized) monodromy $\overline\phi$ as a product of three monodromies $\overline\phi_i$ as (uniquely) determined by multiplicities. Thus, any positive factorization splits as $\overline{\Psi}_1\overline{\Psi}_2\overline{\Psi}_3$ with $\overline\Psi_i$ describing $\overline\phi_i$.

For the second part, we use the turned-over perspective again. We first notice that $K_i$ always occurs among twists containing $\lambda_1^3$, i.e. $K_i\leq n_i$, as otherwise all pairwise multiplicities are reached, and the factorization agrees with $\overline\Phi$. Now, if $p_{K_i}^i$ did not include some $f_{K_i}^i$-hole $\chi$, the pairwise multiplicity of that hole with $\lambda_1^3$ would be strictly smaller than in $\overline\Phi$ ($m'_{\chi\lambda_1^3}(\overline\Psi)<m'_{\chi\lambda_1^3}(\overline\Phi)$); because the number of twists containing $\lambda_1^3$ is fixed, $n_i$, and $f_k^i,k\geq K_i,$ all contain $\chi$, while $f_k^i,k<K_i,$ contains $\chi$ if and only if $p_k^i$ does. Finally, as (in $\mathbf D$) pairwise $M$-multiplicities of holes out of $f_{K_i}^i$ with any other hole are exactly as many as there are twists from $\{p_k^i=f_k^i;k<K_i\}$ around them, neither can be encircled together with any other hole additionally.
\endproof

This reduces the problem of finding a positive factorization to whether any factorization $\overline{\Psi}$ (maybe $\overline\Phi$) from Lemma \ref{l1} can be lifted to a positive factorization of $\phi\in\operatorname{AbMap} \mathbf D$, i.e. in such a way that all $M$-multiplicities of $\Phi$ are preserved (after possibly adding some twists which do not contain any $\rho$).

In the following, we investigate possible lifts of $\overline\Psi$-twists, in particular, which of the $\lambda$-holes they include.

\begin{lem}\label{l2}
If there exists a positive factorization of $\phi$ lifting $\overline\Psi$, then for $i=1,2,$ the last $n_i-k_i$ twists containing $\rho^{in}$ in $\overline\Psi_i$ (the ones which avoid all $\rho_k^i,k\leq k_i$) lift to the twists which additionally contain only $\lambda^i$-holes.
\end{lem}

\proof
Recall that on the disk with the initial outer boundary all multiplicities $m_{\lambda^i\lambda^j}, i\neq j,$ vanish. On $\mathbf D$ this means that whenever some $\lambda^i$ is encircled together with any of $\lambda^j$, the twist needs to contain also the initial outer boundary, the hole $\lambda^{out}$. But the $\rho^{in}\lambda^1$- and $\rho^{in}\lambda^2$-multiplicities are greater than $M_{\rho^{in}\lambda^{out}}=1$, for $\lambda_j^i$-hole the multiplicity is exactly $M_{\rho^{in}\lambda_j^i}=n_i-j+2$. Thus (at least) $n_i-k_i$ Dehn twists which contain $\rho^{in}$ need to lift into twists which include only $\lambda^i$-type $\lambda$-holes. Moreover, as $m_{\rho_k^i\lambda^i}=0$ for $k\leq k_i$, whenever such $\rho_k^i$ is encircled together with $\lambda^i$, the twist contains also $\lambda^{out}$ -- hence the $n_i-k_i$ twists mentioned above are the last $n_i-k_i$ twists from $\overline\Psi_i$ which contain $\rho^{in}$ (and avoid all $\rho_k^i,k\leq k_i$).
\endproof

\begin{remark}\label{r2} 
Considering $m'$-multiplicities in the turned-over perspective, the same (with interchanged role of $\lambda$- and $\rho$-holes) can be concluded for the $n_3-k_3$ twists containing $\lambda^{out}$ and avoiding $\lambda_k^3,k\leq k_3$.
\end{remark}

Let us list some properties of encircling $\lambda^{out}\cup\lambda^3$-holes (viewed in $\mathbf D$):
\begin{enumerate}[leftmargin=.6cm]
\item\label{p1} Their pairwise multiplicity with any of $\rho^{in}\cup\rho^1\cup\rho^2$ is one (and with any of $\rho^3$ one enlarged by the number of twists mentioned in Remark \ref{r2}). 
\item\label{p2} Each of $\lambda_j^3$-holes is encircled by at most $j+1$ non-boundary Dehn twists, $\lambda^{out}$ by at most $n_3+1$ (number of twists around each of them in $\Phi$); for $j>k_3$, $j-k_3$ of them are described by Remark \ref{r2}. 
\item\label{p3} Pairwise multiplicity of each $\lambda_j^3$ with any $\lambda_{\geq j}^3$ is exactly $j$.
\item\label{p4} According to Lemma \ref{l2}, lifts of the twists $p_{k_i+1}^i,\dots,p_{n_i}^i$ for $i=1,2,$ never encircle any of $\lambda^{out}\cup\lambda^3$; denote $\overline\Psi^*:=\overline\Psi\backslash\{p_{k_i+1}^i,\dots,p_{n_i}^i;i=1,2\}$.
\end{enumerate}

So, since we need to enclose each $\lambda^{out}\cup\lambda^3$ with all of $\rho$-holes once (\ref{p1}) and by the bounded number of twists (\ref{p2}), we look (in every factorization $\overline\Psi$) for partitions of $\rho$-holes by the $\overline\Psi^*$-twists, which have appropriate number of parts. If two sets of $\overline\Psi^*$-twists define set-wise the same partition, we say they are parallel as the twists of the two sets need to be parallel (or equal), the equal twists are referred to as shared.

Let us proceed successively, focusing on $\lambda_j^3$ for every $j$ in $1,2,\dots,n_3+1$, here we denote $\lambda_{n_3+1}^3:=\lambda^{out}$. We say that $\overline\Psi$ lifts over $\lambda_{\leq j}^3$, if $\overline\Psi^*$-twists can be lifted to a positive factorization in $\operatorname{AbMap} \mathbf D_{\rho\cup\lambda_{\leq j}^3}$ which satisfies the listed properties.

Recursively define $J_0=0, J_l:=\min\{j;j>J_{l-1},|\lambda_j^3|\geq1\ (|\lambda_1^3|>1\text{ for }j=1)\}.$

\begin{lem}\label{l3}
If $\overline\Psi$ lifts over $\lambda^{out}\cup\lambda^3$-holes, the $\overline\Psi^*$-twists whose lifts encircle $\lambda^{out}\cup\lambda^3$-holes, can all (apart from the ones from Remark \ref{r2}) be chosen from a single $\overline\Psi_i^*, i=1$ or $2$. Furthermore, if any factorization $\overline\Psi_i$ lifts over $\lambda_{\leq j}^3$, so does $\overline\Phi_i$.
\end{lem}

\proof
In order to fulfill the first property (\ref{p1}) that twists partition $\rho$-holes, we need a twist which contains $\rho^{in}$. So, every partition defined by $\overline\Psi^*$-twists consists of some $p_K^I\in\{p_k^i;k\leq k_i,i\in\{1,2\}\}$ and some twists covering all $\rho^I$-holes which are not in $p_K^I$. Now, if there is a partition of less than $J_1+1$ parts, we can extend its defining twists over all $\lambda^{out}\cup\lambda^3$. This choice satisfies the second (\ref{p2}) and the third (\ref{p3}) property (when completed by some twists which do not contain any $\rho$-holes), and the lifted twists obviously come from a single $\overline\Psi_I^*$. If all partitions have more than $J_1+1$ parts, the second (\ref{p2}) property can never be satisfied and there is no positive factorization. Finally, if there is a partition of exactly $J_1+1$ parts, $J_1$ of them are necessarily shared by all $\lambda^{out}\cup\lambda^3$, to fulfill the third (\ref{p3}) property. Since around each hole there can be only one twist which does not contain $\rho^{in}$, the twists other than $p_K^I$ are always shared and partitions at the following levels can only arise from splitting the $p_K^I$-part, which is possible only by $\overline\Psi_I^*$-twists.

Suppose now we are lifting $\overline\Psi_i\neq\overline\Phi_i$. At each level we are looking for partitions with the least possible parts. As long as the twists used in $\overline\Psi_i^*$-partitions agree with some $\overline\Phi_i^*$-twists, the two factorizations lift simultaneously. Otherwise, as soon as we need the largest (as a set) $p_k^i\neq f_k^i, k\leq n_i,$ the second part of Lemma \ref{l1} tells that $\overline\Phi_i^*$ admits at least one more partition of at least one less part. Since by assumption $\overline\Psi_i$ lifts over $\lambda_{\leq j}^3$, this $\overline\Phi_i^*$-partition has less than $j+1$ parts, and can be used for all $\lambda_{\geq j}^3$, fulfilling the properties.
\endproof

\begin{remark}\label{r3}
The concluding statement in Lemma \ref{l3} essentially means that we can focus only on $\overline\Phi$ as the most liftable among $\overline\phi$-factorizations, when looking for obstructions of positive factorization. Moreover, if we number the legs so that $-\frac{1}{q_1}>-\frac{1}{q_2}$, the $\overline\Psi_i$ in Lemma \ref{l3} can be $\overline\Phi_1$ (it lifts whenever any of $\overline\Psi_i$ lifts).
\end{remark}

\begin{lem}\label{l4} At the $l^\text{th}$ level when $J_l\leq k_3$:
\begin{enumerate}[leftmargin=.7cm]
  \item[(i)] If there is a (not-yet-used) $\overline\Phi_1^*$-partition into less than $J_l+1$ parts, the assumptions of the proposition are not satisfied. 
  \item[(ii)] If there is no (not-yet-used) $\overline\Phi_1^*$-partition into less than $J_l+1$ parts, and there are less than $|\lambda_{J_l}^3|$ ($|\lambda_1^3|-1$ for $J_1=1$) of parallel $\overline\Phi_1^*$-partitions into $J_l+1$ parts, there is no positive factorization of $\phi$.
  \item[(iii)] Otherwise, we proceed to the next level.
\end{enumerate}
\end{lem}

\proof
Suppose that $\overline\Phi_1^*$ falls under (iii) for all levels up to $l^\text{th}$.  At the $l^\text{th}$ level, if there are only partitions of more than $J_l+1$ twists or there are less than $|\lambda_{J_l}^3|$ of $J_l+1$-part partitions, there is no positive factorization; because we cannot satisfy the first (\ref{p1}) and the second (\ref{p2}) listed property simultaneously. On the other hand, when we can partition $\rho$-holes into less than $J_l+1$ parts, the structure does not fulfill our assumptions. Indeed, in Remark \ref{rmk:dual} we write out how the coefficients of two legs need to be related in order for corresponding rational numbers to add up to one. In our case, $J_l$'s are the unknots on $L_3$ with surgery coefficient less than $-2$, $J_l-J_{l-1}$ counts number of parallel twists, which is one more than number of $-2$'s preceding $-a_{J_l}^3$. For the levels up to $l^\text{th}$, the conditions of (iii) mean the number of left-out holes exactly agrees with $J_l-J_{l-1}$, corresponding to an unknot of coefficient $-J_l+J_{l-1}-2$, which is followed by exactly $|\lambda_{J_l}^3|$ of $-2$'s on $L_1$. But condition (i) at the $l^\text{th}$ level quit this sequence, having $J_l-J_{l-1}$ parallel twists ($J_l-J_{l-1}-1$ of $-2$'s on $L_3$) but leaving out less than $J_l-J_{l-1}$ holes by the next $a^{1}$-unknot (its coefficient being at least $-J_l+J_{l-1}-1$). Since $J_l\leq k_3$ and we are considering only $\overline\Phi_1^*$-partitions, the two truncated chains correspond to rational numbers smaller than or equal to $-\frac{1}{q_i}$. 
\endproof

 The process eventually stops as we run into an obstruction for positive factorization (ii) or we leave the assumed conditions (i). If not before when we cross the $k_3$-level (possibly $k_3=n_3$), as over that holes we are not allowed to extend more than $k_3+1$ twists -- so, the only possible positive factorizations would arise from partitions into $k_3+1<J_l+1$ twists, but then as always assumptions of the proposition are not satisfied.
\endproof

\section{Surgery links of tight $S^1\times S^2$}\label{Sec3}
\begin{lem}
Whenever $r_i+r_j\geq 1$, there is a sublink $L_{ij}$ of the surgery presentation as in Theorem \ref{thm}, surgery along it smoothly describes $S^1\times S^2$.
\end{lem}

\proof
By basic calculus of continued fractions there exist truncated continued fractions $-\frac{1}{s_i}=[a_1^i,\dots,a_{m_i}^i]<[a_1^i,\dots,a_{n_i}^i]=-\frac{1}{r_i}\ (m_i\leq n_i)$ such that $s_i+s_j=1$ (see \cite[Lemma 3.2]{LL}).

The framed link $L_{ij}$ smoothly consists of four $-1$-linked unknots with framing coefficients $0,0,-\frac{s_i+1}{s_i},-\frac{s_j+1}{s_j}$. Blowing-up once and applying inverse slam-dunks to rationally framed unknots, we obtain a chain of unknots with coefficients $[a_{m_i}^i,\dots,a_1^i,-1,a_1^j,\dots,a_{m_j}^j]$, which can be successively, starting with the middle $-1$-surgery, blown-down ending in a $0$-framed unknot.
\endproof

\begin{remark}\label{rmk:dual}
Notice that the two chains forming the two legs of $L_{ij}$ need to be dual to each other (i.e., describing a lens space and its orientation reversal). Explicitly, the coefficients of the two are related as follows: $$\begin{array}{llccccc}L_i&=&[-b_1-2,&-2^{\times b_2},& -b_3-3,&\dots,&-b_m-2]\\ L_j&=&[-2^{\times b_1},&-b_2-3,&-2^{\times b_3},&\dots,& -2^{\times b_m}]\end{array}.$$ When looked in the presentation of Figure \ref{fig:SFS}, the first unknots of both chains are framed one lower, so $-b_1-3$ and $-3$ respectively.
\end{remark}

\begin{prop}\label{prop:+factor}
Contact surgery presentation by Legendrian link $L_{ij}$ corresponds to the tight $S^1\times S^2$ if and only if all stabilizations on one leg are positive and all stabilizations on the other leg are negative.
\end{prop}

\proof
Necessity of the condition is a special case of Proposition \ref{prop:no+factor}. We prove here that it is also sufficient, describing concrete factorization.

Considering Legendrian link $L_{ij}$ with all stabilizations on the $i^{\text{th}}$ leg $L_i=[a_1^i,\dots,a_{m_i}^i]$ positive and all stabilizations on the $j^{\text{th}}$ leg $L_j=[a_1^j,\dots,a_{m_j}^j]$ negative, all Dehn twists corresponding to $a^i$-unknots lie outside core circle (the negative Dehn twist) and the ones from $a^j$-unknots lie inside. We can rewrite this monodromy by iterative use of the lantern relation as follows (look also at the example given by Figure \ref{fig:+factor}).

\begin{figure}
\begin{tikzpicture}
\begin{scope}[scale=.85]
\begin{scope}[shift={(-1,0)}]
\begin{scope}[gray!70]
\draw[thick] (0,0) circle (1.9cm);\fill (0,0) circle (.07cm);
\fill (.5,.5) circle (.07cm); \fill (.5,-.3) circle (.07cm); \fill (-.5,-1.25) circle (.07cm); \fill (1.3,0) circle (.07cm); \fill (1.6,-.5) circle (.07cm);
\end{scope}
\draw[dashed] (0,0) circle (1.1cm); \draw (0,0) circle (1.53cm); \draw (0,0) circle (1.85cm); 
\draw[rotate around={-35:(.23,-.13)}] (.23,-.13) ellipse (.5cm and .3cm);
\draw (0,0) circle (.1cm); \draw (0,0) circle (.14cm);
\draw (.5,.5) circle (.1cm); \draw (.5,-.3) circle (.1cm); \draw (-.5,-1.25) circle (.1cm); \draw (1.3,0) circle (.1cm); \draw (1.6,-.5) circle (.1cm);
\end{scope}

\begin{scope}[shift={(3,0)}]
\begin{scope}[gray!70]
\fill (1.8,-.7) circle (.07cm);\fill (0,0) circle (.07cm);
\fill (.5,.5) circle (.07cm); \fill (.5,-.3) circle (.07cm); \fill (-.5,-1.25) circle (.07cm); \fill (1.3,0) circle (.07cm); \fill (1.6,-.5) circle (.07cm);
\end{scope}
\draw[dashed] (0.6,0) circle (.9cm); \draw (0,0) circle (1.53cm); \draw (1.8,-.7) circle (.1cm); 
\draw[rotate around={-30:(.73,.43)}] (.91,.37) ellipse (.65cm and .3cm);
\draw (.6,-.05) ellipse (.8cm and .4cm);
\draw (0,0) circle (.1cm); \draw (0,0) circle (.14cm);
\draw (.5,-.3) circle (.1cm); \draw (-.5,-1.25) circle (.1cm); \draw (1.6,-.5) circle (.1cm);
\end{scope}

\begin{scope}[shift={(6,0)}]
\begin{scope}[gray!70]
\fill (1.8,-.7) circle (.07cm);\fill (0,0) circle (.07cm);
\fill (.7,.44) circle (.07cm); \fill (.7,-.3) circle (.07cm); \fill (-.1,-1.25) circle (.07cm); \fill (1.3,0) circle (.07cm); \fill (1.6,-.5) circle (.07cm);
\end{scope}
\draw (.88,.1) circle (.53cm); \draw[rotate around={-45:(1.2,-.1)}] (1.27,-.1) ellipse (.95cm and .25cm);
\draw[rotate around={-20:(1.2,-.4)}] (1.2,-.47) ellipse (.85cm and .35cm);
\draw[dashed] (0.6,0) circle (.9cm); \draw[dashed] (1.2,-.2) circle (1cm);  \draw (1.8,-.7) circle (.1cm); 
\draw (.6,-.05) ellipse (.8cm and .4cm);
\draw (0,0) circle (.1cm); \draw (0,0) circle (.14cm);
\draw (-.1,-1.25) circle (.1cm); \draw (1.6,-.5) circle (.1cm);
\end{scope}

\begin{scope}[shift={(9.5,0)}]
\begin{scope}[gray!70]
\fill (1.8,-.7) circle (.07cm);\fill (0,0) circle (.07cm);
\fill (.7,.44) circle (.07cm); \fill (.7,-.3) circle (.07cm); \fill (1.3,0) circle (.07cm); \fill (1.5,-.5) circle (.07cm);
\draw[thick] (.7,0) circle (1.57cm);
\end{scope} 
\draw[rotate around={-45:(1.2,-.1)}] (1.27,-.1) ellipse (.95cm and .25cm);
\draw[rotate around={-20:(1.2,-.4)}] (1.2,-.47) ellipse (.85cm and .35cm);
\draw (.6,-.05) ellipse (.8cm and .4cm);
\draw plot [smooth cycle] coordinates {(-.11,.05)(.5,.7)(.9,.9)(1.3,.9)(1.5,.5)(1.9,-.7)(1.7,-.7)(1.3,.7)(.9,.8)(.5,.6)(.11,-.05)};
\draw plot [smooth cycle] coordinates {(-.13,.03)(.5,-.7)(1.1,-.9)(1.62,-.5)(1.4,-.43)(1.1,-.7)(.5,-.6)(.11,.03)};
\end{scope}
\end{scope}
\end{tikzpicture}
\caption{Example of positive factorization: $L_i=[-3,-3]$ and $L_j=[-2,-3,-2]$. On the first and the last picture the page is presented as a punctured disk with outer boundary in $\lambda^{out}$ and one of $\lambda_1^i$, respectively. Intermediate steps are presented as punctured spheres.}
\label{fig:+factor}
\end{figure}

In the following we use $b$-notation in the sense of Remark \ref{rmk:dual}. 

One of the two legs, say $L_j$, starts in $-2$'s, say $b_1$ of them. In the first step we consider the associated $b_1$ parallel Dehn twists, a Dehn twist around the hole $\rho_1^j$ responsible for stabilization of the first unknot in this chain, and $b_1$ stabilizations $\lambda_1^i$ of $a_1^i$-unknot (one less than all if we have not reached the end of $L_i$). We apply daisy  relation on them (i.e. repeated lantern as in \cite[Lemma 3.5]{PVH-M}). This pushes the negative twist, from now on call it $N$, over $b_1$ stabilizations ($\lambda_1^i$-holes) of $a_1^i$, and gives additional positive Dehn twist $D_1$ around all considered stabilization holes (the one $\rho_1^j$ from $a_1^j$ and $b_1$ of $\lambda_1^i$ from $a_1^i$). 

From now on, imagine the remaining $\lambda_1^i$-hole as an outer boundary. Now we take positive Dehn twist coming from $a_1^i$ and all of its parallel push-offs, there are $b_2+1$ of them where $b_2$ is the number of $-2$'s following $a_1^i$ on $L_i$. Further, we take positive Dehn twist $D_1$ and all $b_2+1$ stabilization holes $\rho_{b_1+1}^j$ of $a_{b_1+1}^j$. We apply daisy relation as before, resulting in a new negative twist $N'$ around all considered holes, and a positive twist $D_2$ around considered holes which are not initially encircled by $a_1^i$. Concretely, the twist $N'$ goes around the initial outer boundary $\lambda^{out}$, all stabilization holes $\lambda^i$ of $L_i$, and the first two levels stabilization holes ($\rho_1^j\cup\rho_{b_1+1}^j$) of $L_j$, while $D_2$ enlarges $D_1$ over the second level stabilization holes of $L_j$ ($\rho_1^j\cup\lambda_1^i\cup\rho_{b_1+1}^j$).

We continue by interchangeably applying daisy relation from inside (involve some $a_k^j$) and from outside (involve some $a_k^i$), interchangeably ``pushing'' the two negative twists $N$ and $N'$ over always the next level of $L_i$- or $L_j$-holes, respectively. (For the negative twist which arises through a single application of daisy relation we use the name of the negative twist which has been canceled through the same process.) At the same time, each application of daisy relation ``enlarges'' $D_{l-1}$ into $D_l$, additionally encircling the next level of $L_i$- (for $l$ odd) or $L_j$-holes (when $l$ even). After the $l^\text{th}$ application of daisy relation, the twists contain:
$$\begin{array}{ll}
 l=2l'+1:  &   D_l=\{\rho_1^j\cup\lambda_1^i\cup\cdots\cup\rho_{\sum_{k=1}^{l'}b_{2k-1}+l'}^j\cup\lambda_{\sum_{k=1}^{l'}b_{2k}+l'+1}^i\}  \\ & N=\{\rho^{in}\cup\rho^j\cup\lambda_1^i\cup\cdots\cup\lambda_{\sum_{k=1}^{l'}b_{2k}+l'+1}^i\} \\  & N'=N' \text{ after }(l-1)^\text{th}\text{ step} \\ & \\
 l=2l':     &   D_l=\{\rho_1^j\cup\lambda_1^i\cup\cdots\cup\lambda_{\sum_{k=1}^{l'-1}b_{2k}+l'}^i\cup\rho_{\sum_{k=1}^{l'}b_{2k-1}+l'}^j\}  \\ & N'=\{\lambda^{out}\cup\lambda^i\cup\rho_1^j\cdots\cup\rho_{\sum_{k=1}^{l'}b_{2k-1}+l'}^j\} \\  & N=N \text{ after }(l-1)^\text{th}\text{ step} \\
\end{array}$$

In the last level there is one less stabilization hole of the $a^j$- ($a^i$-) unknot in comparison to the number of parallel twists from $-2$'s ending $L_i$ ($L_j$); when applying daisy relation we include also the initial outer (inner) boundary hole. So after the last step, $D_m$ contains  also the initial outer (inner) boundary and it cancels with the negative twist $N'$ ($N$). While the other negative twist, $N, N'$ respectively, encircles all the holes and it cancels with the positive Dehn twist along the current outer boundary.
\endproof

\proof[Proof of Theorem \ref{thm}]
Joining Proposition \ref{prop:no+factor} and Proposition \ref{prop:+factor} we obtain the theorem. Indeed, Legendrian surgeries on tight $S^1\times S^2$ (from Proposition \ref{prop:+factor}) give Stein fillable structures, while all other presentations fall under the conditions of Proposition \ref{prop:no+factor}, thus they do not admit positive factorization of associated planar monodromy, and by that, do not admit any Stein filling.
\endproof

\begin{thebibliography}{99}

\bibitem{G}
P. Ghiggini, {\em On tight contact structures with negative maximal twisting number on small Seifert manifolds}, Algebr. Geom. Topol. \textbf{8} (2008) 381--396.

\bibitem{GLS0}
P. Ghiggini, P. Lisca, and A. Stipsicz, {\em Classification of tight contact structures on small Seifert 3-manifolds with $e_0\geq 0$}, Proc. Amer. Math. Soc. \textbf{134} (2006) 909--916.

\bibitem{GLS}
P. Ghiggini, P. Lisca, and A. Stipsicz, {\em Tight contact structures on some small Seifert fibered 3-manifolds}, Amer. J. Math. \textbf{129}(\textbf{5}) (2007) 1403--1447.

\bibitem{LL}
A. G. Lecuona, and P. Lisca, {\em Stein fillable Seifert fibered 3–manifolds}, Algebr. Geom. Topol. \textbf{11} (2011) 625--642.

\bibitem{LS.III}
P. Lisca, and A. Stipsicz, {\em Ozsváth-Szabó invariants and tight contact 3-manifolds III}, J. Symplectic Geom. \textbf{5}(\textbf{4}) (2007) 357--384.

\bibitem{M}
I. Matkovi\v{c}, {\em Classification of tight contact structures on small Seifert fibered $L$-spaces}, arXiv:1601.05059 [math.GT].

\bibitem{PVH-M}
O. Plamenevskaya, and J. Van Horn-Morris, {\em Planar open books, monodromy factorizations and symplectic fillings}, Geom. Topol. \textbf{14} (2010) 2077--2101.

\bibitem{S}
A. Stipsicz, {\em Ozsváth-Szabó invariants and 3-dimensional contact topology}, Proceedings of the International Congress of Mathematicians, Hyderabad 2010, vol II, 1159--1178.

\bibitem{W}
C. Wendl, {\em Strongly fillable contact manifolds and $J$-holomorphic foliations}, Duke Math. J. \textbf{151} (2010) 337--384.

\bibitem{Wu}
H. Wu, {\em Legendrian vertical circles in small Seifert spaces}, Commun. Contemp. Math. \textbf{8} (2006) 219--246.

\end{thebibliography}

\end{document}

\documentclass[10pt, reqno]{amsart}
\numberwithin{equation}{section}

\usepackage{amssymb,latexsym}
\usepackage{eucal}
\usepackage{tikz}

\newtheorem{thm}{Theorem}[section]
\newtheorem{cor}[thm]{Corollary}
\newtheorem{lem}[thm]{Lemma}

\newtheorem*{TA}{Theorem A}
\newtheorem*{TB}{Theorem B}
\newtheorem*{TC}{Theorem C}

\newtheorem*{T1}{Theorem 1.1}
\newtheorem*{T2}{Theorem 1.2}
\newtheorem*{T3}{Theorem 1.3}

\begin{document}

\title
[Linear programming and intersection of subgroups in free products]
{Linear programming and the intersection of free subgroups in free products of groups}
\author{S. V. Ivanov }
 \address{  Department of Mathematics\\
 University of Illinois \\
 Urbana\\  IL 61801\\ U.S.A. } \email{ivanov@illinois.edu}
\thanks{Supported in part by  the NSF under  grant  DMS 09-01782.}
\keywords{Free products of groups, free and factor-free subgroups, rank of intersection of factor-free subgroups, linear programming.}
\subjclass[2010]{Primary 20E06, 20E07, 20F65, 90C90.}

\begin{abstract}
We study the intersection of finitely generated factor-free subgroups of free products of  groups by utilizing the method of linear programming.  For example, we prove that if $H_1$ is a finitely generated
factor-free noncyclic subgroup of the free product $G_1 * G_2$   of two finite groups $G_1$, $G_2$, then the Walter Neumann  coefficient $\sigma(H_1)$ of $H_1$ is rational and can be computed. This coefficient $\sigma(H_1)$ is the minimal positive real number such that, for every  finitely generated
factor-free subgroup $H_2$ of  $G_1 * G_2$, it is true that
$\bar {\mathrm{r}}(H_1, H_2)  \le  \sigma(H_1) \bar {\mathrm{r}}(H_1) \bar {\mathrm{r}}(H_2)$, where $\bar{ {\rm r}} (H) = \max ( {\rm r} (H)-1,0)$ is reduced rank of $H$, ${\mathrm{r}} (H)$ is rank of $H$, and  $\bar {\mathrm{r}}(H_1, H_2)$ is reduced rank of a generalized intersection of $H_1, H_2$.
\end{abstract}
\maketitle

\section{Introduction}

Let $G_{\alpha}$, ${\alpha} \in I$, be some nontrivial groups and let
 ${\mathcal{F}} = \prod_{\alpha \in I}^* G_\alpha$ denote the free product of these groups. According to the classic Kurosh subgroup
theorem  \cite{K}, \cite{LS}, every subgroup $H$ of ${\mathcal{F}}$ is a free product $ F(H) * \prod^*  t_{{{\alpha},\gamma}} H_{{\alpha},\gamma} t_{{{\alpha},\gamma}}^{-1}$, where $H_{{{\alpha},\gamma}}$ is a subgroup of $G_{\alpha}$, $t_{{{\alpha},\gamma}} \in
{\mathcal{F}}$, and $F(H)$ is a free subgroup of
${\mathcal{F}}$  such that, for every $s \in {\mathcal{F}}$
and $\gamma \in I$,  it is true that $F(H) \cap
s G_\gamma s^{-1} =\{ 1 \}$. We say that $H$
is a {\em factor-free} subgroup of  ${\mathcal{F}}$  if $H=F(H)$ in the above form of $H$, i.e., for every $s \in
{\mathcal{F}}$ and $\gamma \in I$, we have $H \cap s
G_\gamma s^{-1} =\{ 1 \}$. Let ${\rm r} (F)$ denote the rank of a (finitely generated)
 free group $F$.   Since a factor-free subgroup $H$ of
${\mathcal{F}}$  is free,  the  reduced
rank $\bar{ {\rm r}} (H) := \max ( {\rm r} (H)-1,0)$ of $H$, where ${\rm r} (H)$ is the rank of $H$,  is well defined.

Let $q^*= q^*(G_{\alpha}, {\alpha} \in I)$ denote the minimum of orders $>2$ of finite subgroups of groups $G_{\alpha}$, ${\alpha}
\in I$, and  let $q^* := \infty$ if there are no such subgroups.  It is clear that either $q^*$ is an odd prime or $q^* \in \{ 4,  \infty \}$.
If  $q^* = \infty$, define
$ \tfrac{q^*}{q^*-2} := 1$.
Dicks and the author \cite{DIv} proved that if
$H_1$, $H_2$ are finitely generated factor-free subgroups of  ${\mathcal{F}}$,  then
\begin{equation}\label{di}
\bar {\mathrm{r}}(H_1\cap H_2)  \le  2\tfrac{q^*}{q^*-2}
 \bar {\mathrm{r}}(H_1) \bar {\mathrm{r}}(H_2)  .
\end{equation}
Dicks and the author \cite{DIv} conjectured that if groups $G_{\alpha}$, ${\alpha} \in I$,  contain no involutions, then the coefficient 2 could be left out  and
 \begin{equation}\label{conj}
 \bar {\mathrm{r}}(H_1\cap H_2)  \le  \tfrac{q^*}{q^*-2}
 \bar {\mathrm{r}}(H_1) \bar {\mathrm{r}}(H_2)  .
 \end{equation}
This conjecture  can be regarded as a far reaching generalization of the Hanna Neumann conjecture  \cite{N1} on rank of the intersection of subgroups in free groups.  Recall that the  Hanna Neumann conjecture  \cite{N1}   claims that  if  $H_1$, $H_2$
are finitely generated subgroups of a free group, then $\bar { \rm{r} } (H_1 \cap H_2) \le \bar{ {\rm r}} (H_1) \bar { \rm r}  (H_2)$.
For more discussion, partial results and proofs of this conjecture the reader is referred to  \cite{D}, \cite{D2}, \cite{Fr}, \cite{Iv12}, \cite{Min},  \cite{N2}, \cite{St}, \cite{T}.

The conjecture \eqref{conj} is established by Dicks and the author \cite{DIv2} in the case when ${\mathcal{F}}$  is the free product of two groups of order 3 in which case $q^* = 3$ and \eqref{conj} turns into
$$
\bar {\mathrm{r}}(H_1\cap H_2)  \le  \tfrac{q^*}{q^*-2}  \bar {\mathrm{r}}(H_1) \bar {\mathrm{r}}(H_2) = 3  \bar {\mathrm{r}}(H_1) \bar {\mathrm{r}}(H_2) .
$$
Another special case in which the conjecture \eqref{conj} is known to be true is the case when ${\mathcal{F}}$ is the free product of infinite cyclic groups, i.e., ${\mathcal{F}}$ is a free group, as follows from Friedman's \cite{Fr}  and Mineyev's \cite{Min} proofs  of the Hanna Neumann conjecture, see also Dicks's proof \cite{D2}.  In this case  $q^* =\infty$ and the inequality \eqref{conj} turns into
\begin{equation}\label{HNC}
\bar {\mathrm{r}}(H_1\cap H_2)  \le   \bar {\mathrm{r}}(H_1) \bar {\mathrm{r}}(H_2) .
 \end{equation}
More generally, the inequality \eqref{HNC}  also holds in the case when ${\mathcal{F}}$ is the free product  of right  orderable groups as follows from results of  Antol\'in, Martino, and Schwabrow \cite{ABC}, see also \cite{Iv12}.
We mention that it follows from results of \cite{DIv}  that the conjectured inequality  \eqref{conj}  is sharp and may not be improved.

In an attempt to improve on the bound  \eqref{di} in a special  case, Dicks and the author  \cite{DIv2}  showed  that
   \begin{equation}\label{imrn}
  \bar {\mathrm{r}}(H_1\cap H_2)  \le   \left(2 - \tfrac{(4+2\sqrt{3})p}{(2p-3+\sqrt{3})^2} \right) \cdot \tfrac{p}{p-2}  \bar {\mathrm{r}}(H_1) \bar {\mathrm{r}}(H_2)
 \end{equation}
for finitely generated factor-free subgroups $H_1$, $H_2$ of the free product $C_p * C_p$ of two cyclic  groups of prime order $p >2$.

Note that for $p=3$ the inequality  \eqref{imrn}  yields the conjectured inequality  \eqref{conj}.  However,  for prime $p \ge 5$, the problem whether the inequality \eqref{conj} holds for  the free product of two cyclic groups of order $p$ remains open and seems to be the most basic and appealing case of the conjecture \eqref{conj}  for groups with torsion. In this connection, we remark that the ideas of articles \cite{ABC},  \cite{D2},  \cite{Fr},  \cite{Min}, do not look to be applicable to the case of free products with torsion and shed no light on the conjecture \eqref{conj} for free products of groups with torsion, especially, for free products of finite groups.
\medskip

In this article, however, we will not attempt to prove or improve on any upper bounds. Instead, we will look at generalized intersections of finitely generated factor-free subgroups in
free products of groups from a disparate standpoint and prove results of quite different flavor by utilizing the method of linear programming.

First we recall a stronger version of the conjecture \eqref{conj} that generalizes the  strengthened Hanna Neumann
conjecture which was  put forward by
Walter Neumann  \cite{N2}  for subgroups of free groups.  Let  $H_1$, $H_2$ be finitely generated factor-free subgroups of an arbitrary free product ${\mathcal{F}} = \prod_{\alpha \in I}^* G_\alpha$ of groups $G_\alpha$, $\alpha \in I$, let  the number $\tfrac{q^*}{q^*-2}$ be defined for ${\mathcal{F}}$ as above,   and let $S(H_1, H_2)$ denote a set of  representatives of those double cosets $H_1 t H_2$ of ${\mathcal{F}}$, $t \in {\mathcal{F}}$,   that have the property $H_1 \cap t H_2 t^{-1} \ne  \{ 1 \}$.  Then the  strengthened
version of  the conjecture \eqref{conj} claims that
\begin{equation}\label{conjs}
\bar {\mathrm{r}}(H_1, H_2) :=   \sum_{s \in S(H_1, H_2)} \bar {\mathrm{r}}(H_1\cap s H_2 s^{-1})
\le \tfrac{q^*}{q^*-2}   \bar {\mathrm{r}}(H_1) \bar {\mathrm{r}}(H_2)  ,
\end{equation}
where $\bar {\mathrm{r}}(H_1, H_2) $ is reduced rank of the generalized intersection of $H_1$ and $H_2$ consisting of subgroups $H_1\cap s H_2 s^{-1} $, $s \in S(H_1, H_2)$.

Let ${ \mathcal{K}_{\textsf{ff}}(\mathcal{F})}$  denote the set of all finitely generated noncyclic factor-free subgroups of the free product  ${\mathcal{F}}$.
Pick a subgroup $H_1 \in { \mathcal{K}_{\textsf{ff}}(\mathcal{F})}$.  We will say that a real number $\sigma(H_1) >0 $ is the {\em Walter Neumann coefficient} for   $H_1$, or,  briefly, the WN-coefficient for $H_1$,   if, for every  $H_2 \in { \mathcal{K}_{\textsf{ff}}(\mathcal{F})}$, we have
\begin{equation}\label{dfy0}
{\bar {\mathrm{r}}} (H_1 , H_2) \le \sigma(H_1)  {\bar {\mathrm{r}}}(H_1) {\bar {\mathrm{r}}}(H_2)
\end{equation}
and  $\sigma(H_1)$ is minimal  with this property. Clearly,
$$
\sigma(H_1)  = \sup \{ \tfrac{{\bar {\mathrm{r}}} (H_1,  H_2)}{{\bar {\mathrm{r}}} (H_1) {\bar {\mathrm{r}}} (H_2) } \}
$$ over all  subgroups $H_2 \in { \mathcal{K}_{\textsf{ff}}(\mathcal{F})}$.

For every integer $d \ge 3$, we also define the number
\begin{equation}\label{dfy0d}
\sigma_d(H_1)  := \sup \{ \tfrac{{\bar {\mathrm{r}}} (H_1 ,  H_2)}{{\bar {\mathrm{r}}} (H_1) {\bar {\mathrm{r}}} (H_2) } \}
\end{equation}
over all  subgroups $H_2 \in { \mathcal{K}_{\textsf{ff}}(\mathcal{F}, d)}$, where ${ \mathcal{K}_{\textsf{ff}}(\mathcal{F}, d)}$ is a subset of $ { \mathcal{K}_{\textsf{ff}}(\mathcal{F})} $ consisting of those subgroups
whose  irreducible core graphs have all of its vertices of degree at most $d$, see Sect.~2 for definitions.
This number  $\sigma_d(H_1)$ is called the  WN${}_d$-{\em coefficient} for    $H_1$.
 Since  ${ \mathcal{K}_{\textsf{ff}}(\mathcal{F}, d)} \subseteq \mathcal K_{\textsf{ff}}({\mathcal{F}}, d+1)$, it follows from the definitions that  $\sigma_d(H_1) \le \sigma_{d+1}(H_1) \le \sigma(H_1)$ and $\lim_{d \to \infty} \sigma_d(H_1) = \sigma(H_1)$.

For example, it folows from results of   \cite{DIv, DIv2} mentioned above  that if  ${\mathcal{F}} = C_p * C_p$ is the
free product of two cyclic groups of  prime order $p >2$ and  $H_1 \in { \mathcal{K}_{\textsf{ff}}(\mathcal{F})}$, then
$$
\tfrac{p}{p-2} \le   \sigma_d(H_1)\le \sigma(H_1)  \le   \left(2 - \tfrac{(4+2\sqrt{3})p}{(2p-3+\sqrt{3})^2} \right) \cdot \tfrac{p}{p-2} .
$$

The main technical result of this article is the following.

\begin{thm}\label{th1} Suppose that ${\mathcal{F}} =G_1 * G_2$ is the free product of two nontrivial groups $G_1,  G_2$
and  $H_1$ is a  finitely generated factor-free noncyclic subgroup of ${\mathcal{F}}$. Then the following claims are true.

$\rm{(a)}$ For every integer $d \ge 3$,  there exists a linear programming problem (LP-problem)
\begin{equation}\label{lpa}
{\mathcal{P}}(H_1, d) = \max\{ c(d)x(d) \mid A(d)x(d) \le b(d)  \}
\end{equation}
with integer coefficients whose solution is equal to $-\sigma_d(H_1) {\bar {\mathrm{r}}} (H_1)$.

$\rm{(b)}$  There is a finitely generated factor-free subgroup $H_2^* $ of ${\mathcal{F}}$, $H_2^*= H_2^*(H_1)$,   which corresponds to  a vertex solution of the dual problem
$$
{\mathcal{P}}^*(H_1, d) = \min \{ b(d)^{\top}  y(d)  \mid A(d)^{\top}y(d) = c(d)^{\top} , \, y(d) \ge 0  \}
$$
of the primal LP-problem  \eqref{lpa} of part (a) such that   $\bar {\mathrm{r}}(H_1, H_2^*)  =  \sigma_d(H_1)  \bar {\mathrm{r}}(H_1) \bar {\mathrm{r}}( H_2^*)$. In particular,  the WN${}_d$-coefficient $\sigma_d(H_1)$ of $H_1$ is rational. Furthermore, if $\Psi(H_1)$, $\Psi(H_2^*)$ denote irreducible core graphs representing subgroups $H_1, H_2^*$, resp.,
and $| E \Psi |$ is the number of oriented edges in a graph $\Psi$, then
$$
| E  \Psi(H_2^*) | < 2^{  2^{4| E  \Psi(H_1) | + 1+ \log_2 \log_2 (2d)  } } .
$$

$\rm{(c)}$ There exists a linear semi-infinite programming problem (LSIP-problem) ${\mathcal{P}}(H_1) = \sup \{ cx \mid Ax \le b  \}$ with finitely many variables in $x$ and with countably  many constraints in the system $Ax \le b$ whose dual problem
$$
{\mathcal{P}}^*(H_1)  = \inf \{ b^{\top} y \mid A^{\top} y = c^{\top} , \, y \ge 0  \}
$$
has a solution equal to  $-\sigma(H_1) {\bar {\mathrm{r}}} (H_1)$.

$\rm{(d)}$ Let the word problem for both groups $G_1, G_2$ be solvable
and let an  irreducible core graph $\Psi(H_1)$  of $H_1$ be given. Then the
LP-problem \eqref{lpa}  of part (a)   can be effectively written down and the
WN${}_d$-coefficient $\sigma_d(H_1)$ for $H_1$  can be computed.
In addition,  an irreducible core graph $\Psi(H_2^*)$ of the subgroup $H_2^*$ of part (b) can be effectively constructed.

$\rm{(e)}$ Let both $G_1, G_2$ be finite,  let $d := \max( |G_1|, |G_2|) \ge 3$, and
 let an irreducible  core graph $\Psi(H_1)$  of $H_1$ be given.
 Then the LP-problem \eqref{lpa} of part (a)  coincides with the
LSIP-problem ${\mathcal{P}}(H_1)$ of part (c) and the WN-coefficient $\sigma(H_1)$ for $H_1$ is
rational and computable.
 \end{thm}

It is worthwhile to mention that  the correspondence $H_2 \to  y(H_2)$ between subgroups $H_2 \in { \mathcal{K}_{\textsf{ff}}(\mathcal{F}, d)}$, with a special property (Bd), see Sect.~3, and vectors $y(H_2)$ of the feasible polyhedron  $\{ y(d)  \mid A(d)^{\top}y(d) = c(d)^{\top} , \, y(d) \ge 0  \}$  of the dual problem ${\mathcal{P}}^*(H_1, d)$, that  is mentioned in part (b) of Theorem~\ref{th1},  plays an important role in proofs and is reminiscent of the correspondence between (almost) normal surfaces in 3-dimensional manifolds and their (resp. almost)  normal vectors in the Haken theory of normal surfaces and its generalizations, see \cite{Haken, HLP, Hemion, Iv08s, JT}. In particular, the idea of a vertex solution works equally well both in the context of almost normal surfaces \cite{Iv08s}, see also \cite{HLP, JT}, and in the context of factor-free subgroups, providing in either  situation  both the connectedness of the underlying object represented by a vertex solution and an upper bound on size of the underlying object.
\medskip

Relying on the linear programming approach of Theorem~\ref{th1}, in the following Theorem~\ref{th2}, we look at the computational complexity  of the problem to calculate the WN-coefficient $\sigma(H_1)$ for a factor-free subgroup $H_1$  of the free product of two finite groups.

\begin{thm}\label{th2}  Suppose that ${\mathcal{F}} =G_1 * G_2$ is the free product of two nontrivial finite groups $G_1,  G_2$ and $H_1$ is a subgroup of ${\mathcal{F}}$ given by a finite generating set ${\mathcal{S}}$ of words over the alphabet
${\mathcal{A}} = G_1 \cup G_2$. Then the following are true.

$\rm{(a)}$ In deterministic polynomial time of size of ${\mathcal{S}}$, one can
determine whether $H_1$ is factor-free and noncyclic and, if so, one can construct an  irreducible graph $\Psi_o(H_1)$  of $H_1$.

$\rm{(b)}$  If  $H_1$ is factor-free and noncylcic, then, in deterministic exponential time of size of ${\mathcal{S}}$, one can write down and solve an LP-problem ${\mathcal{P}} = \max\{ cx \mid Ax \le b  \}$ whose solution is equal to $-\sigma(H_1) {\bar {\mathrm{r}}} (H_1)$. In particular, the $WN$-coefficient $\sigma(H_1)$ of $H_1$ is computable in deterministic exponential time of size of ${\mathcal{S}}$.

$\rm{(c)}$  If  $H_1$ is factor-free and noncylcic, then, in deterministic double exponential time of size of ${\mathcal{S}}$, one can construct an irreducible core graph $\Psi(H_2^*)$ of $H_2^*$, where $H_2^*$ is a finitely generated factor-free subgroup ${\mathcal{F}}$ such that
$\bar {\mathrm{r}}(H_1, H_2^*) =  \sigma(H_1)  \bar {\mathrm{r}}(H_1) \bar {\mathrm{r}}( H_2^*)$.
In addition, if $| E\Psi |$ denotes the number of oriented edges of a graph $\Psi$, $\Psi(H_1)$ is an irreducible core graph of $H_1$, and   $d := \max( |G_1|, |G_2|)$, then
$
| E \Psi(H_2^*) |  <  2^{ 2^{4| E \Psi(H_1) | +1+ \log_2 \log_2 (2d)  } } .
$
\end{thm}

The situation with free products of more than two factors is more difficult to study and we will make additional efforts to  obtain the following.

\begin{thm}\label{th3}
Suppose  that ${\mathcal{F}} = \prod_{\alpha \in I}^* G_\alpha$ is
the  free product  of  nontrivial groups   $G_{\alpha}$, ${\alpha} \in I$, and $H_1$ is a
finitely generated factor-free noncyclic subgroup of ${\mathcal{F}}$.  Then there are two  disjoint finite subsets
$I_1, I_2$ of $I$ such that if
${\widehat} G_1 := \prod_{\alpha \in I_1}^* G_\alpha$,  ${\widehat} G_2 := \prod_{\alpha \in I_2}^* G_\alpha$,
and ${\widehat} {\mathcal{F}} := {\widehat} G_1 * {\widehat} G_2$, then there exists a finitely generated factor-free subgroup ${\widehat} H_1 $ of ${\widehat} {\mathcal{F}}$  with the following properties.

$\rm{(a)}$  ${\bar {\mathrm{r}}} ({\widehat} H_1  ) = {\bar {\mathrm{r}}} (H_1  ) $,
$\sigma_d({\widehat} H_1) \ge \sigma_d(H_1)$ for every $d \ge 3$,  and
$\sigma({\widehat} H_1) \ge \sigma(H_1)$. In particular, if the conjecture
\eqref{conjs} fails for $H_1$ then the conjecture \eqref{conjs} also fails for ${\widehat} H_1$.

$\rm{(b)}$ If the word problem for all groups $G_{\alpha}$, ${\alpha} \in I_1 \cup I_2$,
is solvable and  a finite irreducible graph  of $H_1$ is given, then the
LP-problem ${\mathcal{P}}({\widehat} H_1, d)$  for  ${\widehat} H_1$   of part (a) of Theorems~\ref{th1}  can be effectively written down and the WN${}_d$-coefficient $\sigma_d({\widehat} H_1) $ for ${\widehat} H_1$  can be computed.

$\rm{(c)}$  Let groups $G_{\alpha}$, where ${\alpha} \in I_1 \cup I_2$,
be finite, let $H_1$ be effectively given either by
a finite irreducible graph or by a finite generating set, and let
$$
d' := \max\{ |I_1 \cup I_2| , \max\{ |G_{\alpha} | \, : \,   {\alpha} \in I_1 \cup I_2 \} \} .
$$
Then $\sigma_{d'}({\widehat} H_1) \ge \sigma(H_1)$
and there is an algorithm that decides whether the conjecture  \eqref{conjs} holds for $H_1$.
\end{thm}

We remark that proofs of Theorems~\ref{th2}--\ref{th3} provide a practical deterministic algorithm
(with exponential running time, though) to compute the WN-coefficient $\sigma(H_1) $ for a finitely generated factor-free subgroup $H_1$ of the free product of two finite groups and to determine whether a certain finitely generated factor-free subgroup of a free product of  finite groups satisfies the  conjecture \eqref{conjs}. It seems to be of interest to implement this algorithm and experiment with it.

The article is structured as follows.
In Sect.~2,  we define basic notions and recall geometric ideas that are used to study finitely generated  factor-free subgroups and their intersections in a free product ${\mathcal{F}}$. In particular, we define a finite labeled graph $\Psi(H)$ associated with such a subgroup $H$ of ${\mathcal{F}}$.  In Sect.~3, we consider the free product  ${\mathcal{F}} = G_1 * G_2$  of two nontrivial groups $G_1, G_2$ and introduce certain linear inequalities associated with the groups $ G_1, G_2 $ and with a graph $\Psi(H_1)$ of $H_1$, where $H_1$ is a finitely generated  factor-free noncyclic subgroup of ${\mathcal{F}}$.
Informally, these inequalities are used for construction of cores of potential fiber product graphs $\Psi(H_1) \times \Psi(H_2)$, where $H_2$ is another finitely generated  factor-free subgroup of ${\mathcal{F}}$, and subsequent translation to linear programming. More formally,  these inequalities enable us to define an  LP-problem
$
\max\{ c(d)x(d) \mid A(d) x(d) \le b(d) \} ,
$
corresponding to $\Psi(H_1)$ and to an integer $d \ge 3$, and to define an LSIP-problem $\sup \{ cx \mid A x \le b \}$, corresponding to $\Psi(H_1)$. We also consider and make use of the dual problems of the primal problems $\max\{ c(d)x(d) \mid A(d) x(d) \le b(d) \}$,  $\sup \{ cx \mid A x \le b \}$.  Basic results and terminology of linear programming are recalled in Sect.~4.
These LP-,  LSIP-problems and their dual problems are investigated in Sects.~3--4. In Sect.~5, we look at the  case of  free products of more than two groups and prove a few more technical lemmas. Proofs of  Theorems~\ref{th1}--\ref{th3}  are given in Sect.~6.

\section{Preliminaries}

Let $G_{\alpha}$, ${\alpha} \in I$,  be  nontrivial
groups,  let ${\mathcal{F}} = \prod_{\alpha \in I}^* G_\alpha$ be their  free product, and let $H$ be  a finitely generated factor-free subgroup of ${\mathcal{F}}$, $H \ne \{ 1\}$.
Consider the alphabet ${\mathcal{A}} = \cup_{{\alpha} \in I} G_{\alpha}$, where $G_{\alpha} \cap G_{{\alpha}'} =\{ 1\}$ if ${\alpha} \ne {\alpha}'$.

Analogously to the graph-theoretic approach of articles \cite{DIv, DIv2, Iv99, Iv01, Iv08, Iv10, Iv12},
we first define a labeled ${\mathcal{A}}$-graph $\Psi(H)$ which  geometrically represents $H$ in a manner similar  to the way Stallings  graphs represent  subgroups of a free group, see \cite{St}.

If $\Gamma$ is a graph, $V \Gamma $ denotes the vertex set of  $\Gamma $ and  $E \Gamma $ denotes the  set of oriented edges of $\Gamma$. For $e \in E \Gamma $ let  $e_-$, $e_+$ denote the initial, terminal, resp., vertices of  $e$ and let $e^{-1}$ be the edge with the opposite (to $e$) orientation, where $e^{-1} \ne  e$ for every $e \in E\Gamma$,  $(e^{-1})_- = e_+$, $(e^{-1})_+ = e_-$.
A  {\em path} $p = e_1 \dots e_k$ in $\Gamma$ is a sequence of edges $e_1, \dots, e_k$ such that $(e_{i})_+ = (e_{i+1})_-$,  $i=1, \dots, k-1$.
Define $p_- := (e_1)_-$, $p_+ := (e_k)_+$, and $|p| := k$, where $|p| $ is the {\em length} of $p$. We  allow
the possibility  that $p = \{ p_- \} = \{ p_+ \}$ and $|p| = 0$. A path $p$ is {\em closed} if $p_- =p_+$.
A path $p  $ is called {\em reduced} if  $p$ contains no subpaths of the form $e e^{-1}$, $e \in E\Gamma$.
A closed path $p = e_1 \dots e_k$  is {\em cyclically reduced} if $|p| >0$ and both $p$ and the cyclic permutation
$e_2 \dots e_k e_1$ of $p$ are reduced paths. The  {\em core} of a graph $\Gamma$,  denoted  $\mbox{core}(\Gamma)$,
is the minimal subgraph of $\Gamma$ that contains every  edge $e$ which
can be included into a cyclically reduced path in $\Gamma$.

Let $\Psi$ be a  graph whose vertex set  $V \Psi$  consists of two disjoint nonempty parts  $V_P \Psi,  V_S \Psi$, so $V \Psi = V_P \Psi \cup V_S \Psi$. Vertices in  $ V_P \Psi$ are called {\em primary} and vertices in  $ V_S \Psi$ are called {\em secondary}.
Every edge $e \in E \Psi$ connects  primary and secondary vertices, hence $\Psi$ is a bipartite graph.   $\Psi$ is called a  {\em labeled ${\mathcal{A}}$-graph}, or briefly {\em ${\mathcal{A}}$-graph},  if  $\Psi$ is equipped with functions
 \begin{equation*}
{\varphi} : E\Psi \to {\mathcal{A}} , \qquad {\theta} :  V_S \Psi \to I ,
\end{equation*}
called  {\em labeling}, such that, for every edge $e \in E\Psi$, it is true that
${\varphi}(e) \in {\mathcal{A}} = \cup_{{\alpha} \in I} G_{\alpha}$, ${\varphi}(e^{-1}) =  {\varphi}(e)^{-1}$,   and,  if
$e_+ \in V_S \Psi$, then ${\varphi}(e) \in G_{\alpha}$ for ${\alpha} = {\theta}(e_+)$. If  $e_+ \in V_S \Psi$, define
${\theta}(e) :={\theta}(e_+)$,   ${\theta}(e^{-1}) :={\theta}(e_+)$ and call ${\theta}(e_+)$, ${\theta}(e) $    the {\em type} of  a vertex $e_+ \in V_S \Psi$ and of an edge $e \in E\Psi$. Thus, for every   $e \in E\Psi$, we have defined  an element
${\varphi}(e) \in {\mathcal{A}}$,  called the { label} of $e$, and an element ${\theta}(e)   \in I$, termed the type of $e$.

The reader familiar with  van Kampen diagrams over a free product of groups,
as defined in \cite{LS}, will recognize that our labeling function ${\varphi} : E\Psi \to {\mathcal{A}} $  is defined in the way analogous to labeling functions on van Kampen diagrams over free products of groups. Recall that van Kampen diagrams are planar 2-complexes whereas  graphs are 1-complexes, however, apart from this, the ideas of cancelations and edge folding  work equally well for both diagrams and  graphs.

An ${\mathcal{A}}$-graph $\Psi$ is called {\em irreducible} if  properties (P1)--(P3) hold true:
\begin{enumerate}
\item[(P1)] If $e, f \in E \Psi$,  $e_- = f_- \in  V_P \Psi$, and $e_+ \ne f_+$, then   ${\theta}(e) \ne {\theta}(f)$.
\item[(P2)] If $e, f \in E \Psi$,  $e \ne f$, and $e_+= f_+ \in V_S \Psi$, then  ${\varphi}(e) \ne  {\varphi}(f)$ in $G_{{\theta}(e)}$.
\item[(P3)] $\Psi$ has no multiple edges, $\mbox{deg}_\Psi v >0$ for every $v \in V\Psi$,
  and there is at most one vertex of degree 1 in $\Psi$ which, if exists, is primary.
    \end{enumerate}

Suppose $\Psi$ is a connected  finite
irreducible ${\mathcal{A}}$-graph and a primary vertex $o \in V_P\Psi$ is distinguished so that $\deg_\Psi o =1$
if $\Psi$ happens to have a vertex of degree 1. Then  $o$ is called the {\em base} vertex of $\Psi = \Psi_o$.
\medskip

As usual, elements of the free product  ${\mathcal{F}} = \prod_{\alpha \in I}^* G_\alpha$
are regarded  as words over the alphabet ${\mathcal{A}} = \cup_{{\alpha} \in I} G_{\alpha}$, where
$G_{\alpha} \cap G_{{\alpha}'} =\{ 1\}$ if ${\alpha} \ne {\alpha}'$.
A {\em syllable} of a word $W$ over ${\mathcal{A}}$ is a maximal nonempty subword of $W$
all of whose letters belong to the same factor $G_{\alpha}$. The {\em syllable length} $\| W \|$ of $W$
is the number of syllables
of $W$, while the {\em length} $|W|$ of $W$ is the number of all
letters in $W$. For example, if $a_1, a_2 \in G_{\alpha}$, then $| a_1 1 a_2 | = 3$, $\| a_1 1 a_2 \| = 1$, and
$| 1| = \| 1 \| = 1$.

A nonempty word $W$ over   ${\mathcal{A}} $ is called {\em reduced} if every
syllable of $W$ consists of a single letter.
Clearly, $|W| = \|
W \|$ if $W$ is reduced. Note that an arbitrary nontrivial element of the
free product ${\mathcal{F}}$  can  be uniquely
written as a reduced word. A word $W$  is called {\em cyclically reduced} if $W^2$ is reduced.
We write $U \overset 0 = W$ if words $U$, $W$
are equal as elements of ${\mathcal{F}}$. The
literal (or letter-by-letter) equality of words $U$, $W$ is denoted $U \equiv W$.

If $p = e_1 \dots e_k$ is a path in an  ${\mathcal{A}}$-graph $\Psi$ and $e_1, \dots, e_k$ are edges of $\Psi$, then the {\em label} ${\varphi}(p)$ of $p$ is the word ${\varphi}(p) := {\varphi}(e_1) \dots {\varphi}(e_k)$.

The significance of irreducible ${\mathcal{A}}$-graphs for geometric interpretation of  factor-free subgroups $H$ of ${\mathcal{F}}$ is given in the following lemma.

\begin{lem}\label{Lm1} Suppose $H$ is a finitely generated factor-free subgroup
of the free product ${\mathcal{F}} =  \prod_{\alpha \in I}^* G_\alpha$, $H \ne \{ 1 \}$. Then there exists a finite connected
irreducible ${\mathcal{A}}$-graph $\Psi = \Psi_o(H)$, with a  base vertex $o$,  such that a reduced word $W$ over the alphabet ${\mathcal{A}}$ belongs to $H$ if and only if there is a reduced path $p$ in $\Psi_o(H)$  such that $p_- =p_+
=o$,  ${\varphi}(p) \overset 0  = W$ in   ${\mathcal{F}}$, and $| p | = 2|W|$.

In addition, assume that all factors $G_\alpha$, $\alpha \in I$, are finite and  $V_1, \dots, V_k$  are  words over ${\mathcal{A}}$. Then there is an algorithm which, in time  polynomial of $|V_1|+ \dots+ |V_k|$, decides whether the subgroup  $H_V = \langle V_1, \dots, V_k \rangle$, generated by $V_1, \dots, V_k$, is factor-free and, if so, constructs an irreducible ${\mathcal{A}}$-graph $\Psi_o(H_V)$ for $H_V$.
\end{lem}

\begin{proof} The proof is based on Stallings's folding techniques and is somewhat analogous to the proof of van Kampen lemma for diagrams over free products of groups, see \cite{LS} (in fact, it is simpler because foldings need not preserve the property of being planar for  diagrams).

Let  $H_V = \langle V_1, \dots, V_k \rangle$ be a subgroup of ${\mathcal{F}}$,
generated by some words $V_1, \dots, V_k$ over  ${\mathcal{A}}$. Without loss of generality we can assume that  $V_1, \dots, V_k$  are  reduced  words. Consider a graph ${\widetilde} \Psi$ which consists of $k$ closed paths $p_1, \dots, p_k$ such that they have a single common vertex $o = (p_i)_-$, and $|p_i| = 2|V_i|$, $i=1, \dots, k$.
We distinguish  $o$ as the base vertex of ${\widetilde} \Psi$ and call $o$  primary, the vertices adjacent to $o$ are called secondary vertices and so on.  Denote $V \equiv a_{i,1} \dots a_{i,\ell_i}$, where $a_{i,j} \in {\mathcal{A}}$ are letters, $i =1, \dots, k$, and let
$p_i =  e_{i,1} f_{i,1} \dots e_{i,\ell_i}f_{i,\ell_i}$, where $e_{i,j}, f_{i,j}$ are edges of the path $p_i$. The labeling functions ${\varphi}, {\theta}$ on the path $p_i$ are defined so that if  $a_{i,j} \in G_{{\alpha}(i,j)}$, then
$$
{\theta}(e_{i,j}) := {\alpha}(i,j),  \quad {\theta}(f_{i,j}) := {\alpha}(i,j) , \quad {\varphi}(e_{i,j}) := a_{i,j}b_{i,j}^{-1},  \  {\varphi}(f_{i,j}) := b_{i,j} ,
$$
where $b_{i,j}$ is an element of the group  $G_{{\alpha}(i,j)}$.

Clearly,  ${\varphi}(p_i) \overset 0 = V_i$ in ${\mathcal{F}}$ for all $i=1, \dots, k$.

It is also clear that ${\widetilde} \Psi={\widetilde} \Psi_o$ is a finite connected
${\mathcal{A}}$-graph with the base vertex $o$ that has the following  property.
\begin{itemize}
\item[(Q)] A word $W  \in {\mathcal{F}}$ belongs to $H$  if and only if there is a path $p$ in ${\widetilde} \Psi_o$ such that $p_- =p_+
=o$ and  ${\varphi}(p) \overset 0 = W$.
  \end{itemize}

However, ${\widetilde} \Psi_o$ need not be irreducible  and we will do
foldings of edges in ${\widetilde} \Psi_o$ which  preserve  property (Q) and which are aimed  to achieve properties (P1)--(P2).

Assume that property (P1) fails for edges $e, f$ with  $e_- = f_- \in V_P{\widetilde} \Psi_o$ so that  $e_+ \ne f_+$ and  ${\theta}(e) ={\theta}(f)$.
Let us redefine the labels  of all edges  $e'$ with $e'_+ = e_+$ so that ${\varphi}(e') {\varphi}(e)^{-1}$ does not change
and ${\varphi}(e) = {\varphi}(f)$ in $G_{{\theta}(e)}$. This can be done by multiplication of ${\varphi}$-labels on the right by ${\varphi}(e)^{-1} {\varphi}(f)$.
Since ${\varphi}(e) = {\varphi}(f)$ and ${\theta}(e) = {\theta}(f)$, we may now identify the edges $e$, $f$ and vertices $e_+$, $f_+$.  Observe that this folding preserves  property (Q)  ((P2) might fail) and decreases the total edge number $|E {\widetilde} \Psi_o|$. This operation changes the labels of edges and can be done in time polynomial in $|V_1|+ \dots+ |V_k|$ if
all factors $G_{\alpha}$, ${\alpha} \in I$, are finite. Note that if $G_{\alpha}$ were not finite, then there would be a problem with increasing space needed to store   ${\varphi}$-labels of edges and subsequent computations with larger labels.

If property (P2) fails for edges $e, f$ and  ${\varphi}(e) = {\varphi}(f)$ in $G_{{\theta}(e)}$, then we  identify the edges $e, f$.  Note property  (Q) still holds ((P1) might fail) and the number $|E {\widetilde} \Psi_o|$ decreases.

Suppose property (P3) fails and there are two distinct edges  $e, f$ in ${\widetilde} \Psi_o$ such that $e_- = f_-$ and $e_+ = f_+ \in V_S {\widetilde} \Psi_o$.
If ${\varphi}(e) \ne {\varphi}(f)$ in $G_{{\theta}(e)}$, then a conjugate of  ${\varphi}(e) {\varphi}(f)^{-1} \in G_{{\theta}(e)}$ is in $H_V$, hence we conclude that
 $H_V$ is not factor-free.  So we may assume that ${\varphi}(e) = {\varphi}(f)$ in $G_{{\theta}(e)}$.
Then we identify the edges $e, f$, thus preserving  property  (Q) and decreasing the number $|E {\widetilde} \Psi_o|$.  If property (P3) fails so that there is a vertex $v$ of degree 1, different from $o$, then we delete $v$ along with the incident edge. Clearly, property  (Q)
still holds and the number $|E {\widetilde} \Psi_o|$ decreases.

Thus,  by induction on $|E {\widetilde} \Psi_o|$ in polynomially many
(relative to $\sum_{i=1}^{k} |V_i|$) steps as described above, we either establish that the subgroup $H_V$ is not factor-free or construct an
irreducible  ${\mathcal{A}}$-graph $\Psi_o$ with property (Q).

It follows from the definitions and from property (Q) of the graph $\Psi_o$ that $H_V$ is factor-free (see also Lemma~\ref{Lm2}). Other stated properties of $\Psi_o$ are straightforward.

Finally, we observe that if  all factors $G_\alpha$, $\alpha \in I$, are finite, then the  space required to store the ${\varphi}$-label of an edge of intermediate graphs is constant and multiplication (or inversion) of  ${\varphi}$-labels would require  time bounded by a constant. Therefore, the above procedure implies the existence of  a polynomial algorithm for finding out whether a subgroup
 $H_V= \langle V_1, \dots, V_k \rangle $ of ${\mathcal{F}}$ is factor-free and for construction of a  finite  irreducible ${\mathcal{A}}$-graph $\Psi_o$  for $H_V$.
\end{proof}

The following lemma further elaborates on the correspondence between finitely generated
factor-free subgroups of the free product  ${\mathcal{F}}$ and finite
irreducible ${\mathcal{A}}$-graphs.

\begin{lem}\label{Lm2}  Let  $\Psi_o$ be a finite connected
irreducible  ${\mathcal{A}}$-graph with the base vertex  $o$ and let
$H= H(\Psi_o)$ be a subgroup of the free product
${\mathcal{F}}$ that consists of all words $\varphi(p)$, where $p$ is
a  path in $\Psi_o$ such that $p_- = p_+ = o$. Then $H$ is a
factor-free subgroup of  ${\mathcal{F}}$   and
$\bar {\mathrm{r}}(H)=- \chi(\Psi_o)$, where $\chi(\Psi_o) =  |V \Psi_o | - \frac 12|E \Psi_o | $ is
the Euler characteristic of~$\Psi_o$.
\end{lem}

\begin{proof} This follows from  the facts that the fundamental group $\pi_1(\Psi_o, o)$ of
$\Psi_o$ at $o$ is free of rank $- \chi(\Psi_o)+1$ and that the homomorphism $\pi_1(\Psi_o, o) \to {\mathcal{F}}$, given by $p \to {\varphi}(p) $, where $p$ is a path with $p_-=p_+= o$, has the trivial kernel in view of properties (P1)--(P2).
\end{proof}

Suppose $H$ is a nontrivial finitely generated factor-free subgroup
of a free product ${\mathcal{F}} = \prod_{\alpha \in I}^* G_\alpha$, and $\Psi_o = \Psi_o(H)$ is a  finite irreducible  ${\mathcal{A}}$-graph for $H$ as in Lemma~\ref{Lm1}. We  say that  $\Psi_o(H)$ is an {\em irreducible graph} of $H$.

Let $\Psi(H) := \operatorname{core}(\Psi_o(H))$ denote the core of an irreducible graph $\Psi_o(H)$ of $H$.
Clearly, $\Psi(H)$
has no vertices of degree $\le 1$ and $\Psi(H)$ is also an irreducible ${\mathcal{A}}$-graph.
We  say that  $\Psi(H)$ is an {\em irreducible core graph} of $H$.

It is easy to see that an irreducible  graph  $\Psi_o(H)$ of $H$ can be obtained back from an irreducible  core graph $\Psi(H)$  of $H$ by attaching a suitable path $p$ to  $\Psi(H)$ so that $p$ starts at a primary vertex $o$, ends in  $p_+ \in V_P\Psi(H)$, and then by doing foldings of edges as in the proof of Lemma~\ref{Lm1}, see Fig.~2.1.
\begin{center}
\usetikzlibrary{arrows}
\begin{tikzpicture}[scale=.92]
\node at (-2,0) {$\Psi_o(H)$};
\node at (0,0.8) {$\Psi(H)=$};
\node at (0,0.3) {$ \mbox{core}\Psi_o(H)$};
\draw  (0,-0.7)[fill = black]circle (0.05);
\draw  (0,.5) ellipse (1.02 and 1.2);
\draw  plot[smooth, tension=.8] coordinates
{(0,-0.7)(-0.5,-1.5) (-1,-1) (-1.5,-1.5) (-2,-1.2) (-2.5,-1.5) (-2.8,-1.5)};
\draw [-latex]  (-1.269,-1.26) -- (-1.21,-1.16);
\draw  (-2.8,-1.5)[fill = black]circle (0.05);
\node at (-1.5,-1.1) {$p$};
\node at (-2.8,-1.1) {$o$};
\node at (.0,-2.2) {Fig.~2.1};
\end{tikzpicture}
\end{center}

Now suppose  $H_1$, $H_2$  are  nontrivial finitely generated factor-free subgroups of ${\mathcal{F}}$.
Consider a set $S(H_1, H_2)$ of representatives of those double cosets
$H_1 t H_2$ of ${\mathcal{F}}$, $t \in {\mathcal{F}}$,   that have the property $H_1 \cap t H_2 t^{-1} \ne  \{ 1 \}$.
For every $s \in S(H_1, H_2)$,  define the subgroup $K_s := H_1 \cap s H_2 s^{-1}$.
Similarly to articles \cite{Iv99, Iv01,  Iv08, Iv10, Iv12} and analogously to  the case of free groups, see  \cite{D, N2}, we now construct a finite irreducible  ${\mathcal{A}}$-graph $\Psi(H_1, H_2)$, also denoted $\operatorname{core}(\Psi(H_1) \times \Psi(H_2))$,  whose connected components are irreducible  core graphs $\Psi(K_s)$, $s \in S(H_1, H_2)$.

First we define an ${\mathcal{A}}$-graph $\Psi_o'(H_1, H_2)$. The set  of primary vertices of
$\Psi_o'(H_1, H_2)$ is
$V_P \Psi_o'(H_1, H_2)   := V_P \Psi_{o_1}(H_1)\times V_P \Psi_{o_2}(H_2)$. Let
$$
\tau_i : V_P \Psi_o'(H_1, H_2) \to V_P \Psi_{o_i}(H_i)
$$
denote the projection map, $\tau_i((v_1, v_2)) = v_i$, $i=1,2$.

The set of secondary vertices $V_S \Psi_o'(H_1, H_2)$ of
$\Psi_o'(H_1, H_2)$ consists of equivalence classes $[u]_{\alpha}$, where  $u \in V_P \Psi_o'(H_1, H_2)$, ${\alpha} \in I$, with respect to  the minimal  equivalence relation generated by   the following relation   $\overset {\alpha} \sim$ on the set  $V_P \Psi_o'(H_1, H_2)$.  Define $v \overset {\alpha} \sim w$ if and only if there are edges $e_i, f_i \in E \Psi_{o_i}(H_i)$ such that $$
(e_i)_- = \tau_i(v) , \  (f_i)_- = \tau_i(w) ,  \ (e_i)_+ = (f_i)_+
$$
for each  $i=1,2$, the edges  $e_i, f_i$ have type ${\alpha}$, and ${\varphi}(e_1) {\varphi}(f_1)^{-1} = {\varphi}(e_2) {\varphi}(f_2)^{-1}$ in $G_{\alpha}$. It is easy to see that the relation  $\overset {\alpha} \sim$  is symmetric and transitive on pairs and triples of distinct elements (but it could lack the reflexive property).

The edges in $\Psi_o'(H_1, H_2)$  are defined so that $u \in  V_P \Psi_o'(H_1, H_2)$ and
$[v]_{\alpha} \in V_S \Psi_o'(H_1, H_2)$ are connected by an edge if and only if $u  \in [v]_{\alpha}$.

The type ${\theta}([v]_{\alpha})$  of a vertex $[v]_{\alpha} \in V_S \Psi_o'(H_1, H_2)$  is ${\alpha}$ and if  $e \in E\Psi_o'(H_1, H_2)$,  $e_- =u$, $e_+ = [v]_{\alpha}$, then ${\varphi}(e) :={\varphi}(e_1)$, where $e_1  \in E\Psi_{o_1}(H_1)$
is an edge of type ${\alpha}$ with $(e_1)_- = \tau_1(u)$, when such an $e_1$ exists, and ${\varphi}(e_1) :=g_{\alpha} $, where $g_{\alpha} \in G_{\alpha}$, $g_{\alpha} \ne 1$,    otherwise.

It follows from the definitions and properties (P1)--(P2) of $\Psi_{o_i}(H_i)$, $i=1,2$, that $\Psi_o'(H_1, H_2)$   is an  ${\mathcal{A}}$-graph with properties (P1)--(P2). Hence, taking the core of
$\Psi_o'(H_1, H_2)$, we obtain a  finite  irreducible  ${\mathcal{A}}$-graph which we  denote by $\Psi(H_1, H_2)$ or by $\operatorname{core}(\Psi(H_1) \times \Psi(H_2))$.

It is not difficult to see that, when taking the connected component $\Psi_o'(H_1, H_2, o)$  of $\Psi_o'(H_1, H_2)$  that contains the vertex $o = (o_1, o_2)$ and inductively removing  from $\Psi_o'(H_1, H_2, o)$  vertices of degree 1 different from $o$, we obtain an
irreducible  ${\mathcal{A}}$-graph $\Psi_o(H_1\cap H_2)$ with the base vertex $o$ that  corresponds to the intersection $H_1\cap H_2$ as in Lemma \ref{Lm1}.

It follows from the definitions and property (P1) for  $\Psi(H_i)$, $i=1,2$,   that, for every edge $e \in E \Psi(H_1, H_2)$ with $e_- \in V_P \Psi(H_1, H_2)$, there are unique edges
$e_i  \in E\Psi(H_i)$ such that  $\tau_i(e_-) =(e_i)_-$, $i=1,2$. Hence, by setting  $\tau_i(e) =e_i$, $\tau_i(e_+) =(e_i)_+$, $i=1,2$,   we extend  $\tau_i$ to the graph map
\begin{gather}\label{tau1}
\tau_i :  \Psi(H_1, H_2) \to  \Psi(H_i)  , \quad i=1,2 \, .
\end{gather}
It follows from the definitions that $\tau_i$ is locally injective and $\tau_i$ preserves  syllables of the word ${\varphi}(p)$ for every path $p$ with primary vertices $p_-, p_+$.

\begin{lem}\label{Lm3} Suppose $H_1$, $H_2$ are  finitely generated factor-free
subgroups of the free product ${\mathcal{F}}$ and
$S(H_1, H_2) \ne  \varnothing$.  Then the connected components of the graph  $\Psi(H_1, H_2)$ are core graphs  $\Psi(H_1 \cap s H_2 s^{-1})$ of subgroups $H_1 \cap s H_2 s^{-1}$, $s \in S(H_1, H_2)$. In particular,
$\bar {\mathrm{r}}(H_1, H_2) :=
\sum_{s \in S(H_1, H_2) } \bar {\mathrm{r}}(H_1 \cap s  H_2 s^{-1}) =
-\chi(\Psi(H_1, H_2) )$.
\end{lem}
\begin{proof} This is straightforward, details can be found in \cite{Iv12}.
\end{proof}

\section{The system of linear inequalities $\operatorname{\textsf{SLI}}[Y_1]$}

In this Section, we  let ${\mathcal{F}}_2 = G_1 * G_2$ be the free product of two nontrivial  groups  $G_1, G_2$, let
${\mathcal{A}} = G_1 \cup G_2$ be the alphabet,  $G_1 \cap G_2 = \{ 1\}$,  and let $Y_1$ be a finite connected
irreducible  ${\mathcal{A}}$-graph such that  $\operatorname{core}(Y_1) = Y_1$ and ${\bar {\mathrm{r}}}(Y_1) := -\chi(Y_1) >0$.
In particular, $Y_1$ has no vertices of degree 1 and $Y_1$ contains a vertex of degree $>2$.

Let $S_2(G_{\alpha})$, where ${\alpha} =1,2$, denote the set of all finite subsets of $G_{\alpha}$ of cardinality $\ge 2$ and let $S_1(V_P Y_1)$ denote the set of all nonempty subsets of $V_P Y_1$.
For every $T \in S_2(G_{\alpha})$, consider  a function
$$
{\Omega}_T : T \to S_1(V_P Y_1) .
$$
We also consider a relation  ${\sim}_{{\Omega}_T}$  on the  set of all pairs $(a, u)$, where $a \in T$ and $u \in {\Omega}_T(a)$, defined as follows.  Two pairs $(a, u)$, $(b, v)$ are related by ${\sim}_{{\Omega}_T}$,  written $(a, u) {\sim}_{{\Omega}_T} (b, v)$, if and only if the following holds. Either $(a, u) = (b, v)$ or, otherwise,
there exist edges $e, f \in E Y_1$ with the properties that $e_- = u$,  $f_- = v$, the secondary vertex $e_+ = f_+$ has type ${\alpha}$, and ${\varphi}(e){\varphi}(f)^{-1} = ab^{-1}$  in $G_{\alpha}$, see an example depicted in Fig.~3.1.
It is easy to see that the relation  ${\sim}_{{\Omega}_T}$  is an equivalence one.

\begin{center}
\begin{tikzpicture}[scale=.64]
\draw  (-2,2) [fill = black]circle (0.05);
\draw  (-2,3) node (v1) {} [fill = black]circle (0.05);
\draw  (-2,4) [fill = black]circle (0.05);
\draw  (-2.3,3) ellipse (1.1 and 1.8);

\draw  (0,6) [fill = black]circle (0.05);
\draw  (1,6.) node (v5) {} [fill = black]circle (0.05);
\draw  (2,6) [fill = black]circle (0.05);
\draw  (1,6.3) ellipse (1.8 and 1.1);

\draw  (4,2) [fill = black]circle (0.05);
\draw  (4,3) node (v3) {} [fill = black]circle (0.05);
\draw  (4,4) [fill = black]circle (0.05);
\draw  (4.3,3) ellipse (1.1 and 1.8);

\draw  (0,0) [fill = black]circle (0.05);
\draw  (1,0) node (v4) {} [fill = black]circle (0.05);
\draw  (2,0) [fill = black]circle (0.055);
\draw  (1,-.3) ellipse (1.8 and 1.1);

\draw  (1,3) node (v2) {} circle (0.08);
\draw [-latex]  (-.6,3) -- (-.4,3);
\draw [-latex]  (2.6,3) -- (2.4,3);
\draw [-latex]  (1,1.4) -- (1,1.6);
\draw [-latex]  (1,4.6) -- (1,4.4);
\draw  (v1) edge (v2);
\draw  (v2) edge (v3);
\draw  (v2) edge (v4);
\draw  (v5) edge (v2);
\node at (-2.5,2) {$u_2$};
\node at (-2.5,3) {$u_5$};
\node at (-2.5,4) {$u_8$};
\node at (0,6.5) {$u_1$};
\node at (1,6.5) {$u_2$};
\node at (2,6.5) {$u_3$};
\node at (4.5,4) {$u_3$};
\node at (4.5,3) {$u_4$};
\node at (4.5,2) {$u_5$};
\node at (2,-0.5) {$u_1$};
\node at (1,-0.5) {$u_6$};
\node at (0,-0.5) {$u_7$};
\node at (1.5,4.5) {$e_1$};
\node at (2.5,2.5) {$e_2$};
\node at (0.5,1.5) {$e_3$};
\node at (-0.5,3.5) {$e_4$};
\node at (-9,6) {$T  = \{ g_1, g_2, g_3, g_4 \} \subseteq G_\alpha,$ };
\node at (-9,5) {$ \Omega_T(g_1) = \{ u_1, u_2, u_3 \},  $ };
\node at (-9,4) {$\Omega_T(g_2) = \{ u_3, u_4, u_5 \}$, };
\node at (-9,3) {$\Omega_T(g_3) = \{ u_1, u_6, u_7 \} $,  };
\node at (-9,2) {$\Omega_T(g_4) = \{ u_2, u_5, u_8 \} $,  };
\node at (-8.4,1) {$\Omega_T(g_i)  \subseteq V_PY_1, \ \varphi(e_i) = g_i,\  i =1,2,3,4$,  };
\node at (-8,0) {$(g_1, u_2) \sim_{\Omega_T} (g_2, u_4) \sim_{\Omega_T}
   (g_3, u_6) \sim_{\Omega_T} (g_4, u_5) $.  };

\node at (4.2,6.4) {$ \Omega_T(g_1) $};
\node at (4.5,0.5) { $\Omega_T(g_2)$ };
\node at (3.7,-1.24) {$ \Omega_T(g_3) $};
\node at (-2.3,5.5) {$ \Omega_T(g_4) $};
\node at (-3,-2.2) {Fig.~3.1};
\end{tikzpicture}
\end{center}

Let
$[(a, u)]_{ {\sim}_{{\Omega}_T} }$ denote the equivalence class of $(a, u)$ and let $| [(a, u)]_{ {\sim}_{{\Omega}_T} } |$ denote the cardinality of $[(a, u)]_{ {\sim}_{{\Omega}_T} }$. It follows from the definitions that
\begin{gather}\label{cd1}
1 \le  | [(a, u)]_{ {\sim}_{{\Omega}_T} } | \le |T|   .
\end{gather}
We will say that  the equivalence class  $[(a, u)]_{ {\sim}_{{\Omega}_T} }$ is {\em associated} with a secondary vertex $w \in V_S Y_1$  of type ${\alpha}$ if $w = e_+$, where  $e \in E Y_1$ and  $e_- = u$. It is easy to see that the definition of $w$ is independent of $u$ in  $[(a, u)]_{ {\sim}_{{\Omega}_T} }$.

A function ${\Omega}_T : T \to S_1(V_P Y_1)$, $T \in S_2(G_{\alpha})$, is called
{\em ${\alpha}$-admissible} if
$$
|[ (a, u)  ]_{{\sim}_{{\Omega}_T}} | \ge 2
$$
for every equivalence class $[(a, u)]_{{\sim}_{{\Omega}_T}}$, where $a \in T$, $u \in {\Omega}_T(a)$. The set of all ${\alpha}$-admissible functions is denoted $\Omega(Y_1, {\alpha})$, ${\alpha} =1,2$.

Let ${\Omega}_T \in  \Omega(Y_1, {\alpha})$ be an ${\alpha}$-admissible  function, $T \in S_2(G_{\alpha})$,  and let
\begin{equation}\label{Nal}
N_{\alpha} ( {\Omega}_T ) := \sum ( |  [  (a, u) ]_{{\sim}_{{\Omega}_T}} | -2 )
\end{equation}
denote the sum of cardinalities minus two over all equivalence classes $[(a, u) ]_{{\sim}_{{\Omega}_T}}$, where  $a \in T$ and $u \in {\Omega}_T(a)$, of the equivalence relation ${\sim}_{{\Omega}_T}$.

Let $r$ be the number of all equivalence classes $[(a, u)]_{{\sim}_{{\Omega}_T}}$ of the  equivalence  relation ${\sim}_{{\Omega}_T}$,  where $a \in T$ and $u \in {\Omega}_T(a)$. If $r \ge |V_P Y_1 |$, then it follows from \eqref{Nal} and definitions that
\begin{equation*}
N_{\alpha} ( {\Omega}_T ) = \sum ( |  [  (a, u) ]_{{\sim}_{{\Omega}_T}} | -2 )  \le
     |T|\cdot |V_P Y_1 | - 2r \le (|T| - 2) |V_P Y_1 | .
\end{equation*}
On the other hand, if $r \le |V_P Y_1 |$, then it follows from \eqref{cd1} and \eqref{Nal} that
\begin{equation*}
N_{\alpha} ( {\Omega}_T ) = \sum ( |  [  (a, u) ]_{{\sim}_{{\Omega}_T}} | -2 )  \le
   (|T| -2 ) r   \le (|T| - 2) |V_P Y_1 |  .
\end{equation*}
Thus, in any case, it is shown that
\begin{equation}\label{cd2}
N_{\alpha} ( {\Omega}_T )\le (|T| - 2) |V_P Y_1 |  .
\end{equation}

For every $A \in S_1(V_P Y_1)$,
we consider a variable $x_A$. We also introduce a special variable $x_s$. Now we will define a system of linear inequalities in these variables.
For every ${\alpha}$-admissible function $\Omega_T$,   where $ T \in S_2(G_{\alpha})$, ${\alpha} =1,2$,  denote $T = \{ b_1, \dots, b_{k} \}$ and set $A_i := {\Omega}_T(b_i)$, $i=1, \dots, k$. If ${\alpha} =1$, then the inequality,  corresponding to the  ${\alpha}$-admissible  function $\Omega_T$, is defined as follows
\begin{equation}\label{inqa}
- x_{A_1} -\dots  - x_{A_{k}}  - (k-2)x_s \le   - N_1(\Omega_T)   .
\end{equation}
If ${\alpha} =2$, then the inequality  corresponding to the  ${\alpha}$-admissible  function $\Omega_T$ is written in the following form
\begin{equation}\label{inqb}
 x_{A_1} +\dots  + x_{A_{k}}  - (k-2)x_s \le   - N_2(\Omega_T)   .
\end{equation}

Let  $d \ge 3$ be an integer and  $\operatorname{\textsf{SLI}}[Y_1]$ denote  the system of linear inequalities \eqref{inqa}--\eqref{inqb} over all ${\alpha}$-admissible functions $\Omega_T$, $\Omega_T \in \Omega(Y_1, {\alpha})$,  ${\alpha} =1,2$. Since the set $S_2(G_{\alpha})$ is in general infinite (unless $G_{\alpha}$ is finite) and the set $S_1(V_P Y_1)$  is finite  (because $Y_1$ is finite),  it follows that  $\operatorname{\textsf{SLI}}[Y_1]$ is an infinite system  of linear inequalities with integer coefficients over a finite set of variables $x_A, A \in S_1(V_P Y_1)$, $x_s$.

Let
\begin{equation}\label{slid}
\operatorname{\textsf{SLI}}_d[Y_1]
\end{equation}
denote  the subsystem of the system $\operatorname{\textsf{SLI}}[Y_1]$  whose  linear inequalities \eqref{inqa}--\eqref{inqb} are defined for all ${\alpha}$-admissible functions $\Omega_T$,  $\Omega_T \in \Omega(Y_1, {\alpha})$,  ${\alpha} =1,2$, such that $|T| \le d$. If $q$ is an inequality of $\operatorname{\textsf{SLI}}_d[Y_1]$
then the coefficient of $x_s$ in the  left-hand side of $q$ is an integer $-k+2$, where $2 \le k = |T| \le d$, and the right-hand side of $q$ is an  integer $-N_{\alpha}({\Omega}_T)$, where    $0 \le N_{\alpha}({\Omega}_T) \le (d-2) | V_P Y_1 |$, as follows from inequality \eqref{cd2}. Also, the number of subsets $A \subseteq V_P Y_1$ that  index variables $\pm  x_A$  in the  left-hand side of $q$  is finite and the total number of occurrences of such variables $\pm  x_A$
in $q$ is $k = |T| \le d$.
Therefore, $\operatorname{\textsf{SLI}}_d[Y_1]$ is a finite system of linear inequalities and
$\operatorname{\textsf{SLI}}[Y_1] = \cup_{d=3}^{\infty}  \operatorname{\textsf{SLI}}_d[Y_1]$.
\medskip

Let $Y_2$ be a finite irreducible   ${\mathcal{A}}$-graph. Consider the following property of $Y_2$.
\begin{enumerate}
\item[(B)] \ The map $\tau_2  :  \operatorname{core} (Y_1  \times  Y_2) \to Y_2$ is surjective, $\operatorname{core}(Y_2) = Y_2$,
     and  $ {\bar {\mathrm{r}}}(Y_2) := - \chi(Y_2) > 0$.
     \end{enumerate}

If $\Gamma$ is a finite graph, let $\deg \Gamma$ denote the maximum
of degrees of vertices of $\Gamma$. We will also use one more  property.
\begin{enumerate}
\item[(Bd)] \  $Y_2$ has property (B) and $\deg Y_2 \le d$, where $d \ge 3$ is an integer.
\end{enumerate}

Suppose $Y_2$ is a graph with property (B).
For every secondary vertex $u \in V_S Y_2$ of type ${\alpha}$,  consider
all edges $e_1, \dots, e_\ell$, where $\deg u = \ell$, such that
 $u = (e_1)_+=  \dots = (e_\ell)_+$ and denote   $v_j = (e_j)_-$, $j =1, \dots, \ell$.  Define   $T_u:= \{ {\varphi}(e_1), \dots , {\varphi}(e_\ell) \}$.
 Clearly,  $T_u \subseteq S_2(G_{\alpha})$.
 Define  the sets $A_j := \tau_1  \tau_2^{-1}(v_j) \subseteq V_P Y_1$ for $j=1, \dots, \ell$ and consider the function $$
 {\Omega}_{T_u} : T_u \to S_1(V_P Y_1)
 $$
 so that ${\Omega}_{T_u}({\varphi}(e_j)) = A_j$.

It is easy to check that  $\Omega_{T_u}$ is ${\alpha}$-admissible.
Since every  ${\alpha}$-admissible function $\Omega \in \Omega(Y_1, {\alpha})$
gives rise to  an inequality \eqref{inqa}  if ${\alpha} =1$ or  to an inequality \eqref{inqb}  if ${\alpha}=2 $ and every
secondary vertex $u \in V_S Y_2$  of type ${\alpha}$  defines, as indicated above, an ${\alpha}$-admissible  function  $\Omega_{T_u}$, it follows that every  $u \in V_S Y_2$ is mapped to a certain inequality
of the system  $\operatorname{\textsf{SLI}}[Y_1]$, denoted $\operatorname{\textsf{inq}}_S(u)$.  Thus we obtain a function
\begin{equation*}
\operatorname{\textsf{inq}}_S : V_S Y_2 \to \operatorname{\textsf{SLI}}[Y_1]
\end{equation*}
defined from the set $V_S Y_2$ of secondary vertices of a finite irreducible ${\mathcal{A}}$-graph  $Y_2$ with property (B)
to the set of inequalities of $\operatorname{\textsf{SLI}}[Y_1]$.

If $q$ is an inequality of the system $\operatorname{\textsf{SLI}}[Y_1]$, denoted  $q \in \operatorname{\textsf{SLI}}[Y_1]$, we let $q^L$ denote the left-hand side of $q$ and let  $q^R$ denote the integer in the right-hand side of $q$.

\begin{lem}\label{lem1}
  In the foregoing notation, if $\deg u \le d$ for every $u \in V_S Y_2$, then $\operatorname{\textsf{inq}}_S(V_S Y_2) \subseteq  \operatorname{\textsf{SLI}}_d[Y_1]$. Furthermore,
 \begin{equation*}\label{inc2}
\sum_{u \in V_S Y_2}  \operatorname{\textsf{inq}}_S(u)^L = -2 {\bar {\mathrm{r}}} (Y_2) x_s  \, ,   \ \mbox{and}  \
\sum_{u \in  V_S Y_2}  \operatorname{\textsf{inq}}_S(u)^R = -2 {\bar {\mathrm{r}}} ( \operatorname{core}(Y_1 \times  Y_2) )   .
\end{equation*}
\end{lem}

\begin{proof}  The inclusion  $\operatorname{\textsf{inq}}_S(V_S Y_2) \subseteq  \operatorname{\textsf{SLI}}_d[Y_1]$   is evident from the definitions.

Suppose $v \in V_P Y_2$ and  let
$e_1$, $e_2$ be the edges such that
$(e_1)_- =  (e_2)_- =  v$  and  ${\varphi}(e_{\alpha})  \in G_{\alpha}$, ${\alpha} =1,2$.
Clearly, $u_{\alpha} := (e_{\alpha})_+$, ${\alpha} =1,2$, is a secondary vertex of $Y_2$.     Denote $A_v := \tau_1 \tau_2^{-1}(v)$. It follows from the definitions that  $A_v \in S_1( V_P Y_1)$ and that the variables   $-x_{A_v}$, $x_{A_v}$  occur in
$\operatorname{\textsf{inq}}_S(u_1)^L$, $\operatorname{\textsf{inq}}_S(u_2)^L$, resp., and will cancel out in the sum $\operatorname{\textsf{inq}}_S(u_1)^L + \operatorname{\textsf{inq}}_S(u_2)^L$. It is easy to see that  all occurrences of variables $\pm x_A$, $A \in S_1( V_P Y_1)$, in the formal sum
$$
\sum_{u \in V_S Y_2 } \operatorname{\textsf{inq}}_S(u)^L ,
$$
before any cancelations are made, can be  paired down by using primary vertices  of $Y_2$ as indicated above.  Since every secondary vertex $u$ of $Y_2$  contributes $-(\deg u -2)$ to the coefficient of $x_s$ in $\sum_{u \in V_S Y_2 } \operatorname{\textsf{inq}}_S(u)^L$ and
\begin{equation}\label{add1a}
\sum_{u \in V_S Y_2}  (\deg u -2) = 2 {\bar {\mathrm{r}}} (Y_2)  ,
\end{equation}
the first equality of Lemma~\ref{lem1} becomes evident.  The second equality follows from
the analogous to \eqref{add1a} formula
\begin{equation*}
\sum_{u \in V_S (\operatorname{core}(Y_1 \times  Y_2))}  (\deg u -2) = 2{\bar {\mathrm{r}}} (\operatorname{core}(Y_1 \times  Y_2)) ,
\end{equation*}
and from the definition \eqref{Nal} of numbers $N_{\alpha}(\Omega(u))$, $u \in V_S Y_2$.
\end{proof}

Let $A = \{ a_1,  \dots, a_n \}$ be a finite set. A {\em combination with repetitions} $B$ of $A$, which we denote $B = [[ b_1, \dots, b_m ]]$, is a finite unordered collection of multiple copies of elements of $A$. Hence $b_i \in A$ and $b_i = b_j$ is possible for $i \ne j$.

Observe that a finite irreducible ${\mathcal{A}}$-graph $Y_2$ with  property (Bd)  can be used to construct a combination  with repetitions, denoted
$\operatorname{\textsf{inq}}(V_S Y_2)$ (note that, in general, $\operatorname{\textsf{inq}}(V_S Y_2) \ne \operatorname{\textsf{inq}}_S(V_S Y_2)$ as $\operatorname{\textsf{inq}}_S(V_S Y_2) \subseteq \operatorname{\textsf{SLI}}_d[Y_1]$),  of the system   $\operatorname{\textsf{SLI}}_d[Y_1]$, whose elements are individual inequalities of $\operatorname{\textsf{SLI}}_d[Y_1]$
so that every inequality $q = \operatorname{\textsf{inq}}_S(u)$ of  $\operatorname{\textsf{SLI}}_d[Y_1]$ occurs in $\operatorname{\textsf{inq}}(V_S Y_2)$ as many times as the number
of preimages of $q$ in $V_SY_2$ under $\operatorname{\textsf{inq}}_S$.  It follows from Lemma~\ref{lem1} that if $\operatorname{\textsf{inq}} (V_S  Y_2)  = [[ q_1, \dots, q_m ]]$, then
$$
\sum_{q \in  \operatorname{\textsf{inq}} (V_S  Y_2) }^{}q^L := \sum_{j=1}^{m}q_j^L = - 2 {\bar {\mathrm{r}}}(Y_2)  x_s  .
$$

In the opposite direction, we will prove the following.

\begin{lem}\label{lem2} Suppose $Q$ is a nonempty combination with repetitions of
$\operatorname{\textsf{SLI}}_d[Y_1]$  and
\begin{equation}\label{l2i11}
\sum_{q \in  Q}^{}q^L = -C x_s  ,
\end{equation}
where $C > 0 $ is an integer. Then there exists a  finite irreducible ${\mathcal{A}}$-graph $Y_Q$ such that $Y_Q$ has  property (Bd) and  $\operatorname{\textsf{inq}}(V_S Y_Q) =Q$. In particular,
\begin{equation}\label{l2i2}
\sum_{q \in  \operatorname{\textsf{inq}}( V_S Y_Q)}^{}q^L = - 2{\bar {\mathrm{r}}}(Y_Q) x_s \quad  \mbox{and}   \quad
\sum_{q \in  \operatorname{\textsf{inq}}(V_S Y_Q)}^{}q^R = - 2 {\bar {\mathrm{r}}}(\operatorname{core}(Y_1  \times  Y_Q))   .
\end{equation}
\end{lem}

\begin{proof} We will construct an ${\mathcal{A}}$-graph $Y_Q$ whose set $V_S Y_Q$   of secondary vertices is in bijective correspondence $u_j \to q_j$, $j=1, \dots, m$, with elements of $Q = [[ q_1, \dots, q_m ]]$ so that vertices of type ${\alpha}$ in  $V_S Y_Q$ correspond to inequalities of type ${\alpha}$ in $Q$, see \eqref{inqa}--\eqref{inqb}. To fix notation, we
let the inequality  $q_j$ be defined by means of an ${\alpha}_j$-admissible function
$$
{\Omega}_j : T_j \to S_1( V_P Y_1) ,
$$
where $T_j \in S_2(G_{{\alpha}_j})$, $T_j = \{ b_{1,j}, \dots, b_{k_j, j}\}$,
 $2 \le  k_j \le d$, $b_{\ell, j} \in  G_{{\alpha}_j}$.

 Consider a secondary vertex $u_j$ of type ${\alpha}_j$ and $k_j$ edges $e_{1,j}, \dots, e_{k_j, j}$ whose terminal vertex is $u_j$ and whose labels are ${\varphi}(e_{1,j}) = b_{1,j}, \dots$, ${\varphi}(e_{k_j, j}) = b_{k_j, j}$.  This is a local structure of the graph $Y_Q$ around its secondary vertices.  Now we will  identify in pairs  initial vertices
 of  edges  $e_{1,j}, \dots, e_{k_j, j}$,  $j=1, \dots, m$,   which will form the set of primary vertices $V_P Y_Q$ of the graph $Y_Q$.  In the notation introduced above, it follows from the definitions \eqref{inqa}--\eqref{inqb} that a typical term  $\pm x_A$ of $q_j^L$ has the form  $(-1)^{{\alpha}_j}x_A$, where $A= { {\Omega}_j(b_{\ell,j})}$ for some $\ell = 1, \dots, k_j$.

It follows from equality \eqref{l2i11} that there is an involution map $\iota$ on the set of $\pm x_A$ terms of the formal sum $\sum_{j=1}^m q_j^L$ that takes every term  $\pm x_A$ of $q_j^L$ to a term  $\mp x_A$ of $q_{j'}^L$, $j \ne j'$, and $\iota^2 = \mbox{id}$. Therefore, if
$$
(-1)^{{\alpha}_j}x_{ {\Omega}_j(b_{\ell,j})} \quad \mbox{and} \quad  (-1)^{{\alpha}_{j'}}x_{ {\Omega}_{j'}(b_{\ell',j'})}
$$
are two terms of the formal sum $\sum_{j=1}^m q_j^L$ which are $\iota$-images of each other, then $\{ {\alpha}_{j},  {\alpha}_{j'} \} = \{ 1,2 \}$ and ${\Omega}_j(b_{\ell,j}) = {\Omega}_{j'}(b_{\ell',j'})$. We identify  the initial vertices of the edges $e_{ \ell,j}$,  $e_{ \ell', j'}$ so that $(e_{ \ell, j})_- = (e_{\ell', j'})_-$ becomes a primary vertex of  $Y_Q$.
Doing this identification  for initial vertices of the pairs of edges, corresponding as above to all pairs of  canceling symbols $\pm x_A$, $\mp x_A$ in $\sum_{j=1}^m q_j^L$, results in an ${\mathcal{A}}$-graph $Y_Q$.
It is not difficult to see from the definitions and from ${\alpha}$-admissibility of functions $\Omega_{1}, \dots,  \Omega_{m}$ that  $Y_Q$ is a  finite  irreducible ${\mathcal{A}}$-graph with $\operatorname{core}(Y_Q) = Y_Q$, that $\operatorname{\textsf{inq}}(V_S Y_Q) = Q$, and that the map
$$
\tau_Q : \operatorname{core}(  Y_1  \times  Y_Q ) \to Y_Q
$$
is surjective.  Note that the map  $\tau_Q$ is the analogue of $\tau_2$, see  \eqref{tau1}.
Hence, $Y_Q$  has property (B). Clearly,  $\deg u \le d$ for every $u \in V_S Y_Q$, whence $Y_Q$ also has property (Bd).
The equalities  \eqref{l2i2}  follow from Lemma~\ref{lem1}.
\end{proof}

We  summarize Lemmas~\ref{lem1}--\ref{lem2} in the following.

\begin{lem}\label{lem3}  The function $\operatorname{\textsf{inq}} : Y_2 \to \operatorname{\textsf{inq}}(V_S Y_2) = Q$ from the set of finite irreducible ${\mathcal{A}}$-graphs $Y_2$ with property (Bd) to the set of combinations $Q$ with repetitions of the system $\operatorname{\textsf{SLI}}_d[Y_1]$ with the property
$\sum_{q \in Q}^{}q^L = -C x_s $,
where $C>0$ is an integer, is surjective. In addition, if $\operatorname{\textsf{inq}}( V_S Y_2) = Q$ then
\begin{equation*}\label{l2i1}
\sum_{q \in Q}^{}q^L   =    -2{\bar {\mathrm{r}}}(Y_2) x_s \quad  \mbox{and}   \quad
\sum_{q \in  Q}^{  }q^R = - 2{\bar {\mathrm{r}}}(\operatorname{core}(Y_1  \times  Y_2))   .
\end{equation*}
\end{lem}

\begin{proof} This is straightforward  from Lemmas~\ref{lem1}--\ref{lem2} and their proofs.
\end{proof}

\section{Utilizing Linear and Linear Semi-infinite Programming}

First we briefly review relevant results from the theory of linear programming (LP) over the field $\mathbb Q$ of rational numbers. Following the notation of Schrijver's monograph \cite{S86}, let $A \in \mathbb Q^{m\times n}$ be an $m\times n$-matrix, let $b \in \mathbb Q^{n\times 1} = \mathbb Q^{n} $ be a column vector,  let  $c \in \mathbb Q^{1\times n} $ be a row vector, $c=(c_1, \ldots, c_n)$,
and let $x$ be a column vector consisting of variables $x_1, \dots, x_n$, so $x = (x_1, \dots, x_n)^{\top}$, where $M^{\top}$ means the transpose of a matrix $M$.
The inequality $x \ge 0$ means that $x_i \ge 0$ for every $i$.

A typical LP-problem asks about the maximal value of the objective linear functional
$$
cx= c_1x_1+\dots +c_nx_n
$$
over all $x \in \mathbb Q^{n}$ subject to a finite system of linear inequalities $Ax \le b$.
This value (and often the LP-problem itself) is denoted
$$
\max\{ cx \mid Ax \le b  \} .
$$

We write $\max\{ cx \mid Ax \le b    \} = -\infty$ if the set $\{ cx \mid Ax \le b    \}$ is empty. We write $\max\{ cx \mid Ax \le b    \} = +\infty$ if the set $\{ cx \mid Ax \le b    \}$ is unbounded from above and say that
 $\max\{ cx \mid Ax \le b   \} $ is finite if the set $\{ cx \mid Ax \le b    \}$ is nonempty and bounded from above. The notation and terminology for an LP-problem
 $$
 \min\{ cx \mid Ax \le b    \} = -  \max\{ -cx \mid Ax \le b    \}
 $$
 is analogous with
 $-\infty$ and $+\infty$ interchanged.

 If $\max\{ cx \mid Ax \le b   \} $ is an LP-problem as defined above, then the problem
 $$
 \min\{ b^{\top}y \mid A^{\top}y = c^{\top}, y\ge 0  \} ,
 $$
where  $y = (y_1, \dots, y_m)^{\top}$, is called the {\em dual} problem of the {\em primal}
LP-problem  $\max\{ cx \mid Ax \le b  \}$.

The (weak) duality theorem of linear programming can be stated as follows, see \cite[Sect. 7.4]{S86}.

\begin{TA} Let  $\max\{ cx \mid Ax \le b   \} $ be an LP-problem and let
$\min\{ b^{\top}y \mid A^{\top}y = c^{\top}, y\ge 0  \} $ be its dual LP-problem.
  Then
 for every $x \in  \mathbb Q^n$ such that  $ Ax \le b$  and every $y \in  \mathbb Q^m$ such that  $A^{\top}y = c^{\top}, y\ge 0$, one has $c x  \le y^{\top} Ax \le  b^{\top} y$ and
 \begin{equation}\label{dt}
\max\{ cx \mid Ax \le b    \}  =  \min\{ b^{\top}y \mid A^{\top}y = c^{\top}, y\ge 0  \}
\end{equation}
provided both polyhedra  $\{ x \mid Ax \le b \}$ and  $\{  y \mid A^{\top}y = c^{\top}, y\ge 0  \}$ are not empty.  In addition,
the minimum, whenever it is finite,  is attained at a vector ${\widehat} y$  which is a vertex of
the polyhedron $\{ y  \mid A^{\top}y = c^{\top}, y\ge 0 \}$.
\end{TA}

Since  the system of inequalities $\operatorname{\textsf{SLI}}[Y_1]$, as defined in Sect.~3, is infinite in general, we also recall basic terminology and results regarding   duality in linear semi-infinite programming (LSIP), see \cite{LSIP1}, \cite{LSIP2}, \cite{LSIP3}. Consider a generalized LP-problem $\max\{ cx \mid Ax \le b    \}$ that has countably many linear inequalities in the system $Ax \le b$ while the number of variables in $x$ is still finite. Hence, in this setting,    $A$ is a matrix with countably many rows and $n$ columns, or $A  \in \mathbb Q^{\infty \times n}$ is an $\infty \times n$-matrix, $b \in \mathbb Q^{\infty \times 1} = \mathbb Q^{\infty }$, or $b$ is an infinite column vector,  $c \in \mathbb Q^{1\times n} $ is a row vector, and  $x = (x_1, \dots, x_n)^{\top}$.

A typical LSIP-problem over $\mathbb Q$ asks about the supremum  of the objective linear functional $cx$  over all $x \in \mathbb Q^{n}$ subject to $Ax \le b$.  This number  and the problem itself is denoted $\sup \{ cx \mid Ax \le b \}$. As above, we write $\sup \{ cx \mid Ax \le b \} = -\infty$ if the set $\{ cx \mid Ax \le b    \}$ is empty,  $\sup \{ cx \mid Ax \le b    \} = +\infty$ if the set $\{ cx \mid Ax \le b    \}$ is not bounded from above and say that
 $\sup \{ cx \mid Ax \le b    \} $ is finite if the set $\{ cx \mid Ax \le b    \}$ is nonempty and bounded from above.  The notation and terminology for an LSIP-problem
 $\inf \{ cx \mid Ax \le b \} = -  \sup \{ -cx \mid Ax \le b    \} $ is analogous with $-\infty$ and $+\infty$ interchanged.  Let $A_i$ denote the submatrix of $A$ of size $i \times n$ whose first $i$ rows are those of $A$ and $b_i$ is the the starting subcolumn of $b$ of  length $i$. Then
 $\max\{ cx \mid A_i x \le b_i \} $ is an LP-problem which is called the  {\em $i$-approximate} of the LSIP-problem  $\sup \{ cx \mid Ax \le b  \}$.

 Let  $M_i = \max\{ cx \mid A_i x \le b_i \}$ denote the optimal value of the $i$-approximate LP-problem  $\max\{ cx \mid A_i x \le b_i \}$ and $M$ is the number $\sup \{ cx \mid Ax \le b    \}$. Clearly, for every $i$, $M_i \ge M_{i+1}  \ge M$.    Note that in general  $\lim_{i \to \infty} M_i \ne M$, see  \cite{LSIP1, LSIP2}.

Similarly  to  \cite{LSIP1}, \cite{LSIP2},  \cite{LSIP3},  we say that
if $\sup \{ cx \mid Ax \le b   \} $ is an LSIP-problem as above, then the problem
$$
\inf \{ b^{\top}y \mid A^{\top}y = c^{\top}, y\ge 0  \} ,
$$
where  $y = (y_1, y_2, \dots )^{\top}$ is an infinite vector whose set of nonzero components is finite, is called the {\em dual problem} of  $\sup \{ cx \mid Ax \le b  \}$.

For later reference, we state an analogue of Theorem~A for linear semi-infinite programming which, in fact, is an easy corollary of Theorem~A.

\begin{TB}  Suppose that $\sup \{ cx \mid Ax \le b    \}$ is an LSIP-problem
whose set $\{ cx \mid Ax \le b    \}$ is nonempty and bounded from above and whose dual problem is  $\inf \{ b^{\top}y \mid A^{\top}y = c^{\top}, y\ge 0  \} $.  Then
 \begin{equation}\label{dtt}
\sup \{ cx \mid Ax \le b    \}  \le   \inf \{ b^{\top}y \mid A^{\top}y = c^{\top}, y\ge 0  \}
\end{equation}
and the equality holds  if and only if
$\sup \{ cx \mid Ax \le b    \}$ is equal to  $\lim_{i \to \infty} M_i$,
where $M_i := \max\{ cx \mid A_i x \le b_i \}$ is the optimal solution of  the
$i$-approximate LP-problem $\max\{ cx \mid A_i x \le b_i \}$
of the primal LSIP-problem  $\sup \{ cx \mid Ax \le b  \}$.
\end{TB}

In the situation when the inequality \eqref{dtt} is strict,  the difference
$\inf \{ b^{\top}y \mid A^{\top}y = c^{\top}, y\ge 0  \} - \sup \{ cx \mid Ax \le b    \} >0$
is called the {\em duality gap} of the LSIP-problem $\sup \{ cx \mid Ax \le b \}$.

We will also need a corollary of the complementary slackness and Carath\'eodory theorem, see  \cite[Corollary 7.11]{S86}.

\begin{TC} If both optima in \eqref{dt} are finite, then the minimum is attained at a vector $\widehat y$, $\widehat y \ge 0$, whose positive components correspond to linear independent  columns of $A^{\top}$, i.e., rows of $A$.
\end{TC}

We now consider the problem of maximizing the objective linear functional $cx :=  -x_s$   over all rational vectors $x$, $x \in \mathbb Q^{n'}$, for a suitable $n'$, subject to the system of  linear inequalities $\operatorname{\textsf{SLI}}[Y_1]$, as an LP-problem $\max \{ cx \mid Ax \le b  \}$.
Note that, in this context, $m' = m_\operatorname{\textsf{inq}}$ and $n' = n_\operatorname{\textsf{inq}}$, where $m_\operatorname{\textsf{inq}}$ is the number of inequalities in $\operatorname{\textsf{SLI}}[Y_1]$
and $n_\operatorname{\textsf{inq}}$ is the number of all variables $x_{j, A}, x_s$ in $\operatorname{\textsf{SLI}}[Y_1]$.

We now consider the problem of maximizing the objective linear functional
$$cx :=  -x_s$$   over all rational vectors $x$, $x \in \mathbb Q^{n}$ for a suitable $n$, subject to the system of linear inequalities $\operatorname{\textsf{SLI}}[Y_1]$ introduced in Sect.~3, as an LSIP-problem $\sup \{ cx \mid Ax \le b  \}$.

We also consider a subsequence of $i_d$-approximate LP-problems $\max \{ cx \mid A_{i_d} x \le b_{i_d}  \} $ of the
LSIP-problem $\sup \{ cx \mid Ax \le b  \}$
 whose systems $A_{i_d}x  \le b_{i_d}$ of inequalities are finite subsystems  $\operatorname{\textsf{SLI}}_d[Y_1]$ of  $\operatorname{\textsf{SLI}}[Y_1]$, where  $d = 3,4, \dots$, as defined in \eqref{slid}.

It is straightforward to verify that the dual problem
$\inf \{ b^{\top} y  \mid  A^{\top} y = c^{\top}, y\ge 0 \}$ of this LSIP-problem $\sup \{ cx \mid Ax \le b  \}$   can be equivalently stated as follows
\begin{equation}\label{dlpp}
\sum_{j=1}^{\infty } y_j q_j^R \to \inf  \quad  \mbox{subject to} \quad   y \ge 0 , \  \
\sum_{j=1}^{\infty } y_j q_j^L = - x_s ,
\end{equation}
where almost all  $y_j$, $j=1,2,\dots$,  are 0's.  We rewrite
\eqref{dlpp} in the form
\begin{equation}\label{dlp}
\inf \{ \sum_{j=1}^{\infty} y_j q_j^R \mid  \sum_{j=1}^{\infty} y_j q_j^L = - x_s , y \ge 0  \} .
\end{equation}

Analogously, the dual problem  of the  $i_d$-approximate LP-problem $\max \{ cx \mid A_{i_d} x \le b_{i_d}  \}$ can be stated in the form
\begin{equation*}
\sum_{j=1}^{i_d } y_j q_j^R \to \min  \quad   \mbox{subject to} \quad  y \ge 0,  \
\sum_{j=1}^{i_d} y_j q_j^L = - x_s
\end{equation*}
which  we will write as
\begin{equation}\label{dlpd}
\min  \{ \sum_{j=1}^{i_d} y_j q_j^R \mid  \sum_{j=1}^{i_d} y_j q_j^L = - x_s , y \ge 0  \} .
\end{equation}

In Lemma~\ref{lem3},  we established the existence of a surjective function
$$
\operatorname{\textsf{inq}}  :   Y_2 \to  \operatorname{\textsf{inq}}(Y_2)
$$
from the set of finite irreducible ${\mathcal{A}}$-graphs $Y_2$ with property (Bd) to a certain set of combinations with  repetitions of $\operatorname{\textsf{SLI}}_d[Y_1]$.
Now we will relate these combinations
with repetitions of $\operatorname{\textsf{SLI}}_d[Y_1]$ to solutions of the dual LP-problem \eqref{dlpd}.

Consider a  combination
 with repetitions $Q$ of $\operatorname{\textsf{SLI}}_d[Y_1]$ that has the property
 \begin{equation}\label{st1}
\sum_{q \in Q} q^L = -C(Q) x_s     ,
\end{equation}
where $C(Q) >0$ is an integer. As before, let all inequalities of $\operatorname{\textsf{SLI}}_d[Y_1]$ be $q_1, \dots, q_{i_{d}}$, let  $\ell_j(Q)$ denote the number of times that $q_j$ occurs in $Q$, and let $\delta_j$ be the coefficient of $x_s$ in $q_j$. Then it follows from the definitions and \eqref{st1} that
\begin{equation}\label{st2}
     \sum_{q \in Q} q^L =  \sum_{j=1 }^{i_d} \delta_j \ell_j(Q) x_s =  -C(Q) x_s    .
\end{equation}
Consider the map
\begin{equation}\label{dfyy}
\operatorname{\textsf{sol}}_d :  Q \to y_Q = ( y_{Q,1}, \dots,  y_{Q, i_{d}} )^{\top}   ,
\end{equation}
where
$y_{Q,j} :=  \tfrac{\ell_j(Q)}{C(Q)}$ for $j =1, \dots, i_{d}$.
It follows from the definitions that $y_Q$ is a rational vector, $y_Q \ge 0$ and, by \eqref{st2}, $y_Q$ satisfies the condition
$$
\sum_{j=1}^{i_{d}} y_{{Q,j}} q_j^L  = - x_s    .
$$
Hence, $y_Q$ is a vector in the polyhedron $\{ y \mid y \ge 0,
\sum_{j=1}^{i_{d}} y_{j} q_j^L   = - x_s \} $ of the dual  LP-problem
\eqref{dlpd}.

Conversely, let ${\widehat} y = ({\widehat} y_1, \dots, {\widehat} y_{i_{d}})^{\top}$ be a vector of the feasible polyhedron
$$
\{ y \mid y \ge 0, \sum_{j=1}^{i_{d}} y_{j} q_j^L   = - x_s \}
$$
of the dual  LP-problem \eqref{dlpd}. Let $C >0 $ be a common multiple of positive denominators
of the rational numbers
${\widehat} y_1, \dots, {\widehat} y_{i_{d}}$. Consider a combination with repetitions
$Q = Q({\widehat} y)$ of  $\operatorname{\textsf{SLI}}_d[Y_1]$ such that every $q_j$ of  $\operatorname{\textsf{SLI}}_d[Y_1]$ occurs
in $Q$ \  $C {\widehat} y_{j} = \ell_j$ many times. Then it follows from the definitions that
$$
\sum_{q \in Q} q^L =   \sum_{j=1}^{i_{d}} \ell_j q_j^L = \sum_{j=1}^{i_{d}} C {\widehat} y_{j}  q_j^L  =  C \sum_{j=1}^{i_{d}} {\widehat} y_{j}  q_j^L  = - C x_s  .
$$
Now we can see that the vector $y_Q = \operatorname{\textsf{sol}}_d(Q({\widehat} y))$,  defined by \eqref{dfyy} for $Q$, is equal to $ {\widehat} y$, i.e., $\operatorname{\textsf{sol}}_d(Q({\widehat} y)) = {\widehat} y$.

We now summarize our findings.

\begin{lem}\label{lem4}
The map $\operatorname{\textsf{sol}}_d : Q \to y_Q$ defined by \eqref{dfyy} is a
surjective function from the set of combinations $Q$ with repetitions of $\operatorname{\textsf{SLI}}_d[Y_1]$ that
satisfy an equation $\sum_{q \in Q} q^L = -C x_s$, where $C >0$ is an integer, to the set of
vectors of  the rational  polyhedron $\{ y \mid y \ge 0,
\sum_{j=1}^{i_{d}} y_{j} q_j^L   = - x_s \} $  of the dual  LP-problem
\eqref{dlpd}. Furthermore, the composition of the maps
$\operatorname{\textsf{inq}}$ and $\operatorname{\textsf{sol}}_d$, \  $\operatorname{\textsf{sol}}_d  \circ \operatorname{\textsf{inq}} : Y_2 \to  {\widehat} y = {\widehat} y(Y_2)$,  provides a surjective function from the set of finite irreducible ${\mathcal{A}}$-graphs with property (Bd) to the set of points of the rational polyhedron $\{ y \mid y \ge 0,
\sum_{j=1}^{i_{d}} y_{j} q_j^L   = - x_s \} $ of  the dual  LP-problem
\eqref{dlpd}. Under this map, the value of the objective function
$
 \sum_{j=1}^{i_d}  {\widehat} y_{j}  q_j^R
$
of the  dual  LP-problem  \eqref{dlpd}  at ${\widehat} y = {\widehat} y(Y_2)$ satisfies
\begin{equation}\label{ps4}
\sum_{j=1 }^{i_d}    {\widehat} y_{j}  q_j^R    =   -\frac{{\bar {\mathrm{r}}} (\operatorname{core}(Y_1 \times  Y_2))}{{\bar {\mathrm{r}}} (Y_2) }     .
\end{equation}
\end{lem}

\begin{proof} As was observed above, $\operatorname{\textsf{sol}}_d$ is surjective, hence, by Lemma~\ref{lem3},  the composition $\operatorname{\textsf{sol}}_d  \circ \operatorname{\textsf{inq}}$ is also surjective.
Consider a finite irreducible ${\mathcal{A}}$-graph  $Y_2$ with property (Bd) and define $Q := \operatorname{\textsf{inq}}(Y_2)$, ${\widehat} y := \operatorname{\textsf{sol}}_d(Q)$. It follows from Lemma~\ref{lem3} that
\begin{equation}\label{eqq1}
\sum_{q \in Q} q^L = - 2{\bar {\mathrm{r}}} (Y_2) x_s  \quad \mbox{ and} \quad  \sum_{q \in Q} q^R = - 2{\bar {\mathrm{r}}} (\operatorname{core}(Y_1 \times  Y_2) ) .
\end{equation}
In view of the definition
\eqref{st2} and Lemma~\ref{lem3}, we have $C(Q ) = 2{\bar {\mathrm{r}}}(Y_2)$. Hence, it follows from the definition \eqref{dfyy} and  equalities  \eqref{eqq1}  that
\begin{equation*}
\sum_{j=1}^{i_d}  {\widehat} y_j q_j^R = ( \sum_{q \in Q } q^R  )
C(Q)^{-1} = -\frac {{\bar {\mathrm{r}}}( \operatorname{core}(Y_1 \times  Y_2) ) }{ {\bar {\mathrm{r}}}(Y_2 ) }   ,
 \end{equation*}
as required.
\end{proof}

We will say that a real number  $\sigma(Y_1)\ge 0$ is the  {\em Walter Neumann
coefficient}, or briefly WN-coefficient,
for $Y_1$ if
\begin{equation*}
{\bar {\mathrm{r}}} ( \operatorname{core}(Y_1 \times  Y_2) )  \le  \sigma(Y_1)  {\bar {\mathrm{r}}}(Y_1) {\bar {\mathrm{r}}}(Y_2)
\end{equation*}
for every  finite irreducible ${\mathcal{A}}$-graph $Y_2$  with property (B) and  $\sigma(Y_1)$ is minimal with this property.

We also consider the  WN$_d$-coefficient   $\sigma_d(Y_1)$, where $d \ge 3$ is an integer, for  $Y_1$ defined so that
\begin{equation*}
{\bar {\mathrm{r}}} ( \operatorname{core}(Y_1 \times  Y_2) )  \le  \sigma_d(Y_1)  {\bar {\mathrm{r}}}(Y_1) {\bar {\mathrm{r}}}(Y_2)
\end{equation*}
for every finite irreducible ${\mathcal{A}}$-graph $Y_2$ with property (Bd) and $\sigma_d(Y_1)$ is minimal with this property. It is clear from the definitions that  $\sigma_d(Y_1) \le \sigma_{d+1}(Y_1) \le \sigma(Y_1)$ for every $d =3,4,\dots$, that  $\sigma(Y_1)  = \sup \frac {{\bar {\mathrm{r}}}(  \operatorname{core}(Y_1 \times  Y_2) ) }{{\bar {\mathrm{r}}}(Y_1 ) {\bar {\mathrm{r}}}(Y_2 ) } $ over all  finite  irreducible ${\mathcal{A}}$-graphs $Y_2$ with property (B)  and that $\sigma_d(Y_1)   = \sup \frac {{\bar {\mathrm{r}}}(\operatorname{core}(Y_1 \times  Y'_2) ) }{{\bar {\mathrm{r}}}(Y_1 ) {\bar {\mathrm{r}}}(Y'_2 ) }$ over all  finite irreducible ${\mathcal{A}}$-graphs $Y'_2$  with property (Bd).

\begin{lem}\label{lem5} Suppose $d \ge \max \{ \deg u \mid u \in V_S Y_1  \} = \deg Y_1$ is an integer. Then both optima $\max \{ - x_s \mid \operatorname{\textsf{SLI}}_d[Y_1]\}$ and $\min   \{  \sum_{j=1}^{i_d}   y_j q_j^R   \mid y \ge 0 , \
\sum_{j=1}^{i_d}   y_j q_j^L  = - x_s \}$ are finite and satisfy the following inequalities and equalities
\begin{multline}
\label{st3}
- 2 \tfrac{q^*}{q^*-2} {\bar {\mathrm{r}}}(Y_1) \le \sup \{ - x_s \mid \operatorname{\textsf{SLI}} [Y_1]\}    \le   \max \{ - x_s \mid \operatorname{\textsf{SLI}}_d[Y_1]\}  = \\  = \min \{ \sum_{j=1}^{i_d}
 y_j q_j^R   \mid y \ge 0 , \ \sum_{j=1}^{i_d}   y_j q_j^L  = - x_s \} =
 - \sigma_d(Y_1) {\bar {\mathrm{r}}}(Y_1)  .
\end{multline}
Furthermore, the minimum is attained at a vector ${\widehat} y(d)$, ${\widehat} y(d) \ge 0$, which is a vertex of the polyhedron  $\{  y \mid y \ge 0 , \ \sum_{j=1}^{i_d}   y_j q_j^L  = - x_s \}$  and
\begin{multline}
\label{st3a}
\inf   \{  \sum_{j=1}^{\infty}   y_j q_j^R   \mid y \ge 0 , \
\sum_{j=1}^{\infty}   y_j q_j^L  = - x_s \} =  - \sigma(Y_1) {\bar {\mathrm{r}}}(Y_1) \\ \le
\min   \{  \sum_{j=1}^{i_d}   y_j q_j^R   \mid y \ge 0 , \
\sum_{j=1}^{i_d}   y_j q_j^L  = - x_s \} = - \sigma_d(Y_1)  {\bar {\mathrm{r}}}(Y_1) .
\end{multline}
In particular,
$\sigma_d(Y_1) \le \sigma(Y_1) \le 2 \tfrac{q^*}{q^*-2}$.
\end{lem}

\begin{proof} Setting $Y_2 := Y_1$, we get a solution ${\widehat} y =  \operatorname{\textsf{sol}}_d ( \operatorname{\textsf{inq}}  (Y_2))$ to the system  $ y \ge 0$, \ $\sum_{j=1}^{\infty}   y_j q_j^L  = - x_s$.  Hence,  both sets
$$
\{  y   \mid y \ge 0 , \sum_{j=1}^{\infty}  y_j q_j^L  = - x_s \} ,   \quad \{  y   \mid y \ge 0 , \sum_{j=1}^{i_d}  y_j q_j^L  = - x_s \}
$$
are  not empty.

To see that the sets  $ \{ x \mid \operatorname{\textsf{SLI}}[Y_1]\}$,
$\{  x \mid \operatorname{\textsf{SLI}}_d[Y_1]\}$  are not empty either,  we will show that ${\widehat} x$, whose components are
 ${\widehat} x_A = 0$ for every $A \subseteq V_P Y_1$ and ${\widehat}  x_{s} = 2\tfrac{q^*}{q^*-2} {\bar {\mathrm{r}}}(Y_1)$, is a solution both to $\operatorname{\textsf{SLI}}_d[Y_1]$ and  to $\operatorname{\textsf{SLI}}[Y_1]$. To do this,  we will check that every inequality of $\operatorname{\textsf{SLI}}[Y_1]$ is satisfied with these values of variables, that is,
 \begin{align}\label{dd1}
 -(k-2)\cdot 2\tfrac{q^*}{q^*-2} {\bar {\mathrm{r}}}(Y_1) \le -   N_{\alpha}(\Omega_T)
\end{align}
 for every ${\alpha}$-admissible function
  $\Omega_T : T \to S_1( V_P Y_1 )$, where $T \in S_2(G_{\alpha})$, $|T| = k$.   Let
 $T = \{ a_1, \dots, a_k\}$, $k \ge 2$, $a_i \in G_{\alpha}$,
  and $\Omega_T(a_i) = A_i$, $i =1,\dots, k$.
Consider a secondary vertex $u$ of $Y_1$, suppose $\deg u = \ell$ and let
$e_1, \dots, e_{\ell}$ be all edges of $Y_1$ with
 $u = (e_1)_+ = \dots = (e_\ell)_+$.
  Denote $B := \{   {\varphi}(e_1),  \dots,  {\varphi}(e_\ell) \}$.
   It is not difficult to see from the  definition  \eqref{Nal} of  $N_{\alpha}({\Omega}_T)$ that  the contribution to the sum $N_{\alpha}(\Omega_T)$  made by those equivalence classes  that are associated with the vertex  $u \in V_S Y_1$ does not exceed
   $   \sum_{g \in G_{\alpha}}  \max( | T \cap  B g | -2 ,  0)$. Hence, it follows from the definition of $\tfrac{q^*}{q^*-2}$ and  from results of Dicks and the author \cite[Corollary 3.5]{DIv}  that
\begin{align}\label{dd2}
\sum_{g \in G_{\alpha}} \max(|T \cap Bg | -2, 0)  \le \tfrac{q^*}{q^*-2}(|T|-2)(|B|-2) =
\tfrac{q^*}{q^*-2}(k-2)(\ell-2) .
\end{align}
Therefore, summing up  inequalities \eqref{dd2} over all vertices $u \in V_S Y_1$, we arrive at
\begin{align*}
N_{\alpha}(\Omega)   \le \tfrac{q^*}{q^*-2} (k-2) \cdot 2 {\bar {\mathrm{r}}}_{\alpha} (Y_1)  \le \tfrac{q^*}{q^*-2} (k-2) \cdot 2 {\bar {\mathrm{r}}} (Y_1)    ,
\end{align*}
where  $2 {\bar {\mathrm{r}}}_{\alpha} (Y_1)$ is the sum $\sum (\deg u -2)$ over all  $u \in V_S Y_2$ of type ${\alpha}$.  This proves \eqref{dd1} and also shows that
\begin{align*}
-x_s \ge  -\tfrac{2q^*}{q^*-2}  {\bar {\mathrm{r}}} (Y_1)  .
\end{align*}

Thus, according to Theorem~A, the maximum and minimum in \eqref{st3}  are finite and equal. Inequalities in  \eqref{st3}  are shown.  Recall that, by Theorem~A,  the minimum  $\min   \{  \sum_{j=1}^{i_d}   y_j q_j^R   \mid y \ge 0 , \
\sum_{j=1}^{i_d}   y_j q_j^L  = - x_s \}$ is attained at a vertex  ${\widehat} y(d)$  of the polyhedron   $\{ y  \mid y \ge 0 , \
\sum_{j=1}^{i_d}   y_j q_j^L  = - x_s \}$.

By Lemma~\ref{lem4}, for every  irreducible ${\mathcal{A}}$-graph $Y_2$ with property
(Bd),  the ratio $-\frac {{\bar {\mathrm{r}}}(\operatorname{core}(Y_1 \times Y_2) ) }{ {\bar {\mathrm{r}}}(Y_2 ) }$
is $\sum_{j=1}^{i_d}  {\widehat} y_j q_j^R $, where ${\widehat} y = \operatorname{\textsf{sol}}_d(\operatorname{\textsf{inq}}(Y_2))$, and the  map
$$
 \operatorname{\textsf{sol}}_d \circ \operatorname{\textsf{inq}} : Y_2 \to    {\widehat} y
$$
covers the feasible polyhedron   $\{ y  \mid y \ge 0 , \ \sum_{j=1}^{i_d}   y_j q_j^L  = - x_s \}$.
Hence, the supremum
\begin{equation*}
\sigma_d(Y_1 )  {\bar {\mathrm{r}}}(Y_1 ) :=  \sup_{Y_2} \tfrac {{\bar {\mathrm{r}}}(  \operatorname{core}(Y_1 \times Y_2) ) }{ {\bar {\mathrm{r}}}(Y_2 ) }  =
- \inf_{Y_2}  \left( - \tfrac {{\bar {\mathrm{r}}}(  \operatorname{core}(Y_1 \times Y_2)  ) }{ {\bar {\mathrm{r}}}(Y_2 ) }  \right)
\end{equation*}
over all  graphs $Y_2$ with property (Bd) is equal to
\begin{align*}
  \sigma_d(Y_1 )  {\bar {\mathrm{r}}}(Y_1 ) &  =  -\inf \{  \sum_{j=1}^{i_d}   y_j q_j^R   \mid y \ge 0 , \ \sum_{j=1}^{i_d}
 y_j q_j^L = - x_s \}   \\
  & =  - \min \{  \sum_{j=1}^{i_d}   y_j q_j^R   \mid y \ge 0 , \ \sum_{j=1}^{i_d}
 y_j q_j^L = - x_s \} \\ &  = -   \sum_{j=1}^{i_d}  {\widehat} y_j(d) q_j^R
  =  - \max \{ - x_s \mid \operatorname{\textsf{SLI}}_d[Y_1]\}   ,
\end{align*}
where  ${\widehat} y(d)$ is a vertex of  the polyhedron   $\{ y  \mid y \ge 0 , \
\sum_{j=1}^{i_d}   y_j q_j^L  = - x_s \}$ at which the minimum is attained.

Since
$$
\inf \{  \sum_{j=1}^{\infty}   y_j q_j^R   \mid y \ge 0 , \
\sum_{j=1}^{\infty}   y_j q_j^L  = - x_s \} =\inf_d  \min   \{  \sum_{j=1}^{i_d}   y_j q_j^R   \mid y \ge 0 , \
\sum_{j=1}^{i_d}   y_j q_j^L  = - x_s \}
$$
and $\sigma(Y_1) = \sup_d ( \sigma_d(Y_1)) $, the equalities and inequalities \eqref{st3a} follow from proven equalities and inequalities \eqref{st3}.
\end{proof}

\section{More Lemmas}

We now let ${\mathcal{F}} = \prod_{{\alpha} \in I}^* G_{\alpha}$ be  an arbitrary free product of nontrivial groups  $G_{\alpha}$,  ${\alpha} \in I$, $|I| >1$. Let $H$ be a finitely generated factor-free subgroup of  ${\mathcal{F}}$. As in Sect.~2, let $\Psi_o(H)$ denote an irreducible
${\mathcal{A}}$-graph  of $H$, where ${\mathcal{A}} =   \cup_{{\alpha} \in I} G_{\alpha}$,  with the base vertex $o$ and
let $\Psi(H)$ denote the core of  $\Psi_o(H)$.
Let $I(H)$ denote a subset of  the index set $I$ such that ${\alpha} \in I(H)$  if and only if  there is a secondary vertex $u \in V_S \Psi(H)$ of type ${\alpha}$. Since $H$ is finitely generated,  it follows that  $I(H)$ is finite.

We fix a finitely generated factor-free subgroup $H_1$ of  ${\mathcal{F}}$ with positive reduced rank
${\bar {\mathrm{r}}}(H_1) = -\chi(\Psi(H_1))$ $>0$.  We say that a  finitely generated factor-free subgroup $H_2$ of  ${\mathcal{F}}$ has  {\em property }(B), resp. {\em property } (Bd),  where $d \ge 3$ is an integer, {\em  relative} to $H_1$  if the core graph $\Psi(H_2)$ of $H_2$ has the original property (B), resp.  property (Bd), in which the graphs $Y_1, Y_2$ are replaced with $\Psi(H_1)$,  $\Psi(H_2)$, resp., i.e., ${\bar {\mathrm{r}}}(H_2) = -\chi(\Psi(H_2)) >0$, $\deg \Psi(H_2)  \le d$ in case of property (Bd), and  the map
$\tau_2 : \operatorname{core} (\Psi(H_1) \times \Psi(H_2)) \to  \Psi(H_2)$ is surjective.

\begin{lem}\label{lemBd} Suppose $H_2$ is a  finitely generated factor-free subgroup of  ${\mathcal{F}}$ such that
$\deg \Psi(H_2) \le d$,  where $d \ge 3$ is an integer or $d = \infty$,  ${\bar {\mathrm{r}}}(H_2) = -\chi(\Psi(H_2)) >0$, and the map  $\tau_2 :  \operatorname{core} (\Psi(H_1) \times \Psi(H_2)) \to  \Psi(H_2)$ is not surjective. Then there exists a finitely generated factor-free subgroup $H_4$ of  ${\mathcal{F}}$ with property (Bd) if $d < \infty$ or
with property (B) if $d = \infty$  such that
\begin{equation}\label{bol}
    \frac{{\bar {\mathrm{r}}}(H_1, H_4) }{  {\bar {\mathrm{r}}}(H_4) }  >  \frac{{\bar {\mathrm{r}}}(H_1, H_2) }{  {\bar {\mathrm{r}}}(H_2) }  . \end{equation}
\end{lem}

\begin{proof} Recall that  $ {\bar {\mathrm{r}}}(  H_1, H_2 )=   {\bar {\mathrm{r}}}( \operatorname{core} (\Psi(H_1) \times \Psi(H_2)) )$ and
$ {\bar {\mathrm{r}}}(  H_i) = {\bar {\mathrm{r}}}(\Psi(  H_i))$, $i=1,2$.
If ${\bar {\mathrm{r}}}( \operatorname{core} (\Psi(H_1) \times \Psi(H_2)) ) = 0$, then we may take $H_4 = H_1$ and the inequality \eqref{bol} holds.  Assume that ${\bar {\mathrm{r}}}( \operatorname{core} (\Psi(H_1) \times \Psi(H_2)) ) > 0$ and that the map
$$
\tau_2 : \operatorname{core} (\Psi(H_1) \times \Psi(H_2)) \to  \Psi(H_2)
$$
is not surjective. Consider the subgraph
$\Gamma := \tau_2 (\operatorname{core} (\Psi(H_1) \times \Psi(H_2)) ) $ of $\Psi(H_2)$. It follows from the definitions and assumptions that ${\bar {\mathrm{r}}}(\Gamma) < {\bar {\mathrm{r}}} (\Psi(H_2))$ and
$$
{\bar {\mathrm{r}}}(  \operatorname{core} (\Psi(H_1) \times \Gamma )  ) = {\bar {\mathrm{r}}}(  \operatorname{core} (\Psi(H_1) \times \Psi(H_2))  ) >0,
$$
whence $ {\bar {\mathrm{r}}}( \Gamma ) > 0$. It is also clear that $\operatorname{core}(\Gamma  ) = \Gamma $.
Therefore,
\begin{equation}\label{e52}
\frac{{\bar {\mathrm{r}}}( \operatorname{core} (  \Psi(H_1) \times \Gamma   ))  }{  {\bar {\mathrm{r}}}(\Gamma ) }  > \frac{{\bar {\mathrm{r}}}( \operatorname{core} (  \Psi(H_1) \times \Psi(H_2)   ))  }{  {\bar {\mathrm{r}}}(\Psi(H_2) ) } .
\end{equation}

Let $\Gamma_1, \dots, \Gamma_k$  be  connected components of the graph $\Gamma $.
Since
$$
{\bar {\mathrm{r}}}( \operatorname{core} (  \Psi(H_1) \times \Gamma )) >0 ,
$$ it follows  that  ${\bar {\mathrm{r}}}( \Gamma ) >0$.
Note that the graph
$
\operatorname{core} (  \Psi(H_1) \times \Gamma )
$
consists of disjoint graphs $\operatorname{core} (  \Psi(H_1) \times \Gamma_j )$, $j=1, \dots, k$.
In particular,
$$
{\bar {\mathrm{r}}}(\Gamma) = \sum_{j=1}^k {\bar {\mathrm{r}}}(\Gamma_j) , \qquad   {\bar {\mathrm{r}}}( \operatorname{core} (  \Psi(H_1) \times \Gamma )) = \sum_{j=1}^k {\bar {\mathrm{r}}}( \operatorname{core} (  \Psi(H_1) \times \Gamma_j )) ,
$$
 hence,
\begin{equation}\label{e53}
\frac{{\bar {\mathrm{r}}}( \operatorname{core} (  \Psi(H_1) \times \Gamma   ))  }{  {\bar {\mathrm{r}}}(\Gamma ) } =  \frac{ \sum_{j=1}^k {\bar {\mathrm{r}}}( \operatorname{core} (  \Psi(H_1) \times \Gamma_j )) }{     \sum_{j=1}^k {\bar {\mathrm{r}}}(  \Gamma_j )      } .
\end{equation}

Note that if $ {\bar {\mathrm{r}}}(  \Gamma_j )  =0$ then $ {\bar {\mathrm{r}}}( \operatorname{core} (  \Psi(H_1) \times \Gamma_j )) =0$.
Let $\Gamma_{j^*}$ be chosen so that ${\bar {\mathrm{r}}}(\Gamma_{j^*}) >0$ and the ratio
$$
\frac{  {\bar {\mathrm{r}}}( \operatorname{core} ( \Psi(H_1) \times \Gamma_{j^*} )) } {  {\bar {\mathrm{r}}}( \Gamma_{j^*} )      }
$$
is maximal over those graphs $\Gamma_j$ with   $ {\bar {\mathrm{r}}}(  \Gamma_j )  >0$.  It follows from
${\bar {\mathrm{r}}}(\Gamma) = \sum_{j=1}^k {\bar {\mathrm{r}}}(\Gamma_j) >0$ that such $j^*$ does exist.  It is not difficult to see that
$$
 \frac{ \sum_{j=1}^k {\bar {\mathrm{r}}}( \operatorname{core} (  \Psi(H_1) \times \Gamma_j )) }{     \sum_{j=1}^k {\bar {\mathrm{r}}}(  \Gamma_j )      }
  \le     \frac{  {\bar {\mathrm{r}}}( \operatorname{core} ( \Psi(H_1) \times \Gamma_{j^*} )) } {  {\bar {\mathrm{r}}}( \Gamma_{j^*} )      } .
  $$
This, together with \eqref{e52} and \eqref{e53},  implies that
$$
 \frac{  {\bar {\mathrm{r}}}( \operatorname{core} ( \Psi(H_1) \times \Gamma_{j^*} )) } {  {\bar {\mathrm{r}}}( \Gamma_{j^*} )      }  \ge
  \frac{  {\bar {\mathrm{r}}}( \operatorname{core} ( \Psi(H_1) \times \Gamma)) } {  {\bar {\mathrm{r}}}( \Gamma ) } >
  \frac{  {\bar {\mathrm{r}}}( \operatorname{core} ( \Psi(H_1) \times \Psi(H_2))) } {  {\bar {\mathrm{r}}}( \Psi(H_2) ) } .
  $$

Hence, picking an arbitrary
primary vertex $v \in V_P \Gamma_{j^*}$ in $\Gamma_{j^*}$ as a base point, and letting $H_4 := H(\Gamma_{j^*, v})$, as in Lemma~\ref{Lm2}, we obtain a subgroup $H_4$ with the desired inequality \eqref{bol}.
\end{proof}

\begin{lem}\label{lemsup} The supremum  $\sup \frac{ {\bar {\mathrm{r}}}(   H_1, H_3   )  }{  {\bar {\mathrm{r}}}(H_3 ) }$
over all finitely generated factor-free subgroups $H_3$ of  ${\mathcal{F}}$ with  ${\bar {\mathrm{r}}}(H_3 ) >0$ and $\deg  \Psi(H_3) \le d$,
where $d \ge 3$ is an integer or $d = \infty$,
is equal to  $\sup \frac{ {\bar {\mathrm{r}}}(   H_1, H_2   )  }{  {\bar {\mathrm{r}}}(H_2 ) }$
over all finitely generated factor-free subgroups $H_2$ of  ${\mathcal{F}}$ that have  property (Bd) (or property (B) if $d = \infty$) relative to $H_1$ and satisfy the condition $I(H_2) \subseteq I(H_1)$. In particular,
$$
\sigma_d(H_1){\bar {\mathrm{r}}}(H_1 )   = \sigma_d( \Psi(H_1)){\bar {\mathrm{r}}}(\Psi(H_1)) \quad \mbox{and}  \quad
\sigma(H_1){\bar {\mathrm{r}}}(H_1) =  \sigma( \Psi(H_1)) {\bar {\mathrm{r}}}(\Psi(H_1)).
$$
\end{lem}

\begin{proof}  This first claim follows from Lemma~\ref{lemBd} and the observation that
if the map  $\tau_2 :  \operatorname{core} (\Psi(H_1) \times \Psi(H_2)) \to  \Psi(H_2)$ is surjective  then $I(H_2) \subseteq I(H_1)$. The equalities follow from the first claim, definitions of numbers  $\sigma_d(H_1)$, $\sigma(H_1)$,
 $\sigma_d(\Psi(H_1))$, $\sigma(\Psi(H_1))$,  and  Lemma~\ref{lemBd}.
\end{proof}

In view of  Lemma~\ref{lemsup}, when investigating
 $$\sup \frac{ {\bar {\mathrm{r}}}(   H_1, H_3  )  }{  {\bar {\mathrm{r}}}(H_3 ) }$$
over all finitely generated factor-free subgroups $H_3$ of  ${\mathcal{F}}$ with
${\bar {\mathrm{r}}}(H_3 ) >0$  and $\deg \Psi(H_3)  \le d$,  we may assume that the index set $I$
is finite, i.e., $I= I(H_1)$,  say, $I = \{ 1, \dots, m\}$,
and so  ${\mathcal{F}}= G_1  *  G_2  * \ldots * G_{m}$.

Furthermore, in order to be able to make use of results of Sects.~3--4,
we consider ${\mathcal{F}}$ as the following  free product
$$
{\mathcal{F}}_2(1) =   G_1 *  G(2, m)
$$
of two groups $G_1$ and  $G(2, m) := G_2  * \ldots * G_{m}$. Let $g_{\alpha} \in G_{\alpha}$ be some nontrivial element of  $G_{\alpha}$, ${\alpha} \in I = \{ 1, \dots, m \}$. For every  $a_{\alpha} \in G_{\alpha}$,  consider the map
\begin{equation}\label{map2}
a_{\alpha} \to
  (g_{{\alpha}+1} \ldots  g_{m}  g_{1} \ldots  g_{\alpha})^{-1}  a_{\alpha}   g_{{\alpha} +1} \ldots  g_{m}  g_{1} \ldots   g_{\alpha}   ,
\end{equation}
where $g_{{\alpha}+1 }  \ldots   g_{m}  g_{1} \ldots  g_{\alpha}$ is a cyclic permutation of the
word $g_{1}  g_{2}  \ldots g_{m}$.

Recall that a subgroup $K$ of a group $G$ is called {\em antinormal}
if, for every $g \in G$, $g K g^{-1} \cap K \ne \{ 1\}$ implies $g \in K$.

\begin{lem}\label{lemmap2} Let $|I| = m \ge 3$ and let $H_1$ be a finitely generated factor-free subgroup of ${\mathcal{F}}$.  Then the map \eqref{map2} extends to monomorphisms
 $$
 \mu : {\mathcal{F}} \to {\mathcal{F}} , \qquad \mu_2 : {\mathcal{F}} \to  {\mathcal{F}}_2(1)
 $$
 that have the following properties.

 $(a)$ A word $U \in {\mathcal{F}}$ with $|U| >1$ is cyclically reduced if and only if $\mu(U)$ is cyclically reduced.

 $(b)$  The subgroups $\mu_2({\mathcal{F}} )$,   $\mu({\mathcal{F}} )$ are  antinormal in ${\mathcal{F}}_2(1)$,  ${\mathcal{F}}$, resp..

 $(c)$   $\mu_2(H_1)$ is a factor-free subgroup of ${\mathcal{F}}_2(1)$ and $\mu(H_1)$ is factor-free in ${\mathcal{F}}$.  Furthermore, $\deg \Psi(H_1) = \deg  \Psi(\mu_2(H_1))$.

 $(d)$   If $K_1, K_2$ are  finitely generated  factor-free subgroups of ${\mathcal{F}}$, then   ${\bar {\mathrm{r}}}(K_1, K_2)   =    {\bar {\mathrm{r}}}(  \mu_2(K_1) , \mu_2(K_2) )$.

$(e)$   The supremum  $\sup \frac{ {\bar {\mathrm{r}}}(  H_1, H_2   )  }{  {\bar {\mathrm{r}}}(H_2 ) }$
over all finitely generated factor-free subgroups $H_2$ of ${\mathcal{F}}$
such that  ${\bar {\mathrm{r}}}(H_2) > 0$  and $ \deg  \Psi(H_2) \le d$,  where $d \ge 3$ is an integer,  does not exceed  $\sup \frac{ {\bar {\mathrm{r}}}(  \mu_2( H_1), K_2   )  }{  {\bar {\mathrm{r}}}(K_2 ) }$  over all finitely generated factor-free subgroups $K_2$ of  ${\mathcal{F}}_2(1)$ with property (Bd) relative to $\mu_2( H_1)$.
In particular, $\sigma_d(H_1) \le  \sigma_d( \mu_2( H_1))$ and   $\sigma(H_1) \le  \sigma( \mu_2( H_1))$.
\end{lem}

\begin{proof}
It is clear that the map \eqref{map2} extends to homomorphisms
 $$
 \mu : {\mathcal{F}} \to {\mathcal{F}} , \quad \mu_2 :  {\mathcal{F}} \to {\mathcal{F}}_2(1) .
 $$
Note that if $a_1 \in G_{{\alpha}_1}$ and $a_2 \in G_{{\alpha}_2}$ are nontrivial elements and ${\alpha}_1 \ne {\alpha}_2$, then
$\mu(a_1) \mu(a_2) $ is a cyclically reduced word. This remark implies that the kernels of the maps $\mu, \mu_2$ are trivial, whence $\mu, \mu_2$  are monomorphisms.
\smallskip

(a) It follows from the foregoing remark that  a word $U \in {\mathcal{F}}$ with $|U| >1$ is cyclically reduced if and only if $\mu(U)$ is cyclically reduced.
\smallskip

(b)  Let $U_1, U_2 \in {\mathcal{F}}$ be reduced words and
$W \mu(U_1) W^{-1} = \mu(U_2)$ in ${\mathcal{F}}$. Using induction on $|U_1| + |U_2|$, we will prove that $W \in \mu({\mathcal{F}})$. Suppose $U_1$ is not cyclically reduced and
$U_1 \equiv a_1 U_3 a_2$, where $a_1, a_2 \in G_{\alpha} \setminus \{ 1\}$ are letters of $U_1$. Then we can replace $U_1$ with $U_1' := U_3 a_3$, where $a_3 \in G_{\alpha}$,    $a_3 = a_2 a_1$ in $G_{\alpha}$ if  $a_3 \ne 1$ or with  $U_1' := U_3$ if $a_3 = 1$, and we replace $W$ with $W' := W \mu_2(a_1)$. This way we obtain an equality
$$
W' \mu(U_1') (W')^{-1} \overset 0 =  \mu(U_2)
$$
in ${\mathcal{F}}$ in which $|U_1'|+ |U_2| < |U_1| + |U_2|$. Hence, it follows from the induction hypothesis that $W \in \mu({\mathcal{F}})$, as required. If  $U_2$ is not cyclically reduced, then, analogously to what we did above for $U_1$, we can decrease the sum  $|U_1| + |U_2|$ and use the induction hypothesis.

Thus we may assume that both words $U_1, U_2$ are
cyclically reduced.  By part (a), the words $\mu(U_1)$, $\mu(U_2)$ are also cyclically reduced.  Observe that if $W V_1 W^{-1} = V_2$  in ${\mathcal{F}}$, where
$V_1, V_2$ are cyclically reduced and $W$ is reduced, then $V_2$ is a cyclic permutation of $V_1$. More specifically, there is a factorization $V_1 \equiv V_{11}V_{12}$ and an integer $k$ such that if $k \ge 0$ then $W \equiv V_{12}V_{1}^k$ and if  $k \le 0$ then $W \equiv V_{11}^{-1}V_{1}^k$. In either case, $V_2 \equiv V_{12} V_{11}$. Applying this observation to the equality
$$
W \mu(U_1) W^{-1} \overset 0 = \mu(U_2)
$$
in ${\mathcal{F}}$, we can see from \eqref{map2}, when $m \ge 3$,  that a cyclic permutation  of  $\mu(U_1)$ equal to $\mu(U_2)$ must have the form
$\mu(\bar U_1)$, where $\bar U_1$  is a cyclic permutation of $U_1$. For similar reasons, $W \equiv \mu(V)$  for some $V \in {\mathcal{F}}$ and part (b) is proven for the subgroup $\mu({\mathcal{F}})$. It now follows that $\mu_2({\mathcal{F}})$  is also antinormal in ${\mathcal{F}}_2(1)$.
\smallskip

(c) Arguing on the contrary, suppose $H$ is a factor-free subgroup of ${\mathcal{F}}$ and one of $\mu(H)$, $\mu_2(H)$ is not factor-free in ${\mathcal{F}}$, ${\mathcal{F}}_2(1)$, resp.  Then it follows from the definitions that  $\mu_2(H)$ is not factor-free in ${\mathcal{F}}_2(1)$. Hence, there is a reduced word $U$
such that $U$ is not conjugate in ${\mathcal{F}}$ to a word of length $\le 1$ and
\begin{equation}\label{e54}
    \mu(U) \overset 0 = WVW^{-1}
\end{equation}
in ${\mathcal{F}}$, where $W$ is either empty or reduced and $V$ is either a letter of $G_1 \setminus \{ 1\}$ or $V$ is
a reduced word with no letters of $G_1$. Thus, $V$ is reduced and either $V \in G_1$ or $V \in G(2,m)$.

Assume that the word $U$ in \eqref{e54} is not cyclically reduced. Then $U \equiv a_1 U_1 a_2$, where $a_1, a_2 \in G_{\alpha} \setminus \{ 1\}$ are letters of $U$. If $a_1 a_2 = a_3$ in $G_{\alpha}$   and $a_3 \in G_{\alpha} \setminus \{ 1\}$, then the word $U' \equiv U_1 a_3$, similarly to $U$, is not conjugate to ${\mathcal{F}}$ to a word of length $\le 1$ and $\mu(U')$, being conjugate to $\mu(U)$ in ${\mathcal{F}}$, has a representation of the form \eqref{e54}, so $U$ can be replaced with
$U'$.  If $a_1 a_2 = 1$ in $G_{\alpha}$, then the word $U_1$ can be taken as $U$. Hence, by induction on $|U|$, we may assume that $U$ is cyclically reduced.

If the word  $WVW^{-1}$  in \eqref{e54} is not reduced, then there are words $W'$, $V'$ such that
$$
\mu(U) \overset 0 = W' V' (W')^{-1} ,
$$
$W'$, $V'$ have the foregoing properties of $W$, $V$, resp., and $2|W'| + |V'| < 2|W| + |V|$. Indeed, if, say
$W \equiv W_1 a_1$ and $V \equiv a_2 V_1$, where $a_1, a_2 \in G_{\alpha} \setminus \{ 1\}$, then we set $W' := W_1$ and $V'$ is a reduced word equal in ${\mathcal{F}}$  to $a_1 a_2 V_1 a_1^{-1}$. Note that $W'$, $V'$ have the foregoing properties of $W$, $V$, resp., and   $|W'| =|W|-1$,   $|V'| \le |V|+1$, whence  $2|W'| + |V'| < 2|W| + |V|$. Thus, by induction on
$2|W| + |V|$, we may assume that the word $W V W^{-1}$ in \eqref{e54} is  reduced.

Since $U$ is cyclically reduced and $|U| >1$, it follows from part (a) that $\mu(U)$ is cyclically reduced. Hence, the word $W$ is empty and  $\mu(U) \equiv V$, where $V$ is a single letter of $G_1 \setminus \{ 1\}$ or $V$ has no letters of $G_1$. However, neither situation is possible by the definition  \eqref{map2}. This contradiction completes the proof of the first statement of part (c).

Now we will prove the equality  $\deg \Psi(H_1) = \deg  \Psi(\mu_2(H_1))$ of part (c).
It follows from the definition  \eqref{map2} that the graph $\Psi(\mu_2(H))$
can be visualized as  a graph obtained from $\Psi(H)$ by subdivision of edges  of $\Psi(H)$ into paths in accordance with formula  \eqref{map2}  and subsequent ``mergers" of edges that have labels in $G_2 \cup \dots \cup G_{m}$. In particular, for every
vertex $v \in V \Psi(H)$ with   $\deg v >2$ there will be a unique vertex $u = u(v) \in V_S \Psi(\mu_2(H))$  of degree $\deg u = \deg v$ and this map $v \to u(v)$ is bijective on  the sets of all vertices of $\Psi(H)$, $\Psi(\mu_2(H))$    of  degree $> 2$. Hence, the maximal degree of vertices of  $\Psi(\mu_2(H))$ is equal to that of $\Psi(H)$, as claimed.
\smallskip

(d) By part (c), the subgroups  $\mu_2(K_1), \mu_2(K_2)$ of ${\mathcal{F}}_2(1)$ are  factor-free
and the subgroups  $\mu(K_1)$, $\mu(K_2)$  of ${\mathcal{F}}$ are also factor-free.
Let $T(\mu_2(K_1), \mu_2(K_2) )$ be a  set of representatives of those double cosets
$\mu_2(K_1) U \mu_2(K_2)$ of ${\mathcal{F}}_2(1)$, $U \in {\mathcal{F}}_2(1)$, that have the property
$\mu_2(K_1) \cap U \mu_2(K_2)U^{-1} \ne  \{ 1 \}$. If $T \in T(\mu_2(K_1), \mu_2(K_2) )$, then it follows from the definition    of   $T(\mu_2(K_1), \mu_2(K_2))$
that there are nontrivial
$V_i \in K_i$, $i =1,2$, such that $T \mu_2(V_2) T^{-1} = \mu_2(V_1) \ne 1$ in ${\mathcal{F}}_2(1)$.
By part (b), such an  equality  implies  $T \in \mu_2({\mathcal{F}})$ (note $\mu_2$ could be replaced with $\mu$).   Now we can see that
there is a set $S(K_1, K_2) \subseteq {\mathcal{F}}$ such that $\mu_2(S(K_1, K_2)) = T(\mu_2(K_1), \mu_2(K_2) )$ and $S(K_1, K_2) $ is   a set of representatives of those double cosets $K_1 S K_2$ of ${\mathcal{F}}$, $S \in {\mathcal{F}}$, that have the property $K_1 \cap S K_2 S^{-1} \ne  \{ 1 \}$. Therefore,
$$
{\bar {\mathrm{r}}}(K_1, K_2)   := \sum_{S \in S(K_1, K_2)}  {\bar {\mathrm{r}}}( K_1 \cap S K_2 S^{-1}) =
{\bar {\mathrm{r}}}(  \mu_2(K_1) , \mu_2(K_2) )  ,
$$
as desired.
\smallskip

(e)  This follows from  Lemma~\ref{lemsup}, parts (c)--(d) and definitions.
\end{proof}

\section{Proofs of Theorems}

For reader's convenience, we restate Theorems~1.1--1.3 before proving them.

\begin{T1} Suppose that ${\mathcal{F}} =G_1 * G_2$ is the free product of two nontrivial groups $G_1,  G_2$
and  $H_1$ is a  finitely generated factor-free noncyclic subgroup of ${\mathcal{F}}$. Then the following claims are true.

$\rm{(a)}$ For every integer $d \ge 3$,  there exists a linear programming problem (LP-problem)
\begin{equation*}\tag{1.8}
{\mathcal{P}}(H_1, d) = \max\{ c(d)x(d) \mid A(d)x(d) \le b(d)  \}
\end{equation*}
with integer coefficients whose solution is equal to $-\sigma_d(H_1) {\bar {\mathrm{r}}} (H_1)$.

$\rm{(b)}$  There is a finitely generated factor-free subgroup $H_2^* $ of ${\mathcal{F}}$, $H_2^*= H_2^*(H_1)$,   which corresponds to  a vertex solution of the dual problem
$$
{\mathcal{P}}^*(H_1, d) = \min \{ b(d)^{\top}  y(d)  \mid A(d)^{\top}y(d) = c(d)^{\top} , \, y(d) \ge 0  \}
$$
of the primal LP-problem  \eqref{lpa} of part (a) such that   $\bar {\mathrm{r}}(H_1, H_2^*)  =  \sigma_d(H_1)  \bar {\mathrm{r}}(H_1) \bar {\mathrm{r}}( H_2^*)$. In particular,  the WN${}_d$-coefficient $\sigma_d(H_1)$ of $H_1$ is rational. Furthermore, if $\Psi(H_1)$, $\Psi(H_2^*)$ denote irreducible core graphs representing subgroups $H_1, H_2^*$, resp.,
and $| E \Psi |$ is the number of oriented edges in a graph $\Psi$, then
$$
| E  \Psi(H_2^*) | < 2^{  2^{4| E  \Psi(H_1) | + 1+ \log_2 \log_2 (2d)  } } .
$$

$\rm{(c)}$ There exists a linear semi-infinite programming problem (LSIP-problem) ${\mathcal{P}}(H_1) = \sup \{ cx \mid Ax \le b  \}$ with finitely many variables in $x$ and with countably  many constraints in the system $Ax \le b$ whose dual problem
$$
{\mathcal{P}}^*(H_1)  = \inf \{ b^{\top} y \mid A^{\top} y = c^{\top} , \, y \ge 0  \}
$$
has a solution equal to  $-\sigma(H_1) {\bar {\mathrm{r}}} (H_1)$.

$\rm{(d)}$ Let the word problem for both groups $G_1, G_2$ be solvable
and let an  irreducible core graph $\Psi(H_1)$  of $H_1$ be given. Then the
LP-problem \eqref{lpa}  of part (a)   can be effectively written down and the
WN${}_d$-coefficient $\sigma_d(H_1)$ for $H_1$  can be computed.
In addition,  an irreducible core graph $\Psi(H_2^*)$ of the subgroup $H_2^*$ of part (b) can be effectively constructed.

$\rm{(e)}$ Let both $G_1, G_2$ be finite,  let $d := \max( |G_1|, |G_2|) \ge 3$, and
 let an irreducible  core graph $\Psi(H_1)$  of $H_1$ be given.
 Then the LP-problem \eqref{lpa} of part (a)  coincides with the
LSIP-problem ${\mathcal{P}}(H_1)$ of part (c) and the WN-coefficient $\sigma(H_1)$ for $H_1$ is
rational and computable.
\end{T1}

\begin{proof}[Proof of Theorem~1.1]
  Assume that $I = \{ 1,2\}$, ${\mathcal{F}}_2 = G_1 * G_2$ and $H_1$ is a finitely generated factor-free noncyclic subgroup of ${\mathcal{F}}_2$. As before, let $\Psi_o(H_1)$ denote a finite irreducible ${\mathcal{A}}$-graph of $H_1$ and $\Psi(H_1)$ denote the core of    $\Psi_o(H_1)$.

Denote $Y_1 := \Psi(H_1)$ and pick an integer $d \ge 3$.
As in Sects.~3--4, consider a system of linear inequalities $\operatorname{\textsf{SLI}}_d[Y_1]$ and an LP-problem
\begin{equation}\label{lppf}
    \max \{ - x_s \mid \operatorname{\textsf{SLI}}_d[Y_1] \}  .
\end{equation}
According to Theorem~A and Lemma~\ref{lem5}, the maximum of the  LP-problem
\eqref{lppf} is equal to $-\sigma_d(Y_1) {\bar {\mathrm{r}}}(Y_1)$, where
$$
\sigma_d(Y_1){\bar {\mathrm{r}}}(Y_1) = \sup \frac{  {\bar {\mathrm{r}}} ( \operatorname{core}(Y_1 \times Y_2) )}{ {\bar {\mathrm{r}}}(Y_2) }
$$
over  all finite irreducible ${\mathcal{A}}$-graphs $Y_2$ with property (Bd) relative to $Y_1$.  By Lemma~\ref{lemsup},
$\sigma_d( Y_1 ) {\bar {\mathrm{r}}}(Y_1)   = \sigma_d(H_1){\bar {\mathrm{r}}}(H_1) $, as desired in part (a).
\medskip

 According to Lemma~\ref{lem5}, the minimum of the dual problem \eqref{dlpd} of \eqref{lppf}
 is attained at a vector ${\widehat} y(d)$ which is a vertex of the feasible polyhedron
\begin{equation}\label{fplh}
 \{  y \mid y \ge 0 , \ \sum_{j=1}^{i_d}   y_j q_j^L  = - x_s \}
 \end{equation}
 of the LP-problem \eqref{dlpd}.   By Lemma~\ref{lem4}, there is a finite irreducible ${\mathcal{A}}$-graph ${\widehat} Y_2$ with property  (Bd)  such that
\begin{equation}\label{hy2}
 \operatorname{\textsf{sol}}_d ( \operatorname{\textsf{inq}}( {\widehat} Y_2)) = {\widehat} y(d) \quad
  \mbox{and} \quad
  {\bar {\mathrm{r}}} (\operatorname{core}(Y_1 \times {\widehat}  Y_2))    =   \sigma_d(Y_1)   {\bar {\mathrm{r}}} (Y_1)  {\bar {\mathrm{r}}} ({\widehat} Y_2) .
 \end{equation}

 Assume that ${\widehat} Y_2$ is not connected and that  ${\widehat} Y_3, {\widehat} Y_4$ are (nonempty) subgraphs of   ${\widehat} Y_2$ so that there is no path in ${\widehat} Y_2$  from any vertex of ${\widehat} Y_3$ to a vertex of ${\widehat} Y_4$.  Clearly, ${\widehat} Y_3, {\widehat} Y_4$ are  finite irreducible ${\mathcal{A}}$-graphs with property (Bd).
  Invoking  Lemma~\ref{lem4}, denote ${\widehat} y^j  := \operatorname{\textsf{sol}}_d  ( \operatorname{\textsf{inq}}( {\widehat} Y_j))$, $j=3,4$.
 By  Lemma~\ref{lem4},   ${\widehat} y^3, {\widehat} y^4$ are points in the   feasible polyhedron \eqref{fplh} of \eqref{dlpd}.
 It is easy to see from the definitions and  Lemma~\ref{lem4} that ${\widehat} y(d) = \lambda_3 {\widehat} y^3+ \lambda_4 {\widehat} y^4 $ with some positive rational numbers $\lambda_3, \lambda_4 $ that satisfy $\lambda_3+ \lambda_4 =1$.
 This, however, is impossible when ${\widehat} y(d)$ is a vertex. This contradiction proves that
  ${\widehat} Y_2$ is a connected graph. Hence, there is a finitely generated factor-free subgroup $H_2^*$ of  ${\mathcal{F}}$ whose
  irreducible graph $\Psi_{o^*}(H_2^*)$  is ${\widehat} Y_2$, as required.
 \medskip

  Now we establish the upper bound
\begin{equation}\label{upp}
| E {\widehat} Y_2 | <   2^{ 2^{4| E Y_1 | +1+ \log_2 (\log_2 2d)  } }
\end{equation}
on size of ${\widehat} Y_2$, where ${\widehat} Y_2$ is defined by equalities \eqref{hy2}.
Recall that it follows from  Lemma~\ref{lem5} that
\begin{align}\label{dlp2}
- \sigma_d(Y_1) {\bar {\mathrm{r}}}(Y_1) & =  \min   \{  \sum_{j=1}^{i_d}   y_j q_j^R   \mid y \ge 0 ,
\sum_{j=1}^{i_d}   y_j q_j^L  = - x_s \}  \\ \label{lp0}
  & =   \min \{ b^{\top}y \mid A^{\top}y = c^{\top}, y\ge 0  \}  \\   \label{lp2}
  & =  \max\{ cx \mid Ax \le b  \} =  \max \{ - x_s \mid \operatorname{\textsf{SLI}}_d[Y_1]\}   .
\end{align}

It is convenient to switch back to the general LP notation as was introduced in Sect.~4
and reminded in \eqref{dlp2}--\eqref{lp2}. In particular, let $A$ be an $m \times n$ matrix, i.e.,
$m=i_d$ and $n$ is the number of variables.

According to Theorem~C, we may assume that
the minimum of the dual LP-problem \eqref{dlp2}
is attained at a vertex ${\widehat} y(d)$,  ${\widehat} y(d) \ge 0$,  of  the feasible polyhedron
$\{ y  \mid y \ge 0 , \ \sum_{j=1}^{ m }   y_j q_j^L  = - x_s \}$ of the LP-problem \eqref{dlp2}
whose positive components ${\widehat} y_i(d)$ correspond to linearly independent left-hand sides $q_i^L$ of the inequalities $q_i$, $i =1, \dots, m$, of $\operatorname{\textsf{SLI}}_d[Y_1]$.
Reordering the inequalities  of $\operatorname{\textsf{SLI}}_d[Y_1]$ if necessary, we may assume that
${\widehat} y_1(d), \dots,  {\widehat} y_r(d)$ are positive, $q_1^L, \dots, q_r^L $   are     linearly independent and that
${\widehat} y_{r+1}(d) = \ldots = {\widehat} y_{m}(d) = 0$.

Recall that  $Ax \le b$ is the matrix form of $\operatorname{\textsf{SLI}}_d[Y_1]$ and $A$ has size $m\times n$.
Let $A_{r, n}$ denote the submatrix of $A$  that consists of the first $r$ rows of $A$, hence, ${\widehat} y_1(d), \dots,  {\widehat} y_r(d)$ correspond to linearly independent rows of $A_{r, n}$. Let $A_{r \times r}$ denote a submatrix of $A_{r, n}$
of size $r \times r$ with  $\det A_{r \times r} \ne 0$ and let  $\bar y(d)  =( {\widehat} y_1(d), \dots,  {\widehat} y_r(d) )^{\top}$ be a truncated version of ${\widehat} y(d)$. Then
$$
A_{r \times r}^{\top}  \bar y(d)  = (c_1, \dots,  c_r)^{\top} = \bar c^{\top} .
$$
Since
$\sum_{i=1}^{m} {\widehat} y_{i}(d) q_i^L  = - x_s$, it follows that $c_{j} =0$ if $c_{j}$ corresponds to the column of a variable $x_{B}$ in $A_{r, n}$ and $c_j =-1$ if $c_j$ corresponds to the column of  $x_{s}$.
Since ${\widehat} y(d)  \ne 0$, we conclude that $\bar c^{\top} \ne 0$, hence, one of $c_j$ in $\bar c^{\top}$
is $-1$ and all other entries of  $\bar c^{\top}$  are equal to $0$. Since every entry of $ A_{r \times r} $ is 0 or $\pm k'$ or  $-(k-2)$, where   $1 \le k' \le d$,  $2 \le k \le d$, and every row of $A_{r \times r} $ contains at most
$d+1$ nonzero entries which are sums of coefficients of variables in an inequality \eqref{inqa}--\eqref{inqb},
it follows that the standard Euclidian norm of any row of $ A_{r \times r} $ is at most
$(d^2 + (d-2)^2)^{1/2} < d^2$ as $d \ge 3$. Hence, by Hadamard's inequality, we obtain
\begin{equation}\label{cr1}
| \det A_{r \times r} | <    d^{2r} .
\end{equation}

Invoking the Cramer's rule, we further have that
\begin{equation}\label{cr2}
{\widehat} y_{i}(d) = \frac{\det A_{r \times r, i}^{\top}(\bar c^{\top})}{  \det A_{r \times r}}  ,
\end{equation}
where  $ A_{r \times r, i}^{\top}(\bar c^{\top})$ is the matrix obtained from $A_{r \times r}$ by replacing the $i$th column with $\bar c^{\top}$, $i=1, \ldots, r$.  Similarly to $A_{r \times r}$, see \eqref{cr1}, we obtain that
\begin{equation}\label{cr3}
| \det A_{r \times r, i}(\bar c^{\top}) | <  d^{2r-2}
\end{equation}
for  $\|  \bar c^{\top}  \| = 1$.

In view of \eqref{cr1}--\eqref{cr3},  we can see that a common denominator
$L >0$ of positive rational numbers ${\widehat} y_1(d), \ldots,  {\widehat} y_r(d)$  satisfies $L < d^{2r}$ and that positive integers $L {\widehat} y_1(d), \ldots,  L{\widehat} y_r(d)$    are less than $d^{2r-2}$.

It follows from the definition of the function $\operatorname{\textsf{sol}}_d$, see also Lemma~\ref{lem4}, that if ${\widehat} y(d) = \operatorname{\textsf{sol}}_d({\widehat}  Q)$ then
\begin{equation}\label{eQ}
|{\widehat}  Q | < r d^{2r-2}  ,
\end{equation}
where $| Q | $ is the cardinality of a combination with repetitions $Q \sqsubseteq \operatorname{\textsf{SLI}}_d[Y_1]$
defined so that every element $q \in Q$ is counted as many times as it occurs in $Q$.

Recall that   $\operatorname{\textsf{inq}}({\widehat}   Y_2) = {\widehat} Q$.
It follows from  the definition of the function $\operatorname{\textsf{inq}}$, see also Lemma~\ref{lem3}, that  $| V_S {\widehat} Y_2 | = |{\widehat} Q|$ and, therefore, by \eqref{eQ}, we get
 \begin{equation}\label{cr4}
|  E {\widehat} Y_2 | \le  2d | V_S {\widehat}  Y_2 | = 2d |{\widehat}  Q | <  2d  r d^{2r-2} < 2 r d^{2r} .
\end{equation}

Note that $r$ does not exceed  the number $n$ of variables $x_{B}, x_s$ of $\operatorname{\textsf{SLI}}_d[Y_1]$.
Since every index $B$ of  $x_{B}$ is a nonempty subset of $V_P Y_1$, it follows that
\begin{equation}\label{cr5}
r \le n \le (2^{|V_P Y_1 |}   - 1) + 1  = 2^{  4| E Y_1 |}
\end{equation}
for $|V_P Y_1 | = 4| E Y_1 |$  as every primary vertex of $Y_1$ has degree 2.

Finally, we obtain from \eqref{cr4}--\eqref{cr5} that
 \begin{align*}
| E {\widehat}  Y_2 |  & < 2 r d^{2r}  =   2^{ 4| E Y_1 | +1}  d^{   2^{  4| E Y_1 | +1}  }  <   2^{ 4| E Y_1|  +1 +  \log_2 d  \cdot 2^{ 4 | E Y_1 | +1 }  } \\ & <  2^{ (\log_2 d +1 )  \cdot 2^{4| E Y_1 | +1}   } \le 2^{  \log_2 (2d) \cdot 2^{4| E Y_1 | +1 } } \\
 & \le  2^{  2^{4| E Y_1 | +1+ \log_2 \log_2 (2d)  } }   .
\end{align*}
This completes the proof of part (b) of Theorem~\ref{th1}.
 \medskip

To prove part (c), we note that it follows from Lemmas~\ref{lem5} and \ref{lemsup}  that the dual problem  \eqref{dlp} of the LSIP-problem $ \sup \{ - x_s \mid \operatorname{\textsf{SLI}}[Y_1] \}$, where $Y_1 = \Psi(H_1)$,  has the infimum equal to
$
-\sigma(Y_1) {\bar {\mathrm{r}}}(Y_1)  = -\sigma(H_1) {\bar {\mathrm{r}}}(H_1) .
$
This proves part (c).
\medskip

Now we turn to parts (d)--(e) of Theorem~1.1.
First we discuss  how to effectively write down inequalities of the system $\operatorname{\textsf{SLI}}_d[Y_1]$, $d=3,4, \dots$. Recall that every inequality of $\operatorname{\textsf{SLI}}[Y_1]$ is written in the form \eqref{inqa}--\eqref{inqb} and there are finitely many subsets $A \subseteq S_1(V_PY_1)$ that are indices of $k$ variables $\pm x_A$ in the left-hand sides of inequalities  \eqref{inqa}--\eqref{inqb}, where $2 \le k = |T| \le d$. The coefficient of $x_s$ is the integer $-(k-2)$ and the right-hand side of \eqref{inqa}--\eqref{inqb} is an integer $-N({{\Omega}}_T)_{\alpha}$, where
$$
0 \le N({{\Omega}}_T)_{\alpha} \le (d-2) |   V_P Y_1| ,
$$
see \eqref{cd2}. This information is sufficient to conclude that the set of inequalities in the system $\operatorname{\textsf{SLI}}_d[Y_1]$ is finite. However, this information is not sufficient to effectively write down
inequalities of $\operatorname{\textsf{SLI}}_d[Y_1]$ because the set of available sets $T$ is infinite whenever $G_1 \cup G_2$ is infinite.

To explicitly write down  the system $\operatorname{\textsf{SLI}}_d[Y_1]$, we assume that
the word problem for both groups $G_1, G_2$ is solvable and we will look more closely into construction of inequalities \eqref{inqa}--\eqref{inqb}.

Recall that inequalities \eqref{inqa}--\eqref{inqb} are defined in Sect.~3 by using an ${\alpha}$-admissible function $\Omega_T : T \to S_1( V_P Y_1)$, where $T \in S_2(G_{\alpha})$ and ${\alpha} \in I = \{1,2\}$.

We also recall that $\sim_{\Omega_T}$ denotes an equivalence relation on the set of all pairs $(a, u)$, where $a \in T$ and $u \in \Omega_T(a)$, see Sect.~3. Making use of the equivalence relation $\sim_{\Omega_T}$, we define a relation  $\approx$ on the set  $T$ so that $a \approx b$ if and only if  there are  $u \in \Omega_T(a)$, $v \in \Omega_T(b)$ such that $(a,u) \sim_{{{\Omega}}_T} (b,v)$.  Note that this relation $\approx $  is reflexive and symmetric. The transitive closure of  the relation $\approx $ is an equivalence relation on $T$ which we denote by  $\approx_{{{{\Omega}}_T} } $. The equivalence class of $a \in T$  is denoted  $[a]_{\approx_{{{\Omega}}_T}}$.   It follows from the definition of  $[a]_{\approx_{{{\Omega}}_T}}$ and from the property of being ${\alpha}$-admissible for ${{\Omega}}_T$ that, for every $b_1 \in
[a]_{\approx_{{{\Omega}}_T}}$, there is an element $b_2 \in [a]_{\approx_{{{\Omega}}_T}}$ such that $b_2 \ne b_1$ and there are edges $e_1, e_2 \in EY_1$ such that  $(e_1)_+ = (e_2)_+ \in V_S Y_1$, the vertex $(e_1)_+$ has  type ${\alpha}$, $(e_i)_- \in {{\Omega}}_T (b_i)$, $i =1,2$, and
\begin{equation}\label{t3t3}
    b_1 b_2^{-1} = {\varphi}(e_1) {\varphi}(e_2)^{-1}
\end{equation}
in $ G_{\alpha}$.
 Note that if we connect every two such elements $b_1, b_2 \in [a]_{\approx_{{{\Omega}}_T}}$  by an edge, then the graph $\Gamma({{\Omega}}_T)$,  whose vertex set is $T$, will have connected components whose vertex sets are equivalence classes  $[a]_{\approx_{{{\Omega}}_T}}$ of $T$.
This connectedness of subgraphs  of $\Gamma({{\Omega}}_T)$ on vertex sets
$[a]_{\approx_{{{\Omega}}_T}}$ obviously implies the following claim.

\begin{lem}\label{lem3t} The equations  \eqref{t3t3} can be used to determine all elements of the equivalence class  $[a]_{\approx_{{{\Omega}}_T}}$
if only one element in $[a]_{\approx_{{{\Omega}}_T}}$  is known.
\end{lem}

\begin{proof} This easily follows from definitions.
\end{proof}

Let
$
C({\alpha}, d)
$
be a subset of $G_{\alpha}$ of cardinality $|C({\alpha}, d) | = d^2 +d$, ${\alpha} =1,2$, where $d \ge 3$ is a fixed integer.
In the arguments below, this set  $C({\alpha}, d)$  will be held fixed. Note that if $|G_{\alpha}| < d^2 +d$, so it is not possible to choose  $d^2 +d$ distinct elements in $G_{\alpha}$, then all inequalities \eqref{inqa} if ${\alpha} =1$ or $\eqref{inqb}$ if  ${\alpha} =2$  for $k \le d$, where as before $k = |T|$,  can be written down effectively for the following reasons: The sets
$S_2(G_{\alpha})$ and $\{ {{\Omega}}_T \mid {{\Omega}}_T : T \to S_1(V_P Y_1), T \in  S_2(G_{\alpha}) \} $ are finite, can be written down explicitly, and it is possible to verify whether given function
${{\Omega}}_T :  T  \to S_1(V_P Y_1) $ is ${\alpha}$-admissible. Clearly, the same conclusion holds if both $G_1, G_2$ are finite but in the arguments below we will only need the inequality $|C({\alpha}, d) | = d^2 +d$, hence we can just assume that $|G_{\alpha}| \ge d^2 +d$.

Consider a subset $C \subset  C({\alpha}, d)$, where $1 \le |C| \le k \le d$ and a set of indeterminates $Z = \{ z_1, \dots, z_{k-|C|} \}$. Note  $| C \cup Z | = k$.
Consider a function
\begin{equation}\label{fscz}
{{\Omega}}_{ C \cup Z} : C \cup Z \to S_1(V_P Y_1) .
\end{equation}
Similarly to the relation $\sim_{\Omega_T}$ defined in Sect.~3, we consider a relation $\sim_{{{\Omega}}_{ C \cup Z}}$ on the set of all pairs $(a, u)$, where $a \in C \cup Z$ and $u \in {{\Omega}}_{ C \cup Z}(a)$, defined as follows. Two pairs
$(a,u)$ and $(b,v)$ are related by $\sim_{{{\Omega}}_{ C \cup Z}}$  if and only if  either
$(a,u)=  (b,v)$ or, otherwise, there exist edges $e, f \in EY_1$ such that $e_- = u$, $f_- = v$ and the secondary vertex $e_+ = f_+$ has type ${\alpha}$.

We also consider an analogue  $\approx_Z$  of the relation $\approx$ defined above so that $a \approx_Z b$, where $a, b \in C\cup Z$, if and only if   there are  $u \in {{\Omega}}_{ C \cup Z}(a)$,  $v \in {{\Omega}}_{ C \cup Z}(b)$ such that $(a, u) \sim_{{{\Omega}}_{ C \cup Z}}  (b, v)$.
As before,  the  relation $\approx_Z$ is reflexive and symmetric.   By taking the transitive closure of the  relation $\approx_Z$ we obtain an equivalence relation on the set $C \cup Z$  which is denoted by $\approx_{{{\Omega}}_{ C \cup Z}}$.

We will say that a function ${{\Omega}}_{{ C \cup Z}}$, as in \eqref{fscz}, is {\em unacceptable} if there is an equivalence class $[(a, u)]_{\sim_{{{\Omega}}_{ C \cup Z}}}$ of $\sim_{{{\Omega}}_{ C \cup Z}}$ with a single element in it or there is  an  equivalence class $[a]_{\approx_{{{\Omega}}_{ C \cup Z}}}$  of the relation $\approx_{{{\Omega}}_{ C \cup Z}}$ that
contains no elements of $C$. Note that, when given a function ${{\Omega}}_{{ C \cup Z}}$ as in \eqref{fscz}, we can algorithmically check whether or not ${{\Omega}}_{{ C \cup Z}}$ is unacceptable.

If now the function ${{\Omega}}_{C \cup Z}$ is not found to be unacceptable, then  we attempt to construct a function $\zeta : Z \to G_{\alpha}$ by using the following algorithm.

First, we set $\zeta_0(c)= c$ if $c \in C $  and let $C_0 := C$, $Z_0 := \varnothing$. Consider the set of all triples $(a, u, \ell)$, where $a \in C \cup Z$, $u \in  {{\Omega}}_{C \cup Z}(a)$, $1 \le \ell  \le d+1$, and do the following. By induction on $i \ge 0$, assume that the sets
$C_i \subseteq G_{\alpha}$, $Z_i \subseteq Z$ are constructed and a bijective function $\zeta_i : C_0 \cup Z_i \to C_i$ is  defined so that the restriction of $\zeta_i$ on $C_0$ is  $\zeta_0$. For every unordered pair
$\{ (a, u, \ell),  (b, v, \ell) \}$ of distinct triples with a fixed $\ell$ (first we use $\ell =1$, then $\ell =2$ and so on up to $\ell  = d+1$), we check whether there are edges $e , f \in EY_1$ such that $e_- = u$, $f_- = v$,  $e_+ = f_+$, and  $e_+ = f_+ \in V_SY_1$ has type ${\alpha}$.  If there are no such edges, then we pass on to the next pair
$\{ (a, u, \ell) ,  (b, v, \ell) \}$. If there are such edges $e, f$, then we consider three Cases 1--3 below, perform the described actions and pass on to the next pair. We remark that these actions can be
algorithmically implemented as follows from the solvability of the word problem for $G_1, G_2$ and the availability of the graph $Y_1 = \Psi(H_1)$.
\smallskip

{\em  Case 1.} \  If both $a, b \in C \cup Z_i$, then we check whether the equality
 $$
 \zeta_i(a) \zeta_i(b)^{-1} = {\varphi}(e) {\varphi}(f)^{-1}
 $$
 holds in $G_{\alpha}$. If this equality is false, then we conclude that the function  ${{\Omega}}_{C \cup Z}$ is  unacceptable and stop. Otherwise, we set
 $$
 Z_{i+1} := Z_{i} ,  \quad C_{i+1} := C_{i}  ,  \quad   \zeta_{i+1} := \zeta_{i} .
 $$
\smallskip

{\em  Case 2.} \  Suppose that exactly one of  $a, b$ is in $ C \cup Z_i$, say $b \in C \cup Z_i$. Then it is clear that
$a \in Z \setminus Z_i$ and  we can uniquely determine an element $\eta(a)$ by solving the equation
  $\eta(a) \zeta_i(b)^{-1} = {\varphi}(e) {\varphi}(f)^{-1} $. If $\eta(a) \in C_i$, then we conclude that the function  ${{\Omega}}_{C \cup Z}$ is  unacceptable and stop. Otherwise, we set
$$
Z_{i+1} := Z_{i}\cup \{ a \}  ,  \quad   C_{i+1} := C_{i}\cup \{ \eta(a) \}
$$
and define a function $\zeta_{i+1}$ on the set $C \cup Z_{i+1}$ so that  $\zeta_{i+1}(a) := \eta(a)$ and the restriction of  $\zeta_{i+1}$ on $C \cup Z_{i}$ is $\zeta_{i}$.
\smallskip

  {\em  Case 3.} \  If both $a, b \not\in C \cup Z_i$, then we  set
$$
Z_{i+1} := Z_{i}   ,  \quad    C_{i+1} := C_{i}  ,  \quad   \zeta_{i+1} := \zeta_{i} .
$$

Cases 1--3 are complete.
\medskip

Since every equivalence class $[a]_{\approx_{{{\Omega}}_{C \cup Z}}}$ contains an element
of $C$,  it follows from the definitions that while this algorithm runs over all pairs for a fixed $\ell' = 1, \dots, d$, one of the following three Cases (C1)--(C3) will occur.

\begin{enumerate}
\item[(C1)] For some $i$,  $|Z_{i+1}| = |Z_{i}|+1$.

\item[(C2)] The set  ${{\Omega}}_{C \cup Z}$ is found to be  unacceptable.

\item[(C3)] For the index $i$, corresponding to the last pair $\{ (a, u, \ell'),  (b, v, \ell') \}$  for parameter $\ell$ equal to $\ell'$,  one has  $Z_{i} = Z$.
\end{enumerate}

Since $|Z| \le d-1$, we can see that it is not possible for Case (C1) to occur for all
$\ell' = 1, \dots, d$. Hence, running this algorithm
consecutively for $\ell' = 1, \dots, d$, results either in conclusion that the function   ${{\Omega}}_{C \cup Z}$ is  unacceptable or in construction of a bijective function
$$
\zeta = \zeta_{i} : C \cup Z \to  C_i \subseteq G_{\alpha} ,
$$
where $Z_{i} =Z$, in which case we say that the function ${{\Omega}}_{C \cup Z}$ is  {\em acceptable}.
Furthermore, setting $T := \zeta(C \cup Z)$ and ${{\Omega}}_T( \zeta(a) ) := {{\Omega}}_{C \cup Z}(a)$ for every $a \in C \cup Z$, we obtain an ${\alpha}$-admissible function ${{\Omega}}_T$ on the set $T \subseteq G_{\alpha}$.

Observe that the set of all such   functions
$
{{\Omega}}_{C \cup Z} :  C \cup Z \to S_1(V_PY_1) ,
$
where $C \subseteq C({\alpha}, d)$  and  $Z = \{ z_1, \dots, z_{k-|C|} \}$, see  \eqref{fscz},  is finite  (recall the set $C({\alpha}, d)$ is fixed) and that all such  functions   can be written down explicitly. Moreover, using the foregoing algorithm, we can verify whether a function ${{\Omega}}_{C \cup Z}$ is acceptable and, when doing so, construct a unique function
$\zeta : C \cup Z \to S_1(V_PY_1)$, where $T := \zeta( C \cup Z )$,
so that ${{\Omega}}_T( \zeta(a) ) := {{\Omega}}_{C \cup Z}(a)$ for every $a \in C \cup Z$ and $\zeta(c) = c$ if $c \in C$. Therefore, in order to establish that inequalities \eqref{inqa}--\eqref{inqb} can be effectively written down, it remains to prove the following.

\begin{lem}\label{lemx} For every ${\alpha}$-admissible function
${{\Omega}}_{T'} : T' \to S_1(V_PY_1)$, where  $T' \subset G_{\alpha}$ and $2 \le |T'|=k \le d$, there exists an acceptable function
$$
{{\Omega}}_{C \cup Z} : C \cup Z \to S_1(V_PY_1) ,
$$
where $C \subset C({\alpha}, d)$ and $Z =   \{ z_1, \dots, z_{k-|C|} \}$,  such that if $T := \zeta(C \cup Z)$ and if ${{\Omega}}_{T} : T \to S_1(V_PY_1)$ is the  ${\alpha}$-admissible function, defined by
 ${{\Omega}}_T( \zeta(a) ) := {{\Omega}}_{C \cup Z}(a)$ for every $a \in C \cup Z$ and $\zeta(c) = c$ for $c \in C$, then the two
 inequalities \eqref{inqa}, that correspond to  ${{\Omega}}_{T'}$ and to ${{\Omega}}_{T}$ if ${\alpha} =1$, or the two
 inequalities \eqref{inqb}, that correspond to  ${{\Omega}}_{T'}$ and to ${{\Omega}}_{T}$ if ${\alpha} =2$, are identical.
\end{lem}

To prove Lemma~\ref{lemx}, we first establish an auxiliary lemma.

\begin{lem}\label{lemy} Suppose ${{\Omega}}_{T'} : T' \to S_1(V_PY_1)$ is an
${\alpha}$-admissible function, where $2 \le |T'| \le d$, and $T' = E_1 \cup \dots \cup  E_r$ is a partition of $T'$ into equivalence classes $[a]_{\approx_{{{\Omega}}_{T'}} }$ of the equivalence relation $\approx_{{{\Omega}}_{T'}}$.
Then there are elements $h_1, \dots, h_r \in G_{\alpha}$ such that the set $T :=  E_1 h_1 \cup \dots \cup E_r h_r$ has cardinality $ |T|= |T'|$ and every set $E_i h_i$, $i = 1, \dots, r$, contains an element from the set  $C({\alpha}, d)$.
\end{lem}

\begin{proof} By induction on $i$, where  $1 \le i \le r$, we will prove the existence of elements
$h_1, \dots, h_i \in G_{\alpha}$ with the property that the set
$E_1 h_1 \cup \dots \cup E_i h_i$ has cardinality $\sum_{j=1}^i |E_j h_j|$ and every set $E_j h_j$, $j = 1, \dots, i$, contains an element from $C({\alpha}, d)$.

If $i=1$, then we set $h_1 := b^{-1} c$, where $b \in E_1$ and $c \in C({\alpha}, d)$.

Making the induction hypothesis, assume that there are elements
$h_1, \dots, h_i \in G_{\alpha}$ with the desired properties.

To make the induction step from $i$ to $i+1$, denote
$$
C_i({\alpha}, d) :=   C({\alpha}, d) \cap  (E_1 h_1 \cup \dots \cup E_i h_i )
$$
and let $b \in E_{i+1}$. For an element $c \in C({\alpha}, d) \setminus C_i({\alpha}, d)$, we consider the set $R_c := E_{i+1} b^{-1} c$. Clearly, $R_c $ contains an element from $C({\alpha}, d)$ and if $R_c $ is disjoint from  $E_1 h_1 \cup \dots \cup E_i h_i$, then we can set $h_{i+1}  :=   b^{-1} c$. Therefore, we may assume that $R_c$ contains an element from $E_1 h_1 \cup \dots \cup E_i h_i$ for every $c \in C({\alpha}, d) \setminus C_i({\alpha}, d)$.

Suppose that elements in $E_1 h_1 \cup \dots \cup E_i h_i$ are indexed by integers from 1 to $|E_1 h_1 \cup \dots \cup E_i h_i|$ and elements in  $R_c = E_{i+1} b^{-1} c$, where $b$ and $c$ are chosen as above,  are indexed by integers from 1 to $|E_{i+1}|$ so that, for every $e \in E_{i+1}$,  the index of $e  b^{-1} c \in R_c$ is equal to that of $e \in E_{i+1}$. Thus, we wish to keep indices stable when multiplying $E_{i+1}$ by  $b^{-1} c$.

Making use of these indices, we fix an element $b \in E_{i+1}$ and, for every $c \in C({\alpha}, d) \setminus C_i({\alpha}, d)$, we consider the pair $( j_R(c),  j_E(c))$ of indices  $j_R(c),  j_E(c)$  in $R_c = E_{i+1} b^{-1} c$  and in  $E_1 h_1 \cup \dots \cup E_i h_i$, resp., of an element of the intersection $R_c  \cap (E_1 h_1 \cup \dots \cup E_i h_i)$ which is not empty as was assumed above.

Suppose that  $( j_R(c_1),  j_E(c_1)) =  ( j_R(c_2),  j_E(c_2))$. Then it follows from the definitions that if $e_1, e_2 \in  E_{i+1}$ are such that
$$
e_1 b^{-1} c_1 \in R_{c_1} \cap (E_1 h_1 \cup \dots \cup E_i h_i) \quad \mbox{and}   \quad e_2 b^{-1} c_2 \in R_{c_2}\cap (E_1 h_1 \cup \dots \cup E_i h_i) ,
$$
then $e_1 = e_2$ and $e_1 b^{-1} c_1 = e_2 b^{-1} c_2$ in $G_{\alpha}$. These equalities imply that $c_1 = c_2$. Therefore,  for distinct elements $c_1, c_2 \in C({\alpha}, d) \setminus C_i({\alpha}, d)$, the pairs $(j_R(c_1),  j_E(c_1))$, $(j_R(c_2),  j_E(c_2))$ are also distinct. However, the number of elements $c$ in   $C({\alpha}, d) \setminus C_i({\alpha}, d)$ is
$$
|C({\alpha}, d)|- | C_i({\alpha}, d)| \ge (d^2 +d)-d = d^2
$$
and the number of all such pairs  $(j_R(c),  j_E(c))$  is less than $d^2$. This contradiction completes the induction step and Lemma~\ref{lemy} is proved.
\end{proof}

\begin{proof}[Proof of Lemma~\ref{lemx}] Utilizing the notation of Lemma~\ref{lemy}, we let $T' = E_1 \cup \dots \cup  E_r$ and let $h_1, \dots, h_r \in G_{\alpha}$ be elements  such that the set $T :=  E_1 h_1 \cup \dots \cup E_r h_r$ has cardinality $ |T|= |T'| = k$ and every set $E_i h_i$, $i = 1, \dots, r$, contains an element from $C({\alpha}, d)$.
 Define a function ${\widehat}  {{\Omega}} : T \to    S_1(V_PY_1)$ so that
 if $a \in E_i$, $i = 1, \dots, r$, then  ${\widehat}  {{\Omega}}(ah_i) :=  {{\Omega}}_{T'}(a)$.
 Define $C := C({\alpha}, d) \cap T$ and let $C = \{ c_1, \dots, c_{|C|} \}$.
Introducing more notation, denote $T = \{ c_1, \dots, c_{|C|}, b_1, \dots, b_{k-|C|} \}$ and $Z = \{ z_1, \dots, z_{k-|C|} \}$. Define a function
${{\Omega}}_{C \cup Z}  \to S_1(V_PY_1)$ by setting ${{\Omega}}_{C \cup Z}(c_i) := {\widehat} {{\Omega}} (c_i)$ and ${{\Omega}}_{C \cup Z} (z_j) :={\widehat}  {{\Omega}} (b_j)$ for all $i,j$. In view of Lemma~\ref{lem3t}, it is not difficult to see that the function ${{\Omega}}_{C \cup Z} $ is acceptable, $\zeta(C \cup Z) = T$, and if
 ${{\Omega}}_{T} : T \to S_1(V_PY_1)$  is the function   defined by
${{\Omega}}_{T}(\zeta(a)) := {{\Omega}}_{C \cup Z}(a)$ for every $a \in  C \cup Z$,
where  $\zeta(c) = c$ for $c \in C$, then the following hold:  ${{\Omega}}_{T}$ is an ${\alpha}$-admissible function, ${{\Omega}}_{T}= {\widehat}  {{\Omega}}$, and   the inequalities \eqref{inqa} if ${\alpha} =1$    or the   inequalities   \eqref{inqb} if ${\alpha} =2$, corresponding to  ${{\Omega}}_{T'}$ and to ${{\Omega}}_{T}$, are identical.
 Lemma~\ref{lemx} is proved.
\end{proof}

To finish the proof of part (d) of Theorem~\ref{th1}, we remark that, by  Lemma~\ref{lemx}, the LP-problem  $\max \{ - x_s \mid \operatorname{\textsf{SLI}}_{d}[Y_1] \}$ can be effectively written down. Solving this  LP-problem   we obtain, by Lemma~\ref{lem5}, the number  $- \sigma_d( Y_1)  {\bar {\mathrm{r}}} (Y_1)$ which is equal to  $- \sigma_d( H_1)  {\bar {\mathrm{r}}} (H_1)$ by Lemma~\ref{lemsup}. Since the number ${\bar {\mathrm{r}}} (Y_1) = {\bar {\mathrm{r}}} (H_1)$ is computable off the graph $Y_1$,  the coefficient $\sigma_d( Y_1)$ is also computable.

Since the LP-problem $\max \{ - x_s \mid \operatorname{\textsf{SLI}}_d[Y_1] \} $ can be effectively written down, its dual problem
\eqref{dlp2} can also  be effectively constructed. Using the notation of the foregoing proof of parts (a)--(c), we observe that a vertex solution ${\widehat} y(d)$ to \eqref{dlp2} can be computed, see \cite{S86}.
Hence, a  combination with repetitions
${\widehat} Q$ such that  $\operatorname{\textsf{sol}}_d({\widehat} Q) = {\widehat} y(d)$ is also computable.
Therefore, a graph ${\widehat} Y_2 = \Psi(H_2^*)$ such that $\operatorname{\textsf{inq}}({\widehat} Y_2) = {\widehat} Q$
can be effectively constructed and the proof of part~(d) is complete.
\medskip

 To show part (e), we note that if both groups $G_1, G_2$ are finite,
 then any irreducible finite ${\mathcal{A}}$-graph  $\Psi$ has the property that $\deg u \le \max \{ |G_1 |, |G_2| \}$
 for every secondary vertex $u \in V_S \Psi$. Hence, letting $d := \max \{ |G_1 |, |G_2| \}$, we obtain  $\operatorname{\textsf{SLI}}[Y_1] =  \operatorname{\textsf{SLI}}_{d}[Y_1]$ and so,
 by Lemma~\ref{lem5}, $\sigma( Y_1) = \sigma_{d}( Y_1)$. Since  the  coefficient  $\sigma_{d}( Y_1)$
 is rational and computable by part (d), the  number $\sigma(H_1) = \sigma( Y_1)$ is also rational and computable.
  Theorem~\ref{th1} is proved.  \end{proof}

\begin{T2}  Suppose that ${\mathcal{F}} =G_1 * G_2$ is the free product of two nontrivial finite groups $G_1,  G_2$ and $H_1$ is a subgroup of ${\mathcal{F}}$ given by a finite generating set ${\mathcal{S}}$ of words over the alphabet
${\mathcal{A}} = G_1 \cup G_2$. Then the following are true.

$\rm{(a)}$ In deterministic polynomial time of size of ${\mathcal{S}}$, one can
determine whether $H_1$ is factor-free and noncyclic and, if so, one can construct an  irreducible graph $\Psi_o(H_1)$  of $H_1$.

$\rm{(b)}$  If  $H_1$ is factor-free and noncylcic, then, in deterministic exponential time of size of ${\mathcal{S}}$, one can write down and solve an LP-problem ${\mathcal{P}} = \max\{ cx \mid Ax \le b  \}$ whose solution is equal to $-\sigma(H_1) {\bar {\mathrm{r}}} (H_1)$. In particular, the $WN$-coefficient $\sigma(H_1)$ of $H_1$ is computable in deterministic exponential time of size of ${\mathcal{S}}$.

$\rm{(c)}$  If  $H_1$ is factor-free and noncylcic, then, in deterministic double exponential time of size of ${\mathcal{S}}$, one can construct an irreducible core graph $\Psi(H_2^*)$ of $H_2^*$, where $H_2^*$ is a finitely generated factor-free subgroup ${\mathcal{F}}$ such that
$\bar {\mathrm{r}}(H_1, H_2^*) =  \sigma(H_1)  \bar {\mathrm{r}}(H_1) \bar {\mathrm{r}}( H_2^*)$.
In addition, if $| E\Psi |$ denotes the number of oriented edges of a graph $\Psi$, $\Psi(H_1)$ is an irreducible core graph of $H_1$, and   $d := \max( |G_1|, |G_2|)$, then
$
| E \Psi(H_2^*) |  <  2^{ 2^{4| E \Psi(H_1) | +1+ \log_2 \log_2 (2d)  } } .
$
\end{T2}

\begin{proof}[Proof of Theorem~1.2] Part (a) follows from Lemma~\ref{Lm1}.

To show part (b), we observe that, in the case when $G_1, G_2$ are finite, we can  effectively write down  the system   $\operatorname{\textsf{SLI}}_{d}[Y_1]$, where $d =\max \{ |G_1|, |G_2| \}$,  and this can be done in exponential time of size of $Y_1$. Furthermore, by  Lemma~\ref{Lm1}, the size of  the graph $Y_1$ is polynomial in size $\sum_{S \in  {\mathcal{S}}} |S|$  of the generating set ${\mathcal{S}}$.
It follows from Theorem~1.1(e) that  $ \operatorname{\textsf{SLI}}[Y_1] =   \operatorname{\textsf{SLI}}_{d}[Y_1]$. The size of the system
$ \operatorname{\textsf{SLI}}[Y_1] =   \operatorname{\textsf{SLI}}_{d}[Y_1]$ is exponential in size of $Y_1$. Hence, the size of the primal LP-problem $\max \{ - x_s \mid \operatorname{\textsf{SLI}}_{d}[Y_1] \}$ as well as the size of the dual problem \eqref{dlpd}  are exponential in size of $Y_1$.
By Theorem~\ref{th1} and Lemma~\ref{lemsup}, an optimal solution to the  dual problem \eqref{dlpd}  is equal to
$$
- \sigma_d( Y_1)  {\bar {\mathrm{r}}} (Y_1) = - \sigma_d( H_1)  {\bar {\mathrm{r}}} (H_1)
= - \sigma( Y_1)  {\bar {\mathrm{r}}} (Y_1) = - \sigma( H_1)  {\bar {\mathrm{r}}} (H_1) .
$$
It remains to mention that an LP-problem $\max \{cx \mid Ax \le b \} $ can be solved in polynomial time  of size of the problem, see \cite{S86}, and that the reduced rank ${\bar {\mathrm{r}}}(Y_1) = {\bar {\mathrm{r}}}(H_1)$ can be computed in polynomial time of size of  $Y_1$.
\medskip

To prove part (c), we note that the size of the dual LP-problem \eqref{dlpd}, similarly to size of
$\max \{ - x_s \mid \operatorname{\textsf{SLI}}_d[Y_1] \} $, is at most exponential in size of $Y_1$ and that
a vertex solution  ${\widehat} y(d)$  to  \eqref{dlpd} can be computed in polynomial time of
size of the dual LP-problem \eqref{dlpd}, see \cite{S86}. Note that here and below we use the notation of the proof of part~(b) of Theorem~\ref{th1}.  Hence, both a vertex solution ${\widehat} y(d)$ and a  combination with repetitions ${\widehat} Q$ such that  $\operatorname{\textsf{sol}}_d({\widehat} Q) = {\widehat} y(d)$ can be computed in deterministic exponential
time of size of $Y_1$. In view of inequalities \eqref{eQ} and \eqref{cr5}, we obtain that
\begin{align*}
| {\widehat} Q| < rd^{2r-2} < 2^{4| EY_1 |} d^{2^{ 4| EY_1 |+1 }} = 2^{2^{O(| EY_1 |)}} .
\end{align*}
Therefore, a desired graph ${\widehat} Y_2 = \Psi(H_2^*)$ such that $\operatorname{\textsf{inq}}({\widehat} Y_2) = {\widehat} Q$
can be constructed from ${\widehat} Q$ (this construction is similar to the proof of Lemma~\ref{lem2})
in deterministic double exponential time of size of $Y_1 = \Psi(H_1)$.

The inequality
$$
| E \Psi(H_2^*) | <  2^{  2^{4| E \Psi(H_1) | + 1+ \log_2 \log_2 (2d)  } },
$$
where, as above, $d = \max (|G_1|, |G_2| )$,  follows from  part~(d) of Theorem~\ref{th1}.
Theorem~\ref{th2} is proved.  \end{proof}

\begin{T3}
Suppose  that ${\mathcal{F}} = \prod_{\alpha \in I}^* G_\alpha$ is
the  free product  of  nontrivial groups   $G_{\alpha}$, ${\alpha} \in I$, and $H_1$ is a
finitely generated factor-free noncyclic subgroup of ${\mathcal{F}}$.  Then there are two  disjoint finite subsets
$I_1, I_2$ of $I$ such that if
${\widehat} G_1 := \prod_{\alpha \in I_1}^* G_\alpha$,  ${\widehat} G_2 := \prod_{\alpha \in I_2}^* G_\alpha$,
and ${\widehat} {\mathcal{F}} := {\widehat} G_1 * {\widehat} G_2$, then there exists a finitely generated factor-free subgroup ${\widehat} H_1 $ of ${\widehat} {\mathcal{F}}$  with the following properties.

$\rm{(a)}$  ${\bar {\mathrm{r}}} ({\widehat} H_1  ) = {\bar {\mathrm{r}}} (H_1  ) $,
$\sigma_d({\widehat} H_1) \ge \sigma_d(H_1)$ for every $d \ge 3$,  and
$\sigma({\widehat} H_1) \ge \sigma(H_1)$. In particular, if the conjecture
\eqref{conjs} fails for $H_1$ then the conjecture \eqref{conjs} also fails for ${\widehat} H_1$.

$\rm{(b)}$ If the word problem for all groups $G_{\alpha}$, ${\alpha} \in I_1 \cup I_2$,
is solvable and  a finite irreducible graph  of $H_1$ is given, then the
LP-problem ${\mathcal{P}}({\widehat} H_1, d)$  for  ${\widehat} H_1$   of part (a) of Theorems~\ref{th1}  can be effectively written down and the WN${}_d$-coefficient $\sigma_d({\widehat} H_1) $ for ${\widehat} H_1$  can be computed.

$\rm{(c)}$  Let groups $G_{\alpha}$, where ${\alpha} \in I_1 \cup I_2$,
be finite, let $H_1$ be effectively given either by
a finite irreducible graph or by a finite generating set, and let
$$
d' := \max\{ |I_1 \cup I_2| , \max\{ |G_{\alpha} | \, : \,   {\alpha} \in I_1 \cup I_2 \} \} .
$$
Then $\sigma_{d'}({\widehat} H_1) \ge \sigma(H_1)$
and there is an algorithm that decides whether the conjecture  \eqref{conjs} holds for $H_1$.
\end{T3}

\begin{proof}[Proof of Theorem~1.3]  (a)  We assume that the  subgroup $H_1$ is given by an irreducible ${\mathcal{A}}$-graph $\Psi(H_1)$  with $\operatorname{core}(\Psi(H_1)) = \Psi(H_1)$.  Note that it is also possible to assume that $H_1$ is defined by a finite generating elements ${\mathcal{S}}$ whose elements  are words over ${\mathcal{A}}$. In the latter case, we would apply Lemma~\ref{Lm1} which, when given
a finite generating set of a subgroup $H$ of ${\mathcal{F}}$, verifies that $H$ is a factor-free subgroup of ${\mathcal{F}}$ and, if so, constructs an irreducible ${\mathcal{A}}$-graph of $H$.

 Making use of the graph $Y_1 := \Psi(H_1)$ of $H_1$, we switch from the original index set $I$ to its finite subset $I(H_1)$ and rename it by $\{ 1, \dots, m \}$.
 Here and below we use the notation introduced in Sect.~5. Without loss of generality, we may assume that $m \ge 3$, otherwise we may set ${\widehat} H_1 := H_1$.
 Consider an  embedding $\mu_2 : {\mathcal{F}} \to {\mathcal{F}}_2(1)$ defined by means of the map \eqref{map2}, where ${\mathcal{F}}_2(1) = G_1 * G(2,m)$ and $G(2,m) = G_2 * \dots  * G_m$. Denote ${\widehat} H_1 := \mu_2(H_1)$.
 Then   ${\bar {\mathrm{r}}}({\widehat} H_1)=  {\bar {\mathrm{r}}}(H_1)$ and, by Lemma~\ref{lemmap2}(e),  $\sigma_d(H_1) \le \sigma_d(  {\widehat} H_1 )$ for every $d \ge 3$.  Consequently,   $\sigma(H_1) \le \sigma(  {\widehat} H_1 )$ as well. This proves part (a).
\smallskip

 (b) Assume that the word problem is solvable in groups $G_{\alpha}$, ${\alpha} \in I(H_1)$.
 Then the word problem is also solvable in factors $G_1,  G(2,m)$ of the free product ${\mathcal{F}}_2(1) = G_1 * G(2,m)$. Furthermore, using the graph $Y_1$ of $H_1$ and the map \eqref{map2}, we can effectively construct a finite irreducible ${\mathcal{A}}$-graph $\Psi({\widehat} H_1)$  with $\operatorname{core}(\Psi({\widehat} H_1)) = \Psi({\widehat} H_1)$. By Theorem~\ref{th1}(d), the LP-problem ${\mathcal{P}}( {\widehat} H_1, d) =  {\mathcal{P}}( \Psi({\widehat} H_1), d) $, associated with ${\widehat} H_1$, can be effectively constructed and the coefficient  $\sigma_d({\widehat} H_1)$ can be computed, as claimed in part (b).
\smallskip

(c) We will continue to use the notation introduced above.
Suppose that all factors $ G_{\alpha}$, where ${\alpha} \in I(H_1)= \{ 1, \dots, m \}$, are finite and $H_1$ is effectively given by its irreducible ${\mathcal{A}}$-graph $Y_1 := \Psi(H_1)$ with $\operatorname{core}(Y_1) = Y_1$ (or by  a finite generating set;  Lemma~\ref{Lm1} reduces the latter case to the former one).
By Lemma~\ref{lemsup}, when computing the number
$\sigma( H_1) = \sup  \tfrac {{\bar {\mathrm{r}}}(H_1, H_2)}{{\bar {\mathrm{r}}}(H_1) {\bar {\mathrm{r}}}(H_2)}$,
we may assume that the subgroup $H_2$ has property (B)   and satisfies the condition $I(H_2) \subseteq I(H_1)$.  The latter condition implies that the degree of any primary vertex of $Y_2 := \Psi(H_2)$ does not exceed  $ | I(H_1)|$. On the other hand,  the degree of every secondary vertex of
$Y_2$ does not exceed   $\max \{ | G_{\alpha} |   :    {\alpha} \in   I(H_1) \}$. Hence, the degree $\deg v$ of every vertex $v$ of $Y_2$ satisfies
\begin{equation}\label{eq11}
 \deg v \le d' := \max \{ | I(H_1)|, \, \max \{   | G_{\alpha} |   :   {\alpha} \in   I(H_1)  \} \, \} .
 \end{equation}
 Thus, by Lemma~\ref{lemsup}, we may conclude that
 $$
 \sigma( H_1) =  \sigma_{d'}( H_1) = \sup \tfrac {{\bar {\mathrm{r}}}(H_1, H_2)}{{\bar {\mathrm{r}}}(H_1)  {\bar {\mathrm{r}}}(H_2)} ,
 $$
where the supremum is taken over all subgroups $H_2$ with property (Bd), where $d = d'$.
Applying Lemma~\ref{lemmap2}(e) to $H_1$, we obtain $\sigma(H_1) =  \sigma_{d'}( H_1) \le \sigma_{d'}({\widehat} H_1)$. Recall that an irreducible graph $\Psi({\widehat} H_1)$ of  ${\widehat} H_1 = \mu_2(H_1)$  can be effectively constructed from $Y_1$ (for details see the proof of Lemma~\ref{lemmap2}(c)) and the word  problem is solvable for factors of the free product ${\mathcal{F}}_2(1) = G_1 * G(2,m)$. Invoking Theorem~\ref{th1}(d), we see that the LP-problem
 $ {\mathcal{P}}({\widehat} H_1, d') = {\mathcal{P}}(\Psi({\widehat}  H_1), d')$ can be written down and hence the coefficient $\sigma_{d'}({\widehat} H_1)$ can be computed, as desired.   The proof of Theorem~\ref{th3} is complete.
\end{proof}

In conclusion, we mention that it is not clear whether there is a duality gap between
the LSIP-problem $ \sup \{ - x_s \mid \operatorname{\textsf{SLI}}[Y_1] \}$, introduced in Sect.~4, and its dual problem \eqref{dlp} and
it would be of interest to find this out. Another natural problem
is to find an algorithm that solves the dual  problem  \eqref{dlp} of the LSIP-problem $ \sup \{ - x_s \mid \operatorname{\textsf{SLI}}[Y_1] \}$
and thereby effectively  computes the WN-coefficient $\sigma(\Psi(H_1)) =  \sigma(H_1)$ for a finitely generated factor-free subgroup $ H_1$ of the free product of two groups (and, perhaps, more than two groups) which are not necessarily finite.  It would also be interesting to find an algorithm  that computes the Hanna Neumann coefficient $\bar \sigma(H_1)$ for a finitely  generated factor-free noncyclic subgroup $ H_1$ of the free product ${\mathcal{F}}$ of two finite groups which is defined as
$
\bar \sigma(H_1) := \sup \tfrac {{\bar {\mathrm{r}}}(H_1 \cap H_2)} {{\bar {\mathrm{r}}}(H_1) {\bar {\mathrm{r}}}(H_2)}
$
over all finitely  generated factor-free noncyclic subgroups $H_2$ of ${\mathcal{F}}$.
\medskip

{\em Acknowledgments.}  The author is grateful to the referee for helpful remarks and suggestions.

\begin{thebibliography}{[10]}

\bibitem[1]{ABC} Y. Antol\'in, A. Martino, and I. Schwabrow,
{\em Kurosh rank of intersections of subgroups of free products of right-orderable groups},
preprint, \texttt{http://arxiv.org/abs/1109.0233v3}

\bibitem[2]{LSIP1}  A. Charnes, W. W. Cooper, K. O. Kortanek, {\em Duality, Haar
programs and finite sequence spaces},  Proc. Nat. Acad. Sci. U.S.A.  {\bf 48}(1962) 783--786.

\bibitem[3]{LSIP2} A. Charnes, W. W. Cooper and K. Kortanek, {\em On representations of semi-infinite program
which have no duality gaps}, Management Sci. {\bf 12}(1965), 113--121.

\bibitem[4]{D}  W. Dicks,  {\em Equivalence of the strengthened
Hanna Neumann conjecture and the amalgamated graph conjecture},
Invent. Math. {\bf  117}(1994), 373--389.

\bibitem[5]{D2}  W. Dicks, {\em Simplified Mineyev}, preprint,
 \texttt{http://mat.uab.cat/$\! \sim$dicks/SimplifiedMineyev.pdf}

\bibitem[6]{DIv}  W. Dicks and S. V. Ivanov, {\em On the intersection
of free subgroups in free products of groups},   Math. Proc. Cambridge
Phil. Soc. {\bf 144}(2008), 511--534.

\bibitem[7]{DIv2}   W. Dicks and S. V. Ivanov, {\em On the intersection
of free subgroups in free products of groups with no
2-torsion},   Illinois J. Math. {\bf 54}(2010), 223--248.

\bibitem[8]{Fr}
J. Friedman, {\em Sheaves on graphs, their homological invariants, and a proof of
the Hanna Neumann conjecture: with an appendix by Warren Dicks},
Mem. Amer. Math. Soc. {\bf  233}(2014), no. 1100. xii+106 pp.

\bibitem[9]{LSIP3}   M. A. Goberna and M. A. L\'opez,
{\em Linear semi-infinite programming theory: an updated survey},
European J. Oper. Res.  {\bf 143}(2002), 390--405.

\bibitem[10]{Haken} W. Haken, {\em Theorie der Normalfl\"achen},
Acta Math. {\bf 105}(1961), 245--375.

 \bibitem[11]{HLP}  J. Hass, J. C. Lagarias  and N. Pippenger,
 {\em The computational complexity of knot and link problems},
 J. Assoc. Comput. Mach. {\bf 46}(1999), 185--211.

\bibitem[12]{Hemion}  G. Hemion,
 {\em  The classification of knots and 3-dimensional spaces},
 Oxford Univ. Press, 1993.

\bibitem[13]{Iv99}  S. V. Ivanov, {\em  On the intersection of
finitely generated subgroups in free products of groups},  {
Internat. J. Algebra and Comp.}  {\bf 9}(1999), 521--528.

\bibitem[14]{Iv01}  S. V. Ivanov, {\em Intersecting free subgroups
in free products of groups},  { Internat. J. Algebra and Comp.}
{\bf 11}(2001), 281--290.

\bibitem[15]{Iv08} S. V. Ivanov,  {\em On the Kurosh rank of the
intersection of subgroups in free products of groups}, Adv.  Math. {\bf
218}(2008), 465--484.

\bibitem[16]{Iv08s} S. V. Ivanov,  {\em The computational complexity of basic decision problems in 3-dimensional topology},  Geom. Dedicata {\bf 131}(2008), 1--26.

\bibitem[17]{Iv10}  S. V. Ivanov,
{\em A property of groups and the Cauchy–-Davenport theorem},
{J. Group Theory} {\bf 13}(2010), 21--39.

\bibitem[18]{Iv12}  S. V. Ivanov, {\em Intersecting free subgroups in free products of left ordered groups}, submitted.

\bibitem[19]{JT}  W. H. Jaco and J. L. Tollefson, {\em Algorithms for the complete decomposition of a closed 3-
manifold}, Illinois J. Math. {\bf 39}(1995), 358--406.

\bibitem[20]{K} A. G. Kurosh, {\em The theory of groups},
Chelsea, 1956.

\bibitem[21]{LS} R. C. Lyndon and P. E. Schupp, {\em Combinatorial
group theory}, Springer-Verlag, 1977.

\bibitem[22]{Min} I. Mineyev, {\em Submultiplicativity and the Hanna Neumann conjecture},  Ann.  Math. {\bf 175}(2012), 393--414.

\bibitem[23]{N1}  H. Neumann, {\em On the intersection of finitely generated
free groups}, {  Publ. Math.} {\bf  4}(1956), 186--189; Addendum,
{ Publ. Math.} {\bf  5}(1957), 128.

\bibitem[24]{N2} W. D. Neumann, {\em On the intersection of finitely
generated subgroups of free groups},   { Lecture Notes in Math.
(Groups-Canberra 1989)}  {\bf  1456}(1990), 161--170.

\bibitem[25]{S86}
A.~Schrijver, {\em  Theory of linear and integer programming},
John Wiley \& Sons, 1986.

\bibitem[26]{St}  J. R. Stallings, {\em Topology of finite
graphs},  {Invent. Math.} {\bf  71}(1983), 551--565.

\bibitem[27]{T}   G. Tardos, {\em On the intersection of subgroups
of a free group}, { Invent. Math.} {\bf  108}(1992), 29--36.
\end{thebibliography}
\end{document}


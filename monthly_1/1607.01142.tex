\documentclass[12pt]{amsart}
\voffset=-.5cm
\textwidth=15cm
\textheight=23cm
\oddsidemargin=1cm
\evensidemargin=1cm

\usepackage{amsmath}
\usepackage{amsxtra}
\usepackage{amscd}
\usepackage{amsthm}
\usepackage{amsfonts}
\usepackage{amssymb}
\usepackage {pstricks}
\usepackage{pstricks,pst-node}
\usepackage[all]{xy}

\usepackage{amsbsy}
\usepackage{mathrsfs}

\usepackage{verbatim}

\usepackage{color}
\usepackage{indentfirst}
\usepackage{latexsym,bm}
\usepackage{graphicx}
\usepackage{cases}
\usepackage{pifont}
\usepackage{txfonts}
\usepackage{xcolor}

\usepackage{longtable}
\usepackage{setspace}

\usepackage{cases}
\usepackage{multirow}
\usepackage{array}
\usepackage{makecell}
\usepackage{supertabular}
\usepackage{booktabs}

\newtheorem{theorem}{Theorem}[section]
\newtheorem{assertion}[theorem]{Assertion}
\newtheorem{lemma}[theorem]{Lemma}

\newtheorem{corollary}[theorem]{Corollary}
\newtheorem{proposition}[theorem]{Proposition}

\theoremstyle{definition}
\newtheorem{definition}[theorem]{Definition}
\newtheorem{example}[theorem]{Example}
\newtheorem{scholium}[theorem]{Scholium}

\theoremstyle{remark}
\newtheorem{remark}[theorem]{Remark}

\numberwithin{equation}{section}
\newtheorem{case}{Case}

\newtheorem{subcase}{Subcase}[case]

  

\allowdisplaybreaks

\begin{document}
\normalfont
\sloppy

\title[Quantum correspondences]{Quantum correspondences of affine Lie superalgebras}
\author{Ying Xu}
\author{R. B. Zhang}
\address[Xu]{School of Mathematics, Hefei University of Technology, Anhui Province, 230009, China}
\address[Xu, Zhang]{School of Mathematics and Statistics,
University of Sydney, NSW 2006, Australia}
\email{xuying@hfut.edu.cn, ruibin.zhang@sydney.edu.au}
\begin{abstract}
An intriguing isomorphism between the quantised universal enveloping algebras of ${{\rm\mathfrak{osp}}}(1|2n)$ and ${{\rm\mathfrak{so}}}(2n+1)$ was discovered in the early 90s.
This same isomorphism emerged in recent work of Mikhaylov and Witten
in the context of string theory as a $T$-duality composed with an S-duality.
We construct similar isomorphisms for families of pairs of quantum affine superalgebras, and show that the representation categories of the quantum affine superalgebras in each pair are equivalent as strict tensor categories.
\end{abstract}
\subjclass[2010]{15A72,17B20}
\keywords{affine Lie superalgebras, quantum affine superalgebras, Hopf superalgebras, tensor categories}
\maketitle

\section{Introduction}\label{sect:introduction}
In the early 80s, Rittenberg and Scheunert  \cite{RS} observed a remarkable connection between the orthogonal Lie algebra ${{\rm\mathfrak{so}}}(2n+1)$ and
the orthosymplectic Lie superalgebra ${{\rm\mathfrak{osp}}}(1|2n)$: there is a one-to-one
correspondence between the finite dimensional  representations of
${{\rm\mathfrak{osp}}}(1|2n)$ and the tensorial representations of ${{\rm\mathfrak{so}}}(2n+1)$,
and the central characters of the two algebras
in the corresponding simple modules are the same.
This remained rather mysterious until
quantum groups \cite{D1, J} and quantum supergroups \cite{BGZ, Y,  ZGB,Z93, Z98}
came to the scene.
It was discovered in \cite{Z3} that the quantised universal enveloping
algebras of these Lie (super)algebras are essentially isomorphic, and the
Rittenberg-Scheunert correspondence is a consequence of this isomorphism
in the semiclassical limits. Similar isomorphisms between some
quantum affine algebras and quantum affine superalgebras were discovered in \cite{Z2}.

The isomorphism between the quantum (super)groups of ${{\rm\mathfrak{osp}}}(1|2n)$ and  ${{\rm\mathfrak{so}}}(2n+1)$ 
was used by Lanzmann to great effect \cite{LE} in the study of primitive ideals of ${{\rm U}}({{\rm\mathfrak{osp}}}(1|2n))$. 
By relating them to the primitive ideals of ${{\rm U}}({{\rm\mathfrak{so}}}(2n+1))$ via the isomorphism, he drastically simplified the proofs of the results first obtained in \cite{GL}.

The same isomorphism emerged in very recent work of Mikhaylov and Witten \cite{MW} on quantum Chern-Simons theories.
The authors gave a description of Chern-Simons theories with super gauge groups
in terms of systems of D3-branes ending on 2-sides of an NS5-brane.
A T-duality composed with an S-duality of the brane systems interchanges
the corresponding quantum Chern-Simons theories
with gauge groups ${{\rm\mathfrak{osp}}}(1|2n)$ and ${{\rm\mathfrak{so}}}(2n+1)$ respectively.
The strong-weak coupling transformation procured by the T-duality corresponds
precisely to the interchange $q\leftrightarrow  -q$ in the quantum group context \cite{Z3}. Furthermore,  a similar duality between
quantum Chern-Simons theories
with gauge groups ${{\rm\mathfrak{osp}}}(2m+1|2n)$ and  ${{\rm\mathfrak{osp}}}(2n+1|2m)$ was constructed for arbitrary $m$ and $n$ in \cite{MW}.

The aim of the present paper is to give a catalogue of the isomorphisms analogous to 
that between the quantum (super)groups of ${{\rm\mathfrak{osp}}}(1|2n)$ and  ${{\rm\mathfrak{so}}}(2n+1)$.

Let ${{\mathfrak g}}$ be a Lie superalgebra or an affine Lie superalgebra in the classical series.
Fixing a fundamental system $\Pi$
of simple roots (Definition \ref {defi:quantised})  corresponding to an arbitrary choice of Borel subalgebra, 
we let ${{\rm U}}_{q}({{\mathfrak g}}, \Pi)$ be the quantised universal enveloping superalgebra of ${{\mathfrak g}}$ with respect to this fundamental system. 
We note that in general, the quantised universal enveloping superalgebras ${{\rm U}}_{q}({{\mathfrak g}}, \tilde\Pi)$ associated with different fundamental systems $\tilde\Pi$
are not isomorphic as Hopf superalgebras.  This already happens in the case of ${{\rm\mathfrak{sl}}}(2|1)$. 

Corresponding to each $\alpha_i$ in $\Pi$, we introduce a ${{\mathbb Z}}_2$ group generated by $\sigma_i$ such that $\sigma_i^2=1$, and let $G$ be the direct product of all 
the ${{\mathbb Z}}_2$ groups, i.e.,  $\mathrm{G}:={{\mathbb Z}}_2\times \dots\times {{\mathbb Z}}_2$ ($|\Pi|$ copies).
We introduce an action of  $\mathrm{G}$ on ${{\rm U}}_q({{\mathfrak g}}, \Pi)$ 
such that there exists an element $u\in \mathrm{G}$ which implements the  ${{\mathbb Z}}_2$-grading.
Let ${{\mathfrak U}}_{q}({{\mathfrak g}}, \Pi)={{\rm U}}_{q}({{\mathfrak g}}, \Pi)\sharp{{\mathbb C}}\mathrm{G}$ be the smash product (Definition \ref {defi:smash product}), which
has a natural Hopf superalgebra structure by Proposition \ref{prop:hopf}. 
We have the following result.
\begin{theorem}\label{thm:main-quan}
Let $({{\mathfrak g}}, {{\mathfrak g}}')$ be any pair of Lie superalgebras or affine Lie superalgebras listed in Table \ref{tbl:q-correspond} (where $m+n>0$).

\vspace{-1mm}
\begin{table}[!hbp]

\caption{Quantum correspondences}
\label{tbl:q-correspond}
\begin{tabular}{c|c|c|c}
\hline
${{\mathfrak g}}$ & ${{\rm\mathfrak{osp}}}(2m+1|2n)$ & ${{\rm\mathfrak{sl}}}(2m+1|2n)^{(2)}$  & ${{\rm\mathfrak{osp}}}(2m+2|2n)^{(2)}$ \\
\hline
${{\mathfrak g}}' $ & ${{\rm\mathfrak{osp}}}(2n+1|2m)$ & ${{\rm\mathfrak{osp}}}(2n+1|2m)^{(1)}$ & ${{\rm\mathfrak{osp}}}(2n+2|2m)^{(2)}$  \\
\hline
\end{tabular}
\end{table}

\vspace{-1mm}
\noindent
For any chosen fundamental system $\Pi$ of ${{\mathfrak g}}$ and the corresponding  fundamental system $\Pi'=\phi(\Pi)$ of ${{\mathfrak g}}'$ (see Lemma \ref{lem:connection phi}),  
there exists an isomorphism of associative algebras
\begin{eqnarray*}
{{\mathfrak U}}_{-q}({{\mathfrak g}}', \Pi')\stackrel{\cong}{\longrightarrow}{{\mathfrak U}}_{q}({{\mathfrak g}}, \Pi),
\end{eqnarray*}
which is defined explicitly in Theorem \ref{thm:iso-main}.
\end{theorem}

Theorem \ref{thm:main-quan} was proved in \cite{Z3, Z2} for the pairs $({{\mathfrak g}}, {{\mathfrak g}}')$ with either ${{\mathfrak g}}$ or ${{\mathfrak g}}'$ being an ordinary (affine) Lie algebra; see Section  \ref{remk:nonisotropic} for more details.  Also, the quantum correspondence 
between ${{\rm\mathfrak{osp}}}(2m+1|2n)$ and ${{\rm\mathfrak{osp}}}(2n+1|2m)$  for arbitrary $m$ and $n$
was discovered in \cite{MW} in the string theory context, as we have already mentioned.

As it stands, the isomorphism in Theorem \ref{thm:main-quan}
does not preserve the ${{\mathbb Z}}_2$-gradings, thus can not be a superalgebra isomorphism, let alone a Hopf superalgebra isomorphism.
However, it relates the Hopf superalgebra structures
at a more fundamental level.

A useful way to look at Hopf superalgebras was advocated by Majid  and others (see e.g., \cite[Chapter 10.1]{Ma} and \cite{AEG}) in the context of braided tensor categories.  
The category of vector superspaces can be considered as the tensor category of representations of the group algebra of ${{\mathbb Z}}_2$ regarded as a  triangular Hopf algebra. 
A Hopf superalgebra is then a Hopf algebra in this category.
Given a Hopf superalgebra, one may change the ${{\mathbb Z}}_2$-action to obtain a new Hopf
superalgebra with the same underlying associative algebra structure, but has different ${{\mathbb Z}}_2$-grading.  We refer to this process as a {\em picture change} (see Definition \ref{def:PC}), which is also loosely known as {\em bosonisation} in the literature   \cite{Ma}
(see Remark \ref{rem:bosonisation}).  
Now the representation category of the new Hopf superalgebra is
equivalent to that of the original Hopf superalgebra as strict tensor category (see  \cite[ Chapter 10.1]{Ma} and \cite[Theorem 3.1.1]{AEG}).

Given a pair $({{\mathfrak g}}, {{\mathfrak g}}')$ of (affine) Lie superalgebras from Theorem \ref{thm:main-quan},
we apply an appropriate picture change
to ${{\mathfrak U}}_q({{\mathfrak g}}, \Pi)$ to obtain a new Hopf superalgebra, which turns out to be isomorphic to
 ${{\mathfrak U}}_{-q}({{\mathfrak g}}', \Pi')$ in a nontrivial way.  The superalgebra isomorphism is that of Theorem \ref{thm:main-quan},  but with ${{\mathfrak U}}_q({{\mathfrak g}}, \Pi)$ endowed a new ${{\mathbb Z}}_2$-grading.
However, the coalgebraic structures are isomorphic via a Drinfeld twist, see Theorem \ref{them: hopf connection} for details.

For easy reference, we shall call this Hopf superalgebra isomorphism a  {\em quantum correspondence} between the (affine) Lie superalgebras ${{\mathfrak g}}$ and ${{\mathfrak g}}'$.
The following result is an immediate consequence of the quantum correspondence.
\begin{theorem}\label{thm:tensor-equiv} Let  $({{\mathfrak g}}, {{\mathfrak g}}')$ be 
any pair  of affine Lie superalgebras in Theorem \ref{thm:main-quan}. 
For any fundamental system $\Pi$ of ${{\mathfrak g}}$ and the corresponding  fundamental system $\Pi'=\phi(\Pi)$ of ${{\mathfrak g}}'$,  the representation categories of
the Hopf superalgebras ${{\mathfrak U}}_{q}({{\mathfrak g}}, \Pi)$ and ${{\mathfrak U}}_{-q}({{\mathfrak g}}', \Pi')$ are 
equivalent as strict tensor categories.
\end{theorem}

The remainder of the paper is devoted to the proofs of Theorems \ref{thm:main-quan} and \ref{thm:tensor-equiv}.  
Both theorems will be incorporated into Theorem \ref{them: hopf connection}. We will also define quantum correspondences more precisely, see Definition \ref{def:correspond}.

 \section{Quantised universal enveloping superalgebras}

\subsection{Root Systems}\label{sect:roots}

We begin with a description of the root data of the classical series of Lie superalgebras and the related twisted and untwisted affine Lie superalgebras.  

For any given pair of nonnegative integers $k$ and $l$, we let ${{\mathcal E}}(k|l)$ be the $(k+l)$-dimensional vector space over ${{\mathbb R}}$
with a basis consisting  of elements $\varepsilon_i$ ($i=1, 2, \dots, k$)
and $\delta_\nu$ ($\nu=1, 2, \dots, l$). We endow the vector space with a symmetric non-degenerate bi-linear form defined by
\begin{equation}\label{eq:bilinear form}
\begin{aligned}
(\varepsilon_i, \varepsilon_j)=(-1)^{\theta}\delta_{i j}, \quad (\delta_\mu, \delta_\nu)=-(-1)^{\theta}\delta_{\mu \nu}, \quad
(\varepsilon_i, \delta_\mu)=(\delta_\mu, \varepsilon_i)=0,
\end{aligned}
\end{equation}
where $\theta$ is $0$ or $1$ which will be fixed in the following way.
Call an order of the basis elements \emph{admissible} if $\varepsilon_i$ appears before
$\varepsilon_{i+1}$ for all $i$,  and $\delta_\nu$ before $\delta_{\nu+1}$
for all $\nu$. Fix an admissible order and denote by $({\mathcal E}_1, {\mathcal E}_2, \dots, {\mathcal E}_{k+l})$ the ordered basis of ${{\mathcal E}}(k|l)$. Then we choose $\theta$ so that
$
({\mathcal E}_1, {\mathcal E}_1)=1.
$

Let ${{\mathfrak g}}$ be either a special linear or orthosymplectic Lie superalgebra. Then the set $\Phi$ of roots of ${{\mathfrak g}}$ can be realized as a subset 
of ${{\mathcal E}}(k|l)$ for appropriate $k$ and $l$, where we will take the $k$ and $l$ to be the smallest possible.
We will call ${{\mathcal E}}(k|l)$ the ambient space of $\Phi$.

Each choice of a Borel subalgebra corresponds to a choice of positive roots, and hence a fundamental system $\Pi=\{\alpha_1, \alpha_2, \dots, \alpha_r\}$ of simple roots, where $r$ is the rank of ${{\mathfrak g}}$.
The Weyl group conjugacy classes of Borel subalgebras correspond bijectively to the admissible ordered bases of the ambient space.

Now the root data of the classical series of simple Lie superalgebras can be described as follows, where the ambient space of $\Phi$ is ${{\mathcal E}}(m|n)$ in each case.

\vspace{-2mm}
\begin{table}[h]

\centering
\caption{Classical series of Lie superalgebras}
\label{table:classical}
\begin{tabular}{c|c}
\hline
${{\mathfrak g}}$ &\text{simple roots}\\
\hline
${{\rm\mathfrak{sl}}}(m|n)$ & $\alpha_i={\mathcal E}_i - {\mathcal E}_{i+1},\ \ 1\le i< m+n$\\
\hline
${{\rm\mathfrak{osp}}}(2m+1|2n)$ & $\alpha_i={\mathcal E}_i - {\mathcal E}_{i+1}, \ \ 1\le i<m+n, \quad
  \alpha_{m+n}= {\mathcal E}_{ m+n}$\\
  \hline
\multirow{3}{*}{${{\rm\mathfrak{osp}}}(2m|2n)$} & $\alpha_i={\mathcal E}_i - {\mathcal E}_{i+1}, \ \ 1\le i<m+n, $ \\
& $\alpha_{m+n}=\begin{cases}
{\mathcal E}_{m+n-1}+{\mathcal E}_{m+n}, &\text{ if ${\mathcal E}_{m+n}=\varepsilon_{m}$},\\
2{\mathcal E}_{m+n}, &\text{ if ${\mathcal E}_{m+n}=\delta_{n}$}.
\end{cases}$\\
\hline
\end{tabular}
\end{table}

\vspace{-4mm}
\begin{remark}
The Lie superalgebras  ${{\rm\mathfrak{osp}}}(m|n)$ and ${{\rm\mathfrak{sl}}}(m|n)$ reduce ordinary Lie algebras
if $m=0$ or $n=0$.
Also, ${{\rm\mathfrak{sl}}}(m|m)$ contains the ideal ${{\mathbb C}} 1_{2m}$, and ${{\rm\mathfrak{sl}}}(m|m)/{{\mathbb C}}1_{2m}$ is simple.
\end{remark}

In order to describe the root data of untwisted and twisted affine Lie superalgebras of the classical series of Lie superalgebras discussed above,  
we introduce the vector space $\mathcal{E}_{\delta}(k|l)$, which has a basis consisting of the basis elements of $\mathcal{E}(k|l)$ 
and the additional element ${\mathcal E}_0=\delta$. We extend the bilinear form on $\mathcal{E}(k|l)$ to $\mathcal{E}_\delta(k|l)$ by setting
\[
(\mathcal{E}_0, \mathcal{E}_i)=(\mathcal{E}_i, \mathcal{E}_0)=0, \ \forall i=0, 1, \dots, k+l.
\]
The resulting form still has rank $k+l$ and is degenerate.
The affine root data can be described as in Table \ref{table:affine} (see \cite{K1, K2, JWV}), where the ambient space of the roots is ${{\mathcal E}}_\delta(m|n)$ in each case.

\begin{table}[h]

\centering
\caption{Classical series of affine Lie superalgebras}
\label{table:affine}
\begin{tabular}{c|c}
\hline
 ${{\mathfrak g}}$ &\text{simple roots}\\
\hline
\multirow{2}{*}{${{\rm\mathfrak{sl}}}(m|n)^{(1)}$} &$\alpha_i={\mathcal E}_i - {\mathcal E}_{i+1}$, \ \ $1\le i< m+n$,\\
& $\alpha_0={\mathcal E}_0-{\mathcal E}_1 + {\mathcal E}_{m+n+1}.$\\
\hline
\multirow{3}{*}{${{\rm\mathfrak{osp}}}(2m+1|2n)^{(1)}$}  &$\alpha_i={\mathcal E}_i - {\mathcal E}_{i+1}$, \ \ $1\le i<m+n$, \ \  $\alpha_{m+n}= {\mathcal E}_{ m+n}$,\\
					& $\alpha_0=\begin{cases}
{\mathcal E}_0-{\mathcal E}_1-{\mathcal E}_{2},&\text{if ${\mathcal E}_1=\varepsilon_{1}$},\\
{\mathcal E}_0-2{{\mathcal E}_1},&\text{if ${\mathcal E}_1=\delta_{1}$}.
\end{cases}$ \\
\hline
\multirow{6}{*}{${{\rm\mathfrak{osp}}}(2m|2n)^{(1)}$}  & $\alpha_i={\mathcal E}_i - {\mathcal E}_{i+1},\ \ 1\le i<m+n,$ \\
&$\alpha_{m+n}=\begin{cases}
{\mathcal E}_{m+n-1}+{\mathcal E}_{m+n}, &\text{ if ${\mathcal E_{m+n}}={\varepsilon}_{m}$},\\
2{\mathcal E_{m+n}}, &\text{ if ${\mathcal E_{m+n}}={\delta}_{n}$},
\end{cases}$\\
& $\alpha_0=\begin{cases}
{\mathcal E}_0-{\mathcal E}_1-{\mathcal E}_{2},&\text{if ${\mathcal E}_1=\varepsilon_{1}$},\\
{\mathcal E}_0-2{\mathcal E}_1,&\text{if ${\mathcal E}_1=\delta_{1}$}.
\end{cases}$\\
\hline
\multirow{3}{*}{${{\rm\mathfrak{sl}}}(2m+1|2n)^{(2)}$}  &$\alpha_i={\mathcal E}_i - {\mathcal E}_{i+1}, \ \ 1\le i<m+n ,\ \ \alpha_{m+n}= {\mathcal E}_{ m+n},$\\
 &$\alpha_0=\begin{cases}
{\mathcal E}_0-2{\mathcal E}_1,&\text{if ${\mathcal E}_1=\varepsilon_{1}$},\\
{\mathcal E}_0-{\mathcal E}_1-{\mathcal E}_2,&\text{if ${\mathcal E}_1=\delta_{1}$}.
\end{cases}$ \\
\hline
\multirow{6}{*}{${{\rm\mathfrak{sl}}}(2m|2n)^{(2)}$}  &$\alpha_i={\mathcal E}_i - {\mathcal E}_{i+1} \ \ 1\le i<m+n,$ \\
						 &$\alpha_{m+n}=\begin{cases}
{\mathcal E}_{m+n-1}+{\mathcal E}_{m+n}, &\text{ if ${\mathcal E_{m+n}}={\varepsilon}_{m}$}, \\
2{\mathcal E_{m+n}}, &\text{ if ${\mathcal E_{m+n}}={\delta}_n$},
\end{cases}$\\
					 &$\alpha_0=\begin{cases}
{\mathcal E}_0-2{{\mathcal E}_1},&\text{if ${\mathcal E}_1=\varepsilon_{1}$},\\
{\mathcal E}_0-{\mathcal E}_1-{\mathcal E}_{2},&\text{if ${\mathcal E}_1=\delta_{1}$}.
\end{cases}$\\
\hline
 \multirow{2}{*}{${{\rm\mathfrak{osp}}}(2m+2|2n)^{(2)}$} &$\alpha_{i}={\mathcal E}_{i}-{\mathcal E}_{i+i} \ \ 1\le  i< m+n ,$\\ &$\alpha_{m+n}={{\mathcal E}}_{m+n},\ \ \ \alpha_{0}={\mathcal E}_0-{\mathcal E}_{1}.$\\
\hline
  \multirow{2}{*}{${{\rm\mathfrak{sl}}}(2m+1|2n+1)^{(4)}$} &$\alpha_{i}={\mathcal E}_{i}-{\mathcal E}_{i+i} \ \ 1\le  i< m+n ,$\\
&$\alpha_{m+n}={{\mathcal E}}_{m+n},\ \ \ \alpha_{0}={\mathcal E}_0-{\mathcal E}_{1}.$\\
\hline
\end{tabular}

\vspace{4mm}
Note: The imaginary root  ${\mathcal E}_0=\delta$ is even for all affine Lie superalgebras except
  ${{\rm\mathfrak{sl}}}(2m+1|2n+1)^{(4)}$, where it is odd.
\end{table}

The vector spaces ${\mathcal E}_{\delta}(m|n)$ and ${\mathcal E}_{\delta}(n|m)$ are both $(m+n+1)$-dimensional. To avoid confusion, we write the basis of ${\mathcal E}_{\delta}(n|m)$ as $\{\delta', {\varepsilonup}'_1,\dots {\varepsilonup}'_n,\delta'_1,\dots \delta'_m\}$. We consider the following
vector space isomorphism
\begin{eqnarray}\label{eq:phi-def}
\phi:  {\mathcal E}_{\delta}(m|n)\longrightarrow {\mathcal E}_{\delta}(n|m), \quad \delta\mapsto\delta', \ \
\varepsilon_i\mapsto \delta'_i, \  \  \delta_j\mapsto\varepsilon'_j, \  \  \forall i, j.
\end{eqnarray}
We will still denote its restriction to ${\mathcal E}(m|n)$ by $\phi$.

Clearly $\phi$ sends an admissible basis of ${\mathcal E}(m|n)$ to an admissible basis of
${\mathcal E}(n|m)$.  If ${\mathcal E}(m|n)$ or ${\mathcal E}_\delta(m|n)$ is the ambient space of the root system of a Lie superalgebra or affine Lie superalgebra ${{\mathfrak g}}$, and $\Pi$ is a fundamental system of ${{\mathfrak g}}$, then the set $\phi(\Pi)$ may be a fundamental system of another (affine) Lie superalgebra with ${\mathcal E}(n|m)$ or ${\mathcal E}_{\delta}(n|m)$ as the ambient space of roots. This happens in the following cases.
\begin{lemma}\label{lem:connection phi}
The map $\phi$ induces a one to one correspondences between fundamental systems of the (affine) Lie superalgebras in  each of the following pairs $({{\mathfrak g}}, {{\mathfrak g}}')$:
\begin{enumerate}
\item[(i).] those listed in Table \ref{tbl:q-correspond};
\item[(ii).] \label{sl-pairs}  and $({{\rm\mathfrak{sl}}}(m|n), {{\rm\mathfrak{sl}}}(n|m))$, $({{\rm\mathfrak{sl}}}(m|n)^{(1)}, {{\rm\mathfrak{sl}}}(n|m)^{(1)})$, $({{\rm\mathfrak{sl}}}(2m|2n)^{(2)}, {{\rm\mathfrak{sl}}}(2n|2m)^{(2)})$,\\  $({{\rm\mathfrak{sl}}}(2m+1|2n)^{(2)}, {{\rm\mathfrak{sl}}}(2n|2m+1)^{(2)})$, $({{\rm\mathfrak{sl}}}(2m+1|2n+1)^{(4)},{{\rm\mathfrak{sl}}}(2n+1|2m+1)^{(4)})$.
\end{enumerate}
\end{lemma}

\begin{remark} For all the pairs in case (ii), we have ${{\mathfrak g}}={{\mathfrak g}}'$. This is why we do not consider them when studying quantum correspondences.
\end{remark}

\subsection{Quantum superalgebras} Hereafter we will only
consider the Lie superalgebras in Table \ref{table:classical} and  affine Lie superalgebras in Table \ref{table:affine}.

Let ${{\mathfrak g}}$ be such a Lie superalgebra or affine Lie superalgebra with  a fundamental system $\Pi$.
For ${{\mathfrak g}}$ in Table \ref{table:classical}, let $\Pi=\{\alpha_i\mid i=1, 2, \dots, m+n\}$, and let  $\tau\subset\{1, 2, \dots, m+n\}$ be the labelling set of the odd simple roots, i.e.,  $\{\alpha_s\mid s\in\tau\}$ is the subset of $\Pi$ consisting the odd simple roots.  Similarly, for ${{\mathfrak g}}$ in Table \ref{table:affine}, let $\Pi=\{\alpha_i\mid i=0, 1, 2, \dots, m+n\}$, and let  $\tau\subset\{0, 1, 2, \dots, m+n\}$ be the labelling set of the odd simple roots.
Define $b_{i j}=({\alpha}_i,{\alpha}_j)$ for all $i, j$. Then the
Cartan matrix of ${{\mathfrak g}}$ corresponding to $\Pi$ is given by
\[
A=(a_{ij}) \quad\text{with}\quad
a_{ij}=\begin{cases}\dfrac{2b_{ij}}{b_{ii}},&\mbox{if}~b_{ii}\neq 0\\b_{ij},&\mbox{if}~b_{ii}=0
\end{cases}.
\]
Note that $a_{i i}=0$ if and only if $\alpha_i$ is an isotropic odd simple root.
We will represent fundamental systems by Dynkin diagrams \cite{K1,K2,JWV, Z1},
and in doing that, we will follow the convention of Kac \cite{K1}.  In particular, a node
$\circ$ corresponds to an even simple root;
$\otimes$ to an odd isotropic simple root;
$\bullet$ to an odd non-isotropic simple root, and
$\times$ stands for $\circ$ or $\otimes$, depending on whether the simple root is even or odd.  Note that
the sub-diagrams
\begin{picture}(72, 20)(4, 6)
\put(10, 10){\circle{10}}
\put(16, 10){\line(1, 0){18}}
\put(15,7){$<$}
\put(35, 6){\Large$\otimes$}
\put(46, 10){\line(1, 0){18}}
\put(26, 12){\tiny $2$}
\put(58,7){$>$}
\put(65, 6){\Large$\otimes$}
\end{picture}
 and
 \begin{picture}(75, 20)(25, 6)
\put(30, 10){\circle{10}}
\put(37, 10){\line(1, 0){18}}
\put(35,7){$<$}
\put(55, 6){\Large$\otimes$}
\put(65, 10){\line(1, 0){18}}
\put(47, 12){\tiny $2$}
\put(78,7){$>$}
\put(90, 10){\circle{10}}
\end{picture}
correspond respectively to the sub-matrices  $\begin{bmatrix} 2 &-1 &0\\ -2& 0& 1\\0 &1 &0 \end{bmatrix}$ and $\begin{bmatrix} 2 &-1 &0\\ -2& 0& 1\\0 &-1 &2 \end{bmatrix}$ in Cartan matrices.

For convenience, we take a slight variation of the usual definition \cite{BGZ, Y, ZGB} of quantised universal enveloping superalgebras (see Remark \ref{rem:def-change} below for further comments).
Let us
fix $q\in{\mathbb{{C}}}$ such that $q\ne 0, \pm 1$, and let $q^{1/2}$ be a fixed
square root of $q$.  Denote
\[
\begin{aligned}
q_i=\begin{cases}q^{\frac{(\alpha_i,\alpha_i)}{2}},&\mbox{if }({\alpha}_i,{\alpha}_i)\neq 0\\q,&\mbox{if }({\alpha}_i,{\alpha}_i)=0,
\end{cases}\qquad
\theta_i=\begin{cases}1,&\mbox{if }({\alpha}_i,{\alpha}_i)=1,2\\2,&\mbox{if }({\alpha}_i,{\alpha}_i)=0,4.\end{cases}
\end{aligned}
\]
Note that $q_i^{a_{ij}}=q_j^{a_{j,i}}=q^{(\alpha_i,\alpha_j)}$. In what follows, $[x,y]_v=xy-(-1)^{[x][y]}v yx$.
\begin{definition}\label{defi:quantised}
The {\em quantised universal enveloping superalgebra} ${{\rm U}}_q({{\mathfrak g}},\Pi)$ of ${{\mathfrak g}}$ with the fundamental system $\Pi$ is an associative superalgebra over ${\mathbb{{C}}}$ with identity, which is defined by the following presentation:
The generators are
$e_i,f_i,k_i^{\pm1}$, where $e_s,f_s, (s\in\tau),$ are odd and the rest are even, and the relations are given by
\begin{enumerate}
\item\qquad\qquad\quad\quad\quad\quad $k_ik_i^{-1}=k_i^{-1}k_i=1,\quad k_ik_j=k_jk_i,$
\[
\begin{array}{l}
k_ie_jk_i^{-1}=q_i^{a_{ij}}e_j,\quad k_if_jk_i^{-1}=q_i^{-a_{ij}}f_j,\\
e_if_j-(-1)^{[e_i][f_j]}f_je_i={\delta}_{ij}\dfrac{k_i-k_i^{-1}}{q^{\theta_i}-q^{-\theta_i}};
\end{array}
\]
\item\qquad\qquad if $a_{ss}=0$, \qquad\quad $(e_s)^2=(f_s)^2=0,$
\[
\begin{aligned}
&\text{ if } a_{ii}\neq0,i\neq j, \quad {\left(}\mbox{Ad}_{e_i}{\right)}^{1-a_{ij}}(e_j)={\left(}\mbox{Ad}_{f_i}{\right)}^{1-a_{ij}}(f_j)=0,
\end{aligned}\]
where $\mbox{Ad}_{e_i}(x)$ and $\mbox{Ad}_{f_i}(x)$ are defined by
\eqref{eq:Ad};

\item  and higher order Serre relations (see \cite{Z1}) associated with the following subdiagrams of Dynkin diagrams
\end{enumerate}

{\selectfont
(A)\quad \label{Serre:case-1}
\begin{picture}(65, 20)(13, 5)
\put(10, 7){$\times$}
\put(15, 10){\line(1, 0){20}}
 \put(35, 6){\Large$\otimes$ }
\put(45, 10){\line(1, 0){20}}
\put(62, 7){$\times$}
\put(8, -2){\tiny $s-1$}
\put(39, -2){\tiny $s$}
\put(60, -2){\tiny $s+1$}
\end{picture}
with $a_{s-1,s}=-a_{s,s+1}$, the
associated higher order Serre relations are
\[\begin{aligned}
\mbox{Ad}_{e_s}\mbox{Ad}_{e_{s-1}}\mbox{Ad}_{e_s}(e_{s+1})=0,\quad
\mbox{Ad}_{f_s}\mbox{Ad}_{f_{s-1}}\mbox{Ad}_{f_s}(f_{s+1})=0;
\end{aligned}\]

(B)\quad \label{Serre:case-2}
\begin{picture}(70, 20)(13, 5)
\put(10, 7){$\times$}
\put(15, 10){\line(1, 0){20}}
\put(35, 6){\Large$\otimes$}
\put(45, 11){\line(1, 0){17}}
\put(45, 9){\line(1, 0){17}}
\put(57, 6.5){$>$}
\put(70,10){\circle{10}}
\put(8, -2){\tiny $s-1$}
\put(39, -2){\tiny $s$}
\put(63, -2){\tiny $s+1$}
\put(76, 7){,}
\end{picture}
the associated higher order Serre elements are
\[\begin{aligned}
\mbox{Ad}_{e_s}\mbox{Ad}_{e_{s-1}}\mbox{Ad}_{e_s}(e_{s+1})=0,\quad
\mbox{Ad}_{f_s}\mbox{Ad}_{f_{s-1}}\mbox{Ad}_{f_s}(f_{s+1})=0;
\end{aligned}\]

(C) \quad \label{case-3}
\begin{picture}(68, 20)(15, 5)
\put(10, 6.5){$\times$}
\put(15, 10){\line(1, 0){20}}
\put(35, 6){\Large$\otimes$ }
\put(45, 11){\line(1, 0){17}}
\put(45, 9){\line(1, 0){17}}
\put(57, 6.5){$>$}
\put(70, 10){\circle*{10}}
\put(8, -2){\tiny $s-1$}
\put(39, -2){\tiny $s$}
\put(63, -2){\tiny $s+1$}
\put(76, 7){,}
\end{picture}
the associated higher order Serre relations are
\[\begin{aligned}
\mbox{Ad}_{e_s}\mbox{Ad}_{e_{s-1}}\mbox{Ad}_{e_s}(e_{s+1})=0,\quad
\mbox{Ad}_{f_s}\mbox{Ad}_{f_{s-1}}\mbox{Ad}_{f_s}(f_{s+1})=0;
\end{aligned}\]

(D)\quad \label{case-4}
\begin{picture}(72, 20)(10, 5)
\put(10, 10){\circle{10}}
\put(16, 10){\line(1, 0){18}}
\put(15,7){$<$}
\put(35, 6){\Large$\otimes$}
\put(46, 10){\line(1, 0){18}}
\put(26, 12){\tiny $2$}
\put(58,7){$>$}
\put(65, 6){\Large$\otimes$}
\put(3, -2){\tiny $s-1$}
\put(39, -2){\tiny $s$}
\put(63, -2){\tiny $s+1$}
\put(76, 7){,}
\end{picture}
the associated higher order Serre relations are
\[
\begin{aligned}
\left[{{\mbox{Ad}}}_{e_{s+1}}(e_{s}), \left[{{\mbox{Ad}}}_{e_{s+1}}(e_s) , {{\mbox{Ad}}}_{e_s}(e_{s-1})\right]_{v_1}\right]_{v_2}=0,\\
\left[{{\mbox{Ad}}}_{f_{s+1}}(e_{s}), \left[{{\mbox{Ad}}}_{f_{s+1}}(f_s) , {{\mbox{Ad}}}_{f_s}(f_{s-1})\right]_{v_1}\right]_{v_2}=0,
\end{aligned}\]
where $v_1=q^{-(\alpha_s,\alpha_{s+1})}, v_2=q^{(\alpha_s,\alpha_{s+1})}$;

(E)\quad \label{case-5}
\begin{picture}(100, 20)(30, 5)
\put(30, 10){\circle{10}}
\put(37, 10){\line(1, 0){18}}
\put(35,7){$<$}
\put(55, 6){\Large$\otimes$}
\put(65, 10){\line(1, 0){18}}
\put(47, 12){\tiny $2$}
\put(78,7){$>$}
\put(90, 10){\circle{10}}
\put(95, 10){\line(1, 0){20}}
\put(115, 6.5){$\times$}
\put(112, -2){\tiny $s+2$}
\put(26, -2){\tiny $s-1$}
\put(59, -2){\tiny $s$}
\put(82, -2){\tiny $s+1$}
\put(123, 7){,}
\end{picture}
the associated higher order Serre relations are
\[
\begin{aligned}
&\left[{{\mbox{Ad}}}_{e_{s+2}}({{\mbox{Ad}}}_{e_{s+1}}e_s),\left[{{\mbox{Ad}}}_{e_{s+1}}e_s, {{\mbox{Ad}}}_{e_s}e_{s-1}
\right]_{v_1}\right]=0,\\
&\left[{{\mbox{Ad}}}_{f_{s+2}}({{\mbox{Ad}}}_{f_{s+1}}f_s),\left[{{\mbox{Ad}}}_{f_{s+1}}f_s, {{\mbox{Ad}}}_{f_s}f_{s-1}
\right]_{v_1}\right]=0, \ \text{ $v_1=q^{-(\alpha_{s},\alpha_{s+1})}$};
\end{aligned}
\]

(F)\quad \label{case-6}
\begin{picture}(65, 30)(0, 3)
\put(9, 6.8){$\times$}
\put(15, 10){\line(1, 1){20}}
\put(15,10){\line(1, -1){20}}
\put(35, 26){\Large$\otimes$}
\put(39, 25){\line(0, -1){30}}
\put(41, 25){\line(0, -1){30}}
\put(35, -14){\Large$\otimes$}
\put(6, -2){\tiny $s-1$}
\put(47, 28){\tiny $s$}
\put(47, -13){\tiny $s+1$}
\put(50, 5){,}
\end{picture}
the associated higher order Serre relations are\\
\[\begin{aligned}
&{{\mbox{Ad}}}_{e_s}{{\mbox{Ad}}}_{e_{s+1}}(e_{s-1})-{{\mbox{Ad}}}_{e_{s+1}}{{\mbox{Ad}}}_{e_s}(e_{s-1})=0,\\
&{{\mbox{Ad}}}_{f_s}{{\mbox{Ad}}}_{f_{s+1}}(f_{s-1})-{{\mbox{Ad}}}_{f_{s+1}}{{\mbox{Ad}}}_{f_s}(f_{s-1})=0;
\end{aligned}\]}
where $\mbox{Ad}_{e_i}(x)$ and $\mbox{Ad}_{f_i}(x)$ are defined by
\begin{eqnarray}\label{eq:Ad}
\mbox{Ad}_{e_i}(x)=e_ix-(-1)^{[e_i][x]}k_ixk_i^{-1}e_i, \quad
\mbox{Ad}_{f_i}(x)=f_ix-(-1)^{[f_i][x]}k_i^{-1}xk_if_i.
\end{eqnarray}
\end{definition}

\begin{remark}\label{rem:def-change}  We have used $\dfrac{k_i-k_i^{-1}}{q^{\theta_i}-q^{-\theta_i}}$ instead of the standard expression $\dfrac{k_i-k_i^{-1}}{q_i-q_i^{-1}}$
in the third relation of (1).
A consequence is that $q^{\pm 1/2}$ never  appears in our definition of the quantised universal enveloping superalgebras.
\end{remark}

Corresponding to each simple root $\alpha_i\in \Pi$ of ${{\mathfrak g}}$,
we introduce a  group ${{\mathbb Z}}_2$ generated by $\sigma_i$ such that $\sigma_i^2=1$, and let $\mathrm{G}$ be the direct product of all such groups. 
Then $G={{\mathbb Z}}_2^{\times |\Pi|}$, where $|\Pi|$ denotes the cardinality of $\Pi$.
We define a $\mathrm{G}$-action on ${{\rm U}}_q({{\mathfrak g}},\Pi)$ by
\[
\begin{aligned}
&\sigma_i\cdot e_j=(-1)^{(\alpha_i,\alpha_j)}e_j, \quad \sigma_i\cdot f_j=(-1)^{-(\alpha_i,\alpha_j)}f_j, \quad \sigma_i\cdot k_j=k_j, \quad\text{$i\ne 0$}, \\
&\sigma_0\cdot e_j=(-1)^{\delta_{i,0}}e_j,\qquad \sigma_0\cdot f_j=(-1)^{-\delta_{i,0}}f_j, \qquad \sigma_0\cdot k_j=k_j, \quad \forall j,
\end{aligned}
\]
where the second line is present only when ${{\mathfrak g}}$ is an affine superalgebra.

\begin{definition}\label{defi:smash product}
Let ${{\mathfrak U}}_q({{\mathfrak g}},\Pi)={{\rm U}}_q({{\mathfrak g}},\Pi)\sharp{{\mathbb C}} \mathrm{G}$
denote the smash product superalgebra with ${{\mathfrak U}}_q({{\mathfrak g}},\Pi)_{\bar 0}={{\rm U}}_q({{\mathfrak g}},\Pi)_{\bar 0}\otimes{{\mathbb C}} \mathrm{G}$ and ${{\mathfrak U}}_q({{\mathfrak g}},\Pi)_{\bar 1}={{\rm U}}_q({{\mathfrak g}},\Pi)_{\bar 1}\otimes{{\mathbb C}} \mathrm{G}$, where the multiplication is defined by
\[
\begin{aligned}
(x\otimes \sigma)(x'\otimes \sigma')=x(\sigma\cdot x')\otimes \sigma\sigma', \quad \forall x, x'\in {{\rm U}}_q({{\mathfrak g}},\Pi), \ \sigma,\sigma'\in {{\mathbb C}}\mathrm{G}.
\end{aligned}
\]
\end{definition}
To simplify the notation, we write $x$ for  $x\otimes 1$ for any $x\in{{\rm U}}_q({{\mathfrak g}},\Pi)$, and
$\sigma$ for $1\otimes \sigma$ for any $\sigma\in{{\mathbb C}}\mathrm{G}$. Then in ${{\mathfrak U}}_q({{\mathfrak g}}, \Pi)$, we have
\begin{eqnarray}\label{eq:sigma-act}
\begin{aligned}
&\sigma_i e_j\sigma_i^{-1}=(-1)^{(\alpha_i,\alpha_j)}e_j,\ \
\sigma_i f_j\sigma_i^{-1}=(-1)^{-(\alpha_i,\alpha_j)}f_j, \ \
\sigma_i k_j\sigma_i^{-1}=k_j, \ \  i\ne 0,\\
&\sigma_0 e_j \sigma^{-1}_0=(-1)^{\delta_{i,0}}e_j,\ \quad \sigma_0 f_j \sigma^{-1}_0=(-1)^{-\delta_{i,0}}f_j,\ \ \quad \sigma_0 k_j \sigma^{-1}_0=k_j,  \ \  \forall j.
\end{aligned}
\end{eqnarray}

The quantised universal enveloping algebra ${{\rm U}}_q({{\mathfrak g}}, \Pi)$ is a Hopf superalgebra, and the group algebra of $G$ has a canonical Hopf algebra structure. Their smash product inherits a Hopf superalgebra structure.
\begin{proposition}\label{prop:hopf}
The quantum superalgebra ${{\mathfrak U}}_q({{\mathfrak g}},\Pi)$ is a Hopf superalgebra with

\noindent
comultiplication ${\Delta}:{{\mathfrak U}}_q({{\mathfrak g}},\Pi)\rightarrow {{\mathfrak U}}_q({{\mathfrak g}},\Pi)\otimes {{\mathfrak U}}_q({{\mathfrak g}},\Pi)$,
\[
\begin{aligned}
{\Delta}(e_i)=e_i\otimes 1+k_i\otimes e_i, &\quad
{\Delta}(f_i)=f_i\otimes k_i^{-1}+1\otimes f_i,\\
{\Delta}(k_i)=k_i\otimes k_i, &\quad
{\Delta}(\sigma_i)=\sigma_i\otimes\sigma_i;
\end{aligned}
\]
counit ${\varepsilonup}:{{\mathfrak U}}_q({{\mathfrak g}},\Pi)\rightarrow {\mathbb{{C}}}$,\ \
$
{\varepsilonup}(e_i)={\varepsilonup}(f_i)=0, \ \ {\varepsilonup}(k_i^{\pm1})=1, \ \ {\varepsilonup}(\sigma_i)=1;
$
and 
 antipode $S:{{\mathfrak U}}_q({{\mathfrak g}},\Pi)\rightarrow {{\mathfrak U}}_q({{\mathfrak g}},\Pi)$, \ \ 
$
\begin{aligned}
S(e_i)=-k_i^{-1}e_i,\ \ S(f_i)=-f_ik_i, \ \ S(k_i)=k_i^{-1}, \ \ S(\sigma_i)=\sigma_i^{-1}.
\end{aligned}
$
\end{proposition}

\section{Algebraic isomorphisms}\label{sect:quantum}

We prove Theorem \ref{thm:main-quan} in this section.  The proof requires
detailed considerations of the structures of the relevant quantum superalgebras.
It is easy conceptually, but is very lengthy, as each pair $({{\mathfrak g}}, {{\mathfrak g}}')$ involves
numerous cases corresponding to different choices of fundamental systems. 
Here we will present only the main steps of the proof, omitting most of the detailed calculations. 

Let $({{\mathfrak g}}, {{\mathfrak g}}')$ be  a pair of  Lie superalgebras or affine Lie superalgebras
in Theorem \ref{thm:main-quan}. Choose any fundamental system $\Pi$  for ${{\mathfrak g}}$ with $\tau$ being the labelling set for the odd simple roots. By Lemma \ref{lem:connection phi},
$\Pi'=\phi(\Pi)$ is the corresponding  fundamental system of ${{\mathfrak g}}'$ with the labelling set $\tau'$ for the odd simple roots.  
We write $\alpha'_i=\phi(\alpha_i)$ for the simple roots of ${{\mathfrak g}}'$.   Note that $\alpha'_s\in\Pi'$ is isotropic if and only if $\alpha_s\in\Pi$ is.

Let  $\{e_i, f_i, k_i^{\pm 1}, \sigma_i\}$ be the set of generators of
the quantum superalgebra ${{\mathfrak U}}_q({{\mathfrak g}}, \Pi)$, and denote by  $\mathrm{G}$
the group generated by the elements $\sigma_i$. Similarly, we let
$\{e'_i, f'_i, {k'}_i^{\pm 1}, \sigma'_i\}$ be the standard generating set of ${{\mathfrak U}}_{-q}({{\mathfrak g}}', \Pi')$, and denote by $\mathrm{G}'$
the group generated by the elements $\sigma'_i$.

For $1\leq i\leq m+n$, we introduce the following elements
\begin{align*}
&\Phi_i=\prod_{k=i}^{m+n}\sigma_k,
\quad \tilde{\Phi}_i=\prod_{k=0}^{m+n}\sigma_{i+2k},
\ \quad \text{in ${{\mathfrak U}}_q({{\mathfrak g}}, \Pi)$};\\
&\Phi'_i=\prod_{k=i}^{m+n}\sigma'_k,\quad \tilde{\Phi}'_i=\prod_{k=0}^{m+n}\sigma'_{i+2k},
\quad \text{in ${{\mathfrak U}}_{-q}({{\mathfrak g}}', \Pi')$},
\end{align*}
where $\sigma_j\in G$ and $\sigma'_j\in G'$ are both $1$ if $j\geq m+n+1$.  Note that
\[
\Phi_{m+n}=\tilde{\Phi}_{m+n}=\sigma_{m+n}, \quad  \Phi'_{m+n}=\tilde{\Phi}'_{m+n}=\sigma'_{m+n}.
\]
Using \eqref{eq:sigma-act}, we can easily verify the following result by direct computations.

For $i=1,2,\dots,m+n$, we define the following elements
\begin{eqnarray}\label{eq:connect B}
\begin{aligned}
&E_{i}=\Phi_{i+1}e_i,\quad F_{i}=\Phi_if_i,\quad i\notin \tau, \\
&E_i=\tilde{\Phi}_{i+2}e_i,\quad F_i=\tilde{\Phi}_{i}f_i,\quad i\in \tau,\\
&K_i=\sigma_i k_i, \quad \text{which belong to ${{\mathfrak U}}_q({{\mathfrak g}}, \Pi)$; and}  \\
&E'_{i}=\Phi'_{i+1}e'_i,\quad F'_{i}=\Phi'_i f'_i,\quad i\notin \tau', \\
&E'_i=\tilde{\Phi}'_{i+2}e'_i,\quad F'_i=\tilde{\Phi}'_{i}f'_i,\quad i\in \tau',\\
&K'_i=\sigma'_i k'_i,  \quad  \text{which belong to ${{\mathfrak U}}_{-q}({{\mathfrak g}}', \Pi')$},
\end{aligned}
\end{eqnarray}
where $\Phi_{m+n+k}=\tilde{\Phi}_{m+n+k}=1$ and
$\Phi'_{m+n+k}=\tilde{\Phi}'_{m+n+k}=1$
for all $k>0$.

If $({{\mathfrak g}}, {{\mathfrak g}}')$ is a pair of affine Lie superalgebras, we will also  define  elements
\[
E_0, F_0, K_0\in {{\mathfrak U}}_q({{\mathfrak g}}, \Pi), \quad \text{and} \quad E'_0, F'_0, K'_0\in{{\mathfrak U}}_{-q}({{\mathfrak g}}', \Pi'),
\]
the explicit expressions of which depend on the affine Lie superalgebras
and fundamental systems, and will be given in the proof of the following result.

\begin{theorem}\label{thm:iso-main} The associative algebra isomorphism
$ {{\mathfrak U}}_{-q}({{\mathfrak g}}', \Pi')\stackrel{\cong}{\longrightarrow}{{\mathfrak U}}_q({{\mathfrak g}}, \Pi)$ of Theorem \ref{thm:main-quan} is given by
\begin{equation}\label{eq:B-map}
\sigma'_i\mapsto \sigma_i, \quad e'_i \mapsto E_i, \quad f'_i \mapsto F_i, \quad k'_i \mapsto K_i, \quad \forall i,
\end{equation}
with the inverse map
\begin{equation}\label{eq:B-map-inv}
\sigma_i\mapsto \sigma'_i, \quad e_i \mapsto E'_i, \quad f_i \mapsto F'_i, \quad k_i \mapsto K'_i, \quad \forall i.
\end{equation}
\end{theorem}
This is a more explicit version of Theorem \ref{thm:main-quan}.
We prove it  below for each pair $({{\mathfrak g}}, {{\mathfrak g}}')$. 

\subsection{The case of  ${{\rm\mathfrak{osp}}}(2m+1|2n)$ and ${{\rm\mathfrak{osp}}}(2n+1|2m)$ }\label{sect:osp}

Recall from Section \ref{sect:roots} that
the ambient space of the roots of ${{\mathfrak g}}={{\rm\mathfrak{osp}}}(2m+1|2n)$ is ${{\mathcal E}}(m|n)$.
Each admissible ordered basis of it leads to a fundamental system $\Pi=\{\alpha_i\mid 1\le i\le m+n\}$ with the $\alpha_i$ given in
Table \ref{table:classical}.
Now ${{\mathfrak g}}'={{\rm\mathfrak{osp}}}(2n+1|2m)$ with the corresponding fundamental system $\Pi'=\{\alpha'_i=\phi(\alpha_i)\mid i=1, 2, \dots, m+n\}$.
The ambient space of the roots is ${{\mathcal E}}(n|m)$. In the case $\alpha_{m+n}=\delta_{n}\in \Pi$, which is odd, $\alpha'_{m+n}=\varepsilon_n$ is an even simple root in $\Pi'$, and the
Dynkin diagrams of $\Pi$ and $\Pi'$ are the Type (1) diagrams in Table \ref{table:Dynkin diagram-B}. In this case,  $\tau'=\tau\backslash\{m+n\}$.   If $\alpha_{m+n}=\varepsilon_m\in\Pi$,
which is even,  $\alpha'_{m+n}=\delta_m$ is an odd simple root in $\Pi'$, and the
Dynkin diagrams of $\Pi$ and $\Pi'$ are the Type (2) diagrams in Table \ref{table:Dynkin diagram-B}.  In this case, $\tau=\tau'\backslash\{m+n\}$.

\begin{table}[h]
\caption{Dynkin diagram of ${{\rm\mathfrak{osp}}}(2m+1|2n)$ and ${{\rm\mathfrak{osp}}}(2n+1|2m)$}
\label{table:Dynkin diagram-B}
\begin{tabular}{ >{\centering\arraybackslash}m{0.4in} | >{\centering\arraybackslash}m{2.2in} | >{\centering\arraybackslash}m{2.2in}  }
\hline
 Type 
&\vspace{2mm} ${{\mathfrak g}}={{\rm\mathfrak{osp}}}(2m+1|2n)$ \vspace{2mm} & ${{\mathfrak g}}'={{\rm\mathfrak{osp}}}(2n+1|2m)$\\
\hline
(1) &
\begin{picture}(0, 25)(75, 2)
\put(10, 10.5){$\times$}
\put(10,2){\tiny$\alpha_1$}
\put(15, 14){\line(1, 0){20}}
 \put(35, 10.5){$\times$}
\put(35,2){\tiny$\alpha_2$}
\put(42, 14){\line(1, 0){20}}
\put(65,11){$\cdots$}
\put(80, 14){\line(1, 0){20}}
\put(100, 11){$\times$}
\put(90,2){\tiny$\alpha_{m+n-1}$}
\put(106, 15){\line(1, 0){17}}
\put(106, 13){\line(1, 0){17}}
\put(118, 10.5){$>$}
\put(130, 14){\circle*{10}}
\put(123,2){\tiny$\alpha_{m+n}$}
\end{picture}
&
\begin{picture}(0, 25)(75, 2)
\put(10, 10.5){$\times$}
\put(10,2){\tiny$\alpha'_1$}
\put(15, 14){\line(1, 0){20}}
 \put(35, 10.5){$\times$}
\put(35,2){\tiny$\alpha'_2$}
\put(42, 14){\line(1, 0){20}}
\put(65,11){$\cdots$}
\put(80, 14){\line(1, 0){20}}
\put(100, 11){$\times$}
\put(90,2){\tiny$\alpha'_{m+n-1}$}
\put(106, 15){\line(1, 0){17}}
\put(106, 13){\line(1, 0){17}}
\put(118, 10.5){$>$}
\put(130, 14){\circle{10}}
\put(123,2){\tiny$\alpha'_{m+n}$}
\end{picture}\\
\hline
(2)
&\begin{picture}(0, 25)(75, 2)
\put(10, 10.5){$\times$}
\put(10,2){\tiny$\alpha_1$}
\put(15, 14){\line(1, 0){20}}
 \put(35, 10.5){$\times$}
\put(35,2){\tiny$\alpha_2$}
\put(42, 14){\line(1, 0){20}}
\put(65,11){$\cdots$}
\put(80, 14){\line(1, 0){20}}
\put(100, 11){$\times$}
\put(90,2){\tiny$\alpha_{m+n-1}$}
\put(106, 15){\line(1, 0){17}}
\put(106, 13){\line(1, 0){17}}
\put(118, 10.5){$>$}
\put(130, 14){\circle{10}}
\put(123,2){\tiny$\alpha_{m+n}$}
\end{picture}
&
\begin{picture}(0, 25)(75, 2)
\put(10, 10.5){$\times$}
\put(10,2){\tiny$\alpha'_1$}
\put(15, 14){\line(1, 0){20}}
 \put(35, 10.5){$\times$}
\put(35,2){\tiny$\alpha'_2$}
\put(42, 14){\line(1, 0){20}}
\put(65,11){$\cdots$}
\put(80, 14){\line(1, 0){20}}
\put(100, 11){$\times$}
\put(90,2){\tiny$\alpha'_{m+n-1}$}
\put(106, 15){\line(1, 0){17}}
\put(106, 13){\line(1, 0){17}}
\put(118, 10.5){$>$}
\put(130, 14){\circle*{10}}
\put(123,2){\tiny$\alpha'_{m+n}$}
\end{picture}\\
\hline
\end{tabular}
\end{table}

To describe the quantum superalgebra ${{\mathfrak U}}_q({{\rm\mathfrak{osp}}}(2m+1|2n), \Pi)$, we first describe the quantum supergroup ${{\rm{U}_q}}({{\rm\mathfrak{osp}}}(2m+1|2n), \Pi)$ of ${{\rm\mathfrak{osp}}}(2m+1|2n)$
associated to $\Pi$.  For this,
we introduce the following notation:
\[\begin{aligned}
&[k]_z=\frac{z^k-z^{-k}}{z-z^{-1}},\quad \{k\}_z=\frac{z^k-(-z)^{-k}}{z+z^{-1}},\quad \mbox{for}\ \  k\in{{\mathbb N}}, \\
&[0]_z!=\{0\}_z!=1, \quad [N]_z!=\prod_{i=1}^N[i]_z,\quad \{N\}_z!=\prod_{i=1}^N\{i\}_z, \mbox{ for}\ \ 1\le N\in{{\mathbb N}},\\
&\begin{bmatrix} N\\k\end{bmatrix}_z=\frac{[N]_z!}{[N-k]_z![k]_z!},\quad \left\{\begin{matrix} n\\k\end{matrix}\right\}_z=\frac{\{N\}_z!}{\{N-k\}_z! \{k\}_z!},\quad \mbox{for}\ \  k\leq N \in {{\mathbb N}},
\end{aligned}
\]
where $z\in{{\mathbb C}}$ such that the expressions above are defined.

By Definition \ref{defi:quantised}, the generators of  ${{\rm U}}_q({{\rm\mathfrak{osp}}}(2m+1|2n), \Pi)$ are
$e_i, \ f_i,  \ k_i, \ k_i^{-1}$ with $1\le i\le m+n$,
where
$e_s$ and $f_s$ with $s\in\tau$ are odd, and the other generators are all even.
The defining  relations are :
\begin{eqnarray}\label{eq:UqB}
\text{relations (1) in Definition \ref{defi:quantised}};
\end{eqnarray}
three types of Serre relations:

 \noindent if $i\notin \tau$, then for all $j\ne i$,
\begin{eqnarray}
&&\label{eq:UqB-even}
\begin{aligned}
&\sum_{k=0}^{1-a_{i j}}
(-1)^k \begin{bmatrix} 1-a_{i j}\\k\end{bmatrix}_{q_i} e_i^{k} e_j
e_i^{1-a_{i j}-k}=0,\quad 
&\sum_{k=0}^{1-a_{i j}}
(-1)^k \begin{bmatrix} 1-a_{i j}\\k\end{bmatrix}_{q_i} f_i^{k} f_j
f_i^{1-a_{i j}-k}=0;
\end{aligned}
\end{eqnarray}
if $s\in\tau$ and $a_{s s}=2$, then for all $j\ne s$,
\begin{eqnarray}
&&\label{eq:UqB-odd2}
\begin{aligned}
&\sum_{k=0}^{1-a_{s j}}
(-1)^{\frac{1}{2}k(k+1)} \left\{\begin{array}{c}1-a_{s j}\\k\end{array}\right\}_{q_s} e_s^{k} e_j
e_s^{1-a_{s j}-k}=0, \\
&\sum_{k=0}^{1-a_{s j}}
(-1)^{\frac{1}{2}k(k+1)}
\left\{\begin{array}{c} 1-a_{s j}\\k\end{array}\right\}_{q_s} f_s^{k} f_j
f_s^{1-a_{s j}-k}=0;
\end{aligned}
\end{eqnarray}
if $a_{s s}=0$, then
\begin{eqnarray}
&&\label{eq:UqB-odd0}
(e_s)^2=0, \quad (f_s)^2=0;
\end{eqnarray}
and higher order Serre relations if the Dynkin diagram
contains sub-diagrams of the following types:
{\selectfont
\begin{enumerate}
\item
\begin{picture}(75, 15)(0, 7)
\put(10, 7){$\times$}
\put(15, 10){\line(1, 0){20}}
 \put(35, 6){\Large$\otimes$ }
\put(45, 10){\line(1, 0){20}}
\put(62, 7){$\times$}
\put(8, -2){\tiny $s-1$}
\put(39, -2){\tiny $s$}
\put(60, -2){\tiny $s+1$}
\put(70, 7){,}
\end{picture}
where the associated higher order Serre relations are
\begin{eqnarray}\label{eq:UqB-higher order1}
\begin{aligned}
&e_s e_{s-1; s; s+1}+(-1)^{[e_{s-1}]+[e_{s+1}]}  e_{s-1; s; s+1}e_s=0,\\
&f_s f_{s-1; s; s+1} +(-1)^{[f_{s-1}]+[f_{s+1}]} f_{s-1; s; s+1}f_s=0;
\end{aligned}
\end{eqnarray}

\item
\begin{picture}(80, 20)(0, 7)
\put(10, 7){$\times$}
\put(15, 10){\line(1, 0){20}}
\put(35, 6){\Large$\otimes$}
\put(45, 11){\line(1, 0){17}}
\put(45, 9){\line(1, 0){17}}
\put(57, 6.5){$>$}
\put(70,10){\circle{10}}
\put(8, -2){\tiny $s-1$}
\put(39, -2){\tiny $s$}
\put(63, -2){\tiny $s+1$}
\put(76, 7){,}
\end{picture}
where $s=m+n-1$ and $\alpha_{m+n}=\varepsilon_m$, and the associated higher

\medskip
\noindent order Serre relations are
\begin{eqnarray}\label{eq:UqB-higher order2}
\begin{aligned}
&e_s e_{s-1; s; s+1}+(-1)^{[e_{s-1}]}  e_{s-1; s; s+1}e_s=0,\\
&f_s f_{s-1; s; s+1} +(-1)^{[f_{s-1}]}  f_{s-1; s; s+1}f_s=0;
\end{aligned}
\end{eqnarray}

\item
\begin{picture}(80, 20)(0, 7)
\put(10, 6.5){$\times$}
\put(15, 10){\line(1, 0){20}}
\put(35, 6){\Large$\otimes$ }
\put(45, 11){\line(1, 0){17}}
\put(45, 9){\line(1, 0){17}}
\put(57, 6.5){$>$}
\put(70, 10){\circle*{10}}
\put(8, -2){\tiny $s-1$}
\put(39, -2){\tiny $s$}
\put(63, -2){\tiny $s+1$}
\put(76, 7){,}
\end{picture}
where $s=m+n-1$ and $\alpha_{m+n}=\delta_n$, and the associated higher

\medskip
\noindent order Serre relations are
\begin{eqnarray}\label{eq:UqB-higher order3}
\begin{aligned}
&e_s e_{s-1; s; s+1}-(-1)^{[e_{s-1}]}  e_{s-1; s; s+1}e_s=0,\\
&f_s f_{s-1; s; s+1} -(-1)^{[f_{s-1}]}  f_{s-1; s; s+1}f_s=0;
\end{aligned}
\end{eqnarray}
\end{enumerate}}
where
\begin{eqnarray}\label{eq:Uab-higher order ef}
\begin{aligned}
e_{i;s;j}=&e_i(e_se_j-(-1)^{[e_j]}q_j^{a_{js}}e_je_s)\\
		&-(-1)^{[e_i](1+[e_j])}q_i^{a_{is}+a_{ij}}(e_se_j-(-1)^{[e_j]}q_j^{a_{js}}e_je_s)e_i,\\
f_{i;s;j}=&f_i(f_sf_j-(-1)^{[f_j]}q_j^{a_{js}}f_jf_s)\\
	&-(-1)^{[f_i](1+[f_j])}q_i^{a_{is}+a_{ij}}(f_sf_j-(-1)^{ [f_j]}q_j^{a_{js}}f_jf_s)f_i.
\end{aligned}
\end{eqnarray}

The quantum superalgebra ${{\mathfrak U}}_q({{\rm\mathfrak{osp}}}(2m+1|2n), \Pi)$ is generated by
${{\rm{U}_q}}({{\rm\mathfrak{osp}}}(2m+1|2n), \Pi)$, and the elements $\sigma_i$, which generate a group $G={{\mathbb Z}}_2^{\times (m+n)}$.
The commutation relations of the $\sigma_i$ with the generators of ${{\rm{U}_q}}({{\rm\mathfrak{osp}}}(2m+1|2n), \Pi)$ are given by \eqref{eq:sigma-act}.
The quantum superalgebra ${{\mathfrak U}}_{-q}({{\rm\mathfrak{osp}}}(2n|2m+1), \Pi')$ can be described similarly.

Let $t=-q$, and let $t^{1/2}$ be a square root  for $t$. Denote $t_i=t^{(\alpha_i,\alpha_i)/2}$. Note that relations among $E_i, F_i, K_i$ depend on $t^{\pm 1}$, but not  on $t^{\pm 1/2}$.

\begin{proof}[Proof of Theorem \ref{thm:iso-main} for $({{\mathfrak g}}, {{\mathfrak g}}')=({{\rm\mathfrak{osp}}}(2m+1|2n), {{\rm\mathfrak{osp}}}(2n+1|2m))$]
If we can prove that the maps \eqref{eq:B-map} and \eqref{eq:B-map-inv} are algebra homomorphisms,
then we immediately see that they are inverses of each other since $\Phi^2_i=1$ for all $i$. 
It is clear from equation \eqref{eq:sigma-act} that the maps preserve the action of $G$ on ${{\rm{U}_q}}({{\mathfrak g}}, \Pi)$ and the action of $G'$ on ${{\rm U}}_{t}({{\mathfrak g}}', \Pi')$.  Thus what remains to be shown is that
\begin{itemize}
\item $E_i, F_i, K^{\pm 1}_i$ ($1\le i\le m+n$)
satisfy the defining relations of ${{\rm U}}_{t}({{\mathfrak g}}', \Pi')$
obeyed by the standard generators $e'_i, f'_i, {k'}_i^{\pm 1}$ ($1\le i\le m+n$);
and
\item
$E'_i, F'_i, {K'}^{\pm 1}_i$ ($1\le i\le m+n$) satisfy the defining relations of ${{\rm U}}_q({{\mathfrak g}}, \Pi)$ obeyed by the standard generators $e_i, f_i, k_i^{\pm 1}$ ($1\le i\le m+n$).
\end{itemize}
The proof simply boils down to deducing the desired relations satisfied
by $E_i, F_i, K^{\pm 1}_i$ (resp.  $E'_i, F'_i, {K'}^{\pm 1}_i$), where $1\le i\le m+n$,
from the defining relations of  ${{\rm U}}_q({{\mathfrak g}}, \Pi)$ (resp. ${{\rm U}}_t({{\mathfrak g}}', \Pi')$).
The proofs for the two statements are exactly the same, thus
we will only present the details for the first one.

\begin{remark} \label{rem:rationale} The rationale of the proof of
Theorem \ref{thm:iso-main} is the same for all pairs $({{\mathfrak g}}, {{\mathfrak g}}')$.
\end{remark}

We begin by considering the Type (1) Dynkin diagrams in Table \ref{table:Dynkin diagram-B}.
In this case, $\alpha_{m+n}=\delta_{n}\in \Pi$. It is easy to see that
\begin{eqnarray}\label{eq:KK-t}
\begin{aligned}
K_i K_j=K_j K_i, \quad
K_iE_jK_i^{-1}=t^{(\alpha_i,\alpha_j)}E_j, \quad
K_iF_jK_i^{-1}=t^{-(\alpha_i,\alpha_j)}F_j, \quad \forall i, j.
\end{aligned}
\end{eqnarray}

Let $\bar{i}=1$ if $i\in\tau'$ and $0$ otherwise. Then we have 
\begin{equation}\label{eq:EF-t}
E_i F_j - (-1)^{\bar{i}\bar{j}}F_j E_i
    =\delta_{i j} \frac{K_i -K_i^{-1}}{t ^{\theta_i}- t^{-\theta_i}}, \quad 1\le i, j\le m+n.
\end{equation}
Consider, for example, 
\[\begin{aligned}
&E_{m+n}F_{m+n}\!\!-\!\!F_{m+n}E_{m+n}=-\sigma_{m+n}(e_{m+n}f_{m+n}\!+\!f_{m+n}e_{m+n})=\!\!\frac{K_{m+n}\!-\!K_{m+n}^{-1}}{t\!-\!t^{-1}}, \\
&E_kF_{m+n}-F_{m+n}E_k=(-1)^{\delta_{k,m+n-1}}\tilde{\Phi}_{k+2}\tilde{\Phi}_{m+n}(e_kf_{m+n}+f_{m+n}e_k)=0, \ \ k\in\tau'.
\end{aligned}\]
The other relations can be proved in exactly the same way. 

It is clear that \eqref{eq:UqB-odd0} immediately leads to 
\begin{eqnarray} \label{eq:nilpotent-t}
E_i^2=F_i^2=0, \quad \text{if $\alpha_i\in\Pi$ is isotropic}.
\end{eqnarray}
In this case $\alpha_i'\in \Pi'$ is an isotropic odd simple root.

We now consider the Serre relations
\eqref{eq:UqB-even}-\eqref{eq:UqB-odd2}, which lead to 
\begin{eqnarray}\label{eq:serre-t}
\begin{aligned}
&\sum_{k=0}^{1-a_{ij}}(-1)^k{\begin{bmatrix}\begin{smallmatrix} 1-a_{ij}\\k\end{smallmatrix}\end{bmatrix}}_{t_i}E_i^kE_jE_i^{1-a_{ij}-k}=0,\\
&\sum_{k=0}^{1-a_{ij}}(-1)^k{\begin{bmatrix}\begin{smallmatrix} 1-a_{ij}\\k\end{smallmatrix}\end{bmatrix}}_{t_i}F_i^kF_jF_i^{1-a_{ij}-k}=0, \quad  i\not\in\tau', \  i\ne j.
\end{aligned}
\end{eqnarray}
To see this, we consider, for example, the case $i=m+n$ and $j=m+n-1$. 
We have
\[
\begin{aligned}
\sum_{k=0}^3(-1)^k\begin{bmatrix} 3\\k\end{bmatrix}_{t_{m+n}}E_{m+n}^kE_{m+n-1} E_{m+n}^{3-k}
=\Phi_{m+n-1+1} \sum_{k=0}^3(-1)^{\frac{k(k+1)}{2}}\left\{\begin{matrix} 3\\k\end{matrix}\right\}_{q_{m+n}} e_{m+n}^k e_{m+n-1} e_{m+n}^{3-k}=0,
\end{aligned}
\]
where we used 
 $
\begin{bmatrix} 3\\k\end{bmatrix}_{t_{m+n}}=(-1)^{\frac{k(k+1)}{2}}\left\{\begin{matrix}3\\k\end{matrix}\right\}_{q_{m+n}}
$
for $0\leq k\leq 3$. 

We proceed to examine the higher order Serre relations, which can arise from two types of sub-diagrams only in the present case.
\begin{case}
\begin{picture}(75, 15)(0, 7)
\put(10, 7){$\times$}
\put(15, 10){\line(1, 0){20}}
 \put(35, 6){\Large$\otimes$ }
\put(45, 10){\line(1, 0){20}}
\put(62, 7){$\times$}
\put(70, 7){,}
\put(7,0){\tiny $s-1$}
\put(38,-1){\tiny $s$}
\put(59,0){\tiny $s+1$}
\end{picture}
with the associated higher order Serre relations given by \eqref{eq:UqB-higher order1}.
\end{case}
\noindent

Let us first assume that $s-1,s+1\notin\tau$.
In this case, $s\neq m+n-1$, $a_{s-1,s}=a_{s+1,s}=-1$  and $q_{s+1}=q_{s-1}^{-1}$,
with $q_{s-1}=q$ or $q^{-1}$ depending on depending on
the value of  $\theta$ in \eqref{eq:bilinear form}.  Then \eqref{eq:UqB-higher order1} is given by
\begin{eqnarray*}
\begin{aligned}
e_s e_{s-1; s; s+1} +  e_{s-1; s; s+1}e_s=0,\quad
f_s f_{s-1; s; s+1} +  f_{s-1; s; s+1}f_s=0, \quad \text{with}\\
e_{s-1; s; s+1} = e_{s-1}(e_s e_{s+1}-q_{s+1}^{-1} e_{s+1} e_s)-
q_{s-1}^{-1}(e_s e_{s+1}-q_{s+1}^{-1} e_{s+1} e_s) e_{s-1},\\
f_{s-1; s; s+1} = f_{s-1}(f_s f_{s+1}-q_{s+1}^{-1} f_{s+1} f_s)-
q_{s-1}^{-1}(f_s f_{s+1}-q_{s+1}^{-1} f_{s+1} f_s) f_{s-1}.
\end{aligned}
\end{eqnarray*}
Write $t_{s\pm 1}= - q_{s\pm 1}$, and let
\[\begin{aligned}
E_{s-1;s;s+1}&:=E_{s-1}(E_sE_{s+1}-t_{s+1}^{-1}E_{s+1}E_s)-t_{s-1}^{-1}(E_sE_{s+1}- t_{s+1}^{-1} E_{s+1}E_s)E_{s-1}, \\
F_{s-1;s;s+1}&:=F_{s-1}(F_s F_{s+1}-t_{s+1}^{-1}F_{s+1}F_s)-t_{s-1}^{-1}(F_s F_{s+1}- t_{s+1}^{-1} F_{s+1}F_s)F_{s-1}.
\end{aligned}
\]
For any mutually distinct $i,j,k$ not in $\tau$,
\[
\begin{aligned}
E_iE_jE_k=(-1)^{\delta_{i,k+1}+\delta_{j,k+1}
+\delta_{i,j+1}}\Phi_{i+1}\Phi_{j+1}\Phi_{k+1}e_ie_je_k,
\end{aligned}\]
and if $j\in\tau$, the identity holds if we replace $\Phi_{j+1}$ by $\tilde{\Phi}_{j+2}$.
There are also similar relations for $F$'s.  Using these facts, we obtain
\[\begin{aligned}
E_{s-1;s;s+1}&=\Phi_s\tilde{\Phi}_{s+2}\Phi_{s+2}e_{s-1;s;s+1},
\quad
F_{s-1;s;s+1}=\Phi_{s-1}\tilde{\Phi}_{s}\Phi_{s+1}f_{s-1;s;s+1}.
\end{aligned}\]
This immediately leads to
\[\begin{aligned}
&E_sE_{s-1;s;s+1}+E_{s-1;s;s+1}E_s=-\sigma_s\sigma_{s+1}(e_se_{s-1;s;s+1}+e_{s-1;s;s+1}e_s)=0,\\
&F_sF_{s-1;s;s+1}+F_{s-1;s;s+1}F_s=-\sigma_{s-1}\sigma_{s}(f_sf_{s-1;s;s+1}+f_{s-1;s;s+1}f_s)=0.
\end{aligned}\]

Without assuming $s-1,s+1\notin\tau$, we can still show that similar relations hold.

In summary,  for $s$ such that $\alpha_s$ is isotropic, we have
\begin{eqnarray}\label{eq:Hserre1-t}
\begin{aligned}
E_sE_{s-1;s;s+1} - (-1)^{1+\overline{s-1}+\overline{s+1}}E_{s-1;s;s+1}E_s=0, \\
F_sF_{s-1;s;s+1} - (-1)^{1+\overline{s-1}+\overline{s+1}}F_{s-1;s;s+1}F_s=0,
\end{aligned}
\end{eqnarray}
where  $\overline{s-1}$ and $\overline{s+1}$ are as in equation \eqref{eq:EF-t}.

\begin{case}
\begin{picture}(82, 15)(0, 7)
\put(10, 7){$\times$}
\put(15, 10){\line(1, 0){20}}
\put(35, 6){\Large$\otimes$}
\put(45, 11){\line(1, 0){17}}
\put(45, 9){\line(1, 0){17}}
\put(57, 6.5){$>$}
\put(70,10){\circle*{10}}
\put(79, 6){,}
\put(7,0){\tiny $s-1$}
\put(38,-1){\tiny $s$}
\put(62,-1){\tiny $s+1$}
\end{picture}
with the associated higher order Serre relations given by \eqref{eq:UqB-higher order3},
\end{case}
\noindent
where $s+1=m+n$. We have  $a_{s+1,s}=-2$, $a_{s-1,s}=-1$ and  $q_{s-1}=q_{s+1}^{-2}$ with $q_{s-1}=q$ or $q^{-1}$ depending on
the value of  $\theta$ in \eqref{eq:bilinear form}.

First let $s-1\notin\tau$. Then equation \eqref{eq:UqB-higher order3} becomes
\begin{eqnarray*}
\begin{aligned}
&e_s e_{s-1; s; s+1} -  e_{s-1; s; s+1}e_s=0,\quad
f_s f_{s-1; s; s+1} -  f_{s-1; s; s+1}f_s=0; \quad \text{with}\\
&e_{s-1; s; s+1} = e_{s-1}(e_s e_{s+1}+q_{s+1}^{a_{s+1, s}} e_{s+1} e_s)-
q_{s-1}^{a_{s-1, s}}(e_s e_{s+1}+q_{s+1}^{a_{s+1, s}} e_{s+1} e_s) e_{s-1},\\
&f_{s-1; s; s+1} =f_{s-1}(f_s f_{s+1}+q_{s+1}^{a_{s+1, s}} f_{s+1} f_s)-
q_{s-1}^{a_{s-1, s}}(f_s f_{s+1}+q_{s+1}^{a_{s+1,s}} f_{s+1} f_s) f_{s-1}.
\end{aligned}
\end{eqnarray*}

Let $t_{s-1}=-q_{s-1}$ and $t_{s+1}^{a_{s+1, s}}=t_{s+1}^{-2}=t_{s-1}$.
Since $\alpha'_{m+n}=\varepsilon_m$ is an even element in $\Pi'$, we follow  \eqref{eq:Uab-higher order ef} to define
\[\begin{aligned}
E_{s-1;s;s+1}
:=&E_{s-1}(E_{s}E_{s+1}-t_{s+1}^{-2}E_{s}E_{s+1})-t_{s-1}^{-1}(E_{s}E_{s+1}
-t_{s+1}^{-2} E_{s+1}E_{s})E_{s-1}\\
F_{s-1;s;s+1}
:=&F_{s-1}(F_{s}F_{s+1}-t_{s+1}^{-2}F_{s}F_{s+1})-t_{s-1}^{-1}(F_{s}F_{s+1}
-t_{s+1}^{-2} F_{s+1} F_{s}) F_{s-1}.
\end{aligned}
\]
Then some tedious calculations show that
\[\begin{aligned}
E_{s-1;s;s+1}=&\Phi_{s}e_{s-1;s;s+1},\\
F_{s-1;s;s+1}=&\Phi_{s-1}\tilde{\Phi}_{s}\Phi_{s+1}f_{s-1;s;s+1}
=\sigma_{s-1}f_{s-1;s;s+1}.
\end{aligned}\]
Using these we easily obtain
\[
\begin{aligned}
E_{s}E_{s-1;s;s+1}+E_{s-1;s;s+1}E_{s}
=&-\Phi_{s}(e_{s}e_{s-1;s;s+1}-e_{s-1;s;s+1}e_{s})=0,\\
F_{s}F_{s-1;s;s+1}+F_{s-1;s;s+1}F_{s}
=&-\sigma_{s}\sigma_{s-1}(f_{s}f_{s-1;s;s+1}-f_{s-1;s;s+1}f_{s})=0.
\end{aligned}
\]

When $s-1\in\tau$, there exist similar relations.

To summarize, we have
\begin{eqnarray}\label{eq:Hserre2-t}
\begin{aligned}
E_sE_{s-1;s;s+1} - (-1)^{1+\overline{s-1}}E_{s-1;s;s+1}E_s=0, \\
F_sF_{s-1;s;s+1} - (-1)^{1+\overline{s-1}}F_{s-1;s;s+1}F_s=0.
\end{aligned}
\end{eqnarray}

Note that equations \eqref{eq:KK-t}, \eqref{eq:EF-t},  \eqref{eq:nilpotent-t},  \eqref{eq:serre-t}, \eqref{eq:Hserre1-t}  and \eqref{eq:Hserre2-t} are
the same as the defining relations of ${{\rm U}}_{-q}({{\mathfrak g}}', \Pi')$ satisfied by
the generators $e'_i, f'_i, k'_i$. Thus we have shown that the map
${{\mathfrak U}}_{-q}({{\mathfrak g}}', \Pi')\longrightarrow {{\mathfrak U}}_q({{\mathfrak g}}, \Pi)$ given by \eqref{eq:B-map}
is indeed an algebra homomorphism if $\alpha_{m+n}=\delta_n$,
i.e., in the case of the Type (1) diagrams in Table \ref{table:Dynkin diagram-B}.
Similarly we can prove this for
Type (2) diagrams in Table \ref{table:Dynkin diagram-B}, where $\alpha_{m+n}=\varepsilon_m$.
\end{proof}

\subsection{The case of  ${{\rm\mathfrak{sl}}}(2m+1|2n)^{(2)}$ and ${{\rm\mathfrak{osp}}}(2n+1|2m)^{(1)}$ }\label{sect:sl-osp}
\setcounter{case}{0}
Given a fundamental system $\Pi=\{\alpha_i\mid i=0, 1, \dots, m+n\}$ of  ${{\mathfrak g}}={{\rm\mathfrak{sl}}}(2m+1|2n)^{(2)}$, we obtain a corresponding fundamental system $\Pi'=\{\alpha'_i=\phi(\alpha_i)\mid i=0, 1, \dots, m+n\}$ of  ${{\mathfrak g}}'={{\rm\mathfrak{osp}}}(2n+1|2m)^{(1)}$ by Lemma \ref{lem:connection phi}. We draw the Dynkin diagrams for $\Pi$ and $\Pi'$ in a row of Table \ref{table:Dynkin diagram-sl2}, with the diagram for $\Pi$ on the left. The Dynkin diagrams corresponding to different choices of fundamental systems are divided into four types in the table.
\begin{table}[h]
\caption{Dynkin diagrams of ${{\rm\mathfrak{sl}}}(2m+1|2n)^{(2)}$ and ${{\rm\mathfrak{osp}}}(2n+1|2m)^{(1)}$}
\label{table:Dynkin diagram-sl2}
\begin{tabular}{ >{\centering\arraybackslash}m{0.4in} | >{\centering\arraybackslash}m{2.5in}|  >{\centering\arraybackslash}m{2.5in}  }
\hline
Type  &\vspace{3mm} ${{\mathfrak g}}={{\rm\mathfrak{sl}}}(2m+1|2n)^{(2)}$ \vspace{3mm} & ${{\mathfrak g}}'={{\rm\mathfrak{osp}}}(2n+1|2m)^{(1)}$\\
\hline
\multirow{3}{*}{(1)}
&\begin{picture}(180, 28)(-5,-12)
\put(7,0){\circle{10}}
\put(5,-12){\tiny\mbox{$\alpha_0$}}
\put(13,1){\line(1, 0){17}}
\put(13,-1){\line(1, 0){17}}
\put(28,-3){$>$}
\put(40, 0){\circle{10}}
\put(38,-12){\tiny\mbox{$\alpha_1$}}
\put(45, 0){\line(1, 0){20}}
\put(64,-3){$\times$}
\put(70, 0){\line(1, 0){20}}
\put(91, -0.5){\dots}
\put(105, 0){\line(1, 0){20}}
\put(124, -3){$\times$}
\put(130,1){\line(1, 0){17}}
\put(130,-1){\line(1, 0){17}}
\put(143,-3){$>$}
\put(155, 0){\circle*{10}}
\put(150,-12){\mbox{\tiny$\alpha_{m+n}$}}
\end{picture}
& \begin{picture}(180, 28)(-5,-12)
\put(7,0){\circle{10}}
\put(5,-12){\tiny\mbox{$\alpha'_0$}}
\put(13,1){\line(1, 0){17}}
\put(13,-1){\line(1, 0){17}}
\put(28,-3){$>$}
\put(40, 0){\circle{10}}
\put(38,-12){\tiny\mbox{$\alpha'_1$}}
\put(45, 0){\line(1, 0){20}}
\put(64,-3){$\times$}
\put(70, 0){\line(1, 0){20}}
\put(91, -0.5){\dots}
\put(105, 0){\line(1, 0){20}}
\put(124, -3){$\times$}
\put(130,1){\line(1, 0){17}}
\put(130,-1){\line(1, 0){17}}
\put(143,-3){$>$}
\put(155, 0){\circle{10}}
\put(150,-12){\mbox{\tiny$\alpha'_{m+n}$}}
\end{picture}  \\

\cline{2-3}
&\begin{picture}(180, 28)(-5,-12)
\put(7,0){\circle{10}}
\put(5,-12){\tiny$\alpha_0$}
\put(13,1){\line(1, 0){17}}
\put(13,-1){\line(1, 0){17}}
\put(28,-3){$>$}
\put(40, 0){\circle{10}}
\put(38,-12){\tiny$\alpha_1$}
\put(45, 0){\line(1, 0){20}}
\put(64,-3){$\times$}
\put(70, 0){\line(1, 0){20}}
\put(91, -0.5){\dots}
\put(105, 0){\line(1, 0){20}}
\put(124, -3){$\times$}
\put(130,1){\line(1, 0){17}}
\put(130,-1){\line(1, 0){17}}
\put(143,-3){$>$}
\put(155, 0){\circle{10}}
\put(150,-12){\tiny$\alpha_{m+n}$}
\end{picture}
&\begin{picture}(180, 28)(-5,-12)
\put(7,0){\circle{10}}
\put(5,-12){\tiny\mbox{$\alpha'_0$}}
\put(13,1){\line(1, 0){17}}
\put(13,-1){\line(1, 0){17}}
\put(28,-3){$>$}
\put(40, 0){\circle{10}}
\put(38,-12){\tiny\mbox{$\alpha'_1$}}
\put(45, 0){\line(1, 0){20}}
\put(64,-3){$\times$}
\put(70, 0){\line(1, 0){20}}
\put(91, -0.5){\dots}
\put(105, 0){\line(1, 0){20}}
\put(124, -3){$\times$}
\put(130,1){\line(1, 0){17}}
\put(130,-1){\line(1, 0){17}}
\put(143,-3){$>$}
\put(155, 0){\circle*{10}}
\put(150,-12){\mbox{\tiny$\alpha'_{m+n}$}}
\end{picture}  \\

\hline

\multirow{3}{*}{(2)}
& \begin{picture}(180, 28)(-10,-12)
\put(7,0){\circle{10}}
\put(5,-12){\tiny$\alpha_0$}
\put(11,-3){$<$}
\put (24,4){\tiny $2$}
\put(14,0){\line(1, 0){22}}
\put(37,-5){\Large$\otimes$}
\put(38,-12){\tiny$\alpha_1$}
\put(47,0){\line(1, 0){22}}
\put(65,-3){$>$}
\put(75, -0.5){\dots}
\put(91,0){\line(1, 0){20}}
\put(110, -3){$\times$}
\put(116,1){\line(1, 0){17}}
\put(116,-1){\line(1, 0){17}}
\put(130,-3){$>$}
\put(142, 0){\circle*{10}}
\put(136,-12){\mbox{\tiny$\alpha_{m+n}$}}
\end{picture}
& \begin{picture}(180, 28)(-10,-12)
\put(7,0){\circle{10}}
\put(5,-12){\tiny$\alpha'_0$}
\put(11,-3){$<$}
\put (24,4){\tiny $2$}
\put(14,0){\line(1, 0){22}}
\put(37,-5){\Large$\otimes$}
\put(38,-12){\tiny$\alpha'_1$}
\put(47,0){\line(1, 0){22}}
\put(65,-3){$>$}
\put(75, -0.5){\dots}
\put(91,0){\line(1, 0){20}}
\put(110, -3){$\times$}
\put(116,1){\line(1, 0){17}}
\put(116,-1){\line(1, 0){17}}
\put(130,-3){$>$}
\put(142, 0){\circle{10}}
\put(136,-12){\mbox{\tiny$\alpha'_{m+n}$}}
\end{picture}  \\

\cline{2-3}
& \begin{picture}(180, 28)(-10,-12)
\put(7,0){\circle{10}}
\put(5,-12){\tiny$\alpha_0$}
\put(11,-3){$<$}
\put (24,4){\tiny $2$}
\put(14,0){\line(1, 0){22}}
\put(37,-5){\Large$\otimes$}
\put(38,-12){\tiny$\alpha_1$}
\put(47,0){\line(1, 0){22}}
\put(65,-3){$>$}
\put(75, -0.5){\dots}
\put(91,0){\line(1, 0){20}}
\put(110, -3){$\times$}
\put(116,1){\line(1, 0){17}}
\put(116,-1){\line(1, 0){17}}
\put(130,-3){$>$}
\put(142, 0){\circle{10}}
\put(136,-12){\mbox{\tiny$\alpha_{m+n}$}}
\end{picture}
& \begin{picture}(180, 28)(-10,-12)
\put(7,0){\circle{10}}
\put(5,-12){\tiny$\alpha'_0$}
\put(11,-3){$<$}
\put (24,4){\tiny $2$}
\put(14,0){\line(1, 0){22}}
\put(37,-5){\Large$\otimes$}
\put(38,-12){\tiny$\alpha'_1$}
\put(47,0){\line(1, 0){22}}
\put(65,-3){$>$}
\put(75, -0.5){\dots}
\put(91,0){\line(1, 0){20}}
\put(110, -3){$\times$}
\put(116,1){\line(1, 0){17}}
\put(116,-1){\line(1, 0){17}}
\put(130,-3){$>$}
\put(142, 0){\circle*{10}}
\put(136,-12){\mbox{\tiny$\alpha'_{m+n}$}}
\end{picture}\\
\hline
\multirow{6}{*}{(3)}
&\begin{picture}(180, 60)(-26,-28)
\put(0, 15){\circle{10}}
\put(-5,23){\tiny$\alpha_0$}
\put(0, -16){\circle{10}}
\put(-4,-28){\tiny$\alpha_1$}
\put(15, -3){\line(-1, -1){10}}
\put(15, 3){\line(-1, 1){10}}
\put(16, -3){$\times$}
\put(24, 0){\line(1, 0){20}}
\put(46, -0.5){\dots}
\put(60,0){\line(1, 0){20}}
\put(79, -3){$\times$}
\put(86,1){\line(1, 0){17}}
\put(86,-1){\line(1, 0){17}}
\put(100,-3){$>$}
\put(112, 0){\circle*{10}}
\put(106,-15){\tiny$\alpha_{m+n}$}
\end{picture}
& \begin{picture}(180, 60)(-26,-28)
\put(0, 15){\circle{10}}
\put(-4,24){\tiny$\alpha'_0$}
\put(0, -16){\circle{10}}
\put(-4,-28){\tiny$\alpha'_1$}
\put(15, -3){\line(-1, -1){10}}
\put(15, 3){\line(-1, 1){10}}
\put(16, -3){$\times$}
\put(24, 0){\line(1, 0){20}}
\put(46, -0.5){\dots}
\put(60,0){\line(1, 0){20}}
\put(79, -3){$\times$}
\put(86,1){\line(1, 0){17}}
\put(86,-1){\line(1, 0){17}}
\put(100,-3){$>$}
\put(112, 0){\circle{10}}
\put(106,-15){\tiny$\alpha'_{m+n}$}
\end{picture} \\
\cline{2-3}
&\begin{picture}(180, 60)(-26,-28)
\put(0, 15){\circle{10}}
\put(-5,23){\tiny$\alpha_0$}
\put(0, -16){\circle{10}}
\put(-4,-28){\tiny$\alpha_1$}
\put(15, -3){\line(-1, -1){10}}
\put(15, 3){\line(-1, 1){10}}
\put(16, -3){$\times$}
\put(24, 0){\line(1, 0){20}}
\put(46, -0.5){\dots}
\put(60,0){\line(1, 0){20}}
\put(79, -3){$\times$}
\put(86,1){\line(1, 0){17}}
\put(86,-1){\line(1, 0){17}}
\put(100,-3){$>$}
\put(112, 0){\circle{10}}
\put(106,-15){\tiny$\alpha_{m+n}$}
\end{picture}
&\begin{picture}(180, 60)(-26,-28)
\put(0, 15){\circle{10}}
\put(-4,24){\tiny$\alpha'_0$}
\put(0, -16){\circle{10}}
\put(-4,-28){\tiny$\alpha'_1$}
\put(15, -3){\line(-1, -1){10}}
\put(15, 3){\line(-1, 1){10}}
\put(16, -3){$\times$}
\put(24, 0){\line(1, 0){20}}
\put(46, -0.5){\dots}
\put(60,0){\line(1, 0){20}}
\put(79, -3){$\times$}
\put(86,1){\line(1, 0){17}}
\put(86,-1){\line(1, 0){17}}
\put(100,-3){$>$}
\put(112, 0){\circle*{10}}
\put(106,-15){\tiny$\alpha'_{m+n}$}
\end{picture} \\
\hline
\multirow{6}{*}{(4)}
&\begin{picture}(180, 63)(-26,-26)
\put(0, 13){\Large$\otimes$}
\put(2,26){\tiny$\alpha_0$}
\put(0, -17){\Large$\otimes$}
\put(2,-26){\tiny$\alpha_1$}
\put(4,-7){\line(0, 1){20}}
\put(6,-7){\line(0, 1){20}}
\put(20, -2){\line(-1, -1){10}}
\put(20, 3){\line(-1, 1){10}}
\put(20, -3){$\times$}
\put(26, 0){\line(1, 0){20}}
\put(50, -0.5){\dots}
\put(65,0){\line(1, 0){20}}
\put(83, -3){$\times$}
\put(90,1){\line(1, 0){17}}
\put(90,-1){\line(1, 0){17}}
\put(105,-3){$>$}
\put(117, 0){\circle*{10}}
\put(110,-15){\tiny$\alpha_{m+n}$}
\end{picture}
&\begin{picture}(180, 63)(-26,-26)
\put(0, 13){\Large$\otimes$}
\put(2,26){\tiny$\alpha'_0$}
\put(0, -17){\Large$\otimes$}
\put(2,-26){\tiny$\alpha'_1$}
\put(4,-7){\line(0, 1){20}}
\put(6,-7){\line(0, 1){20}}
\put(20, -2){\line(-1, -1){10}}
\put(20, 3){\line(-1, 1){10}}
\put(20, -3){$\times$}
\put(26, 0){\line(1, 0){20}}
\put(50, -0.5){\dots}
\put(65,0){\line(1, 0){20}}
\put(83, -3){$\times$}
\put(90,1){\line(1, 0){17}}
\put(90,-1){\line(1, 0){17}}
\put(105,-3){$>$}
\put(117, 0){\circle{10}}
\put(110,-15){\tiny$\alpha'_{m+n}$}
\end{picture} \\
\cline{2-3}
&\begin{picture}(180, 63)(-26,-26)
\put(0, 13){\Large$\otimes$}
\put(2,26){\tiny$\alpha_0$}
\put(0, -17){\Large$\otimes$}
\put(2,-26){\tiny$\alpha_1$}
\put(4,-7){\line(0, 1){20}}
\put(6,-7){\line(0, 1){20}}
\put(20, -2){\line(-1, -1){10}}
\put(20, 3){\line(-1, 1){10}}
\put(20, -3){$\times$}
\put(26, 0){\line(1, 0){20}}
\put(50, -0.5){\dots}
\put(65,0){\line(1, 0){20}}
\put(83, -3){$\times$}
\put(90,1){\line(1, 0){17}}
\put(90,-1){\line(1, 0){17}}
\put(105,-3){$>$}
\put(117, 0){\circle{10}}
\put(110,-15){\tiny$\alpha_{m+n}$}
\end{picture}
&\begin{picture}(180, 63)(-26,-26)
\put(0, 13){\Large$\otimes$}
\put(2,26){\tiny$\alpha'_0$}
\put(0, -17){\Large$\otimes$}
\put(2,-26){\tiny$\alpha'_1$}
\put(4,-7){\line(0, 1){20}}
\put(6,-7){\line(0, 1){20}}
\put(20, -2){\line(-1, -1){10}}
\put(20, 3){\line(-1, 1){10}}
\put(20, -3){$\times$}
\put(26, 0){\line(1, 0){20}}
\put(50, -0.5){\dots}
\put(65,0){\line(1, 0){20}}
\put(83, -3){$\times$}
\put(90,1){\line(1, 0){17}}
\put(90,-1){\line(1, 0){17}}
\put(105,-3){$>$}
\put(117, 0){\circle*{10}}
\put(110,-15){\tiny$\alpha'_{m+n}$}
\end{picture} \\
\hline
\end{tabular}
\end{table}

\begin{proof}[Proof of Theorem \ref{thm:iso-main} for $({{\mathfrak g}}, {{\mathfrak g}}')=({{\rm\mathfrak{sl}}}(2m+1|2n)^{(2)}, {{\rm\mathfrak{osp}}}(2n+1|2m)^{(1)})$] For $1\le i\le m+n$, we define $E_i, F_i, K_i$, $E'_i, F'_i, K'_i$   by \eqref{eq:connect B}.
Note that when the nodes of $\alpha_0$ and $\alpha'_0$ are removed,
the Dynkin diagrams in Table \ref{table:Dynkin diagram-sl2}  reduce to the Dynkin diagrams for the finite dimensional Lie superalgebras ${{\rm\mathfrak{osp}}}(2m+1|2n)$ and ${{\rm\mathfrak{osp}}}(2n+1|2m)$.
Thus by \eqref{eq:connect B}, the same reasoning in Section \ref{sect:osp} can show
that the elements
$E_i, F_i, K_i$ (resp. $E'_i, F'_i, K'_i$)  for $1\le i\le m+n$ have
the desired properties.

What remains to be shown to complete the proof of
Theorem \ref{thm:iso-main}  is to construct elements
$E_0, F_0, K_0\in {{\mathfrak U}}_q({{\mathfrak g}}, \Pi)$ (resp. $E'_0, F'_0, K'_0\in {{\mathfrak U}}_t({{\mathfrak g}}', \Pi')$),
which satisfy the commutation relations obeyed by $e'_0, f'_0, k'_0$ (resp $e_0, f_0, k_0$).  Now we do this for each of the four types of diagrams in
Table \ref{table:Dynkin diagram-sl2}.
The proofs for $E_0, F_0, K_0$  and  for $E'_0, F'_0, K'_0$ are identical, thus we will give the details for the former only.

\begin{case}
{\em Type (1) Dynkin diagrams  in Table \ref{table:Dynkin diagram-sl2}}.
Define 
\begin{equation}\label{eq:connect AB-case1}
\begin{aligned}
&\quad E_0=e_0,\quad F_0=f_0, \quad K_0=k_0, \text{ and,}\ \ E'_0=e'_0,\quad F'_0=f'_0, \quad K'_0=k'_0.
\end{aligned}
\end{equation}
Calculations similar to those in Section \ref{sect:osp} show that 
\begin{equation*}
E_i F_j - (-1)^{\bar{i}\bar{j}}F_j E_i
    =\delta_{i j} \frac{K_i -K_i^{-1}}{t ^{\theta_i}- t^{-\theta_i}}, \quad 0\le i, j\le m+n,
\end{equation*}
\[
\begin{aligned}
&\sum_{k=0}^2(-1)^k\begin{bmatrix} 2\\ k \end{bmatrix}_{t^2}E_0^kE_1E_0^{2-k}
=\Phi_2\sum_{k=0}^2(-1)^k\begin{bmatrix}2\\k\end{bmatrix}_{q^2}e_0^ke_1e_0^{2-k}=0,
\end{aligned}\]
\[\begin{aligned}
&\sum_{k=0}^3(-1)^k\begin{bmatrix} 3\\ k \end{bmatrix}_{t}E_1^kE_0E_1^{3-k}=\Phi_2 \sum_{k=0}^3(-1)^k\begin{bmatrix}3\\k\end{bmatrix}_{q}e_1^ke_0e_1^{3-k}=0,
\end{aligned}
\]
and similar relations among $F_0$ and $F_1$. 
Since there is no higher order Serre relation involving $e'_0$ or $f'_0$ because
 $\alpha_1$ is not an isotropic root,
this completes the proof that the elements
$E_0, F_0$ satisfy the defining relations of ${{\mathfrak U}}_t({{\mathfrak g}}', \Pi')$.
\end{case}

\begin{case}
{\em Type (2) Dynkin diagrams  in Table \ref{table:Dynkin diagram-sl2}}.
The elements $E_0, F_0, K_0, E'_0, F'_0, K'_0$ are still defined by \eqref{eq:connect AB-case1}.

The same proof as that in the last case shows that $E_0$ and $F_0$  satisfy the ordinary Serre relations. The new feature here is that higher order Serre relations involving $E_0$ or $F_0$ arise when $m+n>2$, which we now consider.
The higher order Serre relations are either of type (D) (for $2\in\tau$) or (E) (for $2\notin\tau$)  in Definition \ref{defi:quantised}.

If $\alpha_2$ is an odd isotropic root, then
\[
\begin{aligned}
\left[{{\mbox{Ad}}}_{e_{2}}(e_{1}), \left[{{\mbox{Ad}}}_{e_{2}}(e_1) , {{\mbox{Ad}}}_{e_1}(e_{0})\right]_{q^{-1}}\right]_{q}=0.
\end{aligned}\]
Denote ${{\mbox{Ad}}}_{E_i}(E_j):=E_iE_j-(-1)^{\bar{i}\bar{j}}K_iE_jK_i^{-1}E_i$, where $\bar{i}=1$ if $i\in\tau'$ and $0$ otherwise. By \eqref{eq:connect AB-case1}, we have
\[\begin{aligned}
&{{\mbox{Ad}}}_{E_2}(E_1)=E_2E_1+tE_2E_1=-\Phi_3{{\mbox{Ad}}}_{e_2}(e_1),\\
&{{\mbox{Ad}}}_{E_1}(E_0)=E_1E_0-t^{-2}E_0E_1=\tilde{\Phi}_3{{\mbox{Ad}}}_{e_1}(e_0).
\end{aligned}\]
Hence $\left[{{\mbox{Ad}}}_{E_{2}}(E_1), {{\mbox{Ad}}}_{E_1}(E_{0})\right]_{t^{-1}}=\tilde{\Phi}_4\left[{{\mbox{Ad}}}_{e_{2}}(e_1), {{\mbox{Ad}}}_{e_1}(e_{0})\right]_{q^{-1}}$, and it follows that
\[\begin{aligned}
&\left[{{\mbox{Ad}}}_{E_{2}}(E_{1}), \left[{{\mbox{Ad}}}_{E_{2}}(E_1) , {{\mbox{Ad}}}_{E_1}(E_{0})\right]_{t^{-1}}\right]_t=-\tilde{\Phi}_3\left[{{\mbox{Ad}}}_{e_{2}}e_{1}, \left[{{\mbox{Ad}}}_{e_{2}}e_1 , {{\mbox{Ad}}}_{e_1}e_{0}\right]_{q^{-1}}\right]_{q}=0.
\end{aligned}\]
We can prove the higher order Serre relation (D) for $F_0,F_1,F_2$
in the same way.

If $\alpha_2$ is not an odd isotropic root, the higher order Serre relation (E) in Definition \ref{defi:quantised} is
\[\begin{aligned}
\left[{{\mbox{Ad}}}_{e_3}{{\mbox{Ad}}}_{e_2}(e_1),\left[{{\mbox{Ad}}}_{e_{2}}(e_1), {{\mbox{Ad}}}_{e_1}(e_{0})
\right]_{q^{-1}}\right]=0.
\end{aligned}\]
Note that
$
\left[{{\mbox{Ad}}}_{E_{2}}(E_1), {{\mbox{Ad}}}_{E_1}(E_{0})
\right]_{t^{-1}}=-\Phi_3\left[{{\mbox{Ad}}}_{e_{2}}(e_1), {{\mbox{Ad}}}_{e_1}(e_{0})
\right]_{q^{-1}}.
$

Denote ${{\mbox{Ad}}}_{E_3}{{\mbox{Ad}}}_{E_2}(E_1):=E_3({{\mbox{Ad}}}_{E_2}(E_1))-t({{\mbox{Ad}}}_{E_2}(E_1))E_3$. Then straightforward computations show that
$
{{\mbox{Ad}}}_{E_3}{{\mbox{Ad}}}_{E_2}(E_1)=\Phi\tilde{\Phi}_4{{\mbox{Ad}}}_{e_3}{{\mbox{Ad}}}_{e_2}(e_1),
$
where $\Phi=\Phi_4$ if $3\notin\tau$ or $\Phi=\tilde{\Phi}_5$ if $3\in\tau$. This gives
\[
\begin{aligned}
\left[{{\mbox{Ad}}}_{E_3}{{\mbox{Ad}}}_{E_2}(E_1),\left[{{\mbox{Ad}}}_{E_{2}}(E_1), {{\mbox{Ad}}}_{E_1}(E_{0})
\right]_{t^{-1}}\right]
=-\Phi\tilde{\Phi}_3\left[{{\mbox{Ad}}}_{e_3}{{\mbox{Ad}}}_{e_2}(e_1),\left[{{\mbox{Ad}}}_{e_{2}}e_1, {{\mbox{Ad}}}_{e_1}e_{0}
\right]_{q^{-1}}\right]=0.
\end{aligned}
\]
We can similarly prove the higher order Serre relation (E) for $F_0,F_1,F_2,F_3$.
\end{case}

\begin{case}
{\em Type (3) Dynkin diagrams in Table \ref{table:Dynkin diagram-sl2}}.
In this case, 
\begin{eqnarray}\label{eq:connect AB-case3}
\begin{aligned}
&E_0=\Phi_{2}e_0,\quad F_0=\Phi_1 f_0, \quad K_0=\sigma_1k_0,\\
&E'_0=\Phi'_{2}e'_0,\quad F'_0=\Phi'_1 f'_0, \quad K'_0=\sigma'_1k'_0.
\end{aligned}
\end{eqnarray}
The proof is analogous to that in Case 1. 
\end{case}

\begin{case}
{\em Type (4) Dynkin diagrams in Table \ref{table:Dynkin diagram-sl2}}.
This time $\alpha_0$ is an odd simple root. Define
\begin{eqnarray}\label{eq:connect AB-case4}
\begin{aligned}
&E_0=\tilde{\Phi}_{3}e_0,\quad \ \ F_0= \tilde{\Phi}_1 f_0,\quad K_0=\sigma_1k_0,\\
&E'_0=\tilde{\Phi}'_{3}e'_0,\quad \ \ F'_0= \tilde{\Phi}'_1 f'_0,\quad K'_0=\sigma'_1k'_0.
\end{aligned}
\end{eqnarray}
In this case, $E_0^2=F_0^2=0$.
The proof of the relations is again similar to Case 1, except for the higher order Serre relations (F) for $E_0$ and $F_0$ (see Definition \ref{defi:quantised}). We only give the proof for $E_0$; the same proof works for $F_0$.
Assume $2\notin\tau$. The higher order Serre relation among $e_0,e_1,e_2$  is
\[{{\mbox{Ad}}}_{e_0}{{\mbox{Ad}}}_{e_1}(e_2)-{{\mbox{Ad}}}_{e_1}{{\mbox{Ad}}}_{e_0}(e_2)=0.\]
 We want to deduce from it the following relation
\[
E_{0;1;2}:=E_0{{\mbox{Ad}}}_{E_1}(E_2)+t^{-1}({{\mbox{Ad}}}_{E_1}(E_2))E_0-E_1{{\mbox{Ad}}}_{E_0}(E_2)-t^{-1}({{\mbox{Ad}}}_{E_0}(E_2))E_1=0.
\]
This follows from 
$
E_{0;1;2}=\Phi_3({{\mbox{Ad}}}_{e_0}{{\mbox{Ad}}}_{e_1}(e_2)-{{\mbox{Ad}}}_{e_1}{{\mbox{Ad}}}_{e_0}(e_2)) =0.
$
The case with $2\in\tau$ is similar.
\end{case}
\end{proof}

\subsection{The case of  ${{\rm\mathfrak{osp}}}(2m+2|2n)^{(2)}$ and ${{\rm\mathfrak{osp}}}(2n+2|2m)^{(2)}$}\label{sect:osp2}
\setcounter{case}{0}
The Dynkin diagrams of ${{\mathfrak g}}={{\rm\mathfrak{osp}}}(2m+2|2n)^{(2)}$ and ${{\mathfrak g}}'={{\rm\mathfrak{osp}}}(2n+2|2m)^{(2)}$ are  given in Table \ref{table:Dynkin diagram-osp2}, 
where the diagrams for a fundamental system $\Pi$ of ${{\mathfrak g}}$ and the corresponding fundamental system  $\Pi'$ of ${{\mathfrak g}}'$ are shown in the same row.

\begin{table}[!htbp]
\caption{Dynkin diagrams of ${{\rm\mathfrak{osp}}}(2m+2|2n)^{(2)}$ and ${{\rm\mathfrak{osp}}}(2n+2|2m)^{(2)}$}
\label{table:Dynkin diagram-osp2}
\begin{tabular}{  >{\centering\arraybackslash}m{0.4in} |>{\centering\arraybackslash}m{2.2in}|   >{\centering\arraybackslash}m{2.2in}  }
\hline
Type 
&\vspace{3mm} ${{\mathfrak g}}={{\rm\mathfrak{osp}}}(2m+2|2n)^{(2)}$ \vspace{2mm} & ${{\mathfrak g}}'={{\rm\mathfrak{osp}}}(2n+2|2m)^{(2)}$\\
\hline
\multirow{3}{*}{(1)}
&\begin{picture}(150, 30)(-6,-14)
\put(7,0){\circle{10}}
\put(3,-12){\tiny $\alpha_0$}
\put(16,1){\line(1, 0){18}}
\put(16,-1){\line(1, 0){18}}
\put(12,-3){$<$}
\put(34, -3){$\times$}
\put(40, 0){\line(1, 0){20}}
\put(61, -0.5){\dots}
\put(75, 0){\line(1, 0){20}}
\put(94, -3){$\times$}
\put(100,1){\line(1, 0){17}}
\put(100,-1){\line(1, 0){17}}
\put(113,-3){$>$}
\put(125, 0){\circle*{10}}
\put(115,-12){ \tiny$\alpha_{m+n}$}
\end{picture}
&\begin{picture}(150, 30)(-10,-14)
\put(7,0){\circle*{10}}
\put(3,-12){\tiny $\alpha'_0$}
\put(16,1){\line(1, 0){18}}
\put(16,-1){\line(1, 0){18}}
\put(12,-3){$<$}
\put(34, -3){$\times$}
\put(40, 0){\line(1, 0){20}}
\put(61, -0.5){\dots}
\put(75, 0){\line(1, 0){20}}
\put(94, -3){$\times$}
\put(100,1){\line(1, 0){17}}
\put(100,-1){\line(1, 0){17}}
\put(113,-3){$>$}
\put(125, 0){\circle{10}}
\put(115,-12){ \tiny$\alpha'_{m+n}$}
\end{picture} \\
\cline{2-3}
&\begin{picture}(150, 30)(-6,-14)
\put(7,0){\circle{10}}
\put(3,-12){\tiny $\alpha_0$}
\put(16,1){\line(1, 0){18}}
\put(16,-1){\line(1, 0){18}}
\put(12,-3){$<$}
\put(34, -3){$\times$}
\put(40, 0){\line(1, 0){20}}
\put(61, -0.5){\dots}
\put(75, 0){\line(1, 0){20}}
\put(94, -3){$\times$}
\put(100,1){\line(1, 0){17}}
\put(100,-1){\line(1, 0){17}}
\put(113,-3){$>$}
\put(125, 0){\circle{10}}
\put(115,-12){ \tiny$\alpha_{m+n}$}
\end{picture}
&\begin{picture}(150, 30)(-10,-14)
\put(7,0){\circle*{10}}
\put(3,-12){\tiny $\alpha'_0$}
\put(16,1){\line(1, 0){18}}
\put(16,-1){\line(1, 0){18}}
\put(12,-3){$<$}
\put(34, -3){$\times$}
\put(40, 0){\line(1, 0){20}}
\put(61, -0.5){\dots}
\put(75, 0){\line(1, 0){20}}
\put(94, -3){$\times$}
\put(100,1){\line(1, 0){17}}
\put(100,-1){\line(1, 0){17}}
\put(113,-3){$>$}
\put(125, 0){\circle*{10}}
\put(115,-12){ \tiny$\alpha'_{m+n}$}
\end{picture} \\
\hline
\multirow{3}{*}{(2)}
&\begin{picture}(150, 30)(-6,-14)
\put(7,0){\circle*{10}}
\put(3,-12){\tiny $\alpha_0$}
\put(16,1){\line(1, 0){18}}
\put(16,-1){\line(1, 0){18}}
\put(12,-3){$<$}
\put(34, -3){$\times$}
\put(40, 0){\line(1, 0){20}}
\put(61, -0.5){\dots}
\put(75, 0){\line(1, 0){20}}
\put(94, -3){$\times$}
\put(100,1){\line(1, 0){17}}
\put(100,-1){\line(1, 0){17}}
\put(113,-3){$>$}
\put(125, 0){\circle*{10}}
\put(115,-12){ \tiny$\alpha_{m+n}$}
\end{picture}
&\begin{picture}(150, 30)(-10,-14)
\put(7,0){\circle{10}}
\put(3,-12){\tiny $\alpha'_0$}
\put(16,1){\line(1, 0){18}}
\put(16,-1){\line(1, 0){18}}
\put(12,-3){$<$}
\put(34, -3){$\times$}
\put(40, 0){\line(1, 0){20}}
\put(61, -0.5){\dots}
\put(75, 0){\line(1, 0){20}}
\put(94, -3){$\times$}
\put(100,1){\line(1, 0){17}}
\put(100,-1){\line(1, 0){17}}
\put(113,-3){$>$}
\put(125, 0){\circle{10}}
\put(115,-12){ \tiny$\alpha'_{m+n}$}
\end{picture} \\
\cline{2-3}
&\begin{picture}(150, 30)(-6,-14)
\put(7,0){\circle*{10}}
\put(3,-12){\tiny $\alpha_0$}
\put(16,1){\line(1, 0){18}}
\put(16,-1){\line(1, 0){18}}
\put(12,-3){$<$}
\put(34, -3){$\times$}
\put(40, 0){\line(1, 0){20}}
\put(61, -0.5){\dots}
\put(75, 0){\line(1, 0){20}}
\put(94, -3){$\times$}
\put(100,1){\line(1, 0){17}}
\put(100,-1){\line(1, 0){17}}
\put(113,-3){$>$}
\put(125, 0){\circle{10}}
\put(115,-12){ \tiny$\alpha_{m+n}$}
\end{picture}
&\begin{picture}(150, 30)(-10,-14)
\put(7,0){\circle{10}}
\put(3,-12){\tiny $\alpha'_0$}
\put(16,1){\line(1, 0){18}}
\put(16,-1){\line(1, 0){18}}
\put(12,-3){$<$}
\put(34, -3){$\times$}
\put(40, 0){\line(1, 0){20}}
\put(61, -0.5){\dots}
\put(75, 0){\line(1, 0){20}}
\put(94, -3){$\times$}
\put(100,1){\line(1, 0){17}}
\put(100,-1){\line(1, 0){17}}
\put(113,-3){$>$}
\put(125, 0){\circle*{10}}
\put(115,-12){ \tiny$\alpha'_{m+n}$}
\end{picture} \\
\hline
\end{tabular}
\end{table}

\begin{proof}[Proof of Theorem \ref{thm:iso-main}  for $({{\mathfrak g}}, {{\mathfrak g}}')=
 ({{\rm\mathfrak{osp}}}(2m+2|2n)^{(2)}, {{\rm\mathfrak{osp}}}(2n+2|2m)^{(2)})$]
We will merely construct the elements $E_i, F_i, K_i, E'_i, F'_i, K'_i$ here, as
the proof of Theorem \ref{thm:iso-main} is much the same as in the previous  cases.
For $1\leq i\leq m+n$, the elements $E_i, F_i, K_i, E'_i, F'_i, K'_i$ are given by
\eqref{eq:connect B}; and for $i=0$, they are defined as follows.
\begin{case}
{\em Type (1) Dynkin diagrams in Table \ref{table:Dynkin diagram-osp2}}.
\begin{eqnarray}\label{eq:connect D-case1}
\begin{aligned}
&E_0=\tilde{\Phi}_2\prod_{j\in\tau}(\tilde{\Phi}_1\tilde{\Phi}_{j+1})\cdot e_0,\quad F_0=\tilde{\Phi}_1\prod_{j\in\tau}(\tilde{\Phi}_1\tilde{\Phi}_{j+1})\cdot f_0,\quad K_0=\Phi_1\cdot k_0;\\
&E'_0=\tilde{\Phi}'_2\prod_{j\in\tau'}(\tilde{\Phi}'_1\tilde{\Phi}'_{j+1})\cdot e'_0,\quad F'_0= \tilde{\Phi}'_1\prod_{j\in\tau'}(\tilde{\Phi}'_1\tilde{\Phi}'_{j+1})\cdot f'_0, \quad K'_0=\Phi'_1\cdot k'_0.
\end{aligned}
\end{eqnarray}
In this case, $0\notin\tau, m+n\in\tau$ while $0\in\tau', m+n\notin\tau'$.
\end{case}

\begin{case}
{\em Type (2) Dynkin diagrams in Table \ref{table:Dynkin diagram-osp2}}.
\begin{equation}\label{eq:connect D-case2}
\begin{aligned}
&E_0=\Phi_1\prod_{j\in\tau}(\tilde{\Phi}_1\tilde{\Phi}_{j+1})\cdot e_0,\quad F_0=\prod_{j\in\tau}(\tilde{\Phi}_1\tilde{\Phi}_{j+1})\cdot f_0,\quad K_0=\Phi_1\cdot k_0;\\
&E'_0=\tilde{\Phi}'_1\prod_{j\in\tau'}(\tilde{\Phi}'_1\tilde{\Phi}'_{j+1})\cdot e'_0,\quad F'_0=\prod_{j\in\tau'}(\tilde{\Phi}'_1\tilde{\Phi}'_{j+1})\cdot f'_0,\quad K'_0=\Phi'_1\cdot k'_0.
\end{aligned}
\end{equation}
In this case, $0\in\tau$ and $0\notin\tau'$.
\end{case}
\end{proof}

\subsection{Relationship with earlier work}\label{remk:nonisotropic} 

Theorem \ref{thm:iso-main} was proved in \cite{Z3, Z2}
in the special case where ${{\mathfrak g}}$ does not have any isotropic odd roots, 
and hence ${{\mathfrak g}}'$ is an ordinary (affine) Lie algebra. 
The $({{\mathfrak g}}, {{\mathfrak g}}')$ pairs in this case are given in Table \ref{table:nonisotropic}.
\begin{table}[h]
\caption{Cases of quantum correspondences known before}
\label{table:nonisotropic}
\begin{tabular}{>{\centering\arraybackslash}m{1.2in}|>{\centering\arraybackslash}m{0.8in}|>{\centering\arraybackslash}m{3in}}
\hline
\vspace{2mm}
${{\mathfrak g}}$ \vspace{2mm}& ${{\mathfrak g}}'$& Transformation  \\
\hline
\multirow{2}{*}{${{\rm\mathfrak{osp}}}(1|2n)$} & \multirow{2}{*}{$B_n$ }            & $K_i=\sigma_ik_i$\\
                                              &  & $E_i=\Phi_{i+1}e_i, \ F_i=\Phi_if_i$\\

\hline
\multirow{2}{*}{${{\rm\mathfrak{osp}}}(2|2)^{(2)}$} & \multirow{2}{*}{ $A_1^{(1)}$  }      & $E_1= e_1, \  F_1=\sigma_1f_1,\   K_1=\sigma_1k_1$\\
                                                     & &$E_0= e_0,\  F_0=\sigma_1f_0, \  K_0=\sigma_1k_0$ \\

\hline
${{\rm\mathfrak{osp}}}(2|2n)^{(2)}$ & \multirow{2}{*}{ $D_{n+1}^{(2)}$ }        &$ E_i=\Phi_{i+1}e_i, \   F_i=\Phi_if_i,\   K_i=\sigma_ik_i$ \\
$(n\ge2)$              &  & $ E_0=\tilde{\Phi}_2e_0,\  F_0=\tilde{\Phi}_1f_0,\   K_0=\Phi_1k_0$\\

\hline
\multirow{2}{*}{${{\rm\mathfrak{osp}}}(1|2n)^{(1)}$}   & \multirow{2}{*}{ $A_{2n}^{(2)}$  }     & $ E_0=e_0,\   F_0=f_0, \  K_0= k_0$ \\
                                                        &  & $ E_i= \Phi_{i+1}e_i, \  F_i= \Phi_if_i,\  K_i=\sigma_ik_i$\\

\hline
${{\rm\mathfrak{sl}}}(1|2n)^{(2)}$     & \multirow{2}{*}{ $B_n^{(1)}$  }             & $ E_i=\Phi_{i+1}e_i, \  F_i=\Phi_if_i,\  K_i=\sigma_ik_i$ \\
$(n\geq 3)$            &    & $ E_0=\Phi_2e_0,\  F_0=\sigma_1\Phi_2f_0,
\  K_0=\sigma_1k_0$\\

\hline
\multirow{3}{*}{${{\rm\mathfrak{sl}}}(1|4)^{(2)}$}  & \multirow{3}{*}{$C_2^{(1)}$ } &  $E_2=e_2,\   F_2=\sigma_2f_2,\ K_2=\sigma_2k_2$  \\
                                                   &  & $E_1=\sigma_2e_0,\  F_1=\sigma_1\sigma_2f_1,\  K_1=\sigma_1k_1$\\
                                                   &  & $ E_0=\sigma_2e_0, \  F_0=\sigma_1\sigma_2f_0, \  K_0=\sigma_1k_0$ \\

\hline
\end{tabular}
\end{table}

For ${{\rm\mathfrak{sl}}}(1|4)^{(2)}$, we take  $\{\alpha_0=\delta-\delta_1-\delta_2,\alpha_1=\delta_1-\delta_2,\alpha_2=\delta_2\}$ as the fundamental system.
The formulae for the elements $E_i$, $F_i$ and $K_i$ in the table will be needed later.

The quantum correspondence between ${{\rm\mathfrak{osp}}}(2m+1|2n)$ and ${{\rm\mathfrak{osp}}}(2n+1|2m)$
for arbitrary $m$ and $n$  was discovered \cite{MW}  in the context of string theory,
and was formulated in terms of $T$- and $S$-dualities.

\section{Tensor equivalences of representation categories}\label{sect:tensor-cats}

We will prove Theorem \ref{thm:tensor-equiv} in this section. Let us begin by discussing some general facts on Hopf superalgebras, which will be needed presently.
\subsection{Picture changes and twists for Hopf superalgebras}\label{sect:Hopf}
\subsubsection{Picture changes}\label{sect:picture}

The category of vector superspaces can be regarded as the category of representations of the group algebra of ${{\mathbb Z}}_2:=\{1, u\}$ where $u^2=1$, which is a triangular Hopf algebra with the universal $R$-matrix
\[
R :=\frac{1}{2}\left(1 \otimes 1 + 1  \otimes u + u  \otimes 1- u  \otimes u\right) \in {{\mathbb C}}[{{\mathbb Z}}_2]  \otimes{{\mathbb C}}[{{\mathbb Z}}_2].
\]
A Hopf superalgebra ${{\mathscr H}}$ is then a Hopf algebra in this category. The grading of ${{\mathscr H}}$ is given by the ${{\mathbb Z}}_2$-action such that
\[
u. a = (-1)^{[a]} a,
\]
for any homogeneous $a\in {{\mathscr H}}$. For any  $a, b\in {{\mathscr H}}$, if we write their co-products as
$\Delta(a) = \sum a_{(1)}\otimes a_{(2)}$ and $\Delta(b) = \sum b_{(1)}\otimes b_{(2)}$ respectively, then $\Delta(a b)$ is given by
\[
\begin{aligned}
\Delta(a b) &= (m\otimes m)\left(\sum a_{(1)}\otimes \tau R(a_{(2)}\otimes b_{(1)})\otimes b_{(2)}\right),
\end{aligned}
\]
where $m$ is the multiplication of ${{\mathscr H}}$, and $\tau: v\otimes w\mapsto w\otimes v$ is the usual permutation map (without signs).  Then clearly
\[
\begin{aligned}
\Delta(a b)
&= \sum (-1)^{[b_{(1)})][a_{(2)}]}a_{(1)}b_{(1)}\otimes a_{(2)}b_{(2)}.
\end{aligned}
\]

By changing the category of ${{\mathbb Z}}_2$-representations one obtains a non-isomorphic Hopf superalgebra from any given one, 
such that its category of representations is equivalent to that of the original Hopf superalgebra as tensor category, see
\cite[Theorem 3.1.1]{AEG}  and \cite[ Chapter 10.1]{Ma}.

 Let us describe this more explicitly.

{\bf PC1} (\cite[Theorem 3.1.1]{AEG}).
Let $(H, \Delta, \epsilon, S)$ be an ordinary Hopf algebra with a group like element $u$ such that $u^2 = 1$. Using $u$, we decompose $H$ as a vector space into
$H=H_0\oplus H_1$ with
\begin{eqnarray}\label{eq:PO}
H_i=\left\{x\in H\mid u x u^{-1} = (-1)^i x\right\}.
\end{eqnarray}
This clearly defines a ${{\mathbb Z}}_2$-grading for $H$ as an associative algebra, thus turning it into an superalgebra.  We set $[x]=i$ for $x\in H_i$.
For any $x\in H$, write $\Delta(x)=\Delta_0(x) + \Delta_1(x)$ with  $\Delta_0(x) \in H\otimes H_0$ and $\Delta_1(x)\in H\otimes H_1$.  Define maps
\begin{eqnarray}\label{eq:corresp}
\begin{aligned}
&\Delta_u: H\longrightarrow H\otimes H, &&\quad \Delta_u(x)=\Delta_0(x) + \Delta_1(x)(u\otimes 1), \\
&S_u: H\longrightarrow H, &&\quad S_u(x) = u^{[x]} S(x).
\end{aligned}
\end{eqnarray}
Then $(H, \Delta_u, \epsilon, S_u)$ is a Hopf superalgebra. The
element $u$ acts as the parity operator (PO) of this Hopf superalgebra in the sense of \eqref{eq:PO}.

{\bf PC2} (\cite[Theorem 3.1.1]{AEG}).
Let $({{\mathscr H}}, \Delta, \epsilon, S)$ be a Hopf superalgebra with a group like element $g$ satisfying $g^2=1$, which acts as the parity operator in the sense that
$g xg^{-1} =(-1)^{[x]} x$ for all homogeneous $x\in {{\mathscr H}}$. We define maps
$\Delta_g: {{\mathscr H}}\longrightarrow {{\mathscr H}}\otimes {{\mathscr H}}$ and $S_g: {{\mathscr H}}\longrightarrow {{\mathscr H}}$ in exactly the same way as in \eqref{eq:corresp}. Then $({{\mathscr H}}, \Delta_g, \epsilon, S_g)$ is an ordinary Hopf algebra.

{\bf PC}. Let $({{\mathscr H}}, \Delta, \epsilon, S)$ be a Hopf superalgebra. Suppose that it has two group like elements $g$ and $u$  such that
\[
g^2=1=u^2, \quad g u=u g,\quad \text{and $g$ acts as the parity operator}.
\]
We apply {\bf PC2} to obtain an ordinary Hopf algebra, and then apply {\bf PC1} with $u$ to the ordinary Hopf algebra to obtain a new Hopf superalgebra with parity operator $u$:
\begin{eqnarray*}
\begin{array}{c}
\text{Hopf superalgebra}\\
({{\mathscr H}}, \Delta, \epsilon, S)\\
\text{with PO $g$}
\end{array}
\stackrel{PC2}{\xymatrix{{}\ar@{~>}[r]&{}}}
\begin{array}{c}
\text{Hopf algebra}\\
({{\mathscr H}}, \Delta_g, \epsilon, S_g)\\
\text{with $u$}
\end{array}
\stackrel{PC1}{\xymatrix{{}\ar@{~>}[r]&{}}}
\begin{array}{c}
\text{Hopf superalgebra}\\
({{\mathscr H}}, (\Delta_g)_u, \epsilon, (S_g)_u)\\
\text{with PO $u$}.
\end{array}
\end{eqnarray*}

\begin{definition} \label{def:PC} Call the operation of constructing the new Hopf superalgebra $({{\mathscr H}}, (\Delta_g)_u, \epsilon, (S_g)_u)$ with parity operator $u$ from a given Hopf superalgebra
$({{\mathscr H}}, \Delta, \epsilon, S)$ with parity operator $g$  a {\em picture change} ({\bf PC}) with respect to $g$ and $u$.
\end{definition}

\begin{remark}\label{rem:bosonisation}
This is loosely called ``bosonisation'' in the literature (see \cite{Ma} in particular).
As bosonisation means something very different in quantum field theory, we prefer the term ``picture change''.
\end{remark}

Representation categories of Hopf algebras and Hopf superalgebras are strict tensor categories.
For any ${{\mathscr H}}$-modules $M$ and $N$,  the ${{\mathbb Z}}_2$-graded action of $\Delta(x)$ and  the ordinary action of $\Delta_g(x)$ on $M\otimes N$ coincide for all $x\in {{\mathscr H}}$.
This in essence implies the tensor equivalence of the representation categories of the Hopf
superalgebra $({{\mathscr H}}, \Delta, \epsilon, S)$ and the ordinary Hopf algebra $({{\mathscr H}}, \Delta_g, \epsilon, S_g)$ related by {\bf PC2}.  Similarly one can show the
tensor equivalence of the representation categories of the Hopf
algebra $(H, \Delta, \epsilon, S)$ and Hopf superalgebra $(H, \Delta_u, \epsilon, S_u)$
related by {\bf PC1}. See  \cite[Theorem 3.1.1]{AEG}.

We summarise the above into the following

\begin{theorem} \label{thm:PC} Let $({{\mathscr H}}, \Delta, \epsilon, S)$ be a Hopf superalgebra with group like elements $g$ and $u$ as described above such that $g$ acts as the parity operator.
Then a picture change turns this Hopf superalgebra into a new Hopf superalgebra
$({{\mathscr H}}, (\Delta_g)_u, \epsilon, (S_g)_u)$ with parity operator $u$. The categories of representations of the two Hopf superalgebras are equivalent as strict tensor categories.
\end{theorem}

\subsubsection{Twisting the coalgebra structure}
\setcounter{case}{0}

Recall the following well known fact. Let $({{\mathscr H}}, \Delta, \epsilon, S)$ be a Hopf superalgebra. Given an invertible even element ${{\mathcal J}}\in {{\mathscr H}}\otimes{{\mathscr H}}$ satisfying the conditions
\begin{eqnarray}\label{eq:twist}
\begin{aligned}
&(\Delta\otimes{{\rm{id}}})({{\mathcal J}})({{\mathcal J}}\otimes 1)=({{\rm{id}}}\otimes\Delta)({{\mathcal J}})(1\otimes{{\mathcal J}}), \\
&(\epsilon\otimes{{\rm{id}}})({{\mathcal J}})=({{\rm{id}}}\otimes\epsilon)({{\mathcal J}})=1,
\end{aligned}
\end{eqnarray}
one can twist the coalgebra structure to obtain a new Hopf superalgebra $({{\mathscr H}}, \Delta^{{\mathcal J}}, \epsilon, S^{{\mathcal J}})$ with the same underlying associative superalgebraic structure on ${{\mathscr H}}$.
The new comultiplication $\Delta^{{\mathcal J}}$ and antipode $S^{{\mathcal J}}$ are given by
\[
\Delta^{{\mathcal J}}(x)= {{\mathcal J}}^{-1}\Delta(x) {{\mathcal J}}, \quad S^{{\mathcal J}}(x)={{\mathcal G}}^{-1} S(x) {{\mathcal G}}, \quad \forall x\in {{\mathscr H}},
\]
with ${{\mathcal G}}=m\circ(S\otimes{{\rm{id}}})({{\mathcal J}})$, where $m$ is the multiplication of ${{\mathscr H}}$. The element ${{\mathcal J}}$ is called a {\em twist} for ${{\mathscr H}}$. Note that twisting does not change the counit.

\subsection{Main theorem on quantum correspondences of affine superalgebras}
 Keep the notation in Section \ref{sect:quantum}.
Let ${{\mathfrak g}}$ be a Lie superalgebra or affine Lie superalgebra in Table \ref{table:classical} or Table \ref{table:affine} with a fundamental system $\Pi$. 
Then there exists a corresponding ${{\mathfrak g}}'$ such that $({{\mathfrak g}}, {{\mathfrak g}}')$ is a pair in Theorem \ref{thm:main-quan}.  Now $\Pi'=\phi(\Pi)$ is a fundamental system of ${{\mathfrak g}}'$.

Consider ${{\mathfrak U}}_q({{\mathfrak g}}, \Pi)$ as a Hopf superalgebra with the standard grading. As before, we denote its comultiplication, counit and antipode by $\Delta, \epsilon$ and $S$ respectively. Let
\begin{eqnarray}\label{eq:u1u2}
\begin{aligned}
&u_1:=\prod_{i\in\tau}(\tilde{\Phi}_1\tilde{\Phi}_{i+1}), \quad &u_2:=\tilde{\Phi}_1 u_1,
\end{aligned}
\end{eqnarray}
Then $u_1$ is the parity operator of ${{\mathfrak U}}_q({{\mathfrak g}}, \Pi)$.

Applying a picture change with respect to $u_1$ and $u_2$ to $({{\mathfrak U}}_q({{\mathfrak g}}, \Pi), \Delta, \epsilon, S)$, we obtain the
Hopf superalgebra $({{\mathfrak U}}_q({{\mathfrak g}}, \Pi), (\Delta_{u_1})_{u_2}, \epsilon, (S_{u_1})_{u_2})$
with parity operator $u_2$.
The new ${{\mathbb Z}}_2$-grading of ${{\mathfrak U}}_q({{\mathfrak g}}, \Pi)$, induced by $u_2$,  is given by
\begin{eqnarray}\label{eq:new-grade}
\begin{aligned}
&{{\mathfrak U}}_q({{\mathfrak g}}, \Pi)={{\mathfrak U}}_q({{\mathfrak g}}, \Pi)'_0\oplus{{\mathfrak U}}_q({{\mathfrak g}}, \Pi)'_1  \quad \text{with}\\
&{{\mathfrak U}}_q({{\mathfrak g}}, \Pi)'_\theta=\left\{x\in {{\mathfrak U}}_q({{\mathfrak g}}, \Pi)\mid u_2 x u_2^{-1} = (-1)^\theta x\right\},
\quad \theta=0, 1.
\end{aligned}
\end{eqnarray}
We write $\tilde{\Delta}=(\Delta_{u_1})_{u_2}$ and $\tilde{S}=(S_{u_1})_{u_2}$, and use
$({{\mathfrak U}}_q({{\mathfrak g}}, \Pi), \tilde{\Delta}, \epsilon, \tilde{S})$ to denote
this new Hopf superalgebra with the ${{\mathbb Z}}_2$-grading given by \eqref{eq:new-grade}.

Recall the elements $E_i, F_i, K_i^{\pm 1}$ of ${{\mathfrak U}}_q({{\mathfrak g}}, \Pi)$ introduced in Section \ref{sect:quantum}. 
They together with the elements $\sigma_i$ generate ${{\mathfrak U}}_q({{\mathfrak g}}, \Pi)$. We have the following easy observation.
\begin{lemma} For any fixed $i$, the elements $E_i, F_i$ belong to ${{\mathfrak U}}_q({{\mathfrak g}}, \Pi)'_0$
(resp. ${{\mathfrak U}}_q({{\mathfrak g}}, \Pi)'_1$) if and only if $\phi(\alpha_i)$ is an even (resp. odd) simple root in $\Pi'$.
\end{lemma}
This immediately implies
\begin{corollary} \label{cor:alg-iso} The associative algebra isomorphism ${{\mathfrak U}}_{-q}({{\mathfrak g}}', \Pi')\stackrel{\cong}{\longrightarrow} {{\mathfrak U}}_q({{\mathfrak g}}, \Pi)$ of Theorem \ref{thm:main-quan} defined by \eqref{eq:B-map}
is an isomorphism of superalgebras if
${{\mathfrak U}}_q({{\mathfrak g}}, \Pi)$ is given the ${{\mathbb Z}}_2$-grading \eqref{eq:new-grade} induced by $u_2$, while ${{\mathfrak U}}_{-q}({{\mathfrak g}}', \Pi')$ has  the usual ${{\mathbb Z}}_2$-grading.
\end{corollary}

Recall that $|\Pi|$ denotes the cardinality of $\Pi$. Define
\begin{eqnarray}\label{eq:J}
{{\mathcal J}}:=\frac{1}{2^{|\Pi|}}{{\mathscr T}}, \quad {{\mathscr T}}:={{\mathscr T}}^{(0)}{{\mathscr T}}^{(1)},
\end{eqnarray}
where
\[
\begin{aligned}
{{\mathscr T}}^{(0)}:=&\prod_{i\notin\tau}{{\mathscr T}}^{(0)}_i, \quad {{\mathscr T}}^{(1)}:=\prod_{i\in\tau}{{\mathscr T}}^{(1)}_i, \\
{{\mathscr T}}^{(0)}_i:=&(1+\tilde{\Phi}_1\tilde{\Phi}_{i+1})\otimes 1+(1-\tilde{\Phi}_1\tilde{\Phi}_{i+1})\otimes \Phi_{i+1},\quad i\notin\tau, \\
{{\mathscr T}}^{(1)}_i:=&(1+\tilde{\Phi}_1\tilde{\Phi}_{i+1})\otimes 1+(1-\tilde{\Phi}_1\tilde{\Phi}_{i+1})\otimes \tilde{\Phi}_{i+2}, \quad i\in\tau.
\end{aligned}
\]
\begin{lemma}\label{lem:j-properties}
The element ${{\mathcal J}}$ defined by \eqref{eq:J} satisfies the relations
 \[
 \begin{aligned}
&(\tilde{\Delta}\otimes{{\rm{id}}})({{\mathcal J}})({{\mathcal J}}\otimes 1)=({{\rm{id}}}\otimes\tilde{\Delta})({{\mathcal J}})(1\otimes{{\mathcal J}}), \\
&(\epsilon\otimes{{\rm{id}}})({{\mathcal J}})=({{\rm{id}}}\otimes\epsilon)({{\mathcal J}})=1.
\end{aligned}
\]
\end{lemma}
\begin{proof}
The second relation is clear since $\epsilon(\Phi_i)=\epsilon(\tilde\Phi_i)=1$ for all $i$.

To prove the first relation, note that ${{\mathcal J}}$, $u_1$ and $u_2$ involve only the even elements $\sigma_i$ of ${{\mathfrak U}}_q({{\mathfrak g}}, \Pi)$, which commute  among themselves.
Thus the first relation is equivalent to that obtained by replacing $\tilde{\Delta}$ by $\Delta$.

For any elements $\sigma,\sigma'$ in $\mathrm{G}$, denote $x=(1+\sigma)\otimes 1+(1-\sigma)\otimes \sigma'$. It can be proven by direct computations that
$(\Delta\otimes{{\rm{id}}})(x)(x\otimes 1)=({{\rm{id}}}\otimes\Delta)(x)(1\otimes x)$, and hence $(\tilde{\Delta}\otimes{{\rm{id}}})(x)(x\otimes 1)=({{\rm{id}}}\otimes\tilde{\Delta})(x)(1\otimes x)$. As ${{\mathscr T}}$ is the product of elements of the form $x$, this immediately leads to the first relation.
\end{proof}
\begin{remark}
We have ${{\mathcal J}}^{-1}={{\mathcal J}}$.
\end{remark}

By Lemma \ref{lem:j-properties}, we can twist the Hopf superalgebra $({{\mathfrak U}}_q({{\mathfrak g}}, \Pi), \tilde{\Delta}, \epsilon, \tilde{S})$ using the element ${{\mathcal J}}$ given in \eqref{eq:J} to obtain a new Hopf superalgebra $({{\mathfrak U}}_q({{\mathfrak g}}, \Pi), \tilde{\Delta}^{{\mathcal J}}, \epsilon, \tilde{S}^{{\mathcal J}})$. We emphasize that the ${{\mathbb Z}}_2$-grading is given by \eqref{eq:new-grade}.

\begin{lemma}\label{lem:tilde-Delta}
The comultiplication, counit and antipode of the Hopf superalgebra $({{\mathfrak U}}_q({{\mathfrak g}}, \Pi), \tilde{\Delta}^{{\mathcal J}}, \epsilon, \tilde{S}^{{\mathcal J}})$ are given by
\[
\begin{aligned}
&\tilde{\Delta}^{{\mathcal J}}(\sigma_i)=\sigma_i\otimes\sigma_i, \quad \tilde{\Delta}^{{\mathcal J}}(K_i)=K_i\otimes K_i, \\
&\tilde{\Delta}^{{\mathcal J}}(E_i)=E_i\otimes 1 + K_i  \otimes E_i, \quad \tilde{\Delta}^{{\mathcal J}}(F_i)=F_i\otimes K^{-1}_i + 1 \otimes F_i, \\
&\epsilon(E_i)=0, \quad \epsilon(F_i)=0, \quad \epsilon(K_i)=1,   \quad \epsilon(\sigma_i)=1\\
&\tilde{S}^{{\mathcal J}}(E_i)=- K^{-1}_i E_i, \quad \tilde{S}^{{\mathcal J}}(F_i)=- F_i K_i , \quad \tilde{S}^{{\mathcal J}}(K_i)=K^{-1}_i, \quad \tilde{S}^{{\mathcal J}}(\sigma_i)=\sigma^{-1}_i, \quad \forall i.
\end{aligned}
\]
\end{lemma}
The proof of the lemma is given in Section \ref{pf:key-lemma}.

As an easy consequence of this lemma and Corollary \ref{cor:alg-iso},
we have the following result, which incorporates Theorems \ref{thm:main-quan} and \ref{thm:tensor-equiv} in a single theorem.
\begin{theorem}\label{them: hopf connection}
Let $({{\mathfrak g}}, {{\mathfrak g}}')$ be a pair of (affine) Lie superalgebras  in Theorem \ref{thm:main-quan}. Then the quantum (affine) superalgebra ${{\mathfrak U}}_{-q}({{\mathfrak g}}',\Pi')$ with the standard Hopf superalgebra structure is isomorphic to $({{\mathfrak U}}_q({{\mathfrak g}}, \Pi), \tilde{\Delta}^{{\mathcal J}}, \epsilon, \tilde{S}^{{\mathcal J}})$.
\end{theorem}

\begin{definition}\label{def:correspond} Call this Hopf superalgebra isomorphism a quantum correspondence between the (affine) Lie superalgebras ${{\mathfrak g}}$ and ${{\mathfrak g}}'$.
\end{definition}

\begin{proof}[Proof of Theorem \ref{them: hopf connection}]
By Corollary \ref{cor:alg-iso}, the map \eqref{eq:B-map}
is an isomorphism of associative superalgebras, and by Lemma \ref{lem:tilde-Delta}, it is a Hopf superalgebra map. Hence follows the theorem.
\end{proof}

\begin{proof}[Proof of Theorem \ref{thm:tensor-equiv}]
This immediately follows from Theorem \ref{them: hopf connection} by using 
Theorem \ref{thm:PC}. 
\end{proof}

\begin{remark} 
For a given (affine) Lie superalgebra ${{\mathfrak g}}$, 
the quantised universal enveloping superalgebras ${{\rm{U}_q}}({{\mathfrak g}}, \Pi)$  corresponding to different fundamental systems are non-isomorphic as Hopf superalgebras  in general. 
Thus Theorem \ref{them: hopf connection} depends on the fundamental systems nontrivially. 
\end{remark}

\subsection{Proof of Lemma \ref{lem:tilde-Delta}}\label{pf:key-lemma}

The relations for the counit are clear,
and the antipode relations can be easily obtained from the comultiplication
and the counit. Note that  ${{\mathcal J}}$ depends only on $\sigma_i$.
Since $K_i$ and $\sigma_i$ are all even and
commute among themselves, we immediately have
\[
\begin{aligned}
&\tilde{\Delta}^{{\mathcal J}}(\sigma_i)=\tilde{\Delta}(\sigma_i)=\Delta(\sigma_i)=\sigma_i\otimes\sigma_i, \quad
&\tilde{\Delta}^{{\mathcal J}}(K_i)=\tilde{\Delta}(K_i)=\Delta(K_i)=K_i\otimes K_i.
\end{aligned}
\]
Thus what remains to be proven are the formulae for $\tilde{\Delta}^{{\mathcal J}}(E_i)$
and $\tilde{\Delta}^{{\mathcal J}}(F_i)$.

We  divide the proof into three cases.

\begin{case}
{\em $\Pi$ and $\Pi'$ contain no isotropic odd roots}.
\end{case}
In this case, the  pairs $({{\mathfrak g}}, {{\mathfrak g}}')$ are given by Table \ref{table:nonisotropic}.
Note that $\tau'=\emptyset$ and $u_2=1$. Thus only the special picture change {\bf PC2}  is needed.
We present the details of the proof of Lemma \ref{lem:tilde-Delta} for the comultiplication in the case $({{\mathfrak g}},{{\mathfrak g}}')=({{\rm\mathfrak{osp}}}(1|2n),B_n)$. The proofs for the other cases are very similar, thus we omit the details.

We maintain the notation of Section \ref{remk:nonisotropic} .

The picture change {\bf PC2} with $g=u_1=\tilde{\Phi}_1$ applied to the Hopf superalgebra
$({{\mathfrak U}}_{q}({{\rm\mathfrak{osp}}}(1|2n)),\Delta,{\varepsilonup},S)$ leads to the ordinary Hopf algebra $({{\mathfrak U}}_{q}({{\rm\mathfrak{osp}}}(1|2n)),{\tilde{\Delta}},{\varepsilonup},{\tilde{S}})$, where the comultiplication is given by ${\tilde{\Delta}}(\sigma_i)=\sigma_i\otimes \sigma_i$, ${\tilde{\Delta}}(K_i)=K_i\otimes K_i$, for $1\le i\le n$, and
\[
\begin{aligned}
&{\tilde{\Delta}}(E_j)=E_j\otimes\Phi_{j+1}+\Phi_jK_j\otimes E_j,\quad {\tilde{\Delta}}(F_j)=F_j\otimes\Phi_{j+1}K_j^{-1}+\Phi_j\otimes F_j, \quad \text{$j\ne n$,}\\
&{\tilde{\Delta}}(E_n)=E_n\otimes 1+\tilde{\Phi}_1 \Phi _nK_n\otimes E_n,\quad {\tilde{\Delta}}(F_n)=F_n\otimes K_n^{-1}+\tilde{\Phi}_1\Phi_n\otimes F_n.
\end{aligned}
\]

We can prove the following relations by easy calculations,
\begin{eqnarray}
\label{eq:eti-L}
&&\begin{aligned}
(E_i\otimes 1){{\mathscr T}}_j^{(0)}=\begin{cases}
{{\mathscr T}}_j^{(0)}(E_i\otimes \Phi_{i+1}), & j=i,\\
{{\mathscr T}}_j^{(0)}(E_i\otimes 1),&j\ne i,
\end{cases}
\end{aligned}\\
\label{eq:et1-L}
&&\begin{aligned}
(1\otimes E_1){{\mathscr T}}_j^{(0)}=\begin{cases}
{{\mathscr T}}_1^{(0)}(\Phi_1\otimes E_1), & j=1,\\
{{\mathscr T}}_j^{(0)}(1\otimes E_1),&j\ne 1,
\end{cases}
\end{aligned}\\
\label{eq:eti}
&&\begin{aligned}
(1\otimes E_i){{\mathscr T}}_j^{(0)}&=\begin{cases}
{{\mathscr T}}_j^{(0)}(\tilde{\Phi}_1\tilde{\Phi}_{j+1}\otimes E_i), & j=i, i-1,\\
{{\mathscr T}}_j^{(0)}(1\otimes E_i),&j\ne i, i-1,
\end{cases}\quad \text{for $i\ne 1, n$}
\end{aligned}\\
\label{eq:etn}
&&\begin{aligned}
(1\otimes E_n){{\mathscr T}}^{(0)}_j&=\begin{cases}
{{\mathscr T}}^{(0)}_j(\tilde{\Phi}_1 \Phi _n\otimes E_n), & j= n-1,\\
{{\mathscr T}}^{(0)}_j(1\otimes E_n),&j\ne n-1,
\end{cases}
\end{aligned}
\end{eqnarray}
from which we obtain
\begin{eqnarray*}
&&(E_i\otimes  \Phi_{i+1}){{\mathcal J}}={{\mathcal J}}(E_i\otimes 1), \quad \forall i, \\
&&(\Phi_i\otimes E_i){{\mathscr T}}={{\mathscr T}}(1\otimes E_i), \quad i\ne n, \\
&&(\tilde{\Phi}_1 \Phi _n\otimes E_n){{\mathscr T}}={{\mathscr T}}(1\otimes E_n).
\end{eqnarray*}
These relations together immediately lead to
\[\begin{aligned}
{\tilde{\Delta}}^{{\mathcal J}}(E_i)&={{\mathcal J}}^{-1}{\tilde{\Delta}} (E_i) {{\mathcal J}}=E_i\otimes 1+ K_i\otimes E_i, \quad \forall i.
\end{aligned}\]
Similar arguments as above can be used to prove the formula for ${\tilde{\Delta}}^{{\mathcal J}}(F_i)$ for all $i$. This completes the proof for the case  ${{\mathfrak g}}={{\rm\mathfrak{osp}}}(1|2n)$.

\begin{case}
$({{\mathfrak g}},{{\mathfrak g}}')=({{\rm\mathfrak{osp}}}(2m+1|2n), {{\rm\mathfrak{osp}}}(2n+1|2m))$ with $m\ne 0\ne n$.
\end{case}

In the Hopf superalgebra $({{\mathfrak U}}_q({{\rm\mathfrak{osp}}}(2m+1|2n),\Pi), {\tilde{\Delta}},{\varepsilonup},{\tilde{S}})$,  we have
\[\begin{aligned}
&{\tilde{\Delta}}(E_i)=E_i\otimes\Phi_{i+1}+\Phi_iK_i\otimes E_i, \\
&{\tilde{\Delta}}(F_i)=F_i\otimes\Phi_{i+1}K_i^{-1}+\Phi_i\otimes F_i, \quad m+n\ne i\notin\tau,\\
&{\tilde{\Delta}}(E_j)=E_j\otimes\tilde{\Phi}_{j+2}+\tilde{\Phi}_1\tilde{\Phi}_jK_j\otimes E_j,\\
&{\tilde{\Delta}}(F_j)=F_j\otimes\tilde{\Phi}_{j+2}K_j^{-1}+\tilde{\Phi}_1\Phi_j\otimes F_j, \quad m+n\ne j\in\tau\\
&{\tilde{\Delta}}(E_{m+n})=E_{m+n}\otimes 1+u\Phi_{m+n}K_{m+n}\otimes E_{m+n},\\
&{\tilde{\Delta}}(F_{m+n})=F_{m+n}\otimes K_{m+n}^{-1}+u\Phi_{m+n}\otimes F_{m+n},
\end{aligned}\]
where  $u=\tilde{\Phi}_1\prod_{j\in\tau,j\neq m+n}(\tilde{\Phi}_1\tilde{\Phi}_{j+1})$, which is $u_1$ if $m+n\in\tau$, and is $u_2$ if $m+n\notin\tau$.

Now consider the formula for ${\tilde{\Delta}}^{{\mathcal J}}(E_i)$. Using
\begin{eqnarray*}
&&(1\otimes E_1){{\mathscr T}}^{(1)}_j={{\mathscr T}}^{(1)}_j(1\otimes E_1),\\
&&(1 \otimes E_i) {{\mathscr T}}^{(1)}_j=\begin{cases}
{{\mathscr T}}^{(1)}_j(\tilde{\Phi}_1\tilde{\Phi}_{i}\otimes E_i),& j=i-1,\\
{{\mathscr T}}^{(1)}_j(1 \otimes E_i),&j\ne i-1.
\end{cases}\quad\text{for $i\neq 1,m+n$,}
\\
&&(1\otimes E_{m+n}){{\mathscr T}}^{(1)}_j=\begin{cases}
{{\mathscr T}}^{(1)}_j(\tilde{\Phi}_1\tilde{\Phi}_{j+1}\otimes E_{m+n}),&j\neq m+n-1,m+n,\\
{{\mathscr T}}^{(1)}_j(1\otimes E_{m+n}), & j=m+n-1,m+n,
\end{cases} \\
&&(E_i\otimes 1)
{{\mathscr T}}^{(1)}_j=\begin{cases}
{{\mathscr T}}_j^{(1)}(E_i\otimes\tilde{\Phi}_{i+2}) ,& j=i,\\
{{\mathscr T}}_j^{(1)}(E_i\otimes 1),& j\neq i,
\end{cases}\quad\text{for $\forall i$,}\\
\end{eqnarray*}
and relations \eqref{eq:eti-L}, \eqref{eq:et1-L}, \eqref{eq:eti} and  \eqref{eq:etn}, we obtain
\begin{eqnarray}\label{eq:EmnT}
\begin{aligned}
&(E_i\otimes 1) {{\mathscr T}}=\begin{cases}
{{\mathscr T}}(E_i\otimes \Phi_{i+1}), & i\notin\tau,\\
{{\mathscr T}}(E_i\otimes \tilde{\Phi}_{i+2}), & i\in\tau.
\end{cases}\\
&(1 \otimes E_i) {{\mathscr T}}=\begin{cases}
{{\mathscr T}} (\tilde{\Phi}_1\tilde{\Phi}_i\otimes E_i), & i\in\tau,\\
{{\mathscr T}}(\Phi_i\otimes E_i), &i\notin\tau.
\end{cases}\quad \text{for $i\ne m+n$}\\
&(1\otimes E_{m+n}){{\mathscr T}}={{\mathscr T}}(u\Phi_{m+n}\otimes E_{m+n}).
\end{aligned}
\end{eqnarray}
This allows us to immediately show that
\[\begin{aligned}
{\tilde{\Delta}}^{{\mathcal J}}(E_i)=E_i\otimes 1+ K_i\otimes E_i, \quad \forall i.
\end{aligned}\]
We can similarly prove the formula for ${\tilde{\Delta}}^{{\mathcal J}}(F_i)$.

\begin{case}
{\em Affine Lie superalgebra pairs}.
\end{case}

The same arguments as in Case 2 prove the formulae for ${\tilde{\Delta}}^{{\mathcal J}}(E_i), {\tilde{\Delta}}^{{\mathcal J}}(F_i),
{\tilde{\Delta}}^{{\mathcal J}}(K_i)$ with $i>0$ for any affine Lie superalgebra pair $({{\mathfrak g}},{{\mathfrak g}}')$ in Theorem \ref{thm:main-quan}. 
Thus we only need to consider ${\tilde{\Delta}}^{{\mathcal J}}(E_0)$, ${\tilde{\Delta}}^{{\mathcal J}}(F_0)$ and ${\tilde{\Delta}}^{{\mathcal J}}(K_0)$. The case of $K_0$ is obvious. We now consider $E_0$ in detail, and  we can treat $F_0$ similarly.

\begin{subcase}
$({{\mathfrak g}},{{\mathfrak g}}')=({{\rm\mathfrak{sl}}}(2m+1|2n)^{(2)}, {{\rm\mathfrak{osp}}}(2n+1|2m)^{(1)})$.
\end{subcase}

For the Type (1) and Type (2) Dynkin diagrams in Table \ref{table:Dynkin diagram-sl2},
we have
\[
{\tilde{\Delta}}(E_0)=E_0\otimes 1+ K_0\otimes E_0, \quad {\tilde{\Delta}}(F_0)=F_0\otimes  K_0^{-1}+1 \otimes F_0.
\]
Note that $(E_0\otimes 1){{\mathscr T}}={{\mathscr T}}(E_0\otimes 1)$ and $(1\otimes E_0){{\mathscr T}}={{\mathscr T}}(1\otimes E_0)$. Thus Lemma \ref{lem:tilde-Delta} immediately follows.

Now consider Type (3) Dynkin diagrams in Table \ref{table:Dynkin diagram-sl2}. In this case, $0,1\notin\tau$   and
\[\begin{aligned}
{\tilde{\Delta}}(E_0)=E_0\otimes \Phi_2+\Phi_1K_0\otimes E_0.
\end{aligned}\]
On the other hand,
\[\begin{aligned}
(E_0\otimes 1){{\mathscr T}}&={{\mathscr T}}^{(1)}\prod_{j\notin\tau,j\ne 1}{{\mathscr T}}^{(0)}_j (E_0\otimes 1){{\mathscr T}}^{(0)}_1={{\mathscr T}}(E_0\otimes\Phi_2),\\
(1\otimes E_0){{\mathscr T}}&={{\mathscr T}}^{(1)}\prod_{j\notin\tau,j\ne 1}{{\mathscr T}}^{(0)}_j (1\otimes E_0){{\mathscr T}}^{(0)}_1={{\mathscr T}}(\Phi_1\otimes E_0).
\end{aligned}\]
Therefore,
$
{\tilde{\Delta}}^{{\mathcal J}}(E_0)=E_0\otimes 1+ K_0\otimes E_0.
$

For Type (4) Dynkin diagrams in Table \ref{table:Dynkin diagram-sl2},
we have $0,1\in\tau$,  and hence
\[\begin{aligned}
&{\tilde{\Delta}}(E_0)=E_0\otimes \tilde{\Phi}_3+ K_0\otimes E_0.
\end{aligned}\]
The parity of $E_0$ is altered by the  picture change.
Using
\[\begin{aligned}
(1\otimes E_0){{\mathscr T}}=  {{\mathscr T}}( 1\otimes E_0),\quad (E_0\otimes 1){{\mathscr T}}={{\mathscr T}}(E_0\otimes \tilde{\Phi}_3),
\end{aligned}\]
we immediately obtain  ${\tilde{\Delta}}^{{\mathcal J}}(E_0)=E_0\otimes 1+ K_0\otimes E_0$.

\begin{subcase}
$({{\mathfrak g}},{{\mathfrak g}}')=({{\rm\mathfrak{osp}}}(2m+2|2n)^{(2)}, {{\rm\mathfrak{osp}}}(2n+2|2m)^{(2)})$.
\end{subcase}

The most involved case is associated with the Type (1) Dynkin diagrams in
Table \ref{table:Dynkin diagram-osp2}, and we consider this case only.
The parity of $E_0$ is changed by the picture change. We have
\[\begin{aligned}
&{\tilde{\Delta}}(E_0)=E_0\otimes \tilde{\Phi}_2\prod_{j\in\tau}(\tilde{\Phi}_1\tilde{\Phi}_{j+1})+ K_0\otimes E_0.
\end{aligned}\]

It is easy to show that $(E_0\otimes  1){{\mathscr T}}^{(0)}_0={{\mathscr T}}^{(0)}_0(E_0\otimes 1)$ and for $j\neq 0$,
\[\begin{aligned}
(E_0\otimes  1){{\mathscr T}}^{(0)}_j={{\mathscr T}}^{(0)}_j(E_0\otimes \Phi_{j+1}),\quad (E_0\otimes  1){{\mathscr T}}^{(1)}_j={{\mathscr T}}^{(1)}_j(E_0\otimes \tilde{\Phi}_{j+2}).
\end{aligned}\]
Using these relations, we can prove that
\[\begin{aligned}
&\left(E_0\otimes  \tilde{\Phi}_2\prod_{j\in\tau}(\tilde{\Phi}_1\tilde{\Phi}_{j+1})\right){{\mathscr T}}={{\mathscr T}}(E_0\otimes 1),
\end{aligned}\]
where we have used $\tilde{\Phi}_2=\Pi_{j\ge 1}\Phi_{j+1}$. It is also easily seen that
\[\begin{aligned}
(1\otimes E_0){{\mathscr T}}  ={{\mathscr T}} (1\otimes E_0).
\end{aligned}\]
Combining the two relations above, we obtain ${\tilde{\Delta}}^{{\mathcal J}}(E_0)=E_0\otimes 1+ K_0\otimes E_0$.

\begin{remark}
We expect that for any pair $({{\mathfrak g}}, {{\mathfrak g}}')$ in Theorem  \ref{thm:main-quan}, the 
representation categories of ${{\mathfrak U}}_q({{\mathfrak g}}, \Pi)$ and of ${{\mathfrak U}}_{-q}({{\mathfrak g}}', \Pi')$ are 
equivalent as braided strict tensor categories. One should be able to prove 
this following \cite[Chapter 10.1]{Ma}.
\end{remark}

\section*{Acknowledgements}
This research was supported by National Natural Science Foundation of China Grants No. 11301130,  No. 11431010; 
and Australian Research Council Discovery-Project Grant DP140103239.
Xu wishes to thank the School of Mathematics and Statistics at the University of Sydney for its hospitality.

\begin{thebibliography}{9999}

\bibitem{AEG} Andruskiewitsch, N.; Etingof, P.; Gelaki, S., Triangular Hopf algebras with the Chevalley property. {\sl Michigan Math. J. \bf 49} (2001), 227-198.

\bibitem{BGZ}  Bracken, A. J.; Gould, M. D.; Zhang, R. B.,
Quantum supergroups and solutions of the Yang-Baxter equation.
{\sl Modern Phys. Lett. A. \bf 5} (1990), no. 11, 831--840.

\bibitem{D1} Drinfeld, V.G., Quantum Group. {\sl In Proceeding of the International Congress of Mathematicans, Berkeley} (1986), Vol.1; {\sl Amer. Math. Soc.} (1987), 798-820.

\bibitem{GL} Gorelik, M.; Lanzmann, E.,
The annihilation theorem for the completely reducible Lie superalgebras.
{\sl Invent. Math. \bf 137} (1999), no. 3, 651--680.

\bibitem{J}Jimbo, M., A q-difference analogue of ${{\rm U}} ({{\mathfrak g}})$ and the Yang-Baxter equation. {\sl Lett. Math. Phys. \bf 10} (1985), 63-69.

\bibitem{K1}Kac, V.G., Lie superalgebras. {\sl  Adv. Math. \bf 26} (1977), 8-96.

\bibitem{K2}Kac, V.G., Infinite-dimensional Lie algebras. Third edition. Cambridge University Press, Cambridge, 1990.

\bibitem{LE}Lanzmann, E., The Z transformation and $U_q (osp (1, 2l))$-Verma modules annihilators.  {\sl Alge.  Rep. Theory  {\bf 5}(3)} (2002), 235-258.

\bibitem{Ma} Majid, S., Foundations of Quantum GroupTheory.  Cambridge University Press, 1995.

\bibitem{MW}  Mikhaylov, V.; Witten, E.,
 Branes and supergroups. {\sl Comm. Math. Phys. \bf 340} (2015), 699--832.

\bibitem{RS} Rittenberg, V.; Scheunert, M., A remarkable connection between the representations of the Lie superalgebras ${{\rm\mathfrak{osp}}}(1,2n)$ and the Lie algebras $o(2n+1)$. {\sl Comm. Math. Phys. \bf 83} (1982), no. 1, 1--9.

\bibitem{JWV} van de Leur,  J.W., Contragredient Lie superalgebras of finite growth. Utrecht thesis (1985).

\bibitem{Y}  Yamane, H., Universal $R$-matrices for quantum groups associated to simple Lie superalgebras. {\sl Proc. Japan Academy, Series A, Math Sciences  {\bf 67}(4)}  (1991), 108-112.

\bibitem{ZGB}Zhang, R. B., M D Gould and A J Bracken,
Solutions of the graded classical Yang-Baxter equation and integrable models.
{\sl J. Phys. A: Math. Gen. \bf 24} (1991), 1185--1197.

\bibitem{Z3} Zhang, R. B., Finite-dimensional representations of ${{\rm U}}_q (osp (1/2n))$ and its connection with quantum $so (2n+ 1)$. {\sl Lett. Math. Physics, \bf 25}  (1992),  317-325.

\bibitem{Z93} Zhang, R. B.,
Finite dimensional irreducible representations of the quantum supergroup ${{\rm{U}_q}} (gl (m/n))$.
{\sl J. Math Phys. \bf 34} (1993), 1236-1254.

\bibitem{Z2} Zhang, R. B., Symmetrizable quantum affine superalgebras and their representations.  {\sl J. Math Phys. \bf 38} (1997),  535--543.

\bibitem{Z98} Zhang, R. B., Structure and representations of the quantum general linear supergroup. {\sl Comm. Math. Phys. \bf 195} (1998), no. 3, 525--547.

\bibitem{Z1} Zhang, R.B., Serre presentations of Lie superalgebras. In {\em Advances in Lie Superalgebras}, 235--280, Springer INdAM Ser., 7, Springer, Cham, 2014.

\end{thebibliography}
\end{document}


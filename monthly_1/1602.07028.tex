

\documentclass[leqno]{amsart}
\usepackage{amsmath,amssymb}
\usepackage{booktabs,mathrsfs,enumerate,mathtools}
\usepackage{hyperref,framed}
\hypersetup{pdftitle={Quiver Hecke algebras for alternating groups},
  pdfauthor={Clinton Boys and Andrew Mathas},
  colorlinks = true,
  linkcolor =blue,
  anchorcolor = red,
  citecolor = blue,
  urlcolor = blue
}

\usepackage{todonotes}
\long
\long
\long

\synctex=1

\usepackage{aliascnt}

\theoremstyle{plain}
\newcounter{mainTheorem}

\newtheorem{MainTheorem}[mainTheorem]{Theorem}

\newcounter{mainCorollary}
\numberwithin{mainCorollary}{mainTheorem}

\newtheorem{MainCorollary}[mainCorollary]{Corollary}

\newtheorem*{Notation}{Notation}

\swapnumbers
\numberwithin{equation}{section}
{\newaliascnt{{Assumption}}{equation}
\newtheorem{{Assumption}}[{Assumption}]{{Assumption}}
\aliascntresetthe{{Assumption}}
\expandafterautorefname\endcsname{{Assumption}}
}
{\newaliascnt{{Proposition}}{equation}
\newtheorem{{Proposition}}[{Proposition}]{{Proposition}}
\aliascntresetthe{{Proposition}}
\expandafterautorefname\endcsname{{Proposition}}
}
{\newaliascnt{{Theorem}}{equation}
\newtheorem{{Theorem}}[{Theorem}]{{Theorem}}
\aliascntresetthe{{Theorem}}
\expandafterautorefname\endcsname{{Theorem}}
}
{\newaliascnt{{Corollary}}{equation}
\newtheorem{{Corollary}}[{Corollary}]{{Corollary}}
\aliascntresetthe{{Corollary}}
\expandafterautorefname\endcsname{{Corollary}}
}
{\newaliascnt{{Conjecture}}{equation}
\newtheorem{{Conjecture}}[{Conjecture}]{{Conjecture}}
\aliascntresetthe{{Conjecture}}
\expandafterautorefname\endcsname{{Conjecture}}
}
{\newaliascnt{{Lemma}}{equation}
\newtheorem{{Lemma}}[{Lemma}]{{Lemma}}
\aliascntresetthe{{Lemma}}
\expandafterautorefname\endcsname{{Lemma}}
}
\theoremstyle{definition}
{\newaliascnt{{Definition}}{equation}
\newtheorem{{Definition}}[{Definition}]{{Definition}}
\aliascntresetthe{{Definition}}
\expandafterautorefname\endcsname{{Definition}}
}
\theoremstyle{remark}
{\newaliascnt{{Remark}}{equation}
\newtheorem{{Remark}}[{Remark}]{{Remark}}
\aliascntresetthe{{Remark}}
\expandafterautorefname\endcsname{{Remark}}
}
\newtheorem*{Remark*}{Remark}
{\newaliascnt{{Remarks}}{equation}
\newtheorem{{Remarks}}[{Remarks}]{{Remarks}}
\aliascntresetthe{{Remarks}}
\expandafterautorefname\endcsname{{Remarks}}
}

\newaliascnt{Example}{equation}
\aliascntresetthe{Example}

\newenvironment{Example}{\refstepcounter{Example}\trivlist
\item[\hskip\labelsep\theequation.~\textbf{Example}\space]
\ignorespaces
}{\unskip\nobreak\hfil\penalty50\hskip2em\hbox{}\nobreak\hfil$\Diamond$\parfillskip=0pt\finalhyphendemerits=0\penalty-100\endtrivlist
}
\newaliascnt{Examples}{equation}
\aliascntresetthe{Examples}

\newenvironment{Examples}[1]{\refstepcounter{Examples}\trivlist
\item[\hskip\labelsep\theequation.~\textbf{Examples}\space]
\ignorespaces #1\enumerate
}{\endenumerate\unskip\nobreak\hfil\penalty50\hskip2em\hbox{}\nobreak\hfil$\Diamond$\parfillskip=0pt\finalhyphendemerits=0\penalty-100\endtrivlist
}

\usepackage{xparse,mathtools}

\DeclarePairedDelimiterX{\set}[1]{\{}{\}}{\setargs{#1}}
\NewDocumentCommand{\setargs}{>{\SplitArgument{1}{|}}m}
{\setargsaux#1}
\NewDocumentCommand{\setargsaux}{mm}
{\IfNoValueTF{#2}{#1} {#1\,\delimsize|\,\mathopen{}#2}}

\let\<=\langle
\let\>=\rangle

\usepackage{tikz}
\usetikzlibrary{matrix}

\newcount\tableauRow\newcount\tableauCol

\let\eps=\varepsilon

\let\gedom=\unrhd
\let\notgedom\ntrianglerighteq
\let\gdom=\rhd

\NewDocumentCommand\RAn{D<>{e} O{n}}{\mathscr{R}_{#1}({\mathfrak{A}_{)}}}

\makeindex

\keywords{Alternating groups, alternating Hecke algebras,
Khovanov-Lauda-Rouquier algebras, representation theory}
\subjclass[2000]{20C08, 20D06, 20C30}
\begin{document}
\title{Quiver Hecke algebras for alternating groups}
\author{Clinton Boys}
\address{School of Mathematics and Statistics F07, University of
Sydney, NSW 2006, Australia.}
{\email{\href{mailto:{clinton.boys@sydney.edu.au}}{{clinton.boys@sydney.edu.au}}}}
\author{Andrew Mathas}
{\email{\href{mailto:{andrew.mathas@sydney.edu.au}}{{andrew.mathas@sydney.edu.au}}}}

\begin{abstract}
  The main result of this paper shows that, over large enough fields of
  characteristic different from~$2$, the alternating Hecke algebras are
  ${\mathbb{Z}}$-graded algebras that are isomorphic to fixed-point subalgebras of
  the quiver Hecke algebra of~the symmetric group~${\mathfrak{S}_{n}}. As a special
  case, this shows that the group algebra of the alternating group,
  over large enough fields of characteristic different from~$2$, is a
  ${\mathbb{Z}}$-graded algebra. We give a homogeneous presentation for these
  algebras, compute their graded dimension and show that the blocks of
  these quiver Hecke algebras of the alternating group are graded
  symmetric algebras.
\end{abstract}

\maketitle

\section*{Introduction}

  In a landmark paper, Brundan and Kleshchev \cite{BK:GradedKL}
  constructed an explicit ${\mathbb{Z}}$-grading on the cyclotomic Hecke algebras of
  type~$A$. These algebras include, as special cases, the group algebras
  of the symmetric group and the Iwahori-Hecke algebras of type~$A$. This
  paper extends these results to the group algebras of the alternating
  groups and, more generally, to Mitsuhashi's alternating Hecke
  algebras~\cite{Mitsuhashi:A}.

  Let ${{\mathscr{H}}_\xi({\mathfrak{S}_{n}}} be the Iwahori-Hecke algebra of the symmetric group with
  parameter $\xi\in F^\times$, where $F$ is a field. Then ${{\mathscr{H}}_\xi({\mathfrak{S}_{n}}} is a
  deformation of the group algebra of~${\mathfrak{S}_{n}}. The algebra ${{\mathscr{H}}_\xi({\mathfrak{S}_{n}}} has an
  automorphism $\#$ that can be considered as a $\xi$-deformation of the
  sign automorphism of~$F{\mathfrak{S}_{n}}. The alternating Hecke algebra ${{\mathscr{H}}_\xi({\mathfrak{A}_{n}}}{{\mathscr{H}}_\xi({\mathfrak{S}_{n}}}\#$
  is the fixed-point subalgebra of~${{\mathscr{H}}_\xi({\mathfrak{S}_{n}}} under~$\#$.

  Brundan and Kleshchev showed that ${{\mathscr{H}}_\xi({\mathfrak{S}_{n}}} is a ${\mathbb{Z}}$-graded algebra by
  constructing an explicit family of isomorphisms $\theta:{\mathscr{R}_e({\mathfrak{S}_{n}}}{{\mathscr{H}}_\xi({\mathfrak{S}_{n}}},
  where ${\mathscr{R}_e({\mathfrak{S}_{n}}} is a quiver Hecke algebra
  of~${\mathfrak{S}_{n}}~\cite{BK:GradedKL,BrundanStroppel:KhovanovI,KhovLaud:diagI,Rouq:2KM}.
  Here, $e$ is \textbf{quantum characteristic} of~$\xi$, so $e>0$ is
  minimal such that $1+\xi+\dots+\xi^{e-1}=0$.

  As observed in \cite[(3.14)]{KMR:UniversalSpecht}, the algebra ${\mathscr{R}_e({\mathfrak{S}_{n}}}
  has a homogeneous automorphism ${\mathtt{sgn}}$ that is a graded analogue of the
  sign automorphism of the symmetric group. Let $\RAn={\mathscr{R}_e({\mathfrak{S}_{n}}}{\mathtt{sgn}}$ be the
  fixed-point subalgebra of~${\mathscr{R}_e({\mathfrak{S}_{n}}} under~${\mathtt{sgn}}$. Then $\RAn$ is a
  homogeneous subalgebra of~${\mathscr{R}_e({\mathfrak{S}_{n}}}. It is natural to hope that $\theta$
  restricts to an isomorphism $\RAn{\overset{\simeq}{\longrightarrow}}{{\mathscr{H}}_\xi({\mathfrak{A}_{n}}}.  Unfortunately, the
  isomorphisms constructed by Brundan and Kleshchev do not restrict to
  isomorphisms between the alternating subalgebras; see
  \autoref{Ex:BadBKRestriction}.

  Let $F$ be a field and $\xi\in F$ an element of quantum
  characteristic~$e$. By definition, the field~$F$ is \textbf{large
  enough} for~$\xi$ if~$F$ contains square roots $\sqrt\xi$ and
  $\sqrt{1+\xi+\xi^2}$ whenever $e>3$.

  \begin{MainTheorem}\label{T:Main}
    Suppose that $\xi\in F$ is an element of quantum characteristic~$e\ne2$,
    where~$F$ is a large enough field for~$\xi$ of characteristic
    different from~$2$. Then ${{\mathscr{H}}_\xi({\mathfrak{A}_{n}}}\RAn$.
  \end{MainTheorem}

  To prove this result we construct a new isomorphism ${\mathscr{R}_e({\mathfrak{S}_{n}}}{{\mathscr{H}}_\xi({\mathfrak{S}_{n}}}
  that intertwines the involutions,~${\mathtt{sgn}}$ and~$\#$, on the two algebras. We
  do this using the framework developed by Hu and the second-named
  author~\cite{HuMathas:SeminormalQuiver}, which shows that the KLR grading
  can be described explicitly in terms of seminormal forms.

  As the algebra $\RAn$ is a graded subalgebra of ${\mathscr{R}_e({\mathfrak{S}_{n}}} we immediately
  obtain the following.

  \begin{MainCorollary}\label{C:HAnZGraded}
    Suppose that $\xi\in F$ is an element of quantum
    characteristic~$e\ne2$, where~$F$ is a large enough field
    for~$\xi$ of characteristic different from~$2$. Then ${{\mathscr{H}}_\xi({\mathfrak{A}_{n}}} is a
    ${\mathbb{Z}}$-graded algebra.
  \end{MainCorollary}

  In particular, over large enough fields of characteristic different
  from~$2$ the group algebra $F{\mathfrak{A}_{n}} of the alternating group is
  ${\mathbb{Z}}$-graded, for $n\ge1$. The alternating group corresponds to the
  case when~$\xi=1$, so if~$F$ is a field of characteristic~$p\ne2$ then
  $F{\mathfrak{A}_{n}} \RAn<p>$ if~$3$ has a square root in $F$ whenever $p>3$.

  Applying \autoref{T:Main} twice shows that, up to isomorphism, ${{\mathscr{H}}_\xi({\mathfrak{A}_{n}}}
  depends only on~$e$, the quantum characteristic of~$\xi$, rather than
  on~$\xi$ itself. Hence, the following holds:

  \begin{MainCorollary}
    Let $F$ be a field of characteristic different from~$2$. Suppose
    that $\xi,\xi'\in F$ are elements of quantum characteristic~$e\ne2$,
    where~$F$ is a large enough field for~$\xi$ and for $\xi'$. Then
    ${{\mathscr{H}}_\xi({\mathfrak{A}_{n}}}{\mathscr{H}}_{\xi'}({\mathfrak{A}_{n}}$.
  \end{MainCorollary}

  In particular, over a large enough field, the decomposition matrix
  of~${{\mathscr{H}}_\xi({\mathfrak{A}_{n}}} depends only on~$e$, and the field~$F$, and not on the choice
  of~$\xi$.

  The quiver Hecke algebra ${\mathscr{R}_e({\mathfrak{S}_{n}}} has a homogeneous presentation by
  generators and relations that is described in terms of the quiver
  $\Gamma_e$ with vertex set $I={\mathbb{Z}}/e{\mathbb{Z}}$ and edges $i\to i+1$, for $i\in I$.
  In \autoref{S:KLRAlgebras} we associate to~$\Gamma_e$ a set of
  simple roots
  $\set{\alpha_i|i\in I}$ and a positive root lattice
  $Q^+=\bigoplus_{i\in I}{\mathbb{N}}\alpha_i$. If $\alpha\in Q^+$ let
  $I^\alpha=\set{{\mathbf{i}}\in I^n|\alpha=\alpha_{i_1}+\dots+\alpha_{i_n}}$.
  Let~$\sim$ be the equivalence relation on~$I^n$ generated by ${\mathbf{i}}\sim{\mathbf{j}}$ if ${\mathbf{j}}=-{\mathbf{i}}$.
  This relation induces an equivalence relation on~$Q^+$ where
  $\alpha\sim\beta$ if there exists ${\mathbf{i}}\in I^\alpha$ such that
  $-{\mathbf{i}}\in I^\beta$.
  Let ${Q^\varepsilon_n}=Q^+_n/{\sim}$. If $\gamma\in{Q^\varepsilon_n}$ let
  $I^\gamma=\bigcup_{\alpha\in\gamma}I^\alpha$.

  Using the KLR presentation of~${\mathscr{R}_e({\mathfrak{S}_{n}}}, and the realisation of~$\RAn$ as
  a fixed-point subalgebra of~${\mathscr{R}_e({\mathfrak{S}_{n}}}, gives the following homogeneous
  presentation for~$\RAn$ with respect to the ${\mathbb{Z}}$-grading. For
  ${\mathbf{i}},{\mathbf{j}}\in I^n$ set $\delta_{{\mathbf{i}}\sim{\mathbf{j}}}=1$ if ${\mathbf{i}}\sim{\mathbf{j}}$ and set
  $\delta_{{\mathbf{i}}\sim{\mathbf{j}}}=0$ otherwise.

  \begin{MainTheorem}\label{T:MainRelations}
    Suppose that $e\ne2$ and that $2$ is invertible in~${\mathcal{Z}}$. Then
    \[\RAn=\bigoplus_{\gamma\in {Q^\varepsilon_n}}\RAn_\gamma \]
    where $\RAn_\gamma$ is the unital associative ${\mathcal{Z}}$-algebra generated by
    elements
    \[\set{\Psi_{r}({\mathbf{i}}), Y_{s}({\mathbf{i}}), \eps({\mathbf{i}}) |
    1\le r\le n, 1\le s<n\text{ and }{\mathbf{i}}\in I^\gamma}
    \]
    subject to the relations
    {\setlength{\abovedisplayskip}{2pt}
    \setlength{\belowdisplayskip}{1pt}
    \begin{align*}
      Y_{1}({\mathbf{i}})^{(\Lambda_0,\alpha_{i_1})}&=0,
      &\eps({\mathbf{i}})\eps({\mathbf{j}})&= \delta_{{\mathbf{i}}\sim{\mathbf{j}}}\eps({\mathbf{i}}),
      &\textstyle\sum_{{\mathbf{i}}\in I^\gamma}\frac12\eps({\mathbf{i}})&= 1,\\
      Y_r(-{\mathbf{i}})&=-Y_r({\mathbf{i}})
      &\Psi_r(-{\mathbf{i}})&=-\Psi_r({\mathbf{i}})
      &\eps(-{\mathbf{i}})&=\eps({\mathbf{i}})\\
      \eps({\mathbf{i}})Y_r({\mathbf{j}})\eps({\mathbf{k}})&=\delta_{{\mathbf{i}}{\mathbf{j}}}\delta_{{\mathbf{j}}{\mathbf{k}}} Y_r({\mathbf{j}}),
      &\eps({\mathbf{i}})\Psi_r({\mathbf{j}})\eps({\mathbf{k}})&
      =\delta_{s_r\cdot{\mathbf{i}},{\mathbf{j}}}\delta_{{\mathbf{j}}{\mathbf{k}}}\Psi_r({\mathbf{j}}),
      &Y_r({\mathbf{i}})Y_s({\mathbf{j}})&=\delta_{{\mathbf{i}}{\mathbf{j}}}Y_s({\mathbf{i}})Y_r({\mathbf{j}}),
    \end{align*}
    \begin{align*}
      \Psi_r({\mathbf{i}})Y_{r+1}({\mathbf{i}}) & =\(Y_r(s_r\cdot{\mathbf{i}})\Psi_r({\mathbf{i}})
      +\delta_{i_r i_{r+1}}\eps({\mathbf{i}}){\big)},\\
      Y_{r+1}(s_r\cdot{\mathbf{i}})\Psi_r({\mathbf{i}}) &={\big(}\Psi_r({\mathbf{i}})Y_r({\mathbf{i}})
      +\delta_{i_r i_{r+1}}\eps({\mathbf{i}}){\big)},
    \end{align*}
    \begin{align*}
      \Psi_r({\mathbf{i}})Y_s({\mathbf{i}}) &=Y_s(s_r\cdot{\mathbf{i}})\Psi_r({\mathbf{i}}),
      &&\text{if }s\neq r,r+1,\\
      \Psi_r(s_t\cdot{\mathbf{i}})\Psi_t({\mathbf{i}})
      &=\Psi_t(s_r\cdot{\mathbf{i}})\Psi_r({\mathbf{i}}),&&\text{if }|r-s|>1,
    \end{align*}
    \begin{align*}
      \Psi_r(s_r\cdot{\mathbf{i}})\Psi_r({\mathbf{i}})&= \begin{cases}
        Y_r({\mathbf{i}})-Y_{r+1}({\mathbf{i}}),&\text{if }i_r\to i_{r+1},\\
        Y_{r+1}({\mathbf{i}})-Y_r({\mathbf{i}}),&\text{if }i_r\leftarrow i_{r+1},\\
        0,&\text{if }i_r=i_{r+1},\\
        \eps({\mathbf{i}}),&\text{otherwise}
      \end{cases}
    \end{align*}
    and $\Psi_r(s_{r+1}s_r\cdot{\mathbf{i}})\Psi_{r+1}(s_r\cdot{\mathbf{i}})\Psi_r({\mathbf{i}})
    -\Psi_{r+1}(s_rs_{r+1}\cdot{\mathbf{i}})\Psi_r(s_{r+1}\cdot{\mathbf{i}})\Psi_{r+1}({\mathbf{i}})$
    is equal to
    \begin{align*}
      \begin{cases}
        \eps({\mathbf{i}}), &\text{if }i_r=i_{r+2}\leftarrow i_{r+1},\\
        -\eps({\mathbf{i}}), &\text{if }i_r=i_{r+2}\to i_{r+1},\\
        0,&\text{otherwise,}
      \end{cases}
    \end{align*}
    }    for all ${\mathbf{i}},{\mathbf{j}},{\mathbf{k}}\in I^\gamma$ and all admissible $r,s$ and $t$.
  \end{MainTheorem}

  The fraction~$\frac12$ appears in the relation $\sum_{{\mathbf{i}}\in
  I^\gamma}\frac12\eps({\mathbf{i}})=1$ because $\eps({\mathbf{i}})=\eps(-{\mathbf{i}})$.
  (It follows from the relations in \autoref{T:MainRelations} that
  $\eps({\mathbf{i}})=0$ if ${\mathbf{i}}=-{\mathbf{i}}$, for ${\mathbf{i}}\in I^n$.)

  The relations in \autoref{T:MainRelations} are homogeneous with respect
  to the degree function
  \[
  \deg \eps({\mathbf{i}})=0, \quad \deg Y_r({\mathbf{i}})=2\quad\text{and}\quad
  \deg\Psi_r({\mathbf{i}})=-(\alpha_{i_r},\alpha_{i_{r+1}}),
  \]
  where $(\ ,\ )$ is the Cartan pairing. Hence, $\RAn$ is a ${\mathbb{Z}}$-graded
  algebra. Quite surprisingly, this presentation is almost identical to
  the KLR-presentation of~${\mathscr{R}_e({\mathfrak{S}_{n}}}. The main difference being the three
  ``sign relations'' relating the generators indexed by ${\mathbf{i}}$ and $-{\mathbf{i}}$, for
  ${\mathbf{i}}\in I^\gamma$. The key idea behind the proof
  \autoref{T:MainRelations} is to introduce a $({\mathbb{Z}}_2\times{\mathbb{Z}})$-grading on
  the KLR algebra~${\mathscr{R}_e({\mathfrak{S}_{n}}}. With respect to the $({\mathbb{Z}}_2\times{\mathbb{Z}})$-grading,
  the algebra $\RAn$ is the even part of~${\mathscr{R}_e({\mathfrak{S}_{n}}}. Using this observation
  we can deduce the relations for~$\RAn$ directly from those for~${\mathscr{R}_e({\mathfrak{S}_{n}}}.

  Finally, using the graded cellular bases of Hu and
  Mathas~\cite{HuMathas:GradedCellular} we construct a homogeneous basis
  for the ${\mathbb{Z}}$-graded algebra~$\RAn$. As a corollary we obtain the
  graded dimension of~$\RAn$. See \autoref{S:Basis} below for the
  unexplained notation.

  \begin{MainTheorem}\label{T:GDim}
    Suppose that $e\ne2$ and that $2$ is invertible in~$F$. Then the
    alternating quiver Hecke algebra $\RAn$ has graded dimension
    \[
    \sum_{\substack{({\mathsf{s}},{\mathsf{t}})\in{\mathop{\rm Std}\nolimits}^2({\mathcal{P}_{n}}\\\operatorname{res}({\mathsf{s}})\in I^n_+}}
    q^{\deg{\mathsf{s}}+\deg{\mathsf{t}}}.
    \]
  \end{MainTheorem}

  In fact, using the work of Li~\cite{GeLi:IntegralKLR} it follows that
  over any ring in which $2$ is invertible the algebra $\RAn$ is free
  with the same graded rank. As a second application of our basis
  theorem we show that the blocks of~$\RAn$ are graded symmetric
  algebras.

  The results in this paper exclude the cases when~$F$ is a field of
  characteristic~$2$ and when $e=2$ or, equivalently, $\xi=-1$.  This is
  because most our arguments fail, and most of our results are false,
  when we drop these assumptions.

  The paper is organised as follows. \autoref{S:HeckeAlgebras} starts by
  defining the quiver Hecke algebra~$\RAn$ of the alternating group as
  the fixed-point subalgebra of~${\mathscr{R}_e({\mathfrak{S}_{n}}} under the homogeneous sign
  involution~${\mathtt{sgn}}$. We then prove \autoref{T:MainRelations} by first
  introducing a $({\mathbb{Z}}_2\times{\mathbb{Z}})$-grading on~${\mathscr{R}_e({\mathfrak{S}_{n}}} and showing
  that~$\RAn$ is the \textit{even} part of~${\mathscr{R}_e({\mathfrak{S}_{n}}}, with respect to the
  ${\mathbb{Z}}_2$-grading.  \autoref{S:SeminormalForm} starts by setting up the
  framework of \textit{seminormal coefficient systems} and showing how
  seminormal bases behave under the ungraded sign involution~$\#$.
  Building on ideas from~\cite{HuMathas:SeminormalQuiver}, we give a new
  presentation of~${{\mathscr{H}}^{\mathcal{O}}_{t}}, over a specially chosen ring~${\mathcal{O}}$, that we use
  to construct a new isomorphism $\Theta:{\mathscr{R}_e({\mathfrak{S}_{n}}}{{\mathscr{H}}_\xi({\mathfrak{S}_{n}}} over the
  residue field of~${\mathcal{O}}$. Unlike the known isomorphisms in the
  literature, $\Theta$ intertwines the two sign
  involutions,~${\mathtt{sgn}}$ and~$\#$, implying \autoref{T:Main}. In
  \autoref{S:Basis} we give a homogeneous basis of~$\RAn$ and hence
  prove \autoref{T:GDim}. As an application we show that the blocks
  of~$\RAn$ are graded symmetric algebras. Finally, using Clifford
  theory, the classification of the blocks and irreducible graded
  modules of~$\RAn$ is given.

  \subsection*{Acknowledgments}
  The first-named author was supported by an Australian Postgraduate Award
  and the second-named author by the Australian Research Council. Some of
  the work in the first-named author's PhD thesis \cite{Boys:PhDThesis}
  was based on an earlier version of this paper.

\section{Iwahori-Hecke algebras and quiver Hecke
algebras}\label{S:HeckeAlgebras}
This chapter defines both the alternating Hecke algebras and the
alternating quiver Hecke algebras of type~$A$. Both algebras are defined as fixed
point subalgebras of the corresponding Hecke algebras. In the
final section we prove that \autoref{T:MainRelations} gives a homogeneous
presentation for the alternating quiver Hecke algebra.

\subsection{Iwahori-Hecke algebras and alternating Hecke algebras}
We start by defining the Iwahori-Hecke algebras of the symmetric groups.
These algebras are well-studied deformations of the group algebras of
the symmetric groups that arise naturally in the representation theory
of the general linear groups.

Fix a (unital) integral domain ${\mathcal{Z}}$ and an invertible element $\xi\in {\mathcal{Z}} ^\times$.

\begin{Definition}\label{D:Hecke}
  The \textbf{Iwahori-Hecke algebra} ${{\mathscr{H}}_\xi({\mathfrak{S}_{n}}}{\mathscr{H}}_\xi^{\mathcal{Z}}({\mathfrak{S}_{n}}$ is the (unital)
  associative ${\mathcal{Z}} $-algebra with generators $T_1,\ldots,T_{n-1}$ subject to
  relations
  \begin{align*}
    (T_r-\xi)(T_r+1)&= 0,&\quad\text{for }r=1,\ldots,n-1,\\
    T_rT_s&= T_sT_r,&\quad\text{if }|r-s|>1,\\
    T_rT_{r+1}T_r&= T_{r+1}T_rT_{r+1},&\quad\text{for }r=1,\ldots,n-2.
  \end{align*}
\end{Definition}

For $1\le r<n$ let $s_r=(r,r+1)\in{\mathfrak{S}_{n}}. Then $\{s_1,\dots,s_{n-1}\}$ is
the standard set of Coxeter generators for~${\mathfrak{S}_{n}}.  If $w\in{\mathfrak{S}_{n}} then the
\textbf{length} of $w$ is the integer
$\ell(w)=\min\set{l\ge0|w=s_{r_1}\dots s_{r_l}\text{ with }1\le r_j<n}$.  A
\textbf{reduced expression} for~$w$ is any word $w=s_{r_1}\dots
s_{r_\ell}$ with $\ell=\ell(w)$ and $1\le r_j<n$, for $1\le j\le \ell$.

If $w\in{\mathfrak{S}_{n}} define $T_w=T_{r_1}\dots T_{r_\ell}$, where
$w=s_{r_1}\dots s_{r_\ell}$ is any reduced expression.  As is
well-known, because the braid relations hold in~${{\mathscr{H}}_\xi({\mathfrak{S}_{n}}} the element~$T_w$
depends only on~$w$ and not on the choice of reduced expression.
Moreover, the algebra ${{\mathscr{H}}_\xi({\mathfrak{S}_{n}}} is free as a ${\mathcal{Z}}$-module with basis
$\set{T_w|w\in{\mathfrak{S}_{n}}$. See, for example, \cite[Chapter~1]{M:ULect}.
In particular, if $\xi=1$ then ${{\mathscr{H}}_\xi({\mathfrak{S}_{n}}}{\mathcal{Z}}{\mathfrak{S}_{n}} via the map
$T_w\mapsto w$, for $w\in{\mathfrak{S}_{n}}.

Following Goldman~\cite[Theorem~5.4]{Iwahori:Hecke},
let  $\#{\,{:}\,{{\mathscr{H}}_\xi({\mathfrak{S}_{n}}}\longrightarrow\!{{\mathscr{H}}_\xi({\mathfrak{S}_{n}}}$ be the
unique ${\mathcal{Z}} $-linear automorphism of ${\mathscr{H}}_\xi({\mathfrak{S}_{n}}$ such that
\begin{equation}\label{E:hash}
  T_r^\#=-\xi T_r^{-1}=-T_r+(\xi-1),
\end{equation}
for $1\le r<n$.  It follows directly from the definitions that when
$\xi=1$ the automorphism $\#$ of ${\mathscr{H}}^{\mathcal{Z}}_1({\mathfrak{S}_{n}}\cong {\mathcal{Z}}
{\mathfrak{S}_{n}} is the usual ``sign involution'' which sends each simple
transposition $s_r$ to  $-s_r$, for $1\le r<n$~\cite[p5]{James}.  Since
the group algebra of the alternating group is the fixed-point subalgebra
of the sign automorphism, the following definition gives a $\xi$-analogue
of the group ring ${\mathcal{Z}}{\mathfrak{A}_{n}} of the alternating group~${\mathfrak{A}_{n}}.

\begin{Definition}[Mitsuhashi~\cite{Mitsuhashi:A}]\label{D:AltHecke}
  Suppose that $\xi\ne-1$. Then the \textbf{ alternating Hecke algebra} is the
  fixed-point subalgebra
  \[
  {{\mathscr{H}}_\xi({\mathfrak{A}_{n}}}{\mathscr{H}}_\xi^{\mathcal{Z}}({\mathfrak{A}_{n}}=\set{h\in{{\mathscr{H}}_\xi({\mathfrak{S}_{n}}} h^\#=h}
  \]
  of~${{\mathscr{H}}_\xi({\mathfrak{S}_{n}}} under the hash involution.
\end{Definition}

\begin{Remark}
  Mitsuhashi's~\cite[Definition 4.1]{Mitsuhashi:A} original definition of
  the alternating Hecke algebra was by generators and relations, giving a
  deformation of a well-known presentation of the alternating group.
  \autoref{D:AltHecke} is equivalent to Mitsuhashi's definition by
  \cite[Proposition~1.5]{MathasRatliff}.
\end{Remark}

\subsection{Graded modules and algebras}
The main result of this paper shows that the alternating Hecke algebra
is a graded algebra, so we quickly review this terminology. For the
most part we will work with ${\mathbb{Z}}$-graded modules and algebras, however,
to prove \autoref{T:MainRelations} we consider more general
gradings.

Recall that ${\mathcal{Z}}$ is a unital integral domain. In this paper all
modules will be assumed to be free and of finite rank as
${\mathcal{Z}}$-modules.

Let $(G,+)$ be an abelian group. A \textbf{$G$-graded ${\mathcal{Z}}$-module} is
a ${\mathcal{Z}}$-module $M$ that admits a vector space decomposition
$M=\bigoplus_{g\in G} M_g$. If $g\in G$ and $0\ne m\in M_g$ then $m$ is
\textbf{homogeneous of degree}~$g$.  Similarly, a
\textbf{$G$-graded ${\mathcal{Z}}$-algebra} is a $G$-graded ${\mathcal{Z}}$-module
$A=\bigoplus_{g\in G}A_g$ that is a ${\mathcal{Z}}$-algebra such that
$A_fA_g\subseteq A_{f+g}$, for all $f,g\in G$. A  \textbf{$G$-graded $A$-module} is
a $G$-graded ${\mathcal{Z}}$-module $M$ such that $M_fA_g\subset M_{f+g}$, for
$f,g\in G$.

Unless otherwise specified, $G={\mathbb{Z}}$ and a \textbf{graded module} will mean a
${\mathbb{Z}}$-graded module. Similarly, a \textbf{graded algebra} is a ${\mathbb{Z}}$-graded
algebra.

\subsection{Cyclotomic quiver Hecke algebras of type~$A$}\label{S:KLRAlgebras}
We now define the second class of algebras that we are interested in:
the cyclotomic quiver Hecke algebras of~${\mathfrak{S}_{n}}. These algebras are
certain quotients of the ${\mathbb{Z}}$-graded quiver Hecke algebras introduced,
independently, by Khovanov and Lauda~\cite{KhovLaud:diagI} and
Rouquier~\cite{Rouq:2KM}.

Fix  $e\in\set{3,4,5,\ldots}\cup\set{\infty}$ and define $\Gamma_e$ to
be the quiver with vertex set $I={\mathbb{Z}}/e{\mathbb{Z}}$ and edges $i\to i+1$, for $i\in
I$. (By convention, $I={\mathbb{Z}}$ if $e=\infty$.) Thus, $\Gamma_e$ is the
infinite quiver of type $A_\infty$ if $e=\infty$ and the finite quiver
of Dynkin type $A_{e-1}^{(1)}$ if $e\geq 3$.  We exclude $e=2$ only because
this corresponds to the case $\xi=-1$ in \autoref{D:AltHecke}, which we
do not consider in this paper.

Following Kac~\cite{Kac}, to the quiver $\Gamma_e$ we attach the usual
Lie theoretic data of the positive roots $\set{\alpha_i|i\in I}$, the
fundamental weights $\set{\Lambda_i|i\in I}$,
the positive weight lattice $P^+=\bigoplus_{i\in I}{\mathbb{N}}\Lambda_i$,
the positive root lattice $Q^+=\bigoplus_{i\in I}{\mathbb{N}}\alpha_i$,
the non-degenerate
pairing $(\  ,\ ){\,{:}\,{P^+\times Q^+}\!\longrightarrow\!{\mathbb{Z}}}$ given by
$(\Lambda_i,\alpha_j)=\delta_{ij}$, for $i,j\in I$, and the
\textbf{Cartan matrix} $C=(c_{ij})_{i,j\in I}$ where
\[
c_{ij}=\begin{cases}
  2,&\text{if }i=j,\\
  -1,&\text{if $i\leftarrow j$ or $i\rightarrow j$},\\
  0,&\text{otherwise.}
\end{cases}
\]
The \textbf{height} of $\alpha\in Q^+$ is the
non-negative integer $\operatorname{ht}\alpha=\sum_i(\Lambda_i,\alpha)$. Fix $n\ge0$ and let
$Q^+_n=\set{\alpha\in Q^+|\operatorname{ht}\alpha=n}$. For $\alpha\in Q^+_n$, let
\[
  I^\alpha=\set{{\mathbf{i}}=(i_1,\dots,i_n)\in I^n|\alpha=\alpha_{i_1}+\dots+\alpha_{i_n}}.
\]

\begin{Definition}[Khovanov and Lauda \cite{KhovLaud:diagI}
  and Rouquier~\cite{Rouq:2KM}]
  \label{D:klr}
  Suppose that  $\alpha\in Q^+$ and
  $e\in\set{3,4,5,\ldots}\cup\set{\infty}$. The \textbf{cyclotomic quiver
  Hecke algebra} ${{\mathscr{R}_e({\mathfrak{S}_{n}}}{\alpha}} is the unital associative
  ${\mathcal{Z}}$-algebra  with generators
  \begin{equation*}
    \set{\psi_1,\ldots,\psi_{n-1}}\cup \set{y_1,\ldots,y_n}
    \cup \set{e({{\mathbf{i}}})| {{\mathbf{i}}}\in I^\alpha}
  \end{equation*}
  and relations
  {\setlength{\abovedisplayskip}{2pt}
  \setlength{\belowdisplayskip}{1pt}
  \begin{xalignat*}{3}
    y_1^{(\Lambda_0,\alpha_{i_1})}e({\mathbf{i}})&= 0,& e({{\mathbf{i}}})e({{\mathbf{j}}})&= \delta_{{{\mathbf{i}}}{{\mathbf{j}}}}e({{\mathbf{i}}}),
    &\textstyle\sum_{{{\mathbf{i}}}\in I^\alpha}e({{\mathbf{i}}})&= 1,\\%\label{E:top3}
    y_re({{\mathbf{i}}})&= e({{\mathbf{i}}})y_r,& \psi_re({{\mathbf{i}}})&= e(s_r\cdot{{\mathbf{i}}})\psi_r,& y_ry_s&= y_sy_r,
  \end{xalignat*}
  \begin{xalignat*}{2}
    \psi_ry_{r+1}e({{\mathbf{i}}})&= (y_r\psi_r+\delta_{i_r i_{r+1}})e({{\mathbf{i}}}),
    & y_{r+1}\psi_re({{\mathbf{i}}})&= (\psi_ry_r+\delta_{i_r i_{r+1}})e({{\mathbf{i}}}),
  \end{xalignat*}
  \begin{align*}
    \psi_ry_s&= y_s\psi_r,&\text{if }s\neq r,r+1,\\
    \psi_r\psi_s&= \psi_s\psi_r,&\text{if }|r-s|>1,
  \end{align*}
  \begin{align*}
    \psi_r^2e({{\mathbf{i}}})&= \begin{cases}
      0,&\text{if }i_r=i_{r+1},\\
      (y_r-y_{r+1})e({{\mathbf{i}}}),&\text{if }i_r\to i_{r+1},\\
      (y_{r+1}-y_r)e({{\mathbf{i}}}),&\text{if }i_r\leftarrow i_{r+1},\\
      e({{\mathbf{i}}}),&\text{otherwise},
    \end{cases}\\
    \psi_r\psi_{r+1}\psi_re({{\mathbf{i}}})&=\begin{cases}
      (\psi_{r+1}\psi_r\psi_{r+1}-1)e({{\mathbf{i}}}),&\text{if }i_r=i_{r+2}\to i_{r+1},\\
      (\psi_{r+1}\psi_r\psi_{r+1}+1)e({{\mathbf{i}}}),
      &\text{if }i_r=i_{r+2}\leftarrow i_{r+1},\\
      \psi_{r+1}\psi_r\psi_{r+1}e({{\mathbf{i}}}),&\text{otherwise,}
    \end{cases}
  \end{align*}
  }  for ${{\mathbf{i}}},{{\mathbf{j}}}\in I^\alpha$ and all admissible $r$ and $s$. If $n\ge0$
  then the \textbf{quiver Hecke algebra of~${\mathfrak{S}_{n}}}
  is the algebra
  \[
  {\mathscr{R}_e({\mathfrak{S}_{n}}}\bigoplus_{\alpha\in Q^+_n}{{\mathscr{R}_e({\mathfrak{S}_{n}}}{\alpha}}
  \]
\end{Definition}

Note that the algebra ${{\mathscr{R}_e({\mathfrak{S}_{n}}}{\alpha}} depends on $e$, $\Gamma_e$, $\Lambda_0$
and $\alpha\in Q^+$.

We write ${\mathscr{R}_e({\mathfrak{S}_{n}}}{\mathscr{R}^{{\mathcal{Z}}}_e({\mathfrak{S}_{n}}} when we want to emphasise that ${\mathscr{R}_e({\mathfrak{S}_{n}}} is a
${\mathcal{Z}}$-algebra. The main advantage of the relations in \autoref{D:klr} is
that they are homogeneous with respect
to the following ${\mathbb{Z}}$-valued degree function:
\begin{align}
  \deg e({{\mathbf{i}}})&= 0, &\text{for all }{{\mathbf{i}}}\in I^n,\nonumber\\
  \deg y_r&= 2,&\text{for }1\le r\le n,\label{degfunc}\\
  \deg \psi_re({{\mathbf{i}}})&= -c_{i_r,i_{r+1}},&\text{for $1\le r<n$ and ${{\mathbf{i}}}\in I^n$}.
  \nonumber
\end{align}
Therefore, ${\mathscr{R}_e({\mathfrak{S}_{n}}} is a ${\mathbb{Z}}$-graded algebra.

\begin{Remark}
  There are fewer relations appearing in \autoref{D:klr} than in
  \cite[Theorem~1.1]{BK:GradedKL}. This is because we are assuming that
  $e\ne 2$ (and $\Lambda=\Lambda_0$). We have also made a sign change
  compared with \cite{BK:GradedKL}, which is consistent with
  \cite{HuMathas:SeminormalQuiver}.
\end{Remark}

In examples, we write $e({{\mathbf{i}}})=e(i_1i_2\dots i_n)$ if
${{\mathbf{i}}}=(i_1,i_2,\dots,i_n)$.

\begin{Example}\label{Ex:OS3}
  Let $n=3$, $e=3$ and $\Lambda=\Lambda_0$. First, $y_1=0$ because of the
  cyclotomic relations $y_1^{(\Lambda_0,\alpha_{i_1})}e({{\mathbf{i}}})=0$. It is not
  difficult to see that $\psi_1=0=y_2$ and
  that $e({{\mathbf{i}}})=0$ unless ${{\mathbf{i}}}=(012)$ or $(021)$; see, for example
  \cite[Proposition~2.4.6]{Mathas:Singapore}.
  Hence, ${\mathscr{R}_e({\mathfrak{S}_{3}}} is generated by $\psi_2,y_3,e(012)$ and $e(021)$. Using
  the quadratic relation,
  \[
  \psi_2^2e({{\mathbf{i}}})=\begin{cases}
    -y_3e({{\mathbf{i}}}),&\text{if }{{\mathbf{i}}}=(012),\\
    \phantom{-}y_3e({{\mathbf{i}}}),&\text{if }{{\mathbf{i}}}=(021).
  \end{cases}
  \]
  In turn, this implies that $y_3^2e({{\mathbf{i}}})=\pm y_3\psi_2e({{\mathbf{i}}})=\pm \psi_2y_2e({{\mathbf{i}}})=0$,
  so $y_3^2=0$. Therefore, ${\mathscr{R}_e({\mathfrak{S}_{3}}} is spanned by
  \[
  \set{e(012), e(021), \psi_2e(012), \psi_2e(021), y_3e(012), y_3e(021)}.
  \]
  By \autoref{T:BKiso} below, these elements are a basis of~${\mathscr{R}_e({\mathfrak{S}_{n}}}.
\end{Example}

To connect the algebras ${\mathscr{R}_e({\mathfrak{S}_{n}}} and ${{\mathscr{H}}_\xi({\mathfrak{S}_{n}}} define the \textbf{quantum
characteristic} of $\xi$ to be the smallest non-negative integer $e$
such that $1+\xi+\dots+\xi^{e-1}=0,$ and set $e=\infty$ if no such
integer exists.  By definition,
$e\in\set{2,3,4,5,6,\dots}\cup\set{\infty}$ and $e=2$ if and only if
$\xi=-1$.

Brundan and Kleshchev proved the following remarkable theorem, which
connects the two algebras ${\mathscr{R}_e({\mathfrak{S}_{n}}} and ${{\mathscr{H}}_\xi({\mathfrak{S}_{n}}}.

\begin{Theorem}[\protect{Brundan and Kleshchev~\cite{BK:GradedKL},
  Rouquier~\cite[Corollary~3.20]{Rouq:2KM}}]\label{T:BKiso}
  \leavevmode\newline
  Suppose that $F$ is a field and that $\xi\ne-1$ has quantum
  characteristic~$e>2$. Then ${\mathscr{R}^{F}_e({\mathfrak{S}_{n}}}{{\mathscr{H}}_\xi^{F}({\mathfrak{S}_{n}}}.
\end{Theorem}

Hence, if $F$ is a field then we can consider ${{\mathscr{H}}_\xi^{F}({\mathfrak{S}_{n}}} as a ${\mathbb{Z}}$-graded
algebra via the isomorphism ${{\mathscr{H}}_\xi^{F}({\mathfrak{S}_{n}}}{\mathscr{R}^{F}_e({\mathfrak{S}_{n}}}. We have stated a special
case of Brundan and Kleshchev's result because this is all that we need.
Over a field, Brundan and Kleshchev prove more generally that the cyclotomic Hecke
algebras of type $A$ are isomorphic to cyclotomic quiver Hecke algebras
--- and they also allow $e=2$. To prove \autoref{T:BKiso}
Brundan and Kleshchev construct an explicit isomorphism (in fact, they
construct different isomorphisms for the cases when $\xi=1$ and
$\xi\ne1$). Our proof of \autoref{T:MainRelations} builds from a
variation on their ideas, following~\cite{HuMathas:SeminormalQuiver}.

\subsection{Alternating quiver Hecke algebras of type~$A$}
In this section we introduce a homogeneous analogue of the
$\#$-involution of~${{\mathscr{H}}_\xi({\mathfrak{S}_{n}}} and use it to define the alternating quiver
Hecke algebras of type~$A$.

If ${\mathbf{i}}=(i_1,\dots,i_n)\in I^n$ let $-{\mathbf{i}}=(-i_1,\dots,-i_n)\in I^n$.
Following~\cite[(3.14)]{KMR:UniversalSpecht}, define ${\mathtt{sgn}}$ to be the unique
automorphism of ${\mathscr{R}_e({\mathfrak{S}_{n}}} such that
\[
\psi_r^{\mathtt{sgn}}= -\psi_r,\quad y_s^{\mathtt{sgn}}= -y_s,\quad\text{and}\quad e({{\mathbf{i}}})^{\mathtt{sgn}}= e(-{{\mathbf{i}}}),
\]
for $1\le r<n$, $1\le s\le n$ and ${\mathbf{i}}\in I^n$.

If $\alpha\in Q^+_n$ let $\alpha'$ be the unique element of~$Q^+_n$
such that $(\Lambda_i,\alpha)=(\Lambda_{-i},\alpha')$, for all $i\in I$.
Recall from the introduction that $\sim$ is the equivalence relation
on~$I^n$ generated by ${\mathbf{i}}\sim{\mathbf{j}}$ if ${\mathbf{j}}=-{\mathbf{i}}$ and that if
$\alpha,\beta\in Q^+$ then $\alpha\sim\beta$ if there exists ${\mathbf{i}}\in
I^\alpha$ such that $-{\mathbf{i}}\in I^\beta$. Hence, $\alpha\sim\beta$ if and
only if $\beta\in\set{\alpha,\alpha'}$.

Checking the relations in \autoref{D:klr} reveals the following.

\begin{Proposition}[\protect{\cite[(3.14)]{KMR:UniversalSpecht}}]
  \label{P:SgnHomogeneous}
  The map ${\mathtt{sgn}}$ restricts to a homogeneous isomorphism of ${\mathbb{Z}}$-graded
  algebras
  ${{\mathscr{R}_e({\mathfrak{S}_{n}}}{\alpha}}{{\mathscr{R}_e({\mathfrak{S}_{n}}}{\alpha'}}, for $\alpha\in Q^+_n$.
  Hence, ${\mathtt{sgn}}$ is a homogeneous automorphism of~${\mathscr{R}_e({\mathfrak{S}_{n}}} of order~$2$.
\end{Proposition}

Mirroring \autoref{D:AltHecke}, we define the second algebra
appearing in \autoref{T:Main}.

\begin{Definition}\label{D:sgndef}
  The \textbf{alternating quiver Hecke algebra of~${\mathfrak{A}_{n}}} is the fixed-point
  subalgebra $\RAn=\mathscr{R}_e^{\mathcal{Z}}({\mathfrak{A}_{n}}=\set{a\in{\mathscr{R}_e({\mathfrak{S}_{n}}}a^{\mathtt{sgn}}=a}$ of~${\mathscr{R}_e({\mathfrak{S}_{n}}}
  under the involution~${\mathtt{sgn}}$.
\end{Definition}

Since ${\mathtt{sgn}}$ is a homogeneous involution of~${\mathscr{R}_e({\mathfrak{S}_{n}}}, an immediate and important
consequence of \autoref{D:sgndef} is the following.

\begin{Corollary}\label{P:GradedSubalgebra}
  The alternating quiver Hecke algebra $\RAn$ is a ${\mathbb{Z}}$-graded subalgebra
  of~${\mathscr{R}_e({\mathfrak{S}_{n}}}.
\end{Corollary}

We finish this section with an example of how our main result,
\autoref{T:Main}, works when $n=3=e$.

\begin{Example}\label{Ex:A3Basis}
  Suppose that $e=3$ and $n=3$. Then ${\mathfrak{A}_{3}}{\mathbb{Z}}/3{\mathbb{Z}}$ is the cyclic group of
  order~$3$. By \autoref{Ex:OS3}, $\mathscr{R}_3({\mathfrak{A}_{3}}$ is spanned
  by the three elements
  \[
  1=e(012)+e(021),\quad \Psi=\psi_2(e(012)-e(021)),\quad Y=y_3(e(012)-e(021)).
  \]
  These three elements are homogeneous, with $\deg\Psi=1$ and $\deg
  Y=2$, and \autoref{T:BKiso} implies that they are non-zero. Therefore,
  $\{1,\Psi, Y\}$ is a basis of~$\mathscr{R}_3({\mathfrak{A}_{3}}$.  Using the
  relations in \autoref{D:klr}, $\Psi^2=-Y$ and $\Psi^3=-Y\Psi=0$.
  Therefore, the map $\Psi\mapsto x$ determines an isomorphism of graded
  algebras
  \[\RAn[3]=\<\Psi\mid \Psi^3=0\>\cong{\mathbb{Z}}[x]/(x^3),\]
  where we put $\deg x=1$. It is now easy to see that
  $\mathscr{R}_3({\mathfrak{A}_{3}}\otimes {\mathbb{F}}_3\cong{\mathbb{F}}_3{\mathfrak{A}_{3}}, where
  ${\mathbb{F}}_3={\mathbb{Z}}/3{\mathbb{Z}}$. For example, an isomorphism is determined
  by~$\Psi\mapsto 1-s_1s_2$.

  The isomorphism above is not unique. In the special case when $n=3$
  and $\xi=1\in{\mathbb{F}}_3$, the proof of \autoref{T:Main} in \autoref{S:Main}
  constructs a different isomorphism
  ${\mathbb{F}}_3{\mathfrak{A}_{3}}{{\mathscr{H}}_\xi({\mathfrak{A}_{3}}}\RAn[3]$ that is determined by
  $\Psi\mapsto s_2s_1-s_1s_2$.
\end{Example}

The algebra ${\mathscr{R}_e({\mathfrak{S}_{n}}} is defined in terms of the subalgebras ${{\mathscr{R}_e({\mathfrak{S}_{n}}}{\alpha}}. To
give a presentation for~$\RAn$ we need to work with the blocks of these
algebras. As in the introduction, set ${Q^\varepsilon_n}=Q^+_n/{\sim}$. Using the
notation introduced before \autoref{P:SgnHomogeneous}, if $\alpha\in
Q^+_n$ then $\set{\alpha,\alpha'}$ is its $\sim$-equivalence class. If
$\gamma\in{Q^\varepsilon_n}$ set~${{\mathscr{R}_e({\mathfrak{S}_{n}}}{\gamma}}\bigoplus_{\alpha\in\gamma}{{\mathscr{R}_e({\mathfrak{S}_{n}}}{\alpha}}.
By \autoref{P:SgnHomogeneous}, ${\mathtt{sgn}}$ induces a homogeneous automorphism
of~${{\mathscr{R}_e({\mathfrak{S}_{n}}}{\gamma}}. Define
\begin{equation}\label{E:RAngamma}
  \RAn_\gamma=({{\mathscr{R}_e({\mathfrak{S}_{n}}}{\gamma}}^{\mathtt{sgn}}=\set{a\in{{\mathscr{R}_e({\mathfrak{S}_{n}}}{\gamma}}a=a^{\mathtt{sgn}}}
\end{equation}
to be the fixed-point subalgebra of ${{\mathscr{R}_e({\mathfrak{S}_{n}}}{\gamma}} under the
${\mathtt{sgn}}$ automorphism. Since ${\mathscr{R}_e({\mathfrak{S}_{n}}}\bigoplus_\alpha{{\mathscr{R}_e({\mathfrak{S}_{n}}}{\alpha}} we have the
following decomposition of~$\RAn$ as a direct sum of two-sided
${\mathbb{Z}}$-graded ideals.

\begin{Corollary}\label{C:RAnBlocks}
  As a graded algebra,
  $\displaystyle\RAn=\bigoplus_{\gamma\in{Q^\varepsilon_n}}\RAn_\gamma$.
\end{Corollary}

\subsection{A presentation for alternating quiver Hecke algebras of ${\mathfrak{A}_{n}}}
\label{S:GeneratorsRelations}

In this section we prove \autoref{T:MainRelations}. To do this we  first give
a ``super'' presentation for the quiver Hecke algebra~${\mathscr{R}_e({\mathfrak{S}_{n}}}.
For convenience, we identify the group ${\mathbb{Z}}_2={\mathbb{Z}}/2{\mathbb{Z}}$ with $\{0,1\}$ in
the obvious way.

We start by defining a new algebra ${\mathscr{R}^\varepsilon_{n}} that, it turns out, is
isomorphic to~${\mathscr{R}_e({\mathfrak{S}_{n}}}. The advantage of ${\mathscr{R}^\varepsilon_{n}} is that it is
$({\mathbb{Z}}_2\times{\mathbb{Z}})$-graded where the ${\mathbb{Z}}_2$-grading encodes the effects of~${\mathtt{sgn}}$.
Abusing notation, we use similar notation for the generators of~${\mathscr{R}_e({\mathfrak{S}_{n}}}
and~${\mathscr{R}^\varepsilon_{n}}. This is justified by \autoref{P:RpSnIsomorphism} below.

The sequence ${\mathbf{i}}=(0,\dots,0)\in I^n$, which is the unique sequence such that
${\mathbf{i}}=-{\mathbf{i}}$, is potentially problematic for us. The next result
resolves this.

\begin{Lemma}\label{L:ZeroSequence}
  Suppose that ${\mathbf{i}}\in I^n$ and $e({\mathbf{i}})\ne0$. Then $i_1=0$ and
  $i_2=\pm1$. In particular, $e(0,\dots,0)=0$.
\end{Lemma}

\begin{proof}By \cite[Lemma 4.1c]{HuMathas:SeminormalQuiver},
  $e({\mathbf{i}})\ne0$ if and only ${\mathbf{i}}$ is the residue sequence of some
  standard tableau (see \autoref{S:tableaux}), which readily implies
  the result. As we need this argument later, we give a direct proof
  following \cite[Proposition~2.4.6]{Mathas:Singapore}.  First, $y_1=0$ and
  $e({\mathbf{i}})\ne0$ only if $i_1=0$ by the cyclotomic relation
  $y_1^{(\Lambda_0,\alpha_{i_1})}e({\mathbf{i}})=0$. If $i_1=i_2=0$ then
  $e({\mathbf{i}})=(y_2\psi_1-\psi_1y_1)e({\mathbf{i}})=y_2\psi_1e({\mathbf{i}})=y_2e({\mathbf{i}})\psi_1$,
  so that $e({\mathbf{i}})=y_2^2e({\mathbf{i}})\psi_1^2=0$. Hence,
  $e(0,0,i_3,\dots,i_n)=0$. Finally, suppose that $i_2\ne\pm1,0$. Then
  $e({\mathbf{i}})=\psi_1^2e({\mathbf{i}})=\psi_1e(i_2,i_1,i_3\dots,i_n)\psi_1=0$ since
  $e({\mathbf{j}})=0$ whenever $j_1\ne0$.
\end{proof}

Set $I^n_+=\set{{\mathbf{i}}\in I^n|i_1=0\text{ and }i_2=+1}$ and
$I^n_-=\set{{\mathbf{i}}\in I^n|i_1=0\text{ and }i_2=-1}$. Then
\autoref{L:ZeroSequence}
shows that $e({\mathbf{i}})\ne0$ only if ${\mathbf{i}}\in I^n_+\cup I^n_-$.

Recall that if $\gamma\in{Q^\varepsilon_n}$ then $I^\gamma=\bigcup_{\alpha\in\gamma}I^\alpha$.

\begin{Definition}\label{D:RpSn}
  Suppose that $e\ne2$ and $\gamma\in{Q^\varepsilon_n}$. The algebra
  ${\mathscr{R}^\varepsilon_{\gamma}}{\mathscr{R}^\varepsilon_{\gamma}}\Gamma_e,\Lambda_0)$
  is the unital associative ${\mathcal{Z}}$-algebra with generators
  \[\set{\psi_r, y_s, \eps_a({\mathbf{i}})|1\le r<n, 1\le s\le n,
  {\mathbf{i}}\in I^\gamma\text{ and } a\in{\mathbb{Z}}_2}
  \]
  subject to the relations
  {\setlength{\abovedisplayskip}{2pt}
  \setlength{\belowdisplayskip}{1pt}
  \begin{align*}
    y_1^{(\Lambda_0,\alpha_{i_1})}\eps_0({\mathbf{i}})&= 0,
    & \textstyle\sum_{{\mathbf{i}}\in I^\gamma}\frac12\eps_0({\mathbf{i}})&= 1,
    & \eps_0({\mathbf{i}})\eps_0({\mathbf{j}})&= \delta_{{\mathbf{i}}\sim{\mathbf{j}}}\eps_0({\mathbf{i}}),\\
    \eps_a({\mathbf{i}})\eps_b({\mathbf{i}})&= \eps_{a+b}({\mathbf{i}}),
    & \eps_a({\mathbf{i}})&=(-1)^a\eps_a(-{\mathbf{i}}),
    & \psi_r\eps_a({\mathbf{i}}) &=\eps_a(s_r\cdot{\mathbf{i}})\psi_r,
  \end{align*}
  \begin{align*}
    y_r\eps_a({\mathbf{i}})&= \eps_a({\mathbf{i}})y_r, &
    y_ry_s\eps_1({\mathbf{i}})&= y_sy_r\eps_1({\mathbf{i}}),\\
    \psi_ry_{r+1}\eps_1({\mathbf{i}})&= (y_r\psi_r+\delta_{i_r i_{r+1}})\eps_1({\mathbf{i}}), &
    y_{r+1}\psi_r\eps_1({\mathbf{i}})&= (\psi_ry_r+\delta_{i_r i_{r+1}})\eps_1({\mathbf{i}}),
  \end{align*}
  \begin{align*}
    \psi_ry_s\eps_1({\mathbf{i}})&= y_s\psi_r\eps_1({\mathbf{i}}),&&\text{if }s\neq r,r+1,\\
    \psi_r\psi_s\eps_1({\mathbf{i}})&= \psi_s\psi_r\eps_1({\mathbf{i}}),&&\text{if }|r-s|>1,
  \end{align*}
  \begin{align*}
    \psi_r^2\eps_1({\mathbf{i}})&= \begin{cases}
      0,&\text{if }i_r=i_{r+1},\\
      (y_r-y_{r+1})\eps_0({\mathbf{i}}),&\text{if }i_r\to i_{r+1},\\
      (y_{r+1}-y_r)\eps_0({\mathbf{i}}),&\text{if }i_r\leftarrow i_{r+1},\\
      \eps_1({\mathbf{i}}),&\text{otherwise,}
    \end{cases}\\
    \psi_r\psi_{r+1}\psi_r\eps_0({\mathbf{i}})&=\begin{cases}
      \psi_{r+1}\psi_r\psi_{r+1}\eps_0({\mathbf{i}})-\eps_1({\mathbf{i}}),
      &\text{if }i_r=i_{r+2}\to i_{r+1},\\
      \psi_{r+1}\psi_r\psi_{r+1}\eps_0({\mathbf{i}})+\eps_1({\mathbf{i}}),
      &\text{if }i_r=i_{r+2}\leftarrow i_{r+1},\\
      \psi_{r+1}\psi_r\psi_{r+1}\eps_0({\mathbf{i}}),&\text{otherwise,}
    \end{cases}
  \end{align*}
  }  for ${\mathbf{i}},{\mathbf{j}}\in I^\gamma$, $a,b\in{\mathbb{Z}}_2$ and all
  admissible~$r$ and~$s$. Let
  ${\mathscr{R}^\varepsilon_{n}}\bigoplus_{\gamma\in {Q^\varepsilon_n}}{\mathscr{R}^\varepsilon_{\gamma}}.
\end{Definition}

It is routine to check that the relations in \autoref{D:RpSn}
are homogeneous with respect to the degree function
$\operatorname{Deg}{\,{:}\,{\mathscr{R}^\varepsilon_{n}}\longrightarrow\!{\mathbb{Z}}}_2\times{\mathbb{Z}}$ that is determined by
\[\operatorname{Deg}\psi_r\eps_0({\mathbf{i}})=(1,-c_{i_r,i_{r+1}}),\quad
\operatorname{Deg} y_2=(1,2)\quad \text{and}\quad
\operatorname{Deg}\eps_a({\mathbf{i}})=(a,0),
\]
for $1\le r<n$, $1\le s\le n$ and ${\mathbf{i}}\in I^n$. Hence, ${\mathscr{R}^\varepsilon_{n}} is a
$({\mathbb{Z}}_2\times{\mathbb{Z}})$-graded algebra.

By \autoref{D:RpSn}, if ${\mathbf{i}}\in I^\gamma$ then
$\eps_a({\mathbf{i}})=\pm\eps_a(-{\mathbf{i}})$ so ${\mathscr{R}^\varepsilon_{n}} is generated by the elements
$\set{\psi_1,\dots,\psi_{n-1}}\cup\set{y_1,\dots,y_n}
\cup\set{\eps_a({\mathbf{i}})|a\in{\mathbb{Z}}_2\text{ and }{\mathbf{i}}\in I^\gamma_+}$,
where $I^\gamma_+=I^\gamma\cap I^n_+$. Similarly, set
$I^\gamma_-=I^\gamma\cap I^n_-$. We use~$I^\gamma$ in
\autoref{D:RpSn} because it compactly encodes a sign change
in the relation $\psi_r\eps_a({\mathbf{i}})=\eps_a(s_r\cdot{\mathbf{i}})\psi$
when $r=2$.

We need a partial analogue of \autoref{L:ZeroSequence} for ${\mathscr{R}^\varepsilon_{\gamma}}.

\begin{Lemma}\label{L:ZeroSequence2}
  Suppose that ${\mathbf{i}}\in I^\gamma$ and that ${\mathbf{i}}=-{\mathbf{i}}$. Then
  $\eps_a({\mathbf{i}})=0$, for $a\in{\mathbb{Z}}_2$.
\end{Lemma}

\begin{proof}
  If ${\mathbf{i}}=-{\mathbf{i}}$ then $\eps_1({\mathbf{i}})=-\eps_1({\mathbf{i}})=0$. Hence,
  $\eps_0({\mathbf{i}})=\eps_1({\mathbf{i}})\eps_1({\mathbf{i}})=0$.
\end{proof}

By forgetting the ${\mathbb{Z}}_2$-grading on ${\mathscr{R}^\varepsilon_{\gamma}} we obtain a ${\mathbb{Z}}$-graded
algebra. Given the similarity of the relations in \autoref{D:klr} and
\autoref{D:RpSn} the next result should not surprise the reader.

\begin{Proposition}\label{P:RpSnIsomorphism}
  Suppose that $\gamma\in{Q^\varepsilon_n}$, $n\ge0$, $e\ne2$ and that~$2$ is
  invertible in~${\mathcal{Z}}$. Then, as ${\mathbb{Z}}$-graded algebras,
  ${\mathscr{R}^\varepsilon_{\gamma}}{{\mathscr{R}_e({\mathfrak{S}_{n}}}{\gamma}}.
\end{Proposition}

\begin{proof}
  Define a map $\theta$ from the generators of ${\mathscr{R}^\varepsilon_{\gamma}} to
  ${{\mathscr{R}_e({\mathfrak{S}_{n}}}{\gamma}} by
  \begin{equation*}
    \theta(\psi_r) = \psi_r,\quad
    \theta(y_s) = y_s,\quad\text{ and }\quad
    \theta{\big(}\eps_a({\mathbf{i}}){\big)} = e({\mathbf{i}}) + (-1)^ae(-{\mathbf{i}}),
  \end{equation*}
  for $1\le r<n$, $1\le s\le n$, $a\in Z_2$ and ${\mathbf{i}}\in I^\gamma$.  The
  relations in \autoref{D:RpSn} are very similar to those of
  \autoref{D:klr}, so it is straightforward to check that~$\theta$
  extends to an algebra homomorphism ${\mathscr{R}^\varepsilon_{\gamma}}{{\mathscr{R}_e({\mathfrak{S}_{n}}}{\gamma}}.
  By definition, if ${\mathbf{i}}\in I^\gamma$then
  $e({\mathbf{i}})=\frac12\theta\big(\eps_0({\mathbf{i}})+\eps_1({\mathbf{i}})\big)$.
  Therefore, the image of~$\theta$ contains
  all of the generators of ${{\mathscr{R}_e({\mathfrak{S}_{n}}}{\gamma}}. Hence, $\theta$ is
  surjective.

  Rather than proving directly that $\theta$ is an isomorphism we define
  an inverse map. Define $\vartheta$ to be the map from
  the set of non-zero generators of~${{\mathscr{R}_e({\mathfrak{S}_{n}}}{\gamma}} into~${\mathscr{R}^\varepsilon_{\gamma}} given by
  \[\vartheta(\psi_r)=\psi_r,\quad
  \vartheta(y_s)=y_s, \quad\text{and}\quad
  \vartheta\(e({\mathbf{i}}){\big)}=\tfrac12{\big(}\eps_0({\mathbf{i}})+\eps_1({\mathbf{i}}){\big)},
  \]
  for $1\le r<n$, $1\le s\le n$ and ${\mathbf{i}}\in I^\gamma$. If ${\mathbf{i}}\in I^\gamma$ and
  $a\in{\mathbb{Z}}_2$ then
  \[
  \vartheta\big(e({\mathbf{i}})+(-1)^ae(-{\mathbf{i}})\big)
  = \frac12\big( \eps_0({\mathbf{i}})+\eps_1({\mathbf{i}})
  +(-1)^a \eps_0(-{\mathbf{i}})+(-1)^a\eps_1(-{\mathbf{i}})\big)
  =\eps_a({\mathbf{i}}).
  \]
  Hence, by \autoref{L:ZeroSequence2}, the image of $\vartheta$ contains
  all of the generators of~${\mathscr{R}^\varepsilon_{\gamma}} so, if it is a homomorphism, it is
  surjective.

  Now, since $\eps_a({\mathbf{i}})\eps_b({\mathbf{i}})=\eps_{a+b}({\mathbf{i}})$, for
  all ${\mathbf{i}}\in I^\gamma$
  and $a,b\in{\mathbb{Z}}_2$, by multiplying the relations in \autoref{D:RpSn} on
  the right by $\eps_1({\mathbf{i}})$ the following additional relations
  hold in ${\mathscr{R}^\varepsilon_{\gamma}}:
  {\setlength{\abovedisplayskip}{2pt}
  \setlength{\belowdisplayskip}{1pt}
  \begin{align*}
    y_ry_s\eps_0({\mathbf{i}})&= y_sy_r\eps_0({\mathbf{i}}),
  \end{align*}
  \begin{align*}
    \psi_ry_{r+1}\eps_0({\mathbf{i}})&= (y_r\psi_r+\delta_{i_r i_{r+1}})\eps_0({\mathbf{i}}), &
    y_{r+1}\psi_r\eps_0({\mathbf{i}})&= (\psi_ry_r+\delta_{i_r i_{r+1}})\eps_0({\mathbf{i}}),
  \end{align*}
  \begin{align*}
    \psi_ry_s\eps_0({\mathbf{i}})&= y_s\psi_r\eps_0({\mathbf{i}}),&&\text{if }s\neq r,r+1,\\
    \psi_r\psi_s\eps_0({\mathbf{i}})&= \psi_s\psi_r\eps_0({\mathbf{i}}),&&\text{if }|r-s|>1,
  \end{align*}
  \begin{align*}
    \psi_r^2\eps_0({\mathbf{i}})&= \begin{cases}
      0,&\text{if }i_r=i_{r+1},\\
      (y_r-y_{r+1})\eps_1({\mathbf{i}}),&\text{if }i_r\to i_{r+1},\\
      (y_{r+1}-y_r)\eps_1({\mathbf{i}}),&\text{if }i_r\leftarrow i_{r+1},\\
      \eps_0({\mathbf{i}}),&\text{otherwise,}
    \end{cases}\\
    \psi_r\psi_{r+1}\psi_r\eps_1({\mathbf{i}})&=\begin{cases}
      \psi_{r+1}\psi_r\psi_{r+1}\eps_1({\mathbf{i}})-\eps_0({\mathbf{i}}),
      &\text{if }i_r=i_{r+2}\to i_{r+1},\\
      \psi_{r+1}\psi_r\psi_{r+1}\eps_1({\mathbf{i}})+\eps_0({\mathbf{i}}),
      &\text{if }i_r=i_{r+2}\leftarrow i_{r+1},\\
      \psi_{r+1}\psi_r\psi_{r+1}\eps_1({\mathbf{i}}),&\text{otherwise,}
    \end{cases}
  \end{align*}}  for all admissible $r$ and $s$ and ${\mathbf{i}}\in I^\gamma$. As it was
  for~$\theta$, it is now straightforward to verify that
  $\vartheta$ respects all of the relations of~${{\mathscr{R}_e({\mathfrak{S}_{n}}}{\gamma}}. Consequently,
  $\vartheta$ extends to an algebra homomorphism ${{\mathscr{R}_e({\mathfrak{S}_{n}}}{\gamma}}{\mathscr{R}^\varepsilon_{\gamma}}.

  In view of \autoref{L:ZeroSequence} and \autoref{L:ZeroSequence2}, the
  automorphisms $\theta\circ\vartheta$ and $\vartheta\circ\theta$ act as
  the identity on the non-zero generators of ${{\mathscr{R}_e({\mathfrak{S}_{n}}}{\gamma}} and
  ${\mathscr{R}^\varepsilon_{\gamma}}, respectively. Therefore,~$\theta$ and $\vartheta$
  are mutually inverse isomorphisms and
  ${\mathscr{R}^\varepsilon_{\gamma}}{{\mathscr{R}_e({\mathfrak{S}_{n}}}{\gamma}} as (ungraded) algebras.

  It remains to observe that $\theta$ and
  $\vartheta$ respect the ${\mathbb{Z}}$-gradings on both algebras, but this is
  immediate from the definitions of~$\theta$ and~$\vartheta$. Hence,
  ${\mathscr{R}^{\Lambda}_{\beta}}{\mathscr{R}^\varepsilon_{\gamma}} as ${\mathbb{Z}}$-graded algebras, completing the
  proof.
\end{proof}

Define $h\in{\mathscr{R}^\varepsilon_{n}} to be \textbf{even} if $\operatorname{Deg} h=(0,d)$
and~$h$ is \textbf{odd} if $\operatorname{Deg} h=(1,d)$, for some $d\in{\mathbb{Z}}$. Let~${\mathscr{R}^{\varepsilon+}_\gamma} and
${\mathscr{R}^{\varepsilon-}_\gamma} be the sets of even and odd elements in~${\mathscr{R}^\varepsilon_{\gamma}},
respectively. Then~${\mathscr{R}^{\varepsilon+}_\gamma} is a subalgebra of~${\mathscr{R}^\varepsilon_{\gamma}} and
\begin{equation}\label{clifforddecomp}
  {\mathscr{R}^\varepsilon_{\gamma}}{\mathscr{R}^{\varepsilon+}_\gamma}{\mathscr{R}^{\varepsilon-}_\gamma}{equation}
as ${\mathcal{Z}}$-modules.  Moreover, as we next show, ${\mathscr{R}^{\varepsilon+}_\gamma} is isomorphic
to $\RAn$ under the isomorphism of \autoref{P:RpSnIsomorphism}.

\begin{Corollary}\label{C:EvenBit}
  Suppose that $n\ge0$ and that $2$ is invertible in ${\mathcal{Z}}$. Let
  $\gamma\in {Q^\varepsilon_n}$. Then ${\mathscr{R}^{\varepsilon+}_\gamma}\RAn_\gamma$
  as ${\mathbb{Z}}$-graded algebras.
\end{Corollary}

\begin{proof}
  Under the isomorphism ${\mathscr{R}^\varepsilon_{\gamma}}{{\mathscr{R}_e({\mathfrak{S}_{n}}}{\gamma}} of
  \autoref{P:RpSnIsomorphism}, the images of the
  even generators of~${\mathscr{R}^\varepsilon_{\gamma}} are ${\mathtt{sgn}}$-invariant and ${\mathtt{sgn}}$ multiplies
  the images of the odd generators by~$-1$. Hence, $\theta$ restricts to
  an isomorphism ${\mathscr{R}^{\varepsilon+}_\gamma}\RAn_\gamma$.
\end{proof}

We can now prove \autoref{T:MainRelations} from the introduction.

\begin{proof}[Proof of \autoref{T:MainRelations}]
  Let $A_\gamma$ be the abstract algebra with the presentation given in
  \autoref{T:MainRelations}. By \autoref{C:EvenBit}, to prove
  \autoref{T:MainRelations} it is enough to show that $A_\gamma\cong {\mathscr{R}^{\varepsilon+}_\gamma}.
  Define a map $\Theta{\,{:}\,{A_\gamma}\!\longrightarrow\!{\mathscr{R}^{\varepsilon+}_\gamma}} by
  \[\Psi_r({\mathbf{i}})\mapsto\psi_r\eps_1({\mathbf{i}}),\quad
  Y_s({\mathbf{i}})\mapsto y_s\eps_1({\mathbf{i}})\quad\text{and}\quad
  \eps({\mathbf{i}})\mapsto \eps_0({\mathbf{i}}),
  \]
  for all ${\mathbf{i}}\in I^\gamma$, $1\le r\le n$ and $1\le s<n$. Using
  \autoref{D:RpSn}, and the relations in the proof of
  \autoref{P:RpSnIsomorphism}, it is straightforward to check that all
  of the relations in~$A_\gamma$ are satisfied in ${\mathscr{R}^{\varepsilon+}_\gamma}, so $\Theta$
  extends to an algebra homomorphism from~$A_\gamma$ to~${\mathscr{R}^{\varepsilon+}_\gamma}.

  By \autoref{D:RpSn}, the algebra ${\mathscr{R}^{\varepsilon+}_\gamma} is generated by
  arbitrary products of the generators of~${\mathscr{R}^\varepsilon_{n}} such that the
  resulting element is even. However, the only even generators
  of~${\mathscr{R}^\varepsilon_{n}} are the idempotents $\eps_0({\mathbf{i}})$, for ${\mathbf{i}}\in I^n$, so
  ${\mathscr{R}^{\varepsilon+}_\gamma} is generated by these idempotents together with
  all words of even length in the odd generators of~${\mathscr{R}^\varepsilon_{n}}. As
  $\psi_r\eps_1({\mathbf{i}})=\eps_1(s_r\cdot{\mathbf{j}})\psi_r$ and
  $y_s\eps_1({\mathbf{i}})=\eps_1({\mathbf{i}})y_s$, for admissible $r$, $s$ and ${\mathbf{i}}\in I^n$,
  it follows that ${\mathscr{R}^{\varepsilon+}_\gamma} is generated by the images of
  $A_\gamma$ under $\Theta$. Hence, $\Theta$ is surjective.

  The algebra ${\mathscr{R}^{\varepsilon+}_\gamma} is defined by generators and relations, so
  ${\mathscr{R}^{\varepsilon+}_\gamma} is the subalgebra of ${\mathscr{R}^\varepsilon_{n}} generated by the
  words of even length in the generators of~${\mathscr{R}^\varepsilon_{n}} modulo the even part of the
  relational ideal that defines~${\mathscr{R}^\varepsilon_{n}}. The only even relations
  for~${\mathscr{R}^\varepsilon_{n}} are the idempotent relations and the commutation
  relations (these are the relations appearing in the first three lines of
  the relations in \autoref{D:RpSn}). Hence, up to multiplication by an
  idempotent~$\eps_0({\mathbf{i}})$, all of the even relations are
  essentially trivial. Therefore, the even component of
  the relational ideal for~${\mathscr{R}^\varepsilon_{n}} is generated by words of even
  length in the odd relations for~${\mathscr{R}^\varepsilon_{n}}. In turn, all of the even
  products of the odd relations are products of the even relations given
  in the proof of \autoref{P:RpSnIsomorphism}, together with the even
  idempotent and commutation relations. All of these relations are the
  images under $\Theta$ of the relations of~$A_\gamma$. Hence, $\Theta$ is an
  isomorphism and the result follows.
\end{proof}

\section{The seminormal form}\label{S:SeminormalForm}
  To prove \autoref{T:Main} we will work mainly in the setting of the
  semisimple representation theory of~${{\mathscr{H}}_\xi({\mathfrak{S}_{n}}}. The idea is to show that
  the fixed-point subalgebras of ${{\mathscr{H}}_\xi({\mathfrak{A}_{n}}}{{\mathscr{H}}_\xi({\mathfrak{S}_{n}}}\#$ and ${\mathscr{R}_e({\mathfrak{S}_{n}}}{\mathscr{R}_e({\mathfrak{S}_{n}}}{\mathtt{sgn}}$
  coincide under the Brundan-Kleshchev isomorphism of \autoref{T:BKiso}.
  Unfortunately, as shown by \autoref{Ex:BadBKRestriction}, this is not true. To
  get around this we use the machinery developed in
  \cite{HuMathas:SeminormalQuiver} to construct a new isomorphism
  ${\mathscr{R}_e({\mathfrak{S}_{n}}}{{\mathscr{H}}_\xi({\mathfrak{S}_{n}}} that does restrict to an isomorphism
  $\RAn{\overset{\simeq}{\longrightarrow}}{{\mathscr{H}}_\xi({\mathfrak{A}_{n}}}.

  \subsection{Tableau combinatorics}\label{S:tableaux}
  This section recalls the partition and tableau
  combinatorics  that are needed in this paper.

  A \textbf{ partition} ${\lambda}=(\lambda_1,\lambda_2,\dots)$ is a weakly decreasing
  sequence of non-negative integers. The integers $\lambda_r$ are the
  \textbf{parts} of $\lambda$, for $r\ge1$, and $\lambda$ is a partition of~$n$ if
  $\left|\lambda\right|=n$, where $\left|\lambda\right|=\lambda_1+\lambda_2+\cdots$.

  The \textbf{Young diagram} of a partition $\lambda$ is the set
  $\set{(r,c)|1\le c\le\lambda_r\text{ for }r\ge1}$, which we represent as
  a collection of left-justified boxes in the plane, with $\lambda_r$
  boxes in row~$r$ and with rows ordered from top to bottom by increasing
  row index. We identify a partition with its diagram. The partition
  ${\lambda}'$ with ${\lambda}'_r=\#\set{c\geq 1| \lambda_r\geq c}$ is the partition
  \textbf{conjugate} to ${\lambda}$.

  Suppose that $\lambda\in{\mathcal{P}_{n}}. A \textbf{${\lambda}$-tableau} is a bijective filling of
  the boxes of ${\lambda}$ with the numbers $1,2,\ldots,n$. If ${\mathsf{t}}$ is a
  $\lambda$-tableau then it has \textbf{shape} $\lambda$ and we write
  $\operatorname{\text{Shape}}({\mathsf{t}})=\lambda$. For $m\ge1$ let ${\mathsf{t}}_{\downarrow m}$ be the subtableau
  of~${\mathsf{t}}$ that contains the numbers $1,2,\dots.m$.

  A tableau is \textbf{standard} if its entries increase from left to
  right along each row and from top to bottom down each column.  Hence,
  ${\mathsf{t}}$ is standard if and only if~${\mathsf{t}}_{\downarrow m}$ is standard for
  $1\le m\le n$.  Let~${\mathop{\rm Std}\nolimits}(\lambda)$ be the set of standard ${\lambda}$-tableaux
  and let
  \[{\mathop{\rm Std}\nolimits}({\mathcal{P}_{n}}=\bigcup_{\lambda\in{\mathcal{P}_{n}}{\mathop{\rm Std}\nolimits}(\lambda)\qquad\text{and}\qquad
  {\mathop{\rm Std}\nolimits}^2({\mathcal{P}_{n}}=\bigcup_{\lambda\in{\mathcal{P}_{n}}{\mathop{\rm Std}\nolimits}(\lambda)\times{\mathop{\rm Std}\nolimits}(\lambda).
  \]

  If ${\mathsf{t}}$ is a standard $\lambda$-tableau then the \textbf{conjugate}
  tableau ${\mathsf{t}}'$ is the standard ${\lambda}'$-tableau obtained by swapping the rows and
  columns of~${\mathsf{t}}$.

  The \textbf{initial $\lambda$-tableau} ${\mathsf{t}}^\lambda$ is the
  $\lambda$-tableau obtained by inserting the numbers $1, 2, \ldots,n$ in
  order along the rows of~$\lambda$, from left to right and then top to
  bottom. The \textbf{co-initial} tableau ${\mathsf{t}}_\lambda$ is the conjugate
  of~${\mathsf{t}}^{\lambda'}$. Then ${\mathsf{t}}_\lambda$ is the unique $\lambda$-tableau that has
  the numbers $1,2,\dots,n$ entered in order down the columns
  of~$\lambda$, from left to right.

  Recall that $I={\mathbb{Z}}/e{\mathbb{Z}}$. Let ${\mathsf{t}}$ be a standard tableau and
  suppose that $m$ appears in row~$r$ and column~$c$ of ~${\mathsf{t}}$, where $1\le
  m\le n$. The
  \textbf{content} and \textbf{$e$-residue} of~$m$ in~${\mathsf{t}}$ are given by
  \[
  c_m({\mathsf{t}})=c-r\in{\mathbb{Z}}\qquad\text{and}\qquad \operatorname{res}_m({\mathsf{t}})= c-r+e{\mathbb{Z}}\in I.
  \]
  respectively. The \textbf{$e$-residue sequence} of the tableau~${\mathsf{t}}$ is
  the $n$-tuple
  \[
  \operatorname{res}({\mathsf{t}})={\big(}\operatorname{res}_1({\mathsf{t}}),\dots,\operatorname{res}_n({\mathsf{t}}){\big)}\in I^n.
  \]

  Given a sequence ${\mathbf{i}}\in I^n$ let
  ${\mathop{\rm Std}\nolimits}({\mathbf{i}})=\set{{\mathsf{t}}\in{\mathop{\rm Std}\nolimits}({\mathcal{P}_{n}}\mid\operatorname{res}({\mathsf{t}})={\mathbf{i}}}$ be the set of
  standard tableaux with residue sequence~${\mathbf{i}}$.

  \subsection{Seminormal forms}\label{Sect:Seminormal}
  To prove \autoref{T:MainTheorem} we make extensive use of the
  semisimple representation theory of ${{\mathscr{H}}_\xi({\mathfrak{S}_{n}}} using seminormal forms.
  This section introduces Jucys-Murphy elements and seminormal forms and
  proves some basic facts relating seminormal forms and the
  $\#$-involution.

  Throughout this section we fix a field ${\mathcal{K}}$ and a non-zero scalar
  $t\in{\mathcal{K}}$. Let~${{\mathscr{H}}_t^{\mathcal{K}}({\mathfrak{S}_{n}}}$ be the Iwahori-Hecke algebra of~${\mathfrak{S}_{n}} over~${\mathcal{K}}$ with
  parameter~$t$.

  If $k\in{\mathbb{Z}}$  define the \textbf{quantum integer} to be the scalar
  \[[k]_t=\begin{cases}
    \phantom{-(}(1+t+\dots+t^{k-1}),& \text{if }k\ge0,\\
    -(t^{-1}+t^{-2}+\dots+t^k),&\text{if }k<0.
  \end{cases}
  \]
  When~$t$ is understood we write $[k]=[k]_t$.

  We need a well-known result, which is easily proved by induction
  on~$n$. To state this, define the \textbf{Poincar\'e polynomial} of~${{\mathscr{H}}_t^{\mathcal{K}}({\mathfrak{S}_{n}}}$ to
  be $P_{\mathscr{H}}(t)=[1][2]\dots[n]\in{\mathcal{K}}$.

  \begin{Lemma}[\protect{See \cite[Lemma 3.34]{M:ULect}}]
    \label{L:separation}
    Suppose that $P_{\mathscr{H}}(t)\ne0$ and that
    ${\mathsf{s}},{\mathsf{t}}\in{\mathop{\rm Std}\nolimits}({\mathcal{P}_{n}}$. Then ${\mathsf{s}}={\mathsf{t}}$ if and only if
    $[c_r({\mathsf{s}})]=[c_r({\mathsf{t}})]$, for $1\le r\le n$.
  \end{Lemma}

  We assume for the rest of this section that $P_{\mathscr{H}}(t)\ne0$. In fact,
  the results that follow imply that~${{\mathscr{H}}_t^{\mathcal{K}}({\mathfrak{S}_{n}}}$ is semisimple if and only if
  $P_{\mathscr{H}}(t)\ne0$ and, in turn, this is equivalent to the condition in
  \autoref{L:separation}.

  If ${\mathsf{t}}$ is a tableau and $1\le r\le n$ then the
  \textbf{axial distance} from $r+1$ to $r$ in~${\mathsf{t}}$ is
  \begin{equation}\label{E:rhodef}
    \rho_r({\mathsf{t}})=c_r({\mathsf{t}})-c_{r+1}({\mathsf{t}})\in{\mathbb{Z}}.
  \end{equation}
  By definition, $-n<\rho_r({\mathsf{t}})<n$ so $[\rho_r({\mathsf{t}})]\ne0$ if $P_{\mathscr{H}}(t)\ne0$.

  The next definition will provide us with the framework to prove
  \autoref{T:MainTheorem}.

  \begin{Definition}[\protect{    Hu-Mathas~\cite[Definition~3.5]{HuMathas:SeminormalQuiver}}]
    \label{D:alphaSNCS}
    Suppose that $P_{\mathscr{H}}(t)\ne0$.  A \textbf{$*$-seminormal coefficient
    system} is a set of scalars
    \[{{\boldsymbol{\alpha}}}=\set{\alpha_r({\mathsf{t}})\in{\mathcal{K}}|1\leq r<n\text{ and }
    {\mathsf{t}}\in{\mathop{\rm Std}\nolimits}({\mathcal{P}_{n}}}
    \]
    such that if~${\mathsf{t}}\in{\mathop{\rm Std}\nolimits}({\mathcal{P}_{n}}$ and $1\le r<n$ then:
    \begin{enumerate}
      \item $\alpha_r({\mathsf{t}})=0$ whenever $s_r{\mathsf{t}}$ is not standard.
      \item $\alpha_r({\mathsf{t}})\alpha_k(s_r{\mathsf{t}})=\alpha_k({\mathsf{t}})\alpha_r(s_k{\mathsf{t}})$
      whenever $1\le k<n$ and $|r-k|>1$.
      \item
      $ \alpha_r(s_{r+1}s_r{\mathsf{t}})\alpha_{r+1}(s_r{\mathsf{t}})\alpha_r({\mathsf{t}})
      =\alpha_{r+1}(s_rs_{r+1}{\mathsf{t}})\alpha_r(s_{r+1}{\mathsf{t}})\alpha_{r+1}({\mathsf{t}})
      $
      if $r\ne n-1$,
      \item if ${\mathsf{v}}=s_r{\mathsf{t}}\in{\mathop{\rm Std}\nolimits}({\mathcal{P}_{n}}$ then
      \begin{equation*}\label{E:sncf}
        \alpha_r({\mathsf{t}})\alpha_r({\mathsf{v}})
        =\frac{[1+\rho_r({\mathsf{t}})][1+\rho_r({\mathsf{v}})]}{[\rho_r({\mathsf{t}})][\rho_r({\mathsf{v}})]}.
      \end{equation*}
    \end{enumerate}
  \end{Definition}

  Many examples of seminormal coefficient systems are given in
  \cite[\S3]{HuMathas:SeminormalQuiver}. For example, $\set{\alpha_r({\mathsf{t}})}$
  is seminormal coefficient system, where
  $\alpha_r({\mathsf{t}})=\tfrac{[1+\rho_r({\mathsf{t}})]}{[\rho_r({\mathsf{t}})]}$ whenever ${\mathsf{t}},
  s_r{\mathsf{t}}\in{\mathop{\rm Std}\nolimits}({\mathcal{P}_{n}}$. In \autoref{S:AlternaatingCoefficients} we fix
  a particular choice of seminormal coefficient system but until then
  we will work with an arbitrary coefficient system.

  For $k=1,2,\ldots,n$ the \textbf{Jucys-Murphy} element $L_k\in {{\mathscr{H}}^{\mathcal{O}}_{t}} is
  defined by
  \[
  L_k=\sum_{j=1}^{k-1}t^{j-k} T_{(k-j,k)}.
  \]

  A basis $\set{f_{{\mathsf{s}}{\mathsf{t}}}|({\mathsf{s}},{\mathsf{t}})\in{\mathop{\rm Std}\nolimits}^2({\mathcal{P}_{n}}}$ of ${{\mathscr{H}}_t^{\mathcal{K}}({\mathfrak{S}_{n}}}$ is a \textbf{seminormal
  basis} if
  \[L_kf_{{\mathsf{s}}{\mathsf{t}}}=[c_k({\mathsf{s}})]f_{{\mathsf{s}}{\mathsf{t}}}\qquad\text{and}\qquad
  f_{{\mathsf{s}}{\mathsf{t}}}L_k=[c_k({\mathsf{t}})]f_{{\mathsf{s}}{\mathsf{t}}},
  \]
  for all $({\mathsf{s}},{\mathsf{t}})\in {\mathop{\rm Std}\nolimits}^2({\mathcal{P}_{n}}$ and $1\le k\le n$.
  The basis $\{f_{{\mathsf{s}}{\mathsf{t}}}\}$ is a \textbf{$*$-seminormal basis} if, in addition,
  $f_{{\mathsf{s}}{\mathsf{t}}}^*=f_{{\mathsf{t}}{\mathsf{s}}}$, for all $({\mathsf{s}},{\mathsf{t}})\in{\mathop{\rm Std}\nolimits}^2({\mathcal{P}_{n}}$, where $*$ is
  the unique anti-isomorphism of ${{\mathscr{H}}_t^{\mathcal{K}}({\mathfrak{S}_{n}}}$ that fixes $T_1,\dots,T_{n-1}$.

  Recall that $P_{\mathscr{H}}(t)=[1][2]\dots[n]$.

  \begin{Theorem}[\protect{    The seminormal form~\cite[Theorem 3.9]{HuMathas:SeminormalQuiver}}]
    \label{T:SeminormalForm}
    Suppose that $P_{\mathscr{H}}(t)\ne0$ and that ${{\boldsymbol{\alpha}}}$ is a seminormal
    coefficient system for ${{\mathscr{H}}_t^{\mathcal{K}}({\mathfrak{S}_{n}}}$. Then:
    \begin{enumerate}
      \item The algebra ${{\mathscr{H}}_t^{\mathcal{K}}({\mathfrak{S}_{n}}}$ has a unique $*$-seminormal basis
      $\set{f_{{\mathsf{s}}{\mathsf{t}}}|({\mathsf{s}},{\mathsf{t}})\in{\mathop{\rm Std}\nolimits}^2({\mathcal{P}_{n}}}$ such that
      \[
      f_{{\mathsf{s}}{\mathsf{t}}}^*=f_{{\mathsf{t}}{\mathsf{s}}},\quad L_kf_{{\mathsf{s}}{\mathsf{t}}}=[c_k({\mathsf{s}})]f_{{\mathsf{s}}{\mathsf{t}}}\quad\text{and}\quad
      T_rf_{{\mathsf{s}}{\mathsf{t}}}=\alpha_r({\mathsf{s}})f_{{\mathsf{u}}{\mathsf{t}}}-\frac{1}{[\rho_r({\mathsf{s}})]}f_{{\mathsf{s}}{\mathsf{t}}},
      \]
      where ${\mathsf{u}}=(r,r+1){\mathsf{s}}$. $($Set $f_{{\mathsf{u}}{\mathsf{t}}}=0$ if ${\mathsf{u}}$ is not standard.$)$
      \item For ${\mathsf{t}}\in{\mathop{\rm Std}\nolimits}({\mathcal{P}_{n}}$ there exist non-zero scalars
      $\gamma_{\mathsf{t}}\in{\mathcal{K}}$ such that
      $f_{{\mathsf{s}}{\mathsf{t}}}f_{{\mathsf{u}}{\mathsf{v}}}=\delta_{{\mathsf{t}}{\mathsf{u}}}\gamma_{\mathsf{t}} f_{{\mathsf{s}}{\mathsf{v}}}$ and
      $\set{\tfrac1{\gamma_{\mathsf{t}}}f_{{\mathsf{t}}{\mathsf{t}}}| {\mathsf{t}}\in{\mathop{\rm Std}\nolimits}({\mathcal{P}_{n}}}$ is a complete
      set of pairwise orthogonal primitive idempotents.
      \item The $*$-seminormal basis
      $\set{f_{{\mathsf{s}}{\mathsf{t}}}|({\mathsf{s}},{\mathsf{t}})\in{\mathop{\rm Std}\nolimits}^2({\mathcal{P}_{n}}}$ is uniquely determined by
      the $*$-seminormal coefficient system ${{\boldsymbol{\alpha}}}$ and the scalars
      $\set{\gamma_{{\mathsf{t}}^\lambda}|\lambda\in{\mathcal{P}_{n}}$.
    \end{enumerate}
  \end{Theorem}

  By \autoref{T:SeminormalForm}(b), if ${\mathsf{t}}\in{\mathop{\rm Std}\nolimits}({\mathcal{P}_{n}}$ then
  $F_{\mathsf{t}}=\tfrac1{\gamma_{\mathsf{t}}}f_{{\mathsf{t}}{\mathsf{t}}}$ is a primitive idempotent in ${{\mathscr{H}}_t^{\mathcal{K}}({\mathfrak{S}_{n}}}$. As is
  well-known (see, for example, \cite[(3.2)]{HuMathas:SeminormalQuiver}),
  \begin{equation*}\label{E:Ft}
    F_{\mathsf{t}}= \prod_{k=1}^n\prod_{\substack{{\mathsf{s}}\in{\mathop{\rm Std}\nolimits}({\mathcal{P}_{n}}\\c_k({\mathsf{s}})\ne c_k({\mathsf{t}})}}
    \frac{L_k-[c_k({\mathsf{s}})]}{[c_k({\mathsf{t}})]-[c_k({\mathsf{s}})]}.
  \end{equation*}
  In particular, the idempotent $F_{\mathsf{t}}$ is independent of the choice of
  seminormal basis.

  Let ${\mathscr{L}}$ be the commutative subalgebra generated by the
  Jucys-Murphy elements. \autoref{T:SeminormalForm}(a) implies that, as
  an $({\mathscr{L}},{\mathscr{L}})$-bimodule, ${{\mathscr{H}}_t^{\mathcal{K}}({\mathfrak{S}_{n}}}$ decomposes as
  \begin{equation}\label{E:LLBimodule}
    {{\mathscr{H}}_t^{\mathcal{K}}({\mathfrak{S}_{n}}}=\bigoplus_{({\mathsf{s}},{\mathsf{t}})\in{\mathop{\rm Std}\nolimits}^2({\mathcal{P}_{n}}}H_{{\mathsf{s}}{\mathsf{t}}},
  \end{equation}
  where $H_{{\mathsf{s}}{\mathsf{t}}}={\mathcal{K}} f_{{\mathsf{s}}{\mathsf{t}}}$. Equivalently,
 \[ H_{{\mathsf{s}}{\mathsf{t}}}=\set{h\in{{\mathscr{H}}_t^{\mathcal{K}}({\mathfrak{S}_{n}}}|L_k h=[c_k({\mathsf{s}})]h\text{ and }
    hL_k=[c_k({\mathsf{t}})]h\text{ for }1\le k\le n},
  \]
  for $({\mathsf{s}},{\mathsf{t}})\in{\mathop{\rm Std}\nolimits}^2({\mathcal{P}_{n}}$.

  Let $\#$ be the hash involution from \autoref{E:hash} on~${{\mathscr{H}}_t^{\mathcal{K}}({\mathfrak{S}_{n}}}$. Then
  $T_r^\#=-T_r+t-1$, for $1\le r<n$.

  \begin{Lemma}\label{L:LFHash}
    Suppose that $1\le k\le n$ and ${\mathsf{s}}\in{\mathop{\rm Std}\nolimits}({\mathcal{P}_{n}}$. Then
    \[
    L_k^\# f_{{\mathsf{s}}{\mathsf{s}}}=[c_k({\mathsf{s}}')]f_{{\mathsf{s}}{\mathsf{s}}}.
    \]
  \end{Lemma}
  \begin{proof}
    For $1\le k\le n$ set
    $\hat{L}_k=t^{1-k}T_{k-1}T_{k-2}\cdots T_2T_1^2T_2\cdots T_{k-2}T_{k-1}$.
    It is well-known and easy to prove that
    $\hat{L}_k=(t-1)L_k+1$; see, for example,
    \cite[Exercise~3.6]{M:ULect}.
    By \autoref{T:SeminormalForm}(a), $L_k f_{{\mathsf{s}}{\mathsf{s}}}=[c_k({\mathsf{s}})]f_{{\mathsf{s}}{\mathsf{s}}}$, so
    $\hat L_kf_{{\mathsf{s}}{\mathsf{s}}}=t^{c_k({\mathsf{s}})}f_{{\mathsf{s}}{\mathsf{s}}}$. Now
    $(\hat{L}_k)^\#=\hat{L}_k^{-1}$ since $T_r^\#=-tT_r^{-1}$ by
    \autoref{E:hash}, for $1\le r<k\le n$. Therefore,
    \[\hat{L}_k^\# f_{{\mathsf{s}}{\mathsf{s}}}=\hat{L}_k^{-1}f_{{\mathsf{s}}{\mathsf{s}}}
    =t^{-c_k({\mathsf{s}})}f_{{\mathsf{s}}{\mathsf{s}}}
    =t^{c_k({\mathsf{s}}')}f_{{\mathsf{s}}{\mathsf{s}}},
    \]
    where the last equality follows because $c_k({\mathsf{s}}')=-c_k({\mathsf{s}})$ for $1\le k\le n$.
    Hence, $L_k^\#f_{{\mathsf{s}}{\mathsf{s}}}=[c_k({\mathsf{s}}')]f_{{\mathsf{s}}{\mathsf{s}}}$ as claimed.
  \end{proof}

  \begin{Lemma}\label{L:Fhash}
    Suppose that ${\mathsf{s}}\in {\mathop{\rm Std}\nolimits}({\mathcal{P}_{n}}$. Then $F_{\mathsf{s}}^\#=F_{{\mathsf{s}}'}$.
  \end{Lemma}
  \begin{proof}
    Since $F_{\mathsf{s}}=\frac{1}{\gamma_{\mathsf{s}}}f_{{\mathsf{s}}{\mathsf{s}}}$, applying \autoref{L:LFHash} gives
    \[
    L_kF_{\mathsf{s}}^\#=(L_k^\#F_{\mathsf{s}})^\#=([c_k({\mathsf{s}}')F_{\mathsf{s}})^\#=[c_k({\mathsf{s}}')]F_{\mathsf{s}}^\#.
    \]
    Similarly $F_{\mathsf{s}}^\#L_k=[c_k({\mathsf{s}}')]F_{\mathsf{s}}^\#$. Therefore, $F_{\mathsf{s}}^\#\in H_{{\mathsf{s}}'{\mathsf{s}}'}$ in the
    decomposition of \autoref{E:LLBimodule}. As $F_{\mathsf{s}}$ is an idempotent, and $\#$ is
    an algebra automorphism, it follows that $F_{\mathsf{s}}^\#=F_{{\mathsf{s}}'}$ since this is the
    unique idempotent in $H_{{\mathsf{s}}'{\mathsf{s}}'}={\mathcal{K}} F_{{\mathsf{s}}'}={\mathcal{K}} f_{{\mathsf{s}}'{\mathsf{s}}'}$.
  \end{proof}

  \begin{Corollary}\label{C:fttHash}
    Suppose that ${\mathsf{s}}\in{\mathop{\rm Std}\nolimits}({\mathcal{P}_{n}}$. Then
    $f_{{\mathsf{s}}{\mathsf{s}}}^\#=\dfrac{\gamma_{\mathsf{s}}}{\gamma_{{\mathsf{s}}'}}f_{{\mathsf{s}}'{\mathsf{s}}'}.$
  \end{Corollary}

  \begin{proof} Using \autoref{T:SeminormalForm}(b) and \autoref{L:Fhash},
    $f_{{\mathsf{s}}{\mathsf{s}}}^\#=\frac1{\gamma_{\mathsf{s}}}F_{\mathsf{s}}^\#=\frac1{\gamma_{\mathsf{s}}}F_{{\mathsf{s}}'}
    =\frac{\gamma_{{\mathsf{s}}'}}{\gamma_{\mathsf{s}}}f_{{\mathsf{s}}'{\mathsf{s}}'}$.
  \end{proof}

  \begin{Lemma}\label{L:futhash}
    Let ${\mathsf{s}},{\mathsf{u}}\in{\mathop{\rm Std}\nolimits}({\mathcal{P}_{n}}$ be standard tableaux such that
    ${\mathsf{u}}=(r,r+1){\mathsf{s}}$, for some integer~$r$ with $1\le r<n$. Then
    \[
    f_{{\mathsf{u}}{\mathsf{s}}}^\#=-\frac{\alpha_r({\mathsf{s}}')\gamma_{\mathsf{s}}}{\alpha_r({\mathsf{s}})\gamma_{{\mathsf{s}}'}}f_{{\mathsf{u}}'{\mathsf{s}}'}.
    \]
  \end{Lemma}

  \begin{proof}
    By \autoref{T:SeminormalForm}(a),
    $f_{{\mathsf{u}}{\mathsf{s}}}=\frac{1}{\alpha_r({\mathsf{s}})}\Big(T_r+\frac1{[\rho_r({\mathsf{s}})]}\Big)f_{{\mathsf{s}}{\mathsf{s}}}$.
    Recall that $T_r^\#=-tT_r^{-1}=-T_r+t-1$. Therefore, using
    \autoref{E:hash} and \autoref{C:fttHash} for the second equality,
    \begin{align*}
      f_{{\mathsf{u}}{\mathsf{s}}}^\#&=\frac{1}{\alpha_r({\mathsf{s}})}\Big(T_r
       +\frac1{[\rho_r({\mathsf{s}})]}\Big)^\#f_{{\mathsf{s}}{\mathsf{s}}}^\#\\
      &=\frac{\gamma_{\mathsf{s}}}{\alpha_r({\mathsf{s}})\gamma_{{\mathsf{s}}'}}\Big(-T_r+t-1
       +\frac1{[\rho_r({\mathsf{s}})]}\Big)f_{{\mathsf{s}}'{\mathsf{s}}'}\\
      &=-\frac{\gamma_{\mathsf{s}}}{\alpha_r({\mathsf{s}})\gamma_{{\mathsf{s}}'}}\Big(T_r
       -\frac{t^{\rho_r({\mathsf{s}})}}{[\rho_r({\mathsf{s}})]}\Big)f_{{\mathsf{s}}'{\mathsf{s}}'}\\
      &=-\frac{\gamma_{\mathsf{s}}}{\alpha_r({\mathsf{s}})\gamma_{{\mathsf{s}}'}}\Big(T_r
       +\frac1{[\rho_r({\mathsf{s}}')]}\Big)f_{{\mathsf{s}}'{\mathsf{s}}'},
    \end{align*}
    since
    $[\rho_r({\mathsf{s}})]=-t^{\rho_r({\mathsf{s}})}[-\rho_r({\mathsf{s}})]=-t^{\rho_r({\mathsf{s}})}[\rho_r({\mathsf{s}}')]$.
    Hence, the result follows by another application of \autoref{T:SeminormalForm}(a).
  \end{proof}

  By \autoref{T:SeminormalForm}(c) any $*$-seminormal basis is uniquely
  determined by a seminormal coefficient system and a choice of scalars
  $\set{\gamma_{{\mathsf{t}}^\lambda}|\lambda\in{\mathcal{P}_{n}}$. For completeness we
  determine these scalars for the seminormal basis
  $\set{f_{{\mathsf{s}}{\mathsf{t}}}^\#|({\mathsf{s}},{\mathsf{t}})\in{\mathop{\rm Std}\nolimits}^2({\mathcal{P}_{n}}}$.  Recall from
  \autoref{S:tableaux} that ${\mathsf{t}}_{\lambda}=({\mathsf{t}}^{\lambda'})'$ is the
  co-initial $\lambda$-tableau.

  \begin{Proposition}\label{P:AltCS}
    The seminormal basis $\set{f_{{\mathsf{s}}{\mathsf{t}}}^\#|({\mathsf{s}},{\mathsf{t}})\in{\mathop{\rm Std}\nolimits}({\mathcal{P}_{n}}}$ of
    ${{\mathscr{H}}_t^{\mathcal{K}}({\mathfrak{S}_{n}}}$ is the seminormal basis determined by the seminormal
    coefficient system
    \[
    \set{-\alpha_r({\mathsf{s}}')|{\mathsf{s}}\in{\mathop{\rm Std}\nolimits}({\mathcal{P}_{n}}\text{ and }1\le r<n}
    \]
    together with the $\gamma$-coefficients
    $\set{\gamma_{{\mathsf{t}}_\lambda}|\lambda\in{\mathcal{P}_{n}}$.  That is, if
    $({\mathsf{s}},{\mathsf{t}})\in{\mathop{\rm Std}\nolimits}^2({\mathcal{P}_{n}}$ then
    \[
    L_kf_{{\mathsf{s}}{\mathsf{t}}}^\#=[c_r({\mathsf{s}}')]f_{{\mathsf{s}}{\mathsf{t}}}^\#,\quad
    f_{{\mathsf{s}}{\mathsf{t}}}^\#L_k=[c_r({\mathsf{t}}')]f_{{\mathsf{s}}{\mathsf{t}}}^\#\quad\text{and}\quad
    T_r f_{{\mathsf{s}}{\mathsf{t}}}^\# = -\alpha_r({\mathsf{s}})f_{{\mathsf{u}}{\mathsf{t}}}^\#-\frac1{[\rho_r({\mathsf{s}}')]}f_{{\mathsf{s}}{\mathsf{t}}}^\#,
    \]
    where ${\mathsf{u}}=(r,r+1){\mathsf{s}}$, $1\le k\le n$ and $1\le r<n$. Moreover,
    $f_{{\mathsf{s}}{\mathsf{t}}}^\#f_{{\mathsf{u}}{\mathsf{v}}}^\#=\delta_{{\mathsf{t}}{\mathsf{u}}}\gamma_{\mathsf{t}} f_{{\mathsf{s}}{\mathsf{v}}}^\#$,
    for $({\mathsf{s}},{\mathsf{t}}),({\mathsf{u}},{\mathsf{v}})\in{\mathop{\rm Std}\nolimits}^2({\mathcal{P}_{n}}$.
  \end{Proposition}

  \begin{proof}
    If $1\le k\le n$ then
    $L_kf_{{\mathsf{s}}{\mathsf{t}}}^\#=[c_r({\mathsf{s}}')]f_{{\mathsf{s}}{\mathsf{t}}}^\#$ and
    $f_{{\mathsf{s}}{\mathsf{t}}}^\#L_k=[c_r({\mathsf{t}}')]f_{{\mathsf{s}}{\mathsf{t}}}^\#$ by \autoref{L:LFHash}.
    Using \autoref{T:SeminormalForm}(a),
    \begin{align*}
      T_rf_{{\mathsf{s}}{\mathsf{t}}}^\# &= \Big(T_r^\# f_{{\mathsf{s}}{\mathsf{t}}}\Big)^\#
      = \Big((-T_r+t-1)f_{{\mathsf{s}}{\mathsf{t}}}\Big)^\#\\
      &=\Big(-\alpha_r({\mathsf{s}})f_{{\mathsf{u}}{\mathsf{t}}} +(t-1+\frac1{[\rho_r({\mathsf{s}})]})f_{{\mathsf{s}}{\mathsf{t}}}\Big)^\#\\
      &=\Big(-\alpha_r({\mathsf{s}})f_{{\mathsf{u}}{\mathsf{t}}}-\frac1{[\rho_r({\mathsf{s}}')]}f_{{\mathsf{s}}{\mathsf{t}}}\Big)^\#\\
      &=-\alpha_r({\mathsf{s}})f_{{\mathsf{u}}{\mathsf{t}}}^\# -\frac1{[\rho_r({\mathsf{s}}')]}f_{{\mathsf{s}}{\mathsf{t}}}^\#.
    \end{align*}
    Similarly,
    $f_{{\mathsf{s}}{\mathsf{t}}}^\#f_{{\mathsf{u}}{\mathsf{v}}}^\#=(f_{{\mathsf{s}}{\mathsf{t}}}f_{{\mathsf{u}}{\mathsf{v}}})^\#=\delta_{{\mathsf{t}}{\mathsf{u}}}\gamma_{\mathsf{t}} f_{{\mathsf{s}}{\mathsf{v}}}^\#$.
    By \autoref{L:LFHash} $f_{{\mathsf{s}}{\mathsf{t}}}^\#\in H_{{\mathsf{s}}'{\mathsf{t}}'}$, so the $\alpha$-coefficient
    corresponding to~$f_{{\mathsf{s}}{\mathsf{t}}}^\#$ is naturally indexed by~${\mathsf{s}}'$ (and not by~${\mathsf{s}}$).
    Similarly, the labelling for the $\gamma$-coefficients involves conjugation
    because $F_{\mathsf{t}}=\frac1{\gamma_{{\mathsf{t}}'}}f_{{\mathsf{t}}'{\mathsf{t}}'}^\#$ by \autoref{C:fttHash}. Hence,
    the result follows by \autoref{T:SeminormalForm}.
  \end{proof}

  \subsection{Idempotent subrings and KLR generators}\label{S:KLRGenerators}
  We are almost ready to introduce the generators of ${{\mathscr{H}}^{\mathcal{O}}_{t}} that we need to
  prove \autoref{T:MainTheorem}. This section defines roughly half of
  these generators. The definition of these elements involves lifting idempotents from the
  non-semisimple case to the semisimple case and to do this we need to
  place additional constraints upon the rings that we work over.

  If ${\mathcal{O}}$ is a ring let ${\mathcal{J}}={\mathcal{J}}({\mathcal{O}})$ be the Jacobson radical of~${\mathcal{O}}$.

  \begin{Definition}[\protect{\cite[Definition~4.1]{HuMathas:SeminormalQuiver}}]
    \label{D:idempotentSub}
    Suppose that ${\mathcal{O}}$ is a subring of a field~${\mathcal{K}}$ and that $t\in{\mathcal{O}}^\times$.
    The pair $({\mathcal{O}},t)$ is an \textbf{$e$-idempotent subring} of ${\mathcal{K}}$ if:
    \begin{enumerate}
      \item The Poincar\'e polynomial $P_{\mathscr{H}}(t)$ is a non-zero element of~${\mathcal{O}}$,
      \item If $k\notin e{\mathbb{Z}}$ then $[k]_t$ is invertible in ${\mathcal{O}}$,
      \item If $k\in e{\mathbb{Z}}$ then $[k]_t\in \mathcal{J}({\mathcal{O}})$.
    \end{enumerate}
  \end{Definition}

  Condition~(a) ensures that \autoref{L:separation} and
  \autoref{T:SeminormalForm} apply and, in particular, that ${{\mathscr{H}}_t^{\mathcal{K}}({\mathfrak{S}_{n}}}$ has a
  seminormal basis $\set{f_{{\mathsf{s}}{\mathsf{t}}}}$. As discussed in
  \cite[Example~4.2]{HuMathas:GradedCellular}, when considering the Hecke
  algebra ${{\mathscr{H}}_\xi({\mathfrak{S}_{n}}} defined over the field~$F$ with parameter $\xi\in
  F^\times$, one natural choice of idempotent subring is to let $x$ be an
  indeterminate over~$F$ and set ${\mathcal{K}}=F(x)$, $t=x+\xi$ and
  ${\mathcal{O}}=F[x,x^{-1}]_{(x)}$. Note that ${\mathfrak{m}}=x{\mathcal{O}}$ is the unique maximal ideal of~${\mathcal{O}}$
  and that $({\mathcal{K}},{\mathcal{O}},F)$ is a modular system with ${{\mathscr{H}}_\xi({\mathfrak{S}_{n}}}{{\mathscr{H}}^{\mathcal{O}}_{t}}_{\mathcal{O}}
  F$, where $F$ is considered as an ${\mathcal{O}}$-module by letting~$x$ act as
  multiplication by~$0$.

  The hash involution $\#$ from \autoref{E:hash} is well-defined on ${{\mathscr{H}}^{\mathcal{O}}_{t}}.
  Let ${{\mathscr{H}}^{\mathcal{O}}_{t}}{\mathfrak{A}_{n}}$ the ${\mathcal{O}}$-subalgebra of $\#$-fixed points in~${{\mathscr{H}}^{\mathcal{O}}_{t}}.
  Then ${{\mathscr{H}}_\xi({\mathfrak{A}_{n}}}{{\mathscr{H}}^{\mathcal{O}}_{t}}{\mathfrak{A}_{n}}\otimes_{\mathcal{O}} F$.

  Recall that if ${\mathbf{i}}\in I^n$ then
  ${\mathop{\rm Std}\nolimits}({\mathbf{i}})=\set{{\mathsf{t}}\in{\mathop{\rm Std}\nolimits}({\mathcal{P}_{n}}|\operatorname{res}({\mathsf{t}})={\mathbf{i}}}$. Using an idea that
  goes back to Murphy~\cite{M:Nak}, the \textbf{${\mathbf{i}}$-residue idempotent}
  is defined to be the element
  \begin{equation*}\label{E:ResidueIdempotent}
    {f^{\mathcal{O}}_{\mathbf{i}}} \sum_{{\mathsf{t}}\in{\mathop{\rm Std}\nolimits}({\mathbf{i}})}\frac1{\gamma_{\mathsf{t}}}f_{{\mathsf{t}}{\mathsf{t}}}
    = \sum_{{\mathsf{t}}\in{\mathop{\rm Std}\nolimits}({\mathbf{i}})}F_{\mathsf{t}}.
  \end{equation*}
  By \autoref{T:SeminormalForm}(b), if ${\mathsf{s}}\in{\mathop{\rm Std}\nolimits}({\mathbf{j}})$ and ${\mathsf{t}}\in{\mathop{\rm Std}\nolimits}({\mathbf{k}})$
  are tableaux of the same shape then ${f^{\mathcal{O}}_{\mathbf{i}}}_{{\mathsf{s}}{\mathsf{t}}}=\delta_{{\mathbf{i}}{\mathbf{j}}}f_{{\mathsf{s}}{\mathsf{t}}}$ and
  $f_{{\mathsf{s}}{\mathsf{t}}}{f^{\mathcal{O}}_{\mathbf{i}}}\delta_{{\mathbf{i}}{\mathbf{k}}}f_{{\mathsf{s}}{\mathsf{t}}}$, for ${\mathbf{i}},{\mathbf{j}},{\mathbf{k}}\in I^n$.

  By definition, ${f^{\mathcal{O}}_{\mathbf{i}}}{\mathscr{H}}_t^{\mathcal{K}}({\mathfrak{S}_{n}}$ but, in fact,  ${f^{\mathcal{O}}_{\mathbf{i}}}{{\mathscr{H}}^{\mathcal{O}}_{t}}.

  \begin{Lemma}\label{L:foIdempotents}
    Suppose that ${\mathbf{i}}\in I^n$. Then ${f^{\mathcal{O}}_{\mathbf{i}}}{{\mathscr{H}}^{\mathcal{O}}_{t}} and $({f^{\mathcal{O}}_{\mathbf{i}}}^\#={f^{\mathcal{O}}_{-{\mathbf{i}}}}.
  \end{Lemma}

  \begin{proof}
    Since $({\mathcal{O}},t)$ is an idempotent subring, ${f^{\mathcal{O}}_{\mathbf{i}}}{{\mathscr{H}}^{\mathcal{O}}_{t}} by
    \cite[Lemma~4.5]{HuMathas:SeminormalQuiver}.  To prove that
    $({f^{\mathcal{O}}_{\mathbf{i}}}^\#={f^{\mathcal{O}}_{-{\mathbf{i}}}} first observe that ${\mathsf{s}}\in{\mathop{\rm Std}\nolimits}({\mathbf{i}})$ if and only if
    ${\mathsf{s}}'\in{\mathop{\rm Std}\nolimits}(-{\mathbf{i}})$. Therefore, by \autoref{L:Fhash},
    \[
    ({f^{\mathcal{O}}_{\mathbf{i}}}^\#=\sum_{{\mathsf{s}}\in{\mathop{\rm Std}\nolimits}({\mathbf{i}})}F_{\mathsf{s}}^\#=\sum_{{\mathsf{s}}\in{\mathop{\rm Std}\nolimits}({\mathbf{i}})}F_{{\mathsf{s}}'}={f^{\mathcal{O}}_{-{\mathbf{i}}}}
    as claimed.
  \end{proof}

  Following \cite{BK:GradedKL,HuMathas:SeminormalQuiver}, define
  $M_r=1-L_r+tL_{r+1}$, for $1\le r\le n$.  If $({\mathsf{s}},{\mathsf{t}})\in{\mathop{\rm Std}\nolimits}^2({\mathcal{P}_{n}}$ then
  it follows easily using \autoref{T:SeminormalForm}(a) that
  \begin{equation}\label{E:Mr}
    M_r f_{{\mathsf{s}}{\mathsf{t}}}=t^{c_r({\mathsf{s}})}[1-\rho_r({\mathsf{s}})]f_{{\mathsf{s}}{\mathsf{t}}}.
    
    
  \end{equation}
  The next result says that these elements are invertible when projected
  onto certain residue idempotents ${f^{\mathcal{O}}_{\mathbf{i}}}, for ${\mathbf{i}}\in I^n$.

  \begin{Corollary}[\protect{\cite[Corollary~4.8]{HuMathas:SeminormalQuiver}}]
    \label{C:MrInvert}
    Suppose that ${\mathbf{i}}\in I^n$ and $i_r\ne i_{r+1}+1$, for $1\le r<n$ and . Then
    \[{\displaystyle\sum}_{{\mathsf{s}}\in{\mathop{\rm Std}\nolimits}({\mathbf{i}})}\frac{t^{-c_r({\mathsf{t}})}}{[1-\rho_r({\mathsf{s}})]}F_{\mathsf{s}}\in{{\mathscr{H}}^{\mathcal{O}}_{t}}\]
  \end{Corollary}

  In view of \autoref{C:MrInvert}, if $i_r\ne i_{r+1}+1$ define the formal symbol
  \begin{align*}
    \frac{1}{M_r}{f^{\mathcal{O}}_{\mathbf{i}}}={\displaystyle\sum}_{{\mathsf{s}}\in{\mathop{\rm Std}\nolimits}({\mathbf{i}})}\frac{t^{-c_r({\mathsf{t}})}}{[1-\rho_r({\mathsf{s}})]}F_{\mathsf{s}}.
    
    
    
    
    \end{align*}
    This abuse of notation is justified because
    $\frac1{M_r}{f^{\mathcal{O}}_{\mathbf{i}}}_r={f^{\mathcal{O}}_{\mathbf{i}}} 
    by \autoref{E:Mr}. We will use these elements to define the
    KLR-generators of~${{\mathscr{H}}^{\mathcal{O}}_{t}} that we use to prove \autoref{T:Main}.

    The results of~\cite{HuMathas:SeminormalQuiver} depend upon choosing an
    arbitrary  section of the natural quotient map ${\mathbb{Z}}\twoheadrightarrow{\mathbb{Z}}/e{\mathbb{Z}}$. In
    this paper we are far less flexible and need to use a particular
    section of this map. If
    $i\in I$ let ${\hat\imath}\ge0$ be the smallest non-negative integer such that
    $i={\hat\imath}+e{\mathbb{Z}}$. (If $e=\infty$ set $\hat i=i$.) This defines an
    embedding $I\hookrightarrow{\mathbb{Z}}; i\mapsto{\hat\imath}$.  For ${\mathbf{i}}\in I^n$ set
    $\rho_r({\mathbf{i}})={\hat\imath}_r-{\hat\imath}_{r+1}$, for $1\le r<n$.

    By \autoref{T:SeminormalForm}, the identity element of~${{\mathscr{H}}^{\mathcal{O}}_{t}} can
    be written as $1=\sum_{{\mathbf{i}}\in I^n}{f^{\mathcal{O}}_{\mathbf{i}}}. So if $h\in{{\mathscr{H}}^{\mathcal{O}}_{t}} then
    $h=\sum_{{\mathbf{i}}\in I^n} h{f^{\mathcal{O}}_{\mathbf{i}}} is uniquely determined by its
    projection onto the idempotents ${f^{\mathcal{O}}_{\mathbf{i}}}.

    \begin{Definition}[\protect{\cite[Definition~4.14]{HuMathas:SeminormalQuiver}}]
      \label{D:psip}
      Fix an integer $1\le r<n$ and define the element
      ${\psi^+}_r=\sum_{{\mathbf{i}}\in I^n}{\psi^+}_r{f^{\mathcal{O}}_{\mathbf{i}}} by
      \[
      {\psi^+}_r{f^{\mathcal{O}}_{\mathbf{i}}}\begin{cases}
        (1+T_r)\frac{t^{{\hat\imath}_r}}{M_r}{f^{\mathcal{O}}_{\mathbf{i}}}&\text{if }i_r=i_{r+1},\\
        (T_rL_r-L_rT_r)t^{-{\hat\imath}_r}{f^{\mathcal{O}}_{\mathbf{i}}}&\text{if }i_r\to i_{r+1},\\
        (T_rL_r-L_rT_r)\frac{1}{M_r}{f^{\mathcal{O}}_{\mathbf{i}}}&\text{otherwise}.
      \end{cases}
      \]
      For $1\le s\le n$ define ${y^+}_s=\sum_{{\mathbf{i}}\in I^n}t^{-{\hat\imath}_s}(L_s-[{\hat\imath}_s]){f^{\mathcal{O}}_{\mathbf{i}}}.
    \end{Definition}

    Recall from \autoref{S:GeneratorsRelations} that ${Q^\varepsilon_n}=Q^+_n/{\sim}$.
    For $\alpha\in Q^+$ define
    \[
    {{\mathscr{H}}^{\mathcal{O}}_{\alpha}}{{\mathscr{H}}^{\mathcal{O}}_{t}}[\alpha],\qquad\text{where }\quad
    {f^{\mathcal{O}}_{\alpha}}\sum_{{{\mathbf{i}}}\in I^\alpha}{f^{\mathcal{O}}_{\mathbf{i}}}
    \]
    For $\gamma\in{Q^\varepsilon_n}$ set
    ${f^{\mathcal{O}}_{\gamma}}\sum_{\alpha\in\gamma}{f^{\mathcal{O}}_{\alpha}} and set
    ${{\mathscr{H}}^{\mathcal{O}}_{\gamma}}\bigoplus_{\alpha\in\gamma}{{\mathscr{H}}^{\mathcal{O}}_{\alpha}}{{\mathscr{H}}^{\mathcal{O}}_{t}}[\gamma]$.

    \begin{Proposition}\label{P:HOdecomp}
      Suppose that $({\mathcal{O}},t)$ is an idempotent subring. Then ${f^{\mathcal{O}}_{\gamma}} is
      a central idempotent in ${{\mathscr{H}}^{\mathcal{O}}_{t}} and
      ${{\mathscr{H}}^{\mathcal{O}}_{t}}\displaystyle\bigoplus_{\gamma\in{Q^\varepsilon_n}}{{\mathscr{H}}^{\mathcal{O}}_{\gamma}}.
    \end{Proposition}

      \begin{proof}
        By \autoref{L:foIdempotents}, ${f^{\mathcal{O}}_{\gamma}}{{\mathscr{H}}^{\mathcal{O}}_{t}} and it follows from
        \autoref{T:SeminormalForm} that ${f^{\mathcal{O}}_{\gamma}} is a central
        idempotent and that $1=\sum_{\gamma\in {Q^\varepsilon_n}}{f^{\mathcal{O}}_{\gamma}}. Hence,
        ${{\mathscr{H}}^{\mathcal{O}}_{\gamma}}{f^{\mathcal{O}}_{\gamma}}[\gamma]$ is a subalgebra of~${{\mathscr{H}}^{\mathcal{O}}_{t}} and
        ${{\mathscr{H}}^{\mathcal{O}}_{t}}\bigoplus_{\gamma\in {Q^\varepsilon_n}}{{\mathscr{H}}^{\mathcal{O}}_{\gamma}}.
      \end{proof}

      By \cite{JM:Schaper}, for $\alpha\in Q^+_n$ the algebras
    ${{\mathscr{H}}^{\mathcal{O}}_{\alpha}}_{\mathcal{O}} F$ are indecomposable two-sided
    ideals of~${{\mathscr{H}}_\xi({\mathfrak{S}_{n}}}. Later we need the counterpart of \autoref{C:RAnBlocks}
    for ${{\mathscr{H}}_\xi({\mathfrak{A}_{n}}}. If $\gamma\in{Q^\varepsilon_n}$ then $({f^{\mathcal{O}}_{\gamma}}^\#={f^{\mathcal{O}}_{\gamma}} by
    \autoref{L:foIdempotents}. Therefore, $\#$ restricts to an automorphism
    of~${{\mathscr{H}}^{\mathcal{O}}_{\gamma}}. Define
    \begin{equation}\label{E:HAnBlocks}
      {{\mathscr{H}}^{\mathcal{O}}_{t}}{\mathfrak{A}_{n}}_\gamma=\bigl({{\mathscr{H}}^{\mathcal{O}}_{\gamma}})^\#
                     =\set{h\in{{\mathscr{H}}^{\mathcal{O}}_{\gamma}}h^\#=h}
                     ={{\mathscr{H}}^{\mathcal{O}}_{t}}{\mathfrak{A}_{n}}{f^{\mathcal{O}}_{\gamma}}
    \end{equation}
    As $1=\sum_\gamma{f^{\mathcal{O}}_{\gamma}}, \autoref{P:HOdecomp} immediately implies
    the following.

    \begin{Corollary}\label{C:HAnDecomp}
      Suppose that $({\mathcal{O}},t)$ is an idempotent subring. Then
      \[{{\mathscr{H}}^{\mathcal{O}}_{t}}{\mathfrak{A}_{n}}=\bigoplus_{\gamma\in{Q^\varepsilon_n}}{{\mathscr{H}}^{\mathcal{O}}_{t}}{\mathfrak{A}_{n}}_\gamma.\]
    \end{Corollary}

    The subalgebra ${{\mathscr{H}}^{\mathcal{O}}_{t}}{\mathfrak{A}_{n}}_\gamma$ is a block of~${{\mathscr{H}}^{\mathcal{O}}_{t}}{\mathfrak{A}_{n}}$ in the
    sense that it is a two-sided ideal and a direct summand
    of~${{\mathscr{H}}^{\mathcal{O}}_{t}}{\mathfrak{A}_{n}}$.  Let $F$ be a field that is an ${\mathcal{O}}$-algebra and set
    ${{\mathscr{H}}_\xi^{F}({\mathfrak{A}_{n}}}\gamma={{\mathscr{H}}^{\mathcal{O}}_{t}}{\mathfrak{A}_{n}}_\gamma\otimes_{\mathcal{O}} F$. Then ${{\mathscr{H}}_\xi^{F}({\mathfrak{A}_{n}}}\gamma$ is
    almost always indecomposable. See \autoref{T:RAnBlocks} for
    the precise statement.

      \begin{Theorem}[\protect{        Hu-Mathas~\cite[Theorem~A]{HuMathas:SeminormalQuiver}}]
        \label{T:HMGradedIso}
        Suppose that  $\mathcal{K}$ is a field, $\gamma\in Q^+_n$ and that
        $({\mathcal{O}},t)$ an $e$-idempotent subring of $\mathcal{K}$, where $e\ne2$.
        As an ${\mathcal{O}}$-algebra, the Iwahori-Hecke algebra ${{\mathscr{H}}^{\mathcal{O}}_{\gamma}} is generated
        by the elements
        \[
        \set{{f^{\mathcal{O}}_{\mathbf{i}}}{{\mathbf{i}}}\in I^\gamma}\cup \set{{\psi^+}_r|1\leq r<n}\cup\set{{y^+}_s|1\leq s\leq n}
        \]
        subject to the relations
        {\setlength{\abovedisplayskip}{2pt}
        \setlength{\belowdisplayskip}{1pt}
        \begin{alignat*}{3}
          ({y^+}_1)^{(\Lambda_0,\alpha_{i_1})}{f^{\mathcal{O}}_{\mathbf{i}}}=0,
          &\qquad {f^{\mathcal{O}}_{\mathbf{i}}}[{\mathbf{j}}]&= \delta_{{{\mathbf{i}}}{{\mathbf{j}}}}{f^{\mathcal{O}}_{\mathbf{i}}}
          &\qquad\sum_{{{\mathbf{i}}}\in I^\gamma}{f^{\mathcal{O}}_{\mathbf{i}}}= 1\\
          {y^+}_r {f^{\mathcal{O}}_{\mathbf{i}}}= {f^{\mathcal{O}}_{\mathbf{i}}}_r,&\qquad {\psi^+}_r {f^{\mathcal{O}}_{\mathbf{i}}}= f_{s_r\cdot{{\mathbf{i}}}}^{\mathcal{O}}{\psi^+}_r,&\qquad
          {y^+}_r {y^+}_s&= {y^+}_s {y^+}_r
        \end{alignat*}
        \begin{alignat*}{2}
          {\psi^+}_r {y^+}_{r+1}{f^{\mathcal{O}}_{\mathbf{i}}}= ({y^+}_r {\psi^+}_r+\delta_{i_r i_{r+1}}){f^{\mathcal{O}}_{\mathbf{i}}}
          &\qquad{y^+}_{r+1}{\psi^+}_r {f^{\mathcal{O}}_{\mathbf{i}}}= ({\psi^+}_r {y^+}_r+\delta_{i_r i_{r+1}}){f^{\mathcal{O}}_{\mathbf{i}}}{alignat*}
        \begin{align*}
          {\psi^+}_r {y^+}_s&= {y^+}_s{\psi^+}_r,   &\text{if $s\neq r,r+1$,}\\
          {\psi^+}_r{\psi^+}_s&= {\psi^+}_s{\psi^+}_r,&\text{if $\left|r-s\right|>1$,}
        \end{align*}
        \begin{align*}
          ({\psi^+}_r)^2{f^{\mathcal{O}}_{\mathbf{i}}}= \begin{cases}
            ({y^{\<{1+\rho_r({\mathbf{i}})}\>}}_r-{y^+}_{r+1}){f^{\mathcal{O}}_{\mathbf{i}}}&\text{if }i_r\rightarrow i_{r+1},\\
            ({y^{\<{1-\rho_r({\mathbf{i}})}\>}}_{r+1}-{y^+}_r){f^{\mathcal{O}}_{\mathbf{i}}}&\text{if }i_r\leftarrow i_{r+1},\\
            0,&\text{if }i_r=i_{r+1},\\
            {f^{\mathcal{O}}_{\mathbf{i}}}&\text{otherwise,}
          \end{cases}\\
          {\psi^+}_r{\psi^+}_{r+1}{\psi^+}_r {f^{\mathcal{O}}_{\mathbf{i}}}= \begin{cases}
            ({\psi^+}_{r+1}{\psi^+}_r{\psi^+}_{r+1}-t^{1+\rho_r({\mathbf{i}})}){f^{\mathcal{O}}_{\mathbf{i}}}
            &\text{if }i_r=i_{r+2}\rightarrow i_{r+1},\\
            ({\psi^+}_{r+1}{\psi^+}_r{\psi^+}_{r+1}+1){f^{\mathcal{O}}_{\mathbf{i}}}&\text{if }i_r=i_{r+2}\leftarrow i_{r+1},\\
            {\psi^+}_{r+1}{\psi^+}_r{\psi^+}_{r+1}{f^{\mathcal{O}}_{\mathbf{i}}}&\text{otherwise,}
          \end{cases}
        \end{align*}
        }\noindent        where ${y^{\<{d}\>}}_r{f^{\mathcal{O}}_{\mathbf{i}}}(t^d{y^+}_r-[d]){f^{\mathcal{O}}_{\mathbf{i}}} for $d\in {\mathbb{Z}}$, and
        $\rho_r({\mathbf{i}})={\hat\imath}_r-{\hat\imath}_{r+1}$, for ${{\mathbf{i}}},{{\mathbf{j}}}\in I^\gamma$ and all
        admissible $r,s$.
      \end{Theorem}

      \begin{proof}
        By \cite[Theorem~A]{HuMathas:GradedCellular}, this result holds for the
        algebras ${{\mathscr{H}}^{\mathcal{O}}_{\alpha}}, for $\alpha\in Q^+_n$.
        As~${{\mathscr{H}}^{\mathcal{O}}_{\gamma}}\bigoplus_{\alpha\in\gamma}{{\mathscr{H}}^{\mathcal{O}}_{\alpha}} this gives the
        result for~${{\mathscr{H}}^{\mathcal{O}}_{\gamma}}. The proof of
        \cite[Theorem~4]{HuMathas:GradedCellular} assumes that $e<\infty$ (or
        $e=0$ in the notation of \cite{HuMathas:SeminormalQuiver}), however,
        this assumption is only needed for
        \cite[(3.2)]{HuMathas:SeminormalQuiver} which is automatic in
        level~$1$ when $\Lambda=\Lambda_0$. Alternatively, as in
        \cite[Corollary~2.15]{HuMathas:SeminormalQuiver}, it is enough to
        consider the case when~$n<e<\infty$.
      \end{proof}

      As above, let ${\mathfrak{m}}$ be a maximal ideal of ${\mathcal{O}}$ and set ${\mathbb{F}}={\mathcal{O}}/{\mathfrak{m}}$ and
    $\xi=t+{\mathfrak{m}}\in{\mathbb{F}}$ and let~${{\mathscr{H}}_\xi({\mathfrak{S}_{n}}} be the Iwahori-Hecke algebra over~${\mathbb{F}}$
    with parameter~$\xi$.  Then ${{\mathscr{H}}_\xi^{{\mathbb{F}}}({\mathfrak{S}_{n}}}{{\mathscr{H}}^{\mathcal{O}}_{t}}_O{\mathbb{F}}$. The definition of
    an $e$-idempotent subring ensures that~$\xi$ has quantum
    characteristic~$e$. Comparing the relations in \autoref{D:klr} with
    those in \autoref{T:HMGradedIso}, modulo~${\mathfrak{m}}$, there is an  algebra
    isomorphism $\theta{\,{:}\,{{\mathscr{R}^{{\mathbb{F}}}_e({\mathfrak{S}_{n}}}\!\longrightarrow\!{{\mathscr{H}}_\xi^{F}({\mathfrak{S}_{n}}}[{\mathbb{F}}]$ determined by
    \[
        \psi_r\mapsto{\psi^+}_r\otimes1_{\mathbb{F}},\quad
        y_r\mapsto{y^+}_r\otimes1_{\mathbb{F}} \quad\text{and}\quad
        e({\mathbf{i}})\mapsto {f^{\mathcal{O}}_{\mathbf{i}}}1_{\mathbb{F}},
    \]
    for all admissible $r$ and ${\mathbf{i}}\in I^n$. Unfortunately, as the next
    example shows, we cannot use~$\theta$ to prove \autoref{T:Main}
    because $\theta\circ{\mathtt{sgn}}\ne\#\circ\theta$.

    \begin{Example}\label{Ex:BadBKRestriction}
      Suppose that $n=3$, $\Lambda=\Lambda_0$ and work over ${\mathbb{F}}_3$, the field with
      three elements. By \autoref{Ex:OS3},
      \[
      \set[\big]{e(012),  e(021),  y_3 e(012),  y_3e(021),  \psi_2e(012),
      \psi_2e(021)}
      \]
      is a basis of ${\mathscr{R}_e({\mathfrak{S}_{3}}}{\mathbb{F}}_3{\mathfrak{S}_{3}} and, by \autoref{Ex:A3Basis},
      \[
      \set[\big]{e(012)+e(021),\psi_2(e(012)-e(021)), y_3(e(012)-e(021))}
      \]
      is a basis of $\RAn[3]$. Let $\theta{\,{:}\,{{\mathscr{R}_e({\mathfrak{S}_{3}}}\!\longrightarrow\!{\mathbb{F}}}_3{\mathfrak{S}_{3}} be the
      Brundan-Kleshchev isomorphism induced by \autoref{T:HMGradedIso}. With
      some work it is possible to show that:
      \begin{align*}
        \theta\big(e(012)+e(021)\big)&= 1\\
        \theta\big(y_3(e(012)-e(021))\big)&= 1+s_1s_2+s_2s_1\\
        \theta\big(\psi_2(e(012)-e(021))\big)&= s_2+2s_1s_2s_1.
      \end{align*}
      In particular, $\theta$ does not restrict to an isomorphism
      between $\RAn[3]$ and ${\mathbb{F}}_3{\mathfrak{A}_{3}}.
    \end{Example}

    To obtain an isomorphism ${\mathscr{R}_e({\mathfrak{S}_{n}}}{{\mathscr{H}}_\xi({\mathfrak{A}_{n}}}, which restricts to an
    isomorphism $\RAn\to{{\mathscr{H}}_\xi({\mathfrak{A}_{n}}}, we modify the generators of~${{\mathscr{H}}^{\mathcal{O}}_{t}}
    given in \autoref{T:HMGradedIso}.

    \begin{Definition}\label{D:psim}
      Let ${\psi^-}_r=({\psi^+}_r)^\#$ and ${y^-}_s=({y^+}_s)^\#$, for
      $1\le r<n$ and $1\le s\le n$.
    \end{Definition}

    Notice that $({f^{\mathcal{O}}_{\mathbf{i}}}^\#={f^{\mathcal{O}}_{-{{\mathbf{i}}}}} by \autoref{L:foIdempotents}.
    Therefore, since $\#$ is an automorphism, the elements
    $\set{{\psi^-}_r,{y^-}_s,{f^{\mathcal{O}}_{\mathbf{i}}}$ generate ${{\mathscr{H}}^{\mathcal{O}}_{t}}, subject to essentially the
    same relations as those given in \autoref{T:HMGradedIso} except that
    ${\mathbf{i}}$ should be replaced with $-{\mathbf{i}}$. As we need this result below we
    state it in full for easy reference.

    \begin{Corollary}\label{C:HMGradedIso}
      Suppose that $e>2$, $\gamma\in Q^+_n$ and  $n\ge0$. Let $({\mathcal{O}},t)$ be an $e$-idempotent subring of~${\mathcal{K}}$. Then ${{\mathscr{H}}^{\mathcal{O}}_{\gamma}} is generated as an ${\mathcal{O}}$-algebra by the elements
      \[
      \set{{\psi^-}_r|1\leq r<n}\cup\set{{y^-}_s|1\leq s\leq n}\cup \set{{f^{\mathcal{O}}_{\mathbf{i}}}{{\mathbf{i}}}\in I^\gamma}
      \]
      subject to the relations
      {\setlength{\abovedisplayskip}{2pt}
      \setlength{\belowdisplayskip}{1pt}
      \begin{alignat*}{3}
        ({y^-}_1)^{(\Lambda_0,\alpha_{i_1})}{f^{\mathcal{O}}_{\mathbf{i}}}=0,
        &\qquad {f^{\mathcal{O}}_{\mathbf{i}}}[{\mathbf{j}}]&= \delta_{{{\mathbf{i}}}{{\mathbf{j}}}}{f^{\mathcal{O}}_{\mathbf{i}}}
        &\qquad\sum_{{{\mathbf{i}}}\in I^\gamma}{f^{\mathcal{O}}_{\mathbf{i}}}= 1\\
        {y^-}_r {f^{\mathcal{O}}_{\mathbf{i}}}= {f^{\mathcal{O}}_{\mathbf{i}}}_r,&\qquad {\psi^-}_r {f^{\mathcal{O}}_{\mathbf{i}}}= f_{s_r\cdot{{\mathbf{i}}}}^{\mathcal{O}}{\psi^-}_r,&\qquad
        {y^-}_r {y^-}_s&= {y^-}_s {y^-}_r
      \end{alignat*}
      \begin{alignat*}{2}
        {\psi^-}_r {y^-}_{r+1}{f^{\mathcal{O}}_{\mathbf{i}}}= ({y^-}_r {\psi^-}_r+\delta_{i_r i_{r+1}}){f^{\mathcal{O}}_{\mathbf{i}}}
        &\qquad{y^-}_{r+1}{\psi^-}_r {f^{\mathcal{O}}_{\mathbf{i}}}= ({\psi^-}_r {y^-}_r+\delta_{i_r i_{r+1}}){f^{\mathcal{O}}_{\mathbf{i}}}{alignat*}
      \begin{align*}
        {\psi^-}_r {y^-}_s&= {y^-}_s{\psi^-}_r,   &\text{if $s\neq r,r+1$,}\\
        {\psi^-}_r{\psi^-}_s&= {\psi^-}_s{\psi^-}_r,&\text{if $\left|r-s\right|>1$,}
      \end{align*}
      \begin{align*}
        ({\psi^-}_r)^2{f^{\mathcal{O}}_{\mathbf{i}}}= \begin{cases}
          ({y^{\<{1-\rho_r({\mathbf{i}})}\>}}_r-{y^-}_{r+1}){f^{\mathcal{O}}_{\mathbf{i}}}&\text{if }i_r\leftarrow i_{r+1},\\
          ({y^{\<{1+\rho_r({\mathbf{i}})}\>}}_{r+1}-{y^-}_r){f^{\mathcal{O}}_{\mathbf{i}}}&\text{if }i_r\rightarrow i_{r+1},\\
          0,&\text{if }i_r=i_{r+1},\\
          {f^{\mathcal{O}}_{\mathbf{i}}}&\text{otherwise,}
        \end{cases}\\
        {\psi^-}_r{\psi^-}_{r+1}{\psi^-}_r {f^{\mathcal{O}}_{\mathbf{i}}}= \begin{cases}
          ({\psi^-}_{r+1}{\psi^-}_r{\psi^-}_{r+1}-t^{1-\rho_r({\mathbf{i}})}){f^{\mathcal{O}}_{\mathbf{i}}}
          &\text{if }i_r=i_{r+2}\leftarrow i_{r+1},\\
          ({\psi^-}_{r+1}{\psi^-}_r{\psi^-}_{r+1}+1){f^{\mathcal{O}}_{\mathbf{i}}}&\text{if }i_r=i_{r+2}\rightarrow i_{r+1},\\
          {\psi^-}_{r+1}{\psi^-}_r{\psi^-}_{r+1}{f^{\mathcal{O}}_{\mathbf{i}}}&\text{otherwise,}
        \end{cases}
      \end{align*}
      }\noindent      where ${y^{\<{d}\>}}_r{f^{\mathcal{O}}_{\mathbf{i}}}(t^d{y^-}_r-[d]){f^{\mathcal{O}}_{\mathbf{i}}}, for all $d\in {\mathbb{Z}}$, for
      ${{\mathbf{i}}},{{\mathbf{j}}}\in I^\gamma$ and all admissible~$r$ and~$s$.
    \end{Corollary}

    The elements ${y^{\<{d}\>}}_r$ appearing in \autoref{T:HMGradedIso} and
    \autoref{C:HMGradedIso} are different. In the next section we
    introduce a third variation of this notation. The meaning
    of~${y^{\<{d}\>}}_r$ will always be clear from context.

    \subsection{Signed KLR generators}\label{S:AlternaatingCoefficients}

    This section sets up the machinery that will be used to construct the
    isomorphism $\RAn\to{{\mathscr{H}}_\xi({\mathfrak{A}_{n}}}. The idea is to
    use the results of the last two sections to give a new presentation
    of~${{\mathscr{H}}^{\mathcal{O}}_{t}}, which induces an isomorphism ${\mathscr{R}_e({\mathfrak{S}_{n}}}{{\mathscr{H}}_\xi({\mathfrak{S}_{n}}} that restricts
    to an isomorphism $\RAn\to{{\mathscr{H}}_\xi({\mathfrak{A}_{n}}}. To do this we use the generators of~${{\mathscr{H}}^{\mathcal{O}}_{t}}
    given in \autoref{T:HMGradedIso} and \autoref{C:HMGradedIso} together
    with a particular seminormal coefficient system and idempotent
    subring.

    The seminormal coefficient system that we use to prove
    \autoref{T:Main} forces us to work over a ring that contains
    ``enough'' square roots. We start by defining this ring, following
    \cite[Definition~3.1]{MathasRatliff}. Suppose that the
    Iwahori-Hecke algebra ${\mathscr{H}}$ is defined over the field~$F$ with
    parameter $\xi\in F$ of quantum characteristic~$e$. Recall that
    in this paper we are assuming that characteristic of~$F$
    is not~$2$ and that $e>2$.

    \begin{Definition}\label{D:squareroots}
      Let $x$ be an indeterminate over~$F$ and set $t=x+\xi$. In the
      algebraic closure $\overline{F(x)}$ of $F(x)$ fix square roots
      $\sqrt{-1}$, $\sqrt{t}$ and $\sqrt{[h]}$, for $1<h\le n$. Let
      \[{\mathcal{O}}=
      F\bigr[\sqrt{t},\sqrt{[h]}\ \big|\ 1<h\le n\bigr]_{(x)}
      \]
      be the localization of $F[\sqrt{t},\sqrt{[h]}\mid1< h\le n]$ at
      the maximal ideal generated by~$x$.  Let ${\mathcal{K}}$ be the field
      of fractions of ${\mathcal{O}}$.
    \end{Definition}

    Note that $t$ is invertible in~${\mathcal{O}}$ so that we can consider the
    Iwahori-Hecke algebra~${{\mathscr{H}}^{\mathcal{O}}_{t}} with parameter~$t$. By
    \cite[Corollary~5.12]{MathasRatliff}, the field of fractions~${\mathcal{K}}$
    of~${\mathcal{O}}$ is a splitting field for the semisimple algebra ${{\mathscr{H}}_t^{\mathcal{K}}({\mathfrak{A}_{n}}}$.

    Let ${\mathfrak{m}}=x{\mathcal{O}}$ be the maximal ideal of~${\mathcal{O}}$ and set ${\mathbb{F}}={\mathcal{O}}/{\mathfrak{m}}$.  Then
    $F$ is (isomorphic to) a subfield of~${\mathbb{F}}$.
    Moreover,~$\xi$ is identified with the image of~$t$ under the natural
    map
    ${\mathcal{O}}\twoheadrightarrow{\mathbb{F}}$. Hence,
    ${{\mathscr{H}}_\xi({\mathfrak{S}_{n}}}_F{\mathbb{F}}\cong{{\mathscr{H}}^{\mathcal{O}}_{t}}_{\mathcal{O}}{\mathbb{F}}$. (Note that working over~${\mathbb{F}}$ does not
    change the representation theory of~${{\mathscr{H}}_\xi({\mathfrak{S}_{n}}} because any field is a
    splitting field for~${{\mathscr{H}}_\xi({\mathfrak{S}_{n}}} since it is a cellular
    algebra~\cite{GL,M:ULect}.) By construction, ${\mathbb{F}}$ contains square
    roots $\sqrt{-1}$, $\sqrt{\xi}$ and $\sqrt{[h]_\xi}$, for $-n\le h\le n$.
    In general,~${\mathbb{F}}$ is a non-trivial extension of~$F$.

    For $0<h\le n$ fix a choice of ``negative'' square roots in~${\mathcal{O}}$ by setting
    \begin{equation}\label{E:Squareroots}
      \sqrt{[-h]}=\sqrt{-1}(\sqrt{t})^{-h}\sqrt{[h]}.
    \end{equation}
    Then $\sqrt{[-h]}\in{\mathcal{O}}$ for $-n\le h\le n$.
    If $h>0$ then $[-h]=-t^{-h}[h]$ so the effect of~\autoref{E:Squareroots} is
    to fix the sign of the square root of~$[-h]$.

    In order to apply the results of \autoref{T:HMGradedIso}
    and \autoref{C:HMGradedIso} we need to check that $({\mathcal{O}},t)$ is an
    idempotent subring in the sense of \autoref{D:idempotentSub}. Part~(a)
    of \autoref{D:idempotentSub} is automatic whereas parts~(b) and~(c)
    follow from the observation that if~$k\in{\mathbb{Z}}$ then the
    polynomial $[k]=[k]_t\in{\mathcal{O}}$ has zero constant term, as a polynomial
    in~$x$, if and only if $k\in e{\mathbb{Z}}$. Hence, we have the following.

    \begin{Lemma}\label{L:IdempotentSubring}
      The pair $({\mathcal{O}},t)$ is an idempotent subring.
    \end{Lemma}

    Now that we have fixed an idempotent subring we turn to the proof of
    \autoref{T:Main}. The idea is to use the generators of ${{\mathscr{H}}^{\mathcal{O}}_{\gamma}}
    from \autoref{T:HMGradedIso} for ``half'' of~${{\mathscr{H}}^{\mathcal{O}}_{\gamma}} and to use
    the generators from \autoref{C:HMGradedIso} the rest of the time.  To
    make this more
    precise, recall from after \autoref{L:ZeroSequence} that
    \[
    I^n_+=\set{{\mathbf{i}}\in I_n|i_1=0\text{ and } i_2=1}
    \quad\text{ and }\quad
    I^n_-=\set{{\mathbf{i}}\in I_n|i_1=0\text{ and } i_2=-1}.
    \]
    These sets are disjoint because $e\neq 2$. Set
    $I^\gamma_+=I^\gamma\cap I^n_+$ and $I^\gamma_-=I^\gamma\cap I^n_-$,
    for $\gamma\in{Q^\varepsilon_n}$. Then the map
    ${\mathbf{i}}\mapsto-{\mathbf{i}}$ is a bijection of sets $I^\gamma_+{\overset{\simeq}{\longrightarrow}} I^\gamma_-$.

    \begin{Lemma}\label{L:res}
      Suppose that ${\mathbf{i}}\in I^\gamma$ and ${f^{\mathcal{O}}_{\mathbf{i}}}0$. Then ${\mathbf{i}}\in I^\gamma_+$ or
      ${\mathbf{i}}\in I^\gamma_-$.
    \end{Lemma}

    \begin{proof}
      By definition, ${f^{\mathcal{O}}_{\mathbf{i}}}0$ only if ${\mathop{\rm Std}\nolimits}({\mathbf{i}})\ne\emptyset$ or,
      equivalently, ${\mathbf{i}}=\operatorname{res}({\mathsf{t}})$ for some standard tableau ${\mathsf{t}}\in{\mathop{\rm Std}\nolimits}({\mathcal{P}_{n}}$.
      If ${\mathsf{t}}\in{\mathop{\rm Std}\nolimits}({\mathbf{i}})$ then $i_1=\operatorname{res}_1({\mathsf{t}})=0$ and $i_2=\operatorname{res}_2({\mathsf{t}})=\pm1$, so
      ${\mathbf{i}}\in I^\gamma_+\cup I^\gamma_-$.
    \end{proof}

    In particular, if $h\in{{\mathscr{H}}^{\mathcal{O}}_{\gamma}} then $h=\sum_{{\mathbf{i}}\in
    I^\gamma_+}(h{f^{\mathcal{O}}_{\mathbf{i}}}h{f^{\mathcal{O}}_{-{\mathbf{i}}}}$. In what follows we apply
    \autoref{L:res}, and this observation, without further mention.

    Motivated in part by \autoref{P:AltCS} we make the following
    definition.

    \begin{Definition}\label{D:AltCS}
      An \textbf{alternating coefficient system} is a $*$-seminormal
      coefficient system ${{\boldsymbol{\alpha}}}=\set{\alpha_r({\mathsf{t}})}$ such that
      $\alpha_r({\mathsf{t}})=-\alpha_r({\mathsf{t}}')$, for $1\leq r<n$ and
      ${\mathsf{t}}\in{\mathop{\rm Std}\nolimits}({\mathcal{P}_{n}}$.
    \end{Definition}

    Consider the case when $n=3$ and ${{\boldsymbol{\alpha}}}$ is an alternating coefficient system.
    Let ${\mathsf{t}}={
\begin{tikzpicture}[scale=0.3,draw/.append style={thick,black},baseline=-2mm]
  \tableauRow=0
  \foreach \Row in {{{1,2},{3}}} {
  \tableauCol=1
  \foreach\k in \Row {
  \draw(\the\tableauCol,\the\tableauRow)+(-.5,-.5)rectangle++(.5,.5);
  \draw(\the\tableauCol,\the\tableauRow)node{\k};
  \global\advance\tableauCol by 1
  }
  \global\advance\tableauRow by -1
  }
\end{tikzpicture}
}$ and ${\mathsf{s}}={
\begin{tikzpicture}[scale=0.3,draw/.append style={thick,black},baseline=-2mm]
  \tableauRow=0
  \foreach \Row in {{{1,3},{2}}} {
  \tableauCol=1
  \foreach\k in \Row {
  \draw(\the\tableauCol,\the\tableauRow)+(-.5,-.5)rectangle++(.5,.5);
  \draw(\the\tableauCol,\the\tableauRow)node{\k};
  \global\advance\tableauCol by 1
  }
  \global\advance\tableauRow by -1
  }
\end{tikzpicture}
}$, so that
    $\operatorname{res}({\mathsf{t}})\in I^3_+$. By \autoref{D:AltCS} and \autoref{D:alphaSNCS},
    \[
    \alpha_2({\mathsf{t}})^2 = - \alpha_2({\mathsf{s}})\alpha_2({\mathsf{t}})=-\frac{t[3]}{\hspace*{1em}[2]^2}.
    \]
    So $\alpha_2({\mathsf{t}})=\pm\sqrt{-1}\sqrt{t}\sqrt{[3]}/[2]$. In the
    argument that follows we need  $\alpha_2({\mathsf{t}})$ to have such
    values for all ${\mathsf{t}}\in{\mathop{\rm Std}\nolimits}({\mathcal{P}_{n}}$. Following
    \cite[\S3]{MathasRatliff}, for ${\mathbf{i}}\in I^n$ and $1\le r<n$ define
    \begin{equation}\label{E:AlternatingCS}
      \alpha_r({\mathsf{t}})=\begin{cases}
        \frac{t^{\rho_r({\mathsf{t}})/2}\sqrt{[1+\rho_r({\mathsf{t}})]}\sqrt{[1-\rho_r({\mathsf{t}})]}}{[\rho_r({\mathsf{t}})]},
        &\text{if }{\mathbf{i}}\in I^n_+,\\[2mm]
        -\alpha_r({\mathsf{t}}'), &\text{if }{\mathbf{i}}\in I^n_-,\\
        0,&\text{otherwise}.
      \end{cases}
    \end{equation}
    By \autoref{D:squareroots}, $\alpha_r({\mathsf{t}})\in{\mathcal{O}}$
    for all ${\mathsf{t}}\in{\mathop{\rm Std}\nolimits}({\mathcal{P}_{n}}$ and $1\le r<n$.
    Moreover, if~${\mathbf{i}}\in I^n_\pm$ and ${\mathsf{t}}\in{\mathop{\rm Std}\nolimits}({\mathbf{i}})$ then
    $\alpha_2({\mathsf{t}})=\pm\sqrt{-1}\sqrt{t}\sqrt{[3]}/[2]$ by
    \autoref{E:Squareroots}.

    It is straightforward to check that  $\{\alpha_r({\mathsf{t}})\}$ is an
    alternating coefficient system. In particular, if ${\mathsf{t}}$ is standard,
    $1\le r<n$ and $s_r{\mathsf{t}}$ is not standard then
    $\rho_r({\mathsf{t}})=\pm1$ so that $\alpha_r({\mathsf{t}})=0$.

    Using \autoref{T:SeminormalForm}, we fix an arbitrary seminormal basis
    $\set{f_{{\mathsf{s}}{\mathsf{t}}}}$ for ${{\mathscr{H}}_t^{\mathcal{K}}({\mathfrak{S}_{n}}}$ that is compatible with the
    seminormal coefficient system defined by \autoref{E:AlternatingCS}.
    Note that \autoref{D:psio}, and hence the results that follow, do
    not depend on this choice of seminormal basis.

    \begin{Definition}\label{D:psio}
      Suppose that $1\le r<n$ and  $1\le s\le n$. If $r\ne2$ define
      \[
      {\psi^{\mathcal{O}}}_r=\sum_{{\mathbf{i}}\in I^\gamma_+}({\psi^+}_r {f^{\mathcal{O}}_{\mathbf{i}}}{\psi^-}_r{f^{\mathcal{O}}_{-{\mathbf{i}}}}
      \quad\text{and}\quad
      {y^{\mathcal{O}}}_s=\sum_{{\mathbf{i}}\in I^\gamma_+}({y^+}_s{f^{\mathcal{O}}_{\mathbf{i}}}{y^-}_s{f^{\mathcal{O}}_{-{\mathbf{i}}}},
      \]
      and when $r=2$ set
      ${\psi^{\mathcal{O}}}_2={\displaystyle\sum}_{{\mathbf{i}}\in I^\gamma_+}\kappa_{\mathbf{i}}({\psi^+}_2{f^{\mathcal{O}}_{\mathbf{i}}}{\psi^-}_2{f^{\mathcal{O}}_{-{\mathbf{i}}}}$, where
      \[ \kappa_{\mathbf{i}} = \begin{cases}
        t^{-1},&\text{if }e=3,\\
        \frac{\sqrt{t}}{\sqrt{[3]}},&\text{if }e>3.
      \end{cases}
      \]
    \end{Definition}

    For convenience, set $\kappa_{-{\mathbf{i}}}=\kappa_{\mathbf{i}}$, for ${\mathbf{i}}\in
    I^\gamma_+$.  The scalars $\kappa_{\mathbf{i}}$ are needed to ensure that
    ${\psi^{\mathcal{O}}}_2$ satisfies analogues of the quadratic and braid relations in
    \autoref{D:klr}.

    By \autoref{D:AltCS}, $\kappa_{\mathbf{i}}$ is invertible in~${\mathcal{O}}$ for all
    ${\mathbf{i}}\in I^\gamma$. The reason why $\kappa_{\mathbf{i}}$ depends only on~$e$,
    and not on the quiver~$\Gamma_e$, goes back to \autoref{L:ZeroSequence}: if
    ${\mathbf{i}}\in I^\gamma$ and $e({\mathbf{i}})\ne0$ or, equivalently, ${f^{\mathcal{O}}_{\mathbf{i}}}0$ then
    the possible values for $i_1$, $i_2$ and $i_3$ are tightly constrained.

    By \autoref{D:squareroots}, the elements in $\set{{\psi^{\mathcal{O}}}_r|1\leq
    r<n}\cup\set{{y^{\mathcal{O}}}_s|1\leq s\leq n}$ belong to~${{\mathscr{H}}^{\mathcal{O}}_{\gamma}}.  The aim
    is now to show that these elements, together with the idempotents
    $\set{{f^{\mathcal{O}}_{\mathbf{i}}}{\mathbf{i}}\in I^n}$, generate~${{\mathscr{H}}^{\mathcal{O}}_{\gamma}} subject to relations
    that are similar to those in \autoref{T:HMGradedIso}.  This will
    imply that these elements induce an isomorphism
    ${\mathscr{R}_e({\mathfrak{S}_{n}}}_{\mathbb{Z}}{\mathbb{F}}\cong{{\mathscr{H}}_\xi({\mathfrak{S}_{n}}}_F{\mathbb{F}}$. Before we start the proof we
    note the following consequence of \autoref{D:psio} and
    \autoref{L:foIdempotents}. Ultimately, this observation will imply that
    ${{\mathscr{H}}_\xi({\mathfrak{A}_{n}}}_F{\mathbb{F}}\cong\RAn\otimes_{\mathbb{Z}}{\mathbb{F}}$.

    \begin{Corollary}\label{C:PsioHash}
      Suppose that $1\le r<n$, $1\le s\le n$ and ${\mathbf{i}}\in I^\gamma$. Then
      \[({\psi^{\mathcal{O}}}_r)^\#=-{\psi^{\mathcal{O}}}_r,\qquad
      ({y^{\mathcal{O}}}_s)^\#=-{y^{\mathcal{O}}}_s\qquad\text{and}\qquad
      ({f^{\mathcal{O}}_{\mathbf{i}}}^\# = {f^{\mathcal{O}}_{-{\mathbf{i}}}} \]
    \end{Corollary}

    The first step is to give a new generating set for ${{\mathscr{H}}^{\mathcal{O}}_{\gamma}}.

    \begin{Proposition}\label{P:NewGenerators}
      Suppose $\gamma\in {Q^\varepsilon_n}$. Then ${{\mathscr{H}}^{\mathcal{O}}_{\gamma}} is generated by
      \[
      \set{{\psi^{\mathcal{O}}}_r|1\le r<n}\cup\set{{y^{\mathcal{O}}}_s|1\le s\le n}\cup\set{{f^{\mathcal{O}}_{\mathbf{i}}}{\mathbf{i}}\in I^\gamma}.
      \]
    \end{Proposition}

    \begin{proof}Let $H_\gamma$ be the ${\mathcal{O}}$-subalgebra of ${{\mathscr{H}}^{\mathcal{O}}_{\gamma}}
      generated by the elements in the statement of the proposition.
      It is enough to show that $T_r{f^{\mathcal{O}}_{\mathbf{i}}} H_\gamma$,
      for $1\le r<n$ and~${\mathbf{i}}\in I^\gamma$, since these elements generate
      ${{\mathscr{H}}^{\mathcal{O}}_{\gamma}}. Further, by \autoref{C:PsioHash}, $H_\gamma^\#=H_\gamma$
      so it is enough to show that $T_r{f^{\mathcal{O}}_{\mathbf{i}}} H_\gamma$, for ${\mathbf{i}}\in I^\gamma_+$ and
      $1\le r<n$. Let ${f^{\mathcal{O}}_{+}}\sum_{{\mathbf{i}}\in I^\gamma_+}{f^{\mathcal{O}}_{\mathbf{i}}}. As remarked
      above, $\kappa_{\mathbf{i}}$ is an invertible scalar in~${\mathcal{O}}$. Therefore, the
      ${\mathcal{O}}$-module $H_\gamma{f^{\mathcal{O}}_{+}} contains the elements
      $\set{{\psi^+}_r e({\mathbf{i}}), {y^+}_se({\mathbf{i}}), e({\mathbf{i}})|{\mathbf{i}}\in I^\gamma_+}$. Hence,
      $H_\gamma{f^{\mathcal{O}}_{+}}{{\mathscr{H}}^{\mathcal{O}}_{t}}[+]$ by \autoref{T:HMGradedIso}. This completes the proof.
    \end{proof}

    For the rest of this paper, for $d\in{\mathbb{Z}}$, $1\le r\le n$ and ${\mathbf{i}}\in I^\gamma$ we
    set
    \begin{equation}\label{E:YShiftDef}
        {y^{\<{d}\>}}_r{f^{\mathcal{O}}_{\mathbf{i}}}\begin{dcases*}
                       (t^d{y^{\mathcal{O}}}_r-[d]){f^{\mathcal{O}}_{\mathbf{i}}}& if ${\mathbf{i}}\in I^\gamma_+$,\\
                       (t^d{y^{\mathcal{O}}}_r+[d]){f^{\mathcal{O}}_{\mathbf{i}}}& if ${\mathbf{i}}\in I^\gamma_-$.
                    \end{dcases*}
    \end{equation}
    Since ${y^{\mathcal{O}}}_r=\sum_{{\mathbf{i}}\in I^\gamma_+}({y^+}_r{f^{\mathcal{O}}_{\mathbf{i}}}{y^-}_r{f^{\mathcal{O}}_{-{\mathbf{i}}}}$
    this is compatible with the two definitions of~${y^{\<{d}\>}}_r{f^{\mathcal{O}}_{\mathbf{i}}} used in the
    last section

    The rest of this section determines a set of defining relations
    for ${{\mathscr{H}}^{\mathcal{O}}_{\gamma}} for the generators of~${{\mathscr{H}}^{\mathcal{O}}_{\gamma}} from
    \autoref{P:NewGenerators}.  Fortunately, much of the work has
    already been done because \autoref{T:HMGradedIso} and
    \autoref{C:HMGradedIso} give us a large number of relations.  More
    precisely, they give the following list of relations,
    \textit{not} involving~$\psi_2^{\mathcal{O}}$.

    \begin{Lemma}\label{L:AutomaticRelations}
      Suppose that  $\gamma\in{Q^\varepsilon_n}$. The following
      identities hold in ${{\mathscr{H}}^{\mathcal{O}}_{\gamma}}:
      {\setlength{\abovedisplayskip}{2pt}
      \setlength{\belowdisplayskip}{1pt}
      \begin{alignat*}{3}
        ({y^{\mathcal{O}}}_1)^{(\Lambda_0,\alpha_{i_1})}{f^{\mathcal{O}}_{\mathbf{i}}}=0,
        &\qquad {f^{\mathcal{O}}_{\mathbf{i}}}[{\mathbf{j}}]&= \delta_{{{\mathbf{i}}}{{\mathbf{j}}}}{f^{\mathcal{O}}_{\mathbf{i}}}&\qquad\textstyle\sum_{{{\mathbf{i}}}\in I^\gamma}{f^{\mathcal{O}}_{\mathbf{i}}}= 1\\
        {y^{\mathcal{O}}}_t {f^{\mathcal{O}}_{\mathbf{i}}}= {f^{\mathcal{O}}_{\mathbf{i}}}_t,&\qquad {\psi^{\mathcal{O}}}_r {f^{\mathcal{O}}_{\mathbf{i}}}= {f^{\mathcal{O}}_{s_r\cdot{{\mathbf{i}}}}}_r,&\qquad
        {y^{\mathcal{O}}}_r {y^{\mathcal{O}}}_t&= {y^{\mathcal{O}}}_t {y^{\mathcal{O}}}_r
      \end{alignat*}
      \begin{alignat*}{2}
        {\psi^{\mathcal{O}}}_r {y^{\mathcal{O}}}_{r+1}{f^{\mathcal{O}}_{\mathbf{i}}}= ({y^{\mathcal{O}}}_r {\psi^{\mathcal{O}}}_r+\delta_{i_r i_{r+1}}){f^{\mathcal{O}}_{\mathbf{i}}}
        &\qquad{y^{\mathcal{O}}}_{r+1}{\psi^{\mathcal{O}}}_r {f^{\mathcal{O}}_{\mathbf{i}}}= ({\psi^{\mathcal{O}}}_r {y^{\mathcal{O}}}_r+\delta_{i_r i_{r+1}}){f^{\mathcal{O}}_{\mathbf{i}}}{alignat*}
      \begin{align*}
        {\psi^{\mathcal{O}}}_r {y^{\mathcal{O}}}_t&= {y^{\mathcal{O}}}_t{\psi^{\mathcal{O}}}_r,   &\text{if $t\neq r,r+1$,}\\
        {\psi^{\mathcal{O}}}_r{\psi^{\mathcal{O}}}_s&= {\psi^{\mathcal{O}}}_s{\psi^{\mathcal{O}}}_r,&\text{if $\left|r-s\right|>1$,}
      \end{align*}
      \begin{align*}
        ({\psi^{\mathcal{O}}}_r)^2{f^{\mathcal{O}}_{\mathbf{i}}}= \begin{cases}
          ({y^{\<{1+\rho_r({\mathbf{i}})}\>}}_r-{y^{\mathcal{O}}}_{r+1}){f^{\mathcal{O}}_{\mathbf{i}}}
          &\text{if $i_r\to i_{r+1}$ and ${\mathbf{i}}\in I^\gamma_+$},\\
          ({y^{\mathcal{O}}}_{r+1}-{y^{\<{1-\rho_r({\mathbf{i}})}\>}}_r){f^{\mathcal{O}}_{\mathbf{i}}}
          &\text{if $i_r\leftarrow i_{r+1}$ and ${\mathbf{i}}\in I^\gamma_-$}\\
          ({y^{\<{1-\rho_r({\mathbf{i}})}\>}}_{r+1}-{y^{\mathcal{O}}}_r){f^{\mathcal{O}}_{\mathbf{i}}}
          &\text{if $i_r\leftarrow i_{r+1}$ and ${\mathbf{i}}\in I^\gamma_+$}\\
          ({y^{\mathcal{O}}}_r-{y^{\<{1+\rho_r({\mathbf{i}})}\>}}_{r+1}){f^{\mathcal{O}}_{\mathbf{i}}}
          &\text{if $i_r\to i_{r+1}$ and ${\mathbf{i}}\in I^\gamma_-$},\\
          0,&\text{if }i_r=i_{r+1},\\
          {f^{\mathcal{O}}_{\mathbf{i}}}&\text{otherwise,}
        \end{cases}\\
        {\psi^{\mathcal{O}}}_r{\psi^{\mathcal{O}}}_{r+1}{\psi^{\mathcal{O}}}_r {f^{\mathcal{O}}_{\mathbf{i}}}= \begin{cases}
          ({\psi^{\mathcal{O}}}_{r+1}{\psi^{\mathcal{O}}}_r{\psi^{\mathcal{O}}}_{r+1}-t^{1+\rho_r({\mathbf{i}})}){f^{\mathcal{O}}_{\mathbf{i}}}
          &\text{if $i_r=i_{r+2}\to i_{r+1},$ and ${\mathbf{i}}\in I^\gamma_+$}\\
          ({\psi^{\mathcal{O}}}_{r+1}{\psi^{\mathcal{O}}}_r{\psi^{\mathcal{O}}}_{r+1}+t^{1-\rho_r({\mathbf{i}})}){f^{\mathcal{O}}_{\mathbf{i}}}
          &\text{if $i_r=i_{r+2}\leftarrow i_{r+1}$ and ${\mathbf{i}}\in I^\gamma_-$},\\
          ({\psi^{\mathcal{O}}}_{r+1}{\psi^{\mathcal{O}}}_r{\psi^{\mathcal{O}}}_{r+1}+1){f^{\mathcal{O}}_{\mathbf{i}}}
          &\text{if $i_r=i_{r+2}\leftarrow i_{r+1}$ and ${\mathbf{i}}\in I^\gamma_+$},\\
          ({\psi^{\mathcal{O}}}_{r+1}{\psi^{\mathcal{O}}}_r{\psi^{\mathcal{O}}}_{r+1}-1){f^{\mathcal{O}}_{\mathbf{i}}}
          &\text{if $i_r=i_{r+2}\to i_{r+1},$ and ${\mathbf{i}}\in I^\gamma_-$}\\
          {\psi^{\mathcal{O}}}_{r+1}{\psi^{\mathcal{O}}}_r{\psi^{\mathcal{O}}}_{r+1}{f^{\mathcal{O}}_{\mathbf{i}}}&\text{otherwise,}
        \end{cases}
      \end{align*}
      }\noindent      for all admissible ${\mathbf{i}},{\mathbf{j}}\in I^\gamma$ and $r,s,t$ satisfying
      $2<r,s<n$ and $1\le t\le n$.
    \end{Lemma}

    \begin{proof}
      First notice that if ${\mathbf{i}}\notin I^\gamma_+\cup I^\gamma_-$ then ${f^{\mathcal{O}}_{\mathbf{i}}}0$ by
      \autoref{L:res}, so all of the relations above are trivially true.
      We may assume then that ${\mathbf{i}}\in I^\gamma_+\cup I^\gamma_-$.

      The first three identities follow directly from \autoref{T:HMGradedIso} and
      \autoref{C:HMGradedIso}. For the remaining formulas, observe that if
      $2<r<n$ then ${\mathbf{i}}\in I^\gamma_+$ if and only if $s_r\cdot{\mathbf{i}}\in I^\gamma_+$ and,
      similarly, ${\mathbf{i}}\in I^\gamma_-$ if and only if $s_r\cdot{\mathbf{i}}\in I^\gamma_-$.
      Therefore, if ${\mathbf{i}}\in I^\gamma_+$ the relations hold by virtue of
      \autoref{T:HMGradedIso} and if ${\mathbf{i}}\in I^\gamma_-$ then they hold by
      \autoref{C:HMGradedIso}. Note that if~${\mathbf{i}}\in I^\gamma_-$ then there is a sign
      change in the last two relations, in comparison with \autoref{C:HMGradedIso},
      because ${\psi^{\mathcal{O}}}_r{f^{\mathcal{O}}_{\mathbf{i}}}-{\psi^-}_r{f^{\mathcal{O}}_{\mathbf{i}}} and ${y^{\mathcal{O}}}_t{f^{\mathcal{O}}_{\mathbf{i}}}-{y^-}_s{f^{\mathcal{O}}_{\mathbf{i}}}.
    \end{proof}

    Next we need analogues of the relations in
    \autoref{L:AutomaticRelations} for~${\psi^{\mathcal{O}}}_1$. We could replace the
    next result with the single relation~${\psi^{\mathcal{O}}}_1=0$, however, this
    is not sufficient for our later arguments because the proof of
    \autoref{T:MainTheorem} relies on the fact that the generators
    of~\autoref{P:NewGenerators} satisfy relations that are compatible
    with \autoref{D:klr}.

    \begin{Lemma}\label{L:PsiOne}
      Suppose that $\gamma\in{Q^\varepsilon_n}$. The following
      identities hold in ${{\mathscr{H}}^{\mathcal{O}}_{\gamma}}:
      {\setlength{\abovedisplayskip}{2pt}
      \setlength{\belowdisplayskip}{1pt}
      \begin{alignat*}{3}
        {\psi^{\mathcal{O}}}_1{f^{\mathcal{O}}_{\mathbf{i}}}={f^{\mathcal{O}}_{s_1\cdot{\mathbf{i}}}}_1,
        &\qquad {\psi^{\mathcal{O}}}_1 {y^{\mathcal{O}}}_s&= {y^{\mathcal{O}}}_s{\psi^{\mathcal{O}}}_1,
        &\qquad {\psi^{\mathcal{O}}}_1{\psi^{\mathcal{O}}}_r&= {\psi^{\mathcal{O}}}_r{\psi^{\mathcal{O}}}_1,
      \end{alignat*}
      \begin{alignat*}{2}
        {\psi^{\mathcal{O}}}_1 {y^{\mathcal{O}}}_2{f^{\mathcal{O}}_{\mathbf{i}}}= ({y^{\mathcal{O}}}_1 {\psi^{\mathcal{O}}}_1+\delta_{i_1 i_2}){f^{\mathcal{O}}_{\mathbf{i}}}
        &\qquad{y^{\mathcal{O}}}_2{\psi^{\mathcal{O}}}_1 {f^{\mathcal{O}}_{\mathbf{i}}}= ({\psi^{\mathcal{O}}}_1 {y^{\mathcal{O}}}_1+\delta_{i_1 i_2}){f^{\mathcal{O}}_{\mathbf{i}}}
      \end{alignat*}
      \begin{align*}
        ({\psi^{\mathcal{O}}}_1)^2{f^{\mathcal{O}}_{\mathbf{i}}}= \begin{cases}
          ({y^{\mathcal{O}}}_1-{y^{\mathcal{O}}}_2){f^{\mathcal{O}}_{\mathbf{i}}}
          &\text{if }i_1\to i_2,\\% and ${\mathbf{i}}\in I^\gamma_+$},\\
          ({y^{\mathcal{O}}}_2-{y^{\mathcal{O}}}_1){f^{\mathcal{O}}_{\mathbf{i}}}
          &\text{if }i_1\leftarrow i_2,\\% and ${\mathbf{i}}\in I^\gamma_-$}\\
          
          
          
          
          0,&\text{if }i_1=i_2,\\
          {f^{\mathcal{O}}_{\mathbf{i}}}&\text{otherwise,}
        \end{cases}\\
        {\psi^{\mathcal{O}}}_1{\psi^{\mathcal{O}}}_2{\psi^{\mathcal{O}}}_1 {f^{\mathcal{O}}_{\mathbf{i}}}= \begin{cases}
          ({\psi^{\mathcal{O}}}_2{\psi^{\mathcal{O}}}_1{\psi^{\mathcal{O}}}_2-1){f^{\mathcal{O}}_{\mathbf{i}}}
          &\text{if }i_1=i_3\to i_2,\\% and ${\mathbf{i}}\in I^\gamma_+$}\\
          ({\psi^{\mathcal{O}}}_2{\psi^{\mathcal{O}}}_1{\psi^{\mathcal{O}}}_2+1){f^{\mathcal{O}}_{\mathbf{i}}}
          &\text{if }i_1=i_3\leftarrow i_2,\\% and ${\mathbf{i}}\in I^\gamma_-$},\\
          
          
          
          
          {\psi^{\mathcal{O}}}_2{\psi^{\mathcal{O}}}_1{\psi^{\mathcal{O}}}_2{f^{\mathcal{O}}_{\mathbf{i}}}&\text{otherwise,}
        \end{cases}
      \end{align*}
      }\noindent      for all admissible ${\mathbf{i}}\in I^\gamma$ and $r,s$ satisfying
      $2<r<n$ and $3\le s\le n$.
    \end{Lemma}

    \begin{proof}
      By definition, ${f^{\mathcal{O}}_{\mathbf{i}}}0$ if and only if~${\mathbf{i}}=\operatorname{res}({\mathsf{s}})$ for some
      standard tableau~${\mathsf{s}}$. In particular, if ${\mathbf{i}}=(i_1,\dots,i_n)$ then
      $i_1=0$, $i_2\in\set{-1,1}$ and $i_3\in\set{-2,-1,1,2}$. Hence, it
      follows from \autoref{T:HMGradedIso} and \autoref{C:HMGradedIso}
      that ${\psi^+}_1=0={\psi^-}_1$. Therefore, ${\psi^{\mathcal{O}}}_1=0$ and the first three
      relations are trivially true. The next two relations hold
      because $\delta_{i_1i_2}=0$ whenever ${f^{\mathcal{O}}_{\mathbf{i}}}0$ and the quadratic
      relation for $({\psi^{\mathcal{O}}}_1)^2$ holds in view of \autoref{T:HMGradedIso}
      and \autoref{C:HMGradedIso}. For the final ``braid'' relation,
      if~$i_1=i_3\rightarrow i_2$ or $i_1=i_3\leftarrow i_2$ then
      ${f^{\mathcal{O}}_{\mathbf{i}}}0$ by the remarks at the start of the proof, so the
      braid relation is trivially true in these cases. In the remaining cases
      ${\psi^{\mathcal{O}}}_1{\psi^{\mathcal{O}}}_2{\psi^{\mathcal{O}}}_1=0={\psi^{\mathcal{O}}}_2{\psi^{\mathcal{O}}}_1{\psi^{\mathcal{O}}}_2$ since
      ${\psi^{\mathcal{O}}}_1=0$. This completes the proof.
    \end{proof}

    It remains to determine the relations involving ${\psi^{\mathcal{O}}}_2$. The first step is
    easy.

    \begin{Lemma}\label{L:SimplePsi2Relations}
      Suppose that ${\mathbf{i}}\in I^\gamma$, $2<r<n$ and $1\le s\le n$ with
      $t\ne 2,3$. Then
      \[
            {\psi^{\mathcal{O}}}_2{\psi^{\mathcal{O}}}_r= {\psi^{\mathcal{O}}}_r{\psi^{\mathcal{O}}}_2
                \qquad\text{and}\qquad
            {\psi^{\mathcal{O}}}_2 {y^{\mathcal{O}}}_s= {y^{\mathcal{O}}}_s{\psi^{\mathcal{O}}}_2.
      \]
    \end{Lemma}

    \begin{proof}
      Since ${\psi^{\mathcal{O}}}_2{f^{\mathcal{O}}_{\mathbf{i}}}\pm\kappa_{\mathbf{i}}\psi^\pm_2{f^{\mathcal{O}}_{\mathbf{i}}}, where $\kappa_{\mathbf{i}}\in{\mathcal{O}}$ is
      invertible for ${\mathbf{i}}\in I^\gamma$, the result follows directly from
      \autoref{T:HMGradedIso} and \autoref{C:HMGradedIso}.
    \end{proof}

    For the remaining relations we need a more precise description of how the
    generators of~\autoref{D:psio} act on the seminormal basis. Suppose that
    ${\mathsf{s}}\in{\mathop{\rm Std}\nolimits}({\mathbf{i}})$ and that ${\mathsf{u}}=(r,r+1){\mathsf{s}}$, where $1\le r<n$.
    Following \cite[(4.21)]{HuMathas:SeminormalQuiver}, define
    \begin{equation}\label{E:beta}
      \beta_r({\mathsf{s}})=\begin{cases}
        -\beta_r({\mathsf{s}}'),&\text{if }{\mathbf{i}}\in I^\gamma_-,\\[1pt]
        \dfrac{t^{{\hat\imath}_{r}-c_{r}({\mathsf{s}})}\alpha_r({\mathsf{s}})}{[1-\rho_r({\mathsf{s}})]},
        &\text{if ${\mathbf{i}}\in I^\gamma_+$ and }i_r=i_{r+1},\\[1pt]
        t^{c_{r+1}({\mathsf{s}})-{\hat\imath}_r}\alpha_r({\mathsf{s}})[\rho_r({\mathsf{s}})],
        &\text{if ${\mathbf{i}}\in I^\gamma_+$ and }i_r=i_{r+1}+1,\\
        \dfrac{t^{-\rho_r({\mathsf{s}})}\alpha_r({\mathsf{s}})[\rho_r({\mathsf{s}})]}{[1-\rho_r({\mathsf{s}})]},
        &\text{if ${\mathbf{i}}\in I^\gamma_+$ and }i_r\notin\set{i_{r+1}, i_{r+1}+1}.
      \end{cases}
    \end{equation}
    Note that $\beta_r({\mathsf{s}})=0$ if ${\mathsf{u}}$ is not standard because
    $\alpha_r({\mathsf{u}})=0$ whenever ${\mathsf{u}}\notin{\mathop{\rm Std}\nolimits}({\mathcal{P}_{n}}$. More explicit formulas
    for $\beta_r({\mathsf{s}})$ can be obtained using
    \autoref{E:AlternatingCS}, however, we will only need these in one special case;
    see \autoref{E:BetaTwo} below.

    Note that if ${\mathsf{s}}$ is standard and ${\mathbf{i}}=\operatorname{res}({\mathsf{s}})$ then $\operatorname{res}({\mathsf{s}}')=-{\mathbf{i}}$. Therefore,
    the four cases in \autoref{E:beta} are mutually exclusive.
    We need to be slightly careful, however, because
    if ${\mathbf{i}}=\operatorname{res}({\mathsf{s}})$ and ${\mathbf{j}}=\operatorname{res}({\mathsf{s}}')$ then it is
    not usually true that ${\hat\jmath}_r=-{\hat\imath}_r$, for $1\le r\le n$.

    Following \cite[Lemma~4.23]{HuMathas:SeminormalQuiver} we can now
    describe the action of ${\psi^{\mathcal{O}}}_r$ and ${y^{\mathcal{O}}}_r$ on the seminormal basis
    $\set{f_{{\mathsf{s}}{\mathsf{t}}}}$. This result is the only place where we explicitly use
    the assumption of \autoref{D:AltCS}.

    \begin{Proposition}\label{P:betas}
      Suppose that ${\mathsf{s}},{\mathsf{t}}\in{\mathop{\rm Std}\nolimits}(\lambda)$ for $\lambda\in{\mathcal{P}_{n}} and let
      ${\mathbf{i}}=\operatorname{res}({\mathsf{s}})$, ${\mathbf{j}}=\operatorname{res}({\mathsf{s}}')$ for ${\mathbf{i}},{\mathbf{j}}\in I^\gamma$. Fix $1\leq r<n$
      and let ${\mathsf{u}}=(r,r+1){\mathsf{s}}$. Then
      \[
      {\psi^{\mathcal{O}}}_r f_{{\mathsf{s}}{\mathsf{t}}} =\begin{cases}
        \kappa_{\mathbf{i}}\beta_2({\mathsf{s}})f_{{\mathsf{u}}{\mathsf{t}}},&\text{if }r=2,\\
        \beta_r({\mathsf{s}})f_{{\mathsf{u}}{\mathsf{t}}}-\delta_{i_ri_{r+1}}
        \dfrac{t^{{\hat\imath}_{r+1}-c_{r+1}({\mathsf{s}})}}{[\rho_r({\mathsf{s}})]}f_{{\mathsf{s}}{\mathsf{s}}},
        &\text{if $r\ne2$ and }{\mathbf{i}}\in I^\gamma_+,\\[3mm]
        \beta_r({\mathsf{s}})f_{{\mathsf{u}}{\mathsf{t}}}-\delta_{i_ri_{r+1}}
        \dfrac{t^{{\hat\jmath}_{r+1}-c_{r+1}({\mathsf{s}})}}{[\rho_r({\mathsf{s}})]}f_{{\mathsf{s}}{\mathsf{s}}},
        &\text{if $r\ne2$ and }{\mathbf{i}}\in I^\gamma_-.\\
      \end{cases}
      \]
      Moreover, if $1\le k\le n$ then
      \[{y^{\mathcal{O}}}_k f_{{\mathsf{s}}{\mathsf{t}}}=\begin{cases}
        \phantom{-}[c_k({\mathsf{s}})-{\hat\imath}_k]f_{{\mathsf{s}}{\mathsf{t}}},&\text{if }{\mathbf{i}}\in I^\gamma_+,\\
        -[c_k({\mathsf{s}}')-{\hat\jmath}_k]f_{{\mathsf{s}}{\mathsf{t}}},
        &\text{if }{\mathbf{i}}\in I^\gamma_-.\\
      \end{cases}\]
    \end{Proposition}

    \begin{proof}
      Without loss of generality, we can
      assume that ${\mathsf{t}}={\mathsf{s}}$ by \autoref{T:SeminormalForm}(a), so we need to
      compute ${\psi^{\mathcal{O}}}_rf_{{\mathsf{s}}{\mathsf{s}}}$ and ${y^{\mathcal{O}}}_r f_{{\mathsf{s}}{\mathsf{s}}}$.

      First consider ${\psi^{\mathcal{O}}}_rf_{{\mathsf{s}}{\mathsf{s}}}$ when $r\ne2$. If ${\mathbf{i}}\in I^\gamma_+$ then
      ${\psi^{\mathcal{O}}}_r f_{{\mathsf{s}}{\mathsf{s}}}={\psi^+}_rf_{{\mathsf{s}}{\mathsf{s}}}$ and the lemma is a restatement of
      \cite[Lemma~4.23]{HuMathas:SeminormalQuiver}. Suppose then that
      ${\mathbf{i}}\in I^\gamma_-$,
      so that ${\mathbf{j}}\in I^\gamma_+$ and ${\psi^{\mathcal{O}}}_r f_{{\mathsf{s}}'{\mathsf{s}}'}$ is given
      by the formulas above. As $\#$ is an involution,
      using \autoref{C:fttHash} for the third equality and \autoref{L:futhash}
      for the last equality,
      \begin{align*}
        {\psi^{\mathcal{O}}}_r f_{{\mathsf{s}}{\mathsf{s}}} &=-({\psi^+}_r)^\#f_{{\mathsf{s}}{\mathsf{s}}}
        =-\big({\psi^+}_r f_{{\mathsf{s}}{\mathsf{s}}}^\#\big)^\#
        =-\frac{\gamma_{\mathsf{s}}}{\gamma_{{\mathsf{s}}'}}\big({\psi^+}_r f_{{\mathsf{s}}'{\mathsf{s}}'}\big)^\#,\\
        &=-\frac{\gamma_{\mathsf{s}}}{\gamma_{{\mathsf{s}}'}}\Big(\beta_r({\mathsf{s}}')f_{{\mathsf{u}}'{\mathsf{s}}'}
        -\delta_{j_rj_{r+1}}
        \frac{t^{{\hat\jmath}_{r+1}-c_{r+1}({\mathsf{s}}')}}{[\rho_r({\mathsf{s}}')]}f_{{\mathsf{s}}'{\mathsf{s}}'}\Big)^\#\\
        &=\frac{\alpha_r({\mathsf{s}})\beta_r({\mathsf{s}}')}{\alpha_r({\mathsf{s}}')}f_{{\mathsf{u}}{\mathsf{s}}}
        -\delta_{i_ri_{r+1}}\frac{t^{{\hat\jmath}_{r+1}-c_{r+1}({\mathsf{s}})}}{[\rho_r({\mathsf{s}})]}f_{{\mathsf{s}}{\mathsf{s}}},
      \end{align*}
      since $[\rho_r({\mathsf{s}}')]=[-\rho_r({\mathsf{s}})]=-t^{-\rho_r({\mathsf{s}})}[\rho_r({\mathsf{s}})]$. By
      \autoref{D:AltCS}, $\alpha_r({\mathsf{s}}')=-\alpha_r({\mathsf{s}})$ and
      $\beta_r({\mathsf{s}}')=-\beta_r({\mathsf{s}})$, so this establishes the formula for
      ${\psi^{\mathcal{O}}}_r f_{{\mathsf{s}}{\mathsf{t}}}$ when $r\ne2$.

      Now consider ${\psi^{\mathcal{O}}}_2f_{{\mathsf{s}}{\mathsf{s}}}$. If ${f^{\mathcal{O}}_{\mathbf{i}}}0$ then $i_2\ne i_3$
      because ${f^{\mathcal{O}}_{\mathbf{i}}}0$ only if ${\mathbf{i}}$ is
      the residue sequence of some standard tableau. Therefore,
      if ${\mathbf{i}}\in I^\gamma_+$ and ${f^{\mathcal{O}}_{\mathbf{i}}}0$ then
      the argument of the last paragraph shows that
      ${\psi^+}_2f_{{\mathsf{s}}{\mathsf{t}}}=\beta_2({\mathsf{s}})f_{{\mathsf{u}}{\mathsf{t}}}$ and if ${\mathbf{i}}\in I^\gamma_-$ then
      $-{\psi^-}_2f_{{\mathsf{s}}{\mathsf{t}}}=\beta_2({\mathsf{s}})f_{{\mathsf{u}}{\mathsf{t}}}$. As
      ${\psi^{\mathcal{O}}}_2=\sum_{{\mathbf{i}}\in I^\gamma_+}\kappa_{\mathbf{i}}({\psi^+}_2{f^{\mathcal{O}}_{\mathbf{i}}}{\psi^-}_2{f^{\mathcal{O}}_{-{\mathbf{i}}}}$,
      it follows that ${\psi^{\mathcal{O}}}_2 f_{{\mathsf{s}}{\mathsf{t}}}=\kappa_{\mathbf{i}}\beta_2({\mathsf{s}})f_{{\mathsf{u}}{\mathsf{t}}}$ as
      claimed.

      For the action of ${y^{\mathcal{O}}}_k$, if ${\mathbf{i}}\in I^\gamma_+$ then
      ${y^{\mathcal{O}}}_k f_{{\mathsf{s}}{\mathsf{s}}}={y^+}_k f_{{\mathsf{s}}{\mathsf{s}}}=[c_k({\mathsf{s}})-{\hat\imath}_k]f_{{\mathsf{s}}{\mathsf{s}}}$ by
      \cite[Lemma~4.23]{HuMathas:SeminormalQuiver}. On the other hand, if
      ${\mathbf{i}}\in I^\gamma_-$ then, using \autoref{L:foIdempotents} twice,
      \[{y^{\mathcal{O}}}_k f_{{\mathsf{s}}{\mathsf{s}}}
      = -\frac{\gamma_{\mathsf{s}}}{\gamma_{{\mathsf{s}}'}}({y^+}_k f_{{\mathsf{s}}'{\mathsf{s}}'})^\#
      = -\frac{\gamma_{\mathsf{s}}}{\gamma_{{\mathsf{s}}'}}\big([c_k({\mathsf{s}}')-{\hat\jmath}_k] f_{{\mathsf{s}}'{\mathsf{s}}'}\big)^\#
      =-[c_k({\mathsf{s}}')-{\hat\jmath}_k]f_{{\mathsf{s}}{\mathsf{s}}}.
      \]
      as required.
    \end{proof}

    We can now determine the remaining ``KLR-like'' relations satisfied by
    ${\psi^{\mathcal{O}}}_2$.

    \begin{Lemma}\label{L:psi2Intertwiner}
      Suppose that ${\mathbf{i}}\in I^\gamma$ and let ${\mathbf{j}}=s_2\cdot{\mathbf{i}}$. Then
      ${\psi^{\mathcal{O}}}_2{f^{\mathcal{O}}_{\mathbf{i}}}{f^{\mathcal{O}}_{{\mathbf{j}}}}_2$.
    \end{Lemma}

    \begin{proof}
      If ${\mathbf{i}}\notin I^\gamma_+\cup I^\gamma_-$ then ${f^{\mathcal{O}}_{\mathbf{i}}}0={f^{\mathcal{O}}_{{\mathbf{j}}}} and
      there is nothing to prove. Therefore, we may assume that ${\mathbf{i}}\in
      I^\gamma_+\cup I^\gamma_-$. Recall that
      ${f^{\mathcal{O}}_{\gamma}}\sum_{{\mathbf{k}}\in I^\gamma}{f^{\mathcal{O}}_{{\mathbf{k}}}} is the identity element
      of~${{\mathscr{H}}^{\mathcal{O}}_{\gamma}}.  By \autoref{P:betas},
      \[ {\psi^{\mathcal{O}}}_2={\psi^{\mathcal{O}}}_2{f^{\mathcal{O}}_{\gamma}}\sum_{{\mathbf{k}}\in I^\gamma}{\psi^{\mathcal{O}}}_2{f^{\mathcal{O}}_{{\mathbf{k}}}}\sum_{\substack{{\mathbf{k}}\in I^\gamma\\{\mathsf{t}}\in{\mathop{\rm Std}\nolimits}({\mathbf{k}})}}
                 \frac1{\gamma_{\mathsf{t}}}{\psi^{\mathcal{O}}}_2f_{{\mathsf{t}}{\mathsf{t}}}
              =\sum_{\substack{{\mathbf{k}}\in I^\gamma\\{\mathsf{t}}\in{\mathop{\rm Std}\nolimits}({\mathbf{k}})\\{\mathsf{s}}=s_2{\mathsf{t}}}}
              \frac{\kappa_{\mathbf{k}}\beta_2({\mathsf{t}})}{\gamma_{\mathsf{t}}}f_{{\mathsf{s}}{\mathsf{t}}}.
      \]
      If $f_{{\mathsf{s}}{\mathsf{t}}}$ is a term in the right-hand sum, with
      ${\mathsf{t}}\in{\mathop{\rm Std}\nolimits}({\mathbf{k}})$ and ${\mathsf{s}}=s_2{\mathsf{t}}$, then
      \[
           f_{{\mathsf{s}}{\mathsf{t}}}{f^{\mathcal{O}}_{{\mathbf{i}}}}\delta_{{\mathbf{i}}{\mathbf{k}}}f_{{\mathsf{s}}{\mathsf{t}}}
                           =\delta_{s_2\cdot{\mathbf{i}},s_2\cdot{\mathbf{k}}}f_{{\mathsf{s}}{\mathsf{t}}}
                           ={f^{\mathcal{O}}_{{\mathbf{j}}}}_{{\mathsf{s}}{\mathsf{t}}},
      \]
      by \autoref{T:SeminormalForm}(b). Hence,
      ${\psi^{\mathcal{O}}}_2{f^{\mathcal{O}}_{\mathbf{i}}}\sum_{{\mathsf{t}}\in{\mathop{\rm Std}\nolimits}({\mathbf{i}})}\frac1{\gamma_{\mathsf{t}}}\kappa_{\mathbf{i}}\beta_2({\mathsf{t}})f_{{\mathsf{s}}{\mathsf{t}}}
                 ={f^{\mathcal{O}}_{{\mathbf{j}}}}_2$,
      as required.
    \end{proof}

    Recall from \autoref{E:YShiftDef} that if $d\in{\mathbb{Z}}$ and
    ${\mathbf{i}}\in I^\gamma_\pm$
    then ${y^{\<{d}\>}}_r{f^{\mathcal{O}}_{\mathbf{i}}}(t^d{y^{\mathcal{O}}}_r\mp[d]){f^{\mathcal{O}}_{\mathbf{i}}}.

    \begin{Lemma}\label{L:Mixed}
      Suppose that $\gamma\in{Q^\varepsilon_n}$ and ${\mathbf{i}}\in I^\gamma$. Then:
      \[
        {y^{\mathcal{O}}}_3 {\psi^{\mathcal{O}}}_2{f^{\mathcal{O}}_{\mathbf{i}}}\big({\psi^{\mathcal{O}}}_2{y^{\<{-e}\>}}_2+\delta_{i_2i_3}\big){f^{\mathcal{O}}_{\mathbf{i}}}
          \quad\text{and}\quad
        {\psi^{\mathcal{O}}}_2 {y^{\mathcal{O}}}_3{f^{\mathcal{O}}_{\mathbf{i}}}\big({y^{\<{-e}\>}}_2{\psi^{\mathcal{O}}}_2+\delta_{i_2i_3}\big){f^{\mathcal{O}}_{\mathbf{i}}}
      \]
    \end{Lemma}

    \begin{proof}
      Both identities are proved similarly so we consider only the first
      one.  If~${f^{\mathcal{O}}_{\mathbf{i}}}0$ then ${\mathbf{i}}=\operatorname{res}({\mathsf{s}})$, for some standard
      tableau~${\mathsf{s}}$, in which case $i_2\ne i_3$. Hence, if~${f^{\mathcal{O}}_{\mathbf{i}}}0$ then
      $\delta_{i_2i_3}=0$ so we can assume that $\delta_{i_2i_3}=0$ in
      what follows.  (We include the term for $\delta_{i_2i_3}$
      because to prove \autoref{T:Main} we need to compare the
      identity in the lemma with the relations in \autoref{D:klr}.)
      Without loss of generality, we may assume that ${\mathbf{i}}\in I^\gamma_+$.

      By \autoref{T:SeminormalForm}(b),
      ${f^{\mathcal{O}}_{\mathbf{i}}}\sum_{{\mathsf{s}}\in{\mathop{\rm Std}\nolimits}({\mathbf{i}})}\frac1{\gamma_{\mathsf{s}}}f_{{\mathsf{s}}{\mathsf{s}}}$. Therefore,
      to prove the lemma it is enough to verify that
      ${y^{\mathcal{O}}}_3 {\psi^{\mathcal{O}}}_2f_{{\mathsf{s}}{\mathsf{s}}}={\psi^{\mathcal{O}}}_2{y^{\<{-e}\>}}_2f_{{\mathsf{s}}{\mathsf{s}}}$, for all
      ${\mathsf{s}}\in{\mathop{\rm Std}\nolimits}({\mathbf{i}})$.  Fix ${\mathsf{s}}\in{\mathop{\rm Std}\nolimits}({\mathbf{i}})$ and set ${\mathsf{u}}=(2,3){\mathsf{s}}$ and
      ${\mathbf{j}}=s_2\cdot{\mathbf{i}}\in I^\gamma_-$ so that ${\mathbf{j}}=\operatorname{res}({\mathsf{u}})$ if~${\mathsf{u}}$
      is standard. If~${\mathsf{u}}$ is not standard then $\beta_2({\mathsf{s}})=0$ so
      ${y^{\mathcal{O}}}_3 {\psi^{\mathcal{O}}}_2f_{{\mathsf{s}}{\mathsf{s}}}=0={\psi^{\mathcal{O}}}_2{y^{\<{-e}\>}}_2f_{{\mathsf{s}}{\mathsf{s}}}$ by \autoref{P:betas}.
      Suppose then that ${\mathsf{u}}$ is standard so that
      \[  {\mathsf{s}}_{\downarrow3}={
\begin{tikzpicture}[scale=0.3,draw/.append style={thick,black},baseline=-2mm]
  \tableauRow=0
  \foreach \Row in {{{1,2},{3}}} {
  \tableauCol=1
  \foreach\k in \Row {
  \draw(\the\tableauCol,\the\tableauRow)+(-.5,-.5)rectangle++(.5,.5);
  \draw(\the\tableauCol,\the\tableauRow)node{\k};
  \global\advance\tableauCol by 1
  }
  \global\advance\tableauRow by -1
  }
\end{tikzpicture}
} \quad\text{and}\quad
          {\mathsf{u}}_{\downarrow3}={
\begin{tikzpicture}[scale=0.3,draw/.append style={thick,black},baseline=-2mm]
  \tableauRow=0
  \foreach \Row in {{{1,3},{2}}} {
  \tableauCol=1
  \foreach\k in \Row {
  \draw(\the\tableauCol,\the\tableauRow)+(-.5,-.5)rectangle++(.5,.5);
  \draw(\the\tableauCol,\the\tableauRow)node{\k};
  \global\advance\tableauCol by 1
  }
  \global\advance\tableauRow by -1
  }
\end{tikzpicture}
}.
      \]
      Using \autoref{P:betas} again,
      ${\psi^{\mathcal{O}}}_2{y^{\<{-e}\>}}_2f_{{\mathsf{s}}{\mathsf{s}}}=-\kappa_{\mathbf{i}}\beta_2({\mathsf{s}})[-e]f_{{\mathsf{u}}{\mathsf{s}}}$
      since ${y^{\mathcal{O}}}_2f_{{\mathsf{s}}{\mathsf{s}}}=0$. Similarly,
      ${y^{\mathcal{O}}}_3{\psi^{\mathcal{O}}}_2f_{{\mathsf{s}}{\mathsf{s}}}
               =\kappa_{\mathbf{i}}\beta_2({\mathsf{s}}){y^{\mathcal{O}}}_3f_{{\mathsf{u}}{\mathsf{s}}}
               =-\kappa_{\mathbf{i}}\beta_2({\mathsf{s}})[-e]f_{{\mathsf{u}}{\mathsf{s}}}.
      $
      Hence, ${y^{\mathcal{O}}}_3{\psi^{\mathcal{O}}}_2f_{{\mathsf{s}}{\mathsf{s}}} ={\psi^{\mathcal{O}}}_2{y^{\<{-e}\>}}_2f_{{\mathsf{s}}{\mathsf{s}}}$ in all cases,
      completing the proof.
    \end{proof}

    The proof of the next result explains why $\kappa_{\mathbf{i}}$ is needed in the
    definition of~${\psi^{\mathcal{O}}}_2$. Fix ${\mathsf{s}}\in{\mathop{\rm Std}\nolimits}({\mathbf{i}})$ such that  ${\mathbf{i}}\in I^\gamma_+$
    and $(2,3){\mathsf{s}}$ is standard. Then $\rho_2({\mathsf{s}})=2$ and either
    $e=3$ and $i_2\rightarrow i_3$, or $e>3$ and $i_2\operatorname{\:\rlap{\hspace*{0.25em}/}\text{---}\:} i_3$.  Hence,
    $i_2\not\in\set{i_3,i_3+1}$, so by \autoref{E:beta} and \autoref{D:psio}
    \begin{equation}\label{E:BetaTwo}
      \beta_2({\mathsf{s}})
      =\frac{t^{-\rho_2({\mathsf{s}})}\sqrt{-1}\sqrt{t}\sqrt{[3]}[\rho_2({\mathsf{s}})]}      {[1-\rho_2({\mathsf{s}})][2]}
      =-\frac{\sqrt{-1}\,\sqrt{[3]}}{\sqrt{t}}.
    \end{equation}
    We can now determine the quadratic relation for ${\psi^{\mathcal{O}}}_2$.

    \begin{Lemma}\label{L:Quadratic}
      Suppose that ${\mathbf{i}}\in I^\gamma$. Then
      \[({\psi^{\mathcal{O}}}_2)^2{f^{\mathcal{O}}_{\mathbf{i}}} \begin{dcases*}
        ({y^{\mathcal{O}}}_2-{y^{\mathcal{O}}}_3){f^{\mathcal{O}}_{\mathbf{i}}} &if $i_2\to i_3$,\\
        ({y^{\mathcal{O}}}_3-{y^{\mathcal{O}}}_2){f^{\mathcal{O}}_{\mathbf{i}}} &if $i_2\leftarrow i_3$,\\
        
        
        0,&if $i_2=i_3$,\\
        {f^{\mathcal{O}}_{\mathbf{i}}}&otherwise.
      \end{dcases*}
      \]
    \end{Lemma}

    \begin{proof}
      It is enough to consider the case when ${\mathbf{i}}\in I^\gamma_+$. Since
      ${f^{\mathcal{O}}_{\mathbf{i}}}\sum_{{\mathsf{s}}\in{\mathop{\rm Std}\nolimits}({\mathbf{i}})}\frac1{\gamma_{\mathsf{s}}}f_{{\mathsf{s}}{\mathsf{s}}}$ we are
      reduced to computing $({\psi^{\mathcal{O}}}_2)^2f_{{\mathsf{s}}{\mathsf{s}}}$, for ${\mathsf{s}}\in{\mathop{\rm Std}\nolimits}({\mathbf{i}})$ and
      ${\mathbf{i}}\in I^\gamma_+$. Fix~${\mathsf{s}}\in{\mathop{\rm Std}\nolimits}({\mathbf{i}})$ and let
      ${\mathsf{u}}=(2,3){\mathsf{s}}\in{\mathop{\rm Std}\nolimits}({\mathbf{j}})$. By \autoref{P:betas},
      \[
        ({\psi^{\mathcal{O}}}_2)^2f_{{\mathsf{s}}{\mathsf{s}}} = \kappa_{\mathbf{i}}\beta_2({\mathsf{s}}){\psi^{\mathcal{O}}}_2f_{{\mathsf{u}}{\mathsf{s}}}
                            = -\kappa_{\mathbf{i}}^2\beta_2({\mathsf{s}})^2f_{{\mathsf{s}}{\mathsf{s}}}.
      \]
      If $3$ is in the first row of~${\mathsf{s}}$ then ${\mathsf{u}}$ is not standard so
      $\beta_2({\mathsf{s}})=0$ and $({\psi^{\mathcal{O}}}_2)^2f_{{\mathsf{s}}{\mathsf{s}}}=0$.
      In this case, $i_2\to i_3$ and ${y^{\mathcal{O}}}_2 f_{{\mathsf{s}}{\mathsf{s}}}=0={y^{\mathcal{O}}}_3f_{{\mathsf{s}}{\mathsf{s}}}$, so the
      lemma holds. The only other possibility is that~$3$ is in the first
      column of~${\mathsf{s}}$, so that $\rho_2({\mathsf{s}})=2$. Then $i_2\to i_3$ if~$e=3$ and
      $i_2\operatorname{\:\rlap{\hspace*{0.25em}/}\text{---}\:} i_3$ if~$e>3$. Hence, using \autoref{D:psio} and
      \autoref{E:BetaTwo},
      \[   ({\psi^{\mathcal{O}}}_2)^2f_{{\mathsf{s}}{\mathsf{s}}}=
      -\kappa_{\mathbf{i}}^2\beta_2({\mathsf{s}})^2f_{{\mathsf{s}}{\mathsf{s}}}=\begin{dcases*}
        t^{-3}[3]f_{{\mathsf{s}}{\mathsf{s}}},&if $i_2\to i_3$ (and $e=3$),\\
        f_{{\mathsf{s}}{\mathsf{s}}},&if $i_2\operatorname{\:\rlap{\hspace*{0.25em}/}\text{---}\:} i_3$ (and $e>3$).
      \end{dcases*}
      \]
      Hence, if $i_2\operatorname{\:\rlap{\hspace*{0.25em}/}\text{---}\:} i_3$ then $({\psi^{\mathcal{O}}}_2)^2{f^{\mathcal{O}}_{\mathbf{i}}}{f^{\mathcal{O}}_{\mathbf{i}}} as claimed.
      Finally, if $i_2\to i_3$ then
      \[  ({y^{\mathcal{O}}}_2-{y^{\mathcal{O}}}_3)f_{{\mathsf{s}}{\mathsf{s}}}
                =(0-[-e])f_{{\mathsf{s}}{\mathsf{s}}}
                =t^{-3}[3]f_{{\mathsf{s}}{\mathsf{s}}}
                =({\psi^{\mathcal{O}}}_2)^2f_{{\mathsf{s}}{\mathsf{s}}},
      \]
      where the middle equality holds only because $e=3$. This completes the proof.
    \end{proof}

    \begin{Remark*}
      The proof of \autoref{L:Quadratic} suggests that $\kappa_{\mathbf{i}}$ is
      uniquely determined, for ${\mathbf{i}}\in I^\gamma$. In fact, this is not
      quite true. What the proof shows is that the value of $\kappa_{\mathbf{i}}$
      is uniquely determined by the quadratic relation satisfied
      by~${\psi^{\mathcal{O}}}_2$. For the proof of our main results we only need
      ${\psi^{\mathcal{O}}}_2$ to satisfy a ``deformed'' version of the quadratic
      relation for $\psi_2$ in \autoref{L:Quadratic}. For example, we
      can obtain slightly different relations by replacing~${y^{\mathcal{O}}}_r$
      with~${y^{\<{ke}\>}}_r$, for some $k\in{\mathbb{Z}}$. Such relations would require
      a different value for~$\kappa_{\mathbf{i}}$.
    \end{Remark*}

    Finally, it remains to check the braid relation for ${\psi^{\mathcal{O}}}_2$ and~${\psi^{\mathcal{O}}}_3$.

    \begin{Lemma}
      Suppose that ${\mathbf{i}}\in I^\gamma$. Then
      \begin{align*}
        {\psi^{\mathcal{O}}}_2{\psi^{\mathcal{O}}}_3{\psi^{\mathcal{O}}}_2 {f^{\mathcal{O}}_{\mathbf{i}}}= \begin{cases}
          ({\psi^{\mathcal{O}}}_3{\psi^{\mathcal{O}}}_2{\psi^{\mathcal{O}}}_3-1){f^{\mathcal{O}}_{\mathbf{i}}} &\text{if }i_2=i_4\to i_3,\\
          ({\psi^{\mathcal{O}}}_3{\psi^{\mathcal{O}}}_2{\psi^{\mathcal{O}}}_3+1){f^{\mathcal{O}}_{\mathbf{i}}} &\text{if }i_2=i_4\leftarrow i_3,\\
          
          
          
          
          {\psi^{\mathcal{O}}}_3{\psi^{\mathcal{O}}}_2{\psi^{\mathcal{O}}}_3{f^{\mathcal{O}}_{\mathbf{i}}}&\text{otherwise,}
        \end{cases}
      \end{align*}
    \end{Lemma}

    \begin{proof}Again, it is enough to consider the case when
      ${\mathbf{i}}\in I^\gamma_+$. We fix ${\mathsf{s}}\in{\mathop{\rm Std}\nolimits}({\mathbf{i}})$ and show that the two
      sides of the identity in the lemma act in the same way on~$f_{{\mathsf{s}}{\mathsf{s}}}$.
      Let ${\mathbf{j}}=(2,4)\cdot{\mathbf{i}}\in I^\gamma$.
      By \autoref{L:psi2Intertwiner} and \autoref{L:SimplePsi2Relations},
      ${\psi^{\mathcal{O}}}_2{\psi^{\mathcal{O}}}_3{\psi^{\mathcal{O}}}_2 {f^{\mathcal{O}}_{\mathbf{i}}}{f^{\mathcal{O}}_{{\mathbf{j}}}}_2{\psi^{\mathcal{O}}}_3{\psi^{\mathcal{O}}}_2$,
      so  ${\psi^{\mathcal{O}}}_2{\psi^{\mathcal{O}}}_3{\psi^{\mathcal{O}}}_2 {f^{\mathcal{O}}_{\mathbf{i}}}0$ unless ${\mathbf{j}}$ is the
      residue sequence of a standard tableau. Similarly
      ${\psi^{\mathcal{O}}}_3{\psi^{\mathcal{O}}}_2{\psi^{\mathcal{O}}}_3 {f^{\mathcal{O}}_{\mathbf{i}}}0$ unless ${\mathbf{j}}$ is the
      residue sequence of a standard tableau. Let ${\mathsf{s}}_{\downarrow4}$ be the
      subtableau of~${\mathsf{s}}$ containing the numbers $1,2,3,4$. Then
      \[
      {\mathsf{s}}_{\downarrow4}\in\set[\Bigg]{ {
\begin{tikzpicture}[scale=0.3,draw/.append style={thick,black},baseline=-4mm]
  \tableauRow=0
  \foreach \Row in {{{1,2,3,4}}} {
  \tableauCol=1
  \foreach\k in \Row {
  \draw(\the\tableauCol,\the\tableauRow)+(-.5,-.5)rectangle++(.5,.5);
  \draw(\the\tableauCol,\the\tableauRow)node{\k};
  \global\advance\tableauCol by 1
  }
  \global\advance\tableauRow by -1
  }
\end{tikzpicture}
},  \quad
      {
\begin{tikzpicture}[scale=0.3,draw/.append style={thick,black},baseline=-4mm]
  \tableauRow=0
  \foreach \Row in {{{1,2,3},{4}}} {
  \tableauCol=1
  \foreach\k in \Row {
  \draw(\the\tableauCol,\the\tableauRow)+(-.5,-.5)rectangle++(.5,.5);
  \draw(\the\tableauCol,\the\tableauRow)node{\k};
  \global\advance\tableauCol by 1
  }
  \global\advance\tableauRow by -1
  }
\end{tikzpicture}
},\quad
      {
\begin{tikzpicture}[scale=0.3,draw/.append style={thick,black},baseline=-4mm]
  \tableauRow=0
  \foreach \Row in {{{1,2,4},{3}}} {
  \tableauCol=1
  \foreach\k in \Row {
  \draw(\the\tableauCol,\the\tableauRow)+(-.5,-.5)rectangle++(.5,.5);
  \draw(\the\tableauCol,\the\tableauRow)node{\k};
  \global\advance\tableauCol by 1
  }
  \global\advance\tableauRow by -1
  }
\end{tikzpicture}
},\quad
      {
\begin{tikzpicture}[scale=0.3,draw/.append style={thick,black},baseline=-4mm]
  \tableauRow=0
  \foreach \Row in {{{1,2},{3,4}}} {
  \tableauCol=1
  \foreach\k in \Row {
  \draw(\the\tableauCol,\the\tableauRow)+(-.5,-.5)rectangle++(.5,.5);
  \draw(\the\tableauCol,\the\tableauRow)node{\k};
  \global\advance\tableauCol by 1
  }
  \global\advance\tableauRow by -1
  }
\end{tikzpicture}
},\quad
      {
\begin{tikzpicture}[scale=0.3,draw/.append style={thick,black},baseline=-4mm]
  \tableauRow=0
  \foreach \Row in {{{1,2},{3},{4}}} {
  \tableauCol=1
  \foreach\k in \Row {
  \draw(\the\tableauCol,\the\tableauRow)+(-.5,-.5)rectangle++(.5,.5);
  \draw(\the\tableauCol,\the\tableauRow)node{\k};
  \global\advance\tableauCol by 1
  }
  \global\advance\tableauRow by -1
  }
\end{tikzpicture}
} }
      \]
      since ${\mathbf{i}}\in I^\gamma_+$. We consider two cases.

      \smallskip\noindent\textbf{Case 1: $e>3$:}
      Inspecting the list of possibilities for ${\mathsf{s}}_{\downarrow4}$, in all cases
      $i_2\ne i_4$ and  ${\mathbf{j}}\ne\operatorname{res}({\mathsf{t}})$ for any standard tableau~${\mathsf{t}}$. Therefore,
      \[ {\psi^{\mathcal{O}}}_2{\psi^{\mathcal{O}}}_3{\psi^{\mathcal{O}}}_2 {f^{\mathcal{O}}_{\mathbf{i}}}0={\psi^{\mathcal{O}}}_3{\psi^{\mathcal{O}}}_2{\psi^{\mathcal{O}}}_3{f^{\mathcal{O}}_{\mathbf{i}}}\]
      in agreement with the statement of the lemma.

      \smallskip\noindent\textbf{Case 2: $e=3$:}
      Except for the last tableau in the set above, $i_2\ne i_4$ and
      ${\mathbf{j}}$ is not a residue sequence for a standard tableau. Hence, as
      in Case~1, the lemma holds when $i_2\ne i_4$ as both sides are
      zero. Moreover, if ${f^{\mathcal{O}}_{\mathbf{i}}}0$ then the case $i_2=i_4\leftarrow i_3$
      does not arise, so the lemma is vacuously true in this case.  It
      remains to consider the case when $i_2=i_4\to i_3$, which occurs
      only if $s_{\downarrow4}$ is the last tableau in the set above.
      Noting that ${\psi^{\mathcal{O}}}_3f_{{\mathsf{s}}{\mathsf{s}}}=0$, \autoref{P:betas} and
      \autoref{E:BetaTwo} quickly imply that
      \[ \big({\psi^{\mathcal{O}}}_2{\psi^{\mathcal{O}}}_3{\psi^{\mathcal{O}}}_2-{\psi^{\mathcal{O}}}_3{\psi^{\mathcal{O}}}_2{\psi^{\mathcal{O}}}_3\big)f_{{\mathsf{s}}{\mathsf{s}}}
      =-\kappa_{\mathbf{i}}^2\beta_2({\mathsf{s}})^2\frac{t^3}{[3]}f_{{\mathsf{s}}{\mathsf{s}}}
      =-f_{{\mathsf{s}}{\mathsf{s}}},
      \]
      where the last equality follows using \autoref{E:BetaTwo} exactly as
      in the proof of \autoref{L:Quadratic}. This completes the proof.
    \end{proof}

    \subsection{The isomorphism $\RAn\cong{{\mathscr{H}}_\xi({\mathfrak{A}_{n}}}}\label{S:Main}
    We now have almost everything in place that we need to prove
    \autoref{T:Main}. We first prove a stronger version of
    \autoref{T:BKiso} over~${\mathcal{O}}$. For this we need the following
    definition, which should be viewed as an ${\mathcal{O}}$-deformation of~${\mathscr{R}_e({\mathfrak{S}_{n}}}.
    The reader should compare this result with \autoref{T:HMGradedIso}.

    \begin{Definition}\label{D:RO}
      Suppose that $\gamma\in{Q^\varepsilon_n}$. Let ${\dot R^{\mathcal{O}}_\gamma} be the unital
      associative ${\mathcal{O}}$-algebra generated by the elements
      \[
      \set{{{\dot\psi}^{\mathcal{O}}}_1,{{\dot\psi}^{\mathcal{O}}}_2,\ldots,{{\dot\psi}^{\mathcal{O}}}_{n-1}}
      \cup\set{{\dot{y}^{\mathcal{O}}}_1,{\dot{y}^{\mathcal{O}}}_2,\ldots,{\dot{y}^{\mathcal{O}}}_n}
      \cup\set{{\dot{f}^{\mathcal{O}}_{\mathbf{i}}} {\mathbf{i}}\in I^\gamma}
      \]
      subject to the relations
      {\setlength{\abovedisplayskip}{2pt}
      \setlength{\belowdisplayskip}{1pt}
      \begin{alignat*}{3}
        ({\dot{y}^{\mathcal{O}}}_1)^{(\Lambda_0,\alpha_{i_1})}{\dot{f}^{\mathcal{O}}_{\mathbf{i}}}=0,
        &{\dot{f}^{\mathcal{O}}_{\mathbf{i}}}[{\mathbf{j}}]&= \delta_{{{\mathbf{i}}}{{\mathbf{j}}}}{\dot{f}^{\mathcal{O}}_{\mathbf{i}}}
        &\textstyle\sum_{{{\mathbf{i}}}\in I^\gamma}{\dot{f}^{\mathcal{O}}_{\mathbf{i}}}= 1,\\
        {\dot{y}^{\mathcal{O}}}_t {\dot{f}^{\mathcal{O}}_{\mathbf{i}}}= {\dot{f}^{\mathcal{O}}_{\mathbf{i}}}_t,&\qquad
        {{\dot\psi}^{\mathcal{O}}}_r {\dot{f}^{\mathcal{O}}_{\mathbf{i}}}= {\dot{f}^{\mathcal{O}}_{s_r\cdot{{\mathbf{i}}}}}_r,&
        {\dot{y}^{\mathcal{O}}}_r {\dot{y}^{\mathcal{O}}}_t&= {\dot{y}^{\mathcal{O}}}_t {\dot{y}^{\mathcal{O}}}_r,
      \end{alignat*}
      \begin{align*}
        {{\dot\psi}^{\mathcal{O}}}_1 {\dot{y}^{\mathcal{O}}}_2=\big({\dot{y}^{\mathcal{O}}}_1{{\dot\psi}^{\mathcal{O}}}_1+\delta_{i_1i_2}\big){\dot{f}^{\mathcal{O}}_{\mathbf{i}}}
        \qquad
        {\dot{y}^{\mathcal{O}}}_2 {{\dot\psi}^{\mathcal{O}}}_1=\big({{\dot\psi}^{\mathcal{O}}}_1{\dot{y}^{\mathcal{O}}}_1+\delta_{i_1i_2}\big){\dot{f}^{\mathcal{O}}_{\mathbf{i}}}\\
        {{\dot\psi}^{\mathcal{O}}}_2 {\dot{y}^{\mathcal{O}}}_3=\big({\dot{y}^{\<{-e}\>}}_2{{\dot\psi}^{\mathcal{O}}}_2+\delta_{i_2i_3}\big){\dot{f}^{\mathcal{O}}_{\mathbf{i}}}
        \qquad
        {\dot{y}^{\mathcal{O}}}_3 {{\dot\psi}^{\mathcal{O}}}_2=\big({{\dot\psi}^{\mathcal{O}}}_2{\dot{y}^{\<{-e}\>}}_2+\delta_{i_2i_3}\big){\dot{f}^{\mathcal{O}}_{\mathbf{i}}}
      \end{align*}
      \begin{align*}
        {{\dot\psi}^{\mathcal{O}}}_r {\dot{y}^{\mathcal{O}}}_t&= {\dot{y}^{\mathcal{O}}}_t{{\dot\psi}^{\mathcal{O}}}_r,   \quad\text{if }t\neq r,r+1,\\
        {{\dot\psi}^{\mathcal{O}}}_r{{\dot\psi}^{\mathcal{O}}}_s&= {{\dot\psi}^{\mathcal{O}}}_s{{\dot\psi}^{\mathcal{O}}}_r,\quad\text{if }\left|r-s\right|>1,
      \end{align*}
      if $r=1$ or $r=2$ then
      \begin{align*}
        ({{\dot\psi}^{\mathcal{O}}}_r)^2{\dot{f}^{\mathcal{O}}_{\mathbf{i}}}= \begin{cases}
          ({\dot{y}^{\mathcal{O}}}_r-{\dot{y}^{\mathcal{O}}}_{r+1}){\dot{f}^{\mathcal{O}}_{\mathbf{i}}}
          &\text{if }i_r\rightarrow i_{r+1},\\
          ({\dot{y}^{\mathcal{O}}}_{r+1}-{\dot{y}^{\mathcal{O}}}_r){\dot{f}^{\mathcal{O}}_{\mathbf{i}}}
          &\text{if }i_r\leftarrow i_{r+1},\\
          0,&\text{if }i_r=i_{r+1},\\
          {\dot{f}^{\mathcal{O}}_{\mathbf{i}}}&\text{otherwise,}
        \end{cases}
      \end{align*}
      \begin{align*}
        {{\dot\psi}^{\mathcal{O}}}_r{{\dot\psi}^{\mathcal{O}}}_{r+1}{{\dot\psi}^{\mathcal{O}}}_r {\dot{f}^{\mathcal{O}}_{\mathbf{i}}}= \begin{cases}
          ({{\dot\psi}^{\mathcal{O}}}_{r+1}{{\dot\psi}^{\mathcal{O}}}_r{{\dot\psi}^{\mathcal{O}}}_{r+1}-1){\dot{f}^{\mathcal{O}}_{\mathbf{i}}}
          &\text{if }i_r=i_{r+2}\rightarrow i_{r+1},\\
          ({{\dot\psi}^{\mathcal{O}}}_{r+1}{{\dot\psi}^{\mathcal{O}}}_r{{\dot\psi}^{\mathcal{O}}}_{r+1}+1){\dot{f}^{\mathcal{O}}_{\mathbf{i}}}
          &\text{if }i_r=i_{r+2}\leftarrow i_{r+1},\\
          {{\dot\psi}^{\mathcal{O}}}_{r+1}{{\dot\psi}^{\mathcal{O}}}_r{{\dot\psi}^{\mathcal{O}}}_{r+1}{\dot{f}^{\mathcal{O}}_{\mathbf{i}}}&\text{otherwise},
        \end{cases}
      \end{align*}
      and if $2<r<n$ then
      \begin{align*}
        {{\dot\psi}^{\mathcal{O}}}_r {\dot{y}^{\mathcal{O}}}_{r+1}{\dot{f}^{\mathcal{O}}_{\mathbf{i}}} ({\dot{y}^{\mathcal{O}}}_r {{\dot\psi}^{\mathcal{O}}}_r+\delta_{i_r i_{r+1}}){\dot{f}^{\mathcal{O}}_{\mathbf{i}}}
        \qquad
        {\dot{y}^{\mathcal{O}}}_{r+1}{{\dot\psi}^{\mathcal{O}}}_r {\dot{f}^{\mathcal{O}}_{\mathbf{i}}} ({{\dot\psi}^{\mathcal{O}}}_r {\dot{y}^{\mathcal{O}}}_r+\delta_{i_r i_{r+1}}){\dot{f}^{\mathcal{O}}_{\mathbf{i}}}
      \end{align*}
      \begin{align*}
        ({{\dot\psi}^{\mathcal{O}}}_r)^2{\dot{f}^{\mathcal{O}}_{\mathbf{i}}}= \begin{cases}
          ({\dot{y}^{\<{1+\rho_r({\mathbf{i}})}\>}}_r-{\dot{y}^{\mathcal{O}}}_{r+1}){\dot{f}^{\mathcal{O}}_{\mathbf{i}}}
          &\text{if $i_r\rightarrow i_{r+1}$ and ${\mathbf{i}}\in I^\gamma_+$}\\
          ({\dot{y}^{\mathcal{O}}}_{r+1}-{\dot{y}^{\<{1-\rho_r({\mathbf{i}})}\>}}_r){\dot{f}^{\mathcal{O}}_{\mathbf{i}}}
          &\text{if $i_r\leftarrow i_{r+1}$ and ${\mathbf{i}}\in I^\gamma_-$}\\
          ({\dot{y}^{\<{1-\rho_r({\mathbf{i}})}\>}}_{r+1}-{\dot{y}^{\mathcal{O}}}_{r}){\dot{f}^{\mathcal{O}}_{\mathbf{i}}}
          &\text{if $i_r\leftarrow i_{r+1}$ and ${\mathbf{i}}\in I^\gamma_+$}\\
          ({\dot{y}^{\mathcal{O}}}_{r}-{\dot{y}^{\<{1+\rho_r({\mathbf{i}})}\>}}_{r+1}){\dot{f}^{\mathcal{O}}_{\mathbf{i}}}
          &\text{if $i_r\rightarrow i_{r+1}$ and ${\mathbf{i}}\in I^\gamma_-$}\\
          0,&\text{if }i_r=i_{r+1},\\
          {\dot{f}^{\mathcal{O}}_{\mathbf{i}}}&\text{otherwise,}
        \end{cases}
      \end{align*}
      \begin{align*}
        {{\dot\psi}^{\mathcal{O}}}_r{{\dot\psi}^{\mathcal{O}}}_{r+1}{{\dot\psi}^{\mathcal{O}}}_r {\dot{f}^{\mathcal{O}}_{\mathbf{i}}}= \begin{cases}
          ({{\dot\psi}^{\mathcal{O}}}_{r+1}{{\dot\psi}^{\mathcal{O}}}_r{{\dot\psi}^{\mathcal{O}}}_{r+1}-t^{1+\rho_r({{\mathbf{i}}})}){\dot{f}^{\mathcal{O}}_{\mathbf{i}}}
            &\text{if $i_r=i_{r+2}\rightarrow i_{r+1}$ and ${\mathbf{i}}\in I^\gamma_+$,}\\
          ({{\dot\psi}^{\mathcal{O}}}_{r+1}{{\dot\psi}^{\mathcal{O}}}_r{{\dot\psi}^{\mathcal{O}}}_{r+1}+t^{1-\rho_r({{\mathbf{i}}})}){\dot{f}^{\mathcal{O}}_{\mathbf{i}}}
            &\text{if $i_r=i_{r+2}\leftarrow i_{r+1}$ and ${\mathbf{i}}\in I^\gamma_-$,}\\
          ({{\dot\psi}^{\mathcal{O}}}_{r+1}{{\dot\psi}^{\mathcal{O}}}_r{{\dot\psi}^{\mathcal{O}}}_{r+1}-1){\dot{f}^{\mathcal{O}}_{\mathbf{i}}}
            &\text{if $i_r=i_{r+2}\rightarrow i_{r+1}$ and ${\mathbf{i}}\in I^\gamma_-$,}\\
          ({{\dot\psi}^{\mathcal{O}}}_{r+1}{{\dot\psi}^{\mathcal{O}}}_r{{\dot\psi}^{\mathcal{O}}}_{r+1}+1){\dot{f}^{\mathcal{O}}_{\mathbf{i}}}
            &\text{if $i_r=i_{r+2}\leftarrow i_{r+1}$ and ${\mathbf{i}}\in I^\gamma_+$,}\\
          {{\dot\psi}^{\mathcal{O}}}_{r+1}{{\dot\psi}^{\mathcal{O}}}_r{{\dot\psi}^{\mathcal{O}}}_{r+1}{\dot{f}^{\mathcal{O}}_{\mathbf{i}}}&\text{otherwise,}
        \end{cases}
      \end{align*}
      }\noindent      for all admissible ${\mathbf{i}},{\mathbf{j}}\in I^\gamma$ and all admissible $r,s$ and $t$
      and where for $d\in{\mathbb{Z}}$
      \[   {\dot{y}^{\<{d}\>}}_r{\dot{f}^{\mathcal{O}}_{\mathbf{i}}}\begin{dcases*}
                    (t^d{\dot{y}^{\mathcal{O}}}_r-[d]){\dot{f}^{\mathcal{O}}_{\mathbf{i}}}& if ${\mathbf{i}}\in I^\gamma_+$,\\
                    (t^d{\dot{y}^{\mathcal{O}}}_r+[d]){\dot{f}^{\mathcal{O}}_{\mathbf{i}}}& if ${\mathbf{i}}\in I^\gamma_-$.
           \end{dcases*}
      \]
      If ${\mathbb{F}}$ is an ${\mathcal{O}}$-module let ${\dot R^{\mathbb{F}}_\gamma}{\dot R^{\mathcal{O}}_\gamma}_{\mathcal{O}}{\mathbb{F}}$.
    \end{Definition}

    To show that ${\dot R^{\mathcal{O}}_\gamma} is finitely generated as an ${\mathcal{O}}$-module we need
    the following technical lemma, which is an analogue of
    \cite[Lemma~4.31]{HuMathas:SeminormalQuiver}.

    \begin{Lemma}\label{L:yorder}
      Suppose that $1\leq r\leq n$ and ${{\mathbf{i}}}\in I^\gamma$. If ${\mathbf{i}}\notin
      I^\gamma_+\cup I^\gamma_-$ then ${\dot{f}^{\mathcal{O}}_{\mathbf{i}}}0$ and if ${\mathbf{i}}\in
      I^\gamma_\pm$ then there exists a set $X_r({{\mathbf{i}}})\subseteq
      e{\mathbb{Z}}\times{\mathbb{N}}$ such that
      \[
      \prod_{(c,m)\in X_r({{\mathbf{i}}})}({\dot{y}^{\mathcal{O}}}_r\mp[c])^m{f^{\mathcal{O}}_{\mathbf{i}}}0
      \]
      in ${\dot R^{\mathcal{O}}_\gamma}.
    \end{Lemma}

    \begin{proof}
      Arguing exactly as in proof of \autoref{L:ZeroSequence}, if
      ${\mathbf{i}}\in I^\gamma$ then ${\dot{f}^{\mathcal{O}}_{\mathbf{i}}}0$ only if $i_1=0$
      and $i_2=\pm1$.  That is, ${\dot{f}^{\mathcal{O}}_{\mathbf{i}}}0$ only if ${\mathbf{i}}\in
      I^\gamma_+\cup I^\gamma_-$. Hence, we may assume that
      ${\mathbf{i}}\in I^\gamma_+\cup I^\gamma_-$.

      Checking the relations in \autoref{D:RO}, ${\dot R^{\mathcal{O}}_\gamma} has an
      automorphism~${\dot\#}$ such that
      \[    ({{\dot\psi}^{\mathcal{O}}}_r)^{\dot\#}=-{{\dot\psi}^{\mathcal{O}}}_r, \quad
            ({\dot{y}^{\mathcal{O}}}_s)^{\dot\#}=-{\dot{y}^{\mathcal{O}}}_s\quad\text{and}\quad
            ({\dot{f}^{\mathcal{O}}_{\mathbf{i}}}^{\dot\#}={\dot{f}^{\mathcal{O}}_{-{\mathbf{i}}}}
      \]
      for all $1\le r<n$, $1\le s\le n$ and ${\mathbf{i}}\in I^\gamma$.
      Therefore, it is enough to consider the case when ${\mathbf{i}}\in I^\gamma_+$.

      By \autoref{D:RO}, ${\dot{y}^{\mathcal{O}}}_1{\dot{f}^{\mathcal{O}}_{\mathbf{i}}}0$, so we may take
      $X_1({\mathbf{i}})=\set{(0,1)}$.  As
      ${{\dot\psi}^{\mathcal{O}}}_1{\dot{f}^{\mathcal{O}}_{\mathbf{i}}}{\dot{f}^{\mathcal{O}}_{s_1\cdot{\mathbf{i}}}}_1$, it follows that
      ${{\dot\psi}^{\mathcal{O}}}_1=0$.  Therefore, if ${\dot{f}^{\mathcal{O}}_{\mathbf{i}}}0$ then
      $0=({{\dot\psi}^{\mathcal{O}}}_1)^2{\dot{f}^{\mathcal{O}}_{\mathbf{i}}}({\dot{y}^{\mathcal{O}}}_1-{\dot{y}^{\mathcal{O}}}_2){\dot{f}^{\mathcal{O}}_{\mathbf{i}}}-{\dot{y}^{\mathcal{O}}}_2{\dot{f}^{\mathcal{O}}_{\mathbf{i}}}, so
      ${\dot{y}^{\mathcal{O}}}_2=0$.  Hence, we can set $X_2({\mathbf{i}})=\set{(0,1)}$, for all
      ${\mathbf{i}}\in I^\gamma$.

      Now consider ${\dot{y}^{\mathcal{O}}}_3{\dot{f}^{\mathcal{O}}_{\mathbf{i}}}, for ${\mathbf{i}}\in I^\gamma_+$. If $i_2=i_3$ then the
      commutation relations for~${{\dot\psi}^{\mathcal{O}}}_2$ and~${\dot{y}^{\mathcal{O}}}_3$ give
      ${\dot{f}^{\mathcal{O}}_{\mathbf{i}}}({\dot{y}^{\mathcal{O}}}_3{{\dot\psi}^{\mathcal{O}}}_2-{{\dot\psi}^{\mathcal{O}}}_2{\dot{y}^{\<{-e}\>}}_2){\dot{f}^{\mathcal{O}}_{\mathbf{i}}}({\dot{y}^{\mathcal{O}}}_3+[-e]){{\dot\psi}^{\mathcal{O}}}_2{\dot{f}^{\mathcal{O}}_{\mathbf{i}}} since ${\dot{y}^{\mathcal{O}}}_2=0$.
      Similarly, ${\dot{f}^{\mathcal{O}}_{\mathbf{i}}}{{\dot\psi}^{\mathcal{O}}}_2({\dot{y}^{\mathcal{O}}}_3+[-e]){\dot{f}^{\mathcal{O}}_{\mathbf{i}}}. Therefore,
      ${\dot{f}^{\mathcal{O}}_{\mathbf{i}}}({\dot{y}^{\mathcal{O}}}_3+[e])({{\dot\psi}^{\mathcal{O}}}_2)^2({\dot{y}^{\mathcal{O}}}_3+[-e]){\dot{f}^{\mathcal{O}}_{\mathbf{i}}}0$.
      Hence, we can assume that $i_2\ne i_3$. If $i_2\operatorname{\:\rlap{\hspace*{0.25em}/}\text{---}\:} i_3$ and
      ${\mathbf{i}}\in I^\gamma_+$ then
      \begin{align*}
      ({\dot{y}^{\mathcal{O}}}_3-[-e]){f^{\mathcal{O}}_{\mathbf{i}}}=({\dot{y}^{\mathcal{O}}}_3-[-e])({{\dot\psi}^{\mathcal{O}}}_2)^2{\dot{f}^{\mathcal{O}}_{\mathbf{i}}}({\dot{y}^{\mathcal{O}}}_3-[-e]){{\dot\psi}^{\mathcal{O}}}_2{\dot{f}^{\mathcal{O}}_{s_2\cdot{\mathbf{i}}}}_2\\
                      &={{\dot\psi}^{\mathcal{O}}}_2({\dot{y}^{\mathcal{O}}}_2+[-e]-[-e]){\dot{f}^{\mathcal{O}}_{s_2\cdot{\mathbf{i}}}}_2
                       =0.
      \end{align*}
      Hence, if $i_2\operatorname{\:\rlap{\hspace*{0.25em}/}\text{---}\:} i_3$ set $X_3({\mathbf{i}})=\set{(-e,1)}$.
      Similarly, if $i_2\to i_3$ then
      \begin{align*}
        ({\dot{y}^{\mathcal{O}}}_3-[-e]){\dot{y}^{\mathcal{O}}}_3{\dot{f}^{\mathcal{O}}_{\mathbf{i}}}=({\dot{y}^{\mathcal{O}}}_3-[-e])({\dot{y}^{\mathcal{O}}}_2-{{\dot\psi}^{\mathcal{O}}}_2)^2{\dot{f}^{\mathcal{O}}_{\mathbf{i}}}-({\dot{y}^{\mathcal{O}}}_3-[-e])({{\dot\psi}^{\mathcal{O}}}_2)^2{\dot{f}^{\mathcal{O}}_{\mathbf{i}}}0.
      \end{align*}
      Consequently, we can set $X_3({\mathbf{i}})=\set{(-e,1),(0,1)}$.
      The case when $i_2\leftarrow i_3$ is similar and easier with
      $X_3({\mathbf{i}})=\set{(0,1)}$.

      The last two paragraphs show that if $1\le r\le 3$ and ${\mathbf{i}}\in I^\gamma_+$
      then there exists a set $X_r({\mathbf{i}})\subseteq e{\mathbb{Z}}\times{\mathbb{N}}$ such that
      $\prod_{(c,m)\in X_r({\mathbf{i}})}({\dot{y}^{\mathcal{O}}}_r-[c])^m{\dot{f}^{\mathcal{O}}_{\mathbf{i}}}0$. If $3\le r<n$ and ${\mathbf{i}}\in
      I^\gamma_+$ then $s_r\cdot{\mathbf{i}}\in I^\gamma_+$. Moreover, the
      elements ${{\dot\psi}^{\mathcal{O}}}_3,\dots,{{\dot\psi}^{\mathcal{O}}}_{n-1}$ and ${\dot{y}^{\mathcal{O}}}_4,\dots,{\dot{y}^{\mathcal{O}}}_n$
      satisfy the same defining relations as ${\psi^+}_3,\dots,{\psi^+}_{n-1}$,
      ${y^+}_4,\dots,{y^+}_n$. Therefore, the inductive argument in
      \cite[Lemma~4.31]{HuMathas:SeminormalQuiver} shows that
      there exists a set $X_r({\mathbf{i}})\subset e{\mathbb{Z}}\times{\mathbb{N}}$ such that
      \[\prod_{(c,m)\in X_r({\mathbf{i}})}({\dot{y}^{\mathcal{O}}}_r-[c])^m{\dot{f}^{\mathcal{O}}_{\mathbf{i}}}0,
          \qquad\text{ for }{\mathbf{i}}\in I^\gamma_+\text{ and }1\le r\le n.
      \]
      (Note that if $1\le r\le 3$, or if ${\mathbf{i}}\in I^\gamma_-$, then the
      argument from \cite{HuMathas:SeminormalQuiver} does not apply
      because ${{\dot\psi}^{\mathcal{O}}}_1$ and~${{\dot\psi}^{\mathcal{O}}}_2$ satisfy slightly different
      relations to the corresponding elements considered in that paper.)
    \end{proof}

    Finally, we are able to prove the enhanced version of \autoref{T:BKiso}
    that we use to prove our main result. If $A$ is an ${\mathcal{O}}$-algebra let
    ${{\mathscr{H}}_\xi^{A}({\mathfrak{S}_{n}}}{{\mathscr{H}}^{\mathcal{O}}_{t}}_{\mathcal{O}} A$.

    \begin{Theorem}\label{T:MainTheorem}
      Suppose that $\gamma\in{Q^\varepsilon_n}$ and that $({\mathcal{O}},t)$ is the idempotent
      subring defined in $\autoref{E:AlternatingCS}$. Then
      ${\dot R^{\mathcal{O}}_\gamma}{{\mathscr{H}}^{\mathcal{O}}_{\gamma}} as ${\mathcal{O}}$-algebras.
    \end{Theorem}

  \begin{proof}
    By the results in \autoref{S:AlternaatingCoefficients}, from
    \autoref{P:NewGenerators} onwards,
    there is a unique surjective algebra homomorphism
    ${\dot R^{\mathcal{O}}_\gamma}{{\mathscr{H}}^{\mathcal{O}}_{\gamma}} such that
    \[
    {{\dot\psi}^{\mathcal{O}}}_r\mapsto{\psi^{\mathcal{O}}}_r,\qquad
    {\dot{y}^{\mathcal{O}}}_s\mapsto{y^{\mathcal{O}}}_s\quad\text{and}\quad
    {\dot{f}^{\mathcal{O}}_{\mathbf{i}}}{f^{\mathcal{O}}_{\mathbf{i}}}
    \]
    for $1\le r<n$, $1\le s\le n$ and ${\mathbf{i}}\in I^\gamma$. To prove that this
    map is an isomorphism we use the argument from
    \cite[Theorem~4.32]{HuMathas:SeminormalQuiver} to show that ${\dot R^{\mathcal{O}}_\gamma} is
    free as an ${\mathcal{O}}$-module with the same rank as~${{\mathscr{H}}^{\mathcal{O}}_{\gamma}}.

    First, using the relations in \autoref{D:RO} it is straightforward to
    show that ${\dot R^{\mathcal{O}}_\gamma} is spanned by elements of the form $f_w(\dot y){{\dot\psi}^{\mathcal{O}}}_w{\dot{f}^{\mathcal{O}}_{\mathbf{i}}},
    where $f_w(\dot y)$ is a polynomial in ${\mathcal{O}}[{\dot{y}^{\mathcal{O}}}_1,\dots,{\dot{y}^{\mathcal{O}}}_n]$,
    ${\mathbf{i}}\in I^\gamma$ and for each $w\in{\mathfrak{S}_{n}} we fix a reduced expression
    $w=s_{r_1}\dots s_{r_k}$ and set
    ${{\dot\psi}^{\mathcal{O}}}_w={{\dot\psi}^{\mathcal{O}}}_{r_1}\dots{\psi^{\mathcal{O}}}_{r_k}$. Hence, ${\dot R^{\mathcal{O}}_\gamma} is finitely
    generated as an ${\mathcal{O}}$-module by \autoref{L:yorder}.

    Next, let ${\mathfrak{m}}=x{\mathcal{O}}$ be the maximal ideal of~${\mathcal{O}}$ and set ${\mathbb{F}}={\mathcal{O}}/{\mathfrak{m}}$ and
    $\xi=t+{\mathfrak{m}}\in{\mathbb{F}}$. Then~$\xi$ has quantum characteristic~$e$ because if
    $k\in{\mathbb{Z}}$ then $[k]_t\in{\mathcal{J}}({\mathcal{O}})={\mathfrak{m}}$ if and only if $k\in e{\mathbb{Z}}$ by
    \autoref{D:idempotentSub}. By \autoref{D:RO}, the relations in
    ${\dot R^{\mathbb{F}}_\gamma} collapse and become the KLR relations for ${\mathscr{R}^{{\mathbb{F}}}_e({\mathfrak{S}_{n}}}\gamma$
    given in \autoref{D:klr}. That is, ${\dot R^{\mathbb{F}}_\gamma}{\mathscr{R}^{{\mathbb{F}}}_e({\mathfrak{S}_{n}}}\gamma$ as
    ${\mathbb{F}}$-algebras. Consequently,
    \[\dim_{\mathbb{F}}{\dot R^{\mathbb{F}}_\gamma}\dim_{\mathbb{F}}{\mathscr{R}^{{\mathbb{F}}}_e({\mathfrak{S}_{n}}}\gamma=\operatorname{rank}_{\mathcal{O}}{{\mathscr{H}}^{\mathcal{O}}_{\gamma}}\]
    where the last equality follows by \cite[Theorem
    4.20]{BK:GradedDecomp} (alternatively, use \autoref{T:BKiso}).
    Since ${\mathfrak{m}}$ is the unique maximal ideal~${\mathfrak{m}}$ of ${\mathcal{O}}$, and ${\dot R^{\mathcal{O}}_\gamma} is
    finitely generated as an ${\mathcal{O}}$-module, Nakayama's Lemma implies
    that~${\dot R^{\mathcal{O}}_\gamma} is free as an ${\mathcal{O}}$-module of rank
    $\dim_{\mathbb{F}}{{\mathscr{H}}_\xi^{{\mathbb{F}}}({\mathfrak{S}_{n}}}\gamma=\operatorname{rank}_{\mathcal{O}}{{\mathscr{H}}^{\mathcal{O}}_{\gamma}}. Hence, as an
    ${\mathcal{O}}$-module, ${\dot R^{\mathcal{O}}_\gamma} is free of the same rank as~${{\mathscr{H}}^{\mathcal{O}}_{\gamma}}. Since
    ${{\mathscr{H}}^{\mathcal{O}}_{\gamma}} is also free over~${\mathcal{O}}$, it follows that the surjective
    algebra homomorphism ${\dot R^{\mathcal{O}}_\gamma}{{\mathscr{H}}^{\mathcal{O}}_{\gamma}} given in the
    first paragraph of the proof is actually an isomorphism and the
    theorem is proved.
  \end{proof}

  Recalling \autoref{E:HAnBlocks}, for $\gamma\in{Q^\varepsilon_n}$ define
  ${{\mathscr{H}}_\xi^{\mathbb{F}}({\mathfrak{A}_{n}}}\gamma={{\mathscr{H}}^{\mathcal{O}}_{t}}{\mathfrak{A}_{n}}_\gamma\otimes_{\mathcal{O}}{\mathbb{F}}$. Then
  ${{\mathscr{H}}_\xi^{\mathbb{F}}({\mathfrak{A}_{n}}}\gamma$ is a direct summand of ${{\mathscr{H}}_\xi^{\mathbb{F}}({\mathfrak{A}_{n}}} by
  \autoref{C:HAnDecomp}. By construction, $F$ is (isomorphic to) a subfield of~${\mathbb{F}}$
  and the algebra ${{\mathscr{H}}_\xi^{F}({\mathfrak{A}_{n}}} is the $F$-subalgebra of ${{\mathscr{H}}_\xi^{\mathbb{F}}({\mathfrak{A}_{n}}}
  generated by the elements $T_1,\dots,T_{n-1}$. By \autoref{L:foIdempotents} and
  \autoref{P:HOdecomp}, $e_\gamma={f^{\mathcal{O}}_{\gamma}}1_{\mathbb{F}}$ is central
  idempotent in~${{\mathscr{H}}_\xi^{F}({\mathfrak{A}_{n}}}. Define
  \[ {{\mathscr{H}}_\xi^{F}({\mathfrak{A}_{n}}}\gamma={{\mathscr{H}}_\xi^{F}({\mathfrak{A}_{n}}}_\gamma.  \]
  Then ${{\mathscr{H}}_\xi^{F}({\mathfrak{A}_{n}}}\gamma$ is the $F$-subalgebra of~${{\mathscr{H}}_\xi^{\mathbb{F}}({\mathfrak{A}_{n}}}\gamma$
  generated by $T_1e_\gamma,\dots,T_{n-1}e_\gamma$.

  We are assuming that $F$ is a field and that $\xi\in F$ an element of quantum
  characteristic~$e$.  Recall from before \autoref{T:Main} that a field~$F$ is
  \textbf{large enough} for~$\xi$ if~$F$ contains squareroots $\sqrt{\xi}$
  and $\sqrt{1+\xi+\xi^2}$ whenever $e>3$. (In particular, if $e=3$ then
  any field is large enough for~$\xi$.)

\begin{Theorem}\label{altiso}
  Suppose that $\gamma\in{Q^\varepsilon_n}$, $e>2$ and that $\xi\in F$ an element of
  quantum characteristic~$e$. Let $F$ be a large enough field for~$\xi$ of
  characteristic different from~$2$. Then ${{\mathscr{H}}_\xi^{F}({\mathfrak{A}_{n}}}\gamma\cong{\mathscr{R}^{F}_e({\mathfrak{A}_{n}}}\gamma$.
\end{Theorem}

\begin{proof}
  Let $({\mathcal{O}},t)$ be the idempotent subring given in
  \autoref{L:IdempotentSubring}, starting from~$F$ and~$\xi$, and let
  ${\mathbb{F}}={\mathcal{O}}/{\mathfrak{m}}$, where ${\mathfrak{m}}=x{\mathcal{O}}$ is the maximal ideal of~${\mathcal{O}}$. Now
  ${\dot R^{\mathbb{F}}_\gamma}{{\mathscr{H}}_\xi^{{\mathbb{F}}}({\mathfrak{S}_{n}}}\gamma$ by \autoref{T:MainTheorem}, so there is an
  isomorphism of ${\mathbb{F}}$-algebras $\Theta{\,{:}\,{{\mathscr{R}^{{\mathbb{F}}}_e({\mathfrak{S}_{n}}}\gamma}\!\longrightarrow\!{{{\mathscr{H}}_\xi^{{\mathbb{F}}}({\mathfrak{S}_{n}}}\gamma}}$ given by
  \[
      \psi_r\otimes1_{\mathbb{F}}\mapsto{\psi^{\mathcal{O}}}_r\otimes1_{\mathbb{F}},\qquad
      y_s\otimes1_{\mathbb{F}}\mapsto{y^{\mathcal{O}}}_s\otimes1_{\mathbb{F}}\quad\text{and}\quad
      e({\mathbf{i}})\otimes1_{\mathbb{F}}\mapsto{f^{\mathcal{O}}_{\mathbf{i}}}1_{\mathbb{F}},
  \]
  for $1\le r<n$, $1\le s\le n$ and ${\mathbf{i}}\in I^\gamma$. By
  \autoref{C:PsioHash} the following diagram commutes:
  \begin{center}
    \begin{tikzpicture}[>=stealth,->,shorten >=2pt,looseness=.5,auto]
      \matrix (M)[matrix of math nodes,row sep=1cm,column sep=16mm]{
           {\mathscr{R}^{{\mathbb{F}}}_e({\mathfrak{S}_{n}}}\gamma & {{\mathscr{H}}_\xi^{{\mathbb{F}}}({\mathfrak{S}_{n}}}\gamma \\
           {\mathscr{R}^{{\mathbb{F}}}_e({\mathfrak{S}_{n}}}\gamma & {{\mathscr{H}}_\xi^{{\mathbb{F}}}({\mathfrak{S}_{n}}}\gamma \\
       };
       \draw(M-1-1)--node[above]{$\Theta$}(M-1-2);
       \draw(M-2-1)--node[above]{$\Theta$}(M-2-2);
       \draw(M-1-1)--node[left]{${\mathtt{sgn}}$}(M-2-1);
       \draw(M-1-2)--node[right]{$\#$}(M-2-2);
    \end{tikzpicture}
  \end{center}
  Therefore, $\Theta$ restricts to an isomorphism
  $\Theta{\,{:}\,{{\mathscr{R}^{\mathbb{F}}_e({\mathfrak{A}_{n}}}\gamma}\!\longrightarrow\!{{{\mathscr{H}}_\xi^{\mathbb{F}}({\mathfrak{A}_{n}}}\gamma}}$.

  We have now shown that ${\mathscr{R}^{\mathbb{F}}_e({\mathfrak{A}_{n}}}\gamma$ and ${{\mathscr{H}}_\xi^{\mathbb{F}}({\mathfrak{A}_{n}}}\gamma$ are
  isomorphic over~${\mathbb{F}}$ but, of course, we want the isomorphism over~$F$,
  which is a subfield of~${\mathbb{F}}$. Since~$F$ is large enough for~$\xi$, by
  \autoref{D:psio} the generators of~${{\mathscr{H}}_\xi^{{\mathbb{F}}}({\mathfrak{S}_{n}}}\gamma$ listed in
  \autoref{P:NewGenerators} all belong
  to~${{\mathscr{H}}_\xi^{F}({\mathfrak{S}_{n}}}\gamma$, which we consider as a subalgebra
  of~${\mathscr{R}^{{\mathbb{F}}}_e({\mathfrak{S}_{n}}}\gamma$. The coefficients in the relations of
  \autoref{D:RO} also belong to ${{\mathscr{H}}_\xi^{F}({\mathfrak{S}_{n}}}\gamma$. Hence, there is a
  surjective algebra homomorphism
  ${\mathscr{R}^{F}_e({\mathfrak{S}_{n}}}\gamma\twoheadrightarrow{{\mathscr{H}}_\xi^{F}({\mathfrak{S}_{n}}}\gamma$. Counting
  dimensions, this map is an isomorphism
  so~${\mathscr{R}^{F}_e({\mathfrak{S}_{n}}}\gamma\cong{{\mathscr{H}}_\xi^{F}({\mathfrak{S}_{n}}}\gamma$ as $F$-algebras.  Applying
  \autoref{C:PsioHash}, as above, it follows that
  ${\mathscr{R}^{F}_e({\mathfrak{A}_{n}}}\gamma\cong{{\mathscr{H}}_\xi^{F}({\mathfrak{A}_{n}}}\gamma$ as required.
\end{proof}

In view of \autoref{C:HAnDecomp}, we obtain \autoref{T:Main} from the
introduction.

\begin{Corollary}\label{C:AlternatingHecke}
  Let $F$ be a field of characteristic different from~$2$ and let
  $\xi\in F$ be an element of quantum characteristic~$e$.  Suppose that
  $e>2$ and that $F$ is a large enough field for~$\xi$. Then
  ${{\mathscr{H}}_\xi^{F}({\mathfrak{A}_{n}}}{\mathscr{R}^{F}_e({\mathfrak{A}_{n}}}.
\end{Corollary}

Hence, as noted in \autoref{C:HAnZGraded}, the alternating Hecke algebra
${{\mathscr{H}}_\xi^{F}({\mathfrak{A}_{n}}} is a ${\mathbb{Z}}$-graded algebra. In particular, $F{\mathfrak{A}_{n}} is a
${\mathbb{Z}}$-graded algebra when $F$ is large enough for~$\xi=1$.

\section{A homogeneous basis for ${{\mathscr{H}}_\xi^{F}({\mathfrak{A}_{n}}}}\label{S:Basis}
  We have now proved \autoref{T:Main} and \autoref{T:MainRelations} from
  the introduction. It remains to prove \autoref{T:GDim}, which
  gives the graded dimension of~$\RAn$. To do this we give a homogeneous
  basis for~$\RAn$ by combining the two graded cellular bases of~${\mathscr{R}_e({\mathfrak{S}_{n}}}
  defined by Hu and the second-named author
  \cite{HuMathas:GradedCellular}. In order to define these bases we need some definitions.

Fix a partition $\lambda\in{\mathcal{P}_{n}}. If $A=(r,c)$ and $B=(s,d)$ are nodes
of $\lambda$ then $A$ is \textbf{strictly above} $B$, or $B$ is
\textbf{strictly below} $A$, if $r<s$. Following
Brundan, Kleshchev and Wang~\cite[\S1]{BKW:GradedSpecht}, define
integers
\begin{align*}
    d_A(\lambda)&=\#{\Big\{\vcenter{\hsize34mm\centering{addable $i$-nodes of $\lambda$\\[-2mm]
                            strictly below $A$}}\Big\}}
                 -\#{\Big\{\vcenter{\hsize36mm\centering{removable $i$-nodes of $\lambda$\\[-2mm]
                            strictly below $A$}}\Big\}},\\
    d^A(\lambda)&=\#{\Big\{\vcenter{\hsize34mm\centering{addable $i$-nodes of $\lambda$\\[-2mm]
                            strictly above $A$}}\Big\}}
                 -\#{\Big\{\vcenter{\hsize36mm\centering{removable $i$-nodes of $\lambda$\\[-2mm]
                            strictly above $A$}}\Big\}}.
\end{align*}

\begin{Definition}[\protect{  Brundan, Kleshchev and Wang~\cite[\S1]{BKW:GradedSpecht}}]
  Let ${\mathsf{t}}$ be a standard $\lambda$-tableau, for $\lambda\in{\mathcal{P}_{n}}, and let
  $A={\mathsf{t}}^{-1}(n)$. Then the \textbf{degree} $\deg{\mathsf{t}}$ and $\operatorname{codeg}{\mathsf{t}}$
  of~~${\mathsf{t}}$ are defined inductively by
  \begin{align*}
    \deg{\mathsf{t}}&=\begin{cases}
      \deg{\mathsf{t}}_{\downarrow{(n-1)}}+d_A(\lambda),&\text{if },n>0,\\
              1,&\text{if }n=0,\\
    \end{cases}\\
    \operatorname{codeg}{\mathsf{t}}&=\begin{cases}
      \operatorname{codeg}{\mathsf{t}}_{\downarrow{(n-1)}}+d^A(\lambda),&\text{if }n>0,\\
        1,&\text{if }n=0,
    \end{cases}
  \end{align*}
\end{Definition}

Fix $\lambda\in{\mathcal{P}_{n}}. If $1\le m\le n$ and ${\mathsf{t}}\in{\mathop{\rm Std}\nolimits}(\lambda)$
let $\operatorname{col}_m({\mathsf{t}})=c$ if $m$ appears in column~$c$ of~${\mathsf{t}}$ and let
$\operatorname{row}_m({\mathsf{t}})=r$ if $m$ appears in row~$r$ of~${\mathsf{t}}$.
Following~\cite{HuMathas:GradedCellular} set
\[
  y_\lambda=\prod_{\substack{1\le m\le n\\\operatorname{col}_m({\mathsf{t}}^\lambda)\equiv0{\text{ }(\text{mod } e)\,}}}y_m
  \qquad\text{and}\qquad
  y'_\lambda=\prod_{\substack{1\le m\le n\\\operatorname{row}_m({\mathsf{t}}_\lambda)\equiv0{\text{ }(\text{mod } e)\,}}}y_m.
\]
As we are considering the special case when $\Lambda=\Lambda_0$ the
definitions of~$y_\lambda$ and~$y'_\lambda$ from
\cite{HuMathas:GradedCellular} simplify and are equivalent to the formulas above.

If ${\mathsf{t}}\in{\mathop{\rm Std}\nolimits}(\lambda)$ define
permutations $d({\mathsf{t}})$ and $d'({\mathsf{t}})$ in~${\mathfrak{S}_{n}} by
\[
  {\mathsf{t}}=d({\mathsf{t}}){\mathsf{t}}^\lambda\qquad\text{and}\qquad{\mathsf{t}}=d'({\mathsf{t}}){\mathsf{t}}_\lambda,
\]
where ${\mathfrak{S}_{n}} acts on ${\mathsf{t}}$ by permuting its entries. For each $w\in{\mathfrak{S}_{n}}
fix a reduced expression $w=s_{r_1}\dots s_{r_k}$ and define
$\psi_w=\psi_{r_1}\dots\psi_{r_k}\in{\mathscr{R}_e({\mathfrak{S}_{n}}}. In general, $\psi_w$ depends
upon the choice of reduced expression for~$w$.

\begin{Definition}[\protect{   Hu and Mathas~\cite[Definitions~5.1 and~6.9]{HuMathas:GradedCellular}}]
Suppose that ${\mathsf{s}},{\mathsf{t}}\in{\mathop{\rm Std}\nolimits}(\lambda)$, for $\lambda\in{\mathcal{P}_{n}}. Set
${\mathbf{i}}^\lambda=\operatorname{res}({\mathsf{t}}^\lambda)$ and ${\mathbf{i}}_\lambda=\operatorname{res}({\mathsf{t}}_\lambda)$
and define
\[
\psi_{{\mathsf{s}}{\mathsf{t}}}=\psi_{d({\mathsf{s}})}y_\lambda e({\mathbf{i}}^\lambda) \psi_{d({\mathsf{t}})}^*
\quad\text{and}\quad
\psi_{{\mathsf{s}}{\mathsf{t}}}'= \psi_{d'({\mathsf{s}})}y_\lambda'e({\mathbf{i}}_\lambda)\psi_{d'({\mathsf{t}})}^*.
\]
\end{Definition}

By construction, $\psi_{{\mathsf{s}}{\mathsf{t}}}$ and $\psi'_{{\mathsf{s}}{\mathsf{t}}}$ are homogeneous
elements of~${\mathscr{R}_e({\mathfrak{S}_{n}}}.

\begin{Remark}
For the reasons explained in \cite[Remark~3.12]{HuMathas:QuiverSchurI},
we are following the conventions of
\cite{HuMathas:QuiverSchurI,KMR:UniversalSpecht} here rather than those
of~\cite{HuMathas:GradedCellular}. In particular, the
element~$\psi'_{{\mathsf{s}}{\mathsf{t}}}$ defined above is equal to $\psi'_{{\mathsf{s}}'{\mathsf{t}}'}$ in the
notation of \cite{HuMathas:GradedCellular}.
\end{Remark}

\begin{Theorem}[Hu and Mathas~\cite{HuMathas:GradedCellular}
                and Li~\cite{GeLi:IntegralKLR}]\label{T:SnCellularBasisThm}
Let ${\mathcal{Z}}$ be a commutative ring and suppose that $e>2$ and $n\geq 0$.
Then ${\mathscr{R}_e({\mathfrak{S}_{n}}} is free as a ${\mathcal{Z}}$-module with homogeneous bases
$\set{\psi_{{\mathsf{s}}{\mathsf{t}}}| {\mathsf{s}},{\mathsf{t}}\in{\mathop{\rm Std}\nolimits}^2({\mathcal{P}_{n}}}$ and
$\set{\psi'_{{\mathsf{s}}{\mathsf{t}}}| {\mathsf{s}},{\mathsf{t}}\in{\mathop{\rm Std}\nolimits}^2({\mathcal{P}_{n}}}$.
Moreover, if~$({\mathsf{s}},{\mathsf{t}})\in{\mathop{\rm Std}\nolimits}^2({\mathcal{P}_{n}}$ then
$\deg\psi_{{\mathsf{s}}{\mathsf{t}}}=\deg{\mathsf{s}}+\deg{\mathsf{t}}$ and $\deg\psi'_{{\mathsf{s}}{\mathsf{t}}}=\operatorname{codeg}{\mathsf{s}}+\operatorname{codeg}{\mathsf{t}}$.
\end{Theorem}

This result was first proved in~\cite{HuMathas:GradedCellular} with some
restrictions on the ring~${\mathcal{Z}}$. Li's~\cite{GeLi:IntegralKLR} extension
of this result to arbitrary rings is a difficult theorem. A second proof
of this result, using geometry, is given in \cite{StroppelWebster:QuiverSchur}.

Although we will not need this, by \cite[Theorems~5.8
and~6.11]{HuMathas:GradedCellular} both of the bases of
\autoref{T:SnCellularBasisThm} are graded \textit{cellular} bases of
${\mathscr{R}_e({\mathfrak{S}_{n}}} in the sense of Graham and
Lehrer~\cite{GL,HuMathas:GradedCellular}.

\begin{Example}\label{psibasiscalc}
Set $n=e=3$ and let
${\mathsf{s}}={
\begin{tikzpicture}[scale=0.3,draw/.append style={thick,black},baseline=-1mm]
  \tableauRow=0
  \foreach \Row in {{{1,2,3}}} {
  \tableauCol=1
  \foreach\k in \Row {
  \draw(\the\tableauCol,\the\tableauRow)+(-.5,-.5)rectangle++(.5,.5);
  \draw(\the\tableauCol,\the\tableauRow)node{\k};
  \global\advance\tableauCol by 1
  }
  \global\advance\tableauRow by -1
  }
\end{tikzpicture}
}$, ${\mathsf{t}}={
\begin{tikzpicture}[scale=0.3,draw/.append style={thick,black},baseline=-2mm]
  \tableauRow=0
  \foreach \Row in {{{1,2},{3}}} {
  \tableauCol=1
  \foreach\k in \Row {
  \draw(\the\tableauCol,\the\tableauRow)+(-.5,-.5)rectangle++(.5,.5);
  \draw(\the\tableauCol,\the\tableauRow)node{\k};
  \global\advance\tableauCol by 1
  }
  \global\advance\tableauRow by -1
  }
\end{tikzpicture}
}$, ${\mathsf{u}}={
\begin{tikzpicture}[scale=0.3,draw/.append style={thick,black},baseline=-2mm]
  \tableauRow=0
  \foreach \Row in {{{1,3},{2}}} {
  \tableauCol=1
  \foreach\k in \Row {
  \draw(\the\tableauCol,\the\tableauRow)+(-.5,-.5)rectangle++(.5,.5);
  \draw(\the\tableauCol,\the\tableauRow)node{\k};
  \global\advance\tableauCol by 1
  }
  \global\advance\tableauRow by -1
  }
\end{tikzpicture}
}$
and ${\mathsf{v}}={
\begin{tikzpicture}[scale=0.3,draw/.append style={thick,black},baseline=-4mm]
  \tableauRow=0
  \foreach \Row in {{{1},{2},{3}}} {
  \tableauCol=1
  \foreach\k in \Row {
  \draw(\the\tableauCol,\the\tableauRow)+(-.5,-.5)rectangle++(.5,.5);
  \draw(\the\tableauCol,\the\tableauRow)node{\k};
  \global\advance\tableauCol by 1
  }
  \global\advance\tableauRow by -1
  }
\end{tikzpicture}
}$. Using the definitions,
\begin{xalignat*}{2}
  \psi_{{\mathsf{s}}{\mathsf{s}}}&=y_3e(012),         & \psi_{{\mathsf{v}}{\mathsf{v}}}'&=y_3e(021),\\
  \psi_{{\mathsf{t}}{\mathsf{t}}}&=e(012),            & \psi_{{\mathsf{u}}{\mathsf{u}}}'&=e(021),\\
  \psi_{{\mathsf{t}}{\mathsf{u}}}&=e(012)\psi_2,      & \psi_{{\mathsf{u}}{\mathsf{t}}}'&=e(021)\psi_2,\\
  \psi_{{\mathsf{u}}{\mathsf{t}}}&=\psi_2e(012),      & \psi_{{\mathsf{t}}{\mathsf{u}}}'&=\psi_2e(021),\\
  \psi_{{\mathsf{u}}{\mathsf{u}}}&=\psi_2e(012)\psi_2,& \psi_{{\mathsf{t}}{\mathsf{t}}}'&=\psi_2e(021)\psi_2,\\
  \psi_{{\mathsf{v}}{\mathsf{v}}}&=e(021),            & \psi_{{\mathsf{s}}{\mathsf{s}}}'&=e(012).
\end{xalignat*}
Notice that $\psi_{{\mathsf{t}}{\mathsf{u}}}=\psi_2e(021)=\psi'_{{\mathsf{t}}{\mathsf{u}}}$,
$\psi_{{\mathsf{u}}{\mathsf{u}}}=\psi_2^2e(021)=-y_3e(021)=-\psi'_{{\mathsf{v}}{\mathsf{v}}}$ and, similarly,
$\psi'_{{\mathsf{t}}{\mathsf{t}}}=-y_3e(012)=-\psi_{{\mathsf{s}}{\mathsf{s}}}$. Therefore, up to sign, the
$\psi$ and $\psi'$ bases coincide with the basis of ${\mathscr{R}_e({\mathfrak{S}_{3}}} given
in \autoref{Ex:OS3}.
\end{Example}

We now use the two homogeneous bases for ${\mathscr{R}_e({\mathfrak{S}_{n}}} from
\autoref{T:SnCellularBasisThm} to construct homogeneous bases for
$\RAn$.  The next result, which follows easily from the definitions,
shows that the $\psi$ and $\psi'$-bases are interchanged by the sign
automorphism.

\begin{Lemma}[\protect{  Hu and Mathas~\cite[Proposition 3.26]{HuMathas:QuiverSchurI}}]
  \label{L:psisgn}
  Suppose that ${\mathsf{s}},{\mathsf{t}}\in{\mathop{\rm Std}\nolimits}({\mathcal{P}_{n}}$. Then
  $\psi_{{\mathsf{s}}{\mathsf{t}}}^{\mathtt{sgn}} = (-1)^{\ell(d({\mathsf{s}}))+\ell(d({\mathsf{t}}))+\deg {\mathsf{t}}^\lambda}
         \psi_{{\mathsf{s}}'{\mathsf{t}}'}'$.
  In particular, $\deg\psi_{{\mathsf{s}}{\mathsf{t}}}=\deg\psi_{{\mathsf{s}}'{\mathsf{t}}'}'$.
\end{Lemma}

In fact, it follows easily from the definitions that $\deg{\mathsf{t}}=\operatorname{codeg}{\mathsf{t}}'$
for any standard tableau~${\mathsf{t}}$. Hence,
$\deg\psi_{{\mathsf{s}}'{\mathsf{t}}'}'=\operatorname{codeg}{\mathsf{s}}'+\operatorname{codeg}{\mathsf{t}}'=\deg{\mathsf{s}}+\deg{\mathsf{t}}=\deg\psi_{{\mathsf{s}}{\mathsf{t}}}$
as claimed.

Recall from \autoref{P:RpSnIsomorphism} that
${\mathscr{R}_e({\mathfrak{S}_{n}}}{\mathscr{R}^\varepsilon_{n}}{\mathscr{R}^{\varepsilon+}_\gamma}{\mathscr{R}^{\varepsilon-}_\gamma} is a $({\mathbb{Z}}_2\times{\mathbb{Z}})$-graded
algebra and that $\RAn\cong{\mathscr{R}^{\varepsilon+}_\gamma} by \autoref{C:EvenBit}. To find a
basis for~$\RAn$ we first give a basis of~${\mathscr{R}_e({\mathfrak{S}_{n}}} that is homogeneous
with respect to the $({\mathbb{Z}}_2\times{\mathbb{Z}})$-grading.

\begin{Definition}
Fix $e>2$ and $\lambda\in{\mathcal{P}_{n}}. For ${\mathsf{s}},{\mathsf{t}}\in {\mathop{\rm Std}\nolimits}(\lambda)$ define
elements
\[
    \Psi^+_{{\mathsf{s}}{\mathsf{t}}}=\psi_{{\mathsf{s}}{\mathsf{t}}}+\psi_{{\mathsf{s}}{\mathsf{t}}}^{\mathtt{sgn}} \quad\text{and}\quad
    \Psi^-_{{\mathsf{s}}{\mathsf{t}}}=\psi_{{\mathsf{s}}{\mathsf{t}}}-\psi_{{\mathsf{s}}{\mathsf{t}}}^{\mathtt{sgn}}.
\]
\end{Definition}

Fix $({\mathsf{s}},{\mathsf{t}})\in{\mathop{\rm Std}\nolimits}^2({\mathcal{P}_{n}}$. By \autoref{L:psisgn},
$\Psi^+_{{\mathsf{s}}{\mathsf{t}}}$ and $\Psi^=_{{\mathsf{s}}{\mathsf{t}}}$ are homogeneous with respect to the ${\mathbb{Z}}$-grading
on~${\mathscr{R}_e({\mathfrak{S}_{n}}}. Furthermore, by \autoref{C:EvenBit},~$\Psi^+_{{\mathsf{s}}{\mathsf{t}}}$ is even
and~$\Psi^-_{{\mathsf{s}}{\mathsf{t}}}$ is odd with respect to the ${\mathbb{Z}}_2$-grading. Hence,
the elements $\Psi^\pm_{{\mathsf{s}}{\mathsf{t}}}$ are homogeneous with respect to the
$({\mathbb{Z}}_2\times{\mathbb{Z}})$-grading on~${\mathscr{R}_e({\mathfrak{S}_{n}}}.

Fix $({\mathsf{s}},{\mathsf{t}})\in{\mathop{\rm Std}\nolimits}^2({\mathcal{P}_{n}}$ and set ${\mathbf{i}}^{\mathsf{s}}=\operatorname{res}({\mathsf{s}})$ and
${\mathbf{i}}^{\mathsf{t}}=\operatorname{res}({\mathsf{t}})$.  By \cite[(3.13)]{HuMathas:QuiverSchurI}, if
${\mathbf{i}},{\mathbf{j}}\in I^n$
\begin{equation}\label{E:psiIdempotents}
  e({\mathbf{i}})\psi_{{\mathsf{s}}{\mathsf{t}}}e({\mathbf{j}})=\delta_{{\mathbf{i}}^{\mathsf{s}}{\mathbf{i}}}\delta_{{\mathbf{i}}^{\mathsf{t}}{\mathbf{j}}}\psi_{{\mathsf{s}}{\mathsf{t}}}
  \quad\text{and}\quad
  e({\mathbf{i}})\psi'_{{\mathsf{s}}{\mathsf{t}}}e({\mathbf{j}})=\delta_{{\mathbf{i}}^{\mathsf{s}}{\mathbf{i}}}\delta_{{\mathbf{i}}^{\mathsf{t}}{\mathbf{j}}}\psi'_{{\mathsf{s}}{\mathsf{t}}}
\end{equation}
Set $\eps_1({\mathbf{i}})=e({\mathbf{i}})-e(-{\mathbf{i}})$ and
$\eps_0({\mathbf{i}})=\eps_1({\mathbf{i}})^2=e({\mathbf{i}})+e(-{\mathbf{i}})$.
Observe that \autoref{E:psiIdempotents} implies that if
${\mathbf{i}}\in I^\gamma_+$ and $({\mathsf{s}},{\mathsf{t}})\in{\mathop{\rm Std}\nolimits}({\mathcal{P}_{n}}$ with
${\mathbf{i}}^{\mathsf{s}}=\operatorname{res}({\mathsf{s}})\in I^\gamma$ and ${\mathbf{i}}^{\mathsf{t}}=\operatorname{res}({\mathsf{t}})\in I^\gamma$ then
\begin{equation}\label{E:epsPsi}
\eps_1({\mathbf{i}})\Psi^+_{{\mathsf{s}}{\mathsf{t}}}
        \begin{cases}
          \phantom{-}\Psi^-_{{\mathsf{s}}{\mathsf{t}}}&\text{if }{\mathbf{i}}={\mathbf{i}}^{\mathsf{s}},\\
          -\Psi^-_{{\mathsf{s}}{\mathsf{t}}}&\text{if }{\mathbf{i}}=-{\mathbf{i}}^{\mathsf{s}},\\
          \phantom{-}0,&\text{otherwise},
        \end{cases}
\quad\text{and}\quad
\Psi^+_{{\mathsf{s}}{\mathsf{t}}}\eps_1({\mathbf{i}})=
        \begin{cases}
          \phantom{-}\Psi^-_{{\mathsf{s}}{\mathsf{t}}}&\text{if }{\mathbf{i}}={\mathbf{i}}^{\mathsf{t}},\\
          -\Psi^-_{{\mathsf{s}}{\mathsf{t}}}&\text{if }{\mathbf{i}}=-{\mathbf{i}}^{\mathsf{t}},\\
          \phantom{-}0,&\text{otherwise}.
        \end{cases}
\end{equation}
Since $\eps_1(-{\mathbf{i}})=-\eps_1({\mathbf{i}})$ and $\eps_0({\mathbf{i}})=\eps_1({\mathbf{i}})^2$,
this readily implies the corresponding formulas for the action of
$\eps_a({\mathbf{i}})$ on $\Psi^\pm_{{\mathsf{s}}{\mathsf{t}}}$, for all $a\in{\mathbb{Z}}_2$ and ${\mathbf{i}}\in
I^\gamma$.

The \textbf{dominance} order on~${\mathcal{P}_{n}} is the partial order~$\gedom$
given by $\lambda\gedom\mu$ if
\[
    \sum_{j=1}^k\lambda_j\ge\sum_{{\hat\jmath}=1}^k\mu_j\qquad\text{ for all }k\ge0
\]
Write $\lambda\gdom\mu$ if $\lambda\gedom\mu$ and $\lambda\ne\mu$.

For $\lambda\in{\mathcal{P}_{n}} define
${\mathop{\rm Std}\nolimits}_+(\lambda)=\set{{\mathsf{s}}\in{\mathop{\rm Std}\nolimits}(\lambda)|\operatorname{res}({\mathsf{s}})\in I^n_+}$.  We will
use this set to index a homogeneous basis for~${\mathscr{R}_e({\mathfrak{S}_{n}}}, with respect to its
$({\mathbb{Z}}_2\times{\mathbb{Z}})$-grading.
The following simple combinatorial result is probably well-known.

\begin{Lemma}\label{L:CountingLemma}
  Suppose that $n\ge0$. Then
  \[
     \sum_{\substack{\lambda\in{\mathcal{P}_{n}}\lambda\gedom\lambda'}}
         |{\mathop{\rm Std}\nolimits}_+(\lambda)|\cdot|{\mathop{\rm Std}\nolimits}(\lambda)|
         =\frac{n!}{2}.
  \]
\end{Lemma}

\begin{proof} Implicit in \autoref{T:SnCellularBasisThm}, is the
  well-known
  fact that $n!=\sum_{\lambda\in{\mathcal{P}_{n}}|{\mathop{\rm Std}\nolimits}(\lambda)|^2$.
  Since $|{\mathop{\rm Std}\nolimits}(\lambda)|=|{\mathop{\rm Std}\nolimits}(\lambda')|$, via the map ${\mathsf{t}}\mapsto{\mathsf{t}}'$,
  it follows that
  \begin{align*}
    \frac{n!}2
      &=\sum_{\substack{\lambda\in{\mathcal{P}_{n}}\lambda\gdom\lambda'}}|{\mathop{\rm Std}\nolimits}(\lambda)|^2
         +\frac12\sum_{\substack{\lambda\in{\mathcal{P}_{n}}\lambda=\lambda'}}
                |{\mathop{\rm Std}\nolimits}(\lambda)|^2\\
      &=\sum_{\substack{\lambda\in{\mathcal{P}_{n}}\lambda\gdom\lambda'}}
          \Bigl(|{\mathop{\rm Std}\nolimits}_+(\lambda)|+|{\mathop{\rm Std}\nolimits}_+(\lambda')|\Bigr)\cdot|{\mathop{\rm Std}\nolimits}(\lambda)|
         +\sum_{\substack{\lambda\in{\mathcal{P}_{n}}\lambda=\lambda'}}
               |{\mathop{\rm Std}\nolimits}_+(\lambda)|\cdot |{\mathop{\rm Std}\nolimits}(\lambda)|.
  \end{align*}
\end{proof}

Recall from \autoref{S:GeneratorsRelations} that
$\operatorname{Deg}{\,{:}\,{\mathscr{R}_e({\mathfrak{S}_{n}}}\longrightarrow\!{\mathbb{Z}}}_2\times{\mathbb{Z}}$ is the degree function for the
$({\mathbb{Z}}_2\times{\mathbb{Z}})$-grading on~${\mathscr{R}_e({\mathfrak{S}_{n}}}.

\begin{Theorem}\label{T:OddEvenBasis}
  Let ${\mathcal{Z}}$ be a commutative ring such that $2$ is invertible
  in~${\mathcal{Z}}$ and suppose that $e>2$ and $n\ge0$. Then ${\mathscr{R}_e({\mathfrak{S}_{n}}} is free as
  a ${\mathcal{Z}}$-module with basis
  \[
  \set{\Psi^+_{{\mathsf{s}}{\mathsf{t}}}, \Psi^-_{{\mathsf{s}}{\mathsf{t}}}|{\mathsf{s}}\in{\mathop{\rm Std}\nolimits}_+(\lambda), {\mathsf{t}}\in{\mathop{\rm Std}\nolimits}(\lambda)
           \text{ for } \lambda\in{\mathcal{P}_{n}}.
  \]
  Moreover, this basis is homogeneous with respect to the
  $({\mathbb{Z}}_2\times{\mathbb{Z}})$-grading on~${\mathscr{R}_e({\mathfrak{S}_{n}}}.
\end{Theorem}

\begin{proof}
  We have already noted $\Psi^\pm_{s{\mathsf{t}}}$ is homogeneous with respect to
  the $({\mathbb{Z}}_2\times{\mathbb{Z}})$-grading. More precisely, if
  $({\mathsf{s}},{\mathsf{t}})\in{\mathop{\rm Std}\nolimits}^2({\mathcal{P}_{n}}$ then
  \[\operatorname{Deg}\Psi^+_{{\mathsf{s}}{\mathsf{t}}}=(0,\deg{\mathsf{s}}+\deg{\mathsf{t}})\quad\text{and}\quad
    \operatorname{Deg}\Psi^-_{{\mathsf{s}}{\mathsf{t}}}=(1,\deg{\mathsf{s}}+\deg{\mathsf{t}}).
  \]

  Let $R_n$ be the ${\mathcal{Z}}$-submodule of ${\mathscr{R}_e({\mathfrak{S}_{n}}} spanned by the
  elements in the statement of the theorem.
  Fix ${\mathsf{s}}\in{\mathop{\rm Std}\nolimits}_+(\lambda)$ and ${\mathsf{t}}\in{\mathop{\rm Std}\nolimits}(\lambda)$, for some
  $\lambda\in{\mathcal{P}_{n}}. Since $2$ is invertible in~${\mathcal{Z}}$,
  \[
      \psi_{{\mathsf{s}}{\mathsf{t}}}=\frac12\big(\Psi^+_{{\mathsf{s}}{\mathsf{t}}}+\Psi^-_{{\mathsf{s}}{\mathsf{t}}}\big)
      \quad\text{and}\quad
      \psi^{\mathtt{sgn}}_{{\mathsf{s}}{\mathsf{t}}}=\frac12\big(\Psi^+_{{\mathsf{s}}{\mathsf{t}}}-\Psi^-_{{\mathsf{s}}{\mathsf{t}}}\big).
  \]
  Hence, $\psi_{{\mathsf{s}}{\mathsf{t}}}, \psi'_{{\mathsf{s}}'{\mathsf{t}}'}\in R_n$ since
  $\psi'_{{\mathsf{s}}'{\mathsf{t}}'}=\pm\psi_{{\mathsf{s}}{\mathsf{t}}}^{\mathtt{sgn}}$ by \autoref{L:psisgn}.  Recall
  from \autoref{L:ZeroSequence} that $e({\mathbf{i}})\ne0$ only if
  ${\mathbf{i}}\in I^n_+$ or ${\mathbf{i}}\in I^n_-$. Set
  $e_+=\sum_{{\mathbf{i}}\in I^n_+}e({\mathbf{i}})$ and $e_-=\sum_{{\mathbf{i}}\in I^n_-}e({\mathbf{i}})$.
  By \autoref{T:SnCellularBasisThm} and \autoref{E:psiIdempotents},
  as ${\mathcal{Z}}$-modules,
  \begin{align*}
    e_+{\mathscr{R}_e({\mathfrak{S}_{n}}}=\<\psi_{{\mathsf{s}}{\mathsf{t}}}\mid ({\mathsf{s}},{\mathsf{t}})\in{\mathop{\rm Std}\nolimits}^2({\mathcal{P}_{n}}\text{ and }
            \operatorname{res}({\mathsf{s}})\in I^n_+ \>_{\mathcal{Z}}\subseteq e_+ R_n\\
  \intertext{
  Similarly, since $\operatorname{res}({\mathsf{s}}')=-\operatorname{res}({\mathsf{s}})$, \autoref{T:SnCellularBasisThm}
  also implies that}
      e_-{\mathscr{R}_e({\mathfrak{S}_{n}}}=\<\psi'_{{\mathsf{s}}{\mathsf{t}}}\mid ({\mathsf{s}},{\mathsf{t}})\in{\mathop{\rm Std}\nolimits}^2({\mathcal{P}_{n}}\text{ and }
            \operatorname{res}({\mathsf{s}})\in I^n_- \>_{\mathcal{Z}}\subseteq e_- R_n.
  \end{align*}
  Hence, $e_+R_n\subseteq e_+{\mathscr{R}_e({\mathfrak{S}_{n}}} e_+R_n$ and
  $e_-R_n\subseteq e_-{\mathscr{R}_e({\mathfrak{S}_{n}}} e_-R_n$, so that
  \[  {\mathscr{R}_e({\mathfrak{S}_{n}}}e_+{\mathscr{R}_e({\mathfrak{S}_{n}}} e_-{\mathscr{R}_e({\mathfrak{S}_{n}}}e_+R_n\oplus e_-R_n=R_n. \]
  We have now shown that the set of elements $\set{\Psi^\pm_{{\mathsf{s}}{\mathsf{t}}}}$ in
  the statement of the theorem span~${\mathscr{R}_e({\mathfrak{S}_{n}}}. Let $F$ be the field of
  fractions of~${\mathcal{Z}}$. Using \autoref{L:CountingLemma} to count
  dimensions, it follows that $\set{\Psi^\pm_{{\mathsf{s}}{\mathsf{t}}}\otimes1_F}$ is a
  basis of~${\mathscr{R}^{F}_e({\mathfrak{S}_{n}}}{\mathscr{R}_e({\mathfrak{S}_{n}}}_{\mathcal{Z}} F$. Hence,
  $\set{\Psi^\pm_{{\mathsf{s}}{\mathsf{t}}}}$ is ${\mathcal{Z}}$-linearly independent, completing the proof.
\end{proof}

By \autoref{C:EvenBit}, $\RAn$ is the even component of~${\mathscr{R}_e({\mathfrak{S}_{n}}}, with
respect to the ${\mathbb{Z}}_2$-grading. Hence, we have the following.

\begin{Corollary}\label{C:EvenBasis}
  Let ${\mathcal{Z}}$ be a commutative ring such that $2$ is invertible
  in~${\mathcal{Z}}$ and suppose that $e>2$ and $n\ge0$.
  Then $\RAn$ is free as a ${\mathcal{Z}}$-module with basis
  \[
  \set{\Psi^+_{{\mathsf{s}}{\mathsf{t}}}|{\mathsf{s}}\in{\mathop{\rm Std}\nolimits}_+(\lambda) \text{ and } {\mathsf{t}}\in{\mathop{\rm Std}\nolimits}(\lambda)
           \text{ for } \lambda\in{\mathcal{P}_{n}}.
  \]
  Moreover, this basis is homogeneous with respect to the
  ${\mathbb{Z}}$-grading on~$\RAn$.
\end{Corollary}

\begin{Example}\label{psibasisalt}
Continuing the notation of \autoref{psibasiscalc},
\[
\Psi^+_{{\mathsf{s}}{\mathsf{s}}}= \mathcal{Y}_3= -\Psi^+_{{\mathsf{u}}{\mathsf{u}}}\quad
\Psi^+_{{\mathsf{t}}{\mathsf{t}}}= 1= \Psi^+_{{\mathsf{v}}{\mathsf{v}}}\quad\text{and}\quad
\Psi^+_{{\mathsf{t}}{\mathsf{u}}}= \Psi^+_2= -\Psi^+_{{\mathsf{t}}{\mathsf{u}}}.
\]
Hence,
$\set{\Psi^+_{{\mathsf{a}}{\mathsf{b}}}| {\mathsf{a}}\in{\mathop{\rm Std}\nolimits}_+(\lambda)\text{ and }{\mathsf{b}}\in{\mathop{\rm Std}\nolimits}(\lambda)}
      =\set{\Psi^+_{{\mathsf{s}}{\mathsf{s}}}, \Psi^+_{{\mathsf{t}}{\mathsf{t}}}, \Psi^+_{{\mathsf{t}}{\mathsf{u}}}}$
is the basis of~$\RAn$ constructed in \autoref{Ex:A3Basis}.
\end{Example}

If $M=\bigoplus_{d\in{\mathbb{Z}}}M_d$ is a ${\mathbb{Z}}$-graded module then its
\textbf{graded dimension} is the Laurent polynomial
\[
   \operatorname{\mathrm{dim}_q} M=\sum_{d\in{\mathbb{Z}}} (\dim M_d)\,q^d\in{\mathcal{A}}={\mathbb{N}}[q,q^{-1}],
\]
where $q$ is an indeterminate over~${\mathbb{Z}}$.  Adding up the degrees of the
homogeneous basis elements in \autoref{C:EvenBit} gives \autoref{T:GDim}
from the introduction.

\begin{Corollary}\label{gdim}
  Let $F$ be a field of characteristic different from~$2$ and suppose that
  $e>2$ and $n\ge0$. Then the graded dimension of~$\RAn$ is
\[
\operatorname{\mathrm{dim}_q} \RAn=\sum_{\lambda\in{\mathcal{P}_{n}}
  \sum_{\substack{{\mathsf{s}}\in{\mathop{\rm Std}\nolimits}_+(\lambda)\\{\mathsf{t}}\in{\mathop{\rm Std}\nolimits}(\lambda)}} q^{\deg{\mathsf{s}}+\deg{\mathsf{t}}}.
\]
\end{Corollary}

By \autoref{E:epsPsi}, if $\gamma\in{Q^\varepsilon_n}$ then the basis of $\RAn$ given in
\autoref{C:EvenBasis} restricts to give a basis of~$\RAn_\gamma$. Note
that if $({\mathsf{s}},{\mathsf{t}})\in{\mathop{\rm Std}\nolimits}^2({\mathcal{P}_{n}}$ and $\operatorname{res}({\mathsf{s}})\in I^\gamma$ then
$\operatorname{res}({\mathsf{t}})\in I^\gamma$.

\begin{Corollary}\label{C:BlockBasis}
  Fix $\gamma\in{Q^\varepsilon_n}$ and let ${\mathcal{Z}}$ be a commutative ring such that
  $2$ is invertible in~${\mathcal{Z}}$ and suppose that $e>2$ and $n\ge0$.
  Then $\RAn_\gamma$ is free as a ${\mathcal{Z}}$-module with basis
  $\set{\Psi^+_{{\mathsf{s}}{\mathsf{t}}}|({\mathsf{s}},{\mathsf{t}})\in{\mathop{\rm Std}\nolimits}^2({\mathcal{P}_{n}}\text{ for }
             \operatorname{res}({\mathsf{s}})\in I^\gamma_+}$.
\end{Corollary}

We next show that $\RAn_\gamma$ is a graded symmetric algebra. Repeating
the arguments leading to \autoref{C:BlockBasis} we obtain a second
homogeneous basis for~$\RAn_\gamma$. We will use the two homogeneous
bases of~$\RAn_\gamma$ to prove that $\RAn_\gamma$ is graded symmetric.

\begin{Corollary}\label{C:EvenBasisII}
  Fix $\gamma\in{Q^\varepsilon_n}$ and let ${\mathcal{Z}}$ be a commutative ring such that
  $2$ is invertible in~${\mathcal{Z}}$ and suppose that $e>2$ and $n\ge0$.
  Then $\RAn_\gamma$ is free as a ${\mathcal{Z}}$-module with basis
  $\set{\Psi^+_{{\mathsf{s}}{\mathsf{t}}}|({\mathsf{s}},{\mathsf{t}})\in{\mathop{\rm Std}\nolimits}^2({\mathcal{P}_{n}}\text{ for }
             \operatorname{res}({\mathsf{s}})\in I^\gamma_-}$.
\end{Corollary}

Before we show that the blocks of $\RAn$ are graded symmetric algebras
we recall some definitions. A \textbf{trace form} on an algebra $A$ is a
linear map $\tau{\,{:}\,A\!\longrightarrow\!F}$ such that $\tau(ab)=\tau(ba)$, for all $a,b\in
A$. The algebra~$A$ is \textbf{symmetric} if $A$ is equipped with a
non-degenerate symmetric bilinear form $\theta: A\times A\rightarrow F$
which is associative in the following sense:
\[ \theta(xy,z)=\theta(x,yz),\quad\text{for all }x,y,z\in A. \]
A graded algebra $A$ is a \textbf{graded symmetric} algebra if there
exists a non-degenerate homogeneous trace form $\tau{\,{:}\,A\!\longrightarrow\!F}$. Suppose
that $A$ is equipped with a homogeneous anti-isomorphism $\sigma$ of
order~$2$. In view of \cite[Lemma~6.13]{HuMathas:GradedCellular}, giving
a non-degenerate trace form~$\tau$ is equivalent to requiring that the bilinear
form
\[\<a,b\> = \tau(ab^\sigma)\qquad\text{ for all }a,b\in A\]
is non-degenerate. We work interchangeably with the trace form $\tau$
and its associated bilinear form.

The algebras ${\mathscr{R}_e({\mathfrak{S}_{n}}} and $\RAn$ are both symmetric algebras but their
homogeneous trace forms are defined on the blocks $\RAn_\gamma$ of these
algebras, for~$\gamma\in{Q^\varepsilon_n}$.  Recall that ${Q^\varepsilon_n}=Q^+_n/{\sim}$, where
$\alpha\sim\alpha'$ if $(\Lambda_i,\alpha)=(\Lambda_{-i},\alpha')$, for
all $i\in I$. Fix $\alpha\in Q^+_n$.  The \textbf{defect} of~$\alpha$ is the
non-negative integer
\[\operatorname{\mathrm{def}}\alpha=(\Lambda_0,\alpha)-\frac12(\alpha,\alpha),\]
where $(\ ,\ ){\,{:}\,{P^+\times Q^+}\!\longrightarrow\!{\mathbb{Z}}}$ is the pairing defined in
\autoref{S:KLRAlgebras}. Hence, if $\alpha\sim\alpha'$ then
$\operatorname{\mathrm{def}}\alpha=\operatorname{\mathrm{def}}\alpha'$. Therefore, if~$\gamma\in{Q^\varepsilon_n}$ we can
define the \textbf{defect} of~$\gamma$ to be
$\operatorname{\mathrm{def}}\gamma=\operatorname{\mathrm{def}}\alpha$, for any $\alpha\in\gamma$.  Set
\[{\mathcal{P}_{\alpha}}\set{\lambda\in{\mathcal{P}_{n}}{\mathbf{i}}^\lambda\in I^\alpha}
   \quad\text{and}\quad
  {\mathcal{P}_{\gamma}}\bigcup_{\alpha\in\gamma}{\mathcal{P}_{\alpha}}
\]
for $\alpha\in Q^+_n$ and $\gamma\in{Q^\varepsilon_n}$. In the usual language from
the representation theory of the symmetric groups, the partitions
in~${\mathcal{P}_{\gamma}} have \textbf{$e$-weight} $\operatorname{\mathrm{def}}\gamma$.

The following useful fact is straightforward to establish from the
definitions.

\begin{Lemma}[\protect{  Brundan and Kleshchev~\cite[Lemma 3.11, 3.12]{BKW:GradedSpecht}}]
  \label{L:degcodeg}
  Let $\alpha\in Q^+_n$ and suppose that ${\mathsf{s}}\in{\mathop{\rm Std}\nolimits}({\mathcal{P}_{\alpha}}$.
  Then $\deg{\mathsf{s}}+\operatorname{codeg}{\mathsf{s}}=\operatorname{\mathrm{def}}\alpha$.
\end{Lemma}

  As is well-known and easy to prove (see, for example,
  \cite[Proposition~1.16]{M:ULect}), the Iwhahori-Hecke ${{\mathscr{H}}_\xi({\mathfrak{S}_{n}}} is a
  symmetric algebra with trace form $\tau$. Explicitly, if
  $h=\sum_{w\in{\mathfrak{S}_{n}}a_w T_w\in{{\mathscr{H}}_\xi({\mathfrak{S}_{n}}} then $\tau(h)=a_1$. For $\alpha\in Q^+$
  let $\tau_\alpha$ be the homogeneous component of~$\tau$ of
  degree~$-2\operatorname{\mathrm{def}}\alpha$ restricted to ${{\mathscr{H}}_\xi({\mathfrak{S}_{n}}}\alpha$. Let $\<\ ,\ \>_\alpha$
  be the homogeneous bilinear form associated with~$\tau$.

  Let $\star$ be the unique homogeneous anti-isomorphism of~${\mathscr{R}_e({\mathfrak{S}_{n}}} that
  fixes each of the generators of~${\mathscr{R}_e({\mathfrak{S}_{n}}}. Then
  $\psi_{{\mathsf{s}}{\mathsf{t}}}^\star=\psi_{{\mathsf{t}}{\mathsf{s}}}$ and
  $(\psi'_{{\mathsf{s}}{\mathsf{t}}})^\star=\psi'_{{\mathsf{t}}{\mathsf{s}}}$, for all
  $({\mathsf{s}},{\mathsf{t}})\in{\mathop{\rm Std}\nolimits}^2({\mathcal{P}_{n}}$.

  The following result was first proved in~\cite{HuMathas:GradedCellular}.

  \begin{Theorem}[\protect{      Hu and Mathas~\cite[Theorem~6.7]{HuMathas:GradedCellular}}]
    \label{T:RSnSymmetric}
    Suppose that $\alpha\in Q^+_n$ and that~$F$ is a field. Then the KLR
    algebra~${{\mathscr{R}_e({\mathfrak{S}_{n}}}{\alpha}} is a graded symmetric algebra with homogeneous
    bilinear form $\<\ ,\ \>_\alpha$ of degree $-2\operatorname{\mathrm{def}}\alpha$.
    Moreover, if ${\mathsf{s}},{\mathsf{t}}\in{\mathop{\rm Std}\nolimits}(\lambda)$ and ${\mathsf{u}},{\mathsf{v}}\in{\mathop{\rm Std}\nolimits}(\mu)$ then
    \[   \<\psi_{{\mathsf{s}}{\mathsf{t}}},\psi'_{{\mathsf{u}}{\mathsf{v}}}\>_\alpha
             =\begin{dcases*}
               c_{{\mathsf{s}}{\mathsf{t}}},& if $({\mathsf{u}},{\mathsf{v}})=({\mathsf{s}},{\mathsf{t}})$,\\
                   0,& if $({\mathsf{u}},{\mathsf{v}})\notgedom({\mathsf{s}},{\mathsf{t}})$,
               \end{dcases*}
     \]
     where $c_{{\mathsf{s}}{\mathsf{t}}}$ is a non-zero element of~$F$ that depends only
     on~${\mathsf{s}}$ and~${\mathsf{t}}$.
  \end{Theorem}

  Recall from \autoref{E:RAngamma} that
  $\RAn_\gamma=\big(\bigoplus_{\alpha\in\gamma}{{\mathscr{R}_e({\mathfrak{S}_{n}}}{\alpha}})^{\mathtt{sgn}}$.
  We will use the bilinear forms on ${\mathscr{R}_e({\mathfrak{S}_{n}}}\alpha$, for
  $\alpha\in\gamma$, to define a bilinear form on~$\RAn_\gamma$. We do
  not take the obvious extension of these forms to~${\mathscr{R}_e({\mathfrak{S}_{n}}}\gamma$,
  however, because the arguments below require a ${\mathtt{sgn}}$-invariant form.
  If $h\in{\mathscr{R}_e({\mathfrak{S}_{n}}}\gamma$ then $h=\sum_{{\mathbf{i}}\in I^\gamma}e({\mathbf{i}})h_i$. Hence,
  define the trace form $\tau_\gamma{\,{:}\,{{\mathscr{R}_e({\mathfrak{S}_{n}}}\gamma}\!\longrightarrow\!F}$ by
  \begin{equation}\label{E:SgnInvariance}
       \tau_\gamma\big(e({\mathbf{i}})h\big)=\begin{dcases*}
          \tau_\alpha\big(e({\mathbf{i}})h\big),&
              if ${\mathbf{i}}\in I^\alpha_+\subseteq I^\gamma_+$,\\
          \tau_{\alpha'}\big(e(-{\mathbf{i}})h^{\mathtt{sgn}}\big),&
              if ${\mathbf{i}}\in I^{\alpha}_-\subseteq I^\gamma_-$.
        \end{dcases*}
  \end{equation}
  Importantly, $\tau_\gamma(h)=\tau_\gamma(h^{\mathtt{sgn}})$, for all
  $h\in{\mathscr{R}_e({\mathfrak{S}_{n}}}\gamma$.  Let $\<\ ,\ \>_\gamma$ be the corresponding
  bilinear form on~${\mathscr{R}_e({\mathfrak{S}_{n}}}\gamma$.

  We can now prove that $\RAn_\gamma$ is a graded symmetric algebra.

  \begin{Theorem}\label{T:RAnGradeSymmetric}
    Let $F$ be a field of characteristic different from~$2$ and suppose
    that $e>2$ and $\gamma\in{Q^\varepsilon_n}$. Then $\RAn_\gamma$ is a graded
    symmetric algebra with homogeneous bilinear form $\<\ ,\ \>_\gamma$
    of degree $-2\operatorname{\mathrm{def}}\gamma$.
  \end{Theorem}

  \begin{proof}
    By \autoref{T:RAnGradeSymmetric} and the definitions above,
    $\<\ ,\ \>_\gamma$ is a (not necessarily associative) homogeneous bilinear form
    on~${\mathscr{R}_e({\mathfrak{S}_{n}}}\gamma$ of degree $-2\deg\gamma$. By restriction, we can consider
    $\<\ ,\ \>_\gamma$ as a bilinear form on~$\RAn_\gamma$. By
    construction, $\<\ ,\ \>_\gamma$ is an associative bilinear form on~$\RAn$.

    We need to show that $\<\ ,\ \>_\gamma$ is non-degenerate on~$\RAn_\gamma$.
    To do this we use the two bases of $\RAn_\gamma$ given by
    \autoref{C:BlockBasis} and \autoref{C:EvenBasisII}. Fix
    $\lambda,\mu\in{\mathcal{P}_{n}} and tableaux ${\mathsf{s}},{\mathsf{t}}\in{\mathop{\rm Std}\nolimits}(\lambda)$,
    ${\mathsf{u}},{\mathsf{v}}\in{\mathop{\rm Std}\nolimits}(\mu)$ with $\operatorname{res}({\mathsf{s}})\in I^\gamma_+$
    and $\operatorname{res}({\mathsf{u}})\in I^\gamma_-$. By assumption, $\operatorname{res}({\mathsf{s}})\ne\operatorname{res}({\mathsf{u}})$ so
    $\<\psi_{{\mathsf{s}}{\mathsf{t}}},\psi_{{\mathsf{u}}{\mathsf{v}}}\>_\gamma =\tau_\gamma(\psi_{{\mathsf{s}}{\mathsf{t}}}\psi_{{\mathsf{v}}{\mathsf{u}}})
        =\tau_\gamma(\psi_{{\mathsf{v}}{\mathsf{u}}}\psi_{{\mathsf{s}}{\mathsf{t}}})=0$
    by~\autoref{E:psiIdempotents}. Hence,
    $\<\psi^{\mathtt{sgn}}_{{\mathsf{s}}{\mathsf{t}}},\psi^{\mathtt{sgn}}_{{\mathsf{u}}{\mathsf{v}}}\>_\gamma
               =\tau_\gamma(\psi_{{\mathsf{s}}{\mathsf{t}}}\psi_{{\mathsf{v}}{\mathsf{u}}}) =0$.
    Therefore,
    \begin{align*}
        \<\Psi^+_{{\mathsf{s}}{\mathsf{t}}},\Psi^+_{{\mathsf{u}}{\mathsf{v}}}\>_\gamma
          &=\<\psi_{{\mathsf{s}}{\mathsf{t}}}+\psi_{{\mathsf{s}}{\mathsf{t}}}^{\mathtt{sgn}},\psi_{{\mathsf{u}}{\mathsf{v}}}+\psi_{{\mathsf{u}}{\mathsf{v}}}^{\mathtt{sgn}}\>_\gamma
           =\<\psi_{{\mathsf{s}}{\mathsf{t}}},\psi_{{\mathsf{u}}{\mathsf{v}}}^{\mathtt{sgn}}\>_\gamma
                +\<\psi_{{\mathsf{s}}{\mathsf{t}}}^{\mathtt{sgn}},\psi_{{\mathsf{u}}{\mathsf{v}}}\>_\gamma\\
          &=2\tau_\gamma(\psi_{{\mathsf{s}}{\mathsf{t}}}\psi^{\mathtt{sgn}}_{{\mathsf{u}}{\mathsf{v}}})
           =\begin{dcases*}
            \pm 2c_{{\mathsf{s}}{\mathsf{t}}},& if $({\mathsf{u}}',{\mathsf{v}}')=({\mathsf{s}},{\mathsf{t}})$,\\
            0,& if $({\mathsf{u}}',{\mathsf{v}}')\notgedom({\mathsf{s}},{\mathsf{t}})$.
            \end{dcases*}
    \end{align*}
    The second last equality follows because
    $\tau_\gamma(h)=\tau_\gamma(h^{\mathtt{sgn}})$ for $h\in{\mathscr{R}_e({\mathfrak{S}_{n}}}\gamma$, by
    \autoref{E:SgnInvariance}, and the last equality follows from
    \autoref{T:RAnGradeSymmetric}, since
    $\psi_{{\mathsf{u}}{\mathsf{v}}}^{\mathtt{sgn}}=\pm\psi'_{{\mathsf{u}}'{\mathsf{v}}'}$ by \autoref{L:psisgn}. Hence,
    by ordering the two bases $\set{\Psi^+_{{\mathsf{s}}{\mathsf{t}}}}$ and
    $\set{\Psi^+_{{\mathsf{u}}{\mathsf{v}}}}$ in a way that is compatible with dominance and
    reverse dominance, respectively, it follows that the Gram matrix
    $\big(\<\Psi^+_{{\mathsf{s}}{\mathsf{t}}},\Psi^+_{{\mathsf{u}}{\mathsf{v}}}\>_\gamma\big)$ is triangular
    with non-zero entries on the diagonal. Therefore, $\<\ ,\ \>_\gamma$
    is a non-degenerate associative bilinear form on~${{\mathscr{H}}_\xi^{F}({\mathfrak{A}_{n}}}\gamma$ of
    degree~$-2\operatorname{\mathrm{def}}\gamma$, so the theorem is proved.
  \end{proof}

  Finally, we describe the blocks and irreducible modules
  of~${{\mathscr{H}}_\xi^{F}({\mathfrak{A}_{n}}}.

  Let $\operatorname{Res}=\operatorname{Res}^{\mathscr{R}_e({\mathfrak{S}_{n}}}{\RAn}$ be the restriction functor from
  the category of finitely generated graded ${\mathscr{R}_e({\mathfrak{S}_{n}}}-modules to the
  the category of finitely generated graded $\RAn$-modules.

  Let ${\mathcal{R}_{n}}\set{\mu\in{\mathcal{P}_{n}}\mu_r-\mu_{r+1}<e\text{ for all }r\ge1}$
  be the set of \textbf{$e$-restricted} partitions of~$n$. By
  \cite[Theorem~4.11]{BK:GradedKL}, there is a unique self-dual
  irreducible graded ${\mathscr{R}_e({\mathfrak{S}_{n}}}-module $D^\mu$ for each $e$-restricted
  partition~$\mu$ and
  \[\set{D^\mu\<d\>|\mu\in{\mathcal{R}_{n}}{ and }d\in{\mathbb{Z}}}\]
  is a complete set of pairwise non-isomorphic irreducible graded
  ${\mathscr{R}_e({\mathfrak{S}_{n}}}-modules. By \cite[Corollary~5.11]{HuMathas:GradedCellular}, the
  module $D^\mu$ arises as a quotient of the corresponding graded Specht
  module~\cite{BKW:GradedSpecht,HuMathas:GradedCellular}.

  If $M$ is an ${\mathscr{R}_e({\mathfrak{S}_{n}}}-module let $M^{\mathtt{sgn}}$ be the ${\mathscr{R}_e({\mathfrak{S}_{n}}}-module that is
  isomorphic to~$M$ as a vector space but where the ${\mathscr{R}_e({\mathfrak{S}_{n}}}-action is
  twisted by~${\mathtt{sgn}}$.  By \cite[Theorem~3.6.6]{Mathas:Singapore},
  $(D^\mu)^{\mathtt{sgn}}\cong D^{{\mathbf{m}}(\mu)}$ where ${\mathbf{m}}{\,{:}\,{\mathcal{R}_{n}}\!\longrightarrow\!$} is
  the \textbf{Mullineux map}. Therefore, a straightforward application of
  Clifford theory implies that if $\mu\ne{\mathbf{m}}(\mu)$ then $\operatorname{Res} D^\mu\cong
  \operatorname{Res} D^{{\mathbf{m}}(\mu)}$ is an irreducible graded $\RAn$-module and if
  $\mu={\mathbf{m}}(\mu)$ then, over an algebraically closed field, $\operatorname{Res}
  D^\mu= D^\mu_+\oplus D^\mu_-$, for non-isomorphic irreducible graded
  $\RAn$-modules $D^\mu_+$ and $D^\mu_-$. Set
  \[
     {\mathcal{R}_{n}}{{\mathbf{m}}\gdom}=\set{\mu\in{\mathcal{R}_{n}}{\mathbf{m}}(\mu)\gdom\mu}
        \quad\text{and}\quad
     {\mathcal{R}_{n}}{\mathbf{m}}=\set{\mu\in{\mathcal{R}_{n}}\mu={\mathbf{m}}(\mu)}.
  \]
  Clifford theory implies that every irreducible graded $\RAn$-module arises in
  the manner described above, so we obtain the following.

  \begin{Theorem}\label{T:GRadedSimples}
    Suppose that $F$ is an algebraically closed field of characteristic
    different from~$2$ and that $e>2$. Then
    \[\set{D^\mu\<d\>|d\in{\mathbb{Z}} \text{ and } \mu\in{\mathcal{R}_{n}}{{\mathbf{m}}\gdom}}\cup
    \set{D^\mu_+\<d\>, D^\mu_-\<d\>|d\in{\mathbb{Z}} \text{ and } \mu\in{\mathcal{R}_{n}}{\mathbf{m}}}.\]
   is a complete set of pairwise non-isomorphic irreducible graded $\RAn$-modules.
   \end{Theorem}

   In the semisimple case, ${\mathcal{R}_{n}}{\mathcal{P}_{n}} and ${\mathbf{m}}(\mu)=\mu'$. If
   $\mu\in{\mathcal{P}_{n}} and $\mu=\mu'$ then
   \cite[Proposition~3.9]{MathasRatliff} gives an explicit construction
   of the modules $D^\mu_+$ and $D^\mu_-$ over the field of
   fractions~${\mathcal{K}}$ of the idempotent subring ${\mathcal{O}}$ from
   \autoref{D:squareroots}.

   Finally we turn to the blocks of $\RAn$. By \autoref{C:RAnBlocks}, \[  \RAn =
   \bigoplus_{\gamma\in{Q^\varepsilon_n}}\RAn_\gamma,\]
   where $\RAn_\gamma$ is a two-sided graded ideal of~$\RAn$. Our last result
   says that if $\gamma\in{Q^\varepsilon_n}$ then $\RAn_\gamma$ is a block, or
   indecomposable two-sided ideal, of~$\RAn$ except when
   $\operatorname{\mathrm{def}}\gamma=0$ and $|\gamma|=1$. In the traditional language of
   the symmetric groups, $\operatorname{\mathrm{def}}\gamma=0$ if and only if  the
   partitions in ${\mathcal{P}_{\gamma}} are $e$-cores and, in this case,
   $|\gamma|=1$ if and only if ${\mathcal{P}_{\gamma}}\set{\mu}$, where
   $\mu={\mathbf{m}}(\mu)$ is a Mullineux self-conjugate partition. As~$\mu$ is
   an $e$-core, $D^\mu$ is an irreducible Specht module and $\operatorname{Res}
   D^\mu=D^\mu_+\oplus D^\mu_-$, similar to the situation considered in
   the last paragraph.

  \begin{Theorem}\label{T:RAnBlocks}
    Suppose that $F$ is a field of characteristic different from~$2$
    and that $e>2$. Let $\gamma\in{Q^\varepsilon_n}$. Then:
    \begin{enumerate}
      \item If $|\gamma|=2$ or $\operatorname{\mathrm{def}}\gamma>0$ then
      ${\mathscr{R}^{F}_e({\mathfrak{A}_{n}}}\gamma$ is an indecomposable two-sided graded ideal of~${\mathscr{R}^{F}_e({\mathfrak{A}_{n}}}.
      \item If $F$ is algebraically closed, $|\gamma|=1$ and $\operatorname{\mathrm{def}}\gamma=0$
      then ${\mathscr{R}^{F}_e({\mathfrak{A}_{n}}}\gamma$ is a direct sum of two conjugate matrix
      algebras.
    \end{enumerate}
  \end{Theorem}

  \begin{proof}
    This follows by the general theory of covering blocks for
    ${\mathbb{Z}}_2$-graded algebras as can be found, for example, in
    \cite{Witherspoon:Clifford}. In more detail a block~$A$ of ${{\mathscr{H}}_\xi({\mathfrak{S}_{n}}}
    \textit{covers} a block~$B$ of $\RAn$ if~$B$ is a direct summand of
    the restriction of~$A$ to~$\RAn$. Since $|{\mathbb{Z}}_2|=2$ the blocks
    of~$\RAn$ are covered by at most two blocks of~${{\mathscr{H}}_\xi({\mathfrak{S}_{n}}} and, in
    particular, $\RAn_\gamma$ is indecomposable if $|\gamma|=2$.
    If $|\gamma|=1$ and $\operatorname{\mathrm{def}}\gamma>0$ then there exists a partition
    $\mu\in{\mathcal{R}_{n}}{\mathcal{P}_{\gamma}} such that
    $\mu\ne{\mathbf{m}}(\mu)\in{\mathcal{P}_{\gamma}}. Therefore, $\RAn_\gamma$ is a
    self-conjugate block, so that it is indecomposable. Finally, if~$F$
    is algebraically closed, $\operatorname{\mathrm{def}}\gamma=0$ and $|\gamma|=1$ then
    $\RAn_\gamma$ is the direct sum of two matrix algebras, in view of
    the remarks in the paragraph before the theorem.
  \end{proof}

\bibliographystyle{andrew}

\begin{thebibliography}{10}

\bibitem{Boys:PhDThesis}
{\sc C.~Boys}, {\em Alternating quiver Hecke algebras}, PhD thesis, University
  of Sydney, 2014.

\bibitem{BK:GradedKL}
{\sc J.~Brundan and A.~Kleshchev},
  \href{http://dx.doi.org/10.1007/s00222-009-0204-8}{{\em Blocks of cyclotomic
  {H}ecke algebras and {K}hovanov-{L}auda algebras}}, Invent. Math., {\bf 178}
  (2009), 451--484.

\bibitem{BK:GradedDecomp}
\leavevmode\vrule height 2pt depth -1.6pt width 23pt,
  \href{http://dx.doi.org/10.1016/j.aim.2009.06.018}{{\em Graded decomposition
  numbers for cyclotomic {H}ecke algebras}}, Adv. Math., {\bf 222} (2009),
  1883--1942.

\bibitem{BKW:GradedSpecht}
{\sc J.~Brundan, A.~Kleshchev, and W.~Wang},
  \href{http://dx.doi.org/10.1515/CRELLE.2011.033}{{\em Graded {S}pecht
  modules}}, J. Reine Angew. Math., {\bf 655} (2011), 61--87.
\newblock \href{http://arxiv.org/abs/0901.0218}{arXiv:0901.0218}.

\bibitem{BrundanStroppel:KhovanovI}
{\sc J.~Brundan and C.~Stroppel}, {\em Highest weight categories arising from
  {K}hovanov's diagram algebra {I}: cellularity}, Mosc. Math. J., {\bf 11}
  (2011), 685--722, 821--822.
\newblock \href{http://arxiv.org/abs/0806.1532}{arXiv:0806.1532}.

\bibitem{GL}
{\sc J.~J. Graham and G.~I. Lehrer},
  \href{http://dx.doi.org/10.1007/BF01232365}{{\em Cellular algebras}}, Invent.
  Math., {\bf 123} (1996), 1--34.

\bibitem{HuMathas:GradedCellular}
{\sc J.~Hu and A.~Mathas},
  \href{http://dx.doi.org/10.1016/j.aim.2010.03.002}{{\em Graded cellular bases
  for the cyclotomic {K}hovanov-{L}auda-{R}ouquier algebras of type {$A$}}},
  Adv. Math., {\bf 225} (2010), 598--642.
\newblock \href{http://arxiv.org/abs/0907.2985}{arXiv:0907.2985}.

\bibitem{HuMathas:QuiverSchurI}
\leavevmode\vrule height 2pt depth -1.6pt width 23pt,
  \href{http://dx.doi.org/dx.doi.org/10.1112/plms/pdv007}{{\em Cyclotomic
  quiver {S}chur algebras for linear quivers}}, Proc. London Math. Soc., {\bf
  110} (2015), 1315--1386.
\newblock \href{http://arxiv.org/abs/1110.1699}{arXiv:1110.1699}.

\bibitem{HuMathas:SeminormalQuiver}
\leavevmode\vrule height 2pt depth -1.6pt width 23pt,
  \href{http://dx.doi.org/10.1007/s00208-015-1242-8}{{\em Seminormal forms and
  cyclotomic quiver Hecke algebras of type~{$A$}}}, Mathematische Annalen,
  2015, in press.
\newblock \href{http://arxiv.org/abs/1304.0906}{arXiv:1304.0906}.

\bibitem{Iwahori:Hecke}
{\sc N.~Iwahori}, {\em On the structure of a {H}ecke ring of a {C}hevalley
  group over a finite field}, J. Fac. Sci. Univ. Tokyo Sect. I, {\bf 10}
  (1964), 215--236 (1964).

\bibitem{James}
{\sc G.~James}, \href{http://dx.doi.org/10.1007/BFb0067708}{{\em The
  representation theory of the symmetric groups}}, Lecture Notes in
  Mathematics, {\bf 682}, Springer, Berlin, 1978.

\bibitem{JM:Schaper}
{\sc G.~James and A.~Mathas},
  \href{http://dx.doi.org/10.1112/S0024611597000099}{{\em A {$q$}-analogue of
  the {J}antzen-{S}chaper theorem}}, Proc. London Math. Soc. (3), {\bf 74}
  (1997), 241--274.

\bibitem{Kac}
{\sc V.~G. Kac}, \href{http://dx.doi.org/10.1017/CBO9780511626234}{{\em
  Infinite-dimensional {L}ie algebras}}, Cambridge University Press, Cambridge,
  third~ed., 1990.

\bibitem{KhovLaud:diagI}
{\sc M.~Khovanov and A.~D. Lauda},
  \href{http://dx.doi.org/10.1090/S1088-4165-09-00346-X}{{\em A diagrammatic
  approach to categorification of quantum groups. {I}}}, Represent. Theory,
  {\bf 13} (2009), 309--347.

\bibitem{KMR:UniversalSpecht}
{\sc A.~Kleshchev, A.~Mathas, and A.~Ram},
  \href{http://dx.doi.org/10.1112/plms/pds019}{{\em Universal graded {S}pecht
  modules for cyclotomic {H}ecke algebras}}, Proc. Lond. Math. Soc. (3), {\bf
  105} (2012), 1245--1289.
\newblock \href{http://arxiv.org/abs/1102.3519}{arXiv:1102.3519}.

\bibitem{GeLi:IntegralKLR}
{\sc G.~Li}, {\em Integral Basis Theorem of cyclotomic Khovanov-Lauda-Rouquier
  algebras of Type A}, 2014.
\newblock \href{http://arxiv.org/abs/1412.3747}{arXiv:1412.3747}.

\bibitem{M:ULect}
{\sc A.~Mathas}, \href{http://dx.doi.org/10.1090/ulect/015}{{\em
  Iwahori-{H}ecke algebras and {S}chur algebras of the symmetric group}},
  University Lecture Series, {\bf 15}, American Mathematical Society,
  Providence, RI, 1999.

\bibitem{Mathas:Singapore}
\leavevmode\vrule height 2pt depth -1.6pt width 23pt,
  \href{http://dx.doi.org/10.1142/9789814651813_0005}{{\em Cyclotomic quiver
  {H}ecke algebras of type {A}}}, in Modular representation theory of finite
  and $p$-adic groups, G.~W. Teck and K.~M. Tan, eds., National University of
  Singapore Lecture Notes Series, {\bf 30}, World Scientific, 2015, ch.~5,
  165--266.
\newblock \href{http://arxiv.org/abs/1310.2142}{arXiv:1310.2142}.

\bibitem{MathasRatliff}
{\sc A.~Mathas and L.~Neves}, {\em The irreducible characters of the
  alternating Hecke algebra}, 2015, preprint.

\bibitem{Mitsuhashi:A}
{\sc H.~Mitsuhashi}, \href{http://dx.doi.org/10.1006/jabr.2000.8715}{{\em The
  {$q$}-analogue of the alternating group and its representations}}, J.
  Algebra, {\bf 240} (2001), 535--558.

\bibitem{M:Nak}
{\sc G.~E. Murphy}, \href{http://dx.doi.org/10.1016/0021-8693(83)90219-3}{{\em
  The idempotents of the symmetric group and {N}akayama's conjecture}}, J.
  Algebra, {\bf 81} (1983), 258--265.

\bibitem{Rouq:2KM}
{\sc R.~Rouquier}, {\em 2-Kac-Moody algebras}, 2008, preprint.
\newblock \href{http://arxiv.org/abs/0812.5023}{arXiv:0812.5023}.

\bibitem{StroppelWebster:QuiverSchur}
{\sc C.~Stroppel and B.~Webster}, {\em Quiver {S}chur algebras and $q$-{F}ock
  space}, 2011, preprint.
\newblock \href{http://arxiv.org/abs/1110.1115}{arXiv:1110.1115}.

\bibitem{Witherspoon:Clifford}
{\sc S.~J. Witherspoon},
  \href{http://dx.doi.org/10.1090/S0002-9939-99-05224-7}{{\em A
  module-theoretic approach to {C}lifford theory for blocks}}, Proc. Amer.
  Math. Soc., {\bf 128} (2000), 661--670.

\end{thebibliography}

\end{document}


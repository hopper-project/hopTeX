\documentclass[a4paper,fleqn,11pt]{amsart}

\usepackage{mathrsfs}
\usepackage{mathptmx}
\usepackage{cite}
\usepackage{amssymb}

\DeclareMathAlphabet{\mathpzc}{OT1}{pzc}{m}{it}

\newtheorem{theorem}{Theorem}[section]
\newtheorem{lemma}[theorem]{Lemma}
\newtheorem{claim}[theorem]{Claim}
\newtheorem{corollary}[theorem]{Corollary}
\newtheorem{conjecture}[theorem]{Conjecture}
\newtheorem{proposition}[theorem]{Proposition}
\theoremstyle{definition}
\newtheorem{definition}[theorem]{Definition}
\newtheorem{example}[theorem]{Example}
\theoremstyle{remark}
\newtheorem{remark}[theorem]{Remark}
\newtheorem{notation}[theorem]{Notation}

\numberwithin{equation}{section}

\allowdisplaybreaks[4]

\begin{document}

\title{Deformations of weighted homogeneous polynomials with line singularities and equimultiplicity} 

\author{Christophe Eyral and Maria Aparecida Soares Ruas}
\address{C. Eyral, Institute of Mathematics, Polish Academy of Sciences, \'Sniadeckich~8, 00-656 Warsaw, Poland}
\email{eyralchr@yahoo.com}

\address{M. Ruas, Instituto de Ci\^encias Matem\'aticas e de Computa\c c\~ao, Universidade de S\~ao Paulo, Avenida Trabalhador S\~ao-Carlense, 400 - Centro, Caixa Postal 668, 13566-590 S\~ao Carlos - SP, Brazil}
\email{maasruas@icmc.usp.br}

\subjclass[2010]{32S15, 32S25, 32S05.}

\keywords{Weighted homogeneous polynomials; line singularities; L\^e numbers; polar numbers; topological equisingularity; equimultiplicity, Zariski's multiplicity conjecture.}

\begin{abstract}
Consider a family $\{f_t\}$ of complex polynomial functions with line singularities, and assume that $f_0$ is weighted homogeneous. We show that if the $1$-st L\^e number and the $1$-st polar number of $f_t$ at $\mathbf{0}$ are independent of $t$, then the family $\{f_t\}$ is equimultiple. In particular, $\{f_t\}$ is equimultiple if the $1$-st polar number of $f_t$ at $\mathbf{0}$ and the local ambient topology of the hypersurface $f_t^{-1}(0)$ at~$\mathbf{0}$ are independent of $t$. This partially answers the famous Zariski multiplicity conjecture for this special class of non-isolated singularities.
\end{abstract}

\maketitle

\markboth{C. Eyral and M. Ruas}{Deformations of weighted homogeneous line singularities and equimultiplicity} 

\section{Introduction}\label{intro}

Let $\mathbf{z}:=(z_1,\ldots,z_n)$ be linear coordinates for $\mathbb{C}^n$ ($n\geq 2$), and let 
\begin{equation*}
f_0\colon (\mathbb{C}^n,\mathbf{0})\rightarrow (\mathbb{C},0),\ 
\mathbf{z}\mapsto f_0(\mathbf{z}), 
\end{equation*}
be a \emph{weighted homogeneous} polynomial function. We suppose that $f_0$ is reduced at~$\mathbf{0}$. A \emph{deformation} of $f_0$ is a polynomial function 
\begin{equation*}
f\colon (\mathbb{C} \times \mathbb{C}^n,\mathbb{C} \times \{\mathbf{0}\}) \rightarrow (\mathbb{C},0),\ 
(t,\mathbf{z})\mapsto f(t,\mathbf{z}), 
\end{equation*}
such that the following two conditions hold:
\begin{enumerate}
\item
$f(0,\mathbf{z})=f_0(\mathbf{z})$ for any $\mathbf{z}\in\mathbb{C}^n$; 
\item
if we write $f_t(\mathbf{z}):= f(t,\mathbf{z})$, then each $f_t$ is reduced at~$\mathbf{0}$.
\end{enumerate}
Thus a deformation of $f_0$ may be viewed as a $1$-parameter family of polynomial functions $f_t$ locally reduced at $\mathbf{0}$ and depending polynomially on the parameter~$t$. (Note that, by definition, $f_t(\mathbf{0}) = f(t,\mathbf{0})=0$ for any $t\in\mathbb{C}$.) In this paper, we are interested in the local ambient topology of the hypersurfaces $V(f_t):=f_t^{-1}(0)$ in a~neighbourhood of the origin $\mathbf{0}\in \mathbb{C}^n$ as the parameter $t$ varies from a ``small'' non-zero  value $t_0$ to $t=0$. More precisely, we are looking for easy-to-check conditions on the members $f_t$ of the family $\{f_t\}$ that will guarantee topological equisingularity for the corresponding family of hypersurfaces $\{V(f_t)\}$.

In the case where the functions $f_t$ have an \emph{isolated} singularity at $\mathbf{0}$, G.-M. Greuel \cite{G} and D. O'Shea \cite{O'Sh} proved the following result.

\begin{theorem}[Greuel and O'Shea]\label{thm-GOSh}
Suppose that $\{f_t\}$ is a family of isolated hypersurface singularities such that the polynomial function $f_0$ is weighted homogeneous with respect to a given system of weights~$\mathbf{w}:=(w_1,\ldots,w_n)$ with $w_i\in \mathbb{N}\setminus\{0\}$. Under these assumptions, if, furthermore, the family $\{f_t\}$ is $\mu$-constant  (i.e., if for all sufficiently small $t$, the Milnor number of $f_t$ at $\mathbf{0}$ is independent of $t$), then it is equimultiple (i.e., the multiplicity of $f_t$ at $\mathbf{0}$ is independent of $t$ for all sufficiently small $t$).
\end{theorem}

Here, by the \emph{multiplicity} of $f_t$ at $\mathbf{0}$ (denoted by $\mbox{mult}_{\mathbf{0}}(f_t)$) we mean the number of points of intersection near~$\mathbf{0}$ of~$V(f_t)$ with a generic line in~$\mathbb{C}^n$ passing arbitrarily close to, but not through, the origin. As $f_t$ is reduced at $\mathbf{0}$, this number  coincides with the order of~$f_t$ at~$\mathbf{0}$ (denoted by $\mbox{ord}_{\mathbf{0}}(f_t)$).

In the special case where all the weights $w_i$ ($1\leq i\leq n$) are equal to $1$ (i.e., $f_0$ is a homogeneous polynomial), Theorem \ref{thm-GOSh} was first proved by A.~M. Gabri\`elov and A.~G. Ku\v{s}nirenko in \cite{GK}.

As the Milnor number is a topological invariant (cf.~\cite{L3,Mi,T1,T2,T3}), Theorem \ref{thm-GOSh} implies that any topologically $\mathscr{V}$-equisingular deformation of an isolated hypersurface singularity defined by a weighted homogeneous polynomial function is equimultiple. Thus Theorem \ref{thm-GOSh} partially answers the famous Zariski multiplicity conjecture \cite{Z} for this special class of singularities.\footnote{The Zariski multiplicity conjecture says that any topologically $\mathscr{V}$-equisingular family of (possibly non-isolated) hypersurface singularities is equimultiple (cf.~\cite{Z}). For a survey on this conjecture and related topics, we refer the reader to \cite{E2,EyBook}.}
We recall that a family $\{f_t\}$ is said to be \emph{topologically $\mathscr{V}$-equisingular} if there exist open neighbourhoods $D$ and $U$ of the origins in $\mathbb{C}$ and $\mathbb{C}^n$, respectively, together with a continuous map $\varphi\colon (D\times U, D\times \{\mathbf{0}\})\rightarrow (\mathbb{C}^n,\mathbf{0})$ such that for all sufficiently small $t$, there is an open neighbourhood $U_t\subseteq U$ of $\mathbf{0}\in\mathbb{C}^n$ such that
the map $\varphi_t\colon (U_t,\mathbf{0})\rightarrow (\varphi(\{t\}\times U_t),\mathbf{0})$ defined by $\varphi_t(\mathbf{z}):=\varphi(t,\mathbf{z})$ is a homeomorphism sending $V(f_0)\cap U_t$ onto $V(f_t)\cap \varphi_t(U_t)$.

The proof of Theorem \ref{thm-GOSh} relies on a very deep theorem of A. N. Varchenko~\cite{V}, which says that if the assumptions of Theorem \ref{thm-GOSh} are satisfied, then the deformation family $\{f_t\}$ is \emph{upper}. The word ``upper'' is defined as follows. Expand $f(t,\mathbf{z})$ with respect to the deformation parameter $t$:
\begin{equation*}
f(t,\mathbf{z})=f_0(\mathbf{z})+\sum_{j} t^j g_j(\mathbf{z}).
\end{equation*}
(Here, $g_j\colon (\mathbb{C}^n,\mathbf{0})\to (\mathbb{C},0)$ is a polynomial function.) We say that $\{f_t\}$ is \emph{upper} if each $g_j(\mathbf{z})$ is a linear combination of monomials of weighted degree (with respect to the weights $\mathbf{w}$) greater than or equal to the weighted degree of $f_0$. 

In the present paper, we investigate the same question as Greuel and O'Shea for the simplest class of hypersurfaces with non-isolated singularities---namely, the hypersurfaces with \emph{line} singularities. 
Certainly, for such singularities, the Milnor number is no longer relevant. However, the L\^e numbers of D. Massey \cite{M3,M4,M,M2}, together with the polar numbers, may be used instead.  
Then, our main result says that the theorem of Greuel and O'Shea extends to line singularities provided that the constancy of the Milnor number is replaced by the constancy of both the $1$-st L\^e number and the $1$-st polar number (cf.~Theorem \ref{mt2}). While the L\^e numbers are not topological invariants for arbitrary non-isolated singularities, for line singularities they are constant if the local ambient topological type of $V(f_t)$ at $\mathbf{0}$ is constant (cf.~\cite{M5,M7}). Therefore, combined with our main result, we get that any family $\{f_t\}$ of line singularities, with $f_0$ weighted homogeneous, is equimultiple if the $1$-st polar number of $f_t$ at~$\mathbf{0}$ and the local ambient topological type of $V(f_t)$ at $\mathbf{0}$ are both independent of $t$ (cf.~Theorem \ref{cmt2}). This partially answers the Zariski multiplicity conjecture for this special class of non-isolated singularities. Note that in the case where the polynomial $f_0$ is homogeneous, the first author already proved in \cite{Ey2} the following (stronger) result: ``If $\{f_t\}$ is a topologically $\mathscr{V}$-equisingular family of line singularities---or even a family of line singularities with constant L\^e numbers---and if $f_0$ is homogeneous, then $\{f_t\}$ is equimultiple.'' Theorem \ref{cmt2} may be very useful to decide whether families of hypersurfaces with line singularities are \emph{not} topologically $\mathscr{V}$-equisingular---a question which is, in general, extremely difficult to answer (cf.~Section \ref{appli}).

\section{Statement of the results}\label{sect-sr}

Suppose that $\{f_t\}$ is a family of \emph{line} singularities. As in \cite[\S4]{M7}, by this we mean that for each $t$ near $0\in\mathbb{C}$ the singular locus $\Sigma f_t$ of $f_t$ near the origin $\mathbf{0}\in\mathbb{C}^n$ is given by the $z_1$-axis, and the restriction of $f_t$ to the hyperplane $V(z_1)$ defined by $z_1=0$ has an isolated singularity at the origin. Then, by \cite[Remark 1.29]{M}, the partition of $V(f_t)$ given by
\begin{equation*}
\mathscr{S}_t:=\bigl\{V(f_t)\setminus\Sigma f_t,\Sigma f_t\setminus\{\mathbf{0}\}, \{\mathbf{0}\}\bigr\}
\end{equation*}
is a ``good stratification'' for $f_t$ in a neighbourhood of $\mathbf{0}$, and the hyperplane $V(z_1)$ is a ``prepolar slice'' for $f_t$ at $\mathbf{0}$ with respect to $\mathscr{S}_t$ for all $t$ small enough.\footnote{A \emph{good stratification} for~$f_t$ at a point $\mathbf{p}\in V(f_t)$ is an
analytic stratification $\mathscr{S}$ of $V(f_t)$ in a neighbourhood $U$ of $\mathbf{p}$ such that the following two conditions are satisfied:
\begin{enumerate}
\item the (trace of the) smooth part of $V(f_t)$ is a stratum; 
\item$\mathscr{S}$ satisfies Thom's $a_{f_t}$ condition with respect to $U\setminus \Sigma f_t$, that is, if $\{\mathbf{q}_k\}$ is a sequence of points in $U\setminus \Sigma f_t$ such that $\mathbf{q}_k\rightarrow \mathbf{q}\in S\in\mathscr{S}$ and
$T_{\mathbf{q}_k}V(f_t-f_t(\mathbf{q}_k))\rightarrow T$, then $T_{\mathbf{q}}S\subseteq T$.
\end{enumerate}
As usual, $T_{\mathbf{q}_k} V({f_t-f_t(\mathbf{q}_k)})$ denotes the tangent space at $\mathbf{q}_k$ to the level hypersurface defined by $f_t(\mathbf{z})=f_t(\mathbf{q}_k)$, and $T_{\mathbf{q}}S$ the tangent space at $\mathbf{q}$ to the stratum $S$.
Note that good stratifications always exist (cf.~\cite{HL}).
Now if $\mathscr{S}$ is a good stratification for $f_t$ at a point $\mathbf{p}\in V(f_t)$, then a hyperplane $H$ of $\mathbb{C}^n$ through $\mathbf{p}$ is called a \emph{prepolar slice} for $f_t$ at $\mathbf{p}$ with respect to $\mathscr{S}$ if it transversely intersects all the strata of $\mathscr{S}$---perhaps with the exception of the stratum $\{\mathbf{p}\}$ itself---in a neighbourhood of $\mathbf{p}$. For details, see Chapter 1 of Massey's book \cite{M}.} 
In particular, combined with \cite[Proposition~1.23]{M}, this implies that the L\^e numbers $\lambda^0_{f_t,\mathbf{z}}(\mathbf{0})$ and $\lambda^1_{f_t,\mathbf{z}}(\mathbf{0})$ and the polar number $\gamma^1_{f_t,\mathbf{z}} (\mathbf{0})$
of $f_t$ at~$\mathbf{0}$ with respect to the coordinates $\mathbf{z}$ do exist. (For the definitions of the L\^e numbers and the polar numbers, we refer the reader to Chapter 1 of D. Massey's book \cite{M}. For the reader's convenience, we also briefly recall the definitions in the appendix below.) Note that for line singularities, the only possible non-zero L\^e numbers are precisely $\lambda^0_{f_t,\mathbf{z}}(\mathbf{0})$ and $\lambda^1_{f_t,\mathbf{z}}(\mathbf{0})$; all the other L\^e numbers $\lambda^k_{f_t,\mathbf{z}}(\mathbf{0})$ for $2\leq k\leq n-1$ are defined and equal to zero (cf.~\cite{M}). 

Here is our main result.

\begin{theorem}\label{mt2}
Assume that $\{f_t\}$ is a family of line singularities such that the polynomial function $f_0$ is weighted homogeneous with respect to a given system of weights~$\mathbf{w}:=(w_1,\ldots,w_n)$ with $w_i\in \mathbb{N}\setminus\{0\}$. 
Under these conditions, if, in addition, 
for all sufficiently small $t$, either 
\begin{equation}\label{c1}
\lambda^1_{f_t,\mathbf{z}} (\mathbf{0}) \mbox{ and } 
\gamma^1_{f_t,\mathbf{z}} (\mathbf{0})
\end{equation} 
are independent of $t$ or 
\begin{equation}\label{c2}
\lambda^1_{f_t,\mathbf{z}} (\mathbf{0}) \mbox{ and } 
\gamma^1_{f_t,\mathbf{z}} (\mathbf{0}) + \lambda^0_{f_t,\mathbf{z}} (\mathbf{0})
\end{equation}  
are independent of $t$, then the family $\{f_t\}$ is equimultiple.
\end{theorem}

Theorem \ref{mt2} will be proved in Section \ref{prooftheorem}.

The condition (\ref{c2}) is already used by D. Massey in \cite[\S5]{M7} in his first partial generalization of the L\^e-Ramanujam theorem to line singularities. 
Indeed, by \cite[Proposition 1.23]{M}, 
\begin{displaymath}
\gamma^1_{f_t,\mathbf{z}} (\mathbf{0}) + \lambda^0_{f_t,\mathbf{z}} (\mathbf{0})
= \bigl([\Gamma^1_{f_t,\mathbf{z}}] \cdot [V(f_t)]\bigr)_{\mathbf{0}},
\end{displaymath} 
where $[\Gamma^1_{f_t,\mathbf{z}}]$ (respectively, $[V(f_t)]$) is the analytic cycle associated to the relative $1$-st polar variety $\Gamma^1_{f_t,\mathbf{z}}$ of $f_t$ with respect to $\mathbf{z}$ (respectively, the analytic cycle associated to the hypersurface $V(f_t)$), and where 
$([\Gamma^1_{f_t,\mathbf{z}}] \cdot [V(f_t)])_{\mathbf{0}}$
is the intersection number at $\mathbf{0}$ of these two cycles (cf.~Appendix \ref{appendix}). 
In \cite[Theorem (5.2)]{M7}, Massey showed that if $n\geq 5$ and if $\lambda^1_{f_t,\mathbf{z}} (\mathbf{0})$ and $\bigl([\Gamma^1_{f_t,\mathbf{z}}] \cdot [V(f_t)]\bigr)_{\mathbf{0}}$ are constant (as $t$ varies)---equivalently, if (\ref{c2}) holds---then the diffeomorphism type of the Milnor fibration of $f_t$ at $\mathbf{0}$ is constant too. Note that in \cite[Theorem 9.4]{M}, Massey proved a stronger result: ``If $n\geq 5$ and if $\lambda^0_{f_t,\mathbf{z}} (\mathbf{0})$ and $\lambda^1_{f_t,\mathbf{z}} (\mathbf{0})$ are constant, then the diffeomorphism type of the Milnor fibration of $f_t$ at $\mathbf{0}$ is constant.''

We shall see in the proof of Theorem \ref{mt2} that (\ref{c2}) implies (\ref{c1}), and hence, it also implies the constancy of $\lambda^0_{f_t,\mathbf{z}}(\mathbf{0})$. It would be nice if, as in \cite[Theorem 9.4]{M}, we could replace (\ref{c1}) by the assumption that $\lambda^0_{f_t,\mathbf{z}} (\mathbf{0})$ and $\lambda^1_{f_t,\mathbf{z}} (\mathbf{0})$ are independent of $t$ for all small $t$ (see also Remark \ref{rac22} below). Note that in the special case where the polynomial $f_0$ is homogeneous, this is possible (cf.~\cite{Ey2}).

Theorem \ref{mt2} may be viewed as a generalization to line singularities of Theorem \ref{thm-GOSh} in the sense that if $g(t,z_2,\ldots,z_n)$ defines a $\mu$-constant family of isolated singularities in $\mathbb{C}^{n-1}$, then the corresponding family of line singularities in $\mathbb{C}^{n}$---defined by $f(t,z_1,z_2,\ldots,z_n):=g(t,z_2,\ldots,z_n)$---is such that both $\lambda^1_{f_t,\mathbf{z}} (\mathbf{0})$ and $\gamma^1_{f_t,\mathbf{z}} (\mathbf{0})$ are constant, and therefore, by Theorem \ref{mt2}, the family  $\{f_t\}$ (and hence the family $\{g_t\}$) is equimultiple if $g_0$ is a weighted homogeneous polynomial. The constancy of $\lambda^1_{f_t,\mathbf{z}} (\mathbf{0})$ and $\gamma^1_{f_t,\mathbf{z}} (\mathbf{0})$ under the $\mu$-constancy of $\{g_t\}$ is explained by Massey in \cite[\S 5]{M7}. His argument is as follows. To show that $\gamma^1_{f_t,\mathbf{z}} (\mathbf{0}) $ is constant, it suffices to observe that for line singularities, the relative 1-st polar variety
$\Gamma^1_{f_t,\mathbf{z}}$ of $f_t$ with respect to $\mathbf{z}$ is empty, and therefore
\begin{displaymath}
\gamma^1_{f_t,\mathbf{z}} (\mathbf{0}) = 
([\Gamma^1_{f_t,\mathbf{z}}] \cdot [V(z_1)])_{\mathbf{0}}=0
\end{displaymath}
for all small $t$.
To show that $\lambda^1_{f_t,\mathbf{z}} (\mathbf{0})$ is constant, he observes that for line singularities,
\begin{displaymath}
\lambda^1_{f_t,\mathbf{z}} (\mathbf{0}) = \overset{\circ}{\mu}_{f_t}, 
\end{displaymath}
where $\overset{\circ}{\mu}_{f_t}$ is the Milnor number of a generic hyperplane slice of $f_t$ at a point on $\Sigma f_t$ sufficiently close to the origin (cf.~\cite{L4,M7,M}). But in our case $\overset{\circ}{\mu}_{f_t}$ is nothing but the Milnor number of $g_t$ at the origin, which is constant.  

Also, note that requiring that (\ref{c2})---or (\ref{c1})---holds does not imply the Whitney conditions along the $t$-axis (cf.~\cite[\S 5]{M7}). If the family $\{f_t\}$ were Whitney equisingular (i.e., if there were a Whitney stratification of $V(f):=f^{-1}(0)\subseteq\mathbb{C}\times\mathbb{C}^n$ with the $t$-axis as a stratum), then the result would follow immediately from a famous theorem of H.~Hironaka which says that any reduced complex analytic space endowed with a Whitney stratification  is equimultiple along every stratum (cf.~\cite[Corollary (6.2)]{H}).

In \cite[Corollary 6.6]{GG}, T. Gaffney and R. Gassler proved that in the special case where $\{f_t\}$ is a family of surface singularities in $\mathbb{C}^3$, then the family $\{f_t\}$ is Whitney equisingular if, in addition to the condition (\ref{c1}), the second polar number $\gamma^2_{f_t,\mathbf{z}}(\mathbf{0})$ is independent of $t$ as well. They even showed that, still in the case of surfaces in $\mathbb{C}^3$, Whitney equisingularity does hold true if and only if the L\^e numbers $\lambda^1_{f_t,\mathbf{z}}(\mathbf{0})$, $\lambda^2_{f_t,\mathbf{z}}(\mathbf{0})$ and the polar numbers $\gamma^1_{f_t,\mathbf{z}}(\mathbf{0})$, $\gamma^2_{f_t,\mathbf{z}}(\mathbf{0})$ are all independent of $t$ for all small $t$.

Finally, we observe that Theorem \ref{mt2} has the following important corollary.

\begin{theorem}\label{cmt2}
Assume that $\{f_t\}$ is a family of line singularities such that the polynomial function $f_0$ is weighted homogeneous with respect to a given system of weights~$\mathbf{w}:=(w_1,\ldots,w_n)$ with $w_i\in \mathbb{N}\setminus\{0\}$. 
Under these conditions, if, furthermore, the family $\{f_t\}$ is topologically $\mathscr{V}$-equisingular and the first polar number $\gamma^1_{f_t,\mathbf{z}} (\mathbf{0})$ is independent of $t$ for all small $t$, then $\{f_t\}$ is equimultiple.
\end{theorem}

\begin{proof}
We argue by contradiction. Suppose that the family $\{f_t\}$ is not equimultiple. Then, as $\gamma^1_{f_t,\mathbf{z}} (\mathbf{0})$ is independent of $t$, it follows from Theorem \ref{mt2} that the L\^e number $\lambda^1_{f_t,\mathbf{z}} (\mathbf{0})$ is not constant as $t$ varies from an arbitrary small non-zero value  $t_0$ to $t=0$. But as mentioned above, for line singularities, $\lambda^1_{f_t,\mathbf{z}} (\mathbf{0})$ is nothing but the Milnor number $\overset{\circ}{\mu}_{f_t}$ of a generic hyperplane slice of $f_t$ at a point on $\Sigma f_t$ sufficiently close to the origin. Moreover, for such singularities, $\overset{\circ}{\mu}_{f_t}$ is an invariant of the ambient topological type of $V(f_t)$ at $\mathbf{0}$ (cf.~\cite{M5,M7}). It follows that the family $\{f_t\}$ is not topologically $\mathscr{V}$-equisingular---a contradiction.
\end{proof}

\begin{remark}\label{rac22}
If we could replace the assumption (\ref{c1}) in Theorem \ref{mt2} by the condition that $\lambda^0_{f_t,\mathbf{z}} (\mathbf{0})$ and $\lambda^1_{f_t,\mathbf{z}} (\mathbf{0})$ are constant (as $t$ varies), then we could also avoid the assumption that $\gamma^1_{f_t,\mathbf{z}} (\mathbf{0})$ is constant in Theorem \ref{cmt2}. This follows from the following observation of Massey \cite[\S 4]{M7}: ``The L\^e numbers $\lambda^0_{f_t,\mathbf{z}} (\mathbf{0})$ and $\lambda^1_{f_t,\mathbf{z}} (\mathbf{0})$ are constant if and only if the (embedded) topological invariants $\overset{\circ}{\mu}_{f_t}$ and $\tilde\chi(F_{f_t,\mathbf{0}})$ are constant.'' Here, $\tilde\chi(F_{f_t,\mathbf{0}})$ is the reduced Euler characteristic of the Milnor fibre $F_{f_t,\mathbf{0}}$ of $f_t$ at $\mathbf{0}$, which is given in our case by 
\begin{align*}
\tilde \chi(F_{f_t,\mathbf{0}}) 
& = (-1)^{n-1} \lambda^0_{f_t,\mathbf{z}}(\mathbf{0}) + (-1)^{n-2} \lambda^1_{f_t,\mathbf{z}}(\mathbf{0})
\end{align*}
(see \cite[\S 4]{M7} or \cite[Theorem 3.3]{M}).  
\end{remark}

\begin{remark}	
Suppose that $\{f_t\}$ is a family of \emph{parametrized} surface singularities in $\mathbb{C}^3$, that is, a family for which there is an analytic map $(\mathbb{C}\times\mathbb{C}^2,\mathbb{C}\times\{\mathbf{0}\})  \to
(\mathbb{C}\times\mathbb{C}^3,\mathbb{C}\times\{\mathbf{0}\})$
of the form
\begin{align*}
(t,(\zeta_1,\zeta_2))\in \mathbb{C}\times\mathbb{C}^2 & \mapsto (t,\psi_t(\zeta_1,\zeta_2))\in \mathbb{C}\times\mathbb{C}^3
\end{align*}
satisfying $V(f_t)=\mbox{im}(\psi_t)$. Also, assume there is a neighbourhood $W$ of the origin in $\mathbb{C}^2$ such that the following two conditions hold:
\begin{enumerate}
\item[(a)]
$W\cap \psi_t^{-1}(\mathbf{0})=\{\mathbf{0}\}$;
\item[(b)]
the only singularities of $\psi_t(W)\setminus\{\mathbf{0}\}$ are transverse double points.
\end{enumerate}
Then, by Mather-Gaffney's criterion (cf.~\cite{Wall}), $\psi_t$ is finitely $\mathscr{A}$-determined, and it follows from Theorem 5.3 in \cite{MNP} (see also \cite{Gaffney}) that if $\overset{\circ}{\mu}_{f_t}$ and $\mu(D(\psi_t),\mathbf{0})$ are constant, then, in a neighbourhood of the origin, the partition of $V(f)$ given by
\begin{equation*}
\{V(f)\setminus \Sigma f,\Sigma f\setminus (\mathbb{C}\times \{\mathbf{0}\}), \mathbb{C}\times \{\mathbf{0}\}\}
\end{equation*}
is a Whitney stratification---in particular, the family $\{f_t\}$ is Whitney equisingular. (Here, $\mu(D(\psi_t),\mathbf{0})$ denotes the Milnor number of the double point locus $D(\psi_t)=\psi_t^{-1}(\Sigma f_t)$ of $\psi_t$ at the origin. As usual, $\Sigma f$ is the critical locus of $f$.)

Now if we suppose further that $\{f_t\}$ is topologically $\mathscr{V}$-equi\-singular and such that the first polar number $\gamma^1_{f_t,\mathbf{z}}(\mathbf{0})$ is constant, then, by \cite[Theorem 6.2]{CB-H-R} (see also \cite{BP}), the Milnor number $\mu(D(\psi_t),\mathbf{0})$ is constant, and it follows from \cite[Proposition 3.3]{R}  and \cite[Lemma 5.2]{MNP} that $\overset{\circ}{\mu}_{f_t}$ is constant too. 
Therefore, if $\{f_t\}$ is a topologically $\mathscr{V}$-equi\-singular family of parametrized surface singularities in $\mathbb{C}^3$ with constant first polar number and satisfying the above conditions (a) and (b), then $\{f_t\}$ is Whitney equisingular. For examples of such families that have, in addition, line singularities, we refer the reader to \cite{Mond}.
\end{remark}

\section{Application of Theorem \ref{cmt2}}\label{appli}

Theorem \ref{cmt2} may be very useful to decide whether a family of hypersurfaces with line singularities is \emph{not} topologically $\mathscr{V}$-equisingular---a question which is, in general, extremely difficult to answer. 
Indeed, Theorem \ref{cmt2} says that in order to show that such a family is not topologically $\mathscr{V}$-equisingular, it suffices to observe that it is not equimultiple and such that the first polar number $\gamma^1_{f_t,\mathbf{z}} (\mathbf{0})$ is independent of $t$ for all small $t$---two very simple checks.

\begin{example}
Consider the family defined by 
\begin{equation*}
f_t(z_1,z_2,z_3)=z_1^2z_2^2+z_2^5+z_3^4+tz_1z_2^2+t^2z_1^2z_2^2.
\end{equation*}
A priori, it is far from being obvious to decide whether this family is topologically $\mathscr{V}$-equisingular or not. However, this easily follows from Theorem \ref{cmt2}. Indeed, the polynomial function $f_0(z_1,z_2,z_3)=z_1^2z_2^2+z_2^5+z_3^4$ is weighted homogeneous with respect to the weights $\mathbf{w}:=(6,4,5)$, the singular locus $\Sigma f_t$ of $f_t$ near the origin is given by the $z_1$-axis, and the restriction $f_{t\mid V(z_1)}$ has an isolated singularity at the origin. An easy computation also shows that for any $t$ sufficiently small, the relative 1-st polar variety $\Gamma^1_{f_t,\mathbf{z}}$ of $f_t$ with respect to $\mathbf{z}$ is given by
\begin{align*}
\Gamma^1_{f_t,\mathbf{z}} & = V\biggl( \frac{\partial f_t}{\partial z_2},\frac{\partial f_t}{\partial z_3} \biggr)\lnot\Sigma f_t\\
& = V\bigl(z_2(2z_1^2+5z_2^3+2tz_1+2t^2z_1^2),4z_3^3\bigr)\lnot V(z_2,z_3)\\
& = V(2z_1^2+5z_2^3+2tz_1+2t^2z_1^2,z_3^3),
\end{align*}
and hence,
\begin{equation*}
\gamma^1_{f_t,\mathbf{z}}(\mathbf{0}) = 
([\Gamma^1_{f_t,\mathbf{z}}]\cdot[V(z_1)])_{\mathbf{0}}=9.
\end{equation*}
As the family $\{f_t\}$ is not equimultiple, it follows from Theorem \ref{cmt2} that it is not topologically $\mathscr{V}$-equisingular.
\end{example}

\begin{remark}
By \cite[Corollary 3.7]{ER1}, we know that if $\{f_t\}$ is a non-equimultiple family of line singularities of the form $f_t(\mathbf{z})=f_0(\mathbf{z})+\xi(t)g(\mathbf{z})$, where $\xi\colon (\mathbb{C},0)\to (\mathbb{C},0)$ is a non-constant polynomial function and $g\colon (\mathbb{C}^n,0)\to (\mathbb{C},0)$ is any polynomial function, then $\{f_t\}$ is not topologically $\mathscr{V}$-equisingular. The above example is not a consequence of this result.
\end{remark}

\section{Proof of Theorem \ref{mt2}}\label{prooftheorem}

We first prove the theorem under the condition (\ref{c1}).
Pick $t_0\not=0$ small enough so that $\mbox{mult}_{\mathbf{0}}(f_{t})=\mbox{mult}_{\mathbf{0}}(f_{t_0})$ for all $t\not=0$ with $\vert t\vert\leq\vert t_0\vert$. Take a new variable $z_{n+1}$ disjoint from $z_1,\ldots,z_n$, and write $
\tilde{\mathbf{z}}:=(\mathbf{z},z_{n+1}):=
(z_1,\ldots,z_n,z_{n+1})$. 
Then consider the deformation 
\begin{equation*}
\tilde f\colon (\mathbb{C}\times \mathbb{C}^{n+1},\mathbb{C}\times \{\tilde{\mathbf{0}}\})\to(\mathbb{C},0)
\end{equation*}
defined by
\begin{equation*}
\tilde f(t,\tilde{\mathbf{z}}):=f(t,\mathbf{z})+z_{n+1}^{\mbox{\tiny mult}_{\mathbf{0}}(f_{0})}+tz_{n+1}^{\mbox{\tiny mult}_{\mathbf{0}}(f_{t_0})}.
\end{equation*}
Note that the family $\{f_t\}$ is equimultiple if and only if $\mbox{ord}_{\tilde{\mathbf{0}}}(\tilde f_{t})$ is independent of $t$ for all $t$ small enough. Now to show that $\mbox{ord}_{\tilde{\mathbf{0}}}(\tilde f_{t})$ is constant, we proceed as follows. Consider the system of weights $\tilde{\mathbf{w}}:=(\tilde w_1,\ldots,\tilde w_n,\tilde w_{n+1})$---on the variables $\tilde{\mathbf{z}}=(z_1,\ldots,z_n,z_{n+1})$---defined by
\begin{equation*}
\tilde w_i:=\left\{
\begin{aligned}
w_i \cdot \mbox{mult}_{\mathbf{0}}(f_{0}) &\ \mbox{ if }\ 1\leq i\leq n,\\
d_{\mathbf{w}} &\ \mbox{ if }\ i=n+1,
\end{aligned}
\right.
\end{equation*}
where $d_{\mathbf{w}}$ is the weighted degree of the weighted homogeneous polynomial $f_0$ with respect to the old weights $\mathbf{w}:=(w_1,\ldots,w_n)$. Clearly, $\tilde f_0$ is weighted homogeneous with respect to $\tilde{\mathbf{w}}$ and its weighted degree with respect to these weights is $d_{\tilde{\mathbf{w}}}:=d_{\mathbf{w}}\, \mbox{mult}_{\mathbf{0}}(f_{0})$. 
Moreover, for every $t$ there exists a neighbourhood (depending on $t$) of the origin $\tilde{\mathbf{0}}\in\mathbb{C}^{n+1}$ in which the following equalities hold:
\begin{enumerate}
\item
$\Sigma \tilde f_t=\Sigma f_t\times\{0\}$ (i.e., in a neighbourhood of the origin, the singular set $\Sigma \tilde f_t$ is given by the $z_1$-axis of $\mathbb{C}^{n+1}$); \footnote{Note that $\mbox{mult}_{\mathbf{0}}(f_{t})\geq 2$ for all small $t$. Indeed, if $\mbox{mult}_{\mathbf{0}}(f_{t})=1$, then $V(f_{t})$ would be smooth at the origin, which is not the case.} 
\item
$\Sigma (\tilde f_{t\mid V(z_1)})=\Sigma (f_{t\mid V(z_1)})\times \{0\}=\{\mathbf{\tilde 0}\}$ (i.e., $\tilde f_{t\mid V(z_1)}$ has an isolated singularity at the origin $\tilde{\mathbf{0}}\in\mathbb{C}^{n+1}$);
\item
$\Gamma^i_{\tilde f_t,\tilde{\mathbf{z}}} = 
\Gamma^i_{f_t,\mathbf{z}} \times \{0\}$ (for $i=1,2$), 
where $\Gamma^i_{\tilde f_t,\tilde{\mathbf{z}}}$ (respectively, $\Gamma^i_{f_t,\mathbf{z}}$) is the relative $i$-th polar variety of $\tilde f_t$ with respect to $\tilde{\mathbf{z}}$ (respectively, of $f_t$ with respect to $\mathbf{z}$).\footnote{For the definition, see D. Massey's book \cite{M} or the appendix below.}
\end{enumerate}
From the third assertion, we get:
\begin{equation*}
\qquad \gamma^1_{\tilde f_t,\tilde{\mathbf{z}}}(\tilde{\mathbf{0}}) = 
\gamma^1_{f_t,{\mathbf{z}}}(\mathbf{0})
\mbox{ and }
\lambda^1_{\tilde f_t,\tilde{\mathbf{z}}}(\tilde{\mathbf{0}}) = 
\lambda^1_{{f_t},\mathbf{z}}(\mathbf{0}).
\end{equation*}
Combined with the assumption (\ref{c1}), it follows that for all $t$ small enough, the numbers $\gamma^1_{\tilde f_t,\tilde{\mathbf{z}}}(\tilde{\mathbf{0}})$ and $\lambda^1_{\tilde f_t,\tilde{\mathbf{z}}}(\tilde{\mathbf{0}})$ are independent of $t$. Now, by \cite[Proposition 1.21]{M}, if $\tilde{\mathbf{z}}':=(z_2,\ldots,z_n,z_{n+1})$, then 
\begin{displaymath}
\lambda^0_{\tilde f_{t\mid V(z_1)},\tilde{\mathbf{z}}'} (\tilde{\mathbf{0}}') =
\lambda^1_{\tilde f_t,\tilde{\mathbf{z}}} (\tilde{\mathbf{0}}) + \gamma^1_{\tilde f_t,\tilde{\mathbf{z}}} (\tilde{\mathbf{0}}).
\end{displaymath}
It follows that this number is independent of $t$ for all small $t$. Since $\tilde f_{t\mid V(z_1)}$ has an isolated singularity at the origin, it is reduced at $\tilde{\mathbf{0}}'$ and the $0$-th L\^e number $\lambda^0_{\tilde f_{t\mid V(z_1)},\tilde{\mathbf{z}}'} (\tilde{\mathbf{0}}')$ is nothing but the Milnor number $\mu_{\tilde f_{t\mid V(z_1)}} (\tilde{\mathbf{0}}')$ of $\tilde f_{t\mid V(z_1)}$ at $\tilde{\mathbf{0}}'$ (see \cite[Example 2.1]{M}). In other words, the family $\{\tilde f_{t\mid V(z_1)}\}$ is a $\mu$-constant deformation of the weighted homogeneous polynomial function $\tilde f_{0\mid V(z_1)}$. Therefore, by the Greuel-O'Shea theorem (see Theorem \ref{thm-GOSh} above), $\{\tilde f_{t\mid V(z_1)}\}$ is equimultiple. 

Now, as the $z_{n+1}$-axis is not contained in the tangent cone of $V(\tilde f_t)$ at $\tilde{\mathbf{0}}$ for all small $t$, the order of $\tilde f_t$ at $\tilde{\mathbf{0}}$ is equal to the order at $\tilde{\mathbf{0}}$ of the restriction $\tilde f_{t\mid \mbox{\tiny $z_{n+1}$-axis}}$, and $\tilde f_t(0,\ldots,0,z_{n+1})\not=0$ for any $z_{n+1}\not=0$ close enough to $0$. Therefore, by the Weierstrass preparation theorem, in a neighbourhood of the origin, $\tilde f_t(\tilde{\mathbf{z}})$ can be written as a product
\begin{displaymath}
\tilde f_t(\tilde{\mathbf{z}})=u_t(\tilde{\mathbf{z}})\, v_t(\tilde{\mathbf{z}}),
\end{displaymath} 
where $u_t(\tilde{\mathbf{z}})$ and $v_t(\tilde{\mathbf{z}})$ are polynomial functions such that
$v_t(\tilde{\mathbf{z}})\not=0$ for all $\tilde{\mathbf{z}}$ near $\tilde{\mathbf{0}}$ (i.e., there is a constant $c_t\not=0$ such that $v_t(\tilde{\mathbf{z}})=c_t+\mbox{higher-order terms}$), and 
\begin{displaymath}
u_t(\tilde{\mathbf{z}}) = z_{n+1}^{\mbox{\tiny ord}_{\tilde{\mathbf{0}}}(\tilde f_t)} + \sum_{1\leq i\leq \mbox{\tiny ord}_{\tilde{\mathbf{0}}}(\tilde f_t)}
z_{n+1}^{\mbox{\tiny ord}_{\tilde{\mathbf{0}}}(\tilde f_t)-i} u_{t,i}(z_1,\ldots,z_{n}),
\end{displaymath}
where $u_{t,i}(z_1,\ldots,z_{n})=u_{t,i}(\mathbf{z})$ is a polynomial function such that $u_{t,i}(\mathbf{0})=0$ and $\mbox{ord}_{\mathbf{0}}(u_{t,i})\geq i$. 
This shows that
\begin{equation}\label{equord}
\mbox{ord}_{\tilde{\mathbf{0}}}(\tilde f_{t\mid V(z_1)}) = 
\mbox{ord}_{\tilde{\mathbf{0}}}(\tilde f_t).
\end{equation}
As $\{\tilde f_{t\mid V(z_1)}\}$ is equimultiple, it follows from (\ref{equord}) that $\mbox{ord}_{\tilde{\mathbf{0}}}(\tilde f_t)$ is constant (as $t$ varies), and as mentioned above, this is equivalent to the equimultiplicity of the family $\{f_t\}$. This completes the proof of the theorem under the assumption (\ref{c1}).

To prove that the family $\{f_t\}$ is also equimultiple under the condition (\ref{c2}), it suffices to show that (\ref{c2}) implies (\ref{c1}). By \cite[Corollary 2.4]{M7}, if (\ref{c2}) holds, then, for any integer $j$ sufficiently large, the Milnor numbers $\mu_{f_t+z_1^j}(\mathbf{0})$ and $\mu_{f_{t\mid V(z_1)}}(\mathbf{0})$ are both independent of $t$ for all $t$ sufficiently small. As explained in \cite{M7}, this comes from the uniform Iomdine-L\^e-Massey formulas (see Pro\-position 2.1 and the relation (2.2) in \cite{M7} and Theorem 4.15 in \cite{M}) and the upper-semicontinuity of the Milnor number. Indeed, the Iomdine-L\^e-Massey formulas say that for all $t$ sufficiently small and all $j$ sufficiently large, $f_t+z_1^j$ has an isolated singularity at the origin, and 
\begin{align}
&\label{c3} \mu_{f_t+z_1^j}(\mathbf{0}) = \lambda^0_{f_t,\mathbf{z}} (\mathbf{0}) + (j-1) \lambda^1_{f_t,\mathbf{z}} (\mathbf{0}),\\
&\label{c4} \mu_{f_t+z_1^j}(\mathbf{0}) + \mu_{f_{t\mid V(z_1)}}(\mathbf{0}) =(\gamma^1_{f_t,\mathbf{z}} (\mathbf{0}) + \lambda^0_{f_t,\mathbf{z}} (\mathbf{0})) + j \lambda^1_{f_t,\mathbf{z}} (\mathbf{0}).
\end{align}
Thus, if (\ref{c2}) holds, the sum $\mu_{f_t+z_1^j}(\mathbf{0}) + \mu_{f_{t\mid V(z_1)}}(\mathbf{0})$ is independent of $t$, and hence, by the upper-semicontinuity of the Milnor number, both $\mu_{f_t+z_1^j}(\mathbf{0})$ and $\mu_{f_{t\mid V(z_1)}}(\mathbf{0})$ must be independent of $t$.  
Then, (\ref{c3}) implies that $\lambda^0_{f_t,\mathbf{z}} (\mathbf{0})$ does not depend on $t$, and hence, by (\ref{c4}), $\gamma^1_{f_t,\mathbf{z}} (\mathbf{0})$ is independent of $t$ too.

\appendix

\section{L\^e numbers and polar numbers}\label{appendix}

The L\^e numbers generalize to non-isolated hypersurface singularities the data given by the Milnor number for an isolated singularity. They were introduced about 25 years ago by D. Massey \cite{M3,M4,M,M2}. For the convenience of the reader, we briefly recall the definitions in this appendix. We follow the presentation given in Massey's book \cite{M}.

Throughout, we use the following notation. 

\begin{notation}
Let $(X,\mathscr{O}_X)$ be a complex analytic space, and let $\mathscr{I}$ be a coherent sheaf of ideals in $\mathscr{O}_X$. We denote by $V(\mathscr{I})$ the complex analytic space defined by the vanishing of $\mathscr{I}$. If $W\subseteq X$ is any analytic subset of $X$, then we denote by $\mathscr{I}\lnot W$ the gap sheaf associated to $\mathscr{I}$ and $W$. As usual, the scheme $V(\mathscr{I}\lnot W)$ defined by the vanishing of the gap sheaf will be also denoted by $V(\mathscr{I})\lnot W$. Finally, we write $[X]$ for the analytic cycle associated to $X$---that is, the formal sum $\sum m_V [V]$, where the $V$'s run over all the irreducible components of $X$ and where $m_V$ represents the number of times the component $V$ should be counted.
\end{notation}

Consider a holomorphic function $h\colon (U,\mathbf{0})\rightarrow(\mathbb{C},0)$, where $U$ is an open neighbourhood of $\mathbf{0}$ in $\mathbb{C}^n$, and fix a system of linear coordinates $\mathbf{z}=(z_1,\ldots,z_n)$ for $\mathbb{C}^n$. As usual, write $\Sigma h$ for the critical locus of $h$. For $0\leq i\leq n-1$, the relative $i$-th  \emph{polar variety} of $h$ with respect to the coordinates system $\mathbf{z}$ is the scheme
\begin{equation*}
\Gamma_{h,\mathbf{z}}^i:=V\bigg(\frac{\partial h}{\partial z_{i+1}},\ldots,\frac{\partial h}{\partial z_{n}}\bigg) \lnot \Sigma h.
\end{equation*}
The $i$-th \emph{L\^e cycle} of $h$ with respect to $\mathbf{z}$ is the analytic cycle
\begin{equation*}
[\Lambda^i_{h,\mathbf{z}}]:=\bigg[\Gamma_{h,\mathbf{z}}^{i+1}\cap V\bigg(\frac{\partial h}{\partial z_{i+1}}\bigg)\bigg] - \bigg[\Gamma_{h,\mathbf{z}}^i\bigg].
\end{equation*}

\begin{definition}
The $i$-th \emph{L\^e number} $\lambda^i_{h,\mathbf{z}}(\mathbf{p})$ of $h$ at $\mathbf{p}=(p_1,\ldots,p_n)$ with respect to the coordinates system $\mathbf{z}$ is defined to be the intersection number
\begin{equation}\label{dln}
\lambda^i_{h,\mathbf{z}}(\mathbf{p}):=\big([\Lambda^i_{h,\mathbf{z}}]\cdot [V(z_1-p_1,\ldots,z_i-p_i)]\big)_\mathbf{p}
\end{equation}
provided that this intersection is $0$-dimensional or empty at $\mathbf{p}$; otherwise, $\lambda^i_{h,\mathbf{z}}(\mathbf{p})$ is \emph{undefined}.
\end{definition}

For $i=0$, the relation (\ref{dln}) means
\begin{equation*}
\qquad
\lambda^0_{h,\mathbf{z}}(\mathbf{p}) = \big([\Lambda^0_{h,\mathbf{z}}]\cdot U \big)_\mathbf{p} = \bigg[\Gamma_{h,\mathbf{z}}^{1}\cap V\bigg(\frac{\partial h}{\partial z_{1}}\bigg)\bigg]_\mathbf{p}.
\end{equation*}
For any $i$, with $\dim_\mathbf{p}\Sigma h < i\leq n-1$, the corresponding L\^e number $\lambda^i_{h,\mathbf{z}}(\mathbf{p})$ always exists and is equal to zero. 
Note that if $\mathbf{p}$ is an isolated singularity of $h$, then the $0$-th L\^e number $\lambda^0_{h,\mathbf{z}}(\mathbf{p})$ (which is the only possible non-zero L\^e number) is equal to the Milnor number $\mu_h(\mathbf{0})$ of $h$ at~$\mathbf{p}$.

\begin{definition}
The $i$-th \emph{polar number} $\gamma^i_{h,\mathbf{z}}(\mathbf{p})$ of $h$ at $\mathbf{p}=(p_1,\ldots,p_n)$ with respect to the coordinates system $\mathbf{z}$ is defined to be the intersection number
\begin{equation*}
\gamma^i_{h,\mathbf{z}}(\mathbf{p}):=\big([\Gamma^i_{h,\mathbf{z}}]\cdot [V(z_1-p_1,\ldots,z_i-p_i)]\big)_\mathbf{p}
\end{equation*}
provided that this intersection is $0$-dimensional or empty at $\mathbf{p}$; otherwise, $\gamma^i_{h,\mathbf{z}}(\mathbf{p})$ is \emph{undefined}.
\end{definition}

Note that $\gamma^0_{h,\mathbf{z}}(\mathbf{p})$ is always defined and equal to zero. 

\begin{remark}
For a generic choice of coordinates $\mathbf{z}$, for any point $\mathbf{p}\in V(h)$ near~$\mathbf{0}$ and for any $0\leq i\leq \dim_{\mathbf{p}}\Sigma h$, the L\^e number $\lambda^i_{h,\mathbf{z}}(\mathbf{p})$ and the polar number $\gamma^i_{h,\mathbf{z}}(\mathbf{p})$ do exist (cf.~\cite[Proposition 10.2 and Theorem 1.28]{M}).
\end{remark}

\bibliographystyle{amsplain}
\begin{thebibliography}{10}

\bibitem{CB-H-R} R. Callejas-Bedregal, K. Houston and M. A. S. Ruas, ``Topological triviality of families of singular surfaces,''  arXiv:math/0611699v1 [math.CV] 22 Nov 2006.

\bibitem {E2} C. Eyral, ``Zariski's multiplicity question---a survey,'' New Zealand J. Math. \textbf{36} (2007) 253--276.

\bibitem {EyBook} C. Eyral, ``Topics in equisingularity theory,'' IMPAN Lecture Notes, Polish Academy of Sciences, Institute of Mathematics (to appear).

\bibitem {Ey2} C. Eyral, ``Topologically equisingular families of homogeneous hypersurfaces with line singularities are equimultiple'' (to appear); available at arXiv:1506.07996v1 [math.AG] 26 Jun 2015.

\bibitem {ER1} C. Eyral and M. A. S. Ruas, ``Deformations with constant L\^e numbers and multiplicity of nonisolated hypersurface singularities,''  Nagoya Math. J. \textbf{218} (2015) 29--50.

\bibitem{BP} J. Fern\'andez de Bobadilla and M. Pe Pereira, ``Equisingularity at the normalisation,'' J. Topol. \textbf{1} (2008), no. 4, 879--909. 

\bibitem{Fulton} W. Fulton, ``Intersection Theory,'' Ergebnisse der Mathematik und ihrer Grenzgebiete (3), \textbf{2}, Springer-Verlag, Berlin, 1984.

\bibitem{GK} A. M. Gabri\`elov and A. G. Ku\v{s}nirenko, ``Description of deformations with constant Milnor number for homogeneous functions,'' Funkcional. Anal. i Prilo\v{z}en \textbf{9} (1975), no. 4, 67--68 (Rusian). English translation: Functional Anal. Appl. \textbf{9} (1975), no. 4, 329--331.

\bibitem{Gaffney} T. Gaffney, ``Polar multiplicities and equisingularity of map germs,'' Topology \textbf{32} (1993), no. 1, 185--223.

\bibitem{GG} T. Gaffney and R. Gassler, ``Segre numbers and hypersurface singularities,'' J. Algebraic Geom. \textbf{8} (1999), no. 4, 695--736.

\bibitem{G} G.-M. Greuel, ``Constant Milnor number implies
constant multiplicity for quasihomogeneous singularities,''
Manuscripta Math.~\textbf{56} (1986) 159--166.

\bibitem{HL} H. A. Hamm and L\^e D\~ung Tr\'ang, ``Un th\'eor\`eme de Zariski
du type de Lefschetz,'' Ann. Sci. \'Ecole Norm. Sup. \textbf{6} (1973)
317--366. 

\bibitem{H} H. Hironaka, ``Normal cones in analytic Whitney
stratifications,'' Inst. Hautes \'Etudes Sci. Publ. Math. \textbf{36}
(1969) 127--138.

\bibitem{L3} L\^e D\~ung Tr\'ang, ``Topologie des singularit\'es des hypersurfaces complexes,'' Singularit\'es \`a Carg\`ese (Rencontre Singularit\'es G\'eom. Anal., Inst. \'Etudes Sci., Carg\`ese, 1972), pp. 171--182, Ast\'erisque
\textbf{7} \& \textbf{8}, Soc. Math. France, Paris, 1973.

\bibitem{L4} L\^e D\~ung Tr\'ang, ``Ensembles analytiques complexes avec lieu singulier de dimension un (d'apr\`es I. N. Iomdine),'' Seminar on Singularities (Paris, 1976/77), pp. 87--95, Publ. Math. Univ. Paris VII, 7, Paris, 1980.

\bibitem{MNP} W. L. Marar, J. J. Nu\~{n}o-Ballesteros, G. Pe\~{n}afort-Sanchis, ``Double point curves for corank 2 map germs from $\mathbb C^2$  to $\mathbb C^3$,'' Topology Appl. \textbf{159}  (2012),  no. 2, 526--536. 

\bibitem {M5} D. Massey, ``A reduction theorem for the Zariski
multiplicity conjecture,'' Proc. Amer. Math. Soc. \textbf{106} (1989), no. 2,
379--383.

\bibitem{M7} D. Massey, ``The L\^e-Ramanujam problem for hypersurfaces with one-dimensional singular sets,'' Math. Ann. \textbf{282} (1988), no. 1, 33-49.

\bibitem {M3} D. Massey, ``The L\^e varieties, I,'' Invent. Math.~\textbf{99} (1990), no. 2, 357--376.

\bibitem {M4} D. Massey, ``The L\^e varieties, II,'' Invent. Math.~\textbf{104} (1991), no. 1, 113--148.

\bibitem {M} D. Massey, ``L\^e cycles and hypersurface singularities,'' Lecture Notes Math.~\textbf{1615}, Springer-Verlag, Berlin, 1995.

\bibitem{M2} D. Massey, ``Numerical control over complex analytic singularities,'' Mem. Amer. Math. Soc. \textbf{163} (2003), no. 778.  

\bibitem{Mi} J. Milnor, ``Singular points of complex hypersurfaces,'' Annals of Math. Studies \textbf{61}, Princeton Univ. Press, Princeton, N. J., Univ. Tokyo Press, Tokyo, 1968.

\bibitem{Mond} D. Mond, ``On the classification of germs of maps from $\mathbb{R}^2$ to $\mathbb{R}^3$,'' Proc. London Math. Soc. (3) \textbf{50} (1985), no. 2, 333--369. 

\bibitem{O'Sh} D. O'Shea, ``Topologically trivial deformations
of isolated quasihomogeneous hypersurface singularities are
equimultiple,'' Proc. Amer. Math. Soc. \textbf{101} (1987) 260--262.

\bibitem {R} M. A. S. Ruas, ``Equimultiplicity of topologically equisingular families of parametrized surfaces in $\mathbb{C}^3$,'' ArXiv:1302.5800v1 [math.CV] 23 Feb 2013.

\bibitem{T1} B. Teissier, ``Cycles \'evanescents, sections planes et conditions de Whitney,'' Singularit\'es \`a Carg\`ese (Rencontre Singularit\'es G\'eom. Anal., Inst. \'Etudes Sci., Carg\`ese, 1972), pp. 285--362, Ast\'erisque
\textbf{7} \& \textbf{8}, Soc. Math. France, Paris, 1973.

\bibitem{T2} B. Teissier, ``D\'eformations \`a type topologique constant,'' Quelques probl\`emes de modules (S\'em. de G\'eom\'etrie Analytique, \'Ecole Norm. Sup., Paris, 1971/72), pp. 215--249, Ast\'erisque \textbf{16}, Soc. Math. France, Paris, 1974.

\bibitem{T3} B. Teissier, ``Introduction to equisingularity problem,'' Algebraic geometry (Proc. Sympos. Pure Math. 29, Humboldt State Univ., Arcata, Calif., 1974), pp. 593--632, Amer. Math. Soc., Providence, R.I., 1975.

\bibitem{V} A. N. Varchenko, ``A lower bound for the codimension of the $\mu=\mbox{const}$ stratum in terms of the mixed Hodge structure,'' Vestnik Moskov. Univ. Ser. I Mat. Mekh. \textbf{120} (1982), no. 6, 28--31 (Russian). English translation: Moscow Univ. Math. Bull. \textbf{37} (1982), no. 6, 30--33.

\bibitem {Wall} C. T. C. Wall, ``Finite determinacy of smooth map-germs,'' Bull. London Math. Soc. \textbf{13}  (1981), no. 6, 481--539.

\bibitem {Z} O. Zariski, ``Some open questions in the theory of
singularities,'' Bull. Amer. Math. Soc. \textbf{77} (1971) 481--491.

\end{thebibliography}

\end{document}

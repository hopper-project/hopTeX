\documentclass[11pt]{amsart}

\usepackage{graphicx}
\usepackage{mathptmx}

\usepackage{amscd}
\usepackage{amsthm}
\usepackage{amsxtra}
\usepackage{a4wide}
\usepackage{latexsym}
\usepackage{amssymb}
\usepackage{amsfonts}
\usepackage{amsmath}
\usepackage{amsrefs}
\usepackage{mathrsfs}

\usepackage{upref}
\usepackage{txfonts}
\usepackage{graphicx}

\usepackage[bookmarksnumbered, colorlinks, plainpages]{hyperref}
\hypersetup{colorlinks=true,linkcolor=red, anchorcolor=green, citecolor=cyan, urlcolor=red, filecolor=magenta, pdftoolbar=true}

\usepackage[left=2cm, right=2cm, top=2cm, bottom=2cm]{geometry}

\allowdisplaybreaks

\theoremstyle{plain}
\newtheorem{thm}{Theorem}[section]
\newtheorem{lem}[thm]{Lemma}
\newtheorem{cor}[thm]{Corollary}
\newtheorem{prop}[thm]{Proposition}
\theoremstyle{definition}
\newtheorem{rem}[thm]{Remark}
\newtheorem{rems}[thm]{Remarks}
\newtheorem{defi}[thm]{Definition}
\newtheorem{ex}[thm]{Example}
\newtheorem{exs}[thm]{Examples}
\newtheorem{conv}[thm]{Convention}

\numberwithin{thm}{section}
\numberwithin{equation}{section}

\begin{document}

\title[Embeddings between ${{\,^{^{\bf c}}\!}{LM_{p{\theta},{\omega}}({{\mathbb R}^n},v)}}$ and ${LM_{p{\theta},{\omega}}({{\mathbb R}^n},v)}$]{Embeddings between weighted complementary local Morrey-type spaces and weighted local Morrey-type spaces}

\author[A. Gogatishvili]{Amiran Gogatishvili}
\address{Institute of Mathematics \\
    Academy of Sciences of the Czech Republic \\
    \v Zitn\'a~25 \\
    115~67 Praha~1, Czech Republic} \email{gogatish@math.cas.cz}

\author[R.Ch.Mustafayev]{Rza Mustafayev}
\address{Department of Mathematics \\ Faculty of Science and Arts \\ Kirikkale
    University \\ 71450 Yahsihan, Kirikkale, Turkey}
\email{rzamustafayev@gmail.com}

\author[T.~{\"U}nver]{Tu\v{g}\c{c}e {\"U}nver}
\address{Department of Mathematics \\ Faculty of Science and Arts \\ Kirikkale
    University \\ 71450 Yahsihan, Kirikkale, Turkey}
\email{tugceunver@gmail.com}

\thanks{}

\subjclass[2010]{Primary 46E30; Secondary 26D10.}

\keywords{local Morrey-type spaces, embeddings, iterated Hardy inequalities}

\begin{abstract}
	In this paper embeddings between  weighted complementary local Morrey-type spaces ${\,^{^{\bf c}}\!}LM_{p\theta,\omega}({\mathbb R}^n,v)$ and  weighted local Morrey-type spaces $LM_{p\theta,\omega}({\mathbb R}^n,v)$ are characterized. In particular, two-sided estimates of the optimal constant $c$ in the inequality
	\begin{equation*}
	\bigg( \int_0^{\infty} \bigg( \int_{B(0,t)} f(x)^{p_2}v_2(x)\,dx \bigg)^{\frac{q_2}{p_2}} u_2(t)\,dt\bigg)^{\frac{1}{q_2}} \le c \bigg( \int_0^{\infty} \bigg( \int_{{\,^{^{\bf c}}\!}B(0,t)} f(x)^{p_1} v_1(x)\,dx\bigg)^{\frac{q_1}{p_1}} u_1(t)\,dt\bigg)^{\frac{1}{q_1}}
	\end{equation*}
	are obtained, where $p_1,\,p_2,\,q_1,\,q_2 \in (0,\infty)$, $p_2 \le q_2$ and $u_1,\,u_2$ and $v_1,\,v_2$ are weights on $(0,\infty)$ and ${\mathbb R}^n$, respectively. The proof is based on the combination of duality techniques with estimates of optimal constants of the embeddings between	weighted local Morrey-type and complementary local Morrey-type spaces and weighted Lebesgue spaces, which reduce the problem to the solutions of the iterated Hardy-type inequalities.
\end{abstract}

\maketitle

\section{Introduction}\label{introduction}

The classical Morrey spaces $\mathcal{M}_{p, \lambda} \equiv \mathcal{M}_{p, \lambda} ({{\mathbb R}^n})$,
were  introduced by C.~Morrey in \cite{M1938} in order to study regularity questions which appear in the Calculus of Variations,
and defined as follows:  for $0 \le \lambda \le n$ and $1\le p \le \infty$,
$$
\mathcal{M}_{p,\lambda} : = \left\{ f \in   {L_p^{\rm loc}({{\mathbb R}^n})}:\,\left\| f\right\|_{\mathcal{M}_{p,\lambda }} : =
\sup_{x\in {{\mathbb R}^n}, \; r>0 }
r^{\frac{\lambda-n}{p}} \|f\|_{L_{p}(B(x,r))} <\infty\right\},
$$
where ${{B(x,r)}}$ is the open ball centered at $x$ of radius $r$.

Note that $\mathcal{M}_{p,0}({{\mathbb R}^n}) = L_{\infty}({{\mathbb R}^n})$ and ${\mathcal M}_{p,n}({{\mathbb R}^n}) = L_{p}({{\mathbb R}^n})$.

These spaces describe local regularity more precisely than Lebesgue spaces and appeared to be quite useful in the study of the local
behavior of solutions to partial differential equations, a priori estimates and other topics in PDE (cf. \cite{giltrud}).    

The classical Morrey spaces were widely investigated during the last decades, including the study of classical operators of Harmonic and Real Analysis - maximal, singular and potential operators - in generalizations of these spaces (so-called Morrey-type spaces). The local Morrey-type spaces and the complementary local Morrey-type spaces introduced by Guliyev in his doctoral thesis \cite{GulDoc}. 

The local Morrey-type spaces ${LM_{p{\theta},{\omega}}}$ and the complementary local Morrey-type spaces ${\,^{^{\bf c}}\!} {LM_{p{\theta},{\omega}}}$ were intensively studied during the last decades. The research mainly includes the study of the boundedness of classical operators in these spaces (see, for instance, \cite{Bur1,Bur2,BurGol,BurHus1,BurGulHus2,BurGulHus1,BurGulSerTar,BGGM,BGGM1}), and the
investigation of the functional-analytic properties of them and relation of these spaces with other known
function spaces (see, for instance, \cites{BurNur,batsaw,mu_emb}). We refer the reader to the surveys \cite{Bur1}
and \cite{Bur2} for a comprehensive discussion of the history of ${LM_{p{\theta},{\omega}}}$ and ${\,^{^{\bf c}}\!} {LM_{p{\theta},{\omega}}}$. 

Let $A$ be any measurable subset of ${{\mathbb R}^n}$, $n \ge 1$. By ${{\mathfrak M}} (A)$
we denote the set of all measurable functions on $A$. The symbol
${{\mathfrak M}}^+ (A)$ stands for the collection of all $f\in{{\mathfrak M}} (A)$ which
are non-negative on $A$. The family of all weight functions (also
called just weights) on $A$, that is, measurable, positive and
finite a.e. on $A$, is given by ${{\mathcal W}} (A)$.

For $p\in (0,{\infty}]$, we define the functional
$\|\cdot\|_{p,A}$ on ${{\mathfrak M}} (A)$ by
\begin{equation*}
\|f\|_{p,A} : = \bigg\{\begin{array}{cl}
\left(\int_A |f(x)|^p \,dx \right)^{1/p} & {\qquad}\mbox{if}{\qquad} p<{\infty} \\
{\operatornamewithlimits{ess\,sup}}_{A} |f(x)| & {\qquad}\mbox{if}{\qquad} p={\infty}
\end{array}
\bigg..
\end{equation*}

If $w\in {{\mathcal W}}(A)$, then the weighted Lebesgue space
$L_p(w,A)$ is given by
\begin{equation*}
L_{p}(w,A) \equiv L_{p,w}(A) : = \{f\in {{\mathfrak M}} (A):\,\, \|f\|_{p,w,A} := \|fw\|_{p,A} < \infty \}.
\end{equation*}
When $A={{\mathbb R}^n}$, we often write simply $L_{p,w}$ and $L_p(w)$ instead of $L_{p,w}(A)$ and $L_p(w,A)$, respectively. 

Throughout the paper, we always denote by $c$ and $C$ a positive
constant, which is independent of main parameters but it may vary
from line to line. However a constant with subscript
such as $c_1$ does not change in different occurrences. By
$a\lesssim b$, ($b\gtrsim a$) we mean that $a\leq {\lambda} b$, where
${\lambda}>0$ depends on inessential parameters. If $a\lesssim b$ and
$b\lesssim a$, we write $a\approx b$ and say that $a$ and $b$ are
equivalent.  We will denote by $\bf 1$ the function ${\bf 1}(x) =
1$, $x \in {\mathbb R}$.

Given two quasi-normed vector spaces $X$ and $Y$, we write $X=Y$ if
$X$ and $Y$ are equal in the algebraic and the topological sense
(their quasi-norms are equivalent). The symbol $X\hookrightarrow Y$
($Y \hookleftarrow X$) means that $X\subset Y$ and the natural
embedding ${\operatorname{I}}$ of $X$ in $Y$ is continuous, that is, there exist a
constant $c > 0$ such that $\|z\|_Y \le c\|z\|_X$ for all $z\in X$.
The best constant of the embedding $X\hookrightarrow Y$ is
$\|{\operatorname{I}}\|_{X {\rightarrow} Y}$.

The weighted local Morrey-type spaces ${LM_{p{\theta},{\omega}}({{\mathbb R}^n},v)}$ and weighted complementary
local Morrey-type spaces ${LM_{p{\theta},{\omega}}({{\mathbb R}^n},v)}$ are defined as follows: Let  $0 <p, \theta \le \infty$. 
Assume that ${\omega} \in {{\mathfrak M}}^+ {(0,{\infty})}$ and $v \in{{\mathcal W}}({{\mathbb R}^n})$.      
$$
{LM_{p{\theta},{\omega}}({{\mathbb R}^n},v)} : = \bigg\{f\in L_{p,v}^{\operatorname{loc}}({{\mathbb R}^n}):\, \| f \|_{LM_{p{\theta},{\omega}}({{\mathbb R}^n},v)} < \infty\bigg\}, \\
$$
where
$$
\| f \|_{LM_{p{\theta},{\omega}}({{\mathbb R}^n},v)} : = \big\| \|f\|_{p,v,{{B(0,r)}}} \big\|_{{\theta},{\omega},(0,\infty)},
$$
and
$$
{{\,^{^{\bf c}}\!}{LM_{p{\theta},{\omega}}({{\mathbb R}^n},v)}} :  = \bigg\{f\in {\mathbin{\scalebox{1.7}{\ensuremath{\cap}}}}_{t>0} L_{p,v}({\,^{^{\bf c}}\!} B(0,t)):\,\| f\|_{{\,^{^{\bf c}}\!}{LM_{p{\theta},{\omega}}({{\mathbb R}^n},v)}}  < \infty\bigg\},
$$
where
$$
\| f\|_{{\,^{^{\bf c}}\!}{LM_{p{\theta},{\omega}}({{\mathbb R}^n},v)}} : = \big\| \|f\|_{p,v,{{\,^{^{\bf c}}\!}{{B(0,r)}}}} \big\|_{{\theta},{\omega}, (0,\infty)}.
$$
\begin{rem}
	In \cite{BurHus1} and \cite{BurGulHus1} it were proved that the
	spaces ${LM_{p{\theta},{\omega}}} ({{\mathbb R}^n}) : = {LM_{p{\theta},{\omega}}} ({{\mathbb R}^n}, \bf 1)$ and ${{\,^{^{\bf c}}\!}{LM_{p{\theta},{\omega}}}} ({{\mathbb R}^n}) : = {{\,^{^{\bf c}}\!}{LM_{p{\theta},{\omega}}}}
	({{\mathbb R}^n}, \bf 1)$ are non-trivial, i.e. consists not only of functions
	equivalent to $0$ on ${{\mathbb R}^n}$, if and only if
	\begin{equation}\label{11111}
	\|{\omega}\|_{{\theta},(t,{\infty})} < {\infty}, {\qquad} \mbox{for some} {\qquad} t >0,
	\end{equation}
	and
	\begin{equation}\label{1111111}
	\|{\omega}\|_{{\theta},(0,t)} < {\infty},{\qquad} \mbox{for some} {\qquad} t >0,
	\end{equation}
	respectively. The same conclusion is true for ${LM_{p{\theta},{\omega}}({{\mathbb R}^n},v)}$ and ${{\,^{^{\bf c}}\!}{LM_{p{\theta},{\omega}}({{\mathbb R}^n},v)}}$ for
	any $v \in {{\mathcal W}}({{\mathbb R}^n})$.
\end{rem}

The proof of the following statement is straightforward.
\begin{lem} \label{shrinkagelemma}
	\rm{(i)} If $\|{\omega}\|_{{\theta},(t_1,\infty)} = \infty$ for some $t_1 > 0$, then
	$$
	f\in {LM_{p{\theta},{\omega}}({{\mathbb R}^n},v)} \Rightarrow f = 0 \quad \mbox{a.e. on} \quad B(0,t_1).
	$$
	
	\rm{(ii)} If $\|{\omega}\|_{{\theta},(0,t_2)} = \infty$ for some $t_2 > 0$, then
	$$
	f\in {{\,^{^{\bf c}}\!}{LM_{p{\theta},{\omega}}({{\mathbb R}^n},v)}} \Rightarrow f = 0 \quad \mbox{a.e. on} \quad {\,^{^{\bf c}}\!} B(0,t_2).
	$$
\end{lem}

Let $0< {\theta} \le {\infty}$. We denote by
\begin{align*}
{\Omega_{\theta}} : & = \big\{{\omega} \in {{\mathfrak M}}^+ {(0,{\infty})}:\, 0<\|{\omega}\|_{{\theta},(t,{\infty})}<{\infty},~ t>0 \big\}, \\
{{\,^{^{\bf c}}\!}{\Omega_{\theta}}} : & = \big\{{\omega} \in {{\mathfrak M}}^+ {(0,{\infty})}:\,0<\|{\omega}\|_{{\theta},(0,t)}<{\infty},~ t>0 \big\}.	
\end{align*}

Let $v \in {{\mathcal W}}({{\mathbb R}^n})$. It is easy to see that ${LM_{p{\theta},{\omega}}({{\mathbb R}^n},v)}$ and ${{\,^{^{\bf c}}\!}{LM_{p{\theta},{\omega}}({{\mathbb R}^n},v)}}$ are
quasi-normed vector spaces when ${\omega} \in {\Omega_{\theta}}$ and ${\omega} \in {{\,^{^{\bf c}}\!}{\Omega_{\theta}}}$,
respectively.

The following statements are immediate consequences of Fubini's Theorem and were observed in \cite{BurHus1} and \cite{BurGulHus1}, for $v = 1$, respectively. 
\begin{lem}\label{LMpp}
	Let $0 < p \leq \infty$ and $v\in {{\mathcal W}}({{\mathbb R}^n})$. Then 
	
	{\rm (i)} $LM_{pp,{\omega}}({{\mathbb R}^n},v)=L_p(w)$, where	$w(x) := v(x)\|{\omega}\|_{p,(|x|,\infty)}, ~ x \in {{\mathbb R}^n}$.
	
	{\rm (ii)} ${\,^{^{\bf c}}\!} LM_{pp,{\omega}}({{\mathbb R}^n},v)=L_p(w)$, where	$w(x) := v(x)\|{\omega}\|_{p,(0,|x|)}, ~ x \in {{\mathbb R}^n}$.
\end{lem}

Recall that the embeddings between weighted local Morrey-type spaces and weighted Lebesgue spaces, that
is, the embeddings 
\begin{align}
L_{p_1}(v_1) & {\hookrightarrow} LM_{p_2 {\theta}, {\omega}}({{\mathbb R}^n},v_2), \label{emb1}\\
L_{p_1}(v_1) & {\hookrightarrow} {\,^{^{\bf c}}\!}{LM}_{p_2 {\theta}, {\omega}}({{\mathbb R}^n},v_2), \label{emb2}\\
L_{p_1}(v_1) & \hookleftarrow LM_{p_2{\theta}, {\omega}}({{\mathbb R}^n},v_2), \label{emb3}\\
L_{p_1}(v_1) & \hookleftarrow {\,^{^{\bf c}}\!}{LM}_{p_2 {\theta}, {\omega}}({{\mathbb R}^n},v_2)
\label{emb4}
\end{align}
are completely characterized in \cite{mu_emb}.

Our principal goal in this paper is to investigate the embeddings
between weighted complementary local Morrey-type spaces and weighted local Morrey type spaces and vice versa, that is, the embeddings
\begin{align}
{\,^{^{\bf c}}\!} LM_{p_1 {\theta}_1, {\omega}_1}({{\mathbb R}^n},v_1) & {\hookrightarrow} LM_{p_2 {\theta}_2, {\omega}_2}({{\mathbb R}^n},v_2), \label{mainemb1}\\
LM_{p_1 {\theta}_1, {\omega}_1}({{\mathbb R}^n},v_1) & {\hookrightarrow} {\,^{^{\bf c}}\!} LM_{p_2 {\theta}_2, {\omega}_2}({{\mathbb R}^n},v_2).
\label{mainemb2}
\end{align}
An approach used in this paper consist of a duality argument combined with
estimates of optimal constants of embeddings \eqref{emb1} - \eqref{emb4},
which reduce the problem to the solutions of the iterated Hardy-type
inequalities  
\begin{equation}\label{eq.4.11}
\big\| \|H^* f\|_{p,u,(0,\cdot)}\big\|_{q,w,{(0,{\infty})}}\leq c
\,\|f\|_{\theta,v,{(0,{\infty})}},~f \in {\mathfrak M}^+{(0,{\infty})},
\end{equation}
with
$$
(H^*f)(t):= \int_t^{\infty} f(\tau) \, d\tau, \quad t > 0,
$$
where $u,\,v,\,w$ are weights on $(0,\infty)$ and $0 < p,\,q \le \infty$, $1 < \theta < \infty$. There exists different solutions of these inequalities. We will use  characterizations from  \cite{gmp} and \cite{gop}. 

Note that in view of Lemma \ref{LMpp}, embeddings \eqref{mainemb1} - \eqref{mainemb2} contain embeddings \eqref{emb1} - \eqref{emb4} as a special case. Moreover, by the change of
variables $x = {y} / {|y|^2}$ and $t={1} / {\tau}$, it is easy to see that \eqref{mainemb2} is equivalent to the embedding
$$
{\,^{^{\bf c}}\!} LM_{p_1{\theta}_1,\tilde {\omega}_1}({{\mathbb R}^n},\tilde{v}_1) {\hookrightarrow}
LM_{p_2{\theta}_2,\tilde {\omega}_2}({{\mathbb R}^n},\tilde{v}_2),
$$
where $\tilde{v}_i (y) = v_i(y/|y|^2)|y|^{-2n/p_i}$ and
$\tilde{\omega}_i (\tau) = \tau^{- {2} / {{\theta}_i}} {\omega}_i\big({1} / {\tau}\big)$,
$i=1,2$. This note allows us to concentrate our attention
on characterization of \eqref{mainemb1}. On the negative side of
things we have to admit that the duality approach works only in the
case when, in \eqref{mainemb1} - \eqref{mainemb2}, one has $p_2 \le
\theta_2$. Unfortunately, in the case when $p_2 > \theta_2$ the
characterization of these embeddings remains open.

In particular, we obtain two-sided estimates of the optimal constant
$c$ in the inequality
\begin{equation*}
\bigg( \int_0^{\infty} \bigg( \int_{{B(0,t)}} f(x)^{p_2}v_2(x)\,dx \bigg)^{\frac{q_2}{p_2}} u_2(t)\,dt\bigg)^{\frac{1}{q_2}} \le c \bigg( \int_0^{\infty} \bigg( \int_{{\,^{^{\bf c}}\!}{{B(0,t)}}} f(x)^{p_1} v_1(x)\,dx\bigg)^{\frac{q_1}{p_1}} u_1(t)\,dt\bigg)^{\frac{1}{q_1}},
\end{equation*}
where $p_1,\,p_2,\,q_1,\,q_2 \in (0,\infty)$, $p_2 \le q_2$ and $u_1,\,u_2$ and $v_1,\,v_2$ are weights on ${(0,{\infty})}$ and ${{\mathbb R}^n}$, respectively.

The paper is organized as follows. We start with formulations of our main results  in Section~\ref{main.res.}. The proofs of the main results are presented in Section \ref{s7}.

\section{Statement of the main results}\label{main.res.}

We adopt the following usual conventions.
\begin{conv}\label{Notat.and.prelim.conv.1.1}
	{\rm (i)} Throughout the paper we put $0/0 = 0$, $0 \cdot (\pm {\infty}) =
	0$ and $1 / (\pm{\infty}) =0$.
	
	{\rm (ii)} We put
	$$
	p' : = \left\{\begin{array}{cl} \frac p{1-p} & \text{if} \quad 0<p<1,\\
	\infty &\text{if}\quad p=1, \\
	\frac p{p-1}  &\text{if}\quad 1<p<\infty,\\
	1  &\text{if}\quad p=\infty.
	\end{array}
	\right.
	$$
	
	{\rm (iii)} To state our results we use the notation $p {\rightarrow} q$ for $0 < p,\,q
	\le \infty$ defined by
	$$
	\frac{1}{p {\rightarrow} q} = \frac{1}{q} - \frac{1}{p} {\qquad} \mbox{if} {\qquad} q <
	p,
	$$
	and $p {\rightarrow} q = \infty$ if $q \ge p$.
	
	{\rm (iv)} If $I = (a,b) \subseteq {\mathbb R}$ and $g$ is a monotone
	function on $I$, then by $g(a)$ and $g(b)$ we mean the limits
	$\lim_{t{\rightarrow} a+}g(t)$ and $\lim_{t{\rightarrow} b-}g(t)$, respectively.
\end{conv}

Our main results are the following theorems.
Throughout the paper we will denote
$$
\widetilde{V}(x) : = \| v_1^{-1}v_2\|_{p_1 {\rightarrow} 	p_2,B(0,x)}, \quad \mbox{and} \quad {\mathcal V}(t,x):= \frac{\widetilde{V}(t)}{\widetilde{V}(t)+\widetilde{V}(x)} ~ (t > 0,\,x > 0).
$$
\begin{thm}\label{main01}
	Let $0 < {\theta}_2 = p_2 \leq p_1 = {\theta}_1 < \infty$. Assume that $v_1, v_2\in {{\mathcal W}}({{\mathbb R}^n})$, ${\omega}_1\in {\,^{^{\bf c}}\!}{{\Omega}_{{\theta}_1}}$ and ${\omega}_2 \in {\Omega}_{{\theta}_2}$. Then
	\begin{equation*}
	\|{\operatorname{I}}\|_{{\,^{^{\bf c}}\!} LM_{p_1{\theta}_1,{\omega}_1}({{\mathbb R}^n},v_1){\rightarrow} LM_{p_2{\theta}_2,{\omega}_2}({{\mathbb R}^n},v_2)}{\approx} \big\|\|{\omega}_1\|_{p_1,(0,|\cdot|)}^{-1} \,\, \|{\omega}_2\|_{p_2,(|\cdot|,\infty)} \big\|_{p_1{\rightarrow} p_2,v_1^{-1}v_2,{{\mathbb R}^n}}.
	\end{equation*}
\end{thm}

\begin{thm}\label{main02}
	Let $0 < p_1, \,p_2, \,{\theta}_1, \,{\theta}_2 < \infty$ and ${\theta}_2 \neq p_2 \leq p_1 = {\theta}_1$.  Assume that $v_1, v_2\in {{\mathcal W}}({{\mathbb R}^n})$, ${\omega}_1\in {\,^{^{\bf c}}\!}{{\Omega}_{{\theta}_1}}$ and ${\omega}_2 \in {\Omega}_{{\theta}_2}$.
	
	\rm{(i)} If $p_1\le {\theta}_2$, then
	\begin{equation*}
	\|{\operatorname{I}}\|_{{\,^{^{\bf c}}\!} LM_{p_1{\theta}_1,{\omega}_1}({{\mathbb R}^n},v_1){\rightarrow} LM_{p_2{\theta}_2,{\omega}_2}({{\mathbb R}^n},v_2)} {\approx} \sup_{t\in (0,\infty)}  \big\| \|{\omega}_1\|_{p_1,(0,|\cdot|)}^{-1} \big\|_{p_1 {\rightarrow}  p_2,v_1^{-1}v_2,B(0,t)} \|{\omega}_2\|_{{\theta}_2,(t,\infty)};
	\end{equation*}
	
	\rm{(ii)} If ${\theta}_2 < p_1$, then
	\begin{align*}
	\|{\operatorname{I}}\|_{{\,^{^{\bf c}}\!} LM_{p_1{\theta}_1,{\omega}_1}({{\mathbb R}^n},v_1){\rightarrow} LM_{p_2{\theta}_2,{\omega}_2}({{\mathbb R}^n},v_2)} 
	{\approx} \bigg(\int_0^{\infty} \big\| \|{\omega}_1\|_{p_1,(0,|\cdot|)}^{-1} \big\|_{p_1 {\rightarrow} p_2,v_1^{-1}v_2,B(0,t)}^{p_1{\rightarrow} {\theta}_2} \,d \bigg(-\|{\omega}_2\|_{{\theta}_2,(t,\infty)}^{p_1{\rightarrow} {\theta}_2} \bigg) \bigg)^{\frac{1}{p_1{\rightarrow} {\theta}_2}}.
	\end{align*}
\end{thm}

\begin{thm}\label{main03}
	Let $0<p_1, \,p_2, \,{\theta}_1, \,{\theta}_2 < \infty$ and ${\theta}_2 = p_2 \leq p_1 \neq {\theta}_1$. Assume that $v_1, v_2\in {{\mathcal W}}({{\mathbb R}^n})$, ${\omega}_1\in {\,^{^{\bf c}}\!}{{\Omega}_{{\theta}_1}}$ and ${\omega}_2 \in {\Omega}_{{\theta}_2}$.
	
	{\rm (i)} If ${\theta}_1 \le p_2$, then
	\begin{equation*}
	\|{\operatorname{I}}\|_{{\,^{^{\bf c}}\!} LM_{p_1{\theta}_1,{\omega}_1}({{\mathbb R}^n},v_1){\rightarrow} LM_{p_2{\theta}_2,{\omega}_2}({{\mathbb R}^n},v_2)}{\approx} 
	\sup_{t\in (0,\infty)} \|{\omega}_1\|_{{\theta}_1,(0,t)}^{-1} \, \big\| \|{\omega}_2\|_{p_2,(|\cdot|,\infty)} \big\|_{p_1{\rightarrow} p_2,v_1^{-1}v_2,B(0,t)};
	\end{equation*}
	
	{\rm (ii)} If $p_2 < {\theta}_1$, then
	\begin{align*}
	\|{\operatorname{I}}\|_{{\,^{^{\bf c}}\!} LM_{p_1{\theta}_1,{\omega}_1}({{\mathbb R}^n},v_1){\rightarrow} LM_{p_2{\theta}_2,{\omega}_2}({{\mathbb R}^n},v_2)}  {\approx} & \,\, \bigg(\int_0^\infty \big\| \|{\omega}_2\|_{p_2,(|\cdot|,\infty)} \big\|_{p_1{\rightarrow} p_2,v_1^{-1}v_2,B(0,t)}^{{\theta}_1{\rightarrow} p_2}  d\,\bigg(-\|{\omega}_1\|_{{\theta}_1,(0,t)}^{-{\theta}_1{\rightarrow} p_2}\bigg)\bigg)^{\frac{1}{{\theta}_1{\rightarrow} p_2}} \\
	& + \|{\omega}_1\|_{{\theta}_1,(0,\infty)}^{-1} \big\| \|{\omega}_2\|_{p_2,(|\cdot|,\infty)}\big\|_{p_1{\rightarrow} p_2,v_1^{-1}v_2,{{\mathbb R}^n}}.
	\end{align*}
\end{thm}

In view of Lemma \ref{LMpp}, Theorems \ref{main01} - \ref{main03} are straightforward consequences of \cite[Theorem~3.1]{mu_emb} and \cite[Theorem~4.2]{mu_emb}.

To state further results we need the following definitions.
\begin{defi}
	Let $U$ be a continuous, strictly increasing function on $[0,{\infty})$ such that $U(0)=0$ and $\lim_{t{\rightarrow}{\infty}}U(t)={\infty}$. Then we say that $U$ is admissible.
\end{defi}

Let $U$ be an admissible function. We say that a function ${\varphi}$ is
$U$-quasiconcave if ${\varphi}$ is equivalent to an increasing function on
$(0,{\infty})$ and ${\varphi} / {U}$ is equivalent to a decreasing function on
$(0,\infty)$. We say that a $U$-quasiconcave function ${\varphi}$ is
non-degenerate if
$$
\lim_{t{\rightarrow} 0+} {\varphi}(t) = \lim_{t{\rightarrow} {\infty}} \frac{1}{{\varphi}(t)} = \lim_{t{\rightarrow}
	{\infty}} \frac{{\varphi}(t)}{U(t)} = \lim_{t{\rightarrow} 0+} \frac{U(t)}{{\varphi}(t)} =0.
$$
The family of non-degenerate $U$-quasiconcave functions is denoted
by $Q_U$.
\begin{defi}
	Let $U$ be an admissible function, and let $w$ be a non-negative measurable function on $(0,{\infty})$. We say that the function ${\varphi}$, defined by
	\begin{equation*}
	{\varphi}(t)=U(t)\int_0^{\infty} \frac{w(\tau)\,d\tau}{U(\tau)+U(t)}, {\qquad} t\in (0,{\infty}),
	\end{equation*}
	is a fundamental function of $w$ with respect to $U$. One will also say that $w(\tau)\,d\tau$ is a representation measure of ${\varphi}$ with respect to $U$.
\end{defi}

\begin{rem}\label{nondegrem}
	Let ${\varphi}$ be the fundamental function of $w$ with
	respect to $U$.
	Assume that
	\begin{equation*}
	\int_0^{\infty}\frac{w(\tau)\,d\tau}{U(\tau)+U(t)}<{\infty}, ~ t> 0, {\qquad} \int_0^1 \frac{w(\tau)\,d\tau}{U(\tau)}=\int_1^{\infty}w(\tau)\,d\tau=\infty.
	\end{equation*}
	Then ${\varphi}\in Q_{U}$.
\end{rem}

\begin{rem}\label{limsupcondition}
	Suppose that ${\varphi} (x) < {\infty}$ for all $x \in (0,{\infty})$, where ${\varphi}$ is
	defined by
	\begin{equation*}
	{\varphi}(x)={\operatornamewithlimits{ess\,sup}}_{t\in(0,x)}{U(t)}
	{\operatornamewithlimits{ess\,sup}}_{\tau\in(t,{\infty})}\frac{w(\tau)}{U(\tau)},~~t\in{(0,{\infty})}.
	\end{equation*}
	If
	$$
	\limsup_{t \rightarrow 0 +} w(t) = \limsup_{t \rightarrow +\infty} \frac{1}{w(t)} = \limsup_{t \rightarrow 0 +} \frac{U(t)}{w(t)} = \limsup_{t \rightarrow +\infty} \frac{w(t)}{U(t)} = 0,
	$$
	then
	${\varphi}\in Q_{U}$.
\end{rem}

\begin{thm}\label{maintheorem1}
	Let $0<p_1, \,p_2, \,{\theta}_1, \,{\theta}_2 < \infty$, $p_2 < p_1$, ${\theta}_1 \leq p_2 < {\theta}_2$. Assume that $v_1, v_2\in {{\mathcal W}}({{\mathbb R}^n})$, ${\omega}_1 \in {\,^{^{\bf c}}\!}{{\Omega}_{{\theta}_1}}$ and ${\omega}_2\in {\Omega}_{{\theta}_2}$. 
	Suppose that $\widetilde{V}$ is admissible and
	\begin{equation*}
	{\varphi}_1(x):= \sup_{t\in (0,\infty)} \widetilde{V}(t) \, {\mathcal V}(x,t) \,\, \|{\omega}_1\|_{{\theta}_1,(0,t)}^{-1} \in Q_{ \widetilde{V}^{\frac{1}{p_1 {\rightarrow} p_2}}}.
	\end{equation*}
	
	{\rm (i)} If $p_1 \le {\theta}_2$, then
	\begin{equation*}
	\|{\operatorname{I}}\|_{{\,^{^{\bf c}}\!} LM_{p_1{\theta}_1,{\omega}_1}({{\mathbb R}^n},v_1){\rightarrow} LM_{p_2{\theta}_2,{\omega}_2}({{\mathbb R}^n},v_2)}{\approx} \sup_{x\in(0,\infty)} {\varphi}_1(x) \sup_{t\in (0,\infty)} {\mathcal V} (t,x) \,\, \|{\omega}_2\|_{{\theta}_2,(t,\infty)}.
	\end{equation*}
	
	{\rm (ii)} If ${\theta}_2 < p_1$, then
	\begin{align*}
	\|{\operatorname{I}}\|_{{\,^{^{\bf c}}\!} LM_{p_1{\theta}_1,{\omega}_1}({{\mathbb R}^n},v_1){\rightarrow} LM_{p_2{\theta}_2,{\omega}_2}({{\mathbb R}^n},v_2)} {\approx} \sup_{x\in(0,\infty)} {\varphi}_1(x) \bigg(\int_0^\infty  {\mathcal V}(t,x)^{p_1{\rightarrow} {\theta}_2}\,d\bigg(-\|{\omega}_2\|_{{\theta}_2,(t,\infty)}^{p_1{\rightarrow} {\theta}_2}\bigg) \bigg)^{\frac{1}{p_1{\rightarrow}{\theta}_2}}.
	\end{align*}
\end{thm}

\begin{thm}\label{maintheorem3}
	Let $0 < p_1, \,p_2, \,{\theta}_1, \,{\theta}_2 < \infty$,\ $p_2 < p_1$ and $p_2 < \min\{{\theta}_1,{\theta}_2\}$. Assume that $v_1, v_2\in {{\mathcal W}}({{\mathbb R}^n})$, ${\omega}_1 \in	{\,^{^{\bf c}}\!}{{\Omega}_{{\theta}_1}}$ and ${\omega}_2\in {\Omega}_{{\theta}_2}$.
	Suppose that $\widetilde{V}$ is admissible and
	\begin{equation*}
	{\varphi}_2(x):= \bigg(\int_0^{\infty} [\widetilde{V}(t){\mathcal V}(x,t)]^{{\theta}_1{\rightarrow} p_2}\,d\bigg(-\|{\omega}_1\|_{{\theta}_1,(0,t)}^{-{\theta}_1{\rightarrow} p_2}\bigg)\bigg)^{\frac{1}{{\theta}_1 {\rightarrow} p_2}} \in Q_{{\widetilde{V}}^{\frac{1}{p_1{\rightarrow} p_2}}}.
	\end{equation*}
	
	
	
	
	
	
	
	
	
	
	
	{\rm (i)} If $\max\{p_1,{\theta}_1\}\leq {\theta}_2$, then
	\begin{align*}
	\|{\operatorname{I}}\|_{{\,^{^{\bf c}}\!} LM_{p_1{\theta}_1,{\omega}_1}({{\mathbb R}^n},v_1){\rightarrow} LM_{p_2{\theta}_2,{\omega}_2}({{\mathbb R}^n},v_2)} {\approx} & \,\, \sup_{x\in (0,\infty)} {\varphi}_2(x) \sup_{t\in (0,\infty)} {\mathcal V}(t,x) \|{\omega}_2\|_{{\theta}_2,(t,\infty)} \\
	& + \|{\omega}_1\|_{{\theta}_1,(0,\infty)}^{-1} \sup_{t\in (0,\infty)} \widetilde{V}(t) \|{\omega}_2\|_{{\theta}_2,(t,\infty)};
	\end{align*}
	
	{\rm (ii)} If $p_1 \leq {\theta}_2 < {\theta}_1$, then
	\begin{align*}
	\|{\operatorname{I}}\|_{{\,^{^{\bf c}}\!} LM_{p_1{\theta}_1,{\omega}_1}({{\mathbb R}^n},v_1){\rightarrow} LM_{p_2{\theta}_2,{\omega}_2}({{\mathbb R}^n},v_2)} & \\
	& \hspace{-3.5cm} {\approx} \bigg( \int_0^{\infty} {\varphi}_2(x)^{\frac{{\theta}_1 {\rightarrow} {\theta}_2 \cdot {\theta}_1 {\rightarrow} p_2}{{\theta}_2 {\rightarrow} p_2}} \widetilde{V}(x)^{{\theta}_1 {\rightarrow} p_2} \bigg(\sup_{t\in(0,\infty)} {\mathcal V}(t,x) \|{\omega}_2\|_{{\theta}_2,(t,\infty)}\bigg)^{{\theta}_1 {\rightarrow} {\theta}_2} d \bigg( - \|{\omega}_1\|_{{\theta}_1,(0,x)}^{-{\theta}_1 {\rightarrow} p_2}\bigg) \bigg)^{\frac{1}{{\theta}_1 {\rightarrow} {\theta}_2}} \\
	& \hspace{-3cm} + \|{\omega}_1\|_{{\theta}_1,(0,\infty)}^{-1} \sup_{t\in (0,\infty)} \widetilde{V}(t) \|{\omega}_2\|_{{\theta}_2,(t,\infty)};
	\end{align*}
	
	{\rm (iii)} If ${\theta}_1 \leq {\theta}_2 < p_1$, then
	\begin{align*}
	\|{\operatorname{I}}\|_{{\,^{^{\bf c}}\!} LM_{p_1{\theta}_1,{\omega}_1}({{\mathbb R}^n},v_1) {\rightarrow} LM_{p_2{\theta}_2,{\omega}_2}({{\mathbb R}^n},v_2)} {\approx} & \,\, \sup_{x\in (0,\infty)} {\varphi}_2(x) \bigg(\int_0^{\infty} {\mathcal V}(t,x)^{p_1 {\rightarrow} {\theta}_2} d\bigg(-\|{\omega}_2\|_{{\theta}_2,(t,\infty)}^{p_1 {\rightarrow} {\theta}_2}\bigg)\bigg)^{\frac{1}{p_1 {\rightarrow} {\theta}_2}}\\
	&  + \|{\omega}_1\|_{{\theta}_1,(0,\infty)}^{-1} \bigg( \int_0^{\infty} \widetilde{V}(t)^{p_1 {\rightarrow} {\theta}_2} d\bigg(-\|{\omega}_2\|_{{\theta}_2,(t,\infty)}^{p_1 {\rightarrow} {\theta}_2}\bigg)\bigg)^{\frac{1}{p_1 {\rightarrow} {\theta}_2}};
	\end{align*}
	
	{\rm (iv)} If ${\theta}_2 < \min\{p_1,{\theta}_1\}$, then
	\begin{align*}
	\|{\operatorname{I}}\|_{{\,^{^{\bf c}}\!} LM_{p_1{\theta}_1,{\omega}_1}({{\mathbb R}^n},v_1){\rightarrow} LM_{p_2{\theta}_2,{\omega}_2}({{\mathbb R}^n},v_2)} & \\
	& \hspace{-3.5cm} {\approx} \bigg( \int_0^{\infty} {\varphi}_2(x)^{\frac{{\theta}_1 {\rightarrow} {\theta}_2 \cdot {\theta}_1 {\rightarrow} p_2}{{\theta}_2 {\rightarrow} p_2}} \widetilde{V}(x)^{{\theta}_1 {\rightarrow} p_2} \bigg(\int_0^{\infty}{\mathcal V}(t,x)^{p_1 {\rightarrow}{\theta}_2} d\bigg(-\|{\omega}_2\|_{{\theta}_2,(t,\infty)}^{p_1 {\rightarrow} {\theta}_2} \bigg)\bigg)^{\frac{{\theta}_1 {\rightarrow} {\theta}_2}{p_1 {\rightarrow} {\theta}_2}} d\bigg(-\|{\omega}_1\|_{{\theta}_1,(0,x)}^{-{\theta}_1 {\rightarrow} p_2} \bigg)\bigg)^{\frac{1}{{\theta}_1 {\rightarrow} {\theta}_2}} \\
	& \hspace{-3cm} +\|{\omega}_1\|_{{\theta}_1,(0,\infty)}^{-1} \bigg( \int_0^{\infty} \widetilde{V}(t)^{p_1 {\rightarrow} {\theta}_2} d\bigg(-\|{\omega}_2\|_{{\theta}_2,(t,\infty)}^{p_1 {\rightarrow} {\theta}_2} \bigg)\bigg)^{\frac{1}{p_1 {\rightarrow} {\theta}_2}}.
	\end{align*}
\end{thm}

\begin{thm}\label{maintheorem2}
	Let $0 < {\theta}_1 < p = p_1 = p_2 < {\theta}_2 < \infty$. Assume that $v_1, v_2 \in {{\mathcal W}}({{\mathbb R}^n})\cap C({{\mathbb R}^n})$, ${\omega}_1 \in {\,^{^{\bf c}}\!}{{\Omega}_{{\theta}_1}}$ and ${\omega}_2\in {\Omega}_{{\theta}_2}$.
	\begin{equation*}
	\|{\operatorname{I}}\|_{{\,^{^{\bf c}}\!} LM_{p_1{\theta}_1,{\omega}_1}({{\mathbb R}^n},v_1){\rightarrow} LM_{p_2{\theta}_2,{\omega}_2}({{\mathbb R}^n},v_2)}{\approx} 
	\sup_{t\in {(0,{\infty})}} \big\|\|{\omega}_1\|_{{\theta}_1,(0,|\cdot|)}^{-1} \big\|_{\infty,v_1^{-1}v_2,B(0,t)} \|{\omega}_2\|_{{\theta}_2,(t,\infty)}.
	\end{equation*}
\end{thm}

\begin{thm}\label{maintheorem4}
	Let $0 < {\theta}_1, \,{\theta}_2 < \infty$ and $0 < p = p_1 = p_2 < \min\{{\theta}_1,{\theta}_2\}$. Assume that $v_1, v_2\in {{\mathcal W}}({{\mathbb R}^n})$ such that $v_1^{-1}v_2 \in C({{\mathbb R}^n})$. Suppose that ${\omega}_1 \in {\,^{^{\bf c}}\!}{{\Omega}_{{\theta}_1}}$, ${\omega}_2\in {\Omega}_{{\theta}_2}$ and 
	$$
	0 < \|{\omega}_2^{-1}\|_{{\theta}_2 {\rightarrow} p, (x,\infty)} < \infty
	$$
	holds for all $x>0$.
	
	{\rm (i)} If ${\theta}_1 \le {\theta}_2$, then
	\begin{align*}
	\|{\operatorname{I}}\|_{{\,^{^{\bf c}}\!} LM_{p_1{\theta}_1,{\omega}_1}({{\mathbb R}^n},v_1){\rightarrow} LM_{p_2{\theta}_2,{\omega}_2}({{\mathbb R}^n},v_2)} \\ 
	& \hspace{-4cm} {\approx} \sup_{x\in (0,\infty)} \bigg( \widetilde{V}(x)^{{\theta}_1{\rightarrow} p} \int_x^{\infty} \, d \bigg(-\|{\omega}_1\|_{{\theta}_1,(0,t)}^{-{\theta}_1{\rightarrow} p}\bigg)  + \int_0^x  \widetilde{V}(t)^{{\theta}_1{\rightarrow} p} \, d\bigg(-\|{\omega}_1\|_{{\theta}_1,(0,t)}^{- {\theta}_1{\rightarrow} p}\bigg) \bigg) ^{{\frac{1}{{\theta}_1{\rightarrow} p}}}   \|{\omega}_2\|_{{\theta}_2,(x,\infty)} \\
	&\hspace{-3.5cm} +\|{\omega}_1\|_{{\theta}_1,(0,\infty)}^{-1} \sup_{t \in (0,{\infty})} \widetilde{V}(t) \|{\omega}_2\|_{{\theta}_2,(t,\infty)};
	\end{align*}
	
	{\rm (ii)} If ${\theta}_2 < {\theta}_1$, then
	\begin{align*}
	\|{\operatorname{I}}\|_{{\,^{^{\bf c}}\!} LM_{p_1{\theta}_1,{\omega}_1}({{\mathbb R}^n},v_1){\rightarrow} LM_{p_2{\theta}_2,{\omega}_2}({{\mathbb R}^n},v_2)} & \\
	& \hspace{-4.5cm}{\approx} \bigg(\int_0^{\infty} \bigg(\int_x^{\infty} d\bigg(- \|{\omega}_1\|_{{\theta}_1,(0,t)}^{-{\theta}_1{\rightarrow} p}\bigg)\bigg)^{\frac{{\theta}_1{\rightarrow} {\theta}_2}{{\theta}_2 {\rightarrow} p}}  \bigg(\sup_{0 < \tau \leq x} \widetilde V(\tau) \|{\omega}_2\|_{{\theta}_2,(\tau,\infty)}\bigg)^{{\theta}_1 {\rightarrow} {\theta}_2} d\bigg(-\|{\omega}_1\|_{{\theta}_1,(0,x)}^{-{\theta}_1 {\rightarrow} p}\bigg)\bigg)^{\frac{1}{{\theta}_1 {\rightarrow} {\theta}_2}} \\
	& \hspace{-4cm} + \bigg(\int_0^{\infty} \bigg(\int_0^x \widetilde V(t)^{{\theta}_1{\rightarrow} p} d\bigg(-\|{\omega}_1\|_{{\theta}_1,(0,t)}^{-{\theta}_1{\rightarrow} p} \bigg) \bigg)^{\frac{{\theta}_1{\rightarrow} {\theta}_2}{{\theta}_2{\rightarrow} p}} \widetilde V(x)^{{\theta}_1{\rightarrow} p} \|{\omega}_2\|_{{\theta}_2,(t,\infty)}^{{\theta}_1{\rightarrow} {\theta}_2} d\bigg(-\|{\omega}_1\|_{{\theta}_1,(0,x)}^{-{\theta}_1{\rightarrow} p} \bigg)\bigg)^{\frac{1}{{\theta}_1{\rightarrow} {\theta}_2}}\\
	& \hspace{-4cm} + \|{\omega}_1\|_{{\theta}_1,(0,\infty)}^{-1} \sup_{t\in (0,{\infty})} \widetilde V(t) \|{\omega}_2\|_{{\theta}_2,(t,\infty)}.
	\end{align*}
\end{thm}

\section{Proofs of main results}\label{s7}

Before proceeding to the proof of our main results we recall the following integration in polar coordinates formula. 

We denote the unit sphere $\{x\in {{\mathbb R}^n} : |x|=1\}$ in ${{\mathbb R}^n}$ by $S^{n-1}$. If $x\in {{\mathbb R}^n} \backslash \{0\}$, the polar coordinates of $x$ are
$$
r=|x| \in (0,\infty), \qquad x'=\frac{x}{|x|} \in S^{n-1}.
$$
There is a unique Borel measure $\sigma = \sigma_{n-1}$ on $S^{n-1}$ such that if $f$ is Borel measurable on ${{\mathbb R}^n}$ and $f\geq 0$ or $f\in L^1({{\mathbb R}^n})$, then 
$$
\int_{{\mathbb R}^n} f(x) \,dx = \int_0^{\infty} \int_{S^{n-1}} f(rx') r^{n-1} d\sigma(x') dr
$$
(see, for instance, \cite[p. 78]{Folland}).

\begin{lem}\label{triviality}
	Let $0 < p_1, \,p_2, \,{\theta}_1, \,{\theta}_2 \leq \infty$ and $p_1 < p_2$. Assume that $v_1, v_2\in {{\mathcal W}}({{\mathbb R}^n})$, ${\omega}_1\in {\,^{^{\bf c}}\!}{{\Omega}_{{\theta}_1}}$ and ${\omega}_2\in {\Omega}_{{\theta}_2}$. Then ${\,^{^{\bf c}}\!}{LM_{p_1{\theta}_1,{\omega}_1}({{\mathbb R}^n},v_1)}\not \hookrightarrow LM_{p_2{\theta}_2,{\omega}_2}({{\mathbb R}^n},v_2)$.
\end{lem}
\begin{proof}
	Assume that ${\,^{^{\bf c}}\!}{LM_{p_1{\theta}_1,{\omega}_1}({{\mathbb R}^n},v_1)} \hookrightarrow LM_{p_2{\theta}_2,{\omega}_2}({{\mathbb R}^n},v_2)$ holds. Then there exist $c>0$ such that
	\begin{equation*}
	\|f\|_{LM_{p_2{\theta}_2,{\omega}_2}({{\mathbb R}^n},v_2)} \leq c \ \|f\|_{{\,^{^{\bf c}}\!}{LM_{p_1{\theta}_1,{\omega}_1}({{\mathbb R}^n},v_1)}}
	\end{equation*}
	holds for all $f\in {\mathfrak M}^+({{\mathbb R}^n})$. Let $\tau\in (0,\infty)$ and $f\in {\mathfrak M}({{\mathbb R}^n})$:  ${\operatorname{supp}} f\subset  B(0,\tau)$. It is easy to see that
	\begin{align}
	\|f\|_{LM_{p_2{\theta}_2,{\omega}_2}({{\mathbb R}^n},v_2)}&= \big\| \|f\|_{p_2,v_2,B(0,t)} \big\|_{{\theta}_2,{\omega}_2,(0,\infty)}\notag\\
	&\geq \big\| \|f\|_{p_2,v_2,B(0,t)} \big\|_{{\theta}_2,{\omega}_2,(\tau,\infty)}\notag\\
	&\geq \|{\omega}_2 \|_{{\theta}_2,(\tau,\infty)} \, \|f\|_{p_2,v_2,B(0,\tau)} \label{1}
	\end{align}
	and
	\begin{align}
	\|f\|_{{\,^{^{\bf c}}\!} LM_{p_1{\theta}_1,{\omega}_1}({{\mathbb R}^n},v_1)}&= \big\| \|f\|_{p_1,v_1,{\,^{^{\bf c}}\!} B(0,t)} \big\|_{{\theta}_1,{\omega}_1,(0,\infty)}\notag\\
	&= \big\| \|f\|_{p_1,v_1,{\,^{^{\bf c}}\!} B(0,t)} \big\|_{{\theta}_1,{\omega}_1,(0,\tau)}\notag\\
	&\leq \| {\omega}_1 \|_{{\theta}_1,(0,\tau)} \, \|f\|_{p_1,v_1, B(0,\tau)} \label{2}.
	\end{align}
	
	Combining \eqref{1} with \eqref{2}, we can assert that
	\begin{equation*}
	\| {\omega}_2 \|_{{\theta}_2,(\tau,\infty)} \, \|f\|_{p_2,v_2,B(0,\tau)}\leq c \, \| {\omega}_1 \|_{{\theta}_1,(0,\tau)} \,  \|f\|_{p_1,v_1,B(0,\tau)}.
	\end{equation*}
	Since ${\omega}_1\in {\,^{^{\bf c}}\!}{{\Omega}_{{\theta}_1}}$ and ${\omega}_2\in {\Omega}_{{\theta}_2}$, we conclude that $L_{p_1}(B(0,\tau),v_1)\hookrightarrow L_{p_2}(B(0,\tau),v_2)$, which is a contradiction.
\end{proof}

The following lemma is true.
\begin{lem}\label{mainlemma}
	Let $0 < p_1, \,p_2, \,{\theta}_1, \,{\theta}_2 < \infty$, $p_2 \leq p_1$ and $p_2 < {\theta}_2$. Assume that $v_1, v_2\in {{\mathcal W}}({{\mathbb R}^n})$, ${\omega}_1 \in {\,^{^{\bf c}}\!}{{\Omega}_{{\theta}_1}}$ and ${\omega}_2\in {\Omega}_{{\theta}_2}$. Then
	\begin{align*}
	\|{\operatorname{I}}\|_{{\,^{^{\bf c}}\!} LM_{p_1{\theta}_1,{\omega}_1}({{\mathbb R}^n},v_1){\rightarrow} LM_{p_2{\theta}_2,{\omega}_2}({{\mathbb R}^n},v_2)}
	= \left\{\sup_{g\in {\mathfrak M}^+{(0,{\infty})}} {\frac{\displaystyle {\|{\operatorname{I}}\|_{{\,^{^{\bf c}}\!} LM_{p_1{\theta}_1,{\omega}_1}({{\mathbb R}^n},v_1){\rightarrow} L_{p_2}\big(v_2(\cdot)H^*g(|\cdot|)^{\frac{1}{p_2}}\big)}^{p_2}}}{\displaystyle {\|g\|_{\frac{{\theta}_2}{{\theta}_2-p_2},{\omega}_2^{-p_2},(0,\infty)}}}} \right\}^{\frac{1}{p_2}}.
	\end{align*}
\end{lem}
\begin{proof}
	By duality, interchanging suprema, we have that
	\begin{align*}
	\|{\operatorname{I}}\|_{{\,^{^{\bf c}}\!} LM_{p_1{\theta}_1,{\omega}_1}({{\mathbb R}^n},v_1){\rightarrow} LM_{p_2{\theta}_2,{\omega}_2}({{\mathbb R}^n},v_2)} & \\
	& \hspace{-3cm} =\sup_{f\in {\mathfrak M}^+({{\mathbb R}^n})} {\frac{\displaystyle {\|f\|_{LM_{p_2{\theta}_2,{\omega}_2}({{\mathbb R}^n},v_2)}}}{\displaystyle {\|f\|_{{\,^{^{\bf c}}\!} LM_{p_1{\theta}_1,{\omega}_1}({{\mathbb R}^n},v_1)}}}} \\
	& \hspace{-3cm} = \sup_{f\in {\mathfrak M}^+({{\mathbb R}^n})} {\frac{\displaystyle {1}}{\displaystyle {\|f\|_{{\,^{^{\bf c}}\!} LM_{p_1{\theta}_1,{\omega}_1}({{\mathbb R}^n},v_1)}}}} \sup_{g\in {\mathfrak M}^+{(0,{\infty})}} {\frac{\displaystyle {\bigg(\int_0^\infty \bigg(\int_{B(0,\tau)} f(x)^{p_2}  v_2(x)^{p_2} \, dx \bigg) g(\tau)\,d\tau\bigg)^{\frac{1}{p_2}}}}{\displaystyle {\|g\|_{\frac{{\theta}_2}{{\theta}_2-p_2},{\omega}_2^{-p_2},(0,\infty)}^{\frac{1}{p_2}}}}}\\
	& \hspace{-3cm} = \sup_{g\in {\mathfrak M}^+{(0,{\infty})}} {\frac{\displaystyle {1}}{\displaystyle {\|g\|_{\frac{{\theta}_2}{{\theta}_2-p_2},{\omega}_2^{-p_2},(0,\infty)}^{\frac{1}{p_2}}}}} \sup_{f\in {\mathfrak M}^+({{\mathbb R}^n})} {\frac{\displaystyle {\bigg(\int_0^\infty \bigg(\int_{B(0,\tau)} f(x)^{p_2} v_2(x)^{p_2} \,dx\bigg)  g(\tau) \,d\tau\bigg)^{\frac{1}{p_2}}}}{\displaystyle {\|f\|_{{\,^{^{\bf c}}\!} LM_{p_1{\theta}_1,{\omega}_1}({{\mathbb R}^n},v_1)}}}}.
	\end{align*}
	
	Applying Fubini's Theorem, we get that
	\begin{align}
	\|{\operatorname{I}}\|_{{\,^{^{\bf c}}\!} LM_{p_1{\theta}_1,{\omega}_1}({{\mathbb R}^n},v_1){\rightarrow} LM_{p_2{\theta}_2,{\omega}_2}({{\mathbb R}^n},v_2)} & \notag \\
	& \hspace{-3cm} = \sup_{g\in {\mathfrak M}^+{(0,{\infty})}} {\frac{\displaystyle {1}}{\displaystyle {\|g\|_{\frac{{\theta}_2}{{\theta}_2-p_2},{\omega}_2^{-p_2},(0,\infty)}^{\frac{1}{p_2}}}}} \sup_{f\in {\mathfrak M}^+({{\mathbb R}^n})} {\frac{\displaystyle {\bigg(\int_{{\mathbb R}^n} f(x)^{p_2} v_2(x)^{p_2} \bigg(\int_{|x|}^{\infty} g(\tau) \,d\tau \bigg)\, dx \bigg)^{\frac{1}{p_2}}}}{\displaystyle {\|f\|_{{\,^{^{\bf c}}\!} LM_{p_1{\theta}_1,{\omega}_1}({{\mathbb R}^n},v_1)}}}}\notag\\
	& \hspace{-3cm} = \sup_{g\in {\mathfrak M}^+{(0,{\infty})}} {\frac{\displaystyle {1}}{\displaystyle {\|g\|_{\frac{{\theta}_2}{{\theta}_2-p_2},{\omega}_2^{-p_2},(0,\infty)}^{\frac{1}{p_2}}}}} \|{\operatorname{I}}\|_{{\,^{^{\bf c}}\!} LM_{p_1{\theta}_1,{\omega}_1}({{\mathbb R}^n},v_1){\rightarrow} L_{p_2}\big(v_2(\cdot)H^*g(|\cdot|)^{\frac{1}{p_2}}\big)}. \label{GommeninIlkDenkligi}
	\end{align}
\end{proof}

\noindent{\bf Proof of Theorem \ref{maintheorem1}.}
By Lemma \ref{mainlemma}, we have that
\begin{align*}
\|{\operatorname{I}}\|_{{\,^{^{\bf c}}\!} LM_{p_1{\theta}_1,{\omega}_1}({{\mathbb R}^n},v_1){\rightarrow} LM_{p_2{\theta}_2,{\omega}_2}({{\mathbb R}^n},v_2)} =
\sup_{g\in {\mathfrak M}^+{(0,{\infty})}}
\frac{1}{\|g\|_{\frac{{\theta}_2}{{\theta}_2-p_2},{\omega}_2^{-p_2},(0,\infty)}^{\frac{1}{p_2}}}
\|{\operatorname{I}}\|_{{\,^{^{\bf c}}\!} LM_{p_1{\theta}_1,{\omega}_1}({{\mathbb R}^n},v_1){\rightarrow} L_{p_2}\big(v_2(\cdot)
	H^*g(|\cdot|)^{\frac{1}{p_2}}\big)}.
\end{align*}

Since ${\theta}_1\leq p_2$, applying \cite[Theorem~4.2, (a)]{mu_emb}, we obtain that
\begin{align*}
\|{\operatorname{I}}\|_{{\,^{^{\bf c}}\!} LM_{p_1{\theta}_1,{\omega}_1}({{\mathbb R}^n},v_1){\rightarrow} LM_{p_2{\theta}_2,{\omega}_2}({{\mathbb R}^n},v_2)} {\approx}
\left\{ \sup_{g\in {\mathfrak M}^+(0,\infty)} {\frac{\displaystyle {\sup_{t\in (0,\infty)}
	\|{\omega}_1\|_{{\theta}_1,(0,t)}^{- p_2} \|H^*g(|\cdot|)\|_{\frac{p_1}{p_1 - p_2},(v_1^{-1}v_2)^{p_2},B(0,t)}}}{\displaystyle {\|g\|_{\frac{{\theta}_2}{{\theta}_2-p_2},{\omega}_2^{-p_2},(0,\infty)}}}}\right\}^{\frac{1}{p_2}}.
\end{align*}

Using polar coordinates, we have that
\begin{equation*}
\|H^*g(|\cdot|)\|_{\frac{p_1}{p_1-p_2},(v_1^{-1}v_2)^{p_2},B(0,t)}=\|H^*g\|_{\frac{p_1}{p_1-p_2},{\tilde v}^{\frac{p_1-p_2}{p_1}},(0,t)}, \quad t>0,
\end{equation*}
where
\begin{equation*}
\tilde{v}(r):=\int_{S^{n-1}} (v_1^{-1}v_2)(rx')^{\frac{p_1p_2}{p_1- p_2}}r^{n-1}d\sigma(x'), \quad r > 0.
\end{equation*}

Thus, we obtain that\begin{align*}
\|{\operatorname{I}}\|_{{\,^{^{\bf c}}\!} LM_{p_1{\theta}_1,{\omega}_1}({{\mathbb R}^n},v_1){\rightarrow} LM_{p_2{\theta}_2,{\omega}_2}({{\mathbb R}^n},v_2)} {\approx} \left\{\sup_{g\in {\mathfrak M}^+{(0,{\infty})}} {\frac{\displaystyle {\sup_{t\in(0,\infty)} \|{\omega}_1\|_{{\theta}_1,(0,t)}^{-p_2} \left\|H^*g\right\|_{\frac{p_1}{p_1-p_2},{\tilde v}^{\frac{p_1-p_2}{p_1}},(0,t)}}}{\displaystyle {\|g\|_{\frac{{\theta}_2}{{\theta}_2-p_2},{\omega}_2^{-p_2},(0,\infty)}}}}\right\}^{\frac{1}{p_2}}.
\end{align*}

Taking into account that
\begin{align}
\int_0^t \tilde{v} (r)\,dr & = \int_0^t \int_{S^{n-1}} (v_1^{-1}v_2)(rx')^{\frac{p_1p_2}{p_1-p_2}}\,d\sigma(x') r^{n-1} dr \notag \\
& = \int_{B(0,t)} (v_1^{-1}v_2)^{\frac{p_1p_2}{p_1-p_2}} (x)\,dx = \widetilde V(t)^{\frac{p_1p_2}{p_1-p_2}}, \label{Vtilde}
\end{align}

\rm{(i)} if $p_1\le \theta_2$, then applying \cite[Theorem 3.2,
(i)]{gmp}, we arrive at
\begin{equation*}
\|{\operatorname{I}}\|_{{\,^{^{\bf c}}\!} LM_{p_1{\theta}_1,{\omega}_1}({{\mathbb R}^n},v_1){\rightarrow} LM_{p_2{\theta}_2,{\omega}_2}({{\mathbb R}^n},v_2)} {\approx}
\sup_{x\in (0,\infty)} {\varphi}_1(x) \sup_{t\in (0,\infty)} {\mathcal V}(t,x)
\|{\omega}_2\|_{{\theta}_2,(t,\infty)};
\end{equation*}

\rm{(ii)} if ${\theta}_2 < p_1$, then applying \cite[Theorem 3.2, (ii)]{gmp}, we arrive at
\begin{align*}
\|{\operatorname{I}}\|_{{\,^{^{\bf c}}\!} LM_{p_1{\theta}_1,{\omega}_1}({{\mathbb R}^n},v_1){\rightarrow} LM_{p_2{\theta}_2,{\omega}_2}({{\mathbb R}^n},v_2)} {\approx}
\sup_{x\in (0,\infty)} {\varphi}_1(x) \bigg(\int_0^{\infty} {\mathcal V}(t,x)^{p_1 {\rightarrow}
	{\theta}_2}  d\bigg( - \|{\omega}_2\|_{{\theta}_2,(t,\infty)}^{p_1 {\rightarrow} {\theta}_2}\bigg)
\bigg)^{\frac{1}{p_1 {\rightarrow} {\theta}_2}}.
\end{align*}

The proof is completed.

\hspace{16.9cm}$\square$

\begin{rem}
	In view of Remark~\ref{limsupcondition}, if
	\begin{align*}
	\limsup_{t{\rightarrow} 0+} \widetilde V(t) \|{\omega}_1\|_{{\theta}_1,(0,t)}^{-1} & = \limsup_{t{\rightarrow}
		+\infty} \widetilde V(t)\|{\omega}_1\|_{{\theta}_1,(0,t)} \\
	& = \limsup_{t{\rightarrow} 0+}\|{\omega}_1\|_{{\theta}_1,(0,t)} = \limsup_{t{\rightarrow} +\infty}
	\|{\omega}_1\|_{{\theta}_1,(0,t)}^{-1} = 0,
	\end{align*}
	then ${\varphi}_1 \in Q_{\widetilde V^{\frac{1}{p_1 {\rightarrow} p_2}}}$.
\end{rem}

\noindent{\bf Proof of Theorem \ref{maintheorem3}.}
By  Lemma \ref{mainlemma}, applying \cite[Theorem~4.2, (c)]{mu_emb},  we have that
\begin{align*}
\|{\operatorname{I}}\|_{{\,^{^{\bf c}}\!} LM_{p_1{\theta}_1,{\omega}_1}({{\mathbb R}^n},v_1){\rightarrow} LM_{p_2{\theta}_2,{\omega}_2}({{\mathbb R}^n},v_2)} {\approx} & \,\, \|{\omega}_1\|_{{\theta}_1,(0,\infty)}^{-1} \left\{\sup_{g\in {\mathfrak M}^+{(0,{\infty})}} {\frac{\displaystyle { \|H^*g(|\cdot|)\|_{\frac{p_1}{p_1 - p_2},(v_1^{-1}v_2)^{p_2},{{\mathbb R}^n}}}}{\displaystyle {\|g\|_{\frac{{\theta}_2}{{\theta}_2 - p_2},{\omega}_2^{-p_2},(0,\infty)}}}} \right\}^{\frac{1}{p_2}} \\
& + \left\{\sup_{g\in {\mathfrak M}^+{(0,{\infty})}} {\frac{\displaystyle {\bigg(\int_0^{\infty} \|H^* g(|\cdot|)\|_{\frac{p_1}{p_1 - p_2},(v_1^{-1}v_2)^{p_2},B(0,t)}^{\frac{{\theta}_1}{{\theta}_1 - p_2}} d \bigg( - \|{\omega}_1\|_{{\theta}_1,(0,t)}^{- \frac{{\theta}_1 p_2}{{\theta}_1 - p_2}} \bigg)\bigg)^{\frac{{\theta}_1 - p_2}{{\theta}_1}}}}{\displaystyle {\|g\|_{\frac{{\theta}_2}{{\theta}_2 - p_2},{\omega}_2^{-p_2},(0,\infty)}}}} \right\}^{\frac{1}{p_2}}.
\end{align*}

Using polar coordinates, we have that
\begin{align*}
\|{\operatorname{I}}\|_{{\,^{^{\bf c}}\!} LM_{p_1{\theta}_1,{\omega}_1}({{\mathbb R}^n},v_1){\rightarrow} LM_{p_2{\theta}_2,{\omega}_2}({{\mathbb R}^n},v_2)} {\approx} & \,\, \|{\omega}_1\|_{{\theta}_1,(0,\infty)}^{-1} \left\{\sup_{g\in {\mathfrak M}^+{(0,{\infty})}} {\frac{\displaystyle { \|H^*g\|_{\frac{p_1}{p_1-p_2},{\tilde v}^{\frac{p_1-p_2}{p_1}},(0,\infty)}}}{\displaystyle {\|g\|_{\frac{{\theta}_2}{{\theta}_2-p_2},{\omega}_2^{-p_2},(0,\infty)}}}}\right\}^{\frac{1}{p_2}} \notag \\
& + \left\{\sup_{g\in {\mathfrak M}^+{(0,{\infty})}} {\frac{\displaystyle {\bigg(\int_0^{\infty} \|H^*g\|_{\frac{p_1}{p_1-p_2},{\tilde v}^{\frac{p_1-p_2}{p_1}},(0,t)}^{\frac{{\theta}_1}{{\theta}_1-p_2}} d\bigg(-\|{\omega}_1\|_{{\theta}_1,(0,t)}^{-\frac{{\theta}_1p_2}{{\theta}_1-p_2}}\bigg)\bigg)^{\frac{{\theta}_1-p_2}{{\theta}_1}}}}{\displaystyle {\|g\|_{\frac{{\theta}_2}{{\theta}_2-p_2},{\omega}_2^{-p_2},(0,\infty)}}}}\right\}^{\frac{1}{p_2}}\\
:= & \,\, C_1 + C_2.
\end{align*}

Assume first that $p_1\leq {\theta}_2$. On using the characterization of the boundedness of the operator $H^*$ in weighted Lebesgue spaces (see, for instance, \cites{ok,kp}), we arrive at
\begin{equation*}
C_1 {\approx}  \|{\omega}_1\|_{{\theta}_1,(0,\infty)}^{-1} \sup_{t\in (0,\infty)} \widetilde V(t)\,
\|{\omega}_2\|_{{\theta}_2,(t,\infty)}.
\end{equation*}

{\rm (i)} Let  ${\theta}_1\le {\theta}_2$. Applying \cite[Theorem 3.1, (i)]{gmp}, we obtain that
\begin{equation*}
C_2 {\approx} \sup_{x\in (0,\infty)} {\varphi}_2(x) \, \sup_{t\in (0,\infty)} {\mathcal V} (t,x) \,\|{\omega}_2\|_{{\theta}_2,(t,\infty)}.
\end{equation*}
Consequently, the proof is completed in this case.

{\rm (ii)} Let ${\theta}_2<{\theta}_1$. Using \cite[Theorem 3.1, (ii)]{gmp}, we have that
\begin{align*}
C_2 {\approx} \bigg( \int_0^{\infty} {\varphi}_2(x)^{\frac{{\theta}_1 {\rightarrow} {\theta}_2 \cdot {\theta}_1 {\rightarrow} p_2}{{\theta}_2 {\rightarrow} p_2}} \widetilde{V}(x)^{{\theta}_1 {\rightarrow} p_2}  \bigg(\sup_{t\in(0,\infty)}
{\mathcal V}(t,x) \|{\omega}_2\|_{{\theta}_2,(t,\infty)}\bigg)^{{\theta}_1 {\rightarrow} {\theta}_2}\, d\bigg( - \|{\omega}_1\|_{{\theta}_1,(0,x)}^{- {\theta}_1 {\rightarrow} p_2}\bigg) \bigg)^{\frac{1}{{\theta}_1 {\rightarrow} {\theta}_2}},
\end{align*}
and the statement follows in this case.

Let us now assume that ${\theta}_2 < p_1$. Then, using the characterization of the boundedness of the operator $H^*$ in weighted Lebesgue spaces, we have that
\begin{equation*}
C_1 {\approx} \|{\omega}_1\|_{{\theta}_1,(0,\infty)}^{-1} \bigg( \int_0^{\infty} \widetilde V(t)^{p_1 {\rightarrow} {\theta}_2} \,d\bigg( - \|{\omega}_2\|_{{\theta}_2, (t,\infty)}^{p_1 {\rightarrow} {\theta}_2}\bigg) \bigg)^{\frac{1}{p_1 {\rightarrow} {\theta}_2}}.
\end{equation*}

{\rm (iii)} Let ${\theta}_1\leq {\theta}_2$, then \cite[Theorem 3.1, (iii)]{gmp} yields that
\begin{align*}
C_2{\approx} \sup_{x\in (0,\infty)} {\varphi}_2(x) \bigg(\int_0^{\infty} {\mathcal V}(t,x)^{p_1 {\rightarrow} {\theta}_2} d \bigg( - \|{\omega}_2\|_{{\theta}_2, (t,\infty)}^{p_1 {\rightarrow}
	{\theta}_2}\bigg)\bigg)^{\frac{1}{p_1 {\rightarrow} {\theta}_2}},
\end{align*}
and these completes the proof in this case.

{\rm (iv)} If ${\theta}_2 < {\theta}_1$, then on using \cite[Theorem 3.1,
(iv)]{gmp}, we arrive at
\begin{align*}
C_2 {\approx}  \bigg( \int_0^\infty {\varphi}_2(x)^{\frac{{\theta}_1 {\rightarrow} {\theta}_2 \cdot {\theta}_1 {\rightarrow} p_2}{{\theta}_2 {\rightarrow} p_2}} \widetilde{V}(x)^{{\theta}_1 {\rightarrow} p_2} \bigg(\int_0^{\infty}{\mathcal V}(t,x)^{p_1 {\rightarrow} {\theta}_2} d \bigg(-\|{\omega}_2\|_{{\theta}_2,(t,\infty)}^{p_1 {\rightarrow} {\theta}_2} \bigg)
\bigg)^{\frac{{\theta}_1 {\rightarrow} {\theta}_2}{p_1 {\rightarrow} {\theta}_2}}
d\bigg(-\|{\omega}_1\|_{{\theta}_1,(0,x)}^{- {\theta}_1 {\rightarrow} p_2} \bigg)
\bigg)^{\frac{1}{{\theta}_1 {\rightarrow} {\theta}_2}},
\end{align*}
and in this case the proof is completed. 

\hspace{16.9cm}$\square$

\begin{rem}
	Assume that ${\varphi}_2(x) < \infty, ~ x > 0$. In view of
	Remark~\ref{nondegrem}, if
	$$
	\int_0^1 \bigg(\int_0^t  {\omega}_1^{{\theta}_1} \bigg)^{-\frac{{\theta}_1}{{\theta}_1 - p_2}}
	{\omega}_1^{{\theta}_1}(t) \,dt = \int_1^{\infty} \widetilde V(t)^{\frac{{\theta}_1 p_2}{{\theta}_1 - p_2}}
	\bigg(\int_0^t  {\omega}_1^{{\theta}_1} \bigg)^{-\frac{{\theta}_1}{{\theta}_1-p_2}} {\omega}_1^{{\theta}_1}(t) \,dt = \infty,
	$$
	then ${\varphi}_2 \in Q_{\widetilde V^{\frac{1}{p_1 {\rightarrow} p_2}}}$.
\end{rem}

\noindent{\bf Proof of Theorem \ref{maintheorem2}.}
By  Lemma \ref{mainlemma}, applying \cite[Theorem~4.2, (b)]{mu_emb},  we get that
\begin{align*}
\|{\operatorname{I}}\|_{{\,^{^{\bf c}}\!} LM_{p_1{\theta}_1,{\omega}_1}({{\mathbb R}^n},v_1){\rightarrow} LM_{p_2{\theta}_2,{\omega}_2}({{\mathbb R}^n},v_2)} = \left\{ \sup_{g\in {\mathfrak M}^+(0,\infty)} {\frac{\displaystyle {\sup_{t\in (0,\infty)} \|{\omega}_1\|_{{\theta}_1,(0,t)}^{-p} \|H^*g(|\cdot|)\|_{\infty,(v_1^{-1}v_2)^{p},B(0,t)}}}{\displaystyle {\|g\|_{\frac{{\theta}_2}{{\theta}_2-p},{\omega}_2^{-p},(0,\infty)}}}} \right\}^{\frac{1}{p}}.
\end{align*}

Recall that, whenever $F,G$ are non-negative measurable functions on $(0,\infty)$ and $F$ is non-increasing, then 
\begin{equation}\label{esupFG1}
{\operatornamewithlimits{ess\,sup}}_{t\in (0,\infty)} F(t)G(t) = {\operatornamewithlimits{ess\,sup}}_{t\in (0,\infty)} F(t) {\operatornamewithlimits{ess\,sup}}_{\tau\in (0,t)} G(\tau).
\end{equation}

Observe that 
\begin{equation}\label{polcorforsup}
\|H^*g(|\cdot|)\|_{\infty,(v_1^{-1}v_2)^p,B(0,t)} = \sup_{\tau\in(0,t)} \, \sup_{|y|=\tau} \big( v_1^{-1}(y) v_2(y)\big)^p H^* g (|y|)  = \|H^*g\|_{\infty,\tilde{\tilde {v}}, (0,t)}
\end{equation}
holds for all $t > 0$, where $\tilde{\tilde {v}}(\tau):= \big(\sup_{|y|=\tau} v_1^{-1}(y)v_2(y)\big)^p$, $\tau > 0$. 

On using \eqref{esupFG1}, we get that
\begin{align*}
\|{\operatorname{I}}\|_{{\,^{^{\bf c}}\!} LM_{p_1{\theta}_1,{\omega}_1}({{\mathbb R}^n},v_1){\rightarrow} LM_{p_2{\theta}_2,{\omega}_2}({{\mathbb R}^n},v_2)} & = \left\{ \sup_{g\in {\mathfrak M}^+(0,\infty)} {\frac{\displaystyle {\sup_{t\in (0,\infty)} \|{\omega}_1\|_{{\theta}_1,(0,t)}^{-p} \|H^*g\|_{\infty,\tilde{\tilde {v}}, (0,t)}}}{\displaystyle {\|g\|_{\frac{{\theta}_2}{{\theta}_2-p},{\omega}_2^{-p},(0,\infty)}}}} \right\}^{\frac{1}{p}} \\
& = \left\{ \sup_{g\in {\mathfrak M}^+(0,\infty)} {\frac{\displaystyle { \|H^*g\|_{\infty,\|{\omega}_1\|_{{\theta}_1,(0,\cdot)}^{-p} \tilde{\tilde{v}}(\cdot),(0,\infty)}
}}{\displaystyle {\|g\|_{\frac{{\theta}_2}{{\theta}_2-p},{\omega}_2^{-p},(0,\infty)}}}} \right\}^{\frac{1}{p}}.
\end{align*}

Using the characterization of the boundedness of $H^*$ in weighted Lebesgue spaces, we obtain that
\begin{align*}
\|{\operatorname{I}}\|_{{\,^{^{\bf c}}\!} LM_{p_1{\theta}_1,{\omega}_1}({{\mathbb R}^n},v_1){\rightarrow} LM_{p_2{\theta}_2,{\omega}_2}({{\mathbb R}^n},v_2)} & {\approx} \sup_{t\in (0,\infty)} \|{\omega}_2\|_{{\theta}_2,(t,\infty)} \bigg(\sup_{s\in(0,t)} \|{\omega}_1\|_{{\theta}_1,(0,s)}^{-1} \tilde{\tilde v}(s)^{\frac{1}{p}}\bigg) \\
& = \sup_{t\in (0,\infty)} \|{\omega}_2\|_{{\theta}_2,(t,\infty)}
\bigg(\sup_{s\in (0,t)} \sup_{|y|=s} \|{\omega}_1\|_{{\theta}_1,(0,|y|)}^{-1} v_1^{-1}(y) v_2(y)\bigg)\\
& = \sup_{t\in (0,\infty)} \|{\omega}_2\|_{{\theta}_2,(t,\infty)}
\bigg(\sup_{x\in B(0,t)} \|{\omega}_1\|_{{\theta}_1,(0,|x|)}^{-1} v_1^{-1}(x) v_2(x)\bigg)\\  
& = \sup_{t\in (0,\infty)} \|{\omega}_2\|_{{\theta}_2,(t,\infty)}
\bigg\| \|{\omega}_1\|_{{\theta}_1,(0,|\cdot|)}^{-1} \bigg\|_{\infty,v_1^{-1} v_2,B(0,t)}.
\end{align*}

\hspace{16.9cm}$\square$

\noindent{\bf Proof of Theorem \ref{maintheorem4}.}
By  Lemma \ref{mainlemma}, applying \cite[Theorem~4.2, (d)]{mu_emb}, and using \eqref{polcorforsup}, we get that
\begin{align*}
\|{\operatorname{I}}\|_{{\,^{^{\bf c}}\!} LM_{p_1{\theta}_1,{\omega}_1}({{\mathbb R}^n},v_1){\rightarrow} LM_{p_2{\theta}_2,{\omega}_2}({{\mathbb R}^n},v_2)} {\approx} & \,\, \|{\omega}_1\|_{{\theta}_1,(0,\infty)}^{-1} \left\{\sup_{g\in {\mathfrak M}^+(0,\infty)} {\frac{\displaystyle {\|H^*g\|_{\infty,\tilde{\tilde{v}},(0,\infty)}}}{\displaystyle {\|g\|_{\frac{{\theta}_2}{{\theta}_2-p},{\omega}_2^{-p},(0,\infty)}}}} \right\}^{\frac{1}{p}}\\
&  + \left\{\sup_{g\in {\mathfrak M}^+(0,\infty)} {\frac{\displaystyle {\bigg(\int_0^{\infty} \|H^*g\|_{\infty,\tilde{\tilde {v}}, (0,t)} ^{\frac{{\theta}_1}{{\theta}_1 - p}} d\bigg(-\|{\omega}_1\|_{{\theta}_1,(0,t)}^{-\frac{{\theta}_1 p}{{\theta}_1 - p}}\bigg)\bigg)^{\frac{{\theta}_1 - p}{{\theta}_1}}}}{\displaystyle {\|g\|_{\frac{{\theta}_2}{{\theta}_2 - p},{\omega}_2^{-p},(0,\infty)}}}}\right\}^{\frac{1}{p}}\\
:= & \,\, C_3 + C_4.
\end{align*}

Again, using the characterization of the boundedness of $H^*$ in weighted Lebesgue spaces, we obtain that
\begin{equation*}
C_3{\approx} \|{\omega}_1\|_{{\theta}_1,(0,\infty)}^{-1} \sup_{t\in (0,{\infty})} \widetilde V(t) \|{\omega}_2\|_{{\theta}_2,(t,\infty)}.
\end{equation*}

{\rm (i)} Let  ${\theta}_1 \le {\theta}_2$, then by \cite[Theorem 4.1]{gop}, we have that
\begin{align*}
C_4  {\approx} \sup_{x\in (0,\infty)} \bigg( \widetilde{V}(x)^{{\theta}_1{\rightarrow} p} \int_x^{\infty} \, d \bigg(-\|{\omega}_1\|_{{\theta}_1,(0,t)}^{-{\theta}_1{\rightarrow} p}\bigg) + \int_0^x  \widetilde{V}(t)^{{\theta}_1{\rightarrow} p} \, d\bigg(-\|{\omega}_1\|_{{\theta}_1,(0,t)}^{- {\theta}_1{\rightarrow} p}\bigg) \bigg)^{{\frac{1}{{\theta}_1{\rightarrow} p}}} \|{\omega}_2\|_{{\theta}_2,(x,\infty)},
\end{align*}
and the statement follows in this case.

{\rm (ii)} Let  ${\theta}_2 < {\theta}_1$, then \cite[Theorem 4.4]{gop} yields that
\begin{align*}
C_4  {\approx} & \,\, \bigg(\int_0^{\infty} \bigg(\int_x^{\infty} d\bigg(- \|{\omega}_1\|_{{\theta}_1,(0,t)}^{-{\theta}_1{\rightarrow} p}\bigg)\bigg)^{\frac{{\theta}_1{\rightarrow} {\theta}_2}{{\theta}_2 {\rightarrow} p}} \bigg(\sup_{0 < \tau \leq x} \widetilde V(\tau) \|{\omega}_2\|_{{\theta}_2,(\tau,\infty)}\bigg)^{{\theta}_1 {\rightarrow} {\theta}_2} d\bigg(-\|{\omega}_1\|_{{\theta}_1,(0,x)}^{-{\theta}_1 {\rightarrow} p}\bigg)\bigg)^{\frac{1}{{\theta}_1 {\rightarrow} {\theta}_2}} \\
& + \bigg(\int_0^{\infty} \bigg(\int_0^x \widetilde V(t)^{{\theta}_1{\rightarrow} p} d\bigg(-\|{\omega}_1\|_{{\theta}_1,(0,t)}^{-{\theta}_1{\rightarrow} p} \bigg) \bigg)^{\frac{{\theta}_1{\rightarrow} {\theta}_2}{{\theta}_2{\rightarrow} p}} \widetilde{V}(x)^{{\theta}_1{\rightarrow} p} \|{\omega}_2\|_{{\theta}_2,(t,\infty)}^{{\theta}_1{\rightarrow} {\theta}_2} d\bigg(-\|{\omega}_1\|_{{\theta}_1,(0,x)}^{-{\theta}_1{\rightarrow} p} \bigg)\bigg)^{\frac{1}{{\theta}_1{\rightarrow} {\theta}_2}},
\end{align*}
and the proof is completed in this case.

\hspace{16.9cm}$\square$

\

{\bf Acknowledgments.} The research of A. Gogatishvili was partially supported by the grant P201-13-14743S of the Grant Agency of the Czech Republic and RVO: 67985840 and  by Shota Rustaveli National Science Foundation grants no. DI/9/5-100/13 (Function spaces, weighted inequalities for integral operators and problems of summability of Fourier series).

\begin{bibdiv}
	\begin{biblist}
		
		\bib{batsaw}{article}{
			author={Batbold, Ts.},
			author={Sawano, Y.},
			title={Decompositions for local Morrey spaces},
			journal={Eurasian Math. J.},
			volume={5},
			date={2014},
			number={3},
			pages={9--45},
			issn={2077-9879},
		}
		
		\bib{Bur1}{article}{
			author={Burenkov, V.I.},
			title={Recent progress in studying the boundedness of classical operators
				of real analysis in general Morrey-type spaces. I},
			journal={Eurasian Math. J.},
			volume={3},
			date={2012},
			number={3},
			pages={11--32},
			issn={2077-9879},
		}
		
		\bib{Bur2}{article}{
			author={Burenkov, V. I.},
			title={Recent progress in studying the boundedness of classical operators
				of real analysis in general Morrey-type spaces. II},
			journal={Eurasian Math. J.},
			volume={4},
			date={2013},
			number={1},
			pages={21--45},
			issn={2077-9879},
		}
		
		\bib{BurGol}{article}{
			author={Burenkov, V. I.},
			author={Goldman, M. L.},
			title={Necessary and sufficient conditions for the boundedness of the
				maximal operator from Lebesgue spaces to Morrey-type spaces},
			journal={Math. Inequal. Appl.},
			volume={17},
			date={2014},
			number={2},
			pages={401--418},
			issn={1331-4343},
			doi={10.7153/mia-17-30},
		}
		
		\bib{BurHus1}{article}{
			author={Burenkov, V. I.},
			author={Guliyev, H. V.},
			title={Necessary and sufficient conditions for boundedness of the maximal
				operator in local Morrey-type spaces},
			journal={Studia Math.},
			volume={163},
			date={2004},
			number={2},
			pages={157--176},
			issn={0039-3223},
		}
		
		\bib{BurGulHus2}{article}{
			author={Burenkov, V. I.},
			author={Guliyev, H. V.},
			author={Guliyev, V. S.},
			title={Necessary and sufficient conditions for the boundedness of
				fractional maximal operators in local Morrey-type spaces},
			journal={J. Comput. Appl. Math.},
			volume={208},
			date={2007},
			number={1},
			pages={280--301},
			issn={0377-0427},
		}
		
		\bib{BurGulHus1}{article}{
			author={Burenkov, V.I.},
			author={Guliyev, H.V.},
			author={Guliyev, V.S.},
			title={On boundedness of the fractional maximal operator from
				complementary Morrey-type spaces to Morrey-type spaces},
			conference={
				title={The interaction of analysis and geometry},
			},
			book={
				series={Contemp. Math.},
				volume={424},
				publisher={Amer. Math. Soc.},
				place={Providence, RI},
			},
			date={2007},
			pages={17--32},
		}
		
		\bib{BurGulSerTar}{article}{
			author={Burenkov, V.I.},
			author={Guliyev, V.S.},
			author={Serbetci, A.},
			author={Tararykova, T.V.},
			title={Necessary and sufficient conditions for the boundedness of genuine
				singular integral operators in local Morrey-type spaces},
			journal={Eurasian Math. J.},
			volume={1},
			date={2010},
			number={1},
			pages={32--53},
			issn={2077-9879},
		}
		
		\bib{BGGM}{article}{
			author={Burenkov, V.I.},
			author={Gogatishvili, A.},
			author={Guliyev, V.S.},
			author={Mustafayev, R.Ch.},
			title={Boundedness of the fractional maximal operator in local
				Morrey-type spaces},
			journal={Complex Var. Elliptic Equ.},
			volume={55},
			date={2010},
			number={8-10},
			pages={739--758},
			issn={1747-6933},
		}
		
		\bib{BGGM1}{article}{
			author={Burenkov, V.I.},
			author={Gogatishvili, A.,}
			author={Guliyev, V.S.},
			author={Mustafayev, R.Ch.},
			title={Boundedness of the Riesz potential in local Morrey-type spaces},
			journal={Potential Anal.},
			volume={35},
			date={2011},
			number={1},
			pages={67--87},
			issn={0926-2601},
		}
		
		\bib{BurNur}{article}{
			author={Burenkov, V.I.},
			author={Nursultanov, E.D.},
			title={Description of interpolation spaces for local Morrey-type spaces},
			language={Russian, with Russian summary},
			journal={Tr. Mat. Inst. Steklova},
			volume={269},
			date={2010},
			number={Teoriya Funktsii i Differentsialnye Uravneniya},
			pages={52--62},
			issn={0371-9685},
			translation={
				journal={Proc. Steklov Inst. Math.},
				volume={269},
				date={2010},
				number={1},
				pages={46--56},
				issn={0081-5438},
			},
		}
		
		\bib{Folland}{book}{
			author={Folland, G. B.},
			title={Real analysis},
			series={Pure and Applied Mathematics (New York)},
			edition={2},
			note={Modern techniques and their applications;
				A Wiley-Interscience Publication},
			publisher={John Wiley \& Sons, Inc., New York},
			date={1999},
			pages={xvi+386},
			isbn={0-471-31716-0},
		}
		
		\bib{giltrud}{book}{
			author={Gilbarg, D.},
			author={Trudinger, N. S.},
			title={Elliptic partial differential equations of second order},
			edition={2},
			publisher={Springer-Verlag, Berlin},
			date={1983},
			pages={xiii+513},
			isbn={3-540-13025-X},
		}
		
		\bib{gmp}{article}{
			author={Gogatishvili, A.}, 
			author={Mustafayev, R. Ch.},
			author={Persson, L.-E.}, 
			title={Some new iterated Hardy-type
				inequalities}, 
			journal={J. Funct. Spaces Appl.}, 
			date={2012},
			pages={Art. ID 734194, 30}, }
		
		\bib{gop}{article}{
			author={Gogatishvili, A.}, author={Opic, B.}, author={Pick, L.},
			title={Weighted inequalities for Hardy-type operators involving
				suprema}, journal={Collect. Math.}, volume={57}, date={2006},
			number={3}, pages={227--255}, }
		
		\bib{GulDoc}{book}{
			author={Guliyev, V.S.},
			title={Integral operators on function spaces on the homogeneous groups and on domains in ${\mathbb R}^n$},
			publisher={Doctor's degree dissertation. Mat. Inst. Steklov.},
			place={Moscow},
			language={Russian},
			date={1994},
			pages={329 pp.},
		}
		
		\bib{kp}{book}{
			author={Kufner, A.}, author={Persson, L.-E.}, title={Weighted
				inequalities of Hardy type}, publisher={World Scientific Publishing
				Co. Inc.}, place={River Edge, NJ}, date={2003}, pages={xviii+357},
			isbn={981-238-195-3}, review={\MR{1982932 (2004c:42034)}}, }
		
		\bib{M1938}{article}{
			author={Morrey, C. B.},
			title={On the solutions of quasi-linear elliptic partial differential
				equations},
			journal={Trans. Amer. Math. Soc.},
			volume={43},
			date={1938},
			number={1},
			pages={126--166},
			issn={0002-9947},
			doi={10.2307/1989904},
		}
		
		\bib{mu_emb}{article}{
			author={Mustafayev, R. Ch.},
			author={{\"U}nver, T.},
			title={Embeddings between weighted local Morrey-type spaces and weighted
				Lebesgue spaces},
			journal={J. Math. Inequal.},
			volume={9},
			date={2015},
			number={1},
			pages={277--296},
			issn={1846-579X},
			doi={10.7153/jmi-09-24},
		}
		
		\bib{ok}{book}{
			author={Opic, B.}, author={Kufner, A.}, title={Hardy-type
				inequalities}, series={Pitman Research Notes in Mathematics Series},
			volume={219}, publisher={Longman Scientific \& Technical},
			place={Harlow}, date={1990}, pages={xii+333}, isbn={0-582-05198-3},
		}
		
		
	\end{biblist}
\end{bibdiv}

\end{document}


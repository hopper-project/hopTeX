\begin{document}

\thispagestyle{empty}

\hfill December 21, 1997
\bigskip\bigskip

\begin{center}

{\LARGE{\bf{Central Extensions of the families of
\\[0.3cm]
Quasi-unitary Lie algebras}}}
\end{center}

\bigskip

\begin{center}
F.J. HerranzEQIX0Q,
J.C. P\'erez BuenoEQIX1Q
and M. SantanderEQIX2Q
\end{center}

\begin{center}
{\it EQIX0Q Departamento de F\'{\i}sica, E.U. Polit\'ecnica \\
Universidad de Burgos, E--09006 Burgos, Spain}
\end{center}

\begin{center}
{\it EQIX1Q Departamento de F\'{\i}sica Te\'orica and IFIC \\
Centro Mixto Universidad de Valencia--CSIC \\
E--46100 Burjassot, Valencia, Spain}
\end{center}

\begin{center}
{\it EQIX3Q Departamento de F\'{\i}sica Te\'orica,
Universidad de Valladolid \\
E--47011, Valladolid, Spain}
\end{center}

\begin{abstract}
The most general possible central extensions of two whole families of  Lie
algebras, which can be obtained by contracting the special pseudo-unitary
algebras EQIX4Q of the  Cartan series EQIX5Q and the pseudo-unitary
algebras
EQIX6Q, are completely determined and classified for arbitrary EQIX7Q.
In addition to the EQIX4Q and EQIX6Q algebras, whose second cohomology
group is well known to be trivial, each family includes many non-semisimple
algebras; their central extensions, which are explicitly given, can be
classified into three types as far as their properties under contraction are
involved. A  closed expression for the dimension of the second cohomology group
of any member of these families of algebras is given.
\end{abstract}

\section{Introduction}

This paper investigates the Lie algebra cohomology of the unitary
Cayley--Klein (CK) families of Lie algebras
in any dimension.  These families, also called `quasi-unitary' algebras,
include both the special \mbox{(pseudo-)}unitary
EQIX4Q and \mbox{(pseudo-)}uni\-tary EQIX6Q algebras
---which have only trivial central
extensions\mbox{---,} as well as many other obtained from these  by a sequence
of contractions, which are no longer semisimple and may have
non-trivial central extensions.

The paper can be considered as a further step in a
series of studies on the CK families of Lie algebras. These
have both
mathematical interest and physical relevance. The families of CK
algebras provide a frame to describe the behaviour of mathematical
properties of algebras under contraction; in physical terms this is related to
some kind of approximation. The central extensions for the family of
quasi-orthogonal  algebras, also in the general situation and for any
dimension, have been determined in a previous paper \cite{Azc.Her.Bue.San:96}.
We refer to this work for references and for physical
motivations; we simply remark here that there are three main reasons behind the
interest in the second cohomology groups for Lie algebras. First, in any
quantum theory the relevant representations of any symmetry group are
projective instead of linear ones. Second, homogeneous symplectic manifolds
under a group appear as orbits of the coadjoint representation of either the
group itself or of a central extension. And third, quasi-invariant Lagrangians
are also directly linked to the central extensions of the group; these can be
related also to Wess--Zumino terms. In
addition to the references in
\cite{Azc.Her.Bue.San:96}, we may add that Wess--Zumino--Witten models leading
to central extensions have also been studied
(see \emph{e.g.}
\cite{Azc.Izq.Mac:90,Fig.Sta:94} and references therein).

The knowledge of the second cohomology
group for a Lie algebra relies on the general solution of a set of linear
equations, yet some general results allow to bypass the calculations in
special cases. For instance, the second cohomology group is trivial for
semisimple Lie algebras. But once a contraction is made, the semisimple
character disappears, and the contracted algebra might have non-trivial central
extensions. Instead of finding the general solution for the extension equations
on a case-by-case basis, our approach is to do these calculations for a
whole family including a large number of algebras simultaneously. This program
has been developed for the quasi-orthogonal algebras, and here we discuss the
`next' quasi-unitary case. There are two main advantages in this approach.
First, it allows to record, in a form easily retrievable, a large number of
results which can be needed in applications, both in mathematics and in
physics. This avoids at once and for all the case-by-case type computation of
the
central extensions of algebras included in the unitary families. And second, it
sheds some further light on the interrelations between cohomology and
contractions, by discussing in particular examples how and when a contraction
increases the cohomology of the algebra: central extensions can be classed into
three types, with different behaviour under contraction.

The section~\ref{sec.2} is devoted to the description of the  two families of
unitary CK algebras. We show how to obtain these as graded contractions of the
compact algebras EQIX8Q and
EQIX9Q, and we provide some details on their structure. It should be remarked
that the CK unitary algebras are associated to the complex hermitian spaces
with metrics of different signatures and to their contractions. In
section~\ref{sec.3} the general solution to the central extension problem for
these algebras is given; this includes the completely explicit description of
all possible central extensions and the discussion of their triviality. A
closed formula for the dimension of the second cohomology group is also
obtained. Computational details on the procedure to
solve the central extension problem are given in an Appendix. The results are
illustrated in section~\ref{sec.4} for the lowest dimensional examples.
Finally, some remarks close the paper.

\section{The CK families of quasi-unitary algebras}
\label{sec.2}

The family of special quasi-unitary algebras, which involves the simple
Lie algebras EQIX10Q, as well as many non-simple algebras obtained
by \.In\"on\"u--Wigner \cite{Ino.Wig:53} contraction from EQIX10Q can be
easily described in terms of graded contraction theory
\cite{Mon.Pat:91,Moo.Pat:91}, taking the compact real form
EQIX8Q of the simple algebras in the series
EQIX11Q as starting point. As it is well known, the special unitary algebra
can be realised by complex antihermitian and traceless matrices, and is the
quotient of the algebra of all complex antihermitian matrices by its
center (generated by the pure imaginary multiples of the identity). It
 will be convenient to consider the family of quasi-unitary algebras
altogether; these can be similarly described in terms of graded
contractions of EQIX9Q, and will include algebras obtained from EQIX12Q
by \.In\"on\"u--Wigner contractions. Let us consider the (fundamental)
matrix representation of the algebras
EQIX8Q and
EQIX9Q, as given by the complex matrices
EQIX13Q and EQIX14Q:
EQDS0Q
where EQIX15Q, EQIX16Q,\  EQIX17Q, and where EQIX18Q means
the EQIX19Q matrix with a single  1 entry in row EQIX20Q and
column EQIX21Q. The commutation relations involved in either of these
algebras are given by
EQDS1Q
EQDS2Q
EQDS3Q

The algebra EQIX8Q has a grading by a group
EQIX22Q
related to a set of EQIX23Q commuting involutions in the
subalgebra EQIX24Q generated by EQIX25Q
\cite{Her.Mon.Olm.San:94,Her.San:96b}.
If EQIX26Q denotes any subset of the set of indices EQIX27Q,
and EQIX28Q denotes the characteristic function over EQIX26Q,
then each of the linear mappings given by
EQDS4Q
is an involutive automorphism of the algebra EQIX8Q; by
considering all possible subsets of indices we get EQIX29Q different
automorphisms defining a EQIX22Q grading for this
algebra. The corresponding graded contractions of
EQIX8Q constitute a large set of Lie algebras, but there exists
a particular subset or family of these graded contractions, nearer
to the simple ones, which essentially
preserves the properties
associated to simplicity, and which belong to the so termed
\cite{Roz:88, Roz:97} `quasi-simple' algebras. This family, to be defined
below, encompasses the special pseudo-unitary algebras (in the EQIX11Q Cartan
series)  as well as their nearest non-simple contractions. By taking the
generator EQIX30Q as invariant under all involutions, this grading can
be extended to the algebra EQIX9Q, whose graded contractions include the
pseudo-unitary algebras as well as many non-semisimple algebras;  again a
particular family of these graded contractions, to be introduced below,
preserves properties associated to semi-simplicity.  Collectively, all these
algebras (special or not) are called
\emph{quasi-unitary}; these are also called Cayley--Klein algebras of unitary
type, or unitary CK algebras, since they are exactly those
algebras behind the geometries of a complex hermitian space  with a
projective metric in the CK sense \cite{Roz:97}. Another
view to these algebras is given in \cite{Gro.Man:90}.

The overall details on the structure of this family are similar
to the orthogonal case. The set of unitary CK algebras is parametrised by
EQIX23Q real coefficients EQIX31Q (EQIX32Q), whose values codify in a
convenient way the pertinent information on the Lie algebra structure
\cite{Her:95,Her.San:96}. In terms of the EQIX33Q
two-index coefficients
EQIX34Q defined by
EQDS5Q
which verify
EQDS6Q
the algebras to be denoted EQIX35Q and EQIX36Q,
EQIX37Q, of
dimensions EQIX38Q and EQIX39Q, are generated by  and EQIX14Q (EQIX15Q), with commutators:
EQDS7Q
EQDS8Q
EQDS9Q
where EQIX40Q and EQIX41Q; we assume
EQIX42Q for each set of three  indices  EQIX43Q, and EQIX44Q
for each set of four indices EQIX45Q which are also assumed to
be \emph{different}.

\subsection{The unitary CK groups}

The connection with groups of isometries of a hermitian metric is as
follows: for a generic choice, with \emph{all} EQIX46Q, let us
consider the space EQIX47Q endowed with a hermitian
(sesqui)linear form  associated to the matrix
EQDS10Q
this is, for any pair of vectors EQIX48Q,
EQDS11Q
Let us define the group  as the group of linear isometries of
the hermitian metric (\ref{suMetricMatrix}). The isometry
condition
EQDS12Q
implies for the matrix
EQIX49Q the condition
EQDS13Q
For the corresponding Lie algebra the above relation leads to
EQDS14Q
This Lie algebra is generated by
the complex matrices (cf. (\ref{usuRealMat}))
EQDS15Q
with EQIX15Q, EQIX16Q, \ EQIX17Q.

The group EQIX50Q is
defined similarly by adding the unimodularity condition EQIX51Q;
this leads for the Lie algebra to the condition , so the algebra EQIX35Q is generated by
EQIX52Q alone.

The action of the groups EQIX53Q and EQIX54Q in
EQIX47Q is not transitive, and the `sphere' with equation
EQDS16Q
is stable. For the action of
EQIX54Q, the isotropy subgroup of a reference point in this
sphere, say EQIX55Q, is easily shown to be isomorphic to
EQIX56Q, and the isotropy subgroup of the
\emph{ray} of a reference point is
EQIX57Q, locally isomorphic to . The quotient spaces
 are a family of  hermitian spaces which
includes examples with non-definite and/or
degenerate hermitian metrics; the CK scheme provides a common frame to discuss
all them jointly. The most familiar corresponds to EQIX58Q,
and depends on a single parameter EQIX59Q; when EQIX60Q or EQIX61Q these
are the usual elliptic or hyperbolic complex hermitian spaces of (holomorphic
constant) curvature EQIX62Q; when EQIX63Q we get
the `Euclidean' flat hermitian space (finite-dimensional Hilbert space).

When the constants EQIX31Q are allowed to vanish, the set of
isometries of the hermitian metric (\ref{suMetricMatrix}) is larger
than the group generated by the matrices
EQIX14Q. In this case, there are additional
geometric structures in EQIX47Q (related to the existence of
additional invariant foliations similar to the one implied by
(\ref{suSphere})), and the proper definition of the automorphism
group of these structures leads again to the group generated by the
matrix Lie algebra (\ref{CKusuRealMat}) with the commutation relations
 (\ref{CKusuCR})--(\ref{CKuCR}). These matrix
realisations can be considered as the fundamental representation of
the  unitary CK Lie algebras
EQIX35Q and EQIX36Q.

Since each coefficient EQIX31Q can be positive, negative or zero,
each unitary CK family comprises  EQIX64Q Lie algebras  although some of
them may be isomorphic. For instance, the map
EQDS17Q
provides an isomorphism
EQDS18Q

\subsection{Structure of the unitary CK algebras}

The  unitary CK algebras EQIX35Q contain many subalgebras
isomorphic to algebras in both families EQIX65Q and
EQIX66Q, EQIX67Q. To best describe this,  we introduce a new set
of Cartan subalgebra generators for EQIX35Q,
EQIX68Q (EQIX32Q), defined by
EQDS19Q
In the matrix realisation (\ref{CKusuRealMat}) EQIX68Q is given by
EQDS20Q
so each EQIX68Q appears as a direct sum of two blocks,
each proportional with a pure imaginary coefficient to the identity matrix.

Denoting by EQIX69Q the pair of generators
EQIX70Q, we can check that the set 
closes a Lie subalgebra
EQIX71Q.
Furthermore,
EQIX68Q commutes with all the generators in this subalgebra, so that
the former generators plus EQIX72Q close an algebra isomorphic to
EQIX73Q.

Similarly, the set  closes the  special unitary CK Lie algebra
EQIX74Q, and by adding EQIX75Q we
get an algebra isomorphic to EQIX76Q.

This structure can be visualised by arranging the basis generators
as in Fig.~\ref{fig2.1}.

\begin{figure}[ht]
\begin{center}
\begin{tabular}{ccccc|cccccc}
& EQIX77Q&EQIX78Q&EQIX79Q&EQIX80Q&
   EQIX81Q&EQIX82Q&EQIX79Q&EQIX79Q&EQIX83Q\\
EQIX84Q && EQIX85Q&EQIX79Q&EQIX86Q&
   EQIX87Q&EQIX88Q& EQIX79Q&EQIX79Q&EQIX89Q\\
& EQIX90Q &EQIX91Q&EQIX92Q&EQIX93Q&
   EQIX93Q&EQIX93Q&EQIX94Q&&EQIX93Q\\
& &EQIX94Q&EQIX94Q&EQIX95Q&  EQIX96Q&EQIX97Q&
  EQIX79Q&EQIX79Q&EQIX98Q\\
& &EQIX94Q& EQIX99Q &&  EQIX100Q&EQIX101Q&
   EQIX79Q&EQIX79Q&EQIX102Q\\
\cline{6-10}
& &EQIX94Q&\multicolumn{2}{c}{\,} EQIX103Q
   &&EQIX104Q& EQIX79Q&EQIX79Q&EQIX105Q\\
& &EQIX94Q&\multicolumn{2}{c}{\,}&  EQIX106Q&
  EQIX94Q& EQIX107Q&&EQIX93Q\\
& &EQIX94Q&\multicolumn{2}{c}{\,}&  EQIX94Q&
  EQIX94Q EQIX91Q&&EQIX108Q&EQIX109Q\\
& &EQIX94Q&\multicolumn{2}{c}{\,}&  EQIX94Q&
  EQIX94Q &EQIX110Q&&EQIX111Q\\
& &EQIX94Q&\multicolumn{2}{c}{\,}&  EQIX94Q& EQIX94Q &&EQIX112Q\\
\end{tabular}
\end{center}
\caption{Generators of the (special) unitary CK algebras}
\label{fig2.1}
\end{figure}

The special unitary subalgebras EQIX71Q  and
EQIX74Q correspond, in this order, to the
two triangles to the left and below the rectangle, both excluding the
generator EQIX68Q.  The unitary subalgebras
EQIX73Q  and EQIX76Q
correspond, in this order, to the two triangles to the left and
below the rectangle, both including the generator EQIX68Q.
This generator EQIX68Q closes a EQIX113Q subalgebra.

We sum up the details relative to the structure of the special unitary
CK algebras in two statements
\begin{itemize}
\item
When all EQIX31Q  are different from zero,
EQIX35Q is a  pseudo-unitary simple Lie algebra
EQIX4Q in the  Cartan series EQIX11Q (EQIX114Q and EQIX115Q are the number
of positive and negative signs in diagonal of the metric matrix
(\ref{suMetricMatrix}), EQIX116Q).

\item
If a coefficient EQIX31Q vanishes, the CK algebra is a
non-simple Lie algebra which has a semidirect structure
EQDS21Q
where the subalgebras appearing in (\ref{IsotropyDecom}) are
generated by
EQDS22Q
We note that EQIX117Q is an abelian subalgebra of dimension EQIX118Q. In
terms of the triangular arrangement of generators (Fig.~\ref{fig2.1}), EQIX117Q
is spanned by the generators inside the rectangle; we remark that these
generators do not close a subalgebra when EQIX119Q. The three remaining
sets are always subalgebras, no matter of whether or not EQIX120Q.
\end{itemize}

For the particular case EQIX63Q (or, \emph{mutatis mutandis}, EQIX121Q) the
contracted algebra is a quasi-unitary inhomogeneous algebra,
 EQDS23Q 
The subindex EQIX122Q in EQIX117Q denotes the real dimension of
EQIX123Q which can be identified with the space
EQIX124Q, with the
natural action of  EQIX125Q (locally isomorphic to ) over EQIX126Q. This
direct product appeared as the isotropy subalgebra of a ray for the natural
action of EQIX127Q on EQIX47Q discussed after
(\ref{suSphere}). In the case where
EQIX128Q are all different from zero, the algebra is an
ordinary inhomogeneous pseudo-unitary (not special) algebra:
 EQDS24Q 
and in this case
EQIX129Q can be identified to the
EQIX23Q-dimensional flat complex hermitian space with signature
EQIX130Q determined as the number of positive and negative terms in the sequence
EQIX131Q.

When several coefficients EQIX31Q are equal to zero the algebra
EQIX132Q has simultaneously several such
decompositions. The more contracted case corresponds
to taking all EQIX31Q equal to zero; this gives rise to the  special
unitary flag algebra.

\section{Central extensions}
\label{sec.3}

Now we proceed to compute in a unified way all the central extensions
for the two unitary families of CK algebras, for arbitrary choices of
the constants EQIX31Q and  in any dimension. Let EQIX133Q  be an arbitrary
EQIX134Q-dimensional Lie algebra with generators EQIX135Q and
structure constants
EQIX136Q. A central extension   EQIX137Q of the algebra EQIX133Q
by the one-dimensional algebra generated by
 EQIX138Q will have EQIX139Q generators
EQIX140Q with commutation relations given by
EQDS25Q
The   extension coefficients or  central charges
EQIX141Q  must be antisymmetric in the indices EQIX142Q,
EQIX143Q and must fulfil the following conditions
coming from the Jacobi identities in the extended Lie algebra:
EQDS26Q

These  extension coefficients  are the coordinates
EQIX144Q of the antisymmetric
two-tensor EQIX145Q which is the two-cocycle of the specific extension
being considered, and  (\ref{JacobiEqs}) is
the two-cocycle condition for the Lie algebra cohomology.

Let us consider  the `abstract' extended Lie algebra EQIX137Q
with the  Lie brackets (\ref{CentExt}) and let us perform a change of
generators:
EQDS27Q
where EQIX146Q are arbitrary real
numbers. The commutation rules
for the generators EQIX147Q become
EQDS28Q
Thus, the general expression for the two-coboundary EQIX148Q
generated  by EQIX149Q is
EQDS29Q
Two two-cocycles differing by a two-coboundary lead to equivalent
extensions; the classes of equivalence of non-trivial two-cocycles
associated with the tensors EQIX145Q determine the second cohomology group
EQIX150Q.

\subsection{The  general solution to the extension problem for the
 unitary CK algebras}

In a previous paper \cite{Azc.Her.Bue.San:96}
we have given the general solution to
the extension equations for the case of the  orthogonal CK algebras.
The same approach can be used for the   family of
quasi-unitary algebras. However, and in order not to burden the
exposition, the main details on the procedure have been placed in the Appendix.
The results obtained there give the  general solution to the problem of
finding the central extensions for the unitary CK
algebras. They are summed up as

\begin{theorem}
\label{theor3.1}
The most general central extension EQIX151Q of
any algebra in the family of special unitary CK algebras
EQIX152Q is determined by the following \emph{basic}
coefficients:
\begin{description}
\item[Type I.]
EQIX33Q basic extension coefficients EQIX153Q and
EQIX33Q basic extension coefficients EQIX154Q  (EQIX15Q,
EQIX155Q). These coefficients are not subjected to any
further relationship.

\item[Type II.] EQIX23Q basic extension coefficients EQIX156Q
(EQIX157Q), not subjected to any further relationship.

\item[Type III.] EQIX158Q basic extension coefficients EQIX159Q
(EQIX160Q, EQIX161Q) which must satisfy the conditions
EQDS30Q
\end{description}
\end{theorem}

\begin{theorem}
\label{theor3.2}
The most general central extension EQIX162Q of
any algebra in the unitary CK family
EQIX163Q, is determined by the basic extension
coefficients given in Theorem~\ref{theor3.1}, and by an additional set of

\begin{description}
\item[Type III.]
EQIX23Q basic extension coefficients EQIX164Q
(EQIX157Q), subjected to the relation
EQDS31Q
\end{description}
\end{theorem}
For any given choice of the constants EQIX31Q, these basic extension
coefficients determine two-cocycles for the algebras
EQIX152Q and
EQIX163Q. The Lie brackets of the extended algebras
EQIX151Q and EQIX162Q are
given by
EQDS32Q
EQDS33Q
where EQIX42Q, EQIX160Q, EQIX165Q  and EQIX166Q are all different.

The complete expression for the two-cocycles for EQIX152Q and
EQIX163Q can be read directly
from these commutators; for future convenience, we collect some
expressions relating the basic extension coefficients with
particular values of the two-cocycles determining the extensions
(however, and as it can be seen in (\ref{CKusuGenExtCR}), most of
these basic coefficients appears related to the values of the
cocycle in several ways)
EQDS34Q
EQDS35Q

\subsection{Equivalence of extensions}

According to the general discussion in the beginning of this section,
we now look for the more general coboundary for EQIX152Q or
EQIX163Q. We write a change of basis (see (\ref{ChangeGens}))
for the generators as
EQDS36Q
EQDS37Q
where EQIX167Q are the values
of EQIX149Q on the generators EQIX168Q.
By using (\ref{Cobound}) and the structure constants of the algebras
EQIX152Q or EQIX163Q
read from (\ref{CKusuCR})--(\ref{CKuCR}),
we find for the associated coboundaries EQIX148Q,
EQDS38Q
EQDS39Q
We shall not need the remaining values of the coboundaries EQIX148Q for
EQIX152Q or EQIX163Q;  each
EQIX148Q being a two-cocycle, it must necessarily appear as a
particular case of the most general two-cocycles which are completely
determined by the basic extension coefficients (\ref{suBasicExtCoef}).

The question of whether a general two-cocycle for a CK
algebra in Theorem~\ref{theor3.1} defines a trivial extension
amounts to checking whether it is a coboundary, which will allow to
eliminate the central EQIX138Q term from (\ref{CKusuGenExtCR}). This
may depend on the values of the constants EQIX31Q. In fact, the
three types of extensions  behave in three different ways, which
mimics the pattern found in the orthogonal case \cite{Azc.Her.Bue.San:96}:

\begin{itemize}
\item
Type I extensions can be done for all  unitary CK
algebras, since there is not any EQIX31Q-dependent
restriction to the basic Type I coefficients
EQIX169Q. However, as seen in
(\ref{CKusuCobound}), these extensions are \emph{always} trivial. A
considerable simplification of all expressions can be gained
if these trivial extensions are simply discarded, as we shall
do from now on. Hence for the \emph{extended}
algebra, the whole block of commutation
relations in (\ref{CKusuCR}) will hold and only those commutators in
(\ref{CKsuCR}) or
(\ref{CKuCR}) may change.

\item
Type II extensions appear also in all  unitary CK
algebras, as there is not any EQIX31Q-dependent restriction to the
basic Type II coefficients
EQIX156Q. The triviality of these extensions is
EQIX31Q-dependent, and (\ref{CKusuCobound}) shows that the extension
determined by the  coefficient EQIX156Q is
non-trivial if EQIX170Q, and trivial otherwise. It is within
this type of extensions that a \emph{pseudoextension} (trivial
extension by a two-coboundary) may become a non-trivial extension
by contraction
\cite{Ald.Azc:85b,Azc.Izq:95}.

\item
Type III extensions behave in a completely different way.
Due to the additional conditions (\ref{ECtaRestrict}) and
(\ref{urestrict}) that Type III extension coefficients must fulfil, some
of them might be necessarily equal to zero. Hence, these extensions do not
exist for all  unitary CK algebras. But those allowed (one
EQIX159Q for each pair of vanishing constants  and for the (non-special)  unitary case one additional EQIX164Q
for each vanishing constant EQIX171Q) are
always non-trivial, as the last equation in (\ref{CKusuCobound}) and
(\ref{CKuCobound}) show. Therefore, Type III extensions  do
not appear through the pseudoextension mechanism.
\end{itemize}

\subsection{The second cohomology groups of the unitary CK algebras}
\label{sec4}

If we disregard Type I extensions, which are trivial for all members in
the two CK families of  unitary algebras, the above  results can be
summarised in the following

\begin{theorem}
\label{theor4.1}
The  commutation relations of any central extension
EQIX151Q of the special unitary CK algebra
EQIX152Q can be written as the commutation relations in
(\ref{CKusuCR}), together with:
EQDS40Q
which will replace those in (\ref{CKsuCR}). The extension is
completely characterised by

\begin{itemize}
\item EQIX23Q Type II coefficients EQIX172Q (EQIX157Q).
Each of them gives rise to a non-trivial extension if
EQIX173Q and to a trivial one otherwise.

\item EQIX158Q Type III extension coefficients EQIX159Q
(EQIX160Q and
EQIX161Q), satisfying
EQDS30Q
Thus, EQIX159Q must be equal to zero when at least one of the
constants EQIX174Q is different from zero. When EQIX159Q is
non-zero, the extension that it determines is always non-trivial.
\end{itemize}
\end{theorem}

\begin{theorem}
\label{theor4.2}
The  commutation relations of any central extension
EQIX162Q of the unitary CK algebra
EQIX163Q can be written as the commutation relations in
the preceding statement, together with
EQDS41Q
which will replace those in (\ref{CKuCR}). In addition to the
extension coefficients EQIX172Q and EQIX159Q, the
extension is completely characterised by
\begin{itemize}
\item
EQIX23Q Type III coefficients EQIX175Q (EQIX157Q) satisfying
EQDS31Q
When EQIX176Q is non-zero, the extension that it determines is non-trivial.
\end{itemize}
\end{theorem}

All Type II extensions come from the pseudocohomology mechanism
\cite{Ald.Azc:85b,Azc.Izq:95}. We can write (\ref{CKsuExtCR}) as
EQDS42Q
which is well defined even if any of the EQIX177Q
(EQIX178Q) is equal to zero.
This clearly shows that when a given
EQIX177Q  is different from zero, the extension
coefficient
EQIX179Q gives rise to a trivial extension, which can be
removed by the one-cochain
EQIX180Q (all other coordinates of the
one-cochain being zero). However, when EQIX177Q goes to zero, the
corresponding extension is non-trivial, as the cochain  defined above
diverges, but the term EQIX181Q
in (\ref{TypeIIPseudoExt}) does not.

In terms of the triangular arrangement for the generators of
EQIX152Q (see Fig.~\ref{fig2.1}), it is also  worth remarking
that   Type III extensions only affect the commutators of the Cartan
generators in the outermost `EQIX182Q' diagonal, while the Type~II extension
EQIX183Q  only modifies the commutators of each those pairs
EQIX184Q with EQIX185Q, {\it i.e.} those pairs
contained inside a rectangle
with left-down corner EQIX186Q.

As a by-product of these results we can give closed expressions for
the dimension of the second cohomology group of any Lie algebra in
the unitary CK families.

\begin{proposition}
\label{prop4.1}
Let EQIX152Q or EQIX163Q
be a Lie algebra belonging
to a family of unitary CK algebras, and let EQIX187Q be the
number of coefficients EQIX188Q equal to zero. The dimension of
its second cohomology group is given by
EQDS43Q
EQDS44Q
\end{proposition}

The first term EQIX187Q in the sum of (\ref{dimH}), (\ref{uec}) corresponds to the
central extensions EQIX172Q, the second term  to the EQIX159Q and the third term EQIX187Q in (\ref{uec})
to the central extensions EQIX176Q. We recall that the analogous
expression for the quasi-orthogonal case is far more complicated,
and depends not only on the number of constants equal to zero, but
also on the detailed arrangement of zeros in the sequence  \cite{Azc.Her.Bue.San:96}.

As expected for the simple
EQIX4Q or the semisimple EQIX6Q algebras, which appear within the
two unitary CK families when all
EQIX189Q, the second cohomology group is trivial. The inhomogeneous
EQIX190Q algebras, appearing in the special unitary family when either
EQIX63Q or
EQIX121Q, with all other constants , have, in any dimension, a single non-trivial extension:
EQIX191Q when EQIX63Q or EQIX192Q if EQIX121Q.
The  special unitary flag algebra (when all EQIX193Q) has the maximum
number of non-trivial extensions within the special unitary family, that
is, EQIX194Q.

\section{Examples}
\label{sec.4}

Let us illustrate the general results of the above section for the
EQIX152Q algebras in the three lowest dimensional cases,
EQIX195Q. A completely similar discussion can be performed for the
EQIX163Q algebras.

\subsection{EQIX196Q }

We simply mention this example for the sake of completeness. The
results for the extensions of EQIX197Q could be also obtained
from those in
\cite{Azc.Her.Bue.San:96}
by using the isomorphism  provided by
EQIX198Q,
EQIX199Q,
EQIX200Q.
The most general extension is defined
by the extension coefficient EQIX191Q
and the non-zero Lie brackets
EQDS45Q
The extension is non-trivial for EQIX201Q and trivial otherwise, the
triviality being exhibited by the redefinition
EQDS46Q

\subsection{EQIX202Q }

The most general extended  special unitary CK algebra
EQIX202Q  has nine generators
, and it is determined by three possible extension
coefficients EQIX203Q, with
EQIX204Q.  Their  commutators are:
EQDS47Q
EQDS48Q

The triviality of Type II extensions is governed by the values of the
constants EQIX205Q. We analyse this problem for each specific CK
algebra within EQIX202Q. The extension
determined by EQIX191Q is trivial when EQIX206Q, and the
extension determined by EQIX207Q is trivial when EQIX208Q, the
triviality being exhibited by the redefinitions
EQDS49Q
Thus, EQIX209Q  is equal to

\begin{itemize}
\item
0 when both EQIX210Q. Here both
EQIX211Q produce trivial extensions, and EQIX212Q
must vanish. This case corresponds to the extensions of EQIX213Q for
EQIX214Q,  and  EQIX215Q for
EQIX216Q and the result is in agreement
with Whitehead's lemma, according to which simple algebras have no
non-trivial extensions.

\item
1 for the inhomogeneous unitary algebras EQIX217Q and
EQIX218Q. These algebras appear twice in the CK family, namely
for EQIX219Q and for EQIX220Q. In the
first case the only non-trivial extension coefficient is
EQIX191Q and the extended Lie brackets (\ref{CKsuiiiCR}) reduce to
EQDS50Q
The second case  is related to the former one due to the isomorphism
(\ref{PolarityIsom}). Here there is a single non-trivial extension
coefficient EQIX207Q and the extended Lie brackets are
EQDS51Q

\item
3 for the special unitary flag algebra EQIX221Q
when EQIX222Q. The three extensions are  non-trivial
EQDS52Q
\end{itemize}

\subsection{EQIX223Q }

We consider now the extensions EQIX224Q of
the CK algebra EQIX225Q. There are six possible
basic extension coefficients, , which must satisfy the
conditions
EQDS53Q
and the Lie brackets of the extension are given by the non-extended
ones in (\ref{CKusuCR}) and by the extended ones
EQDS54Q

The results for each one of the 27 CK algebras
EQIX224Q are displayed in Table~\ref{table4.1}.
The columns in this Table show, in this order, the number of coefficients
EQIX31Q set equal to zero (number of contractions), the centrally
extended Lie algebras, the signs EQIX226Q of each
coefficient EQIX227Q  together with the  non-trivial central
extensions  allowed for the algebra with these signs for the coefficients,
and finally, the dimension of the second cohomology group as a sum of
the number of non-trivial extensions of Types II and III, coming
respectively from the coefficients EQIX156Q and EQIX159Q. In the
table EQIX228Q (EQIX229Q) denotes a positive (negative) EQIX31Q coefficient which could be
rescaled to EQIX230Q (EQIX231Q).

\begin{table}
\begin{center}
\begin{tabular}{|l|l|l|l|}
\hline
\multicolumn{1}{|c|}{EQIX232Q\#EQIX232Q} &\multicolumn{1}{c|}{Extended
algebra}& \multicolumn{1}{c|}{(CK constants) [Non-trivial
extensions]EQIX232Q}& \multicolumn{1}{c|}{EQIX232QdimEQIX233Q}\\[0.1cm]
\hline
0&EQIX234Q&EQIX235Q&  0\\[0.1cm]
 &EQIX236Q&EQIX237Q   & \cr
 &EQIX238Q &EQIX239Q& \\[0.1cm]
\hline
1 &EQIX240Q&(EQIX241Q) EQIX242Q
or (EQIX243Q) EQIX244Q  & 1+0\\[0.1cm]
  &EQIX245Q&EQIX246Q  EQIX242Q
or&  \\[0.1cm]
 & &EQIX247Q  EQIX244Q&  \\[0.1cm]
  &EQIX248Q&EQIX249Q
EQIX250Q &  \\[0.1cm]
  &EQIX251Q&EQIX252Q
EQIX250Q  &  \\[0.1cm]
  &EQIX253Q&EQIX254Q
EQIX250Q &  \\[0.1cm]
\hline
2 & &(EQIX255Q) EQIX256Q
or (EQIX257Q) EQIX258Q   & 2+1\\[0.1cm]
  & &(EQIX259Q) EQIX256Q
or (EQIX260Q) EQIX258Q   &  \\[0.1cm]
 & &(EQIX261Q) EQIX262Q &  \\[0.1cm]
 & &(EQIX263Q) EQIX262Q &  \\[0.1cm]
\hline
3 &Flag algebra&(EQIX264Q)     & 3+3\\[0.1cm]
\hline
\end{tabular}
\end{center}
\caption{Non-trivial central extensions
EQIX224Q.}
\label{table4.1}
\end{table}

\section{Conclusions and outlook}

We restrict here to a couple of remarks. First, the pattern of three
types of extensions behaving under contractions in three different ways,
first found for the quasi-orthogonal family \cite{Azc.Her.Bue.San:96}, appears
also in the quasi-unitary
case. This seems likely to be a general phenomenon, not restricted to a single
family of contractions of some Lie algebras. The analysis of the extensions for
the
third CK main series of algebras, which embraces the symplectic
EQIX265Q in
the EQIX266Q series and their contractions, would be required to complete the
study of the relationships between cohomology and contractions undertaken
in \cite{Azc.Her.Bue.San:96} and continued in this paper.
These algebras can be adequately realised by quaternionic antihermitian
matrices, or, alternatively, by quaternionic antihermitian traceless matrices
plus
the Lie algebra of derivations of the quaternion
division algebra. Work in this area is in progress. Second, as compared to the
quasi-orthogonal case, the
quasi-unitary algebras have a comparatively smaller set of
extensions, whose description in terms of the values taken by the CK constants
EQIX31Q is straightforward. The suitability of a CK approach to the study of the
central extensions of a complete family is therefore put forward more clearly
than in the orthogonal case.
While the ordinary inhomogeneous orthogonal algebras
EQIX267Q associated to the real orthogonal EQIX268Q
dimensional flat spaces have
non-trivial extensions only in the case EQIX269Q, the algebras EQIX270Q
associated to
the complex pseudo-Euclidean hermitian flat spaces have a
single non-trivial extension, in any dimension. The
relevance of this fact in relation with the classical limit of quantum
mechanics
will be discussed elsewhere.

\section*{Acknowledgements}
The authors wish to acknowledge J. A. de Azc\'arraga for his comments on the
manuscript.
This research has been partially supported by the Spanish DGES
projects PB96--0756, PB94--1115
from the Ministerio de Educaci\'on y Cultura and by Junta de Castilla y
Le\'on (Projects CO1/396 and CO2/297). J.C.P.B.
wishes to thank an FPI grant from the Ministerio de Educaci\'on y
Cultura and the CSIC.

\section*{Appendix: The general solution to the Jacobi identities}
\setcounter{equation}{0}

In order to get the general solution of the set of linear equations
determining the possible extensions of the  unitary CK
algebras, we first introduce a suitable notation for the central
extension coefficients, which is `adapted' to the structure of the
algebras EQIX35Q  (\ref{CKusuCR})--(\ref{CKsuCR})  and
EQIX36Q (\ref{CKusuCR})--(\ref{CKuCR})
whose basic generators come  naturally divided in either three
or four `kinds' EQIX168Q. The symbol
corresponding to EQIX271Q will have one or two letters taken
from EQIX272Q, determined by the kind of the basis generators . To this symbol we append two groups of indices, each coming from
those of the corresponding generators.  The complete list of all
extension coefficients as written in this notation is
EQDS55Q
where we implicitly assume . We remark that  are single, unbreakable symbols, and are not products. In the
course of the derivation we will find useful to sort these
coefficients into several subsets, as follows

\begin{itemize}
\item Coefficients
EQIX273Q
involving \emph{four} different indices. If we write these four indices
as EQIX274Q the coefficients are
EQDS56Q
\item Coefficients
EQIX275Q involving
\emph{three} different indices. If we write the three indices as
EQIX42Q these coefficients are
EQDS57Q
\item Coefficients EQIX276Q
involving \emph{two} different indices
EQDS58Q
\item Coefficients EQIX277Q with \emph{two}
different indices EQIX15Q and a third index EQIX278Q
EQDS59Q
\item Coefficients EQIX279Q with \emph{two}
different indices EQIX15Q and a third index EQIX280Q
EQDS60Q
\item Coefficients EQIX281Q with two different indices EQIX160Q
EQDS61Q
\item Coefficients EQIX282Q and EQIX283Q with \emph{two}
different indices EQIX15Q
EQDS62Q
\item Coefficients EQIX284Q with a \emph{single} index
EQDS63Q
\end{itemize}
The Lie brackets of the extended CK algebra
EQIX151Q  and
EQIX162Q   read
EQDS64Q
EQDS65Q
EQDS66Q
EQDS67Q
where as indicated before, the relations EQIX42Q, EQIX285Q, EQIX286Q,   ,
EQIX160Q for the indices EQIX287Q
and EQIX288Q are all different, will be assumed without saying.

Our strategy here will be to enforce the complete set of Jacobi
identities  first for EQIX35Q and then for EQIX36Q, in
a carefully selected order which actually allows to explicitly
solve the rather large set of linear equations. The first stage
will be to identify many extension coefficients which are forced to
vanish; the remaining Jacobi equations will drastically simplify
and will either produce relations allowing to express certain
\emph{derived} extension coefficients in terms of the so-called
\emph{basic} ones, or further relations to be satisfied by the basic
extension coefficients.

To begin with, we show that \emph{all} coefficients in
(\ref{ECa}) vanish.  Denoting by EQIX289Q the Jacobi
identity for the generators EQIX290Q,
EQIX291Q and EQIX292Q, we display several choices for them
and the equations ensuing from these choices:
EQDS68Q
EQDS69Q
EQDS70Q
By substituting (\ref{JEa}) and (\ref{JEb}) in (\ref{JEc}), we find
that \emph{all} coefficients in (\ref{ECa})  are necessarily equal to
zero. From now on, substitution of the already known information in
further equations will be automatically assumed.

The coefficients  in (\ref{ECe}) turn out to be also equal to zero:
EQDS71Q

Now we look for equations involving the  coefficients EQIX281Q
in (\ref{ECf}). We find:
EQDS72Q
so the EQIX158Q coefficients of the type EQIX281Q might be
different from zero. We denote them as
EQDS73Q
and from  (\ref{JEf}) they must fulfil two additional conditions
EQDS30Q

We now look for Jacobi identities involving the  extension
coefficients in (\ref{ECd}):
EQDS74Q
which hold no matter of either EQIX293Q or EQIX294Q. These
equations show that
EQDS75Q
and therefore these coefficients  only depend on the first pair of
indices. These common values must be considered as another set of
\emph{basic} coefficients
EQDS76Q

Now we consider Jacobi identities leading to equations which involve
the coefficients in  (\ref{ECb}), those  with \emph{three} different indices. This
is the most tedious part of the process, due to the need of paying
minute attention to the index ranges.  Let us first look for
equations involving the coefficients with indices EQIX295Q, which
appear in the middle line of (\ref{ECb})
EQDS77Q
EQDS78Q
so in all cases, and no matter on the value of the middle index EQIX21Q,
we have
EQDS79Q

For the coefficients in the first line of (\ref{ECb}) we obtain that
EQDS80Q
EQDS81Q
These equations are summarised in
EQDS82Q
so again these are derived extension coefficients, expressible in
terms of EQIX296Q and EQIX297Q.

For the coefficients in the third line of (\ref{ECb}) with indices
EQIX298Q we get
EQDS83Q
EQDS84Q
These equations lead to
EQDS85Q
so these coefficients are also derived.

Finally we look for equations involving the coefficients in
(\ref{ECc}), this is EQIX299Q. Whenever there exists an index
EQIX21Q between EQIX20Q and EQIX300Q, the choice
EQDS86Q
leads to an expression for EQIX299Q in terms of EQIX276Q and
EQIX301Q. By iterating while possible, we find that the
coefficients   EQIX299Q with EQIX20Q and EQIX300Q not contiguous can be
written in terms of EQIX276Q with EQIX20Q and EQIX21Q contiguous. These
must be considered as basic ones
EQDS87Q
and the remaining coefficients in (\ref{ECc}) are given, recalling that
EQIX302Q, by
EQDS88Q

As far as EQIX35Q is concerned, the final step in this
process is to ascertain that there is no  any relation for the
extensions coefficients further to the ones yet considered. It can
be checked that \emph{all} remaining Jacobi equations
involving the generators EQIX52Q are
identically satisfied, so the process has indeed terminated.

Now we deal with the EQIX36Q case; as Jacobi equations
involving EQIX52Q have been already considered,
we must take into account only the extra generator EQIX30Q and the
associated extension coefficients. For these, successively we obtain
EQDS89Q
so the extension coefficients in (\ref{ECg}) are equal to zero,  and those in
(\ref{ECh}) are basic, to be denoted as
EQDS90Q
and must satisfy
EQDS31Q

Again in this case, it is easy to check that all  remaining Jacobi
equations involving the generator EQIX30Q are satisfied and  do not lead
to any further relation.

\begin{thebibliography}{30}

\bibitem{Azc.Her.Bue.San:96}
de Azc\'arraga J A,   Herranz F J,   P\'erez Bueno J C  and
Santander M  {Central extensions of the quasi-orthogonal Lie algebras}
{\em J. Phys. A: Math. Gen.} to appear

\bibitem{Azc.Izq.Mac:90}
de Azc\'arraga J A, Izquierdo J M, Macfarlane A J 1990
{\em Ann. Phys. (N.Y.)} {\bf 202} 1

\bibitem{Fig.Sta:94}
Figueroa-O'Farrill J M and Stanciu S 1994 {\em Phys. Lett.} {\bf B327} 40

\bibitem{Ino.Wig:53}
\.In\"on\"u  E  and   Wigner E P 1953 {\em Proc. Natl.
Acad. Sci., USA}  {\bf 39}  510

\bibitem{Mon.Pat:91}
   de Montigny M  and     Patera J 1991
{\em J. Phys. A: Math. Gen.} {\bf 24}   525

\bibitem{Moo.Pat:91}
 Moody  R V  and    Patera J 1991
{\em J. Phys. A: Math. Gen.}  {\bf 24}  2227

\bibitem{Her.Mon.Olm.San:94}
  Herranz F J,   de Montigny M,   del Olmo M A   and   Santander M
1994 {\em J. Phys. A: Math. Gen.}  {\bf 27}  2515

\bibitem{Her.San:96b}
  Herranz F J and   Santander M 1996
{\em J. Phys. A: Math. Gen.}  {\bf 29} 6643

\bibitem{Roz:88}
    Rozenfel'd B A 1988
{\em A History of Non-Euclidean Geometry} (New York: Springer)

\bibitem{Roz:97}
    Rozenfel'd B A 1997
{\em The Geometry of Lie Groups} (New York: Kluwer)

\bibitem{Gro.Man:90}
  Gromov N A and   Man'ko V I 1990 {\em J. Math. Phys.} {\bf 31}  1054

\bibitem{Her:95}
  Herranz F J 1995 {\em PhD Thesis}  Universidad de Valladolid

\bibitem{Her.San:96}
  Herranz F J and  Santander M 1996
{Cayley--Klein  schemes for all quasisimple real Lie algebras
  and Freudhental magic squares}
{\em Group  21: Physical Applications and Mathematical Aspects of Geometry,
Groups, and  Algebra} ed  H D  Doebner, P Nattermann  and W Scherer
  vol. I  pp 151  (Singapore: World Scientific)

\bibitem{Ald.Azc:85b}
 Aldaya V and   de Azc\'arraga J A 1985
{\em Int. J. of Theor. Phys.} {\bf 24}  141

\bibitem{Azc.Izq:95}
  de Azc\'arraga J A and  Izquierdo J M 1995
{\em Lie groups, Lie algebras, cohomology and some applications in
  physics}   (Cambridge: Cambridge  University Press)

\end{thebibliography}
\end{document}
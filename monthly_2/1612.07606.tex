\documentclass{amsart}
\usepackage{amsmath, amssymb, amsthm, amscd, amsfonts, eucal, hyperref}
\usepackage[all]{xy}
\usepackage{enumerate}
\usepackage[T5, T1]{fontenc}
\usepackage{color, charter}

\newtheorem{main}{Theorem}
\newtheorem{theorem}{Theorem}[section]
\newtheorem{proposition}[theorem]{Proposition}
\newtheorem{lemma}[theorem]{Lemma}
\newtheorem{notation}[theorem]{Notation}
\newtheorem {corollary}[theorem]{Corollary}
\theoremstyle {definition}
\newtheorem {definition}[theorem]{Definition}
\newtheorem {example}[theorem]{Example}
\newtheorem {question}[theorem]{Question}
\newtheorem {problem}[theorem]{Problem}
\newtheorem {conjecture}[theorem]{Conjecture}
\theoremstyle {remark}
\newtheorem{remark}[theorem]{Remark}

\begin{document}

\title[Saturated Hilbert polynomial of ideals in local rings]{Saturated Hilbert polynomial of ideals in local rings}

\author{\fontencoding{T5}\selectfont \DJ o\`an Trung C\uhorn{}\`\ohorn ng}
\address{\fontencoding{T5}\selectfont \DJ o\`an Trung C\uhorn{}\`\ohorn ng.  Institute of Mathematics, Vietnam Academy of Science and Technology, 18 Hoang Quoc Viet, 10307 Hanoi, Vietnam.} \email{dtcuong@math.ac.vn}

\author{\fontencoding{T5}\selectfont Ph\d am H\`\ocircumflex ng Nam} \address{\fontencoding{T5}\selectfont Ph\d am H\`\ocircumflex ng Nam, University of Sciences, Thai Nguyen University, Thai Nguyen, Vietnam.} \email{phamhongnam2106@gmail.com}

\author{\fontencoding{T5}\selectfont Ph\d am H\`ung Qu\'y} \address{\fontencoding{T5}\selectfont Ph\d am H\`ung Qu\'y, Department of Mathematics, FPT University, and Thang Long Institute of Mathematics and Applied sciences, Thang Long University, Hanoi, Vietnam.} \email{quyph@fe.edu.vn}

\thanks{The first and the second authors are funded by Vietnam National Foundation for Science and Technology Development (NAFOSTED) under grant number 101.04-2015.26. This paper was written while the third author was visiting Vietnam Institute for Advanced Study in Mathematics. He would like to thank the VIASM for hospitality and financial support.}

\subjclass[2010]{13H15, 13D40, 13D45}
\keywords{saturated Hilbert polynomial, Hilbert polynomial of Artinian modules, Rees polynomial, generalized Cohen-Macaulay ring}

\begin{abstract}
For an ideal $I$ in a local ring $(R, {\ensuremath{\mathfrak m}})$, we prove that the integer-valued function $\ell_R(H^0_{\ensuremath{\mathfrak m}}(R/I^{n+1}))$ is a polynomial for $n$ big enough if either $I$ is a principle ideal or $I$ is generated by part of an almost p-standard system of parameters. Furthermore, we are able to compute the coefficients of this polynomial in terms of length of certain local cohomology modules and usual multiplicity if either the ideal is principal or it is generated by part of a standard system of parameters in a generalized Cohen-Macaulay ring. We also give an example of an ideal generated by part of a (general) system of parameters such that the function $\ell_R(H^0_{\ensuremath{\mathfrak m}}(R/I^{n+1}))$ is not a polynomial for $n\gg 0$.
\end{abstract}

\maketitle

\section{Introduction}

Let $(R, {\ensuremath{\mathfrak m}})$ be a Noetherian local ring and $I\subset R$ be an ideal. There is a numerical function attached to $I$,
$$h^0_I: {\ensuremath{\mathbb Z}}_{\geq 0}\rightarrow {\ensuremath{\mathbb Z}}_{\geq 0}, n\mapsto \ell_R(H^0_{\ensuremath{\mathfrak m}}(R/I^{n+1})).$$
If $I$ is ${\ensuremath{\mathfrak m}}$-primary then $h^0_I(n)=\ell(R/I^{n+1})$ is the Hilbert-Samuel function. So it is of polynomial type, that means, there is a polynomial $H_I(n)$ such that $h^0_I(n)=H_I(n)$ for all $n\gg 0$. For a general ideal $I$, it is natural to ask whether the function $h^0_I(n)$ is of polynomial type. Unfortunately, it is not always the case. In \cite{UV} Ulrich and Validashti prove that the limit
$$\limsup_n d!\frac{h^0_I(n)}{n^d},$$
is a finite number. It is then called the $\epsilon$-multiplicity of the ideal $I$ as
a generalization of the Buchsbaum-Rim multiplicity. This numerical invariant $\epsilon$-multiplicity is denoted by $\epsilon(I)$ and is very useful in the theory of equisingularity (see \cite{KUV}). Nevertheless, Cutkosky, Ha, Srinivasan and Theodorescu \cite[Theorem 2.2]{CTST} give an interesting example of a $4$-dimensional local ring $R$ and an ideal $I$ such that
$$\epsilon(I)=\lim_{n\to \infty}4!\frac{h^0_I(n)}{n^4},$$
is an irrational number. Thus in this example $h^0_I(n)$ is not of polynomial type. So the answer to the previous question is no in general.

In this short note, we address ourself on the question under which assumption the function $h^0_I(n)$ is of polynomial type, that is, there is a polynomial $P_I(n)$ such tht $h^0_I(n)=P_I(n)$ for all $n\gg 0$. Note that $H^0_{\ensuremath{\mathfrak m}}(R/I^n)=\cup_{s>0}(I^n:_R{\ensuremath{\mathfrak m}}^s)/I^n$ and we call $P_I(n)$ the saturated Hilbert polynomial of $I$. As the main results, we will show that the saturated Hilbert polynomial exists in the following cases

\begin{enumerate}
\item[(a)] The ideal $I$ is principal (Theorem \ref{22}). The main idea of the proof is to relate the function $h^0_I(n)$ to the Hilbert functions of some Artinian modules arising from the first local cohomology module of $R$. In this case the leading coefficient of the saturated Hilbert polynomial could be expressed precisely in terms of usual multiplicity and length of certain local cohomology modules;

\item[(b)] The ring $R$ is a quotient of a Cohen-Macaulay local ring  and the ideal $I$ is generated by part of an almost p-standard system of parameters of $R$ (Theorem \ref{37}). Almost p-standard systems of parameters are introduced by the first two authors in \cite{DTCNam}, they are systems of parameters and at the same time a dd-sequence on the ring, the later notion is given by N. T. Cuong and the first author in \cite{NTCDTC1} as a special class of strong d-sequences.

We will show that $h^0_I(n)=\ell_R(J^{n+1}/I^{n+1})$ for an ideal $J$ such that $I$ is a reduction of $J$ and $\ell_R(J/I)$ is finite. Then a theorem of Amao \cite{AMAO} concludes that there is a polynomial $P_I(n)$ such that $h^0_I(n)=P_I(n)$ for all $n\gg 0$. It should be remarked that $h^0_I(n)$ and $P_I(n)$ are also called the Rees function and Rees polynomial of the pair $(I, J)$ by Herzog-Puthenpurakal-Verma \cite{HPV}.
\end{enumerate}

A particular case of (b) is when the ideal $I$ is generated by part of a standard system of parameters in a generalized Cohen-Macaulay local ring. With this assumption, we are able to give a precise formula for the saturated Hilbert polynomial $P_I(n)$ (Theorem \ref{39}).

In the general case, relying on the example of Cutkosky, Ha, Srinivasan and Theodorescu \cite[Theorem 2.2]{CTST}, we give an example of an ideal $I$ generated by part of a (general) system of parameters such that the function $h^0_I(n)$ is not of polynomial type (Example \ref{310}).

About the structure of the note, we treat the case of principal ideals in Section 2, while the case of ideals generated by part of a system of parameters is considered in Section 3.

Throughout this note, $(R, {\ensuremath{\mathfrak m}})$ is always a Noetherian local ring.

\section{Saturated Hilbert polynomial of a principal ideal}

In this section, we study the existence of a saturated Hilbert polynomial of the ring $R$ with respect to a principal ideal. Before stating the main result of this section, we first recall several facts about Hilbert functions and multiplicities of Artinian modules from \cite{Kir, NTCNhan} which will be used in the subsequent sections.

\begin{remark}\label{21} Let $A$ be an Artinian $R$-module and $I\subset R$ be a proper ideal. Suppose that $0:_AI$ is of finite length.
\begin{enumerate}
\item[(a)] D. Kirby \cite[Proposition 2]{Kir} proves that the module $0:_AI^{n+1}$ is also of finite length for any $n\geq 0$ and moreover, there is a polynomial $P_{A, I}(n)$ such that $P_{A, I}(n)=\ell_R(0:_AI^{n+1})$ for all $n\gg 0$. The degree of $P_{A, I}(n)$ is bounded above by $\mu(I)$.
\item[(b)] (See \cite{NTCNhan}) The multiplicity $e^\prime(I; A)$ of the Artinian module $A$ with respect to $I$ is defined such that $d!e^\prime(I;A)$ is the leading coefficient of the polynomial $P_{A, I}(n)$, here $d$ is the degree of $P_{A, I}(n)$. Given a short exact sequence of Artinian modules
$$0\rightarrow A_1\rightarrow A\rightarrow A_2\rightarrow 0,$$
we have $e^\prime(I, A)=\delta_{dd_1}e^\prime(I, A_1)+\delta_{dd_2}e^\prime(I, A_2)$, where $d_i$ is the degree of the polynomial $P_{I, A_i}(n)$ and $\delta_{dd_i}$ is the Kronecker delta, $i=1, 2$.
\item[(c)] Assume that $R$ is a homomorphic image of a Cohen-Macaulay local ring. Equivalently, as shown in \cite{NTCDTC2}, $R$ is universally catenary and all its formal fibers are Cohen-Macaulay. In \cite{BS1} Brodmann and Sharp define the $i$-th pseudo-support of $R$ to be the set
$${\operatorname{Psupp}}^i(R)=\{{\ensuremath{\mathfrak p}}\in {\operatorname{Spec}}(R): H^{i-\dim R/{\ensuremath{\mathfrak p}}}_{{\ensuremath{\mathfrak p}} R_{\ensuremath{\mathfrak p}}}(R_{\ensuremath{\mathfrak p}})\not=0\}.$$
In particular, ${\operatorname{Psupp}}^1(R)=\{{\ensuremath{\mathfrak m}}\}\cup\{{\ensuremath{\mathfrak p}}:
\dim R/{\ensuremath{\mathfrak p}}=1 \text{ and }H^0_{{\ensuremath{\mathfrak p}} R_{\ensuremath{\mathfrak p}}}(R_{\ensuremath{\mathfrak p}})\not=0\}$. Put $\mathrm{psd}^i(R)=\max\{\dim R/{\ensuremath{\mathfrak p}}: {\ensuremath{\mathfrak p}}\in {\operatorname{Psupp}}^i(R)\}$. They then show that (see \cite[Theorem 2.4]{BS1})
$$e^\prime(I; H^i_{\ensuremath{\mathfrak m}}(R))=\sum_{\substack{{\ensuremath{\mathfrak p}}\in{\operatorname{Psupp}}^i(R)\\ \dim R/{\ensuremath{\mathfrak p}}=\mathrm{psd}^i(R)}}\ell_{R_{\ensuremath{\mathfrak p}}}(H^{i-\dim R/{\ensuremath{\mathfrak p}}}_{{\ensuremath{\mathfrak p}} R_{\ensuremath{\mathfrak p}}}(R_{\ensuremath{\mathfrak p}}))e(I, R/{\ensuremath{\mathfrak p}}).$$
\end{enumerate}
\end{remark}

The main result of this section is the following theorem

\begin{theorem}\label{22}
Let $I=aR$ be a principal ideal of $R$. There is a polynomial $P_I(n)$ with $\deg P_I(n)\leq 1$ such that $P_I(n)=h^0_I(n)$ for all $n\gg 0$.
\end{theorem}
\begin{proof}
As $R$ is Noetherian, there is a number $t>0$ such that $0:_Ra^t=0:_Ra^{n+t+1}$ for any $n\geq 0$. The short exact sequence
$$0\longrightarrow R/0:_Ra^t\xrightarrow{\ *a^{n+t+1}\ } R\longrightarrow R/a^{n+t+1}R\longrightarrow 0,$$
leads to an exact sequence of local cohomology modules
\begin{multline*}
0\longrightarrow H^0_{\ensuremath{\mathfrak m}}(R/0:_Ra^t)\xrightarrow{\ *a^{n+t+1}\ } H^0_{\ensuremath{\mathfrak m}}(R)\longrightarrow H^0_{\ensuremath{\mathfrak m}}(R/a^{n+t+1}R)\\
\longrightarrow H^1_{\ensuremath{\mathfrak m}}(R/0:_Ra^t)\xrightarrow{\psi_{n+t+1}} H^1_{\ensuremath{\mathfrak m}}(R)\longrightarrow \ldots.
\end{multline*}
Here $H^0_{\ensuremath{\mathfrak m}}(R/0:_Ra^t)=0$ as $a$ is a regular element on $R/0:_Ra^t$. The map $\psi_{n+t+1}$ is derived from the multiplication by $a^{n+t+1}$ on $R$. We  get
$$h^0_I(n+t)=\ell({\operatorname{Ker}} \psi_{n+t+1})+\ell_R(H^0_{\ensuremath{\mathfrak m}}(R)).$$
From the commutative triangle\medskip

\centerline{
\xymatrix@1{
R/0:_Ra^t \ \ \ar[rr]^{*a^{n+t+1}} \ar[dr]_{*a^{n+1}} && \ R \\
&R/0:_Ra^t\ar[ur]_{*a^t}
}}
\bigskip

\noindent there is induced a commutative triangle of local cohomology\medskip

\centerline{
\xymatrix@1{
H^1_{\ensuremath{\mathfrak m}}(R/0:_Ra^t) \ \ \ar[rr]^{\psi_{n+t+1}} \ar[dr]_{*a^{n+1}} && \ H^1_{\ensuremath{\mathfrak m}}(R). \\
&H^1_{\ensuremath{\mathfrak m}}(R/0:_Ra^t)\ar[ur]_{\psi_t}
}}
\bigskip

\noindent Hence ${\operatorname{Ker}}(\psi_{n+t+1})={\operatorname{Ker}}(\psi_t):_Aa^{n+1}$, where $A=H^1_{\ensuremath{\mathfrak m}}(R/0:_Ra^t)$ is an Artinian module (see, for example, \cite[Theorem 7.1.3]{BS}). Let $\bar A=A/{\operatorname{Ker}}(\psi_t)$. We obtain
$$h^0_I(n+t)=\ell(0:_{\bar A}a^{n+1})+\ell({\operatorname{Ker}} \psi_t)+\ell_R(H^0_{\ensuremath{\mathfrak m}}(R)),$$
for any $n\geq 0$. Since the module $\bar A$ is Artinian, the conclusion is implied from Remark \ref{21}(a).
\end{proof}

Let $I$ be a principal ideal and let $P_I(n)$ be the polynomial in Theorem \ref{22} so that $P_I(n)=h^0_I(n)$ for all $n\gg 0$. We call $P_I(n)$ the cohomological Hilbert-Samuel polynomial of the ring $R$ with respect to the ideal $I$. As in the proof of Theorem \ref{22}, this polynomial is closely related to a Hilbert polynomial of a first local cohomology module, thus, an Artinian module. Since $\deg P_I(n)\leq 1$, if the dimension of $R$ is bigger than $1$ then
$$\lim_nd!\frac{P_I(n)}{n^d}=0.$$
So the $\epsilon$-multiplicity of a principal ideal in $R$ is zero.

In the next, we will investigate more on the coefficients of the saturated Hilbert polynomials $P_{aR}(n)$. Let $I=aR$ be a principal ideal of $R$. Since the degree of $P_I(n)$ is bounded above by $1$, we write
$$P_I(n)=ne^{sat}_0(a; R)+e_1^{sat}(a; R),$$
for any $n\geq 0$, where the coefficients $e^{sat}_0(a; R), e^{sat}_1(a; R)$ are given integers. We will show that the leading coefficient $e_0^{sat}(a; R)$ could be expressed explicitly by means of usual multiplicities and length of certain local cohomology modules. We need first a lemma.

\begin{lemma}\label{23}
Denote ${\operatorname{Ass}}(R)_1=\{{\ensuremath{\mathfrak p}}\in{\operatorname{Ass}}(R): \dim R/{\ensuremath{\mathfrak p}}=1\}$. Then
$${\operatorname{Ass}}(R)_1={\operatorname{Psupp}}^1(R)\setminus\{{\ensuremath{\mathfrak m}}\}.$$
\end{lemma}
\begin{proof}
Let ${\ensuremath{\mathfrak p}}$ be a prime ideal of codimension $1$. Then $H^0_{{\ensuremath{\mathfrak p}} R_{\ensuremath{\mathfrak p}}}(R_{\ensuremath{\mathfrak p}})\not=0$ if and only if ${\ensuremath{\mathfrak p}} R_{\ensuremath{\mathfrak p}}$ is an associated prime ideal of $R_{\ensuremath{\mathfrak p}}$. This turns out to be equivalent to that ${\ensuremath{\mathfrak p}}$ is an associated prime ideal of $R$.
\end{proof}

From now on, we will need to assume that $R$ is a quotient of a Cohen-Macaulay local ring.

\begin{theorem}\label{24}
Let $R$ be a quotient of a Cohen-Macaulay Noetherian local ring. Let $I=aR$ be a principal ideal of $R$. We have
$$e^{sat}_0(a; R)=\sum_{{\ensuremath{\mathfrak p}}\in{\operatorname{Ass}}(R)_1\setminus V(I)}\ell_{R_{\ensuremath{\mathfrak p}}}(H^0_{{\ensuremath{\mathfrak p}} R_{\ensuremath{\mathfrak p}}}(R_{\ensuremath{\mathfrak p}}))e(a; R/{\ensuremath{\mathfrak p}}).$$
\end{theorem}
\begin{proof}
Using the notations in the proof of Theorem \ref{22}, from the equality
$$h^0_I(n+t)=\ell(0:_{\bar A}a^{n+1})+\ell({\operatorname{Ker}} \psi_t)+\ell_R(H^0_{\ensuremath{\mathfrak m}}(R)),$$
where $A=H^1_{\ensuremath{\mathfrak m}}(R/0:_Ra^t)$ and $\bar A=A/{\operatorname{Ker}}(\psi_t)$, we get that $$e^{sat}_0(a; R)=e^\prime (a; \bar A)=e^\prime(a; H^1_{\ensuremath{\mathfrak m}}(R/0:_Ra^t)/{\operatorname{Ker}}(\psi_t)).$$
Since the module ${\operatorname{Ker}}(\psi_t)$ is of finite length as it is the image of the connecting map $H^0_{\ensuremath{\mathfrak m}}(R/a^tR)\rightarrow H^1_{\ensuremath{\mathfrak m}}(R/0:_Ra^t)$, Remark \ref{21}(b) gives us
$$e^{sat}_0(a; R)=e^\prime (a; H_{\ensuremath{\mathfrak m}} ^1(R/0:_Ra^t)).$$
Using Brodmann-Sharp's associativity formula for multiplicities of local cohomology modules in Remark \ref{21}(c) we have
\begin{displaymath}
e^\prime (a, H_{\ensuremath{\mathfrak m}} ^1(R/0:_Ra^t))=
\sum_{\substack{{\ensuremath{\mathfrak p}} \in {\operatorname{Psupp}}^1(R/0:_Ra^t) \\ {\ensuremath{\mathfrak p}} \not={\ensuremath{\mathfrak m}}}}
\ell_{R_{\ensuremath{\mathfrak p}} }(H_{{\ensuremath{\mathfrak p}} R_{\ensuremath{\mathfrak p}} }^0(R_{\ensuremath{\mathfrak p}}/(0:_Ra^t)_{\ensuremath{\mathfrak p}}))e(a; R/{\ensuremath{\mathfrak p}} ).
\end{displaymath}

Now the inclusion $R/0:_Ra^t\stackrel{*a^t}{\longrightarrow} R$ leads to an inclusion
${\operatorname{Psupp}}^1(R/0:_Ra^t)\subseteq {\operatorname{Psupp}}^1(R)$. Take a prime ideal ${\ensuremath{\mathfrak p}} \in {\operatorname{Psupp}}^1(R)$ with $\dim R/{\ensuremath{\mathfrak p}} =1$. If $a\not\in{\ensuremath{\mathfrak p}}$ then $(0:_Ra^t)_{\ensuremath{\mathfrak p}}=0$ and $R_{\ensuremath{\mathfrak p}}\simeq (R/0:_Ra^t)_{\ensuremath{\mathfrak p}}$. So $H_{{\ensuremath{\mathfrak p}} R_{\ensuremath{\mathfrak p}} }^0(R_{\ensuremath{\mathfrak p}})\simeq H_{{\ensuremath{\mathfrak p}} R_{\ensuremath{\mathfrak p}} }^0((R/0:_Ra^t)_{\ensuremath{\mathfrak p}})$. This implies that
$${\operatorname{Psupp}}^1(R)\setminus V(I)={\operatorname{Psupp}}^1(R/0:_Ra^t)\setminus V(I).$$
On the other hand, if $a\in {\ensuremath{\mathfrak p}}$ then $a$ is a regular element on $R/0:_Ra^t$, hence $\frac{a}{1}$ is a regular element on $(R/0:_Ra^t)_{\ensuremath{\mathfrak p}}$. This shows particularly that $H_{{\ensuremath{\mathfrak p}} R_{\ensuremath{\mathfrak p}} }^0((R/0:_Ra^t)_{\ensuremath{\mathfrak p}})=0$ and thus
$${\operatorname{Psupp}}^1(R/0:_Ra^t)\setminus V(I)={\operatorname{Psupp}}^1(R/0:_Ra^t)\setminus \{{\ensuremath{\mathfrak m}}\}.$$
Therefore we get

\begin{displaymath}
\begin{split}
e^{sat}_0(a; R)
&= \sum_{\substack{{\ensuremath{\mathfrak p}} \in {\operatorname{Psupp}}^1(R/0:_Ra^t) \\ {\ensuremath{\mathfrak p}} \not={\ensuremath{\mathfrak m}}}}
\ell_{R_{\ensuremath{\mathfrak p}} }(H_{{\ensuremath{\mathfrak p}} R_{\ensuremath{\mathfrak p}} }^0(R_{\ensuremath{\mathfrak p}}/(0:_Ra^t)_{\ensuremath{\mathfrak p}}))e(a; R/{\ensuremath{\mathfrak p}} )\\
&= \sum_{\substack{{\ensuremath{\mathfrak p}} \in {\operatorname{Psupp}}^1(R/0:_Ra^t) \\ {\ensuremath{\mathfrak p}} \not\in V(I)}}
\ell_{R_{\ensuremath{\mathfrak p}} }(H_{{\ensuremath{\mathfrak p}} R_{\ensuremath{\mathfrak p}} }^0(R_{\ensuremath{\mathfrak p}}/(0:_Ra^t)_{\ensuremath{\mathfrak p}}))e(a; R/{\ensuremath{\mathfrak p}} )\\
&=\sum_{{\ensuremath{\mathfrak p}} \in {\operatorname{Psupp}}^1(R)\setminus V(I)}
\ell_{R_{\ensuremath{\mathfrak p}} }(H_{{\ensuremath{\mathfrak p}} R_{\ensuremath{\mathfrak p}} }^0(R_{\ensuremath{\mathfrak p}}))e(a; R/{\ensuremath{\mathfrak p}})\\
&=\sum_{{\ensuremath{\mathfrak p}} \in {\operatorname{Ass}}(R)_1\setminus V(I)}
\ell_{R_{\ensuremath{\mathfrak p}} }(H_{{\ensuremath{\mathfrak p}} R_{\ensuremath{\mathfrak p}} }^0(R_{\ensuremath{\mathfrak p}}))e(a; R/{\ensuremath{\mathfrak p}}),
\end{split}
\end{displaymath}
as required, here the last equality follows from Lemma \ref{23}.
\end{proof}

We have $\emptyset \subseteq {\operatorname{Ass}}(R)_1\setminus V(I)\subseteq {\operatorname{Ass}}(R)_1$. A direct consequence of Theorem \ref{24} is that the cohomological Hilbert-Samuel polynomial $P_I(n)$ is of degree one if and only if ${\operatorname{Ass}}(R)_1\setminus V(I)\not=\emptyset$, or equivalently, $a$ is in some associated prime ideal of codimension one. In the next, for a given ring $R$ and let $a$ vary, we consider the two extremal cases of ${\operatorname{Ass}}(R)_1\setminus V(I)$, that is, $\emptyset$ and ${\operatorname{Ass}}(R)_1$.

\begin{corollary}\label{25}
Let $a\in {\ensuremath{\mathfrak m}}$ and let $t>0$ such that $0:_Ra^{n+t+1}=0:_Ra^t$ for all $n\geq 0$.
\begin{enumerate}[(i)]
\item If $a$ is a filter-regular element of $R$ then
$$e_0^{sat}(a; R)=e^\prime(a; H^1_{\ensuremath{\mathfrak m}}(R)).$$
The right hand side is the multiplicity of a local cohomology module defined in Remark \ref{21}(b).

\item The element $a$ is in the radical of ${\operatorname{Ann}}_RH^1(R)$ if and only if $e_0^{sat}(a; R)=0$. In that case,
$$P_I(n)=\ell(H^1_{\ensuremath{\mathfrak m}}(R/0:_Ra^t))+\ell(H^0_{\ensuremath{\mathfrak m}}(R)),$$
is a constant polynomial.
\end{enumerate}
\end{corollary}
\begin{proof}
\item(i) Since $a$ is a filter-regular element of $R$, it is obvious that $0:_{H^1_{\ensuremath{\mathfrak m}}(R)}a$ is of finite length, thus the multiplicity $e^\prime(a; H^1_{\ensuremath{\mathfrak m}}(R))$ is determined by Remark \ref{21}(b).

On the other hand, being a filter-regular element of $R$ means that $a\not\in{\ensuremath{\mathfrak p}}$ for any associated prime ideal ${\ensuremath{\mathfrak p}}\not={\ensuremath{\mathfrak m}}$. So ${\operatorname{Ass}}(R)_1\setminus V(I)={\operatorname{Ass}}(R)_1$. From Theorem \ref{24} and the associativity formula in Remark \ref{21}(c) we obtain the equality $e_0^{sat}(a; R)=e^\prime(a; H^1_{\ensuremath{\mathfrak m}}(R))$.

\item(ii) By Theorem \ref{24}, the leading coefficient $e_0^{sat}(a; R)$ vanishes if and only if ${\operatorname{Ass}}(R)_1\subset V(I)$. For the necessary condition, we take an associated prime ideal ${\ensuremath{\mathfrak p}}\in {\operatorname{Ass}}(R)_1$. There is an inclusion $\psi: R/{\ensuremath{\mathfrak p}}\hookrightarrow R$ which give an exact sequence
$$H^0_{\ensuremath{\mathfrak m}}({\operatorname{Coker}} \psi)\rightarrow H^1_{\ensuremath{\mathfrak m}}(R/{\ensuremath{\mathfrak p}})\rightarrow H^1_{\ensuremath{\mathfrak m}}(R).$$
Since $a$ is in the radical of ${\operatorname{Ann}}_RH^1_{\ensuremath{\mathfrak m}}(R)$, the exact sequence implies that $a^rH^1(R/{\ensuremath{\mathfrak p}})=0$ for some $r\gg0$. This implies in particular that $a\in {\ensuremath{\mathfrak p}}$ since $\dim R/{\ensuremath{\mathfrak p}}=1$. Hence ${\operatorname{Ass}}(R)_1\subseteq V(I)$.

Conversely, suppose ${\operatorname{Ass}}(R)_1\subseteq V(I)$. Since $R$ is a homomorphic image of a Cohen-Macaulay local ring, by \cite[Proposition 2.5]{BS1}, we have $$V({\operatorname{Ann}}_RH^1_{\ensuremath{\mathfrak m}}(R))=\bigcup_{\substack{{\ensuremath{\mathfrak p}}\in {\operatorname{Ass}}(R)\\ \dim R/{\ensuremath{\mathfrak p}}\leq 1}} V({\ensuremath{\mathfrak p}})\subseteq V(I).$$
So $I\subseteq \sqrt{{\operatorname{Ann}}_RH^1_{\ensuremath{\mathfrak m}}(R)}$.
\end{proof}

If we put more restriction on the ring $R$, assuming that it is a homomorphic image of a Gorenstein local ring $(R^\prime, {\ensuremath{\mathfrak m}}^\prime)$, then the coefficient $e_0^{sat}(a; R)$ could be computed as a multiplicity of the first module of deficiency $K^1(R)={\operatorname{Ext}}^{d^\prime-1}_{R^\prime}(R, R^\prime)$, where $d^\prime=\dim R^\prime$. The module of deficiency $K^1(R)$ is a finitely generated $R$-module which is dual to $H^1_{\ensuremath{\mathfrak m}}(R)$ via the Local Duality Theorem \cite[11.2.6]{BS}.

\begin{corollary}\label{26}
Let $(R, {\ensuremath{\mathfrak m}})$ be a homomorphic image of a Gorenstein local ring. Let $I=aR$ be a principal ideal. We have
$$e_0^{sat}(a; R)=e(a; K^1(R)),$$
where the multiplicity on the right hand side is defined by
$$e(a; K^1(R))=\sum_{\substack{{\ensuremath{\mathfrak p}}\in{\operatorname{Ass}} K^1(R)\setminus V(I)\\ \dim R/{\ensuremath{\mathfrak p}}=1}}\ell_{R_{\ensuremath{\mathfrak p}} }(K^1(R)_{\ensuremath{\mathfrak p}})e(a; R/{\ensuremath{\mathfrak p}}).$$
\end{corollary}
\begin{proof}
Following \cite[Proposition 1.2]{BS1}, we have ${\operatorname{Psupp}}^1(M)={\operatorname{Supp}} K^1(R)$ being a closed subset of ${\operatorname{Spec}} R$ of dimension at most $1$. Moreover, for a prime ideal ${\ensuremath{\mathfrak p}}\in {\operatorname{Psupp}}^1(R)_1={\operatorname{Supp}} K^1(R)_1={\operatorname{Ass}} K^1(R)_1$, we have
$\ell_{R_{\ensuremath{\mathfrak p}}}(H^0_{{\ensuremath{\mathfrak p}} R_{\ensuremath{\mathfrak p}}}(R_{\ensuremath{\mathfrak p}}))=\ell_{R_{\ensuremath{\mathfrak p}}}(K^1(R)_{\ensuremath{\mathfrak p}})$. The conclusion is therefore a consequence of Theorem \ref{24}.
\end{proof}

\section{Saturated Hilbert polynomial of ideals generated by part of a system of parameters}

In this section we study the existence of saturated Hilbert polynomial for ideals generated by part of a system of parameters. In the first part of the section, we restrict ourself to the case of almost p-standard systems of parameters which are defined in \cite{DTCNam}. Through this section, again we assume $(R, {\ensuremath{\mathfrak m}})$ is a Noetherian local ring.

We recall the definition of almost p-standard system of parameters.

\begin{definition}\cite[Definition 2.1]{DTCNam}\label{31}
Let $M$ be a finitely generated $R$-module of dimension $d$. A system of parameters $x_1, \ldots, x_d$ of $M$ is a called an almost p-standard system of parameters if there are given integers $\lambda_0, \ldots, \lambda_d$ such that
$$\ell_R(M/(x_1^{n_1}, \ldots, x_d^{n_d})M)=\sum_{i=0}^dn_1\ldots n_i\lambda_i,$$
for all $n_1, \ldots, n_d>0$.
\end{definition}

In \cite{NTCDTC1} the notion of dd-sequence on a finitely generated module is introduced as a special class of d-sequences, the later notion is given by Huneke \cite{HU}. It is shown that a system of parameters is a dd-sequence if and only if its length function satisfies the equality in the definition above, hence is an almost p-standard system of parameters. In more detail, we have

\begin{proposition}\cite[Corollary 3.6]{NTCDTC1}\label{32}
Let $x_1, \ldots, x_d$ be a system of parameters on a finitely generated module $M$. The following statements are equivalent.
\begin{enumerate}[(a)]
\item $x_1, \ldots, x_d$ is an almost p-standard system of parameters;
\item $x_1^{n_1}, \ldots, x_i^{n_i}$ is a d-sequence on $M/(x_{i+1}^{n_{i+1}}, \ldots, x_d^{n_d})M$ for $i=1, \ldots, d$ and for all $n_1, \ldots, n_d>0$.
\end{enumerate}
\end{proposition}

Recall that a sequence $x_1, \ldots, x_s$ is a d-sequence on $M$ if $(x_1, \ldots, x_{i-1})M:x_ix_j=(x_1, \ldots, x_{i-1})M:x_j$ for all $1\leq i\leq j\leq s$. It is a strong d-sequence if $x_1^{n_1}, \ldots, x_s^{n_s}$ is a d-sequence on $M$ for any $n_1, \ldots, n_s>0$.

A local ring has an almost p-standard system of parameters if and only if it is a quotient of a Cohen-Macaulay local ring (see \cite{NTCDTC2}). The existence of almost p-standard systems of parameters therefore is at large. The following proposition is the first step to prove the existence of saturated Hilbert polynomial.

\begin{proposition}\label{33}
Let $x_1, \ldots, x_d$ be an almost p-standard system of parameters of a finitely generated module $M$ over $R$. For $0\leq i<j\leq d$, denote $I=(x_{i+1}, \ldots, x_j)$. The sequence $x_1, \ldots, x_i, x_{j+1}^2, \ldots, x_d^2$ is an almost p-standard system of parameters of $M/I^nM$ for all $n>0$.
\end{proposition}
\begin{proof} Let $n_1, \ldots, n_d$ be positive integers. By \cite{NTCDTC1}, $x_{i+1}, \ldots, x_j$ is an almost p-standard system of parameters of $M_1=M/(x_1^{n_1}, \ldots, x_i^{n_i}, x_{j+1}^{n_{j+1}}, \ldots, x_d^{n_d})M$, in particular, it is a strong d-sequence on $M_1$. Due to \cite[Theorem 4.1]{NVT1}, we have
$$\ell(M_1/I^{n+1}M_1)=\sum_{t=0}^{j-i}f_{j-i-t}(I; M_1)\binom{n+t}{t},$$
where $f_{j-i}(I; M_1)=h^0(M_1)$ and for $t>0$,
\[\begin{aligned}
f_{j-i-t}(I; M_1)=&h^0(M/(x_1^{n_1}, \ldots, x_i^{n_i}, x_{i+1}, \ldots, x_{t+1}, x_{j+1}^{n_{j+1}}, \ldots, x_d^{n_d})M)\\
&-h^0(M/(x_1^{n_1}, \ldots, x_i^{n_i}, x_{i+1}, \ldots, x_t, x_{j+1}^{n_{j+1}}, \ldots, x_d^{n_d})M).
\end{aligned}\]
Denote $M_2=M/(x_{j+1}^{n_{j+1}}, \ldots, x_d^{n_d})M$. The length of the zero-th local cohomology module is computed in \cite[Corollary 4.4]{DTCNam} and we have
\[\begin{aligned}
h^0(M/(x_1^{n_1}, \ldots, x_i^{n_i}, &x_{i+1}, \ldots, x_t, x_{j+1}^{n_{j+1}}, \ldots, x_d^{n_d})M)\\
&=h^0(M_2/((x_1^{n_1}, \ldots, x_i^{n_i}, x_{i+1}, \ldots, x_t)M_2)\\
&=\sum_{v=0}^in_1\ldots n_ve(x_1, \ldots, x_v; (0:x_{v+1})_{M_2/(x_{v+2}, \ldots, x_t)M_2}),
\end{aligned}\]
which does not depend on $n_{j+1}, \ldots, n_d\geq 2$ by \cite[Proposition 3.2]{DTCNam}.

On the other hand, using Lech's theorem and the definition of almost p-standard system of parameters \ref{31}, we have
\[\begin{aligned}
f_0(I; M_1)
=&\lim_{m \to \infty}\ \ \frac{1}{m^{j-i}}\ell(M_1/(x_{i+1}^m, \ldots, x_j^m)M_1)\\
=&\lim_{m \to \infty}\ \ \frac{1}{m^{j-i}}\ell(M/(x_1^{n_1}, \ldots, x_i^{n_i}, x_{i+1}^m, \ldots, x_j^m, x_{j+1}^{n_{j+1}}, \ldots, x_d^{n_d})M)\\
=&n_1\ldots n_i\sum_{v=j}^dn_{j+1}\ldots n_ve(x_1, \ldots, x_v; (0:x_{v+1})_{M/(x_{v+2}, \ldots, x_d)M}).
\end{aligned}\]
Replacing $n_{j+1}, \ldots, n_d$ by $2n_{j+1}, \ldots, 2n_d$, there are integers $\lambda_0, \ldots, \lambda_i$, $\lambda_{j+1}, \ldots, \lambda_d$ such that
$$\ell\left(\frac{M/I^{n+1}M}{(x_{i+1}^{n_{i+1}}, \ldots, x_j^{n_j})M/I^{n+1}M}\right)=\sum_{v=0}^in_1\ldots n_v\lambda_v+ n_1\ldots n_i\sum_{v=j}^dn_{j+1}\ldots n_v\lambda_v,$$
for all $n_1, \ldots, n_d\geq 1$. Therefore, $x_1, \ldots, x_i, x_{j+1}^2, \ldots, x_d^2$ is an almost p-standard system of parameters of $M/I^{n+1}M$ for all $n\geq 0$.
\end{proof}

If $R$ is a generalized Cohen-Macaulay module then an almost p-standard system of parameters is the same as a standard system of parameters. Particularly it is an unconditioned strong d-sequence. An immediate consequence of Proposition \ref{33} is (compare \cite[Theorem 1.2]{LT})

\begin{corollary}\label{34}
Let $R$ be a generalized Cohen-Macaulay local ring with a standard system of parameters $x_1, \ldots, x_d$. For some $0<i<d$, put $I=(x_1, \ldots, x_i)$. Then $R/I^{n+1}$ is a generalized Cohen-Macaulay ring for any $n\geq 0$ and $x_{i+1}, \ldots, x_d$ is its standard system of parameters.

In particular, if $R$ is a Cohen-Macaulay local ring then for each $n>0$, $R/I^n$ is a Cohen-Macaulay ring and $h^0_I(n)=0$.
\end{corollary}

In order to prove the existence of saturated Hilbert polynomial, we need the following key lemma of which the proof is rather technical.

\begin{lemma}\label{35}
Let $x_1,\ldots, x_d$ be an almost p-standard system of parameters of $R$. For $0\leq i<j\leq d$, denote $I=(x_{i+1}, \ldots, x_j)$. Let $t\in \{1,\ldots, i\}\cup\{j+1,\ldots, d\}$. Then
$$I^{n+1}:_Rx_t=I^n(I:_Rx_t)+0:_Rx_t.$$
\end{lemma}
\begin{proof}
The conclusion is proved by induction on $j-i$. If $j-i=0$ then the conclusion is obvious. Suppose $j-i>0$. It suffices to show that
$$I^{n+1}:x_t\subseteq I(I^n:x_t)+0:_Rx_t,$$
for all $n>0$. We have
$$I^{n+1}=x_{i+1}^2I^{n-1}+x_{i+1}J^n+J^{n+1},$$
where $J=(x_{i+2}, \ldots, x_j)$. Taking an element $a\in I^{n+1}:x_t$, we write
$$x_ta=x_{i+1}^2a_1+x_{i+1}a_2+a_3,$$
where $a_1\in I^{n-1}$, $a_2\in J^n$ and $a_3\in J^{n+1}$. Then
$$a_1\in (x_t, J^n):x_{i+1}^2=(x_t, J^n):x_{i+1}.$$
The last equality follows from the fact that the sequence $x_1, \ldots, x_i, x_{i+1}, x_{j+1}^2, \ldots, x_d^2$ (with $x_t$ being removed) is an almost p-standard system of parameters of $R/(x_t, J^n)$ due to Proposition \ref{33}. We are able to write $x_{i+1}a_1=x_tb+b_2$, where $b_2\in J^n$. In particular, $b\in (x_{i+1}I^{n-1}+J^n):x_t= I^n:x_t$, since $a_1\in I^{n-1}$. Hence
$$x_t(a-x_{i+1}b)=x_{i+1}a_2^\prime +a_3,$$
where $a_2^\prime=a_2+b_2\in J^n$. Note that $x_{i+1}b\in I(I^n:x_t)$. So we only need to show that $a-x_{i+1}b\in I(I^n:x_t)+0:_Rx_t$, or in other words, from beginning we might assume without lost of generality that $a_1=0$, or equivalently,
$$x_ta=x_{i+1}a_2+a_3.$$

We have $a_2\in [(x_t, J^{n+1}):x_{i+1}]\cap J^n$. The induction assumption applying to the ring $R/x_tR$ and the ideal $J=(x_{i+2}, \ldots, x_j)$ yields
$$(x_t, J^{n+1}):x_{i+1}=J^n((x_t, J):x_{i+1})+x_tR:x_{i+1}.$$
Hence
$$a_2\in [(x_t, J^{n+1}):x_{i+1}]\cap J^n=J^n((x_t, J):x_{i+1})+(x_tR:x_{i+1})\cap J^n.$$
Write
$$a_2=\sum_k\lambda_ku_k+u,$$
for $\lambda_k\in J^n$, $u_k\in (x_t, J):x_{i+1}$ and $u\in (x_tR:x_{i+1})\cap J^n$. We have $x_{i+1}u_k=x_tv_k+w_k$ for some $w_k\in J$ and $v_k\in (x_{i+1}, J):x_t=I:x_t$. Thus
$$x_{i+1}a_2=x_t\sum_k\lambda_kv_k+\sum_k\lambda_kw_k+x_{i+1}u.$$
Put $v=\sum_k\lambda_kv_k\in J^n(I:x_t)$ and $w=\sum_k\lambda_kw_k\in J^{n+1}$. We obtain
$$x_ta=x_tv+x_{i+1}u+w+a_3.$$

Now, it is worth noting that $x_1, \ldots, x_{t-1}, x_{t+1}, \ldots, x_d$ is an almost p-standard system of parameters of $R/x_tR$ due to Proposition \ref{32} and \cite[Proposition 3.4]{NTCDTC1}. We obtain $(0:_{R/x_tR}x_{i+1})\cap J^n(R/x_tR)=0$ as being shown in \cite[Corollary 3.7]{NTCDTC3}, or equivalently,
$$(x_tR:x_{i+1})\cap (x_t, J^n)=x_tR.$$
It shows that
$$u\in  (x_tR:x_{i+1})\cap J^n= x_tR\cap J^n.$$
Let $u=x_tu^\prime$, then $u^\prime\in J^n:x_t$. So $x_{i+1}u=x_t.x_{i+1}u^\prime$ where
$$x_{i+1}u^\prime\in x_{i+1}(J^n:x_t)\subseteq I(I^n:x_t).$$
To sum up, we have $x_ta=x_{i+1}a_2+a_3=x_tv+x_tx_{i+1}u^\prime+(w+a_3)$. Then
$$a-v-x_{i+1}u^\prime\in J^{n+1}:x_t=J(J^n:x_t)+0:_Rx_t.$$
The last equality holds by using the induction assumption to $J=(x_{i+2}, \ldots, x_j)$. Therefore $a\in I(I^n:x_t)+0:_Rx_t$ and
$$I^{n+1}:x_t\subseteq I(I^n:x_t)+0:_Rx_t.$$
\end{proof}

\begin{corollary}\label{36}
Keep the notations and assumption as in Lemma \ref{35}. Let $t=1$ if $i\geq 1$ or $t=j+1$ if $i=0$. For each $n\geq 0$, we have
$$\bigcup_{s>0}I^{n+1}:{\ensuremath{\mathfrak m}}^s=I^{n+1}:_Rx_t=I^n(I:_Rx_t)+0:_Rx_t.$$
\end{corollary}
\begin{proof}
By Proposition \ref{33}, $x_1, \ldots, x_i, x_{j+1}^2, \ldots, x_d^2$ is an almost p-standard system of parameters of $R/I^{n+1}$. Particularly they are strong d-sequence on $R/I^{n+1}$. Hence

$$
\bigcup_{s>0}I^{n+1}:{\ensuremath{\mathfrak m}}^s=
\begin{cases}
I^{n+1}:_Rx_1&\mbox{ if } i\geq 1;\\
I^{n+1}:_Rx_{j+1}^2&\mbox{ if } i=0.
\end{cases}
$$
In the second case, following Lemma \ref{35}, we have
$$I^{n+1}:_Rx_{j+1}^2=I^n(I:_Rx_{j+1}^2)+0:_Rx_{j+1}^2=I^n(I:_Rx_{j+1})+0:_Rx_{j+1}=I^{n+1}:_Rx_{j+1}.$$
It proves the corollary.
\end{proof}

Keep the notations as in Lemma \ref{35}, so $x_1, \ldots, x_d$ is an almost p-standard system of parameters of $R$ and $I=(x_{i+1}, \ldots, x_j)$ for $0\leq i<j\leq d$.  Let $t=1$ if $i\geq 1$ or $t=j+1$ if $i=0$. Denote $I_n=I^n:x_t$. We note that $I_1^2=I_2$. Then from Lemma \ref{35} we get that $I_nI_m=I_{n+m}$ for all $n, m>0$. This observation together with Corollary \ref{36} show that for all $n\geq 0$, $I_n=I_1^n$ and
$$h^0_I(n)=\ell_R(I_1^{n+1}/I^{n+1}).$$
Therefore $h^0_I(n)$ is the Rees function of the pair $(I, I_1)$ in the sense of \cite{HPV}. Combining with the main theorem of Amao in \cite{AMAO} and Herzog-Puthenpurakal-Verma \cite[Corollary 4.7]{HPV}, we obtain the main theorem of this section.

\begin{theorem}\label{37}
Let $R$ be a quotient of a Cohen-Macaulay local ring. Let $x_1, \ldots, x_d$ be an almost p-standard system of parameters of $R$. For $0\leq i<j\leq d$, let $I$ be the ideal generated by $x_{i+1}, \ldots, x_j$. Then there is a polynomial $P_I(n)$ such that $h^0_I(n)=P_I(n)$ for all $n\geq 0$. Moreover, the saturated Hilbert polynomial $P_I(n)$ has degree equal to the dimension of the graded module $\bigoplus_{n=1}^\infty I_1^n/I^n$ over the Rees algebra $\mathcal R(I)$.
\end{theorem}

A typical example of almost p-standard system of parameters is the standard system of parameters of a generalized Cohen-Macaulay ring. The theorem particularly asserts that if $I$ is an ideal generated by part of a standard system of parameters in a generalized Cohen-Macaulay local ring, then the saturated Hilbert polynomial of $I$ exists. We even can do more, the polynomial can be computed in terms of length for certain local cohomology modules of the ring. This is done in the next part of this section. For the definitions and results on generalized Cohen-Macaulay local rings, we refer to \cite{NVT2}.

From Lemma \ref{35}, we obtain the following well known property of standard system of parameters.

\begin{corollary}[\cite{NVT2}, Corollary 2.6 (v)] \label{38}
Let $R$ be a generalized Cohen-Macaulay local ring. Let $x_1, \ldots, x_d$ be a standard system of parameters of $R$. Denote $I=(x_1, \ldots, x_i)$ for some $0\leq i\leq d$. Then $$I^{n+1}:_Rx_1=I^n+0:_Rx_1,$$ for all $n\geq 0$.
\end{corollary}
\begin{proof} It suffices to show that $I^{n+1}:_Rx_1\subseteq I^n + 0:_R x_1$.

We have $I^{n+1}=x_1I^n+J^{n+1}$ where $J=(x_2, \ldots, x_i)$. Let $a\in R$ such that $ax_1\in I^{n+1}$, hence $ax_1=x_1b+c$ for some $b\in I^n$, $c\in J^{n+1}$. We have $a-b\in J^{n+1}:x_1$. Lemma \ref{35} applies and we have
$$a-b\in J^n+0:_Rx_1.$$
So $a\in I^n+0:_Rx_1$.
\end{proof}

\begin{theorem}\label{39}
Let $R$ be a generalized Cohen-Macaulay local ring and let $I$ be an ideal of $R$ generated by $i$ elements in a standard system of parameters of $R$. Then
$$h_I^0(n)=h^0(R)+\sum_{t=0}^{i-1}\Big(\sum_{j=0}^t\binom{t}jh^{j+1}(R)\Big)\binom{n+t}{t},$$
where $h^j(R)$ is the length of the $j$-th local cohomology module $H^j_{\ensuremath{\mathfrak m}}(R)$. In particular, $h_I^0(n)=0$ if $i<\mathrm{depth}(R)$ and $\deg(h_I^0(n))=i-1$ if $i\geq \mathrm{depth}(R)$.
\end{theorem}
\begin{proof}
Let $x_1, \ldots, x_d$ be a standard system of parameters of $R$ so that $x_1, \ldots, x_i$ generate $I$. Denote $R_1=R/x_1R$ and $I_1=(x_2, \ldots, x_i)R_1=IR_1$. The ring $R_1$ is again a generalized Cohen-Macaulay module. Recall that we denote $h^j(R)=\ell_R(H^j_{\ensuremath{\mathfrak m}}(R))$ for $j\not=\dim R$ and $h^0_{I, R}(n):=h^0_I(n)=\ell(H^0_{\ensuremath{\mathfrak m}}(R/I^{n+1}))$. We first prove the following claim which is necessary for induction process later.
\medskip

\noindent {\bf Claim.} $h_{I, R}^0(n)-h_{I, R}^0(n-1)=h_{I_1, R_1}^0(n)-h^0(R)-h^1(R),$ for all $n\geq 1$.
\medskip

In oder to prove the claim, we note by using Corollary \ref{38} that the map $R/I^{n+1}\stackrel{x_1}{\longrightarrow}R/I^{n+1}$ induces an exact sequence
$$0\longrightarrow R/(I^{n}+0:x_1) \stackrel{x_1}{\longrightarrow}R/I^{n+1}\longrightarrow R_1/I_1^{n+1}\longrightarrow 0.$$
Then the commutative diagram
\begin{displaymath}
\xymatrix{
0\ar[r]&R/(I^n+0:x_1)  \ar[r]^{x_1} \ar[d]_{x_{i+1}} &
R/I^{n+1}\ar[d]^{x_{i+1}}\ar[r]&R_1/I_1^{n+1}\ar[d]^{x_{i+1}}\ar[r]&0\\ 0\ar[r]& R/(I^n+0:x_1) \ar[r]^{x_1} & R/I^{n+1}\ar[r]&R_1/I_1^{n+1}\ar[r]&0}
\end{displaymath}
gives rise to an exact sequence
\begin{multline*}
0\longrightarrow  ((I^n+0:x_1):x_{i+1})/(I^n+0:x_1)\longrightarrow  I^{n+1}:x_{i+1}/I^{n+1}\\
\longrightarrow I^{n+1}R_1:x_{i+1}/ I_1^{n+1}
\longrightarrow {\operatorname{Ker}}(f)\longrightarrow  0,
\end{multline*}
where $f$ is the multiplicative map
$$R/(x_{i+1}R+ I^n+0:x_1)\stackrel{x_1}{\longrightarrow}R/(x_{i+1}R+I^{n+1}).$$
We have
$$\ell((I^{n+1}:x_{i+1})/I^{n+1})=\ell(H^0_{\ensuremath{\mathfrak m}}(R/I^{n+1})),$$
$$\ell(I^{n+1}R_1:x_{i+1}/I_1^{n+1})=\ell(H^0_{\ensuremath{\mathfrak m}}(R_1/I_1^{n+1})),$$
\[\begin{aligned}
\ell(((I^n+0:x_1):x_{i+1})/(I^n+0:x_1))
&=\ell(H^0_{\ensuremath{\mathfrak m}}(R/I^n+0:x_1))\\
&=\ell(H^0_{\ensuremath{\mathfrak m}}(R/I^n))-\ell(H^0_{\ensuremath{\mathfrak m}}(R)),
\end{aligned}\]
where $H^0_{\ensuremath{\mathfrak m}}(R)=0:_Rx_1$. Moreover, we have $(x_{i+1}R+I^{n+1}):x_1=I^n+x_{i+1}R:x_1$ by applying Corollary \ref{38} to the ring $R/x_{i+1}R$ and the sequence $x_1, \ldots, x_i$. Hence
\[\begin{aligned}
{\operatorname{Ker}}(f)&=((x_{i+1}R+I^{n+1}):x_1)/(x_{i+1}R+I^n+0:_Rx_1)\\
&\simeq (I^n+x_{i+1}R:x_1)/(I^n+x_{i+1}R+0:_Rx_1)\\
&\simeq x_{i+1}R:x_1/(x_{i+1}R+0:_Rx_1)\\
&\simeq H^0_{\ensuremath{\mathfrak m}}(R/(x_{i+1}R+0:_Rx_1)).
\end{aligned}\]
It implies that $\ell({\operatorname{Ker}}(f))=\ell(H^0_{\ensuremath{\mathfrak m}}(R/x_{i+1}R))-\ell(H^0_{\ensuremath{\mathfrak m}}(R))=\ell(H^1_{\ensuremath{\mathfrak m}}(R))$. So the claim is proved.
\medskip

The theorem is now proved by induction on $i$. If $i=1$, we have from the claim $h_I^0(n)-h_I^0(n-1)=h^0(R/x_1R)-h^1(R)-h^0(R)=0$, hence
$$h_I^0(n)=h^0_I(n-1)=h^0_I(0)=h^0(R)+h^1(R).$$
Assume $i>1$. Combining the induction assumption with the claim, we have
\[\begin{aligned}
h_I^0(n)-h_I^0(n-1)=&h^0(R/x_1R)-h^0(R)-h^1(R)\\
&+\sum_{t=0}^{i-2}\Big(\sum_{j=0}^t\binom{t}jh^{j+1}(R/x_1R)\Big)\binom{n+t}{t}.\end{aligned}\]
For the generalized Cohen-Macaulay ring $R$, we have $h^j(R/x_1R)=h^j(R)+h^{j+1}(R)$. A simple computation then gives us
$$h_I^0(n)=h^0(R)+\sum_{t=0}^{i-1}\Big(\sum_{j=0}^t\binom{t}jh^{j+1}(R)\Big)\binom{n+t}{t}.$$
\end{proof}

In the general case when $R$ is not necessarily a generalized Cohen-Macaulay local ring and the ideal is not generated by part of an almost p-standard system of parameters, the assumption that $I$ is generated by part of a system of parameters is not enough for the existence of the saturated Hilbert polynomial. We end with the following example.

\begin{example}\label{310}
By Cutkosky, Ha, Srinivasan and Theodorescu \cite[Theorem 2.2]{CTST}, there exist a regular local ring $(A,\frak n)$ and an ideal $I$ such that $h^0_I(n)$ is not asymptotic to a polynomial. We choose a system of generators of $I$, say $a_1, ..., a_r$. Let $S = A[T_1, ..., T_r]_{(T_1,..., T_r)+\frak n}$, we have $\dim S = \dim A + r$. Consider $A$ as an $S$-module we have $\mathrm{Ann}_SA = (T_1, ..., T_r)$. Let $J = (T_1+a_1, ..., T_r+a_r)$. It is easy to check that $S/J \cong A$. So $J$ is generated by a part of a system of parameters of $S$. Since $S$ is regular, we have  $S/J^n$ is Cohen-Macaulay for all $n$.

On the other hand we have $A/J^nA = A/I^n$. Let $\frak m = ((T_1,..., T_r)+\frak n)S$ we have $H^0_{\frak m}(S/J^n) = 0$ and $H^0_{\frak m}(A/J^nA) \cong H^0_{\frak n}(A/I^n)$.

Take the idealization $R = S \ltimes A$ and the ideal $\frak q = ((T_1+a_1, 0), \ldots, (T_r+a_r, 0))$. We have $\frak q$ is generated by part of a system of parameters of $R$ and
$$R/\frak q^n = S/J^n \ltimes A/I^n,$$
for all $n$. Let ${\ensuremath{\mathfrak m}}^\prime$ be the maximal ideal of $R$ we have
$$\ell(H^0_{{\ensuremath{\mathfrak m}}^\prime}(R/\frak q^n)) = \ell(H^0_{\frak n}(A/I^n)),$$
which is not asymptotic to a polynomial in $n$.

Finally, it should be remarked that the ring $R$ is not a domain. It would be interesting to find a counterexample to be a domain.
\end{example}

\begin{thebibliography}{99}
\bibitem{AMAO}
J. O. Amao, On a certain Hilbert polynomial. {\it J. London Math. Soc.} (2) {\bf 14} (1976), no. 1, 13–20.

\bibitem{BS}
M. Brodmann and R. Y. Sharp, {\it Local Cohomology: An Algebraic Introduction with Geometric Applications}, Cambridge University Press 1998.

\bibitem{BS1}
M. Brodmann and R. Y. Sharp, On the dimension and multiplicity of local cohomology modules. {\it Nagoya Math. J.} {\bf 167} (2002), 217-233.

\bibitem{DTCNam}
D. T. Cuong and P. H. Nam, Hilbert coefficients and partial Euler-Poincar\'e characteristics of Koszul complexes of d-sequences. {\it J. Algebra} {\bf 441} (2015), 125-158.

\bibitem{NTCDTC1} N. T. Cuong and D. T. Cuong, dd-Sequences and partial Euler-Poincar\'{e} characteristics of Koszul complex. {\it J. Algebra and Its Applications} {\bf 6}(2) (2007), 207-231.

\bibitem{NTCDTC3}
N. T. Cuong, D. T. Cuong, On sequentially Cohen–Macaulay modules. {\it Kodai Math. J.} {\bf 30} (2007) 409-428.

\bibitem{NTCDTC2} N. T. Cuong and D. T. Cuong, Local cohomology annihilators and Macaulayfication. To appear in {\it Acta Mathematica Vietnamica} (2016). Doi:10.1007/s40306-016-0185-9.

\bibitem{NTCNhan} N. T. Cuong and L. T. Nhan, Dimension, multiplicity and Hilbert function of Artinian modules. {\it East-West J. Math.} {\bf 1}(2) (1999), 179-196.

\bibitem{CTST} S.D. Cutkosky, H\`a Huy T\`ai, H. Srinivasan, E. Theodorescu, Asymptotic behavior of the length of local cohomology. {\it Canad. J. Math.} {\bf 57}(6) (2005), 1178-1192.

\bibitem{HPV}
J. Herzog, T. J. Puthenpurakal, J. K. Verma, Hilbert polynomials and powers of ideals. {\it Math. Proc. Camb. Phil. Soc.} {\bf 145}(3) (2008), 623-642.

\bibitem{HU}
C. Huneke, Theory of d-sequences and powers of ideals. {\it Adv. in Math.} {\bf 46} (1982), 249-279.

\bibitem{Kir} D. Kirby, Artinian modules and Hilbert polynomials. {\it Quart. J. Math. Oxford} {\bf 24}(2) (1973), 47-57.

\bibitem{KUV} S. Kleiman, B. Ulrich and J. Validashti, Multiplicities, integral dependence and equisingularity. Preprint.

\bibitem{LT} C. H. Linh and N. V. Trung, Uniform bounds in generalized Cohen-Macaulay rings. {\it Journal of Algebra} {\bf 304} (2) (2006), 1147-1159.

\bibitem{NVT1} N. V. Trung, Absolutely superficial sequences. {\it Math. Proc. Cambridge Philos. Soc.} {\bf 93} (1983), 35-47.

\bibitem{NVT2}
N. V. Trung, Toward a theory of generalized Cohen-Macaulay modules. {\it Nagoya Math. J.} {\bf 102} (1986), 1-49.

\bibitem{UV} B. Ulrich and J. Validashti, Numerical criteria for integral dependence. {\it  Math. Proc. Camb. Phil. Soc.} {\bf 151}(1) (2011), 95-102.
\end{thebibliography}

\end{document}


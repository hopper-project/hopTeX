

\documentclass[12pt]{amsart}

\usepackage{mathrsfs}
\usepackage{txfonts}
\usepackage{amssymb,dsfont}
\usepackage{enumerate}
\usepackage{slashed}
\usepackage{fancyhdr}
\usepackage{aliascnt}
\usepackage[backref,pagebackref,breaklinks,colorlinks,linkcolor=red,citecolor=green,filecolor=magenta,urlcolor=cyan]{hyperref}
\usepackage[square,super,sort,longnamesfirst]{natbib}
\usepackage[dvips,dvipsnames]{xcolor}
\usepackage[dvips]{graphicx}
\usepackage[all,cmtip]{xy}
\usepackage{pdfsync}

\makeatletter

\makeatother

\textwidth 16.20cm \textheight 22cm \topmargin 0.1cm
\oddsidemargin 0.1cm \evensidemargin 0.1cm
\parskip 0.0cm

\theoremstyle{plain}
\newtheorem{known}{Theorem}
\providecommand*{\knownautorefname}{Theorem}

\newtheorem{theorem}{Theorem}[section]

\newaliascnt{lem}{theorem}
\newtheorem{lem}[lem]{Lemma}
 \aliascntresetthe{lem}
\providecommand*{\lemautorefname}{Lemma}

\newaliascnt{cor}{theorem}
\newtheorem{cor}[cor]{Corollary}
 \aliascntresetthe{cor}
\providecommand*{\corautorefname}{Corollary}

\newaliascnt{prop}{theorem}
\newtheorem{prop}[prop]{Proposition}
 \aliascntresetthe{prop}
\providecommand*{\propautorefname}{Proposition}

\newtheorem*{conj}{Conjecture}

\theoremstyle{remark}
\newtheorem{rem}{Remark}[section]
\providecommand*{\remautorefname}{Remark}
\newtheorem*{claim}{Claim}
\newtheorem*{fact}{Fact}
\newtheorem{case}{Case}
\providecommand*{\caseautorefname}{Case}

\theoremstyle{definition}
\newtheorem{defn}{Definition}[section]
\providecommand*{\defnautorefname}{Definition}
\newtheorem{propy}{Property}[section]
\providecommand*{\propyautorefname}{Property}
\newtheorem{probm}{Problem}[section]
\providecommand*{\probmautorefname}{Problem}
\newtheorem{eg}{Example}[section]
\providecommand*{\egautorefname}{Example}

\numberwithin{equation}{section}

\begin{document}

\title[lower bounds of eigenvalues]{A sharp lower bounds of eigenvalues for differential forms and  homology sphere theorems}
\thanks{This work is partially supported by National Nature Science Foundation
 of China (Grant No. 11601442) and
Fundamental Research Funds for the Central Universities (Grant No. 2682016CX114, WK0010000055).}

\author{Qing Cui}
\address{School of Mathematics,
Southwest Jiaotong University, 611756
Chengdu, Sichuan, China}
\email{qingcui@impa.br}

\author{Linlin Sun}
\address{School of Mathematics Sciences,
University of Science and Technology of China,
230026 Hefei, Anhui, China}
\email{sunll@ustc.edu.cn}

\subjclass[2010]{58J50; 53C24; 53C40}

\keywords{Hodge Laplacian, eigenvalue estimate, rigidity theorem,
homology sphere theorem}

\maketitle
\begin{abstract}
In this paper, we investigate two main problems. First, we
obtain an extrinsic sharp lower bounds of eigenvalues for Hodge-Laplacian operator on
a Riemannian manifold isometrically immersed into another for arbitrary
codimension, which is a natural generalization for Savo's result \cite{Sav14}. As applications, we obtain some
rigidity results and a homology sphere theorem under the assumption that the second fundamental form
bounded from above. Second,
we prove another homology sphere theorem when the ambient manifold
is more general than space form, which is a
 generalization of Gu-Xu's result \cite[Theorem 4.1]{XuGu13} in the homology sense.

\end{abstract}\section{Introduction}
The aim of the present paper is two-fold. First, we wish to give an extrinsic eigenvalue estimate
for Hodge-Laplacian operator on a Riemannian manifold $M$ isometrically immersed into a Riemannian
manifold $\bar{M}$ with arbitrary codimension. As applications, we will prove some results including a rigidity theorem
and a homology sphere theorem when the second fundamental form bounded from above.
Second, we are interested in finding the relation between the Ricci curvature and the topology of a submanifold
immersed into a manifold more general than a space form.

Eigenvalue estimate for Hodge-Laplacian play an important role in topology and geometry.
It has been being active since  its appear and  plenty of
works concerned on this topic are published. Most of these works are devoted to the  intrinsic eigenvalue
estimate for Hodge-Laplacian, we refer the reader to a classical survey \cite{Cha84}.
Recent years,  extrinsic eigenvalue estimate for Hodge-Laplacian attract several authors' attention,
such as (uncomplete list) Savo \cite{Sav05,Sav14}, Raulot and Savo
\cite{RauSav11,RauSav12} and Kwong \cite{Kwo16}.
In particular, Savo obtain the following eigenvalue estimate when the ambient manifold having
curvature operator bounded from below.

\begin{known}[Savo\cite{Sav14}]
 Let $M$ be a closed hypersurface of $\bar{M}^{n+1}$, a manifold with
 curvature operator bounded from below by $c\in\mathbb R$. Let $ 1\le p \le \frac{n}{2}$. Then
$$
\lambda_{1,p}(M)\ge p(n-p+1)(c+\beta_p(\bar{M})),
$$
where $\beta_p(M)$ is a constant depending on the infimum of the $p$-th curvature
of $\bar{M}$ (for the definition, see \cite{Sav14}). If $M$ is a geodesic
sphere in a simply connected
manifold of constant curvature $c$, then equality holds.
\end{known}
It is natural to ask whether the above eigenvalue estimate is still true if the
codimension is high.
To this end, we obtain the following result.

\begin{theorem}\label{eigenest}
Suppose $M^n$ is a closed submanifold in $\bar{M}^{n+m}$
with $i^*\bar{W}^{[p]}\geq c$. Let $1\le p\le\frac{n}{2}$. Then
\begin{align*}
\lambda_{1,p}(M)\geq&p(n-p+1)
\left(c+\gamma_p\right).
\end{align*}
where $$\gamma_p=\inf_{x\in M}\left\{\left(-\dfrac{1}{n}\abs{\mathring{B}}^2
-\dfrac{(n-2p)\abs{H}}{\sqrt{np(n-p)}}
\abs{\mathring{B}}+\abs{H}^2\right)(x)\right\},$$ and $H$ is mean curvature vector and $\mathring{B}$ is the
traceless part of the second fundamental form $B$.
\end{theorem}
It is easy to see the above eigenvalue estimate is sharp, since the case
 $M=\mathbb S^n$ and $\bar{M}=\mathbb R^{n+m}$ attains the equality.
It is also worth pointing out that the condition  $i^*\bar{W}^{[p]}\ge c$
($i^*\bar{W}^{[p]}$ is the pull back Weitzenb\"{o}ck operator defined in section 2)
 is weaker
than the condition that curvature operator bounded from below by $c$.
As an application, we have the following rigidity result.
\begin{theorem}\label{rigidthm}
Suppose $M$ is a closed submanifold in $\bar M$, $p\in\set{1,\dotsc, n-1}, i^*\bar W^{[p]}\geq c\geq0$ and $\abs{B}^2\leq\alpha(c,p,n,H)$,
then the $p$-th betti number $b_p\leq\binom{n}{p}$,
where
\begin{align*}
\alpha(c,p,n,H)\coloneqq nc+\dfrac{n^3\abs{H}^2}{2p(n-p)}-\dfrac{n\abs{n-2p}\abs{H}\sqrt{n^2\abs{H}^2+4cp(n-p)}}{2p(n-p)}.
\end{align*}
Moreover, if $b_p>0$, then $\vert B\vert^2\equiv\alpha(c,p,n,H)$.

As a consequence, if $\abs{B}^2\leq\alpha(c,1,n,H)$, then $b_p\leq\binom{n}{p}$ for every $p\in\set{0,\dotsc,n}$. Moreover, if $b_p>0$ for some $p\in\set{1,\dotsc, n-1}$, then $\abs{B}^2\equiv\alpha(c,1,n,H)$. In particular, if $\chi(M)\neq1+(-1)^n$, then $\abs{B}^2\equiv\alpha(c,1,n,H)$.
\end{theorem}
The following two corollaries are direct consequences of the above rigidity theorem.
\begin{cor}If the assumptions of Theorem \ref{rigidthm} hold, and moreover if
 $n=0 \ \ (\mod4)$, $\abs{B}^2\leq\alpha(c,1,n,H)$ and the signature $sig(M)$ of $M$ is nonzero, then $\abs{B}^2\equiv\alpha(c,1,n,H)$.
\end{cor}
\begin{cor}If the assumptions of Theorem \ref{rigidthm} hold, and moreover if
 $\abs{B}^2\leq\alpha(c,1,n,H)$ and $b_p>0$ for some $1<p<n-1$, then either $\mathring{B}\equiv0$ or $M$ is minimal satisfying $\abs{B}^2\equiv nc$. In particular, if $\chi(M)\neq 1+(-1)^n$, then there is some $b_p>0$ for some $1<p<n-1$.
\end{cor}
Applying the rational Hurewicz theorem, we have the following homology sphere theorem.
\begin{theorem}\label{rationalsphere1}
If the assumptions of Theorem \ref{rigidthm} hold, and moreover if
$\abs{B}^2<\alpha(c,1,n,H)$ and $M$ is simply connected,
then $M$ is a rational homotopy sphere, i.e.,
$\pi_i(M,\Q)=0$ for all $1\leq i\leq n-1$.
\end{theorem}

In the second main part (Section 4), we will show another homology
sphere theorem.
Sphere theorems are studied extensively recent years, we refer the reader to \cite{XuGu13}
for a survey. There are several type of
sphere theorems depending on the curvature assumptions
 added on the submanifold or the ambient manifold. These curvature assumptions include
  pinched sectional curvature,
 bounded Ricci curvature, etc.
In Section 4, we will restrict our attention to the case that the Ricci curvature of the
submanifold bounded from below.
As far as we know, the first such type result was given by Ejiri in 1979
for minimal submanifolds of a sphere. Recently, Gu-Xu generalize Ejiri's result to submanifolds of space forms with parallel mean curvature vector.

\begin{known}[Ejiri\cite{Eji79}, Gu-Xu\cite{XuGu13}]
Let $M$ be an $n(\geq3)$-dimensional complete submanifold with parallel mean curvature vector $H$ in $F^{n+m}(c)$ with $c+\abs{H}^2>0$. If the Ricci curvature of $M$ satisfies
\begin{align*}
Ric_M\geq(n-2)\left(c+\abs{H}^2\right),
\end{align*}
then $M$ is either the totally geodesic submanifold $\mathbb{S}^n\left(\tfrac{1}{\sqrt{c+\abs{H}^2}}\right)$, the Clifford torus $\mathbb{S}^l\left(\tfrac{1}{\sqrt{2(c+\abs{H}^2)}}\right)\times\Lg{S}^l\left(\tfrac{1}{\sqrt{2(c+\abs{H}^2)}}\right)$ in $\Lg{S}^{n+1}\left(\tfrac{1}{\sqrt{c+\abs{H}^2}}\right)$ with $n=2l$, or $\Com P^2\left(\tfrac{4}{3}(c+\abs{H}^2)\right)$ in $\mathbb{S}^7\left(\tfrac{1}{c+\abs{H}^2}\right)$. Here $\Com P^2\left(\tfrac{4}{3}(c+\abs{H}^2)\right)$ denotes the $2$-dimensional complex projective space minimally immersed into $\mathbb{S}^7\left(\tfrac{1}{c+\abs{H}^2}\right)$ with constant holomorphic sectional curvature $\tfrac{4}{3}(1+\abs{H}^2)$.
\end{known}
Gu-Xu \cite{XuGu13} also obtain the following topological sphere theorem without the assumption of parallel mean curvature vector.
\begin{known}[Gu-Xu\cite{XuGu13}]\label{GXthm}Let $M$ be an $n$-dimensional closed submanifold with mean curvature vector $H$ in $F^{n+m}(c)$ with $c\geq0$. If the Ricci curvature of $M$ satisfies
\begin{align}\label{eq:condition-eriji}
Ric_M>(n-2)\left(c+\abs{H}^2\right),
\end{align}
then $M$ is homoemorphic to a sphere.
\end{known}

The original version of Gu-Xu's theorem assume that $n\geq4$. The case $n=2$ is a consequence of Gauss-Bonnet formula. The case $n=3$ is a consequence of Lawson-Simons theorem and Perelman's solution of Poincar\'{e} conjecture.

The key idea to prove Theorem \ref{GXthm} is to claim that there is no
stable integral $p$-currents for $0<p<n$ under the assumption \eqref{eq:condition-eriji}. The {\it $p$-th weak Ricci curvature} of the $p$-plane $e_1\wedge e_2\wedge\dotsm\wedge e_p$ introduced by Xu-Gu\cite{GuXu12} is defined by
\begin{align*}
Ric(e_1\wedge e_2\wedge\dotsm\wedge e_p)\coloneqq\sum_{i=1}^pRic_{ii}.
\end{align*}
One can verify that $Ric(e_1\wedge e_2\wedge\dotsm\wedge e_p)$ is well defined, i.e., it is depending only on the $p$-plane $e_1\wedge e_2\wedge\dotsm\wedge e_p$. With an obvious modification of original results of Gu-Xu\cite{XuLenGu14,XuGu13}, one can obtain the following Theorem (for readers' convenience, we list a proof in Section 4).
\begin{known}\cite{XuLenGu14,XuGu13}\label{gclaim}
Let $M$ be an $n(\geq4)$-dimensional closed submanifold  with mean curvature vector $H$  in $F^{n+m}(c)$ with $c\geq0$.
If
\begin{align*}
\dfrac{Ric_{(p)}}{p}>\left(n-1-\dfrac{(n-2)p(n-p)}{(n+2)p(n-p)-n^2}\right)\left(c+\abs{H}^2\right),\quad 1<p<n-1,
\end{align*}
where $Ric_{(p)}$ is the lower bound of the $p$-th Ricci curvature, then there is no stable integral $p$-currents.
In particular, if
\begin{align*}
Ric>(n-2)\left(c+\abs{H}^2\right)\geq0,
\end{align*}
then $M$ is homoemorphic to a sphere.
\end{known}

Note that the ambient manifold in the above three results are still of constant curvature $c$.
In Section 4, we also replace this condition with a weaker curvature condition and give a homology sphere theorem via the method of eigenvalue estimates for differential forms.
\begin{theorem}\label{thm:eriji}Suppose $M^n$ is a closed submanifold of $\bar{M}^{n+m}$ with $i^*\bar{W}^{[p]}\geq p(n-p)c_*$ and $i^*\bar{W}^{[1]}\leq (n-1)c^*$. If
\begin{align*}
Ric>(n-1)\left(c^*+\abs{H}^2\right)-\dfrac{(n-2)p(n-p)}{(n+2)p(n-p)-n^2}\left(c_*+\abs{H}^2\right),
\end{align*}
holds for some $0<p<n$, then the $p$-th Betti number is zero.

As a consequence, if $M$ is simply connected, $i^*\bar{W}^{[p]}\geq p(n-p)c_*$  for every $0<p<n$, $i^*\bar W^{[1]}\leq (n-1)c^*$ and
\begin{align*}
Ric>(n-1)\left(c^*+\abs{H}^2\right)-\left(c_*+\abs{H}^2\right),
\end{align*}
then $M$ is a homology sphere.
\end{theorem}
We call the above result a {\it weak Ejiri type theorem}. It is weak in the sense that
$M$ is just a homology sphere in our conclusion.
Therefore, it can be seen as a generalization
of Theorem \ref{GXthm} in the rational homotopy sense.
We emphasize that the proof of Theorem \ref{thm:eriji}
based on the cohomology theory and some algebraic Lemmas
 which is quite
different from Ejiri's and Gu-Xu's.

The paper is organized as follows. In Section 2, we set up notation and terminology, and review some
of the standard facts on submanifolds geometry and  Hodge-Laplace operator. We also prove two lemmas
which are crucial in the proof of Theorem \ref{eigenest}.
In Section 3, we give the proof of Theorem \ref{eigenest}, Theorem \ref{rigidthm}. We also give
another two applications of Theorem \ref{eigenest} and an example which attains the equality in the
rigidity theorem. In Section 4, we give a useful algebraic lemma and  prove  Theorem \ref{thm:eriji}.

{\bf Acknowledgements}: The authors would like to thank Professor Gu Juanru for helpful suggestions.

\section{Preliminaries}
In this section, we first recall some of the standard facts on submanifold geometry.

Let $i:M^n\to \bar{M}^{n+m}$ be an isometric immersion from a closed $n$-dimensional
Riemannian manifold $M$ to an $(n+m)$-dimensional Riemannian manifold $\bar{M}$. Let $e_1, \cdots, e_n, \nu_1, \cdots,\nu_m$ be an orthonormal frame on $\bar{M}$ such that $e_1, \cdots, e_n$ are tangent to $M$ and $\nu_1, \cdots,\nu_m$ are perpendicular to $M$, and $\eta^1, \cdots, \eta^n$ be the dual of $e_1, \cdots, e_n$. Let $R$ (resp. $\bar{R}$) be the (0,4)-type curvature tensor of $M$ (resp. $\bar{M}$), and $\mathcal{R}:\Lambda^2TM\to \Lambda^2TM$ be the curvature operator defined by
$$
\langle \mathcal{R}(e_i\wedge e_j),e_k\wedge e_l\rangle = R(e_i,e_j,e_k,e_l)\eqqcolon R_{ijkl}.
$$
From now on, we assume the Latin subscripts (or superscripts) $i,j,k,l,\cdots$  range from 1 to $n$, and the Greek subscripts (or superscripts) $\alpha,\beta, \gamma, \cdots$ range from 1 to $m$, and we will adopt the Einstein summation rule.
The second fundamental form and the mean curvature vector are given by
$$
B=h^\alpha_{ij} \eta^i\otimes\eta^j\otimes\nu_\alpha,\quad H=\frac{1}{n}\sum_i h_{ii}^\alpha\nu_\alpha\eqqcolon H^\alpha\nu_\alpha.
$$
and write $\mathring{B} = B - H\otimes g$ which is the traceless part of $B$, where $g$ is the metric on $M$.
Let $A$ be the shape operator defined by
$$
\hin{B(X,Y)}{\nu}= \hin{A^{\nu}(X)}{Y}, \ \  \text{for all }\ \  X,Y\in TM \ \ \text{and }\ \ \nu\in T^*M,
$$
Write $$
A^\alpha:= A^{\nu^\alpha}\quad \text{and} \quad \mathring{A}^\alpha := A^\alpha - g.$$
Recall the  Gauss equation
\begin{align}\label{eq:gauss}
R_{ijkl}=\bar R_{ijkl}+\sum_{\alpha=1}^m\left(h_{ik}^{\alpha}h_{jl}^{\alpha}-h_{il}^{\alpha}h_{jk}^{\alpha}\right).
\end{align}

Second, we summarize the relevant material on Hodge-Laplace operator and its first eigenvalue.

Let $\Delta$ be the Hodge Laplacian operator, i.e., $\Delta={\mathrm d}\delta+\delta{\mathrm d}$. Since $[{\mathrm d},\Delta]=[\delta,\Delta]=0$, we have
\begin{align*}
\Delta:{\mathrm d}\Omega^p(M){\longrightarrow}{\mathrm d}\Omega^p(M),\\
\Delta:\delta\Omega^p(M){\longrightarrow}\delta\Omega^p(M).
\end{align*}
Let $\lambda_{1,p}^e$ and $\lambda_{1,p}^{ce}$ be the first eigenvalue of $\Delta$ acting on the exact and co-exact $p$-form on $M$ respectively. It is easy to check that the first eigenvalue $\lambda_{1,p}$ of $\Delta$ satisfying
\begin{align*}
\lambda_{1,p}\leq\min\set{\lambda_{1,p}^e,\lambda_{1,p}^{ce}}.
\end{align*}
By Hodge decomposition, we know that
\begin{align*}
\lambda_{1,p}=\min\set{\lambda_{1,p}^e,\lambda_{1,p}^{ce}},
\end{align*}
provided $H^p(M,{\field{R}})=0$.
 By Hodge duality,
\begin{align*}
\lambda_{1,p}^{e}=\lambda_{1,n-p}^{ce}.
\end{align*}
Thus, by differentiating eigenforms, we obtain that if $H^p(M,{\field{R}})=0$, then
\begin{align*}
\lambda_{1,p-1}^{ce}\leq\lambda_{1,p}^{e},\quad\lambda_{1,p+1}^{e}\leq\lambda_{1,p}^{ce}.
\end{align*}
In particular, if $H^{p-1}(M,{\field{R}})=H^{p}(M,{\field{R}})=0$, we have
\begin{align*}
\lambda_{1,p-1}^{ce}=\lambda_{1,p}^{e}.
\end{align*}
Moreover, if $H^{p}(M,{\field{R}})=0$, then
\begin{align*}
\min\set{\lambda_{1,p-1},\lambda_{1,p+1}}\leq\lambda_{1,p}.
\end{align*}
For example,
\begin{align*}
\lambda_{1,p}^{e}\left(\Lg{S}^n(1)\right)=p(n-p+1),\quad\lambda_{1,p}^{ce}\left(\Lg{S}^n(1)\right)=(p+1)(n-p).
\end{align*}

Third, we briefly sketch the Weitzenb\"{o}ck formula and Bochner formula for differential forms.

For every $p$-form $\omega$ on $M$, the operator $W^{[p]}:\Lambda^pT^*M\to \Lambda^pT^*M$ is defined by
$$
W^{[p]}(\omega)=\eta^i\wedge\iota_{e_j}R(e_j,e_i)\omega.
$$
The following two  formulas for $p$-forms are well known,
\begin{align}\label{Wformula}
\Delta\omega&=\nabla^\star\nabla\omega+W^{[p]}(\omega),\\
\label{Bformula}\dfrac12\Delta\abs{\omega}^2&=\abs{\nabla\omega}^2
-\hin{\Delta\omega}{\omega}
+\hin{W^{[p]}(\omega)}{\omega},
\end{align}
where   $\nabla^\star\nabla$ is the connection Laplacian.
Equalities (\ref{Wformula}) and (\ref{Bformula}) are usually called Weitzenb\"{o}ck formula and Bochner formula.
A direct computation gives (c.f. \cite{LawMic89})
\begin{align*}
\hin{W^{[p]}(\omega)}{\omega}=\dfrac14\hin{\mathcal{R}(\theta_I)}{\theta_J}\hin{\La{ad}_{\theta_I}\omega}{\La{ad}_{\theta_J}\omega},
\end{align*}
where $\{\theta_I\}$ is a local orthonormal frame of $\Lambda^2TM$.

Next we will prove two lemmas which are crucial in the eigenvalue estimate.
For stating the lemma, we need introduce two notations,
\begin{align*}
S(\omega)\coloneqq\eta^i\wedge\iota_{A^{\alpha}(e_i)}\omega\otimes\nu_{\alpha},
\quad
\mathring{S}(\omega)\coloneqq\eta^i\wedge\iota_{\mathring{A^{\alpha}}(e_i)}\omega\otimes\nu_{\alpha}.
\end{align*}

\begin{lem}For every $p$-form on $M$,
\begin{align*}
\hin{W^{[p]}(\omega)}{\omega}=\hin{i^*\bar W^{[p]}(\omega)}{\omega}-\abs{S(\omega)-\dfrac{n}{2}H\omega}^2+\dfrac{n^2}{4}\abs{H}^2,
\end{align*}
where
\begin{align*}
i^*\bar W^{[p]}(\omega)=\eta^i\wedge\iota_{e_j}\bar R(e_j,e_i)\omega.
\end{align*}
\end{lem}
\begin{proof}It is a  direct calculation. Since
\begin{align*}
W^{[p]}=&\eta^i\wedge\iota_{e_j}R(e_j,e_i)\\
=&R_{ijkl}\eta^j\wedge\iota_{e_i}\left(\eta^k\wedge\iota_{e_l}\right).
\end{align*}
Hence, by using Gauss equation \eqref{eq:gauss}
\begin{align*}
&\hin{W^{[p]}(\omega)}{\omega}-\hin{i^*\bar W^{[p]}(\omega)}{\omega}\\
=&\left(R_{ijkl}-\bar R_{ijkl}\right)\hin{\eta^i\wedge\iota_{e_j}\omega}{\eta^k\wedge\iota_{e_l}\omega}\\
=&\sum_{\alpha}\left(h^{\alpha}_{ik}h^{\alpha}_{jl}-h^{\alpha}_{il}h^{\alpha}_{jk}\right)\hin{\eta^i\wedge\iota_{e_j}\omega}{\eta^k\wedge\iota_{e_l}\omega}\\
=&\sum_{\alpha}\hin{\eta^i\wedge\iota_{A^{\alpha}(e_l)}\omega}{A^{\alpha}(e_i)\wedge\iota_{e_l}\omega}-\sum_{\alpha}\hin{\eta^i\wedge\iota_{A^{\alpha}(e_k)}\omega}{\eta^k\wedge\iota_{A^{\alpha}(e_i)}\omega}\\
=&\sum_{\alpha}\hin{\iota_{A^{\alpha}(e_l)}\omega}{nH^{\alpha}\iota_{e_l}\omega+A^{\alpha}(e_i)\wedge\iota_{e_l}\iota_{e_i}\omega}-\sum_{\alpha}\hin{\iota_{A^{\alpha}(e_k)}\omega}{\iota_{A^{\alpha}(e_k)}\omega-\eta^k\wedge\iota_{e_i}\iota_{A^{\alpha}(e_i)}\omega}\\
=&\sum_{\alpha}\hin{\eta^l\wedge\iota_{nH^{\alpha}\iota_{A^{\alpha}(e_l)}}\omega}{\omega}-\sum_{\alpha}\abs{\sum_l\eta^l\wedge\iota_{A^{\alpha}(e_l)}\omega}^2+\sum_{\alpha}\abs{\sum_{k}\iota_{e_k}\iota_{A^{\alpha}(e_k)}\omega}^2\\
=&\sum_{\alpha}\hin{\eta^l\wedge\iota_{nH^{\alpha}\iota_{A^{\alpha}(e_l)}}\omega}{\omega}-\sum_{\alpha}\abs{\sum_l\eta^l\wedge\iota_{A^{\alpha}(e_l)}\omega}^2\\
=&-\sum_{\alpha}\abs{\sum_l\eta^l\wedge\iota_{A^{\alpha}(e_l)}\omega-\dfrac{n}{2}H^{\alpha}\omega}^2+\dfrac{n^2}{4}\abs{H}^2.
\end{align*}
\end{proof}

Now we can state the following estimate.
\begin{lem}\label{lem:W3}Assume $i^*\bar W^{[p]}\geq p(n-p)c$, then restricted on $M$,
\begin{align*}
W^{[p]}\geq p(n-p)\left(c-\dfrac{1}{n}\abs{\mathring{B}}^2-\dfrac{\abs{n-2p}\abs{H}}{\sqrt{np(n-p)}}\abs{\mathring{B}}+\abs{H}^2\right).
\end{align*}
\end{lem}
\begin{proof}It is sufficient to prove that when acting on $p$-forms,
\begin{align*}
\abs{\mathring{S}\vert_{\Lambda^pT^*M}}^2_{op}\leq\dfrac{p(n-p)}{n}\abs{\mathring{A}}^2,
\end{align*}
where $\abs{\cdot}_{op}$ stands for the operator norm.

By definition, acting on $p$-forms,
\begin{align*}
\mathring{S}(\omega)=\eta^i\wedge\iota_{\mathring{A}^{\alpha}(e_i)}\omega\otimes\nu_{\alpha}\eqqcolon\mathring{S}^{\alpha}(\omega)\otimes\nu_{\alpha},
\end{align*}
we have
\begin{align*}
\abs{\mathring{S}}_{op}^2=\left(\sup_{0\neq\omega\in\Lambda^pT^*M}\dfrac{\abs{\mathring{S}(\omega)}}{\abs{\omega}}\right)^2\leq\sum_{\alpha=1}^m\abs{\mathring{S}^{\alpha}}_{op}^2.
\end{align*}
Moreover, $\abs{\mathring{A}}^2=\sum_{\alpha=1}^m\abs{\mathring{A}^{\alpha}}^2$. Hence, to complete the proof, it is sufficient to consider the hypersurface case, i.e., $m=1$.
For the proof, we refer the reader to \cite{RauSav11} for example.
\end{proof}

In the end of this section, let us recall the following Theorem due to Lawson-Simons \cite{LawSim73} ($c>0$) and Xin \cite{Xin84} ($c=0$).
\begin{known}[Lawson-Simons, Xin]Suppose $M^n\subset F^{n+m}(c), c\geq0$ and for every orthonormal frame $\set{e_i}$ of $TM$,
\begin{align}\label{eq:L-S}
\sum_{i=1}^p\sum_{j=p+1}^n\sum_{\alpha=1}^m\left(2\left(h_{ij}^{\alpha}\right)^2-h_{ii}^{\alpha}h_{jj}^{\alpha}\right)<p(n-p)c,
\end{align}
then there is  no stable integral $p$-currents, where $F^{n+m}(c)$ is the $(n+m)$-dimensional space form with sectional curvature $c$.
\end{known}

\section{Eigenvalue estimate and its applications}
In this section, we give the proofs of Theorem \ref{eigenest} and Theorem \ref{rigidthm} and
their applications.
\begin{proof}[Proof of Theorem \ref{eigenest}]
Introduce the twistor operator $P$ on $M$ acting on $p$-form $\omega$ by
\begin{equation*}
P_X\omega\coloneqq\nabla_X\omega-\dfrac{1}{p+1}\iota_X{\mathrm d}\omega+\dfrac{1}{n+1-p}X^{\flat}\wedge\delta\omega,
\end{equation*}
where $X^{\flat}$ is the dual $1$-form defined by $X^{\flat}(e_i)=\hin{X}{e_i}$. Then the following identity holds,
\begin{align*}
\abs{\nabla\omega}^2=\abs{P\omega}^2+\dfrac{1}{p+1}\abs{{\mathrm d}\omega}^2+\dfrac{1}{n+1-p}\abs{\delta\omega}^2.
\end{align*}
Now applying Bochner formula (\ref{Bformula}), we obtain
\begin{align*}
\dfrac{p}{p+1}\int_M\abs{{\mathrm d}\omega}^2+\dfrac{n-p}{n+1-p}\int_M\abs{\delta\omega}^2=\int_M\abs{P\omega}^2+\hin{W^{[p]}(\omega)}{\omega}.
\end{align*}

Suppose $W^{[p]}\geq p(n-p)f$, then by variational characteristic of the first eigenvalue, we have
\begin{align*}
\lambda_{1,p}\geq&p(n+1-p)\times\min f.
\end{align*}
According to \autoref{lem:W3}, we know that
\begin{align*}
f\geq c+\dfrac{n^2\abs{H}^2}{4p(n-p)}-\dfrac{1}{p(n-p)}\abs{\mathring{S}\vert_{\Lambda^pT^*M}-\dfrac{n-2p}{2}H}^2_{op}.
\end{align*}
\end{proof}

\begin{proof}[Proof of Theorem \ref{rigidthm}]
Notice that
\begin{align*}
c\geq\max\set{\dfrac{1}{n}\abs{\mathring{B}}^2+\dfrac{\abs{n-2p}\abs{H}}{\sqrt{np(n-p)}}\abs{\mathring{B}}-\abs{H}^2}
\end{align*}
is equivalent to
\begin{align*}
\abs{B}^2\leq nc+\dfrac{n^3\abs{H}^2}{2p(n-p)}-\dfrac{n\abs{n-2p}\abs{H}\sqrt{n^2\abs{H}^2+4cp(n-p)}}{2p(n-p)}.
\end{align*}
Hence, by assumption,  $\lambda_{1,p}(M)\geq0$. Thus, every harmonic $p$-form is a conformal killing form, i.e., $P\omega=0$, and is parallel. Moreover, if for some point, the strictly inequality holds, then there is no nontrivial harmonic $p$-form. In other words, $H^p(M,{\field{R}})=0$.

Notice that
\begin{align*}
&\min_{p\in\set{1,\dotsc,n-1}}\set{nc+\dfrac{n^3\abs{H}^2}{2p(n-p)}-\dfrac{n\abs{n-2p}\abs{H}\sqrt{n^2\abs{H}^2+4cp(n-p)}}{2p(n-p)}}\\
=&nc+\dfrac{n^3\abs{H}^2}{2(n-1)}-\dfrac{n(n-2)\abs{H}\sqrt{n^2\abs{H}^2+4c(n-1)}}{2(n-1)}.
\end{align*}
Consequently, if
\begin{align*}
\abs{B}^2\leq nc+\dfrac{n^3\abs{H}^2}{2(n-1)}-\dfrac{n(n-2)\abs{H}\sqrt{n^2\abs{H}^2+4c(n-1)}}{2(n-1)},
\end{align*}
then
\begin{align*}
b_p\coloneqq\dim_\RH^p(M,{\field{R}})\leq\binom{n}{p},\quad p\in\set{0,1,\dotsc, n}.
\end{align*}
Moreover, if the inequality holds strictly at some point, then
\begin{align*}
b_p\coloneqq\dim_\RH^p(M,{\field{R}})=0,\quad p\in\set{1,\dotsc, n-1}.
\end{align*}

Finally, since $n$ is even and $\chi(M)\neq 2$, there must be some $p\in\set{1,\dotsc, n-1}$ such that the betti number $b_p>0$. We finished the proof.
\end{proof}

\begin{proof}[Proof of Theorem \ref{rationalsphere1}]Since $\abs{B}^2<\alpha(c,1,n,H)$, applying the estimate of the lower bound of the first $p$-eigenvalue, we know that the $p$-th betti number is zero for $0<p<n$, i.e., $M$ is a homology sphere.  By using rational Hurewicz theorem, we know that $\pi_i(M,\Q)=0$ for all $1\leq i\leq n-1$.
\end{proof}

Besides the corollaries and theorems mentioned in the introduction, we have two more applications of Theomre \ref{eigenest}.
\begin{theorem}If the assumptions of Theorem \ref{rigidthm} hold, and moreover if $\abs{\mathring{B}}^2\leq4cp(n-p)/n$ and the strictly inequality holds at some point, then $b_p=0$ for $p=1,\dotsc, n-1$. Therefore, if $\abs{\mathring{B}}^2<4c(n-1)/n$ and $M$ is simply connected, then $M$ is a rational homotopy sphere.
\end{theorem}
\begin{proof}
A direct computation gives
\begin{align*}
\min_{H}\set{\alpha(c,p,n,H)-n\abs{H}^2}=\dfrac{4cp(n-p)}{n},
\end{align*}
and the equality holds if and only if
\begin{align*}
n^2\abs{H}^2+4cp(n-p)=n^2c.
\end{align*}
Therefore, if $\abs{\mathring{B}}^2\leq\dfrac{4cp(n-p)}{n}$ and strictly inequality holds at some point, then $b_p=0$.
\end{proof}
Similarly,
\begin{theorem}If the assumptions of Theorem \ref{rigidthm} hold, and moreover if
$\abs{B}^2\leq 2c\sqrt{p(n-p)}$ and the strictly inequality holds at some point, then $b_p=0$ for $p=1,\dotsc, n-1$. Therefore, if $\abs{B}^2<2c\sqrt{n-1}$ and $M$ is simply connected, then $M$ is a rational homotopy sphere.
\end{theorem}
\begin{proof}
Since
\begin{align*}
\min_{H}\alpha(c,p,n,H)=2c\sqrt{p(n-p)},
\end{align*}
we obtain the theorem.
\end{proof}

\section{Ejiri's type Theorem}
In this section, we will prove Theorem \ref{thm:eriji}.

First, for readers' convenience,  we first provide here a complete proof of Theorem \ref{gclaim}.
\begin{proof}[Proof of Theorem \ref{gclaim}]
We need to verify Lawson-Simons condition \eqref{eq:L-S}.

We first study the case of $p=1$.
\begin{align*}
\sum_{i=1}^p\sum_{j=p+1}^n\sum_{\alpha=1}^m\left(2\left(h_{ij}^{\alpha}\right)^2-h_{ii}^{\alpha}h_{jj}^{\alpha}\right)=&\sum_{j=2}^n\sum_{\alpha=1}^m\left(h^{\alpha}_{1j}\right)^2-Ric_{11}+(n-1)c\\
\leq&\dfrac12\left(\abs{B}^2-n\abs{H}^2\right)-Ric_{11}+(n-1)c\\
=&\dfrac12\left(n(n-1)\left(c+\abs{H}^2\right)-\sum_{i=1}^nR_{ii}\right)-Ric_{11}+(n-1)c\\
=&\dfrac12n(n-1)\left(c+\abs{H}^2\right)-\dfrac12\sum_{i=1}^nR_{ii}-Ric_{11}+(n-1)c.
\end{align*}
Hence, if
\begin{align*}
Ric>\dfrac{n(n-1)}{n+2}\left(c+\abs{H}^2\right),
\end{align*}
then
\begin{align*}
\sum_{j=2}^n\sum_{\alpha=1}^m\left(2\left(h_{1j}^{\alpha}\right)^2-h_{11}^{\alpha}h_{jj}^{\alpha}\right)<(n-1)c,
\end{align*}
which means that there is no stable integral $1$-currents.

Now we consider the case of $2\leq p\leq n/2$.
\begin{align*}
&\sum_{i=1}^p\sum_{j=p+1}^n\sum_{\alpha=1}^m\left(2\left(h_{ij}^{\alpha}\right)^2-h_{ii}^{\alpha}h_{jj}^{\alpha}\right)\\
=&\sum_{i=1}^p\sum_{j=p+1}^n\sum_{\alpha=1}^m2\left(\mathring{h}_{ij}^{\alpha}\right)^2+\sum_{\alpha=1}^m\left(\left(\sum_{i=1}^p\mathring{h}^{\alpha}_{ii}\right)^2-(n-2p)H^{\alpha}\sum_{i=1}^p\mathring{h}^{\alpha}_{ii}\right)-p(n-p)\abs{H}^2\\
=&\sum_{i=1}^p\sum_{j=p+1}^n\sum_{\alpha=1}^m2\left(\mathring{h}_{ij}^{\alpha}\right)^2+\sum_{\alpha=1}^m\left(\left(\sum_{i=1}^p\mathring{h}^{\alpha}_{ii}-\dfrac{n-2}{2}pH^{\alpha}\right)^2+(p-1)nH^{\alpha}\sum_{i=1}^p\mathring{h}^{\alpha}_{ii}\right)\\
&-\dfrac{(n-2)^2}{4}p^2\abs{H}^2-p(n-p)\abs{H}^2.
\end{align*}
Similarly,
\begin{align*}
&\sum_{i=1}^p\sum_{j=p+1}^n\sum_{\alpha=1}^m\left(2\left(h_{ij}^{\alpha}\right)^2-h_{ii}^{\alpha}h_{jj}^{\alpha}\right)\\
=&\sum_{i=1}^p\sum_{j=p+1}^n\sum_{\alpha=1}^m2\left(\mathring{h}_{ij}^{\alpha}\right)^2+\sum_{\alpha=1}^m\left(\left(\sum_{j=p+1}^n\mathring{h}^{\alpha}_{jj}-\dfrac{n-2}{2}(n-p)H^{\alpha}\right)^2+(n-p-1)nH^{\alpha}\sum_{j=p+1}^n\mathring{h}^{\alpha}_{jj}\right)\\
&-\dfrac{(n-2)^2}{4}(n-p)^2\abs{H}^2-p(n-p)\abs{H}^2.
\end{align*}
Therefore,
\begin{align*}
&\sum_{i=1}^p\sum_{j=p+1}^n\sum_{\alpha=1}^m\left(2\left(h_{ij}^{\alpha}\right)^2-h_{ii}^{\alpha}h_{jj}^{\alpha}\right)\\
=&\dfrac{n-p-1}{n-2}\left(\sum_{i=1}^p\sum_{j=p+1}^n\sum_{\alpha=1}^m2\left(\mathring{h}_{ij}^{\alpha}\right)^2+\sum_{\alpha=1}^m\left(\sum_{i=1}^p\mathring{h}^{\alpha}_{ii}-\dfrac{n-2}{2}pH^{\alpha}\right)^2-\dfrac{(n-2)^2}{4}p^2\abs{H}^2\right)\\
&+\dfrac{p-1}{n-2}\left(\sum_{i=1}^p\sum_{j=p+1}^n\sum_{\alpha=1}^m2\left(\mathring{h}_{ij}^{\alpha}\right)^2+\sum_{\alpha=1}^m\left(\sum_{j=p+1}^n\mathring{h}^{\alpha}_{jj}-\dfrac{n-2}{2}(n-p)H^{\alpha}\right)^2-\dfrac{(n-2)^2}{4}(n-p)^2\abs{H}^2\right)\\
&-p(n-p)\abs{H}^2.
\end{align*}
 Thus, for $2\leq p\leq n/2$,
\begin{align*}
&\sum_{i=1}^p\sum_{j=p+1}^n\sum_{\alpha=1}^m\left(2\left(h_{ij}^{\alpha}\right)^2-h_{ii}^{\alpha}h_{jj}^{\alpha}\right)\\
\leq&\dfrac{n-p-1}{n-2}\left(\sum_{i=1}^p\sum_{j\neq i}\sum_{\alpha=1}^mp\left(\mathring{h}_{ij}^{\alpha}\right)^2+\sum_{\alpha=1}^mp\sum_{i=1}^p\left(\mathring{h}^{\alpha}_{ii}-\dfrac{n-2}{2}H^{\alpha}\right)^2-\dfrac{(n-2)^2}{4}p^2\abs{H}^2\right)\\
&+\dfrac{p-1}{n-2}\left(\sum_{j=p+1}^n\sum_{i\neq j}\sum_{\alpha=1}^m(n-p)\left(\mathring{h}_{ij}^{\alpha}\right)^2+\sum_{\alpha=1}^m(n-p)\sum_{j=p+1}^n\left(\mathring{h}^{\alpha}_{jj}-\dfrac{n-2}{2}H^{\alpha}\right)^2-\dfrac{(n-2)^2}{4}(n-p)^2\abs{H}^2\right)\\
&-p(n-p)\abs{H}^2\\
=&\dfrac{(n-p-1)p}{n-2}\sum_{i=1}^p\sum_{\alpha=1}^m\left(\sum_{j\neq i}\left(\mathring{h}_{ij}^{\alpha}\right)^2+\left(\mathring{h}^{\alpha}_{ii}-\dfrac{n-2}{2}H^{\alpha}\right)^2-\dfrac{n^2}{4}\abs{H^{\alpha}}^2+(n-1)\abs{H^{\alpha}}^2\right)\\
&+\dfrac{(p-1)(n-p)}{n-2}\sum_{j=p+1}^n\sum_{\alpha=1}^m\left(\sum_{i\neq j}\left(\mathring{h}_{ij}^{\alpha}\right)^2+\left(\mathring{h}^{\alpha}_{jj}-\dfrac{n-2}{2}H^{\alpha}\right)^2-\dfrac{n^2}{4}\abs{H^{\alpha}}^2+(n-1)\abs{H^{\alpha}}^2\right)\\
&-p(n-p)\abs{H}^2\\
\leq&\dfrac{(n-p-1)p}{n-2}(-K_p+(n-1)p\abs{H}^2)+\dfrac{(p-1)(n-p)}{n-2}\dfrac{n-p}{p}(-K_p+(n-1)p\abs{H}^2)-p(n-p)\abs{H}^2\\
=&-\dfrac{(n+2)p(n-p)-n^2}{(n-2)p}(K_p-(n-1)p\abs{H}^2)-p(n-p)\abs{H}^2,
\end{align*}
where
\begin{align*}
K_p=\min_{\set{e_i}}\set{\sum_{i=1}^pRic_{ii}-(n-1)pc}\eqqcolon Ric_{(p)}-(n-1)pc.
\end{align*}
Hence,
\begin{align*}
\sum_{i=1}^p\sum_{j=p+1}^n\sum_{\alpha=1}^m\left(2\left(h_{ij}^{\alpha}\right)^2-h_{ii}^{\alpha}h_{jj}^{\alpha}\right)\leq\dfrac{(n+2)p(n-p)-n^2}{(n-2)}\left((n-1)(c+\abs{H}^2)-\dfrac{Ric_{(p)}}{p}\right)-p(n-p)\abs{H}^2.
\end{align*}
Consequently, if
\begin{align*}
\dfrac{(n+2)p(n-p)-n^2}{(n-2)}\left((n-1)(c+\abs{H}^2)-\dfrac{Ric_{(p)}}{p}\right)-p(n-p)\abs{H}^2<p(n-p)c,
\end{align*}
or equivalently,
\begin{align*}
\dfrac{Ric_{(p)}}{p}>\left(n-1-\dfrac{(n-2)p(n-p)}{(n+2)p(n-p)-n^2}\right)\left(c+\abs{H}^2\right),
\end{align*}
we have
\begin{align*}
\sum_{i=1}^p\sum_{j=p+1}^n\sum_{\alpha=1}^m\left(2\left(h_{ij}^{\alpha}\right)^2-h_{ii}^{\alpha}h_{jj}^{\alpha}\right)<p(n-p)c.
\end{align*}

Finally, since $n\geq4$ and $c+\abs{H}^2\geq0$, we have
\begin{align*}
\dfrac{Ric_{(p)}}{p}\geq Ric_{\min},\\
(n-2)\left(c+\abs{H}^2\right)\geq\max_{1<p<n-1}\left(n-1-\dfrac{(n-2)p(n-p)}{(n+2)p(n-p)-n^2}\right)\left(c+\abs{H}^2\right),\\
(n-2)\left(c+\abs{H}^2\right)\geq\dfrac{n(n-1)}{n+2}\left(c+\abs{H}^2\right).
\end{align*}
Thus, if $Ric>(n-2)\left(c+\abs{H}^2\right)$ and $n\geq4$, then there is no stable integral $p$-currents for $0<p<n$.
\end{proof}

Before proving Theorem \ref{thm:eriji}, we need an algebraic lemma.

Given a matrix $A\in M_{n\times n}({\field{R}})$, we extend $A$ linearly into an operator $A:\Lambda^*{\field{R}}^n{\longrightarrow} \Lambda^*{\field{R}}^n$ satisfying
\begin{align*}
A(\omega\wedge\eta)=A(\omega)\wedge\eta+\omega\wedge A(\eta),\quad\forall\omega, \eta\in \Lambda^*{\field{R}}^n.
\end{align*}
For the matrix $A$, as an operator of ${\field{R}}^n$ to itself, we denote $\abs{A}_2$ by its operator norm, i.e.,
\begin{align*}
\abs{A}_2\coloneqq\max_{0\neq x\in{\field{R}}^n}\dfrac{\abs{Ax}}{\abs{x}}.
\end{align*}
It is obvious that $\abs{A}_2^2$ is the largest eigenvalue of $A^*A$.

Now we state the following algebraic Lemma.
\begin{lem}\label{lem:algebraic-eriji}For every symmetric matrices $A^{\alpha}\in M_{n\times n}({\field{R}}), 1\leq\alpha\leq m$, we have
\begin{align*}
\sum_{\alpha=1}^m\abs{A^{\alpha}\omega}^2\leq p^2\abs{\sum_{\alpha=1}^m\left(A_{\alpha}\right)^2}_{2}\abs{\omega}^2,\quad\forall \omega\in\Lambda^p{\field{R}}^n, \quad 0\leq p\leq n.
\end{align*}
\end{lem}
\begin{proof} If $p=1$, a direct verification claims that this conjecture is true, i.e., for every real numbers $x_1,\dotsc, x_n$, we have
\begin{align*}
\sum_{\alpha=1}^m\sum_{i=1}^n\abs{\sum_{j=1}^nA^{\alpha}_{ij}x_j}^2\leq\abs{\sum_{\alpha=1}^m(A^{\alpha})^2}_2\sum_{j=1}^n\abs{x_j}^2.
\end{align*}

Set
\begin{align*}
\omega=\dfrac{1}{p!}\omega_{i_1\dotsc i_p}e_{i_1}\wedge\dotsm\wedge e_{i_p}=\sum_{1\leq i_1<\dotsm<i_p\leq n}\omega_{i_1\dotsc i_p}e_{i_1}\wedge\dotsm\wedge e_{i_p},
\end{align*}
then
\begin{align*}
A^{\alpha}\omega=&\dfrac{1}{p!}\omega_{i_1\dotsc i_p}A^{\alpha}(e_{i_1})\wedge\dotsm\wedge e_{i_p}+\dotsm+\dfrac{1}{p!}\omega_{i_1\dotsc i_p}e_{i_1}\wedge\dotsm\wedge A^{\alpha}(e_{i_p})\\
=&\dfrac{1}{(p-1)!}\omega_{i_1i_2\dotsc i_p}A^{\alpha}_{i_1k}e_k\wedge e_{i_2}\dotsm\wedge e_{i_p}\\
=&p\sum_{1\leq k<i_2<\dotsm<i_p\leq n}\tilde\omega_{k i_2\dotsc i_p}e_k\wedge e_{i_2}\dotsm\wedge e_{i_p},
\end{align*}
where $\tilde\omega_{k i_2\dotsc i_p}$ is the antisymmetrizer of $\sum_{i_1=1}^n\omega_{i_1i_2\dotsc i_p}A^{\alpha}_{i_1k}$, i.e.,
\begin{align*}
\tilde\omega_{k i_2\dotsc i_p}=\dfrac1p\left(\sum_{i_1=1}^n\omega_{i_1i_2\dotsc i_p}A^{\alpha}_{i_1k}-\sum_{i_1=1}^n\omega_{i_1i_2\dotsc i_{p-1}k}A^{\alpha}_{i_1i_p}-\dotsm-\sum_{i_1=1}^n\omega_{i_1ki_3\dotsc i_p}A^{\alpha}_{i_1i_2}\right).
\end{align*}
Therefore, we obtain
\begin{align*}
\sum_{\alpha=1}^m\abs{A^{\alpha}\omega}^2\leq&\dfrac{ p^2}{p!}\sum_{\alpha=1}^m\sum_{i_2,\dotsc,i_p,k}\abs{\tilde\omega_{ki_2\dotsc i_p}}^2\\
\leq&\dfrac{ p}{p!}\sum_{\alpha=1}^m\sum_{\#\set{i_2,\dotsc,i_p,k}=p}\left(\abs{\sum_{i_1}\omega_{i_1i_2\dotsc i_p}A^{\alpha}_{i_1k}}^2+\abs{\sum_{i_1}\omega_{i_1i_2\dotsc i_{p-1}k}A^{\alpha}_{i_1i_p}}^2+\dotsm+\abs{\sum_{i_1}\omega_{i_1ki_3\dotsc i_p}A^{\alpha}_{i_1i_2}}^2\right)\\
=&\dfrac{ p^2}{p!}\sum_{i_2,\dotsc,i_p}\sum_{\alpha=1}^m\sum_{k\notin\set{i_2,\dotsc,i_p}}\abs{\sum_{i_1}\omega_{i_1i_2\dotsc i_p}A^{\alpha}_{i_1k}}^2\\
\leq&\dfrac{ p^2}{p!}\sum_{i_2,\dotsc,i_p}\abs{\sum_{\alpha=1}^m(A^{\alpha})^2}_2\sum_{i_1}\omega_{i_1i_2\dotsc i_p}^2\\
=&p^2\abs{\sum_{\alpha=1}^m(A^{\alpha})^2}_2\abs{\omega}^2.
\end{align*}
\end{proof}

\begin{proof}[Proof of weak Eriji's type \autoref{thm:eriji}]
Let $\abs{\omega}\in\Lambda^pT^*M$ with $\abs{\omega}=1$.
First,
\begin{align*}
&\abs{\mathring{S}(\omega)-\dfrac{n-2p}{2}H\omega}^2-\dfrac{n^2}{4}\abs{H}^2\\
=&\abs{\mathring{S}(\omega)}^2-(n-2p)\hin{\mathring{S}(\omega)}{H\omega}-p(n-p)\abs{H}^2\\
=&\abs{\mathring{S}(\omega)-\dfrac{n-2}{2}pH\omega}^2+(p-1)n\hin{\mathring{S}(\omega)}{H\omega}-\dfrac{(n-2)^2}{4}p^2\abs{H}^2-p(n-p)\abs{H}^2.
\end{align*}

Moreover, a direct calculation yields
\begin{align*}
&\abs{\mathring{S}(\omega)-\dfrac{n-2p}{2}H\omega}^2-\dfrac{n^2}{4}\abs{H}^2\\
=&\abs{\mathring{S}(*\omega)-\dfrac{n-2(n-p)}{2}H*\omega}^2-\dfrac{n^2}{4}\abs{H}^2\\
=&\abs{\mathring{S}(*\omega)-\dfrac{n-2}{2}(n-p)H*\omega}^2+(n-p-1)n\hin{\mathring{S}(*\omega)}{H*\omega}\\
&-\dfrac{(n-2)^2}{4}(n-p)^2\abs{H}^2-p(n-p)\abs{H}^2\\
=&\abs{\mathring{S}(\omega)+\dfrac{n-2}{2}(n-p)H\omega}^2-(n-p-1)n\hin{\mathring{S}(\omega)}{H\omega}\\
&-\dfrac{(n-2)^2}{4}(n-p)^2\abs{H}^2-p(n-p)\abs{H}^2.
\end{align*}
Therefore,
\begin{align*}
&\abs{\mathring{S}(\omega)-\dfrac{n-2p}{2}H\omega}^2-\dfrac{n^2}{4}\abs{H}^2\\\\
=&\dfrac{n-p-1}{n-2}\left(\abs{\mathring{S}(\omega)-\dfrac{n-2}{2}pH\omega}^2-\dfrac{(n-2)^2}{4}p^2\abs{H}^2\right)\\
&+\dfrac{p-1}{n-2}\left(\abs{\mathring{S}(\omega)+\dfrac{n-2}{2}(n-p)H\omega}^2-\dfrac{(n-2)^2}{4}(n-p)^2\abs{H}^2\right)\\
&-p(n-p)\abs{H}^2.
\end{align*}

Notice that acting on $p$-forms, $\mathring{S}-\dfrac{n-2}{2}pH$ can be viewed as a linearly extension operator of $\left(\mathring{A}^{\alpha}-\dfrac{n-2}{2}H^{\alpha}\operatorname{Id}\right)\otimes\nu_{\alpha}$. In particular, applying \autoref{lem:algebraic-eriji}, we obtain
\begin{align*}
\abs{\mathring{S}(\omega)-\dfrac{n-2}{2}pH\omega}^2\leq p^2\abs{\sum_{\alpha=1}^m\left(\mathring{A}^{\alpha}-\dfrac{n-2}{2}H^{\alpha}\operatorname{Id}\right)^2}_2.
\end{align*}
Similarly,
\begin{align*}
\abs{\mathring{S}(\omega)+\dfrac{n-2}{2}(n-p)H\omega}^2=\abs{\mathring{S}(*\omega)-\dfrac{n-2}{2}(n-p)H*\omega}^2\leq (n-p)^2\abs{\sum_{\alpha=1}^m\left(\mathring{A}^{\alpha}-\dfrac{n-2}{2}H^{\alpha}\operatorname{Id}\right)^2}_2.
\end{align*}
Consequently,
\begin{align*}
&\abs{\mathring{S}\vert_{\Lambda^p}-\dfrac{n-2p}{2}H}_{op}^2-\dfrac{n^2}{4}\abs{H}^2\\
\leq&\dfrac{(n-p-1)p^2+(p-1)(n-p)^2}{n-2}\left(\abs{\sum_{\alpha=1}^m\left(\mathring{A}^{\alpha}-\dfrac{n-2}{2}H^{\alpha}\operatorname{Id}\right)^2}_2-\dfrac{(n-2)^2}{4}\abs{H}^2\right)\\
=&\dfrac{(n+2)p(n-p)-n^2}{n-2}\left(\abs{\sum_{\alpha=1}^m\left(\mathring{A}^{\alpha}-\dfrac{n-2}{2}H^{\alpha}\operatorname{Id}\right)^2}_2-\dfrac{(n-2)^2}{4}\abs{H}^2\right).
\end{align*}

By Gauss equation, we know that
\begin{align*}
Ric_{ii}=&\sum_{j=1}^n\bar R_{ijij}-\abs{\mathring{B}_{ii}}^2+(n-2)\hin{\mathring{B}_{ii}}{H}-\sum_{j\neq i}\abs{\mathring{B}_{ij}}^2+(n-1)\abs{H}^2\\
=&\sum_{j=1}^n\bar R_{ijij}-\abs{\mathring{B}_{ii}-\dfrac{n-2}{2}H}^2-\sum_{j\neq i}\abs{\mathring{B}_{ij}}^2+\dfrac{n^2}{4}\abs{H}^2.
\end{align*}
By choosing $\set{e_i}$ so that
\begin{align*}
\sum_{\alpha=1}^m\left(\mathring{A}^{\alpha}-\dfrac{n-2}{2}H^{\alpha}\operatorname{Id}\right)^2
\end{align*}
is a diagonalizing matrix, without loss of generality, assume the largest eigenvalue is $\abs{\mathring{B}_{11}-\dfrac{n-2}{2}H}^2$ we obtain
\begin{align*}
Ric_{\min}\leq Ric_{11}=&\sum_{j=1}^n\bar R_{1j1j}-\abs{\mathring{B}_{11}-\dfrac{n-2}{2}H}^2+\dfrac{n^2}{4}\abs{H}^2\\
=&\sum_{j=1}^n\bar R_{1j1j}-\abs{\sum_{\alpha=1}^m\left(\mathring{A}^{\alpha}-\dfrac{n-2}{2}H^{\alpha}\operatorname{Id}\right)^2}_2+\dfrac{n^2}{4}\abs{H}^2\\
\leq&(n-1)c^*-\abs{\sum_{\alpha=1}^m\left(\mathring{A}^{\alpha}-\dfrac{n-2}{2}H^{\alpha}\operatorname{Id}\right)^2}_2+\dfrac{n^2}{4}\abs{H}^2.
\end{align*}

As a consequence, we obtain
\begin{align*}
\abs{\mathring{S}\vert_{\Lambda^p}-\dfrac{n-2p}{2}H}_{op}^2-\dfrac{n^2}{4}\abs{H}^2\leq&\dfrac{(n+2)p(n-p)-n^2}{n-2}\left((n-1)(c^*+\abs{H}^2)-Ric_{\min}\right)-p(n-p)\abs{H}^2.
\end{align*}
Hence, by assumption
\begin{align*}
W^{[p]}\geq& p(n-p)c_*-\dfrac{(n+2)p(n-p)-n^2}{n-2}\left((n-1)(c^*+\abs{H}^2)-Ric_{\min}\right)+p(n-p)\abs{H}^2\\
=&\dfrac{(n+2)p(n-p)-n^2}{n-2}\left(Ric_{\min}-(n-1)\left(c^*+\abs{H}^2\right)+\dfrac{(n-2)p(n-p)}{(n+2)p(n-p)-n^2}\left(c_*+\abs{H}^2\right)\right)\\
>&0.
\end{align*}
The rest of the proof is obvious.
\end{proof}
\begin{rem}
\begin{enumerate}
\item Suppose the sectional curvature of $\bar M$ is bounded below by $\bar K_{\min}$ and above by $\bar K_{\max}$, then we can take
\begin{align*}
c^*=&(n-1)\bar K_{\max},\\
c_*=&\dfrac{2[n/2]+1}{3}\left(\bar K_{\min}-\dfrac{2[n/2]-2}{2[n/2]+1}\bar K_{\max}\right).
\end{align*}
\item The condition is optimal when $\bar M=F^{n+m}(c)$ and $n=2p$.

\end{enumerate}

\end{rem}

\appendix
\section{An Example}
The following example shows that the conditions mentioned in this paper are sharp.
\begin{eg}Consider the following Clifford torus
\begin{align*}\Lg{S}^p\left(\dfrac{\mu}{\sqrt{1+\mu^2}}\right)\times\Lg{S}^{n-p}\left(\dfrac{1}{\sqrt{1+\mu^2}}\right)\subset\Lg{S}^{n+1}(1)\subset{\field{R}}^{n+2},\quad p=1,2,\dotsc, n-1,\quad \mu>0.
\end{align*}
Let $\phi=(x,y)$ be the position vector, then the first fundamental form is given by
\begin{align*}
{\mathrm d} s^2={\mathrm d} x{\mathrm d} x+{\mathrm d} y{\mathrm d} y.
\end{align*}
A unit norm vector field is $\nu=(-\mu^{-1}x,\mu y)$. Hence, the second fundamental form $B$ is
\begin{align*}
B=&-\hin{{\mathrm d}(x,y)}{{\mathrm d}(-\mu^{-1}x,\mu y)}\\
=&\mu^{-1}{\mathrm d} x{\mathrm d} x-\mu{\mathrm d} y{\mathrm d} y.
\end{align*}
Hence the principal curvatures are $\mu^{-1}$ and $-\mu$ with multiplicity $p$ and $n-p$ respectively. In particular,
\begin{align*}
H=\dfrac{1}{n}\left(p\mu^{-1}-(n-p)\mu\right),\\
\abs{B}^2=p\mu^{-2}+(n-p)\mu^2,\\
\abs{\mathring{B}}^2=\dfrac{p(n-p)}{n}\left(\mu^{-1}+\mu\right)^2.
\end{align*}
Moreover, the sectional curvature satisfies
\begin{equation*}
K_{ij}=
\begin{cases}
1+\mu^{-2},& 1\leq i< j\leq p;\\
1+\mu^{2},&p+1\leq i\neq j\leq n;\\
0,&1\leq i\leq p, p+1\leq j\leq n.
\end{cases}
\end{equation*}
The Ricci curvature satisfies
\begin{equation*}
Ric_{ii}=
\begin{cases}
n-1+(p-1)\mu^{-2}+(n-p)\mu^2,&1\leq i\leq p;\\
n-1+p\mu^{-2}+(n-p-1)\mu^2,&p+1\leq i\leq n.
\end{cases}
\end{equation*}

 A direct computation gives the following: if $(n-2p)(p\mu^{-1}-(n-p)\mu)\leq0$, then
 \begin{align*}
&\dfrac{\abs{\mathring{B}}}{\sqrt{n}}+\dfrac{\abs{n-2p}\abs{H}}{2\sqrt{p(n-p)}}-\sqrt{1+\dfrac{n^2\abs{H}^2}{4p(n-p)}}\\
=&\dfrac{\sqrt{p(n-p)}(\mu^{-1}+\mu)}{n}+\dfrac{\abs{n-2p}\abs{p\mu^{-1}-(n-p)\mu}}{2n\sqrt{p(n-p)}}-\dfrac{(p\mu^{-1}+(n-p)\mu)}{2\sqrt{p(n-p)}}\\
=&\dfrac{2p(n-p)(\mu^{-1}+\mu)+\abs{n-2p}\abs{p\mu^{-1}-(n-p)\mu}-(np\mu^{-1}+n(n-p)\mu)}{2n\sqrt{p(n-p)}}\\
=&\dfrac{2p(n-p)(\mu^{-1}+\mu)-(n-2p)(p\mu^{-1}-(n-p)\mu)-(np\mu^{-1}+n(n-p)\mu)}{2n\sqrt{p(n-p)}}\\
=&0.
\end{align*}
Hence, if $(n-2p)(p\mu^{-1}-(n-p)\mu)\leq0$, we obtain that $\abs{B}^2=\alpha(1,p,n,H)$.

Finally, it is obvious that $b_p\geq\max\set{p,n-p}$.
\end{eg}

 

\begin{thebibliography}{10}

\bibitem{Cha84}
I.~Chavel, \emph{Eigenvalues in {R}iemannian geometry}, Pure and Applied
  Mathematics, vol. 115, Academic Press, Inc., Orlando, FL, 1984, Including a
  chapter by Burton Randol, With an appendix by Jozef Dodziuk. \MR{768584}

\bibitem{Eji79}
N.~Ejiri, \emph{Compact minimal submanifolds of a sphere with positive {R}icci
  curvature}, J. Math. Soc. Japan \textbf{31} (1979), no.~2, 251--256.
  \MR{527542}

\bibitem{GuXu12}
J.~R. Gu and H.~W. Xu, \emph{The sphere theorems for manifolds with positive
  scalar curvature}, J. Differential Geom. \textbf{92} (2012), no.~3, 507--545.
  \MR{3005061}

\bibitem{Kwo16}
K.~K. Kwong, \emph{Some sharp {H}odge {L}aplacian and {S}teklov eigenvalue
  estimates for differential forms}, Calc. Var. Partial Differential Equations
  \textbf{55} (2016), no.~2, Art. 38, 14. \MR{3478292}

\bibitem{LawMic89}
H.~B. Lawson, Jr. and M.~L. Michelsohn, \emph{Spin geometry}, Princeton
  Mathematical Series, vol.~38, Princeton University Press, Princeton, NJ,
  1989. \MR{1031992}

\bibitem{LawSim73}
H.~Blaine Lawson, Jr. and James Simons, \emph{On stable currents and their
  application to global problems in real and complex geometry}, Ann. of Math.
  (2) \textbf{98} (1973), 427--450. \MR{0324529}

\bibitem{RauSav11}
S.~Raulot and A.~Savo, \emph{A {R}eilly formula and eigenvalue estimates for
  differential forms}, J. Geom. Anal. \textbf{21} (2011), no.~3, 620--640.
  \MR{2810846}

\bibitem{RauSav12}
\bysame, \emph{On the first eigenvalue of the {D}irichlet-to-{N}eumann operator
  on forms}, J. Funct. Anal. \textbf{262} (2012), no.~3, 889--914. \MR{2863852}

\bibitem{Sav05}
A.~Savo, \emph{On the first {H}odge eigenvalue of isometric immersions}, Proc.
  Amer. Math. Soc. \textbf{133} (2005), no.~2, 587--594. \MR{2093083}

\bibitem{Sav14}
\bysame, \emph{The {B}ochner formula for isometric immersions}, Pacific J.
  Math. \textbf{272} (2014), no.~2, 395--422. \MR{3284892}

\bibitem{Xin84}
Y.~L. Xin, \emph{An application of integral currents to the vanishing
  theorems}, Sci. Sinica Ser. A \textbf{27} (1984), no.~3, 233--241.
  \MR{763966}

\bibitem{XuGu13}
H.~W. Xu and J.~R. Gu, \emph{Geometric, topological and differentiable rigidity
  of submanifolds in space forms}, Geom. Funct. Anal. \textbf{23} (2013),
  no.~5, 1684--1703. \MR{3102915}

\bibitem{XuLenGu14}
H.~W. Xu, Y.~Leng, and J.~R. Gu, \emph{Geometric and topological rigidity for
  compact submanifolds of odd dimension}, Sci. China Math. \textbf{57} (2014),
  no.~7, 1525--1538. \MR{3213887}

\end{thebibliography}

\end{document}


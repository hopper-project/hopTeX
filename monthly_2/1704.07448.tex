\documentclass[11 pt]{amsart}  

\usepackage{graphicx}
\usepackage{amsmath,amsthm,amscd,amssymb}
\usepackage{amsfonts}
\usepackage{latexsym}
\usepackage{epsfig,epstopdf}
\usepackage{enumerate}

\textwidth 6.2 in
\textheight 8in
\parindent 7mm
\oddsidemargin 0.67cm
\evensidemargin 0.67cm

\setlength{\parskip}{10pt}

            
	

\newtheorem{definition}{\sc Definition}[section]
\newtheorem{teo}{\sc Theorem}[section]
\newtheorem{prop}{\sc Proposition}[section]
\newtheorem{coro}{\sc Corollary}[section]
\newtheorem{ejemplo}{\sc Example}[section]
\newtheorem{lemma}{\sc Lemma}[section]
\newtheorem{obs}{\sc Remark}[section]

\begin{document}

\title[Controllability of the Semilinear Beam Equation]
{Controllability of the Impulsive Semilinear Beam Equation with Memory and Delay}
\date{\today}
\author[A. CARRASCO, C. GUEVARA  AND H. LEIVA ]{A. Carrasco$^1$, C. Guevara$^2$ and H. Leiva$^3$  }
\address{$^{1}$ Universidad Centroccidental Lisandro Alvarado \\
          Decanato de Ciencias y Tecnologia, Departamento de Matem\'aticas \\
          Barquisimeto 3001-Venezuela}
          \email {acarrasco@ucla.edu.ve}
\address{$^{2}$ Louisiana State University \\
          College of Science, Department of Mathematics \\
          Baton Rouge, LA 70803-USA} \email{cguevara@lsu.edu, cristi.guevara@asu.edu}
\address{$^{3}$ School of Mathematical Sciences and Information Technology, \\
           Universidad Yachay Tech, \\
         San Miguel de Urcuqui, Ecuador}
          \email{hleiva@yachaytech.edu.ec, hleiva@ula.ve}

\thanks{$\dagger$ This work has been supported by YachayTech, UCLA, UCV and BCV}
\subjclass[2010]{primary: 93B05; secondary:  93C10.} \keywords{ approximate controllability, strongly continuous semigroup, impulsive semilinear beam equation with memory and delay.}

\begin{abstract}
The semilinear beam equation with impulses, memory and delay is considered. We obtain the approximate controllability. This is done by employing a technique that avoids fixed point theorems and pulling back the control solution to a fixed curve in a short time interval. Demonstrating, once again, that  the controllability of a system is robust under the influence of impulses and delays.

\end{abstract}

\maketitle 

\section{Introduction}

Beams have been used since ancient times in bridges, buildings, for example. Through the millennia, understanding the dynamics and controllability of beams, including bending and vibration, has been  of great importance. One of the pioneers  in studying them was Galileo presenting how they should be approached,  however, it was not until the late 17th century with the progress of elasticity that  Leonhard Euler and Daniel Bernoulli provided  a second-order spatial derivatives mathematical model that later, in 1921, Stephen Timoshenko improved by including a shear deformation and rotational inertia effects, obtaining  fourth order  mathematical model (see \cite{Timoshenko:1921aa,Timoshenko:1922aa, Timoshenko:1953aa} for  details).

In particular, the impulsive semilinear beam equations of the form \eqref{eq:beam} is of greatly interest  in mechanical engineer and nanotechnology design \cite{Wang-Tan:2006aa,Wang-Zhang:2006aa}, and the memory and delay are characterized by the viscoelasticity property  and response of the materials. In this paper, we are exploring the  approximate  controllability of
\begin{equation}\label{eq:beam}
 w_{tt}  -  2\beta\Delta w_t + \Delta^{2}w = u(t,x) + f(t,w(t-r),w_{t}(t-r),u)+{\displaystyle} \int_{0}^{t}M(t-s)g(w(s-r,x))ds,
\end{equation}
$t \neq t_k$, subjected to the initial-boundary conditions and impulses
\begin{equation}\label{eq:initial}
\left\{\begin{array}{lcl}
 w(t,x)=\Delta w (t,x)=0, &&\mbox{in}\; (0, \tau) \times \partial \Omega,\\
 \begin{split}
 &w(s,x)=\phi_1(s,x),\\
 &w_{t}(s,x)=\phi_2(s,x),
 \end{split} && \mbox{in} \; [-r,0] \times \Omega,\\
 w_{t}(t_{k}^{+},x) = w_{t}(t_{k}^{-},x)+I_{k}(t_k,w(t_{k},x),w_{t}(t_{k},x),u(t_{k},x)),& & k=1, \dots, p,
  \end{array}
 \right.
\end{equation}
where   $\Omega \subseteq {\mathbb{R}}^{N} \,(N\geq1)$ is a bounded  domain, the constant  $\beta > 1$ and the real-valued functions $w = w(t,x)$  in $(0, \tau] \times  \Omega$ denotes the deflection of a beam , $u$ in $(0, \tau] \times  \Omega$ represents the distributed control, $M$ acts as convolution kernel with respect to the time variable, $I_k$ is defined on $[0, \tau] \times {\mathbb{R}}^3$  and $g$ on ${\mathbb{R}}$,  $f$ on $[0, \tau] \times {\mathbb{R}}^3$ are the nonlinearities. Under the assumption
that $M\in L^{\infty}((0,\tau)\times \Omega)$ and  $g, f, I_{k}$  are smooth enough so that, for all $\phi, \psi\in {\mathcal{C}}([-r,0],L^{2}(\Omega))$ and $u\in L^{2}([0,\tau]; L^{2}(\Omega))$ the equation \eqref{eq:beam} admits only one mild solution on $[-r,\tau]$ and
\begin{equation}\label{eq:f1}
    \begin{array}{ll}
|f(t,y,v,u)|  & \leq a\sqrt{|y |^2+ |v |^2} +b,
\end{array}
\end{equation}
with $t\in [0,\tau]$, $y,v, u\in {\mathbb{R}}$  ,  $a_{0},b_{0}\geq 0$.

This article has been inspired by the series of papers from Carrasco, Leiva, Merentes and
Sanchez \cite{Carrasco-Leiva:2013aa,Carrasco-Leiva:2014aa,Carrasco-Leiva:2016aa} on the approximate controllability of semilinear beam and the works of Guevara and Leiva  \cite{Guevara-Leiva:2016aa, Guevara-Leiva:2017aa} on the approximate controllability for the semilinear heat and strongly damped wave equations with memory and delays.
Here we prove the  approximate controllability of the beam equation  \eqref{eq:beam} under the initial-boundary condition \eqref{eq:initial}  with memory, impulses and delay terms  by applying   A.E. Bashirov, N. Ghahramanlou, N. Mahmudov, N. Semi and H. Etikan technique \cite{Bashirov-Ghahramanlou:2014aa, Bashirov-Jneid:2013aa, Bashirov-Mahmudov:2007aa},  and avoiding the Rothe's fixed point theorem used in \cite{Carrasco-Leiva:2013aa,Carrasco-Leiva:2016aa} and the   Schauder fixed point theorem applied in \cite{Carrasco-Leiva:2014aa}.

The structure of this paper is as follow: In section \ref{sec:formulation}, we present the abstract formulation of the beam equation \eqref{eq:beam}. Section \ref{sec:lineal}, recalls the linear controllability characterization of the problem. In section \ref{sec:semilineal}, the approximated controllability of the beam equation with memory, delay and impulses is proved.

\section{Abstract Formulation of the Problem} \label{sec:formulation}

In this section we choose the appropiate Hilbert space where the Cauchy problem \eqref{eq:beam}-\eqref{eq:initial}  can be written as an abstract differential equation.

 First of all, notice that the term $-2\beta\Delta w_t$ in the equation \eqref{eq:beam} acts as a damping force, thus the energy space used to set up the wave equation is not suitable here. Even so,  in   \cite{Oliveira:1998aa}, Oliveira shows that the uncontrolled linear  equation can be transformed into a system of parabolic equations of the form $w_{t} = D \Delta w$,  obtaining that corresponding space for the abstract formulation of the problem is ${\mathcal{Z}}^{1}=\left[H^{2}(\Omega) \bigcap H^{1}_{0}(\Omega) \right] \times L^{2}(\Omega)$ and proving that the linear part of this system generates a strongly continuous analytic semigroup in this space.

Consider the Hilbert space ${\mathcal{X}} = L^{2}(\Omega)$,  and denote  ${\mathcal{A}}=-\Delta$ with the eigenvalues  $0 <
\lambda_{1}<\lambda_{2}<...<\lambda_{j}\to \infty,$ each one with multiplicity $\gamma_{j}<\infty$ equal to its corresponding eigenspace dimension. Recall, ${\mathcal{A}}$ satisfies the following properties:
\begin{enumerate}[(i)]
\item There exists a complete orthonormal set $\left\{
\phi_{j_k} \right\}$ of eigenvectors of ${\mathcal{A}}$.\\

\item For all $x \in D({\mathcal{A}})$,
\begin{equation*} \label{prop}
{\mathcal{A}} x = \sum_{j = 1}^{\infty} \lambda_{j} \sum_{k =
1}^{\gamma_j} {\langle {\xi, \phi_{j_k}}\rangle} \phi_{j_k} =\sum_{j = 1}^{\infty}
\lambda_{j}E_{j}x,
\end{equation*}
where ${\langle {\cdot, \cdot}\rangle}$ denotes the inner product in ${\mathcal{X}}$, ${\displaystyle} E_{n}x = \sum_{k = 1}^{\gamma_j} {\langle {z, \phi_{j_k}}\rangle} \phi_{j_k},$ and  $\{ E_j \}$ is a family of complete orthogonal projections in
${\mathcal{X}}$.
 \item $-{\mathcal{A}}$ generates an analytic semigroup $\{
S(t) \}_{t \geq 0}$ given by
$$
S(t)x =  \sum_{j = 1}^{\infty} e^{-\lambda_j t}E_{j}x \quad  \mbox{and} \quad  {\left\| {S(t)}\right\|} \leq e^{-\lambda_{1}t}.
$$

\item For $\alpha\geq0$ the fractional powered spaces ${\mathcal{X}}^{\alpha}$ are given by
\begin{equation*}
   {\mathcal{X}}^{\alpha} =D({\mathcal{A}}^{\alpha}) = \left\{x \in {\mathcal{X}} : \sum_{j = 1}^{\infty} \lambda_{j} ^{2
\alpha} {\left\| { E_{j}x}\right\|}^2 < \infty \right\}
\end{equation*}
equipped with the norm ${\displaystyle}
{\left\| {x}\right\|}_{\alpha}^2 = {\left\| {{\mathcal{A}}^{\alpha}x}\right\|}^2=  \sum_{j = 1}^{\infty}
\lambda_{j}^{2 \alpha} {\left\| { E_{j}x}\right\|}^2
$, where ${\displaystyle} {\mathcal{A}}^{\alpha}x = \sum_{j = 1}^{\infty} \lambda_{j}^{ \alpha}  E_{j}x$.

\end{enumerate}

In particular, taking $\alpha=1,$ the Hilbert space ${\mathcal{Z}}^{1}={\mathcal{X}}^{1}\times {\mathcal{X}}$ has the  norm
$$
{\displaystyle} {\left\| {\left(
                  \begin{array}{c}
                    w \\
                    v \\
                  \end{array}
                \right)}\right\|}^{2}_{{\mathcal{Z}}^{1}}=\|w\|^{2}_{1}+\|v\|^{2}.
$$

Using the above notation, we rewrite the system \eqref{eq:beam}-\eqref{eq:initial}  as the second-order ordinary differential equations in the Hilbert
space ${\mathcal{X}}$
\begin{equation} \label{eq:ODE}
\left\{\begin{array}{lll}
    \begin{split}
    w''(t) = &-{\mathcal{A}}^{2}w(t) - 2\beta {\mathcal{A}} w'(t) + u(t) + {\displaystyle} \int_{0}^{t}M(t,s)g^{e}(w(s-r))ds \\
 &\ \ +  f^{e}(t,w(t-r),w'(t-r),u(t)),
 \end{split}
 & & t>0, t \neq t_k,\\
     w(s)  =  \phi_1(s), \qquad  w'(s)=\phi_2(s),
& &s \in [-r,0],\\
 w' (t_{k}^{+}) = w' (t_{k}^{-})+I^{e}_{k}(t_k,w(t_{k}),w' (t_{k}),u(t_{k},)),& & k=1, \dots, p,
    \end{array}\right.
\end{equation}
where ${\mathcal{U}}={\mathcal{X}}=L^{2}(\Omega)$, and
\begin{eqnarray*}
I_{k}^{e}:&[0, \tau]\times {\mathcal{Z}}^{1} \times {\mathcal{U}} &\longrightarrow \qquad {\mathcal{X}} \\
&(t,w,v,u)(\cdot)&\longmapsto \quad I_{k}(t,w(\cdot),v(\cdot),u(\cdot)),
\end{eqnarray*}
\begin{eqnarray*}
f^{e}:&[0, \tau]\times {\mathcal{C}} (-r,0;  {\mathcal{Z}}^{1} ) \times {\mathcal{U}} & \longrightarrow \qquad {\mathcal{X}}\\
&(t,\Phi,u)(\cdot)&\longmapsto \quad f(t,\phi_1(-r, \cdot),\phi_2(-r, \cdot),u(\cdot)),
\end{eqnarray*}
and
\begin{eqnarray*}
g^{e}:&{\mathcal{C}}(-r,0;  {\mathcal{Z}}^{1} )  &\longrightarrow  {\mathcal{Z}}^{1} \\
&\Phi=\left(\begin{array}{c}
             \phi_1\\
             \phi_2
        \end{array}\right)&\longmapsto  g(\phi_1(\cdot-r)).
\end{eqnarray*}

\noindent Finally, by changing variables, $v=w',$  the systems \eqref{eq:ODE} can be written as an abstract first order
functional differential equations with memory and impulses in ${\mathcal{Z}}^{1}$
\begin{equation}\label{eq:abstract}
\left\{
\begin{array}{ll}
z' =  -{\mathbb{A}} z+ {\mathbb{B}} u + {\displaystyle} \int_{0}^{t}{\mathbb{M}_g}(t,s ,z_{s}(-r))ds + {\mathbb{F}}(t,z_{t}(-r),u(s)) ,& z\in Z^{1},\; t\geq 0, \\
z(s)  = \Phi(s), \ \ s \in [-r,0],\\
z(t_{k}^{+})  = z(t_{k}^{-})+{\mathbb{I}}_{k}(t_k, z(t_{k}),u(t_{k})), \quad  k=1,2,3, \dots, p,
\end{array}
\right.
\end{equation}
where ${\mathbb{A}} = \left(
   \begin{array}{rr}
     0 & I_{\mathcal{X}} \\ - {\mathcal{A}}^2 & -2\beta {\mathcal{A}}
   \end{array}\right)$ is a unbounded linear operator  with domain
$$
D({\mathbb{A}})=\{w\in H^{4}(\Omega):\:w=\Delta w=0\}\times D({\mathcal{A}}),
$$ with  $I_{\mathcal{X}}$ the identity in ${\mathcal{X}} $,
$ z =(w,v)^{T}$,
$u\in L^{2}(0,\tau;{\mathcal{U}})$,
   $\Phi=\left(\begin{array}{c}
             \phi_1\\
             \phi_2
        \end{array}\right) \in {\mathcal{C}}\left(-r,0;  {\mathcal{Z}}^{1} \right),$   ${\mathbb{B}}: {\mathcal{U}} \longrightarrow {\mathcal{Z}}^{1}$ is the bounded linear operator defined by
${\mathbb{B}} u=
\left(\begin{array}{c}
             0\\
             u
        \end{array}\right),$ and the functions
\begin{eqnarray*}
{\mathbb{I}}_{k}:&[0, \tau]\times {\mathcal{Z}}^{1} \times {\mathcal{U}}& \longrightarrow \qquad {\mathcal{Z}}^{1} \\
&(t, z,u)&\longmapsto \quad
\left(\begin{array}{c}
             0\\
             I_{k}^{e}(t,w,v,u)
        \end{array}\right)
        \end{eqnarray*}
\begin{eqnarray}
\label{functionF}
{\mathbb{F}}:& [0, \tau] \times {\mathcal{C}}(-r,0;  {\mathcal{Z}}^{1} ) \times {\mathcal{U}} & \longrightarrow \qquad {\mathcal{Z}}^{1}\\
&(t, \Phi,u)&\longmapsto \quad
\left(
  \begin{array}{c}
    0 \\f^{e}(t,\phi_1(-r),\phi_2(-r),u)
  \end{array}
  \right),\notag
  \end{eqnarray}
and
\begin{eqnarray*}
{\mathbb{M}_g}:&[0,\tau]\times [0,\tau]\times {\mathcal{C}}(-r,0;  {\mathcal{Z}}^{1} )  &\longrightarrow  {\mathcal{Z}}^{1} \\
&(t,s,\Phi)&\longmapsto \left(
  \begin{array}{c}
    0 \\M(t,s) g^{e}(\Phi)
  \end{array}
  \right).
\end{eqnarray*}
Moreover, this abstract formulation together with condition \eqref{eq:f1} and the continous imbeding ${\mathcal{X}}^1 \subset {\mathcal{X}}$ yields
\begin{prop}\label{prop:cotaF}
There exist constants  $\tilde{a},\tilde{b}>0$ such that, for all $(t, \Phi,u) \in [0, \tau] \times {\mathcal{C}}(-r,0;  {\mathcal{Z}}^{1} ) \times {\mathcal{U}}$ the following inequality holds
\begin{equation}\label{eq:bound}
{\left\| {{\mathbb{F}}(t,\Phi,u)}\right\|}_{Z^{1}}  \leq   \tilde{a}\| \Phi(-r) \|_{{\mathcal{Z}}^1}+\tilde{b}.
\end{equation}
\end{prop}
In \cite{Carrasco-Leiva:2013aa} is proved that the linear unbounded operator
${\mathbb{A}}$  generates a strongly continuous compact semigroup
 in the space ${\mathcal{Z}}^1$ which decays exponentially to zero, precisely:
\begin{prop}\label{semigroup}
The operator $\mathcal{A}$ is the infinitesimal generator
of a strongly continuous compact semigroup $\{T(t)\}_{t\geq0}$
represented by
\begin{equation}\label{repre}
T(t)z=\displaystyle\sum_{j=1}^{\infty}e^{\mathbb{A}_{j}t}P_{j}z,\;z\in
Z_{1},\;t\geq 0,
\end{equation}
where $\{P_{j}\}_{j\geq0}$ is a complete family of orthogonal
projections in the Hilbert space $Z_{1}$ given by
\begin{equation}\label{proyecciones}
    P_{j} = diag(E_{j},E_{j}),
\end{equation}
and
$$
\mathbb{A}_{j}=K_{j}P_{j},\;
K_{j}=\left(
                           \begin{array}{cc}
                             0 & 1 \\
                             -\lambda_{j}^{2}  & -2\beta\lambda_{j} \\
                           \end{array}
                         \right)\;\;j\geq 1,
$$
and there exists $M \geq 1$ and $\mu >0$ such that
$$
\parallel T(t)\parallel\leq Me^{-\mu t},\;t\geq0,
$$

\end{prop}

\section{Approximate Controllability of the Linear System }\label{sec:lineal}
 This section is devoted to characterize the approximate controllability of the linear system. Thus,  for all $z_{0}\in {\mathcal{Z}}^{1}$ and $u\in
L^{2}([0,\tau];{\mathcal{U}})$ consider the initial value problem
\begin{equation}\label{eq:linear}
\left\{\begin{array}{lll}
    z'(t) = \mathbb{A} z(t) + {\mathbb{B}} u(t),\\
    z(t_{0}) = z_{0},
    \end{array}\right.
\end{equation}
obtained from \eqref{eq:abstract}. It admits only one mild solution on $0\leq t_{0}\leq t\leq \tau$ given by
\begin{equation}\label{eq:mild}
    z(t)=T(t-t_{0})z_{0} + \displaystyle\int_{t_{0}}^{t}T(t-s){\mathbb{B}} u(s)ds;\;t\in[t_{0},\tau],\:0\leq t_{0}\leq \tau.
\end{equation}
\begin{definition} \label{def2}
({\bf Approximate Controllability of (\ref{eq:linear})}) The system (\ref{eq:linear}) is said
to be approximately controllable on $[t_{0},\tau]$ if for every $z_0$,
$z_1\in Z$, $\varepsilon>0$ there exists $u\in L_{2}(t_{0},\tau;U)$ such
that the solution $z(t)$ of (\ref{eq:mild}) corresponding to $u$
verifies: $$\|z(\tau)-z_1\|<\varepsilon.$$
\end{definition}

For the system \eqref{eq:linear} and $\tau>0$, we have the following notions:
\begin{enumerate}
\item $G_{\tau\delta}$ is the controllability  operator defined by
\begin{eqnarray*}
G_{\tau\delta}: L^2(\tau-\delta,\tau;{\mathcal{U}}) \longrightarrow& {\mathcal{Z}}^{1}\\
u\longmapsto&{\displaystyle} \int_{\tau-\delta}^{\tau}T(\tau-s){\mathbb{B}} u(s)ds,
\end{eqnarray*}
with corresponding adjoint $G^*_{\tau\delta}$ given by
\begin{eqnarray*}
G^*_{\tau\delta}:  {\mathcal{Z}}^{1} \longrightarrow& L^2(\tau-\delta,\tau;{\mathcal{U}})\\
z\longmapsto& {\mathbb{B}} ^{*}T^{*}(\tau-\cdot)z.
\end{eqnarray*}
\item The Gramian controllability operator is
\begin{equation*}
Q_{\tau \delta*} = G_{\tau\delta}G_{\tau\delta}^{*}= \int_{\tau-\delta}^{\tau}T(\tau-t){\mathbb{B}} {\mathbb{B}} ^{*}T^{*}(\tau-t)dt.
\end{equation*}
\end{enumerate}

In general, for linear bounded operator $G$ between Hilbert spaces $\mathcal{W}$ and $\mathcal{Z}$, the following lemma holds (see \cite{Bashirov-Kerimov:1997aa,Bashirov-Mahmudov:1999aa, Leiva-Merentes:2013aa}).
\begin{lemma}
The approximate controllability of the linear system \eqref{eq:linear}  on $[\tau-\delta,\tau]$ is equivalent to  any of the following statements
\begin{enumerate}[(a)]
\item $\overline{Rang(G_{\tau\delta})}={\mathcal{Z}}^{1}.$
\item $\ker(G_{\tau\delta}^{*})={0}.$
\item For $0\neq z \in\ Z^{1},  \  \ {\langle { Q_{\tau\delta}z,z}\rangle}>0.$
\end{enumerate}
\end{lemma}

 The controllability of the linear system \eqref{eq:linear} on $[0,\tau]$ was proved  by A. Carrasco and H. Leiva in \cite{Carrasco-Leiva:2013aa}.  Theorem \ref{A1.5}   and Lemma \ref{lema} characterized the controllability of the system \eqref{eq:linear}, their proofs and details can be found in \cite{Bashirov-Kerimov:1997aa,Bashirov-Mahmudov:1999aa, Curtain-Pritchard:1978aa, Curtain-Zwart:1995aa, Leiva-Merentes:2013aa}

 \begin{teo}\label{A1.5} The system \eqref{eq:linear} is approximately
controllable on $[0,\tau]$ if and only if any one of the following conditions hold:
\begin{enumerate}
\item ${\displaystyle}\lim_{\alpha \to 0^+} \alpha(\alpha I +Q_{\tau\delta}^{*})^{-1}z =0 $.\\
\item If $z\in Z^{1}$,  $0<\alpha \leq 1$ and $u_{\alpha}=G_{\tau\delta}^{*}(\alpha I +
Q_{\tau\delta}^{*})^{-1}z$, then
$$
G_{\tau\delta}u_{\alpha}=z - \alpha(\alpha I+ Q_{\tau\delta})^{-1}z \quad \mbox{and} \quad
\displaystyle\lim_{\alpha\to 0}G_{\tau\delta}u_{\alpha}=z.
$$
Moreover, for each $v\in L^{2}([\tau-\delta,\tau];{\mathcal{U}})$, the sequence of controls
$$
u_{\alpha}=G_{\tau\delta}^{*}(\alpha I +
Q_{\tau\delta}^{*})^{-1}z + (v-G_{\tau\delta}^{*}(\alpha I +
Q_{\tau\delta}^{*})^{-1}G_{\tau\delta}v),
$$
satisfies
$$
G_{\tau\delta}u_{\alpha}=z-\alpha(\alpha I +
Q_{\tau\delta}^{*})^{-1}(z-G_{\tau\delta}v)
\quad
\mbox{and}
\quad
\displaystyle\lim_{\alpha\to 0}G_{\tau\delta}u_{\alpha}=z,
$$
with the error $E_{\tau\delta}z=\alpha(\alpha I +
Q_{\tau\delta})^{-1}(z+G_{\tau\delta}v),\;\alpha\in(0,1].
$
\end{enumerate}
\end{teo}
Theorem \ref{A1.5} indicates that the family of linear operators $
\Gamma_{\tau\delta}=G_{\tau\delta}^{*}(\alpha I +
Q_{\tau\delta}^{*})^{-1}
$
is an approximate right inverse for the $G_{\tau\delta}$, in
the sense that
$$
\displaystyle\lim_{\alpha\longrightarrow 0}G_{\tau\delta}\Gamma_{\tau\delta}=I,
$$
in the strong topology.

\begin{lemma}\label{lema}
$Q_{\tau\delta}> 0$, if and only if, the linear system \eqref{eq:linear} is controllable on $[\tau-\delta, \tau]$.
Moreover, for  given initial state $y_0$ and  final state $z_{1}$, there exists a sequence of controls $\{u_{\alpha}^{\delta}\}_{0 <\alpha \leq 1}$ in the space $L^2(\tau-\delta,\tau;{\mathcal{U}})$, defined by
$$
u_{\alpha}=u_{\alpha}^{\delta}= G_{\tau\delta}^{*}(\alpha I+ G_{\tau\delta}G_{\tau\delta}^{*})^{-1}(z_{1} - T(\tau)y_0),
$$
such that the solutions $y(t)=y(t,\tau-\delta, y_0, u_{\alpha}^{\delta})$ of the initial value problem
\begin{equation}\label{IVL}
\left\{
\begin{array}{l}
y'={\mathbb{A}} y+{\mathbb{B}} u_{\alpha}(t), \ \  y \in {\mathcal{Z}}^{1}, \ \ t>0,\\
y(\tau-\delta) = y_0,
\end{array}
\right.
\end{equation}
satisfies
\begin{equation}\label{eq:limit}
\lim_{\alpha \to 0^{+}}y(\tau)
= \lim_{\alpha \to 0^{+}}\left(T(\delta)y_0 + \int_{\tau-\delta}^{\tau}T(\tau-s){\mathbb{B}} u_{\alpha}(s)ds \right)= z_{1}.
\end{equation}
\end{lemma}

\section{Controllability of the Semilinear System}\label{sec:semilineal}

This section is devoted to prove the main result of this paper,   the approximate controllability of the beam equation  Theorem \ref{main}, which is  it is equivalent to prove the controllability of the abstract system \eqref{eq:abstract}  given in the following definition

\begin{definition} \label{def2}{\rm (}{\sf Approximate Controllability}{\rm )}  The system \eqref{eq:abstract} is said
to be approximately controllable on $[0,\tau]$
if  for every $\epsilon>0$,   every
$\Phi\in {\mathcal{C}}\left(-r,0;  {\mathcal{Z}}^{1} \right)$ and a given initial state $z_{1}\in {\mathcal{Z}}^{1}$ there exists $u\in L^{2}(0,\tau;{\mathcal{U}})$, such that,  the corresponding  mild solution
\begin{align}\label{eq:mild}
z^{u}(t) = & {\displaystyle} T(t)\Phi(0)+\int_{0}^{t}T(t-s)\left[{\mathbb{B}} u(s)+\left(\int_{0}^{s}{\mathbb{M}_g}(s,l,z(l-r))dl\right)\right]ds \\
& \;+   {\displaystyle} \int_{0}^{t}T(t-s){\mathbb{F}}(s,z(s-r),u(s))ds  +   \sum_{0 < t_k < t} T(t-t_k ){\mathbb{I}}_{k}(t_k,z(t_k), u(t_k)),  \nonumber
\end{align}
satisfies $z(0)=\Phi(0)$ and
\begin{equation}\label{eq:goal}
{\left\| { z^u(\tau) - z_{1}}\right\|}_{{\mathcal{Z}}^1}<\epsilon.
\end{equation}

\end{definition}

The approach to obtain \eqref{eq:goal} consist in construct a sequence of controls conducting the system from the initial condition $\Phi$ to a small ball around $z_1,$ taking advantage of the delay, which  allows us to pullback the corresponding family of  solutions to a fixed trajectory in short time interval. Now, we are ready to present the proof of our main result

\begin{teo} \label{main}
Under the condition (\ref{eq:bound}) the impulsive semilinear beam equation with memory and delay (\ref{eq:beam}) is approximately
controllable on $[0,\tau]$.
\end{teo}
{\noindent {\bf Proof.} \mbox{}} Let $\epsilon>0$, and given $\Phi\in \mathcal{C}$ and a final state $z_{1}$.  Consider any $u\in L^{2}([0,\tau];{\mathcal{U}})$ and the corresponding mild solution \eqref{eq:mild}  of the initial value problem \eqref{eq:abstract}, denoted by $z(t)=z(t,0,\Phi,u)$. For $0\leq\alpha \leq 1,$ define the control $u_{\alpha}^{\delta}\in L^{2}([0,\tau];{\mathcal{U}})$  as follows
$$
u_{\alpha}^{\delta}(t)=\left\{\begin{array}{ccl}
                         u(t), &&0\leq t\leq \tau-\delta, \\
                         u_{\alpha}(t), &\quad& \tau-\delta\leq t\leq \tau,
                       \end{array}\right.
$$
with $
u_{\alpha}= {\mathbb{B}}^{*}T^{*}(\tau-t)(\alpha I+ G_{\tau\delta}G_{\tau\delta}^{*})^{-1}(z_{1} - T(\delta)z(\tau-\delta)).
$
Thus,

 $0<\delta<\tau-t_{p}$ and the corresponding mild solution at time $\tau$ can be written as follows:
 \begin{eqnarray*}
 {\displaystyle}
z^{\delta,\alpha}(\tau) &=&
 {\displaystyle} T(\tau)\Phi(0) +\int_{0}^{\tau}T(\tau-s)
	\left[ {\mathbb{B}}   u_{\alpha}^{\delta} (s)
		+ \int_0^s {\mathbb{M}_g}(z^{\delta,\alpha}(l-r))dl\right]ds+ \\
	&&+  \int_{0}^{\tau}T(\tau-s){\mathbb{F}}(s,z^{\delta,\alpha}(s-r),u_{\alpha}^{\delta}(s))ds+
\sum_{0 < t_k < \tau} T(t-t_k ){\mathbb{I}}_{k}(t_k,z^{\delta,\alpha}(t_k), u_{\alpha}^{\delta}(t_k))\\
&=&T(\delta)\left\{T(\tau-\delta)\Phi(0)
+\int_{0}^{\tau-\delta}T(\tau-\delta-s) \left({\mathbb{B}}   u_{\alpha}^{\delta} (s)+{\mathbb{F}}(s,z^{\delta,\alpha}(s-r),u_{\alpha}^{\delta}(s))\right)ds\right.\\
&&\qquad\quad+\int_{0}^{\tau-\delta}T(\tau-\delta-s)  \int_0^s {\mathbb{M}_g}(s,l, z^{\delta,\alpha}(l-r))dlds\\
&&\qquad\quad\left.+ \sum_{0 < t_k < \tau-\delta} T(t-\delta-t_k ){\mathbb{I}}_{k}(t_k,z^{\delta,\alpha}(t_k), u_{\alpha}^{\delta}(t_k))\right\}+\\
&& + \int_{\tau-\delta}^{\tau}T(\tau-s)\left({\mathbb{B}} u_{\alpha}(s)+
{\mathbb{F}}(s,z^{\delta,\alpha}(s-r),u_{\alpha}^{\delta}(s))+\int_0^s{\mathbb{M}_g}(s,l,z^{\delta,\alpha}(l-r))dl\right)ds.
\end{eqnarray*}
Thus,
\begin{align*}
z^{\delta,\alpha}(\tau)  = &T(\delta)z(\tau-\delta)+ \int_{\tau-\delta}^{\tau}T(\tau-s)\left({\mathbb{B}} u_{\alpha}(s)+
{\mathbb{F}}(s,z^{\delta,\alpha}(s-r),u_{\alpha}^{\delta}(s))\right)ds \\
&+ \int_{\tau-\delta}^{\tau}T(\tau-s)\int_0^s{\mathbb{M}_g}(s,l,z^{\delta,\alpha}(l-r))dlds.
\end{align*}
Note that the corresponding solution $y^{\delta,\alpha}(t)=y(t,\tau-\delta,z(\tau-\delta),u_{\alpha})$ of the initial value problem \eqref{IVL} at time $\tau$ is:
$$
y^{\delta,\alpha}(\tau)=T(\delta)z(\tau-\delta)+ \int_{\tau-\delta}^{\tau}T(\tau-s){\mathbb{B}}_{\varpi} u_{\alpha}(s)d.,
$$
Hence,
$$
z^{\delta,\alpha}(\tau)-y^{\delta,\alpha}(\tau)=
\int_{\tau-\delta}^{\tau}T(\tau-s)\left(\int_{0}^{s}{\mathbb{F}}(s,z^{\delta,\alpha}(s-r),u_{\alpha}^{\delta}(s)+{\mathbb{M}_g}(s,l,z^{\delta,\alpha}(l-r))dl)\right)ds,
$$
 and from condition (\ref{eq:bound})we obtain that
\begin{align*}
  {\left\| { z^{\delta,\alpha}(\tau)-y^{\delta,\alpha}(\tau)}\right\|} & \leq \int_{\tau-\delta}^{\tau} {\left\| { T(\tau-s)}\right\|}\left( \tilde{a}{\left\| {\Phi(s-r)}\right\|}+\tilde{b}\right)ds \\
   & +  \int_{\tau-\delta}^{\tau}{\left\| { T(\tau-s)}\right\|}\int_{0}^{s}{\left\| {{\mathbb{M}_g}(s,l,z^{\delta,\alpha}(l-r))}\right\|}dlds.
\end{align*}
Observe that
 $0< \delta< r$ and $\tau-\delta \leq s\leq \tau$, thus $$l-r \leq s-r \leq \tau-r< \tau-\delta.
 $$
Therefore,
$
z^{\delta,\alpha}(l-r)=z(l-r) $  and $ z^{\delta,\alpha}(s-r)=z(s-r),
$ implying that for $\epsilon>0$ there exists $\delta>0$ such that
\begin{align*}
 {\left\| {z^{\delta,\alpha}(\tau)-y^{\delta,\alpha}(\tau)}\right\|} & \leq \int_{\tau-\delta}^{\tau}{\left\| { T(\tau-s)}\right\|}\left( \tilde{a}{\left\| {z(s-r)}\right\|}+\tilde{b}\right)ds \\
  &\quad +  \int_{\tau-\delta}^{\tau}{\left\| {T(\tau-s)}\right\|}\int_{0}^{s}{\left\| { {\mathbb{M}_g}(s,l,z(l-r))}\right\|} dlds  \\
   & <  \displaystyle\frac{\epsilon}{2}.
\end{align*}
Additionally, for $0<\alpha <1$, Lemma \ref{lema}  \eqref{eq:limit} yields
$$
 {\left\| { y^{\delta,\alpha}(\tau)-z_{1}}\right\|}  <  \frac{\epsilon}{2}.
$$
Thus,
$$
\begin{array}{lll}
 {\left\| { z^{\delta,\alpha}(\tau)-z_{1}}\right\|}  & \leq &  {\left\| { z^{\delta,\alpha}(\tau)-y^{\delta,\alpha}(\tau)}\right\|} +  {\left\| { y^{\delta,\alpha}(\tau)-z_{1}}\right\|}  <  \frac{\epsilon}{2}+ \frac{\epsilon}{2}=\epsilon,
\end{array}
$$
which completes our proof.
\section{Final Remarks}\label{final}
\noindent

We believe that the same technique  can be applied for  controlling diffusion processes systems involving compact semigroups. In particular, our result can be formulated in a more general setting  for the semilinear evolution equation with impulses, delay and memory  in a Hilbert space ${\mathcal{Z}}$
\begin{equation*}\label{eq:formulation1}
\left\{
\begin{array}{ll}
z' = {\mathbb{A}} z + {\mathbb{B}} u +{\displaystyle}\int_0^t {\mathbb{M}_g}(t,s,z(s-r))ds+ {\mathbb{F}}(t,z(t-r),u(t)),&  z \in  {\mathcal{Z}},  0< t \neq t_k, \\
z(s)=\Phi(s), &s \in [-r,0]\\
z(t_{k}^{+}) = z(t_{k}^{-})+{\mathbb{I}}_{k}(t_k, z(t_{k}),u(t_{k})), &  k=1,2, \dots, p,
\end{array}
\right.
\end{equation*}
where $u\in L^{2}(0,\tau;{\mathcal{U}})$, ${\mathcal{U}}$ is another Hilbert space, ${\mathbb{B}} :{\mathcal{U}} \longrightarrow {\mathcal{Z}}$ is a bounded linear operator,${\mathbb{I}}_{k}, {\mathbb{F}}:[0, \tau]\times C(-r,0; {\mathcal{Z}}) \times {\mathcal{U}} \rightarrow {\mathcal{Z}}$, ${\mathbb{A}} :D({\mathbb{A}}) \subset {\mathcal{Z}} \rightarrow {\mathcal{Z}}$ is an unbounded linear operator in ${\mathcal{Z}}$ that generates a strongly continuous semigroup according to
Lemma 2.1 from \cite{Leiva:2003aa}:
\begin{equation}\label{damp2}
  T(t)z =\sum_{nj=1}^{\infty}e^{A_{j}t}P_jz  \mbox{, } \ \ z\in {\mathcal{Z}} \mbox{, } \ \ t \geq 0,     
\end{equation}
where  $\left\{ P_j\right\} _{j \geq 0}$ is a complete family of orthogonal projections in the Hilbert space ${\mathcal{Z}}$ and
\begin{equation}
\|F(t,\Phi,u) \|_{\mathcal{Z}}  \leq   \tilde{a} \|\Phi(-r)\|_{\mathcal{Z}} +\tilde{b}, \quad \forall (t, \Phi, u) \in [0, \tau]\times C(-r,0;  {\mathcal{Z}} ) \times {\mathcal{U}}.
\end{equation}\begin{thebibliography}{99}

\bibitem{Bashirov-Ghahramanlou:2014aa}
A.~E. Bashirov and N.~Ghahramanlou.
\newblock {\em On partial approximate controllability of semilinear systems.}
\newblock { Cogent Engineering}, 1(1):965947, 2014.

\bibitem{Bashirov-Jneid:2013aa}
A.~E. Bashirov and M.~Jneid.
\newblock {\em On partial complete controllability of semilinear systems.}
\newblock In {Abstract and Applied Analysis}, volume 2013. Hindawi
  Publishing Corporation, 2013.

\bibitem{Bashirov-Kerimov:1997aa}
A.~E. Bashirov and K.~R. Kerimov.
\newblock {\em On controllability conception for stochastic systems.}
\newblock { SIAM journal on control and optimization}, 35(2):384--398, 1997.

\bibitem{Bashirov-Mahmudov:1999aa}
A.~E. Bashirov and N.~I. Mahmudov.
\newblock {\em On concepts of controllability for deterministic and stochastic}
  systems.
\newblock { SIAM Journal on Control and Optimization}, 37(6):1808--1821,
  1999.

\bibitem{Bashirov-Mahmudov:2007aa}
A.~E. Bashirov, N.~Mahmudov, N.~{\c{S}}em{\i}, and H.~Et{\i}kan.
\newblock {\em Partial controllability concepts.}
\newblock { International Journal of Control}, 80(1):1--7, 2007.

\bibitem{Carrasco-Leiva:2016aa}
A.~Carrasco, H.~Leiva, and N.~Merentes.
\newblock {\em Controllability of the perturbed beam equation.}
\newblock { IMA Journal of Mathematical Control and Information},
  33(3):603--615, 2016.

\bibitem{Carrasco-Leiva:2013aa}
A.~Carrasco, H.~Leiva, and J.~Sanchez.
\newblock {\em Controllability of the semilinear beam equation.}
\newblock { Journal of Dynamical and Control Systems}, 19(4):553--568, 2013.

\bibitem{Carrasco-Leiva:2014aa}
A.~Carrasco, H.~Leiva, J.~Sanchez, and M.~Tineo.
\newblock {\em Approximate controllability of the semilinear impulsive beam equation.}
\newblock {Transaction on IoT and Cloud Computing}, 2(3):70--88, 2014.

\bibitem{Curtain-Pritchard:1978aa}
R.~F. Curtain and A.~J. Pritchard.
\newblock {\em Infinite dimensional linear systems theory}, volume~8.
\newblock Springer-Verlag Berlin, 1978.

\bibitem{Curtain-Zwart:1995aa}
R.~Curtain and H.~Zwart.
\newblock {\em An introduction to infinite-dimensional linear systems theory.}
\newblock { Texts in Applied Mathematics, Springer-Verlag, New York}, 1995.

\bibitem{Guevara-Leiva:2016aa}
C.~Guevara and H.~Leiva.
\newblock Controllability of the impulsive semilinear heat equation with memory
  and delay.
\newblock {\em Journal of Dynamical and Control Systems}, pages 1--11, 2016.

\bibitem{Guevara-Leiva:2017aa}
C.~Guevara and H.~Leiva.
\newblock { Controllability of the strongly damped impulsive semilinear wave
  equation with memory and delay.}
  \newblock https://arxiv.org/abs/1704.02561
\newblock 2017.

\bibitem{Leiva:2003aa}
H.~Leiva.
\newblock {\em A lemma on $C_0$-semigroups and applications.}
\newblock { Quaestiones Mathematicae}, 26(3):247--265, 2003.

\bibitem{Leiva-Merentes:2013aa}
H.~Leiva, N.~Merentes, and J.~Sanchez.
\newblock {\em A characterization of semilinear dense range operators and
  applications.}
\newblock In Abstract and Applied Analysis, volume 2013. Hindawi
  Publishing Corporation, 2013.

\bibitem{Oliveira:1998aa}
L.~A.~F. de~Oliveira.
\newblock {\em On reaction-diffusion systems.}
\newblock { Electron. J. Diff. Eqns}, (24):1--10, 1998.

\bibitem{Timoshenko:1921aa}
S.~P. Timoshenko.
\newblock {\em On the correction for shear of the differential equation for
  transverse vibrations of prismatic bars.}
\newblock { The London, Edinburgh, and Dublin Philosophical Magazine and
  Journal of Science}, 41(245):744--746, 1921.

\bibitem{Timoshenko:1922aa}
S.~P. Timoshenko.
\newblock {\em On the transverse vibrations of bars of uniform cross-section.}
\newblock { The London, Edinburgh, and Dublin Philosophical Magazine and
  Journal of Science}, 43(253):125--131, 1922.

\bibitem{Timoshenko:1953aa}
S.~Timoshenko.
\newblock {\em History of strength of materials: with a brief account of the
  history of theory of elasticity and theory of structures}.
\newblock Courier Corporation, 1953.

\bibitem{Wang-Tan:2006aa}
C.~Wang, V.~Tan, and Y.~Zhang.
\newblock {\em Timoshenko beam model for vibration analysis of multi-walled carbon
  nanotubes.}
\newblock { Journal of Sound and Vibration}, 294(4):1060--1072, 2006.

\bibitem{Wang-Zhang:2006aa}
C.~Wang, Y.~Zhang, S.~S. Ramesh, and S.~Kitipornchai.
\newblock {\em Buckling analysis of micro-and nano-rods/tubes based on nonlocal
  timoshenko beam theory.}
\newblock { Journal of Physics D: Applied Physics}, 39(17):3904, 2006.

\end{thebibliography}

\end{document}

